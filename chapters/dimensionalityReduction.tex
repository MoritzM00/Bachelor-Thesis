%% ==============================
\chapter{Dimensionsreduktion}
\label{ch:Dimensionsreduktion}
%% ==============================

\idea{umgangssprachlich und mathematisch formulieren}

Oftmals liegen hochdimensionale Daten in einem niedrigdimensionalem Unterraum. Wir beziehen uns auf die hochdimensionalen Daten $\vect{x_1},\ldots,\vect{x_D}$ mit der Datenmatrix $\mat{X} \in \mathbb{R}^{n \times D}$. Jeder Datenpunkt $\vect{x_i}$ ist also ein $D$-dimensionaler Vektor. Die niedrigdimensionale Repräsentation bezeichnen wir mit $\vect{y_1},\ldots,\vect{y_d}$, die in einer $n \times d$-Datenmatrix $\mat{Y}$ angeordnet sind. Hierbei bezeichnet $d$ die
\newterm{intrinsische Dimension}\rewrite{hier braucht man mehr Informationen. Was ist die Intuition hinter der intrins. Dimension?} des Datensatzes, wobei $d < D$ gilt, um eine Dimensionsreduktion zu erreichen.


Das Ziel der Dimensionsreduktion ist es nun, diesen Unterraum zu finden und die Daten in diese niedrigdimensionale Repräsentation zu projizieren, wobei möglichst wenig Informationen gegenüber der ursprünglichen (hochdimensionalen) Repräsentation verloren gehen sollen.

\idea{generelle Ansätze grob formulieren, also nichtlinear vs. linear, topology vs. distance preservation, convex vs. non-convex}