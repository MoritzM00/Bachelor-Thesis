%% ==============================
\chapter{Dimensionsreduktion}
\label{ch:Dimensionsreduktion}
%% ==============================

Die Dimensionsreduktion hat im Kern das Ziel der Abbildung eines hochdimensionalen Datensatzes auf
eine niedrigdimensionale, sogenannte \newterm{latente} Repräsentation, wobei möglichst wenig
Information über die Daten verloren gehen soll \parencite[2]{Lee.2007}. Dies ist darin begründet, dass Daten oft nur künstlich hochdimensional, also
\textit{redundant} sind. Dies bedeutet, dass die Daten effizienter über eine kleinere Menge von
Merkmalen $y_1,\ldots,y_d$ ausgedrückt werden kann, als über die ursprüngliche Repräsentation durch
die Merkmale $x_1,\ldots,x_D$ mit $d < D$. Hierbei bezeichnet $D$ die \newterm{extrinsische
	Dimension} des zugehörigen Ursprungsraumes $\mathcal{X}$ (welcher üblicherweise dem $\real^D$
entspricht) und $d$ bezeichnet die \newterm{intrinsische Dimension} der Daten. Die intrinsische
Dimension wird teilweise auch als latente Dimension bezeichnet und beschreibt die minimale Anzahl
an Merkmalsvariablen $y_i$, die für die Generierung der Daten benötigt werden \parencite[47]{Lee.2007}. Die intrinsische Dimension kann neben dieser intuitiven Sichtweise auch über
topologische Überlegungen der zugrundeliegenden Verteilung der Daten definiert werden. Diese Idee
wird in \secref{ch:Dimensionsreduktion:MannigfaltigkeitenIntrinsDim} durch das Konzept von
Mannigfaltigkeiten erläutert.

Formell wird die ursprüngliche (hochdimensionale) Repräsentation mit dem $D$-dimensionalen
Zufallsvektor $\rvect{x} = \tr{(x_1, \ldots, x_D)}$ und die latente (niedrigdimensionale)
Repräsentation mit dem $d$-dimensionalen Zufallsvektor $\rvect{y}$ gekennzeichnet. Hierbei
bezeichnet $\tr{\,(\cdot)\,}$ die Transponierte. Liegt eine konkrete Stichprobe vor, so werden die
einzelnen Stichproben $\vect{x}_i, i = 1,\ldots,n$ als Spaltenvektoren in der $n \times D$
Datenmatrix $\mat{X}$ angeordnet. Analog dazu werden die transformierten (auch: projizierten) Daten
in der Matrix $\mat{Y} \in \real^{n \times d}$ angeordnet. Es wird angenommen, dass die Datenmatrix
$\mat{X}$ \textit{zentriert} ist. Dies kann jederzeit durch Subtraktion des Erwartungswertes
$\Exp[x_i]$ einer Variable $x_i$ von Spalte $i$ sichergestellt werden.

Nachdem nun die grundlegende Terminologie und das Ziel der Dimensionsreduktion geklärt wurde,
werden im Folgenden einige weitere wichtige Ideen und Konzepte erläutert. Dazu wird in
\secref{ch:Dimensionsreduktion:FluchDerDim} der Fluch der Dimensionalität sowie in
\secref{ch:Dimensionsreduktion:MannigfaltigkeitenIntrinsDim} die Idee von Mannigfaltigkeiten und
der intrinsischen Dimension behandelt. In \secref{ch:Dimensionsreduktion:Ansaetze} werden kurz
verschiedene Ansätze zur Dimensionsreduktion vorgestellt und letztlich wird in
\secref{ch:Dimensionsreduktion:Merkmalsextrahierung} die Relation zum Gebiet der
Merksmalsextrahierung (engl. \textit{feature extraction}) besprochen.
\section{Der Fluch der Dimensionalität}
\label{ch:Dimensionsreduktion:FluchDerDim}

Sehr hochdimensionale Räume weisen einige Phänomene auf, die in niedrigdimensionalen Räumen nicht
anzutreffen sind. Diese Phänomene bereiten statistische und algorithmische Probleme und werden
unter dem Fluch der Dimensionalität (engl. \textit{Curse of Dimensionality}) zusammengefasst.
Hauptsächlich meint man damit aber die Konzentration von Normen und das Phänomen der leeren Räume.
\ldots

\section{Mannigfaltigkeiten und intrinsische Dimension}
\label{ch:Dimensionsreduktion:MannigfaltigkeitenIntrinsDim}

Wie eingangs besprochen wird bei hochdimensionalen Daten oft von Redundanz oder
Abhängigkeitsstrukturen in den Merkmalen ausgegangen. Eng damit verbunden ist die Idee, dass Daten
auf einer sogenannten \newterm{Mannigfaltigkeit} (engl. \textit{manifold}) liegen.
Dimensionsreduktionsmethoden, die auf dieser Idee basieren, gehören zu einem wichtigen Teilgebiet
der Dimensionreduktion: dem Erlernen von Mannigfaltigkeiten \parencite{Cayton.2005}. Motiviert werden diese Ansätze durch die Hypothese, dass reale
hochdimensionale Daten in sehr vielen Fällen auf einer in diesem hochdimensionalen Raum
\textit{eingebetteten} Mannigfaltigkeit $\mathcal{M}$ der Dimensionalität $d$ < $D$ liegen \parencite[vgl.][1]{Cayton.2005}. In diesem Abschnitt wird ein kurzer Überblick über den abstrakten
Begriff einer Mannigfaltigkeit gegeben, wodurch ein geometrischer Bezug der intrinsischen Dimension
hergestellt werden kann.

Eine $d$-dimensionale Mannigfaltigkeit $\mathcal{M}$ ist lokal \textit{homöomorph} zum $\real^d$,
das heißt $\mathcal{M}$ ähnelt \textit{lokal} dem $\real^d$ \parencite[3]{Lee.2011}. Das bedeutet, dass es für jeden Punkt $\vect{z} \in \mathcal{M}$ eine stetige
Abbildung $\phi: B_\epsilon(\vect{z}) \rightarrow \real^d$ gibt, deren Inverse ebenfalls stetig
ist. Hierbei ist $B_\epsilon(\vect{z})$ ein Ball mit Radius $\epsilon > 0$ um $\vect{z}$. Die
Abbildung $\phi$ heißt \textit{Karte} und die Gesamtheit aller Karten ergibt den \textit{Atlas} von
$\mathcal{M}$ \parencite[4]{Cayton.2005}. Intuitiv können diese Begriffe besser anhand eines anschaulichen Beispiels
erklärt werden, weshalb in \figref{fig:Torus} ein Torus dargestellt ist. Dieses Objekt ist eine
zweidimensionale Mannigfaltigkeit eingebettet im $\real^D$.
\begin{figure}[ht]
	\centering
	%% Creator: Matplotlib, PGF backend
%%
%% To include the figure in your LaTeX document, write
%%   \input{<filename>.pgf}
%%
%% Make sure the required packages are loaded in your preamble
%%   \usepackage{pgf}
%%
%% Also ensure that all the required font packages are loaded; for instance,
%% the lmodern package is sometimes necessary when using math font.
%%   \usepackage{lmodern}
%%
%% Figures using additional raster images can only be included by \input if
%% they are in the same directory as the main LaTeX file. For loading figures
%% from other directories you can use the `import` package
%%   \usepackage{import}
%%
%% and then include the figures with
%%   \import{<path to file>}{<filename>.pgf}
%%
%% Matplotlib used the following preamble
%%   
%%   \usepackage{fontspec}
%%   \setmainfont{DejaVuSerif.ttf}[Path=\detokenize{/Users/moritzmistol/.pyenv/versions/3.9.13/envs/bachelor-thesis-latex/lib/python3.9/site-packages/matplotlib/mpl-data/fonts/ttf/}]
%%   \setsansfont{DejaVuSans.ttf}[Path=\detokenize{/Users/moritzmistol/.pyenv/versions/3.9.13/envs/bachelor-thesis-latex/lib/python3.9/site-packages/matplotlib/mpl-data/fonts/ttf/}]
%%   \setmonofont{DejaVuSansMono.ttf}[Path=\detokenize{/Users/moritzmistol/.pyenv/versions/3.9.13/envs/bachelor-thesis-latex/lib/python3.9/site-packages/matplotlib/mpl-data/fonts/ttf/}]
%%   \makeatletter\@ifpackageloaded{underscore}{}{\usepackage[strings]{underscore}}\makeatother
%%
\begingroup%
\makeatletter%
\begin{pgfpicture}%
\pgfpathrectangle{\pgfpointorigin}{\pgfqpoint{4.680250in}{2.181932in}}%
\pgfusepath{use as bounding box, clip}%
\begin{pgfscope}%
\pgfsetbuttcap%
\pgfsetmiterjoin%
\definecolor{currentfill}{rgb}{1.000000,1.000000,1.000000}%
\pgfsetfillcolor{currentfill}%
\pgfsetlinewidth{0.000000pt}%
\definecolor{currentstroke}{rgb}{1.000000,1.000000,1.000000}%
\pgfsetstrokecolor{currentstroke}%
\pgfsetdash{}{0pt}%
\pgfpathmoveto{\pgfqpoint{0.000000in}{0.000000in}}%
\pgfpathlineto{\pgfqpoint{4.680250in}{0.000000in}}%
\pgfpathlineto{\pgfqpoint{4.680250in}{2.181932in}}%
\pgfpathlineto{\pgfqpoint{0.000000in}{2.181932in}}%
\pgfpathlineto{\pgfqpoint{0.000000in}{0.000000in}}%
\pgfpathclose%
\pgfusepath{fill}%
\end{pgfscope}%
\begin{pgfscope}%
\pgfsetbuttcap%
\pgfsetmiterjoin%
\definecolor{currentfill}{rgb}{1.000000,1.000000,1.000000}%
\pgfsetfillcolor{currentfill}%
\pgfsetlinewidth{0.000000pt}%
\definecolor{currentstroke}{rgb}{0.000000,0.000000,0.000000}%
\pgfsetstrokecolor{currentstroke}%
\pgfsetstrokeopacity{0.000000}%
\pgfsetdash{}{0pt}%
\pgfpathmoveto{\pgfqpoint{0.050000in}{0.050000in}}%
\pgfpathlineto{\pgfqpoint{2.131932in}{0.050000in}}%
\pgfpathlineto{\pgfqpoint{2.131932in}{2.131932in}}%
\pgfpathlineto{\pgfqpoint{0.050000in}{2.131932in}}%
\pgfpathlineto{\pgfqpoint{0.050000in}{0.050000in}}%
\pgfpathclose%
\pgfusepath{fill}%
\end{pgfscope}%
\begin{pgfscope}%
\pgfpathrectangle{\pgfqpoint{0.050000in}{0.050000in}}{\pgfqpoint{2.081932in}{2.081932in}}%
\pgfusepath{clip}%
\pgfsetbuttcap%
\pgfsetroundjoin%
\definecolor{currentfill}{rgb}{0.206756,0.371758,0.553117}%
\pgfsetfillcolor{currentfill}%
\pgfsetlinewidth{0.000000pt}%
\definecolor{currentstroke}{rgb}{0.000000,0.000000,0.000000}%
\pgfsetstrokecolor{currentstroke}%
\pgfsetdash{}{0pt}%
\pgfpathmoveto{\pgfqpoint{1.174624in}{1.361876in}}%
\pgfpathlineto{\pgfqpoint{1.175261in}{1.372216in}}%
\pgfpathlineto{\pgfqpoint{1.175832in}{1.382432in}}%
\pgfpathlineto{\pgfqpoint{1.176336in}{1.392485in}}%
\pgfpathlineto{\pgfqpoint{1.176771in}{1.402337in}}%
\pgfpathlineto{\pgfqpoint{1.177134in}{1.411949in}}%
\pgfpathlineto{\pgfqpoint{1.142047in}{1.413401in}}%
\pgfpathlineto{\pgfqpoint{1.106879in}{1.413590in}}%
\pgfpathlineto{\pgfqpoint{1.071755in}{1.412515in}}%
\pgfpathlineto{\pgfqpoint{1.036798in}{1.410178in}}%
\pgfpathlineto{\pgfqpoint{1.002134in}{1.406587in}}%
\pgfpathlineto{\pgfqpoint{1.002869in}{1.397002in}}%
\pgfpathlineto{\pgfqpoint{1.003747in}{1.387184in}}%
\pgfpathlineto{\pgfqpoint{1.004765in}{1.377171in}}%
\pgfpathlineto{\pgfqpoint{1.005919in}{1.367001in}}%
\pgfpathlineto{\pgfqpoint{1.007204in}{1.356714in}}%
\pgfpathlineto{\pgfqpoint{1.040365in}{1.360171in}}%
\pgfpathlineto{\pgfqpoint{1.073807in}{1.362421in}}%
\pgfpathlineto{\pgfqpoint{1.107411in}{1.363457in}}%
\pgfpathlineto{\pgfqpoint{1.141057in}{1.363275in}}%
\pgfpathlineto{\pgfqpoint{1.174624in}{1.361876in}}%
\pgfpathclose%
\pgfusepath{fill}%
\end{pgfscope}%
\begin{pgfscope}%
\pgfpathrectangle{\pgfqpoint{0.050000in}{0.050000in}}{\pgfqpoint{2.081932in}{2.081932in}}%
\pgfusepath{clip}%
\pgfsetbuttcap%
\pgfsetroundjoin%
\definecolor{currentfill}{rgb}{0.267968,0.223549,0.512008}%
\pgfsetfillcolor{currentfill}%
\pgfsetlinewidth{0.000000pt}%
\definecolor{currentstroke}{rgb}{0.000000,0.000000,0.000000}%
\pgfsetstrokecolor{currentstroke}%
\pgfsetdash{}{0pt}%
\pgfpathmoveto{\pgfqpoint{1.170558in}{1.309723in}}%
\pgfpathlineto{\pgfqpoint{1.171479in}{1.320081in}}%
\pgfpathlineto{\pgfqpoint{1.172350in}{1.330515in}}%
\pgfpathlineto{\pgfqpoint{1.173166in}{1.340986in}}%
\pgfpathlineto{\pgfqpoint{1.173925in}{1.351453in}}%
\pgfpathlineto{\pgfqpoint{1.174624in}{1.361876in}}%
\pgfpathlineto{\pgfqpoint{1.141057in}{1.363275in}}%
\pgfpathlineto{\pgfqpoint{1.107411in}{1.363457in}}%
\pgfpathlineto{\pgfqpoint{1.073807in}{1.362421in}}%
\pgfpathlineto{\pgfqpoint{1.040365in}{1.360171in}}%
\pgfpathlineto{\pgfqpoint{1.007204in}{1.356714in}}%
\pgfpathlineto{\pgfqpoint{1.008617in}{1.346349in}}%
\pgfpathlineto{\pgfqpoint{1.010150in}{1.335946in}}%
\pgfpathlineto{\pgfqpoint{1.011799in}{1.325545in}}%
\pgfpathlineto{\pgfqpoint{1.013557in}{1.315186in}}%
\pgfpathlineto{\pgfqpoint{1.015417in}{1.304908in}}%
\pgfpathlineto{\pgfqpoint{1.046143in}{1.308133in}}%
\pgfpathlineto{\pgfqpoint{1.077132in}{1.310231in}}%
\pgfpathlineto{\pgfqpoint{1.108272in}{1.311198in}}%
\pgfpathlineto{\pgfqpoint{1.139452in}{1.311028in}}%
\pgfpathlineto{\pgfqpoint{1.170558in}{1.309723in}}%
\pgfpathclose%
\pgfusepath{fill}%
\end{pgfscope}%
\begin{pgfscope}%
\pgfpathrectangle{\pgfqpoint{0.050000in}{0.050000in}}{\pgfqpoint{2.081932in}{2.081932in}}%
\pgfusepath{clip}%
\pgfsetbuttcap%
\pgfsetroundjoin%
\definecolor{currentfill}{rgb}{0.150476,0.504369,0.557430}%
\pgfsetfillcolor{currentfill}%
\pgfsetlinewidth{0.000000pt}%
\definecolor{currentstroke}{rgb}{0.000000,0.000000,0.000000}%
\pgfsetstrokecolor{currentstroke}%
\pgfsetdash{}{0pt}%
\pgfpathmoveto{\pgfqpoint{1.177134in}{1.411949in}}%
\pgfpathlineto{\pgfqpoint{1.177425in}{1.421285in}}%
\pgfpathlineto{\pgfqpoint{1.177643in}{1.430307in}}%
\pgfpathlineto{\pgfqpoint{1.177786in}{1.438982in}}%
\pgfpathlineto{\pgfqpoint{1.177854in}{1.447276in}}%
\pgfpathlineto{\pgfqpoint{1.142331in}{1.448740in}}%
\pgfpathlineto{\pgfqpoint{1.106727in}{1.448930in}}%
\pgfpathlineto{\pgfqpoint{1.071166in}{1.447846in}}%
\pgfpathlineto{\pgfqpoint{1.035776in}{1.445490in}}%
\pgfpathlineto{\pgfqpoint{1.000681in}{1.441871in}}%
\pgfpathlineto{\pgfqpoint{1.000818in}{1.433579in}}%
\pgfpathlineto{\pgfqpoint{1.001106in}{1.424911in}}%
\pgfpathlineto{\pgfqpoint{1.001546in}{1.415902in}}%
\pgfpathlineto{\pgfqpoint{1.002134in}{1.406587in}}%
\pgfpathlineto{\pgfqpoint{1.036798in}{1.410178in}}%
\pgfpathlineto{\pgfqpoint{1.071755in}{1.412515in}}%
\pgfpathlineto{\pgfqpoint{1.106879in}{1.413590in}}%
\pgfpathlineto{\pgfqpoint{1.142047in}{1.413401in}}%
\pgfpathlineto{\pgfqpoint{1.177134in}{1.411949in}}%
\pgfpathclose%
\pgfusepath{fill}%
\end{pgfscope}%
\begin{pgfscope}%
\pgfpathrectangle{\pgfqpoint{0.050000in}{0.050000in}}{\pgfqpoint{2.081932in}{2.081932in}}%
\pgfusepath{clip}%
\pgfsetbuttcap%
\pgfsetroundjoin%
\definecolor{currentfill}{rgb}{0.206756,0.371758,0.553117}%
\pgfsetfillcolor{currentfill}%
\pgfsetlinewidth{0.000000pt}%
\definecolor{currentstroke}{rgb}{0.000000,0.000000,0.000000}%
\pgfsetstrokecolor{currentstroke}%
\pgfsetdash{}{0pt}%
\pgfpathmoveto{\pgfqpoint{1.337116in}{1.336916in}}%
\pgfpathlineto{\pgfqpoint{1.339626in}{1.347001in}}%
\pgfpathlineto{\pgfqpoint{1.341878in}{1.356992in}}%
\pgfpathlineto{\pgfqpoint{1.343865in}{1.366851in}}%
\pgfpathlineto{\pgfqpoint{1.345579in}{1.376539in}}%
\pgfpathlineto{\pgfqpoint{1.347013in}{1.386020in}}%
\pgfpathlineto{\pgfqpoint{1.314188in}{1.393647in}}%
\pgfpathlineto{\pgfqpoint{1.280665in}{1.400072in}}%
\pgfpathlineto{\pgfqpoint{1.246567in}{1.405274in}}%
\pgfpathlineto{\pgfqpoint{1.212016in}{1.409237in}}%
\pgfpathlineto{\pgfqpoint{1.177134in}{1.411949in}}%
\pgfpathlineto{\pgfqpoint{1.176771in}{1.402337in}}%
\pgfpathlineto{\pgfqpoint{1.176336in}{1.392485in}}%
\pgfpathlineto{\pgfqpoint{1.175832in}{1.382432in}}%
\pgfpathlineto{\pgfqpoint{1.175261in}{1.372216in}}%
\pgfpathlineto{\pgfqpoint{1.174624in}{1.361876in}}%
\pgfpathlineto{\pgfqpoint{1.207994in}{1.359265in}}%
\pgfpathlineto{\pgfqpoint{1.241046in}{1.355450in}}%
\pgfpathlineto{\pgfqpoint{1.273662in}{1.350442in}}%
\pgfpathlineto{\pgfqpoint{1.305725in}{1.344258in}}%
\pgfpathlineto{\pgfqpoint{1.337116in}{1.336916in}}%
\pgfpathclose%
\pgfusepath{fill}%
\end{pgfscope}%
\begin{pgfscope}%
\pgfpathrectangle{\pgfqpoint{0.050000in}{0.050000in}}{\pgfqpoint{2.081932in}{2.081932in}}%
\pgfusepath{clip}%
\pgfsetbuttcap%
\pgfsetroundjoin%
\definecolor{currentfill}{rgb}{0.267968,0.223549,0.512008}%
\pgfsetfillcolor{currentfill}%
\pgfsetlinewidth{0.000000pt}%
\definecolor{currentstroke}{rgb}{0.000000,0.000000,0.000000}%
\pgfsetstrokecolor{currentstroke}%
\pgfsetdash{}{0pt}%
\pgfpathmoveto{\pgfqpoint{1.321092in}{1.286446in}}%
\pgfpathlineto{\pgfqpoint{1.324722in}{1.296416in}}%
\pgfpathlineto{\pgfqpoint{1.328151in}{1.306487in}}%
\pgfpathlineto{\pgfqpoint{1.331368in}{1.316620in}}%
\pgfpathlineto{\pgfqpoint{1.334361in}{1.326776in}}%
\pgfpathlineto{\pgfqpoint{1.337116in}{1.336916in}}%
\pgfpathlineto{\pgfqpoint{1.305725in}{1.344258in}}%
\pgfpathlineto{\pgfqpoint{1.273662in}{1.350442in}}%
\pgfpathlineto{\pgfqpoint{1.241046in}{1.355450in}}%
\pgfpathlineto{\pgfqpoint{1.207994in}{1.359265in}}%
\pgfpathlineto{\pgfqpoint{1.174624in}{1.361876in}}%
\pgfpathlineto{\pgfqpoint{1.173925in}{1.351453in}}%
\pgfpathlineto{\pgfqpoint{1.173166in}{1.340986in}}%
\pgfpathlineto{\pgfqpoint{1.172350in}{1.330515in}}%
\pgfpathlineto{\pgfqpoint{1.171479in}{1.320081in}}%
\pgfpathlineto{\pgfqpoint{1.170558in}{1.309723in}}%
\pgfpathlineto{\pgfqpoint{1.201480in}{1.307288in}}%
\pgfpathlineto{\pgfqpoint{1.232104in}{1.303729in}}%
\pgfpathlineto{\pgfqpoint{1.262321in}{1.299059in}}%
\pgfpathlineto{\pgfqpoint{1.292020in}{1.293292in}}%
\pgfpathlineto{\pgfqpoint{1.321092in}{1.286446in}}%
\pgfpathclose%
\pgfusepath{fill}%
\end{pgfscope}%
\begin{pgfscope}%
\pgfpathrectangle{\pgfqpoint{0.050000in}{0.050000in}}{\pgfqpoint{2.081932in}{2.081932in}}%
\pgfusepath{clip}%
\pgfsetbuttcap%
\pgfsetroundjoin%
\definecolor{currentfill}{rgb}{0.282327,0.094955,0.417331}%
\pgfsetfillcolor{currentfill}%
\pgfsetlinewidth{0.000000pt}%
\definecolor{currentstroke}{rgb}{0.000000,0.000000,0.000000}%
\pgfsetstrokecolor{currentstroke}%
\pgfsetdash{}{0pt}%
\pgfpathmoveto{\pgfqpoint{1.165322in}{1.260470in}}%
\pgfpathlineto{\pgfqpoint{1.166440in}{1.269854in}}%
\pgfpathlineto{\pgfqpoint{1.167527in}{1.279509in}}%
\pgfpathlineto{\pgfqpoint{1.168579in}{1.289398in}}%
\pgfpathlineto{\pgfqpoint{1.169590in}{1.299483in}}%
\pgfpathlineto{\pgfqpoint{1.170558in}{1.309723in}}%
\pgfpathlineto{\pgfqpoint{1.139452in}{1.311028in}}%
\pgfpathlineto{\pgfqpoint{1.108272in}{1.311198in}}%
\pgfpathlineto{\pgfqpoint{1.077132in}{1.310231in}}%
\pgfpathlineto{\pgfqpoint{1.046143in}{1.308133in}}%
\pgfpathlineto{\pgfqpoint{1.015417in}{1.304908in}}%
\pgfpathlineto{\pgfqpoint{1.017372in}{1.294752in}}%
\pgfpathlineto{\pgfqpoint{1.019416in}{1.284757in}}%
\pgfpathlineto{\pgfqpoint{1.021539in}{1.274961in}}%
\pgfpathlineto{\pgfqpoint{1.023734in}{1.265403in}}%
\pgfpathlineto{\pgfqpoint{1.025994in}{1.256119in}}%
\pgfpathlineto{\pgfqpoint{1.053584in}{1.259033in}}%
\pgfpathlineto{\pgfqpoint{1.081414in}{1.260929in}}%
\pgfpathlineto{\pgfqpoint{1.109382in}{1.261802in}}%
\pgfpathlineto{\pgfqpoint{1.137385in}{1.261649in}}%
\pgfpathlineto{\pgfqpoint{1.165322in}{1.260470in}}%
\pgfpathclose%
\pgfusepath{fill}%
\end{pgfscope}%
\begin{pgfscope}%
\pgfpathrectangle{\pgfqpoint{0.050000in}{0.050000in}}{\pgfqpoint{2.081932in}{2.081932in}}%
\pgfusepath{clip}%
\pgfsetbuttcap%
\pgfsetroundjoin%
\definecolor{currentfill}{rgb}{0.150476,0.504369,0.557430}%
\pgfsetfillcolor{currentfill}%
\pgfsetlinewidth{0.000000pt}%
\definecolor{currentstroke}{rgb}{0.000000,0.000000,0.000000}%
\pgfsetstrokecolor{currentstroke}%
\pgfsetdash{}{0pt}%
\pgfpathmoveto{\pgfqpoint{1.347013in}{1.386020in}}%
\pgfpathlineto{\pgfqpoint{1.348161in}{1.395255in}}%
\pgfpathlineto{\pgfqpoint{1.349019in}{1.404211in}}%
\pgfpathlineto{\pgfqpoint{1.349583in}{1.412851in}}%
\pgfpathlineto{\pgfqpoint{1.349851in}{1.421141in}}%
\pgfpathlineto{\pgfqpoint{1.316614in}{1.428829in}}%
\pgfpathlineto{\pgfqpoint{1.282673in}{1.435305in}}%
\pgfpathlineto{\pgfqpoint{1.248150in}{1.440548in}}%
\pgfpathlineto{\pgfqpoint{1.213168in}{1.444542in}}%
\pgfpathlineto{\pgfqpoint{1.177854in}{1.447276in}}%
\pgfpathlineto{\pgfqpoint{1.177786in}{1.438982in}}%
\pgfpathlineto{\pgfqpoint{1.177643in}{1.430307in}}%
\pgfpathlineto{\pgfqpoint{1.177425in}{1.421285in}}%
\pgfpathlineto{\pgfqpoint{1.177134in}{1.411949in}}%
\pgfpathlineto{\pgfqpoint{1.212016in}{1.409237in}}%
\pgfpathlineto{\pgfqpoint{1.246567in}{1.405274in}}%
\pgfpathlineto{\pgfqpoint{1.280665in}{1.400072in}}%
\pgfpathlineto{\pgfqpoint{1.314188in}{1.393647in}}%
\pgfpathlineto{\pgfqpoint{1.347013in}{1.386020in}}%
\pgfpathclose%
\pgfusepath{fill}%
\end{pgfscope}%
\begin{pgfscope}%
\pgfpathrectangle{\pgfqpoint{0.050000in}{0.050000in}}{\pgfqpoint{2.081932in}{2.081932in}}%
\pgfusepath{clip}%
\pgfsetbuttcap%
\pgfsetroundjoin%
\definecolor{currentfill}{rgb}{0.267968,0.223549,0.512008}%
\pgfsetfillcolor{currentfill}%
\pgfsetlinewidth{0.000000pt}%
\definecolor{currentstroke}{rgb}{0.000000,0.000000,0.000000}%
\pgfsetstrokecolor{currentstroke}%
\pgfsetdash{}{0pt}%
\pgfpathmoveto{\pgfqpoint{1.015417in}{1.304908in}}%
\pgfpathlineto{\pgfqpoint{1.013557in}{1.315186in}}%
\pgfpathlineto{\pgfqpoint{1.011799in}{1.325545in}}%
\pgfpathlineto{\pgfqpoint{1.010150in}{1.335946in}}%
\pgfpathlineto{\pgfqpoint{1.008617in}{1.346349in}}%
\pgfpathlineto{\pgfqpoint{1.007204in}{1.356714in}}%
\pgfpathlineto{\pgfqpoint{0.974444in}{1.352061in}}%
\pgfpathlineto{\pgfqpoint{0.942203in}{1.346225in}}%
\pgfpathlineto{\pgfqpoint{0.910596in}{1.339227in}}%
\pgfpathlineto{\pgfqpoint{0.879741in}{1.331087in}}%
\pgfpathlineto{\pgfqpoint{0.849750in}{1.321831in}}%
\pgfpathlineto{\pgfqpoint{0.853163in}{1.311863in}}%
\pgfpathlineto{\pgfqpoint{0.856869in}{1.301896in}}%
\pgfpathlineto{\pgfqpoint{0.860852in}{1.291968in}}%
\pgfpathlineto{\pgfqpoint{0.865099in}{1.282118in}}%
\pgfpathlineto{\pgfqpoint{0.869592in}{1.272384in}}%
\pgfpathlineto{\pgfqpoint{0.897355in}{1.281011in}}%
\pgfpathlineto{\pgfqpoint{0.925926in}{1.288601in}}%
\pgfpathlineto{\pgfqpoint{0.955199in}{1.295127in}}%
\pgfpathlineto{\pgfqpoint{0.985065in}{1.300568in}}%
\pgfpathlineto{\pgfqpoint{1.015417in}{1.304908in}}%
\pgfpathclose%
\pgfusepath{fill}%
\end{pgfscope}%
\begin{pgfscope}%
\pgfpathrectangle{\pgfqpoint{0.050000in}{0.050000in}}{\pgfqpoint{2.081932in}{2.081932in}}%
\pgfusepath{clip}%
\pgfsetbuttcap%
\pgfsetroundjoin%
\definecolor{currentfill}{rgb}{0.206756,0.371758,0.553117}%
\pgfsetfillcolor{currentfill}%
\pgfsetlinewidth{0.000000pt}%
\definecolor{currentstroke}{rgb}{0.000000,0.000000,0.000000}%
\pgfsetstrokecolor{currentstroke}%
\pgfsetdash{}{0pt}%
\pgfpathmoveto{\pgfqpoint{1.007204in}{1.356714in}}%
\pgfpathlineto{\pgfqpoint{1.005919in}{1.367001in}}%
\pgfpathlineto{\pgfqpoint{1.004765in}{1.377171in}}%
\pgfpathlineto{\pgfqpoint{1.003747in}{1.387184in}}%
\pgfpathlineto{\pgfqpoint{1.002869in}{1.397002in}}%
\pgfpathlineto{\pgfqpoint{1.002134in}{1.406587in}}%
\pgfpathlineto{\pgfqpoint{0.967886in}{1.401753in}}%
\pgfpathlineto{\pgfqpoint{0.934177in}{1.395691in}}%
\pgfpathlineto{\pgfqpoint{0.901129in}{1.388421in}}%
\pgfpathlineto{\pgfqpoint{0.868861in}{1.379963in}}%
\pgfpathlineto{\pgfqpoint{0.837493in}{1.370346in}}%
\pgfpathlineto{\pgfqpoint{0.839269in}{1.360946in}}%
\pgfpathlineto{\pgfqpoint{0.841392in}{1.351357in}}%
\pgfpathlineto{\pgfqpoint{0.843853in}{1.341616in}}%
\pgfpathlineto{\pgfqpoint{0.846643in}{1.331762in}}%
\pgfpathlineto{\pgfqpoint{0.849750in}{1.321831in}}%
\pgfpathlineto{\pgfqpoint{0.879741in}{1.331087in}}%
\pgfpathlineto{\pgfqpoint{0.910596in}{1.339227in}}%
\pgfpathlineto{\pgfqpoint{0.942203in}{1.346225in}}%
\pgfpathlineto{\pgfqpoint{0.974444in}{1.352061in}}%
\pgfpathlineto{\pgfqpoint{1.007204in}{1.356714in}}%
\pgfpathclose%
\pgfusepath{fill}%
\end{pgfscope}%
\begin{pgfscope}%
\pgfpathrectangle{\pgfqpoint{0.050000in}{0.050000in}}{\pgfqpoint{2.081932in}{2.081932in}}%
\pgfusepath{clip}%
\pgfsetbuttcap%
\pgfsetroundjoin%
\definecolor{currentfill}{rgb}{0.282327,0.094955,0.417331}%
\pgfsetfillcolor{currentfill}%
\pgfsetlinewidth{0.000000pt}%
\definecolor{currentstroke}{rgb}{0.000000,0.000000,0.000000}%
\pgfsetstrokecolor{currentstroke}%
\pgfsetdash{}{0pt}%
\pgfpathmoveto{\pgfqpoint{1.300463in}{1.239442in}}%
\pgfpathlineto{\pgfqpoint{1.304869in}{1.248340in}}%
\pgfpathlineto{\pgfqpoint{1.309150in}{1.257525in}}%
\pgfpathlineto{\pgfqpoint{1.313292in}{1.266963in}}%
\pgfpathlineto{\pgfqpoint{1.317278in}{1.276616in}}%
\pgfpathlineto{\pgfqpoint{1.321092in}{1.286446in}}%
\pgfpathlineto{\pgfqpoint{1.292020in}{1.293292in}}%
\pgfpathlineto{\pgfqpoint{1.262321in}{1.299059in}}%
\pgfpathlineto{\pgfqpoint{1.232104in}{1.303729in}}%
\pgfpathlineto{\pgfqpoint{1.201480in}{1.307288in}}%
\pgfpathlineto{\pgfqpoint{1.170558in}{1.309723in}}%
\pgfpathlineto{\pgfqpoint{1.169590in}{1.299483in}}%
\pgfpathlineto{\pgfqpoint{1.168579in}{1.289398in}}%
\pgfpathlineto{\pgfqpoint{1.167527in}{1.279509in}}%
\pgfpathlineto{\pgfqpoint{1.166440in}{1.269854in}}%
\pgfpathlineto{\pgfqpoint{1.165322in}{1.260470in}}%
\pgfpathlineto{\pgfqpoint{1.193090in}{1.258270in}}%
\pgfpathlineto{\pgfqpoint{1.220589in}{1.255054in}}%
\pgfpathlineto{\pgfqpoint{1.247717in}{1.250835in}}%
\pgfpathlineto{\pgfqpoint{1.274375in}{1.245625in}}%
\pgfpathlineto{\pgfqpoint{1.300463in}{1.239442in}}%
\pgfpathclose%
\pgfusepath{fill}%
\end{pgfscope}%
\begin{pgfscope}%
\pgfpathrectangle{\pgfqpoint{0.050000in}{0.050000in}}{\pgfqpoint{2.081932in}{2.081932in}}%
\pgfusepath{clip}%
\pgfsetbuttcap%
\pgfsetroundjoin%
\definecolor{currentfill}{rgb}{0.124780,0.640461,0.527068}%
\pgfsetfillcolor{currentfill}%
\pgfsetlinewidth{0.000000pt}%
\definecolor{currentstroke}{rgb}{0.000000,0.000000,0.000000}%
\pgfsetstrokecolor{currentstroke}%
\pgfsetdash{}{0pt}%
\pgfpathmoveto{\pgfqpoint{1.177854in}{1.447276in}}%
\pgfpathlineto{\pgfqpoint{1.177846in}{1.455154in}}%
\pgfpathlineto{\pgfqpoint{1.177763in}{1.462587in}}%
\pgfpathlineto{\pgfqpoint{1.177604in}{1.469545in}}%
\pgfpathlineto{\pgfqpoint{1.177370in}{1.476000in}}%
\pgfpathlineto{\pgfqpoint{1.177062in}{1.481925in}}%
\pgfpathlineto{\pgfqpoint{1.142019in}{1.483363in}}%
\pgfpathlineto{\pgfqpoint{1.106895in}{1.483550in}}%
\pgfpathlineto{\pgfqpoint{1.071815in}{1.482485in}}%
\pgfpathlineto{\pgfqpoint{1.036902in}{1.480172in}}%
\pgfpathlineto{\pgfqpoint{1.002282in}{1.476617in}}%
\pgfpathlineto{\pgfqpoint{1.001659in}{1.470660in}}%
\pgfpathlineto{\pgfqpoint{1.001186in}{1.464179in}}%
\pgfpathlineto{\pgfqpoint{1.000865in}{1.457202in}}%
\pgfpathlineto{\pgfqpoint{1.000697in}{1.449756in}}%
\pgfpathlineto{\pgfqpoint{1.000681in}{1.441871in}}%
\pgfpathlineto{\pgfqpoint{1.035776in}{1.445490in}}%
\pgfpathlineto{\pgfqpoint{1.071166in}{1.447846in}}%
\pgfpathlineto{\pgfqpoint{1.106727in}{1.448930in}}%
\pgfpathlineto{\pgfqpoint{1.142331in}{1.448740in}}%
\pgfpathlineto{\pgfqpoint{1.177854in}{1.447276in}}%
\pgfpathclose%
\pgfusepath{fill}%
\end{pgfscope}%
\begin{pgfscope}%
\pgfpathrectangle{\pgfqpoint{0.050000in}{0.050000in}}{\pgfqpoint{2.081932in}{2.081932in}}%
\pgfusepath{clip}%
\pgfsetbuttcap%
\pgfsetroundjoin%
\definecolor{currentfill}{rgb}{0.150476,0.504369,0.557430}%
\pgfsetfillcolor{currentfill}%
\pgfsetlinewidth{0.000000pt}%
\definecolor{currentstroke}{rgb}{0.000000,0.000000,0.000000}%
\pgfsetstrokecolor{currentstroke}%
\pgfsetdash{}{0pt}%
\pgfpathmoveto{\pgfqpoint{1.002134in}{1.406587in}}%
\pgfpathlineto{\pgfqpoint{1.001546in}{1.415902in}}%
\pgfpathlineto{\pgfqpoint{1.001106in}{1.424911in}}%
\pgfpathlineto{\pgfqpoint{1.000818in}{1.433579in}}%
\pgfpathlineto{\pgfqpoint{1.000681in}{1.441871in}}%
\pgfpathlineto{\pgfqpoint{0.966006in}{1.436999in}}%
\pgfpathlineto{\pgfqpoint{0.931877in}{1.430890in}}%
\pgfpathlineto{\pgfqpoint{0.898415in}{1.423561in}}%
\pgfpathlineto{\pgfqpoint{0.865742in}{1.415036in}}%
\pgfpathlineto{\pgfqpoint{0.833978in}{1.405342in}}%
\pgfpathlineto{\pgfqpoint{0.834309in}{1.397054in}}%
\pgfpathlineto{\pgfqpoint{0.835008in}{1.388435in}}%
\pgfpathlineto{\pgfqpoint{0.836071in}{1.379521in}}%
\pgfpathlineto{\pgfqpoint{0.837493in}{1.370346in}}%
\pgfpathlineto{\pgfqpoint{0.868861in}{1.379963in}}%
\pgfpathlineto{\pgfqpoint{0.901129in}{1.388421in}}%
\pgfpathlineto{\pgfqpoint{0.934177in}{1.395691in}}%
\pgfpathlineto{\pgfqpoint{0.967886in}{1.401753in}}%
\pgfpathlineto{\pgfqpoint{1.002134in}{1.406587in}}%
\pgfpathclose%
\pgfusepath{fill}%
\end{pgfscope}%
\begin{pgfscope}%
\pgfpathrectangle{\pgfqpoint{0.050000in}{0.050000in}}{\pgfqpoint{2.081932in}{2.081932in}}%
\pgfusepath{clip}%
\pgfsetbuttcap%
\pgfsetroundjoin%
\definecolor{currentfill}{rgb}{0.282327,0.094955,0.417331}%
\pgfsetfillcolor{currentfill}%
\pgfsetlinewidth{0.000000pt}%
\definecolor{currentstroke}{rgb}{0.000000,0.000000,0.000000}%
\pgfsetstrokecolor{currentstroke}%
\pgfsetdash{}{0pt}%
\pgfpathmoveto{\pgfqpoint{1.025994in}{1.256119in}}%
\pgfpathlineto{\pgfqpoint{1.023734in}{1.265403in}}%
\pgfpathlineto{\pgfqpoint{1.021539in}{1.274961in}}%
\pgfpathlineto{\pgfqpoint{1.019416in}{1.284757in}}%
\pgfpathlineto{\pgfqpoint{1.017372in}{1.294752in}}%
\pgfpathlineto{\pgfqpoint{1.015417in}{1.304908in}}%
\pgfpathlineto{\pgfqpoint{0.985065in}{1.300568in}}%
\pgfpathlineto{\pgfqpoint{0.955199in}{1.295127in}}%
\pgfpathlineto{\pgfqpoint{0.925926in}{1.288601in}}%
\pgfpathlineto{\pgfqpoint{0.897355in}{1.281011in}}%
\pgfpathlineto{\pgfqpoint{0.869592in}{1.272384in}}%
\pgfpathlineto{\pgfqpoint{0.874315in}{1.262803in}}%
\pgfpathlineto{\pgfqpoint{0.879249in}{1.253412in}}%
\pgfpathlineto{\pgfqpoint{0.884376in}{1.244248in}}%
\pgfpathlineto{\pgfqpoint{0.889676in}{1.235348in}}%
\pgfpathlineto{\pgfqpoint{0.895129in}{1.226744in}}%
\pgfpathlineto{\pgfqpoint{0.920029in}{1.234534in}}%
\pgfpathlineto{\pgfqpoint{0.945662in}{1.241388in}}%
\pgfpathlineto{\pgfqpoint{0.971933in}{1.247282in}}%
\pgfpathlineto{\pgfqpoint{0.998743in}{1.252198in}}%
\pgfpathlineto{\pgfqpoint{1.025994in}{1.256119in}}%
\pgfpathclose%
\pgfusepath{fill}%
\end{pgfscope}%
\begin{pgfscope}%
\pgfpathrectangle{\pgfqpoint{0.050000in}{0.050000in}}{\pgfqpoint{2.081932in}{2.081932in}}%
\pgfusepath{clip}%
\pgfsetbuttcap%
\pgfsetroundjoin%
\definecolor{currentfill}{rgb}{0.268510,0.009605,0.335427}%
\pgfsetfillcolor{currentfill}%
\pgfsetlinewidth{0.000000pt}%
\definecolor{currentstroke}{rgb}{0.000000,0.000000,0.000000}%
\pgfsetstrokecolor{currentstroke}%
\pgfsetdash{}{0pt}%
\pgfpathmoveto{\pgfqpoint{1.159410in}{1.218879in}}%
\pgfpathlineto{\pgfqpoint{1.160620in}{1.226374in}}%
\pgfpathlineto{\pgfqpoint{1.161820in}{1.234313in}}%
\pgfpathlineto{\pgfqpoint{1.163007in}{1.242664in}}%
\pgfpathlineto{\pgfqpoint{1.164176in}{1.251395in}}%
\pgfpathlineto{\pgfqpoint{1.165322in}{1.260470in}}%
\pgfpathlineto{\pgfqpoint{1.137385in}{1.261649in}}%
\pgfpathlineto{\pgfqpoint{1.109382in}{1.261802in}}%
\pgfpathlineto{\pgfqpoint{1.081414in}{1.260929in}}%
\pgfpathlineto{\pgfqpoint{1.053584in}{1.259033in}}%
\pgfpathlineto{\pgfqpoint{1.025994in}{1.256119in}}%
\pgfpathlineto{\pgfqpoint{1.028308in}{1.247148in}}%
\pgfpathlineto{\pgfqpoint{1.030668in}{1.238523in}}%
\pgfpathlineto{\pgfqpoint{1.033065in}{1.230279in}}%
\pgfpathlineto{\pgfqpoint{1.035489in}{1.222450in}}%
\pgfpathlineto{\pgfqpoint{1.037932in}{1.215066in}}%
\pgfpathlineto{\pgfqpoint{1.061985in}{1.217619in}}%
\pgfpathlineto{\pgfqpoint{1.086249in}{1.219281in}}%
\pgfpathlineto{\pgfqpoint{1.110634in}{1.220046in}}%
\pgfpathlineto{\pgfqpoint{1.135051in}{1.219912in}}%
\pgfpathlineto{\pgfqpoint{1.159410in}{1.218879in}}%
\pgfpathclose%
\pgfusepath{fill}%
\end{pgfscope}%
\begin{pgfscope}%
\pgfpathrectangle{\pgfqpoint{0.050000in}{0.050000in}}{\pgfqpoint{2.081932in}{2.081932in}}%
\pgfusepath{clip}%
\pgfsetbuttcap%
\pgfsetroundjoin%
\definecolor{currentfill}{rgb}{0.124780,0.640461,0.527068}%
\pgfsetfillcolor{currentfill}%
\pgfsetlinewidth{0.000000pt}%
\definecolor{currentstroke}{rgb}{0.000000,0.000000,0.000000}%
\pgfsetstrokecolor{currentstroke}%
\pgfsetdash{}{0pt}%
\pgfpathmoveto{\pgfqpoint{1.349851in}{1.421141in}}%
\pgfpathlineto{\pgfqpoint{1.349820in}{1.429049in}}%
\pgfpathlineto{\pgfqpoint{1.349491in}{1.436544in}}%
\pgfpathlineto{\pgfqpoint{1.348864in}{1.443596in}}%
\pgfpathlineto{\pgfqpoint{1.347941in}{1.450176in}}%
\pgfpathlineto{\pgfqpoint{1.346726in}{1.456258in}}%
\pgfpathlineto{\pgfqpoint{1.313942in}{1.463808in}}%
\pgfpathlineto{\pgfqpoint{1.280462in}{1.470168in}}%
\pgfpathlineto{\pgfqpoint{1.246407in}{1.475317in}}%
\pgfpathlineto{\pgfqpoint{1.211899in}{1.479240in}}%
\pgfpathlineto{\pgfqpoint{1.177062in}{1.481925in}}%
\pgfpathlineto{\pgfqpoint{1.177370in}{1.476000in}}%
\pgfpathlineto{\pgfqpoint{1.177604in}{1.469545in}}%
\pgfpathlineto{\pgfqpoint{1.177763in}{1.462587in}}%
\pgfpathlineto{\pgfqpoint{1.177846in}{1.455154in}}%
\pgfpathlineto{\pgfqpoint{1.177854in}{1.447276in}}%
\pgfpathlineto{\pgfqpoint{1.213168in}{1.444542in}}%
\pgfpathlineto{\pgfqpoint{1.248150in}{1.440548in}}%
\pgfpathlineto{\pgfqpoint{1.282673in}{1.435305in}}%
\pgfpathlineto{\pgfqpoint{1.316614in}{1.428829in}}%
\pgfpathlineto{\pgfqpoint{1.349851in}{1.421141in}}%
\pgfpathclose%
\pgfusepath{fill}%
\end{pgfscope}%
\begin{pgfscope}%
\pgfpathrectangle{\pgfqpoint{0.050000in}{0.050000in}}{\pgfqpoint{2.081932in}{2.081932in}}%
\pgfusepath{clip}%
\pgfsetbuttcap%
\pgfsetroundjoin%
\definecolor{currentfill}{rgb}{0.268510,0.009605,0.335427}%
\pgfsetfillcolor{currentfill}%
\pgfsetlinewidth{0.000000pt}%
\definecolor{currentstroke}{rgb}{0.000000,0.000000,0.000000}%
\pgfsetstrokecolor{currentstroke}%
\pgfsetdash{}{0pt}%
\pgfpathmoveto{\pgfqpoint{1.277190in}{1.200455in}}%
\pgfpathlineto{\pgfqpoint{1.281951in}{1.207412in}}%
\pgfpathlineto{\pgfqpoint{1.286678in}{1.214820in}}%
\pgfpathlineto{\pgfqpoint{1.291350in}{1.222649in}}%
\pgfpathlineto{\pgfqpoint{1.295952in}{1.230867in}}%
\pgfpathlineto{\pgfqpoint{1.300463in}{1.239442in}}%
\pgfpathlineto{\pgfqpoint{1.274375in}{1.245625in}}%
\pgfpathlineto{\pgfqpoint{1.247717in}{1.250835in}}%
\pgfpathlineto{\pgfqpoint{1.220589in}{1.255054in}}%
\pgfpathlineto{\pgfqpoint{1.193090in}{1.258270in}}%
\pgfpathlineto{\pgfqpoint{1.165322in}{1.260470in}}%
\pgfpathlineto{\pgfqpoint{1.164176in}{1.251395in}}%
\pgfpathlineto{\pgfqpoint{1.163007in}{1.242664in}}%
\pgfpathlineto{\pgfqpoint{1.161820in}{1.234313in}}%
\pgfpathlineto{\pgfqpoint{1.160620in}{1.226374in}}%
\pgfpathlineto{\pgfqpoint{1.159410in}{1.218879in}}%
\pgfpathlineto{\pgfqpoint{1.183620in}{1.216950in}}%
\pgfpathlineto{\pgfqpoint{1.207591in}{1.214132in}}%
\pgfpathlineto{\pgfqpoint{1.231235in}{1.210435in}}%
\pgfpathlineto{\pgfqpoint{1.254464in}{1.205871in}}%
\pgfpathlineto{\pgfqpoint{1.277190in}{1.200455in}}%
\pgfpathclose%
\pgfusepath{fill}%
\end{pgfscope}%
\begin{pgfscope}%
\pgfpathrectangle{\pgfqpoint{0.050000in}{0.050000in}}{\pgfqpoint{2.081932in}{2.081932in}}%
\pgfusepath{clip}%
\pgfsetbuttcap%
\pgfsetroundjoin%
\definecolor{currentfill}{rgb}{0.267968,0.223549,0.512008}%
\pgfsetfillcolor{currentfill}%
\pgfsetlinewidth{0.000000pt}%
\definecolor{currentstroke}{rgb}{0.000000,0.000000,0.000000}%
\pgfsetstrokecolor{currentstroke}%
\pgfsetdash{}{0pt}%
\pgfpathmoveto{\pgfqpoint{1.453365in}{1.236942in}}%
\pgfpathlineto{\pgfqpoint{1.459410in}{1.246079in}}%
\pgfpathlineto{\pgfqpoint{1.465123in}{1.255370in}}%
\pgfpathlineto{\pgfqpoint{1.470483in}{1.264776in}}%
\pgfpathlineto{\pgfqpoint{1.475470in}{1.274264in}}%
\pgfpathlineto{\pgfqpoint{1.480063in}{1.283794in}}%
\pgfpathlineto{\pgfqpoint{1.453708in}{1.296498in}}%
\pgfpathlineto{\pgfqpoint{1.426129in}{1.308198in}}%
\pgfpathlineto{\pgfqpoint{1.397431in}{1.318857in}}%
\pgfpathlineto{\pgfqpoint{1.367723in}{1.328440in}}%
\pgfpathlineto{\pgfqpoint{1.337116in}{1.336916in}}%
\pgfpathlineto{\pgfqpoint{1.334361in}{1.326776in}}%
\pgfpathlineto{\pgfqpoint{1.331368in}{1.316620in}}%
\pgfpathlineto{\pgfqpoint{1.328151in}{1.306487in}}%
\pgfpathlineto{\pgfqpoint{1.324722in}{1.296416in}}%
\pgfpathlineto{\pgfqpoint{1.321092in}{1.286446in}}%
\pgfpathlineto{\pgfqpoint{1.349431in}{1.278544in}}%
\pgfpathlineto{\pgfqpoint{1.376929in}{1.269612in}}%
\pgfpathlineto{\pgfqpoint{1.403485in}{1.259678in}}%
\pgfpathlineto{\pgfqpoint{1.428996in}{1.248776in}}%
\pgfpathlineto{\pgfqpoint{1.453365in}{1.236942in}}%
\pgfpathclose%
\pgfusepath{fill}%
\end{pgfscope}%
\begin{pgfscope}%
\pgfpathrectangle{\pgfqpoint{0.050000in}{0.050000in}}{\pgfqpoint{2.081932in}{2.081932in}}%
\pgfusepath{clip}%
\pgfsetbuttcap%
\pgfsetroundjoin%
\definecolor{currentfill}{rgb}{0.124780,0.640461,0.527068}%
\pgfsetfillcolor{currentfill}%
\pgfsetlinewidth{0.000000pt}%
\definecolor{currentstroke}{rgb}{0.000000,0.000000,0.000000}%
\pgfsetstrokecolor{currentstroke}%
\pgfsetdash{}{0pt}%
\pgfpathmoveto{\pgfqpoint{1.000681in}{1.441871in}}%
\pgfpathlineto{\pgfqpoint{1.000697in}{1.449756in}}%
\pgfpathlineto{\pgfqpoint{1.000865in}{1.457202in}}%
\pgfpathlineto{\pgfqpoint{1.001186in}{1.464179in}}%
\pgfpathlineto{\pgfqpoint{1.001659in}{1.470660in}}%
\pgfpathlineto{\pgfqpoint{1.002282in}{1.476617in}}%
\pgfpathlineto{\pgfqpoint{0.968077in}{1.471832in}}%
\pgfpathlineto{\pgfqpoint{0.934411in}{1.465832in}}%
\pgfpathlineto{\pgfqpoint{0.901405in}{1.458635in}}%
\pgfpathlineto{\pgfqpoint{0.869178in}{1.450263in}}%
\pgfpathlineto{\pgfqpoint{0.837850in}{1.440744in}}%
\pgfpathlineto{\pgfqpoint{0.836344in}{1.434567in}}%
\pgfpathlineto{\pgfqpoint{0.835201in}{1.427910in}}%
\pgfpathlineto{\pgfqpoint{0.834424in}{1.420801in}}%
\pgfpathlineto{\pgfqpoint{0.834017in}{1.413269in}}%
\pgfpathlineto{\pgfqpoint{0.833978in}{1.405342in}}%
\pgfpathlineto{\pgfqpoint{0.865742in}{1.415036in}}%
\pgfpathlineto{\pgfqpoint{0.898415in}{1.423561in}}%
\pgfpathlineto{\pgfqpoint{0.931877in}{1.430890in}}%
\pgfpathlineto{\pgfqpoint{0.966006in}{1.436999in}}%
\pgfpathlineto{\pgfqpoint{1.000681in}{1.441871in}}%
\pgfpathclose%
\pgfusepath{fill}%
\end{pgfscope}%
\begin{pgfscope}%
\pgfpathrectangle{\pgfqpoint{0.050000in}{0.050000in}}{\pgfqpoint{2.081932in}{2.081932in}}%
\pgfusepath{clip}%
\pgfsetbuttcap%
\pgfsetroundjoin%
\definecolor{currentfill}{rgb}{0.206756,0.371758,0.553117}%
\pgfsetfillcolor{currentfill}%
\pgfsetlinewidth{0.000000pt}%
\definecolor{currentstroke}{rgb}{0.000000,0.000000,0.000000}%
\pgfsetstrokecolor{currentstroke}%
\pgfsetdash{}{0pt}%
\pgfpathmoveto{\pgfqpoint{1.480063in}{1.283794in}}%
\pgfpathlineto{\pgfqpoint{1.484246in}{1.293332in}}%
\pgfpathlineto{\pgfqpoint{1.488001in}{1.302839in}}%
\pgfpathlineto{\pgfqpoint{1.491315in}{1.312279in}}%
\pgfpathlineto{\pgfqpoint{1.494173in}{1.321616in}}%
\pgfpathlineto{\pgfqpoint{1.496565in}{1.330812in}}%
\pgfpathlineto{\pgfqpoint{1.468979in}{1.344018in}}%
\pgfpathlineto{\pgfqpoint{1.440119in}{1.356179in}}%
\pgfpathlineto{\pgfqpoint{1.410096in}{1.367256in}}%
\pgfpathlineto{\pgfqpoint{1.379022in}{1.377213in}}%
\pgfpathlineto{\pgfqpoint{1.347013in}{1.386020in}}%
\pgfpathlineto{\pgfqpoint{1.345579in}{1.376539in}}%
\pgfpathlineto{\pgfqpoint{1.343865in}{1.366851in}}%
\pgfpathlineto{\pgfqpoint{1.341878in}{1.356992in}}%
\pgfpathlineto{\pgfqpoint{1.339626in}{1.347001in}}%
\pgfpathlineto{\pgfqpoint{1.337116in}{1.336916in}}%
\pgfpathlineto{\pgfqpoint{1.367723in}{1.328440in}}%
\pgfpathlineto{\pgfqpoint{1.397431in}{1.318857in}}%
\pgfpathlineto{\pgfqpoint{1.426129in}{1.308198in}}%
\pgfpathlineto{\pgfqpoint{1.453708in}{1.296498in}}%
\pgfpathlineto{\pgfqpoint{1.480063in}{1.283794in}}%
\pgfpathclose%
\pgfusepath{fill}%
\end{pgfscope}%
\begin{pgfscope}%
\pgfpathrectangle{\pgfqpoint{0.050000in}{0.050000in}}{\pgfqpoint{2.081932in}{2.081932in}}%
\pgfusepath{clip}%
\pgfsetbuttcap%
\pgfsetroundjoin%
\definecolor{currentfill}{rgb}{0.268510,0.009605,0.335427}%
\pgfsetfillcolor{currentfill}%
\pgfsetlinewidth{0.000000pt}%
\definecolor{currentstroke}{rgb}{0.000000,0.000000,0.000000}%
\pgfsetstrokecolor{currentstroke}%
\pgfsetdash{}{0pt}%
\pgfpathmoveto{\pgfqpoint{1.037932in}{1.215066in}}%
\pgfpathlineto{\pgfqpoint{1.035489in}{1.222450in}}%
\pgfpathlineto{\pgfqpoint{1.033065in}{1.230279in}}%
\pgfpathlineto{\pgfqpoint{1.030668in}{1.238523in}}%
\pgfpathlineto{\pgfqpoint{1.028308in}{1.247148in}}%
\pgfpathlineto{\pgfqpoint{1.025994in}{1.256119in}}%
\pgfpathlineto{\pgfqpoint{0.998743in}{1.252198in}}%
\pgfpathlineto{\pgfqpoint{0.971933in}{1.247282in}}%
\pgfpathlineto{\pgfqpoint{0.945662in}{1.241388in}}%
\pgfpathlineto{\pgfqpoint{0.920029in}{1.234534in}}%
\pgfpathlineto{\pgfqpoint{0.895129in}{1.226744in}}%
\pgfpathlineto{\pgfqpoint{0.900713in}{1.218473in}}%
\pgfpathlineto{\pgfqpoint{0.906407in}{1.210565in}}%
\pgfpathlineto{\pgfqpoint{0.912189in}{1.203054in}}%
\pgfpathlineto{\pgfqpoint{0.918037in}{1.195968in}}%
\pgfpathlineto{\pgfqpoint{0.923928in}{1.189336in}}%
\pgfpathlineto{\pgfqpoint{0.945605in}{1.196157in}}%
\pgfpathlineto{\pgfqpoint{0.967930in}{1.202160in}}%
\pgfpathlineto{\pgfqpoint{0.990817in}{1.207323in}}%
\pgfpathlineto{\pgfqpoint{1.014180in}{1.211630in}}%
\pgfpathlineto{\pgfqpoint{1.037932in}{1.215066in}}%
\pgfpathclose%
\pgfusepath{fill}%
\end{pgfscope}%
\begin{pgfscope}%
\pgfpathrectangle{\pgfqpoint{0.050000in}{0.050000in}}{\pgfqpoint{2.081932in}{2.081932in}}%
\pgfusepath{clip}%
\pgfsetbuttcap%
\pgfsetroundjoin%
\definecolor{currentfill}{rgb}{0.282327,0.094955,0.417331}%
\pgfsetfillcolor{currentfill}%
\pgfsetlinewidth{0.000000pt}%
\definecolor{currentstroke}{rgb}{0.000000,0.000000,0.000000}%
\pgfsetstrokecolor{currentstroke}%
\pgfsetdash{}{0pt}%
\pgfpathmoveto{\pgfqpoint{1.419034in}{1.194763in}}%
\pgfpathlineto{\pgfqpoint{1.426362in}{1.202619in}}%
\pgfpathlineto{\pgfqpoint{1.433486in}{1.210798in}}%
\pgfpathlineto{\pgfqpoint{1.440379in}{1.219267in}}%
\pgfpathlineto{\pgfqpoint{1.447014in}{1.227993in}}%
\pgfpathlineto{\pgfqpoint{1.453365in}{1.236942in}}%
\pgfpathlineto{\pgfqpoint{1.428996in}{1.248776in}}%
\pgfpathlineto{\pgfqpoint{1.403485in}{1.259678in}}%
\pgfpathlineto{\pgfqpoint{1.376929in}{1.269612in}}%
\pgfpathlineto{\pgfqpoint{1.349431in}{1.278544in}}%
\pgfpathlineto{\pgfqpoint{1.321092in}{1.286446in}}%
\pgfpathlineto{\pgfqpoint{1.317278in}{1.276616in}}%
\pgfpathlineto{\pgfqpoint{1.313292in}{1.266963in}}%
\pgfpathlineto{\pgfqpoint{1.309150in}{1.257525in}}%
\pgfpathlineto{\pgfqpoint{1.304869in}{1.248340in}}%
\pgfpathlineto{\pgfqpoint{1.300463in}{1.239442in}}%
\pgfpathlineto{\pgfqpoint{1.325886in}{1.232306in}}%
\pgfpathlineto{\pgfqpoint{1.350546in}{1.224242in}}%
\pgfpathlineto{\pgfqpoint{1.374352in}{1.215276in}}%
\pgfpathlineto{\pgfqpoint{1.397211in}{1.205438in}}%
\pgfpathlineto{\pgfqpoint{1.419034in}{1.194763in}}%
\pgfpathclose%
\pgfusepath{fill}%
\end{pgfscope}%
\begin{pgfscope}%
\pgfpathrectangle{\pgfqpoint{0.050000in}{0.050000in}}{\pgfqpoint{2.081932in}{2.081932in}}%
\pgfusepath{clip}%
\pgfsetbuttcap%
\pgfsetroundjoin%
\definecolor{currentfill}{rgb}{0.327796,0.773980,0.406640}%
\pgfsetfillcolor{currentfill}%
\pgfsetlinewidth{0.000000pt}%
\definecolor{currentstroke}{rgb}{0.000000,0.000000,0.000000}%
\pgfsetstrokecolor{currentstroke}%
\pgfsetdash{}{0pt}%
\pgfpathmoveto{\pgfqpoint{1.177062in}{1.481925in}}%
\pgfpathlineto{\pgfqpoint{1.176680in}{1.487296in}}%
\pgfpathlineto{\pgfqpoint{1.176227in}{1.492092in}}%
\pgfpathlineto{\pgfqpoint{1.175703in}{1.496292in}}%
\pgfpathlineto{\pgfqpoint{1.175111in}{1.499877in}}%
\pgfpathlineto{\pgfqpoint{1.174453in}{1.502832in}}%
\pgfpathlineto{\pgfqpoint{1.140989in}{1.504202in}}%
\pgfpathlineto{\pgfqpoint{1.107448in}{1.504379in}}%
\pgfpathlineto{\pgfqpoint{1.073948in}{1.503365in}}%
\pgfpathlineto{\pgfqpoint{1.040610in}{1.501162in}}%
\pgfpathlineto{\pgfqpoint{1.007552in}{1.497777in}}%
\pgfpathlineto{\pgfqpoint{1.006222in}{1.494760in}}%
\pgfpathlineto{\pgfqpoint{1.005026in}{1.491118in}}%
\pgfpathlineto{\pgfqpoint{1.003968in}{1.486868in}}%
\pgfpathlineto{\pgfqpoint{1.003053in}{1.482027in}}%
\pgfpathlineto{\pgfqpoint{1.002282in}{1.476617in}}%
\pgfpathlineto{\pgfqpoint{1.036902in}{1.480172in}}%
\pgfpathlineto{\pgfqpoint{1.071815in}{1.482485in}}%
\pgfpathlineto{\pgfqpoint{1.106895in}{1.483550in}}%
\pgfpathlineto{\pgfqpoint{1.142019in}{1.483363in}}%
\pgfpathlineto{\pgfqpoint{1.177062in}{1.481925in}}%
\pgfpathclose%
\pgfusepath{fill}%
\end{pgfscope}%
\begin{pgfscope}%
\pgfpathrectangle{\pgfqpoint{0.050000in}{0.050000in}}{\pgfqpoint{2.081932in}{2.081932in}}%
\pgfusepath{clip}%
\pgfsetbuttcap%
\pgfsetroundjoin%
\definecolor{currentfill}{rgb}{0.150476,0.504369,0.557430}%
\pgfsetfillcolor{currentfill}%
\pgfsetlinewidth{0.000000pt}%
\definecolor{currentstroke}{rgb}{0.000000,0.000000,0.000000}%
\pgfsetstrokecolor{currentstroke}%
\pgfsetdash{}{0pt}%
\pgfpathmoveto{\pgfqpoint{1.496565in}{1.330812in}}%
\pgfpathlineto{\pgfqpoint{1.498480in}{1.339832in}}%
\pgfpathlineto{\pgfqpoint{1.499911in}{1.348642in}}%
\pgfpathlineto{\pgfqpoint{1.500852in}{1.357205in}}%
\pgfpathlineto{\pgfqpoint{1.501298in}{1.365489in}}%
\pgfpathlineto{\pgfqpoint{1.473359in}{1.378802in}}%
\pgfpathlineto{\pgfqpoint{1.444132in}{1.391061in}}%
\pgfpathlineto{\pgfqpoint{1.413728in}{1.402227in}}%
\pgfpathlineto{\pgfqpoint{1.382262in}{1.412264in}}%
\pgfpathlineto{\pgfqpoint{1.349851in}{1.421141in}}%
\pgfpathlineto{\pgfqpoint{1.349583in}{1.412851in}}%
\pgfpathlineto{\pgfqpoint{1.349019in}{1.404211in}}%
\pgfpathlineto{\pgfqpoint{1.348161in}{1.395255in}}%
\pgfpathlineto{\pgfqpoint{1.347013in}{1.386020in}}%
\pgfpathlineto{\pgfqpoint{1.379022in}{1.377213in}}%
\pgfpathlineto{\pgfqpoint{1.410096in}{1.367256in}}%
\pgfpathlineto{\pgfqpoint{1.440119in}{1.356179in}}%
\pgfpathlineto{\pgfqpoint{1.468979in}{1.344018in}}%
\pgfpathlineto{\pgfqpoint{1.496565in}{1.330812in}}%
\pgfpathclose%
\pgfusepath{fill}%
\end{pgfscope}%
\begin{pgfscope}%
\pgfpathrectangle{\pgfqpoint{0.050000in}{0.050000in}}{\pgfqpoint{2.081932in}{2.081932in}}%
\pgfusepath{clip}%
\pgfsetbuttcap%
\pgfsetroundjoin%
\definecolor{currentfill}{rgb}{0.267968,0.223549,0.512008}%
\pgfsetfillcolor{currentfill}%
\pgfsetlinewidth{0.000000pt}%
\definecolor{currentstroke}{rgb}{0.000000,0.000000,0.000000}%
\pgfsetstrokecolor{currentstroke}%
\pgfsetdash{}{0pt}%
\pgfpathmoveto{\pgfqpoint{0.869592in}{1.272384in}}%
\pgfpathlineto{\pgfqpoint{0.865099in}{1.282118in}}%
\pgfpathlineto{\pgfqpoint{0.860852in}{1.291968in}}%
\pgfpathlineto{\pgfqpoint{0.856869in}{1.301896in}}%
\pgfpathlineto{\pgfqpoint{0.853163in}{1.311863in}}%
\pgfpathlineto{\pgfqpoint{0.849750in}{1.321831in}}%
\pgfpathlineto{\pgfqpoint{0.820736in}{1.311490in}}%
\pgfpathlineto{\pgfqpoint{0.792808in}{1.300097in}}%
\pgfpathlineto{\pgfqpoint{0.766073in}{1.287688in}}%
\pgfpathlineto{\pgfqpoint{0.740634in}{1.274305in}}%
\pgfpathlineto{\pgfqpoint{0.716593in}{1.259993in}}%
\pgfpathlineto{\pgfqpoint{0.721732in}{1.250738in}}%
\pgfpathlineto{\pgfqpoint{0.727310in}{1.241554in}}%
\pgfpathlineto{\pgfqpoint{0.733306in}{1.232475in}}%
\pgfpathlineto{\pgfqpoint{0.739695in}{1.223538in}}%
\pgfpathlineto{\pgfqpoint{0.746455in}{1.214778in}}%
\pgfpathlineto{\pgfqpoint{0.768666in}{1.228105in}}%
\pgfpathlineto{\pgfqpoint{0.792181in}{1.240569in}}%
\pgfpathlineto{\pgfqpoint{0.816905in}{1.252129in}}%
\pgfpathlineto{\pgfqpoint{0.842741in}{1.262746in}}%
\pgfpathlineto{\pgfqpoint{0.869592in}{1.272384in}}%
\pgfpathclose%
\pgfusepath{fill}%
\end{pgfscope}%
\begin{pgfscope}%
\pgfpathrectangle{\pgfqpoint{0.050000in}{0.050000in}}{\pgfqpoint{2.081932in}{2.081932in}}%
\pgfusepath{clip}%
\pgfsetbuttcap%
\pgfsetroundjoin%
\definecolor{currentfill}{rgb}{0.327796,0.773980,0.406640}%
\pgfsetfillcolor{currentfill}%
\pgfsetlinewidth{0.000000pt}%
\definecolor{currentstroke}{rgb}{0.000000,0.000000,0.000000}%
\pgfsetstrokecolor{currentstroke}%
\pgfsetdash{}{0pt}%
\pgfpathmoveto{\pgfqpoint{1.346726in}{1.456258in}}%
\pgfpathlineto{\pgfqpoint{1.345221in}{1.461817in}}%
\pgfpathlineto{\pgfqpoint{1.343434in}{1.466830in}}%
\pgfpathlineto{\pgfqpoint{1.341369in}{1.471275in}}%
\pgfpathlineto{\pgfqpoint{1.339035in}{1.475135in}}%
\pgfpathlineto{\pgfqpoint{1.336440in}{1.478391in}}%
\pgfpathlineto{\pgfqpoint{1.305146in}{1.485580in}}%
\pgfpathlineto{\pgfqpoint{1.273184in}{1.491636in}}%
\pgfpathlineto{\pgfqpoint{1.240669in}{1.496539in}}%
\pgfpathlineto{\pgfqpoint{1.207719in}{1.500275in}}%
\pgfpathlineto{\pgfqpoint{1.174453in}{1.502832in}}%
\pgfpathlineto{\pgfqpoint{1.175111in}{1.499877in}}%
\pgfpathlineto{\pgfqpoint{1.175703in}{1.496292in}}%
\pgfpathlineto{\pgfqpoint{1.176227in}{1.492092in}}%
\pgfpathlineto{\pgfqpoint{1.176680in}{1.487296in}}%
\pgfpathlineto{\pgfqpoint{1.177062in}{1.481925in}}%
\pgfpathlineto{\pgfqpoint{1.211899in}{1.479240in}}%
\pgfpathlineto{\pgfqpoint{1.246407in}{1.475317in}}%
\pgfpathlineto{\pgfqpoint{1.280462in}{1.470168in}}%
\pgfpathlineto{\pgfqpoint{1.313942in}{1.463808in}}%
\pgfpathlineto{\pgfqpoint{1.346726in}{1.456258in}}%
\pgfpathclose%
\pgfusepath{fill}%
\end{pgfscope}%
\begin{pgfscope}%
\pgfpathrectangle{\pgfqpoint{0.050000in}{0.050000in}}{\pgfqpoint{2.081932in}{2.081932in}}%
\pgfusepath{clip}%
\pgfsetbuttcap%
\pgfsetroundjoin%
\definecolor{currentfill}{rgb}{0.206756,0.371758,0.553117}%
\pgfsetfillcolor{currentfill}%
\pgfsetlinewidth{0.000000pt}%
\definecolor{currentstroke}{rgb}{0.000000,0.000000,0.000000}%
\pgfsetstrokecolor{currentstroke}%
\pgfsetdash{}{0pt}%
\pgfpathmoveto{\pgfqpoint{0.849750in}{1.321831in}}%
\pgfpathlineto{\pgfqpoint{0.846643in}{1.331762in}}%
\pgfpathlineto{\pgfqpoint{0.843853in}{1.341616in}}%
\pgfpathlineto{\pgfqpoint{0.841392in}{1.351357in}}%
\pgfpathlineto{\pgfqpoint{0.839269in}{1.360946in}}%
\pgfpathlineto{\pgfqpoint{0.837493in}{1.370346in}}%
\pgfpathlineto{\pgfqpoint{0.807140in}{1.359600in}}%
\pgfpathlineto{\pgfqpoint{0.777918in}{1.347758in}}%
\pgfpathlineto{\pgfqpoint{0.749937in}{1.334860in}}%
\pgfpathlineto{\pgfqpoint{0.723305in}{1.320947in}}%
\pgfpathlineto{\pgfqpoint{0.698129in}{1.306065in}}%
\pgfpathlineto{\pgfqpoint{0.700805in}{1.296998in}}%
\pgfpathlineto{\pgfqpoint{0.704004in}{1.287821in}}%
\pgfpathlineto{\pgfqpoint{0.707711in}{1.278571in}}%
\pgfpathlineto{\pgfqpoint{0.711913in}{1.269283in}}%
\pgfpathlineto{\pgfqpoint{0.716593in}{1.259993in}}%
\pgfpathlineto{\pgfqpoint{0.740634in}{1.274305in}}%
\pgfpathlineto{\pgfqpoint{0.766073in}{1.287688in}}%
\pgfpathlineto{\pgfqpoint{0.792808in}{1.300097in}}%
\pgfpathlineto{\pgfqpoint{0.820736in}{1.311490in}}%
\pgfpathlineto{\pgfqpoint{0.849750in}{1.321831in}}%
\pgfpathclose%
\pgfusepath{fill}%
\end{pgfscope}%
\begin{pgfscope}%
\pgfpathrectangle{\pgfqpoint{0.050000in}{0.050000in}}{\pgfqpoint{2.081932in}{2.081932in}}%
\pgfusepath{clip}%
\pgfsetbuttcap%
\pgfsetroundjoin%
\definecolor{currentfill}{rgb}{0.267004,0.004874,0.329415}%
\pgfsetfillcolor{currentfill}%
\pgfsetlinewidth{0.000000pt}%
\definecolor{currentstroke}{rgb}{0.000000,0.000000,0.000000}%
\pgfsetstrokecolor{currentstroke}%
\pgfsetdash{}{0pt}%
\pgfpathmoveto{\pgfqpoint{1.153391in}{1.189078in}}%
\pgfpathlineto{\pgfqpoint{1.154575in}{1.193929in}}%
\pgfpathlineto{\pgfqpoint{1.155773in}{1.199358in}}%
\pgfpathlineto{\pgfqpoint{1.156982in}{1.205343in}}%
\pgfpathlineto{\pgfqpoint{1.158196in}{1.211859in}}%
\pgfpathlineto{\pgfqpoint{1.159410in}{1.218879in}}%
\pgfpathlineto{\pgfqpoint{1.135051in}{1.219912in}}%
\pgfpathlineto{\pgfqpoint{1.110634in}{1.220046in}}%
\pgfpathlineto{\pgfqpoint{1.086249in}{1.219281in}}%
\pgfpathlineto{\pgfqpoint{1.061985in}{1.217619in}}%
\pgfpathlineto{\pgfqpoint{1.037932in}{1.215066in}}%
\pgfpathlineto{\pgfqpoint{1.040384in}{1.208158in}}%
\pgfpathlineto{\pgfqpoint{1.042835in}{1.201754in}}%
\pgfpathlineto{\pgfqpoint{1.045275in}{1.195881in}}%
\pgfpathlineto{\pgfqpoint{1.047695in}{1.190564in}}%
\pgfpathlineto{\pgfqpoint{1.050085in}{1.185824in}}%
\pgfpathlineto{\pgfqpoint{1.070537in}{1.188003in}}%
\pgfpathlineto{\pgfqpoint{1.091171in}{1.189422in}}%
\pgfpathlineto{\pgfqpoint{1.111910in}{1.190075in}}%
\pgfpathlineto{\pgfqpoint{1.132676in}{1.189960in}}%
\pgfpathlineto{\pgfqpoint{1.153391in}{1.189078in}}%
\pgfpathclose%
\pgfusepath{fill}%
\end{pgfscope}%
\begin{pgfscope}%
\pgfpathrectangle{\pgfqpoint{0.050000in}{0.050000in}}{\pgfqpoint{2.081932in}{2.081932in}}%
\pgfusepath{clip}%
\pgfsetbuttcap%
\pgfsetroundjoin%
\definecolor{currentfill}{rgb}{0.282327,0.094955,0.417331}%
\pgfsetfillcolor{currentfill}%
\pgfsetlinewidth{0.000000pt}%
\definecolor{currentstroke}{rgb}{0.000000,0.000000,0.000000}%
\pgfsetstrokecolor{currentstroke}%
\pgfsetdash{}{0pt}%
\pgfpathmoveto{\pgfqpoint{0.895129in}{1.226744in}}%
\pgfpathlineto{\pgfqpoint{0.889676in}{1.235348in}}%
\pgfpathlineto{\pgfqpoint{0.884376in}{1.244248in}}%
\pgfpathlineto{\pgfqpoint{0.879249in}{1.253412in}}%
\pgfpathlineto{\pgfqpoint{0.874315in}{1.262803in}}%
\pgfpathlineto{\pgfqpoint{0.869592in}{1.272384in}}%
\pgfpathlineto{\pgfqpoint{0.842741in}{1.262746in}}%
\pgfpathlineto{\pgfqpoint{0.816905in}{1.252129in}}%
\pgfpathlineto{\pgfqpoint{0.792181in}{1.240569in}}%
\pgfpathlineto{\pgfqpoint{0.768666in}{1.228105in}}%
\pgfpathlineto{\pgfqpoint{0.746455in}{1.214778in}}%
\pgfpathlineto{\pgfqpoint{0.753558in}{1.206227in}}%
\pgfpathlineto{\pgfqpoint{0.760976in}{1.197919in}}%
\pgfpathlineto{\pgfqpoint{0.768682in}{1.189888in}}%
\pgfpathlineto{\pgfqpoint{0.776646in}{1.182163in}}%
\pgfpathlineto{\pgfqpoint{0.784836in}{1.174777in}}%
\pgfpathlineto{\pgfqpoint{0.804707in}{1.186792in}}%
\pgfpathlineto{\pgfqpoint{0.825757in}{1.198034in}}%
\pgfpathlineto{\pgfqpoint{0.847902in}{1.208464in}}%
\pgfpathlineto{\pgfqpoint{0.871056in}{1.218044in}}%
\pgfpathlineto{\pgfqpoint{0.895129in}{1.226744in}}%
\pgfpathclose%
\pgfusepath{fill}%
\end{pgfscope}%
\begin{pgfscope}%
\pgfpathrectangle{\pgfqpoint{0.050000in}{0.050000in}}{\pgfqpoint{2.081932in}{2.081932in}}%
\pgfusepath{clip}%
\pgfsetbuttcap%
\pgfsetroundjoin%
\definecolor{currentfill}{rgb}{0.268510,0.009605,0.335427}%
\pgfsetfillcolor{currentfill}%
\pgfsetlinewidth{0.000000pt}%
\definecolor{currentstroke}{rgb}{0.000000,0.000000,0.000000}%
\pgfsetstrokecolor{currentstroke}%
\pgfsetdash{}{0pt}%
\pgfpathmoveto{\pgfqpoint{1.380354in}{1.161351in}}%
\pgfpathlineto{\pgfqpoint{1.388263in}{1.167159in}}%
\pgfpathlineto{\pgfqpoint{1.396115in}{1.173430in}}%
\pgfpathlineto{\pgfqpoint{1.403882in}{1.180139in}}%
\pgfpathlineto{\pgfqpoint{1.411531in}{1.187259in}}%
\pgfpathlineto{\pgfqpoint{1.419034in}{1.194763in}}%
\pgfpathlineto{\pgfqpoint{1.397211in}{1.205438in}}%
\pgfpathlineto{\pgfqpoint{1.374352in}{1.215276in}}%
\pgfpathlineto{\pgfqpoint{1.350546in}{1.224242in}}%
\pgfpathlineto{\pgfqpoint{1.325886in}{1.232306in}}%
\pgfpathlineto{\pgfqpoint{1.300463in}{1.239442in}}%
\pgfpathlineto{\pgfqpoint{1.295952in}{1.230867in}}%
\pgfpathlineto{\pgfqpoint{1.291350in}{1.222649in}}%
\pgfpathlineto{\pgfqpoint{1.286678in}{1.214820in}}%
\pgfpathlineto{\pgfqpoint{1.281951in}{1.207412in}}%
\pgfpathlineto{\pgfqpoint{1.277190in}{1.200455in}}%
\pgfpathlineto{\pgfqpoint{1.299328in}{1.194206in}}%
\pgfpathlineto{\pgfqpoint{1.320795in}{1.187146in}}%
\pgfpathlineto{\pgfqpoint{1.341507in}{1.179298in}}%
\pgfpathlineto{\pgfqpoint{1.361386in}{1.170689in}}%
\pgfpathlineto{\pgfqpoint{1.380354in}{1.161351in}}%
\pgfpathclose%
\pgfusepath{fill}%
\end{pgfscope}%
\begin{pgfscope}%
\pgfpathrectangle{\pgfqpoint{0.050000in}{0.050000in}}{\pgfqpoint{2.081932in}{2.081932in}}%
\pgfusepath{clip}%
\pgfsetbuttcap%
\pgfsetroundjoin%
\definecolor{currentfill}{rgb}{0.267004,0.004874,0.329415}%
\pgfsetfillcolor{currentfill}%
\pgfsetlinewidth{0.000000pt}%
\definecolor{currentstroke}{rgb}{0.000000,0.000000,0.000000}%
\pgfsetstrokecolor{currentstroke}%
\pgfsetdash{}{0pt}%
\pgfpathmoveto{\pgfqpoint{1.253512in}{1.173357in}}%
\pgfpathlineto{\pgfqpoint{1.258168in}{1.177673in}}%
\pgfpathlineto{\pgfqpoint{1.262882in}{1.182562in}}%
\pgfpathlineto{\pgfqpoint{1.267637in}{1.188004in}}%
\pgfpathlineto{\pgfqpoint{1.272412in}{1.193977in}}%
\pgfpathlineto{\pgfqpoint{1.277190in}{1.200455in}}%
\pgfpathlineto{\pgfqpoint{1.254464in}{1.205871in}}%
\pgfpathlineto{\pgfqpoint{1.231235in}{1.210435in}}%
\pgfpathlineto{\pgfqpoint{1.207591in}{1.214132in}}%
\pgfpathlineto{\pgfqpoint{1.183620in}{1.216950in}}%
\pgfpathlineto{\pgfqpoint{1.159410in}{1.218879in}}%
\pgfpathlineto{\pgfqpoint{1.158196in}{1.211859in}}%
\pgfpathlineto{\pgfqpoint{1.156982in}{1.205343in}}%
\pgfpathlineto{\pgfqpoint{1.155773in}{1.199358in}}%
\pgfpathlineto{\pgfqpoint{1.154575in}{1.193929in}}%
\pgfpathlineto{\pgfqpoint{1.153391in}{1.189078in}}%
\pgfpathlineto{\pgfqpoint{1.173978in}{1.187432in}}%
\pgfpathlineto{\pgfqpoint{1.194360in}{1.185027in}}%
\pgfpathlineto{\pgfqpoint{1.214460in}{1.181872in}}%
\pgfpathlineto{\pgfqpoint{1.234202in}{1.177977in}}%
\pgfpathlineto{\pgfqpoint{1.253512in}{1.173357in}}%
\pgfpathclose%
\pgfusepath{fill}%
\end{pgfscope}%
\begin{pgfscope}%
\pgfpathrectangle{\pgfqpoint{0.050000in}{0.050000in}}{\pgfqpoint{2.081932in}{2.081932in}}%
\pgfusepath{clip}%
\pgfsetbuttcap%
\pgfsetroundjoin%
\definecolor{currentfill}{rgb}{0.327796,0.773980,0.406640}%
\pgfsetfillcolor{currentfill}%
\pgfsetlinewidth{0.000000pt}%
\definecolor{currentstroke}{rgb}{0.000000,0.000000,0.000000}%
\pgfsetstrokecolor{currentstroke}%
\pgfsetdash{}{0pt}%
\pgfpathmoveto{\pgfqpoint{1.002282in}{1.476617in}}%
\pgfpathlineto{\pgfqpoint{1.003053in}{1.482027in}}%
\pgfpathlineto{\pgfqpoint{1.003968in}{1.486868in}}%
\pgfpathlineto{\pgfqpoint{1.005026in}{1.491118in}}%
\pgfpathlineto{\pgfqpoint{1.006222in}{1.494760in}}%
\pgfpathlineto{\pgfqpoint{1.007552in}{1.497777in}}%
\pgfpathlineto{\pgfqpoint{0.974894in}{1.493221in}}%
\pgfpathlineto{\pgfqpoint{0.942752in}{1.487507in}}%
\pgfpathlineto{\pgfqpoint{0.911245in}{1.480654in}}%
\pgfpathlineto{\pgfqpoint{0.880486in}{1.472683in}}%
\pgfpathlineto{\pgfqpoint{0.850590in}{1.463621in}}%
\pgfpathlineto{\pgfqpoint{0.847375in}{1.460181in}}%
\pgfpathlineto{\pgfqpoint{0.844484in}{1.456155in}}%
\pgfpathlineto{\pgfqpoint{0.841927in}{1.451560in}}%
\pgfpathlineto{\pgfqpoint{0.839713in}{1.446416in}}%
\pgfpathlineto{\pgfqpoint{0.837850in}{1.440744in}}%
\pgfpathlineto{\pgfqpoint{0.869178in}{1.450263in}}%
\pgfpathlineto{\pgfqpoint{0.901405in}{1.458635in}}%
\pgfpathlineto{\pgfqpoint{0.934411in}{1.465832in}}%
\pgfpathlineto{\pgfqpoint{0.968077in}{1.471832in}}%
\pgfpathlineto{\pgfqpoint{1.002282in}{1.476617in}}%
\pgfpathclose%
\pgfusepath{fill}%
\end{pgfscope}%
\begin{pgfscope}%
\pgfpathrectangle{\pgfqpoint{0.050000in}{0.050000in}}{\pgfqpoint{2.081932in}{2.081932in}}%
\pgfusepath{clip}%
\pgfsetbuttcap%
\pgfsetroundjoin%
\definecolor{currentfill}{rgb}{0.124780,0.640461,0.527068}%
\pgfsetfillcolor{currentfill}%
\pgfsetlinewidth{0.000000pt}%
\definecolor{currentstroke}{rgb}{0.000000,0.000000,0.000000}%
\pgfsetstrokecolor{currentstroke}%
\pgfsetdash{}{0pt}%
\pgfpathmoveto{\pgfqpoint{1.501298in}{1.365489in}}%
\pgfpathlineto{\pgfqpoint{1.501246in}{1.373461in}}%
\pgfpathlineto{\pgfqpoint{1.500697in}{1.381089in}}%
\pgfpathlineto{\pgfqpoint{1.499652in}{1.388342in}}%
\pgfpathlineto{\pgfqpoint{1.498112in}{1.395192in}}%
\pgfpathlineto{\pgfqpoint{1.496085in}{1.401610in}}%
\pgfpathlineto{\pgfqpoint{1.468535in}{1.414682in}}%
\pgfpathlineto{\pgfqpoint{1.439713in}{1.426720in}}%
\pgfpathlineto{\pgfqpoint{1.409728in}{1.437684in}}%
\pgfpathlineto{\pgfqpoint{1.378693in}{1.447541in}}%
\pgfpathlineto{\pgfqpoint{1.346726in}{1.456258in}}%
\pgfpathlineto{\pgfqpoint{1.347941in}{1.450176in}}%
\pgfpathlineto{\pgfqpoint{1.348864in}{1.443596in}}%
\pgfpathlineto{\pgfqpoint{1.349491in}{1.436544in}}%
\pgfpathlineto{\pgfqpoint{1.349820in}{1.429049in}}%
\pgfpathlineto{\pgfqpoint{1.349851in}{1.421141in}}%
\pgfpathlineto{\pgfqpoint{1.382262in}{1.412264in}}%
\pgfpathlineto{\pgfqpoint{1.413728in}{1.402227in}}%
\pgfpathlineto{\pgfqpoint{1.444132in}{1.391061in}}%
\pgfpathlineto{\pgfqpoint{1.473359in}{1.378802in}}%
\pgfpathlineto{\pgfqpoint{1.501298in}{1.365489in}}%
\pgfpathclose%
\pgfusepath{fill}%
\end{pgfscope}%
\begin{pgfscope}%
\pgfpathrectangle{\pgfqpoint{0.050000in}{0.050000in}}{\pgfqpoint{2.081932in}{2.081932in}}%
\pgfusepath{clip}%
\pgfsetbuttcap%
\pgfsetroundjoin%
\definecolor{currentfill}{rgb}{0.150476,0.504369,0.557430}%
\pgfsetfillcolor{currentfill}%
\pgfsetlinewidth{0.000000pt}%
\definecolor{currentstroke}{rgb}{0.000000,0.000000,0.000000}%
\pgfsetstrokecolor{currentstroke}%
\pgfsetdash{}{0pt}%
\pgfpathmoveto{\pgfqpoint{0.837493in}{1.370346in}}%
\pgfpathlineto{\pgfqpoint{0.836071in}{1.379521in}}%
\pgfpathlineto{\pgfqpoint{0.835008in}{1.388435in}}%
\pgfpathlineto{\pgfqpoint{0.834309in}{1.397054in}}%
\pgfpathlineto{\pgfqpoint{0.833978in}{1.405342in}}%
\pgfpathlineto{\pgfqpoint{0.803242in}{1.394510in}}%
\pgfpathlineto{\pgfqpoint{0.773648in}{1.382573in}}%
\pgfpathlineto{\pgfqpoint{0.745309in}{1.369570in}}%
\pgfpathlineto{\pgfqpoint{0.718335in}{1.355544in}}%
\pgfpathlineto{\pgfqpoint{0.692832in}{1.340540in}}%
\pgfpathlineto{\pgfqpoint{0.693331in}{1.332259in}}%
\pgfpathlineto{\pgfqpoint{0.694384in}{1.323731in}}%
\pgfpathlineto{\pgfqpoint{0.695985in}{1.314988in}}%
\pgfpathlineto{\pgfqpoint{0.698129in}{1.306065in}}%
\pgfpathlineto{\pgfqpoint{0.723305in}{1.320947in}}%
\pgfpathlineto{\pgfqpoint{0.749937in}{1.334860in}}%
\pgfpathlineto{\pgfqpoint{0.777918in}{1.347758in}}%
\pgfpathlineto{\pgfqpoint{0.807140in}{1.359600in}}%
\pgfpathlineto{\pgfqpoint{0.837493in}{1.370346in}}%
\pgfpathclose%
\pgfusepath{fill}%
\end{pgfscope}%
\begin{pgfscope}%
\pgfpathrectangle{\pgfqpoint{0.050000in}{0.050000in}}{\pgfqpoint{2.081932in}{2.081932in}}%
\pgfusepath{clip}%
\pgfsetbuttcap%
\pgfsetroundjoin%
\definecolor{currentfill}{rgb}{0.267004,0.004874,0.329415}%
\pgfsetfillcolor{currentfill}%
\pgfsetlinewidth{0.000000pt}%
\definecolor{currentstroke}{rgb}{0.000000,0.000000,0.000000}%
\pgfsetstrokecolor{currentstroke}%
\pgfsetdash{}{0pt}%
\pgfpathmoveto{\pgfqpoint{1.050085in}{1.185824in}}%
\pgfpathlineto{\pgfqpoint{1.047695in}{1.190564in}}%
\pgfpathlineto{\pgfqpoint{1.045275in}{1.195881in}}%
\pgfpathlineto{\pgfqpoint{1.042835in}{1.201754in}}%
\pgfpathlineto{\pgfqpoint{1.040384in}{1.208158in}}%
\pgfpathlineto{\pgfqpoint{1.037932in}{1.215066in}}%
\pgfpathlineto{\pgfqpoint{1.014180in}{1.211630in}}%
\pgfpathlineto{\pgfqpoint{0.990817in}{1.207323in}}%
\pgfpathlineto{\pgfqpoint{0.967930in}{1.202160in}}%
\pgfpathlineto{\pgfqpoint{0.945605in}{1.196157in}}%
\pgfpathlineto{\pgfqpoint{0.923928in}{1.189336in}}%
\pgfpathlineto{\pgfqpoint{0.929838in}{1.183186in}}%
\pgfpathlineto{\pgfqpoint{0.935746in}{1.177542in}}%
\pgfpathlineto{\pgfqpoint{0.941627in}{1.172429in}}%
\pgfpathlineto{\pgfqpoint{0.947458in}{1.167866in}}%
\pgfpathlineto{\pgfqpoint{0.953216in}{1.163875in}}%
\pgfpathlineto{\pgfqpoint{0.971623in}{1.169691in}}%
\pgfpathlineto{\pgfqpoint{0.990587in}{1.174811in}}%
\pgfpathlineto{\pgfqpoint{1.010034in}{1.179216in}}%
\pgfpathlineto{\pgfqpoint{1.029892in}{1.182891in}}%
\pgfpathlineto{\pgfqpoint{1.050085in}{1.185824in}}%
\pgfpathclose%
\pgfusepath{fill}%
\end{pgfscope}%
\begin{pgfscope}%
\pgfpathrectangle{\pgfqpoint{0.050000in}{0.050000in}}{\pgfqpoint{2.081932in}{2.081932in}}%
\pgfusepath{clip}%
\pgfsetbuttcap%
\pgfsetroundjoin%
\definecolor{currentfill}{rgb}{0.268510,0.009605,0.335427}%
\pgfsetfillcolor{currentfill}%
\pgfsetlinewidth{0.000000pt}%
\definecolor{currentstroke}{rgb}{0.000000,0.000000,0.000000}%
\pgfsetstrokecolor{currentstroke}%
\pgfsetdash{}{0pt}%
\pgfpathmoveto{\pgfqpoint{0.923928in}{1.189336in}}%
\pgfpathlineto{\pgfqpoint{0.918037in}{1.195968in}}%
\pgfpathlineto{\pgfqpoint{0.912189in}{1.203054in}}%
\pgfpathlineto{\pgfqpoint{0.906407in}{1.210565in}}%
\pgfpathlineto{\pgfqpoint{0.900713in}{1.218473in}}%
\pgfpathlineto{\pgfqpoint{0.895129in}{1.226744in}}%
\pgfpathlineto{\pgfqpoint{0.871056in}{1.218044in}}%
\pgfpathlineto{\pgfqpoint{0.847902in}{1.208464in}}%
\pgfpathlineto{\pgfqpoint{0.825757in}{1.198034in}}%
\pgfpathlineto{\pgfqpoint{0.804707in}{1.186792in}}%
\pgfpathlineto{\pgfqpoint{0.784836in}{1.174777in}}%
\pgfpathlineto{\pgfqpoint{0.793222in}{1.167756in}}%
\pgfpathlineto{\pgfqpoint{0.801769in}{1.161131in}}%
\pgfpathlineto{\pgfqpoint{0.810447in}{1.154926in}}%
\pgfpathlineto{\pgfqpoint{0.819220in}{1.149167in}}%
\pgfpathlineto{\pgfqpoint{0.828054in}{1.143876in}}%
\pgfpathlineto{\pgfqpoint{0.845304in}{1.154381in}}%
\pgfpathlineto{\pgfqpoint{0.863590in}{1.164212in}}%
\pgfpathlineto{\pgfqpoint{0.882841in}{1.173336in}}%
\pgfpathlineto{\pgfqpoint{0.902979in}{1.181721in}}%
\pgfpathlineto{\pgfqpoint{0.923928in}{1.189336in}}%
\pgfpathclose%
\pgfusepath{fill}%
\end{pgfscope}%
\begin{pgfscope}%
\pgfpathrectangle{\pgfqpoint{0.050000in}{0.050000in}}{\pgfqpoint{2.081932in}{2.081932in}}%
\pgfusepath{clip}%
\pgfsetbuttcap%
\pgfsetroundjoin%
\definecolor{currentfill}{rgb}{0.636902,0.856542,0.216620}%
\pgfsetfillcolor{currentfill}%
\pgfsetlinewidth{0.000000pt}%
\definecolor{currentstroke}{rgb}{0.000000,0.000000,0.000000}%
\pgfsetstrokecolor{currentstroke}%
\pgfsetdash{}{0pt}%
\pgfpathmoveto{\pgfqpoint{1.174453in}{1.502832in}}%
\pgfpathlineto{\pgfqpoint{1.173731in}{1.505144in}}%
\pgfpathlineto{\pgfqpoint{1.172948in}{1.506802in}}%
\pgfpathlineto{\pgfqpoint{1.172106in}{1.507796in}}%
\pgfpathlineto{\pgfqpoint{1.171210in}{1.508123in}}%
\pgfpathlineto{\pgfqpoint{1.170261in}{1.507777in}}%
\pgfpathlineto{\pgfqpoint{1.139335in}{1.509041in}}%
\pgfpathlineto{\pgfqpoint{1.108336in}{1.509206in}}%
\pgfpathlineto{\pgfqpoint{1.077376in}{1.508269in}}%
\pgfpathlineto{\pgfqpoint{1.046567in}{1.506234in}}%
\pgfpathlineto{\pgfqpoint{1.016019in}{1.503108in}}%
\pgfpathlineto{\pgfqpoint{1.014104in}{1.503368in}}%
\pgfpathlineto{\pgfqpoint{1.012293in}{1.502959in}}%
\pgfpathlineto{\pgfqpoint{1.010593in}{1.501887in}}%
\pgfpathlineto{\pgfqpoint{1.009011in}{1.500157in}}%
\pgfpathlineto{\pgfqpoint{1.007552in}{1.497777in}}%
\pgfpathlineto{\pgfqpoint{1.040610in}{1.501162in}}%
\pgfpathlineto{\pgfqpoint{1.073948in}{1.503365in}}%
\pgfpathlineto{\pgfqpoint{1.107448in}{1.504379in}}%
\pgfpathlineto{\pgfqpoint{1.140989in}{1.504202in}}%
\pgfpathlineto{\pgfqpoint{1.174453in}{1.502832in}}%
\pgfpathclose%
\pgfusepath{fill}%
\end{pgfscope}%
\begin{pgfscope}%
\pgfpathrectangle{\pgfqpoint{0.050000in}{0.050000in}}{\pgfqpoint{2.081932in}{2.081932in}}%
\pgfusepath{clip}%
\pgfsetbuttcap%
\pgfsetroundjoin%
\definecolor{currentfill}{rgb}{0.124780,0.640461,0.527068}%
\pgfsetfillcolor{currentfill}%
\pgfsetlinewidth{0.000000pt}%
\definecolor{currentstroke}{rgb}{0.000000,0.000000,0.000000}%
\pgfsetstrokecolor{currentstroke}%
\pgfsetdash{}{0pt}%
\pgfpathmoveto{\pgfqpoint{0.833978in}{1.405342in}}%
\pgfpathlineto{\pgfqpoint{0.834017in}{1.413269in}}%
\pgfpathlineto{\pgfqpoint{0.834424in}{1.420801in}}%
\pgfpathlineto{\pgfqpoint{0.835201in}{1.427910in}}%
\pgfpathlineto{\pgfqpoint{0.836344in}{1.434567in}}%
\pgfpathlineto{\pgfqpoint{0.837850in}{1.440744in}}%
\pgfpathlineto{\pgfqpoint{0.807536in}{1.430106in}}%
\pgfpathlineto{\pgfqpoint{0.778351in}{1.418385in}}%
\pgfpathlineto{\pgfqpoint{0.750406in}{1.405617in}}%
\pgfpathlineto{\pgfqpoint{0.723810in}{1.391846in}}%
\pgfpathlineto{\pgfqpoint{0.698666in}{1.377115in}}%
\pgfpathlineto{\pgfqpoint{0.696397in}{1.370544in}}%
\pgfpathlineto{\pgfqpoint{0.694675in}{1.363573in}}%
\pgfpathlineto{\pgfqpoint{0.693504in}{1.356228in}}%
\pgfpathlineto{\pgfqpoint{0.692890in}{1.348540in}}%
\pgfpathlineto{\pgfqpoint{0.692832in}{1.340540in}}%
\pgfpathlineto{\pgfqpoint{0.718335in}{1.355544in}}%
\pgfpathlineto{\pgfqpoint{0.745309in}{1.369570in}}%
\pgfpathlineto{\pgfqpoint{0.773648in}{1.382573in}}%
\pgfpathlineto{\pgfqpoint{0.803242in}{1.394510in}}%
\pgfpathlineto{\pgfqpoint{0.833978in}{1.405342in}}%
\pgfpathclose%
\pgfusepath{fill}%
\end{pgfscope}%
\begin{pgfscope}%
\pgfpathrectangle{\pgfqpoint{0.050000in}{0.050000in}}{\pgfqpoint{2.081932in}{2.081932in}}%
\pgfusepath{clip}%
\pgfsetbuttcap%
\pgfsetroundjoin%
\definecolor{currentfill}{rgb}{0.267004,0.004874,0.329415}%
\pgfsetfillcolor{currentfill}%
\pgfsetlinewidth{0.000000pt}%
\definecolor{currentstroke}{rgb}{0.000000,0.000000,0.000000}%
\pgfsetstrokecolor{currentstroke}%
\pgfsetdash{}{0pt}%
\pgfpathmoveto{\pgfqpoint{1.341057in}{1.140025in}}%
\pgfpathlineto{\pgfqpoint{1.348779in}{1.143199in}}%
\pgfpathlineto{\pgfqpoint{1.356601in}{1.146935in}}%
\pgfpathlineto{\pgfqpoint{1.364492in}{1.151218in}}%
\pgfpathlineto{\pgfqpoint{1.372420in}{1.156030in}}%
\pgfpathlineto{\pgfqpoint{1.380354in}{1.161351in}}%
\pgfpathlineto{\pgfqpoint{1.361386in}{1.170689in}}%
\pgfpathlineto{\pgfqpoint{1.341507in}{1.179298in}}%
\pgfpathlineto{\pgfqpoint{1.320795in}{1.187146in}}%
\pgfpathlineto{\pgfqpoint{1.299328in}{1.194206in}}%
\pgfpathlineto{\pgfqpoint{1.277190in}{1.200455in}}%
\pgfpathlineto{\pgfqpoint{1.272412in}{1.193977in}}%
\pgfpathlineto{\pgfqpoint{1.267637in}{1.188004in}}%
\pgfpathlineto{\pgfqpoint{1.262882in}{1.182562in}}%
\pgfpathlineto{\pgfqpoint{1.258168in}{1.177673in}}%
\pgfpathlineto{\pgfqpoint{1.253512in}{1.173357in}}%
\pgfpathlineto{\pgfqpoint{1.272315in}{1.168028in}}%
\pgfpathlineto{\pgfqpoint{1.290541in}{1.162007in}}%
\pgfpathlineto{\pgfqpoint{1.308118in}{1.155317in}}%
\pgfpathlineto{\pgfqpoint{1.324979in}{1.147981in}}%
\pgfpathlineto{\pgfqpoint{1.341057in}{1.140025in}}%
\pgfpathclose%
\pgfusepath{fill}%
\end{pgfscope}%
\begin{pgfscope}%
\pgfpathrectangle{\pgfqpoint{0.050000in}{0.050000in}}{\pgfqpoint{2.081932in}{2.081932in}}%
\pgfusepath{clip}%
\pgfsetbuttcap%
\pgfsetroundjoin%
\definecolor{currentfill}{rgb}{0.636902,0.856542,0.216620}%
\pgfsetfillcolor{currentfill}%
\pgfsetlinewidth{0.000000pt}%
\definecolor{currentstroke}{rgb}{0.000000,0.000000,0.000000}%
\pgfsetstrokecolor{currentstroke}%
\pgfsetdash{}{0pt}%
\pgfpathmoveto{\pgfqpoint{1.336440in}{1.478391in}}%
\pgfpathlineto{\pgfqpoint{1.333593in}{1.481030in}}%
\pgfpathlineto{\pgfqpoint{1.330506in}{1.483040in}}%
\pgfpathlineto{\pgfqpoint{1.327189in}{1.484411in}}%
\pgfpathlineto{\pgfqpoint{1.323656in}{1.485135in}}%
\pgfpathlineto{\pgfqpoint{1.319920in}{1.485208in}}%
\pgfpathlineto{\pgfqpoint{1.291017in}{1.491845in}}%
\pgfpathlineto{\pgfqpoint{1.261491in}{1.497437in}}%
\pgfpathlineto{\pgfqpoint{1.231450in}{1.501965in}}%
\pgfpathlineto{\pgfqpoint{1.201004in}{1.505415in}}%
\pgfpathlineto{\pgfqpoint{1.170261in}{1.507777in}}%
\pgfpathlineto{\pgfqpoint{1.171210in}{1.508123in}}%
\pgfpathlineto{\pgfqpoint{1.172106in}{1.507796in}}%
\pgfpathlineto{\pgfqpoint{1.172948in}{1.506802in}}%
\pgfpathlineto{\pgfqpoint{1.173731in}{1.505144in}}%
\pgfpathlineto{\pgfqpoint{1.174453in}{1.502832in}}%
\pgfpathlineto{\pgfqpoint{1.207719in}{1.500275in}}%
\pgfpathlineto{\pgfqpoint{1.240669in}{1.496539in}}%
\pgfpathlineto{\pgfqpoint{1.273184in}{1.491636in}}%
\pgfpathlineto{\pgfqpoint{1.305146in}{1.485580in}}%
\pgfpathlineto{\pgfqpoint{1.336440in}{1.478391in}}%
\pgfpathclose%
\pgfusepath{fill}%
\end{pgfscope}%
\begin{pgfscope}%
\pgfpathrectangle{\pgfqpoint{0.050000in}{0.050000in}}{\pgfqpoint{2.081932in}{2.081932in}}%
\pgfusepath{clip}%
\pgfsetbuttcap%
\pgfsetroundjoin%
\definecolor{currentfill}{rgb}{0.327796,0.773980,0.406640}%
\pgfsetfillcolor{currentfill}%
\pgfsetlinewidth{0.000000pt}%
\definecolor{currentstroke}{rgb}{0.000000,0.000000,0.000000}%
\pgfsetstrokecolor{currentstroke}%
\pgfsetdash{}{0pt}%
\pgfpathmoveto{\pgfqpoint{1.496085in}{1.401610in}}%
\pgfpathlineto{\pgfqpoint{1.493576in}{1.407571in}}%
\pgfpathlineto{\pgfqpoint{1.490595in}{1.413050in}}%
\pgfpathlineto{\pgfqpoint{1.487153in}{1.418025in}}%
\pgfpathlineto{\pgfqpoint{1.483261in}{1.422473in}}%
\pgfpathlineto{\pgfqpoint{1.478935in}{1.426377in}}%
\pgfpathlineto{\pgfqpoint{1.452664in}{1.438816in}}%
\pgfpathlineto{\pgfqpoint{1.425172in}{1.450272in}}%
\pgfpathlineto{\pgfqpoint{1.396565in}{1.460709in}}%
\pgfpathlineto{\pgfqpoint{1.366950in}{1.470092in}}%
\pgfpathlineto{\pgfqpoint{1.336440in}{1.478391in}}%
\pgfpathlineto{\pgfqpoint{1.339035in}{1.475135in}}%
\pgfpathlineto{\pgfqpoint{1.341369in}{1.471275in}}%
\pgfpathlineto{\pgfqpoint{1.343434in}{1.466830in}}%
\pgfpathlineto{\pgfqpoint{1.345221in}{1.461817in}}%
\pgfpathlineto{\pgfqpoint{1.346726in}{1.456258in}}%
\pgfpathlineto{\pgfqpoint{1.378693in}{1.447541in}}%
\pgfpathlineto{\pgfqpoint{1.409728in}{1.437684in}}%
\pgfpathlineto{\pgfqpoint{1.439713in}{1.426720in}}%
\pgfpathlineto{\pgfqpoint{1.468535in}{1.414682in}}%
\pgfpathlineto{\pgfqpoint{1.496085in}{1.401610in}}%
\pgfpathclose%
\pgfusepath{fill}%
\end{pgfscope}%
\begin{pgfscope}%
\pgfpathrectangle{\pgfqpoint{0.050000in}{0.050000in}}{\pgfqpoint{2.081932in}{2.081932in}}%
\pgfusepath{clip}%
\pgfsetbuttcap%
\pgfsetroundjoin%
\definecolor{currentfill}{rgb}{0.278791,0.062145,0.386592}%
\pgfsetfillcolor{currentfill}%
\pgfsetlinewidth{0.000000pt}%
\definecolor{currentstroke}{rgb}{0.000000,0.000000,0.000000}%
\pgfsetstrokecolor{currentstroke}%
\pgfsetdash{}{0pt}%
\pgfpathmoveto{\pgfqpoint{1.147858in}{1.174163in}}%
\pgfpathlineto{\pgfqpoint{1.148898in}{1.175851in}}%
\pgfpathlineto{\pgfqpoint{1.149976in}{1.178199in}}%
\pgfpathlineto{\pgfqpoint{1.151087in}{1.181196in}}%
\pgfpathlineto{\pgfqpoint{1.152227in}{1.184828in}}%
\pgfpathlineto{\pgfqpoint{1.153391in}{1.189078in}}%
\pgfpathlineto{\pgfqpoint{1.132676in}{1.189960in}}%
\pgfpathlineto{\pgfqpoint{1.111910in}{1.190075in}}%
\pgfpathlineto{\pgfqpoint{1.091171in}{1.189422in}}%
\pgfpathlineto{\pgfqpoint{1.070537in}{1.188003in}}%
\pgfpathlineto{\pgfqpoint{1.050085in}{1.185824in}}%
\pgfpathlineto{\pgfqpoint{1.052436in}{1.181682in}}%
\pgfpathlineto{\pgfqpoint{1.054738in}{1.178158in}}%
\pgfpathlineto{\pgfqpoint{1.056981in}{1.175265in}}%
\pgfpathlineto{\pgfqpoint{1.059157in}{1.173019in}}%
\pgfpathlineto{\pgfqpoint{1.061257in}{1.171430in}}%
\pgfpathlineto{\pgfqpoint{1.078399in}{1.173260in}}%
\pgfpathlineto{\pgfqpoint{1.095697in}{1.174451in}}%
\pgfpathlineto{\pgfqpoint{1.113083in}{1.175000in}}%
\pgfpathlineto{\pgfqpoint{1.130492in}{1.174903in}}%
\pgfpathlineto{\pgfqpoint{1.147858in}{1.174163in}}%
\pgfpathclose%
\pgfusepath{fill}%
\end{pgfscope}%
\begin{pgfscope}%
\pgfpathrectangle{\pgfqpoint{0.050000in}{0.050000in}}{\pgfqpoint{2.081932in}{2.081932in}}%
\pgfusepath{clip}%
\pgfsetbuttcap%
\pgfsetroundjoin%
\definecolor{currentfill}{rgb}{0.267968,0.223549,0.512008}%
\pgfsetfillcolor{currentfill}%
\pgfsetlinewidth{0.000000pt}%
\definecolor{currentstroke}{rgb}{0.000000,0.000000,0.000000}%
\pgfsetstrokecolor{currentstroke}%
\pgfsetdash{}{0pt}%
\pgfpathmoveto{\pgfqpoint{1.554857in}{1.165300in}}%
\pgfpathlineto{\pgfqpoint{1.562808in}{1.173212in}}%
\pgfpathlineto{\pgfqpoint{1.570326in}{1.181353in}}%
\pgfpathlineto{\pgfqpoint{1.577382in}{1.189691in}}%
\pgfpathlineto{\pgfqpoint{1.583947in}{1.198192in}}%
\pgfpathlineto{\pgfqpoint{1.589997in}{1.206824in}}%
\pgfpathlineto{\pgfqpoint{1.571234in}{1.223841in}}%
\pgfpathlineto{\pgfqpoint{1.550769in}{1.240102in}}%
\pgfpathlineto{\pgfqpoint{1.528691in}{1.255549in}}%
\pgfpathlineto{\pgfqpoint{1.505091in}{1.270129in}}%
\pgfpathlineto{\pgfqpoint{1.480063in}{1.283794in}}%
\pgfpathlineto{\pgfqpoint{1.475470in}{1.274264in}}%
\pgfpathlineto{\pgfqpoint{1.470483in}{1.264776in}}%
\pgfpathlineto{\pgfqpoint{1.465123in}{1.255370in}}%
\pgfpathlineto{\pgfqpoint{1.459410in}{1.246079in}}%
\pgfpathlineto{\pgfqpoint{1.453365in}{1.236942in}}%
\pgfpathlineto{\pgfqpoint{1.476497in}{1.224216in}}%
\pgfpathlineto{\pgfqpoint{1.498298in}{1.210640in}}%
\pgfpathlineto{\pgfqpoint{1.518680in}{1.196262in}}%
\pgfpathlineto{\pgfqpoint{1.537560in}{1.181131in}}%
\pgfpathlineto{\pgfqpoint{1.554857in}{1.165300in}}%
\pgfpathclose%
\pgfusepath{fill}%
\end{pgfscope}%
\begin{pgfscope}%
\pgfpathrectangle{\pgfqpoint{0.050000in}{0.050000in}}{\pgfqpoint{2.081932in}{2.081932in}}%
\pgfusepath{clip}%
\pgfsetbuttcap%
\pgfsetroundjoin%
\definecolor{currentfill}{rgb}{0.636902,0.856542,0.216620}%
\pgfsetfillcolor{currentfill}%
\pgfsetlinewidth{0.000000pt}%
\definecolor{currentstroke}{rgb}{0.000000,0.000000,0.000000}%
\pgfsetstrokecolor{currentstroke}%
\pgfsetdash{}{0pt}%
\pgfpathmoveto{\pgfqpoint{1.007552in}{1.497777in}}%
\pgfpathlineto{\pgfqpoint{1.009011in}{1.500157in}}%
\pgfpathlineto{\pgfqpoint{1.010593in}{1.501887in}}%
\pgfpathlineto{\pgfqpoint{1.012293in}{1.502959in}}%
\pgfpathlineto{\pgfqpoint{1.014104in}{1.503368in}}%
\pgfpathlineto{\pgfqpoint{1.016019in}{1.503108in}}%
\pgfpathlineto{\pgfqpoint{0.985844in}{1.498900in}}%
\pgfpathlineto{\pgfqpoint{0.956151in}{1.493624in}}%
\pgfpathlineto{\pgfqpoint{0.927050in}{1.487297in}}%
\pgfpathlineto{\pgfqpoint{0.898646in}{1.479940in}}%
\pgfpathlineto{\pgfqpoint{0.871046in}{1.471575in}}%
\pgfpathlineto{\pgfqpoint{0.866420in}{1.471248in}}%
\pgfpathlineto{\pgfqpoint{0.862046in}{1.470282in}}%
\pgfpathlineto{\pgfqpoint{0.857939in}{1.468683in}}%
\pgfpathlineto{\pgfqpoint{0.854115in}{1.466459in}}%
\pgfpathlineto{\pgfqpoint{0.850590in}{1.463621in}}%
\pgfpathlineto{\pgfqpoint{0.880486in}{1.472683in}}%
\pgfpathlineto{\pgfqpoint{0.911245in}{1.480654in}}%
\pgfpathlineto{\pgfqpoint{0.942752in}{1.487507in}}%
\pgfpathlineto{\pgfqpoint{0.974894in}{1.493221in}}%
\pgfpathlineto{\pgfqpoint{1.007552in}{1.497777in}}%
\pgfpathclose%
\pgfusepath{fill}%
\end{pgfscope}%
\begin{pgfscope}%
\pgfpathrectangle{\pgfqpoint{0.050000in}{0.050000in}}{\pgfqpoint{2.081932in}{2.081932in}}%
\pgfusepath{clip}%
\pgfsetbuttcap%
\pgfsetroundjoin%
\definecolor{currentfill}{rgb}{0.278791,0.062145,0.386592}%
\pgfsetfillcolor{currentfill}%
\pgfsetlinewidth{0.000000pt}%
\definecolor{currentstroke}{rgb}{0.000000,0.000000,0.000000}%
\pgfsetstrokecolor{currentstroke}%
\pgfsetdash{}{0pt}%
\pgfpathmoveto{\pgfqpoint{1.231757in}{1.160963in}}%
\pgfpathlineto{\pgfqpoint{1.235845in}{1.162175in}}%
\pgfpathlineto{\pgfqpoint{1.240082in}{1.164030in}}%
\pgfpathlineto{\pgfqpoint{1.244450in}{1.166521in}}%
\pgfpathlineto{\pgfqpoint{1.248933in}{1.169635in}}%
\pgfpathlineto{\pgfqpoint{1.253512in}{1.173357in}}%
\pgfpathlineto{\pgfqpoint{1.234202in}{1.177977in}}%
\pgfpathlineto{\pgfqpoint{1.214460in}{1.181872in}}%
\pgfpathlineto{\pgfqpoint{1.194360in}{1.185027in}}%
\pgfpathlineto{\pgfqpoint{1.173978in}{1.187432in}}%
\pgfpathlineto{\pgfqpoint{1.153391in}{1.189078in}}%
\pgfpathlineto{\pgfqpoint{1.152227in}{1.184828in}}%
\pgfpathlineto{\pgfqpoint{1.151087in}{1.181196in}}%
\pgfpathlineto{\pgfqpoint{1.149976in}{1.178199in}}%
\pgfpathlineto{\pgfqpoint{1.148898in}{1.175851in}}%
\pgfpathlineto{\pgfqpoint{1.147858in}{1.174163in}}%
\pgfpathlineto{\pgfqpoint{1.165115in}{1.172780in}}%
\pgfpathlineto{\pgfqpoint{1.182198in}{1.170761in}}%
\pgfpathlineto{\pgfqpoint{1.199042in}{1.168111in}}%
\pgfpathlineto{\pgfqpoint{1.215583in}{1.164841in}}%
\pgfpathlineto{\pgfqpoint{1.231757in}{1.160963in}}%
\pgfpathclose%
\pgfusepath{fill}%
\end{pgfscope}%
\begin{pgfscope}%
\pgfpathrectangle{\pgfqpoint{0.050000in}{0.050000in}}{\pgfqpoint{2.081932in}{2.081932in}}%
\pgfusepath{clip}%
\pgfsetbuttcap%
\pgfsetroundjoin%
\definecolor{currentfill}{rgb}{0.282327,0.094955,0.417331}%
\pgfsetfillcolor{currentfill}%
\pgfsetlinewidth{0.000000pt}%
\definecolor{currentstroke}{rgb}{0.000000,0.000000,0.000000}%
\pgfsetstrokecolor{currentstroke}%
\pgfsetdash{}{0pt}%
\pgfpathmoveto{\pgfqpoint{1.509743in}{1.130202in}}%
\pgfpathlineto{\pgfqpoint{1.519366in}{1.136533in}}%
\pgfpathlineto{\pgfqpoint{1.528724in}{1.143234in}}%
\pgfpathlineto{\pgfqpoint{1.537782in}{1.150282in}}%
\pgfpathlineto{\pgfqpoint{1.546504in}{1.157646in}}%
\pgfpathlineto{\pgfqpoint{1.554857in}{1.165300in}}%
\pgfpathlineto{\pgfqpoint{1.537560in}{1.181131in}}%
\pgfpathlineto{\pgfqpoint{1.518680in}{1.196262in}}%
\pgfpathlineto{\pgfqpoint{1.498298in}{1.210640in}}%
\pgfpathlineto{\pgfqpoint{1.476497in}{1.224216in}}%
\pgfpathlineto{\pgfqpoint{1.453365in}{1.236942in}}%
\pgfpathlineto{\pgfqpoint{1.447014in}{1.227993in}}%
\pgfpathlineto{\pgfqpoint{1.440379in}{1.219267in}}%
\pgfpathlineto{\pgfqpoint{1.433486in}{1.210798in}}%
\pgfpathlineto{\pgfqpoint{1.426362in}{1.202619in}}%
\pgfpathlineto{\pgfqpoint{1.419034in}{1.194763in}}%
\pgfpathlineto{\pgfqpoint{1.439737in}{1.183286in}}%
\pgfpathlineto{\pgfqpoint{1.459237in}{1.171047in}}%
\pgfpathlineto{\pgfqpoint{1.477454in}{1.158089in}}%
\pgfpathlineto{\pgfqpoint{1.494313in}{1.144458in}}%
\pgfpathlineto{\pgfqpoint{1.509743in}{1.130202in}}%
\pgfpathclose%
\pgfusepath{fill}%
\end{pgfscope}%
\begin{pgfscope}%
\pgfpathrectangle{\pgfqpoint{0.050000in}{0.050000in}}{\pgfqpoint{2.081932in}{2.081932in}}%
\pgfusepath{clip}%
\pgfsetbuttcap%
\pgfsetroundjoin%
\definecolor{currentfill}{rgb}{0.267004,0.004874,0.329415}%
\pgfsetfillcolor{currentfill}%
\pgfsetlinewidth{0.000000pt}%
\definecolor{currentstroke}{rgb}{0.000000,0.000000,0.000000}%
\pgfsetstrokecolor{currentstroke}%
\pgfsetdash{}{0pt}%
\pgfpathmoveto{\pgfqpoint{0.953216in}{1.163875in}}%
\pgfpathlineto{\pgfqpoint{0.947458in}{1.167866in}}%
\pgfpathlineto{\pgfqpoint{0.941627in}{1.172429in}}%
\pgfpathlineto{\pgfqpoint{0.935746in}{1.177542in}}%
\pgfpathlineto{\pgfqpoint{0.929838in}{1.183186in}}%
\pgfpathlineto{\pgfqpoint{0.923928in}{1.189336in}}%
\pgfpathlineto{\pgfqpoint{0.902979in}{1.181721in}}%
\pgfpathlineto{\pgfqpoint{0.882841in}{1.173336in}}%
\pgfpathlineto{\pgfqpoint{0.863590in}{1.164212in}}%
\pgfpathlineto{\pgfqpoint{0.845304in}{1.154381in}}%
\pgfpathlineto{\pgfqpoint{0.828054in}{1.143876in}}%
\pgfpathlineto{\pgfqpoint{0.836915in}{1.139077in}}%
\pgfpathlineto{\pgfqpoint{0.845768in}{1.134787in}}%
\pgfpathlineto{\pgfqpoint{0.854579in}{1.131025in}}%
\pgfpathlineto{\pgfqpoint{0.863312in}{1.127806in}}%
\pgfpathlineto{\pgfqpoint{0.871932in}{1.125146in}}%
\pgfpathlineto{\pgfqpoint{0.886537in}{1.134089in}}%
\pgfpathlineto{\pgfqpoint{0.902031in}{1.142463in}}%
\pgfpathlineto{\pgfqpoint{0.918352in}{1.150236in}}%
\pgfpathlineto{\pgfqpoint{0.935436in}{1.157382in}}%
\pgfpathlineto{\pgfqpoint{0.953216in}{1.163875in}}%
\pgfpathclose%
\pgfusepath{fill}%
\end{pgfscope}%
\begin{pgfscope}%
\pgfpathrectangle{\pgfqpoint{0.050000in}{0.050000in}}{\pgfqpoint{2.081932in}{2.081932in}}%
\pgfusepath{clip}%
\pgfsetbuttcap%
\pgfsetroundjoin%
\definecolor{currentfill}{rgb}{0.206756,0.371758,0.553117}%
\pgfsetfillcolor{currentfill}%
\pgfsetlinewidth{0.000000pt}%
\definecolor{currentstroke}{rgb}{0.000000,0.000000,0.000000}%
\pgfsetstrokecolor{currentstroke}%
\pgfsetdash{}{0pt}%
\pgfpathmoveto{\pgfqpoint{1.589997in}{1.206824in}}%
\pgfpathlineto{\pgfqpoint{1.595506in}{1.215554in}}%
\pgfpathlineto{\pgfqpoint{1.600455in}{1.224346in}}%
\pgfpathlineto{\pgfqpoint{1.604821in}{1.233168in}}%
\pgfpathlineto{\pgfqpoint{1.608588in}{1.241984in}}%
\pgfpathlineto{\pgfqpoint{1.611741in}{1.250760in}}%
\pgfpathlineto{\pgfqpoint{1.592066in}{1.268464in}}%
\pgfpathlineto{\pgfqpoint{1.570617in}{1.285378in}}%
\pgfpathlineto{\pgfqpoint{1.547485in}{1.301443in}}%
\pgfpathlineto{\pgfqpoint{1.522768in}{1.316605in}}%
\pgfpathlineto{\pgfqpoint{1.496565in}{1.330812in}}%
\pgfpathlineto{\pgfqpoint{1.494173in}{1.321616in}}%
\pgfpathlineto{\pgfqpoint{1.491315in}{1.312279in}}%
\pgfpathlineto{\pgfqpoint{1.488001in}{1.302839in}}%
\pgfpathlineto{\pgfqpoint{1.484246in}{1.293332in}}%
\pgfpathlineto{\pgfqpoint{1.480063in}{1.283794in}}%
\pgfpathlineto{\pgfqpoint{1.505091in}{1.270129in}}%
\pgfpathlineto{\pgfqpoint{1.528691in}{1.255549in}}%
\pgfpathlineto{\pgfqpoint{1.550769in}{1.240102in}}%
\pgfpathlineto{\pgfqpoint{1.571234in}{1.223841in}}%
\pgfpathlineto{\pgfqpoint{1.589997in}{1.206824in}}%
\pgfpathclose%
\pgfusepath{fill}%
\end{pgfscope}%
\begin{pgfscope}%
\pgfpathrectangle{\pgfqpoint{0.050000in}{0.050000in}}{\pgfqpoint{2.081932in}{2.081932in}}%
\pgfusepath{clip}%
\pgfsetbuttcap%
\pgfsetroundjoin%
\definecolor{currentfill}{rgb}{0.278791,0.062145,0.386592}%
\pgfsetfillcolor{currentfill}%
\pgfsetlinewidth{0.000000pt}%
\definecolor{currentstroke}{rgb}{0.000000,0.000000,0.000000}%
\pgfsetstrokecolor{currentstroke}%
\pgfsetdash{}{0pt}%
\pgfpathmoveto{\pgfqpoint{1.061257in}{1.171430in}}%
\pgfpathlineto{\pgfqpoint{1.059157in}{1.173019in}}%
\pgfpathlineto{\pgfqpoint{1.056981in}{1.175265in}}%
\pgfpathlineto{\pgfqpoint{1.054738in}{1.178158in}}%
\pgfpathlineto{\pgfqpoint{1.052436in}{1.181682in}}%
\pgfpathlineto{\pgfqpoint{1.050085in}{1.185824in}}%
\pgfpathlineto{\pgfqpoint{1.029892in}{1.182891in}}%
\pgfpathlineto{\pgfqpoint{1.010034in}{1.179216in}}%
\pgfpathlineto{\pgfqpoint{0.990587in}{1.174811in}}%
\pgfpathlineto{\pgfqpoint{0.971623in}{1.169691in}}%
\pgfpathlineto{\pgfqpoint{0.953216in}{1.163875in}}%
\pgfpathlineto{\pgfqpoint{0.958878in}{1.160471in}}%
\pgfpathlineto{\pgfqpoint{0.964420in}{1.157671in}}%
\pgfpathlineto{\pgfqpoint{0.969822in}{1.155487in}}%
\pgfpathlineto{\pgfqpoint{0.975061in}{1.153929in}}%
\pgfpathlineto{\pgfqpoint{0.980114in}{1.153006in}}%
\pgfpathlineto{\pgfqpoint{0.995523in}{1.157886in}}%
\pgfpathlineto{\pgfqpoint{1.011404in}{1.162184in}}%
\pgfpathlineto{\pgfqpoint{1.027695in}{1.165882in}}%
\pgfpathlineto{\pgfqpoint{1.044334in}{1.168967in}}%
\pgfpathlineto{\pgfqpoint{1.061257in}{1.171430in}}%
\pgfpathclose%
\pgfusepath{fill}%
\end{pgfscope}%
\begin{pgfscope}%
\pgfpathrectangle{\pgfqpoint{0.050000in}{0.050000in}}{\pgfqpoint{2.081932in}{2.081932in}}%
\pgfusepath{clip}%
\pgfsetbuttcap%
\pgfsetroundjoin%
\definecolor{currentfill}{rgb}{0.268510,0.009605,0.335427}%
\pgfsetfillcolor{currentfill}%
\pgfsetlinewidth{0.000000pt}%
\definecolor{currentstroke}{rgb}{0.000000,0.000000,0.000000}%
\pgfsetstrokecolor{currentstroke}%
\pgfsetdash{}{0pt}%
\pgfpathmoveto{\pgfqpoint{1.459013in}{1.104944in}}%
\pgfpathlineto{\pgfqpoint{1.469377in}{1.109073in}}%
\pgfpathlineto{\pgfqpoint{1.479672in}{1.113683in}}%
\pgfpathlineto{\pgfqpoint{1.489858in}{1.118755in}}%
\pgfpathlineto{\pgfqpoint{1.499895in}{1.124269in}}%
\pgfpathlineto{\pgfqpoint{1.509743in}{1.130202in}}%
\pgfpathlineto{\pgfqpoint{1.494313in}{1.144458in}}%
\pgfpathlineto{\pgfqpoint{1.477454in}{1.158089in}}%
\pgfpathlineto{\pgfqpoint{1.459237in}{1.171047in}}%
\pgfpathlineto{\pgfqpoint{1.439737in}{1.183286in}}%
\pgfpathlineto{\pgfqpoint{1.419034in}{1.194763in}}%
\pgfpathlineto{\pgfqpoint{1.411531in}{1.187259in}}%
\pgfpathlineto{\pgfqpoint{1.403882in}{1.180139in}}%
\pgfpathlineto{\pgfqpoint{1.396115in}{1.173430in}}%
\pgfpathlineto{\pgfqpoint{1.388263in}{1.167159in}}%
\pgfpathlineto{\pgfqpoint{1.380354in}{1.161351in}}%
\pgfpathlineto{\pgfqpoint{1.398335in}{1.151315in}}%
\pgfpathlineto{\pgfqpoint{1.415259in}{1.140617in}}%
\pgfpathlineto{\pgfqpoint{1.431055in}{1.129295in}}%
\pgfpathlineto{\pgfqpoint{1.445661in}{1.117389in}}%
\pgfpathlineto{\pgfqpoint{1.459013in}{1.104944in}}%
\pgfpathclose%
\pgfusepath{fill}%
\end{pgfscope}%
\begin{pgfscope}%
\pgfpathrectangle{\pgfqpoint{0.050000in}{0.050000in}}{\pgfqpoint{2.081932in}{2.081932in}}%
\pgfusepath{clip}%
\pgfsetbuttcap%
\pgfsetroundjoin%
\definecolor{currentfill}{rgb}{0.327796,0.773980,0.406640}%
\pgfsetfillcolor{currentfill}%
\pgfsetlinewidth{0.000000pt}%
\definecolor{currentstroke}{rgb}{0.000000,0.000000,0.000000}%
\pgfsetstrokecolor{currentstroke}%
\pgfsetdash{}{0pt}%
\pgfpathmoveto{\pgfqpoint{0.837850in}{1.440744in}}%
\pgfpathlineto{\pgfqpoint{0.839713in}{1.446416in}}%
\pgfpathlineto{\pgfqpoint{0.841927in}{1.451560in}}%
\pgfpathlineto{\pgfqpoint{0.844484in}{1.456155in}}%
\pgfpathlineto{\pgfqpoint{0.847375in}{1.460181in}}%
\pgfpathlineto{\pgfqpoint{0.850590in}{1.463621in}}%
\pgfpathlineto{\pgfqpoint{0.821667in}{1.453495in}}%
\pgfpathlineto{\pgfqpoint{0.793827in}{1.442339in}}%
\pgfpathlineto{\pgfqpoint{0.767177in}{1.430190in}}%
\pgfpathlineto{\pgfqpoint{0.741820in}{1.417086in}}%
\pgfpathlineto{\pgfqpoint{0.717857in}{1.403073in}}%
\pgfpathlineto{\pgfqpoint{0.713016in}{1.398876in}}%
\pgfpathlineto{\pgfqpoint{0.708662in}{1.394161in}}%
\pgfpathlineto{\pgfqpoint{0.704810in}{1.388948in}}%
\pgfpathlineto{\pgfqpoint{0.701474in}{1.383258in}}%
\pgfpathlineto{\pgfqpoint{0.698666in}{1.377115in}}%
\pgfpathlineto{\pgfqpoint{0.723810in}{1.391846in}}%
\pgfpathlineto{\pgfqpoint{0.750406in}{1.405617in}}%
\pgfpathlineto{\pgfqpoint{0.778351in}{1.418385in}}%
\pgfpathlineto{\pgfqpoint{0.807536in}{1.430106in}}%
\pgfpathlineto{\pgfqpoint{0.837850in}{1.440744in}}%
\pgfpathclose%
\pgfusepath{fill}%
\end{pgfscope}%
\begin{pgfscope}%
\pgfpathrectangle{\pgfqpoint{0.050000in}{0.050000in}}{\pgfqpoint{2.081932in}{2.081932in}}%
\pgfusepath{clip}%
\pgfsetbuttcap%
\pgfsetroundjoin%
\definecolor{currentfill}{rgb}{0.150476,0.504369,0.557430}%
\pgfsetfillcolor{currentfill}%
\pgfsetlinewidth{0.000000pt}%
\definecolor{currentstroke}{rgb}{0.000000,0.000000,0.000000}%
\pgfsetstrokecolor{currentstroke}%
\pgfsetdash{}{0pt}%
\pgfpathmoveto{\pgfqpoint{1.611741in}{1.250760in}}%
\pgfpathlineto{\pgfqpoint{1.614266in}{1.259461in}}%
\pgfpathlineto{\pgfqpoint{1.616153in}{1.268053in}}%
\pgfpathlineto{\pgfqpoint{1.617394in}{1.276503in}}%
\pgfpathlineto{\pgfqpoint{1.617982in}{1.284776in}}%
\pgfpathlineto{\pgfqpoint{1.598043in}{1.302628in}}%
\pgfpathlineto{\pgfqpoint{1.576311in}{1.319683in}}%
\pgfpathlineto{\pgfqpoint{1.552877in}{1.335880in}}%
\pgfpathlineto{\pgfqpoint{1.527839in}{1.351166in}}%
\pgfpathlineto{\pgfqpoint{1.501298in}{1.365489in}}%
\pgfpathlineto{\pgfqpoint{1.500852in}{1.357205in}}%
\pgfpathlineto{\pgfqpoint{1.499911in}{1.348642in}}%
\pgfpathlineto{\pgfqpoint{1.498480in}{1.339832in}}%
\pgfpathlineto{\pgfqpoint{1.496565in}{1.330812in}}%
\pgfpathlineto{\pgfqpoint{1.522768in}{1.316605in}}%
\pgfpathlineto{\pgfqpoint{1.547485in}{1.301443in}}%
\pgfpathlineto{\pgfqpoint{1.570617in}{1.285378in}}%
\pgfpathlineto{\pgfqpoint{1.592066in}{1.268464in}}%
\pgfpathlineto{\pgfqpoint{1.611741in}{1.250760in}}%
\pgfpathclose%
\pgfusepath{fill}%
\end{pgfscope}%
\begin{pgfscope}%
\pgfpathrectangle{\pgfqpoint{0.050000in}{0.050000in}}{\pgfqpoint{2.081932in}{2.081932in}}%
\pgfusepath{clip}%
\pgfsetbuttcap%
\pgfsetroundjoin%
\definecolor{currentfill}{rgb}{0.876168,0.891125,0.095250}%
\pgfsetfillcolor{currentfill}%
\pgfsetlinewidth{0.000000pt}%
\definecolor{currentstroke}{rgb}{0.000000,0.000000,0.000000}%
\pgfsetstrokecolor{currentstroke}%
\pgfsetdash{}{0pt}%
\pgfpathmoveto{\pgfqpoint{1.170261in}{1.507777in}}%
\pgfpathlineto{\pgfqpoint{1.169265in}{1.506758in}}%
\pgfpathlineto{\pgfqpoint{1.168225in}{1.505069in}}%
\pgfpathlineto{\pgfqpoint{1.167145in}{1.502714in}}%
\pgfpathlineto{\pgfqpoint{1.166029in}{1.499701in}}%
\pgfpathlineto{\pgfqpoint{1.164882in}{1.496041in}}%
\pgfpathlineto{\pgfqpoint{1.137212in}{1.497174in}}%
\pgfpathlineto{\pgfqpoint{1.109476in}{1.497321in}}%
\pgfpathlineto{\pgfqpoint{1.081776in}{1.496482in}}%
\pgfpathlineto{\pgfqpoint{1.054212in}{1.494659in}}%
\pgfpathlineto{\pgfqpoint{1.026885in}{1.491858in}}%
\pgfpathlineto{\pgfqpoint{1.024568in}{1.495415in}}%
\pgfpathlineto{\pgfqpoint{1.022314in}{1.498328in}}%
\pgfpathlineto{\pgfqpoint{1.020132in}{1.500585in}}%
\pgfpathlineto{\pgfqpoint{1.018031in}{1.502180in}}%
\pgfpathlineto{\pgfqpoint{1.016019in}{1.503108in}}%
\pgfpathlineto{\pgfqpoint{1.046567in}{1.506234in}}%
\pgfpathlineto{\pgfqpoint{1.077376in}{1.508269in}}%
\pgfpathlineto{\pgfqpoint{1.108336in}{1.509206in}}%
\pgfpathlineto{\pgfqpoint{1.139335in}{1.509041in}}%
\pgfpathlineto{\pgfqpoint{1.170261in}{1.507777in}}%
\pgfpathclose%
\pgfusepath{fill}%
\end{pgfscope}%
\begin{pgfscope}%
\pgfpathrectangle{\pgfqpoint{0.050000in}{0.050000in}}{\pgfqpoint{2.081932in}{2.081932in}}%
\pgfusepath{clip}%
\pgfsetbuttcap%
\pgfsetroundjoin%
\definecolor{currentfill}{rgb}{0.278791,0.062145,0.386592}%
\pgfsetfillcolor{currentfill}%
\pgfsetlinewidth{0.000000pt}%
\definecolor{currentstroke}{rgb}{0.000000,0.000000,0.000000}%
\pgfsetstrokecolor{currentstroke}%
\pgfsetdash{}{0pt}%
\pgfpathmoveto{\pgfqpoint{1.305002in}{1.133004in}}%
\pgfpathlineto{\pgfqpoint{1.311774in}{1.133202in}}%
\pgfpathlineto{\pgfqpoint{1.318794in}{1.134008in}}%
\pgfpathlineto{\pgfqpoint{1.326034in}{1.135419in}}%
\pgfpathlineto{\pgfqpoint{1.333465in}{1.137428in}}%
\pgfpathlineto{\pgfqpoint{1.341057in}{1.140025in}}%
\pgfpathlineto{\pgfqpoint{1.324979in}{1.147981in}}%
\pgfpathlineto{\pgfqpoint{1.308118in}{1.155317in}}%
\pgfpathlineto{\pgfqpoint{1.290541in}{1.162007in}}%
\pgfpathlineto{\pgfqpoint{1.272315in}{1.168028in}}%
\pgfpathlineto{\pgfqpoint{1.253512in}{1.173357in}}%
\pgfpathlineto{\pgfqpoint{1.248933in}{1.169635in}}%
\pgfpathlineto{\pgfqpoint{1.244450in}{1.166521in}}%
\pgfpathlineto{\pgfqpoint{1.240082in}{1.164030in}}%
\pgfpathlineto{\pgfqpoint{1.235845in}{1.162175in}}%
\pgfpathlineto{\pgfqpoint{1.231757in}{1.160963in}}%
\pgfpathlineto{\pgfqpoint{1.247502in}{1.156490in}}%
\pgfpathlineto{\pgfqpoint{1.262757in}{1.151438in}}%
\pgfpathlineto{\pgfqpoint{1.277464in}{1.145826in}}%
\pgfpathlineto{\pgfqpoint{1.291564in}{1.139674in}}%
\pgfpathlineto{\pgfqpoint{1.305002in}{1.133004in}}%
\pgfpathclose%
\pgfusepath{fill}%
\end{pgfscope}%
\begin{pgfscope}%
\pgfpathrectangle{\pgfqpoint{0.050000in}{0.050000in}}{\pgfqpoint{2.081932in}{2.081932in}}%
\pgfusepath{clip}%
\pgfsetbuttcap%
\pgfsetroundjoin%
\definecolor{currentfill}{rgb}{0.267968,0.223549,0.512008}%
\pgfsetfillcolor{currentfill}%
\pgfsetlinewidth{0.000000pt}%
\definecolor{currentstroke}{rgb}{0.000000,0.000000,0.000000}%
\pgfsetstrokecolor{currentstroke}%
\pgfsetdash{}{0pt}%
\pgfpathmoveto{\pgfqpoint{0.746455in}{1.214778in}}%
\pgfpathlineto{\pgfqpoint{0.739695in}{1.223538in}}%
\pgfpathlineto{\pgfqpoint{0.733306in}{1.232475in}}%
\pgfpathlineto{\pgfqpoint{0.727310in}{1.241554in}}%
\pgfpathlineto{\pgfqpoint{0.721732in}{1.250738in}}%
\pgfpathlineto{\pgfqpoint{0.716593in}{1.259993in}}%
\pgfpathlineto{\pgfqpoint{0.694046in}{1.244799in}}%
\pgfpathlineto{\pgfqpoint{0.673085in}{1.228776in}}%
\pgfpathlineto{\pgfqpoint{0.653800in}{1.211978in}}%
\pgfpathlineto{\pgfqpoint{0.636274in}{1.194465in}}%
\pgfpathlineto{\pgfqpoint{0.620585in}{1.176299in}}%
\pgfpathlineto{\pgfqpoint{0.627015in}{1.168029in}}%
\pgfpathlineto{\pgfqpoint{0.633993in}{1.159924in}}%
\pgfpathlineto{\pgfqpoint{0.641490in}{1.152016in}}%
\pgfpathlineto{\pgfqpoint{0.649478in}{1.144337in}}%
\pgfpathlineto{\pgfqpoint{0.657925in}{1.136916in}}%
\pgfpathlineto{\pgfqpoint{0.672365in}{1.153806in}}%
\pgfpathlineto{\pgfqpoint{0.688511in}{1.170094in}}%
\pgfpathlineto{\pgfqpoint{0.706294in}{1.185722in}}%
\pgfpathlineto{\pgfqpoint{0.725635in}{1.200634in}}%
\pgfpathlineto{\pgfqpoint{0.746455in}{1.214778in}}%
\pgfpathclose%
\pgfusepath{fill}%
\end{pgfscope}%
\begin{pgfscope}%
\pgfpathrectangle{\pgfqpoint{0.050000in}{0.050000in}}{\pgfqpoint{2.081932in}{2.081932in}}%
\pgfusepath{clip}%
\pgfsetbuttcap%
\pgfsetroundjoin%
\definecolor{currentfill}{rgb}{0.282327,0.094955,0.417331}%
\pgfsetfillcolor{currentfill}%
\pgfsetlinewidth{0.000000pt}%
\definecolor{currentstroke}{rgb}{0.000000,0.000000,0.000000}%
\pgfsetstrokecolor{currentstroke}%
\pgfsetdash{}{0pt}%
\pgfpathmoveto{\pgfqpoint{0.784836in}{1.174777in}}%
\pgfpathlineto{\pgfqpoint{0.776646in}{1.182163in}}%
\pgfpathlineto{\pgfqpoint{0.768682in}{1.189888in}}%
\pgfpathlineto{\pgfqpoint{0.760976in}{1.197919in}}%
\pgfpathlineto{\pgfqpoint{0.753558in}{1.206227in}}%
\pgfpathlineto{\pgfqpoint{0.746455in}{1.214778in}}%
\pgfpathlineto{\pgfqpoint{0.725635in}{1.200634in}}%
\pgfpathlineto{\pgfqpoint{0.706294in}{1.185722in}}%
\pgfpathlineto{\pgfqpoint{0.688511in}{1.170094in}}%
\pgfpathlineto{\pgfqpoint{0.672365in}{1.153806in}}%
\pgfpathlineto{\pgfqpoint{0.657925in}{1.136916in}}%
\pgfpathlineto{\pgfqpoint{0.666798in}{1.129782in}}%
\pgfpathlineto{\pgfqpoint{0.676063in}{1.122964in}}%
\pgfpathlineto{\pgfqpoint{0.685682in}{1.116486in}}%
\pgfpathlineto{\pgfqpoint{0.695618in}{1.110376in}}%
\pgfpathlineto{\pgfqpoint{0.705834in}{1.104657in}}%
\pgfpathlineto{\pgfqpoint{0.718687in}{1.119856in}}%
\pgfpathlineto{\pgfqpoint{0.733079in}{1.134519in}}%
\pgfpathlineto{\pgfqpoint{0.748948in}{1.148594in}}%
\pgfpathlineto{\pgfqpoint{0.766224in}{1.162029in}}%
\pgfpathlineto{\pgfqpoint{0.784836in}{1.174777in}}%
\pgfpathclose%
\pgfusepath{fill}%
\end{pgfscope}%
\begin{pgfscope}%
\pgfpathrectangle{\pgfqpoint{0.050000in}{0.050000in}}{\pgfqpoint{2.081932in}{2.081932in}}%
\pgfusepath{clip}%
\pgfsetbuttcap%
\pgfsetroundjoin%
\definecolor{currentfill}{rgb}{0.636902,0.856542,0.216620}%
\pgfsetfillcolor{currentfill}%
\pgfsetlinewidth{0.000000pt}%
\definecolor{currentstroke}{rgb}{0.000000,0.000000,0.000000}%
\pgfsetstrokecolor{currentstroke}%
\pgfsetdash{}{0pt}%
\pgfpathmoveto{\pgfqpoint{1.478935in}{1.426377in}}%
\pgfpathlineto{\pgfqpoint{1.474190in}{1.429720in}}%
\pgfpathlineto{\pgfqpoint{1.469045in}{1.432486in}}%
\pgfpathlineto{\pgfqpoint{1.463519in}{1.434664in}}%
\pgfpathlineto{\pgfqpoint{1.457634in}{1.436243in}}%
\pgfpathlineto{\pgfqpoint{1.451412in}{1.437215in}}%
\pgfpathlineto{\pgfqpoint{1.427188in}{1.448688in}}%
\pgfpathlineto{\pgfqpoint{1.401828in}{1.459257in}}%
\pgfpathlineto{\pgfqpoint{1.375429in}{1.468888in}}%
\pgfpathlineto{\pgfqpoint{1.348092in}{1.477548in}}%
\pgfpathlineto{\pgfqpoint{1.319920in}{1.485208in}}%
\pgfpathlineto{\pgfqpoint{1.323656in}{1.485135in}}%
\pgfpathlineto{\pgfqpoint{1.327189in}{1.484411in}}%
\pgfpathlineto{\pgfqpoint{1.330506in}{1.483040in}}%
\pgfpathlineto{\pgfqpoint{1.333593in}{1.481030in}}%
\pgfpathlineto{\pgfqpoint{1.336440in}{1.478391in}}%
\pgfpathlineto{\pgfqpoint{1.366950in}{1.470092in}}%
\pgfpathlineto{\pgfqpoint{1.396565in}{1.460709in}}%
\pgfpathlineto{\pgfqpoint{1.425172in}{1.450272in}}%
\pgfpathlineto{\pgfqpoint{1.452664in}{1.438816in}}%
\pgfpathlineto{\pgfqpoint{1.478935in}{1.426377in}}%
\pgfpathclose%
\pgfusepath{fill}%
\end{pgfscope}%
\begin{pgfscope}%
\pgfpathrectangle{\pgfqpoint{0.050000in}{0.050000in}}{\pgfqpoint{2.081932in}{2.081932in}}%
\pgfusepath{clip}%
\pgfsetbuttcap%
\pgfsetroundjoin%
\definecolor{currentfill}{rgb}{0.876168,0.891125,0.095250}%
\pgfsetfillcolor{currentfill}%
\pgfsetlinewidth{0.000000pt}%
\definecolor{currentstroke}{rgb}{0.000000,0.000000,0.000000}%
\pgfsetstrokecolor{currentstroke}%
\pgfsetdash{}{0pt}%
\pgfpathmoveto{\pgfqpoint{1.319920in}{1.485208in}}%
\pgfpathlineto{\pgfqpoint{1.315995in}{1.484628in}}%
\pgfpathlineto{\pgfqpoint{1.311897in}{1.483395in}}%
\pgfpathlineto{\pgfqpoint{1.307642in}{1.481513in}}%
\pgfpathlineto{\pgfqpoint{1.303247in}{1.478986in}}%
\pgfpathlineto{\pgfqpoint{1.298728in}{1.475824in}}%
\pgfpathlineto{\pgfqpoint{1.272891in}{1.481768in}}%
\pgfpathlineto{\pgfqpoint{1.246489in}{1.486777in}}%
\pgfpathlineto{\pgfqpoint{1.219620in}{1.490833in}}%
\pgfpathlineto{\pgfqpoint{1.192385in}{1.493925in}}%
\pgfpathlineto{\pgfqpoint{1.164882in}{1.496041in}}%
\pgfpathlineto{\pgfqpoint{1.166029in}{1.499701in}}%
\pgfpathlineto{\pgfqpoint{1.167145in}{1.502714in}}%
\pgfpathlineto{\pgfqpoint{1.168225in}{1.505069in}}%
\pgfpathlineto{\pgfqpoint{1.169265in}{1.506758in}}%
\pgfpathlineto{\pgfqpoint{1.170261in}{1.507777in}}%
\pgfpathlineto{\pgfqpoint{1.201004in}{1.505415in}}%
\pgfpathlineto{\pgfqpoint{1.231450in}{1.501965in}}%
\pgfpathlineto{\pgfqpoint{1.261491in}{1.497437in}}%
\pgfpathlineto{\pgfqpoint{1.291017in}{1.491845in}}%
\pgfpathlineto{\pgfqpoint{1.319920in}{1.485208in}}%
\pgfpathclose%
\pgfusepath{fill}%
\end{pgfscope}%
\begin{pgfscope}%
\pgfpathrectangle{\pgfqpoint{0.050000in}{0.050000in}}{\pgfqpoint{2.081932in}{2.081932in}}%
\pgfusepath{clip}%
\pgfsetbuttcap%
\pgfsetroundjoin%
\definecolor{currentfill}{rgb}{0.206756,0.371758,0.553117}%
\pgfsetfillcolor{currentfill}%
\pgfsetlinewidth{0.000000pt}%
\definecolor{currentstroke}{rgb}{0.000000,0.000000,0.000000}%
\pgfsetstrokecolor{currentstroke}%
\pgfsetdash{}{0pt}%
\pgfpathmoveto{\pgfqpoint{0.716593in}{1.259993in}}%
\pgfpathlineto{\pgfqpoint{0.711913in}{1.269283in}}%
\pgfpathlineto{\pgfqpoint{0.707711in}{1.278571in}}%
\pgfpathlineto{\pgfqpoint{0.704004in}{1.287821in}}%
\pgfpathlineto{\pgfqpoint{0.700805in}{1.296998in}}%
\pgfpathlineto{\pgfqpoint{0.698129in}{1.306065in}}%
\pgfpathlineto{\pgfqpoint{0.674509in}{1.290264in}}%
\pgfpathlineto{\pgfqpoint{0.652543in}{1.273598in}}%
\pgfpathlineto{\pgfqpoint{0.632323in}{1.256123in}}%
\pgfpathlineto{\pgfqpoint{0.613938in}{1.237900in}}%
\pgfpathlineto{\pgfqpoint{0.597469in}{1.218993in}}%
\pgfpathlineto{\pgfqpoint{0.600821in}{1.210387in}}%
\pgfpathlineto{\pgfqpoint{0.604827in}{1.201781in}}%
\pgfpathlineto{\pgfqpoint{0.609469in}{1.193208in}}%
\pgfpathlineto{\pgfqpoint{0.614729in}{1.184703in}}%
\pgfpathlineto{\pgfqpoint{0.620585in}{1.176299in}}%
\pgfpathlineto{\pgfqpoint{0.636274in}{1.194465in}}%
\pgfpathlineto{\pgfqpoint{0.653800in}{1.211978in}}%
\pgfpathlineto{\pgfqpoint{0.673085in}{1.228776in}}%
\pgfpathlineto{\pgfqpoint{0.694046in}{1.244799in}}%
\pgfpathlineto{\pgfqpoint{0.716593in}{1.259993in}}%
\pgfpathclose%
\pgfusepath{fill}%
\end{pgfscope}%
\begin{pgfscope}%
\pgfpathrectangle{\pgfqpoint{0.050000in}{0.050000in}}{\pgfqpoint{2.081932in}{2.081932in}}%
\pgfusepath{clip}%
\pgfsetbuttcap%
\pgfsetroundjoin%
\definecolor{currentfill}{rgb}{0.267004,0.004874,0.329415}%
\pgfsetfillcolor{currentfill}%
\pgfsetlinewidth{0.000000pt}%
\definecolor{currentstroke}{rgb}{0.000000,0.000000,0.000000}%
\pgfsetstrokecolor{currentstroke}%
\pgfsetdash{}{0pt}%
\pgfpathmoveto{\pgfqpoint{1.407581in}{1.092030in}}%
\pgfpathlineto{\pgfqpoint{1.417679in}{1.093542in}}%
\pgfpathlineto{\pgfqpoint{1.427913in}{1.095600in}}%
\pgfpathlineto{\pgfqpoint{1.438240in}{1.098193in}}%
\pgfpathlineto{\pgfqpoint{1.448620in}{1.101313in}}%
\pgfpathlineto{\pgfqpoint{1.459013in}{1.104944in}}%
\pgfpathlineto{\pgfqpoint{1.445661in}{1.117389in}}%
\pgfpathlineto{\pgfqpoint{1.431055in}{1.129295in}}%
\pgfpathlineto{\pgfqpoint{1.415259in}{1.140617in}}%
\pgfpathlineto{\pgfqpoint{1.398335in}{1.151315in}}%
\pgfpathlineto{\pgfqpoint{1.380354in}{1.161351in}}%
\pgfpathlineto{\pgfqpoint{1.372420in}{1.156030in}}%
\pgfpathlineto{\pgfqpoint{1.364492in}{1.151218in}}%
\pgfpathlineto{\pgfqpoint{1.356601in}{1.146935in}}%
\pgfpathlineto{\pgfqpoint{1.348779in}{1.143199in}}%
\pgfpathlineto{\pgfqpoint{1.341057in}{1.140025in}}%
\pgfpathlineto{\pgfqpoint{1.356289in}{1.131478in}}%
\pgfpathlineto{\pgfqpoint{1.370614in}{1.122371in}}%
\pgfpathlineto{\pgfqpoint{1.383973in}{1.112737in}}%
\pgfpathlineto{\pgfqpoint{1.396313in}{1.102610in}}%
\pgfpathlineto{\pgfqpoint{1.407581in}{1.092030in}}%
\pgfpathclose%
\pgfusepath{fill}%
\end{pgfscope}%
\begin{pgfscope}%
\pgfpathrectangle{\pgfqpoint{0.050000in}{0.050000in}}{\pgfqpoint{2.081932in}{2.081932in}}%
\pgfusepath{clip}%
\pgfsetbuttcap%
\pgfsetroundjoin%
\definecolor{currentfill}{rgb}{0.278791,0.062145,0.386592}%
\pgfsetfillcolor{currentfill}%
\pgfsetlinewidth{0.000000pt}%
\definecolor{currentstroke}{rgb}{0.000000,0.000000,0.000000}%
\pgfsetstrokecolor{currentstroke}%
\pgfsetdash{}{0pt}%
\pgfpathmoveto{\pgfqpoint{0.980114in}{1.153006in}}%
\pgfpathlineto{\pgfqpoint{0.975061in}{1.153929in}}%
\pgfpathlineto{\pgfqpoint{0.969822in}{1.155487in}}%
\pgfpathlineto{\pgfqpoint{0.964420in}{1.157671in}}%
\pgfpathlineto{\pgfqpoint{0.958878in}{1.160471in}}%
\pgfpathlineto{\pgfqpoint{0.953216in}{1.163875in}}%
\pgfpathlineto{\pgfqpoint{0.935436in}{1.157382in}}%
\pgfpathlineto{\pgfqpoint{0.918352in}{1.150236in}}%
\pgfpathlineto{\pgfqpoint{0.902031in}{1.142463in}}%
\pgfpathlineto{\pgfqpoint{0.886537in}{1.134089in}}%
\pgfpathlineto{\pgfqpoint{0.871932in}{1.125146in}}%
\pgfpathlineto{\pgfqpoint{0.880406in}{1.123054in}}%
\pgfpathlineto{\pgfqpoint{0.888700in}{1.121541in}}%
\pgfpathlineto{\pgfqpoint{0.896779in}{1.120614in}}%
\pgfpathlineto{\pgfqpoint{0.904612in}{1.120278in}}%
\pgfpathlineto{\pgfqpoint{0.912166in}{1.120536in}}%
\pgfpathlineto{\pgfqpoint{0.924359in}{1.128029in}}%
\pgfpathlineto{\pgfqpoint{0.937303in}{1.135048in}}%
\pgfpathlineto{\pgfqpoint{0.950947in}{1.141566in}}%
\pgfpathlineto{\pgfqpoint{0.965236in}{1.147559in}}%
\pgfpathlineto{\pgfqpoint{0.980114in}{1.153006in}}%
\pgfpathclose%
\pgfusepath{fill}%
\end{pgfscope}%
\begin{pgfscope}%
\pgfpathrectangle{\pgfqpoint{0.050000in}{0.050000in}}{\pgfqpoint{2.081932in}{2.081932in}}%
\pgfusepath{clip}%
\pgfsetbuttcap%
\pgfsetroundjoin%
\definecolor{currentfill}{rgb}{0.278012,0.180367,0.486697}%
\pgfsetfillcolor{currentfill}%
\pgfsetlinewidth{0.000000pt}%
\definecolor{currentstroke}{rgb}{0.000000,0.000000,0.000000}%
\pgfsetstrokecolor{currentstroke}%
\pgfsetdash{}{0pt}%
\pgfpathmoveto{\pgfqpoint{1.143371in}{1.175841in}}%
\pgfpathlineto{\pgfqpoint{1.144160in}{1.174149in}}%
\pgfpathlineto{\pgfqpoint{1.145007in}{1.173135in}}%
\pgfpathlineto{\pgfqpoint{1.145908in}{1.172800in}}%
\pgfpathlineto{\pgfqpoint{1.146860in}{1.173144in}}%
\pgfpathlineto{\pgfqpoint{1.147858in}{1.174163in}}%
\pgfpathlineto{\pgfqpoint{1.130492in}{1.174903in}}%
\pgfpathlineto{\pgfqpoint{1.113083in}{1.175000in}}%
\pgfpathlineto{\pgfqpoint{1.095697in}{1.174451in}}%
\pgfpathlineto{\pgfqpoint{1.078399in}{1.173260in}}%
\pgfpathlineto{\pgfqpoint{1.061257in}{1.171430in}}%
\pgfpathlineto{\pgfqpoint{1.063271in}{1.170505in}}%
\pgfpathlineto{\pgfqpoint{1.065192in}{1.170252in}}%
\pgfpathlineto{\pgfqpoint{1.067012in}{1.170673in}}%
\pgfpathlineto{\pgfqpoint{1.068722in}{1.171768in}}%
\pgfpathlineto{\pgfqpoint{1.070316in}{1.173535in}}%
\pgfpathlineto{\pgfqpoint{1.084775in}{1.175079in}}%
\pgfpathlineto{\pgfqpoint{1.099367in}{1.176084in}}%
\pgfpathlineto{\pgfqpoint{1.114034in}{1.176547in}}%
\pgfpathlineto{\pgfqpoint{1.128720in}{1.176465in}}%
\pgfpathlineto{\pgfqpoint{1.143371in}{1.175841in}}%
\pgfpathclose%
\pgfusepath{fill}%
\end{pgfscope}%
\begin{pgfscope}%
\pgfpathrectangle{\pgfqpoint{0.050000in}{0.050000in}}{\pgfqpoint{2.081932in}{2.081932in}}%
\pgfusepath{clip}%
\pgfsetbuttcap%
\pgfsetroundjoin%
\definecolor{currentfill}{rgb}{0.268510,0.009605,0.335427}%
\pgfsetfillcolor{currentfill}%
\pgfsetlinewidth{0.000000pt}%
\definecolor{currentstroke}{rgb}{0.000000,0.000000,0.000000}%
\pgfsetstrokecolor{currentstroke}%
\pgfsetdash{}{0pt}%
\pgfpathmoveto{\pgfqpoint{0.828054in}{1.143876in}}%
\pgfpathlineto{\pgfqpoint{0.819220in}{1.149167in}}%
\pgfpathlineto{\pgfqpoint{0.810447in}{1.154926in}}%
\pgfpathlineto{\pgfqpoint{0.801769in}{1.161131in}}%
\pgfpathlineto{\pgfqpoint{0.793222in}{1.167756in}}%
\pgfpathlineto{\pgfqpoint{0.784836in}{1.174777in}}%
\pgfpathlineto{\pgfqpoint{0.766224in}{1.162029in}}%
\pgfpathlineto{\pgfqpoint{0.748948in}{1.148594in}}%
\pgfpathlineto{\pgfqpoint{0.733079in}{1.134519in}}%
\pgfpathlineto{\pgfqpoint{0.718687in}{1.119856in}}%
\pgfpathlineto{\pgfqpoint{0.705834in}{1.104657in}}%
\pgfpathlineto{\pgfqpoint{0.716288in}{1.099351in}}%
\pgfpathlineto{\pgfqpoint{0.726940in}{1.094480in}}%
\pgfpathlineto{\pgfqpoint{0.737748in}{1.090062in}}%
\pgfpathlineto{\pgfqpoint{0.748671in}{1.086116in}}%
\pgfpathlineto{\pgfqpoint{0.759665in}{1.082657in}}%
\pgfpathlineto{\pgfqpoint{0.770760in}{1.095915in}}%
\pgfpathlineto{\pgfqpoint{0.783203in}{1.108712in}}%
\pgfpathlineto{\pgfqpoint{0.796939in}{1.121001in}}%
\pgfpathlineto{\pgfqpoint{0.811910in}{1.132737in}}%
\pgfpathlineto{\pgfqpoint{0.828054in}{1.143876in}}%
\pgfpathclose%
\pgfusepath{fill}%
\end{pgfscope}%
\begin{pgfscope}%
\pgfpathrectangle{\pgfqpoint{0.050000in}{0.050000in}}{\pgfqpoint{2.081932in}{2.081932in}}%
\pgfusepath{clip}%
\pgfsetbuttcap%
\pgfsetroundjoin%
\definecolor{currentfill}{rgb}{0.124780,0.640461,0.527068}%
\pgfsetfillcolor{currentfill}%
\pgfsetlinewidth{0.000000pt}%
\definecolor{currentstroke}{rgb}{0.000000,0.000000,0.000000}%
\pgfsetstrokecolor{currentstroke}%
\pgfsetdash{}{0pt}%
\pgfpathmoveto{\pgfqpoint{1.617982in}{1.284776in}}%
\pgfpathlineto{\pgfqpoint{1.617914in}{1.292839in}}%
\pgfpathlineto{\pgfqpoint{1.617190in}{1.300662in}}%
\pgfpathlineto{\pgfqpoint{1.615811in}{1.308211in}}%
\pgfpathlineto{\pgfqpoint{1.613782in}{1.315458in}}%
\pgfpathlineto{\pgfqpoint{1.611109in}{1.322372in}}%
\pgfpathlineto{\pgfqpoint{1.591460in}{1.339896in}}%
\pgfpathlineto{\pgfqpoint{1.570039in}{1.356638in}}%
\pgfpathlineto{\pgfqpoint{1.546939in}{1.372540in}}%
\pgfpathlineto{\pgfqpoint{1.522254in}{1.387548in}}%
\pgfpathlineto{\pgfqpoint{1.496085in}{1.401610in}}%
\pgfpathlineto{\pgfqpoint{1.498112in}{1.395192in}}%
\pgfpathlineto{\pgfqpoint{1.499652in}{1.388342in}}%
\pgfpathlineto{\pgfqpoint{1.500697in}{1.381089in}}%
\pgfpathlineto{\pgfqpoint{1.501246in}{1.373461in}}%
\pgfpathlineto{\pgfqpoint{1.501298in}{1.365489in}}%
\pgfpathlineto{\pgfqpoint{1.527839in}{1.351166in}}%
\pgfpathlineto{\pgfqpoint{1.552877in}{1.335880in}}%
\pgfpathlineto{\pgfqpoint{1.576311in}{1.319683in}}%
\pgfpathlineto{\pgfqpoint{1.598043in}{1.302628in}}%
\pgfpathlineto{\pgfqpoint{1.617982in}{1.284776in}}%
\pgfpathclose%
\pgfusepath{fill}%
\end{pgfscope}%
\begin{pgfscope}%
\pgfpathrectangle{\pgfqpoint{0.050000in}{0.050000in}}{\pgfqpoint{2.081932in}{2.081932in}}%
\pgfusepath{clip}%
\pgfsetbuttcap%
\pgfsetroundjoin%
\definecolor{currentfill}{rgb}{0.876168,0.891125,0.095250}%
\pgfsetfillcolor{currentfill}%
\pgfsetlinewidth{0.000000pt}%
\definecolor{currentstroke}{rgb}{0.000000,0.000000,0.000000}%
\pgfsetstrokecolor{currentstroke}%
\pgfsetdash{}{0pt}%
\pgfpathmoveto{\pgfqpoint{1.016019in}{1.503108in}}%
\pgfpathlineto{\pgfqpoint{1.018031in}{1.502180in}}%
\pgfpathlineto{\pgfqpoint{1.020132in}{1.500585in}}%
\pgfpathlineto{\pgfqpoint{1.022314in}{1.498328in}}%
\pgfpathlineto{\pgfqpoint{1.024568in}{1.495415in}}%
\pgfpathlineto{\pgfqpoint{1.026885in}{1.491858in}}%
\pgfpathlineto{\pgfqpoint{0.999895in}{1.488088in}}%
\pgfpathlineto{\pgfqpoint{0.973342in}{1.483362in}}%
\pgfpathlineto{\pgfqpoint{0.947324in}{1.477695in}}%
\pgfpathlineto{\pgfqpoint{0.921938in}{1.471106in}}%
\pgfpathlineto{\pgfqpoint{0.897278in}{1.463617in}}%
\pgfpathlineto{\pgfqpoint{0.891686in}{1.466477in}}%
\pgfpathlineto{\pgfqpoint{0.886245in}{1.468708in}}%
\pgfpathlineto{\pgfqpoint{0.880978in}{1.470305in}}%
\pgfpathlineto{\pgfqpoint{0.875905in}{1.471261in}}%
\pgfpathlineto{\pgfqpoint{0.871046in}{1.471575in}}%
\pgfpathlineto{\pgfqpoint{0.898646in}{1.479940in}}%
\pgfpathlineto{\pgfqpoint{0.927050in}{1.487297in}}%
\pgfpathlineto{\pgfqpoint{0.956151in}{1.493624in}}%
\pgfpathlineto{\pgfqpoint{0.985844in}{1.498900in}}%
\pgfpathlineto{\pgfqpoint{1.016019in}{1.503108in}}%
\pgfpathclose%
\pgfusepath{fill}%
\end{pgfscope}%
\begin{pgfscope}%
\pgfpathrectangle{\pgfqpoint{0.050000in}{0.050000in}}{\pgfqpoint{2.081932in}{2.081932in}}%
\pgfusepath{clip}%
\pgfsetbuttcap%
\pgfsetroundjoin%
\definecolor{currentfill}{rgb}{0.278012,0.180367,0.486697}%
\pgfsetfillcolor{currentfill}%
\pgfsetlinewidth{0.000000pt}%
\definecolor{currentstroke}{rgb}{0.000000,0.000000,0.000000}%
\pgfsetstrokecolor{currentstroke}%
\pgfsetdash{}{0pt}%
\pgfpathmoveto{\pgfqpoint{1.214124in}{1.164710in}}%
\pgfpathlineto{\pgfqpoint{1.217226in}{1.162653in}}%
\pgfpathlineto{\pgfqpoint{1.220554in}{1.161247in}}%
\pgfpathlineto{\pgfqpoint{1.224096in}{1.160497in}}%
\pgfpathlineto{\pgfqpoint{1.227835in}{1.160402in}}%
\pgfpathlineto{\pgfqpoint{1.231757in}{1.160963in}}%
\pgfpathlineto{\pgfqpoint{1.215583in}{1.164841in}}%
\pgfpathlineto{\pgfqpoint{1.199042in}{1.168111in}}%
\pgfpathlineto{\pgfqpoint{1.182198in}{1.170761in}}%
\pgfpathlineto{\pgfqpoint{1.165115in}{1.172780in}}%
\pgfpathlineto{\pgfqpoint{1.147858in}{1.174163in}}%
\pgfpathlineto{\pgfqpoint{1.146860in}{1.173144in}}%
\pgfpathlineto{\pgfqpoint{1.145908in}{1.172800in}}%
\pgfpathlineto{\pgfqpoint{1.145007in}{1.173135in}}%
\pgfpathlineto{\pgfqpoint{1.144160in}{1.174149in}}%
\pgfpathlineto{\pgfqpoint{1.143371in}{1.175841in}}%
\pgfpathlineto{\pgfqpoint{1.157928in}{1.174674in}}%
\pgfpathlineto{\pgfqpoint{1.172337in}{1.172971in}}%
\pgfpathlineto{\pgfqpoint{1.186542in}{1.170737in}}%
\pgfpathlineto{\pgfqpoint{1.200490in}{1.167980in}}%
\pgfpathlineto{\pgfqpoint{1.214124in}{1.164710in}}%
\pgfpathclose%
\pgfusepath{fill}%
\end{pgfscope}%
\begin{pgfscope}%
\pgfpathrectangle{\pgfqpoint{0.050000in}{0.050000in}}{\pgfqpoint{2.081932in}{2.081932in}}%
\pgfusepath{clip}%
\pgfsetbuttcap%
\pgfsetroundjoin%
\definecolor{currentfill}{rgb}{0.278012,0.180367,0.486697}%
\pgfsetfillcolor{currentfill}%
\pgfsetlinewidth{0.000000pt}%
\definecolor{currentstroke}{rgb}{0.000000,0.000000,0.000000}%
\pgfsetstrokecolor{currentstroke}%
\pgfsetdash{}{0pt}%
\pgfpathmoveto{\pgfqpoint{1.070316in}{1.173535in}}%
\pgfpathlineto{\pgfqpoint{1.068722in}{1.171768in}}%
\pgfpathlineto{\pgfqpoint{1.067012in}{1.170673in}}%
\pgfpathlineto{\pgfqpoint{1.065192in}{1.170252in}}%
\pgfpathlineto{\pgfqpoint{1.063271in}{1.170505in}}%
\pgfpathlineto{\pgfqpoint{1.061257in}{1.171430in}}%
\pgfpathlineto{\pgfqpoint{1.044334in}{1.168967in}}%
\pgfpathlineto{\pgfqpoint{1.027695in}{1.165882in}}%
\pgfpathlineto{\pgfqpoint{1.011404in}{1.162184in}}%
\pgfpathlineto{\pgfqpoint{0.995523in}{1.157886in}}%
\pgfpathlineto{\pgfqpoint{0.980114in}{1.153006in}}%
\pgfpathlineto{\pgfqpoint{0.984962in}{1.152722in}}%
\pgfpathlineto{\pgfqpoint{0.989584in}{1.153081in}}%
\pgfpathlineto{\pgfqpoint{0.993961in}{1.154083in}}%
\pgfpathlineto{\pgfqpoint{0.998075in}{1.155725in}}%
\pgfpathlineto{\pgfqpoint{1.001908in}{1.158003in}}%
\pgfpathlineto{\pgfqpoint{1.014892in}{1.162117in}}%
\pgfpathlineto{\pgfqpoint{1.028278in}{1.165739in}}%
\pgfpathlineto{\pgfqpoint{1.042013in}{1.168857in}}%
\pgfpathlineto{\pgfqpoint{1.056043in}{1.171459in}}%
\pgfpathlineto{\pgfqpoint{1.070316in}{1.173535in}}%
\pgfpathclose%
\pgfusepath{fill}%
\end{pgfscope}%
\begin{pgfscope}%
\pgfpathrectangle{\pgfqpoint{0.050000in}{0.050000in}}{\pgfqpoint{2.081932in}{2.081932in}}%
\pgfusepath{clip}%
\pgfsetbuttcap%
\pgfsetroundjoin%
\definecolor{currentfill}{rgb}{0.150476,0.504369,0.557430}%
\pgfsetfillcolor{currentfill}%
\pgfsetlinewidth{0.000000pt}%
\definecolor{currentstroke}{rgb}{0.000000,0.000000,0.000000}%
\pgfsetstrokecolor{currentstroke}%
\pgfsetdash{}{0pt}%
\pgfpathmoveto{\pgfqpoint{0.698129in}{1.306065in}}%
\pgfpathlineto{\pgfqpoint{0.695985in}{1.314988in}}%
\pgfpathlineto{\pgfqpoint{0.694384in}{1.323731in}}%
\pgfpathlineto{\pgfqpoint{0.693331in}{1.332259in}}%
\pgfpathlineto{\pgfqpoint{0.692832in}{1.340540in}}%
\pgfpathlineto{\pgfqpoint{0.668904in}{1.324609in}}%
\pgfpathlineto{\pgfqpoint{0.646649in}{1.307804in}}%
\pgfpathlineto{\pgfqpoint{0.626160in}{1.290184in}}%
\pgfpathlineto{\pgfqpoint{0.607527in}{1.271807in}}%
\pgfpathlineto{\pgfqpoint{0.590833in}{1.252741in}}%
\pgfpathlineto{\pgfqpoint{0.591458in}{1.244473in}}%
\pgfpathlineto{\pgfqpoint{0.592777in}{1.236070in}}%
\pgfpathlineto{\pgfqpoint{0.594784in}{1.227565in}}%
\pgfpathlineto{\pgfqpoint{0.597469in}{1.218993in}}%
\pgfpathlineto{\pgfqpoint{0.613938in}{1.237900in}}%
\pgfpathlineto{\pgfqpoint{0.632323in}{1.256123in}}%
\pgfpathlineto{\pgfqpoint{0.652543in}{1.273598in}}%
\pgfpathlineto{\pgfqpoint{0.674509in}{1.290264in}}%
\pgfpathlineto{\pgfqpoint{0.698129in}{1.306065in}}%
\pgfpathclose%
\pgfusepath{fill}%
\end{pgfscope}%
\begin{pgfscope}%
\pgfpathrectangle{\pgfqpoint{0.050000in}{0.050000in}}{\pgfqpoint{2.081932in}{2.081932in}}%
\pgfusepath{clip}%
\pgfsetbuttcap%
\pgfsetroundjoin%
\definecolor{currentfill}{rgb}{0.636902,0.856542,0.216620}%
\pgfsetfillcolor{currentfill}%
\pgfsetlinewidth{0.000000pt}%
\definecolor{currentstroke}{rgb}{0.000000,0.000000,0.000000}%
\pgfsetstrokecolor{currentstroke}%
\pgfsetdash{}{0pt}%
\pgfpathmoveto{\pgfqpoint{0.850590in}{1.463621in}}%
\pgfpathlineto{\pgfqpoint{0.854115in}{1.466459in}}%
\pgfpathlineto{\pgfqpoint{0.857939in}{1.468683in}}%
\pgfpathlineto{\pgfqpoint{0.862046in}{1.470282in}}%
\pgfpathlineto{\pgfqpoint{0.866420in}{1.471248in}}%
\pgfpathlineto{\pgfqpoint{0.871046in}{1.471575in}}%
\pgfpathlineto{\pgfqpoint{0.844354in}{1.462231in}}%
\pgfpathlineto{\pgfqpoint{0.818670in}{1.451939in}}%
\pgfpathlineto{\pgfqpoint{0.794093in}{1.440732in}}%
\pgfpathlineto{\pgfqpoint{0.770719in}{1.428648in}}%
\pgfpathlineto{\pgfqpoint{0.748641in}{1.415728in}}%
\pgfpathlineto{\pgfqpoint{0.741683in}{1.414350in}}%
\pgfpathlineto{\pgfqpoint{0.735101in}{1.412385in}}%
\pgfpathlineto{\pgfqpoint{0.728920in}{1.409842in}}%
\pgfpathlineto{\pgfqpoint{0.723165in}{1.406734in}}%
\pgfpathlineto{\pgfqpoint{0.717857in}{1.403073in}}%
\pgfpathlineto{\pgfqpoint{0.741820in}{1.417086in}}%
\pgfpathlineto{\pgfqpoint{0.767177in}{1.430190in}}%
\pgfpathlineto{\pgfqpoint{0.793827in}{1.442339in}}%
\pgfpathlineto{\pgfqpoint{0.821667in}{1.453495in}}%
\pgfpathlineto{\pgfqpoint{0.850590in}{1.463621in}}%
\pgfpathclose%
\pgfusepath{fill}%
\end{pgfscope}%
\begin{pgfscope}%
\pgfpathrectangle{\pgfqpoint{0.050000in}{0.050000in}}{\pgfqpoint{2.081932in}{2.081932in}}%
\pgfusepath{clip}%
\pgfsetbuttcap%
\pgfsetroundjoin%
\definecolor{currentfill}{rgb}{0.267004,0.004874,0.329415}%
\pgfsetfillcolor{currentfill}%
\pgfsetlinewidth{0.000000pt}%
\definecolor{currentstroke}{rgb}{0.000000,0.000000,0.000000}%
\pgfsetstrokecolor{currentstroke}%
\pgfsetdash{}{0pt}%
\pgfpathmoveto{\pgfqpoint{0.871932in}{1.125146in}}%
\pgfpathlineto{\pgfqpoint{0.863312in}{1.127806in}}%
\pgfpathlineto{\pgfqpoint{0.854579in}{1.131025in}}%
\pgfpathlineto{\pgfqpoint{0.845768in}{1.134787in}}%
\pgfpathlineto{\pgfqpoint{0.836915in}{1.139077in}}%
\pgfpathlineto{\pgfqpoint{0.828054in}{1.143876in}}%
\pgfpathlineto{\pgfqpoint{0.811910in}{1.132737in}}%
\pgfpathlineto{\pgfqpoint{0.796939in}{1.121001in}}%
\pgfpathlineto{\pgfqpoint{0.783203in}{1.108712in}}%
\pgfpathlineto{\pgfqpoint{0.770760in}{1.095915in}}%
\pgfpathlineto{\pgfqpoint{0.759665in}{1.082657in}}%
\pgfpathlineto{\pgfqpoint{0.770688in}{1.079699in}}%
\pgfpathlineto{\pgfqpoint{0.781695in}{1.077255in}}%
\pgfpathlineto{\pgfqpoint{0.792645in}{1.075334in}}%
\pgfpathlineto{\pgfqpoint{0.803493in}{1.073945in}}%
\pgfpathlineto{\pgfqpoint{0.814197in}{1.073093in}}%
\pgfpathlineto{\pgfqpoint{0.823536in}{1.084356in}}%
\pgfpathlineto{\pgfqpoint{0.834027in}{1.095233in}}%
\pgfpathlineto{\pgfqpoint{0.845623in}{1.105682in}}%
\pgfpathlineto{\pgfqpoint{0.858276in}{1.115665in}}%
\pgfpathlineto{\pgfqpoint{0.871932in}{1.125146in}}%
\pgfpathclose%
\pgfusepath{fill}%
\end{pgfscope}%
\begin{pgfscope}%
\pgfpathrectangle{\pgfqpoint{0.050000in}{0.050000in}}{\pgfqpoint{2.081932in}{2.081932in}}%
\pgfusepath{clip}%
\pgfsetbuttcap%
\pgfsetroundjoin%
\definecolor{currentfill}{rgb}{0.278012,0.180367,0.486697}%
\pgfsetfillcolor{currentfill}%
\pgfsetlinewidth{0.000000pt}%
\definecolor{currentstroke}{rgb}{0.000000,0.000000,0.000000}%
\pgfsetstrokecolor{currentstroke}%
\pgfsetdash{}{0pt}%
\pgfpathmoveto{\pgfqpoint{1.275815in}{1.141154in}}%
\pgfpathlineto{\pgfqpoint{1.280946in}{1.138319in}}%
\pgfpathlineto{\pgfqpoint{1.286455in}{1.136080in}}%
\pgfpathlineto{\pgfqpoint{1.292317in}{1.134445in}}%
\pgfpathlineto{\pgfqpoint{1.298508in}{1.133419in}}%
\pgfpathlineto{\pgfqpoint{1.305002in}{1.133004in}}%
\pgfpathlineto{\pgfqpoint{1.291564in}{1.139674in}}%
\pgfpathlineto{\pgfqpoint{1.277464in}{1.145826in}}%
\pgfpathlineto{\pgfqpoint{1.262757in}{1.151438in}}%
\pgfpathlineto{\pgfqpoint{1.247502in}{1.156490in}}%
\pgfpathlineto{\pgfqpoint{1.231757in}{1.160963in}}%
\pgfpathlineto{\pgfqpoint{1.227835in}{1.160402in}}%
\pgfpathlineto{\pgfqpoint{1.224096in}{1.160497in}}%
\pgfpathlineto{\pgfqpoint{1.220554in}{1.161247in}}%
\pgfpathlineto{\pgfqpoint{1.217226in}{1.162653in}}%
\pgfpathlineto{\pgfqpoint{1.214124in}{1.164710in}}%
\pgfpathlineto{\pgfqpoint{1.227394in}{1.160940in}}%
\pgfpathlineto{\pgfqpoint{1.240248in}{1.156683in}}%
\pgfpathlineto{\pgfqpoint{1.252635in}{1.151954in}}%
\pgfpathlineto{\pgfqpoint{1.264506in}{1.146771in}}%
\pgfpathlineto{\pgfqpoint{1.275815in}{1.141154in}}%
\pgfpathclose%
\pgfusepath{fill}%
\end{pgfscope}%
\begin{pgfscope}%
\pgfpathrectangle{\pgfqpoint{0.050000in}{0.050000in}}{\pgfqpoint{2.081932in}{2.081932in}}%
\pgfusepath{clip}%
\pgfsetbuttcap%
\pgfsetroundjoin%
\definecolor{currentfill}{rgb}{0.993248,0.906157,0.143936}%
\pgfsetfillcolor{currentfill}%
\pgfsetlinewidth{0.000000pt}%
\definecolor{currentstroke}{rgb}{0.000000,0.000000,0.000000}%
\pgfsetstrokecolor{currentstroke}%
\pgfsetdash{}{0pt}%
\pgfpathmoveto{\pgfqpoint{1.164882in}{1.496041in}}%
\pgfpathlineto{\pgfqpoint{1.163708in}{1.491745in}}%
\pgfpathlineto{\pgfqpoint{1.162512in}{1.486830in}}%
\pgfpathlineto{\pgfqpoint{1.161298in}{1.481314in}}%
\pgfpathlineto{\pgfqpoint{1.160073in}{1.475218in}}%
\pgfpathlineto{\pgfqpoint{1.158840in}{1.468565in}}%
\pgfpathlineto{\pgfqpoint{1.134827in}{1.469552in}}%
\pgfpathlineto{\pgfqpoint{1.110757in}{1.469681in}}%
\pgfpathlineto{\pgfqpoint{1.086717in}{1.468950in}}%
\pgfpathlineto{\pgfqpoint{1.062797in}{1.467362in}}%
\pgfpathlineto{\pgfqpoint{1.039087in}{1.464923in}}%
\pgfpathlineto{\pgfqpoint{1.036597in}{1.471465in}}%
\pgfpathlineto{\pgfqpoint{1.034122in}{1.477451in}}%
\pgfpathlineto{\pgfqpoint{1.031672in}{1.482859in}}%
\pgfpathlineto{\pgfqpoint{1.029256in}{1.487667in}}%
\pgfpathlineto{\pgfqpoint{1.026885in}{1.491858in}}%
\pgfpathlineto{\pgfqpoint{1.054212in}{1.494659in}}%
\pgfpathlineto{\pgfqpoint{1.081776in}{1.496482in}}%
\pgfpathlineto{\pgfqpoint{1.109476in}{1.497321in}}%
\pgfpathlineto{\pgfqpoint{1.137212in}{1.497174in}}%
\pgfpathlineto{\pgfqpoint{1.164882in}{1.496041in}}%
\pgfpathclose%
\pgfusepath{fill}%
\end{pgfscope}%
\begin{pgfscope}%
\pgfpathrectangle{\pgfqpoint{0.050000in}{0.050000in}}{\pgfqpoint{2.081932in}{2.081932in}}%
\pgfusepath{clip}%
\pgfsetbuttcap%
\pgfsetroundjoin%
\definecolor{currentfill}{rgb}{0.278791,0.062145,0.386592}%
\pgfsetfillcolor{currentfill}%
\pgfsetlinewidth{0.000000pt}%
\definecolor{currentstroke}{rgb}{0.000000,0.000000,0.000000}%
\pgfsetstrokecolor{currentstroke}%
\pgfsetdash{}{0pt}%
\pgfpathmoveto{\pgfqpoint{1.360486in}{1.092811in}}%
\pgfpathlineto{\pgfqpoint{1.369324in}{1.091538in}}%
\pgfpathlineto{\pgfqpoint{1.378490in}{1.090822in}}%
\pgfpathlineto{\pgfqpoint{1.387947in}{1.090666in}}%
\pgfpathlineto{\pgfqpoint{1.397657in}{1.091069in}}%
\pgfpathlineto{\pgfqpoint{1.407581in}{1.092030in}}%
\pgfpathlineto{\pgfqpoint{1.396313in}{1.102610in}}%
\pgfpathlineto{\pgfqpoint{1.383973in}{1.112737in}}%
\pgfpathlineto{\pgfqpoint{1.370614in}{1.122371in}}%
\pgfpathlineto{\pgfqpoint{1.356289in}{1.131478in}}%
\pgfpathlineto{\pgfqpoint{1.341057in}{1.140025in}}%
\pgfpathlineto{\pgfqpoint{1.333465in}{1.137428in}}%
\pgfpathlineto{\pgfqpoint{1.326034in}{1.135419in}}%
\pgfpathlineto{\pgfqpoint{1.318794in}{1.134008in}}%
\pgfpathlineto{\pgfqpoint{1.311774in}{1.133202in}}%
\pgfpathlineto{\pgfqpoint{1.305002in}{1.133004in}}%
\pgfpathlineto{\pgfqpoint{1.317726in}{1.125841in}}%
\pgfpathlineto{\pgfqpoint{1.329683in}{1.118211in}}%
\pgfpathlineto{\pgfqpoint{1.340825in}{1.110143in}}%
\pgfpathlineto{\pgfqpoint{1.351107in}{1.101666in}}%
\pgfpathlineto{\pgfqpoint{1.360486in}{1.092811in}}%
\pgfpathclose%
\pgfusepath{fill}%
\end{pgfscope}%
\begin{pgfscope}%
\pgfpathrectangle{\pgfqpoint{0.050000in}{0.050000in}}{\pgfqpoint{2.081932in}{2.081932in}}%
\pgfusepath{clip}%
\pgfsetbuttcap%
\pgfsetroundjoin%
\definecolor{currentfill}{rgb}{0.327796,0.773980,0.406640}%
\pgfsetfillcolor{currentfill}%
\pgfsetlinewidth{0.000000pt}%
\definecolor{currentstroke}{rgb}{0.000000,0.000000,0.000000}%
\pgfsetstrokecolor{currentstroke}%
\pgfsetdash{}{0pt}%
\pgfpathmoveto{\pgfqpoint{1.611109in}{1.322372in}}%
\pgfpathlineto{\pgfqpoint{1.607802in}{1.328925in}}%
\pgfpathlineto{\pgfqpoint{1.603873in}{1.335090in}}%
\pgfpathlineto{\pgfqpoint{1.599336in}{1.340843in}}%
\pgfpathlineto{\pgfqpoint{1.594209in}{1.346159in}}%
\pgfpathlineto{\pgfqpoint{1.588510in}{1.351016in}}%
\pgfpathlineto{\pgfqpoint{1.569809in}{1.367677in}}%
\pgfpathlineto{\pgfqpoint{1.549412in}{1.383597in}}%
\pgfpathlineto{\pgfqpoint{1.527406in}{1.398721in}}%
\pgfpathlineto{\pgfqpoint{1.503882in}{1.412997in}}%
\pgfpathlineto{\pgfqpoint{1.478935in}{1.426377in}}%
\pgfpathlineto{\pgfqpoint{1.483261in}{1.422473in}}%
\pgfpathlineto{\pgfqpoint{1.487153in}{1.418025in}}%
\pgfpathlineto{\pgfqpoint{1.490595in}{1.413050in}}%
\pgfpathlineto{\pgfqpoint{1.493576in}{1.407571in}}%
\pgfpathlineto{\pgfqpoint{1.496085in}{1.401610in}}%
\pgfpathlineto{\pgfqpoint{1.522254in}{1.387548in}}%
\pgfpathlineto{\pgfqpoint{1.546939in}{1.372540in}}%
\pgfpathlineto{\pgfqpoint{1.570039in}{1.356638in}}%
\pgfpathlineto{\pgfqpoint{1.591460in}{1.339896in}}%
\pgfpathlineto{\pgfqpoint{1.611109in}{1.322372in}}%
\pgfpathclose%
\pgfusepath{fill}%
\end{pgfscope}%
\begin{pgfscope}%
\pgfpathrectangle{\pgfqpoint{0.050000in}{0.050000in}}{\pgfqpoint{2.081932in}{2.081932in}}%
\pgfusepath{clip}%
\pgfsetbuttcap%
\pgfsetroundjoin%
\definecolor{currentfill}{rgb}{0.876168,0.891125,0.095250}%
\pgfsetfillcolor{currentfill}%
\pgfsetlinewidth{0.000000pt}%
\definecolor{currentstroke}{rgb}{0.000000,0.000000,0.000000}%
\pgfsetstrokecolor{currentstroke}%
\pgfsetdash{}{0pt}%
\pgfpathmoveto{\pgfqpoint{1.451412in}{1.437215in}}%
\pgfpathlineto{\pgfqpoint{1.444878in}{1.437576in}}%
\pgfpathlineto{\pgfqpoint{1.438056in}{1.437321in}}%
\pgfpathlineto{\pgfqpoint{1.430975in}{1.436450in}}%
\pgfpathlineto{\pgfqpoint{1.423663in}{1.434966in}}%
\pgfpathlineto{\pgfqpoint{1.416148in}{1.432874in}}%
\pgfpathlineto{\pgfqpoint{1.394538in}{1.443136in}}%
\pgfpathlineto{\pgfqpoint{1.371902in}{1.452593in}}%
\pgfpathlineto{\pgfqpoint{1.348328in}{1.461212in}}%
\pgfpathlineto{\pgfqpoint{1.323906in}{1.468964in}}%
\pgfpathlineto{\pgfqpoint{1.298728in}{1.475824in}}%
\pgfpathlineto{\pgfqpoint{1.303247in}{1.478986in}}%
\pgfpathlineto{\pgfqpoint{1.307642in}{1.481513in}}%
\pgfpathlineto{\pgfqpoint{1.311897in}{1.483395in}}%
\pgfpathlineto{\pgfqpoint{1.315995in}{1.484628in}}%
\pgfpathlineto{\pgfqpoint{1.319920in}{1.485208in}}%
\pgfpathlineto{\pgfqpoint{1.348092in}{1.477548in}}%
\pgfpathlineto{\pgfqpoint{1.375429in}{1.468888in}}%
\pgfpathlineto{\pgfqpoint{1.401828in}{1.459257in}}%
\pgfpathlineto{\pgfqpoint{1.427188in}{1.448688in}}%
\pgfpathlineto{\pgfqpoint{1.451412in}{1.437215in}}%
\pgfpathclose%
\pgfusepath{fill}%
\end{pgfscope}%
\begin{pgfscope}%
\pgfpathrectangle{\pgfqpoint{0.050000in}{0.050000in}}{\pgfqpoint{2.081932in}{2.081932in}}%
\pgfusepath{clip}%
\pgfsetbuttcap%
\pgfsetroundjoin%
\definecolor{currentfill}{rgb}{0.124780,0.640461,0.527068}%
\pgfsetfillcolor{currentfill}%
\pgfsetlinewidth{0.000000pt}%
\definecolor{currentstroke}{rgb}{0.000000,0.000000,0.000000}%
\pgfsetstrokecolor{currentstroke}%
\pgfsetdash{}{0pt}%
\pgfpathmoveto{\pgfqpoint{0.692832in}{1.340540in}}%
\pgfpathlineto{\pgfqpoint{0.692890in}{1.348540in}}%
\pgfpathlineto{\pgfqpoint{0.693504in}{1.356228in}}%
\pgfpathlineto{\pgfqpoint{0.694675in}{1.363573in}}%
\pgfpathlineto{\pgfqpoint{0.696397in}{1.370544in}}%
\pgfpathlineto{\pgfqpoint{0.698666in}{1.377115in}}%
\pgfpathlineto{\pgfqpoint{0.675078in}{1.361475in}}%
\pgfpathlineto{\pgfqpoint{0.653141in}{1.344977in}}%
\pgfpathlineto{\pgfqpoint{0.632948in}{1.327680in}}%
\pgfpathlineto{\pgfqpoint{0.614588in}{1.309642in}}%
\pgfpathlineto{\pgfqpoint{0.598142in}{1.290928in}}%
\pgfpathlineto{\pgfqpoint{0.595300in}{1.283816in}}%
\pgfpathlineto{\pgfqpoint{0.593142in}{1.276410in}}%
\pgfpathlineto{\pgfqpoint{0.591675in}{1.268742in}}%
\pgfpathlineto{\pgfqpoint{0.590905in}{1.260841in}}%
\pgfpathlineto{\pgfqpoint{0.590833in}{1.252741in}}%
\pgfpathlineto{\pgfqpoint{0.607527in}{1.271807in}}%
\pgfpathlineto{\pgfqpoint{0.626160in}{1.290184in}}%
\pgfpathlineto{\pgfqpoint{0.646649in}{1.307804in}}%
\pgfpathlineto{\pgfqpoint{0.668904in}{1.324609in}}%
\pgfpathlineto{\pgfqpoint{0.692832in}{1.340540in}}%
\pgfpathclose%
\pgfusepath{fill}%
\end{pgfscope}%
\begin{pgfscope}%
\pgfpathrectangle{\pgfqpoint{0.050000in}{0.050000in}}{\pgfqpoint{2.081932in}{2.081932in}}%
\pgfusepath{clip}%
\pgfsetbuttcap%
\pgfsetroundjoin%
\definecolor{currentfill}{rgb}{0.993248,0.906157,0.143936}%
\pgfsetfillcolor{currentfill}%
\pgfsetlinewidth{0.000000pt}%
\definecolor{currentstroke}{rgb}{0.000000,0.000000,0.000000}%
\pgfsetstrokecolor{currentstroke}%
\pgfsetdash{}{0pt}%
\pgfpathmoveto{\pgfqpoint{1.298728in}{1.475824in}}%
\pgfpathlineto{\pgfqpoint{1.294105in}{1.472038in}}%
\pgfpathlineto{\pgfqpoint{1.289396in}{1.467641in}}%
\pgfpathlineto{\pgfqpoint{1.284620in}{1.462650in}}%
\pgfpathlineto{\pgfqpoint{1.279795in}{1.457085in}}%
\pgfpathlineto{\pgfqpoint{1.274944in}{1.450965in}}%
\pgfpathlineto{\pgfqpoint{1.252542in}{1.456139in}}%
\pgfpathlineto{\pgfqpoint{1.229644in}{1.460499in}}%
\pgfpathlineto{\pgfqpoint{1.206336in}{1.464031in}}%
\pgfpathlineto{\pgfqpoint{1.182706in}{1.466723in}}%
\pgfpathlineto{\pgfqpoint{1.158840in}{1.468565in}}%
\pgfpathlineto{\pgfqpoint{1.160073in}{1.475218in}}%
\pgfpathlineto{\pgfqpoint{1.161298in}{1.481314in}}%
\pgfpathlineto{\pgfqpoint{1.162512in}{1.486830in}}%
\pgfpathlineto{\pgfqpoint{1.163708in}{1.491745in}}%
\pgfpathlineto{\pgfqpoint{1.164882in}{1.496041in}}%
\pgfpathlineto{\pgfqpoint{1.192385in}{1.493925in}}%
\pgfpathlineto{\pgfqpoint{1.219620in}{1.490833in}}%
\pgfpathlineto{\pgfqpoint{1.246489in}{1.486777in}}%
\pgfpathlineto{\pgfqpoint{1.272891in}{1.481768in}}%
\pgfpathlineto{\pgfqpoint{1.298728in}{1.475824in}}%
\pgfpathclose%
\pgfusepath{fill}%
\end{pgfscope}%
\begin{pgfscope}%
\pgfpathrectangle{\pgfqpoint{0.050000in}{0.050000in}}{\pgfqpoint{2.081932in}{2.081932in}}%
\pgfusepath{clip}%
\pgfsetbuttcap%
\pgfsetroundjoin%
\definecolor{currentfill}{rgb}{0.278012,0.180367,0.486697}%
\pgfsetfillcolor{currentfill}%
\pgfsetlinewidth{0.000000pt}%
\definecolor{currentstroke}{rgb}{0.000000,0.000000,0.000000}%
\pgfsetstrokecolor{currentstroke}%
\pgfsetdash{}{0pt}%
\pgfpathmoveto{\pgfqpoint{1.001908in}{1.158003in}}%
\pgfpathlineto{\pgfqpoint{0.998075in}{1.155725in}}%
\pgfpathlineto{\pgfqpoint{0.993961in}{1.154083in}}%
\pgfpathlineto{\pgfqpoint{0.989584in}{1.153081in}}%
\pgfpathlineto{\pgfqpoint{0.984962in}{1.152722in}}%
\pgfpathlineto{\pgfqpoint{0.980114in}{1.153006in}}%
\pgfpathlineto{\pgfqpoint{0.965236in}{1.147559in}}%
\pgfpathlineto{\pgfqpoint{0.950947in}{1.141566in}}%
\pgfpathlineto{\pgfqpoint{0.937303in}{1.135048in}}%
\pgfpathlineto{\pgfqpoint{0.924359in}{1.128029in}}%
\pgfpathlineto{\pgfqpoint{0.912166in}{1.120536in}}%
\pgfpathlineto{\pgfqpoint{0.919410in}{1.121387in}}%
\pgfpathlineto{\pgfqpoint{0.926316in}{1.122831in}}%
\pgfpathlineto{\pgfqpoint{0.932854in}{1.124862in}}%
\pgfpathlineto{\pgfqpoint{0.938996in}{1.127473in}}%
\pgfpathlineto{\pgfqpoint{0.944718in}{1.130656in}}%
\pgfpathlineto{\pgfqpoint{0.954971in}{1.136964in}}%
\pgfpathlineto{\pgfqpoint{0.965860in}{1.142874in}}%
\pgfpathlineto{\pgfqpoint{0.977344in}{1.148364in}}%
\pgfpathlineto{\pgfqpoint{0.989375in}{1.153413in}}%
\pgfpathlineto{\pgfqpoint{1.001908in}{1.158003in}}%
\pgfpathclose%
\pgfusepath{fill}%
\end{pgfscope}%
\begin{pgfscope}%
\pgfpathrectangle{\pgfqpoint{0.050000in}{0.050000in}}{\pgfqpoint{2.081932in}{2.081932in}}%
\pgfusepath{clip}%
\pgfsetbuttcap%
\pgfsetroundjoin%
\definecolor{currentfill}{rgb}{0.227802,0.326594,0.546532}%
\pgfsetfillcolor{currentfill}%
\pgfsetlinewidth{0.000000pt}%
\definecolor{currentstroke}{rgb}{0.000000,0.000000,0.000000}%
\pgfsetstrokecolor{currentstroke}%
\pgfsetdash{}{0pt}%
\pgfpathmoveto{\pgfqpoint{1.140400in}{1.194180in}}%
\pgfpathlineto{\pgfqpoint{1.140855in}{1.189234in}}%
\pgfpathlineto{\pgfqpoint{1.141381in}{1.184913in}}%
\pgfpathlineto{\pgfqpoint{1.141978in}{1.181232in}}%
\pgfpathlineto{\pgfqpoint{1.142642in}{1.178204in}}%
\pgfpathlineto{\pgfqpoint{1.143371in}{1.175841in}}%
\pgfpathlineto{\pgfqpoint{1.128720in}{1.176465in}}%
\pgfpathlineto{\pgfqpoint{1.114034in}{1.176547in}}%
\pgfpathlineto{\pgfqpoint{1.099367in}{1.176084in}}%
\pgfpathlineto{\pgfqpoint{1.084775in}{1.175079in}}%
\pgfpathlineto{\pgfqpoint{1.070316in}{1.173535in}}%
\pgfpathlineto{\pgfqpoint{1.071786in}{1.175969in}}%
\pgfpathlineto{\pgfqpoint{1.073126in}{1.179060in}}%
\pgfpathlineto{\pgfqpoint{1.074331in}{1.182800in}}%
\pgfpathlineto{\pgfqpoint{1.075395in}{1.187172in}}%
\pgfpathlineto{\pgfqpoint{1.076313in}{1.192162in}}%
\pgfpathlineto{\pgfqpoint{1.088997in}{1.193513in}}%
\pgfpathlineto{\pgfqpoint{1.101797in}{1.194393in}}%
\pgfpathlineto{\pgfqpoint{1.114664in}{1.194798in}}%
\pgfpathlineto{\pgfqpoint{1.127548in}{1.194726in}}%
\pgfpathlineto{\pgfqpoint{1.140400in}{1.194180in}}%
\pgfpathclose%
\pgfusepath{fill}%
\end{pgfscope}%
\begin{pgfscope}%
\pgfpathrectangle{\pgfqpoint{0.050000in}{0.050000in}}{\pgfqpoint{2.081932in}{2.081932in}}%
\pgfusepath{clip}%
\pgfsetbuttcap%
\pgfsetroundjoin%
\definecolor{currentfill}{rgb}{0.993248,0.906157,0.143936}%
\pgfsetfillcolor{currentfill}%
\pgfsetlinewidth{0.000000pt}%
\definecolor{currentstroke}{rgb}{0.000000,0.000000,0.000000}%
\pgfsetstrokecolor{currentstroke}%
\pgfsetdash{}{0pt}%
\pgfpathmoveto{\pgfqpoint{1.026885in}{1.491858in}}%
\pgfpathlineto{\pgfqpoint{1.029256in}{1.487667in}}%
\pgfpathlineto{\pgfqpoint{1.031672in}{1.482859in}}%
\pgfpathlineto{\pgfqpoint{1.034122in}{1.477451in}}%
\pgfpathlineto{\pgfqpoint{1.036597in}{1.471465in}}%
\pgfpathlineto{\pgfqpoint{1.039087in}{1.464923in}}%
\pgfpathlineto{\pgfqpoint{1.015672in}{1.461640in}}%
\pgfpathlineto{\pgfqpoint{0.992642in}{1.457526in}}%
\pgfpathlineto{\pgfqpoint{0.970082in}{1.452593in}}%
\pgfpathlineto{\pgfqpoint{0.948076in}{1.446859in}}%
\pgfpathlineto{\pgfqpoint{0.926709in}{1.440344in}}%
\pgfpathlineto{\pgfqpoint{0.920706in}{1.446141in}}%
\pgfpathlineto{\pgfqpoint{0.914738in}{1.451385in}}%
\pgfpathlineto{\pgfqpoint{0.908827in}{1.456058in}}%
\pgfpathlineto{\pgfqpoint{0.903000in}{1.460140in}}%
\pgfpathlineto{\pgfqpoint{0.897278in}{1.463617in}}%
\pgfpathlineto{\pgfqpoint{0.921938in}{1.471106in}}%
\pgfpathlineto{\pgfqpoint{0.947324in}{1.477695in}}%
\pgfpathlineto{\pgfqpoint{0.973342in}{1.483362in}}%
\pgfpathlineto{\pgfqpoint{0.999895in}{1.488088in}}%
\pgfpathlineto{\pgfqpoint{1.026885in}{1.491858in}}%
\pgfpathclose%
\pgfusepath{fill}%
\end{pgfscope}%
\begin{pgfscope}%
\pgfpathrectangle{\pgfqpoint{0.050000in}{0.050000in}}{\pgfqpoint{2.081932in}{2.081932in}}%
\pgfusepath{clip}%
\pgfsetbuttcap%
\pgfsetroundjoin%
\definecolor{currentfill}{rgb}{0.278791,0.062145,0.386592}%
\pgfsetfillcolor{currentfill}%
\pgfsetlinewidth{0.000000pt}%
\definecolor{currentstroke}{rgb}{0.000000,0.000000,0.000000}%
\pgfsetstrokecolor{currentstroke}%
\pgfsetdash{}{0pt}%
\pgfpathmoveto{\pgfqpoint{0.912166in}{1.120536in}}%
\pgfpathlineto{\pgfqpoint{0.904612in}{1.120278in}}%
\pgfpathlineto{\pgfqpoint{0.896779in}{1.120614in}}%
\pgfpathlineto{\pgfqpoint{0.888700in}{1.121541in}}%
\pgfpathlineto{\pgfqpoint{0.880406in}{1.123054in}}%
\pgfpathlineto{\pgfqpoint{0.871932in}{1.125146in}}%
\pgfpathlineto{\pgfqpoint{0.858276in}{1.115665in}}%
\pgfpathlineto{\pgfqpoint{0.845623in}{1.105682in}}%
\pgfpathlineto{\pgfqpoint{0.834027in}{1.095233in}}%
\pgfpathlineto{\pgfqpoint{0.823536in}{1.084356in}}%
\pgfpathlineto{\pgfqpoint{0.814197in}{1.073093in}}%
\pgfpathlineto{\pgfqpoint{0.824714in}{1.072784in}}%
\pgfpathlineto{\pgfqpoint{0.835002in}{1.073019in}}%
\pgfpathlineto{\pgfqpoint{0.845021in}{1.073797in}}%
\pgfpathlineto{\pgfqpoint{0.854730in}{1.075117in}}%
\pgfpathlineto{\pgfqpoint{0.864090in}{1.076975in}}%
\pgfpathlineto{\pgfqpoint{0.871846in}{1.086393in}}%
\pgfpathlineto{\pgfqpoint{0.880571in}{1.095491in}}%
\pgfpathlineto{\pgfqpoint{0.890227in}{1.104236in}}%
\pgfpathlineto{\pgfqpoint{0.900773in}{1.112595in}}%
\pgfpathlineto{\pgfqpoint{0.912166in}{1.120536in}}%
\pgfpathclose%
\pgfusepath{fill}%
\end{pgfscope}%
\begin{pgfscope}%
\pgfpathrectangle{\pgfqpoint{0.050000in}{0.050000in}}{\pgfqpoint{2.081932in}{2.081932in}}%
\pgfusepath{clip}%
\pgfsetbuttcap%
\pgfsetroundjoin%
\definecolor{currentfill}{rgb}{0.282327,0.094955,0.417331}%
\pgfsetfillcolor{currentfill}%
\pgfsetlinewidth{0.000000pt}%
\definecolor{currentstroke}{rgb}{0.000000,0.000000,0.000000}%
\pgfsetstrokecolor{currentstroke}%
\pgfsetdash{}{0pt}%
\pgfpathmoveto{\pgfqpoint{1.563303in}{1.051445in}}%
\pgfpathlineto{\pgfqpoint{1.574353in}{1.055878in}}%
\pgfpathlineto{\pgfqpoint{1.585105in}{1.060743in}}%
\pgfpathlineto{\pgfqpoint{1.595516in}{1.066019in}}%
\pgfpathlineto{\pgfqpoint{1.605547in}{1.071688in}}%
\pgfpathlineto{\pgfqpoint{1.615156in}{1.077725in}}%
\pgfpathlineto{\pgfqpoint{1.606793in}{1.096151in}}%
\pgfpathlineto{\pgfqpoint{1.596526in}{1.114188in}}%
\pgfpathlineto{\pgfqpoint{1.584407in}{1.131768in}}%
\pgfpathlineto{\pgfqpoint{1.570496in}{1.148826in}}%
\pgfpathlineto{\pgfqpoint{1.554857in}{1.165300in}}%
\pgfpathlineto{\pgfqpoint{1.546504in}{1.157646in}}%
\pgfpathlineto{\pgfqpoint{1.537782in}{1.150282in}}%
\pgfpathlineto{\pgfqpoint{1.528724in}{1.143234in}}%
\pgfpathlineto{\pgfqpoint{1.519366in}{1.136533in}}%
\pgfpathlineto{\pgfqpoint{1.509743in}{1.130202in}}%
\pgfpathlineto{\pgfqpoint{1.523678in}{1.115374in}}%
\pgfpathlineto{\pgfqpoint{1.536056in}{1.100026in}}%
\pgfpathlineto{\pgfqpoint{1.546819in}{1.084215in}}%
\pgfpathlineto{\pgfqpoint{1.555917in}{1.068001in}}%
\pgfpathlineto{\pgfqpoint{1.563303in}{1.051445in}}%
\pgfpathclose%
\pgfusepath{fill}%
\end{pgfscope}%
\begin{pgfscope}%
\pgfpathrectangle{\pgfqpoint{0.050000in}{0.050000in}}{\pgfqpoint{2.081932in}{2.081932in}}%
\pgfusepath{clip}%
\pgfsetbuttcap%
\pgfsetroundjoin%
\definecolor{currentfill}{rgb}{0.227802,0.326594,0.546532}%
\pgfsetfillcolor{currentfill}%
\pgfsetlinewidth{0.000000pt}%
\definecolor{currentstroke}{rgb}{0.000000,0.000000,0.000000}%
\pgfsetstrokecolor{currentstroke}%
\pgfsetdash{}{0pt}%
\pgfpathmoveto{\pgfqpoint{1.202456in}{1.184439in}}%
\pgfpathlineto{\pgfqpoint{1.204242in}{1.179279in}}%
\pgfpathlineto{\pgfqpoint{1.206311in}{1.174710in}}%
\pgfpathlineto{\pgfqpoint{1.208655in}{1.170749in}}%
\pgfpathlineto{\pgfqpoint{1.211263in}{1.167412in}}%
\pgfpathlineto{\pgfqpoint{1.214124in}{1.164710in}}%
\pgfpathlineto{\pgfqpoint{1.200490in}{1.167980in}}%
\pgfpathlineto{\pgfqpoint{1.186542in}{1.170737in}}%
\pgfpathlineto{\pgfqpoint{1.172337in}{1.172971in}}%
\pgfpathlineto{\pgfqpoint{1.157928in}{1.174674in}}%
\pgfpathlineto{\pgfqpoint{1.143371in}{1.175841in}}%
\pgfpathlineto{\pgfqpoint{1.142642in}{1.178204in}}%
\pgfpathlineto{\pgfqpoint{1.141978in}{1.181232in}}%
\pgfpathlineto{\pgfqpoint{1.141381in}{1.184913in}}%
\pgfpathlineto{\pgfqpoint{1.140855in}{1.189234in}}%
\pgfpathlineto{\pgfqpoint{1.140400in}{1.194180in}}%
\pgfpathlineto{\pgfqpoint{1.153170in}{1.193159in}}%
\pgfpathlineto{\pgfqpoint{1.165809in}{1.191668in}}%
\pgfpathlineto{\pgfqpoint{1.178268in}{1.189713in}}%
\pgfpathlineto{\pgfqpoint{1.190500in}{1.187300in}}%
\pgfpathlineto{\pgfqpoint{1.202456in}{1.184439in}}%
\pgfpathclose%
\pgfusepath{fill}%
\end{pgfscope}%
\begin{pgfscope}%
\pgfpathrectangle{\pgfqpoint{0.050000in}{0.050000in}}{\pgfqpoint{2.081932in}{2.081932in}}%
\pgfusepath{clip}%
\pgfsetbuttcap%
\pgfsetroundjoin%
\definecolor{currentfill}{rgb}{0.268510,0.009605,0.335427}%
\pgfsetfillcolor{currentfill}%
\pgfsetlinewidth{0.000000pt}%
\definecolor{currentstroke}{rgb}{0.000000,0.000000,0.000000}%
\pgfsetstrokecolor{currentstroke}%
\pgfsetdash{}{0pt}%
\pgfpathmoveto{\pgfqpoint{1.505133in}{1.036290in}}%
\pgfpathlineto{\pgfqpoint{1.517005in}{1.038343in}}%
\pgfpathlineto{\pgfqpoint{1.528804in}{1.040896in}}%
\pgfpathlineto{\pgfqpoint{1.540484in}{1.043939in}}%
\pgfpathlineto{\pgfqpoint{1.551999in}{1.047460in}}%
\pgfpathlineto{\pgfqpoint{1.563303in}{1.051445in}}%
\pgfpathlineto{\pgfqpoint{1.555917in}{1.068001in}}%
\pgfpathlineto{\pgfqpoint{1.546819in}{1.084215in}}%
\pgfpathlineto{\pgfqpoint{1.536056in}{1.100026in}}%
\pgfpathlineto{\pgfqpoint{1.523678in}{1.115374in}}%
\pgfpathlineto{\pgfqpoint{1.509743in}{1.130202in}}%
\pgfpathlineto{\pgfqpoint{1.499895in}{1.124269in}}%
\pgfpathlineto{\pgfqpoint{1.489858in}{1.118755in}}%
\pgfpathlineto{\pgfqpoint{1.479672in}{1.113683in}}%
\pgfpathlineto{\pgfqpoint{1.469377in}{1.109073in}}%
\pgfpathlineto{\pgfqpoint{1.459013in}{1.104944in}}%
\pgfpathlineto{\pgfqpoint{1.471055in}{1.092005in}}%
\pgfpathlineto{\pgfqpoint{1.481734in}{1.078618in}}%
\pgfpathlineto{\pgfqpoint{1.491001in}{1.064836in}}%
\pgfpathlineto{\pgfqpoint{1.498813in}{1.050708in}}%
\pgfpathlineto{\pgfqpoint{1.505133in}{1.036290in}}%
\pgfpathclose%
\pgfusepath{fill}%
\end{pgfscope}%
\begin{pgfscope}%
\pgfpathrectangle{\pgfqpoint{0.050000in}{0.050000in}}{\pgfqpoint{2.081932in}{2.081932in}}%
\pgfusepath{clip}%
\pgfsetbuttcap%
\pgfsetroundjoin%
\definecolor{currentfill}{rgb}{0.267968,0.223549,0.512008}%
\pgfsetfillcolor{currentfill}%
\pgfsetlinewidth{0.000000pt}%
\definecolor{currentstroke}{rgb}{0.000000,0.000000,0.000000}%
\pgfsetstrokecolor{currentstroke}%
\pgfsetdash{}{0pt}%
\pgfpathmoveto{\pgfqpoint{1.615156in}{1.077725in}}%
\pgfpathlineto{\pgfqpoint{1.624307in}{1.084109in}}%
\pgfpathlineto{\pgfqpoint{1.632962in}{1.090814in}}%
\pgfpathlineto{\pgfqpoint{1.641089in}{1.097814in}}%
\pgfpathlineto{\pgfqpoint{1.648653in}{1.105081in}}%
\pgfpathlineto{\pgfqpoint{1.655625in}{1.112588in}}%
\pgfpathlineto{\pgfqpoint{1.646483in}{1.132429in}}%
\pgfpathlineto{\pgfqpoint{1.635288in}{1.151844in}}%
\pgfpathlineto{\pgfqpoint{1.622098in}{1.170761in}}%
\pgfpathlineto{\pgfqpoint{1.606977in}{1.189109in}}%
\pgfpathlineto{\pgfqpoint{1.589997in}{1.206824in}}%
\pgfpathlineto{\pgfqpoint{1.583947in}{1.198192in}}%
\pgfpathlineto{\pgfqpoint{1.577382in}{1.189691in}}%
\pgfpathlineto{\pgfqpoint{1.570326in}{1.181353in}}%
\pgfpathlineto{\pgfqpoint{1.562808in}{1.173212in}}%
\pgfpathlineto{\pgfqpoint{1.554857in}{1.165300in}}%
\pgfpathlineto{\pgfqpoint{1.570496in}{1.148826in}}%
\pgfpathlineto{\pgfqpoint{1.584407in}{1.131768in}}%
\pgfpathlineto{\pgfqpoint{1.596526in}{1.114188in}}%
\pgfpathlineto{\pgfqpoint{1.606793in}{1.096151in}}%
\pgfpathlineto{\pgfqpoint{1.615156in}{1.077725in}}%
\pgfpathclose%
\pgfusepath{fill}%
\end{pgfscope}%
\begin{pgfscope}%
\pgfpathrectangle{\pgfqpoint{0.050000in}{0.050000in}}{\pgfqpoint{2.081932in}{2.081932in}}%
\pgfusepath{clip}%
\pgfsetbuttcap%
\pgfsetroundjoin%
\definecolor{currentfill}{rgb}{0.227802,0.326594,0.546532}%
\pgfsetfillcolor{currentfill}%
\pgfsetlinewidth{0.000000pt}%
\definecolor{currentstroke}{rgb}{0.000000,0.000000,0.000000}%
\pgfsetstrokecolor{currentstroke}%
\pgfsetdash{}{0pt}%
\pgfpathmoveto{\pgfqpoint{1.076313in}{1.192162in}}%
\pgfpathlineto{\pgfqpoint{1.075395in}{1.187172in}}%
\pgfpathlineto{\pgfqpoint{1.074331in}{1.182800in}}%
\pgfpathlineto{\pgfqpoint{1.073126in}{1.179060in}}%
\pgfpathlineto{\pgfqpoint{1.071786in}{1.175969in}}%
\pgfpathlineto{\pgfqpoint{1.070316in}{1.173535in}}%
\pgfpathlineto{\pgfqpoint{1.056043in}{1.171459in}}%
\pgfpathlineto{\pgfqpoint{1.042013in}{1.168857in}}%
\pgfpathlineto{\pgfqpoint{1.028278in}{1.165739in}}%
\pgfpathlineto{\pgfqpoint{1.014892in}{1.162117in}}%
\pgfpathlineto{\pgfqpoint{1.001908in}{1.158003in}}%
\pgfpathlineto{\pgfqpoint{1.005444in}{1.160909in}}%
\pgfpathlineto{\pgfqpoint{1.008667in}{1.164433in}}%
\pgfpathlineto{\pgfqpoint{1.011563in}{1.168562in}}%
\pgfpathlineto{\pgfqpoint{1.014121in}{1.173281in}}%
\pgfpathlineto{\pgfqpoint{1.016327in}{1.178571in}}%
\pgfpathlineto{\pgfqpoint{1.027709in}{1.182170in}}%
\pgfpathlineto{\pgfqpoint{1.039445in}{1.185339in}}%
\pgfpathlineto{\pgfqpoint{1.051489in}{1.188068in}}%
\pgfpathlineto{\pgfqpoint{1.063794in}{1.190345in}}%
\pgfpathlineto{\pgfqpoint{1.076313in}{1.192162in}}%
\pgfpathclose%
\pgfusepath{fill}%
\end{pgfscope}%
\begin{pgfscope}%
\pgfpathrectangle{\pgfqpoint{0.050000in}{0.050000in}}{\pgfqpoint{2.081932in}{2.081932in}}%
\pgfusepath{clip}%
\pgfsetbuttcap%
\pgfsetroundjoin%
\definecolor{currentfill}{rgb}{0.876168,0.891125,0.095250}%
\pgfsetfillcolor{currentfill}%
\pgfsetlinewidth{0.000000pt}%
\definecolor{currentstroke}{rgb}{0.000000,0.000000,0.000000}%
\pgfsetstrokecolor{currentstroke}%
\pgfsetdash{}{0pt}%
\pgfpathmoveto{\pgfqpoint{0.871046in}{1.471575in}}%
\pgfpathlineto{\pgfqpoint{0.875905in}{1.471261in}}%
\pgfpathlineto{\pgfqpoint{0.880978in}{1.470305in}}%
\pgfpathlineto{\pgfqpoint{0.886245in}{1.468708in}}%
\pgfpathlineto{\pgfqpoint{0.891686in}{1.466477in}}%
\pgfpathlineto{\pgfqpoint{0.897278in}{1.463617in}}%
\pgfpathlineto{\pgfqpoint{0.873439in}{1.455254in}}%
\pgfpathlineto{\pgfqpoint{0.850510in}{1.446044in}}%
\pgfpathlineto{\pgfqpoint{0.828582in}{1.436018in}}%
\pgfpathlineto{\pgfqpoint{0.807739in}{1.425212in}}%
\pgfpathlineto{\pgfqpoint{0.788064in}{1.413662in}}%
\pgfpathlineto{\pgfqpoint{0.779665in}{1.415273in}}%
\pgfpathlineto{\pgfqpoint{0.771491in}{1.416286in}}%
\pgfpathlineto{\pgfqpoint{0.763575in}{1.416700in}}%
\pgfpathlineto{\pgfqpoint{0.755948in}{1.416513in}}%
\pgfpathlineto{\pgfqpoint{0.748641in}{1.415728in}}%
\pgfpathlineto{\pgfqpoint{0.770719in}{1.428648in}}%
\pgfpathlineto{\pgfqpoint{0.794093in}{1.440732in}}%
\pgfpathlineto{\pgfqpoint{0.818670in}{1.451939in}}%
\pgfpathlineto{\pgfqpoint{0.844354in}{1.462231in}}%
\pgfpathlineto{\pgfqpoint{0.871046in}{1.471575in}}%
\pgfpathclose%
\pgfusepath{fill}%
\end{pgfscope}%
\begin{pgfscope}%
\pgfpathrectangle{\pgfqpoint{0.050000in}{0.050000in}}{\pgfqpoint{2.081932in}{2.081932in}}%
\pgfusepath{clip}%
\pgfsetbuttcap%
\pgfsetroundjoin%
\definecolor{currentfill}{rgb}{0.278012,0.180367,0.486697}%
\pgfsetfillcolor{currentfill}%
\pgfsetlinewidth{0.000000pt}%
\definecolor{currentstroke}{rgb}{0.000000,0.000000,0.000000}%
\pgfsetstrokecolor{currentstroke}%
\pgfsetdash{}{0pt}%
\pgfpathmoveto{\pgfqpoint{1.322427in}{1.107333in}}%
\pgfpathlineto{\pgfqpoint{1.329114in}{1.103374in}}%
\pgfpathlineto{\pgfqpoint{1.336294in}{1.099930in}}%
\pgfpathlineto{\pgfqpoint{1.343937in}{1.097015in}}%
\pgfpathlineto{\pgfqpoint{1.352012in}{1.094639in}}%
\pgfpathlineto{\pgfqpoint{1.360486in}{1.092811in}}%
\pgfpathlineto{\pgfqpoint{1.351107in}{1.101666in}}%
\pgfpathlineto{\pgfqpoint{1.340825in}{1.110143in}}%
\pgfpathlineto{\pgfqpoint{1.329683in}{1.118211in}}%
\pgfpathlineto{\pgfqpoint{1.317726in}{1.125841in}}%
\pgfpathlineto{\pgfqpoint{1.305002in}{1.133004in}}%
\pgfpathlineto{\pgfqpoint{1.298508in}{1.133419in}}%
\pgfpathlineto{\pgfqpoint{1.292317in}{1.134445in}}%
\pgfpathlineto{\pgfqpoint{1.286455in}{1.136080in}}%
\pgfpathlineto{\pgfqpoint{1.280946in}{1.138319in}}%
\pgfpathlineto{\pgfqpoint{1.275815in}{1.141154in}}%
\pgfpathlineto{\pgfqpoint{1.286517in}{1.135123in}}%
\pgfpathlineto{\pgfqpoint{1.296569in}{1.128700in}}%
\pgfpathlineto{\pgfqpoint{1.305929in}{1.121910in}}%
\pgfpathlineto{\pgfqpoint{1.314561in}{1.114779in}}%
\pgfpathlineto{\pgfqpoint{1.322427in}{1.107333in}}%
\pgfpathclose%
\pgfusepath{fill}%
\end{pgfscope}%
\begin{pgfscope}%
\pgfpathrectangle{\pgfqpoint{0.050000in}{0.050000in}}{\pgfqpoint{2.081932in}{2.081932in}}%
\pgfusepath{clip}%
\pgfsetbuttcap%
\pgfsetroundjoin%
\definecolor{currentfill}{rgb}{0.327796,0.773980,0.406640}%
\pgfsetfillcolor{currentfill}%
\pgfsetlinewidth{0.000000pt}%
\definecolor{currentstroke}{rgb}{0.000000,0.000000,0.000000}%
\pgfsetstrokecolor{currentstroke}%
\pgfsetdash{}{0pt}%
\pgfpathmoveto{\pgfqpoint{0.698666in}{1.377115in}}%
\pgfpathlineto{\pgfqpoint{0.701474in}{1.383258in}}%
\pgfpathlineto{\pgfqpoint{0.704810in}{1.388948in}}%
\pgfpathlineto{\pgfqpoint{0.708662in}{1.394161in}}%
\pgfpathlineto{\pgfqpoint{0.713016in}{1.398876in}}%
\pgfpathlineto{\pgfqpoint{0.717857in}{1.403073in}}%
\pgfpathlineto{\pgfqpoint{0.695383in}{1.388197in}}%
\pgfpathlineto{\pgfqpoint{0.674491in}{1.372508in}}%
\pgfpathlineto{\pgfqpoint{0.655270in}{1.356062in}}%
\pgfpathlineto{\pgfqpoint{0.637802in}{1.338916in}}%
\pgfpathlineto{\pgfqpoint{0.622167in}{1.321130in}}%
\pgfpathlineto{\pgfqpoint{0.616109in}{1.315890in}}%
\pgfpathlineto{\pgfqpoint{0.610659in}{1.310226in}}%
\pgfpathlineto{\pgfqpoint{0.605835in}{1.304161in}}%
\pgfpathlineto{\pgfqpoint{0.601658in}{1.297719in}}%
\pgfpathlineto{\pgfqpoint{0.598142in}{1.290928in}}%
\pgfpathlineto{\pgfqpoint{0.614588in}{1.309642in}}%
\pgfpathlineto{\pgfqpoint{0.632948in}{1.327680in}}%
\pgfpathlineto{\pgfqpoint{0.653141in}{1.344977in}}%
\pgfpathlineto{\pgfqpoint{0.675078in}{1.361475in}}%
\pgfpathlineto{\pgfqpoint{0.698666in}{1.377115in}}%
\pgfpathclose%
\pgfusepath{fill}%
\end{pgfscope}%
\begin{pgfscope}%
\pgfpathrectangle{\pgfqpoint{0.050000in}{0.050000in}}{\pgfqpoint{2.081932in}{2.081932in}}%
\pgfusepath{clip}%
\pgfsetbuttcap%
\pgfsetroundjoin%
\definecolor{currentfill}{rgb}{0.267004,0.004874,0.329415}%
\pgfsetfillcolor{currentfill}%
\pgfsetlinewidth{0.000000pt}%
\definecolor{currentstroke}{rgb}{0.000000,0.000000,0.000000}%
\pgfsetstrokecolor{currentstroke}%
\pgfsetdash{}{0pt}%
\pgfpathmoveto{\pgfqpoint{1.446306in}{1.033747in}}%
\pgfpathlineto{\pgfqpoint{1.457845in}{1.033215in}}%
\pgfpathlineto{\pgfqpoint{1.469543in}{1.033205in}}%
\pgfpathlineto{\pgfqpoint{1.481355in}{1.033716in}}%
\pgfpathlineto{\pgfqpoint{1.493234in}{1.034745in}}%
\pgfpathlineto{\pgfqpoint{1.505133in}{1.036290in}}%
\pgfpathlineto{\pgfqpoint{1.498813in}{1.050708in}}%
\pgfpathlineto{\pgfqpoint{1.491001in}{1.064836in}}%
\pgfpathlineto{\pgfqpoint{1.481734in}{1.078618in}}%
\pgfpathlineto{\pgfqpoint{1.471055in}{1.092005in}}%
\pgfpathlineto{\pgfqpoint{1.459013in}{1.104944in}}%
\pgfpathlineto{\pgfqpoint{1.448620in}{1.101313in}}%
\pgfpathlineto{\pgfqpoint{1.438240in}{1.098193in}}%
\pgfpathlineto{\pgfqpoint{1.427913in}{1.095600in}}%
\pgfpathlineto{\pgfqpoint{1.417679in}{1.093542in}}%
\pgfpathlineto{\pgfqpoint{1.407581in}{1.092030in}}%
\pgfpathlineto{\pgfqpoint{1.417729in}{1.081033in}}%
\pgfpathlineto{\pgfqpoint{1.426713in}{1.069663in}}%
\pgfpathlineto{\pgfqpoint{1.434494in}{1.057962in}}%
\pgfpathlineto{\pgfqpoint{1.441035in}{1.045975in}}%
\pgfpathlineto{\pgfqpoint{1.446306in}{1.033747in}}%
\pgfpathclose%
\pgfusepath{fill}%
\end{pgfscope}%
\begin{pgfscope}%
\pgfpathrectangle{\pgfqpoint{0.050000in}{0.050000in}}{\pgfqpoint{2.081932in}{2.081932in}}%
\pgfusepath{clip}%
\pgfsetbuttcap%
\pgfsetroundjoin%
\definecolor{currentfill}{rgb}{0.636902,0.856542,0.216620}%
\pgfsetfillcolor{currentfill}%
\pgfsetlinewidth{0.000000pt}%
\definecolor{currentstroke}{rgb}{0.000000,0.000000,0.000000}%
\pgfsetstrokecolor{currentstroke}%
\pgfsetdash{}{0pt}%
\pgfpathmoveto{\pgfqpoint{1.588510in}{1.351016in}}%
\pgfpathlineto{\pgfqpoint{1.582262in}{1.355393in}}%
\pgfpathlineto{\pgfqpoint{1.575488in}{1.359272in}}%
\pgfpathlineto{\pgfqpoint{1.568216in}{1.362637in}}%
\pgfpathlineto{\pgfqpoint{1.560472in}{1.365472in}}%
\pgfpathlineto{\pgfqpoint{1.552288in}{1.367766in}}%
\pgfpathlineto{\pgfqpoint{1.535097in}{1.383111in}}%
\pgfpathlineto{\pgfqpoint{1.516333in}{1.397779in}}%
\pgfpathlineto{\pgfqpoint{1.496074in}{1.411717in}}%
\pgfpathlineto{\pgfqpoint{1.474405in}{1.424878in}}%
\pgfpathlineto{\pgfqpoint{1.451412in}{1.437215in}}%
\pgfpathlineto{\pgfqpoint{1.457634in}{1.436243in}}%
\pgfpathlineto{\pgfqpoint{1.463519in}{1.434664in}}%
\pgfpathlineto{\pgfqpoint{1.469045in}{1.432486in}}%
\pgfpathlineto{\pgfqpoint{1.474190in}{1.429720in}}%
\pgfpathlineto{\pgfqpoint{1.478935in}{1.426377in}}%
\pgfpathlineto{\pgfqpoint{1.503882in}{1.412997in}}%
\pgfpathlineto{\pgfqpoint{1.527406in}{1.398721in}}%
\pgfpathlineto{\pgfqpoint{1.549412in}{1.383597in}}%
\pgfpathlineto{\pgfqpoint{1.569809in}{1.367677in}}%
\pgfpathlineto{\pgfqpoint{1.588510in}{1.351016in}}%
\pgfpathclose%
\pgfusepath{fill}%
\end{pgfscope}%
\begin{pgfscope}%
\pgfpathrectangle{\pgfqpoint{0.050000in}{0.050000in}}{\pgfqpoint{2.081932in}{2.081932in}}%
\pgfusepath{clip}%
\pgfsetbuttcap%
\pgfsetroundjoin%
\definecolor{currentfill}{rgb}{0.206756,0.371758,0.553117}%
\pgfsetfillcolor{currentfill}%
\pgfsetlinewidth{0.000000pt}%
\definecolor{currentstroke}{rgb}{0.000000,0.000000,0.000000}%
\pgfsetstrokecolor{currentstroke}%
\pgfsetdash{}{0pt}%
\pgfpathmoveto{\pgfqpoint{1.655625in}{1.112588in}}%
\pgfpathlineto{\pgfqpoint{1.661977in}{1.120305in}}%
\pgfpathlineto{\pgfqpoint{1.667682in}{1.128201in}}%
\pgfpathlineto{\pgfqpoint{1.672719in}{1.136246in}}%
\pgfpathlineto{\pgfqpoint{1.677065in}{1.144408in}}%
\pgfpathlineto{\pgfqpoint{1.680702in}{1.152654in}}%
\pgfpathlineto{\pgfqpoint{1.671070in}{1.173319in}}%
\pgfpathlineto{\pgfqpoint{1.659293in}{1.193535in}}%
\pgfpathlineto{\pgfqpoint{1.645433in}{1.213228in}}%
\pgfpathlineto{\pgfqpoint{1.629557in}{1.232326in}}%
\pgfpathlineto{\pgfqpoint{1.611741in}{1.250760in}}%
\pgfpathlineto{\pgfqpoint{1.608588in}{1.241984in}}%
\pgfpathlineto{\pgfqpoint{1.604821in}{1.233168in}}%
\pgfpathlineto{\pgfqpoint{1.600455in}{1.224346in}}%
\pgfpathlineto{\pgfqpoint{1.595506in}{1.215554in}}%
\pgfpathlineto{\pgfqpoint{1.589997in}{1.206824in}}%
\pgfpathlineto{\pgfqpoint{1.606977in}{1.189109in}}%
\pgfpathlineto{\pgfqpoint{1.622098in}{1.170761in}}%
\pgfpathlineto{\pgfqpoint{1.635288in}{1.151844in}}%
\pgfpathlineto{\pgfqpoint{1.646483in}{1.132429in}}%
\pgfpathlineto{\pgfqpoint{1.655625in}{1.112588in}}%
\pgfpathclose%
\pgfusepath{fill}%
\end{pgfscope}%
\begin{pgfscope}%
\pgfpathrectangle{\pgfqpoint{0.050000in}{0.050000in}}{\pgfqpoint{2.081932in}{2.081932in}}%
\pgfusepath{clip}%
\pgfsetbuttcap%
\pgfsetroundjoin%
\definecolor{currentfill}{rgb}{0.227802,0.326594,0.546532}%
\pgfsetfillcolor{currentfill}%
\pgfsetlinewidth{0.000000pt}%
\definecolor{currentstroke}{rgb}{0.000000,0.000000,0.000000}%
\pgfsetstrokecolor{currentstroke}%
\pgfsetdash{}{0pt}%
\pgfpathmoveto{\pgfqpoint{1.256517in}{1.163835in}}%
\pgfpathlineto{\pgfqpoint{1.259470in}{1.158218in}}%
\pgfpathlineto{\pgfqpoint{1.262892in}{1.153123in}}%
\pgfpathlineto{\pgfqpoint{1.266768in}{1.148570in}}%
\pgfpathlineto{\pgfqpoint{1.271082in}{1.144575in}}%
\pgfpathlineto{\pgfqpoint{1.275815in}{1.141154in}}%
\pgfpathlineto{\pgfqpoint{1.264506in}{1.146771in}}%
\pgfpathlineto{\pgfqpoint{1.252635in}{1.151954in}}%
\pgfpathlineto{\pgfqpoint{1.240248in}{1.156683in}}%
\pgfpathlineto{\pgfqpoint{1.227394in}{1.160940in}}%
\pgfpathlineto{\pgfqpoint{1.214124in}{1.164710in}}%
\pgfpathlineto{\pgfqpoint{1.211263in}{1.167412in}}%
\pgfpathlineto{\pgfqpoint{1.208655in}{1.170749in}}%
\pgfpathlineto{\pgfqpoint{1.206311in}{1.174710in}}%
\pgfpathlineto{\pgfqpoint{1.204242in}{1.179279in}}%
\pgfpathlineto{\pgfqpoint{1.202456in}{1.184439in}}%
\pgfpathlineto{\pgfqpoint{1.214090in}{1.181140in}}%
\pgfpathlineto{\pgfqpoint{1.225356in}{1.177416in}}%
\pgfpathlineto{\pgfqpoint{1.236211in}{1.173279in}}%
\pgfpathlineto{\pgfqpoint{1.246612in}{1.168747in}}%
\pgfpathlineto{\pgfqpoint{1.256517in}{1.163835in}}%
\pgfpathclose%
\pgfusepath{fill}%
\end{pgfscope}%
\begin{pgfscope}%
\pgfpathrectangle{\pgfqpoint{0.050000in}{0.050000in}}{\pgfqpoint{2.081932in}{2.081932in}}%
\pgfusepath{clip}%
\pgfsetbuttcap%
\pgfsetroundjoin%
\definecolor{currentfill}{rgb}{0.993248,0.906157,0.143936}%
\pgfsetfillcolor{currentfill}%
\pgfsetlinewidth{0.000000pt}%
\definecolor{currentstroke}{rgb}{0.000000,0.000000,0.000000}%
\pgfsetstrokecolor{currentstroke}%
\pgfsetdash{}{0pt}%
\pgfpathmoveto{\pgfqpoint{1.158840in}{1.468565in}}%
\pgfpathlineto{\pgfqpoint{1.157605in}{1.461383in}}%
\pgfpathlineto{\pgfqpoint{1.156373in}{1.453698in}}%
\pgfpathlineto{\pgfqpoint{1.155148in}{1.445542in}}%
\pgfpathlineto{\pgfqpoint{1.153937in}{1.436948in}}%
\pgfpathlineto{\pgfqpoint{1.152745in}{1.427949in}}%
\pgfpathlineto{\pgfqpoint{1.132421in}{1.428789in}}%
\pgfpathlineto{\pgfqpoint{1.112048in}{1.428898in}}%
\pgfpathlineto{\pgfqpoint{1.091702in}{1.428276in}}%
\pgfpathlineto{\pgfqpoint{1.071458in}{1.426926in}}%
\pgfpathlineto{\pgfqpoint{1.051394in}{1.424851in}}%
\pgfpathlineto{\pgfqpoint{1.048985in}{1.433742in}}%
\pgfpathlineto{\pgfqpoint{1.046540in}{1.442229in}}%
\pgfpathlineto{\pgfqpoint{1.044069in}{1.450276in}}%
\pgfpathlineto{\pgfqpoint{1.041581in}{1.457850in}}%
\pgfpathlineto{\pgfqpoint{1.039087in}{1.464923in}}%
\pgfpathlineto{\pgfqpoint{1.062797in}{1.467362in}}%
\pgfpathlineto{\pgfqpoint{1.086717in}{1.468950in}}%
\pgfpathlineto{\pgfqpoint{1.110757in}{1.469681in}}%
\pgfpathlineto{\pgfqpoint{1.134827in}{1.469552in}}%
\pgfpathlineto{\pgfqpoint{1.158840in}{1.468565in}}%
\pgfpathclose%
\pgfusepath{fill}%
\end{pgfscope}%
\begin{pgfscope}%
\pgfpathrectangle{\pgfqpoint{0.050000in}{0.050000in}}{\pgfqpoint{2.081932in}{2.081932in}}%
\pgfusepath{clip}%
\pgfsetbuttcap%
\pgfsetroundjoin%
\definecolor{currentfill}{rgb}{0.993248,0.906157,0.143936}%
\pgfsetfillcolor{currentfill}%
\pgfsetlinewidth{0.000000pt}%
\definecolor{currentstroke}{rgb}{0.000000,0.000000,0.000000}%
\pgfsetstrokecolor{currentstroke}%
\pgfsetdash{}{0pt}%
\pgfpathmoveto{\pgfqpoint{1.416148in}{1.432874in}}%
\pgfpathlineto{\pgfqpoint{1.408461in}{1.430179in}}%
\pgfpathlineto{\pgfqpoint{1.400633in}{1.426892in}}%
\pgfpathlineto{\pgfqpoint{1.392695in}{1.423025in}}%
\pgfpathlineto{\pgfqpoint{1.384681in}{1.418593in}}%
\pgfpathlineto{\pgfqpoint{1.376622in}{1.413612in}}%
\pgfpathlineto{\pgfqpoint{1.357930in}{1.422532in}}%
\pgfpathlineto{\pgfqpoint{1.338338in}{1.430755in}}%
\pgfpathlineto{\pgfqpoint{1.317924in}{1.438251in}}%
\pgfpathlineto{\pgfqpoint{1.296765in}{1.444996in}}%
\pgfpathlineto{\pgfqpoint{1.274944in}{1.450965in}}%
\pgfpathlineto{\pgfqpoint{1.279795in}{1.457085in}}%
\pgfpathlineto{\pgfqpoint{1.284620in}{1.462650in}}%
\pgfpathlineto{\pgfqpoint{1.289396in}{1.467641in}}%
\pgfpathlineto{\pgfqpoint{1.294105in}{1.472038in}}%
\pgfpathlineto{\pgfqpoint{1.298728in}{1.475824in}}%
\pgfpathlineto{\pgfqpoint{1.323906in}{1.468964in}}%
\pgfpathlineto{\pgfqpoint{1.348328in}{1.461212in}}%
\pgfpathlineto{\pgfqpoint{1.371902in}{1.452593in}}%
\pgfpathlineto{\pgfqpoint{1.394538in}{1.443136in}}%
\pgfpathlineto{\pgfqpoint{1.416148in}{1.432874in}}%
\pgfpathclose%
\pgfusepath{fill}%
\end{pgfscope}%
\begin{pgfscope}%
\pgfpathrectangle{\pgfqpoint{0.050000in}{0.050000in}}{\pgfqpoint{2.081932in}{2.081932in}}%
\pgfusepath{clip}%
\pgfsetbuttcap%
\pgfsetroundjoin%
\definecolor{currentfill}{rgb}{0.993248,0.906157,0.143936}%
\pgfsetfillcolor{currentfill}%
\pgfsetlinewidth{0.000000pt}%
\definecolor{currentstroke}{rgb}{0.000000,0.000000,0.000000}%
\pgfsetstrokecolor{currentstroke}%
\pgfsetdash{}{0pt}%
\pgfpathmoveto{\pgfqpoint{1.274944in}{1.450965in}}%
\pgfpathlineto{\pgfqpoint{1.270084in}{1.444316in}}%
\pgfpathlineto{\pgfqpoint{1.265236in}{1.437163in}}%
\pgfpathlineto{\pgfqpoint{1.260420in}{1.429536in}}%
\pgfpathlineto{\pgfqpoint{1.255657in}{1.421464in}}%
\pgfpathlineto{\pgfqpoint{1.250966in}{1.412981in}}%
\pgfpathlineto{\pgfqpoint{1.232024in}{1.417380in}}%
\pgfpathlineto{\pgfqpoint{1.212657in}{1.421088in}}%
\pgfpathlineto{\pgfqpoint{1.192938in}{1.424092in}}%
\pgfpathlineto{\pgfqpoint{1.172942in}{1.426382in}}%
\pgfpathlineto{\pgfqpoint{1.152745in}{1.427949in}}%
\pgfpathlineto{\pgfqpoint{1.153937in}{1.436948in}}%
\pgfpathlineto{\pgfqpoint{1.155148in}{1.445542in}}%
\pgfpathlineto{\pgfqpoint{1.156373in}{1.453698in}}%
\pgfpathlineto{\pgfqpoint{1.157605in}{1.461383in}}%
\pgfpathlineto{\pgfqpoint{1.158840in}{1.468565in}}%
\pgfpathlineto{\pgfqpoint{1.182706in}{1.466723in}}%
\pgfpathlineto{\pgfqpoint{1.206336in}{1.464031in}}%
\pgfpathlineto{\pgfqpoint{1.229644in}{1.460499in}}%
\pgfpathlineto{\pgfqpoint{1.252542in}{1.456139in}}%
\pgfpathlineto{\pgfqpoint{1.274944in}{1.450965in}}%
\pgfpathclose%
\pgfusepath{fill}%
\end{pgfscope}%
\begin{pgfscope}%
\pgfpathrectangle{\pgfqpoint{0.050000in}{0.050000in}}{\pgfqpoint{2.081932in}{2.081932in}}%
\pgfusepath{clip}%
\pgfsetbuttcap%
\pgfsetroundjoin%
\definecolor{currentfill}{rgb}{0.227802,0.326594,0.546532}%
\pgfsetfillcolor{currentfill}%
\pgfsetlinewidth{0.000000pt}%
\definecolor{currentstroke}{rgb}{0.000000,0.000000,0.000000}%
\pgfsetstrokecolor{currentstroke}%
\pgfsetdash{}{0pt}%
\pgfpathmoveto{\pgfqpoint{1.016327in}{1.178571in}}%
\pgfpathlineto{\pgfqpoint{1.014121in}{1.173281in}}%
\pgfpathlineto{\pgfqpoint{1.011563in}{1.168562in}}%
\pgfpathlineto{\pgfqpoint{1.008667in}{1.164433in}}%
\pgfpathlineto{\pgfqpoint{1.005444in}{1.160909in}}%
\pgfpathlineto{\pgfqpoint{1.001908in}{1.158003in}}%
\pgfpathlineto{\pgfqpoint{0.989375in}{1.153413in}}%
\pgfpathlineto{\pgfqpoint{0.977344in}{1.148364in}}%
\pgfpathlineto{\pgfqpoint{0.965860in}{1.142874in}}%
\pgfpathlineto{\pgfqpoint{0.954971in}{1.136964in}}%
\pgfpathlineto{\pgfqpoint{0.944718in}{1.130656in}}%
\pgfpathlineto{\pgfqpoint{0.949996in}{1.134399in}}%
\pgfpathlineto{\pgfqpoint{0.954805in}{1.138688in}}%
\pgfpathlineto{\pgfqpoint{0.959127in}{1.143507in}}%
\pgfpathlineto{\pgfqpoint{0.962942in}{1.148837in}}%
\pgfpathlineto{\pgfqpoint{0.966234in}{1.154657in}}%
\pgfpathlineto{\pgfqpoint{0.975207in}{1.160172in}}%
\pgfpathlineto{\pgfqpoint{0.984743in}{1.165339in}}%
\pgfpathlineto{\pgfqpoint{0.994802in}{1.170140in}}%
\pgfpathlineto{\pgfqpoint{1.005344in}{1.174556in}}%
\pgfpathlineto{\pgfqpoint{1.016327in}{1.178571in}}%
\pgfpathclose%
\pgfusepath{fill}%
\end{pgfscope}%
\begin{pgfscope}%
\pgfpathrectangle{\pgfqpoint{0.050000in}{0.050000in}}{\pgfqpoint{2.081932in}{2.081932in}}%
\pgfusepath{clip}%
\pgfsetbuttcap%
\pgfsetroundjoin%
\definecolor{currentfill}{rgb}{0.278012,0.180367,0.486697}%
\pgfsetfillcolor{currentfill}%
\pgfsetlinewidth{0.000000pt}%
\definecolor{currentstroke}{rgb}{0.000000,0.000000,0.000000}%
\pgfsetstrokecolor{currentstroke}%
\pgfsetdash{}{0pt}%
\pgfpathmoveto{\pgfqpoint{0.944718in}{1.130656in}}%
\pgfpathlineto{\pgfqpoint{0.938996in}{1.127473in}}%
\pgfpathlineto{\pgfqpoint{0.932854in}{1.124862in}}%
\pgfpathlineto{\pgfqpoint{0.926316in}{1.122831in}}%
\pgfpathlineto{\pgfqpoint{0.919410in}{1.121387in}}%
\pgfpathlineto{\pgfqpoint{0.912166in}{1.120536in}}%
\pgfpathlineto{\pgfqpoint{0.900773in}{1.112595in}}%
\pgfpathlineto{\pgfqpoint{0.890227in}{1.104236in}}%
\pgfpathlineto{\pgfqpoint{0.880571in}{1.095491in}}%
\pgfpathlineto{\pgfqpoint{0.871846in}{1.086393in}}%
\pgfpathlineto{\pgfqpoint{0.864090in}{1.076975in}}%
\pgfpathlineto{\pgfqpoint{0.873063in}{1.079363in}}%
\pgfpathlineto{\pgfqpoint{0.881613in}{1.082274in}}%
\pgfpathlineto{\pgfqpoint{0.889705in}{1.085696in}}%
\pgfpathlineto{\pgfqpoint{0.897306in}{1.089617in}}%
\pgfpathlineto{\pgfqpoint{0.904383in}{1.094022in}}%
\pgfpathlineto{\pgfqpoint{0.910877in}{1.101937in}}%
\pgfpathlineto{\pgfqpoint{0.918189in}{1.109587in}}%
\pgfpathlineto{\pgfqpoint{0.926290in}{1.116942in}}%
\pgfpathlineto{\pgfqpoint{0.935145in}{1.123974in}}%
\pgfpathlineto{\pgfqpoint{0.944718in}{1.130656in}}%
\pgfpathclose%
\pgfusepath{fill}%
\end{pgfscope}%
\begin{pgfscope}%
\pgfpathrectangle{\pgfqpoint{0.050000in}{0.050000in}}{\pgfqpoint{2.081932in}{2.081932in}}%
\pgfusepath{clip}%
\pgfsetbuttcap%
\pgfsetroundjoin%
\definecolor{currentfill}{rgb}{0.162142,0.474838,0.558140}%
\pgfsetfillcolor{currentfill}%
\pgfsetlinewidth{0.000000pt}%
\definecolor{currentstroke}{rgb}{0.000000,0.000000,0.000000}%
\pgfsetstrokecolor{currentstroke}%
\pgfsetdash{}{0pt}%
\pgfpathmoveto{\pgfqpoint{1.139270in}{1.227507in}}%
\pgfpathlineto{\pgfqpoint{1.139339in}{1.219780in}}%
\pgfpathlineto{\pgfqpoint{1.139488in}{1.212556in}}%
\pgfpathlineto{\pgfqpoint{1.139715in}{1.205863in}}%
\pgfpathlineto{\pgfqpoint{1.140019in}{1.199730in}}%
\pgfpathlineto{\pgfqpoint{1.140400in}{1.194180in}}%
\pgfpathlineto{\pgfqpoint{1.127548in}{1.194726in}}%
\pgfpathlineto{\pgfqpoint{1.114664in}{1.194798in}}%
\pgfpathlineto{\pgfqpoint{1.101797in}{1.194393in}}%
\pgfpathlineto{\pgfqpoint{1.088997in}{1.193513in}}%
\pgfpathlineto{\pgfqpoint{1.076313in}{1.192162in}}%
\pgfpathlineto{\pgfqpoint{1.077081in}{1.197750in}}%
\pgfpathlineto{\pgfqpoint{1.077695in}{1.203914in}}%
\pgfpathlineto{\pgfqpoint{1.078154in}{1.210629in}}%
\pgfpathlineto{\pgfqpoint{1.078454in}{1.217869in}}%
\pgfpathlineto{\pgfqpoint{1.078593in}{1.225605in}}%
\pgfpathlineto{\pgfqpoint{1.090602in}{1.226879in}}%
\pgfpathlineto{\pgfqpoint{1.102721in}{1.227708in}}%
\pgfpathlineto{\pgfqpoint{1.114903in}{1.228090in}}%
\pgfpathlineto{\pgfqpoint{1.127102in}{1.228023in}}%
\pgfpathlineto{\pgfqpoint{1.139270in}{1.227507in}}%
\pgfpathclose%
\pgfusepath{fill}%
\end{pgfscope}%
\begin{pgfscope}%
\pgfpathrectangle{\pgfqpoint{0.050000in}{0.050000in}}{\pgfqpoint{2.081932in}{2.081932in}}%
\pgfusepath{clip}%
\pgfsetbuttcap%
\pgfsetroundjoin%
\definecolor{currentfill}{rgb}{0.282327,0.094955,0.417331}%
\pgfsetfillcolor{currentfill}%
\pgfsetlinewidth{0.000000pt}%
\definecolor{currentstroke}{rgb}{0.000000,0.000000,0.000000}%
\pgfsetstrokecolor{currentstroke}%
\pgfsetdash{}{0pt}%
\pgfpathmoveto{\pgfqpoint{0.705834in}{1.104657in}}%
\pgfpathlineto{\pgfqpoint{0.695618in}{1.110376in}}%
\pgfpathlineto{\pgfqpoint{0.685682in}{1.116486in}}%
\pgfpathlineto{\pgfqpoint{0.676063in}{1.122964in}}%
\pgfpathlineto{\pgfqpoint{0.666798in}{1.129782in}}%
\pgfpathlineto{\pgfqpoint{0.657925in}{1.136916in}}%
\pgfpathlineto{\pgfqpoint{0.645260in}{1.119485in}}%
\pgfpathlineto{\pgfqpoint{0.634429in}{1.101578in}}%
\pgfpathlineto{\pgfqpoint{0.625487in}{1.083261in}}%
\pgfpathlineto{\pgfqpoint{0.618481in}{1.064603in}}%
\pgfpathlineto{\pgfqpoint{0.613452in}{1.045677in}}%
\pgfpathlineto{\pgfqpoint{0.623288in}{1.040239in}}%
\pgfpathlineto{\pgfqpoint{0.633552in}{1.035201in}}%
\pgfpathlineto{\pgfqpoint{0.644205in}{1.030581in}}%
\pgfpathlineto{\pgfqpoint{0.655204in}{1.026398in}}%
\pgfpathlineto{\pgfqpoint{0.666506in}{1.022667in}}%
\pgfpathlineto{\pgfqpoint{0.670899in}{1.039659in}}%
\pgfpathlineto{\pgfqpoint{0.677066in}{1.056418in}}%
\pgfpathlineto{\pgfqpoint{0.684973in}{1.072879in}}%
\pgfpathlineto{\pgfqpoint{0.694578in}{1.088979in}}%
\pgfpathlineto{\pgfqpoint{0.705834in}{1.104657in}}%
\pgfpathclose%
\pgfusepath{fill}%
\end{pgfscope}%
\begin{pgfscope}%
\pgfpathrectangle{\pgfqpoint{0.050000in}{0.050000in}}{\pgfqpoint{2.081932in}{2.081932in}}%
\pgfusepath{clip}%
\pgfsetbuttcap%
\pgfsetroundjoin%
\definecolor{currentfill}{rgb}{0.268510,0.009605,0.335427}%
\pgfsetfillcolor{currentfill}%
\pgfsetlinewidth{0.000000pt}%
\definecolor{currentstroke}{rgb}{0.000000,0.000000,0.000000}%
\pgfsetstrokecolor{currentstroke}%
\pgfsetdash{}{0pt}%
\pgfpathmoveto{\pgfqpoint{0.759665in}{1.082657in}}%
\pgfpathlineto{\pgfqpoint{0.748671in}{1.086116in}}%
\pgfpathlineto{\pgfqpoint{0.737748in}{1.090062in}}%
\pgfpathlineto{\pgfqpoint{0.726940in}{1.094480in}}%
\pgfpathlineto{\pgfqpoint{0.716288in}{1.099351in}}%
\pgfpathlineto{\pgfqpoint{0.705834in}{1.104657in}}%
\pgfpathlineto{\pgfqpoint{0.694578in}{1.088979in}}%
\pgfpathlineto{\pgfqpoint{0.684973in}{1.072879in}}%
\pgfpathlineto{\pgfqpoint{0.677066in}{1.056418in}}%
\pgfpathlineto{\pgfqpoint{0.670899in}{1.039659in}}%
\pgfpathlineto{\pgfqpoint{0.666506in}{1.022667in}}%
\pgfpathlineto{\pgfqpoint{0.678066in}{1.019403in}}%
\pgfpathlineto{\pgfqpoint{0.689840in}{1.016620in}}%
\pgfpathlineto{\pgfqpoint{0.701780in}{1.014327in}}%
\pgfpathlineto{\pgfqpoint{0.713840in}{1.012533in}}%
\pgfpathlineto{\pgfqpoint{0.725972in}{1.011247in}}%
\pgfpathlineto{\pgfqpoint{0.729683in}{1.026031in}}%
\pgfpathlineto{\pgfqpoint{0.734939in}{1.040620in}}%
\pgfpathlineto{\pgfqpoint{0.741712in}{1.054957in}}%
\pgfpathlineto{\pgfqpoint{0.749967in}{1.068987in}}%
\pgfpathlineto{\pgfqpoint{0.759665in}{1.082657in}}%
\pgfpathclose%
\pgfusepath{fill}%
\end{pgfscope}%
\begin{pgfscope}%
\pgfpathrectangle{\pgfqpoint{0.050000in}{0.050000in}}{\pgfqpoint{2.081932in}{2.081932in}}%
\pgfusepath{clip}%
\pgfsetbuttcap%
\pgfsetroundjoin%
\definecolor{currentfill}{rgb}{0.150476,0.504369,0.557430}%
\pgfsetfillcolor{currentfill}%
\pgfsetlinewidth{0.000000pt}%
\definecolor{currentstroke}{rgb}{0.000000,0.000000,0.000000}%
\pgfsetstrokecolor{currentstroke}%
\pgfsetdash{}{0pt}%
\pgfpathmoveto{\pgfqpoint{1.680702in}{1.152654in}}%
\pgfpathlineto{\pgfqpoint{1.683616in}{1.160953in}}%
\pgfpathlineto{\pgfqpoint{1.685794in}{1.169271in}}%
\pgfpathlineto{\pgfqpoint{1.687226in}{1.177575in}}%
\pgfpathlineto{\pgfqpoint{1.687904in}{1.185832in}}%
\pgfpathlineto{\pgfqpoint{1.678131in}{1.206676in}}%
\pgfpathlineto{\pgfqpoint{1.666186in}{1.227066in}}%
\pgfpathlineto{\pgfqpoint{1.652132in}{1.246927in}}%
\pgfpathlineto{\pgfqpoint{1.636038in}{1.266187in}}%
\pgfpathlineto{\pgfqpoint{1.617982in}{1.284776in}}%
\pgfpathlineto{\pgfqpoint{1.617394in}{1.276503in}}%
\pgfpathlineto{\pgfqpoint{1.616153in}{1.268053in}}%
\pgfpathlineto{\pgfqpoint{1.614266in}{1.259461in}}%
\pgfpathlineto{\pgfqpoint{1.611741in}{1.250760in}}%
\pgfpathlineto{\pgfqpoint{1.629557in}{1.232326in}}%
\pgfpathlineto{\pgfqpoint{1.645433in}{1.213228in}}%
\pgfpathlineto{\pgfqpoint{1.659293in}{1.193535in}}%
\pgfpathlineto{\pgfqpoint{1.671070in}{1.173319in}}%
\pgfpathlineto{\pgfqpoint{1.680702in}{1.152654in}}%
\pgfpathclose%
\pgfusepath{fill}%
\end{pgfscope}%
\begin{pgfscope}%
\pgfpathrectangle{\pgfqpoint{0.050000in}{0.050000in}}{\pgfqpoint{2.081932in}{2.081932in}}%
\pgfusepath{clip}%
\pgfsetbuttcap%
\pgfsetroundjoin%
\definecolor{currentfill}{rgb}{0.993248,0.906157,0.143936}%
\pgfsetfillcolor{currentfill}%
\pgfsetlinewidth{0.000000pt}%
\definecolor{currentstroke}{rgb}{0.000000,0.000000,0.000000}%
\pgfsetstrokecolor{currentstroke}%
\pgfsetdash{}{0pt}%
\pgfpathmoveto{\pgfqpoint{1.039087in}{1.464923in}}%
\pgfpathlineto{\pgfqpoint{1.041581in}{1.457850in}}%
\pgfpathlineto{\pgfqpoint{1.044069in}{1.450276in}}%
\pgfpathlineto{\pgfqpoint{1.046540in}{1.442229in}}%
\pgfpathlineto{\pgfqpoint{1.048985in}{1.433742in}}%
\pgfpathlineto{\pgfqpoint{1.051394in}{1.424851in}}%
\pgfpathlineto{\pgfqpoint{1.031583in}{1.422059in}}%
\pgfpathlineto{\pgfqpoint{1.012102in}{1.418559in}}%
\pgfpathlineto{\pgfqpoint{0.993024in}{1.414365in}}%
\pgfpathlineto{\pgfqpoint{0.974422in}{1.409491in}}%
\pgfpathlineto{\pgfqpoint{0.956366in}{1.403953in}}%
\pgfpathlineto{\pgfqpoint{0.950565in}{1.412124in}}%
\pgfpathlineto{\pgfqpoint{0.944674in}{1.419880in}}%
\pgfpathlineto{\pgfqpoint{0.938718in}{1.427187in}}%
\pgfpathlineto{\pgfqpoint{0.932721in}{1.434018in}}%
\pgfpathlineto{\pgfqpoint{0.926709in}{1.440344in}}%
\pgfpathlineto{\pgfqpoint{0.948076in}{1.446859in}}%
\pgfpathlineto{\pgfqpoint{0.970082in}{1.452593in}}%
\pgfpathlineto{\pgfqpoint{0.992642in}{1.457526in}}%
\pgfpathlineto{\pgfqpoint{1.015672in}{1.461640in}}%
\pgfpathlineto{\pgfqpoint{1.039087in}{1.464923in}}%
\pgfpathclose%
\pgfusepath{fill}%
\end{pgfscope}%
\begin{pgfscope}%
\pgfpathrectangle{\pgfqpoint{0.050000in}{0.050000in}}{\pgfqpoint{2.081932in}{2.081932in}}%
\pgfusepath{clip}%
\pgfsetbuttcap%
\pgfsetroundjoin%
\definecolor{currentfill}{rgb}{0.162142,0.474838,0.558140}%
\pgfsetfillcolor{currentfill}%
\pgfsetlinewidth{0.000000pt}%
\definecolor{currentstroke}{rgb}{0.000000,0.000000,0.000000}%
\pgfsetstrokecolor{currentstroke}%
\pgfsetdash{}{0pt}%
\pgfpathmoveto{\pgfqpoint{1.198019in}{1.218324in}}%
\pgfpathlineto{\pgfqpoint{1.198291in}{1.210556in}}%
\pgfpathlineto{\pgfqpoint{1.198874in}{1.203255in}}%
\pgfpathlineto{\pgfqpoint{1.199766in}{1.196451in}}%
\pgfpathlineto{\pgfqpoint{1.200961in}{1.190171in}}%
\pgfpathlineto{\pgfqpoint{1.202456in}{1.184439in}}%
\pgfpathlineto{\pgfqpoint{1.190500in}{1.187300in}}%
\pgfpathlineto{\pgfqpoint{1.178268in}{1.189713in}}%
\pgfpathlineto{\pgfqpoint{1.165809in}{1.191668in}}%
\pgfpathlineto{\pgfqpoint{1.153170in}{1.193159in}}%
\pgfpathlineto{\pgfqpoint{1.140400in}{1.194180in}}%
\pgfpathlineto{\pgfqpoint{1.140019in}{1.199730in}}%
\pgfpathlineto{\pgfqpoint{1.139715in}{1.205863in}}%
\pgfpathlineto{\pgfqpoint{1.139488in}{1.212556in}}%
\pgfpathlineto{\pgfqpoint{1.139339in}{1.219780in}}%
\pgfpathlineto{\pgfqpoint{1.139270in}{1.227507in}}%
\pgfpathlineto{\pgfqpoint{1.151361in}{1.226545in}}%
\pgfpathlineto{\pgfqpoint{1.163327in}{1.225139in}}%
\pgfpathlineto{\pgfqpoint{1.175122in}{1.223296in}}%
\pgfpathlineto{\pgfqpoint{1.186702in}{1.221021in}}%
\pgfpathlineto{\pgfqpoint{1.198019in}{1.218324in}}%
\pgfpathclose%
\pgfusepath{fill}%
\end{pgfscope}%
\begin{pgfscope}%
\pgfpathrectangle{\pgfqpoint{0.050000in}{0.050000in}}{\pgfqpoint{2.081932in}{2.081932in}}%
\pgfusepath{clip}%
\pgfsetbuttcap%
\pgfsetroundjoin%
\definecolor{currentfill}{rgb}{0.278791,0.062145,0.386592}%
\pgfsetfillcolor{currentfill}%
\pgfsetlinewidth{0.000000pt}%
\definecolor{currentstroke}{rgb}{0.000000,0.000000,0.000000}%
\pgfsetstrokecolor{currentstroke}%
\pgfsetdash{}{0pt}%
\pgfpathmoveto{\pgfqpoint{1.392571in}{1.044107in}}%
\pgfpathlineto{\pgfqpoint{1.402645in}{1.041031in}}%
\pgfpathlineto{\pgfqpoint{1.413098in}{1.038448in}}%
\pgfpathlineto{\pgfqpoint{1.423888in}{1.036368in}}%
\pgfpathlineto{\pgfqpoint{1.434972in}{1.034800in}}%
\pgfpathlineto{\pgfqpoint{1.446306in}{1.033747in}}%
\pgfpathlineto{\pgfqpoint{1.441035in}{1.045975in}}%
\pgfpathlineto{\pgfqpoint{1.434494in}{1.057962in}}%
\pgfpathlineto{\pgfqpoint{1.426713in}{1.069663in}}%
\pgfpathlineto{\pgfqpoint{1.417729in}{1.081033in}}%
\pgfpathlineto{\pgfqpoint{1.407581in}{1.092030in}}%
\pgfpathlineto{\pgfqpoint{1.397657in}{1.091069in}}%
\pgfpathlineto{\pgfqpoint{1.387947in}{1.090666in}}%
\pgfpathlineto{\pgfqpoint{1.378490in}{1.090822in}}%
\pgfpathlineto{\pgfqpoint{1.369324in}{1.091538in}}%
\pgfpathlineto{\pgfqpoint{1.360486in}{1.092811in}}%
\pgfpathlineto{\pgfqpoint{1.368923in}{1.083614in}}%
\pgfpathlineto{\pgfqpoint{1.376380in}{1.074107in}}%
\pgfpathlineto{\pgfqpoint{1.382826in}{1.064329in}}%
\pgfpathlineto{\pgfqpoint{1.388231in}{1.054316in}}%
\pgfpathlineto{\pgfqpoint{1.392571in}{1.044107in}}%
\pgfpathclose%
\pgfusepath{fill}%
\end{pgfscope}%
\begin{pgfscope}%
\pgfpathrectangle{\pgfqpoint{0.050000in}{0.050000in}}{\pgfqpoint{2.081932in}{2.081932in}}%
\pgfusepath{clip}%
\pgfsetbuttcap%
\pgfsetroundjoin%
\definecolor{currentfill}{rgb}{0.162142,0.474838,0.558140}%
\pgfsetfillcolor{currentfill}%
\pgfsetlinewidth{0.000000pt}%
\definecolor{currentstroke}{rgb}{0.000000,0.000000,0.000000}%
\pgfsetstrokecolor{currentstroke}%
\pgfsetdash{}{0pt}%
\pgfpathmoveto{\pgfqpoint{1.078593in}{1.225605in}}%
\pgfpathlineto{\pgfqpoint{1.078454in}{1.217869in}}%
\pgfpathlineto{\pgfqpoint{1.078154in}{1.210629in}}%
\pgfpathlineto{\pgfqpoint{1.077695in}{1.203914in}}%
\pgfpathlineto{\pgfqpoint{1.077081in}{1.197750in}}%
\pgfpathlineto{\pgfqpoint{1.076313in}{1.192162in}}%
\pgfpathlineto{\pgfqpoint{1.063794in}{1.190345in}}%
\pgfpathlineto{\pgfqpoint{1.051489in}{1.188068in}}%
\pgfpathlineto{\pgfqpoint{1.039445in}{1.185339in}}%
\pgfpathlineto{\pgfqpoint{1.027709in}{1.182170in}}%
\pgfpathlineto{\pgfqpoint{1.016327in}{1.178571in}}%
\pgfpathlineto{\pgfqpoint{1.018174in}{1.184412in}}%
\pgfpathlineto{\pgfqpoint{1.019651in}{1.190781in}}%
\pgfpathlineto{\pgfqpoint{1.020753in}{1.197652in}}%
\pgfpathlineto{\pgfqpoint{1.021473in}{1.204999in}}%
\pgfpathlineto{\pgfqpoint{1.021809in}{1.212792in}}%
\pgfpathlineto{\pgfqpoint{1.032582in}{1.216184in}}%
\pgfpathlineto{\pgfqpoint{1.043691in}{1.219172in}}%
\pgfpathlineto{\pgfqpoint{1.055092in}{1.221745in}}%
\pgfpathlineto{\pgfqpoint{1.066741in}{1.223892in}}%
\pgfpathlineto{\pgfqpoint{1.078593in}{1.225605in}}%
\pgfpathclose%
\pgfusepath{fill}%
\end{pgfscope}%
\begin{pgfscope}%
\pgfpathrectangle{\pgfqpoint{0.050000in}{0.050000in}}{\pgfqpoint{2.081932in}{2.081932in}}%
\pgfusepath{clip}%
\pgfsetbuttcap%
\pgfsetroundjoin%
\definecolor{currentfill}{rgb}{0.267968,0.223549,0.512008}%
\pgfsetfillcolor{currentfill}%
\pgfsetlinewidth{0.000000pt}%
\definecolor{currentstroke}{rgb}{0.000000,0.000000,0.000000}%
\pgfsetstrokecolor{currentstroke}%
\pgfsetdash{}{0pt}%
\pgfpathmoveto{\pgfqpoint{0.657925in}{1.136916in}}%
\pgfpathlineto{\pgfqpoint{0.649478in}{1.144337in}}%
\pgfpathlineto{\pgfqpoint{0.641490in}{1.152016in}}%
\pgfpathlineto{\pgfqpoint{0.633993in}{1.159924in}}%
\pgfpathlineto{\pgfqpoint{0.627015in}{1.168029in}}%
\pgfpathlineto{\pgfqpoint{0.620585in}{1.176299in}}%
\pgfpathlineto{\pgfqpoint{0.606806in}{1.157545in}}%
\pgfpathlineto{\pgfqpoint{0.595003in}{1.138271in}}%
\pgfpathlineto{\pgfqpoint{0.585236in}{1.118550in}}%
\pgfpathlineto{\pgfqpoint{0.577559in}{1.098454in}}%
\pgfpathlineto{\pgfqpoint{0.572017in}{1.078061in}}%
\pgfpathlineto{\pgfqpoint{0.579157in}{1.070973in}}%
\pgfpathlineto{\pgfqpoint{0.586903in}{1.064165in}}%
\pgfpathlineto{\pgfqpoint{0.595224in}{1.057663in}}%
\pgfpathlineto{\pgfqpoint{0.604085in}{1.051493in}}%
\pgfpathlineto{\pgfqpoint{0.613452in}{1.045677in}}%
\pgfpathlineto{\pgfqpoint{0.618481in}{1.064603in}}%
\pgfpathlineto{\pgfqpoint{0.625487in}{1.083261in}}%
\pgfpathlineto{\pgfqpoint{0.634429in}{1.101578in}}%
\pgfpathlineto{\pgfqpoint{0.645260in}{1.119485in}}%
\pgfpathlineto{\pgfqpoint{0.657925in}{1.136916in}}%
\pgfpathclose%
\pgfusepath{fill}%
\end{pgfscope}%
\begin{pgfscope}%
\pgfpathrectangle{\pgfqpoint{0.050000in}{0.050000in}}{\pgfqpoint{2.081932in}{2.081932in}}%
\pgfusepath{clip}%
\pgfsetbuttcap%
\pgfsetroundjoin%
\definecolor{currentfill}{rgb}{0.993248,0.906157,0.143936}%
\pgfsetfillcolor{currentfill}%
\pgfsetlinewidth{0.000000pt}%
\definecolor{currentstroke}{rgb}{0.000000,0.000000,0.000000}%
\pgfsetstrokecolor{currentstroke}%
\pgfsetdash{}{0pt}%
\pgfpathmoveto{\pgfqpoint{0.897278in}{1.463617in}}%
\pgfpathlineto{\pgfqpoint{0.903000in}{1.460140in}}%
\pgfpathlineto{\pgfqpoint{0.908827in}{1.456058in}}%
\pgfpathlineto{\pgfqpoint{0.914738in}{1.451385in}}%
\pgfpathlineto{\pgfqpoint{0.920706in}{1.446141in}}%
\pgfpathlineto{\pgfqpoint{0.926709in}{1.440344in}}%
\pgfpathlineto{\pgfqpoint{0.906062in}{1.433069in}}%
\pgfpathlineto{\pgfqpoint{0.886214in}{1.425060in}}%
\pgfpathlineto{\pgfqpoint{0.867242in}{1.416346in}}%
\pgfpathlineto{\pgfqpoint{0.849222in}{1.406955in}}%
\pgfpathlineto{\pgfqpoint{0.832224in}{1.396922in}}%
\pgfpathlineto{\pgfqpoint{0.823223in}{1.401390in}}%
\pgfpathlineto{\pgfqpoint{0.814270in}{1.405312in}}%
\pgfpathlineto{\pgfqpoint{0.805402in}{1.408673in}}%
\pgfpathlineto{\pgfqpoint{0.796655in}{1.411459in}}%
\pgfpathlineto{\pgfqpoint{0.788064in}{1.413662in}}%
\pgfpathlineto{\pgfqpoint{0.807739in}{1.425212in}}%
\pgfpathlineto{\pgfqpoint{0.828582in}{1.436018in}}%
\pgfpathlineto{\pgfqpoint{0.850510in}{1.446044in}}%
\pgfpathlineto{\pgfqpoint{0.873439in}{1.455254in}}%
\pgfpathlineto{\pgfqpoint{0.897278in}{1.463617in}}%
\pgfpathclose%
\pgfusepath{fill}%
\end{pgfscope}%
\begin{pgfscope}%
\pgfpathrectangle{\pgfqpoint{0.050000in}{0.050000in}}{\pgfqpoint{2.081932in}{2.081932in}}%
\pgfusepath{clip}%
\pgfsetbuttcap%
\pgfsetroundjoin%
\definecolor{currentfill}{rgb}{0.267004,0.004874,0.329415}%
\pgfsetfillcolor{currentfill}%
\pgfsetlinewidth{0.000000pt}%
\definecolor{currentstroke}{rgb}{0.000000,0.000000,0.000000}%
\pgfsetstrokecolor{currentstroke}%
\pgfsetdash{}{0pt}%
\pgfpathmoveto{\pgfqpoint{0.814197in}{1.073093in}}%
\pgfpathlineto{\pgfqpoint{0.803493in}{1.073945in}}%
\pgfpathlineto{\pgfqpoint{0.792645in}{1.075334in}}%
\pgfpathlineto{\pgfqpoint{0.781695in}{1.077255in}}%
\pgfpathlineto{\pgfqpoint{0.770688in}{1.079699in}}%
\pgfpathlineto{\pgfqpoint{0.759665in}{1.082657in}}%
\pgfpathlineto{\pgfqpoint{0.749967in}{1.068987in}}%
\pgfpathlineto{\pgfqpoint{0.741712in}{1.054957in}}%
\pgfpathlineto{\pgfqpoint{0.734939in}{1.040620in}}%
\pgfpathlineto{\pgfqpoint{0.729683in}{1.026031in}}%
\pgfpathlineto{\pgfqpoint{0.725972in}{1.011247in}}%
\pgfpathlineto{\pgfqpoint{0.738129in}{1.010472in}}%
\pgfpathlineto{\pgfqpoint{0.750263in}{1.010212in}}%
\pgfpathlineto{\pgfqpoint{0.762327in}{1.010468in}}%
\pgfpathlineto{\pgfqpoint{0.774273in}{1.011240in}}%
\pgfpathlineto{\pgfqpoint{0.786054in}{1.012524in}}%
\pgfpathlineto{\pgfqpoint{0.789107in}{1.025051in}}%
\pgfpathlineto{\pgfqpoint{0.793474in}{1.037419in}}%
\pgfpathlineto{\pgfqpoint{0.799131in}{1.049580in}}%
\pgfpathlineto{\pgfqpoint{0.806049in}{1.061486in}}%
\pgfpathlineto{\pgfqpoint{0.814197in}{1.073093in}}%
\pgfpathclose%
\pgfusepath{fill}%
\end{pgfscope}%
\begin{pgfscope}%
\pgfpathrectangle{\pgfqpoint{0.050000in}{0.050000in}}{\pgfqpoint{2.081932in}{2.081932in}}%
\pgfusepath{clip}%
\pgfsetbuttcap%
\pgfsetroundjoin%
\definecolor{currentfill}{rgb}{0.636902,0.856542,0.216620}%
\pgfsetfillcolor{currentfill}%
\pgfsetlinewidth{0.000000pt}%
\definecolor{currentstroke}{rgb}{0.000000,0.000000,0.000000}%
\pgfsetstrokecolor{currentstroke}%
\pgfsetdash{}{0pt}%
\pgfpathmoveto{\pgfqpoint{0.717857in}{1.403073in}}%
\pgfpathlineto{\pgfqpoint{0.723165in}{1.406734in}}%
\pgfpathlineto{\pgfqpoint{0.728920in}{1.409842in}}%
\pgfpathlineto{\pgfqpoint{0.735101in}{1.412385in}}%
\pgfpathlineto{\pgfqpoint{0.741683in}{1.414350in}}%
\pgfpathlineto{\pgfqpoint{0.748641in}{1.415728in}}%
\pgfpathlineto{\pgfqpoint{0.727948in}{1.402017in}}%
\pgfpathlineto{\pgfqpoint{0.708724in}{1.387562in}}%
\pgfpathlineto{\pgfqpoint{0.691051in}{1.372413in}}%
\pgfpathlineto{\pgfqpoint{0.675005in}{1.356625in}}%
\pgfpathlineto{\pgfqpoint{0.660657in}{1.340253in}}%
\pgfpathlineto{\pgfqpoint{0.651962in}{1.337430in}}%
\pgfpathlineto{\pgfqpoint{0.643735in}{1.334090in}}%
\pgfpathlineto{\pgfqpoint{0.636006in}{1.330249in}}%
\pgfpathlineto{\pgfqpoint{0.628808in}{1.325923in}}%
\pgfpathlineto{\pgfqpoint{0.622167in}{1.321130in}}%
\pgfpathlineto{\pgfqpoint{0.637802in}{1.338916in}}%
\pgfpathlineto{\pgfqpoint{0.655270in}{1.356062in}}%
\pgfpathlineto{\pgfqpoint{0.674491in}{1.372508in}}%
\pgfpathlineto{\pgfqpoint{0.695383in}{1.388197in}}%
\pgfpathlineto{\pgfqpoint{0.717857in}{1.403073in}}%
\pgfpathclose%
\pgfusepath{fill}%
\end{pgfscope}%
\begin{pgfscope}%
\pgfpathrectangle{\pgfqpoint{0.050000in}{0.050000in}}{\pgfqpoint{2.081932in}{2.081932in}}%
\pgfusepath{clip}%
\pgfsetbuttcap%
\pgfsetroundjoin%
\definecolor{currentfill}{rgb}{0.855810,0.888601,0.097452}%
\pgfsetfillcolor{currentfill}%
\pgfsetlinewidth{0.000000pt}%
\definecolor{currentstroke}{rgb}{0.000000,0.000000,0.000000}%
\pgfsetstrokecolor{currentstroke}%
\pgfsetdash{}{0pt}%
\pgfpathmoveto{\pgfqpoint{1.152745in}{1.427949in}}%
\pgfpathlineto{\pgfqpoint{1.151575in}{1.418584in}}%
\pgfpathlineto{\pgfqpoint{1.150434in}{1.408891in}}%
\pgfpathlineto{\pgfqpoint{1.149325in}{1.398909in}}%
\pgfpathlineto{\pgfqpoint{1.148255in}{1.388681in}}%
\pgfpathlineto{\pgfqpoint{1.147226in}{1.378248in}}%
\pgfpathlineto{\pgfqpoint{1.130243in}{1.378954in}}%
\pgfpathlineto{\pgfqpoint{1.113218in}{1.379046in}}%
\pgfpathlineto{\pgfqpoint{1.096215in}{1.378523in}}%
\pgfpathlineto{\pgfqpoint{1.079299in}{1.377387in}}%
\pgfpathlineto{\pgfqpoint{1.062534in}{1.375642in}}%
\pgfpathlineto{\pgfqpoint{1.060458in}{1.385983in}}%
\pgfpathlineto{\pgfqpoint{1.058297in}{1.396116in}}%
\pgfpathlineto{\pgfqpoint{1.056059in}{1.405999in}}%
\pgfpathlineto{\pgfqpoint{1.053755in}{1.415590in}}%
\pgfpathlineto{\pgfqpoint{1.051394in}{1.424851in}}%
\pgfpathlineto{\pgfqpoint{1.071458in}{1.426926in}}%
\pgfpathlineto{\pgfqpoint{1.091702in}{1.428276in}}%
\pgfpathlineto{\pgfqpoint{1.112048in}{1.428898in}}%
\pgfpathlineto{\pgfqpoint{1.132421in}{1.428789in}}%
\pgfpathlineto{\pgfqpoint{1.152745in}{1.427949in}}%
\pgfpathclose%
\pgfusepath{fill}%
\end{pgfscope}%
\begin{pgfscope}%
\pgfpathrectangle{\pgfqpoint{0.050000in}{0.050000in}}{\pgfqpoint{2.081932in}{2.081932in}}%
\pgfusepath{clip}%
\pgfsetbuttcap%
\pgfsetroundjoin%
\definecolor{currentfill}{rgb}{0.227802,0.326594,0.546532}%
\pgfsetfillcolor{currentfill}%
\pgfsetlinewidth{0.000000pt}%
\definecolor{currentstroke}{rgb}{0.000000,0.000000,0.000000}%
\pgfsetstrokecolor{currentstroke}%
\pgfsetdash{}{0pt}%
\pgfpathmoveto{\pgfqpoint{1.297296in}{1.134278in}}%
\pgfpathlineto{\pgfqpoint{1.301139in}{1.128003in}}%
\pgfpathlineto{\pgfqpoint{1.305595in}{1.122149in}}%
\pgfpathlineto{\pgfqpoint{1.310643in}{1.116739in}}%
\pgfpathlineto{\pgfqpoint{1.316262in}{1.111794in}}%
\pgfpathlineto{\pgfqpoint{1.322427in}{1.107333in}}%
\pgfpathlineto{\pgfqpoint{1.314561in}{1.114779in}}%
\pgfpathlineto{\pgfqpoint{1.305929in}{1.121910in}}%
\pgfpathlineto{\pgfqpoint{1.296569in}{1.128700in}}%
\pgfpathlineto{\pgfqpoint{1.286517in}{1.135123in}}%
\pgfpathlineto{\pgfqpoint{1.275815in}{1.141154in}}%
\pgfpathlineto{\pgfqpoint{1.271082in}{1.144575in}}%
\pgfpathlineto{\pgfqpoint{1.266768in}{1.148570in}}%
\pgfpathlineto{\pgfqpoint{1.262892in}{1.153123in}}%
\pgfpathlineto{\pgfqpoint{1.259470in}{1.158218in}}%
\pgfpathlineto{\pgfqpoint{1.256517in}{1.163835in}}%
\pgfpathlineto{\pgfqpoint{1.265887in}{1.158562in}}%
\pgfpathlineto{\pgfqpoint{1.274684in}{1.152948in}}%
\pgfpathlineto{\pgfqpoint{1.282873in}{1.147014in}}%
\pgfpathlineto{\pgfqpoint{1.290421in}{1.140783in}}%
\pgfpathlineto{\pgfqpoint{1.297296in}{1.134278in}}%
\pgfpathclose%
\pgfusepath{fill}%
\end{pgfscope}%
\begin{pgfscope}%
\pgfpathrectangle{\pgfqpoint{0.050000in}{0.050000in}}{\pgfqpoint{2.081932in}{2.081932in}}%
\pgfusepath{clip}%
\pgfsetbuttcap%
\pgfsetroundjoin%
\definecolor{currentfill}{rgb}{0.876168,0.891125,0.095250}%
\pgfsetfillcolor{currentfill}%
\pgfsetlinewidth{0.000000pt}%
\definecolor{currentstroke}{rgb}{0.000000,0.000000,0.000000}%
\pgfsetstrokecolor{currentstroke}%
\pgfsetdash{}{0pt}%
\pgfpathmoveto{\pgfqpoint{1.552288in}{1.367766in}}%
\pgfpathlineto{\pgfqpoint{1.543696in}{1.369507in}}%
\pgfpathlineto{\pgfqpoint{1.534730in}{1.370687in}}%
\pgfpathlineto{\pgfqpoint{1.525425in}{1.371302in}}%
\pgfpathlineto{\pgfqpoint{1.515821in}{1.371346in}}%
\pgfpathlineto{\pgfqpoint{1.505954in}{1.370819in}}%
\pgfpathlineto{\pgfqpoint{1.490680in}{1.384521in}}%
\pgfpathlineto{\pgfqpoint{1.473989in}{1.397622in}}%
\pgfpathlineto{\pgfqpoint{1.455954in}{1.410077in}}%
\pgfpathlineto{\pgfqpoint{1.436647in}{1.421841in}}%
\pgfpathlineto{\pgfqpoint{1.416148in}{1.432874in}}%
\pgfpathlineto{\pgfqpoint{1.423663in}{1.434966in}}%
\pgfpathlineto{\pgfqpoint{1.430975in}{1.436450in}}%
\pgfpathlineto{\pgfqpoint{1.438056in}{1.437321in}}%
\pgfpathlineto{\pgfqpoint{1.444878in}{1.437576in}}%
\pgfpathlineto{\pgfqpoint{1.451412in}{1.437215in}}%
\pgfpathlineto{\pgfqpoint{1.474405in}{1.424878in}}%
\pgfpathlineto{\pgfqpoint{1.496074in}{1.411717in}}%
\pgfpathlineto{\pgfqpoint{1.516333in}{1.397779in}}%
\pgfpathlineto{\pgfqpoint{1.535097in}{1.383111in}}%
\pgfpathlineto{\pgfqpoint{1.552288in}{1.367766in}}%
\pgfpathclose%
\pgfusepath{fill}%
\end{pgfscope}%
\begin{pgfscope}%
\pgfpathrectangle{\pgfqpoint{0.050000in}{0.050000in}}{\pgfqpoint{2.081932in}{2.081932in}}%
\pgfusepath{clip}%
\pgfsetbuttcap%
\pgfsetroundjoin%
\definecolor{currentfill}{rgb}{0.162142,0.474838,0.558140}%
\pgfsetfillcolor{currentfill}%
\pgfsetlinewidth{0.000000pt}%
\definecolor{currentstroke}{rgb}{0.000000,0.000000,0.000000}%
\pgfsetstrokecolor{currentstroke}%
\pgfsetdash{}{0pt}%
\pgfpathmoveto{\pgfqpoint{1.249183in}{1.198901in}}%
\pgfpathlineto{\pgfqpoint{1.249632in}{1.191047in}}%
\pgfpathlineto{\pgfqpoint{1.250596in}{1.183584in}}%
\pgfpathlineto{\pgfqpoint{1.252070in}{1.176543in}}%
\pgfpathlineto{\pgfqpoint{1.254046in}{1.169951in}}%
\pgfpathlineto{\pgfqpoint{1.256517in}{1.163835in}}%
\pgfpathlineto{\pgfqpoint{1.246612in}{1.168747in}}%
\pgfpathlineto{\pgfqpoint{1.236211in}{1.173279in}}%
\pgfpathlineto{\pgfqpoint{1.225356in}{1.177416in}}%
\pgfpathlineto{\pgfqpoint{1.214090in}{1.181140in}}%
\pgfpathlineto{\pgfqpoint{1.202456in}{1.184439in}}%
\pgfpathlineto{\pgfqpoint{1.200961in}{1.190171in}}%
\pgfpathlineto{\pgfqpoint{1.199766in}{1.196451in}}%
\pgfpathlineto{\pgfqpoint{1.198874in}{1.203255in}}%
\pgfpathlineto{\pgfqpoint{1.198291in}{1.210556in}}%
\pgfpathlineto{\pgfqpoint{1.198019in}{1.218324in}}%
\pgfpathlineto{\pgfqpoint{1.209032in}{1.215214in}}%
\pgfpathlineto{\pgfqpoint{1.219695in}{1.211702in}}%
\pgfpathlineto{\pgfqpoint{1.229969in}{1.207803in}}%
\pgfpathlineto{\pgfqpoint{1.239811in}{1.203531in}}%
\pgfpathlineto{\pgfqpoint{1.249183in}{1.198901in}}%
\pgfpathclose%
\pgfusepath{fill}%
\end{pgfscope}%
\begin{pgfscope}%
\pgfpathrectangle{\pgfqpoint{0.050000in}{0.050000in}}{\pgfqpoint{2.081932in}{2.081932in}}%
\pgfusepath{clip}%
\pgfsetbuttcap%
\pgfsetroundjoin%
\definecolor{currentfill}{rgb}{0.206756,0.371758,0.553117}%
\pgfsetfillcolor{currentfill}%
\pgfsetlinewidth{0.000000pt}%
\definecolor{currentstroke}{rgb}{0.000000,0.000000,0.000000}%
\pgfsetstrokecolor{currentstroke}%
\pgfsetdash{}{0pt}%
\pgfpathmoveto{\pgfqpoint{0.620585in}{1.176299in}}%
\pgfpathlineto{\pgfqpoint{0.614729in}{1.184703in}}%
\pgfpathlineto{\pgfqpoint{0.609469in}{1.193208in}}%
\pgfpathlineto{\pgfqpoint{0.604827in}{1.201781in}}%
\pgfpathlineto{\pgfqpoint{0.600821in}{1.210387in}}%
\pgfpathlineto{\pgfqpoint{0.597469in}{1.218993in}}%
\pgfpathlineto{\pgfqpoint{0.582993in}{1.199470in}}%
\pgfpathlineto{\pgfqpoint{0.570581in}{1.179403in}}%
\pgfpathlineto{\pgfqpoint{0.560296in}{1.158864in}}%
\pgfpathlineto{\pgfqpoint{0.552195in}{1.137931in}}%
\pgfpathlineto{\pgfqpoint{0.546327in}{1.116683in}}%
\pgfpathlineto{\pgfqpoint{0.550054in}{1.108634in}}%
\pgfpathlineto{\pgfqpoint{0.554507in}{1.100717in}}%
\pgfpathlineto{\pgfqpoint{0.559666in}{1.092962in}}%
\pgfpathlineto{\pgfqpoint{0.565511in}{1.085400in}}%
\pgfpathlineto{\pgfqpoint{0.572017in}{1.078061in}}%
\pgfpathlineto{\pgfqpoint{0.577559in}{1.098454in}}%
\pgfpathlineto{\pgfqpoint{0.585236in}{1.118550in}}%
\pgfpathlineto{\pgfqpoint{0.595003in}{1.138271in}}%
\pgfpathlineto{\pgfqpoint{0.606806in}{1.157545in}}%
\pgfpathlineto{\pgfqpoint{0.620585in}{1.176299in}}%
\pgfpathclose%
\pgfusepath{fill}%
\end{pgfscope}%
\begin{pgfscope}%
\pgfpathrectangle{\pgfqpoint{0.050000in}{0.050000in}}{\pgfqpoint{2.081932in}{2.081932in}}%
\pgfusepath{clip}%
\pgfsetbuttcap%
\pgfsetroundjoin%
\definecolor{currentfill}{rgb}{0.855810,0.888601,0.097452}%
\pgfsetfillcolor{currentfill}%
\pgfsetlinewidth{0.000000pt}%
\definecolor{currentstroke}{rgb}{0.000000,0.000000,0.000000}%
\pgfsetstrokecolor{currentstroke}%
\pgfsetdash{}{0pt}%
\pgfpathmoveto{\pgfqpoint{1.250966in}{1.412981in}}%
\pgfpathlineto{\pgfqpoint{1.246367in}{1.404121in}}%
\pgfpathlineto{\pgfqpoint{1.241880in}{1.394921in}}%
\pgfpathlineto{\pgfqpoint{1.237523in}{1.385418in}}%
\pgfpathlineto{\pgfqpoint{1.233314in}{1.375652in}}%
\pgfpathlineto{\pgfqpoint{1.229272in}{1.365664in}}%
\pgfpathlineto{\pgfqpoint{1.213456in}{1.369361in}}%
\pgfpathlineto{\pgfqpoint{1.197281in}{1.372478in}}%
\pgfpathlineto{\pgfqpoint{1.180809in}{1.375004in}}%
\pgfpathlineto{\pgfqpoint{1.164103in}{1.376930in}}%
\pgfpathlineto{\pgfqpoint{1.147226in}{1.378248in}}%
\pgfpathlineto{\pgfqpoint{1.148255in}{1.388681in}}%
\pgfpathlineto{\pgfqpoint{1.149325in}{1.398909in}}%
\pgfpathlineto{\pgfqpoint{1.150434in}{1.408891in}}%
\pgfpathlineto{\pgfqpoint{1.151575in}{1.418584in}}%
\pgfpathlineto{\pgfqpoint{1.152745in}{1.427949in}}%
\pgfpathlineto{\pgfqpoint{1.172942in}{1.426382in}}%
\pgfpathlineto{\pgfqpoint{1.192938in}{1.424092in}}%
\pgfpathlineto{\pgfqpoint{1.212657in}{1.421088in}}%
\pgfpathlineto{\pgfqpoint{1.232024in}{1.417380in}}%
\pgfpathlineto{\pgfqpoint{1.250966in}{1.412981in}}%
\pgfpathclose%
\pgfusepath{fill}%
\end{pgfscope}%
\begin{pgfscope}%
\pgfpathrectangle{\pgfqpoint{0.050000in}{0.050000in}}{\pgfqpoint{2.081932in}{2.081932in}}%
\pgfusepath{clip}%
\pgfsetbuttcap%
\pgfsetroundjoin%
\definecolor{currentfill}{rgb}{0.124780,0.640461,0.527068}%
\pgfsetfillcolor{currentfill}%
\pgfsetlinewidth{0.000000pt}%
\definecolor{currentstroke}{rgb}{0.000000,0.000000,0.000000}%
\pgfsetstrokecolor{currentstroke}%
\pgfsetdash{}{0pt}%
\pgfpathmoveto{\pgfqpoint{1.687904in}{1.185832in}}%
\pgfpathlineto{\pgfqpoint{1.687826in}{1.194009in}}%
\pgfpathlineto{\pgfqpoint{1.686990in}{1.202073in}}%
\pgfpathlineto{\pgfqpoint{1.685399in}{1.209992in}}%
\pgfpathlineto{\pgfqpoint{1.683057in}{1.217733in}}%
\pgfpathlineto{\pgfqpoint{1.679973in}{1.225266in}}%
\pgfpathlineto{\pgfqpoint{1.670355in}{1.245720in}}%
\pgfpathlineto{\pgfqpoint{1.658595in}{1.265730in}}%
\pgfpathlineto{\pgfqpoint{1.644754in}{1.285222in}}%
\pgfpathlineto{\pgfqpoint{1.628900in}{1.304125in}}%
\pgfpathlineto{\pgfqpoint{1.611109in}{1.322372in}}%
\pgfpathlineto{\pgfqpoint{1.613782in}{1.315458in}}%
\pgfpathlineto{\pgfqpoint{1.615811in}{1.308211in}}%
\pgfpathlineto{\pgfqpoint{1.617190in}{1.300662in}}%
\pgfpathlineto{\pgfqpoint{1.617914in}{1.292839in}}%
\pgfpathlineto{\pgfqpoint{1.617982in}{1.284776in}}%
\pgfpathlineto{\pgfqpoint{1.636038in}{1.266187in}}%
\pgfpathlineto{\pgfqpoint{1.652132in}{1.246927in}}%
\pgfpathlineto{\pgfqpoint{1.666186in}{1.227066in}}%
\pgfpathlineto{\pgfqpoint{1.678131in}{1.206676in}}%
\pgfpathlineto{\pgfqpoint{1.687904in}{1.185832in}}%
\pgfpathclose%
\pgfusepath{fill}%
\end{pgfscope}%
\begin{pgfscope}%
\pgfpathrectangle{\pgfqpoint{0.050000in}{0.050000in}}{\pgfqpoint{2.081932in}{2.081932in}}%
\pgfusepath{clip}%
\pgfsetbuttcap%
\pgfsetroundjoin%
\definecolor{currentfill}{rgb}{0.120638,0.625828,0.533488}%
\pgfsetfillcolor{currentfill}%
\pgfsetlinewidth{0.000000pt}%
\definecolor{currentstroke}{rgb}{0.000000,0.000000,0.000000}%
\pgfsetstrokecolor{currentstroke}%
\pgfsetdash{}{0pt}%
\pgfpathmoveto{\pgfqpoint{1.140117in}{1.272513in}}%
\pgfpathlineto{\pgfqpoint{1.139790in}{1.262785in}}%
\pgfpathlineto{\pgfqpoint{1.139541in}{1.253382in}}%
\pgfpathlineto{\pgfqpoint{1.139371in}{1.244343in}}%
\pgfpathlineto{\pgfqpoint{1.139281in}{1.235706in}}%
\pgfpathlineto{\pgfqpoint{1.139270in}{1.227507in}}%
\pgfpathlineto{\pgfqpoint{1.127102in}{1.228023in}}%
\pgfpathlineto{\pgfqpoint{1.114903in}{1.228090in}}%
\pgfpathlineto{\pgfqpoint{1.102721in}{1.227708in}}%
\pgfpathlineto{\pgfqpoint{1.090602in}{1.226879in}}%
\pgfpathlineto{\pgfqpoint{1.078593in}{1.225605in}}%
\pgfpathlineto{\pgfqpoint{1.078572in}{1.233805in}}%
\pgfpathlineto{\pgfqpoint{1.078390in}{1.242435in}}%
\pgfpathlineto{\pgfqpoint{1.078047in}{1.251460in}}%
\pgfpathlineto{\pgfqpoint{1.077544in}{1.260842in}}%
\pgfpathlineto{\pgfqpoint{1.076884in}{1.270542in}}%
\pgfpathlineto{\pgfqpoint{1.089399in}{1.271862in}}%
\pgfpathlineto{\pgfqpoint{1.102028in}{1.272722in}}%
\pgfpathlineto{\pgfqpoint{1.114724in}{1.273117in}}%
\pgfpathlineto{\pgfqpoint{1.127436in}{1.273048in}}%
\pgfpathlineto{\pgfqpoint{1.140117in}{1.272513in}}%
\pgfpathclose%
\pgfusepath{fill}%
\end{pgfscope}%
\begin{pgfscope}%
\pgfpathrectangle{\pgfqpoint{0.050000in}{0.050000in}}{\pgfqpoint{2.081932in}{2.081932in}}%
\pgfusepath{clip}%
\pgfsetbuttcap%
\pgfsetroundjoin%
\definecolor{currentfill}{rgb}{0.993248,0.906157,0.143936}%
\pgfsetfillcolor{currentfill}%
\pgfsetlinewidth{0.000000pt}%
\definecolor{currentstroke}{rgb}{0.000000,0.000000,0.000000}%
\pgfsetstrokecolor{currentstroke}%
\pgfsetdash{}{0pt}%
\pgfpathmoveto{\pgfqpoint{1.376622in}{1.413612in}}%
\pgfpathlineto{\pgfqpoint{1.368553in}{1.408103in}}%
\pgfpathlineto{\pgfqpoint{1.360507in}{1.402087in}}%
\pgfpathlineto{\pgfqpoint{1.352516in}{1.395588in}}%
\pgfpathlineto{\pgfqpoint{1.344615in}{1.388632in}}%
\pgfpathlineto{\pgfqpoint{1.336835in}{1.381248in}}%
\pgfpathlineto{\pgfqpoint{1.321067in}{1.388821in}}%
\pgfpathlineto{\pgfqpoint{1.304530in}{1.395805in}}%
\pgfpathlineto{\pgfqpoint{1.287289in}{1.402175in}}%
\pgfpathlineto{\pgfqpoint{1.269411in}{1.407907in}}%
\pgfpathlineto{\pgfqpoint{1.250966in}{1.412981in}}%
\pgfpathlineto{\pgfqpoint{1.255657in}{1.421464in}}%
\pgfpathlineto{\pgfqpoint{1.260420in}{1.429536in}}%
\pgfpathlineto{\pgfqpoint{1.265236in}{1.437163in}}%
\pgfpathlineto{\pgfqpoint{1.270084in}{1.444316in}}%
\pgfpathlineto{\pgfqpoint{1.274944in}{1.450965in}}%
\pgfpathlineto{\pgfqpoint{1.296765in}{1.444996in}}%
\pgfpathlineto{\pgfqpoint{1.317924in}{1.438251in}}%
\pgfpathlineto{\pgfqpoint{1.338338in}{1.430755in}}%
\pgfpathlineto{\pgfqpoint{1.357930in}{1.422532in}}%
\pgfpathlineto{\pgfqpoint{1.376622in}{1.413612in}}%
\pgfpathclose%
\pgfusepath{fill}%
\end{pgfscope}%
\begin{pgfscope}%
\pgfpathrectangle{\pgfqpoint{0.050000in}{0.050000in}}{\pgfqpoint{2.081932in}{2.081932in}}%
\pgfusepath{clip}%
\pgfsetbuttcap%
\pgfsetroundjoin%
\definecolor{currentfill}{rgb}{0.278791,0.062145,0.386592}%
\pgfsetfillcolor{currentfill}%
\pgfsetlinewidth{0.000000pt}%
\definecolor{currentstroke}{rgb}{0.000000,0.000000,0.000000}%
\pgfsetstrokecolor{currentstroke}%
\pgfsetdash{}{0pt}%
\pgfpathmoveto{\pgfqpoint{0.864090in}{1.076975in}}%
\pgfpathlineto{\pgfqpoint{0.854730in}{1.075117in}}%
\pgfpathlineto{\pgfqpoint{0.845021in}{1.073797in}}%
\pgfpathlineto{\pgfqpoint{0.835002in}{1.073019in}}%
\pgfpathlineto{\pgfqpoint{0.824714in}{1.072784in}}%
\pgfpathlineto{\pgfqpoint{0.814197in}{1.073093in}}%
\pgfpathlineto{\pgfqpoint{0.806049in}{1.061486in}}%
\pgfpathlineto{\pgfqpoint{0.799131in}{1.049580in}}%
\pgfpathlineto{\pgfqpoint{0.793474in}{1.037419in}}%
\pgfpathlineto{\pgfqpoint{0.789107in}{1.025051in}}%
\pgfpathlineto{\pgfqpoint{0.786054in}{1.012524in}}%
\pgfpathlineto{\pgfqpoint{0.797623in}{1.014316in}}%
\pgfpathlineto{\pgfqpoint{0.808935in}{1.016610in}}%
\pgfpathlineto{\pgfqpoint{0.819946in}{1.019395in}}%
\pgfpathlineto{\pgfqpoint{0.830610in}{1.022663in}}%
\pgfpathlineto{\pgfqpoint{0.840887in}{1.026400in}}%
\pgfpathlineto{\pgfqpoint{0.843369in}{1.036850in}}%
\pgfpathlineto{\pgfqpoint{0.846951in}{1.047172in}}%
\pgfpathlineto{\pgfqpoint{0.851614in}{1.057326in}}%
\pgfpathlineto{\pgfqpoint{0.857336in}{1.067274in}}%
\pgfpathlineto{\pgfqpoint{0.864090in}{1.076975in}}%
\pgfpathclose%
\pgfusepath{fill}%
\end{pgfscope}%
\begin{pgfscope}%
\pgfpathrectangle{\pgfqpoint{0.050000in}{0.050000in}}{\pgfqpoint{2.081932in}{2.081932in}}%
\pgfusepath{clip}%
\pgfsetbuttcap%
\pgfsetroundjoin%
\definecolor{currentfill}{rgb}{0.162142,0.474838,0.558140}%
\pgfsetfillcolor{currentfill}%
\pgfsetlinewidth{0.000000pt}%
\definecolor{currentstroke}{rgb}{0.000000,0.000000,0.000000}%
\pgfsetstrokecolor{currentstroke}%
\pgfsetdash{}{0pt}%
\pgfpathmoveto{\pgfqpoint{1.021809in}{1.212792in}}%
\pgfpathlineto{\pgfqpoint{1.021473in}{1.204999in}}%
\pgfpathlineto{\pgfqpoint{1.020753in}{1.197652in}}%
\pgfpathlineto{\pgfqpoint{1.019651in}{1.190781in}}%
\pgfpathlineto{\pgfqpoint{1.018174in}{1.184412in}}%
\pgfpathlineto{\pgfqpoint{1.016327in}{1.178571in}}%
\pgfpathlineto{\pgfqpoint{1.005344in}{1.174556in}}%
\pgfpathlineto{\pgfqpoint{0.994802in}{1.170140in}}%
\pgfpathlineto{\pgfqpoint{0.984743in}{1.165339in}}%
\pgfpathlineto{\pgfqpoint{0.975207in}{1.160172in}}%
\pgfpathlineto{\pgfqpoint{0.966234in}{1.154657in}}%
\pgfpathlineto{\pgfqpoint{0.968987in}{1.160946in}}%
\pgfpathlineto{\pgfqpoint{0.971190in}{1.167677in}}%
\pgfpathlineto{\pgfqpoint{0.972833in}{1.174825in}}%
\pgfpathlineto{\pgfqpoint{0.973907in}{1.182360in}}%
\pgfpathlineto{\pgfqpoint{0.974408in}{1.190252in}}%
\pgfpathlineto{\pgfqpoint{0.982897in}{1.195449in}}%
\pgfpathlineto{\pgfqpoint{0.991919in}{1.200319in}}%
\pgfpathlineto{\pgfqpoint{1.001437in}{1.204844in}}%
\pgfpathlineto{\pgfqpoint{1.011413in}{1.209007in}}%
\pgfpathlineto{\pgfqpoint{1.021809in}{1.212792in}}%
\pgfpathclose%
\pgfusepath{fill}%
\end{pgfscope}%
\begin{pgfscope}%
\pgfpathrectangle{\pgfqpoint{0.050000in}{0.050000in}}{\pgfqpoint{2.081932in}{2.081932in}}%
\pgfusepath{clip}%
\pgfsetbuttcap%
\pgfsetroundjoin%
\definecolor{currentfill}{rgb}{0.120638,0.625828,0.533488}%
\pgfsetfillcolor{currentfill}%
\pgfsetlinewidth{0.000000pt}%
\definecolor{currentstroke}{rgb}{0.000000,0.000000,0.000000}%
\pgfsetstrokecolor{currentstroke}%
\pgfsetdash{}{0pt}%
\pgfpathmoveto{\pgfqpoint{1.201345in}{1.262996in}}%
\pgfpathlineto{\pgfqpoint{1.200061in}{1.253405in}}%
\pgfpathlineto{\pgfqpoint{1.199083in}{1.244104in}}%
\pgfpathlineto{\pgfqpoint{1.198416in}{1.235132in}}%
\pgfpathlineto{\pgfqpoint{1.198061in}{1.226527in}}%
\pgfpathlineto{\pgfqpoint{1.198019in}{1.218324in}}%
\pgfpathlineto{\pgfqpoint{1.186702in}{1.221021in}}%
\pgfpathlineto{\pgfqpoint{1.175122in}{1.223296in}}%
\pgfpathlineto{\pgfqpoint{1.163327in}{1.225139in}}%
\pgfpathlineto{\pgfqpoint{1.151361in}{1.226545in}}%
\pgfpathlineto{\pgfqpoint{1.139270in}{1.227507in}}%
\pgfpathlineto{\pgfqpoint{1.139281in}{1.235706in}}%
\pgfpathlineto{\pgfqpoint{1.139371in}{1.244343in}}%
\pgfpathlineto{\pgfqpoint{1.139541in}{1.253382in}}%
\pgfpathlineto{\pgfqpoint{1.139790in}{1.262785in}}%
\pgfpathlineto{\pgfqpoint{1.140117in}{1.272513in}}%
\pgfpathlineto{\pgfqpoint{1.152717in}{1.271516in}}%
\pgfpathlineto{\pgfqpoint{1.165188in}{1.270060in}}%
\pgfpathlineto{\pgfqpoint{1.177481in}{1.268149in}}%
\pgfpathlineto{\pgfqpoint{1.189549in}{1.265792in}}%
\pgfpathlineto{\pgfqpoint{1.201345in}{1.262996in}}%
\pgfpathclose%
\pgfusepath{fill}%
\end{pgfscope}%
\begin{pgfscope}%
\pgfpathrectangle{\pgfqpoint{0.050000in}{0.050000in}}{\pgfqpoint{2.081932in}{2.081932in}}%
\pgfusepath{clip}%
\pgfsetbuttcap%
\pgfsetroundjoin%
\definecolor{currentfill}{rgb}{0.278012,0.180367,0.486697}%
\pgfsetfillcolor{currentfill}%
\pgfsetlinewidth{0.000000pt}%
\definecolor{currentstroke}{rgb}{0.000000,0.000000,0.000000}%
\pgfsetstrokecolor{currentstroke}%
\pgfsetdash{}{0pt}%
\pgfpathmoveto{\pgfqpoint{1.349236in}{1.066421in}}%
\pgfpathlineto{\pgfqpoint{1.356844in}{1.061088in}}%
\pgfpathlineto{\pgfqpoint{1.365016in}{1.056172in}}%
\pgfpathlineto{\pgfqpoint{1.373718in}{1.051692in}}%
\pgfpathlineto{\pgfqpoint{1.382915in}{1.047665in}}%
\pgfpathlineto{\pgfqpoint{1.392571in}{1.044107in}}%
\pgfpathlineto{\pgfqpoint{1.388231in}{1.054316in}}%
\pgfpathlineto{\pgfqpoint{1.382826in}{1.064329in}}%
\pgfpathlineto{\pgfqpoint{1.376380in}{1.074107in}}%
\pgfpathlineto{\pgfqpoint{1.368923in}{1.083614in}}%
\pgfpathlineto{\pgfqpoint{1.360486in}{1.092811in}}%
\pgfpathlineto{\pgfqpoint{1.352012in}{1.094639in}}%
\pgfpathlineto{\pgfqpoint{1.343937in}{1.097015in}}%
\pgfpathlineto{\pgfqpoint{1.336294in}{1.099930in}}%
\pgfpathlineto{\pgfqpoint{1.329114in}{1.103374in}}%
\pgfpathlineto{\pgfqpoint{1.322427in}{1.107333in}}%
\pgfpathlineto{\pgfqpoint{1.329496in}{1.099601in}}%
\pgfpathlineto{\pgfqpoint{1.335737in}{1.091613in}}%
\pgfpathlineto{\pgfqpoint{1.341123in}{1.083398in}}%
\pgfpathlineto{\pgfqpoint{1.345629in}{1.074990in}}%
\pgfpathlineto{\pgfqpoint{1.349236in}{1.066421in}}%
\pgfpathclose%
\pgfusepath{fill}%
\end{pgfscope}%
\begin{pgfscope}%
\pgfpathrectangle{\pgfqpoint{0.050000in}{0.050000in}}{\pgfqpoint{2.081932in}{2.081932in}}%
\pgfusepath{clip}%
\pgfsetbuttcap%
\pgfsetroundjoin%
\definecolor{currentfill}{rgb}{0.855810,0.888601,0.097452}%
\pgfsetfillcolor{currentfill}%
\pgfsetlinewidth{0.000000pt}%
\definecolor{currentstroke}{rgb}{0.000000,0.000000,0.000000}%
\pgfsetstrokecolor{currentstroke}%
\pgfsetdash{}{0pt}%
\pgfpathmoveto{\pgfqpoint{1.051394in}{1.424851in}}%
\pgfpathlineto{\pgfqpoint{1.053755in}{1.415590in}}%
\pgfpathlineto{\pgfqpoint{1.056059in}{1.405999in}}%
\pgfpathlineto{\pgfqpoint{1.058297in}{1.396116in}}%
\pgfpathlineto{\pgfqpoint{1.060458in}{1.385983in}}%
\pgfpathlineto{\pgfqpoint{1.062534in}{1.375642in}}%
\pgfpathlineto{\pgfqpoint{1.045985in}{1.373295in}}%
\pgfpathlineto{\pgfqpoint{1.029714in}{1.370353in}}%
\pgfpathlineto{\pgfqpoint{1.013784in}{1.366827in}}%
\pgfpathlineto{\pgfqpoint{0.998255in}{1.362731in}}%
\pgfpathlineto{\pgfqpoint{0.983188in}{1.358078in}}%
\pgfpathlineto{\pgfqpoint{0.978191in}{1.367797in}}%
\pgfpathlineto{\pgfqpoint{0.972988in}{1.377283in}}%
\pgfpathlineto{\pgfqpoint{0.967601in}{1.386496in}}%
\pgfpathlineto{\pgfqpoint{0.962052in}{1.395399in}}%
\pgfpathlineto{\pgfqpoint{0.956366in}{1.403953in}}%
\pgfpathlineto{\pgfqpoint{0.974422in}{1.409491in}}%
\pgfpathlineto{\pgfqpoint{0.993024in}{1.414365in}}%
\pgfpathlineto{\pgfqpoint{1.012102in}{1.418559in}}%
\pgfpathlineto{\pgfqpoint{1.031583in}{1.422059in}}%
\pgfpathlineto{\pgfqpoint{1.051394in}{1.424851in}}%
\pgfpathclose%
\pgfusepath{fill}%
\end{pgfscope}%
\begin{pgfscope}%
\pgfpathrectangle{\pgfqpoint{0.050000in}{0.050000in}}{\pgfqpoint{2.081932in}{2.081932in}}%
\pgfusepath{clip}%
\pgfsetbuttcap%
\pgfsetroundjoin%
\definecolor{currentfill}{rgb}{0.227802,0.326594,0.546532}%
\pgfsetfillcolor{currentfill}%
\pgfsetlinewidth{0.000000pt}%
\definecolor{currentstroke}{rgb}{0.000000,0.000000,0.000000}%
\pgfsetstrokecolor{currentstroke}%
\pgfsetdash{}{0pt}%
\pgfpathmoveto{\pgfqpoint{0.966234in}{1.154657in}}%
\pgfpathlineto{\pgfqpoint{0.962942in}{1.148837in}}%
\pgfpathlineto{\pgfqpoint{0.959127in}{1.143507in}}%
\pgfpathlineto{\pgfqpoint{0.954805in}{1.138688in}}%
\pgfpathlineto{\pgfqpoint{0.949996in}{1.134399in}}%
\pgfpathlineto{\pgfqpoint{0.944718in}{1.130656in}}%
\pgfpathlineto{\pgfqpoint{0.935145in}{1.123974in}}%
\pgfpathlineto{\pgfqpoint{0.926290in}{1.116942in}}%
\pgfpathlineto{\pgfqpoint{0.918189in}{1.109587in}}%
\pgfpathlineto{\pgfqpoint{0.910877in}{1.101937in}}%
\pgfpathlineto{\pgfqpoint{0.904383in}{1.094022in}}%
\pgfpathlineto{\pgfqpoint{0.910909in}{1.098894in}}%
\pgfpathlineto{\pgfqpoint{0.916855in}{1.104215in}}%
\pgfpathlineto{\pgfqpoint{0.922196in}{1.109964in}}%
\pgfpathlineto{\pgfqpoint{0.926911in}{1.116118in}}%
\pgfpathlineto{\pgfqpoint{0.930977in}{1.122653in}}%
\pgfpathlineto{\pgfqpoint{0.936645in}{1.129565in}}%
\pgfpathlineto{\pgfqpoint{0.943032in}{1.136247in}}%
\pgfpathlineto{\pgfqpoint{0.950113in}{1.142672in}}%
\pgfpathlineto{\pgfqpoint{0.957857in}{1.148817in}}%
\pgfpathlineto{\pgfqpoint{0.966234in}{1.154657in}}%
\pgfpathclose%
\pgfusepath{fill}%
\end{pgfscope}%
\begin{pgfscope}%
\pgfpathrectangle{\pgfqpoint{0.050000in}{0.050000in}}{\pgfqpoint{2.081932in}{2.081932in}}%
\pgfusepath{clip}%
\pgfsetbuttcap%
\pgfsetroundjoin%
\definecolor{currentfill}{rgb}{0.120638,0.625828,0.533488}%
\pgfsetfillcolor{currentfill}%
\pgfsetlinewidth{0.000000pt}%
\definecolor{currentstroke}{rgb}{0.000000,0.000000,0.000000}%
\pgfsetstrokecolor{currentstroke}%
\pgfsetdash{}{0pt}%
\pgfpathmoveto{\pgfqpoint{1.076884in}{1.270542in}}%
\pgfpathlineto{\pgfqpoint{1.077544in}{1.260842in}}%
\pgfpathlineto{\pgfqpoint{1.078047in}{1.251460in}}%
\pgfpathlineto{\pgfqpoint{1.078390in}{1.242435in}}%
\pgfpathlineto{\pgfqpoint{1.078572in}{1.233805in}}%
\pgfpathlineto{\pgfqpoint{1.078593in}{1.225605in}}%
\pgfpathlineto{\pgfqpoint{1.066741in}{1.223892in}}%
\pgfpathlineto{\pgfqpoint{1.055092in}{1.221745in}}%
\pgfpathlineto{\pgfqpoint{1.043691in}{1.219172in}}%
\pgfpathlineto{\pgfqpoint{1.032582in}{1.216184in}}%
\pgfpathlineto{\pgfqpoint{1.021809in}{1.212792in}}%
\pgfpathlineto{\pgfqpoint{1.021757in}{1.220998in}}%
\pgfpathlineto{\pgfqpoint{1.021319in}{1.229584in}}%
\pgfpathlineto{\pgfqpoint{1.020494in}{1.238515in}}%
\pgfpathlineto{\pgfqpoint{1.019287in}{1.247754in}}%
\pgfpathlineto{\pgfqpoint{1.017701in}{1.257262in}}%
\pgfpathlineto{\pgfqpoint{1.028930in}{1.260779in}}%
\pgfpathlineto{\pgfqpoint{1.040509in}{1.263876in}}%
\pgfpathlineto{\pgfqpoint{1.052392in}{1.266541in}}%
\pgfpathlineto{\pgfqpoint{1.064533in}{1.268766in}}%
\pgfpathlineto{\pgfqpoint{1.076884in}{1.270542in}}%
\pgfpathclose%
\pgfusepath{fill}%
\end{pgfscope}%
\begin{pgfscope}%
\pgfpathrectangle{\pgfqpoint{0.050000in}{0.050000in}}{\pgfqpoint{2.081932in}{2.081932in}}%
\pgfusepath{clip}%
\pgfsetbuttcap%
\pgfsetroundjoin%
\definecolor{currentfill}{rgb}{0.606045,0.850733,0.236712}%
\pgfsetfillcolor{currentfill}%
\pgfsetlinewidth{0.000000pt}%
\definecolor{currentstroke}{rgb}{0.000000,0.000000,0.000000}%
\pgfsetstrokecolor{currentstroke}%
\pgfsetdash{}{0pt}%
\pgfpathmoveto{\pgfqpoint{1.147226in}{1.378248in}}%
\pgfpathlineto{\pgfqpoint{1.146244in}{1.367654in}}%
\pgfpathlineto{\pgfqpoint{1.145314in}{1.356945in}}%
\pgfpathlineto{\pgfqpoint{1.144437in}{1.346164in}}%
\pgfpathlineto{\pgfqpoint{1.143620in}{1.335357in}}%
\pgfpathlineto{\pgfqpoint{1.142864in}{1.324570in}}%
\pgfpathlineto{\pgfqpoint{1.128521in}{1.325171in}}%
\pgfpathlineto{\pgfqpoint{1.114142in}{1.325249in}}%
\pgfpathlineto{\pgfqpoint{1.099782in}{1.324804in}}%
\pgfpathlineto{\pgfqpoint{1.085497in}{1.323838in}}%
\pgfpathlineto{\pgfqpoint{1.071340in}{1.322355in}}%
\pgfpathlineto{\pgfqpoint{1.069815in}{1.333074in}}%
\pgfpathlineto{\pgfqpoint{1.068164in}{1.343807in}}%
\pgfpathlineto{\pgfqpoint{1.066396in}{1.354510in}}%
\pgfpathlineto{\pgfqpoint{1.064516in}{1.365136in}}%
\pgfpathlineto{\pgfqpoint{1.062534in}{1.375642in}}%
\pgfpathlineto{\pgfqpoint{1.079299in}{1.377387in}}%
\pgfpathlineto{\pgfqpoint{1.096215in}{1.378523in}}%
\pgfpathlineto{\pgfqpoint{1.113218in}{1.379046in}}%
\pgfpathlineto{\pgfqpoint{1.130243in}{1.378954in}}%
\pgfpathlineto{\pgfqpoint{1.147226in}{1.378248in}}%
\pgfpathclose%
\pgfusepath{fill}%
\end{pgfscope}%
\begin{pgfscope}%
\pgfpathrectangle{\pgfqpoint{0.050000in}{0.050000in}}{\pgfqpoint{2.081932in}{2.081932in}}%
\pgfusepath{clip}%
\pgfsetbuttcap%
\pgfsetroundjoin%
\definecolor{currentfill}{rgb}{0.296479,0.761561,0.424223}%
\pgfsetfillcolor{currentfill}%
\pgfsetlinewidth{0.000000pt}%
\definecolor{currentstroke}{rgb}{0.000000,0.000000,0.000000}%
\pgfsetstrokecolor{currentstroke}%
\pgfsetdash{}{0pt}%
\pgfpathmoveto{\pgfqpoint{1.142864in}{1.324570in}}%
\pgfpathlineto{\pgfqpoint{1.142173in}{1.313849in}}%
\pgfpathlineto{\pgfqpoint{1.141551in}{1.303238in}}%
\pgfpathlineto{\pgfqpoint{1.140999in}{1.292783in}}%
\pgfpathlineto{\pgfqpoint{1.140521in}{1.282527in}}%
\pgfpathlineto{\pgfqpoint{1.140117in}{1.272513in}}%
\pgfpathlineto{\pgfqpoint{1.127436in}{1.273048in}}%
\pgfpathlineto{\pgfqpoint{1.114724in}{1.273117in}}%
\pgfpathlineto{\pgfqpoint{1.102028in}{1.272722in}}%
\pgfpathlineto{\pgfqpoint{1.089399in}{1.271862in}}%
\pgfpathlineto{\pgfqpoint{1.076884in}{1.270542in}}%
\pgfpathlineto{\pgfqpoint{1.076070in}{1.280520in}}%
\pgfpathlineto{\pgfqpoint{1.075104in}{1.290733in}}%
\pgfpathlineto{\pgfqpoint{1.073990in}{1.301140in}}%
\pgfpathlineto{\pgfqpoint{1.072734in}{1.311695in}}%
\pgfpathlineto{\pgfqpoint{1.071340in}{1.322355in}}%
\pgfpathlineto{\pgfqpoint{1.085497in}{1.323838in}}%
\pgfpathlineto{\pgfqpoint{1.099782in}{1.324804in}}%
\pgfpathlineto{\pgfqpoint{1.114142in}{1.325249in}}%
\pgfpathlineto{\pgfqpoint{1.128521in}{1.325171in}}%
\pgfpathlineto{\pgfqpoint{1.142864in}{1.324570in}}%
\pgfpathclose%
\pgfusepath{fill}%
\end{pgfscope}%
\begin{pgfscope}%
\pgfpathrectangle{\pgfqpoint{0.050000in}{0.050000in}}{\pgfqpoint{2.081932in}{2.081932in}}%
\pgfusepath{clip}%
\pgfsetbuttcap%
\pgfsetroundjoin%
\definecolor{currentfill}{rgb}{0.606045,0.850733,0.236712}%
\pgfsetfillcolor{currentfill}%
\pgfsetlinewidth{0.000000pt}%
\definecolor{currentstroke}{rgb}{0.000000,0.000000,0.000000}%
\pgfsetstrokecolor{currentstroke}%
\pgfsetdash{}{0pt}%
\pgfpathmoveto{\pgfqpoint{1.229272in}{1.365664in}}%
\pgfpathlineto{\pgfqpoint{1.225414in}{1.355495in}}%
\pgfpathlineto{\pgfqpoint{1.221756in}{1.345188in}}%
\pgfpathlineto{\pgfqpoint{1.218313in}{1.334786in}}%
\pgfpathlineto{\pgfqpoint{1.215101in}{1.324333in}}%
\pgfpathlineto{\pgfqpoint{1.212132in}{1.313873in}}%
\pgfpathlineto{\pgfqpoint{1.198784in}{1.317016in}}%
\pgfpathlineto{\pgfqpoint{1.185130in}{1.319665in}}%
\pgfpathlineto{\pgfqpoint{1.171223in}{1.321812in}}%
\pgfpathlineto{\pgfqpoint{1.157116in}{1.323449in}}%
\pgfpathlineto{\pgfqpoint{1.142864in}{1.324570in}}%
\pgfpathlineto{\pgfqpoint{1.143620in}{1.335357in}}%
\pgfpathlineto{\pgfqpoint{1.144437in}{1.346164in}}%
\pgfpathlineto{\pgfqpoint{1.145314in}{1.356945in}}%
\pgfpathlineto{\pgfqpoint{1.146244in}{1.367654in}}%
\pgfpathlineto{\pgfqpoint{1.147226in}{1.378248in}}%
\pgfpathlineto{\pgfqpoint{1.164103in}{1.376930in}}%
\pgfpathlineto{\pgfqpoint{1.180809in}{1.375004in}}%
\pgfpathlineto{\pgfqpoint{1.197281in}{1.372478in}}%
\pgfpathlineto{\pgfqpoint{1.213456in}{1.369361in}}%
\pgfpathlineto{\pgfqpoint{1.229272in}{1.365664in}}%
\pgfpathclose%
\pgfusepath{fill}%
\end{pgfscope}%
\begin{pgfscope}%
\pgfpathrectangle{\pgfqpoint{0.050000in}{0.050000in}}{\pgfqpoint{2.081932in}{2.081932in}}%
\pgfusepath{clip}%
\pgfsetbuttcap%
\pgfsetroundjoin%
\definecolor{currentfill}{rgb}{0.993248,0.906157,0.143936}%
\pgfsetfillcolor{currentfill}%
\pgfsetlinewidth{0.000000pt}%
\definecolor{currentstroke}{rgb}{0.000000,0.000000,0.000000}%
\pgfsetstrokecolor{currentstroke}%
\pgfsetdash{}{0pt}%
\pgfpathmoveto{\pgfqpoint{0.926709in}{1.440344in}}%
\pgfpathlineto{\pgfqpoint{0.932721in}{1.434018in}}%
\pgfpathlineto{\pgfqpoint{0.938718in}{1.427187in}}%
\pgfpathlineto{\pgfqpoint{0.944674in}{1.419880in}}%
\pgfpathlineto{\pgfqpoint{0.950565in}{1.412124in}}%
\pgfpathlineto{\pgfqpoint{0.956366in}{1.403953in}}%
\pgfpathlineto{\pgfqpoint{0.938926in}{1.397772in}}%
\pgfpathlineto{\pgfqpoint{0.922170in}{1.390969in}}%
\pgfpathlineto{\pgfqpoint{0.906163in}{1.383568in}}%
\pgfpathlineto{\pgfqpoint{0.890968in}{1.375597in}}%
\pgfpathlineto{\pgfqpoint{0.876647in}{1.367084in}}%
\pgfpathlineto{\pgfqpoint{0.867963in}{1.373975in}}%
\pgfpathlineto{\pgfqpoint{0.859142in}{1.380429in}}%
\pgfpathlineto{\pgfqpoint{0.850221in}{1.386421in}}%
\pgfpathlineto{\pgfqpoint{0.841235in}{1.391926in}}%
\pgfpathlineto{\pgfqpoint{0.832224in}{1.396922in}}%
\pgfpathlineto{\pgfqpoint{0.849222in}{1.406955in}}%
\pgfpathlineto{\pgfqpoint{0.867242in}{1.416346in}}%
\pgfpathlineto{\pgfqpoint{0.886214in}{1.425060in}}%
\pgfpathlineto{\pgfqpoint{0.906062in}{1.433069in}}%
\pgfpathlineto{\pgfqpoint{0.926709in}{1.440344in}}%
\pgfpathclose%
\pgfusepath{fill}%
\end{pgfscope}%
\begin{pgfscope}%
\pgfpathrectangle{\pgfqpoint{0.050000in}{0.050000in}}{\pgfqpoint{2.081932in}{2.081932in}}%
\pgfusepath{clip}%
\pgfsetbuttcap%
\pgfsetroundjoin%
\definecolor{currentfill}{rgb}{0.296479,0.761561,0.424223}%
\pgfsetfillcolor{currentfill}%
\pgfsetlinewidth{0.000000pt}%
\definecolor{currentstroke}{rgb}{0.000000,0.000000,0.000000}%
\pgfsetstrokecolor{currentstroke}%
\pgfsetdash{}{0pt}%
\pgfpathmoveto{\pgfqpoint{1.212132in}{1.313873in}}%
\pgfpathlineto{\pgfqpoint{1.209420in}{1.303450in}}%
\pgfpathlineto{\pgfqpoint{1.206975in}{1.293107in}}%
\pgfpathlineto{\pgfqpoint{1.204809in}{1.282889in}}%
\pgfpathlineto{\pgfqpoint{1.202929in}{1.272838in}}%
\pgfpathlineto{\pgfqpoint{1.201345in}{1.262996in}}%
\pgfpathlineto{\pgfqpoint{1.189549in}{1.265792in}}%
\pgfpathlineto{\pgfqpoint{1.177481in}{1.268149in}}%
\pgfpathlineto{\pgfqpoint{1.165188in}{1.270060in}}%
\pgfpathlineto{\pgfqpoint{1.152717in}{1.271516in}}%
\pgfpathlineto{\pgfqpoint{1.140117in}{1.272513in}}%
\pgfpathlineto{\pgfqpoint{1.140521in}{1.282527in}}%
\pgfpathlineto{\pgfqpoint{1.140999in}{1.292783in}}%
\pgfpathlineto{\pgfqpoint{1.141551in}{1.303238in}}%
\pgfpathlineto{\pgfqpoint{1.142173in}{1.313849in}}%
\pgfpathlineto{\pgfqpoint{1.142864in}{1.324570in}}%
\pgfpathlineto{\pgfqpoint{1.157116in}{1.323449in}}%
\pgfpathlineto{\pgfqpoint{1.171223in}{1.321812in}}%
\pgfpathlineto{\pgfqpoint{1.185130in}{1.319665in}}%
\pgfpathlineto{\pgfqpoint{1.198784in}{1.317016in}}%
\pgfpathlineto{\pgfqpoint{1.212132in}{1.313873in}}%
\pgfpathclose%
\pgfusepath{fill}%
\end{pgfscope}%
\begin{pgfscope}%
\pgfpathrectangle{\pgfqpoint{0.050000in}{0.050000in}}{\pgfqpoint{2.081932in}{2.081932in}}%
\pgfusepath{clip}%
\pgfsetbuttcap%
\pgfsetroundjoin%
\definecolor{currentfill}{rgb}{0.150476,0.504369,0.557430}%
\pgfsetfillcolor{currentfill}%
\pgfsetlinewidth{0.000000pt}%
\definecolor{currentstroke}{rgb}{0.000000,0.000000,0.000000}%
\pgfsetstrokecolor{currentstroke}%
\pgfsetdash{}{0pt}%
\pgfpathmoveto{\pgfqpoint{0.597469in}{1.218993in}}%
\pgfpathlineto{\pgfqpoint{0.594784in}{1.227565in}}%
\pgfpathlineto{\pgfqpoint{0.592777in}{1.236070in}}%
\pgfpathlineto{\pgfqpoint{0.591458in}{1.244473in}}%
\pgfpathlineto{\pgfqpoint{0.590833in}{1.252741in}}%
\pgfpathlineto{\pgfqpoint{0.576157in}{1.233052in}}%
\pgfpathlineto{\pgfqpoint{0.563569in}{1.212812in}}%
\pgfpathlineto{\pgfqpoint{0.553134in}{1.192096in}}%
\pgfpathlineto{\pgfqpoint{0.544911in}{1.170980in}}%
\pgfpathlineto{\pgfqpoint{0.538947in}{1.149546in}}%
\pgfpathlineto{\pgfqpoint{0.539643in}{1.141295in}}%
\pgfpathlineto{\pgfqpoint{0.541110in}{1.133046in}}%
\pgfpathlineto{\pgfqpoint{0.543341in}{1.124831in}}%
\pgfpathlineto{\pgfqpoint{0.546327in}{1.116683in}}%
\pgfpathlineto{\pgfqpoint{0.552195in}{1.137931in}}%
\pgfpathlineto{\pgfqpoint{0.560296in}{1.158864in}}%
\pgfpathlineto{\pgfqpoint{0.570581in}{1.179403in}}%
\pgfpathlineto{\pgfqpoint{0.582993in}{1.199470in}}%
\pgfpathlineto{\pgfqpoint{0.597469in}{1.218993in}}%
\pgfpathclose%
\pgfusepath{fill}%
\end{pgfscope}%
\begin{pgfscope}%
\pgfpathrectangle{\pgfqpoint{0.050000in}{0.050000in}}{\pgfqpoint{2.081932in}{2.081932in}}%
\pgfusepath{clip}%
\pgfsetbuttcap%
\pgfsetroundjoin%
\definecolor{currentfill}{rgb}{0.606045,0.850733,0.236712}%
\pgfsetfillcolor{currentfill}%
\pgfsetlinewidth{0.000000pt}%
\definecolor{currentstroke}{rgb}{0.000000,0.000000,0.000000}%
\pgfsetstrokecolor{currentstroke}%
\pgfsetdash{}{0pt}%
\pgfpathmoveto{\pgfqpoint{1.062534in}{1.375642in}}%
\pgfpathlineto{\pgfqpoint{1.064516in}{1.365136in}}%
\pgfpathlineto{\pgfqpoint{1.066396in}{1.354510in}}%
\pgfpathlineto{\pgfqpoint{1.068164in}{1.343807in}}%
\pgfpathlineto{\pgfqpoint{1.069815in}{1.333074in}}%
\pgfpathlineto{\pgfqpoint{1.071340in}{1.322355in}}%
\pgfpathlineto{\pgfqpoint{1.057367in}{1.320359in}}%
\pgfpathlineto{\pgfqpoint{1.043631in}{1.317858in}}%
\pgfpathlineto{\pgfqpoint{1.030186in}{1.314862in}}%
\pgfpathlineto{\pgfqpoint{1.017082in}{1.311381in}}%
\pgfpathlineto{\pgfqpoint{1.004371in}{1.307427in}}%
\pgfpathlineto{\pgfqpoint{1.000703in}{1.317690in}}%
\pgfpathlineto{\pgfqpoint{0.996733in}{1.327930in}}%
\pgfpathlineto{\pgfqpoint{0.992478in}{1.338102in}}%
\pgfpathlineto{\pgfqpoint{0.987957in}{1.348166in}}%
\pgfpathlineto{\pgfqpoint{0.983188in}{1.358078in}}%
\pgfpathlineto{\pgfqpoint{0.998255in}{1.362731in}}%
\pgfpathlineto{\pgfqpoint{1.013784in}{1.366827in}}%
\pgfpathlineto{\pgfqpoint{1.029714in}{1.370353in}}%
\pgfpathlineto{\pgfqpoint{1.045985in}{1.373295in}}%
\pgfpathlineto{\pgfqpoint{1.062534in}{1.375642in}}%
\pgfpathclose%
\pgfusepath{fill}%
\end{pgfscope}%
\begin{pgfscope}%
\pgfpathrectangle{\pgfqpoint{0.050000in}{0.050000in}}{\pgfqpoint{2.081932in}{2.081932in}}%
\pgfusepath{clip}%
\pgfsetbuttcap%
\pgfsetroundjoin%
\definecolor{currentfill}{rgb}{0.120638,0.625828,0.533488}%
\pgfsetfillcolor{currentfill}%
\pgfsetlinewidth{0.000000pt}%
\definecolor{currentstroke}{rgb}{0.000000,0.000000,0.000000}%
\pgfsetstrokecolor{currentstroke}%
\pgfsetdash{}{0pt}%
\pgfpathmoveto{\pgfqpoint{1.254680in}{1.242864in}}%
\pgfpathlineto{\pgfqpoint{1.252558in}{1.233564in}}%
\pgfpathlineto{\pgfqpoint{1.250942in}{1.224481in}}%
\pgfpathlineto{\pgfqpoint{1.249839in}{1.215652in}}%
\pgfpathlineto{\pgfqpoint{1.249252in}{1.207114in}}%
\pgfpathlineto{\pgfqpoint{1.249183in}{1.198901in}}%
\pgfpathlineto{\pgfqpoint{1.239811in}{1.203531in}}%
\pgfpathlineto{\pgfqpoint{1.229969in}{1.207803in}}%
\pgfpathlineto{\pgfqpoint{1.219695in}{1.211702in}}%
\pgfpathlineto{\pgfqpoint{1.209032in}{1.215214in}}%
\pgfpathlineto{\pgfqpoint{1.198019in}{1.218324in}}%
\pgfpathlineto{\pgfqpoint{1.198061in}{1.226527in}}%
\pgfpathlineto{\pgfqpoint{1.198416in}{1.235132in}}%
\pgfpathlineto{\pgfqpoint{1.199083in}{1.244104in}}%
\pgfpathlineto{\pgfqpoint{1.200061in}{1.253405in}}%
\pgfpathlineto{\pgfqpoint{1.201345in}{1.262996in}}%
\pgfpathlineto{\pgfqpoint{1.212823in}{1.259773in}}%
\pgfpathlineto{\pgfqpoint{1.223938in}{1.256133in}}%
\pgfpathlineto{\pgfqpoint{1.234647in}{1.252092in}}%
\pgfpathlineto{\pgfqpoint{1.244908in}{1.247663in}}%
\pgfpathlineto{\pgfqpoint{1.254680in}{1.242864in}}%
\pgfpathclose%
\pgfusepath{fill}%
\end{pgfscope}%
\begin{pgfscope}%
\pgfpathrectangle{\pgfqpoint{0.050000in}{0.050000in}}{\pgfqpoint{2.081932in}{2.081932in}}%
\pgfusepath{clip}%
\pgfsetbuttcap%
\pgfsetroundjoin%
\definecolor{currentfill}{rgb}{0.296479,0.761561,0.424223}%
\pgfsetfillcolor{currentfill}%
\pgfsetlinewidth{0.000000pt}%
\definecolor{currentstroke}{rgb}{0.000000,0.000000,0.000000}%
\pgfsetstrokecolor{currentstroke}%
\pgfsetdash{}{0pt}%
\pgfpathmoveto{\pgfqpoint{1.071340in}{1.322355in}}%
\pgfpathlineto{\pgfqpoint{1.072734in}{1.311695in}}%
\pgfpathlineto{\pgfqpoint{1.073990in}{1.301140in}}%
\pgfpathlineto{\pgfqpoint{1.075104in}{1.290733in}}%
\pgfpathlineto{\pgfqpoint{1.076070in}{1.280520in}}%
\pgfpathlineto{\pgfqpoint{1.076884in}{1.270542in}}%
\pgfpathlineto{\pgfqpoint{1.064533in}{1.268766in}}%
\pgfpathlineto{\pgfqpoint{1.052392in}{1.266541in}}%
\pgfpathlineto{\pgfqpoint{1.040509in}{1.263876in}}%
\pgfpathlineto{\pgfqpoint{1.028930in}{1.260779in}}%
\pgfpathlineto{\pgfqpoint{1.017701in}{1.257262in}}%
\pgfpathlineto{\pgfqpoint{1.015743in}{1.267001in}}%
\pgfpathlineto{\pgfqpoint{1.013420in}{1.276928in}}%
\pgfpathlineto{\pgfqpoint{1.010744in}{1.287003in}}%
\pgfpathlineto{\pgfqpoint{1.007723in}{1.297184in}}%
\pgfpathlineto{\pgfqpoint{1.004371in}{1.307427in}}%
\pgfpathlineto{\pgfqpoint{1.017082in}{1.311381in}}%
\pgfpathlineto{\pgfqpoint{1.030186in}{1.314862in}}%
\pgfpathlineto{\pgfqpoint{1.043631in}{1.317858in}}%
\pgfpathlineto{\pgfqpoint{1.057367in}{1.320359in}}%
\pgfpathlineto{\pgfqpoint{1.071340in}{1.322355in}}%
\pgfpathclose%
\pgfusepath{fill}%
\end{pgfscope}%
\begin{pgfscope}%
\pgfpathrectangle{\pgfqpoint{0.050000in}{0.050000in}}{\pgfqpoint{2.081932in}{2.081932in}}%
\pgfusepath{clip}%
\pgfsetbuttcap%
\pgfsetroundjoin%
\definecolor{currentfill}{rgb}{0.876168,0.891125,0.095250}%
\pgfsetfillcolor{currentfill}%
\pgfsetlinewidth{0.000000pt}%
\definecolor{currentstroke}{rgb}{0.000000,0.000000,0.000000}%
\pgfsetstrokecolor{currentstroke}%
\pgfsetdash{}{0pt}%
\pgfpathmoveto{\pgfqpoint{0.748641in}{1.415728in}}%
\pgfpathlineto{\pgfqpoint{0.755948in}{1.416513in}}%
\pgfpathlineto{\pgfqpoint{0.763575in}{1.416700in}}%
\pgfpathlineto{\pgfqpoint{0.771491in}{1.416286in}}%
\pgfpathlineto{\pgfqpoint{0.779665in}{1.415273in}}%
\pgfpathlineto{\pgfqpoint{0.788064in}{1.413662in}}%
\pgfpathlineto{\pgfqpoint{0.769637in}{1.401409in}}%
\pgfpathlineto{\pgfqpoint{0.752534in}{1.388496in}}%
\pgfpathlineto{\pgfqpoint{0.736825in}{1.374968in}}%
\pgfpathlineto{\pgfqpoint{0.722579in}{1.360875in}}%
\pgfpathlineto{\pgfqpoint{0.709858in}{1.346268in}}%
\pgfpathlineto{\pgfqpoint{0.699384in}{1.346168in}}%
\pgfpathlineto{\pgfqpoint{0.689186in}{1.345512in}}%
\pgfpathlineto{\pgfqpoint{0.679306in}{1.344303in}}%
\pgfpathlineto{\pgfqpoint{0.669783in}{1.342547in}}%
\pgfpathlineto{\pgfqpoint{0.660657in}{1.340253in}}%
\pgfpathlineto{\pgfqpoint{0.675005in}{1.356625in}}%
\pgfpathlineto{\pgfqpoint{0.691051in}{1.372413in}}%
\pgfpathlineto{\pgfqpoint{0.708724in}{1.387562in}}%
\pgfpathlineto{\pgfqpoint{0.727948in}{1.402017in}}%
\pgfpathlineto{\pgfqpoint{0.748641in}{1.415728in}}%
\pgfpathclose%
\pgfusepath{fill}%
\end{pgfscope}%
\begin{pgfscope}%
\pgfpathrectangle{\pgfqpoint{0.050000in}{0.050000in}}{\pgfqpoint{2.081932in}{2.081932in}}%
\pgfusepath{clip}%
\pgfsetbuttcap%
\pgfsetroundjoin%
\definecolor{currentfill}{rgb}{0.278012,0.180367,0.486697}%
\pgfsetfillcolor{currentfill}%
\pgfsetlinewidth{0.000000pt}%
\definecolor{currentstroke}{rgb}{0.000000,0.000000,0.000000}%
\pgfsetstrokecolor{currentstroke}%
\pgfsetdash{}{0pt}%
\pgfpathmoveto{\pgfqpoint{0.904383in}{1.094022in}}%
\pgfpathlineto{\pgfqpoint{0.897306in}{1.089617in}}%
\pgfpathlineto{\pgfqpoint{0.889705in}{1.085696in}}%
\pgfpathlineto{\pgfqpoint{0.881613in}{1.082274in}}%
\pgfpathlineto{\pgfqpoint{0.873063in}{1.079363in}}%
\pgfpathlineto{\pgfqpoint{0.864090in}{1.076975in}}%
\pgfpathlineto{\pgfqpoint{0.857336in}{1.067274in}}%
\pgfpathlineto{\pgfqpoint{0.851614in}{1.057326in}}%
\pgfpathlineto{\pgfqpoint{0.846951in}{1.047172in}}%
\pgfpathlineto{\pgfqpoint{0.843369in}{1.036850in}}%
\pgfpathlineto{\pgfqpoint{0.840887in}{1.026400in}}%
\pgfpathlineto{\pgfqpoint{0.850734in}{1.030592in}}%
\pgfpathlineto{\pgfqpoint{0.860114in}{1.035223in}}%
\pgfpathlineto{\pgfqpoint{0.868987in}{1.040276in}}%
\pgfpathlineto{\pgfqpoint{0.877318in}{1.045731in}}%
\pgfpathlineto{\pgfqpoint{0.885073in}{1.051566in}}%
\pgfpathlineto{\pgfqpoint{0.887115in}{1.060331in}}%
\pgfpathlineto{\pgfqpoint{0.890083in}{1.068993in}}%
\pgfpathlineto{\pgfqpoint{0.893963in}{1.077518in}}%
\pgfpathlineto{\pgfqpoint{0.898737in}{1.085872in}}%
\pgfpathlineto{\pgfqpoint{0.904383in}{1.094022in}}%
\pgfpathclose%
\pgfusepath{fill}%
\end{pgfscope}%
\begin{pgfscope}%
\pgfpathrectangle{\pgfqpoint{0.050000in}{0.050000in}}{\pgfqpoint{2.081932in}{2.081932in}}%
\pgfusepath{clip}%
\pgfsetbuttcap%
\pgfsetroundjoin%
\definecolor{currentfill}{rgb}{0.162142,0.474838,0.558140}%
\pgfsetfillcolor{currentfill}%
\pgfsetlinewidth{0.000000pt}%
\definecolor{currentstroke}{rgb}{0.000000,0.000000,0.000000}%
\pgfsetstrokecolor{currentstroke}%
\pgfsetdash{}{0pt}%
\pgfpathmoveto{\pgfqpoint{1.287752in}{1.171049in}}%
\pgfpathlineto{\pgfqpoint{1.288337in}{1.163071in}}%
\pgfpathlineto{\pgfqpoint{1.289591in}{1.155375in}}%
\pgfpathlineto{\pgfqpoint{1.291508in}{1.147992in}}%
\pgfpathlineto{\pgfqpoint{1.294081in}{1.140950in}}%
\pgfpathlineto{\pgfqpoint{1.297296in}{1.134278in}}%
\pgfpathlineto{\pgfqpoint{1.290421in}{1.140783in}}%
\pgfpathlineto{\pgfqpoint{1.282873in}{1.147014in}}%
\pgfpathlineto{\pgfqpoint{1.274684in}{1.152948in}}%
\pgfpathlineto{\pgfqpoint{1.265887in}{1.158562in}}%
\pgfpathlineto{\pgfqpoint{1.256517in}{1.163835in}}%
\pgfpathlineto{\pgfqpoint{1.254046in}{1.169951in}}%
\pgfpathlineto{\pgfqpoint{1.252070in}{1.176543in}}%
\pgfpathlineto{\pgfqpoint{1.250596in}{1.183584in}}%
\pgfpathlineto{\pgfqpoint{1.249632in}{1.191047in}}%
\pgfpathlineto{\pgfqpoint{1.249183in}{1.198901in}}%
\pgfpathlineto{\pgfqpoint{1.258048in}{1.193931in}}%
\pgfpathlineto{\pgfqpoint{1.266370in}{1.188640in}}%
\pgfpathlineto{\pgfqpoint{1.274116in}{1.183048in}}%
\pgfpathlineto{\pgfqpoint{1.281253in}{1.177177in}}%
\pgfpathlineto{\pgfqpoint{1.287752in}{1.171049in}}%
\pgfpathclose%
\pgfusepath{fill}%
\end{pgfscope}%
\begin{pgfscope}%
\pgfpathrectangle{\pgfqpoint{0.050000in}{0.050000in}}{\pgfqpoint{2.081932in}{2.081932in}}%
\pgfusepath{clip}%
\pgfsetbuttcap%
\pgfsetroundjoin%
\definecolor{currentfill}{rgb}{0.855810,0.888601,0.097452}%
\pgfsetfillcolor{currentfill}%
\pgfsetlinewidth{0.000000pt}%
\definecolor{currentstroke}{rgb}{0.000000,0.000000,0.000000}%
\pgfsetstrokecolor{currentstroke}%
\pgfsetdash{}{0pt}%
\pgfpathmoveto{\pgfqpoint{1.336835in}{1.381248in}}%
\pgfpathlineto{\pgfqpoint{1.329211in}{1.373465in}}%
\pgfpathlineto{\pgfqpoint{1.321773in}{1.365316in}}%
\pgfpathlineto{\pgfqpoint{1.314553in}{1.356834in}}%
\pgfpathlineto{\pgfqpoint{1.307581in}{1.348054in}}%
\pgfpathlineto{\pgfqpoint{1.300887in}{1.339013in}}%
\pgfpathlineto{\pgfqpoint{1.287749in}{1.345370in}}%
\pgfpathlineto{\pgfqpoint{1.273964in}{1.351235in}}%
\pgfpathlineto{\pgfqpoint{1.259585in}{1.356584in}}%
\pgfpathlineto{\pgfqpoint{1.244668in}{1.361400in}}%
\pgfpathlineto{\pgfqpoint{1.229272in}{1.365664in}}%
\pgfpathlineto{\pgfqpoint{1.233314in}{1.375652in}}%
\pgfpathlineto{\pgfqpoint{1.237523in}{1.385418in}}%
\pgfpathlineto{\pgfqpoint{1.241880in}{1.394921in}}%
\pgfpathlineto{\pgfqpoint{1.246367in}{1.404121in}}%
\pgfpathlineto{\pgfqpoint{1.250966in}{1.412981in}}%
\pgfpathlineto{\pgfqpoint{1.269411in}{1.407907in}}%
\pgfpathlineto{\pgfqpoint{1.287289in}{1.402175in}}%
\pgfpathlineto{\pgfqpoint{1.304530in}{1.395805in}}%
\pgfpathlineto{\pgfqpoint{1.321067in}{1.388821in}}%
\pgfpathlineto{\pgfqpoint{1.336835in}{1.381248in}}%
\pgfpathclose%
\pgfusepath{fill}%
\end{pgfscope}%
\begin{pgfscope}%
\pgfpathrectangle{\pgfqpoint{0.050000in}{0.050000in}}{\pgfqpoint{2.081932in}{2.081932in}}%
\pgfusepath{clip}%
\pgfsetbuttcap%
\pgfsetroundjoin%
\definecolor{currentfill}{rgb}{0.993248,0.906157,0.143936}%
\pgfsetfillcolor{currentfill}%
\pgfsetlinewidth{0.000000pt}%
\definecolor{currentstroke}{rgb}{0.000000,0.000000,0.000000}%
\pgfsetstrokecolor{currentstroke}%
\pgfsetdash{}{0pt}%
\pgfpathmoveto{\pgfqpoint{1.505954in}{1.370819in}}%
\pgfpathlineto{\pgfqpoint{1.495865in}{1.369722in}}%
\pgfpathlineto{\pgfqpoint{1.485596in}{1.368059in}}%
\pgfpathlineto{\pgfqpoint{1.475187in}{1.365835in}}%
\pgfpathlineto{\pgfqpoint{1.464682in}{1.363059in}}%
\pgfpathlineto{\pgfqpoint{1.454124in}{1.359742in}}%
\pgfpathlineto{\pgfqpoint{1.440971in}{1.371626in}}%
\pgfpathlineto{\pgfqpoint{1.426582in}{1.382996in}}%
\pgfpathlineto{\pgfqpoint{1.411018in}{1.393809in}}%
\pgfpathlineto{\pgfqpoint{1.394342in}{1.404027in}}%
\pgfpathlineto{\pgfqpoint{1.376622in}{1.413612in}}%
\pgfpathlineto{\pgfqpoint{1.384681in}{1.418593in}}%
\pgfpathlineto{\pgfqpoint{1.392695in}{1.423025in}}%
\pgfpathlineto{\pgfqpoint{1.400633in}{1.426892in}}%
\pgfpathlineto{\pgfqpoint{1.408461in}{1.430179in}}%
\pgfpathlineto{\pgfqpoint{1.416148in}{1.432874in}}%
\pgfpathlineto{\pgfqpoint{1.436647in}{1.421841in}}%
\pgfpathlineto{\pgfqpoint{1.455954in}{1.410077in}}%
\pgfpathlineto{\pgfqpoint{1.473989in}{1.397622in}}%
\pgfpathlineto{\pgfqpoint{1.490680in}{1.384521in}}%
\pgfpathlineto{\pgfqpoint{1.505954in}{1.370819in}}%
\pgfpathclose%
\pgfusepath{fill}%
\end{pgfscope}%
\begin{pgfscope}%
\pgfpathrectangle{\pgfqpoint{0.050000in}{0.050000in}}{\pgfqpoint{2.081932in}{2.081932in}}%
\pgfusepath{clip}%
\pgfsetbuttcap%
\pgfsetroundjoin%
\definecolor{currentfill}{rgb}{0.327796,0.773980,0.406640}%
\pgfsetfillcolor{currentfill}%
\pgfsetlinewidth{0.000000pt}%
\definecolor{currentstroke}{rgb}{0.000000,0.000000,0.000000}%
\pgfsetstrokecolor{currentstroke}%
\pgfsetdash{}{0pt}%
\pgfpathmoveto{\pgfqpoint{1.679973in}{1.225266in}}%
\pgfpathlineto{\pgfqpoint{1.676157in}{1.232559in}}%
\pgfpathlineto{\pgfqpoint{1.671625in}{1.239583in}}%
\pgfpathlineto{\pgfqpoint{1.666393in}{1.246308in}}%
\pgfpathlineto{\pgfqpoint{1.660481in}{1.252708in}}%
\pgfpathlineto{\pgfqpoint{1.653911in}{1.258755in}}%
\pgfpathlineto{\pgfqpoint{1.644803in}{1.278180in}}%
\pgfpathlineto{\pgfqpoint{1.633647in}{1.297188in}}%
\pgfpathlineto{\pgfqpoint{1.620503in}{1.315708in}}%
\pgfpathlineto{\pgfqpoint{1.605433in}{1.333672in}}%
\pgfpathlineto{\pgfqpoint{1.588510in}{1.351016in}}%
\pgfpathlineto{\pgfqpoint{1.594209in}{1.346159in}}%
\pgfpathlineto{\pgfqpoint{1.599336in}{1.340843in}}%
\pgfpathlineto{\pgfqpoint{1.603873in}{1.335090in}}%
\pgfpathlineto{\pgfqpoint{1.607802in}{1.328925in}}%
\pgfpathlineto{\pgfqpoint{1.611109in}{1.322372in}}%
\pgfpathlineto{\pgfqpoint{1.628900in}{1.304125in}}%
\pgfpathlineto{\pgfqpoint{1.644754in}{1.285222in}}%
\pgfpathlineto{\pgfqpoint{1.658595in}{1.265730in}}%
\pgfpathlineto{\pgfqpoint{1.670355in}{1.245720in}}%
\pgfpathlineto{\pgfqpoint{1.679973in}{1.225266in}}%
\pgfpathclose%
\pgfusepath{fill}%
\end{pgfscope}%
\begin{pgfscope}%
\pgfpathrectangle{\pgfqpoint{0.050000in}{0.050000in}}{\pgfqpoint{2.081932in}{2.081932in}}%
\pgfusepath{clip}%
\pgfsetbuttcap%
\pgfsetroundjoin%
\definecolor{currentfill}{rgb}{0.120638,0.625828,0.533488}%
\pgfsetfillcolor{currentfill}%
\pgfsetlinewidth{0.000000pt}%
\definecolor{currentstroke}{rgb}{0.000000,0.000000,0.000000}%
\pgfsetstrokecolor{currentstroke}%
\pgfsetdash{}{0pt}%
\pgfpathmoveto{\pgfqpoint{1.017701in}{1.257262in}}%
\pgfpathlineto{\pgfqpoint{1.019287in}{1.247754in}}%
\pgfpathlineto{\pgfqpoint{1.020494in}{1.238515in}}%
\pgfpathlineto{\pgfqpoint{1.021319in}{1.229584in}}%
\pgfpathlineto{\pgfqpoint{1.021757in}{1.220998in}}%
\pgfpathlineto{\pgfqpoint{1.021809in}{1.212792in}}%
\pgfpathlineto{\pgfqpoint{1.011413in}{1.209007in}}%
\pgfpathlineto{\pgfqpoint{1.001437in}{1.204844in}}%
\pgfpathlineto{\pgfqpoint{0.991919in}{1.200319in}}%
\pgfpathlineto{\pgfqpoint{0.982897in}{1.195449in}}%
\pgfpathlineto{\pgfqpoint{0.974408in}{1.190252in}}%
\pgfpathlineto{\pgfqpoint{0.974331in}{1.198469in}}%
\pgfpathlineto{\pgfqpoint{0.973677in}{1.206977in}}%
\pgfpathlineto{\pgfqpoint{0.972448in}{1.215742in}}%
\pgfpathlineto{\pgfqpoint{0.970647in}{1.224728in}}%
\pgfpathlineto{\pgfqpoint{0.968282in}{1.233898in}}%
\pgfpathlineto{\pgfqpoint{0.977135in}{1.239285in}}%
\pgfpathlineto{\pgfqpoint{0.986541in}{1.244334in}}%
\pgfpathlineto{\pgfqpoint{0.996464in}{1.249024in}}%
\pgfpathlineto{\pgfqpoint{1.006865in}{1.253339in}}%
\pgfpathlineto{\pgfqpoint{1.017701in}{1.257262in}}%
\pgfpathclose%
\pgfusepath{fill}%
\end{pgfscope}%
\begin{pgfscope}%
\pgfpathrectangle{\pgfqpoint{0.050000in}{0.050000in}}{\pgfqpoint{2.081932in}{2.081932in}}%
\pgfusepath{clip}%
\pgfsetbuttcap%
\pgfsetroundjoin%
\definecolor{currentfill}{rgb}{0.227802,0.326594,0.546532}%
\pgfsetfillcolor{currentfill}%
\pgfsetlinewidth{0.000000pt}%
\definecolor{currentstroke}{rgb}{0.000000,0.000000,0.000000}%
\pgfsetstrokecolor{currentstroke}%
\pgfsetdash{}{0pt}%
\pgfpathmoveto{\pgfqpoint{1.320666in}{1.098565in}}%
\pgfpathlineto{\pgfqpoint{1.325033in}{1.091487in}}%
\pgfpathlineto{\pgfqpoint{1.330097in}{1.084708in}}%
\pgfpathlineto{\pgfqpoint{1.335835in}{1.078254in}}%
\pgfpathlineto{\pgfqpoint{1.342224in}{1.072151in}}%
\pgfpathlineto{\pgfqpoint{1.349236in}{1.066421in}}%
\pgfpathlineto{\pgfqpoint{1.345629in}{1.074990in}}%
\pgfpathlineto{\pgfqpoint{1.341123in}{1.083398in}}%
\pgfpathlineto{\pgfqpoint{1.335737in}{1.091613in}}%
\pgfpathlineto{\pgfqpoint{1.329496in}{1.099601in}}%
\pgfpathlineto{\pgfqpoint{1.322427in}{1.107333in}}%
\pgfpathlineto{\pgfqpoint{1.316262in}{1.111794in}}%
\pgfpathlineto{\pgfqpoint{1.310643in}{1.116739in}}%
\pgfpathlineto{\pgfqpoint{1.305595in}{1.122149in}}%
\pgfpathlineto{\pgfqpoint{1.301139in}{1.128003in}}%
\pgfpathlineto{\pgfqpoint{1.297296in}{1.134278in}}%
\pgfpathlineto{\pgfqpoint{1.303469in}{1.127525in}}%
\pgfpathlineto{\pgfqpoint{1.308915in}{1.120550in}}%
\pgfpathlineto{\pgfqpoint{1.313609in}{1.113379in}}%
\pgfpathlineto{\pgfqpoint{1.317532in}{1.106041in}}%
\pgfpathlineto{\pgfqpoint{1.320666in}{1.098565in}}%
\pgfpathclose%
\pgfusepath{fill}%
\end{pgfscope}%
\begin{pgfscope}%
\pgfpathrectangle{\pgfqpoint{0.050000in}{0.050000in}}{\pgfqpoint{2.081932in}{2.081932in}}%
\pgfusepath{clip}%
\pgfsetbuttcap%
\pgfsetroundjoin%
\definecolor{currentfill}{rgb}{0.268510,0.009605,0.335427}%
\pgfsetfillcolor{currentfill}%
\pgfsetlinewidth{0.000000pt}%
\definecolor{currentstroke}{rgb}{0.000000,0.000000,0.000000}%
\pgfsetstrokecolor{currentstroke}%
\pgfsetdash{}{0pt}%
\pgfpathmoveto{\pgfqpoint{1.513384in}{0.961832in}}%
\pgfpathlineto{\pgfqpoint{1.525623in}{0.961593in}}%
\pgfpathlineto{\pgfqpoint{1.537793in}{0.961872in}}%
\pgfpathlineto{\pgfqpoint{1.549847in}{0.962670in}}%
\pgfpathlineto{\pgfqpoint{1.561737in}{0.963982in}}%
\pgfpathlineto{\pgfqpoint{1.573415in}{0.965806in}}%
\pgfpathlineto{\pgfqpoint{1.575045in}{0.983090in}}%
\pgfpathlineto{\pgfqpoint{1.574833in}{1.000364in}}%
\pgfpathlineto{\pgfqpoint{1.572791in}{1.017560in}}%
\pgfpathlineto{\pgfqpoint{1.568939in}{1.034609in}}%
\pgfpathlineto{\pgfqpoint{1.563303in}{1.051445in}}%
\pgfpathlineto{\pgfqpoint{1.551999in}{1.047460in}}%
\pgfpathlineto{\pgfqpoint{1.540484in}{1.043939in}}%
\pgfpathlineto{\pgfqpoint{1.528804in}{1.040896in}}%
\pgfpathlineto{\pgfqpoint{1.517005in}{1.038343in}}%
\pgfpathlineto{\pgfqpoint{1.505133in}{1.036290in}}%
\pgfpathlineto{\pgfqpoint{1.509926in}{1.021637in}}%
\pgfpathlineto{\pgfqpoint{1.513166in}{1.006805in}}%
\pgfpathlineto{\pgfqpoint{1.514831in}{0.991854in}}%
\pgfpathlineto{\pgfqpoint{1.514907in}{0.976843in}}%
\pgfpathlineto{\pgfqpoint{1.513384in}{0.961832in}}%
\pgfpathclose%
\pgfusepath{fill}%
\end{pgfscope}%
\begin{pgfscope}%
\pgfpathrectangle{\pgfqpoint{0.050000in}{0.050000in}}{\pgfqpoint{2.081932in}{2.081932in}}%
\pgfusepath{clip}%
\pgfsetbuttcap%
\pgfsetroundjoin%
\definecolor{currentfill}{rgb}{0.267004,0.004874,0.329415}%
\pgfsetfillcolor{currentfill}%
\pgfsetlinewidth{0.000000pt}%
\definecolor{currentstroke}{rgb}{0.000000,0.000000,0.000000}%
\pgfsetstrokecolor{currentstroke}%
\pgfsetdash{}{0pt}%
\pgfpathmoveto{\pgfqpoint{1.452841in}{0.970705in}}%
\pgfpathlineto{\pgfqpoint{1.464703in}{0.967928in}}%
\pgfpathlineto{\pgfqpoint{1.476736in}{0.965643in}}%
\pgfpathlineto{\pgfqpoint{1.488893in}{0.963862in}}%
\pgfpathlineto{\pgfqpoint{1.501125in}{0.962590in}}%
\pgfpathlineto{\pgfqpoint{1.513384in}{0.961832in}}%
\pgfpathlineto{\pgfqpoint{1.514907in}{0.976843in}}%
\pgfpathlineto{\pgfqpoint{1.514831in}{0.991854in}}%
\pgfpathlineto{\pgfqpoint{1.513166in}{1.006805in}}%
\pgfpathlineto{\pgfqpoint{1.509926in}{1.021637in}}%
\pgfpathlineto{\pgfqpoint{1.505133in}{1.036290in}}%
\pgfpathlineto{\pgfqpoint{1.493234in}{1.034745in}}%
\pgfpathlineto{\pgfqpoint{1.481355in}{1.033716in}}%
\pgfpathlineto{\pgfqpoint{1.469543in}{1.033205in}}%
\pgfpathlineto{\pgfqpoint{1.457845in}{1.033215in}}%
\pgfpathlineto{\pgfqpoint{1.446306in}{1.033747in}}%
\pgfpathlineto{\pgfqpoint{1.450279in}{1.021327in}}%
\pgfpathlineto{\pgfqpoint{1.452933in}{1.008762in}}%
\pgfpathlineto{\pgfqpoint{1.454251in}{0.996104in}}%
\pgfpathlineto{\pgfqpoint{1.454223in}{0.983401in}}%
\pgfpathlineto{\pgfqpoint{1.452841in}{0.970705in}}%
\pgfpathclose%
\pgfusepath{fill}%
\end{pgfscope}%
\begin{pgfscope}%
\pgfpathrectangle{\pgfqpoint{0.050000in}{0.050000in}}{\pgfqpoint{2.081932in}{2.081932in}}%
\pgfusepath{clip}%
\pgfsetbuttcap%
\pgfsetroundjoin%
\definecolor{currentfill}{rgb}{0.296479,0.761561,0.424223}%
\pgfsetfillcolor{currentfill}%
\pgfsetlinewidth{0.000000pt}%
\definecolor{currentstroke}{rgb}{0.000000,0.000000,0.000000}%
\pgfsetstrokecolor{currentstroke}%
\pgfsetdash{}{0pt}%
\pgfpathmoveto{\pgfqpoint{1.272519in}{1.291235in}}%
\pgfpathlineto{\pgfqpoint{1.268033in}{1.281446in}}%
\pgfpathlineto{\pgfqpoint{1.263990in}{1.271673in}}%
\pgfpathlineto{\pgfqpoint{1.260407in}{1.261958in}}%
\pgfpathlineto{\pgfqpoint{1.257300in}{1.252342in}}%
\pgfpathlineto{\pgfqpoint{1.254680in}{1.242864in}}%
\pgfpathlineto{\pgfqpoint{1.244908in}{1.247663in}}%
\pgfpathlineto{\pgfqpoint{1.234647in}{1.252092in}}%
\pgfpathlineto{\pgfqpoint{1.223938in}{1.256133in}}%
\pgfpathlineto{\pgfqpoint{1.212823in}{1.259773in}}%
\pgfpathlineto{\pgfqpoint{1.201345in}{1.262996in}}%
\pgfpathlineto{\pgfqpoint{1.202929in}{1.272838in}}%
\pgfpathlineto{\pgfqpoint{1.204809in}{1.282889in}}%
\pgfpathlineto{\pgfqpoint{1.206975in}{1.293107in}}%
\pgfpathlineto{\pgfqpoint{1.209420in}{1.303450in}}%
\pgfpathlineto{\pgfqpoint{1.212132in}{1.313873in}}%
\pgfpathlineto{\pgfqpoint{1.225123in}{1.310250in}}%
\pgfpathlineto{\pgfqpoint{1.237705in}{1.306158in}}%
\pgfpathlineto{\pgfqpoint{1.249830in}{1.301614in}}%
\pgfpathlineto{\pgfqpoint{1.261450in}{1.296633in}}%
\pgfpathlineto{\pgfqpoint{1.272519in}{1.291235in}}%
\pgfpathclose%
\pgfusepath{fill}%
\end{pgfscope}%
\begin{pgfscope}%
\pgfpathrectangle{\pgfqpoint{0.050000in}{0.050000in}}{\pgfqpoint{2.081932in}{2.081932in}}%
\pgfusepath{clip}%
\pgfsetbuttcap%
\pgfsetroundjoin%
\definecolor{currentfill}{rgb}{0.162142,0.474838,0.558140}%
\pgfsetfillcolor{currentfill}%
\pgfsetlinewidth{0.000000pt}%
\definecolor{currentstroke}{rgb}{0.000000,0.000000,0.000000}%
\pgfsetstrokecolor{currentstroke}%
\pgfsetdash{}{0pt}%
\pgfpathmoveto{\pgfqpoint{0.974408in}{1.190252in}}%
\pgfpathlineto{\pgfqpoint{0.973907in}{1.182360in}}%
\pgfpathlineto{\pgfqpoint{0.972833in}{1.174825in}}%
\pgfpathlineto{\pgfqpoint{0.971190in}{1.167677in}}%
\pgfpathlineto{\pgfqpoint{0.968987in}{1.160946in}}%
\pgfpathlineto{\pgfqpoint{0.966234in}{1.154657in}}%
\pgfpathlineto{\pgfqpoint{0.957857in}{1.148817in}}%
\pgfpathlineto{\pgfqpoint{0.950113in}{1.142672in}}%
\pgfpathlineto{\pgfqpoint{0.943032in}{1.136247in}}%
\pgfpathlineto{\pgfqpoint{0.936645in}{1.129565in}}%
\pgfpathlineto{\pgfqpoint{0.930977in}{1.122653in}}%
\pgfpathlineto{\pgfqpoint{0.934379in}{1.129545in}}%
\pgfpathlineto{\pgfqpoint{0.937100in}{1.136764in}}%
\pgfpathlineto{\pgfqpoint{0.939129in}{1.144283in}}%
\pgfpathlineto{\pgfqpoint{0.940455in}{1.152071in}}%
\pgfpathlineto{\pgfqpoint{0.941073in}{1.160097in}}%
\pgfpathlineto{\pgfqpoint{0.946429in}{1.166608in}}%
\pgfpathlineto{\pgfqpoint{0.952467in}{1.172903in}}%
\pgfpathlineto{\pgfqpoint{0.959161in}{1.178957in}}%
\pgfpathlineto{\pgfqpoint{0.966485in}{1.184748in}}%
\pgfpathlineto{\pgfqpoint{0.974408in}{1.190252in}}%
\pgfpathclose%
\pgfusepath{fill}%
\end{pgfscope}%
\begin{pgfscope}%
\pgfpathrectangle{\pgfqpoint{0.050000in}{0.050000in}}{\pgfqpoint{2.081932in}{2.081932in}}%
\pgfusepath{clip}%
\pgfsetbuttcap%
\pgfsetroundjoin%
\definecolor{currentfill}{rgb}{0.606045,0.850733,0.236712}%
\pgfsetfillcolor{currentfill}%
\pgfsetlinewidth{0.000000pt}%
\definecolor{currentstroke}{rgb}{0.000000,0.000000,0.000000}%
\pgfsetstrokecolor{currentstroke}%
\pgfsetdash{}{0pt}%
\pgfpathmoveto{\pgfqpoint{1.300887in}{1.339013in}}%
\pgfpathlineto{\pgfqpoint{1.294499in}{1.329748in}}%
\pgfpathlineto{\pgfqpoint{1.288443in}{1.320298in}}%
\pgfpathlineto{\pgfqpoint{1.282745in}{1.310702in}}%
\pgfpathlineto{\pgfqpoint{1.277430in}{1.301001in}}%
\pgfpathlineto{\pgfqpoint{1.272519in}{1.291235in}}%
\pgfpathlineto{\pgfqpoint{1.261450in}{1.296633in}}%
\pgfpathlineto{\pgfqpoint{1.249830in}{1.301614in}}%
\pgfpathlineto{\pgfqpoint{1.237705in}{1.306158in}}%
\pgfpathlineto{\pgfqpoint{1.225123in}{1.310250in}}%
\pgfpathlineto{\pgfqpoint{1.212132in}{1.313873in}}%
\pgfpathlineto{\pgfqpoint{1.215101in}{1.324333in}}%
\pgfpathlineto{\pgfqpoint{1.218313in}{1.334786in}}%
\pgfpathlineto{\pgfqpoint{1.221756in}{1.345188in}}%
\pgfpathlineto{\pgfqpoint{1.225414in}{1.355495in}}%
\pgfpathlineto{\pgfqpoint{1.229272in}{1.365664in}}%
\pgfpathlineto{\pgfqpoint{1.244668in}{1.361400in}}%
\pgfpathlineto{\pgfqpoint{1.259585in}{1.356584in}}%
\pgfpathlineto{\pgfqpoint{1.273964in}{1.351235in}}%
\pgfpathlineto{\pgfqpoint{1.287749in}{1.345370in}}%
\pgfpathlineto{\pgfqpoint{1.300887in}{1.339013in}}%
\pgfpathclose%
\pgfusepath{fill}%
\end{pgfscope}%
\begin{pgfscope}%
\pgfpathrectangle{\pgfqpoint{0.050000in}{0.050000in}}{\pgfqpoint{2.081932in}{2.081932in}}%
\pgfusepath{clip}%
\pgfsetbuttcap%
\pgfsetroundjoin%
\definecolor{currentfill}{rgb}{0.855810,0.888601,0.097452}%
\pgfsetfillcolor{currentfill}%
\pgfsetlinewidth{0.000000pt}%
\definecolor{currentstroke}{rgb}{0.000000,0.000000,0.000000}%
\pgfsetstrokecolor{currentstroke}%
\pgfsetdash{}{0pt}%
\pgfpathmoveto{\pgfqpoint{0.956366in}{1.403953in}}%
\pgfpathlineto{\pgfqpoint{0.962052in}{1.395399in}}%
\pgfpathlineto{\pgfqpoint{0.967601in}{1.386496in}}%
\pgfpathlineto{\pgfqpoint{0.972988in}{1.377283in}}%
\pgfpathlineto{\pgfqpoint{0.978191in}{1.367797in}}%
\pgfpathlineto{\pgfqpoint{0.983188in}{1.358078in}}%
\pgfpathlineto{\pgfqpoint{0.968640in}{1.352886in}}%
\pgfpathlineto{\pgfqpoint{0.954670in}{1.347173in}}%
\pgfpathlineto{\pgfqpoint{0.941331in}{1.340960in}}%
\pgfpathlineto{\pgfqpoint{0.928677in}{1.334270in}}%
\pgfpathlineto{\pgfqpoint{0.916758in}{1.327128in}}%
\pgfpathlineto{\pgfqpoint{0.909291in}{1.335745in}}%
\pgfpathlineto{\pgfqpoint{0.901513in}{1.344083in}}%
\pgfpathlineto{\pgfqpoint{0.893457in}{1.352107in}}%
\pgfpathlineto{\pgfqpoint{0.885156in}{1.359785in}}%
\pgfpathlineto{\pgfqpoint{0.876647in}{1.367084in}}%
\pgfpathlineto{\pgfqpoint{0.890968in}{1.375597in}}%
\pgfpathlineto{\pgfqpoint{0.906163in}{1.383568in}}%
\pgfpathlineto{\pgfqpoint{0.922170in}{1.390969in}}%
\pgfpathlineto{\pgfqpoint{0.938926in}{1.397772in}}%
\pgfpathlineto{\pgfqpoint{0.956366in}{1.403953in}}%
\pgfpathclose%
\pgfusepath{fill}%
\end{pgfscope}%
\begin{pgfscope}%
\pgfpathrectangle{\pgfqpoint{0.050000in}{0.050000in}}{\pgfqpoint{2.081932in}{2.081932in}}%
\pgfusepath{clip}%
\pgfsetbuttcap%
\pgfsetroundjoin%
\definecolor{currentfill}{rgb}{0.124780,0.640461,0.527068}%
\pgfsetfillcolor{currentfill}%
\pgfsetlinewidth{0.000000pt}%
\definecolor{currentstroke}{rgb}{0.000000,0.000000,0.000000}%
\pgfsetstrokecolor{currentstroke}%
\pgfsetdash{}{0pt}%
\pgfpathmoveto{\pgfqpoint{0.590833in}{1.252741in}}%
\pgfpathlineto{\pgfqpoint{0.590905in}{1.260841in}}%
\pgfpathlineto{\pgfqpoint{0.591675in}{1.268742in}}%
\pgfpathlineto{\pgfqpoint{0.593142in}{1.276410in}}%
\pgfpathlineto{\pgfqpoint{0.595300in}{1.283816in}}%
\pgfpathlineto{\pgfqpoint{0.598142in}{1.290928in}}%
\pgfpathlineto{\pgfqpoint{0.583687in}{1.271605in}}%
\pgfpathlineto{\pgfqpoint{0.571292in}{1.251741in}}%
\pgfpathlineto{\pgfqpoint{0.561023in}{1.231412in}}%
\pgfpathlineto{\pgfqpoint{0.552934in}{1.210693in}}%
\pgfpathlineto{\pgfqpoint{0.547075in}{1.189662in}}%
\pgfpathlineto{\pgfqpoint{0.543915in}{1.181899in}}%
\pgfpathlineto{\pgfqpoint{0.541515in}{1.173974in}}%
\pgfpathlineto{\pgfqpoint{0.539884in}{1.165918in}}%
\pgfpathlineto{\pgfqpoint{0.539028in}{1.157765in}}%
\pgfpathlineto{\pgfqpoint{0.538947in}{1.149546in}}%
\pgfpathlineto{\pgfqpoint{0.544911in}{1.170980in}}%
\pgfpathlineto{\pgfqpoint{0.553134in}{1.192096in}}%
\pgfpathlineto{\pgfqpoint{0.563569in}{1.212812in}}%
\pgfpathlineto{\pgfqpoint{0.576157in}{1.233052in}}%
\pgfpathlineto{\pgfqpoint{0.590833in}{1.252741in}}%
\pgfpathclose%
\pgfusepath{fill}%
\end{pgfscope}%
\begin{pgfscope}%
\pgfpathrectangle{\pgfqpoint{0.050000in}{0.050000in}}{\pgfqpoint{2.081932in}{2.081932in}}%
\pgfusepath{clip}%
\pgfsetbuttcap%
\pgfsetroundjoin%
\definecolor{currentfill}{rgb}{0.282327,0.094955,0.417331}%
\pgfsetfillcolor{currentfill}%
\pgfsetlinewidth{0.000000pt}%
\definecolor{currentstroke}{rgb}{0.000000,0.000000,0.000000}%
\pgfsetstrokecolor{currentstroke}%
\pgfsetdash{}{0pt}%
\pgfpathmoveto{\pgfqpoint{1.573415in}{0.965806in}}%
\pgfpathlineto{\pgfqpoint{1.584837in}{0.968133in}}%
\pgfpathlineto{\pgfqpoint{1.595956in}{0.970956in}}%
\pgfpathlineto{\pgfqpoint{1.606729in}{0.974263in}}%
\pgfpathlineto{\pgfqpoint{1.617113in}{0.978041in}}%
\pgfpathlineto{\pgfqpoint{1.627065in}{0.982278in}}%
\pgfpathlineto{\pgfqpoint{1.628762in}{1.001560in}}%
\pgfpathlineto{\pgfqpoint{1.628400in}{1.020822in}}%
\pgfpathlineto{\pgfqpoint{1.625995in}{1.039987in}}%
\pgfpathlineto{\pgfqpoint{1.621570in}{1.058980in}}%
\pgfpathlineto{\pgfqpoint{1.615156in}{1.077725in}}%
\pgfpathlineto{\pgfqpoint{1.605547in}{1.071688in}}%
\pgfpathlineto{\pgfqpoint{1.595516in}{1.066019in}}%
\pgfpathlineto{\pgfqpoint{1.585105in}{1.060743in}}%
\pgfpathlineto{\pgfqpoint{1.574353in}{1.055878in}}%
\pgfpathlineto{\pgfqpoint{1.563303in}{1.051445in}}%
\pgfpathlineto{\pgfqpoint{1.568939in}{1.034609in}}%
\pgfpathlineto{\pgfqpoint{1.572791in}{1.017560in}}%
\pgfpathlineto{\pgfqpoint{1.574833in}{1.000364in}}%
\pgfpathlineto{\pgfqpoint{1.575045in}{0.983090in}}%
\pgfpathlineto{\pgfqpoint{1.573415in}{0.965806in}}%
\pgfpathclose%
\pgfusepath{fill}%
\end{pgfscope}%
\begin{pgfscope}%
\pgfpathrectangle{\pgfqpoint{0.050000in}{0.050000in}}{\pgfqpoint{2.081932in}{2.081932in}}%
\pgfusepath{clip}%
\pgfsetbuttcap%
\pgfsetroundjoin%
\definecolor{currentfill}{rgb}{0.278791,0.062145,0.386592}%
\pgfsetfillcolor{currentfill}%
\pgfsetlinewidth{0.000000pt}%
\definecolor{currentstroke}{rgb}{0.000000,0.000000,0.000000}%
\pgfsetstrokecolor{currentstroke}%
\pgfsetdash{}{0pt}%
\pgfpathmoveto{\pgfqpoint{1.397683in}{0.991552in}}%
\pgfpathlineto{\pgfqpoint{1.408014in}{0.986506in}}%
\pgfpathlineto{\pgfqpoint{1.418738in}{0.981881in}}%
\pgfpathlineto{\pgfqpoint{1.429813in}{0.977696in}}%
\pgfpathlineto{\pgfqpoint{1.441196in}{0.973965in}}%
\pgfpathlineto{\pgfqpoint{1.452841in}{0.970705in}}%
\pgfpathlineto{\pgfqpoint{1.454223in}{0.983401in}}%
\pgfpathlineto{\pgfqpoint{1.454251in}{0.996104in}}%
\pgfpathlineto{\pgfqpoint{1.452933in}{1.008762in}}%
\pgfpathlineto{\pgfqpoint{1.450279in}{1.021327in}}%
\pgfpathlineto{\pgfqpoint{1.446306in}{1.033747in}}%
\pgfpathlineto{\pgfqpoint{1.434972in}{1.034800in}}%
\pgfpathlineto{\pgfqpoint{1.423888in}{1.036368in}}%
\pgfpathlineto{\pgfqpoint{1.413098in}{1.038448in}}%
\pgfpathlineto{\pgfqpoint{1.402645in}{1.041031in}}%
\pgfpathlineto{\pgfqpoint{1.392571in}{1.044107in}}%
\pgfpathlineto{\pgfqpoint{1.395823in}{1.033742in}}%
\pgfpathlineto{\pgfqpoint{1.397971in}{1.023263in}}%
\pgfpathlineto{\pgfqpoint{1.399002in}{1.012709in}}%
\pgfpathlineto{\pgfqpoint{1.398907in}{1.002125in}}%
\pgfpathlineto{\pgfqpoint{1.397683in}{0.991552in}}%
\pgfpathclose%
\pgfusepath{fill}%
\end{pgfscope}%
\begin{pgfscope}%
\pgfpathrectangle{\pgfqpoint{0.050000in}{0.050000in}}{\pgfqpoint{2.081932in}{2.081932in}}%
\pgfusepath{clip}%
\pgfsetbuttcap%
\pgfsetroundjoin%
\definecolor{currentfill}{rgb}{0.296479,0.761561,0.424223}%
\pgfsetfillcolor{currentfill}%
\pgfsetlinewidth{0.000000pt}%
\definecolor{currentstroke}{rgb}{0.000000,0.000000,0.000000}%
\pgfsetstrokecolor{currentstroke}%
\pgfsetdash{}{0pt}%
\pgfpathmoveto{\pgfqpoint{1.004371in}{1.307427in}}%
\pgfpathlineto{\pgfqpoint{1.007723in}{1.297184in}}%
\pgfpathlineto{\pgfqpoint{1.010744in}{1.287003in}}%
\pgfpathlineto{\pgfqpoint{1.013420in}{1.276928in}}%
\pgfpathlineto{\pgfqpoint{1.015743in}{1.267001in}}%
\pgfpathlineto{\pgfqpoint{1.017701in}{1.257262in}}%
\pgfpathlineto{\pgfqpoint{1.006865in}{1.253339in}}%
\pgfpathlineto{\pgfqpoint{0.996464in}{1.249024in}}%
\pgfpathlineto{\pgfqpoint{0.986541in}{1.244334in}}%
\pgfpathlineto{\pgfqpoint{0.977135in}{1.239285in}}%
\pgfpathlineto{\pgfqpoint{0.968282in}{1.233898in}}%
\pgfpathlineto{\pgfqpoint{0.965362in}{1.243213in}}%
\pgfpathlineto{\pgfqpoint{0.961898in}{1.252635in}}%
\pgfpathlineto{\pgfqpoint{0.957904in}{1.262124in}}%
\pgfpathlineto{\pgfqpoint{0.953397in}{1.271642in}}%
\pgfpathlineto{\pgfqpoint{0.948395in}{1.281148in}}%
\pgfpathlineto{\pgfqpoint{0.958428in}{1.287210in}}%
\pgfpathlineto{\pgfqpoint{0.969086in}{1.292889in}}%
\pgfpathlineto{\pgfqpoint{0.980326in}{1.298164in}}%
\pgfpathlineto{\pgfqpoint{0.992103in}{1.303016in}}%
\pgfpathlineto{\pgfqpoint{1.004371in}{1.307427in}}%
\pgfpathclose%
\pgfusepath{fill}%
\end{pgfscope}%
\begin{pgfscope}%
\pgfpathrectangle{\pgfqpoint{0.050000in}{0.050000in}}{\pgfqpoint{2.081932in}{2.081932in}}%
\pgfusepath{clip}%
\pgfsetbuttcap%
\pgfsetroundjoin%
\definecolor{currentfill}{rgb}{0.606045,0.850733,0.236712}%
\pgfsetfillcolor{currentfill}%
\pgfsetlinewidth{0.000000pt}%
\definecolor{currentstroke}{rgb}{0.000000,0.000000,0.000000}%
\pgfsetstrokecolor{currentstroke}%
\pgfsetdash{}{0pt}%
\pgfpathmoveto{\pgfqpoint{0.983188in}{1.358078in}}%
\pgfpathlineto{\pgfqpoint{0.987957in}{1.348166in}}%
\pgfpathlineto{\pgfqpoint{0.992478in}{1.338102in}}%
\pgfpathlineto{\pgfqpoint{0.996733in}{1.327930in}}%
\pgfpathlineto{\pgfqpoint{1.000703in}{1.317690in}}%
\pgfpathlineto{\pgfqpoint{1.004371in}{1.307427in}}%
\pgfpathlineto{\pgfqpoint{0.992103in}{1.303016in}}%
\pgfpathlineto{\pgfqpoint{0.980326in}{1.298164in}}%
\pgfpathlineto{\pgfqpoint{0.969086in}{1.292889in}}%
\pgfpathlineto{\pgfqpoint{0.958428in}{1.287210in}}%
\pgfpathlineto{\pgfqpoint{0.948395in}{1.281148in}}%
\pgfpathlineto{\pgfqpoint{0.942919in}{1.290603in}}%
\pgfpathlineto{\pgfqpoint{0.936992in}{1.299968in}}%
\pgfpathlineto{\pgfqpoint{0.930638in}{1.309202in}}%
\pgfpathlineto{\pgfqpoint{0.923884in}{1.318268in}}%
\pgfpathlineto{\pgfqpoint{0.916758in}{1.327128in}}%
\pgfpathlineto{\pgfqpoint{0.928677in}{1.334270in}}%
\pgfpathlineto{\pgfqpoint{0.941331in}{1.340960in}}%
\pgfpathlineto{\pgfqpoint{0.954670in}{1.347173in}}%
\pgfpathlineto{\pgfqpoint{0.968640in}{1.352886in}}%
\pgfpathlineto{\pgfqpoint{0.983188in}{1.358078in}}%
\pgfpathclose%
\pgfusepath{fill}%
\end{pgfscope}%
\begin{pgfscope}%
\pgfpathrectangle{\pgfqpoint{0.050000in}{0.050000in}}{\pgfqpoint{2.081932in}{2.081932in}}%
\pgfusepath{clip}%
\pgfsetbuttcap%
\pgfsetroundjoin%
\definecolor{currentfill}{rgb}{0.227802,0.326594,0.546532}%
\pgfsetfillcolor{currentfill}%
\pgfsetlinewidth{0.000000pt}%
\definecolor{currentstroke}{rgb}{0.000000,0.000000,0.000000}%
\pgfsetstrokecolor{currentstroke}%
\pgfsetdash{}{0pt}%
\pgfpathmoveto{\pgfqpoint{0.930977in}{1.122653in}}%
\pgfpathlineto{\pgfqpoint{0.926911in}{1.116118in}}%
\pgfpathlineto{\pgfqpoint{0.922196in}{1.109964in}}%
\pgfpathlineto{\pgfqpoint{0.916855in}{1.104215in}}%
\pgfpathlineto{\pgfqpoint{0.910909in}{1.098894in}}%
\pgfpathlineto{\pgfqpoint{0.904383in}{1.094022in}}%
\pgfpathlineto{\pgfqpoint{0.898737in}{1.085872in}}%
\pgfpathlineto{\pgfqpoint{0.893963in}{1.077518in}}%
\pgfpathlineto{\pgfqpoint{0.890083in}{1.068993in}}%
\pgfpathlineto{\pgfqpoint{0.887115in}{1.060331in}}%
\pgfpathlineto{\pgfqpoint{0.885073in}{1.051566in}}%
\pgfpathlineto{\pgfqpoint{0.892221in}{1.057759in}}%
\pgfpathlineto{\pgfqpoint{0.898732in}{1.064286in}}%
\pgfpathlineto{\pgfqpoint{0.904580in}{1.071121in}}%
\pgfpathlineto{\pgfqpoint{0.909739in}{1.078238in}}%
\pgfpathlineto{\pgfqpoint{0.914189in}{1.085608in}}%
\pgfpathlineto{\pgfqpoint{0.915950in}{1.093252in}}%
\pgfpathlineto{\pgfqpoint{0.918523in}{1.100809in}}%
\pgfpathlineto{\pgfqpoint{0.921896in}{1.108247in}}%
\pgfpathlineto{\pgfqpoint{0.926054in}{1.115538in}}%
\pgfpathlineto{\pgfqpoint{0.930977in}{1.122653in}}%
\pgfpathclose%
\pgfusepath{fill}%
\end{pgfscope}%
\begin{pgfscope}%
\pgfpathrectangle{\pgfqpoint{0.050000in}{0.050000in}}{\pgfqpoint{2.081932in}{2.081932in}}%
\pgfusepath{clip}%
\pgfsetbuttcap%
\pgfsetroundjoin%
\definecolor{currentfill}{rgb}{0.120638,0.625828,0.533488}%
\pgfsetfillcolor{currentfill}%
\pgfsetlinewidth{0.000000pt}%
\definecolor{currentstroke}{rgb}{0.000000,0.000000,0.000000}%
\pgfsetstrokecolor{currentstroke}%
\pgfsetdash{}{0pt}%
\pgfpathmoveto{\pgfqpoint{1.294905in}{1.213988in}}%
\pgfpathlineto{\pgfqpoint{1.292143in}{1.205109in}}%
\pgfpathlineto{\pgfqpoint{1.290041in}{1.196340in}}%
\pgfpathlineto{\pgfqpoint{1.288605in}{1.187717in}}%
\pgfpathlineto{\pgfqpoint{1.287842in}{1.179275in}}%
\pgfpathlineto{\pgfqpoint{1.287752in}{1.171049in}}%
\pgfpathlineto{\pgfqpoint{1.281253in}{1.177177in}}%
\pgfpathlineto{\pgfqpoint{1.274116in}{1.183048in}}%
\pgfpathlineto{\pgfqpoint{1.266370in}{1.188640in}}%
\pgfpathlineto{\pgfqpoint{1.258048in}{1.193931in}}%
\pgfpathlineto{\pgfqpoint{1.249183in}{1.198901in}}%
\pgfpathlineto{\pgfqpoint{1.249252in}{1.207114in}}%
\pgfpathlineto{\pgfqpoint{1.249839in}{1.215652in}}%
\pgfpathlineto{\pgfqpoint{1.250942in}{1.224481in}}%
\pgfpathlineto{\pgfqpoint{1.252558in}{1.233564in}}%
\pgfpathlineto{\pgfqpoint{1.254680in}{1.242864in}}%
\pgfpathlineto{\pgfqpoint{1.263923in}{1.237712in}}%
\pgfpathlineto{\pgfqpoint{1.272601in}{1.232227in}}%
\pgfpathlineto{\pgfqpoint{1.280679in}{1.226430in}}%
\pgfpathlineto{\pgfqpoint{1.288124in}{1.220342in}}%
\pgfpathlineto{\pgfqpoint{1.294905in}{1.213988in}}%
\pgfpathclose%
\pgfusepath{fill}%
\end{pgfscope}%
\begin{pgfscope}%
\pgfpathrectangle{\pgfqpoint{0.050000in}{0.050000in}}{\pgfqpoint{2.081932in}{2.081932in}}%
\pgfusepath{clip}%
\pgfsetbuttcap%
\pgfsetroundjoin%
\definecolor{currentfill}{rgb}{0.993248,0.906157,0.143936}%
\pgfsetfillcolor{currentfill}%
\pgfsetlinewidth{0.000000pt}%
\definecolor{currentstroke}{rgb}{0.000000,0.000000,0.000000}%
\pgfsetstrokecolor{currentstroke}%
\pgfsetdash{}{0pt}%
\pgfpathmoveto{\pgfqpoint{0.788064in}{1.413662in}}%
\pgfpathlineto{\pgfqpoint{0.796655in}{1.411459in}}%
\pgfpathlineto{\pgfqpoint{0.805402in}{1.408673in}}%
\pgfpathlineto{\pgfqpoint{0.814270in}{1.405312in}}%
\pgfpathlineto{\pgfqpoint{0.823223in}{1.401390in}}%
\pgfpathlineto{\pgfqpoint{0.832224in}{1.396922in}}%
\pgfpathlineto{\pgfqpoint{0.816317in}{1.386283in}}%
\pgfpathlineto{\pgfqpoint{0.801567in}{1.375075in}}%
\pgfpathlineto{\pgfqpoint{0.788036in}{1.363340in}}%
\pgfpathlineto{\pgfqpoint{0.775779in}{1.351119in}}%
\pgfpathlineto{\pgfqpoint{0.764852in}{1.338459in}}%
\pgfpathlineto{\pgfqpoint{0.753653in}{1.341113in}}%
\pgfpathlineto{\pgfqpoint{0.742509in}{1.343228in}}%
\pgfpathlineto{\pgfqpoint{0.731465in}{1.344796in}}%
\pgfpathlineto{\pgfqpoint{0.720566in}{1.345810in}}%
\pgfpathlineto{\pgfqpoint{0.709858in}{1.346268in}}%
\pgfpathlineto{\pgfqpoint{0.722579in}{1.360875in}}%
\pgfpathlineto{\pgfqpoint{0.736825in}{1.374968in}}%
\pgfpathlineto{\pgfqpoint{0.752534in}{1.388496in}}%
\pgfpathlineto{\pgfqpoint{0.769637in}{1.401409in}}%
\pgfpathlineto{\pgfqpoint{0.788064in}{1.413662in}}%
\pgfpathclose%
\pgfusepath{fill}%
\end{pgfscope}%
\begin{pgfscope}%
\pgfpathrectangle{\pgfqpoint{0.050000in}{0.050000in}}{\pgfqpoint{2.081932in}{2.081932in}}%
\pgfusepath{clip}%
\pgfsetbuttcap%
\pgfsetroundjoin%
\definecolor{currentfill}{rgb}{0.993248,0.906157,0.143936}%
\pgfsetfillcolor{currentfill}%
\pgfsetlinewidth{0.000000pt}%
\definecolor{currentstroke}{rgb}{0.000000,0.000000,0.000000}%
\pgfsetstrokecolor{currentstroke}%
\pgfsetdash{}{0pt}%
\pgfpathmoveto{\pgfqpoint{1.454124in}{1.359742in}}%
\pgfpathlineto{\pgfqpoint{1.443557in}{1.355895in}}%
\pgfpathlineto{\pgfqpoint{1.433023in}{1.351536in}}%
\pgfpathlineto{\pgfqpoint{1.422567in}{1.346681in}}%
\pgfpathlineto{\pgfqpoint{1.412232in}{1.341349in}}%
\pgfpathlineto{\pgfqpoint{1.402061in}{1.335563in}}%
\pgfpathlineto{\pgfqpoint{1.391016in}{1.345634in}}%
\pgfpathlineto{\pgfqpoint{1.378919in}{1.355272in}}%
\pgfpathlineto{\pgfqpoint{1.365820in}{1.364443in}}%
\pgfpathlineto{\pgfqpoint{1.351773in}{1.373112in}}%
\pgfpathlineto{\pgfqpoint{1.336835in}{1.381248in}}%
\pgfpathlineto{\pgfqpoint{1.344615in}{1.388632in}}%
\pgfpathlineto{\pgfqpoint{1.352516in}{1.395588in}}%
\pgfpathlineto{\pgfqpoint{1.360507in}{1.402087in}}%
\pgfpathlineto{\pgfqpoint{1.368553in}{1.408103in}}%
\pgfpathlineto{\pgfqpoint{1.376622in}{1.413612in}}%
\pgfpathlineto{\pgfqpoint{1.394342in}{1.404027in}}%
\pgfpathlineto{\pgfqpoint{1.411018in}{1.393809in}}%
\pgfpathlineto{\pgfqpoint{1.426582in}{1.382996in}}%
\pgfpathlineto{\pgfqpoint{1.440971in}{1.371626in}}%
\pgfpathlineto{\pgfqpoint{1.454124in}{1.359742in}}%
\pgfpathclose%
\pgfusepath{fill}%
\end{pgfscope}%
\begin{pgfscope}%
\pgfpathrectangle{\pgfqpoint{0.050000in}{0.050000in}}{\pgfqpoint{2.081932in}{2.081932in}}%
\pgfusepath{clip}%
\pgfsetbuttcap%
\pgfsetroundjoin%
\definecolor{currentfill}{rgb}{0.636902,0.856542,0.216620}%
\pgfsetfillcolor{currentfill}%
\pgfsetlinewidth{0.000000pt}%
\definecolor{currentstroke}{rgb}{0.000000,0.000000,0.000000}%
\pgfsetstrokecolor{currentstroke}%
\pgfsetdash{}{0pt}%
\pgfpathmoveto{\pgfqpoint{1.653911in}{1.258755in}}%
\pgfpathlineto{\pgfqpoint{1.646711in}{1.264425in}}%
\pgfpathlineto{\pgfqpoint{1.638908in}{1.269694in}}%
\pgfpathlineto{\pgfqpoint{1.630532in}{1.274539in}}%
\pgfpathlineto{\pgfqpoint{1.621617in}{1.278941in}}%
\pgfpathlineto{\pgfqpoint{1.612200in}{1.282881in}}%
\pgfpathlineto{\pgfqpoint{1.603893in}{1.300740in}}%
\pgfpathlineto{\pgfqpoint{1.593693in}{1.318223in}}%
\pgfpathlineto{\pgfqpoint{1.581652in}{1.335263in}}%
\pgfpathlineto{\pgfqpoint{1.567829in}{1.351797in}}%
\pgfpathlineto{\pgfqpoint{1.552288in}{1.367766in}}%
\pgfpathlineto{\pgfqpoint{1.560472in}{1.365472in}}%
\pgfpathlineto{\pgfqpoint{1.568216in}{1.362637in}}%
\pgfpathlineto{\pgfqpoint{1.575488in}{1.359272in}}%
\pgfpathlineto{\pgfqpoint{1.582262in}{1.355393in}}%
\pgfpathlineto{\pgfqpoint{1.588510in}{1.351016in}}%
\pgfpathlineto{\pgfqpoint{1.605433in}{1.333672in}}%
\pgfpathlineto{\pgfqpoint{1.620503in}{1.315708in}}%
\pgfpathlineto{\pgfqpoint{1.633647in}{1.297188in}}%
\pgfpathlineto{\pgfqpoint{1.644803in}{1.278180in}}%
\pgfpathlineto{\pgfqpoint{1.653911in}{1.258755in}}%
\pgfpathclose%
\pgfusepath{fill}%
\end{pgfscope}%
\begin{pgfscope}%
\pgfpathrectangle{\pgfqpoint{0.050000in}{0.050000in}}{\pgfqpoint{2.081932in}{2.081932in}}%
\pgfusepath{clip}%
\pgfsetbuttcap%
\pgfsetroundjoin%
\definecolor{currentfill}{rgb}{0.267968,0.223549,0.512008}%
\pgfsetfillcolor{currentfill}%
\pgfsetlinewidth{0.000000pt}%
\definecolor{currentstroke}{rgb}{0.000000,0.000000,0.000000}%
\pgfsetstrokecolor{currentstroke}%
\pgfsetdash{}{0pt}%
\pgfpathmoveto{\pgfqpoint{1.627065in}{0.982278in}}%
\pgfpathlineto{\pgfqpoint{1.636546in}{0.986955in}}%
\pgfpathlineto{\pgfqpoint{1.645518in}{0.992056in}}%
\pgfpathlineto{\pgfqpoint{1.653945in}{0.997560in}}%
\pgfpathlineto{\pgfqpoint{1.661791in}{1.003447in}}%
\pgfpathlineto{\pgfqpoint{1.669026in}{1.009693in}}%
\pgfpathlineto{\pgfqpoint{1.670758in}{1.030495in}}%
\pgfpathlineto{\pgfqpoint{1.670261in}{1.051268in}}%
\pgfpathlineto{\pgfqpoint{1.667554in}{1.071929in}}%
\pgfpathlineto{\pgfqpoint{1.662663in}{1.092395in}}%
\pgfpathlineto{\pgfqpoint{1.655625in}{1.112588in}}%
\pgfpathlineto{\pgfqpoint{1.648653in}{1.105081in}}%
\pgfpathlineto{\pgfqpoint{1.641089in}{1.097814in}}%
\pgfpathlineto{\pgfqpoint{1.632962in}{1.090814in}}%
\pgfpathlineto{\pgfqpoint{1.624307in}{1.084109in}}%
\pgfpathlineto{\pgfqpoint{1.615156in}{1.077725in}}%
\pgfpathlineto{\pgfqpoint{1.621570in}{1.058980in}}%
\pgfpathlineto{\pgfqpoint{1.625995in}{1.039987in}}%
\pgfpathlineto{\pgfqpoint{1.628400in}{1.020822in}}%
\pgfpathlineto{\pgfqpoint{1.628762in}{1.001560in}}%
\pgfpathlineto{\pgfqpoint{1.627065in}{0.982278in}}%
\pgfpathclose%
\pgfusepath{fill}%
\end{pgfscope}%
\begin{pgfscope}%
\pgfpathrectangle{\pgfqpoint{0.050000in}{0.050000in}}{\pgfqpoint{2.081932in}{2.081932in}}%
\pgfusepath{clip}%
\pgfsetbuttcap%
\pgfsetroundjoin%
\definecolor{currentfill}{rgb}{0.278012,0.180367,0.486697}%
\pgfsetfillcolor{currentfill}%
\pgfsetlinewidth{0.000000pt}%
\definecolor{currentstroke}{rgb}{0.000000,0.000000,0.000000}%
\pgfsetstrokecolor{currentstroke}%
\pgfsetdash{}{0pt}%
\pgfpathmoveto{\pgfqpoint{1.353302in}{1.022360in}}%
\pgfpathlineto{\pgfqpoint{1.361088in}{1.015532in}}%
\pgfpathlineto{\pgfqpoint{1.369453in}{1.009013in}}%
\pgfpathlineto{\pgfqpoint{1.378364in}{1.002827in}}%
\pgfpathlineto{\pgfqpoint{1.387787in}{0.996999in}}%
\pgfpathlineto{\pgfqpoint{1.397683in}{0.991552in}}%
\pgfpathlineto{\pgfqpoint{1.398907in}{1.002125in}}%
\pgfpathlineto{\pgfqpoint{1.399002in}{1.012709in}}%
\pgfpathlineto{\pgfqpoint{1.397971in}{1.023263in}}%
\pgfpathlineto{\pgfqpoint{1.395823in}{1.033742in}}%
\pgfpathlineto{\pgfqpoint{1.392571in}{1.044107in}}%
\pgfpathlineto{\pgfqpoint{1.382915in}{1.047665in}}%
\pgfpathlineto{\pgfqpoint{1.373718in}{1.051692in}}%
\pgfpathlineto{\pgfqpoint{1.365016in}{1.056172in}}%
\pgfpathlineto{\pgfqpoint{1.356844in}{1.061088in}}%
\pgfpathlineto{\pgfqpoint{1.349236in}{1.066421in}}%
\pgfpathlineto{\pgfqpoint{1.351927in}{1.057724in}}%
\pgfpathlineto{\pgfqpoint{1.353687in}{1.048935in}}%
\pgfpathlineto{\pgfqpoint{1.354506in}{1.040087in}}%
\pgfpathlineto{\pgfqpoint{1.354379in}{1.031217in}}%
\pgfpathlineto{\pgfqpoint{1.353302in}{1.022360in}}%
\pgfpathclose%
\pgfusepath{fill}%
\end{pgfscope}%
\begin{pgfscope}%
\pgfpathrectangle{\pgfqpoint{0.050000in}{0.050000in}}{\pgfqpoint{2.081932in}{2.081932in}}%
\pgfusepath{clip}%
\pgfsetbuttcap%
\pgfsetroundjoin%
\definecolor{currentfill}{rgb}{0.162142,0.474838,0.558140}%
\pgfsetfillcolor{currentfill}%
\pgfsetlinewidth{0.000000pt}%
\definecolor{currentstroke}{rgb}{0.000000,0.000000,0.000000}%
\pgfsetstrokecolor{currentstroke}%
\pgfsetdash{}{0pt}%
\pgfpathmoveto{\pgfqpoint{1.309826in}{1.137408in}}%
\pgfpathlineto{\pgfqpoint{1.310489in}{1.129281in}}%
\pgfpathlineto{\pgfqpoint{1.311913in}{1.121301in}}%
\pgfpathlineto{\pgfqpoint{1.314091in}{1.113501in}}%
\pgfpathlineto{\pgfqpoint{1.317013in}{1.105912in}}%
\pgfpathlineto{\pgfqpoint{1.320666in}{1.098565in}}%
\pgfpathlineto{\pgfqpoint{1.317532in}{1.106041in}}%
\pgfpathlineto{\pgfqpoint{1.313609in}{1.113379in}}%
\pgfpathlineto{\pgfqpoint{1.308915in}{1.120550in}}%
\pgfpathlineto{\pgfqpoint{1.303469in}{1.127525in}}%
\pgfpathlineto{\pgfqpoint{1.297296in}{1.134278in}}%
\pgfpathlineto{\pgfqpoint{1.294081in}{1.140950in}}%
\pgfpathlineto{\pgfqpoint{1.291508in}{1.147992in}}%
\pgfpathlineto{\pgfqpoint{1.289591in}{1.155375in}}%
\pgfpathlineto{\pgfqpoint{1.288337in}{1.163071in}}%
\pgfpathlineto{\pgfqpoint{1.287752in}{1.171049in}}%
\pgfpathlineto{\pgfqpoint{1.293587in}{1.164686in}}%
\pgfpathlineto{\pgfqpoint{1.298733in}{1.158115in}}%
\pgfpathlineto{\pgfqpoint{1.303167in}{1.151361in}}%
\pgfpathlineto{\pgfqpoint{1.306870in}{1.144450in}}%
\pgfpathlineto{\pgfqpoint{1.309826in}{1.137408in}}%
\pgfpathclose%
\pgfusepath{fill}%
\end{pgfscope}%
\begin{pgfscope}%
\pgfpathrectangle{\pgfqpoint{0.050000in}{0.050000in}}{\pgfqpoint{2.081932in}{2.081932in}}%
\pgfusepath{clip}%
\pgfsetbuttcap%
\pgfsetroundjoin%
\definecolor{currentfill}{rgb}{0.120638,0.625828,0.533488}%
\pgfsetfillcolor{currentfill}%
\pgfsetlinewidth{0.000000pt}%
\definecolor{currentstroke}{rgb}{0.000000,0.000000,0.000000}%
\pgfsetstrokecolor{currentstroke}%
\pgfsetdash{}{0pt}%
\pgfpathmoveto{\pgfqpoint{0.968282in}{1.233898in}}%
\pgfpathlineto{\pgfqpoint{0.970647in}{1.224728in}}%
\pgfpathlineto{\pgfqpoint{0.972448in}{1.215742in}}%
\pgfpathlineto{\pgfqpoint{0.973677in}{1.206977in}}%
\pgfpathlineto{\pgfqpoint{0.974331in}{1.198469in}}%
\pgfpathlineto{\pgfqpoint{0.974408in}{1.190252in}}%
\pgfpathlineto{\pgfqpoint{0.966485in}{1.184748in}}%
\pgfpathlineto{\pgfqpoint{0.959161in}{1.178957in}}%
\pgfpathlineto{\pgfqpoint{0.952467in}{1.172903in}}%
\pgfpathlineto{\pgfqpoint{0.946429in}{1.166608in}}%
\pgfpathlineto{\pgfqpoint{0.941073in}{1.160097in}}%
\pgfpathlineto{\pgfqpoint{0.940979in}{1.168329in}}%
\pgfpathlineto{\pgfqpoint{0.940171in}{1.176732in}}%
\pgfpathlineto{\pgfqpoint{0.938653in}{1.185274in}}%
\pgfpathlineto{\pgfqpoint{0.936429in}{1.193919in}}%
\pgfpathlineto{\pgfqpoint{0.933508in}{1.202631in}}%
\pgfpathlineto{\pgfqpoint{0.939097in}{1.209383in}}%
\pgfpathlineto{\pgfqpoint{0.945397in}{1.215911in}}%
\pgfpathlineto{\pgfqpoint{0.952381in}{1.222188in}}%
\pgfpathlineto{\pgfqpoint{0.960020in}{1.228191in}}%
\pgfpathlineto{\pgfqpoint{0.968282in}{1.233898in}}%
\pgfpathclose%
\pgfusepath{fill}%
\end{pgfscope}%
\begin{pgfscope}%
\pgfpathrectangle{\pgfqpoint{0.050000in}{0.050000in}}{\pgfqpoint{2.081932in}{2.081932in}}%
\pgfusepath{clip}%
\pgfsetbuttcap%
\pgfsetroundjoin%
\definecolor{currentfill}{rgb}{0.267004,0.004874,0.329415}%
\pgfsetfillcolor{currentfill}%
\pgfsetlinewidth{0.000000pt}%
\definecolor{currentstroke}{rgb}{0.000000,0.000000,0.000000}%
\pgfsetstrokecolor{currentstroke}%
\pgfsetdash{}{0pt}%
\pgfpathmoveto{\pgfqpoint{0.786054in}{1.012524in}}%
\pgfpathlineto{\pgfqpoint{0.774273in}{1.011240in}}%
\pgfpathlineto{\pgfqpoint{0.762327in}{1.010468in}}%
\pgfpathlineto{\pgfqpoint{0.750263in}{1.010212in}}%
\pgfpathlineto{\pgfqpoint{0.738129in}{1.010472in}}%
\pgfpathlineto{\pgfqpoint{0.725972in}{1.011247in}}%
\pgfpathlineto{\pgfqpoint{0.723830in}{0.996325in}}%
\pgfpathlineto{\pgfqpoint{0.723274in}{0.981326in}}%
\pgfpathlineto{\pgfqpoint{0.724314in}{0.966308in}}%
\pgfpathlineto{\pgfqpoint{0.726956in}{0.951334in}}%
\pgfpathlineto{\pgfqpoint{0.731197in}{0.936464in}}%
\pgfpathlineto{\pgfqpoint{0.743298in}{0.938015in}}%
\pgfpathlineto{\pgfqpoint{0.755369in}{0.940080in}}%
\pgfpathlineto{\pgfqpoint{0.767363in}{0.942651in}}%
\pgfpathlineto{\pgfqpoint{0.779233in}{0.945717in}}%
\pgfpathlineto{\pgfqpoint{0.790933in}{0.949266in}}%
\pgfpathlineto{\pgfqpoint{0.787254in}{0.961831in}}%
\pgfpathlineto{\pgfqpoint{0.784927in}{0.974490in}}%
\pgfpathlineto{\pgfqpoint{0.783953in}{0.987193in}}%
\pgfpathlineto{\pgfqpoint{0.784331in}{0.999888in}}%
\pgfpathlineto{\pgfqpoint{0.786054in}{1.012524in}}%
\pgfpathclose%
\pgfusepath{fill}%
\end{pgfscope}%
\begin{pgfscope}%
\pgfpathrectangle{\pgfqpoint{0.050000in}{0.050000in}}{\pgfqpoint{2.081932in}{2.081932in}}%
\pgfusepath{clip}%
\pgfsetbuttcap%
\pgfsetroundjoin%
\definecolor{currentfill}{rgb}{0.278791,0.062145,0.386592}%
\pgfsetfillcolor{currentfill}%
\pgfsetlinewidth{0.000000pt}%
\definecolor{currentstroke}{rgb}{0.000000,0.000000,0.000000}%
\pgfsetstrokecolor{currentstroke}%
\pgfsetdash{}{0pt}%
\pgfpathmoveto{\pgfqpoint{0.840887in}{1.026400in}}%
\pgfpathlineto{\pgfqpoint{0.830610in}{1.022663in}}%
\pgfpathlineto{\pgfqpoint{0.819946in}{1.019395in}}%
\pgfpathlineto{\pgfqpoint{0.808935in}{1.016610in}}%
\pgfpathlineto{\pgfqpoint{0.797623in}{1.014316in}}%
\pgfpathlineto{\pgfqpoint{0.786054in}{1.012524in}}%
\pgfpathlineto{\pgfqpoint{0.784331in}{0.999888in}}%
\pgfpathlineto{\pgfqpoint{0.783953in}{0.987193in}}%
\pgfpathlineto{\pgfqpoint{0.784927in}{0.974490in}}%
\pgfpathlineto{\pgfqpoint{0.787254in}{0.961831in}}%
\pgfpathlineto{\pgfqpoint{0.790933in}{0.949266in}}%
\pgfpathlineto{\pgfqpoint{0.802417in}{0.953285in}}%
\pgfpathlineto{\pgfqpoint{0.813640in}{0.957756in}}%
\pgfpathlineto{\pgfqpoint{0.824557in}{0.962664in}}%
\pgfpathlineto{\pgfqpoint{0.835127in}{0.967989in}}%
\pgfpathlineto{\pgfqpoint{0.845307in}{0.973709in}}%
\pgfpathlineto{\pgfqpoint{0.842169in}{0.984164in}}%
\pgfpathlineto{\pgfqpoint{0.840156in}{0.994704in}}%
\pgfpathlineto{\pgfqpoint{0.839273in}{1.005285in}}%
\pgfpathlineto{\pgfqpoint{0.839518in}{1.015864in}}%
\pgfpathlineto{\pgfqpoint{0.840887in}{1.026400in}}%
\pgfpathclose%
\pgfusepath{fill}%
\end{pgfscope}%
\begin{pgfscope}%
\pgfpathrectangle{\pgfqpoint{0.050000in}{0.050000in}}{\pgfqpoint{2.081932in}{2.081932in}}%
\pgfusepath{clip}%
\pgfsetbuttcap%
\pgfsetroundjoin%
\definecolor{currentfill}{rgb}{0.268510,0.009605,0.335427}%
\pgfsetfillcolor{currentfill}%
\pgfsetlinewidth{0.000000pt}%
\definecolor{currentstroke}{rgb}{0.000000,0.000000,0.000000}%
\pgfsetstrokecolor{currentstroke}%
\pgfsetdash{}{0pt}%
\pgfpathmoveto{\pgfqpoint{0.725972in}{1.011247in}}%
\pgfpathlineto{\pgfqpoint{0.713840in}{1.012533in}}%
\pgfpathlineto{\pgfqpoint{0.701780in}{1.014327in}}%
\pgfpathlineto{\pgfqpoint{0.689840in}{1.016620in}}%
\pgfpathlineto{\pgfqpoint{0.678066in}{1.019403in}}%
\pgfpathlineto{\pgfqpoint{0.666506in}{1.022667in}}%
\pgfpathlineto{\pgfqpoint{0.663916in}{1.005508in}}%
\pgfpathlineto{\pgfqpoint{0.663151in}{0.988249in}}%
\pgfpathlineto{\pgfqpoint{0.664224in}{0.970961in}}%
\pgfpathlineto{\pgfqpoint{0.667145in}{0.953713in}}%
\pgfpathlineto{\pgfqpoint{0.671910in}{0.936575in}}%
\pgfpathlineto{\pgfqpoint{0.683449in}{0.935500in}}%
\pgfpathlineto{\pgfqpoint{0.695193in}{0.934950in}}%
\pgfpathlineto{\pgfqpoint{0.707097in}{0.934928in}}%
\pgfpathlineto{\pgfqpoint{0.719114in}{0.935434in}}%
\pgfpathlineto{\pgfqpoint{0.731197in}{0.936464in}}%
\pgfpathlineto{\pgfqpoint{0.726956in}{0.951334in}}%
\pgfpathlineto{\pgfqpoint{0.724314in}{0.966308in}}%
\pgfpathlineto{\pgfqpoint{0.723274in}{0.981326in}}%
\pgfpathlineto{\pgfqpoint{0.723830in}{0.996325in}}%
\pgfpathlineto{\pgfqpoint{0.725972in}{1.011247in}}%
\pgfpathclose%
\pgfusepath{fill}%
\end{pgfscope}%
\begin{pgfscope}%
\pgfpathrectangle{\pgfqpoint{0.050000in}{0.050000in}}{\pgfqpoint{2.081932in}{2.081932in}}%
\pgfusepath{clip}%
\pgfsetbuttcap%
\pgfsetroundjoin%
\definecolor{currentfill}{rgb}{0.296479,0.761561,0.424223}%
\pgfsetfillcolor{currentfill}%
\pgfsetlinewidth{0.000000pt}%
\definecolor{currentstroke}{rgb}{0.000000,0.000000,0.000000}%
\pgfsetstrokecolor{currentstroke}%
\pgfsetdash{}{0pt}%
\pgfpathmoveto{\pgfqpoint{1.318133in}{1.258739in}}%
\pgfpathlineto{\pgfqpoint{1.312289in}{1.249865in}}%
\pgfpathlineto{\pgfqpoint{1.307025in}{1.240916in}}%
\pgfpathlineto{\pgfqpoint{1.302360in}{1.231928in}}%
\pgfpathlineto{\pgfqpoint{1.298315in}{1.222940in}}%
\pgfpathlineto{\pgfqpoint{1.294905in}{1.213988in}}%
\pgfpathlineto{\pgfqpoint{1.288124in}{1.220342in}}%
\pgfpathlineto{\pgfqpoint{1.280679in}{1.226430in}}%
\pgfpathlineto{\pgfqpoint{1.272601in}{1.232227in}}%
\pgfpathlineto{\pgfqpoint{1.263923in}{1.237712in}}%
\pgfpathlineto{\pgfqpoint{1.254680in}{1.242864in}}%
\pgfpathlineto{\pgfqpoint{1.257300in}{1.252342in}}%
\pgfpathlineto{\pgfqpoint{1.260407in}{1.261958in}}%
\pgfpathlineto{\pgfqpoint{1.263990in}{1.271673in}}%
\pgfpathlineto{\pgfqpoint{1.268033in}{1.281446in}}%
\pgfpathlineto{\pgfqpoint{1.272519in}{1.291235in}}%
\pgfpathlineto{\pgfqpoint{1.282993in}{1.285440in}}%
\pgfpathlineto{\pgfqpoint{1.292830in}{1.279269in}}%
\pgfpathlineto{\pgfqpoint{1.301990in}{1.272745in}}%
\pgfpathlineto{\pgfqpoint{1.310436in}{1.265893in}}%
\pgfpathlineto{\pgfqpoint{1.318133in}{1.258739in}}%
\pgfpathclose%
\pgfusepath{fill}%
\end{pgfscope}%
\begin{pgfscope}%
\pgfpathrectangle{\pgfqpoint{0.050000in}{0.050000in}}{\pgfqpoint{2.081932in}{2.081932in}}%
\pgfusepath{clip}%
\pgfsetbuttcap%
\pgfsetroundjoin%
\definecolor{currentfill}{rgb}{0.327796,0.773980,0.406640}%
\pgfsetfillcolor{currentfill}%
\pgfsetlinewidth{0.000000pt}%
\definecolor{currentstroke}{rgb}{0.000000,0.000000,0.000000}%
\pgfsetstrokecolor{currentstroke}%
\pgfsetdash{}{0pt}%
\pgfpathmoveto{\pgfqpoint{0.598142in}{1.290928in}}%
\pgfpathlineto{\pgfqpoint{0.601658in}{1.297719in}}%
\pgfpathlineto{\pgfqpoint{0.605835in}{1.304161in}}%
\pgfpathlineto{\pgfqpoint{0.610659in}{1.310226in}}%
\pgfpathlineto{\pgfqpoint{0.616109in}{1.315890in}}%
\pgfpathlineto{\pgfqpoint{0.622167in}{1.321130in}}%
\pgfpathlineto{\pgfqpoint{0.608435in}{1.302769in}}%
\pgfpathlineto{\pgfqpoint{0.596673in}{1.283899in}}%
\pgfpathlineto{\pgfqpoint{0.586942in}{1.264592in}}%
\pgfpathlineto{\pgfqpoint{0.579293in}{1.244918in}}%
\pgfpathlineto{\pgfqpoint{0.573773in}{1.224954in}}%
\pgfpathlineto{\pgfqpoint{0.567044in}{1.218464in}}%
\pgfpathlineto{\pgfqpoint{0.560988in}{1.211661in}}%
\pgfpathlineto{\pgfqpoint{0.555628in}{1.204573in}}%
\pgfpathlineto{\pgfqpoint{0.550985in}{1.197230in}}%
\pgfpathlineto{\pgfqpoint{0.547075in}{1.189662in}}%
\pgfpathlineto{\pgfqpoint{0.552934in}{1.210693in}}%
\pgfpathlineto{\pgfqpoint{0.561023in}{1.231412in}}%
\pgfpathlineto{\pgfqpoint{0.571292in}{1.251741in}}%
\pgfpathlineto{\pgfqpoint{0.583687in}{1.271605in}}%
\pgfpathlineto{\pgfqpoint{0.598142in}{1.290928in}}%
\pgfpathclose%
\pgfusepath{fill}%
\end{pgfscope}%
\begin{pgfscope}%
\pgfpathrectangle{\pgfqpoint{0.050000in}{0.050000in}}{\pgfqpoint{2.081932in}{2.081932in}}%
\pgfusepath{clip}%
\pgfsetbuttcap%
\pgfsetroundjoin%
\definecolor{currentfill}{rgb}{0.855810,0.888601,0.097452}%
\pgfsetfillcolor{currentfill}%
\pgfsetlinewidth{0.000000pt}%
\definecolor{currentstroke}{rgb}{0.000000,0.000000,0.000000}%
\pgfsetstrokecolor{currentstroke}%
\pgfsetdash{}{0pt}%
\pgfpathmoveto{\pgfqpoint{1.402061in}{1.335563in}}%
\pgfpathlineto{\pgfqpoint{1.392097in}{1.329346in}}%
\pgfpathlineto{\pgfqpoint{1.382380in}{1.322724in}}%
\pgfpathlineto{\pgfqpoint{1.372952in}{1.315724in}}%
\pgfpathlineto{\pgfqpoint{1.363851in}{1.308374in}}%
\pgfpathlineto{\pgfqpoint{1.355116in}{1.300706in}}%
\pgfpathlineto{\pgfqpoint{1.345951in}{1.309144in}}%
\pgfpathlineto{\pgfqpoint{1.335903in}{1.317223in}}%
\pgfpathlineto{\pgfqpoint{1.325012in}{1.324913in}}%
\pgfpathlineto{\pgfqpoint{1.313325in}{1.332185in}}%
\pgfpathlineto{\pgfqpoint{1.300887in}{1.339013in}}%
\pgfpathlineto{\pgfqpoint{1.307581in}{1.348054in}}%
\pgfpathlineto{\pgfqpoint{1.314553in}{1.356834in}}%
\pgfpathlineto{\pgfqpoint{1.321773in}{1.365316in}}%
\pgfpathlineto{\pgfqpoint{1.329211in}{1.373465in}}%
\pgfpathlineto{\pgfqpoint{1.336835in}{1.381248in}}%
\pgfpathlineto{\pgfqpoint{1.351773in}{1.373112in}}%
\pgfpathlineto{\pgfqpoint{1.365820in}{1.364443in}}%
\pgfpathlineto{\pgfqpoint{1.378919in}{1.355272in}}%
\pgfpathlineto{\pgfqpoint{1.391016in}{1.345634in}}%
\pgfpathlineto{\pgfqpoint{1.402061in}{1.335563in}}%
\pgfpathclose%
\pgfusepath{fill}%
\end{pgfscope}%
\begin{pgfscope}%
\pgfpathrectangle{\pgfqpoint{0.050000in}{0.050000in}}{\pgfqpoint{2.081932in}{2.081932in}}%
\pgfusepath{clip}%
\pgfsetbuttcap%
\pgfsetroundjoin%
\definecolor{currentfill}{rgb}{0.993248,0.906157,0.143936}%
\pgfsetfillcolor{currentfill}%
\pgfsetlinewidth{0.000000pt}%
\definecolor{currentstroke}{rgb}{0.000000,0.000000,0.000000}%
\pgfsetstrokecolor{currentstroke}%
\pgfsetdash{}{0pt}%
\pgfpathmoveto{\pgfqpoint{0.832224in}{1.396922in}}%
\pgfpathlineto{\pgfqpoint{0.841235in}{1.391926in}}%
\pgfpathlineto{\pgfqpoint{0.850221in}{1.386421in}}%
\pgfpathlineto{\pgfqpoint{0.859142in}{1.380429in}}%
\pgfpathlineto{\pgfqpoint{0.867963in}{1.373975in}}%
\pgfpathlineto{\pgfqpoint{0.876647in}{1.367084in}}%
\pgfpathlineto{\pgfqpoint{0.863256in}{1.358059in}}%
\pgfpathlineto{\pgfqpoint{0.850851in}{1.348557in}}%
\pgfpathlineto{\pgfqpoint{0.839484in}{1.338612in}}%
\pgfpathlineto{\pgfqpoint{0.829201in}{1.328260in}}%
\pgfpathlineto{\pgfqpoint{0.820048in}{1.317542in}}%
\pgfpathlineto{\pgfqpoint{0.809269in}{1.322692in}}%
\pgfpathlineto{\pgfqpoint{0.798314in}{1.327377in}}%
\pgfpathlineto{\pgfqpoint{0.787229in}{1.331577in}}%
\pgfpathlineto{\pgfqpoint{0.776060in}{1.335276in}}%
\pgfpathlineto{\pgfqpoint{0.764852in}{1.338459in}}%
\pgfpathlineto{\pgfqpoint{0.775779in}{1.351119in}}%
\pgfpathlineto{\pgfqpoint{0.788036in}{1.363340in}}%
\pgfpathlineto{\pgfqpoint{0.801567in}{1.375075in}}%
\pgfpathlineto{\pgfqpoint{0.816317in}{1.386283in}}%
\pgfpathlineto{\pgfqpoint{0.832224in}{1.396922in}}%
\pgfpathclose%
\pgfusepath{fill}%
\end{pgfscope}%
\begin{pgfscope}%
\pgfpathrectangle{\pgfqpoint{0.050000in}{0.050000in}}{\pgfqpoint{2.081932in}{2.081932in}}%
\pgfusepath{clip}%
\pgfsetbuttcap%
\pgfsetroundjoin%
\definecolor{currentfill}{rgb}{0.606045,0.850733,0.236712}%
\pgfsetfillcolor{currentfill}%
\pgfsetlinewidth{0.000000pt}%
\definecolor{currentstroke}{rgb}{0.000000,0.000000,0.000000}%
\pgfsetstrokecolor{currentstroke}%
\pgfsetdash{}{0pt}%
\pgfpathmoveto{\pgfqpoint{1.355116in}{1.300706in}}%
\pgfpathlineto{\pgfqpoint{1.346783in}{1.292751in}}%
\pgfpathlineto{\pgfqpoint{1.338886in}{1.284542in}}%
\pgfpathlineto{\pgfqpoint{1.331459in}{1.276114in}}%
\pgfpathlineto{\pgfqpoint{1.324531in}{1.267501in}}%
\pgfpathlineto{\pgfqpoint{1.318133in}{1.258739in}}%
\pgfpathlineto{\pgfqpoint{1.310436in}{1.265893in}}%
\pgfpathlineto{\pgfqpoint{1.301990in}{1.272745in}}%
\pgfpathlineto{\pgfqpoint{1.292830in}{1.279269in}}%
\pgfpathlineto{\pgfqpoint{1.282993in}{1.285440in}}%
\pgfpathlineto{\pgfqpoint{1.272519in}{1.291235in}}%
\pgfpathlineto{\pgfqpoint{1.277430in}{1.301001in}}%
\pgfpathlineto{\pgfqpoint{1.282745in}{1.310702in}}%
\pgfpathlineto{\pgfqpoint{1.288443in}{1.320298in}}%
\pgfpathlineto{\pgfqpoint{1.294499in}{1.329748in}}%
\pgfpathlineto{\pgfqpoint{1.300887in}{1.339013in}}%
\pgfpathlineto{\pgfqpoint{1.313325in}{1.332185in}}%
\pgfpathlineto{\pgfqpoint{1.325012in}{1.324913in}}%
\pgfpathlineto{\pgfqpoint{1.335903in}{1.317223in}}%
\pgfpathlineto{\pgfqpoint{1.345951in}{1.309144in}}%
\pgfpathlineto{\pgfqpoint{1.355116in}{1.300706in}}%
\pgfpathclose%
\pgfusepath{fill}%
\end{pgfscope}%
\begin{pgfscope}%
\pgfpathrectangle{\pgfqpoint{0.050000in}{0.050000in}}{\pgfqpoint{2.081932in}{2.081932in}}%
\pgfusepath{clip}%
\pgfsetbuttcap%
\pgfsetroundjoin%
\definecolor{currentfill}{rgb}{0.162142,0.474838,0.558140}%
\pgfsetfillcolor{currentfill}%
\pgfsetlinewidth{0.000000pt}%
\definecolor{currentstroke}{rgb}{0.000000,0.000000,0.000000}%
\pgfsetstrokecolor{currentstroke}%
\pgfsetdash{}{0pt}%
\pgfpathmoveto{\pgfqpoint{0.941073in}{1.160097in}}%
\pgfpathlineto{\pgfqpoint{0.940455in}{1.152071in}}%
\pgfpathlineto{\pgfqpoint{0.939129in}{1.144283in}}%
\pgfpathlineto{\pgfqpoint{0.937100in}{1.136764in}}%
\pgfpathlineto{\pgfqpoint{0.934379in}{1.129545in}}%
\pgfpathlineto{\pgfqpoint{0.930977in}{1.122653in}}%
\pgfpathlineto{\pgfqpoint{0.926054in}{1.115538in}}%
\pgfpathlineto{\pgfqpoint{0.921896in}{1.108247in}}%
\pgfpathlineto{\pgfqpoint{0.918523in}{1.100809in}}%
\pgfpathlineto{\pgfqpoint{0.915950in}{1.093252in}}%
\pgfpathlineto{\pgfqpoint{0.914189in}{1.085608in}}%
\pgfpathlineto{\pgfqpoint{0.917911in}{1.093202in}}%
\pgfpathlineto{\pgfqpoint{0.920887in}{1.100991in}}%
\pgfpathlineto{\pgfqpoint{0.923106in}{1.108942in}}%
\pgfpathlineto{\pgfqpoint{0.924557in}{1.117025in}}%
\pgfpathlineto{\pgfqpoint{0.925233in}{1.125208in}}%
\pgfpathlineto{\pgfqpoint{0.926889in}{1.132406in}}%
\pgfpathlineto{\pgfqpoint{0.929315in}{1.139522in}}%
\pgfpathlineto{\pgfqpoint{0.932497in}{1.146527in}}%
\pgfpathlineto{\pgfqpoint{0.936423in}{1.153395in}}%
\pgfpathlineto{\pgfqpoint{0.941073in}{1.160097in}}%
\pgfpathclose%
\pgfusepath{fill}%
\end{pgfscope}%
\begin{pgfscope}%
\pgfpathrectangle{\pgfqpoint{0.050000in}{0.050000in}}{\pgfqpoint{2.081932in}{2.081932in}}%
\pgfusepath{clip}%
\pgfsetbuttcap%
\pgfsetroundjoin%
\definecolor{currentfill}{rgb}{0.206756,0.371758,0.553117}%
\pgfsetfillcolor{currentfill}%
\pgfsetlinewidth{0.000000pt}%
\definecolor{currentstroke}{rgb}{0.000000,0.000000,0.000000}%
\pgfsetstrokecolor{currentstroke}%
\pgfsetdash{}{0pt}%
\pgfpathmoveto{\pgfqpoint{1.669026in}{1.009693in}}%
\pgfpathlineto{\pgfqpoint{1.675620in}{1.016274in}}%
\pgfpathlineto{\pgfqpoint{1.681544in}{1.023164in}}%
\pgfpathlineto{\pgfqpoint{1.686774in}{1.030337in}}%
\pgfpathlineto{\pgfqpoint{1.691289in}{1.037763in}}%
\pgfpathlineto{\pgfqpoint{1.695069in}{1.045413in}}%
\pgfpathlineto{\pgfqpoint{1.696814in}{1.067105in}}%
\pgfpathlineto{\pgfqpoint{1.696225in}{1.088760in}}%
\pgfpathlineto{\pgfqpoint{1.693322in}{1.110293in}}%
\pgfpathlineto{\pgfqpoint{1.688135in}{1.131619in}}%
\pgfpathlineto{\pgfqpoint{1.680702in}{1.152654in}}%
\pgfpathlineto{\pgfqpoint{1.677065in}{1.144408in}}%
\pgfpathlineto{\pgfqpoint{1.672719in}{1.136246in}}%
\pgfpathlineto{\pgfqpoint{1.667682in}{1.128201in}}%
\pgfpathlineto{\pgfqpoint{1.661977in}{1.120305in}}%
\pgfpathlineto{\pgfqpoint{1.655625in}{1.112588in}}%
\pgfpathlineto{\pgfqpoint{1.662663in}{1.092395in}}%
\pgfpathlineto{\pgfqpoint{1.667554in}{1.071929in}}%
\pgfpathlineto{\pgfqpoint{1.670261in}{1.051268in}}%
\pgfpathlineto{\pgfqpoint{1.670758in}{1.030495in}}%
\pgfpathlineto{\pgfqpoint{1.669026in}{1.009693in}}%
\pgfpathclose%
\pgfusepath{fill}%
\end{pgfscope}%
\begin{pgfscope}%
\pgfpathrectangle{\pgfqpoint{0.050000in}{0.050000in}}{\pgfqpoint{2.081932in}{2.081932in}}%
\pgfusepath{clip}%
\pgfsetbuttcap%
\pgfsetroundjoin%
\definecolor{currentfill}{rgb}{0.278012,0.180367,0.486697}%
\pgfsetfillcolor{currentfill}%
\pgfsetlinewidth{0.000000pt}%
\definecolor{currentstroke}{rgb}{0.000000,0.000000,0.000000}%
\pgfsetstrokecolor{currentstroke}%
\pgfsetdash{}{0pt}%
\pgfpathmoveto{\pgfqpoint{0.885073in}{1.051566in}}%
\pgfpathlineto{\pgfqpoint{0.877318in}{1.045731in}}%
\pgfpathlineto{\pgfqpoint{0.868987in}{1.040276in}}%
\pgfpathlineto{\pgfqpoint{0.860114in}{1.035223in}}%
\pgfpathlineto{\pgfqpoint{0.850734in}{1.030592in}}%
\pgfpathlineto{\pgfqpoint{0.840887in}{1.026400in}}%
\pgfpathlineto{\pgfqpoint{0.839518in}{1.015864in}}%
\pgfpathlineto{\pgfqpoint{0.839273in}{1.005285in}}%
\pgfpathlineto{\pgfqpoint{0.840156in}{0.994704in}}%
\pgfpathlineto{\pgfqpoint{0.842169in}{0.984164in}}%
\pgfpathlineto{\pgfqpoint{0.845307in}{0.973709in}}%
\pgfpathlineto{\pgfqpoint{0.855058in}{0.979803in}}%
\pgfpathlineto{\pgfqpoint{0.864341in}{0.986247in}}%
\pgfpathlineto{\pgfqpoint{0.873119in}{0.993016in}}%
\pgfpathlineto{\pgfqpoint{0.881358in}{1.000083in}}%
\pgfpathlineto{\pgfqpoint{0.889024in}{1.007421in}}%
\pgfpathlineto{\pgfqpoint{0.886340in}{1.016173in}}%
\pgfpathlineto{\pgfqpoint{0.884601in}{1.025000in}}%
\pgfpathlineto{\pgfqpoint{0.883811in}{1.033864in}}%
\pgfpathlineto{\pgfqpoint{0.883969in}{1.042732in}}%
\pgfpathlineto{\pgfqpoint{0.885073in}{1.051566in}}%
\pgfpathclose%
\pgfusepath{fill}%
\end{pgfscope}%
\begin{pgfscope}%
\pgfpathrectangle{\pgfqpoint{0.050000in}{0.050000in}}{\pgfqpoint{2.081932in}{2.081932in}}%
\pgfusepath{clip}%
\pgfsetbuttcap%
\pgfsetroundjoin%
\definecolor{currentfill}{rgb}{0.282327,0.094955,0.417331}%
\pgfsetfillcolor{currentfill}%
\pgfsetlinewidth{0.000000pt}%
\definecolor{currentstroke}{rgb}{0.000000,0.000000,0.000000}%
\pgfsetstrokecolor{currentstroke}%
\pgfsetdash{}{0pt}%
\pgfpathmoveto{\pgfqpoint{0.666506in}{1.022667in}}%
\pgfpathlineto{\pgfqpoint{0.655204in}{1.026398in}}%
\pgfpathlineto{\pgfqpoint{0.644205in}{1.030581in}}%
\pgfpathlineto{\pgfqpoint{0.633552in}{1.035201in}}%
\pgfpathlineto{\pgfqpoint{0.623288in}{1.040239in}}%
\pgfpathlineto{\pgfqpoint{0.613452in}{1.045677in}}%
\pgfpathlineto{\pgfqpoint{0.610435in}{1.026555in}}%
\pgfpathlineto{\pgfqpoint{0.609455in}{1.007314in}}%
\pgfpathlineto{\pgfqpoint{0.610531in}{0.988029in}}%
\pgfpathlineto{\pgfqpoint{0.613672in}{0.968780in}}%
\pgfpathlineto{\pgfqpoint{0.618880in}{0.949646in}}%
\pgfpathlineto{\pgfqpoint{0.628721in}{0.946035in}}%
\pgfpathlineto{\pgfqpoint{0.638986in}{0.942912in}}%
\pgfpathlineto{\pgfqpoint{0.649635in}{0.940288in}}%
\pgfpathlineto{\pgfqpoint{0.660624in}{0.938173in}}%
\pgfpathlineto{\pgfqpoint{0.671910in}{0.936575in}}%
\pgfpathlineto{\pgfqpoint{0.667145in}{0.953713in}}%
\pgfpathlineto{\pgfqpoint{0.664224in}{0.970961in}}%
\pgfpathlineto{\pgfqpoint{0.663151in}{0.988249in}}%
\pgfpathlineto{\pgfqpoint{0.663916in}{1.005508in}}%
\pgfpathlineto{\pgfqpoint{0.666506in}{1.022667in}}%
\pgfpathclose%
\pgfusepath{fill}%
\end{pgfscope}%
\begin{pgfscope}%
\pgfpathrectangle{\pgfqpoint{0.050000in}{0.050000in}}{\pgfqpoint{2.081932in}{2.081932in}}%
\pgfusepath{clip}%
\pgfsetbuttcap%
\pgfsetroundjoin%
\definecolor{currentfill}{rgb}{0.227802,0.326594,0.546532}%
\pgfsetfillcolor{currentfill}%
\pgfsetlinewidth{0.000000pt}%
\definecolor{currentstroke}{rgb}{0.000000,0.000000,0.000000}%
\pgfsetstrokecolor{currentstroke}%
\pgfsetdash{}{0pt}%
\pgfpathmoveto{\pgfqpoint{1.324090in}{1.060151in}}%
\pgfpathlineto{\pgfqpoint{1.328553in}{1.052203in}}%
\pgfpathlineto{\pgfqpoint{1.333729in}{1.044420in}}%
\pgfpathlineto{\pgfqpoint{1.339595in}{1.036832in}}%
\pgfpathlineto{\pgfqpoint{1.346129in}{1.029469in}}%
\pgfpathlineto{\pgfqpoint{1.353302in}{1.022360in}}%
\pgfpathlineto{\pgfqpoint{1.354379in}{1.031217in}}%
\pgfpathlineto{\pgfqpoint{1.354506in}{1.040087in}}%
\pgfpathlineto{\pgfqpoint{1.353687in}{1.048935in}}%
\pgfpathlineto{\pgfqpoint{1.351927in}{1.057724in}}%
\pgfpathlineto{\pgfqpoint{1.349236in}{1.066421in}}%
\pgfpathlineto{\pgfqpoint{1.342224in}{1.072151in}}%
\pgfpathlineto{\pgfqpoint{1.335835in}{1.078254in}}%
\pgfpathlineto{\pgfqpoint{1.330097in}{1.084708in}}%
\pgfpathlineto{\pgfqpoint{1.325033in}{1.091487in}}%
\pgfpathlineto{\pgfqpoint{1.320666in}{1.098565in}}%
\pgfpathlineto{\pgfqpoint{1.322995in}{1.090979in}}%
\pgfpathlineto{\pgfqpoint{1.324509in}{1.083314in}}%
\pgfpathlineto{\pgfqpoint{1.325199in}{1.075600in}}%
\pgfpathlineto{\pgfqpoint{1.325060in}{1.067869in}}%
\pgfpathlineto{\pgfqpoint{1.324090in}{1.060151in}}%
\pgfpathclose%
\pgfusepath{fill}%
\end{pgfscope}%
\begin{pgfscope}%
\pgfpathrectangle{\pgfqpoint{0.050000in}{0.050000in}}{\pgfqpoint{2.081932in}{2.081932in}}%
\pgfusepath{clip}%
\pgfsetbuttcap%
\pgfsetroundjoin%
\definecolor{currentfill}{rgb}{0.296479,0.761561,0.424223}%
\pgfsetfillcolor{currentfill}%
\pgfsetlinewidth{0.000000pt}%
\definecolor{currentstroke}{rgb}{0.000000,0.000000,0.000000}%
\pgfsetstrokecolor{currentstroke}%
\pgfsetdash{}{0pt}%
\pgfpathmoveto{\pgfqpoint{0.948395in}{1.281148in}}%
\pgfpathlineto{\pgfqpoint{0.953397in}{1.271642in}}%
\pgfpathlineto{\pgfqpoint{0.957904in}{1.262124in}}%
\pgfpathlineto{\pgfqpoint{0.961898in}{1.252635in}}%
\pgfpathlineto{\pgfqpoint{0.965362in}{1.243213in}}%
\pgfpathlineto{\pgfqpoint{0.968282in}{1.233898in}}%
\pgfpathlineto{\pgfqpoint{0.960020in}{1.228191in}}%
\pgfpathlineto{\pgfqpoint{0.952381in}{1.222188in}}%
\pgfpathlineto{\pgfqpoint{0.945397in}{1.215911in}}%
\pgfpathlineto{\pgfqpoint{0.939097in}{1.209383in}}%
\pgfpathlineto{\pgfqpoint{0.933508in}{1.202631in}}%
\pgfpathlineto{\pgfqpoint{0.929900in}{1.211375in}}%
\pgfpathlineto{\pgfqpoint{0.925620in}{1.220115in}}%
\pgfpathlineto{\pgfqpoint{0.920685in}{1.228815in}}%
\pgfpathlineto{\pgfqpoint{0.915114in}{1.237439in}}%
\pgfpathlineto{\pgfqpoint{0.908930in}{1.245950in}}%
\pgfpathlineto{\pgfqpoint{0.915281in}{1.253554in}}%
\pgfpathlineto{\pgfqpoint{0.922436in}{1.260904in}}%
\pgfpathlineto{\pgfqpoint{0.930362in}{1.267970in}}%
\pgfpathlineto{\pgfqpoint{0.939027in}{1.274727in}}%
\pgfpathlineto{\pgfqpoint{0.948395in}{1.281148in}}%
\pgfpathclose%
\pgfusepath{fill}%
\end{pgfscope}%
\begin{pgfscope}%
\pgfpathrectangle{\pgfqpoint{0.050000in}{0.050000in}}{\pgfqpoint{2.081932in}{2.081932in}}%
\pgfusepath{clip}%
\pgfsetbuttcap%
\pgfsetroundjoin%
\definecolor{currentfill}{rgb}{0.876168,0.891125,0.095250}%
\pgfsetfillcolor{currentfill}%
\pgfsetlinewidth{0.000000pt}%
\definecolor{currentstroke}{rgb}{0.000000,0.000000,0.000000}%
\pgfsetstrokecolor{currentstroke}%
\pgfsetdash{}{0pt}%
\pgfpathmoveto{\pgfqpoint{1.612200in}{1.282881in}}%
\pgfpathlineto{\pgfqpoint{1.602316in}{1.286342in}}%
\pgfpathlineto{\pgfqpoint{1.592007in}{1.289309in}}%
\pgfpathlineto{\pgfqpoint{1.581315in}{1.291770in}}%
\pgfpathlineto{\pgfqpoint{1.570282in}{1.293713in}}%
\pgfpathlineto{\pgfqpoint{1.558953in}{1.295131in}}%
\pgfpathlineto{\pgfqpoint{1.551647in}{1.311041in}}%
\pgfpathlineto{\pgfqpoint{1.542647in}{1.326622in}}%
\pgfpathlineto{\pgfqpoint{1.531996in}{1.341817in}}%
\pgfpathlineto{\pgfqpoint{1.519747in}{1.356567in}}%
\pgfpathlineto{\pgfqpoint{1.505954in}{1.370819in}}%
\pgfpathlineto{\pgfqpoint{1.515821in}{1.371346in}}%
\pgfpathlineto{\pgfqpoint{1.525425in}{1.371302in}}%
\pgfpathlineto{\pgfqpoint{1.534730in}{1.370687in}}%
\pgfpathlineto{\pgfqpoint{1.543696in}{1.369507in}}%
\pgfpathlineto{\pgfqpoint{1.552288in}{1.367766in}}%
\pgfpathlineto{\pgfqpoint{1.567829in}{1.351797in}}%
\pgfpathlineto{\pgfqpoint{1.581652in}{1.335263in}}%
\pgfpathlineto{\pgfqpoint{1.593693in}{1.318223in}}%
\pgfpathlineto{\pgfqpoint{1.603893in}{1.300740in}}%
\pgfpathlineto{\pgfqpoint{1.612200in}{1.282881in}}%
\pgfpathclose%
\pgfusepath{fill}%
\end{pgfscope}%
\begin{pgfscope}%
\pgfpathrectangle{\pgfqpoint{0.050000in}{0.050000in}}{\pgfqpoint{2.081932in}{2.081932in}}%
\pgfusepath{clip}%
\pgfsetbuttcap%
\pgfsetroundjoin%
\definecolor{currentfill}{rgb}{0.855810,0.888601,0.097452}%
\pgfsetfillcolor{currentfill}%
\pgfsetlinewidth{0.000000pt}%
\definecolor{currentstroke}{rgb}{0.000000,0.000000,0.000000}%
\pgfsetstrokecolor{currentstroke}%
\pgfsetdash{}{0pt}%
\pgfpathmoveto{\pgfqpoint{0.876647in}{1.367084in}}%
\pgfpathlineto{\pgfqpoint{0.885156in}{1.359785in}}%
\pgfpathlineto{\pgfqpoint{0.893457in}{1.352107in}}%
\pgfpathlineto{\pgfqpoint{0.901513in}{1.344083in}}%
\pgfpathlineto{\pgfqpoint{0.909291in}{1.335745in}}%
\pgfpathlineto{\pgfqpoint{0.916758in}{1.327128in}}%
\pgfpathlineto{\pgfqpoint{0.905623in}{1.319560in}}%
\pgfpathlineto{\pgfqpoint{0.895316in}{1.311594in}}%
\pgfpathlineto{\pgfqpoint{0.885880in}{1.303260in}}%
\pgfpathlineto{\pgfqpoint{0.877356in}{1.294589in}}%
\pgfpathlineto{\pgfqpoint{0.869779in}{1.285615in}}%
\pgfpathlineto{\pgfqpoint{0.860528in}{1.292738in}}%
\pgfpathlineto{\pgfqpoint{0.850889in}{1.299520in}}%
\pgfpathlineto{\pgfqpoint{0.840902in}{1.305932in}}%
\pgfpathlineto{\pgfqpoint{0.830607in}{1.311947in}}%
\pgfpathlineto{\pgfqpoint{0.820048in}{1.317542in}}%
\pgfpathlineto{\pgfqpoint{0.829201in}{1.328260in}}%
\pgfpathlineto{\pgfqpoint{0.839484in}{1.338612in}}%
\pgfpathlineto{\pgfqpoint{0.850851in}{1.348557in}}%
\pgfpathlineto{\pgfqpoint{0.863256in}{1.358059in}}%
\pgfpathlineto{\pgfqpoint{0.876647in}{1.367084in}}%
\pgfpathclose%
\pgfusepath{fill}%
\end{pgfscope}%
\begin{pgfscope}%
\pgfpathrectangle{\pgfqpoint{0.050000in}{0.050000in}}{\pgfqpoint{2.081932in}{2.081932in}}%
\pgfusepath{clip}%
\pgfsetbuttcap%
\pgfsetroundjoin%
\definecolor{currentfill}{rgb}{0.120638,0.625828,0.533488}%
\pgfsetfillcolor{currentfill}%
\pgfsetlinewidth{0.000000pt}%
\definecolor{currentstroke}{rgb}{0.000000,0.000000,0.000000}%
\pgfsetstrokecolor{currentstroke}%
\pgfsetdash{}{0pt}%
\pgfpathmoveto{\pgfqpoint{1.317949in}{1.179100in}}%
\pgfpathlineto{\pgfqpoint{1.314812in}{1.170734in}}%
\pgfpathlineto{\pgfqpoint{1.312424in}{1.162348in}}%
\pgfpathlineto{\pgfqpoint{1.310794in}{1.153975in}}%
\pgfpathlineto{\pgfqpoint{1.309927in}{1.145651in}}%
\pgfpathlineto{\pgfqpoint{1.309826in}{1.137408in}}%
\pgfpathlineto{\pgfqpoint{1.306870in}{1.144450in}}%
\pgfpathlineto{\pgfqpoint{1.303167in}{1.151361in}}%
\pgfpathlineto{\pgfqpoint{1.298733in}{1.158115in}}%
\pgfpathlineto{\pgfqpoint{1.293587in}{1.164686in}}%
\pgfpathlineto{\pgfqpoint{1.287752in}{1.171049in}}%
\pgfpathlineto{\pgfqpoint{1.287842in}{1.179275in}}%
\pgfpathlineto{\pgfqpoint{1.288605in}{1.187717in}}%
\pgfpathlineto{\pgfqpoint{1.290041in}{1.196340in}}%
\pgfpathlineto{\pgfqpoint{1.292143in}{1.205109in}}%
\pgfpathlineto{\pgfqpoint{1.294905in}{1.213988in}}%
\pgfpathlineto{\pgfqpoint{1.300993in}{1.207391in}}%
\pgfpathlineto{\pgfqpoint{1.306363in}{1.200576in}}%
\pgfpathlineto{\pgfqpoint{1.310992in}{1.193572in}}%
\pgfpathlineto{\pgfqpoint{1.314860in}{1.186403in}}%
\pgfpathlineto{\pgfqpoint{1.317949in}{1.179100in}}%
\pgfpathclose%
\pgfusepath{fill}%
\end{pgfscope}%
\begin{pgfscope}%
\pgfpathrectangle{\pgfqpoint{0.050000in}{0.050000in}}{\pgfqpoint{2.081932in}{2.081932in}}%
\pgfusepath{clip}%
\pgfsetbuttcap%
\pgfsetroundjoin%
\definecolor{currentfill}{rgb}{0.606045,0.850733,0.236712}%
\pgfsetfillcolor{currentfill}%
\pgfsetlinewidth{0.000000pt}%
\definecolor{currentstroke}{rgb}{0.000000,0.000000,0.000000}%
\pgfsetstrokecolor{currentstroke}%
\pgfsetdash{}{0pt}%
\pgfpathmoveto{\pgfqpoint{0.916758in}{1.327128in}}%
\pgfpathlineto{\pgfqpoint{0.923884in}{1.318268in}}%
\pgfpathlineto{\pgfqpoint{0.930638in}{1.309202in}}%
\pgfpathlineto{\pgfqpoint{0.936992in}{1.299968in}}%
\pgfpathlineto{\pgfqpoint{0.942919in}{1.290603in}}%
\pgfpathlineto{\pgfqpoint{0.948395in}{1.281148in}}%
\pgfpathlineto{\pgfqpoint{0.939027in}{1.274727in}}%
\pgfpathlineto{\pgfqpoint{0.930362in}{1.267970in}}%
\pgfpathlineto{\pgfqpoint{0.922436in}{1.260904in}}%
\pgfpathlineto{\pgfqpoint{0.915281in}{1.253554in}}%
\pgfpathlineto{\pgfqpoint{0.908930in}{1.245950in}}%
\pgfpathlineto{\pgfqpoint{0.902158in}{1.254314in}}%
\pgfpathlineto{\pgfqpoint{0.894825in}{1.262496in}}%
\pgfpathlineto{\pgfqpoint{0.886963in}{1.270462in}}%
\pgfpathlineto{\pgfqpoint{0.878603in}{1.278179in}}%
\pgfpathlineto{\pgfqpoint{0.869779in}{1.285615in}}%
\pgfpathlineto{\pgfqpoint{0.877356in}{1.294589in}}%
\pgfpathlineto{\pgfqpoint{0.885880in}{1.303260in}}%
\pgfpathlineto{\pgfqpoint{0.895316in}{1.311594in}}%
\pgfpathlineto{\pgfqpoint{0.905623in}{1.319560in}}%
\pgfpathlineto{\pgfqpoint{0.916758in}{1.327128in}}%
\pgfpathclose%
\pgfusepath{fill}%
\end{pgfscope}%
\begin{pgfscope}%
\pgfpathrectangle{\pgfqpoint{0.050000in}{0.050000in}}{\pgfqpoint{2.081932in}{2.081932in}}%
\pgfusepath{clip}%
\pgfsetbuttcap%
\pgfsetroundjoin%
\definecolor{currentfill}{rgb}{0.636902,0.856542,0.216620}%
\pgfsetfillcolor{currentfill}%
\pgfsetlinewidth{0.000000pt}%
\definecolor{currentstroke}{rgb}{0.000000,0.000000,0.000000}%
\pgfsetstrokecolor{currentstroke}%
\pgfsetdash{}{0pt}%
\pgfpathmoveto{\pgfqpoint{0.622167in}{1.321130in}}%
\pgfpathlineto{\pgfqpoint{0.628808in}{1.325923in}}%
\pgfpathlineto{\pgfqpoint{0.636006in}{1.330249in}}%
\pgfpathlineto{\pgfqpoint{0.643735in}{1.334090in}}%
\pgfpathlineto{\pgfqpoint{0.651962in}{1.337430in}}%
\pgfpathlineto{\pgfqpoint{0.660657in}{1.340253in}}%
\pgfpathlineto{\pgfqpoint{0.648072in}{1.323357in}}%
\pgfpathlineto{\pgfqpoint{0.637312in}{1.306000in}}%
\pgfpathlineto{\pgfqpoint{0.628429in}{1.288247in}}%
\pgfpathlineto{\pgfqpoint{0.621472in}{1.270163in}}%
\pgfpathlineto{\pgfqpoint{0.616480in}{1.251820in}}%
\pgfpathlineto{\pgfqpoint{0.606840in}{1.247269in}}%
\pgfpathlineto{\pgfqpoint{0.597714in}{1.242285in}}%
\pgfpathlineto{\pgfqpoint{0.589138in}{1.236889in}}%
\pgfpathlineto{\pgfqpoint{0.581147in}{1.231104in}}%
\pgfpathlineto{\pgfqpoint{0.573773in}{1.224954in}}%
\pgfpathlineto{\pgfqpoint{0.579293in}{1.244918in}}%
\pgfpathlineto{\pgfqpoint{0.586942in}{1.264592in}}%
\pgfpathlineto{\pgfqpoint{0.596673in}{1.283899in}}%
\pgfpathlineto{\pgfqpoint{0.608435in}{1.302769in}}%
\pgfpathlineto{\pgfqpoint{0.622167in}{1.321130in}}%
\pgfpathclose%
\pgfusepath{fill}%
\end{pgfscope}%
\begin{pgfscope}%
\pgfpathrectangle{\pgfqpoint{0.050000in}{0.050000in}}{\pgfqpoint{2.081932in}{2.081932in}}%
\pgfusepath{clip}%
\pgfsetbuttcap%
\pgfsetroundjoin%
\definecolor{currentfill}{rgb}{0.227802,0.326594,0.546532}%
\pgfsetfillcolor{currentfill}%
\pgfsetlinewidth{0.000000pt}%
\definecolor{currentstroke}{rgb}{0.000000,0.000000,0.000000}%
\pgfsetstrokecolor{currentstroke}%
\pgfsetdash{}{0pt}%
\pgfpathmoveto{\pgfqpoint{0.914189in}{1.085608in}}%
\pgfpathlineto{\pgfqpoint{0.909739in}{1.078238in}}%
\pgfpathlineto{\pgfqpoint{0.904580in}{1.071121in}}%
\pgfpathlineto{\pgfqpoint{0.898732in}{1.064286in}}%
\pgfpathlineto{\pgfqpoint{0.892221in}{1.057759in}}%
\pgfpathlineto{\pgfqpoint{0.885073in}{1.051566in}}%
\pgfpathlineto{\pgfqpoint{0.883969in}{1.042732in}}%
\pgfpathlineto{\pgfqpoint{0.883811in}{1.033864in}}%
\pgfpathlineto{\pgfqpoint{0.884601in}{1.025000in}}%
\pgfpathlineto{\pgfqpoint{0.886340in}{1.016173in}}%
\pgfpathlineto{\pgfqpoint{0.889024in}{1.007421in}}%
\pgfpathlineto{\pgfqpoint{0.896088in}{1.015001in}}%
\pgfpathlineto{\pgfqpoint{0.902521in}{1.022794in}}%
\pgfpathlineto{\pgfqpoint{0.908296in}{1.030769in}}%
\pgfpathlineto{\pgfqpoint{0.913390in}{1.038894in}}%
\pgfpathlineto{\pgfqpoint{0.917783in}{1.047139in}}%
\pgfpathlineto{\pgfqpoint{0.915408in}{1.054762in}}%
\pgfpathlineto{\pgfqpoint{0.913858in}{1.062451in}}%
\pgfpathlineto{\pgfqpoint{0.913139in}{1.070176in}}%
\pgfpathlineto{\pgfqpoint{0.913250in}{1.077905in}}%
\pgfpathlineto{\pgfqpoint{0.914189in}{1.085608in}}%
\pgfpathclose%
\pgfusepath{fill}%
\end{pgfscope}%
\begin{pgfscope}%
\pgfpathrectangle{\pgfqpoint{0.050000in}{0.050000in}}{\pgfqpoint{2.081932in}{2.081932in}}%
\pgfusepath{clip}%
\pgfsetbuttcap%
\pgfsetroundjoin%
\definecolor{currentfill}{rgb}{0.267968,0.223549,0.512008}%
\pgfsetfillcolor{currentfill}%
\pgfsetlinewidth{0.000000pt}%
\definecolor{currentstroke}{rgb}{0.000000,0.000000,0.000000}%
\pgfsetstrokecolor{currentstroke}%
\pgfsetdash{}{0pt}%
\pgfpathmoveto{\pgfqpoint{0.613452in}{1.045677in}}%
\pgfpathlineto{\pgfqpoint{0.604085in}{1.051493in}}%
\pgfpathlineto{\pgfqpoint{0.595224in}{1.057663in}}%
\pgfpathlineto{\pgfqpoint{0.586903in}{1.064165in}}%
\pgfpathlineto{\pgfqpoint{0.579157in}{1.070973in}}%
\pgfpathlineto{\pgfqpoint{0.572017in}{1.078061in}}%
\pgfpathlineto{\pgfqpoint{0.568648in}{1.057450in}}%
\pgfpathlineto{\pgfqpoint{0.567482in}{1.036702in}}%
\pgfpathlineto{\pgfqpoint{0.568541in}{1.015899in}}%
\pgfpathlineto{\pgfqpoint{0.571836in}{0.995127in}}%
\pgfpathlineto{\pgfqpoint{0.577373in}{0.974470in}}%
\pgfpathlineto{\pgfqpoint{0.584531in}{0.968663in}}%
\pgfpathlineto{\pgfqpoint{0.592294in}{0.963257in}}%
\pgfpathlineto{\pgfqpoint{0.600630in}{0.958273in}}%
\pgfpathlineto{\pgfqpoint{0.609504in}{0.953730in}}%
\pgfpathlineto{\pgfqpoint{0.618880in}{0.949646in}}%
\pgfpathlineto{\pgfqpoint{0.613672in}{0.968780in}}%
\pgfpathlineto{\pgfqpoint{0.610531in}{0.988029in}}%
\pgfpathlineto{\pgfqpoint{0.609455in}{1.007314in}}%
\pgfpathlineto{\pgfqpoint{0.610435in}{1.026555in}}%
\pgfpathlineto{\pgfqpoint{0.613452in}{1.045677in}}%
\pgfpathclose%
\pgfusepath{fill}%
\end{pgfscope}%
\begin{pgfscope}%
\pgfpathrectangle{\pgfqpoint{0.050000in}{0.050000in}}{\pgfqpoint{2.081932in}{2.081932in}}%
\pgfusepath{clip}%
\pgfsetbuttcap%
\pgfsetroundjoin%
\definecolor{currentfill}{rgb}{0.150476,0.504369,0.557430}%
\pgfsetfillcolor{currentfill}%
\pgfsetlinewidth{0.000000pt}%
\definecolor{currentstroke}{rgb}{0.000000,0.000000,0.000000}%
\pgfsetstrokecolor{currentstroke}%
\pgfsetdash{}{0pt}%
\pgfpathmoveto{\pgfqpoint{1.695069in}{1.045413in}}%
\pgfpathlineto{\pgfqpoint{1.698097in}{1.053258in}}%
\pgfpathlineto{\pgfqpoint{1.700360in}{1.061266in}}%
\pgfpathlineto{\pgfqpoint{1.701848in}{1.069404in}}%
\pgfpathlineto{\pgfqpoint{1.702554in}{1.077640in}}%
\pgfpathlineto{\pgfqpoint{1.704302in}{1.099527in}}%
\pgfpathlineto{\pgfqpoint{1.703685in}{1.121376in}}%
\pgfpathlineto{\pgfqpoint{1.700725in}{1.143100in}}%
\pgfpathlineto{\pgfqpoint{1.695452in}{1.164613in}}%
\pgfpathlineto{\pgfqpoint{1.687904in}{1.185832in}}%
\pgfpathlineto{\pgfqpoint{1.687226in}{1.177575in}}%
\pgfpathlineto{\pgfqpoint{1.685794in}{1.169271in}}%
\pgfpathlineto{\pgfqpoint{1.683616in}{1.160953in}}%
\pgfpathlineto{\pgfqpoint{1.680702in}{1.152654in}}%
\pgfpathlineto{\pgfqpoint{1.688135in}{1.131619in}}%
\pgfpathlineto{\pgfqpoint{1.693322in}{1.110293in}}%
\pgfpathlineto{\pgfqpoint{1.696225in}{1.088760in}}%
\pgfpathlineto{\pgfqpoint{1.696814in}{1.067105in}}%
\pgfpathlineto{\pgfqpoint{1.695069in}{1.045413in}}%
\pgfpathclose%
\pgfusepath{fill}%
\end{pgfscope}%
\begin{pgfscope}%
\pgfpathrectangle{\pgfqpoint{0.050000in}{0.050000in}}{\pgfqpoint{2.081932in}{2.081932in}}%
\pgfusepath{clip}%
\pgfsetbuttcap%
\pgfsetroundjoin%
\definecolor{currentfill}{rgb}{0.120638,0.625828,0.533488}%
\pgfsetfillcolor{currentfill}%
\pgfsetlinewidth{0.000000pt}%
\definecolor{currentstroke}{rgb}{0.000000,0.000000,0.000000}%
\pgfsetstrokecolor{currentstroke}%
\pgfsetdash{}{0pt}%
\pgfpathmoveto{\pgfqpoint{0.933508in}{1.202631in}}%
\pgfpathlineto{\pgfqpoint{0.936429in}{1.193919in}}%
\pgfpathlineto{\pgfqpoint{0.938653in}{1.185274in}}%
\pgfpathlineto{\pgfqpoint{0.940171in}{1.176732in}}%
\pgfpathlineto{\pgfqpoint{0.940979in}{1.168329in}}%
\pgfpathlineto{\pgfqpoint{0.941073in}{1.160097in}}%
\pgfpathlineto{\pgfqpoint{0.936423in}{1.153395in}}%
\pgfpathlineto{\pgfqpoint{0.932497in}{1.146527in}}%
\pgfpathlineto{\pgfqpoint{0.929315in}{1.139522in}}%
\pgfpathlineto{\pgfqpoint{0.926889in}{1.132406in}}%
\pgfpathlineto{\pgfqpoint{0.925233in}{1.125208in}}%
\pgfpathlineto{\pgfqpoint{0.925130in}{1.133456in}}%
\pgfpathlineto{\pgfqpoint{0.924247in}{1.141737in}}%
\pgfpathlineto{\pgfqpoint{0.922586in}{1.150018in}}%
\pgfpathlineto{\pgfqpoint{0.920153in}{1.158265in}}%
\pgfpathlineto{\pgfqpoint{0.916958in}{1.166443in}}%
\pgfpathlineto{\pgfqpoint{0.918692in}{1.173911in}}%
\pgfpathlineto{\pgfqpoint{0.921228in}{1.181292in}}%
\pgfpathlineto{\pgfqpoint{0.924554in}{1.188558in}}%
\pgfpathlineto{\pgfqpoint{0.928653in}{1.195680in}}%
\pgfpathlineto{\pgfqpoint{0.933508in}{1.202631in}}%
\pgfpathclose%
\pgfusepath{fill}%
\end{pgfscope}%
\begin{pgfscope}%
\pgfpathrectangle{\pgfqpoint{0.050000in}{0.050000in}}{\pgfqpoint{2.081932in}{2.081932in}}%
\pgfusepath{clip}%
\pgfsetbuttcap%
\pgfsetroundjoin%
\definecolor{currentfill}{rgb}{0.162142,0.474838,0.558140}%
\pgfsetfillcolor{currentfill}%
\pgfsetlinewidth{0.000000pt}%
\definecolor{currentstroke}{rgb}{0.000000,0.000000,0.000000}%
\pgfsetstrokecolor{currentstroke}%
\pgfsetdash{}{0pt}%
\pgfpathmoveto{\pgfqpoint{1.313017in}{1.101243in}}%
\pgfpathlineto{\pgfqpoint{1.313695in}{1.092953in}}%
\pgfpathlineto{\pgfqpoint{1.315149in}{1.084666in}}%
\pgfpathlineto{\pgfqpoint{1.317374in}{1.076415in}}%
\pgfpathlineto{\pgfqpoint{1.320359in}{1.068232in}}%
\pgfpathlineto{\pgfqpoint{1.324090in}{1.060151in}}%
\pgfpathlineto{\pgfqpoint{1.325060in}{1.067869in}}%
\pgfpathlineto{\pgfqpoint{1.325199in}{1.075600in}}%
\pgfpathlineto{\pgfqpoint{1.324509in}{1.083314in}}%
\pgfpathlineto{\pgfqpoint{1.322995in}{1.090979in}}%
\pgfpathlineto{\pgfqpoint{1.320666in}{1.098565in}}%
\pgfpathlineto{\pgfqpoint{1.317013in}{1.105912in}}%
\pgfpathlineto{\pgfqpoint{1.314091in}{1.113501in}}%
\pgfpathlineto{\pgfqpoint{1.311913in}{1.121301in}}%
\pgfpathlineto{\pgfqpoint{1.310489in}{1.129281in}}%
\pgfpathlineto{\pgfqpoint{1.309826in}{1.137408in}}%
\pgfpathlineto{\pgfqpoint{1.312020in}{1.130265in}}%
\pgfpathlineto{\pgfqpoint{1.313443in}{1.123048in}}%
\pgfpathlineto{\pgfqpoint{1.314086in}{1.115786in}}%
\pgfpathlineto{\pgfqpoint{1.313945in}{1.108508in}}%
\pgfpathlineto{\pgfqpoint{1.313017in}{1.101243in}}%
\pgfpathclose%
\pgfusepath{fill}%
\end{pgfscope}%
\begin{pgfscope}%
\pgfpathrectangle{\pgfqpoint{0.050000in}{0.050000in}}{\pgfqpoint{2.081932in}{2.081932in}}%
\pgfusepath{clip}%
\pgfsetbuttcap%
\pgfsetroundjoin%
\definecolor{currentfill}{rgb}{0.993248,0.906157,0.143936}%
\pgfsetfillcolor{currentfill}%
\pgfsetlinewidth{0.000000pt}%
\definecolor{currentstroke}{rgb}{0.000000,0.000000,0.000000}%
\pgfsetstrokecolor{currentstroke}%
\pgfsetdash{}{0pt}%
\pgfpathmoveto{\pgfqpoint{1.558953in}{1.295131in}}%
\pgfpathlineto{\pgfqpoint{1.547375in}{1.296017in}}%
\pgfpathlineto{\pgfqpoint{1.535596in}{1.296366in}}%
\pgfpathlineto{\pgfqpoint{1.523663in}{1.296177in}}%
\pgfpathlineto{\pgfqpoint{1.511626in}{1.295450in}}%
\pgfpathlineto{\pgfqpoint{1.499535in}{1.294188in}}%
\pgfpathlineto{\pgfqpoint{1.493317in}{1.307954in}}%
\pgfpathlineto{\pgfqpoint{1.485626in}{1.321442in}}%
\pgfpathlineto{\pgfqpoint{1.476502in}{1.334603in}}%
\pgfpathlineto{\pgfqpoint{1.465985in}{1.347385in}}%
\pgfpathlineto{\pgfqpoint{1.454124in}{1.359742in}}%
\pgfpathlineto{\pgfqpoint{1.464682in}{1.363059in}}%
\pgfpathlineto{\pgfqpoint{1.475187in}{1.365835in}}%
\pgfpathlineto{\pgfqpoint{1.485596in}{1.368059in}}%
\pgfpathlineto{\pgfqpoint{1.495865in}{1.369722in}}%
\pgfpathlineto{\pgfqpoint{1.505954in}{1.370819in}}%
\pgfpathlineto{\pgfqpoint{1.519747in}{1.356567in}}%
\pgfpathlineto{\pgfqpoint{1.531996in}{1.341817in}}%
\pgfpathlineto{\pgfqpoint{1.542647in}{1.326622in}}%
\pgfpathlineto{\pgfqpoint{1.551647in}{1.311041in}}%
\pgfpathlineto{\pgfqpoint{1.558953in}{1.295131in}}%
\pgfpathclose%
\pgfusepath{fill}%
\end{pgfscope}%
\begin{pgfscope}%
\pgfpathrectangle{\pgfqpoint{0.050000in}{0.050000in}}{\pgfqpoint{2.081932in}{2.081932in}}%
\pgfusepath{clip}%
\pgfsetbuttcap%
\pgfsetroundjoin%
\definecolor{currentfill}{rgb}{0.296479,0.761561,0.424223}%
\pgfsetfillcolor{currentfill}%
\pgfsetlinewidth{0.000000pt}%
\definecolor{currentstroke}{rgb}{0.000000,0.000000,0.000000}%
\pgfsetstrokecolor{currentstroke}%
\pgfsetdash{}{0pt}%
\pgfpathmoveto{\pgfqpoint{1.344352in}{1.219436in}}%
\pgfpathlineto{\pgfqpoint{1.337707in}{1.211680in}}%
\pgfpathlineto{\pgfqpoint{1.331722in}{1.203735in}}%
\pgfpathlineto{\pgfqpoint{1.326420in}{1.195634in}}%
\pgfpathlineto{\pgfqpoint{1.321823in}{1.187411in}}%
\pgfpathlineto{\pgfqpoint{1.317949in}{1.179100in}}%
\pgfpathlineto{\pgfqpoint{1.314860in}{1.186403in}}%
\pgfpathlineto{\pgfqpoint{1.310992in}{1.193572in}}%
\pgfpathlineto{\pgfqpoint{1.306363in}{1.200576in}}%
\pgfpathlineto{\pgfqpoint{1.300993in}{1.207391in}}%
\pgfpathlineto{\pgfqpoint{1.294905in}{1.213988in}}%
\pgfpathlineto{\pgfqpoint{1.298315in}{1.222940in}}%
\pgfpathlineto{\pgfqpoint{1.302360in}{1.231928in}}%
\pgfpathlineto{\pgfqpoint{1.307025in}{1.240916in}}%
\pgfpathlineto{\pgfqpoint{1.312289in}{1.249865in}}%
\pgfpathlineto{\pgfqpoint{1.318133in}{1.258739in}}%
\pgfpathlineto{\pgfqpoint{1.325048in}{1.251310in}}%
\pgfpathlineto{\pgfqpoint{1.331153in}{1.243636in}}%
\pgfpathlineto{\pgfqpoint{1.336420in}{1.235744in}}%
\pgfpathlineto{\pgfqpoint{1.340826in}{1.227667in}}%
\pgfpathlineto{\pgfqpoint{1.344352in}{1.219436in}}%
\pgfpathclose%
\pgfusepath{fill}%
\end{pgfscope}%
\begin{pgfscope}%
\pgfpathrectangle{\pgfqpoint{0.050000in}{0.050000in}}{\pgfqpoint{2.081932in}{2.081932in}}%
\pgfusepath{clip}%
\pgfsetbuttcap%
\pgfsetroundjoin%
\definecolor{currentfill}{rgb}{0.278791,0.062145,0.386592}%
\pgfsetfillcolor{currentfill}%
\pgfsetlinewidth{0.000000pt}%
\definecolor{currentstroke}{rgb}{0.000000,0.000000,0.000000}%
\pgfsetstrokecolor{currentstroke}%
\pgfsetdash{}{0pt}%
\pgfpathmoveto{\pgfqpoint{1.374809in}{0.940360in}}%
\pgfpathlineto{\pgfqpoint{1.384351in}{0.933370in}}%
\pgfpathlineto{\pgfqpoint{1.394262in}{0.926729in}}%
\pgfpathlineto{\pgfqpoint{1.404502in}{0.920462in}}%
\pgfpathlineto{\pgfqpoint{1.415032in}{0.914593in}}%
\pgfpathlineto{\pgfqpoint{1.425811in}{0.909146in}}%
\pgfpathlineto{\pgfqpoint{1.433854in}{0.921030in}}%
\pgfpathlineto{\pgfqpoint{1.440599in}{0.933179in}}%
\pgfpathlineto{\pgfqpoint{1.446021in}{0.945542in}}%
\pgfpathlineto{\pgfqpoint{1.450105in}{0.958068in}}%
\pgfpathlineto{\pgfqpoint{1.452841in}{0.970705in}}%
\pgfpathlineto{\pgfqpoint{1.441196in}{0.973965in}}%
\pgfpathlineto{\pgfqpoint{1.429813in}{0.977696in}}%
\pgfpathlineto{\pgfqpoint{1.418738in}{0.981881in}}%
\pgfpathlineto{\pgfqpoint{1.408014in}{0.986506in}}%
\pgfpathlineto{\pgfqpoint{1.397683in}{0.991552in}}%
\pgfpathlineto{\pgfqpoint{1.395331in}{0.981033in}}%
\pgfpathlineto{\pgfqpoint{1.391854in}{0.970612in}}%
\pgfpathlineto{\pgfqpoint{1.387265in}{0.960331in}}%
\pgfpathlineto{\pgfqpoint{1.381576in}{0.950232in}}%
\pgfpathlineto{\pgfqpoint{1.374809in}{0.940360in}}%
\pgfpathclose%
\pgfusepath{fill}%
\end{pgfscope}%
\begin{pgfscope}%
\pgfpathrectangle{\pgfqpoint{0.050000in}{0.050000in}}{\pgfqpoint{2.081932in}{2.081932in}}%
\pgfusepath{clip}%
\pgfsetbuttcap%
\pgfsetroundjoin%
\definecolor{currentfill}{rgb}{0.278012,0.180367,0.486697}%
\pgfsetfillcolor{currentfill}%
\pgfsetlinewidth{0.000000pt}%
\definecolor{currentstroke}{rgb}{0.000000,0.000000,0.000000}%
\pgfsetstrokecolor{currentstroke}%
\pgfsetdash{}{0pt}%
\pgfpathmoveto{\pgfqpoint{1.333862in}{0.979527in}}%
\pgfpathlineto{\pgfqpoint{1.341039in}{0.971231in}}%
\pgfpathlineto{\pgfqpoint{1.348753in}{0.963135in}}%
\pgfpathlineto{\pgfqpoint{1.356975in}{0.955271in}}%
\pgfpathlineto{\pgfqpoint{1.365671in}{0.947669in}}%
\pgfpathlineto{\pgfqpoint{1.374809in}{0.940360in}}%
\pgfpathlineto{\pgfqpoint{1.381576in}{0.950232in}}%
\pgfpathlineto{\pgfqpoint{1.387265in}{0.960331in}}%
\pgfpathlineto{\pgfqpoint{1.391854in}{0.970612in}}%
\pgfpathlineto{\pgfqpoint{1.395331in}{0.981033in}}%
\pgfpathlineto{\pgfqpoint{1.397683in}{0.991552in}}%
\pgfpathlineto{\pgfqpoint{1.387787in}{0.996999in}}%
\pgfpathlineto{\pgfqpoint{1.378364in}{1.002827in}}%
\pgfpathlineto{\pgfqpoint{1.369453in}{1.009013in}}%
\pgfpathlineto{\pgfqpoint{1.361088in}{1.015532in}}%
\pgfpathlineto{\pgfqpoint{1.353302in}{1.022360in}}%
\pgfpathlineto{\pgfqpoint{1.351278in}{1.013552in}}%
\pgfpathlineto{\pgfqpoint{1.348310in}{1.004829in}}%
\pgfpathlineto{\pgfqpoint{1.344409in}{0.996227in}}%
\pgfpathlineto{\pgfqpoint{1.339587in}{0.987781in}}%
\pgfpathlineto{\pgfqpoint{1.333862in}{0.979527in}}%
\pgfpathclose%
\pgfusepath{fill}%
\end{pgfscope}%
\begin{pgfscope}%
\pgfpathrectangle{\pgfqpoint{0.050000in}{0.050000in}}{\pgfqpoint{2.081932in}{2.081932in}}%
\pgfusepath{clip}%
\pgfsetbuttcap%
\pgfsetroundjoin%
\definecolor{currentfill}{rgb}{0.267004,0.004874,0.329415}%
\pgfsetfillcolor{currentfill}%
\pgfsetlinewidth{0.000000pt}%
\definecolor{currentstroke}{rgb}{0.000000,0.000000,0.000000}%
\pgfsetstrokecolor{currentstroke}%
\pgfsetdash{}{0pt}%
\pgfpathmoveto{\pgfqpoint{1.425811in}{0.909146in}}%
\pgfpathlineto{\pgfqpoint{1.436796in}{0.904141in}}%
\pgfpathlineto{\pgfqpoint{1.447944in}{0.899600in}}%
\pgfpathlineto{\pgfqpoint{1.459214in}{0.895540in}}%
\pgfpathlineto{\pgfqpoint{1.470559in}{0.891978in}}%
\pgfpathlineto{\pgfqpoint{1.481936in}{0.888928in}}%
\pgfpathlineto{\pgfqpoint{1.491355in}{0.903017in}}%
\pgfpathlineto{\pgfqpoint{1.499231in}{0.917414in}}%
\pgfpathlineto{\pgfqpoint{1.505539in}{0.932056in}}%
\pgfpathlineto{\pgfqpoint{1.510260in}{0.946883in}}%
\pgfpathlineto{\pgfqpoint{1.513384in}{0.961832in}}%
\pgfpathlineto{\pgfqpoint{1.501125in}{0.962590in}}%
\pgfpathlineto{\pgfqpoint{1.488893in}{0.963862in}}%
\pgfpathlineto{\pgfqpoint{1.476736in}{0.965643in}}%
\pgfpathlineto{\pgfqpoint{1.464703in}{0.967928in}}%
\pgfpathlineto{\pgfqpoint{1.452841in}{0.970705in}}%
\pgfpathlineto{\pgfqpoint{1.450105in}{0.958068in}}%
\pgfpathlineto{\pgfqpoint{1.446021in}{0.945542in}}%
\pgfpathlineto{\pgfqpoint{1.440599in}{0.933179in}}%
\pgfpathlineto{\pgfqpoint{1.433854in}{0.921030in}}%
\pgfpathlineto{\pgfqpoint{1.425811in}{0.909146in}}%
\pgfpathclose%
\pgfusepath{fill}%
\end{pgfscope}%
\begin{pgfscope}%
\pgfpathrectangle{\pgfqpoint{0.050000in}{0.050000in}}{\pgfqpoint{2.081932in}{2.081932in}}%
\pgfusepath{clip}%
\pgfsetbuttcap%
\pgfsetroundjoin%
\definecolor{currentfill}{rgb}{0.876168,0.891125,0.095250}%
\pgfsetfillcolor{currentfill}%
\pgfsetlinewidth{0.000000pt}%
\definecolor{currentstroke}{rgb}{0.000000,0.000000,0.000000}%
\pgfsetstrokecolor{currentstroke}%
\pgfsetdash{}{0pt}%
\pgfpathmoveto{\pgfqpoint{0.660657in}{1.340253in}}%
\pgfpathlineto{\pgfqpoint{0.669783in}{1.342547in}}%
\pgfpathlineto{\pgfqpoint{0.679306in}{1.344303in}}%
\pgfpathlineto{\pgfqpoint{0.689186in}{1.345512in}}%
\pgfpathlineto{\pgfqpoint{0.699384in}{1.346168in}}%
\pgfpathlineto{\pgfqpoint{0.709858in}{1.346268in}}%
\pgfpathlineto{\pgfqpoint{0.698720in}{1.331200in}}%
\pgfpathlineto{\pgfqpoint{0.689217in}{1.315728in}}%
\pgfpathlineto{\pgfqpoint{0.681396in}{1.299910in}}%
\pgfpathlineto{\pgfqpoint{0.675298in}{1.283806in}}%
\pgfpathlineto{\pgfqpoint{0.670957in}{1.267478in}}%
\pgfpathlineto{\pgfqpoint{0.659370in}{1.265341in}}%
\pgfpathlineto{\pgfqpoint{0.648084in}{1.262694in}}%
\pgfpathlineto{\pgfqpoint{0.637144in}{1.259549in}}%
\pgfpathlineto{\pgfqpoint{0.626595in}{1.255919in}}%
\pgfpathlineto{\pgfqpoint{0.616480in}{1.251820in}}%
\pgfpathlineto{\pgfqpoint{0.621472in}{1.270163in}}%
\pgfpathlineto{\pgfqpoint{0.628429in}{1.288247in}}%
\pgfpathlineto{\pgfqpoint{0.637312in}{1.306000in}}%
\pgfpathlineto{\pgfqpoint{0.648072in}{1.323357in}}%
\pgfpathlineto{\pgfqpoint{0.660657in}{1.340253in}}%
\pgfpathclose%
\pgfusepath{fill}%
\end{pgfscope}%
\begin{pgfscope}%
\pgfpathrectangle{\pgfqpoint{0.050000in}{0.050000in}}{\pgfqpoint{2.081932in}{2.081932in}}%
\pgfusepath{clip}%
\pgfsetbuttcap%
\pgfsetroundjoin%
\definecolor{currentfill}{rgb}{0.993248,0.906157,0.143936}%
\pgfsetfillcolor{currentfill}%
\pgfsetlinewidth{0.000000pt}%
\definecolor{currentstroke}{rgb}{0.000000,0.000000,0.000000}%
\pgfsetstrokecolor{currentstroke}%
\pgfsetdash{}{0pt}%
\pgfpathmoveto{\pgfqpoint{1.499535in}{1.294188in}}%
\pgfpathlineto{\pgfqpoint{1.487438in}{1.292396in}}%
\pgfpathlineto{\pgfqpoint{1.475387in}{1.290081in}}%
\pgfpathlineto{\pgfqpoint{1.463431in}{1.287252in}}%
\pgfpathlineto{\pgfqpoint{1.451620in}{1.283921in}}%
\pgfpathlineto{\pgfqpoint{1.440002in}{1.280101in}}%
\pgfpathlineto{\pgfqpoint{1.434842in}{1.291736in}}%
\pgfpathlineto{\pgfqpoint{1.428435in}{1.303143in}}%
\pgfpathlineto{\pgfqpoint{1.420812in}{1.314278in}}%
\pgfpathlineto{\pgfqpoint{1.412008in}{1.325098in}}%
\pgfpathlineto{\pgfqpoint{1.402061in}{1.335563in}}%
\pgfpathlineto{\pgfqpoint{1.412232in}{1.341349in}}%
\pgfpathlineto{\pgfqpoint{1.422567in}{1.346681in}}%
\pgfpathlineto{\pgfqpoint{1.433023in}{1.351536in}}%
\pgfpathlineto{\pgfqpoint{1.443557in}{1.355895in}}%
\pgfpathlineto{\pgfqpoint{1.454124in}{1.359742in}}%
\pgfpathlineto{\pgfqpoint{1.465985in}{1.347385in}}%
\pgfpathlineto{\pgfqpoint{1.476502in}{1.334603in}}%
\pgfpathlineto{\pgfqpoint{1.485626in}{1.321442in}}%
\pgfpathlineto{\pgfqpoint{1.493317in}{1.307954in}}%
\pgfpathlineto{\pgfqpoint{1.499535in}{1.294188in}}%
\pgfpathclose%
\pgfusepath{fill}%
\end{pgfscope}%
\begin{pgfscope}%
\pgfpathrectangle{\pgfqpoint{0.050000in}{0.050000in}}{\pgfqpoint{2.081932in}{2.081932in}}%
\pgfusepath{clip}%
\pgfsetbuttcap%
\pgfsetroundjoin%
\definecolor{currentfill}{rgb}{0.162142,0.474838,0.558140}%
\pgfsetfillcolor{currentfill}%
\pgfsetlinewidth{0.000000pt}%
\definecolor{currentstroke}{rgb}{0.000000,0.000000,0.000000}%
\pgfsetstrokecolor{currentstroke}%
\pgfsetdash{}{0pt}%
\pgfpathmoveto{\pgfqpoint{0.925233in}{1.125208in}}%
\pgfpathlineto{\pgfqpoint{0.924557in}{1.117025in}}%
\pgfpathlineto{\pgfqpoint{0.923106in}{1.108942in}}%
\pgfpathlineto{\pgfqpoint{0.920887in}{1.100991in}}%
\pgfpathlineto{\pgfqpoint{0.917911in}{1.093202in}}%
\pgfpathlineto{\pgfqpoint{0.914189in}{1.085608in}}%
\pgfpathlineto{\pgfqpoint{0.913250in}{1.077905in}}%
\pgfpathlineto{\pgfqpoint{0.913139in}{1.070176in}}%
\pgfpathlineto{\pgfqpoint{0.913858in}{1.062451in}}%
\pgfpathlineto{\pgfqpoint{0.915408in}{1.054762in}}%
\pgfpathlineto{\pgfqpoint{0.917783in}{1.047139in}}%
\pgfpathlineto{\pgfqpoint{0.921456in}{1.055470in}}%
\pgfpathlineto{\pgfqpoint{0.924394in}{1.063854in}}%
\pgfpathlineto{\pgfqpoint{0.926583in}{1.072259in}}%
\pgfpathlineto{\pgfqpoint{0.928015in}{1.080651in}}%
\pgfpathlineto{\pgfqpoint{0.928681in}{1.088996in}}%
\pgfpathlineto{\pgfqpoint{0.926425in}{1.096170in}}%
\pgfpathlineto{\pgfqpoint{0.924949in}{1.103408in}}%
\pgfpathlineto{\pgfqpoint{0.924259in}{1.110680in}}%
\pgfpathlineto{\pgfqpoint{0.924354in}{1.117956in}}%
\pgfpathlineto{\pgfqpoint{0.925233in}{1.125208in}}%
\pgfpathclose%
\pgfusepath{fill}%
\end{pgfscope}%
\begin{pgfscope}%
\pgfpathrectangle{\pgfqpoint{0.050000in}{0.050000in}}{\pgfqpoint{2.081932in}{2.081932in}}%
\pgfusepath{clip}%
\pgfsetbuttcap%
\pgfsetroundjoin%
\definecolor{currentfill}{rgb}{0.206756,0.371758,0.553117}%
\pgfsetfillcolor{currentfill}%
\pgfsetlinewidth{0.000000pt}%
\definecolor{currentstroke}{rgb}{0.000000,0.000000,0.000000}%
\pgfsetstrokecolor{currentstroke}%
\pgfsetdash{}{0pt}%
\pgfpathmoveto{\pgfqpoint{0.572017in}{1.078061in}}%
\pgfpathlineto{\pgfqpoint{0.565511in}{1.085400in}}%
\pgfpathlineto{\pgfqpoint{0.559666in}{1.092962in}}%
\pgfpathlineto{\pgfqpoint{0.554507in}{1.100717in}}%
\pgfpathlineto{\pgfqpoint{0.550054in}{1.108634in}}%
\pgfpathlineto{\pgfqpoint{0.546327in}{1.116683in}}%
\pgfpathlineto{\pgfqpoint{0.542732in}{1.095203in}}%
\pgfpathlineto{\pgfqpoint{0.541443in}{1.073575in}}%
\pgfpathlineto{\pgfqpoint{0.542483in}{1.051885in}}%
\pgfpathlineto{\pgfqpoint{0.545867in}{1.030222in}}%
\pgfpathlineto{\pgfqpoint{0.551598in}{1.008674in}}%
\pgfpathlineto{\pgfqpoint{0.555340in}{1.001232in}}%
\pgfpathlineto{\pgfqpoint{0.559808in}{0.994063in}}%
\pgfpathlineto{\pgfqpoint{0.564985in}{0.987195in}}%
\pgfpathlineto{\pgfqpoint{0.570848in}{0.980655in}}%
\pgfpathlineto{\pgfqpoint{0.577373in}{0.974470in}}%
\pgfpathlineto{\pgfqpoint{0.571836in}{0.995127in}}%
\pgfpathlineto{\pgfqpoint{0.568541in}{1.015899in}}%
\pgfpathlineto{\pgfqpoint{0.567482in}{1.036702in}}%
\pgfpathlineto{\pgfqpoint{0.568648in}{1.057450in}}%
\pgfpathlineto{\pgfqpoint{0.572017in}{1.078061in}}%
\pgfpathclose%
\pgfusepath{fill}%
\end{pgfscope}%
\begin{pgfscope}%
\pgfpathrectangle{\pgfqpoint{0.050000in}{0.050000in}}{\pgfqpoint{2.081932in}{2.081932in}}%
\pgfusepath{clip}%
\pgfsetbuttcap%
\pgfsetroundjoin%
\definecolor{currentfill}{rgb}{0.606045,0.850733,0.236712}%
\pgfsetfillcolor{currentfill}%
\pgfsetlinewidth{0.000000pt}%
\definecolor{currentstroke}{rgb}{0.000000,0.000000,0.000000}%
\pgfsetstrokecolor{currentstroke}%
\pgfsetdash{}{0pt}%
\pgfpathmoveto{\pgfqpoint{1.386451in}{1.254299in}}%
\pgfpathlineto{\pgfqpoint{1.376958in}{1.247947in}}%
\pgfpathlineto{\pgfqpoint{1.367966in}{1.241257in}}%
\pgfpathlineto{\pgfqpoint{1.359512in}{1.234255in}}%
\pgfpathlineto{\pgfqpoint{1.351630in}{1.226971in}}%
\pgfpathlineto{\pgfqpoint{1.344352in}{1.219436in}}%
\pgfpathlineto{\pgfqpoint{1.340826in}{1.227667in}}%
\pgfpathlineto{\pgfqpoint{1.336420in}{1.235744in}}%
\pgfpathlineto{\pgfqpoint{1.331153in}{1.243636in}}%
\pgfpathlineto{\pgfqpoint{1.325048in}{1.251310in}}%
\pgfpathlineto{\pgfqpoint{1.318133in}{1.258739in}}%
\pgfpathlineto{\pgfqpoint{1.324531in}{1.267501in}}%
\pgfpathlineto{\pgfqpoint{1.331459in}{1.276114in}}%
\pgfpathlineto{\pgfqpoint{1.338886in}{1.284542in}}%
\pgfpathlineto{\pgfqpoint{1.346783in}{1.292751in}}%
\pgfpathlineto{\pgfqpoint{1.355116in}{1.300706in}}%
\pgfpathlineto{\pgfqpoint{1.363359in}{1.291941in}}%
\pgfpathlineto{\pgfqpoint{1.370643in}{1.282883in}}%
\pgfpathlineto{\pgfqpoint{1.376939in}{1.273565in}}%
\pgfpathlineto{\pgfqpoint{1.382216in}{1.264025in}}%
\pgfpathlineto{\pgfqpoint{1.386451in}{1.254299in}}%
\pgfpathclose%
\pgfusepath{fill}%
\end{pgfscope}%
\begin{pgfscope}%
\pgfpathrectangle{\pgfqpoint{0.050000in}{0.050000in}}{\pgfqpoint{2.081932in}{2.081932in}}%
\pgfusepath{clip}%
\pgfsetbuttcap%
\pgfsetroundjoin%
\definecolor{currentfill}{rgb}{0.124780,0.640461,0.527068}%
\pgfsetfillcolor{currentfill}%
\pgfsetlinewidth{0.000000pt}%
\definecolor{currentstroke}{rgb}{0.000000,0.000000,0.000000}%
\pgfsetstrokecolor{currentstroke}%
\pgfsetdash{}{0pt}%
\pgfpathmoveto{\pgfqpoint{1.702554in}{1.077640in}}%
\pgfpathlineto{\pgfqpoint{1.702472in}{1.085941in}}%
\pgfpathlineto{\pgfqpoint{1.701604in}{1.094274in}}%
\pgfpathlineto{\pgfqpoint{1.699949in}{1.102604in}}%
\pgfpathlineto{\pgfqpoint{1.697515in}{1.110898in}}%
\pgfpathlineto{\pgfqpoint{1.694310in}{1.119122in}}%
\pgfpathlineto{\pgfqpoint{1.696056in}{1.140591in}}%
\pgfpathlineto{\pgfqpoint{1.695469in}{1.162024in}}%
\pgfpathlineto{\pgfqpoint{1.692572in}{1.183337in}}%
\pgfpathlineto{\pgfqpoint{1.687394in}{1.204445in}}%
\pgfpathlineto{\pgfqpoint{1.679973in}{1.225266in}}%
\pgfpathlineto{\pgfqpoint{1.683057in}{1.217733in}}%
\pgfpathlineto{\pgfqpoint{1.685399in}{1.209992in}}%
\pgfpathlineto{\pgfqpoint{1.686990in}{1.202073in}}%
\pgfpathlineto{\pgfqpoint{1.687826in}{1.194009in}}%
\pgfpathlineto{\pgfqpoint{1.687904in}{1.185832in}}%
\pgfpathlineto{\pgfqpoint{1.695452in}{1.164613in}}%
\pgfpathlineto{\pgfqpoint{1.700725in}{1.143100in}}%
\pgfpathlineto{\pgfqpoint{1.703685in}{1.121376in}}%
\pgfpathlineto{\pgfqpoint{1.704302in}{1.099527in}}%
\pgfpathlineto{\pgfqpoint{1.702554in}{1.077640in}}%
\pgfpathclose%
\pgfusepath{fill}%
\end{pgfscope}%
\begin{pgfscope}%
\pgfpathrectangle{\pgfqpoint{0.050000in}{0.050000in}}{\pgfqpoint{2.081932in}{2.081932in}}%
\pgfusepath{clip}%
\pgfsetbuttcap%
\pgfsetroundjoin%
\definecolor{currentfill}{rgb}{0.296479,0.761561,0.424223}%
\pgfsetfillcolor{currentfill}%
\pgfsetlinewidth{0.000000pt}%
\definecolor{currentstroke}{rgb}{0.000000,0.000000,0.000000}%
\pgfsetstrokecolor{currentstroke}%
\pgfsetdash{}{0pt}%
\pgfpathmoveto{\pgfqpoint{0.908930in}{1.245950in}}%
\pgfpathlineto{\pgfqpoint{0.915114in}{1.237439in}}%
\pgfpathlineto{\pgfqpoint{0.920685in}{1.228815in}}%
\pgfpathlineto{\pgfqpoint{0.925620in}{1.220115in}}%
\pgfpathlineto{\pgfqpoint{0.929900in}{1.211375in}}%
\pgfpathlineto{\pgfqpoint{0.933508in}{1.202631in}}%
\pgfpathlineto{\pgfqpoint{0.928653in}{1.195680in}}%
\pgfpathlineto{\pgfqpoint{0.924554in}{1.188558in}}%
\pgfpathlineto{\pgfqpoint{0.921228in}{1.181292in}}%
\pgfpathlineto{\pgfqpoint{0.918692in}{1.173911in}}%
\pgfpathlineto{\pgfqpoint{0.916958in}{1.166443in}}%
\pgfpathlineto{\pgfqpoint{0.913010in}{1.174521in}}%
\pgfpathlineto{\pgfqpoint{0.908327in}{1.182465in}}%
\pgfpathlineto{\pgfqpoint{0.902925in}{1.190241in}}%
\pgfpathlineto{\pgfqpoint{0.896825in}{1.197819in}}%
\pgfpathlineto{\pgfqpoint{0.890053in}{1.205167in}}%
\pgfpathlineto{\pgfqpoint{0.892046in}{1.213586in}}%
\pgfpathlineto{\pgfqpoint{0.894946in}{1.221907in}}%
\pgfpathlineto{\pgfqpoint{0.898739in}{1.230095in}}%
\pgfpathlineto{\pgfqpoint{0.903408in}{1.238120in}}%
\pgfpathlineto{\pgfqpoint{0.908930in}{1.245950in}}%
\pgfpathclose%
\pgfusepath{fill}%
\end{pgfscope}%
\begin{pgfscope}%
\pgfpathrectangle{\pgfqpoint{0.050000in}{0.050000in}}{\pgfqpoint{2.081932in}{2.081932in}}%
\pgfusepath{clip}%
\pgfsetbuttcap%
\pgfsetroundjoin%
\definecolor{currentfill}{rgb}{0.855810,0.888601,0.097452}%
\pgfsetfillcolor{currentfill}%
\pgfsetlinewidth{0.000000pt}%
\definecolor{currentstroke}{rgb}{0.000000,0.000000,0.000000}%
\pgfsetstrokecolor{currentstroke}%
\pgfsetdash{}{0pt}%
\pgfpathmoveto{\pgfqpoint{1.440002in}{1.280101in}}%
\pgfpathlineto{\pgfqpoint{1.428625in}{1.275809in}}%
\pgfpathlineto{\pgfqpoint{1.417536in}{1.271062in}}%
\pgfpathlineto{\pgfqpoint{1.406782in}{1.265880in}}%
\pgfpathlineto{\pgfqpoint{1.396406in}{1.260285in}}%
\pgfpathlineto{\pgfqpoint{1.386451in}{1.254299in}}%
\pgfpathlineto{\pgfqpoint{1.382216in}{1.264025in}}%
\pgfpathlineto{\pgfqpoint{1.376939in}{1.273565in}}%
\pgfpathlineto{\pgfqpoint{1.370643in}{1.282883in}}%
\pgfpathlineto{\pgfqpoint{1.363359in}{1.291941in}}%
\pgfpathlineto{\pgfqpoint{1.355116in}{1.300706in}}%
\pgfpathlineto{\pgfqpoint{1.363851in}{1.308374in}}%
\pgfpathlineto{\pgfqpoint{1.372952in}{1.315724in}}%
\pgfpathlineto{\pgfqpoint{1.382380in}{1.322724in}}%
\pgfpathlineto{\pgfqpoint{1.392097in}{1.329346in}}%
\pgfpathlineto{\pgfqpoint{1.402061in}{1.335563in}}%
\pgfpathlineto{\pgfqpoint{1.412008in}{1.325098in}}%
\pgfpathlineto{\pgfqpoint{1.420812in}{1.314278in}}%
\pgfpathlineto{\pgfqpoint{1.428435in}{1.303143in}}%
\pgfpathlineto{\pgfqpoint{1.434842in}{1.291736in}}%
\pgfpathlineto{\pgfqpoint{1.440002in}{1.280101in}}%
\pgfpathclose%
\pgfusepath{fill}%
\end{pgfscope}%
\begin{pgfscope}%
\pgfpathrectangle{\pgfqpoint{0.050000in}{0.050000in}}{\pgfqpoint{2.081932in}{2.081932in}}%
\pgfusepath{clip}%
\pgfsetbuttcap%
\pgfsetroundjoin%
\definecolor{currentfill}{rgb}{0.227802,0.326594,0.546532}%
\pgfsetfillcolor{currentfill}%
\pgfsetlinewidth{0.000000pt}%
\definecolor{currentstroke}{rgb}{0.000000,0.000000,0.000000}%
\pgfsetstrokecolor{currentstroke}%
\pgfsetdash{}{0pt}%
\pgfpathmoveto{\pgfqpoint{1.306954in}{1.022857in}}%
\pgfpathlineto{\pgfqpoint{1.311063in}{1.014057in}}%
\pgfpathlineto{\pgfqpoint{1.315828in}{1.005289in}}%
\pgfpathlineto{\pgfqpoint{1.321231in}{0.996589in}}%
\pgfpathlineto{\pgfqpoint{1.327251in}{0.987990in}}%
\pgfpathlineto{\pgfqpoint{1.333862in}{0.979527in}}%
\pgfpathlineto{\pgfqpoint{1.339587in}{0.987781in}}%
\pgfpathlineto{\pgfqpoint{1.344409in}{0.996227in}}%
\pgfpathlineto{\pgfqpoint{1.348310in}{1.004829in}}%
\pgfpathlineto{\pgfqpoint{1.351278in}{1.013552in}}%
\pgfpathlineto{\pgfqpoint{1.353302in}{1.022360in}}%
\pgfpathlineto{\pgfqpoint{1.346129in}{1.029469in}}%
\pgfpathlineto{\pgfqpoint{1.339595in}{1.036832in}}%
\pgfpathlineto{\pgfqpoint{1.333729in}{1.044420in}}%
\pgfpathlineto{\pgfqpoint{1.328553in}{1.052203in}}%
\pgfpathlineto{\pgfqpoint{1.324090in}{1.060151in}}%
\pgfpathlineto{\pgfqpoint{1.322291in}{1.052478in}}%
\pgfpathlineto{\pgfqpoint{1.319668in}{1.044882in}}%
\pgfpathlineto{\pgfqpoint{1.316229in}{1.037392in}}%
\pgfpathlineto{\pgfqpoint{1.311986in}{1.030040in}}%
\pgfpathlineto{\pgfqpoint{1.306954in}{1.022857in}}%
\pgfpathclose%
\pgfusepath{fill}%
\end{pgfscope}%
\begin{pgfscope}%
\pgfpathrectangle{\pgfqpoint{0.050000in}{0.050000in}}{\pgfqpoint{2.081932in}{2.081932in}}%
\pgfusepath{clip}%
\pgfsetbuttcap%
\pgfsetroundjoin%
\definecolor{currentfill}{rgb}{0.268510,0.009605,0.335427}%
\pgfsetfillcolor{currentfill}%
\pgfsetlinewidth{0.000000pt}%
\definecolor{currentstroke}{rgb}{0.000000,0.000000,0.000000}%
\pgfsetstrokecolor{currentstroke}%
\pgfsetdash{}{0pt}%
\pgfpathmoveto{\pgfqpoint{1.481936in}{0.888928in}}%
\pgfpathlineto{\pgfqpoint{1.493300in}{0.886402in}}%
\pgfpathlineto{\pgfqpoint{1.504607in}{0.884412in}}%
\pgfpathlineto{\pgfqpoint{1.515811in}{0.882965in}}%
\pgfpathlineto{\pgfqpoint{1.526869in}{0.882069in}}%
\pgfpathlineto{\pgfqpoint{1.537736in}{0.881728in}}%
\pgfpathlineto{\pgfqpoint{1.548494in}{0.897994in}}%
\pgfpathlineto{\pgfqpoint{1.557463in}{0.914606in}}%
\pgfpathlineto{\pgfqpoint{1.564618in}{0.931493in}}%
\pgfpathlineto{\pgfqpoint{1.569939in}{0.948583in}}%
\pgfpathlineto{\pgfqpoint{1.573415in}{0.965806in}}%
\pgfpathlineto{\pgfqpoint{1.561737in}{0.963982in}}%
\pgfpathlineto{\pgfqpoint{1.549847in}{0.962670in}}%
\pgfpathlineto{\pgfqpoint{1.537793in}{0.961872in}}%
\pgfpathlineto{\pgfqpoint{1.525623in}{0.961593in}}%
\pgfpathlineto{\pgfqpoint{1.513384in}{0.961832in}}%
\pgfpathlineto{\pgfqpoint{1.510260in}{0.946883in}}%
\pgfpathlineto{\pgfqpoint{1.505539in}{0.932056in}}%
\pgfpathlineto{\pgfqpoint{1.499231in}{0.917414in}}%
\pgfpathlineto{\pgfqpoint{1.491355in}{0.903017in}}%
\pgfpathlineto{\pgfqpoint{1.481936in}{0.888928in}}%
\pgfpathclose%
\pgfusepath{fill}%
\end{pgfscope}%
\begin{pgfscope}%
\pgfpathrectangle{\pgfqpoint{0.050000in}{0.050000in}}{\pgfqpoint{2.081932in}{2.081932in}}%
\pgfusepath{clip}%
\pgfsetbuttcap%
\pgfsetroundjoin%
\definecolor{currentfill}{rgb}{0.120638,0.625828,0.533488}%
\pgfsetfillcolor{currentfill}%
\pgfsetlinewidth{0.000000pt}%
\definecolor{currentstroke}{rgb}{0.000000,0.000000,0.000000}%
\pgfsetstrokecolor{currentstroke}%
\pgfsetdash{}{0pt}%
\pgfpathmoveto{\pgfqpoint{1.321315in}{1.141579in}}%
\pgfpathlineto{\pgfqpoint{1.318110in}{1.133770in}}%
\pgfpathlineto{\pgfqpoint{1.315671in}{1.125799in}}%
\pgfpathlineto{\pgfqpoint{1.314006in}{1.117699in}}%
\pgfpathlineto{\pgfqpoint{1.313121in}{1.109503in}}%
\pgfpathlineto{\pgfqpoint{1.313017in}{1.101243in}}%
\pgfpathlineto{\pgfqpoint{1.313945in}{1.108508in}}%
\pgfpathlineto{\pgfqpoint{1.314086in}{1.115786in}}%
\pgfpathlineto{\pgfqpoint{1.313443in}{1.123048in}}%
\pgfpathlineto{\pgfqpoint{1.312020in}{1.130265in}}%
\pgfpathlineto{\pgfqpoint{1.309826in}{1.137408in}}%
\pgfpathlineto{\pgfqpoint{1.309927in}{1.145651in}}%
\pgfpathlineto{\pgfqpoint{1.310794in}{1.153975in}}%
\pgfpathlineto{\pgfqpoint{1.312424in}{1.162348in}}%
\pgfpathlineto{\pgfqpoint{1.314812in}{1.170734in}}%
\pgfpathlineto{\pgfqpoint{1.317949in}{1.179100in}}%
\pgfpathlineto{\pgfqpoint{1.320245in}{1.171690in}}%
\pgfpathlineto{\pgfqpoint{1.321736in}{1.164203in}}%
\pgfpathlineto{\pgfqpoint{1.322414in}{1.156668in}}%
\pgfpathlineto{\pgfqpoint{1.322274in}{1.149117in}}%
\pgfpathlineto{\pgfqpoint{1.321315in}{1.141579in}}%
\pgfpathclose%
\pgfusepath{fill}%
\end{pgfscope}%
\begin{pgfscope}%
\pgfpathrectangle{\pgfqpoint{0.050000in}{0.050000in}}{\pgfqpoint{2.081932in}{2.081932in}}%
\pgfusepath{clip}%
\pgfsetbuttcap%
\pgfsetroundjoin%
\definecolor{currentfill}{rgb}{0.278791,0.062145,0.386592}%
\pgfsetfillcolor{currentfill}%
\pgfsetlinewidth{0.000000pt}%
\definecolor{currentstroke}{rgb}{0.000000,0.000000,0.000000}%
\pgfsetstrokecolor{currentstroke}%
\pgfsetdash{}{0pt}%
\pgfpathmoveto{\pgfqpoint{0.845307in}{0.973709in}}%
\pgfpathlineto{\pgfqpoint{0.835127in}{0.967989in}}%
\pgfpathlineto{\pgfqpoint{0.824557in}{0.962664in}}%
\pgfpathlineto{\pgfqpoint{0.813640in}{0.957756in}}%
\pgfpathlineto{\pgfqpoint{0.802417in}{0.953285in}}%
\pgfpathlineto{\pgfqpoint{0.790933in}{0.949266in}}%
\pgfpathlineto{\pgfqpoint{0.795954in}{0.936849in}}%
\pgfpathlineto{\pgfqpoint{0.802302in}{0.924630in}}%
\pgfpathlineto{\pgfqpoint{0.809957in}{0.912661in}}%
\pgfpathlineto{\pgfqpoint{0.818892in}{0.900994in}}%
\pgfpathlineto{\pgfqpoint{0.829077in}{0.889676in}}%
\pgfpathlineto{\pgfqpoint{0.839296in}{0.895822in}}%
\pgfpathlineto{\pgfqpoint{0.849278in}{0.902373in}}%
\pgfpathlineto{\pgfqpoint{0.858984in}{0.909304in}}%
\pgfpathlineto{\pgfqpoint{0.868376in}{0.916588in}}%
\pgfpathlineto{\pgfqpoint{0.877418in}{0.924195in}}%
\pgfpathlineto{\pgfqpoint{0.868874in}{0.933590in}}%
\pgfpathlineto{\pgfqpoint{0.861367in}{0.943280in}}%
\pgfpathlineto{\pgfqpoint{0.854922in}{0.953224in}}%
\pgfpathlineto{\pgfqpoint{0.849563in}{0.963382in}}%
\pgfpathlineto{\pgfqpoint{0.845307in}{0.973709in}}%
\pgfpathclose%
\pgfusepath{fill}%
\end{pgfscope}%
\begin{pgfscope}%
\pgfpathrectangle{\pgfqpoint{0.050000in}{0.050000in}}{\pgfqpoint{2.081932in}{2.081932in}}%
\pgfusepath{clip}%
\pgfsetbuttcap%
\pgfsetroundjoin%
\definecolor{currentfill}{rgb}{0.278012,0.180367,0.486697}%
\pgfsetfillcolor{currentfill}%
\pgfsetlinewidth{0.000000pt}%
\definecolor{currentstroke}{rgb}{0.000000,0.000000,0.000000}%
\pgfsetstrokecolor{currentstroke}%
\pgfsetdash{}{0pt}%
\pgfpathmoveto{\pgfqpoint{0.889024in}{1.007421in}}%
\pgfpathlineto{\pgfqpoint{0.881358in}{1.000083in}}%
\pgfpathlineto{\pgfqpoint{0.873119in}{0.993016in}}%
\pgfpathlineto{\pgfqpoint{0.864341in}{0.986247in}}%
\pgfpathlineto{\pgfqpoint{0.855058in}{0.979803in}}%
\pgfpathlineto{\pgfqpoint{0.845307in}{0.973709in}}%
\pgfpathlineto{\pgfqpoint{0.849563in}{0.963382in}}%
\pgfpathlineto{\pgfqpoint{0.854922in}{0.953224in}}%
\pgfpathlineto{\pgfqpoint{0.861367in}{0.943280in}}%
\pgfpathlineto{\pgfqpoint{0.868874in}{0.933590in}}%
\pgfpathlineto{\pgfqpoint{0.877418in}{0.924195in}}%
\pgfpathlineto{\pgfqpoint{0.886075in}{0.932097in}}%
\pgfpathlineto{\pgfqpoint{0.894313in}{0.940263in}}%
\pgfpathlineto{\pgfqpoint{0.902100in}{0.948660in}}%
\pgfpathlineto{\pgfqpoint{0.909406in}{0.957257in}}%
\pgfpathlineto{\pgfqpoint{0.916202in}{0.966019in}}%
\pgfpathlineto{\pgfqpoint{0.908991in}{0.973869in}}%
\pgfpathlineto{\pgfqpoint{0.902646in}{0.981968in}}%
\pgfpathlineto{\pgfqpoint{0.897191in}{0.990283in}}%
\pgfpathlineto{\pgfqpoint{0.892645in}{0.998779in}}%
\pgfpathlineto{\pgfqpoint{0.889024in}{1.007421in}}%
\pgfpathclose%
\pgfusepath{fill}%
\end{pgfscope}%
\begin{pgfscope}%
\pgfpathrectangle{\pgfqpoint{0.050000in}{0.050000in}}{\pgfqpoint{2.081932in}{2.081932in}}%
\pgfusepath{clip}%
\pgfsetbuttcap%
\pgfsetroundjoin%
\definecolor{currentfill}{rgb}{0.993248,0.906157,0.143936}%
\pgfsetfillcolor{currentfill}%
\pgfsetlinewidth{0.000000pt}%
\definecolor{currentstroke}{rgb}{0.000000,0.000000,0.000000}%
\pgfsetstrokecolor{currentstroke}%
\pgfsetdash{}{0pt}%
\pgfpathmoveto{\pgfqpoint{0.709858in}{1.346268in}}%
\pgfpathlineto{\pgfqpoint{0.720566in}{1.345810in}}%
\pgfpathlineto{\pgfqpoint{0.731465in}{1.344796in}}%
\pgfpathlineto{\pgfqpoint{0.742509in}{1.343228in}}%
\pgfpathlineto{\pgfqpoint{0.753653in}{1.341113in}}%
\pgfpathlineto{\pgfqpoint{0.764852in}{1.338459in}}%
\pgfpathlineto{\pgfqpoint{0.755303in}{1.325407in}}%
\pgfpathlineto{\pgfqpoint{0.747177in}{1.312011in}}%
\pgfpathlineto{\pgfqpoint{0.740511in}{1.298322in}}%
\pgfpathlineto{\pgfqpoint{0.735341in}{1.284394in}}%
\pgfpathlineto{\pgfqpoint{0.731694in}{1.270280in}}%
\pgfpathlineto{\pgfqpoint{0.719339in}{1.270783in}}%
\pgfpathlineto{\pgfqpoint{0.707037in}{1.270753in}}%
\pgfpathlineto{\pgfqpoint{0.694839in}{1.270191in}}%
\pgfpathlineto{\pgfqpoint{0.682796in}{1.269097in}}%
\pgfpathlineto{\pgfqpoint{0.670957in}{1.267478in}}%
\pgfpathlineto{\pgfqpoint{0.675298in}{1.283806in}}%
\pgfpathlineto{\pgfqpoint{0.681396in}{1.299910in}}%
\pgfpathlineto{\pgfqpoint{0.689217in}{1.315728in}}%
\pgfpathlineto{\pgfqpoint{0.698720in}{1.331200in}}%
\pgfpathlineto{\pgfqpoint{0.709858in}{1.346268in}}%
\pgfpathclose%
\pgfusepath{fill}%
\end{pgfscope}%
\begin{pgfscope}%
\pgfpathrectangle{\pgfqpoint{0.050000in}{0.050000in}}{\pgfqpoint{2.081932in}{2.081932in}}%
\pgfusepath{clip}%
\pgfsetbuttcap%
\pgfsetroundjoin%
\definecolor{currentfill}{rgb}{0.606045,0.850733,0.236712}%
\pgfsetfillcolor{currentfill}%
\pgfsetlinewidth{0.000000pt}%
\definecolor{currentstroke}{rgb}{0.000000,0.000000,0.000000}%
\pgfsetstrokecolor{currentstroke}%
\pgfsetdash{}{0pt}%
\pgfpathmoveto{\pgfqpoint{0.869779in}{1.285615in}}%
\pgfpathlineto{\pgfqpoint{0.878603in}{1.278179in}}%
\pgfpathlineto{\pgfqpoint{0.886963in}{1.270462in}}%
\pgfpathlineto{\pgfqpoint{0.894825in}{1.262496in}}%
\pgfpathlineto{\pgfqpoint{0.902158in}{1.254314in}}%
\pgfpathlineto{\pgfqpoint{0.908930in}{1.245950in}}%
\pgfpathlineto{\pgfqpoint{0.903408in}{1.238120in}}%
\pgfpathlineto{\pgfqpoint{0.898739in}{1.230095in}}%
\pgfpathlineto{\pgfqpoint{0.894946in}{1.221907in}}%
\pgfpathlineto{\pgfqpoint{0.892046in}{1.213586in}}%
\pgfpathlineto{\pgfqpoint{0.890053in}{1.205167in}}%
\pgfpathlineto{\pgfqpoint{0.882634in}{1.212254in}}%
\pgfpathlineto{\pgfqpoint{0.874599in}{1.219052in}}%
\pgfpathlineto{\pgfqpoint{0.865980in}{1.225531in}}%
\pgfpathlineto{\pgfqpoint{0.856811in}{1.231666in}}%
\pgfpathlineto{\pgfqpoint{0.847130in}{1.237430in}}%
\pgfpathlineto{\pgfqpoint{0.849549in}{1.247384in}}%
\pgfpathlineto{\pgfqpoint{0.853043in}{1.257219in}}%
\pgfpathlineto{\pgfqpoint{0.857595in}{1.266893in}}%
\pgfpathlineto{\pgfqpoint{0.863182in}{1.276371in}}%
\pgfpathlineto{\pgfqpoint{0.869779in}{1.285615in}}%
\pgfpathclose%
\pgfusepath{fill}%
\end{pgfscope}%
\begin{pgfscope}%
\pgfpathrectangle{\pgfqpoint{0.050000in}{0.050000in}}{\pgfqpoint{2.081932in}{2.081932in}}%
\pgfusepath{clip}%
\pgfsetbuttcap%
\pgfsetroundjoin%
\definecolor{currentfill}{rgb}{0.267004,0.004874,0.329415}%
\pgfsetfillcolor{currentfill}%
\pgfsetlinewidth{0.000000pt}%
\definecolor{currentstroke}{rgb}{0.000000,0.000000,0.000000}%
\pgfsetstrokecolor{currentstroke}%
\pgfsetdash{}{0pt}%
\pgfpathmoveto{\pgfqpoint{0.790933in}{0.949266in}}%
\pgfpathlineto{\pgfqpoint{0.779233in}{0.945717in}}%
\pgfpathlineto{\pgfqpoint{0.767363in}{0.942651in}}%
\pgfpathlineto{\pgfqpoint{0.755369in}{0.940080in}}%
\pgfpathlineto{\pgfqpoint{0.743298in}{0.938015in}}%
\pgfpathlineto{\pgfqpoint{0.731197in}{0.936464in}}%
\pgfpathlineto{\pgfqpoint{0.737028in}{0.921760in}}%
\pgfpathlineto{\pgfqpoint{0.744433in}{0.907284in}}%
\pgfpathlineto{\pgfqpoint{0.753391in}{0.893097in}}%
\pgfpathlineto{\pgfqpoint{0.763870in}{0.879259in}}%
\pgfpathlineto{\pgfqpoint{0.775836in}{0.865830in}}%
\pgfpathlineto{\pgfqpoint{0.786632in}{0.869615in}}%
\pgfpathlineto{\pgfqpoint{0.797396in}{0.873910in}}%
\pgfpathlineto{\pgfqpoint{0.808086in}{0.878699in}}%
\pgfpathlineto{\pgfqpoint{0.818660in}{0.883961in}}%
\pgfpathlineto{\pgfqpoint{0.829077in}{0.889676in}}%
\pgfpathlineto{\pgfqpoint{0.818892in}{0.900994in}}%
\pgfpathlineto{\pgfqpoint{0.809957in}{0.912661in}}%
\pgfpathlineto{\pgfqpoint{0.802302in}{0.924630in}}%
\pgfpathlineto{\pgfqpoint{0.795954in}{0.936849in}}%
\pgfpathlineto{\pgfqpoint{0.790933in}{0.949266in}}%
\pgfpathclose%
\pgfusepath{fill}%
\end{pgfscope}%
\begin{pgfscope}%
\pgfpathrectangle{\pgfqpoint{0.050000in}{0.050000in}}{\pgfqpoint{2.081932in}{2.081932in}}%
\pgfusepath{clip}%
\pgfsetbuttcap%
\pgfsetroundjoin%
\definecolor{currentfill}{rgb}{0.150476,0.504369,0.557430}%
\pgfsetfillcolor{currentfill}%
\pgfsetlinewidth{0.000000pt}%
\definecolor{currentstroke}{rgb}{0.000000,0.000000,0.000000}%
\pgfsetstrokecolor{currentstroke}%
\pgfsetdash{}{0pt}%
\pgfpathmoveto{\pgfqpoint{0.546327in}{1.116683in}}%
\pgfpathlineto{\pgfqpoint{0.543341in}{1.124831in}}%
\pgfpathlineto{\pgfqpoint{0.541110in}{1.133046in}}%
\pgfpathlineto{\pgfqpoint{0.539643in}{1.141295in}}%
\pgfpathlineto{\pgfqpoint{0.538947in}{1.149546in}}%
\pgfpathlineto{\pgfqpoint{0.535286in}{1.127876in}}%
\pgfpathlineto{\pgfqpoint{0.533961in}{1.106055in}}%
\pgfpathlineto{\pgfqpoint{0.534994in}{1.084171in}}%
\pgfpathlineto{\pgfqpoint{0.538402in}{1.062311in}}%
\pgfpathlineto{\pgfqpoint{0.544189in}{1.040567in}}%
\pgfpathlineto{\pgfqpoint{0.544887in}{1.032338in}}%
\pgfpathlineto{\pgfqpoint{0.546360in}{1.024259in}}%
\pgfpathlineto{\pgfqpoint{0.548601in}{1.016360in}}%
\pgfpathlineto{\pgfqpoint{0.551598in}{1.008674in}}%
\pgfpathlineto{\pgfqpoint{0.545867in}{1.030222in}}%
\pgfpathlineto{\pgfqpoint{0.542483in}{1.051885in}}%
\pgfpathlineto{\pgfqpoint{0.541443in}{1.073575in}}%
\pgfpathlineto{\pgfqpoint{0.542732in}{1.095203in}}%
\pgfpathlineto{\pgfqpoint{0.546327in}{1.116683in}}%
\pgfpathclose%
\pgfusepath{fill}%
\end{pgfscope}%
\begin{pgfscope}%
\pgfpathrectangle{\pgfqpoint{0.050000in}{0.050000in}}{\pgfqpoint{2.081932in}{2.081932in}}%
\pgfusepath{clip}%
\pgfsetbuttcap%
\pgfsetroundjoin%
\definecolor{currentfill}{rgb}{0.227802,0.326594,0.546532}%
\pgfsetfillcolor{currentfill}%
\pgfsetlinewidth{0.000000pt}%
\definecolor{currentstroke}{rgb}{0.000000,0.000000,0.000000}%
\pgfsetstrokecolor{currentstroke}%
\pgfsetdash{}{0pt}%
\pgfpathmoveto{\pgfqpoint{0.917783in}{1.047139in}}%
\pgfpathlineto{\pgfqpoint{0.913390in}{1.038894in}}%
\pgfpathlineto{\pgfqpoint{0.908296in}{1.030769in}}%
\pgfpathlineto{\pgfqpoint{0.902521in}{1.022794in}}%
\pgfpathlineto{\pgfqpoint{0.896088in}{1.015001in}}%
\pgfpathlineto{\pgfqpoint{0.889024in}{1.007421in}}%
\pgfpathlineto{\pgfqpoint{0.892645in}{0.998779in}}%
\pgfpathlineto{\pgfqpoint{0.897191in}{0.990283in}}%
\pgfpathlineto{\pgfqpoint{0.902646in}{0.981968in}}%
\pgfpathlineto{\pgfqpoint{0.908991in}{0.973869in}}%
\pgfpathlineto{\pgfqpoint{0.916202in}{0.966019in}}%
\pgfpathlineto{\pgfqpoint{0.922461in}{0.974913in}}%
\pgfpathlineto{\pgfqpoint{0.928160in}{0.983904in}}%
\pgfpathlineto{\pgfqpoint{0.933275in}{0.992956in}}%
\pgfpathlineto{\pgfqpoint{0.937786in}{1.002036in}}%
\pgfpathlineto{\pgfqpoint{0.941675in}{1.011107in}}%
\pgfpathlineto{\pgfqpoint{0.935348in}{1.017935in}}%
\pgfpathlineto{\pgfqpoint{0.929776in}{1.024982in}}%
\pgfpathlineto{\pgfqpoint{0.924980in}{1.032218in}}%
\pgfpathlineto{\pgfqpoint{0.920978in}{1.039614in}}%
\pgfpathlineto{\pgfqpoint{0.917783in}{1.047139in}}%
\pgfpathclose%
\pgfusepath{fill}%
\end{pgfscope}%
\begin{pgfscope}%
\pgfpathrectangle{\pgfqpoint{0.050000in}{0.050000in}}{\pgfqpoint{2.081932in}{2.081932in}}%
\pgfusepath{clip}%
\pgfsetbuttcap%
\pgfsetroundjoin%
\definecolor{currentfill}{rgb}{0.993248,0.906157,0.143936}%
\pgfsetfillcolor{currentfill}%
\pgfsetlinewidth{0.000000pt}%
\definecolor{currentstroke}{rgb}{0.000000,0.000000,0.000000}%
\pgfsetstrokecolor{currentstroke}%
\pgfsetdash{}{0pt}%
\pgfpathmoveto{\pgfqpoint{0.764852in}{1.338459in}}%
\pgfpathlineto{\pgfqpoint{0.776060in}{1.335276in}}%
\pgfpathlineto{\pgfqpoint{0.787229in}{1.331577in}}%
\pgfpathlineto{\pgfqpoint{0.798314in}{1.327377in}}%
\pgfpathlineto{\pgfqpoint{0.809269in}{1.322692in}}%
\pgfpathlineto{\pgfqpoint{0.820048in}{1.317542in}}%
\pgfpathlineto{\pgfqpoint{0.812065in}{1.306496in}}%
\pgfpathlineto{\pgfqpoint{0.805288in}{1.295166in}}%
\pgfpathlineto{\pgfqpoint{0.799750in}{1.283595in}}%
\pgfpathlineto{\pgfqpoint{0.795476in}{1.271827in}}%
\pgfpathlineto{\pgfqpoint{0.792491in}{1.259909in}}%
\pgfpathlineto{\pgfqpoint{0.780631in}{1.263005in}}%
\pgfpathlineto{\pgfqpoint{0.768571in}{1.265600in}}%
\pgfpathlineto{\pgfqpoint{0.756361in}{1.267683in}}%
\pgfpathlineto{\pgfqpoint{0.744052in}{1.269245in}}%
\pgfpathlineto{\pgfqpoint{0.731694in}{1.270280in}}%
\pgfpathlineto{\pgfqpoint{0.735341in}{1.284394in}}%
\pgfpathlineto{\pgfqpoint{0.740511in}{1.298322in}}%
\pgfpathlineto{\pgfqpoint{0.747177in}{1.312011in}}%
\pgfpathlineto{\pgfqpoint{0.755303in}{1.325407in}}%
\pgfpathlineto{\pgfqpoint{0.764852in}{1.338459in}}%
\pgfpathclose%
\pgfusepath{fill}%
\end{pgfscope}%
\begin{pgfscope}%
\pgfpathrectangle{\pgfqpoint{0.050000in}{0.050000in}}{\pgfqpoint{2.081932in}{2.081932in}}%
\pgfusepath{clip}%
\pgfsetbuttcap%
\pgfsetroundjoin%
\definecolor{currentfill}{rgb}{0.120638,0.625828,0.533488}%
\pgfsetfillcolor{currentfill}%
\pgfsetlinewidth{0.000000pt}%
\definecolor{currentstroke}{rgb}{0.000000,0.000000,0.000000}%
\pgfsetstrokecolor{currentstroke}%
\pgfsetdash{}{0pt}%
\pgfpathmoveto{\pgfqpoint{0.916958in}{1.166443in}}%
\pgfpathlineto{\pgfqpoint{0.920153in}{1.158265in}}%
\pgfpathlineto{\pgfqpoint{0.922586in}{1.150018in}}%
\pgfpathlineto{\pgfqpoint{0.924247in}{1.141737in}}%
\pgfpathlineto{\pgfqpoint{0.925130in}{1.133456in}}%
\pgfpathlineto{\pgfqpoint{0.925233in}{1.125208in}}%
\pgfpathlineto{\pgfqpoint{0.924354in}{1.117956in}}%
\pgfpathlineto{\pgfqpoint{0.924259in}{1.110680in}}%
\pgfpathlineto{\pgfqpoint{0.924949in}{1.103408in}}%
\pgfpathlineto{\pgfqpoint{0.926425in}{1.096170in}}%
\pgfpathlineto{\pgfqpoint{0.928681in}{1.088996in}}%
\pgfpathlineto{\pgfqpoint{0.928580in}{1.097262in}}%
\pgfpathlineto{\pgfqpoint{0.927708in}{1.105415in}}%
\pgfpathlineto{\pgfqpoint{0.926070in}{1.113422in}}%
\pgfpathlineto{\pgfqpoint{0.923670in}{1.121251in}}%
\pgfpathlineto{\pgfqpoint{0.920516in}{1.128870in}}%
\pgfpathlineto{\pgfqpoint{0.918170in}{1.136315in}}%
\pgfpathlineto{\pgfqpoint{0.916639in}{1.143826in}}%
\pgfpathlineto{\pgfqpoint{0.915927in}{1.151371in}}%
\pgfpathlineto{\pgfqpoint{0.916034in}{1.158920in}}%
\pgfpathlineto{\pgfqpoint{0.916958in}{1.166443in}}%
\pgfpathclose%
\pgfusepath{fill}%
\end{pgfscope}%
\begin{pgfscope}%
\pgfpathrectangle{\pgfqpoint{0.050000in}{0.050000in}}{\pgfqpoint{2.081932in}{2.081932in}}%
\pgfusepath{clip}%
\pgfsetbuttcap%
\pgfsetroundjoin%
\definecolor{currentfill}{rgb}{0.855810,0.888601,0.097452}%
\pgfsetfillcolor{currentfill}%
\pgfsetlinewidth{0.000000pt}%
\definecolor{currentstroke}{rgb}{0.000000,0.000000,0.000000}%
\pgfsetstrokecolor{currentstroke}%
\pgfsetdash{}{0pt}%
\pgfpathmoveto{\pgfqpoint{0.820048in}{1.317542in}}%
\pgfpathlineto{\pgfqpoint{0.830607in}{1.311947in}}%
\pgfpathlineto{\pgfqpoint{0.840902in}{1.305932in}}%
\pgfpathlineto{\pgfqpoint{0.850889in}{1.299520in}}%
\pgfpathlineto{\pgfqpoint{0.860528in}{1.292738in}}%
\pgfpathlineto{\pgfqpoint{0.869779in}{1.285615in}}%
\pgfpathlineto{\pgfqpoint{0.863182in}{1.276371in}}%
\pgfpathlineto{\pgfqpoint{0.857595in}{1.266893in}}%
\pgfpathlineto{\pgfqpoint{0.853043in}{1.257219in}}%
\pgfpathlineto{\pgfqpoint{0.849549in}{1.247384in}}%
\pgfpathlineto{\pgfqpoint{0.847130in}{1.237430in}}%
\pgfpathlineto{\pgfqpoint{0.836977in}{1.242799in}}%
\pgfpathlineto{\pgfqpoint{0.826392in}{1.247751in}}%
\pgfpathlineto{\pgfqpoint{0.815419in}{1.252266in}}%
\pgfpathlineto{\pgfqpoint{0.804103in}{1.256324in}}%
\pgfpathlineto{\pgfqpoint{0.792491in}{1.259909in}}%
\pgfpathlineto{\pgfqpoint{0.795476in}{1.271827in}}%
\pgfpathlineto{\pgfqpoint{0.799750in}{1.283595in}}%
\pgfpathlineto{\pgfqpoint{0.805288in}{1.295166in}}%
\pgfpathlineto{\pgfqpoint{0.812065in}{1.306496in}}%
\pgfpathlineto{\pgfqpoint{0.820048in}{1.317542in}}%
\pgfpathclose%
\pgfusepath{fill}%
\end{pgfscope}%
\begin{pgfscope}%
\pgfpathrectangle{\pgfqpoint{0.050000in}{0.050000in}}{\pgfqpoint{2.081932in}{2.081932in}}%
\pgfusepath{clip}%
\pgfsetbuttcap%
\pgfsetroundjoin%
\definecolor{currentfill}{rgb}{0.327796,0.773980,0.406640}%
\pgfsetfillcolor{currentfill}%
\pgfsetlinewidth{0.000000pt}%
\definecolor{currentstroke}{rgb}{0.000000,0.000000,0.000000}%
\pgfsetstrokecolor{currentstroke}%
\pgfsetdash{}{0pt}%
\pgfpathmoveto{\pgfqpoint{1.694310in}{1.119122in}}%
\pgfpathlineto{\pgfqpoint{1.690346in}{1.127241in}}%
\pgfpathlineto{\pgfqpoint{1.685638in}{1.135224in}}%
\pgfpathlineto{\pgfqpoint{1.680205in}{1.143037in}}%
\pgfpathlineto{\pgfqpoint{1.674066in}{1.150648in}}%
\pgfpathlineto{\pgfqpoint{1.667248in}{1.158025in}}%
\pgfpathlineto{\pgfqpoint{1.668979in}{1.178389in}}%
\pgfpathlineto{\pgfqpoint{1.668488in}{1.198725in}}%
\pgfpathlineto{\pgfqpoint{1.665793in}{1.218951in}}%
\pgfpathlineto{\pgfqpoint{1.660923in}{1.238987in}}%
\pgfpathlineto{\pgfqpoint{1.653911in}{1.258755in}}%
\pgfpathlineto{\pgfqpoint{1.660481in}{1.252708in}}%
\pgfpathlineto{\pgfqpoint{1.666393in}{1.246308in}}%
\pgfpathlineto{\pgfqpoint{1.671625in}{1.239583in}}%
\pgfpathlineto{\pgfqpoint{1.676157in}{1.232559in}}%
\pgfpathlineto{\pgfqpoint{1.679973in}{1.225266in}}%
\pgfpathlineto{\pgfqpoint{1.687394in}{1.204445in}}%
\pgfpathlineto{\pgfqpoint{1.692572in}{1.183337in}}%
\pgfpathlineto{\pgfqpoint{1.695469in}{1.162024in}}%
\pgfpathlineto{\pgfqpoint{1.696056in}{1.140591in}}%
\pgfpathlineto{\pgfqpoint{1.694310in}{1.119122in}}%
\pgfpathclose%
\pgfusepath{fill}%
\end{pgfscope}%
\begin{pgfscope}%
\pgfpathrectangle{\pgfqpoint{0.050000in}{0.050000in}}{\pgfqpoint{2.081932in}{2.081932in}}%
\pgfusepath{clip}%
\pgfsetbuttcap%
\pgfsetroundjoin%
\definecolor{currentfill}{rgb}{0.162142,0.474838,0.558140}%
\pgfsetfillcolor{currentfill}%
\pgfsetlinewidth{0.000000pt}%
\definecolor{currentstroke}{rgb}{0.000000,0.000000,0.000000}%
\pgfsetstrokecolor{currentstroke}%
\pgfsetdash{}{0pt}%
\pgfpathmoveto{\pgfqpoint{1.296763in}{1.066149in}}%
\pgfpathlineto{\pgfqpoint{1.297387in}{1.057700in}}%
\pgfpathlineto{\pgfqpoint{1.298725in}{1.049112in}}%
\pgfpathlineto{\pgfqpoint{1.300772in}{1.040420in}}%
\pgfpathlineto{\pgfqpoint{1.303519in}{1.031657in}}%
\pgfpathlineto{\pgfqpoint{1.306954in}{1.022857in}}%
\pgfpathlineto{\pgfqpoint{1.311986in}{1.030040in}}%
\pgfpathlineto{\pgfqpoint{1.316229in}{1.037392in}}%
\pgfpathlineto{\pgfqpoint{1.319668in}{1.044882in}}%
\pgfpathlineto{\pgfqpoint{1.322291in}{1.052478in}}%
\pgfpathlineto{\pgfqpoint{1.324090in}{1.060151in}}%
\pgfpathlineto{\pgfqpoint{1.320359in}{1.068232in}}%
\pgfpathlineto{\pgfqpoint{1.317374in}{1.076415in}}%
\pgfpathlineto{\pgfqpoint{1.315149in}{1.084666in}}%
\pgfpathlineto{\pgfqpoint{1.313695in}{1.092953in}}%
\pgfpathlineto{\pgfqpoint{1.313017in}{1.101243in}}%
\pgfpathlineto{\pgfqpoint{1.311306in}{1.094021in}}%
\pgfpathlineto{\pgfqpoint{1.308815in}{1.086872in}}%
\pgfpathlineto{\pgfqpoint{1.305553in}{1.079824in}}%
\pgfpathlineto{\pgfqpoint{1.301531in}{1.072907in}}%
\pgfpathlineto{\pgfqpoint{1.296763in}{1.066149in}}%
\pgfpathclose%
\pgfusepath{fill}%
\end{pgfscope}%
\begin{pgfscope}%
\pgfpathrectangle{\pgfqpoint{0.050000in}{0.050000in}}{\pgfqpoint{2.081932in}{2.081932in}}%
\pgfusepath{clip}%
\pgfsetbuttcap%
\pgfsetroundjoin%
\definecolor{currentfill}{rgb}{0.282327,0.094955,0.417331}%
\pgfsetfillcolor{currentfill}%
\pgfsetlinewidth{0.000000pt}%
\definecolor{currentstroke}{rgb}{0.000000,0.000000,0.000000}%
\pgfsetstrokecolor{currentstroke}%
\pgfsetdash{}{0pt}%
\pgfpathmoveto{\pgfqpoint{1.537736in}{0.881728in}}%
\pgfpathlineto{\pgfqpoint{1.548370in}{0.881943in}}%
\pgfpathlineto{\pgfqpoint{1.558727in}{0.882716in}}%
\pgfpathlineto{\pgfqpoint{1.568766in}{0.884043in}}%
\pgfpathlineto{\pgfqpoint{1.578447in}{0.885921in}}%
\pgfpathlineto{\pgfqpoint{1.587730in}{0.888343in}}%
\pgfpathlineto{\pgfqpoint{1.599662in}{0.906534in}}%
\pgfpathlineto{\pgfqpoint{1.609584in}{0.925102in}}%
\pgfpathlineto{\pgfqpoint{1.617470in}{0.943969in}}%
\pgfpathlineto{\pgfqpoint{1.623300in}{0.963054in}}%
\pgfpathlineto{\pgfqpoint{1.627065in}{0.982278in}}%
\pgfpathlineto{\pgfqpoint{1.617113in}{0.978041in}}%
\pgfpathlineto{\pgfqpoint{1.606729in}{0.974263in}}%
\pgfpathlineto{\pgfqpoint{1.595956in}{0.970956in}}%
\pgfpathlineto{\pgfqpoint{1.584837in}{0.968133in}}%
\pgfpathlineto{\pgfqpoint{1.573415in}{0.965806in}}%
\pgfpathlineto{\pgfqpoint{1.569939in}{0.948583in}}%
\pgfpathlineto{\pgfqpoint{1.564618in}{0.931493in}}%
\pgfpathlineto{\pgfqpoint{1.557463in}{0.914606in}}%
\pgfpathlineto{\pgfqpoint{1.548494in}{0.897994in}}%
\pgfpathlineto{\pgfqpoint{1.537736in}{0.881728in}}%
\pgfpathclose%
\pgfusepath{fill}%
\end{pgfscope}%
\begin{pgfscope}%
\pgfpathrectangle{\pgfqpoint{0.050000in}{0.050000in}}{\pgfqpoint{2.081932in}{2.081932in}}%
\pgfusepath{clip}%
\pgfsetbuttcap%
\pgfsetroundjoin%
\definecolor{currentfill}{rgb}{0.296479,0.761561,0.424223}%
\pgfsetfillcolor{currentfill}%
\pgfsetlinewidth{0.000000pt}%
\definecolor{currentstroke}{rgb}{0.000000,0.000000,0.000000}%
\pgfsetstrokecolor{currentstroke}%
\pgfsetdash{}{0pt}%
\pgfpathmoveto{\pgfqpoint{1.348305in}{1.177117in}}%
\pgfpathlineto{\pgfqpoint{1.341509in}{1.170576in}}%
\pgfpathlineto{\pgfqpoint{1.335390in}{1.163724in}}%
\pgfpathlineto{\pgfqpoint{1.329971in}{1.156587in}}%
\pgfpathlineto{\pgfqpoint{1.325273in}{1.149195in}}%
\pgfpathlineto{\pgfqpoint{1.321315in}{1.141579in}}%
\pgfpathlineto{\pgfqpoint{1.322274in}{1.149117in}}%
\pgfpathlineto{\pgfqpoint{1.322414in}{1.156668in}}%
\pgfpathlineto{\pgfqpoint{1.321736in}{1.164203in}}%
\pgfpathlineto{\pgfqpoint{1.320245in}{1.171690in}}%
\pgfpathlineto{\pgfqpoint{1.317949in}{1.179100in}}%
\pgfpathlineto{\pgfqpoint{1.321823in}{1.187411in}}%
\pgfpathlineto{\pgfqpoint{1.326420in}{1.195634in}}%
\pgfpathlineto{\pgfqpoint{1.331722in}{1.203735in}}%
\pgfpathlineto{\pgfqpoint{1.337707in}{1.211680in}}%
\pgfpathlineto{\pgfqpoint{1.344352in}{1.219436in}}%
\pgfpathlineto{\pgfqpoint{1.346980in}{1.211082in}}%
\pgfpathlineto{\pgfqpoint{1.348697in}{1.202640in}}%
\pgfpathlineto{\pgfqpoint{1.349494in}{1.194142in}}%
\pgfpathlineto{\pgfqpoint{1.349364in}{1.185623in}}%
\pgfpathlineto{\pgfqpoint{1.348305in}{1.177117in}}%
\pgfpathclose%
\pgfusepath{fill}%
\end{pgfscope}%
\begin{pgfscope}%
\pgfpathrectangle{\pgfqpoint{0.050000in}{0.050000in}}{\pgfqpoint{2.081932in}{2.081932in}}%
\pgfusepath{clip}%
\pgfsetbuttcap%
\pgfsetroundjoin%
\definecolor{currentfill}{rgb}{0.268510,0.009605,0.335427}%
\pgfsetfillcolor{currentfill}%
\pgfsetlinewidth{0.000000pt}%
\definecolor{currentstroke}{rgb}{0.000000,0.000000,0.000000}%
\pgfsetstrokecolor{currentstroke}%
\pgfsetdash{}{0pt}%
\pgfpathmoveto{\pgfqpoint{0.731197in}{0.936464in}}%
\pgfpathlineto{\pgfqpoint{0.719114in}{0.935434in}}%
\pgfpathlineto{\pgfqpoint{0.707097in}{0.934928in}}%
\pgfpathlineto{\pgfqpoint{0.695193in}{0.934950in}}%
\pgfpathlineto{\pgfqpoint{0.683449in}{0.935500in}}%
\pgfpathlineto{\pgfqpoint{0.671910in}{0.936575in}}%
\pgfpathlineto{\pgfqpoint{0.678514in}{0.919620in}}%
\pgfpathlineto{\pgfqpoint{0.686939in}{0.902919in}}%
\pgfpathlineto{\pgfqpoint{0.697161in}{0.886542in}}%
\pgfpathlineto{\pgfqpoint{0.709148in}{0.870560in}}%
\pgfpathlineto{\pgfqpoint{0.722859in}{0.855043in}}%
\pgfpathlineto{\pgfqpoint{0.733180in}{0.856080in}}%
\pgfpathlineto{\pgfqpoint{0.743680in}{0.857686in}}%
\pgfpathlineto{\pgfqpoint{0.754318in}{0.859853in}}%
\pgfpathlineto{\pgfqpoint{0.765051in}{0.862572in}}%
\pgfpathlineto{\pgfqpoint{0.775836in}{0.865830in}}%
\pgfpathlineto{\pgfqpoint{0.763870in}{0.879259in}}%
\pgfpathlineto{\pgfqpoint{0.753391in}{0.893097in}}%
\pgfpathlineto{\pgfqpoint{0.744433in}{0.907284in}}%
\pgfpathlineto{\pgfqpoint{0.737028in}{0.921760in}}%
\pgfpathlineto{\pgfqpoint{0.731197in}{0.936464in}}%
\pgfpathclose%
\pgfusepath{fill}%
\end{pgfscope}%
\begin{pgfscope}%
\pgfpathrectangle{\pgfqpoint{0.050000in}{0.050000in}}{\pgfqpoint{2.081932in}{2.081932in}}%
\pgfusepath{clip}%
\pgfsetbuttcap%
\pgfsetroundjoin%
\definecolor{currentfill}{rgb}{0.162142,0.474838,0.558140}%
\pgfsetfillcolor{currentfill}%
\pgfsetlinewidth{0.000000pt}%
\definecolor{currentstroke}{rgb}{0.000000,0.000000,0.000000}%
\pgfsetstrokecolor{currentstroke}%
\pgfsetdash{}{0pt}%
\pgfpathmoveto{\pgfqpoint{0.928681in}{1.088996in}}%
\pgfpathlineto{\pgfqpoint{0.928015in}{1.080651in}}%
\pgfpathlineto{\pgfqpoint{0.926583in}{1.072259in}}%
\pgfpathlineto{\pgfqpoint{0.924394in}{1.063854in}}%
\pgfpathlineto{\pgfqpoint{0.921456in}{1.055470in}}%
\pgfpathlineto{\pgfqpoint{0.917783in}{1.047139in}}%
\pgfpathlineto{\pgfqpoint{0.920978in}{1.039614in}}%
\pgfpathlineto{\pgfqpoint{0.924980in}{1.032218in}}%
\pgfpathlineto{\pgfqpoint{0.929776in}{1.024982in}}%
\pgfpathlineto{\pgfqpoint{0.935348in}{1.017935in}}%
\pgfpathlineto{\pgfqpoint{0.941675in}{1.011107in}}%
\pgfpathlineto{\pgfqpoint{0.944926in}{1.020133in}}%
\pgfpathlineto{\pgfqpoint{0.947526in}{1.029081in}}%
\pgfpathlineto{\pgfqpoint{0.949464in}{1.037913in}}%
\pgfpathlineto{\pgfqpoint{0.950730in}{1.046596in}}%
\pgfpathlineto{\pgfqpoint{0.951320in}{1.055095in}}%
\pgfpathlineto{\pgfqpoint{0.945329in}{1.061518in}}%
\pgfpathlineto{\pgfqpoint{0.940051in}{1.068147in}}%
\pgfpathlineto{\pgfqpoint{0.935506in}{1.074956in}}%
\pgfpathlineto{\pgfqpoint{0.931712in}{1.081915in}}%
\pgfpathlineto{\pgfqpoint{0.928681in}{1.088996in}}%
\pgfpathclose%
\pgfusepath{fill}%
\end{pgfscope}%
\begin{pgfscope}%
\pgfpathrectangle{\pgfqpoint{0.050000in}{0.050000in}}{\pgfqpoint{2.081932in}{2.081932in}}%
\pgfusepath{clip}%
\pgfsetbuttcap%
\pgfsetroundjoin%
\definecolor{currentfill}{rgb}{0.124780,0.640461,0.527068}%
\pgfsetfillcolor{currentfill}%
\pgfsetlinewidth{0.000000pt}%
\definecolor{currentstroke}{rgb}{0.000000,0.000000,0.000000}%
\pgfsetstrokecolor{currentstroke}%
\pgfsetdash{}{0pt}%
\pgfpathmoveto{\pgfqpoint{0.538947in}{1.149546in}}%
\pgfpathlineto{\pgfqpoint{0.539028in}{1.157765in}}%
\pgfpathlineto{\pgfqpoint{0.539884in}{1.165918in}}%
\pgfpathlineto{\pgfqpoint{0.541515in}{1.173974in}}%
\pgfpathlineto{\pgfqpoint{0.543915in}{1.181899in}}%
\pgfpathlineto{\pgfqpoint{0.547075in}{1.189662in}}%
\pgfpathlineto{\pgfqpoint{0.543487in}{1.168402in}}%
\pgfpathlineto{\pgfqpoint{0.542202in}{1.146995in}}%
\pgfpathlineto{\pgfqpoint{0.543242in}{1.125527in}}%
\pgfpathlineto{\pgfqpoint{0.546624in}{1.104086in}}%
\pgfpathlineto{\pgfqpoint{0.552350in}{1.082759in}}%
\pgfpathlineto{\pgfqpoint{0.549177in}{1.074295in}}%
\pgfpathlineto{\pgfqpoint{0.546767in}{1.065809in}}%
\pgfpathlineto{\pgfqpoint{0.545130in}{1.057336in}}%
\pgfpathlineto{\pgfqpoint{0.544269in}{1.048911in}}%
\pgfpathlineto{\pgfqpoint{0.544189in}{1.040567in}}%
\pgfpathlineto{\pgfqpoint{0.538402in}{1.062311in}}%
\pgfpathlineto{\pgfqpoint{0.534994in}{1.084171in}}%
\pgfpathlineto{\pgfqpoint{0.533961in}{1.106055in}}%
\pgfpathlineto{\pgfqpoint{0.535286in}{1.127876in}}%
\pgfpathlineto{\pgfqpoint{0.538947in}{1.149546in}}%
\pgfpathclose%
\pgfusepath{fill}%
\end{pgfscope}%
\begin{pgfscope}%
\pgfpathrectangle{\pgfqpoint{0.050000in}{0.050000in}}{\pgfqpoint{2.081932in}{2.081932in}}%
\pgfusepath{clip}%
\pgfsetbuttcap%
\pgfsetroundjoin%
\definecolor{currentfill}{rgb}{0.278012,0.180367,0.486697}%
\pgfsetfillcolor{currentfill}%
\pgfsetlinewidth{0.000000pt}%
\definecolor{currentstroke}{rgb}{0.000000,0.000000,0.000000}%
\pgfsetstrokecolor{currentstroke}%
\pgfsetdash{}{0pt}%
\pgfpathmoveto{\pgfqpoint{1.292542in}{0.942307in}}%
\pgfpathlineto{\pgfqpoint{1.298365in}{0.932723in}}%
\pgfpathlineto{\pgfqpoint{1.304625in}{0.923243in}}%
\pgfpathlineto{\pgfqpoint{1.311300in}{0.913904in}}%
\pgfpathlineto{\pgfqpoint{1.318362in}{0.904743in}}%
\pgfpathlineto{\pgfqpoint{1.325785in}{0.895795in}}%
\pgfpathlineto{\pgfqpoint{1.337496in}{0.903938in}}%
\pgfpathlineto{\pgfqpoint{1.348294in}{0.912503in}}%
\pgfpathlineto{\pgfqpoint{1.358136in}{0.921455in}}%
\pgfpathlineto{\pgfqpoint{1.366986in}{0.930753in}}%
\pgfpathlineto{\pgfqpoint{1.374809in}{0.940360in}}%
\pgfpathlineto{\pgfqpoint{1.365671in}{0.947669in}}%
\pgfpathlineto{\pgfqpoint{1.356975in}{0.955271in}}%
\pgfpathlineto{\pgfqpoint{1.348753in}{0.963135in}}%
\pgfpathlineto{\pgfqpoint{1.341039in}{0.971231in}}%
\pgfpathlineto{\pgfqpoint{1.333862in}{0.979527in}}%
\pgfpathlineto{\pgfqpoint{1.327254in}{0.971498in}}%
\pgfpathlineto{\pgfqpoint{1.319787in}{0.963730in}}%
\pgfpathlineto{\pgfqpoint{1.311491in}{0.956254in}}%
\pgfpathlineto{\pgfqpoint{1.302398in}{0.949103in}}%
\pgfpathlineto{\pgfqpoint{1.292542in}{0.942307in}}%
\pgfpathclose%
\pgfusepath{fill}%
\end{pgfscope}%
\begin{pgfscope}%
\pgfpathrectangle{\pgfqpoint{0.050000in}{0.050000in}}{\pgfqpoint{2.081932in}{2.081932in}}%
\pgfusepath{clip}%
\pgfsetbuttcap%
\pgfsetroundjoin%
\definecolor{currentfill}{rgb}{0.636902,0.856542,0.216620}%
\pgfsetfillcolor{currentfill}%
\pgfsetlinewidth{0.000000pt}%
\definecolor{currentstroke}{rgb}{0.000000,0.000000,0.000000}%
\pgfsetstrokecolor{currentstroke}%
\pgfsetdash{}{0pt}%
\pgfpathmoveto{\pgfqpoint{1.667248in}{1.158025in}}%
\pgfpathlineto{\pgfqpoint{1.659777in}{1.165138in}}%
\pgfpathlineto{\pgfqpoint{1.651683in}{1.171958in}}%
\pgfpathlineto{\pgfqpoint{1.642998in}{1.178455in}}%
\pgfpathlineto{\pgfqpoint{1.633759in}{1.184604in}}%
\pgfpathlineto{\pgfqpoint{1.624002in}{1.190378in}}%
\pgfpathlineto{\pgfqpoint{1.625697in}{1.209064in}}%
\pgfpathlineto{\pgfqpoint{1.625344in}{1.227732in}}%
\pgfpathlineto{\pgfqpoint{1.622960in}{1.246306in}}%
\pgfpathlineto{\pgfqpoint{1.618568in}{1.264713in}}%
\pgfpathlineto{\pgfqpoint{1.612200in}{1.282881in}}%
\pgfpathlineto{\pgfqpoint{1.621617in}{1.278941in}}%
\pgfpathlineto{\pgfqpoint{1.630532in}{1.274539in}}%
\pgfpathlineto{\pgfqpoint{1.638908in}{1.269694in}}%
\pgfpathlineto{\pgfqpoint{1.646711in}{1.264425in}}%
\pgfpathlineto{\pgfqpoint{1.653911in}{1.258755in}}%
\pgfpathlineto{\pgfqpoint{1.660923in}{1.238987in}}%
\pgfpathlineto{\pgfqpoint{1.665793in}{1.218951in}}%
\pgfpathlineto{\pgfqpoint{1.668488in}{1.198725in}}%
\pgfpathlineto{\pgfqpoint{1.668979in}{1.178389in}}%
\pgfpathlineto{\pgfqpoint{1.667248in}{1.158025in}}%
\pgfpathclose%
\pgfusepath{fill}%
\end{pgfscope}%
\begin{pgfscope}%
\pgfpathrectangle{\pgfqpoint{0.050000in}{0.050000in}}{\pgfqpoint{2.081932in}{2.081932in}}%
\pgfusepath{clip}%
\pgfsetbuttcap%
\pgfsetroundjoin%
\definecolor{currentfill}{rgb}{0.296479,0.761561,0.424223}%
\pgfsetfillcolor{currentfill}%
\pgfsetlinewidth{0.000000pt}%
\definecolor{currentstroke}{rgb}{0.000000,0.000000,0.000000}%
\pgfsetstrokecolor{currentstroke}%
\pgfsetdash{}{0pt}%
\pgfpathmoveto{\pgfqpoint{0.890053in}{1.205167in}}%
\pgfpathlineto{\pgfqpoint{0.896825in}{1.197819in}}%
\pgfpathlineto{\pgfqpoint{0.902925in}{1.190241in}}%
\pgfpathlineto{\pgfqpoint{0.908327in}{1.182465in}}%
\pgfpathlineto{\pgfqpoint{0.913010in}{1.174521in}}%
\pgfpathlineto{\pgfqpoint{0.916958in}{1.166443in}}%
\pgfpathlineto{\pgfqpoint{0.916034in}{1.158920in}}%
\pgfpathlineto{\pgfqpoint{0.915927in}{1.151371in}}%
\pgfpathlineto{\pgfqpoint{0.916639in}{1.143826in}}%
\pgfpathlineto{\pgfqpoint{0.918170in}{1.136315in}}%
\pgfpathlineto{\pgfqpoint{0.920516in}{1.128870in}}%
\pgfpathlineto{\pgfqpoint{0.916620in}{1.136249in}}%
\pgfpathlineto{\pgfqpoint{0.911996in}{1.143357in}}%
\pgfpathlineto{\pgfqpoint{0.906662in}{1.150165in}}%
\pgfpathlineto{\pgfqpoint{0.900637in}{1.156645in}}%
\pgfpathlineto{\pgfqpoint{0.893946in}{1.162771in}}%
\pgfpathlineto{\pgfqpoint{0.891314in}{1.171176in}}%
\pgfpathlineto{\pgfqpoint{0.889607in}{1.179652in}}%
\pgfpathlineto{\pgfqpoint{0.888828in}{1.188166in}}%
\pgfpathlineto{\pgfqpoint{0.888978in}{1.196682in}}%
\pgfpathlineto{\pgfqpoint{0.890053in}{1.205167in}}%
\pgfpathclose%
\pgfusepath{fill}%
\end{pgfscope}%
\begin{pgfscope}%
\pgfpathrectangle{\pgfqpoint{0.050000in}{0.050000in}}{\pgfqpoint{2.081932in}{2.081932in}}%
\pgfusepath{clip}%
\pgfsetbuttcap%
\pgfsetroundjoin%
\definecolor{currentfill}{rgb}{0.227802,0.326594,0.546532}%
\pgfsetfillcolor{currentfill}%
\pgfsetlinewidth{0.000000pt}%
\definecolor{currentstroke}{rgb}{0.000000,0.000000,0.000000}%
\pgfsetstrokecolor{currentstroke}%
\pgfsetdash{}{0pt}%
\pgfpathmoveto{\pgfqpoint{1.270728in}{0.990490in}}%
\pgfpathlineto{\pgfqpoint{1.274057in}{0.980943in}}%
\pgfpathlineto{\pgfqpoint{1.277920in}{0.971314in}}%
\pgfpathlineto{\pgfqpoint{1.282300in}{0.961640in}}%
\pgfpathlineto{\pgfqpoint{1.287180in}{0.951959in}}%
\pgfpathlineto{\pgfqpoint{1.292542in}{0.942307in}}%
\pgfpathlineto{\pgfqpoint{1.302398in}{0.949103in}}%
\pgfpathlineto{\pgfqpoint{1.311491in}{0.956254in}}%
\pgfpathlineto{\pgfqpoint{1.319787in}{0.963730in}}%
\pgfpathlineto{\pgfqpoint{1.327254in}{0.971498in}}%
\pgfpathlineto{\pgfqpoint{1.333862in}{0.979527in}}%
\pgfpathlineto{\pgfqpoint{1.327251in}{0.987990in}}%
\pgfpathlineto{\pgfqpoint{1.321231in}{0.996589in}}%
\pgfpathlineto{\pgfqpoint{1.315828in}{1.005289in}}%
\pgfpathlineto{\pgfqpoint{1.311063in}{1.014057in}}%
\pgfpathlineto{\pgfqpoint{1.306954in}{1.022857in}}%
\pgfpathlineto{\pgfqpoint{1.301152in}{1.015873in}}%
\pgfpathlineto{\pgfqpoint{1.294601in}{1.009116in}}%
\pgfpathlineto{\pgfqpoint{1.287328in}{1.002615in}}%
\pgfpathlineto{\pgfqpoint{1.279360in}{0.996398in}}%
\pgfpathlineto{\pgfqpoint{1.270728in}{0.990490in}}%
\pgfpathclose%
\pgfusepath{fill}%
\end{pgfscope}%
\begin{pgfscope}%
\pgfpathrectangle{\pgfqpoint{0.050000in}{0.050000in}}{\pgfqpoint{2.081932in}{2.081932in}}%
\pgfusepath{clip}%
\pgfsetbuttcap%
\pgfsetroundjoin%
\definecolor{currentfill}{rgb}{0.278791,0.062145,0.386592}%
\pgfsetfillcolor{currentfill}%
\pgfsetlinewidth{0.000000pt}%
\definecolor{currentstroke}{rgb}{0.000000,0.000000,0.000000}%
\pgfsetstrokecolor{currentstroke}%
\pgfsetdash{}{0pt}%
\pgfpathmoveto{\pgfqpoint{1.325785in}{0.895795in}}%
\pgfpathlineto{\pgfqpoint{1.333541in}{0.887094in}}%
\pgfpathlineto{\pgfqpoint{1.341598in}{0.878675in}}%
\pgfpathlineto{\pgfqpoint{1.349928in}{0.870571in}}%
\pgfpathlineto{\pgfqpoint{1.358496in}{0.862814in}}%
\pgfpathlineto{\pgfqpoint{1.367271in}{0.855433in}}%
\pgfpathlineto{\pgfqpoint{1.381282in}{0.865256in}}%
\pgfpathlineto{\pgfqpoint{1.394189in}{0.875585in}}%
\pgfpathlineto{\pgfqpoint{1.405941in}{0.886374in}}%
\pgfpathlineto{\pgfqpoint{1.416495in}{0.897577in}}%
\pgfpathlineto{\pgfqpoint{1.425811in}{0.909146in}}%
\pgfpathlineto{\pgfqpoint{1.415032in}{0.914593in}}%
\pgfpathlineto{\pgfqpoint{1.404502in}{0.920462in}}%
\pgfpathlineto{\pgfqpoint{1.394262in}{0.926729in}}%
\pgfpathlineto{\pgfqpoint{1.384351in}{0.933370in}}%
\pgfpathlineto{\pgfqpoint{1.374809in}{0.940360in}}%
\pgfpathlineto{\pgfqpoint{1.366986in}{0.930753in}}%
\pgfpathlineto{\pgfqpoint{1.358136in}{0.921455in}}%
\pgfpathlineto{\pgfqpoint{1.348294in}{0.912503in}}%
\pgfpathlineto{\pgfqpoint{1.337496in}{0.903938in}}%
\pgfpathlineto{\pgfqpoint{1.325785in}{0.895795in}}%
\pgfpathclose%
\pgfusepath{fill}%
\end{pgfscope}%
\begin{pgfscope}%
\pgfpathrectangle{\pgfqpoint{0.050000in}{0.050000in}}{\pgfqpoint{2.081932in}{2.081932in}}%
\pgfusepath{clip}%
\pgfsetbuttcap%
\pgfsetroundjoin%
\definecolor{currentfill}{rgb}{0.120638,0.625828,0.533488}%
\pgfsetfillcolor{currentfill}%
\pgfsetlinewidth{0.000000pt}%
\definecolor{currentstroke}{rgb}{0.000000,0.000000,0.000000}%
\pgfsetstrokecolor{currentstroke}%
\pgfsetdash{}{0pt}%
\pgfpathmoveto{\pgfqpoint{1.304399in}{1.105156in}}%
\pgfpathlineto{\pgfqpoint{1.301450in}{1.097893in}}%
\pgfpathlineto{\pgfqpoint{1.299205in}{1.090329in}}%
\pgfpathlineto{\pgfqpoint{1.297673in}{1.082496in}}%
\pgfpathlineto{\pgfqpoint{1.296859in}{1.074425in}}%
\pgfpathlineto{\pgfqpoint{1.296763in}{1.066149in}}%
\pgfpathlineto{\pgfqpoint{1.301531in}{1.072907in}}%
\pgfpathlineto{\pgfqpoint{1.305553in}{1.079824in}}%
\pgfpathlineto{\pgfqpoint{1.308815in}{1.086872in}}%
\pgfpathlineto{\pgfqpoint{1.311306in}{1.094021in}}%
\pgfpathlineto{\pgfqpoint{1.313017in}{1.101243in}}%
\pgfpathlineto{\pgfqpoint{1.313121in}{1.109503in}}%
\pgfpathlineto{\pgfqpoint{1.314006in}{1.117699in}}%
\pgfpathlineto{\pgfqpoint{1.315671in}{1.125799in}}%
\pgfpathlineto{\pgfqpoint{1.318110in}{1.133770in}}%
\pgfpathlineto{\pgfqpoint{1.321315in}{1.141579in}}%
\pgfpathlineto{\pgfqpoint{1.319538in}{1.134085in}}%
\pgfpathlineto{\pgfqpoint{1.316947in}{1.126666in}}%
\pgfpathlineto{\pgfqpoint{1.313553in}{1.119351in}}%
\pgfpathlineto{\pgfqpoint{1.309365in}{1.112171in}}%
\pgfpathlineto{\pgfqpoint{1.304399in}{1.105156in}}%
\pgfpathclose%
\pgfusepath{fill}%
\end{pgfscope}%
\begin{pgfscope}%
\pgfpathrectangle{\pgfqpoint{0.050000in}{0.050000in}}{\pgfqpoint{2.081932in}{2.081932in}}%
\pgfusepath{clip}%
\pgfsetbuttcap%
\pgfsetroundjoin%
\definecolor{currentfill}{rgb}{0.267968,0.223549,0.512008}%
\pgfsetfillcolor{currentfill}%
\pgfsetlinewidth{0.000000pt}%
\definecolor{currentstroke}{rgb}{0.000000,0.000000,0.000000}%
\pgfsetstrokecolor{currentstroke}%
\pgfsetdash{}{0pt}%
\pgfpathmoveto{\pgfqpoint{1.587730in}{0.888343in}}%
\pgfpathlineto{\pgfqpoint{1.596578in}{0.891301in}}%
\pgfpathlineto{\pgfqpoint{1.604954in}{0.894783in}}%
\pgfpathlineto{\pgfqpoint{1.612823in}{0.898777in}}%
\pgfpathlineto{\pgfqpoint{1.620154in}{0.903268in}}%
\pgfpathlineto{\pgfqpoint{1.626916in}{0.908239in}}%
\pgfpathlineto{\pgfqpoint{1.639751in}{0.927900in}}%
\pgfpathlineto{\pgfqpoint{1.650404in}{0.947963in}}%
\pgfpathlineto{\pgfqpoint{1.658845in}{0.968341in}}%
\pgfpathlineto{\pgfqpoint{1.665055in}{0.988946in}}%
\pgfpathlineto{\pgfqpoint{1.669026in}{1.009693in}}%
\pgfpathlineto{\pgfqpoint{1.661791in}{1.003447in}}%
\pgfpathlineto{\pgfqpoint{1.653945in}{0.997560in}}%
\pgfpathlineto{\pgfqpoint{1.645518in}{0.992056in}}%
\pgfpathlineto{\pgfqpoint{1.636546in}{0.986955in}}%
\pgfpathlineto{\pgfqpoint{1.627065in}{0.982278in}}%
\pgfpathlineto{\pgfqpoint{1.623300in}{0.963054in}}%
\pgfpathlineto{\pgfqpoint{1.617470in}{0.943969in}}%
\pgfpathlineto{\pgfqpoint{1.609584in}{0.925102in}}%
\pgfpathlineto{\pgfqpoint{1.599662in}{0.906534in}}%
\pgfpathlineto{\pgfqpoint{1.587730in}{0.888343in}}%
\pgfpathclose%
\pgfusepath{fill}%
\end{pgfscope}%
\begin{pgfscope}%
\pgfpathrectangle{\pgfqpoint{0.050000in}{0.050000in}}{\pgfqpoint{2.081932in}{2.081932in}}%
\pgfusepath{clip}%
\pgfsetbuttcap%
\pgfsetroundjoin%
\definecolor{currentfill}{rgb}{0.606045,0.850733,0.236712}%
\pgfsetfillcolor{currentfill}%
\pgfsetlinewidth{0.000000pt}%
\definecolor{currentstroke}{rgb}{0.000000,0.000000,0.000000}%
\pgfsetstrokecolor{currentstroke}%
\pgfsetdash{}{0pt}%
\pgfpathmoveto{\pgfqpoint{1.391411in}{1.204236in}}%
\pgfpathlineto{\pgfqpoint{1.381684in}{1.199635in}}%
\pgfpathlineto{\pgfqpoint{1.372474in}{1.194601in}}%
\pgfpathlineto{\pgfqpoint{1.363818in}{1.189154in}}%
\pgfpathlineto{\pgfqpoint{1.355751in}{1.183318in}}%
\pgfpathlineto{\pgfqpoint{1.348305in}{1.177117in}}%
\pgfpathlineto{\pgfqpoint{1.349364in}{1.185623in}}%
\pgfpathlineto{\pgfqpoint{1.349494in}{1.194142in}}%
\pgfpathlineto{\pgfqpoint{1.348697in}{1.202640in}}%
\pgfpathlineto{\pgfqpoint{1.346980in}{1.211082in}}%
\pgfpathlineto{\pgfqpoint{1.344352in}{1.219436in}}%
\pgfpathlineto{\pgfqpoint{1.351630in}{1.226971in}}%
\pgfpathlineto{\pgfqpoint{1.359512in}{1.234255in}}%
\pgfpathlineto{\pgfqpoint{1.367966in}{1.241257in}}%
\pgfpathlineto{\pgfqpoint{1.376958in}{1.247947in}}%
\pgfpathlineto{\pgfqpoint{1.386451in}{1.254299in}}%
\pgfpathlineto{\pgfqpoint{1.389623in}{1.244424in}}%
\pgfpathlineto{\pgfqpoint{1.391715in}{1.234441in}}%
\pgfpathlineto{\pgfqpoint{1.392715in}{1.224388in}}%
\pgfpathlineto{\pgfqpoint{1.392615in}{1.214306in}}%
\pgfpathlineto{\pgfqpoint{1.391411in}{1.204236in}}%
\pgfpathclose%
\pgfusepath{fill}%
\end{pgfscope}%
\begin{pgfscope}%
\pgfpathrectangle{\pgfqpoint{0.050000in}{0.050000in}}{\pgfqpoint{2.081932in}{2.081932in}}%
\pgfusepath{clip}%
\pgfsetbuttcap%
\pgfsetroundjoin%
\definecolor{currentfill}{rgb}{0.278012,0.180367,0.486697}%
\pgfsetfillcolor{currentfill}%
\pgfsetlinewidth{0.000000pt}%
\definecolor{currentstroke}{rgb}{0.000000,0.000000,0.000000}%
\pgfsetstrokecolor{currentstroke}%
\pgfsetdash{}{0pt}%
\pgfpathmoveto{\pgfqpoint{0.916202in}{0.966019in}}%
\pgfpathlineto{\pgfqpoint{0.909406in}{0.957257in}}%
\pgfpathlineto{\pgfqpoint{0.902100in}{0.948660in}}%
\pgfpathlineto{\pgfqpoint{0.894313in}{0.940263in}}%
\pgfpathlineto{\pgfqpoint{0.886075in}{0.932097in}}%
\pgfpathlineto{\pgfqpoint{0.877418in}{0.924195in}}%
\pgfpathlineto{\pgfqpoint{0.886965in}{0.915136in}}%
\pgfpathlineto{\pgfqpoint{0.897479in}{0.906451in}}%
\pgfpathlineto{\pgfqpoint{0.908919in}{0.898178in}}%
\pgfpathlineto{\pgfqpoint{0.921241in}{0.890353in}}%
\pgfpathlineto{\pgfqpoint{0.934394in}{0.883010in}}%
\pgfpathlineto{\pgfqpoint{0.941041in}{0.892431in}}%
\pgfpathlineto{\pgfqpoint{0.947366in}{0.902042in}}%
\pgfpathlineto{\pgfqpoint{0.953342in}{0.911806in}}%
\pgfpathlineto{\pgfqpoint{0.958947in}{0.921685in}}%
\pgfpathlineto{\pgfqpoint{0.964159in}{0.931641in}}%
\pgfpathlineto{\pgfqpoint{0.953102in}{0.937767in}}%
\pgfpathlineto{\pgfqpoint{0.942737in}{0.944296in}}%
\pgfpathlineto{\pgfqpoint{0.933108in}{0.951201in}}%
\pgfpathlineto{\pgfqpoint{0.924251in}{0.958453in}}%
\pgfpathlineto{\pgfqpoint{0.916202in}{0.966019in}}%
\pgfpathclose%
\pgfusepath{fill}%
\end{pgfscope}%
\begin{pgfscope}%
\pgfpathrectangle{\pgfqpoint{0.050000in}{0.050000in}}{\pgfqpoint{2.081932in}{2.081932in}}%
\pgfusepath{clip}%
\pgfsetbuttcap%
\pgfsetroundjoin%
\definecolor{currentfill}{rgb}{0.282327,0.094955,0.417331}%
\pgfsetfillcolor{currentfill}%
\pgfsetlinewidth{0.000000pt}%
\definecolor{currentstroke}{rgb}{0.000000,0.000000,0.000000}%
\pgfsetstrokecolor{currentstroke}%
\pgfsetdash{}{0pt}%
\pgfpathmoveto{\pgfqpoint{0.671910in}{0.936575in}}%
\pgfpathlineto{\pgfqpoint{0.660624in}{0.938173in}}%
\pgfpathlineto{\pgfqpoint{0.649635in}{0.940288in}}%
\pgfpathlineto{\pgfqpoint{0.638986in}{0.942912in}}%
\pgfpathlineto{\pgfqpoint{0.628721in}{0.946035in}}%
\pgfpathlineto{\pgfqpoint{0.618880in}{0.949646in}}%
\pgfpathlineto{\pgfqpoint{0.626148in}{0.930705in}}%
\pgfpathlineto{\pgfqpoint{0.635460in}{0.912039in}}%
\pgfpathlineto{\pgfqpoint{0.646789in}{0.893728in}}%
\pgfpathlineto{\pgfqpoint{0.660103in}{0.875849in}}%
\pgfpathlineto{\pgfqpoint{0.675357in}{0.858483in}}%
\pgfpathlineto{\pgfqpoint{0.684180in}{0.856646in}}%
\pgfpathlineto{\pgfqpoint{0.693380in}{0.855381in}}%
\pgfpathlineto{\pgfqpoint{0.702919in}{0.854692in}}%
\pgfpathlineto{\pgfqpoint{0.712759in}{0.854579in}}%
\pgfpathlineto{\pgfqpoint{0.722859in}{0.855043in}}%
\pgfpathlineto{\pgfqpoint{0.709148in}{0.870560in}}%
\pgfpathlineto{\pgfqpoint{0.697161in}{0.886542in}}%
\pgfpathlineto{\pgfqpoint{0.686939in}{0.902919in}}%
\pgfpathlineto{\pgfqpoint{0.678514in}{0.919620in}}%
\pgfpathlineto{\pgfqpoint{0.671910in}{0.936575in}}%
\pgfpathclose%
\pgfusepath{fill}%
\end{pgfscope}%
\begin{pgfscope}%
\pgfpathrectangle{\pgfqpoint{0.050000in}{0.050000in}}{\pgfqpoint{2.081932in}{2.081932in}}%
\pgfusepath{clip}%
\pgfsetbuttcap%
\pgfsetroundjoin%
\definecolor{currentfill}{rgb}{0.227802,0.326594,0.546532}%
\pgfsetfillcolor{currentfill}%
\pgfsetlinewidth{0.000000pt}%
\definecolor{currentstroke}{rgb}{0.000000,0.000000,0.000000}%
\pgfsetstrokecolor{currentstroke}%
\pgfsetdash{}{0pt}%
\pgfpathmoveto{\pgfqpoint{0.941675in}{1.011107in}}%
\pgfpathlineto{\pgfqpoint{0.937786in}{1.002036in}}%
\pgfpathlineto{\pgfqpoint{0.933275in}{0.992956in}}%
\pgfpathlineto{\pgfqpoint{0.928160in}{0.983904in}}%
\pgfpathlineto{\pgfqpoint{0.922461in}{0.974913in}}%
\pgfpathlineto{\pgfqpoint{0.916202in}{0.966019in}}%
\pgfpathlineto{\pgfqpoint{0.924251in}{0.958453in}}%
\pgfpathlineto{\pgfqpoint{0.933108in}{0.951201in}}%
\pgfpathlineto{\pgfqpoint{0.942737in}{0.944296in}}%
\pgfpathlineto{\pgfqpoint{0.953102in}{0.937767in}}%
\pgfpathlineto{\pgfqpoint{0.964159in}{0.931641in}}%
\pgfpathlineto{\pgfqpoint{0.968959in}{0.941635in}}%
\pgfpathlineto{\pgfqpoint{0.973327in}{0.951628in}}%
\pgfpathlineto{\pgfqpoint{0.977248in}{0.961583in}}%
\pgfpathlineto{\pgfqpoint{0.980704in}{0.971460in}}%
\pgfpathlineto{\pgfqpoint{0.983684in}{0.981222in}}%
\pgfpathlineto{\pgfqpoint{0.974006in}{0.986544in}}%
\pgfpathlineto{\pgfqpoint{0.964931in}{0.992219in}}%
\pgfpathlineto{\pgfqpoint{0.956496in}{0.998222in}}%
\pgfpathlineto{\pgfqpoint{0.948734in}{1.004527in}}%
\pgfpathlineto{\pgfqpoint{0.941675in}{1.011107in}}%
\pgfpathclose%
\pgfusepath{fill}%
\end{pgfscope}%
\begin{pgfscope}%
\pgfpathrectangle{\pgfqpoint{0.050000in}{0.050000in}}{\pgfqpoint{2.081932in}{2.081932in}}%
\pgfusepath{clip}%
\pgfsetbuttcap%
\pgfsetroundjoin%
\definecolor{currentfill}{rgb}{0.876168,0.891125,0.095250}%
\pgfsetfillcolor{currentfill}%
\pgfsetlinewidth{0.000000pt}%
\definecolor{currentstroke}{rgb}{0.000000,0.000000,0.000000}%
\pgfsetstrokecolor{currentstroke}%
\pgfsetdash{}{0pt}%
\pgfpathmoveto{\pgfqpoint{1.624002in}{1.190378in}}%
\pgfpathlineto{\pgfqpoint{1.613768in}{1.195753in}}%
\pgfpathlineto{\pgfqpoint{1.603098in}{1.200707in}}%
\pgfpathlineto{\pgfqpoint{1.592036in}{1.205220in}}%
\pgfpathlineto{\pgfqpoint{1.580628in}{1.209272in}}%
\pgfpathlineto{\pgfqpoint{1.568920in}{1.212846in}}%
\pgfpathlineto{\pgfqpoint{1.570543in}{1.229452in}}%
\pgfpathlineto{\pgfqpoint{1.570343in}{1.246049in}}%
\pgfpathlineto{\pgfqpoint{1.568330in}{1.262572in}}%
\pgfpathlineto{\pgfqpoint{1.564525in}{1.278954in}}%
\pgfpathlineto{\pgfqpoint{1.558953in}{1.295131in}}%
\pgfpathlineto{\pgfqpoint{1.570282in}{1.293713in}}%
\pgfpathlineto{\pgfqpoint{1.581315in}{1.291770in}}%
\pgfpathlineto{\pgfqpoint{1.592007in}{1.289309in}}%
\pgfpathlineto{\pgfqpoint{1.602316in}{1.286342in}}%
\pgfpathlineto{\pgfqpoint{1.612200in}{1.282881in}}%
\pgfpathlineto{\pgfqpoint{1.618568in}{1.264713in}}%
\pgfpathlineto{\pgfqpoint{1.622960in}{1.246306in}}%
\pgfpathlineto{\pgfqpoint{1.625344in}{1.227732in}}%
\pgfpathlineto{\pgfqpoint{1.625697in}{1.209064in}}%
\pgfpathlineto{\pgfqpoint{1.624002in}{1.190378in}}%
\pgfpathclose%
\pgfusepath{fill}%
\end{pgfscope}%
\begin{pgfscope}%
\pgfpathrectangle{\pgfqpoint{0.050000in}{0.050000in}}{\pgfqpoint{2.081932in}{2.081932in}}%
\pgfusepath{clip}%
\pgfsetbuttcap%
\pgfsetroundjoin%
\definecolor{currentfill}{rgb}{0.120638,0.625828,0.533488}%
\pgfsetfillcolor{currentfill}%
\pgfsetlinewidth{0.000000pt}%
\definecolor{currentstroke}{rgb}{0.000000,0.000000,0.000000}%
\pgfsetstrokecolor{currentstroke}%
\pgfsetdash{}{0pt}%
\pgfpathmoveto{\pgfqpoint{0.920516in}{1.128870in}}%
\pgfpathlineto{\pgfqpoint{0.923670in}{1.121251in}}%
\pgfpathlineto{\pgfqpoint{0.926070in}{1.113422in}}%
\pgfpathlineto{\pgfqpoint{0.927708in}{1.105415in}}%
\pgfpathlineto{\pgfqpoint{0.928580in}{1.097262in}}%
\pgfpathlineto{\pgfqpoint{0.928681in}{1.088996in}}%
\pgfpathlineto{\pgfqpoint{0.931712in}{1.081915in}}%
\pgfpathlineto{\pgfqpoint{0.935506in}{1.074956in}}%
\pgfpathlineto{\pgfqpoint{0.940051in}{1.068147in}}%
\pgfpathlineto{\pgfqpoint{0.945329in}{1.061518in}}%
\pgfpathlineto{\pgfqpoint{0.951320in}{1.055095in}}%
\pgfpathlineto{\pgfqpoint{0.951230in}{1.063377in}}%
\pgfpathlineto{\pgfqpoint{0.950459in}{1.071408in}}%
\pgfpathlineto{\pgfqpoint{0.949010in}{1.079156in}}%
\pgfpathlineto{\pgfqpoint{0.946885in}{1.086591in}}%
\pgfpathlineto{\pgfqpoint{0.944094in}{1.093681in}}%
\pgfpathlineto{\pgfqpoint{0.937851in}{1.100349in}}%
\pgfpathlineto{\pgfqpoint{0.932352in}{1.107231in}}%
\pgfpathlineto{\pgfqpoint{0.927619in}{1.114298in}}%
\pgfpathlineto{\pgfqpoint{0.923669in}{1.121521in}}%
\pgfpathlineto{\pgfqpoint{0.920516in}{1.128870in}}%
\pgfpathclose%
\pgfusepath{fill}%
\end{pgfscope}%
\begin{pgfscope}%
\pgfpathrectangle{\pgfqpoint{0.050000in}{0.050000in}}{\pgfqpoint{2.081932in}{2.081932in}}%
\pgfusepath{clip}%
\pgfsetbuttcap%
\pgfsetroundjoin%
\definecolor{currentfill}{rgb}{0.162142,0.474838,0.558140}%
\pgfsetfillcolor{currentfill}%
\pgfsetlinewidth{0.000000pt}%
\definecolor{currentstroke}{rgb}{0.000000,0.000000,0.000000}%
\pgfsetstrokecolor{currentstroke}%
\pgfsetdash{}{0pt}%
\pgfpathmoveto{\pgfqpoint{1.262473in}{1.035705in}}%
\pgfpathlineto{\pgfqpoint{1.262978in}{1.027117in}}%
\pgfpathlineto{\pgfqpoint{1.264062in}{1.018267in}}%
\pgfpathlineto{\pgfqpoint{1.265720in}{1.009188in}}%
\pgfpathlineto{\pgfqpoint{1.267945in}{0.999917in}}%
\pgfpathlineto{\pgfqpoint{1.270728in}{0.990490in}}%
\pgfpathlineto{\pgfqpoint{1.279360in}{0.996398in}}%
\pgfpathlineto{\pgfqpoint{1.287328in}{1.002615in}}%
\pgfpathlineto{\pgfqpoint{1.294601in}{1.009116in}}%
\pgfpathlineto{\pgfqpoint{1.301152in}{1.015873in}}%
\pgfpathlineto{\pgfqpoint{1.306954in}{1.022857in}}%
\pgfpathlineto{\pgfqpoint{1.303519in}{1.031657in}}%
\pgfpathlineto{\pgfqpoint{1.300772in}{1.040420in}}%
\pgfpathlineto{\pgfqpoint{1.298725in}{1.049112in}}%
\pgfpathlineto{\pgfqpoint{1.297387in}{1.057700in}}%
\pgfpathlineto{\pgfqpoint{1.296763in}{1.066149in}}%
\pgfpathlineto{\pgfqpoint{1.291268in}{1.059578in}}%
\pgfpathlineto{\pgfqpoint{1.285066in}{1.053222in}}%
\pgfpathlineto{\pgfqpoint{1.278181in}{1.047108in}}%
\pgfpathlineto{\pgfqpoint{1.270640in}{1.041260in}}%
\pgfpathlineto{\pgfqpoint{1.262473in}{1.035705in}}%
\pgfpathclose%
\pgfusepath{fill}%
\end{pgfscope}%
\begin{pgfscope}%
\pgfpathrectangle{\pgfqpoint{0.050000in}{0.050000in}}{\pgfqpoint{2.081932in}{2.081932in}}%
\pgfusepath{clip}%
\pgfsetbuttcap%
\pgfsetroundjoin%
\definecolor{currentfill}{rgb}{0.267004,0.004874,0.329415}%
\pgfsetfillcolor{currentfill}%
\pgfsetlinewidth{0.000000pt}%
\definecolor{currentstroke}{rgb}{0.000000,0.000000,0.000000}%
\pgfsetstrokecolor{currentstroke}%
\pgfsetdash{}{0pt}%
\pgfpathmoveto{\pgfqpoint{1.367271in}{0.855433in}}%
\pgfpathlineto{\pgfqpoint{1.376218in}{0.848459in}}%
\pgfpathlineto{\pgfqpoint{1.385302in}{0.841918in}}%
\pgfpathlineto{\pgfqpoint{1.394489in}{0.835838in}}%
\pgfpathlineto{\pgfqpoint{1.403742in}{0.830243in}}%
\pgfpathlineto{\pgfqpoint{1.413026in}{0.825156in}}%
\pgfpathlineto{\pgfqpoint{1.429554in}{0.836830in}}%
\pgfpathlineto{\pgfqpoint{1.444765in}{0.849100in}}%
\pgfpathlineto{\pgfqpoint{1.458599in}{0.861910in}}%
\pgfpathlineto{\pgfqpoint{1.471004in}{0.875206in}}%
\pgfpathlineto{\pgfqpoint{1.481936in}{0.888928in}}%
\pgfpathlineto{\pgfqpoint{1.470559in}{0.891978in}}%
\pgfpathlineto{\pgfqpoint{1.459214in}{0.895540in}}%
\pgfpathlineto{\pgfqpoint{1.447944in}{0.899600in}}%
\pgfpathlineto{\pgfqpoint{1.436796in}{0.904141in}}%
\pgfpathlineto{\pgfqpoint{1.425811in}{0.909146in}}%
\pgfpathlineto{\pgfqpoint{1.416495in}{0.897577in}}%
\pgfpathlineto{\pgfqpoint{1.405941in}{0.886374in}}%
\pgfpathlineto{\pgfqpoint{1.394189in}{0.875585in}}%
\pgfpathlineto{\pgfqpoint{1.381282in}{0.865256in}}%
\pgfpathlineto{\pgfqpoint{1.367271in}{0.855433in}}%
\pgfpathclose%
\pgfusepath{fill}%
\end{pgfscope}%
\begin{pgfscope}%
\pgfpathrectangle{\pgfqpoint{0.050000in}{0.050000in}}{\pgfqpoint{2.081932in}{2.081932in}}%
\pgfusepath{clip}%
\pgfsetbuttcap%
\pgfsetroundjoin%
\definecolor{currentfill}{rgb}{0.855810,0.888601,0.097452}%
\pgfsetfillcolor{currentfill}%
\pgfsetlinewidth{0.000000pt}%
\definecolor{currentstroke}{rgb}{0.000000,0.000000,0.000000}%
\pgfsetstrokecolor{currentstroke}%
\pgfsetdash{}{0pt}%
\pgfpathmoveto{\pgfqpoint{1.446363in}{1.220127in}}%
\pgfpathlineto{\pgfqpoint{1.434677in}{1.217946in}}%
\pgfpathlineto{\pgfqpoint{1.423293in}{1.215254in}}%
\pgfpathlineto{\pgfqpoint{1.412257in}{1.212062in}}%
\pgfpathlineto{\pgfqpoint{1.401616in}{1.208384in}}%
\pgfpathlineto{\pgfqpoint{1.391411in}{1.204236in}}%
\pgfpathlineto{\pgfqpoint{1.392615in}{1.214306in}}%
\pgfpathlineto{\pgfqpoint{1.392715in}{1.224388in}}%
\pgfpathlineto{\pgfqpoint{1.391715in}{1.234441in}}%
\pgfpathlineto{\pgfqpoint{1.389623in}{1.244424in}}%
\pgfpathlineto{\pgfqpoint{1.386451in}{1.254299in}}%
\pgfpathlineto{\pgfqpoint{1.396406in}{1.260285in}}%
\pgfpathlineto{\pgfqpoint{1.406782in}{1.265880in}}%
\pgfpathlineto{\pgfqpoint{1.417536in}{1.271062in}}%
\pgfpathlineto{\pgfqpoint{1.428625in}{1.275809in}}%
\pgfpathlineto{\pgfqpoint{1.440002in}{1.280101in}}%
\pgfpathlineto{\pgfqpoint{1.443889in}{1.268284in}}%
\pgfpathlineto{\pgfqpoint{1.446482in}{1.256330in}}%
\pgfpathlineto{\pgfqpoint{1.447765in}{1.244287in}}%
\pgfpathlineto{\pgfqpoint{1.447727in}{1.232203in}}%
\pgfpathlineto{\pgfqpoint{1.446363in}{1.220127in}}%
\pgfpathclose%
\pgfusepath{fill}%
\end{pgfscope}%
\begin{pgfscope}%
\pgfpathrectangle{\pgfqpoint{0.050000in}{0.050000in}}{\pgfqpoint{2.081932in}{2.081932in}}%
\pgfusepath{clip}%
\pgfsetbuttcap%
\pgfsetroundjoin%
\definecolor{currentfill}{rgb}{0.327796,0.773980,0.406640}%
\pgfsetfillcolor{currentfill}%
\pgfsetlinewidth{0.000000pt}%
\definecolor{currentstroke}{rgb}{0.000000,0.000000,0.000000}%
\pgfsetstrokecolor{currentstroke}%
\pgfsetdash{}{0pt}%
\pgfpathmoveto{\pgfqpoint{0.547075in}{1.189662in}}%
\pgfpathlineto{\pgfqpoint{0.550985in}{1.197230in}}%
\pgfpathlineto{\pgfqpoint{0.555628in}{1.204573in}}%
\pgfpathlineto{\pgfqpoint{0.560988in}{1.211661in}}%
\pgfpathlineto{\pgfqpoint{0.567044in}{1.218464in}}%
\pgfpathlineto{\pgfqpoint{0.573773in}{1.224954in}}%
\pgfpathlineto{\pgfqpoint{0.570419in}{1.204776in}}%
\pgfpathlineto{\pgfqpoint{0.569262in}{1.184465in}}%
\pgfpathlineto{\pgfqpoint{0.570321in}{1.164100in}}%
\pgfpathlineto{\pgfqpoint{0.573611in}{1.143766in}}%
\pgfpathlineto{\pgfqpoint{0.579133in}{1.123545in}}%
\pgfpathlineto{\pgfqpoint{0.572386in}{1.115706in}}%
\pgfpathlineto{\pgfqpoint{0.566312in}{1.107674in}}%
\pgfpathlineto{\pgfqpoint{0.560934in}{1.099483in}}%
\pgfpathlineto{\pgfqpoint{0.556274in}{1.091166in}}%
\pgfpathlineto{\pgfqpoint{0.552350in}{1.082759in}}%
\pgfpathlineto{\pgfqpoint{0.546624in}{1.104086in}}%
\pgfpathlineto{\pgfqpoint{0.543242in}{1.125527in}}%
\pgfpathlineto{\pgfqpoint{0.542202in}{1.146995in}}%
\pgfpathlineto{\pgfqpoint{0.543487in}{1.168402in}}%
\pgfpathlineto{\pgfqpoint{0.547075in}{1.189662in}}%
\pgfpathclose%
\pgfusepath{fill}%
\end{pgfscope}%
\begin{pgfscope}%
\pgfpathrectangle{\pgfqpoint{0.050000in}{0.050000in}}{\pgfqpoint{2.081932in}{2.081932in}}%
\pgfusepath{clip}%
\pgfsetbuttcap%
\pgfsetroundjoin%
\definecolor{currentfill}{rgb}{0.278791,0.062145,0.386592}%
\pgfsetfillcolor{currentfill}%
\pgfsetlinewidth{0.000000pt}%
\definecolor{currentstroke}{rgb}{0.000000,0.000000,0.000000}%
\pgfsetstrokecolor{currentstroke}%
\pgfsetdash{}{0pt}%
\pgfpathmoveto{\pgfqpoint{0.877418in}{0.924195in}}%
\pgfpathlineto{\pgfqpoint{0.868376in}{0.916588in}}%
\pgfpathlineto{\pgfqpoint{0.858984in}{0.909304in}}%
\pgfpathlineto{\pgfqpoint{0.849278in}{0.902373in}}%
\pgfpathlineto{\pgfqpoint{0.839296in}{0.895822in}}%
\pgfpathlineto{\pgfqpoint{0.829077in}{0.889676in}}%
\pgfpathlineto{\pgfqpoint{0.840472in}{0.878759in}}%
\pgfpathlineto{\pgfqpoint{0.853037in}{0.868288in}}%
\pgfpathlineto{\pgfqpoint{0.866721in}{0.858309in}}%
\pgfpathlineto{\pgfqpoint{0.881470in}{0.848867in}}%
\pgfpathlineto{\pgfqpoint{0.897227in}{0.840003in}}%
\pgfpathlineto{\pgfqpoint{0.905090in}{0.847943in}}%
\pgfpathlineto{\pgfqpoint{0.912767in}{0.856247in}}%
\pgfpathlineto{\pgfqpoint{0.920229in}{0.864882in}}%
\pgfpathlineto{\pgfqpoint{0.927447in}{0.873815in}}%
\pgfpathlineto{\pgfqpoint{0.934394in}{0.883010in}}%
\pgfpathlineto{\pgfqpoint{0.921241in}{0.890353in}}%
\pgfpathlineto{\pgfqpoint{0.908919in}{0.898178in}}%
\pgfpathlineto{\pgfqpoint{0.897479in}{0.906451in}}%
\pgfpathlineto{\pgfqpoint{0.886965in}{0.915136in}}%
\pgfpathlineto{\pgfqpoint{0.877418in}{0.924195in}}%
\pgfpathclose%
\pgfusepath{fill}%
\end{pgfscope}%
\begin{pgfscope}%
\pgfpathrectangle{\pgfqpoint{0.050000in}{0.050000in}}{\pgfqpoint{2.081932in}{2.081932in}}%
\pgfusepath{clip}%
\pgfsetbuttcap%
\pgfsetroundjoin%
\definecolor{currentfill}{rgb}{0.606045,0.850733,0.236712}%
\pgfsetfillcolor{currentfill}%
\pgfsetlinewidth{0.000000pt}%
\definecolor{currentstroke}{rgb}{0.000000,0.000000,0.000000}%
\pgfsetstrokecolor{currentstroke}%
\pgfsetdash{}{0pt}%
\pgfpathmoveto{\pgfqpoint{0.847130in}{1.237430in}}%
\pgfpathlineto{\pgfqpoint{0.856811in}{1.231666in}}%
\pgfpathlineto{\pgfqpoint{0.865980in}{1.225531in}}%
\pgfpathlineto{\pgfqpoint{0.874599in}{1.219052in}}%
\pgfpathlineto{\pgfqpoint{0.882634in}{1.212254in}}%
\pgfpathlineto{\pgfqpoint{0.890053in}{1.205167in}}%
\pgfpathlineto{\pgfqpoint{0.888978in}{1.196682in}}%
\pgfpathlineto{\pgfqpoint{0.888828in}{1.188166in}}%
\pgfpathlineto{\pgfqpoint{0.889607in}{1.179652in}}%
\pgfpathlineto{\pgfqpoint{0.891314in}{1.171176in}}%
\pgfpathlineto{\pgfqpoint{0.893946in}{1.162771in}}%
\pgfpathlineto{\pgfqpoint{0.886614in}{1.168517in}}%
\pgfpathlineto{\pgfqpoint{0.878670in}{1.173858in}}%
\pgfpathlineto{\pgfqpoint{0.870145in}{1.178773in}}%
\pgfpathlineto{\pgfqpoint{0.861073in}{1.183241in}}%
\pgfpathlineto{\pgfqpoint{0.851490in}{1.187243in}}%
\pgfpathlineto{\pgfqpoint{0.848415in}{1.197200in}}%
\pgfpathlineto{\pgfqpoint{0.846439in}{1.207238in}}%
\pgfpathlineto{\pgfqpoint{0.845568in}{1.217316in}}%
\pgfpathlineto{\pgfqpoint{0.845800in}{1.227393in}}%
\pgfpathlineto{\pgfqpoint{0.847130in}{1.237430in}}%
\pgfpathclose%
\pgfusepath{fill}%
\end{pgfscope}%
\begin{pgfscope}%
\pgfpathrectangle{\pgfqpoint{0.050000in}{0.050000in}}{\pgfqpoint{2.081932in}{2.081932in}}%
\pgfusepath{clip}%
\pgfsetbuttcap%
\pgfsetroundjoin%
\definecolor{currentfill}{rgb}{0.993248,0.906157,0.143936}%
\pgfsetfillcolor{currentfill}%
\pgfsetlinewidth{0.000000pt}%
\definecolor{currentstroke}{rgb}{0.000000,0.000000,0.000000}%
\pgfsetstrokecolor{currentstroke}%
\pgfsetdash{}{0pt}%
\pgfpathmoveto{\pgfqpoint{1.568920in}{1.212846in}}%
\pgfpathlineto{\pgfqpoint{1.556962in}{1.215929in}}%
\pgfpathlineto{\pgfqpoint{1.544801in}{1.218506in}}%
\pgfpathlineto{\pgfqpoint{1.532490in}{1.220567in}}%
\pgfpathlineto{\pgfqpoint{1.520077in}{1.222104in}}%
\pgfpathlineto{\pgfqpoint{1.507616in}{1.223111in}}%
\pgfpathlineto{\pgfqpoint{1.509127in}{1.237439in}}%
\pgfpathlineto{\pgfqpoint{1.509062in}{1.251768in}}%
\pgfpathlineto{\pgfqpoint{1.507431in}{1.266040in}}%
\pgfpathlineto{\pgfqpoint{1.504248in}{1.280199in}}%
\pgfpathlineto{\pgfqpoint{1.499535in}{1.294188in}}%
\pgfpathlineto{\pgfqpoint{1.511626in}{1.295450in}}%
\pgfpathlineto{\pgfqpoint{1.523663in}{1.296177in}}%
\pgfpathlineto{\pgfqpoint{1.535596in}{1.296366in}}%
\pgfpathlineto{\pgfqpoint{1.547375in}{1.296017in}}%
\pgfpathlineto{\pgfqpoint{1.558953in}{1.295131in}}%
\pgfpathlineto{\pgfqpoint{1.564525in}{1.278954in}}%
\pgfpathlineto{\pgfqpoint{1.568330in}{1.262572in}}%
\pgfpathlineto{\pgfqpoint{1.570343in}{1.246049in}}%
\pgfpathlineto{\pgfqpoint{1.570543in}{1.229452in}}%
\pgfpathlineto{\pgfqpoint{1.568920in}{1.212846in}}%
\pgfpathclose%
\pgfusepath{fill}%
\end{pgfscope}%
\begin{pgfscope}%
\pgfpathrectangle{\pgfqpoint{0.050000in}{0.050000in}}{\pgfqpoint{2.081932in}{2.081932in}}%
\pgfusepath{clip}%
\pgfsetbuttcap%
\pgfsetroundjoin%
\definecolor{currentfill}{rgb}{0.993248,0.906157,0.143936}%
\pgfsetfillcolor{currentfill}%
\pgfsetlinewidth{0.000000pt}%
\definecolor{currentstroke}{rgb}{0.000000,0.000000,0.000000}%
\pgfsetstrokecolor{currentstroke}%
\pgfsetdash{}{0pt}%
\pgfpathmoveto{\pgfqpoint{1.507616in}{1.223111in}}%
\pgfpathlineto{\pgfqpoint{1.495156in}{1.223583in}}%
\pgfpathlineto{\pgfqpoint{1.482750in}{1.223519in}}%
\pgfpathlineto{\pgfqpoint{1.470449in}{1.222920in}}%
\pgfpathlineto{\pgfqpoint{1.458303in}{1.221787in}}%
\pgfpathlineto{\pgfqpoint{1.446363in}{1.220127in}}%
\pgfpathlineto{\pgfqpoint{1.447727in}{1.232203in}}%
\pgfpathlineto{\pgfqpoint{1.447765in}{1.244287in}}%
\pgfpathlineto{\pgfqpoint{1.446482in}{1.256330in}}%
\pgfpathlineto{\pgfqpoint{1.443889in}{1.268284in}}%
\pgfpathlineto{\pgfqpoint{1.440002in}{1.280101in}}%
\pgfpathlineto{\pgfqpoint{1.451620in}{1.283921in}}%
\pgfpathlineto{\pgfqpoint{1.463431in}{1.287252in}}%
\pgfpathlineto{\pgfqpoint{1.475387in}{1.290081in}}%
\pgfpathlineto{\pgfqpoint{1.487438in}{1.292396in}}%
\pgfpathlineto{\pgfqpoint{1.499535in}{1.294188in}}%
\pgfpathlineto{\pgfqpoint{1.504248in}{1.280199in}}%
\pgfpathlineto{\pgfqpoint{1.507431in}{1.266040in}}%
\pgfpathlineto{\pgfqpoint{1.509062in}{1.251768in}}%
\pgfpathlineto{\pgfqpoint{1.509127in}{1.237439in}}%
\pgfpathlineto{\pgfqpoint{1.507616in}{1.223111in}}%
\pgfpathclose%
\pgfusepath{fill}%
\end{pgfscope}%
\begin{pgfscope}%
\pgfpathrectangle{\pgfqpoint{0.050000in}{0.050000in}}{\pgfqpoint{2.081932in}{2.081932in}}%
\pgfusepath{clip}%
\pgfsetbuttcap%
\pgfsetroundjoin%
\definecolor{currentfill}{rgb}{0.162142,0.474838,0.558140}%
\pgfsetfillcolor{currentfill}%
\pgfsetlinewidth{0.000000pt}%
\definecolor{currentstroke}{rgb}{0.000000,0.000000,0.000000}%
\pgfsetstrokecolor{currentstroke}%
\pgfsetdash{}{0pt}%
\pgfpathmoveto{\pgfqpoint{0.951320in}{1.055095in}}%
\pgfpathlineto{\pgfqpoint{0.950730in}{1.046596in}}%
\pgfpathlineto{\pgfqpoint{0.949464in}{1.037913in}}%
\pgfpathlineto{\pgfqpoint{0.947526in}{1.029081in}}%
\pgfpathlineto{\pgfqpoint{0.944926in}{1.020133in}}%
\pgfpathlineto{\pgfqpoint{0.941675in}{1.011107in}}%
\pgfpathlineto{\pgfqpoint{0.948734in}{1.004527in}}%
\pgfpathlineto{\pgfqpoint{0.956496in}{0.998222in}}%
\pgfpathlineto{\pgfqpoint{0.964931in}{0.992219in}}%
\pgfpathlineto{\pgfqpoint{0.974006in}{0.986544in}}%
\pgfpathlineto{\pgfqpoint{0.983684in}{0.981222in}}%
\pgfpathlineto{\pgfqpoint{0.986174in}{0.990829in}}%
\pgfpathlineto{\pgfqpoint{0.988165in}{1.000247in}}%
\pgfpathlineto{\pgfqpoint{0.989649in}{1.009436in}}%
\pgfpathlineto{\pgfqpoint{0.990619in}{1.018362in}}%
\pgfpathlineto{\pgfqpoint{0.991071in}{1.026989in}}%
\pgfpathlineto{\pgfqpoint{0.981916in}{1.031994in}}%
\pgfpathlineto{\pgfqpoint{0.973330in}{1.037331in}}%
\pgfpathlineto{\pgfqpoint{0.965348in}{1.042976in}}%
\pgfpathlineto{\pgfqpoint{0.958002in}{1.048906in}}%
\pgfpathlineto{\pgfqpoint{0.951320in}{1.055095in}}%
\pgfpathclose%
\pgfusepath{fill}%
\end{pgfscope}%
\begin{pgfscope}%
\pgfpathrectangle{\pgfqpoint{0.050000in}{0.050000in}}{\pgfqpoint{2.081932in}{2.081932in}}%
\pgfusepath{clip}%
\pgfsetbuttcap%
\pgfsetroundjoin%
\definecolor{currentfill}{rgb}{0.296479,0.761561,0.424223}%
\pgfsetfillcolor{currentfill}%
\pgfsetlinewidth{0.000000pt}%
\definecolor{currentstroke}{rgb}{0.000000,0.000000,0.000000}%
\pgfsetstrokecolor{currentstroke}%
\pgfsetdash{}{0pt}%
\pgfpathmoveto{\pgfqpoint{1.329256in}{1.135987in}}%
\pgfpathlineto{\pgfqpoint{1.322995in}{1.130640in}}%
\pgfpathlineto{\pgfqpoint{1.317358in}{1.124859in}}%
\pgfpathlineto{\pgfqpoint{1.312368in}{1.118667in}}%
\pgfpathlineto{\pgfqpoint{1.308043in}{1.112091in}}%
\pgfpathlineto{\pgfqpoint{1.304399in}{1.105156in}}%
\pgfpathlineto{\pgfqpoint{1.309365in}{1.112171in}}%
\pgfpathlineto{\pgfqpoint{1.313553in}{1.119351in}}%
\pgfpathlineto{\pgfqpoint{1.316947in}{1.126666in}}%
\pgfpathlineto{\pgfqpoint{1.319538in}{1.134085in}}%
\pgfpathlineto{\pgfqpoint{1.321315in}{1.141579in}}%
\pgfpathlineto{\pgfqpoint{1.325273in}{1.149195in}}%
\pgfpathlineto{\pgfqpoint{1.329971in}{1.156587in}}%
\pgfpathlineto{\pgfqpoint{1.335390in}{1.163724in}}%
\pgfpathlineto{\pgfqpoint{1.341509in}{1.170576in}}%
\pgfpathlineto{\pgfqpoint{1.348305in}{1.177117in}}%
\pgfpathlineto{\pgfqpoint{1.346319in}{1.168658in}}%
\pgfpathlineto{\pgfqpoint{1.343410in}{1.160282in}}%
\pgfpathlineto{\pgfqpoint{1.339587in}{1.152022in}}%
\pgfpathlineto{\pgfqpoint{1.334864in}{1.143912in}}%
\pgfpathlineto{\pgfqpoint{1.329256in}{1.135987in}}%
\pgfpathclose%
\pgfusepath{fill}%
\end{pgfscope}%
\begin{pgfscope}%
\pgfpathrectangle{\pgfqpoint{0.050000in}{0.050000in}}{\pgfqpoint{2.081932in}{2.081932in}}%
\pgfusepath{clip}%
\pgfsetbuttcap%
\pgfsetroundjoin%
\definecolor{currentfill}{rgb}{0.206756,0.371758,0.553117}%
\pgfsetfillcolor{currentfill}%
\pgfsetlinewidth{0.000000pt}%
\definecolor{currentstroke}{rgb}{0.000000,0.000000,0.000000}%
\pgfsetstrokecolor{currentstroke}%
\pgfsetdash{}{0pt}%
\pgfpathmoveto{\pgfqpoint{1.626916in}{0.908239in}}%
\pgfpathlineto{\pgfqpoint{1.633080in}{0.913670in}}%
\pgfpathlineto{\pgfqpoint{1.638620in}{0.919540in}}%
\pgfpathlineto{\pgfqpoint{1.643512in}{0.925827in}}%
\pgfpathlineto{\pgfqpoint{1.647736in}{0.932506in}}%
\pgfpathlineto{\pgfqpoint{1.651273in}{0.939551in}}%
\pgfpathlineto{\pgfqpoint{1.664662in}{0.960076in}}%
\pgfpathlineto{\pgfqpoint{1.675761in}{0.981016in}}%
\pgfpathlineto{\pgfqpoint{1.684539in}{1.002279in}}%
\pgfpathlineto{\pgfqpoint{1.690977in}{1.023775in}}%
\pgfpathlineto{\pgfqpoint{1.695069in}{1.045413in}}%
\pgfpathlineto{\pgfqpoint{1.691289in}{1.037763in}}%
\pgfpathlineto{\pgfqpoint{1.686774in}{1.030337in}}%
\pgfpathlineto{\pgfqpoint{1.681544in}{1.023164in}}%
\pgfpathlineto{\pgfqpoint{1.675620in}{1.016274in}}%
\pgfpathlineto{\pgfqpoint{1.669026in}{1.009693in}}%
\pgfpathlineto{\pgfqpoint{1.665055in}{0.988946in}}%
\pgfpathlineto{\pgfqpoint{1.658845in}{0.968341in}}%
\pgfpathlineto{\pgfqpoint{1.650404in}{0.947963in}}%
\pgfpathlineto{\pgfqpoint{1.639751in}{0.927900in}}%
\pgfpathlineto{\pgfqpoint{1.626916in}{0.908239in}}%
\pgfpathclose%
\pgfusepath{fill}%
\end{pgfscope}%
\begin{pgfscope}%
\pgfpathrectangle{\pgfqpoint{0.050000in}{0.050000in}}{\pgfqpoint{2.081932in}{2.081932in}}%
\pgfusepath{clip}%
\pgfsetbuttcap%
\pgfsetroundjoin%
\definecolor{currentfill}{rgb}{0.227802,0.326594,0.546532}%
\pgfsetfillcolor{currentfill}%
\pgfsetlinewidth{0.000000pt}%
\definecolor{currentstroke}{rgb}{0.000000,0.000000,0.000000}%
\pgfsetstrokecolor{currentstroke}%
\pgfsetdash{}{0pt}%
\pgfpathmoveto{\pgfqpoint{1.218921in}{0.966434in}}%
\pgfpathlineto{\pgfqpoint{1.221118in}{0.956329in}}%
\pgfpathlineto{\pgfqpoint{1.223668in}{0.946055in}}%
\pgfpathlineto{\pgfqpoint{1.226561in}{0.935652in}}%
\pgfpathlineto{\pgfqpoint{1.229784in}{0.925159in}}%
\pgfpathlineto{\pgfqpoint{1.233326in}{0.914618in}}%
\pgfpathlineto{\pgfqpoint{1.246338in}{0.919233in}}%
\pgfpathlineto{\pgfqpoint{1.258815in}{0.924333in}}%
\pgfpathlineto{\pgfqpoint{1.270707in}{0.929895in}}%
\pgfpathlineto{\pgfqpoint{1.281965in}{0.935895in}}%
\pgfpathlineto{\pgfqpoint{1.292542in}{0.942307in}}%
\pgfpathlineto{\pgfqpoint{1.287180in}{0.951959in}}%
\pgfpathlineto{\pgfqpoint{1.282300in}{0.961640in}}%
\pgfpathlineto{\pgfqpoint{1.277920in}{0.971314in}}%
\pgfpathlineto{\pgfqpoint{1.274057in}{0.980943in}}%
\pgfpathlineto{\pgfqpoint{1.270728in}{0.990490in}}%
\pgfpathlineto{\pgfqpoint{1.261468in}{0.984918in}}%
\pgfpathlineto{\pgfqpoint{1.251615in}{0.979705in}}%
\pgfpathlineto{\pgfqpoint{1.241211in}{0.974873in}}%
\pgfpathlineto{\pgfqpoint{1.230298in}{0.970443in}}%
\pgfpathlineto{\pgfqpoint{1.218921in}{0.966434in}}%
\pgfpathclose%
\pgfusepath{fill}%
\end{pgfscope}%
\begin{pgfscope}%
\pgfpathrectangle{\pgfqpoint{0.050000in}{0.050000in}}{\pgfqpoint{2.081932in}{2.081932in}}%
\pgfusepath{clip}%
\pgfsetbuttcap%
\pgfsetroundjoin%
\definecolor{currentfill}{rgb}{0.278012,0.180367,0.486697}%
\pgfsetfillcolor{currentfill}%
\pgfsetlinewidth{0.000000pt}%
\definecolor{currentstroke}{rgb}{0.000000,0.000000,0.000000}%
\pgfsetstrokecolor{currentstroke}%
\pgfsetdash{}{0pt}%
\pgfpathmoveto{\pgfqpoint{1.233326in}{0.914618in}}%
\pgfpathlineto{\pgfqpoint{1.237173in}{0.904067in}}%
\pgfpathlineto{\pgfqpoint{1.241310in}{0.893549in}}%
\pgfpathlineto{\pgfqpoint{1.245722in}{0.883104in}}%
\pgfpathlineto{\pgfqpoint{1.250392in}{0.872771in}}%
\pgfpathlineto{\pgfqpoint{1.255301in}{0.862591in}}%
\pgfpathlineto{\pgfqpoint{1.270802in}{0.868129in}}%
\pgfpathlineto{\pgfqpoint{1.285660in}{0.874246in}}%
\pgfpathlineto{\pgfqpoint{1.299815in}{0.880916in}}%
\pgfpathlineto{\pgfqpoint{1.313208in}{0.888110in}}%
\pgfpathlineto{\pgfqpoint{1.325785in}{0.895795in}}%
\pgfpathlineto{\pgfqpoint{1.318362in}{0.904743in}}%
\pgfpathlineto{\pgfqpoint{1.311300in}{0.913904in}}%
\pgfpathlineto{\pgfqpoint{1.304625in}{0.923243in}}%
\pgfpathlineto{\pgfqpoint{1.298365in}{0.932723in}}%
\pgfpathlineto{\pgfqpoint{1.292542in}{0.942307in}}%
\pgfpathlineto{\pgfqpoint{1.281965in}{0.935895in}}%
\pgfpathlineto{\pgfqpoint{1.270707in}{0.929895in}}%
\pgfpathlineto{\pgfqpoint{1.258815in}{0.924333in}}%
\pgfpathlineto{\pgfqpoint{1.246338in}{0.919233in}}%
\pgfpathlineto{\pgfqpoint{1.233326in}{0.914618in}}%
\pgfpathclose%
\pgfusepath{fill}%
\end{pgfscope}%
\begin{pgfscope}%
\pgfpathrectangle{\pgfqpoint{0.050000in}{0.050000in}}{\pgfqpoint{2.081932in}{2.081932in}}%
\pgfusepath{clip}%
\pgfsetbuttcap%
\pgfsetroundjoin%
\definecolor{currentfill}{rgb}{0.636902,0.856542,0.216620}%
\pgfsetfillcolor{currentfill}%
\pgfsetlinewidth{0.000000pt}%
\definecolor{currentstroke}{rgb}{0.000000,0.000000,0.000000}%
\pgfsetstrokecolor{currentstroke}%
\pgfsetdash{}{0pt}%
\pgfpathmoveto{\pgfqpoint{0.573773in}{1.224954in}}%
\pgfpathlineto{\pgfqpoint{0.581147in}{1.231104in}}%
\pgfpathlineto{\pgfqpoint{0.589138in}{1.236889in}}%
\pgfpathlineto{\pgfqpoint{0.597714in}{1.242285in}}%
\pgfpathlineto{\pgfqpoint{0.606840in}{1.247269in}}%
\pgfpathlineto{\pgfqpoint{0.616480in}{1.251820in}}%
\pgfpathlineto{\pgfqpoint{0.613488in}{1.233288in}}%
\pgfpathlineto{\pgfqpoint{0.612521in}{1.214641in}}%
\pgfpathlineto{\pgfqpoint{0.613598in}{1.195952in}}%
\pgfpathlineto{\pgfqpoint{0.616727in}{1.177298in}}%
\pgfpathlineto{\pgfqpoint{0.621910in}{1.158755in}}%
\pgfpathlineto{\pgfqpoint{0.612262in}{1.152345in}}%
\pgfpathlineto{\pgfqpoint{0.603124in}{1.145590in}}%
\pgfpathlineto{\pgfqpoint{0.594533in}{1.138518in}}%
\pgfpathlineto{\pgfqpoint{0.586526in}{1.131160in}}%
\pgfpathlineto{\pgfqpoint{0.579133in}{1.123545in}}%
\pgfpathlineto{\pgfqpoint{0.573611in}{1.143766in}}%
\pgfpathlineto{\pgfqpoint{0.570321in}{1.164100in}}%
\pgfpathlineto{\pgfqpoint{0.569262in}{1.184465in}}%
\pgfpathlineto{\pgfqpoint{0.570419in}{1.204776in}}%
\pgfpathlineto{\pgfqpoint{0.573773in}{1.224954in}}%
\pgfpathclose%
\pgfusepath{fill}%
\end{pgfscope}%
\begin{pgfscope}%
\pgfpathrectangle{\pgfqpoint{0.050000in}{0.050000in}}{\pgfqpoint{2.081932in}{2.081932in}}%
\pgfusepath{clip}%
\pgfsetbuttcap%
\pgfsetroundjoin%
\definecolor{currentfill}{rgb}{0.267004,0.004874,0.329415}%
\pgfsetfillcolor{currentfill}%
\pgfsetlinewidth{0.000000pt}%
\definecolor{currentstroke}{rgb}{0.000000,0.000000,0.000000}%
\pgfsetstrokecolor{currentstroke}%
\pgfsetdash{}{0pt}%
\pgfpathmoveto{\pgfqpoint{0.829077in}{0.889676in}}%
\pgfpathlineto{\pgfqpoint{0.818660in}{0.883961in}}%
\pgfpathlineto{\pgfqpoint{0.808086in}{0.878699in}}%
\pgfpathlineto{\pgfqpoint{0.797396in}{0.873910in}}%
\pgfpathlineto{\pgfqpoint{0.786632in}{0.869615in}}%
\pgfpathlineto{\pgfqpoint{0.775836in}{0.865830in}}%
\pgfpathlineto{\pgfqpoint{0.789245in}{0.852869in}}%
\pgfpathlineto{\pgfqpoint{0.804046in}{0.840432in}}%
\pgfpathlineto{\pgfqpoint{0.820184in}{0.828575in}}%
\pgfpathlineto{\pgfqpoint{0.837595in}{0.817350in}}%
\pgfpathlineto{\pgfqpoint{0.856209in}{0.806808in}}%
\pgfpathlineto{\pgfqpoint{0.864534in}{0.812487in}}%
\pgfpathlineto{\pgfqpoint{0.872830in}{0.818672in}}%
\pgfpathlineto{\pgfqpoint{0.881066in}{0.825338in}}%
\pgfpathlineto{\pgfqpoint{0.889209in}{0.832458in}}%
\pgfpathlineto{\pgfqpoint{0.897227in}{0.840003in}}%
\pgfpathlineto{\pgfqpoint{0.881470in}{0.848867in}}%
\pgfpathlineto{\pgfqpoint{0.866721in}{0.858309in}}%
\pgfpathlineto{\pgfqpoint{0.853037in}{0.868288in}}%
\pgfpathlineto{\pgfqpoint{0.840472in}{0.878759in}}%
\pgfpathlineto{\pgfqpoint{0.829077in}{0.889676in}}%
\pgfpathclose%
\pgfusepath{fill}%
\end{pgfscope}%
\begin{pgfscope}%
\pgfpathrectangle{\pgfqpoint{0.050000in}{0.050000in}}{\pgfqpoint{2.081932in}{2.081932in}}%
\pgfusepath{clip}%
\pgfsetbuttcap%
\pgfsetroundjoin%
\definecolor{currentfill}{rgb}{0.855810,0.888601,0.097452}%
\pgfsetfillcolor{currentfill}%
\pgfsetlinewidth{0.000000pt}%
\definecolor{currentstroke}{rgb}{0.000000,0.000000,0.000000}%
\pgfsetstrokecolor{currentstroke}%
\pgfsetdash{}{0pt}%
\pgfpathmoveto{\pgfqpoint{0.792491in}{1.259909in}}%
\pgfpathlineto{\pgfqpoint{0.804103in}{1.256324in}}%
\pgfpathlineto{\pgfqpoint{0.815419in}{1.252266in}}%
\pgfpathlineto{\pgfqpoint{0.826392in}{1.247751in}}%
\pgfpathlineto{\pgfqpoint{0.836977in}{1.242799in}}%
\pgfpathlineto{\pgfqpoint{0.847130in}{1.237430in}}%
\pgfpathlineto{\pgfqpoint{0.845800in}{1.227393in}}%
\pgfpathlineto{\pgfqpoint{0.845568in}{1.217316in}}%
\pgfpathlineto{\pgfqpoint{0.846439in}{1.207238in}}%
\pgfpathlineto{\pgfqpoint{0.848415in}{1.197200in}}%
\pgfpathlineto{\pgfqpoint{0.851490in}{1.187243in}}%
\pgfpathlineto{\pgfqpoint{0.841434in}{1.190761in}}%
\pgfpathlineto{\pgfqpoint{0.830947in}{1.193782in}}%
\pgfpathlineto{\pgfqpoint{0.820069in}{1.196291in}}%
\pgfpathlineto{\pgfqpoint{0.808847in}{1.198278in}}%
\pgfpathlineto{\pgfqpoint{0.797324in}{1.199735in}}%
\pgfpathlineto{\pgfqpoint{0.793707in}{1.211685in}}%
\pgfpathlineto{\pgfqpoint{0.791415in}{1.223727in}}%
\pgfpathlineto{\pgfqpoint{0.790450in}{1.235811in}}%
\pgfpathlineto{\pgfqpoint{0.790812in}{1.247888in}}%
\pgfpathlineto{\pgfqpoint{0.792491in}{1.259909in}}%
\pgfpathclose%
\pgfusepath{fill}%
\end{pgfscope}%
\begin{pgfscope}%
\pgfpathrectangle{\pgfqpoint{0.050000in}{0.050000in}}{\pgfqpoint{2.081932in}{2.081932in}}%
\pgfusepath{clip}%
\pgfsetbuttcap%
\pgfsetroundjoin%
\definecolor{currentfill}{rgb}{0.267968,0.223549,0.512008}%
\pgfsetfillcolor{currentfill}%
\pgfsetlinewidth{0.000000pt}%
\definecolor{currentstroke}{rgb}{0.000000,0.000000,0.000000}%
\pgfsetstrokecolor{currentstroke}%
\pgfsetdash{}{0pt}%
\pgfpathmoveto{\pgfqpoint{0.618880in}{0.949646in}}%
\pgfpathlineto{\pgfqpoint{0.609504in}{0.953730in}}%
\pgfpathlineto{\pgfqpoint{0.600630in}{0.958273in}}%
\pgfpathlineto{\pgfqpoint{0.592294in}{0.963257in}}%
\pgfpathlineto{\pgfqpoint{0.584531in}{0.968663in}}%
\pgfpathlineto{\pgfqpoint{0.577373in}{0.974470in}}%
\pgfpathlineto{\pgfqpoint{0.585143in}{0.954016in}}%
\pgfpathlineto{\pgfqpoint{0.595132in}{0.933850in}}%
\pgfpathlineto{\pgfqpoint{0.607313in}{0.914059in}}%
\pgfpathlineto{\pgfqpoint{0.621651in}{0.894730in}}%
\pgfpathlineto{\pgfqpoint{0.638099in}{0.875949in}}%
\pgfpathlineto{\pgfqpoint{0.644529in}{0.871391in}}%
\pgfpathlineto{\pgfqpoint{0.651501in}{0.867351in}}%
\pgfpathlineto{\pgfqpoint{0.658983in}{0.863845in}}%
\pgfpathlineto{\pgfqpoint{0.666946in}{0.860886in}}%
\pgfpathlineto{\pgfqpoint{0.675357in}{0.858483in}}%
\pgfpathlineto{\pgfqpoint{0.660103in}{0.875849in}}%
\pgfpathlineto{\pgfqpoint{0.646789in}{0.893728in}}%
\pgfpathlineto{\pgfqpoint{0.635460in}{0.912039in}}%
\pgfpathlineto{\pgfqpoint{0.626148in}{0.930705in}}%
\pgfpathlineto{\pgfqpoint{0.618880in}{0.949646in}}%
\pgfpathclose%
\pgfusepath{fill}%
\end{pgfscope}%
\begin{pgfscope}%
\pgfpathrectangle{\pgfqpoint{0.050000in}{0.050000in}}{\pgfqpoint{2.081932in}{2.081932in}}%
\pgfusepath{clip}%
\pgfsetbuttcap%
\pgfsetroundjoin%
\definecolor{currentfill}{rgb}{0.120638,0.625828,0.533488}%
\pgfsetfillcolor{currentfill}%
\pgfsetlinewidth{0.000000pt}%
\definecolor{currentstroke}{rgb}{0.000000,0.000000,0.000000}%
\pgfsetstrokecolor{currentstroke}%
\pgfsetdash{}{0pt}%
\pgfpathmoveto{\pgfqpoint{1.268658in}{1.073549in}}%
\pgfpathlineto{\pgfqpoint{1.266269in}{1.066763in}}%
\pgfpathlineto{\pgfqpoint{1.264451in}{1.059556in}}%
\pgfpathlineto{\pgfqpoint{1.263210in}{1.051956in}}%
\pgfpathlineto{\pgfqpoint{1.262550in}{1.043996in}}%
\pgfpathlineto{\pgfqpoint{1.262473in}{1.035705in}}%
\pgfpathlineto{\pgfqpoint{1.270640in}{1.041260in}}%
\pgfpathlineto{\pgfqpoint{1.278181in}{1.047108in}}%
\pgfpathlineto{\pgfqpoint{1.285066in}{1.053222in}}%
\pgfpathlineto{\pgfqpoint{1.291268in}{1.059578in}}%
\pgfpathlineto{\pgfqpoint{1.296763in}{1.066149in}}%
\pgfpathlineto{\pgfqpoint{1.296859in}{1.074425in}}%
\pgfpathlineto{\pgfqpoint{1.297673in}{1.082496in}}%
\pgfpathlineto{\pgfqpoint{1.299205in}{1.090329in}}%
\pgfpathlineto{\pgfqpoint{1.301450in}{1.097893in}}%
\pgfpathlineto{\pgfqpoint{1.304399in}{1.105156in}}%
\pgfpathlineto{\pgfqpoint{1.298674in}{1.098335in}}%
\pgfpathlineto{\pgfqpoint{1.292211in}{1.091737in}}%
\pgfpathlineto{\pgfqpoint{1.285035in}{1.085389in}}%
\pgfpathlineto{\pgfqpoint{1.277173in}{1.079317in}}%
\pgfpathlineto{\pgfqpoint{1.268658in}{1.073549in}}%
\pgfpathclose%
\pgfusepath{fill}%
\end{pgfscope}%
\begin{pgfscope}%
\pgfpathrectangle{\pgfqpoint{0.050000in}{0.050000in}}{\pgfqpoint{2.081932in}{2.081932in}}%
\pgfusepath{clip}%
\pgfsetbuttcap%
\pgfsetroundjoin%
\definecolor{currentfill}{rgb}{0.268510,0.009605,0.335427}%
\pgfsetfillcolor{currentfill}%
\pgfsetlinewidth{0.000000pt}%
\definecolor{currentstroke}{rgb}{0.000000,0.000000,0.000000}%
\pgfsetstrokecolor{currentstroke}%
\pgfsetdash{}{0pt}%
\pgfpathmoveto{\pgfqpoint{1.413026in}{0.825156in}}%
\pgfpathlineto{\pgfqpoint{1.422303in}{0.820597in}}%
\pgfpathlineto{\pgfqpoint{1.431538in}{0.816586in}}%
\pgfpathlineto{\pgfqpoint{1.440694in}{0.813140in}}%
\pgfpathlineto{\pgfqpoint{1.449734in}{0.810273in}}%
\pgfpathlineto{\pgfqpoint{1.458623in}{0.807998in}}%
\pgfpathlineto{\pgfqpoint{1.477639in}{0.821508in}}%
\pgfpathlineto{\pgfqpoint{1.495121in}{0.835700in}}%
\pgfpathlineto{\pgfqpoint{1.511003in}{0.850512in}}%
\pgfpathlineto{\pgfqpoint{1.525225in}{0.865877in}}%
\pgfpathlineto{\pgfqpoint{1.537736in}{0.881728in}}%
\pgfpathlineto{\pgfqpoint{1.526869in}{0.882069in}}%
\pgfpathlineto{\pgfqpoint{1.515811in}{0.882965in}}%
\pgfpathlineto{\pgfqpoint{1.504607in}{0.884412in}}%
\pgfpathlineto{\pgfqpoint{1.493300in}{0.886402in}}%
\pgfpathlineto{\pgfqpoint{1.481936in}{0.888928in}}%
\pgfpathlineto{\pgfqpoint{1.471004in}{0.875206in}}%
\pgfpathlineto{\pgfqpoint{1.458599in}{0.861910in}}%
\pgfpathlineto{\pgfqpoint{1.444765in}{0.849100in}}%
\pgfpathlineto{\pgfqpoint{1.429554in}{0.836830in}}%
\pgfpathlineto{\pgfqpoint{1.413026in}{0.825156in}}%
\pgfpathclose%
\pgfusepath{fill}%
\end{pgfscope}%
\begin{pgfscope}%
\pgfpathrectangle{\pgfqpoint{0.050000in}{0.050000in}}{\pgfqpoint{2.081932in}{2.081932in}}%
\pgfusepath{clip}%
\pgfsetbuttcap%
\pgfsetroundjoin%
\definecolor{currentfill}{rgb}{0.227802,0.326594,0.546532}%
\pgfsetfillcolor{currentfill}%
\pgfsetlinewidth{0.000000pt}%
\definecolor{currentstroke}{rgb}{0.000000,0.000000,0.000000}%
\pgfsetstrokecolor{currentstroke}%
\pgfsetdash{}{0pt}%
\pgfpathmoveto{\pgfqpoint{0.983684in}{0.981222in}}%
\pgfpathlineto{\pgfqpoint{0.980704in}{0.971460in}}%
\pgfpathlineto{\pgfqpoint{0.977248in}{0.961583in}}%
\pgfpathlineto{\pgfqpoint{0.973327in}{0.951628in}}%
\pgfpathlineto{\pgfqpoint{0.968959in}{0.941635in}}%
\pgfpathlineto{\pgfqpoint{0.964159in}{0.931641in}}%
\pgfpathlineto{\pgfqpoint{0.975865in}{0.925946in}}%
\pgfpathlineto{\pgfqpoint{0.988171in}{0.920705in}}%
\pgfpathlineto{\pgfqpoint{1.001027in}{0.915943in}}%
\pgfpathlineto{\pgfqpoint{1.014381in}{0.911680in}}%
\pgfpathlineto{\pgfqpoint{1.017633in}{0.922316in}}%
\pgfpathlineto{\pgfqpoint{1.020592in}{0.932895in}}%
\pgfpathlineto{\pgfqpoint{1.023248in}{0.943376in}}%
\pgfpathlineto{\pgfqpoint{1.025589in}{0.953719in}}%
\pgfpathlineto{\pgfqpoint{1.027607in}{0.963883in}}%
\pgfpathlineto{\pgfqpoint{1.015933in}{0.967586in}}%
\pgfpathlineto{\pgfqpoint{1.004690in}{0.971722in}}%
\pgfpathlineto{\pgfqpoint{0.993926in}{0.976274in}}%
\pgfpathlineto{\pgfqpoint{0.983684in}{0.981222in}}%
\pgfpathclose%
\pgfusepath{fill}%
\end{pgfscope}%
\begin{pgfscope}%
\pgfpathrectangle{\pgfqpoint{0.050000in}{0.050000in}}{\pgfqpoint{2.081932in}{2.081932in}}%
\pgfusepath{clip}%
\pgfsetbuttcap%
\pgfsetroundjoin%
\definecolor{currentfill}{rgb}{0.278012,0.180367,0.486697}%
\pgfsetfillcolor{currentfill}%
\pgfsetlinewidth{0.000000pt}%
\definecolor{currentstroke}{rgb}{0.000000,0.000000,0.000000}%
\pgfsetstrokecolor{currentstroke}%
\pgfsetdash{}{0pt}%
\pgfpathmoveto{\pgfqpoint{0.964159in}{0.931641in}}%
\pgfpathlineto{\pgfqpoint{0.958947in}{0.921685in}}%
\pgfpathlineto{\pgfqpoint{0.953342in}{0.911806in}}%
\pgfpathlineto{\pgfqpoint{0.947366in}{0.902042in}}%
\pgfpathlineto{\pgfqpoint{0.941041in}{0.892431in}}%
\pgfpathlineto{\pgfqpoint{0.934394in}{0.883010in}}%
\pgfpathlineto{\pgfqpoint{0.948325in}{0.876180in}}%
\pgfpathlineto{\pgfqpoint{0.962977in}{0.869895in}}%
\pgfpathlineto{\pgfqpoint{0.978291in}{0.864182in}}%
\pgfpathlineto{\pgfqpoint{0.994202in}{0.859066in}}%
\pgfpathlineto{\pgfqpoint{0.998710in}{0.869377in}}%
\pgfpathlineto{\pgfqpoint{1.002998in}{0.879834in}}%
\pgfpathlineto{\pgfqpoint{1.007049in}{0.890398in}}%
\pgfpathlineto{\pgfqpoint{1.010848in}{0.901027in}}%
\pgfpathlineto{\pgfqpoint{1.014381in}{0.911680in}}%
\pgfpathlineto{\pgfqpoint{1.001027in}{0.915943in}}%
\pgfpathlineto{\pgfqpoint{0.988171in}{0.920705in}}%
\pgfpathlineto{\pgfqpoint{0.975865in}{0.925946in}}%
\pgfpathlineto{\pgfqpoint{0.964159in}{0.931641in}}%
\pgfpathclose%
\pgfusepath{fill}%
\end{pgfscope}%
\begin{pgfscope}%
\pgfpathrectangle{\pgfqpoint{0.050000in}{0.050000in}}{\pgfqpoint{2.081932in}{2.081932in}}%
\pgfusepath{clip}%
\pgfsetbuttcap%
\pgfsetroundjoin%
\definecolor{currentfill}{rgb}{0.296479,0.761561,0.424223}%
\pgfsetfillcolor{currentfill}%
\pgfsetlinewidth{0.000000pt}%
\definecolor{currentstroke}{rgb}{0.000000,0.000000,0.000000}%
\pgfsetstrokecolor{currentstroke}%
\pgfsetdash{}{0pt}%
\pgfpathmoveto{\pgfqpoint{0.893946in}{1.162771in}}%
\pgfpathlineto{\pgfqpoint{0.900637in}{1.156645in}}%
\pgfpathlineto{\pgfqpoint{0.906662in}{1.150165in}}%
\pgfpathlineto{\pgfqpoint{0.911996in}{1.143357in}}%
\pgfpathlineto{\pgfqpoint{0.916620in}{1.136249in}}%
\pgfpathlineto{\pgfqpoint{0.920516in}{1.128870in}}%
\pgfpathlineto{\pgfqpoint{0.923669in}{1.121521in}}%
\pgfpathlineto{\pgfqpoint{0.927619in}{1.114298in}}%
\pgfpathlineto{\pgfqpoint{0.932352in}{1.107231in}}%
\pgfpathlineto{\pgfqpoint{0.937851in}{1.100349in}}%
\pgfpathlineto{\pgfqpoint{0.944094in}{1.093681in}}%
\pgfpathlineto{\pgfqpoint{0.940645in}{1.100399in}}%
\pgfpathlineto{\pgfqpoint{0.936552in}{1.106717in}}%
\pgfpathlineto{\pgfqpoint{0.931828in}{1.112609in}}%
\pgfpathlineto{\pgfqpoint{0.926492in}{1.118050in}}%
\pgfpathlineto{\pgfqpoint{0.920564in}{1.123018in}}%
\pgfpathlineto{\pgfqpoint{0.913504in}{1.130554in}}%
\pgfpathlineto{\pgfqpoint{0.907291in}{1.138331in}}%
\pgfpathlineto{\pgfqpoint{0.901948in}{1.146314in}}%
\pgfpathlineto{\pgfqpoint{0.897495in}{1.154473in}}%
\pgfpathlineto{\pgfqpoint{0.893946in}{1.162771in}}%
\pgfpathclose%
\pgfusepath{fill}%
\end{pgfscope}%
\begin{pgfscope}%
\pgfpathrectangle{\pgfqpoint{0.050000in}{0.050000in}}{\pgfqpoint{2.081932in}{2.081932in}}%
\pgfusepath{clip}%
\pgfsetbuttcap%
\pgfsetroundjoin%
\definecolor{currentfill}{rgb}{0.162142,0.474838,0.558140}%
\pgfsetfillcolor{currentfill}%
\pgfsetlinewidth{0.000000pt}%
\definecolor{currentstroke}{rgb}{0.000000,0.000000,0.000000}%
\pgfsetstrokecolor{currentstroke}%
\pgfsetdash{}{0pt}%
\pgfpathmoveto{\pgfqpoint{1.213472in}{1.013087in}}%
\pgfpathlineto{\pgfqpoint{1.213806in}{1.004396in}}%
\pgfpathlineto{\pgfqpoint{1.214521in}{0.995349in}}%
\pgfpathlineto{\pgfqpoint{1.215615in}{0.985982in}}%
\pgfpathlineto{\pgfqpoint{1.217084in}{0.976331in}}%
\pgfpathlineto{\pgfqpoint{1.218921in}{0.966434in}}%
\pgfpathlineto{\pgfqpoint{1.230298in}{0.970443in}}%
\pgfpathlineto{\pgfqpoint{1.241211in}{0.974873in}}%
\pgfpathlineto{\pgfqpoint{1.251615in}{0.979705in}}%
\pgfpathlineto{\pgfqpoint{1.261468in}{0.984918in}}%
\pgfpathlineto{\pgfqpoint{1.270728in}{0.990490in}}%
\pgfpathlineto{\pgfqpoint{1.267945in}{0.999917in}}%
\pgfpathlineto{\pgfqpoint{1.265720in}{1.009188in}}%
\pgfpathlineto{\pgfqpoint{1.264062in}{1.018267in}}%
\pgfpathlineto{\pgfqpoint{1.262978in}{1.027117in}}%
\pgfpathlineto{\pgfqpoint{1.262473in}{1.035705in}}%
\pgfpathlineto{\pgfqpoint{1.253712in}{1.030465in}}%
\pgfpathlineto{\pgfqpoint{1.244392in}{1.025563in}}%
\pgfpathlineto{\pgfqpoint{1.234552in}{1.021020in}}%
\pgfpathlineto{\pgfqpoint{1.224231in}{1.016855in}}%
\pgfpathlineto{\pgfqpoint{1.213472in}{1.013087in}}%
\pgfpathclose%
\pgfusepath{fill}%
\end{pgfscope}%
\begin{pgfscope}%
\pgfpathrectangle{\pgfqpoint{0.050000in}{0.050000in}}{\pgfqpoint{2.081932in}{2.081932in}}%
\pgfusepath{clip}%
\pgfsetbuttcap%
\pgfsetroundjoin%
\definecolor{currentfill}{rgb}{0.876168,0.891125,0.095250}%
\pgfsetfillcolor{currentfill}%
\pgfsetlinewidth{0.000000pt}%
\definecolor{currentstroke}{rgb}{0.000000,0.000000,0.000000}%
\pgfsetstrokecolor{currentstroke}%
\pgfsetdash{}{0pt}%
\pgfpathmoveto{\pgfqpoint{0.616480in}{1.251820in}}%
\pgfpathlineto{\pgfqpoint{0.626595in}{1.255919in}}%
\pgfpathlineto{\pgfqpoint{0.637144in}{1.259549in}}%
\pgfpathlineto{\pgfqpoint{0.648084in}{1.262694in}}%
\pgfpathlineto{\pgfqpoint{0.659370in}{1.265341in}}%
\pgfpathlineto{\pgfqpoint{0.670957in}{1.267478in}}%
\pgfpathlineto{\pgfqpoint{0.668402in}{1.250991in}}%
\pgfpathlineto{\pgfqpoint{0.667653in}{1.234409in}}%
\pgfpathlineto{\pgfqpoint{0.668726in}{1.217799in}}%
\pgfpathlineto{\pgfqpoint{0.671626in}{1.201228in}}%
\pgfpathlineto{\pgfqpoint{0.676354in}{1.184764in}}%
\pgfpathlineto{\pgfqpoint{0.664786in}{1.180444in}}%
\pgfpathlineto{\pgfqpoint{0.653512in}{1.175661in}}%
\pgfpathlineto{\pgfqpoint{0.642578in}{1.170438in}}%
\pgfpathlineto{\pgfqpoint{0.632030in}{1.164795in}}%
\pgfpathlineto{\pgfqpoint{0.621910in}{1.158755in}}%
\pgfpathlineto{\pgfqpoint{0.616727in}{1.177298in}}%
\pgfpathlineto{\pgfqpoint{0.613598in}{1.195952in}}%
\pgfpathlineto{\pgfqpoint{0.612521in}{1.214641in}}%
\pgfpathlineto{\pgfqpoint{0.613488in}{1.233288in}}%
\pgfpathlineto{\pgfqpoint{0.616480in}{1.251820in}}%
\pgfpathclose%
\pgfusepath{fill}%
\end{pgfscope}%
\begin{pgfscope}%
\pgfpathrectangle{\pgfqpoint{0.050000in}{0.050000in}}{\pgfqpoint{2.081932in}{2.081932in}}%
\pgfusepath{clip}%
\pgfsetbuttcap%
\pgfsetroundjoin%
\definecolor{currentfill}{rgb}{0.278791,0.062145,0.386592}%
\pgfsetfillcolor{currentfill}%
\pgfsetlinewidth{0.000000pt}%
\definecolor{currentstroke}{rgb}{0.000000,0.000000,0.000000}%
\pgfsetstrokecolor{currentstroke}%
\pgfsetdash{}{0pt}%
\pgfpathmoveto{\pgfqpoint{1.255301in}{0.862591in}}%
\pgfpathlineto{\pgfqpoint{1.260432in}{0.852604in}}%
\pgfpathlineto{\pgfqpoint{1.265765in}{0.842847in}}%
\pgfpathlineto{\pgfqpoint{1.271278in}{0.833360in}}%
\pgfpathlineto{\pgfqpoint{1.276952in}{0.824179in}}%
\pgfpathlineto{\pgfqpoint{1.282765in}{0.815340in}}%
\pgfpathlineto{\pgfqpoint{1.301368in}{0.822031in}}%
\pgfpathlineto{\pgfqpoint{1.319192in}{0.829420in}}%
\pgfpathlineto{\pgfqpoint{1.336163in}{0.837475in}}%
\pgfpathlineto{\pgfqpoint{1.352211in}{0.846159in}}%
\pgfpathlineto{\pgfqpoint{1.367271in}{0.855433in}}%
\pgfpathlineto{\pgfqpoint{1.358496in}{0.862814in}}%
\pgfpathlineto{\pgfqpoint{1.349928in}{0.870571in}}%
\pgfpathlineto{\pgfqpoint{1.341598in}{0.878675in}}%
\pgfpathlineto{\pgfqpoint{1.333541in}{0.887094in}}%
\pgfpathlineto{\pgfqpoint{1.325785in}{0.895795in}}%
\pgfpathlineto{\pgfqpoint{1.313208in}{0.888110in}}%
\pgfpathlineto{\pgfqpoint{1.299815in}{0.880916in}}%
\pgfpathlineto{\pgfqpoint{1.285660in}{0.874246in}}%
\pgfpathlineto{\pgfqpoint{1.270802in}{0.868129in}}%
\pgfpathlineto{\pgfqpoint{1.255301in}{0.862591in}}%
\pgfpathclose%
\pgfusepath{fill}%
\end{pgfscope}%
\begin{pgfscope}%
\pgfpathrectangle{\pgfqpoint{0.050000in}{0.050000in}}{\pgfqpoint{2.081932in}{2.081932in}}%
\pgfusepath{clip}%
\pgfsetbuttcap%
\pgfsetroundjoin%
\definecolor{currentfill}{rgb}{0.993248,0.906157,0.143936}%
\pgfsetfillcolor{currentfill}%
\pgfsetlinewidth{0.000000pt}%
\definecolor{currentstroke}{rgb}{0.000000,0.000000,0.000000}%
\pgfsetstrokecolor{currentstroke}%
\pgfsetdash{}{0pt}%
\pgfpathmoveto{\pgfqpoint{0.731694in}{1.270280in}}%
\pgfpathlineto{\pgfqpoint{0.744052in}{1.269245in}}%
\pgfpathlineto{\pgfqpoint{0.756361in}{1.267683in}}%
\pgfpathlineto{\pgfqpoint{0.768571in}{1.265600in}}%
\pgfpathlineto{\pgfqpoint{0.780631in}{1.263005in}}%
\pgfpathlineto{\pgfqpoint{0.792491in}{1.259909in}}%
\pgfpathlineto{\pgfqpoint{0.790812in}{1.247888in}}%
\pgfpathlineto{\pgfqpoint{0.790450in}{1.235811in}}%
\pgfpathlineto{\pgfqpoint{0.791415in}{1.223727in}}%
\pgfpathlineto{\pgfqpoint{0.793707in}{1.211685in}}%
\pgfpathlineto{\pgfqpoint{0.797324in}{1.199735in}}%
\pgfpathlineto{\pgfqpoint{0.785548in}{1.200653in}}%
\pgfpathlineto{\pgfqpoint{0.773568in}{1.201030in}}%
\pgfpathlineto{\pgfqpoint{0.761432in}{1.200863in}}%
\pgfpathlineto{\pgfqpoint{0.749190in}{1.200151in}}%
\pgfpathlineto{\pgfqpoint{0.736893in}{1.198899in}}%
\pgfpathlineto{\pgfqpoint{0.732705in}{1.213091in}}%
\pgfpathlineto{\pgfqpoint{0.730091in}{1.227383in}}%
\pgfpathlineto{\pgfqpoint{0.729056in}{1.241718in}}%
\pgfpathlineto{\pgfqpoint{0.729593in}{1.256036in}}%
\pgfpathlineto{\pgfqpoint{0.731694in}{1.270280in}}%
\pgfpathclose%
\pgfusepath{fill}%
\end{pgfscope}%
\begin{pgfscope}%
\pgfpathrectangle{\pgfqpoint{0.050000in}{0.050000in}}{\pgfqpoint{2.081932in}{2.081932in}}%
\pgfusepath{clip}%
\pgfsetbuttcap%
\pgfsetroundjoin%
\definecolor{currentfill}{rgb}{0.120638,0.625828,0.533488}%
\pgfsetfillcolor{currentfill}%
\pgfsetlinewidth{0.000000pt}%
\definecolor{currentstroke}{rgb}{0.000000,0.000000,0.000000}%
\pgfsetstrokecolor{currentstroke}%
\pgfsetdash{}{0pt}%
\pgfpathmoveto{\pgfqpoint{0.944094in}{1.093681in}}%
\pgfpathlineto{\pgfqpoint{0.946885in}{1.086591in}}%
\pgfpathlineto{\pgfqpoint{0.949010in}{1.079156in}}%
\pgfpathlineto{\pgfqpoint{0.950459in}{1.071408in}}%
\pgfpathlineto{\pgfqpoint{0.951230in}{1.063377in}}%
\pgfpathlineto{\pgfqpoint{0.951320in}{1.055095in}}%
\pgfpathlineto{\pgfqpoint{0.958002in}{1.048906in}}%
\pgfpathlineto{\pgfqpoint{0.965348in}{1.042976in}}%
\pgfpathlineto{\pgfqpoint{0.973330in}{1.037331in}}%
\pgfpathlineto{\pgfqpoint{0.981916in}{1.031994in}}%
\pgfpathlineto{\pgfqpoint{0.991071in}{1.026989in}}%
\pgfpathlineto{\pgfqpoint{0.991002in}{1.035284in}}%
\pgfpathlineto{\pgfqpoint{0.990412in}{1.043213in}}%
\pgfpathlineto{\pgfqpoint{0.989302in}{1.050745in}}%
\pgfpathlineto{\pgfqpoint{0.987675in}{1.057850in}}%
\pgfpathlineto{\pgfqpoint{0.985537in}{1.064498in}}%
\pgfpathlineto{\pgfqpoint{0.975990in}{1.069696in}}%
\pgfpathlineto{\pgfqpoint{0.967038in}{1.075237in}}%
\pgfpathlineto{\pgfqpoint{0.958716in}{1.081099in}}%
\pgfpathlineto{\pgfqpoint{0.951058in}{1.087255in}}%
\pgfpathlineto{\pgfqpoint{0.944094in}{1.093681in}}%
\pgfpathclose%
\pgfusepath{fill}%
\end{pgfscope}%
\begin{pgfscope}%
\pgfpathrectangle{\pgfqpoint{0.050000in}{0.050000in}}{\pgfqpoint{2.081932in}{2.081932in}}%
\pgfusepath{clip}%
\pgfsetbuttcap%
\pgfsetroundjoin%
\definecolor{currentfill}{rgb}{0.993248,0.906157,0.143936}%
\pgfsetfillcolor{currentfill}%
\pgfsetlinewidth{0.000000pt}%
\definecolor{currentstroke}{rgb}{0.000000,0.000000,0.000000}%
\pgfsetstrokecolor{currentstroke}%
\pgfsetdash{}{0pt}%
\pgfpathmoveto{\pgfqpoint{0.670957in}{1.267478in}}%
\pgfpathlineto{\pgfqpoint{0.682796in}{1.269097in}}%
\pgfpathlineto{\pgfqpoint{0.694839in}{1.270191in}}%
\pgfpathlineto{\pgfqpoint{0.707037in}{1.270753in}}%
\pgfpathlineto{\pgfqpoint{0.719339in}{1.270783in}}%
\pgfpathlineto{\pgfqpoint{0.731694in}{1.270280in}}%
\pgfpathlineto{\pgfqpoint{0.729593in}{1.256036in}}%
\pgfpathlineto{\pgfqpoint{0.729056in}{1.241718in}}%
\pgfpathlineto{\pgfqpoint{0.730091in}{1.227383in}}%
\pgfpathlineto{\pgfqpoint{0.732705in}{1.213091in}}%
\pgfpathlineto{\pgfqpoint{0.736893in}{1.198899in}}%
\pgfpathlineto{\pgfqpoint{0.724592in}{1.197110in}}%
\pgfpathlineto{\pgfqpoint{0.712337in}{1.194792in}}%
\pgfpathlineto{\pgfqpoint{0.700178in}{1.191953in}}%
\pgfpathlineto{\pgfqpoint{0.688168in}{1.188606in}}%
\pgfpathlineto{\pgfqpoint{0.676354in}{1.184764in}}%
\pgfpathlineto{\pgfqpoint{0.671626in}{1.201228in}}%
\pgfpathlineto{\pgfqpoint{0.668726in}{1.217799in}}%
\pgfpathlineto{\pgfqpoint{0.667653in}{1.234409in}}%
\pgfpathlineto{\pgfqpoint{0.668402in}{1.250991in}}%
\pgfpathlineto{\pgfqpoint{0.670957in}{1.267478in}}%
\pgfpathclose%
\pgfusepath{fill}%
\end{pgfscope}%
\begin{pgfscope}%
\pgfpathrectangle{\pgfqpoint{0.050000in}{0.050000in}}{\pgfqpoint{2.081932in}{2.081932in}}%
\pgfusepath{clip}%
\pgfsetbuttcap%
\pgfsetroundjoin%
\definecolor{currentfill}{rgb}{0.162142,0.474838,0.558140}%
\pgfsetfillcolor{currentfill}%
\pgfsetlinewidth{0.000000pt}%
\definecolor{currentstroke}{rgb}{0.000000,0.000000,0.000000}%
\pgfsetstrokecolor{currentstroke}%
\pgfsetdash{}{0pt}%
\pgfpathmoveto{\pgfqpoint{0.991071in}{1.026989in}}%
\pgfpathlineto{\pgfqpoint{0.990619in}{1.018362in}}%
\pgfpathlineto{\pgfqpoint{0.989649in}{1.009436in}}%
\pgfpathlineto{\pgfqpoint{0.988165in}{1.000247in}}%
\pgfpathlineto{\pgfqpoint{0.986174in}{0.990829in}}%
\pgfpathlineto{\pgfqpoint{0.983684in}{0.981222in}}%
\pgfpathlineto{\pgfqpoint{0.993926in}{0.976274in}}%
\pgfpathlineto{\pgfqpoint{1.004690in}{0.971722in}}%
\pgfpathlineto{\pgfqpoint{1.015933in}{0.967586in}}%
\pgfpathlineto{\pgfqpoint{1.027607in}{0.963883in}}%
\pgfpathlineto{\pgfqpoint{1.029293in}{0.973830in}}%
\pgfpathlineto{\pgfqpoint{1.030642in}{0.983521in}}%
\pgfpathlineto{\pgfqpoint{1.031646in}{0.992919in}}%
\pgfpathlineto{\pgfqpoint{1.032303in}{1.001986in}}%
\pgfpathlineto{\pgfqpoint{1.032609in}{1.010688in}}%
\pgfpathlineto{\pgfqpoint{1.021570in}{1.014169in}}%
\pgfpathlineto{\pgfqpoint{1.010939in}{1.018057in}}%
\pgfpathlineto{\pgfqpoint{1.000758in}{1.022337in}}%
\pgfpathlineto{\pgfqpoint{0.991071in}{1.026989in}}%
\pgfpathclose%
\pgfusepath{fill}%
\end{pgfscope}%
\begin{pgfscope}%
\pgfpathrectangle{\pgfqpoint{0.050000in}{0.050000in}}{\pgfqpoint{2.081932in}{2.081932in}}%
\pgfusepath{clip}%
\pgfsetbuttcap%
\pgfsetroundjoin%
\definecolor{currentfill}{rgb}{0.150476,0.504369,0.557430}%
\pgfsetfillcolor{currentfill}%
\pgfsetlinewidth{0.000000pt}%
\definecolor{currentstroke}{rgb}{0.000000,0.000000,0.000000}%
\pgfsetstrokecolor{currentstroke}%
\pgfsetdash{}{0pt}%
\pgfpathmoveto{\pgfqpoint{1.651273in}{0.939551in}}%
\pgfpathlineto{\pgfqpoint{1.654107in}{0.946933in}}%
\pgfpathlineto{\pgfqpoint{1.656225in}{0.954623in}}%
\pgfpathlineto{\pgfqpoint{1.657618in}{0.962590in}}%
\pgfpathlineto{\pgfqpoint{1.658278in}{0.970803in}}%
\pgfpathlineto{\pgfqpoint{1.671826in}{0.991520in}}%
\pgfpathlineto{\pgfqpoint{1.683052in}{1.012654in}}%
\pgfpathlineto{\pgfqpoint{1.691926in}{1.034113in}}%
\pgfpathlineto{\pgfqpoint{1.698429in}{1.055806in}}%
\pgfpathlineto{\pgfqpoint{1.702554in}{1.077640in}}%
\pgfpathlineto{\pgfqpoint{1.701848in}{1.069404in}}%
\pgfpathlineto{\pgfqpoint{1.700360in}{1.061266in}}%
\pgfpathlineto{\pgfqpoint{1.698097in}{1.053258in}}%
\pgfpathlineto{\pgfqpoint{1.695069in}{1.045413in}}%
\pgfpathlineto{\pgfqpoint{1.690977in}{1.023775in}}%
\pgfpathlineto{\pgfqpoint{1.684539in}{1.002279in}}%
\pgfpathlineto{\pgfqpoint{1.675761in}{0.981016in}}%
\pgfpathlineto{\pgfqpoint{1.664662in}{0.960076in}}%
\pgfpathlineto{\pgfqpoint{1.651273in}{0.939551in}}%
\pgfpathclose%
\pgfusepath{fill}%
\end{pgfscope}%
\begin{pgfscope}%
\pgfpathrectangle{\pgfqpoint{0.050000in}{0.050000in}}{\pgfqpoint{2.081932in}{2.081932in}}%
\pgfusepath{clip}%
\pgfsetbuttcap%
\pgfsetroundjoin%
\definecolor{currentfill}{rgb}{0.606045,0.850733,0.236712}%
\pgfsetfillcolor{currentfill}%
\pgfsetlinewidth{0.000000pt}%
\definecolor{currentstroke}{rgb}{0.000000,0.000000,0.000000}%
\pgfsetstrokecolor{currentstroke}%
\pgfsetdash{}{0pt}%
\pgfpathmoveto{\pgfqpoint{1.369016in}{1.155485in}}%
\pgfpathlineto{\pgfqpoint{1.360038in}{1.152609in}}%
\pgfpathlineto{\pgfqpoint{1.351540in}{1.149206in}}%
\pgfpathlineto{\pgfqpoint{1.343557in}{1.145289in}}%
\pgfpathlineto{\pgfqpoint{1.336119in}{1.140877in}}%
\pgfpathlineto{\pgfqpoint{1.329256in}{1.135987in}}%
\pgfpathlineto{\pgfqpoint{1.334864in}{1.143912in}}%
\pgfpathlineto{\pgfqpoint{1.339587in}{1.152022in}}%
\pgfpathlineto{\pgfqpoint{1.343410in}{1.160282in}}%
\pgfpathlineto{\pgfqpoint{1.346319in}{1.168658in}}%
\pgfpathlineto{\pgfqpoint{1.348305in}{1.177117in}}%
\pgfpathlineto{\pgfqpoint{1.355751in}{1.183318in}}%
\pgfpathlineto{\pgfqpoint{1.363818in}{1.189154in}}%
\pgfpathlineto{\pgfqpoint{1.372474in}{1.194601in}}%
\pgfpathlineto{\pgfqpoint{1.381684in}{1.199635in}}%
\pgfpathlineto{\pgfqpoint{1.391411in}{1.204236in}}%
\pgfpathlineto{\pgfqpoint{1.389103in}{1.194218in}}%
\pgfpathlineto{\pgfqpoint{1.385698in}{1.184293in}}%
\pgfpathlineto{\pgfqpoint{1.381205in}{1.174502in}}%
\pgfpathlineto{\pgfqpoint{1.375638in}{1.164886in}}%
\pgfpathlineto{\pgfqpoint{1.369016in}{1.155485in}}%
\pgfpathclose%
\pgfusepath{fill}%
\end{pgfscope}%
\begin{pgfscope}%
\pgfpathrectangle{\pgfqpoint{0.050000in}{0.050000in}}{\pgfqpoint{2.081932in}{2.081932in}}%
\pgfusepath{clip}%
\pgfsetbuttcap%
\pgfsetroundjoin%
\definecolor{currentfill}{rgb}{0.227802,0.326594,0.546532}%
\pgfsetfillcolor{currentfill}%
\pgfsetlinewidth{0.000000pt}%
\definecolor{currentstroke}{rgb}{0.000000,0.000000,0.000000}%
\pgfsetstrokecolor{currentstroke}%
\pgfsetdash{}{0pt}%
\pgfpathmoveto{\pgfqpoint{1.156772in}{0.953254in}}%
\pgfpathlineto{\pgfqpoint{1.157602in}{0.942842in}}%
\pgfpathlineto{\pgfqpoint{1.158565in}{0.932213in}}%
\pgfpathlineto{\pgfqpoint{1.159657in}{0.921408in}}%
\pgfpathlineto{\pgfqpoint{1.160875in}{0.910469in}}%
\pgfpathlineto{\pgfqpoint{1.162213in}{0.899437in}}%
\pgfpathlineto{\pgfqpoint{1.177046in}{0.901371in}}%
\pgfpathlineto{\pgfqpoint{1.191635in}{0.903869in}}%
\pgfpathlineto{\pgfqpoint{1.205917in}{0.906919in}}%
\pgfpathlineto{\pgfqpoint{1.219834in}{0.910507in}}%
\pgfpathlineto{\pgfqpoint{1.233326in}{0.914618in}}%
\pgfpathlineto{\pgfqpoint{1.229784in}{0.925159in}}%
\pgfpathlineto{\pgfqpoint{1.226561in}{0.935652in}}%
\pgfpathlineto{\pgfqpoint{1.223668in}{0.946055in}}%
\pgfpathlineto{\pgfqpoint{1.221118in}{0.956329in}}%
\pgfpathlineto{\pgfqpoint{1.218921in}{0.966434in}}%
\pgfpathlineto{\pgfqpoint{1.207125in}{0.962865in}}%
\pgfpathlineto{\pgfqpoint{1.194962in}{0.959749in}}%
\pgfpathlineto{\pgfqpoint{1.182480in}{0.957102in}}%
\pgfpathlineto{\pgfqpoint{1.169732in}{0.954933in}}%
\pgfpathlineto{\pgfqpoint{1.156772in}{0.953254in}}%
\pgfpathclose%
\pgfusepath{fill}%
\end{pgfscope}%
\begin{pgfscope}%
\pgfpathrectangle{\pgfqpoint{0.050000in}{0.050000in}}{\pgfqpoint{2.081932in}{2.081932in}}%
\pgfusepath{clip}%
\pgfsetbuttcap%
\pgfsetroundjoin%
\definecolor{currentfill}{rgb}{0.278791,0.062145,0.386592}%
\pgfsetfillcolor{currentfill}%
\pgfsetlinewidth{0.000000pt}%
\definecolor{currentstroke}{rgb}{0.000000,0.000000,0.000000}%
\pgfsetstrokecolor{currentstroke}%
\pgfsetdash{}{0pt}%
\pgfpathmoveto{\pgfqpoint{0.934394in}{0.883010in}}%
\pgfpathlineto{\pgfqpoint{0.927447in}{0.873815in}}%
\pgfpathlineto{\pgfqpoint{0.920229in}{0.864882in}}%
\pgfpathlineto{\pgfqpoint{0.912767in}{0.856247in}}%
\pgfpathlineto{\pgfqpoint{0.905090in}{0.847943in}}%
\pgfpathlineto{\pgfqpoint{0.897227in}{0.840003in}}%
\pgfpathlineto{\pgfqpoint{0.913927in}{0.831756in}}%
\pgfpathlineto{\pgfqpoint{0.931502in}{0.824164in}}%
\pgfpathlineto{\pgfqpoint{0.949878in}{0.817262in}}%
\pgfpathlineto{\pgfqpoint{0.968980in}{0.811079in}}%
\pgfpathlineto{\pgfqpoint{0.974318in}{0.820074in}}%
\pgfpathlineto{\pgfqpoint{0.979529in}{0.829407in}}%
\pgfpathlineto{\pgfqpoint{0.984593in}{0.839042in}}%
\pgfpathlineto{\pgfqpoint{0.989490in}{0.848941in}}%
\pgfpathlineto{\pgfqpoint{0.994202in}{0.859066in}}%
\pgfpathlineto{\pgfqpoint{0.978291in}{0.864182in}}%
\pgfpathlineto{\pgfqpoint{0.962977in}{0.869895in}}%
\pgfpathlineto{\pgfqpoint{0.948325in}{0.876180in}}%
\pgfpathlineto{\pgfqpoint{0.934394in}{0.883010in}}%
\pgfpathclose%
\pgfusepath{fill}%
\end{pgfscope}%
\begin{pgfscope}%
\pgfpathrectangle{\pgfqpoint{0.050000in}{0.050000in}}{\pgfqpoint{2.081932in}{2.081932in}}%
\pgfusepath{clip}%
\pgfsetbuttcap%
\pgfsetroundjoin%
\definecolor{currentfill}{rgb}{0.227802,0.326594,0.546532}%
\pgfsetfillcolor{currentfill}%
\pgfsetlinewidth{0.000000pt}%
\definecolor{currentstroke}{rgb}{0.000000,0.000000,0.000000}%
\pgfsetstrokecolor{currentstroke}%
\pgfsetdash{}{0pt}%
\pgfpathmoveto{\pgfqpoint{1.027607in}{0.963883in}}%
\pgfpathlineto{\pgfqpoint{1.025589in}{0.953719in}}%
\pgfpathlineto{\pgfqpoint{1.023248in}{0.943376in}}%
\pgfpathlineto{\pgfqpoint{1.020592in}{0.932895in}}%
\pgfpathlineto{\pgfqpoint{1.017633in}{0.922316in}}%
\pgfpathlineto{\pgfqpoint{1.014381in}{0.911680in}}%
\pgfpathlineto{\pgfqpoint{1.028175in}{0.907934in}}%
\pgfpathlineto{\pgfqpoint{1.042354in}{0.904722in}}%
\pgfpathlineto{\pgfqpoint{1.056857in}{0.902058in}}%
\pgfpathlineto{\pgfqpoint{1.071623in}{0.899955in}}%
\pgfpathlineto{\pgfqpoint{1.086590in}{0.898420in}}%
\pgfpathlineto{\pgfqpoint{1.087603in}{0.909485in}}%
\pgfpathlineto{\pgfqpoint{1.088524in}{0.920455in}}%
\pgfpathlineto{\pgfqpoint{1.089351in}{0.931286in}}%
\pgfpathlineto{\pgfqpoint{1.090079in}{0.941939in}}%
\pgfpathlineto{\pgfqpoint{1.090708in}{0.952372in}}%
\pgfpathlineto{\pgfqpoint{1.077631in}{0.953704in}}%
\pgfpathlineto{\pgfqpoint{1.064729in}{0.955530in}}%
\pgfpathlineto{\pgfqpoint{1.052056in}{0.957842in}}%
\pgfpathlineto{\pgfqpoint{1.039665in}{0.960631in}}%
\pgfpathlineto{\pgfqpoint{1.027607in}{0.963883in}}%
\pgfpathclose%
\pgfusepath{fill}%
\end{pgfscope}%
\begin{pgfscope}%
\pgfpathrectangle{\pgfqpoint{0.050000in}{0.050000in}}{\pgfqpoint{2.081932in}{2.081932in}}%
\pgfusepath{clip}%
\pgfsetbuttcap%
\pgfsetroundjoin%
\definecolor{currentfill}{rgb}{0.278012,0.180367,0.486697}%
\pgfsetfillcolor{currentfill}%
\pgfsetlinewidth{0.000000pt}%
\definecolor{currentstroke}{rgb}{0.000000,0.000000,0.000000}%
\pgfsetstrokecolor{currentstroke}%
\pgfsetdash{}{0pt}%
\pgfpathmoveto{\pgfqpoint{1.162213in}{0.899437in}}%
\pgfpathlineto{\pgfqpoint{1.163666in}{0.888354in}}%
\pgfpathlineto{\pgfqpoint{1.165229in}{0.877264in}}%
\pgfpathlineto{\pgfqpoint{1.166896in}{0.866208in}}%
\pgfpathlineto{\pgfqpoint{1.168661in}{0.855229in}}%
\pgfpathlineto{\pgfqpoint{1.170517in}{0.844370in}}%
\pgfpathlineto{\pgfqpoint{1.188209in}{0.846692in}}%
\pgfpathlineto{\pgfqpoint{1.205606in}{0.849691in}}%
\pgfpathlineto{\pgfqpoint{1.222635in}{0.853352in}}%
\pgfpathlineto{\pgfqpoint{1.239223in}{0.857658in}}%
\pgfpathlineto{\pgfqpoint{1.255301in}{0.862591in}}%
\pgfpathlineto{\pgfqpoint{1.250392in}{0.872771in}}%
\pgfpathlineto{\pgfqpoint{1.245722in}{0.883104in}}%
\pgfpathlineto{\pgfqpoint{1.241310in}{0.893549in}}%
\pgfpathlineto{\pgfqpoint{1.237173in}{0.904067in}}%
\pgfpathlineto{\pgfqpoint{1.233326in}{0.914618in}}%
\pgfpathlineto{\pgfqpoint{1.219834in}{0.910507in}}%
\pgfpathlineto{\pgfqpoint{1.205917in}{0.906919in}}%
\pgfpathlineto{\pgfqpoint{1.191635in}{0.903869in}}%
\pgfpathlineto{\pgfqpoint{1.177046in}{0.901371in}}%
\pgfpathlineto{\pgfqpoint{1.162213in}{0.899437in}}%
\pgfpathclose%
\pgfusepath{fill}%
\end{pgfscope}%
\begin{pgfscope}%
\pgfpathrectangle{\pgfqpoint{0.050000in}{0.050000in}}{\pgfqpoint{2.081932in}{2.081932in}}%
\pgfusepath{clip}%
\pgfsetbuttcap%
\pgfsetroundjoin%
\definecolor{currentfill}{rgb}{0.278012,0.180367,0.486697}%
\pgfsetfillcolor{currentfill}%
\pgfsetlinewidth{0.000000pt}%
\definecolor{currentstroke}{rgb}{0.000000,0.000000,0.000000}%
\pgfsetstrokecolor{currentstroke}%
\pgfsetdash{}{0pt}%
\pgfpathmoveto{\pgfqpoint{1.014381in}{0.911680in}}%
\pgfpathlineto{\pgfqpoint{1.010848in}{0.901027in}}%
\pgfpathlineto{\pgfqpoint{1.007049in}{0.890398in}}%
\pgfpathlineto{\pgfqpoint{1.002998in}{0.879834in}}%
\pgfpathlineto{\pgfqpoint{0.998710in}{0.869377in}}%
\pgfpathlineto{\pgfqpoint{0.994202in}{0.859066in}}%
\pgfpathlineto{\pgfqpoint{1.010644in}{0.854570in}}%
\pgfpathlineto{\pgfqpoint{1.027547in}{0.850715in}}%
\pgfpathlineto{\pgfqpoint{1.044841in}{0.847517in}}%
\pgfpathlineto{\pgfqpoint{1.062452in}{0.844991in}}%
\pgfpathlineto{\pgfqpoint{1.080305in}{0.843149in}}%
\pgfpathlineto{\pgfqpoint{1.081710in}{0.854054in}}%
\pgfpathlineto{\pgfqpoint{1.083045in}{0.865076in}}%
\pgfpathlineto{\pgfqpoint{1.084307in}{0.876173in}}%
\pgfpathlineto{\pgfqpoint{1.085490in}{0.887302in}}%
\pgfpathlineto{\pgfqpoint{1.086590in}{0.898420in}}%
\pgfpathlineto{\pgfqpoint{1.071623in}{0.899955in}}%
\pgfpathlineto{\pgfqpoint{1.056857in}{0.902058in}}%
\pgfpathlineto{\pgfqpoint{1.042354in}{0.904722in}}%
\pgfpathlineto{\pgfqpoint{1.028175in}{0.907934in}}%
\pgfpathlineto{\pgfqpoint{1.014381in}{0.911680in}}%
\pgfpathclose%
\pgfusepath{fill}%
\end{pgfscope}%
\begin{pgfscope}%
\pgfpathrectangle{\pgfqpoint{0.050000in}{0.050000in}}{\pgfqpoint{2.081932in}{2.081932in}}%
\pgfusepath{clip}%
\pgfsetbuttcap%
\pgfsetroundjoin%
\definecolor{currentfill}{rgb}{0.268510,0.009605,0.335427}%
\pgfsetfillcolor{currentfill}%
\pgfsetlinewidth{0.000000pt}%
\definecolor{currentstroke}{rgb}{0.000000,0.000000,0.000000}%
\pgfsetstrokecolor{currentstroke}%
\pgfsetdash{}{0pt}%
\pgfpathmoveto{\pgfqpoint{0.775836in}{0.865830in}}%
\pgfpathlineto{\pgfqpoint{0.765051in}{0.862572in}}%
\pgfpathlineto{\pgfqpoint{0.754318in}{0.859853in}}%
\pgfpathlineto{\pgfqpoint{0.743680in}{0.857686in}}%
\pgfpathlineto{\pgfqpoint{0.733180in}{0.856080in}}%
\pgfpathlineto{\pgfqpoint{0.722859in}{0.855043in}}%
\pgfpathlineto{\pgfqpoint{0.738247in}{0.840059in}}%
\pgfpathlineto{\pgfqpoint{0.755253in}{0.825675in}}%
\pgfpathlineto{\pgfqpoint{0.773815in}{0.811955in}}%
\pgfpathlineto{\pgfqpoint{0.793859in}{0.798961in}}%
\pgfpathlineto{\pgfqpoint{0.815306in}{0.786753in}}%
\pgfpathlineto{\pgfqpoint{0.823282in}{0.789591in}}%
\pgfpathlineto{\pgfqpoint{0.831393in}{0.793032in}}%
\pgfpathlineto{\pgfqpoint{0.839606in}{0.797060in}}%
\pgfpathlineto{\pgfqpoint{0.847889in}{0.801659in}}%
\pgfpathlineto{\pgfqpoint{0.856209in}{0.806808in}}%
\pgfpathlineto{\pgfqpoint{0.837595in}{0.817350in}}%
\pgfpathlineto{\pgfqpoint{0.820184in}{0.828575in}}%
\pgfpathlineto{\pgfqpoint{0.804046in}{0.840432in}}%
\pgfpathlineto{\pgfqpoint{0.789245in}{0.852869in}}%
\pgfpathlineto{\pgfqpoint{0.775836in}{0.865830in}}%
\pgfpathclose%
\pgfusepath{fill}%
\end{pgfscope}%
\begin{pgfscope}%
\pgfpathrectangle{\pgfqpoint{0.050000in}{0.050000in}}{\pgfqpoint{2.081932in}{2.081932in}}%
\pgfusepath{clip}%
\pgfsetbuttcap%
\pgfsetroundjoin%
\definecolor{currentfill}{rgb}{0.227802,0.326594,0.546532}%
\pgfsetfillcolor{currentfill}%
\pgfsetlinewidth{0.000000pt}%
\definecolor{currentstroke}{rgb}{0.000000,0.000000,0.000000}%
\pgfsetstrokecolor{currentstroke}%
\pgfsetdash{}{0pt}%
\pgfpathmoveto{\pgfqpoint{1.090708in}{0.952372in}}%
\pgfpathlineto{\pgfqpoint{1.090079in}{0.941939in}}%
\pgfpathlineto{\pgfqpoint{1.089351in}{0.931286in}}%
\pgfpathlineto{\pgfqpoint{1.088524in}{0.920455in}}%
\pgfpathlineto{\pgfqpoint{1.087603in}{0.909485in}}%
\pgfpathlineto{\pgfqpoint{1.086590in}{0.898420in}}%
\pgfpathlineto{\pgfqpoint{1.101695in}{0.897462in}}%
\pgfpathlineto{\pgfqpoint{1.116874in}{0.897084in}}%
\pgfpathlineto{\pgfqpoint{1.132063in}{0.897288in}}%
\pgfpathlineto{\pgfqpoint{1.147197in}{0.898074in}}%
\pgfpathlineto{\pgfqpoint{1.162213in}{0.899437in}}%
\pgfpathlineto{\pgfqpoint{1.160875in}{0.910469in}}%
\pgfpathlineto{\pgfqpoint{1.159657in}{0.921408in}}%
\pgfpathlineto{\pgfqpoint{1.158565in}{0.932213in}}%
\pgfpathlineto{\pgfqpoint{1.157602in}{0.942842in}}%
\pgfpathlineto{\pgfqpoint{1.156772in}{0.953254in}}%
\pgfpathlineto{\pgfqpoint{1.143654in}{0.952071in}}%
\pgfpathlineto{\pgfqpoint{1.130433in}{0.951389in}}%
\pgfpathlineto{\pgfqpoint{1.117164in}{0.951212in}}%
\pgfpathlineto{\pgfqpoint{1.103904in}{0.951540in}}%
\pgfpathlineto{\pgfqpoint{1.090708in}{0.952372in}}%
\pgfpathclose%
\pgfusepath{fill}%
\end{pgfscope}%
\begin{pgfscope}%
\pgfpathrectangle{\pgfqpoint{0.050000in}{0.050000in}}{\pgfqpoint{2.081932in}{2.081932in}}%
\pgfusepath{clip}%
\pgfsetbuttcap%
\pgfsetroundjoin%
\definecolor{currentfill}{rgb}{0.162142,0.474838,0.558140}%
\pgfsetfillcolor{currentfill}%
\pgfsetlinewidth{0.000000pt}%
\definecolor{currentstroke}{rgb}{0.000000,0.000000,0.000000}%
\pgfsetstrokecolor{currentstroke}%
\pgfsetdash{}{0pt}%
\pgfpathmoveto{\pgfqpoint{1.154715in}{1.000697in}}%
\pgfpathlineto{\pgfqpoint{1.154841in}{0.991949in}}%
\pgfpathlineto{\pgfqpoint{1.155111in}{0.982795in}}%
\pgfpathlineto{\pgfqpoint{1.155524in}{0.973269in}}%
\pgfpathlineto{\pgfqpoint{1.156079in}{0.963410in}}%
\pgfpathlineto{\pgfqpoint{1.156772in}{0.953254in}}%
\pgfpathlineto{\pgfqpoint{1.169732in}{0.954933in}}%
\pgfpathlineto{\pgfqpoint{1.182480in}{0.957102in}}%
\pgfpathlineto{\pgfqpoint{1.194962in}{0.959749in}}%
\pgfpathlineto{\pgfqpoint{1.207125in}{0.962865in}}%
\pgfpathlineto{\pgfqpoint{1.218921in}{0.966434in}}%
\pgfpathlineto{\pgfqpoint{1.217084in}{0.976331in}}%
\pgfpathlineto{\pgfqpoint{1.215615in}{0.985982in}}%
\pgfpathlineto{\pgfqpoint{1.214521in}{0.995349in}}%
\pgfpathlineto{\pgfqpoint{1.213806in}{1.004396in}}%
\pgfpathlineto{\pgfqpoint{1.213472in}{1.013087in}}%
\pgfpathlineto{\pgfqpoint{1.202320in}{1.009731in}}%
\pgfpathlineto{\pgfqpoint{1.190819in}{1.006802in}}%
\pgfpathlineto{\pgfqpoint{1.179018in}{1.004314in}}%
\pgfpathlineto{\pgfqpoint{1.166967in}{1.002276in}}%
\pgfpathlineto{\pgfqpoint{1.154715in}{1.000697in}}%
\pgfpathclose%
\pgfusepath{fill}%
\end{pgfscope}%
\begin{pgfscope}%
\pgfpathrectangle{\pgfqpoint{0.050000in}{0.050000in}}{\pgfqpoint{2.081932in}{2.081932in}}%
\pgfusepath{clip}%
\pgfsetbuttcap%
\pgfsetroundjoin%
\definecolor{currentfill}{rgb}{0.206756,0.371758,0.553117}%
\pgfsetfillcolor{currentfill}%
\pgfsetlinewidth{0.000000pt}%
\definecolor{currentstroke}{rgb}{0.000000,0.000000,0.000000}%
\pgfsetstrokecolor{currentstroke}%
\pgfsetdash{}{0pt}%
\pgfpathmoveto{\pgfqpoint{0.577373in}{0.974470in}}%
\pgfpathlineto{\pgfqpoint{0.570848in}{0.980655in}}%
\pgfpathlineto{\pgfqpoint{0.564985in}{0.987195in}}%
\pgfpathlineto{\pgfqpoint{0.559808in}{0.994063in}}%
\pgfpathlineto{\pgfqpoint{0.555340in}{1.001232in}}%
\pgfpathlineto{\pgfqpoint{0.551598in}{1.008674in}}%
\pgfpathlineto{\pgfqpoint{0.559673in}{0.987332in}}%
\pgfpathlineto{\pgfqpoint{0.570075in}{0.966286in}}%
\pgfpathlineto{\pgfqpoint{0.582778in}{0.945628in}}%
\pgfpathlineto{\pgfqpoint{0.597745in}{0.925446in}}%
\pgfpathlineto{\pgfqpoint{0.614930in}{0.905832in}}%
\pgfpathlineto{\pgfqpoint{0.618294in}{0.898984in}}%
\pgfpathlineto{\pgfqpoint{0.622313in}{0.892548in}}%
\pgfpathlineto{\pgfqpoint{0.626967in}{0.886547in}}%
\pgfpathlineto{\pgfqpoint{0.632236in}{0.881007in}}%
\pgfpathlineto{\pgfqpoint{0.638099in}{0.875949in}}%
\pgfpathlineto{\pgfqpoint{0.621651in}{0.894730in}}%
\pgfpathlineto{\pgfqpoint{0.607313in}{0.914059in}}%
\pgfpathlineto{\pgfqpoint{0.595132in}{0.933850in}}%
\pgfpathlineto{\pgfqpoint{0.585143in}{0.954016in}}%
\pgfpathlineto{\pgfqpoint{0.577373in}{0.974470in}}%
\pgfpathclose%
\pgfusepath{fill}%
\end{pgfscope}%
\begin{pgfscope}%
\pgfpathrectangle{\pgfqpoint{0.050000in}{0.050000in}}{\pgfqpoint{2.081932in}{2.081932in}}%
\pgfusepath{clip}%
\pgfsetbuttcap%
\pgfsetroundjoin%
\definecolor{currentfill}{rgb}{0.162142,0.474838,0.558140}%
\pgfsetfillcolor{currentfill}%
\pgfsetlinewidth{0.000000pt}%
\definecolor{currentstroke}{rgb}{0.000000,0.000000,0.000000}%
\pgfsetstrokecolor{currentstroke}%
\pgfsetdash{}{0pt}%
\pgfpathmoveto{\pgfqpoint{1.032609in}{1.010688in}}%
\pgfpathlineto{\pgfqpoint{1.032303in}{1.001986in}}%
\pgfpathlineto{\pgfqpoint{1.031646in}{0.992919in}}%
\pgfpathlineto{\pgfqpoint{1.030642in}{0.983521in}}%
\pgfpathlineto{\pgfqpoint{1.029293in}{0.973830in}}%
\pgfpathlineto{\pgfqpoint{1.027607in}{0.963883in}}%
\pgfpathlineto{\pgfqpoint{1.039665in}{0.960631in}}%
\pgfpathlineto{\pgfqpoint{1.052056in}{0.957842in}}%
\pgfpathlineto{\pgfqpoint{1.064729in}{0.955530in}}%
\pgfpathlineto{\pgfqpoint{1.077631in}{0.953704in}}%
\pgfpathlineto{\pgfqpoint{1.090708in}{0.952372in}}%
\pgfpathlineto{\pgfqpoint{1.091232in}{0.962545in}}%
\pgfpathlineto{\pgfqpoint{1.091652in}{0.972418in}}%
\pgfpathlineto{\pgfqpoint{1.091965in}{0.981954in}}%
\pgfpathlineto{\pgfqpoint{1.092169in}{0.991116in}}%
\pgfpathlineto{\pgfqpoint{1.092264in}{0.999868in}}%
\pgfpathlineto{\pgfqpoint{1.079903in}{1.001120in}}%
\pgfpathlineto{\pgfqpoint{1.067706in}{1.002836in}}%
\pgfpathlineto{\pgfqpoint{1.055725in}{1.005010in}}%
\pgfpathlineto{\pgfqpoint{1.044010in}{1.007631in}}%
\pgfpathlineto{\pgfqpoint{1.032609in}{1.010688in}}%
\pgfpathclose%
\pgfusepath{fill}%
\end{pgfscope}%
\begin{pgfscope}%
\pgfpathrectangle{\pgfqpoint{0.050000in}{0.050000in}}{\pgfqpoint{2.081932in}{2.081932in}}%
\pgfusepath{clip}%
\pgfsetbuttcap%
\pgfsetroundjoin%
\definecolor{currentfill}{rgb}{0.296479,0.761561,0.424223}%
\pgfsetfillcolor{currentfill}%
\pgfsetlinewidth{0.000000pt}%
\definecolor{currentstroke}{rgb}{0.000000,0.000000,0.000000}%
\pgfsetstrokecolor{currentstroke}%
\pgfsetdash{}{0pt}%
\pgfpathmoveto{\pgfqpoint{1.288807in}{1.100254in}}%
\pgfpathlineto{\pgfqpoint{1.283729in}{1.095954in}}%
\pgfpathlineto{\pgfqpoint{1.279160in}{1.091112in}}%
\pgfpathlineto{\pgfqpoint{1.275115in}{1.085749in}}%
\pgfpathlineto{\pgfqpoint{1.271610in}{1.079886in}}%
\pgfpathlineto{\pgfqpoint{1.268658in}{1.073549in}}%
\pgfpathlineto{\pgfqpoint{1.277173in}{1.079317in}}%
\pgfpathlineto{\pgfqpoint{1.285035in}{1.085389in}}%
\pgfpathlineto{\pgfqpoint{1.292211in}{1.091737in}}%
\pgfpathlineto{\pgfqpoint{1.298674in}{1.098335in}}%
\pgfpathlineto{\pgfqpoint{1.304399in}{1.105156in}}%
\pgfpathlineto{\pgfqpoint{1.308043in}{1.112091in}}%
\pgfpathlineto{\pgfqpoint{1.312368in}{1.118667in}}%
\pgfpathlineto{\pgfqpoint{1.317358in}{1.124859in}}%
\pgfpathlineto{\pgfqpoint{1.322995in}{1.130640in}}%
\pgfpathlineto{\pgfqpoint{1.329256in}{1.135987in}}%
\pgfpathlineto{\pgfqpoint{1.322786in}{1.128279in}}%
\pgfpathlineto{\pgfqpoint{1.315476in}{1.120820in}}%
\pgfpathlineto{\pgfqpoint{1.307354in}{1.113643in}}%
\pgfpathlineto{\pgfqpoint{1.298453in}{1.106778in}}%
\pgfpathlineto{\pgfqpoint{1.288807in}{1.100254in}}%
\pgfpathclose%
\pgfusepath{fill}%
\end{pgfscope}%
\begin{pgfscope}%
\pgfpathrectangle{\pgfqpoint{0.050000in}{0.050000in}}{\pgfqpoint{2.081932in}{2.081932in}}%
\pgfusepath{clip}%
\pgfsetbuttcap%
\pgfsetroundjoin%
\definecolor{currentfill}{rgb}{0.120638,0.625828,0.533488}%
\pgfsetfillcolor{currentfill}%
\pgfsetlinewidth{0.000000pt}%
\definecolor{currentstroke}{rgb}{0.000000,0.000000,0.000000}%
\pgfsetstrokecolor{currentstroke}%
\pgfsetdash{}{0pt}%
\pgfpathmoveto{\pgfqpoint{1.217555in}{1.050060in}}%
\pgfpathlineto{\pgfqpoint{1.215978in}{1.043631in}}%
\pgfpathlineto{\pgfqpoint{1.214778in}{1.036691in}}%
\pgfpathlineto{\pgfqpoint{1.213959in}{1.029266in}}%
\pgfpathlineto{\pgfqpoint{1.213523in}{1.021388in}}%
\pgfpathlineto{\pgfqpoint{1.213472in}{1.013087in}}%
\pgfpathlineto{\pgfqpoint{1.224231in}{1.016855in}}%
\pgfpathlineto{\pgfqpoint{1.234552in}{1.021020in}}%
\pgfpathlineto{\pgfqpoint{1.244392in}{1.025563in}}%
\pgfpathlineto{\pgfqpoint{1.253712in}{1.030465in}}%
\pgfpathlineto{\pgfqpoint{1.262473in}{1.035705in}}%
\pgfpathlineto{\pgfqpoint{1.262550in}{1.043996in}}%
\pgfpathlineto{\pgfqpoint{1.263210in}{1.051956in}}%
\pgfpathlineto{\pgfqpoint{1.264451in}{1.059556in}}%
\pgfpathlineto{\pgfqpoint{1.266269in}{1.066763in}}%
\pgfpathlineto{\pgfqpoint{1.268658in}{1.073549in}}%
\pgfpathlineto{\pgfqpoint{1.259523in}{1.068108in}}%
\pgfpathlineto{\pgfqpoint{1.249804in}{1.063017in}}%
\pgfpathlineto{\pgfqpoint{1.239542in}{1.058299in}}%
\pgfpathlineto{\pgfqpoint{1.228777in}{1.053974in}}%
\pgfpathlineto{\pgfqpoint{1.217555in}{1.050060in}}%
\pgfpathclose%
\pgfusepath{fill}%
\end{pgfscope}%
\begin{pgfscope}%
\pgfpathrectangle{\pgfqpoint{0.050000in}{0.050000in}}{\pgfqpoint{2.081932in}{2.081932in}}%
\pgfusepath{clip}%
\pgfsetbuttcap%
\pgfsetroundjoin%
\definecolor{currentfill}{rgb}{0.282327,0.094955,0.417331}%
\pgfsetfillcolor{currentfill}%
\pgfsetlinewidth{0.000000pt}%
\definecolor{currentstroke}{rgb}{0.000000,0.000000,0.000000}%
\pgfsetstrokecolor{currentstroke}%
\pgfsetdash{}{0pt}%
\pgfpathmoveto{\pgfqpoint{1.458623in}{0.807998in}}%
\pgfpathlineto{\pgfqpoint{1.467324in}{0.806325in}}%
\pgfpathlineto{\pgfqpoint{1.475803in}{0.805264in}}%
\pgfpathlineto{\pgfqpoint{1.484025in}{0.804818in}}%
\pgfpathlineto{\pgfqpoint{1.491957in}{0.804992in}}%
\pgfpathlineto{\pgfqpoint{1.499566in}{0.805786in}}%
\pgfpathlineto{\pgfqpoint{1.520798in}{0.820926in}}%
\pgfpathlineto{\pgfqpoint{1.540301in}{0.836825in}}%
\pgfpathlineto{\pgfqpoint{1.557998in}{0.853411in}}%
\pgfpathlineto{\pgfqpoint{1.573827in}{0.870610in}}%
\pgfpathlineto{\pgfqpoint{1.587730in}{0.888343in}}%
\pgfpathlineto{\pgfqpoint{1.578447in}{0.885921in}}%
\pgfpathlineto{\pgfqpoint{1.568766in}{0.884043in}}%
\pgfpathlineto{\pgfqpoint{1.558727in}{0.882716in}}%
\pgfpathlineto{\pgfqpoint{1.548370in}{0.881943in}}%
\pgfpathlineto{\pgfqpoint{1.537736in}{0.881728in}}%
\pgfpathlineto{\pgfqpoint{1.525225in}{0.865877in}}%
\pgfpathlineto{\pgfqpoint{1.511003in}{0.850512in}}%
\pgfpathlineto{\pgfqpoint{1.495121in}{0.835700in}}%
\pgfpathlineto{\pgfqpoint{1.477639in}{0.821508in}}%
\pgfpathlineto{\pgfqpoint{1.458623in}{0.807998in}}%
\pgfpathclose%
\pgfusepath{fill}%
\end{pgfscope}%
\begin{pgfscope}%
\pgfpathrectangle{\pgfqpoint{0.050000in}{0.050000in}}{\pgfqpoint{2.081932in}{2.081932in}}%
\pgfusepath{clip}%
\pgfsetbuttcap%
\pgfsetroundjoin%
\definecolor{currentfill}{rgb}{0.278012,0.180367,0.486697}%
\pgfsetfillcolor{currentfill}%
\pgfsetlinewidth{0.000000pt}%
\definecolor{currentstroke}{rgb}{0.000000,0.000000,0.000000}%
\pgfsetstrokecolor{currentstroke}%
\pgfsetdash{}{0pt}%
\pgfpathmoveto{\pgfqpoint{1.086590in}{0.898420in}}%
\pgfpathlineto{\pgfqpoint{1.085490in}{0.887302in}}%
\pgfpathlineto{\pgfqpoint{1.084307in}{0.876173in}}%
\pgfpathlineto{\pgfqpoint{1.083045in}{0.865076in}}%
\pgfpathlineto{\pgfqpoint{1.081710in}{0.854054in}}%
\pgfpathlineto{\pgfqpoint{1.080305in}{0.843149in}}%
\pgfpathlineto{\pgfqpoint{1.098324in}{0.841998in}}%
\pgfpathlineto{\pgfqpoint{1.116432in}{0.841545in}}%
\pgfpathlineto{\pgfqpoint{1.134552in}{0.841790in}}%
\pgfpathlineto{\pgfqpoint{1.152606in}{0.842733in}}%
\pgfpathlineto{\pgfqpoint{1.170517in}{0.844370in}}%
\pgfpathlineto{\pgfqpoint{1.168661in}{0.855229in}}%
\pgfpathlineto{\pgfqpoint{1.166896in}{0.866208in}}%
\pgfpathlineto{\pgfqpoint{1.165229in}{0.877264in}}%
\pgfpathlineto{\pgfqpoint{1.163666in}{0.888354in}}%
\pgfpathlineto{\pgfqpoint{1.162213in}{0.899437in}}%
\pgfpathlineto{\pgfqpoint{1.147197in}{0.898074in}}%
\pgfpathlineto{\pgfqpoint{1.132063in}{0.897288in}}%
\pgfpathlineto{\pgfqpoint{1.116874in}{0.897084in}}%
\pgfpathlineto{\pgfqpoint{1.101695in}{0.897462in}}%
\pgfpathlineto{\pgfqpoint{1.086590in}{0.898420in}}%
\pgfpathclose%
\pgfusepath{fill}%
\end{pgfscope}%
\begin{pgfscope}%
\pgfpathrectangle{\pgfqpoint{0.050000in}{0.050000in}}{\pgfqpoint{2.081932in}{2.081932in}}%
\pgfusepath{clip}%
\pgfsetbuttcap%
\pgfsetroundjoin%
\definecolor{currentfill}{rgb}{0.606045,0.850733,0.236712}%
\pgfsetfillcolor{currentfill}%
\pgfsetlinewidth{0.000000pt}%
\definecolor{currentstroke}{rgb}{0.000000,0.000000,0.000000}%
\pgfsetstrokecolor{currentstroke}%
\pgfsetdash{}{0pt}%
\pgfpathmoveto{\pgfqpoint{0.851490in}{1.187243in}}%
\pgfpathlineto{\pgfqpoint{0.861073in}{1.183241in}}%
\pgfpathlineto{\pgfqpoint{0.870145in}{1.178773in}}%
\pgfpathlineto{\pgfqpoint{0.878670in}{1.173858in}}%
\pgfpathlineto{\pgfqpoint{0.886614in}{1.168517in}}%
\pgfpathlineto{\pgfqpoint{0.893946in}{1.162771in}}%
\pgfpathlineto{\pgfqpoint{0.897495in}{1.154473in}}%
\pgfpathlineto{\pgfqpoint{0.901948in}{1.146314in}}%
\pgfpathlineto{\pgfqpoint{0.907291in}{1.138331in}}%
\pgfpathlineto{\pgfqpoint{0.913504in}{1.130554in}}%
\pgfpathlineto{\pgfqpoint{0.920564in}{1.123018in}}%
\pgfpathlineto{\pgfqpoint{0.914065in}{1.127492in}}%
\pgfpathlineto{\pgfqpoint{0.907022in}{1.131452in}}%
\pgfpathlineto{\pgfqpoint{0.899462in}{1.134883in}}%
\pgfpathlineto{\pgfqpoint{0.891413in}{1.137768in}}%
\pgfpathlineto{\pgfqpoint{0.882907in}{1.140095in}}%
\pgfpathlineto{\pgfqpoint{0.874552in}{1.149039in}}%
\pgfpathlineto{\pgfqpoint{0.867207in}{1.158266in}}%
\pgfpathlineto{\pgfqpoint{0.860902in}{1.167735in}}%
\pgfpathlineto{\pgfqpoint{0.855657in}{1.177408in}}%
\pgfpathlineto{\pgfqpoint{0.851490in}{1.187243in}}%
\pgfpathclose%
\pgfusepath{fill}%
\end{pgfscope}%
\begin{pgfscope}%
\pgfpathrectangle{\pgfqpoint{0.050000in}{0.050000in}}{\pgfqpoint{2.081932in}{2.081932in}}%
\pgfusepath{clip}%
\pgfsetbuttcap%
\pgfsetroundjoin%
\definecolor{currentfill}{rgb}{0.267004,0.004874,0.329415}%
\pgfsetfillcolor{currentfill}%
\pgfsetlinewidth{0.000000pt}%
\definecolor{currentstroke}{rgb}{0.000000,0.000000,0.000000}%
\pgfsetstrokecolor{currentstroke}%
\pgfsetdash{}{0pt}%
\pgfpathmoveto{\pgfqpoint{1.282765in}{0.815340in}}%
\pgfpathlineto{\pgfqpoint{1.288693in}{0.806878in}}%
\pgfpathlineto{\pgfqpoint{1.294715in}{0.798828in}}%
\pgfpathlineto{\pgfqpoint{1.300807in}{0.791220in}}%
\pgfpathlineto{\pgfqpoint{1.306945in}{0.784087in}}%
\pgfpathlineto{\pgfqpoint{1.313106in}{0.777457in}}%
\pgfpathlineto{\pgfqpoint{1.335127in}{0.785423in}}%
\pgfpathlineto{\pgfqpoint{1.356216in}{0.794218in}}%
\pgfpathlineto{\pgfqpoint{1.376282in}{0.803802in}}%
\pgfpathlineto{\pgfqpoint{1.395245in}{0.814130in}}%
\pgfpathlineto{\pgfqpoint{1.413026in}{0.825156in}}%
\pgfpathlineto{\pgfqpoint{1.403742in}{0.830243in}}%
\pgfpathlineto{\pgfqpoint{1.394489in}{0.835838in}}%
\pgfpathlineto{\pgfqpoint{1.385302in}{0.841918in}}%
\pgfpathlineto{\pgfqpoint{1.376218in}{0.848459in}}%
\pgfpathlineto{\pgfqpoint{1.367271in}{0.855433in}}%
\pgfpathlineto{\pgfqpoint{1.352211in}{0.846159in}}%
\pgfpathlineto{\pgfqpoint{1.336163in}{0.837475in}}%
\pgfpathlineto{\pgfqpoint{1.319192in}{0.829420in}}%
\pgfpathlineto{\pgfqpoint{1.301368in}{0.822031in}}%
\pgfpathlineto{\pgfqpoint{1.282765in}{0.815340in}}%
\pgfpathclose%
\pgfusepath{fill}%
\end{pgfscope}%
\begin{pgfscope}%
\pgfpathrectangle{\pgfqpoint{0.050000in}{0.050000in}}{\pgfqpoint{2.081932in}{2.081932in}}%
\pgfusepath{clip}%
\pgfsetbuttcap%
\pgfsetroundjoin%
\definecolor{currentfill}{rgb}{0.162142,0.474838,0.558140}%
\pgfsetfillcolor{currentfill}%
\pgfsetlinewidth{0.000000pt}%
\definecolor{currentstroke}{rgb}{0.000000,0.000000,0.000000}%
\pgfsetstrokecolor{currentstroke}%
\pgfsetdash{}{0pt}%
\pgfpathmoveto{\pgfqpoint{1.092264in}{0.999868in}}%
\pgfpathlineto{\pgfqpoint{1.092169in}{0.991116in}}%
\pgfpathlineto{\pgfqpoint{1.091965in}{0.981954in}}%
\pgfpathlineto{\pgfqpoint{1.091652in}{0.972418in}}%
\pgfpathlineto{\pgfqpoint{1.091232in}{0.962545in}}%
\pgfpathlineto{\pgfqpoint{1.090708in}{0.952372in}}%
\pgfpathlineto{\pgfqpoint{1.103904in}{0.951540in}}%
\pgfpathlineto{\pgfqpoint{1.117164in}{0.951212in}}%
\pgfpathlineto{\pgfqpoint{1.130433in}{0.951389in}}%
\pgfpathlineto{\pgfqpoint{1.143654in}{0.952071in}}%
\pgfpathlineto{\pgfqpoint{1.156772in}{0.953254in}}%
\pgfpathlineto{\pgfqpoint{1.156079in}{0.963410in}}%
\pgfpathlineto{\pgfqpoint{1.155524in}{0.973269in}}%
\pgfpathlineto{\pgfqpoint{1.155111in}{0.982795in}}%
\pgfpathlineto{\pgfqpoint{1.154841in}{0.991949in}}%
\pgfpathlineto{\pgfqpoint{1.154715in}{1.000697in}}%
\pgfpathlineto{\pgfqpoint{1.142314in}{0.999585in}}%
\pgfpathlineto{\pgfqpoint{1.129816in}{0.998945in}}%
\pgfpathlineto{\pgfqpoint{1.117274in}{0.998778in}}%
\pgfpathlineto{\pgfqpoint{1.104739in}{0.999086in}}%
\pgfpathlineto{\pgfqpoint{1.092264in}{0.999868in}}%
\pgfpathclose%
\pgfusepath{fill}%
\end{pgfscope}%
\begin{pgfscope}%
\pgfpathrectangle{\pgfqpoint{0.050000in}{0.050000in}}{\pgfqpoint{2.081932in}{2.081932in}}%
\pgfusepath{clip}%
\pgfsetbuttcap%
\pgfsetroundjoin%
\definecolor{currentfill}{rgb}{0.124780,0.640461,0.527068}%
\pgfsetfillcolor{currentfill}%
\pgfsetlinewidth{0.000000pt}%
\definecolor{currentstroke}{rgb}{0.000000,0.000000,0.000000}%
\pgfsetstrokecolor{currentstroke}%
\pgfsetdash{}{0pt}%
\pgfpathmoveto{\pgfqpoint{1.658278in}{0.970803in}}%
\pgfpathlineto{\pgfqpoint{1.658202in}{0.979227in}}%
\pgfpathlineto{\pgfqpoint{1.657389in}{0.987829in}}%
\pgfpathlineto{\pgfqpoint{1.655840in}{0.996573in}}%
\pgfpathlineto{\pgfqpoint{1.653562in}{1.005424in}}%
\pgfpathlineto{\pgfqpoint{1.650563in}{1.014345in}}%
\pgfpathlineto{\pgfqpoint{1.663937in}{1.034659in}}%
\pgfpathlineto{\pgfqpoint{1.675022in}{1.055384in}}%
\pgfpathlineto{\pgfqpoint{1.683790in}{1.076429in}}%
\pgfpathlineto{\pgfqpoint{1.690223in}{1.097705in}}%
\pgfpathlineto{\pgfqpoint{1.694310in}{1.119122in}}%
\pgfpathlineto{\pgfqpoint{1.697515in}{1.110898in}}%
\pgfpathlineto{\pgfqpoint{1.699949in}{1.102604in}}%
\pgfpathlineto{\pgfqpoint{1.701604in}{1.094274in}}%
\pgfpathlineto{\pgfqpoint{1.702472in}{1.085941in}}%
\pgfpathlineto{\pgfqpoint{1.702554in}{1.077640in}}%
\pgfpathlineto{\pgfqpoint{1.698429in}{1.055806in}}%
\pgfpathlineto{\pgfqpoint{1.691926in}{1.034113in}}%
\pgfpathlineto{\pgfqpoint{1.683052in}{1.012654in}}%
\pgfpathlineto{\pgfqpoint{1.671826in}{0.991520in}}%
\pgfpathlineto{\pgfqpoint{1.658278in}{0.970803in}}%
\pgfpathclose%
\pgfusepath{fill}%
\end{pgfscope}%
\begin{pgfscope}%
\pgfpathrectangle{\pgfqpoint{0.050000in}{0.050000in}}{\pgfqpoint{2.081932in}{2.081932in}}%
\pgfusepath{clip}%
\pgfsetbuttcap%
\pgfsetroundjoin%
\definecolor{currentfill}{rgb}{0.120638,0.625828,0.533488}%
\pgfsetfillcolor{currentfill}%
\pgfsetlinewidth{0.000000pt}%
\definecolor{currentstroke}{rgb}{0.000000,0.000000,0.000000}%
\pgfsetstrokecolor{currentstroke}%
\pgfsetdash{}{0pt}%
\pgfpathmoveto{\pgfqpoint{0.985537in}{1.064498in}}%
\pgfpathlineto{\pgfqpoint{0.987675in}{1.057850in}}%
\pgfpathlineto{\pgfqpoint{0.989302in}{1.050745in}}%
\pgfpathlineto{\pgfqpoint{0.990412in}{1.043213in}}%
\pgfpathlineto{\pgfqpoint{0.991002in}{1.035284in}}%
\pgfpathlineto{\pgfqpoint{0.991071in}{1.026989in}}%
\pgfpathlineto{\pgfqpoint{1.000758in}{1.022337in}}%
\pgfpathlineto{\pgfqpoint{1.010939in}{1.018057in}}%
\pgfpathlineto{\pgfqpoint{1.021570in}{1.014169in}}%
\pgfpathlineto{\pgfqpoint{1.032609in}{1.010688in}}%
\pgfpathlineto{\pgfqpoint{1.032563in}{1.018991in}}%
\pgfpathlineto{\pgfqpoint{1.032163in}{1.026860in}}%
\pgfpathlineto{\pgfqpoint{1.031411in}{1.034266in}}%
\pgfpathlineto{\pgfqpoint{1.030310in}{1.041178in}}%
\pgfpathlineto{\pgfqpoint{1.028862in}{1.047569in}}%
\pgfpathlineto{\pgfqpoint{1.017347in}{1.051184in}}%
\pgfpathlineto{\pgfqpoint{1.006258in}{1.055222in}}%
\pgfpathlineto{\pgfqpoint{0.995640in}{1.059667in}}%
\pgfpathlineto{\pgfqpoint{0.985537in}{1.064498in}}%
\pgfpathclose%
\pgfusepath{fill}%
\end{pgfscope}%
\begin{pgfscope}%
\pgfpathrectangle{\pgfqpoint{0.050000in}{0.050000in}}{\pgfqpoint{2.081932in}{2.081932in}}%
\pgfusepath{clip}%
\pgfsetbuttcap%
\pgfsetroundjoin%
\definecolor{currentfill}{rgb}{0.855810,0.888601,0.097452}%
\pgfsetfillcolor{currentfill}%
\pgfsetlinewidth{0.000000pt}%
\definecolor{currentstroke}{rgb}{0.000000,0.000000,0.000000}%
\pgfsetstrokecolor{currentstroke}%
\pgfsetdash{}{0pt}%
\pgfpathmoveto{\pgfqpoint{1.419814in}{1.161579in}}%
\pgfpathlineto{\pgfqpoint{1.409001in}{1.161489in}}%
\pgfpathlineto{\pgfqpoint{1.398473in}{1.160829in}}%
\pgfpathlineto{\pgfqpoint{1.388273in}{1.159604in}}%
\pgfpathlineto{\pgfqpoint{1.378441in}{1.157820in}}%
\pgfpathlineto{\pgfqpoint{1.369016in}{1.155485in}}%
\pgfpathlineto{\pgfqpoint{1.375638in}{1.164886in}}%
\pgfpathlineto{\pgfqpoint{1.381205in}{1.174502in}}%
\pgfpathlineto{\pgfqpoint{1.385698in}{1.184293in}}%
\pgfpathlineto{\pgfqpoint{1.389103in}{1.194218in}}%
\pgfpathlineto{\pgfqpoint{1.391411in}{1.204236in}}%
\pgfpathlineto{\pgfqpoint{1.401616in}{1.208384in}}%
\pgfpathlineto{\pgfqpoint{1.412257in}{1.212062in}}%
\pgfpathlineto{\pgfqpoint{1.423293in}{1.215254in}}%
\pgfpathlineto{\pgfqpoint{1.434677in}{1.217946in}}%
\pgfpathlineto{\pgfqpoint{1.446363in}{1.220127in}}%
\pgfpathlineto{\pgfqpoint{1.443671in}{1.208107in}}%
\pgfpathlineto{\pgfqpoint{1.439656in}{1.196193in}}%
\pgfpathlineto{\pgfqpoint{1.434331in}{1.184435in}}%
\pgfpathlineto{\pgfqpoint{1.427709in}{1.172881in}}%
\pgfpathlineto{\pgfqpoint{1.419814in}{1.161579in}}%
\pgfpathclose%
\pgfusepath{fill}%
\end{pgfscope}%
\begin{pgfscope}%
\pgfpathrectangle{\pgfqpoint{0.050000in}{0.050000in}}{\pgfqpoint{2.081932in}{2.081932in}}%
\pgfusepath{clip}%
\pgfsetbuttcap%
\pgfsetroundjoin%
\definecolor{currentfill}{rgb}{0.278791,0.062145,0.386592}%
\pgfsetfillcolor{currentfill}%
\pgfsetlinewidth{0.000000pt}%
\definecolor{currentstroke}{rgb}{0.000000,0.000000,0.000000}%
\pgfsetstrokecolor{currentstroke}%
\pgfsetdash{}{0pt}%
\pgfpathmoveto{\pgfqpoint{1.170517in}{0.844370in}}%
\pgfpathlineto{\pgfqpoint{1.172457in}{0.833671in}}%
\pgfpathlineto{\pgfqpoint{1.174473in}{0.823176in}}%
\pgfpathlineto{\pgfqpoint{1.176558in}{0.812924in}}%
\pgfpathlineto{\pgfqpoint{1.178705in}{0.802955in}}%
\pgfpathlineto{\pgfqpoint{1.180904in}{0.793310in}}%
\pgfpathlineto{\pgfqpoint{1.202169in}{0.796119in}}%
\pgfpathlineto{\pgfqpoint{1.223076in}{0.799745in}}%
\pgfpathlineto{\pgfqpoint{1.243536in}{0.804171in}}%
\pgfpathlineto{\pgfqpoint{1.263460in}{0.809378in}}%
\pgfpathlineto{\pgfqpoint{1.282765in}{0.815340in}}%
\pgfpathlineto{\pgfqpoint{1.276952in}{0.824179in}}%
\pgfpathlineto{\pgfqpoint{1.271278in}{0.833360in}}%
\pgfpathlineto{\pgfqpoint{1.265765in}{0.842847in}}%
\pgfpathlineto{\pgfqpoint{1.260432in}{0.852604in}}%
\pgfpathlineto{\pgfqpoint{1.255301in}{0.862591in}}%
\pgfpathlineto{\pgfqpoint{1.239223in}{0.857658in}}%
\pgfpathlineto{\pgfqpoint{1.222635in}{0.853352in}}%
\pgfpathlineto{\pgfqpoint{1.205606in}{0.849691in}}%
\pgfpathlineto{\pgfqpoint{1.188209in}{0.846692in}}%
\pgfpathlineto{\pgfqpoint{1.170517in}{0.844370in}}%
\pgfpathclose%
\pgfusepath{fill}%
\end{pgfscope}%
\begin{pgfscope}%
\pgfpathrectangle{\pgfqpoint{0.050000in}{0.050000in}}{\pgfqpoint{2.081932in}{2.081932in}}%
\pgfusepath{clip}%
\pgfsetbuttcap%
\pgfsetroundjoin%
\definecolor{currentfill}{rgb}{0.296479,0.761561,0.424223}%
\pgfsetfillcolor{currentfill}%
\pgfsetlinewidth{0.000000pt}%
\definecolor{currentstroke}{rgb}{0.000000,0.000000,0.000000}%
\pgfsetstrokecolor{currentstroke}%
\pgfsetdash{}{0pt}%
\pgfpathmoveto{\pgfqpoint{0.920564in}{1.123018in}}%
\pgfpathlineto{\pgfqpoint{0.926492in}{1.118050in}}%
\pgfpathlineto{\pgfqpoint{0.931828in}{1.112609in}}%
\pgfpathlineto{\pgfqpoint{0.936552in}{1.106717in}}%
\pgfpathlineto{\pgfqpoint{0.940645in}{1.100399in}}%
\pgfpathlineto{\pgfqpoint{0.944094in}{1.093681in}}%
\pgfpathlineto{\pgfqpoint{0.951058in}{1.087255in}}%
\pgfpathlineto{\pgfqpoint{0.958716in}{1.081099in}}%
\pgfpathlineto{\pgfqpoint{0.967038in}{1.075237in}}%
\pgfpathlineto{\pgfqpoint{0.975990in}{1.069696in}}%
\pgfpathlineto{\pgfqpoint{0.985537in}{1.064498in}}%
\pgfpathlineto{\pgfqpoint{0.982895in}{1.070664in}}%
\pgfpathlineto{\pgfqpoint{0.979758in}{1.076321in}}%
\pgfpathlineto{\pgfqpoint{0.976139in}{1.081446in}}%
\pgfpathlineto{\pgfqpoint{0.972049in}{1.086017in}}%
\pgfpathlineto{\pgfqpoint{0.967504in}{1.090016in}}%
\pgfpathlineto{\pgfqpoint{0.956682in}{1.095896in}}%
\pgfpathlineto{\pgfqpoint{0.946539in}{1.102164in}}%
\pgfpathlineto{\pgfqpoint{0.937113in}{1.108793in}}%
\pgfpathlineto{\pgfqpoint{0.928444in}{1.115754in}}%
\pgfpathlineto{\pgfqpoint{0.920564in}{1.123018in}}%
\pgfpathclose%
\pgfusepath{fill}%
\end{pgfscope}%
\begin{pgfscope}%
\pgfpathrectangle{\pgfqpoint{0.050000in}{0.050000in}}{\pgfqpoint{2.081932in}{2.081932in}}%
\pgfusepath{clip}%
\pgfsetbuttcap%
\pgfsetroundjoin%
\definecolor{currentfill}{rgb}{0.278791,0.062145,0.386592}%
\pgfsetfillcolor{currentfill}%
\pgfsetlinewidth{0.000000pt}%
\definecolor{currentstroke}{rgb}{0.000000,0.000000,0.000000}%
\pgfsetstrokecolor{currentstroke}%
\pgfsetdash{}{0pt}%
\pgfpathmoveto{\pgfqpoint{0.994202in}{0.859066in}}%
\pgfpathlineto{\pgfqpoint{0.989490in}{0.848941in}}%
\pgfpathlineto{\pgfqpoint{0.984593in}{0.839042in}}%
\pgfpathlineto{\pgfqpoint{0.979529in}{0.829407in}}%
\pgfpathlineto{\pgfqpoint{0.974318in}{0.820074in}}%
\pgfpathlineto{\pgfqpoint{0.968980in}{0.811079in}}%
\pgfpathlineto{\pgfqpoint{0.988726in}{0.805645in}}%
\pgfpathlineto{\pgfqpoint{1.009034in}{0.800983in}}%
\pgfpathlineto{\pgfqpoint{1.029815in}{0.797116in}}%
\pgfpathlineto{\pgfqpoint{1.050982in}{0.794062in}}%
\pgfpathlineto{\pgfqpoint{1.072444in}{0.791833in}}%
\pgfpathlineto{\pgfqpoint{1.074108in}{0.801533in}}%
\pgfpathlineto{\pgfqpoint{1.075733in}{0.811554in}}%
\pgfpathlineto{\pgfqpoint{1.077311in}{0.821858in}}%
\pgfpathlineto{\pgfqpoint{1.078837in}{0.832403in}}%
\pgfpathlineto{\pgfqpoint{1.080305in}{0.843149in}}%
\pgfpathlineto{\pgfqpoint{1.062452in}{0.844991in}}%
\pgfpathlineto{\pgfqpoint{1.044841in}{0.847517in}}%
\pgfpathlineto{\pgfqpoint{1.027547in}{0.850715in}}%
\pgfpathlineto{\pgfqpoint{1.010644in}{0.854570in}}%
\pgfpathlineto{\pgfqpoint{0.994202in}{0.859066in}}%
\pgfpathclose%
\pgfusepath{fill}%
\end{pgfscope}%
\begin{pgfscope}%
\pgfpathrectangle{\pgfqpoint{0.050000in}{0.050000in}}{\pgfqpoint{2.081932in}{2.081932in}}%
\pgfusepath{clip}%
\pgfsetbuttcap%
\pgfsetroundjoin%
\definecolor{currentfill}{rgb}{0.267004,0.004874,0.329415}%
\pgfsetfillcolor{currentfill}%
\pgfsetlinewidth{0.000000pt}%
\definecolor{currentstroke}{rgb}{0.000000,0.000000,0.000000}%
\pgfsetstrokecolor{currentstroke}%
\pgfsetdash{}{0pt}%
\pgfpathmoveto{\pgfqpoint{0.897227in}{0.840003in}}%
\pgfpathlineto{\pgfqpoint{0.889209in}{0.832458in}}%
\pgfpathlineto{\pgfqpoint{0.881066in}{0.825338in}}%
\pgfpathlineto{\pgfqpoint{0.872830in}{0.818672in}}%
\pgfpathlineto{\pgfqpoint{0.864534in}{0.812487in}}%
\pgfpathlineto{\pgfqpoint{0.856209in}{0.806808in}}%
\pgfpathlineto{\pgfqpoint{0.875952in}{0.796998in}}%
\pgfpathlineto{\pgfqpoint{0.896741in}{0.787963in}}%
\pgfpathlineto{\pgfqpoint{0.918491in}{0.779745in}}%
\pgfpathlineto{\pgfqpoint{0.941111in}{0.772383in}}%
\pgfpathlineto{\pgfqpoint{0.946770in}{0.779178in}}%
\pgfpathlineto{\pgfqpoint{0.952408in}{0.786476in}}%
\pgfpathlineto{\pgfqpoint{0.958003in}{0.794247in}}%
\pgfpathlineto{\pgfqpoint{0.963535in}{0.802458in}}%
\pgfpathlineto{\pgfqpoint{0.968980in}{0.811079in}}%
\pgfpathlineto{\pgfqpoint{0.949878in}{0.817262in}}%
\pgfpathlineto{\pgfqpoint{0.931502in}{0.824164in}}%
\pgfpathlineto{\pgfqpoint{0.913927in}{0.831756in}}%
\pgfpathlineto{\pgfqpoint{0.897227in}{0.840003in}}%
\pgfpathclose%
\pgfusepath{fill}%
\end{pgfscope}%
\begin{pgfscope}%
\pgfpathrectangle{\pgfqpoint{0.050000in}{0.050000in}}{\pgfqpoint{2.081932in}{2.081932in}}%
\pgfusepath{clip}%
\pgfsetbuttcap%
\pgfsetroundjoin%
\definecolor{currentfill}{rgb}{0.120638,0.625828,0.533488}%
\pgfsetfillcolor{currentfill}%
\pgfsetlinewidth{0.000000pt}%
\definecolor{currentstroke}{rgb}{0.000000,0.000000,0.000000}%
\pgfsetstrokecolor{currentstroke}%
\pgfsetdash{}{0pt}%
\pgfpathmoveto{\pgfqpoint{1.156257in}{1.037191in}}%
\pgfpathlineto{\pgfqpoint{1.155661in}{1.030959in}}%
\pgfpathlineto{\pgfqpoint{1.155208in}{1.024165in}}%
\pgfpathlineto{\pgfqpoint{1.154899in}{1.016837in}}%
\pgfpathlineto{\pgfqpoint{1.154735in}{1.009004in}}%
\pgfpathlineto{\pgfqpoint{1.154715in}{1.000697in}}%
\pgfpathlineto{\pgfqpoint{1.166967in}{1.002276in}}%
\pgfpathlineto{\pgfqpoint{1.179018in}{1.004314in}}%
\pgfpathlineto{\pgfqpoint{1.190819in}{1.006802in}}%
\pgfpathlineto{\pgfqpoint{1.202320in}{1.009731in}}%
\pgfpathlineto{\pgfqpoint{1.213472in}{1.013087in}}%
\pgfpathlineto{\pgfqpoint{1.213523in}{1.021388in}}%
\pgfpathlineto{\pgfqpoint{1.213959in}{1.029266in}}%
\pgfpathlineto{\pgfqpoint{1.214778in}{1.036691in}}%
\pgfpathlineto{\pgfqpoint{1.215978in}{1.043631in}}%
\pgfpathlineto{\pgfqpoint{1.217555in}{1.050060in}}%
\pgfpathlineto{\pgfqpoint{1.205920in}{1.046575in}}%
\pgfpathlineto{\pgfqpoint{1.193923in}{1.043533in}}%
\pgfpathlineto{\pgfqpoint{1.181612in}{1.040948in}}%
\pgfpathlineto{\pgfqpoint{1.169039in}{1.038831in}}%
\pgfpathlineto{\pgfqpoint{1.156257in}{1.037191in}}%
\pgfpathclose%
\pgfusepath{fill}%
\end{pgfscope}%
\begin{pgfscope}%
\pgfpathrectangle{\pgfqpoint{0.050000in}{0.050000in}}{\pgfqpoint{2.081932in}{2.081932in}}%
\pgfusepath{clip}%
\pgfsetbuttcap%
\pgfsetroundjoin%
\definecolor{currentfill}{rgb}{0.278791,0.062145,0.386592}%
\pgfsetfillcolor{currentfill}%
\pgfsetlinewidth{0.000000pt}%
\definecolor{currentstroke}{rgb}{0.000000,0.000000,0.000000}%
\pgfsetstrokecolor{currentstroke}%
\pgfsetdash{}{0pt}%
\pgfpathmoveto{\pgfqpoint{1.080305in}{0.843149in}}%
\pgfpathlineto{\pgfqpoint{1.078837in}{0.832403in}}%
\pgfpathlineto{\pgfqpoint{1.077311in}{0.821858in}}%
\pgfpathlineto{\pgfqpoint{1.075733in}{0.811554in}}%
\pgfpathlineto{\pgfqpoint{1.074108in}{0.801533in}}%
\pgfpathlineto{\pgfqpoint{1.072444in}{0.791833in}}%
\pgfpathlineto{\pgfqpoint{1.094108in}{0.790441in}}%
\pgfpathlineto{\pgfqpoint{1.115879in}{0.789892in}}%
\pgfpathlineto{\pgfqpoint{1.137665in}{0.790189in}}%
\pgfpathlineto{\pgfqpoint{1.159371in}{0.791330in}}%
\pgfpathlineto{\pgfqpoint{1.180904in}{0.793310in}}%
\pgfpathlineto{\pgfqpoint{1.178705in}{0.802955in}}%
\pgfpathlineto{\pgfqpoint{1.176558in}{0.812924in}}%
\pgfpathlineto{\pgfqpoint{1.174473in}{0.823176in}}%
\pgfpathlineto{\pgfqpoint{1.172457in}{0.833671in}}%
\pgfpathlineto{\pgfqpoint{1.170517in}{0.844370in}}%
\pgfpathlineto{\pgfqpoint{1.152606in}{0.842733in}}%
\pgfpathlineto{\pgfqpoint{1.134552in}{0.841790in}}%
\pgfpathlineto{\pgfqpoint{1.116432in}{0.841545in}}%
\pgfpathlineto{\pgfqpoint{1.098324in}{0.841998in}}%
\pgfpathlineto{\pgfqpoint{1.080305in}{0.843149in}}%
\pgfpathclose%
\pgfusepath{fill}%
\end{pgfscope}%
\begin{pgfscope}%
\pgfpathrectangle{\pgfqpoint{0.050000in}{0.050000in}}{\pgfqpoint{2.081932in}{2.081932in}}%
\pgfusepath{clip}%
\pgfsetbuttcap%
\pgfsetroundjoin%
\definecolor{currentfill}{rgb}{0.120638,0.625828,0.533488}%
\pgfsetfillcolor{currentfill}%
\pgfsetlinewidth{0.000000pt}%
\definecolor{currentstroke}{rgb}{0.000000,0.000000,0.000000}%
\pgfsetstrokecolor{currentstroke}%
\pgfsetdash{}{0pt}%
\pgfpathmoveto{\pgfqpoint{1.028862in}{1.047569in}}%
\pgfpathlineto{\pgfqpoint{1.030310in}{1.041178in}}%
\pgfpathlineto{\pgfqpoint{1.031411in}{1.034266in}}%
\pgfpathlineto{\pgfqpoint{1.032163in}{1.026860in}}%
\pgfpathlineto{\pgfqpoint{1.032563in}{1.018991in}}%
\pgfpathlineto{\pgfqpoint{1.032609in}{1.010688in}}%
\pgfpathlineto{\pgfqpoint{1.044010in}{1.007631in}}%
\pgfpathlineto{\pgfqpoint{1.055725in}{1.005010in}}%
\pgfpathlineto{\pgfqpoint{1.067706in}{1.002836in}}%
\pgfpathlineto{\pgfqpoint{1.079903in}{1.001120in}}%
\pgfpathlineto{\pgfqpoint{1.092264in}{0.999868in}}%
\pgfpathlineto{\pgfqpoint{1.092250in}{1.008175in}}%
\pgfpathlineto{\pgfqpoint{1.092126in}{1.016005in}}%
\pgfpathlineto{\pgfqpoint{1.091892in}{1.023327in}}%
\pgfpathlineto{\pgfqpoint{1.091549in}{1.030111in}}%
\pgfpathlineto{\pgfqpoint{1.091098in}{1.036330in}}%
\pgfpathlineto{\pgfqpoint{1.078201in}{1.037630in}}%
\pgfpathlineto{\pgfqpoint{1.065476in}{1.039413in}}%
\pgfpathlineto{\pgfqpoint{1.052977in}{1.041671in}}%
\pgfpathlineto{\pgfqpoint{1.040755in}{1.044393in}}%
\pgfpathlineto{\pgfqpoint{1.028862in}{1.047569in}}%
\pgfpathclose%
\pgfusepath{fill}%
\end{pgfscope}%
\begin{pgfscope}%
\pgfpathrectangle{\pgfqpoint{0.050000in}{0.050000in}}{\pgfqpoint{2.081932in}{2.081932in}}%
\pgfusepath{clip}%
\pgfsetbuttcap%
\pgfsetroundjoin%
\definecolor{currentfill}{rgb}{0.150476,0.504369,0.557430}%
\pgfsetfillcolor{currentfill}%
\pgfsetlinewidth{0.000000pt}%
\definecolor{currentstroke}{rgb}{0.000000,0.000000,0.000000}%
\pgfsetstrokecolor{currentstroke}%
\pgfsetdash{}{0pt}%
\pgfpathmoveto{\pgfqpoint{0.551598in}{1.008674in}}%
\pgfpathlineto{\pgfqpoint{0.548601in}{1.016360in}}%
\pgfpathlineto{\pgfqpoint{0.546360in}{1.024259in}}%
\pgfpathlineto{\pgfqpoint{0.544887in}{1.032338in}}%
\pgfpathlineto{\pgfqpoint{0.544189in}{1.040567in}}%
\pgfpathlineto{\pgfqpoint{0.552350in}{1.019029in}}%
\pgfpathlineto{\pgfqpoint{0.562869in}{0.997788in}}%
\pgfpathlineto{\pgfqpoint{0.575721in}{0.976937in}}%
\pgfpathlineto{\pgfqpoint{0.590869in}{0.956566in}}%
\pgfpathlineto{\pgfqpoint{0.608264in}{0.936766in}}%
\pgfpathlineto{\pgfqpoint{0.608892in}{0.928562in}}%
\pgfpathlineto{\pgfqpoint{0.610218in}{0.920651in}}%
\pgfpathlineto{\pgfqpoint{0.612233in}{0.913064in}}%
\pgfpathlineto{\pgfqpoint{0.614930in}{0.905832in}}%
\pgfpathlineto{\pgfqpoint{0.597745in}{0.925446in}}%
\pgfpathlineto{\pgfqpoint{0.582778in}{0.945628in}}%
\pgfpathlineto{\pgfqpoint{0.570075in}{0.966286in}}%
\pgfpathlineto{\pgfqpoint{0.559673in}{0.987332in}}%
\pgfpathlineto{\pgfqpoint{0.551598in}{1.008674in}}%
\pgfpathclose%
\pgfusepath{fill}%
\end{pgfscope}%
\begin{pgfscope}%
\pgfpathrectangle{\pgfqpoint{0.050000in}{0.050000in}}{\pgfqpoint{2.081932in}{2.081932in}}%
\pgfusepath{clip}%
\pgfsetbuttcap%
\pgfsetroundjoin%
\definecolor{currentfill}{rgb}{0.327796,0.773980,0.406640}%
\pgfsetfillcolor{currentfill}%
\pgfsetlinewidth{0.000000pt}%
\definecolor{currentstroke}{rgb}{0.000000,0.000000,0.000000}%
\pgfsetstrokecolor{currentstroke}%
\pgfsetdash{}{0pt}%
\pgfpathmoveto{\pgfqpoint{1.650563in}{1.014345in}}%
\pgfpathlineto{\pgfqpoint{1.646854in}{1.023299in}}%
\pgfpathlineto{\pgfqpoint{1.642449in}{1.032250in}}%
\pgfpathlineto{\pgfqpoint{1.637367in}{1.041161in}}%
\pgfpathlineto{\pgfqpoint{1.631628in}{1.049994in}}%
\pgfpathlineto{\pgfqpoint{1.625254in}{1.058713in}}%
\pgfpathlineto{\pgfqpoint{1.638051in}{1.077959in}}%
\pgfpathlineto{\pgfqpoint{1.648673in}{1.097598in}}%
\pgfpathlineto{\pgfqpoint{1.657091in}{1.117545in}}%
\pgfpathlineto{\pgfqpoint{1.663285in}{1.137716in}}%
\pgfpathlineto{\pgfqpoint{1.667248in}{1.158025in}}%
\pgfpathlineto{\pgfqpoint{1.674066in}{1.150648in}}%
\pgfpathlineto{\pgfqpoint{1.680205in}{1.143037in}}%
\pgfpathlineto{\pgfqpoint{1.685638in}{1.135224in}}%
\pgfpathlineto{\pgfqpoint{1.690346in}{1.127241in}}%
\pgfpathlineto{\pgfqpoint{1.694310in}{1.119122in}}%
\pgfpathlineto{\pgfqpoint{1.690223in}{1.097705in}}%
\pgfpathlineto{\pgfqpoint{1.683790in}{1.076429in}}%
\pgfpathlineto{\pgfqpoint{1.675022in}{1.055384in}}%
\pgfpathlineto{\pgfqpoint{1.663937in}{1.034659in}}%
\pgfpathlineto{\pgfqpoint{1.650563in}{1.014345in}}%
\pgfpathclose%
\pgfusepath{fill}%
\end{pgfscope}%
\begin{pgfscope}%
\pgfpathrectangle{\pgfqpoint{0.050000in}{0.050000in}}{\pgfqpoint{2.081932in}{2.081932in}}%
\pgfusepath{clip}%
\pgfsetbuttcap%
\pgfsetroundjoin%
\definecolor{currentfill}{rgb}{0.120638,0.625828,0.533488}%
\pgfsetfillcolor{currentfill}%
\pgfsetlinewidth{0.000000pt}%
\definecolor{currentstroke}{rgb}{0.000000,0.000000,0.000000}%
\pgfsetstrokecolor{currentstroke}%
\pgfsetdash{}{0pt}%
\pgfpathmoveto{\pgfqpoint{1.091098in}{1.036330in}}%
\pgfpathlineto{\pgfqpoint{1.091549in}{1.030111in}}%
\pgfpathlineto{\pgfqpoint{1.091892in}{1.023327in}}%
\pgfpathlineto{\pgfqpoint{1.092126in}{1.016005in}}%
\pgfpathlineto{\pgfqpoint{1.092250in}{1.008175in}}%
\pgfpathlineto{\pgfqpoint{1.092264in}{0.999868in}}%
\pgfpathlineto{\pgfqpoint{1.104739in}{0.999086in}}%
\pgfpathlineto{\pgfqpoint{1.117274in}{0.998778in}}%
\pgfpathlineto{\pgfqpoint{1.129816in}{0.998945in}}%
\pgfpathlineto{\pgfqpoint{1.142314in}{0.999585in}}%
\pgfpathlineto{\pgfqpoint{1.154715in}{1.000697in}}%
\pgfpathlineto{\pgfqpoint{1.154735in}{1.009004in}}%
\pgfpathlineto{\pgfqpoint{1.154899in}{1.016837in}}%
\pgfpathlineto{\pgfqpoint{1.155208in}{1.024165in}}%
\pgfpathlineto{\pgfqpoint{1.155661in}{1.030959in}}%
\pgfpathlineto{\pgfqpoint{1.156257in}{1.037191in}}%
\pgfpathlineto{\pgfqpoint{1.143318in}{1.036036in}}%
\pgfpathlineto{\pgfqpoint{1.130278in}{1.035370in}}%
\pgfpathlineto{\pgfqpoint{1.117192in}{1.035197in}}%
\pgfpathlineto{\pgfqpoint{1.104113in}{1.035518in}}%
\pgfpathlineto{\pgfqpoint{1.091098in}{1.036330in}}%
\pgfpathclose%
\pgfusepath{fill}%
\end{pgfscope}%
\begin{pgfscope}%
\pgfpathrectangle{\pgfqpoint{0.050000in}{0.050000in}}{\pgfqpoint{2.081932in}{2.081932in}}%
\pgfusepath{clip}%
\pgfsetbuttcap%
\pgfsetroundjoin%
\definecolor{currentfill}{rgb}{0.993248,0.906157,0.143936}%
\pgfsetfillcolor{currentfill}%
\pgfsetlinewidth{0.000000pt}%
\definecolor{currentstroke}{rgb}{0.000000,0.000000,0.000000}%
\pgfsetstrokecolor{currentstroke}%
\pgfsetdash{}{0pt}%
\pgfpathmoveto{\pgfqpoint{1.476582in}{1.153535in}}%
\pgfpathlineto{\pgfqpoint{1.465022in}{1.156261in}}%
\pgfpathlineto{\pgfqpoint{1.453519in}{1.158436in}}%
\pgfpathlineto{\pgfqpoint{1.442118in}{1.160051in}}%
\pgfpathlineto{\pgfqpoint{1.430868in}{1.161100in}}%
\pgfpathlineto{\pgfqpoint{1.419814in}{1.161579in}}%
\pgfpathlineto{\pgfqpoint{1.427709in}{1.172881in}}%
\pgfpathlineto{\pgfqpoint{1.434331in}{1.184435in}}%
\pgfpathlineto{\pgfqpoint{1.439656in}{1.196193in}}%
\pgfpathlineto{\pgfqpoint{1.443671in}{1.208107in}}%
\pgfpathlineto{\pgfqpoint{1.446363in}{1.220127in}}%
\pgfpathlineto{\pgfqpoint{1.458303in}{1.221787in}}%
\pgfpathlineto{\pgfqpoint{1.470449in}{1.222920in}}%
\pgfpathlineto{\pgfqpoint{1.482750in}{1.223519in}}%
\pgfpathlineto{\pgfqpoint{1.495156in}{1.223583in}}%
\pgfpathlineto{\pgfqpoint{1.507616in}{1.223111in}}%
\pgfpathlineto{\pgfqpoint{1.504527in}{1.208842in}}%
\pgfpathlineto{\pgfqpoint{1.499865in}{1.194692in}}%
\pgfpathlineto{\pgfqpoint{1.493641in}{1.180718in}}%
\pgfpathlineto{\pgfqpoint{1.485871in}{1.166980in}}%
\pgfpathlineto{\pgfqpoint{1.476582in}{1.153535in}}%
\pgfpathclose%
\pgfusepath{fill}%
\end{pgfscope}%
\begin{pgfscope}%
\pgfpathrectangle{\pgfqpoint{0.050000in}{0.050000in}}{\pgfqpoint{2.081932in}{2.081932in}}%
\pgfusepath{clip}%
\pgfsetbuttcap%
\pgfsetroundjoin%
\definecolor{currentfill}{rgb}{0.282327,0.094955,0.417331}%
\pgfsetfillcolor{currentfill}%
\pgfsetlinewidth{0.000000pt}%
\definecolor{currentstroke}{rgb}{0.000000,0.000000,0.000000}%
\pgfsetstrokecolor{currentstroke}%
\pgfsetdash{}{0pt}%
\pgfpathmoveto{\pgfqpoint{0.722859in}{0.855043in}}%
\pgfpathlineto{\pgfqpoint{0.712759in}{0.854579in}}%
\pgfpathlineto{\pgfqpoint{0.702919in}{0.854692in}}%
\pgfpathlineto{\pgfqpoint{0.693380in}{0.855381in}}%
\pgfpathlineto{\pgfqpoint{0.684180in}{0.856646in}}%
\pgfpathlineto{\pgfqpoint{0.675357in}{0.858483in}}%
\pgfpathlineto{\pgfqpoint{0.692498in}{0.841707in}}%
\pgfpathlineto{\pgfqpoint{0.711464in}{0.825594in}}%
\pgfpathlineto{\pgfqpoint{0.732183in}{0.810220in}}%
\pgfpathlineto{\pgfqpoint{0.754577in}{0.795655in}}%
\pgfpathlineto{\pgfqpoint{0.778554in}{0.781965in}}%
\pgfpathlineto{\pgfqpoint{0.785386in}{0.781647in}}%
\pgfpathlineto{\pgfqpoint{0.792507in}{0.781970in}}%
\pgfpathlineto{\pgfqpoint{0.799887in}{0.782934in}}%
\pgfpathlineto{\pgfqpoint{0.807497in}{0.784530in}}%
\pgfpathlineto{\pgfqpoint{0.815306in}{0.786753in}}%
\pgfpathlineto{\pgfqpoint{0.793859in}{0.798961in}}%
\pgfpathlineto{\pgfqpoint{0.773815in}{0.811955in}}%
\pgfpathlineto{\pgfqpoint{0.755253in}{0.825675in}}%
\pgfpathlineto{\pgfqpoint{0.738247in}{0.840059in}}%
\pgfpathlineto{\pgfqpoint{0.722859in}{0.855043in}}%
\pgfpathclose%
\pgfusepath{fill}%
\end{pgfscope}%
\begin{pgfscope}%
\pgfpathrectangle{\pgfqpoint{0.050000in}{0.050000in}}{\pgfqpoint{2.081932in}{2.081932in}}%
\pgfusepath{clip}%
\pgfsetbuttcap%
\pgfsetroundjoin%
\definecolor{currentfill}{rgb}{0.855810,0.888601,0.097452}%
\pgfsetfillcolor{currentfill}%
\pgfsetlinewidth{0.000000pt}%
\definecolor{currentstroke}{rgb}{0.000000,0.000000,0.000000}%
\pgfsetstrokecolor{currentstroke}%
\pgfsetdash{}{0pt}%
\pgfpathmoveto{\pgfqpoint{0.797324in}{1.199735in}}%
\pgfpathlineto{\pgfqpoint{0.808847in}{1.198278in}}%
\pgfpathlineto{\pgfqpoint{0.820069in}{1.196291in}}%
\pgfpathlineto{\pgfqpoint{0.830947in}{1.193782in}}%
\pgfpathlineto{\pgfqpoint{0.841434in}{1.190761in}}%
\pgfpathlineto{\pgfqpoint{0.851490in}{1.187243in}}%
\pgfpathlineto{\pgfqpoint{0.855657in}{1.177408in}}%
\pgfpathlineto{\pgfqpoint{0.860902in}{1.167735in}}%
\pgfpathlineto{\pgfqpoint{0.867207in}{1.158266in}}%
\pgfpathlineto{\pgfqpoint{0.874552in}{1.149039in}}%
\pgfpathlineto{\pgfqpoint{0.882907in}{1.140095in}}%
\pgfpathlineto{\pgfqpoint{0.873978in}{1.141853in}}%
\pgfpathlineto{\pgfqpoint{0.864662in}{1.143034in}}%
\pgfpathlineto{\pgfqpoint{0.854995in}{1.143632in}}%
\pgfpathlineto{\pgfqpoint{0.845015in}{1.143643in}}%
\pgfpathlineto{\pgfqpoint{0.834764in}{1.143066in}}%
\pgfpathlineto{\pgfqpoint{0.824772in}{1.153828in}}%
\pgfpathlineto{\pgfqpoint{0.816003in}{1.164923in}}%
\pgfpathlineto{\pgfqpoint{0.808489in}{1.176304in}}%
\pgfpathlineto{\pgfqpoint{0.802256in}{1.187925in}}%
\pgfpathlineto{\pgfqpoint{0.797324in}{1.199735in}}%
\pgfpathclose%
\pgfusepath{fill}%
\end{pgfscope}%
\begin{pgfscope}%
\pgfpathrectangle{\pgfqpoint{0.050000in}{0.050000in}}{\pgfqpoint{2.081932in}{2.081932in}}%
\pgfusepath{clip}%
\pgfsetbuttcap%
\pgfsetroundjoin%
\definecolor{currentfill}{rgb}{0.296479,0.761561,0.424223}%
\pgfsetfillcolor{currentfill}%
\pgfsetlinewidth{0.000000pt}%
\definecolor{currentstroke}{rgb}{0.000000,0.000000,0.000000}%
\pgfsetstrokecolor{currentstroke}%
\pgfsetdash{}{0pt}%
\pgfpathmoveto{\pgfqpoint{1.230859in}{1.073676in}}%
\pgfpathlineto{\pgfqpoint{1.227505in}{1.070160in}}%
\pgfpathlineto{\pgfqpoint{1.224487in}{1.066022in}}%
\pgfpathlineto{\pgfqpoint{1.221817in}{1.061279in}}%
\pgfpathlineto{\pgfqpoint{1.219503in}{1.055951in}}%
\pgfpathlineto{\pgfqpoint{1.217555in}{1.050060in}}%
\pgfpathlineto{\pgfqpoint{1.228777in}{1.053974in}}%
\pgfpathlineto{\pgfqpoint{1.239542in}{1.058299in}}%
\pgfpathlineto{\pgfqpoint{1.249804in}{1.063017in}}%
\pgfpathlineto{\pgfqpoint{1.259523in}{1.068108in}}%
\pgfpathlineto{\pgfqpoint{1.268658in}{1.073549in}}%
\pgfpathlineto{\pgfqpoint{1.271610in}{1.079886in}}%
\pgfpathlineto{\pgfqpoint{1.275115in}{1.085749in}}%
\pgfpathlineto{\pgfqpoint{1.279160in}{1.091112in}}%
\pgfpathlineto{\pgfqpoint{1.283729in}{1.095954in}}%
\pgfpathlineto{\pgfqpoint{1.288807in}{1.100254in}}%
\pgfpathlineto{\pgfqpoint{1.278455in}{1.094099in}}%
\pgfpathlineto{\pgfqpoint{1.267438in}{1.088340in}}%
\pgfpathlineto{\pgfqpoint{1.255800in}{1.083001in}}%
\pgfpathlineto{\pgfqpoint{1.243591in}{1.078106in}}%
\pgfpathlineto{\pgfqpoint{1.230859in}{1.073676in}}%
\pgfpathclose%
\pgfusepath{fill}%
\end{pgfscope}%
\begin{pgfscope}%
\pgfpathrectangle{\pgfqpoint{0.050000in}{0.050000in}}{\pgfqpoint{2.081932in}{2.081932in}}%
\pgfusepath{clip}%
\pgfsetbuttcap%
\pgfsetroundjoin%
\definecolor{currentfill}{rgb}{0.606045,0.850733,0.236712}%
\pgfsetfillcolor{currentfill}%
\pgfsetlinewidth{0.000000pt}%
\definecolor{currentstroke}{rgb}{0.000000,0.000000,0.000000}%
\pgfsetstrokecolor{currentstroke}%
\pgfsetdash{}{0pt}%
\pgfpathmoveto{\pgfqpoint{1.321079in}{1.113057in}}%
\pgfpathlineto{\pgfqpoint{1.313787in}{1.111699in}}%
\pgfpathlineto{\pgfqpoint{1.306888in}{1.109729in}}%
\pgfpathlineto{\pgfqpoint{1.300408in}{1.107157in}}%
\pgfpathlineto{\pgfqpoint{1.294374in}{1.103994in}}%
\pgfpathlineto{\pgfqpoint{1.288807in}{1.100254in}}%
\pgfpathlineto{\pgfqpoint{1.298453in}{1.106778in}}%
\pgfpathlineto{\pgfqpoint{1.307354in}{1.113643in}}%
\pgfpathlineto{\pgfqpoint{1.315476in}{1.120820in}}%
\pgfpathlineto{\pgfqpoint{1.322786in}{1.128279in}}%
\pgfpathlineto{\pgfqpoint{1.329256in}{1.135987in}}%
\pgfpathlineto{\pgfqpoint{1.336119in}{1.140877in}}%
\pgfpathlineto{\pgfqpoint{1.343557in}{1.145289in}}%
\pgfpathlineto{\pgfqpoint{1.351540in}{1.149206in}}%
\pgfpathlineto{\pgfqpoint{1.360038in}{1.152609in}}%
\pgfpathlineto{\pgfqpoint{1.369016in}{1.155485in}}%
\pgfpathlineto{\pgfqpoint{1.361364in}{1.146339in}}%
\pgfpathlineto{\pgfqpoint{1.352710in}{1.137485in}}%
\pgfpathlineto{\pgfqpoint{1.343085in}{1.128963in}}%
\pgfpathlineto{\pgfqpoint{1.332528in}{1.120809in}}%
\pgfpathlineto{\pgfqpoint{1.321079in}{1.113057in}}%
\pgfpathclose%
\pgfusepath{fill}%
\end{pgfscope}%
\begin{pgfscope}%
\pgfpathrectangle{\pgfqpoint{0.050000in}{0.050000in}}{\pgfqpoint{2.081932in}{2.081932in}}%
\pgfusepath{clip}%
\pgfsetbuttcap%
\pgfsetroundjoin%
\definecolor{currentfill}{rgb}{0.267968,0.223549,0.512008}%
\pgfsetfillcolor{currentfill}%
\pgfsetlinewidth{0.000000pt}%
\definecolor{currentstroke}{rgb}{0.000000,0.000000,0.000000}%
\pgfsetstrokecolor{currentstroke}%
\pgfsetdash{}{0pt}%
\pgfpathmoveto{\pgfqpoint{1.499566in}{0.805786in}}%
\pgfpathlineto{\pgfqpoint{1.506821in}{0.807198in}}%
\pgfpathlineto{\pgfqpoint{1.513691in}{0.809226in}}%
\pgfpathlineto{\pgfqpoint{1.520148in}{0.811861in}}%
\pgfpathlineto{\pgfqpoint{1.526166in}{0.815094in}}%
\pgfpathlineto{\pgfqpoint{1.531717in}{0.818915in}}%
\pgfpathlineto{\pgfqpoint{1.554679in}{0.835307in}}%
\pgfpathlineto{\pgfqpoint{1.575754in}{0.852516in}}%
\pgfpathlineto{\pgfqpoint{1.594863in}{0.870462in}}%
\pgfpathlineto{\pgfqpoint{1.611937in}{0.889064in}}%
\pgfpathlineto{\pgfqpoint{1.626916in}{0.908239in}}%
\pgfpathlineto{\pgfqpoint{1.620154in}{0.903268in}}%
\pgfpathlineto{\pgfqpoint{1.612823in}{0.898777in}}%
\pgfpathlineto{\pgfqpoint{1.604954in}{0.894783in}}%
\pgfpathlineto{\pgfqpoint{1.596578in}{0.891301in}}%
\pgfpathlineto{\pgfqpoint{1.587730in}{0.888343in}}%
\pgfpathlineto{\pgfqpoint{1.573827in}{0.870610in}}%
\pgfpathlineto{\pgfqpoint{1.557998in}{0.853411in}}%
\pgfpathlineto{\pgfqpoint{1.540301in}{0.836825in}}%
\pgfpathlineto{\pgfqpoint{1.520798in}{0.820926in}}%
\pgfpathlineto{\pgfqpoint{1.499566in}{0.805786in}}%
\pgfpathclose%
\pgfusepath{fill}%
\end{pgfscope}%
\begin{pgfscope}%
\pgfpathrectangle{\pgfqpoint{0.050000in}{0.050000in}}{\pgfqpoint{2.081932in}{2.081932in}}%
\pgfusepath{clip}%
\pgfsetbuttcap%
\pgfsetroundjoin%
\definecolor{currentfill}{rgb}{0.268510,0.009605,0.335427}%
\pgfsetfillcolor{currentfill}%
\pgfsetlinewidth{0.000000pt}%
\definecolor{currentstroke}{rgb}{0.000000,0.000000,0.000000}%
\pgfsetstrokecolor{currentstroke}%
\pgfsetdash{}{0pt}%
\pgfpathmoveto{\pgfqpoint{1.313106in}{0.777457in}}%
\pgfpathlineto{\pgfqpoint{1.319264in}{0.771357in}}%
\pgfpathlineto{\pgfqpoint{1.325397in}{0.765813in}}%
\pgfpathlineto{\pgfqpoint{1.331479in}{0.760848in}}%
\pgfpathlineto{\pgfqpoint{1.337486in}{0.756483in}}%
\pgfpathlineto{\pgfqpoint{1.343394in}{0.752738in}}%
\pgfpathlineto{\pgfqpoint{1.368819in}{0.761973in}}%
\pgfpathlineto{\pgfqpoint{1.393153in}{0.772167in}}%
\pgfpathlineto{\pgfqpoint{1.416294in}{0.783271in}}%
\pgfpathlineto{\pgfqpoint{1.438147in}{0.795233in}}%
\pgfpathlineto{\pgfqpoint{1.458623in}{0.807998in}}%
\pgfpathlineto{\pgfqpoint{1.449734in}{0.810273in}}%
\pgfpathlineto{\pgfqpoint{1.440694in}{0.813140in}}%
\pgfpathlineto{\pgfqpoint{1.431538in}{0.816586in}}%
\pgfpathlineto{\pgfqpoint{1.422303in}{0.820597in}}%
\pgfpathlineto{\pgfqpoint{1.413026in}{0.825156in}}%
\pgfpathlineto{\pgfqpoint{1.395245in}{0.814130in}}%
\pgfpathlineto{\pgfqpoint{1.376282in}{0.803802in}}%
\pgfpathlineto{\pgfqpoint{1.356216in}{0.794218in}}%
\pgfpathlineto{\pgfqpoint{1.335127in}{0.785423in}}%
\pgfpathlineto{\pgfqpoint{1.313106in}{0.777457in}}%
\pgfpathclose%
\pgfusepath{fill}%
\end{pgfscope}%
\begin{pgfscope}%
\pgfpathrectangle{\pgfqpoint{0.050000in}{0.050000in}}{\pgfqpoint{2.081932in}{2.081932in}}%
\pgfusepath{clip}%
\pgfsetbuttcap%
\pgfsetroundjoin%
\definecolor{currentfill}{rgb}{0.636902,0.856542,0.216620}%
\pgfsetfillcolor{currentfill}%
\pgfsetlinewidth{0.000000pt}%
\definecolor{currentstroke}{rgb}{0.000000,0.000000,0.000000}%
\pgfsetstrokecolor{currentstroke}%
\pgfsetdash{}{0pt}%
\pgfpathmoveto{\pgfqpoint{1.625254in}{1.058713in}}%
\pgfpathlineto{\pgfqpoint{1.618272in}{1.067282in}}%
\pgfpathlineto{\pgfqpoint{1.610711in}{1.075665in}}%
\pgfpathlineto{\pgfqpoint{1.602601in}{1.083827in}}%
\pgfpathlineto{\pgfqpoint{1.593977in}{1.091734in}}%
\pgfpathlineto{\pgfqpoint{1.584873in}{1.099353in}}%
\pgfpathlineto{\pgfqpoint{1.596738in}{1.116980in}}%
\pgfpathlineto{\pgfqpoint{1.606607in}{1.134972in}}%
\pgfpathlineto{\pgfqpoint{1.614452in}{1.153255in}}%
\pgfpathlineto{\pgfqpoint{1.620254in}{1.171749in}}%
\pgfpathlineto{\pgfqpoint{1.624002in}{1.190378in}}%
\pgfpathlineto{\pgfqpoint{1.633759in}{1.184604in}}%
\pgfpathlineto{\pgfqpoint{1.642998in}{1.178455in}}%
\pgfpathlineto{\pgfqpoint{1.651683in}{1.171958in}}%
\pgfpathlineto{\pgfqpoint{1.659777in}{1.165138in}}%
\pgfpathlineto{\pgfqpoint{1.667248in}{1.158025in}}%
\pgfpathlineto{\pgfqpoint{1.663285in}{1.137716in}}%
\pgfpathlineto{\pgfqpoint{1.657091in}{1.117545in}}%
\pgfpathlineto{\pgfqpoint{1.648673in}{1.097598in}}%
\pgfpathlineto{\pgfqpoint{1.638051in}{1.077959in}}%
\pgfpathlineto{\pgfqpoint{1.625254in}{1.058713in}}%
\pgfpathclose%
\pgfusepath{fill}%
\end{pgfscope}%
\begin{pgfscope}%
\pgfpathrectangle{\pgfqpoint{0.050000in}{0.050000in}}{\pgfqpoint{2.081932in}{2.081932in}}%
\pgfusepath{clip}%
\pgfsetbuttcap%
\pgfsetroundjoin%
\definecolor{currentfill}{rgb}{0.296479,0.761561,0.424223}%
\pgfsetfillcolor{currentfill}%
\pgfsetlinewidth{0.000000pt}%
\definecolor{currentstroke}{rgb}{0.000000,0.000000,0.000000}%
\pgfsetstrokecolor{currentstroke}%
\pgfsetdash{}{0pt}%
\pgfpathmoveto{\pgfqpoint{0.967504in}{1.090016in}}%
\pgfpathlineto{\pgfqpoint{0.972049in}{1.086017in}}%
\pgfpathlineto{\pgfqpoint{0.976139in}{1.081446in}}%
\pgfpathlineto{\pgfqpoint{0.979758in}{1.076321in}}%
\pgfpathlineto{\pgfqpoint{0.982895in}{1.070664in}}%
\pgfpathlineto{\pgfqpoint{0.985537in}{1.064498in}}%
\pgfpathlineto{\pgfqpoint{0.995640in}{1.059667in}}%
\pgfpathlineto{\pgfqpoint{1.006258in}{1.055222in}}%
\pgfpathlineto{\pgfqpoint{1.017347in}{1.051184in}}%
\pgfpathlineto{\pgfqpoint{1.028862in}{1.047569in}}%
\pgfpathlineto{\pgfqpoint{1.027073in}{1.053412in}}%
\pgfpathlineto{\pgfqpoint{1.024949in}{1.058683in}}%
\pgfpathlineto{\pgfqpoint{1.022497in}{1.063360in}}%
\pgfpathlineto{\pgfqpoint{1.019726in}{1.067424in}}%
\pgfpathlineto{\pgfqpoint{1.016647in}{1.070856in}}%
\pgfpathlineto{\pgfqpoint{1.003582in}{1.074948in}}%
\pgfpathlineto{\pgfqpoint{0.991001in}{1.079519in}}%
\pgfpathlineto{\pgfqpoint{0.978959in}{1.084549in}}%
\pgfpathlineto{\pgfqpoint{0.967504in}{1.090016in}}%
\pgfpathclose%
\pgfusepath{fill}%
\end{pgfscope}%
\begin{pgfscope}%
\pgfpathrectangle{\pgfqpoint{0.050000in}{0.050000in}}{\pgfqpoint{2.081932in}{2.081932in}}%
\pgfusepath{clip}%
\pgfsetbuttcap%
\pgfsetroundjoin%
\definecolor{currentfill}{rgb}{0.993248,0.906157,0.143936}%
\pgfsetfillcolor{currentfill}%
\pgfsetlinewidth{0.000000pt}%
\definecolor{currentstroke}{rgb}{0.000000,0.000000,0.000000}%
\pgfsetstrokecolor{currentstroke}%
\pgfsetdash{}{0pt}%
\pgfpathmoveto{\pgfqpoint{1.533553in}{1.132077in}}%
\pgfpathlineto{\pgfqpoint{1.522427in}{1.137358in}}%
\pgfpathlineto{\pgfqpoint{1.511120in}{1.142163in}}%
\pgfpathlineto{\pgfqpoint{1.499679in}{1.146471in}}%
\pgfpathlineto{\pgfqpoint{1.488150in}{1.150267in}}%
\pgfpathlineto{\pgfqpoint{1.476582in}{1.153535in}}%
\pgfpathlineto{\pgfqpoint{1.485871in}{1.166980in}}%
\pgfpathlineto{\pgfqpoint{1.493641in}{1.180718in}}%
\pgfpathlineto{\pgfqpoint{1.499865in}{1.194692in}}%
\pgfpathlineto{\pgfqpoint{1.504527in}{1.208842in}}%
\pgfpathlineto{\pgfqpoint{1.507616in}{1.223111in}}%
\pgfpathlineto{\pgfqpoint{1.520077in}{1.222104in}}%
\pgfpathlineto{\pgfqpoint{1.532490in}{1.220567in}}%
\pgfpathlineto{\pgfqpoint{1.544801in}{1.218506in}}%
\pgfpathlineto{\pgfqpoint{1.556962in}{1.215929in}}%
\pgfpathlineto{\pgfqpoint{1.568920in}{1.212846in}}%
\pgfpathlineto{\pgfqpoint{1.565469in}{1.196300in}}%
\pgfpathlineto{\pgfqpoint{1.560192in}{1.179882in}}%
\pgfpathlineto{\pgfqpoint{1.553100in}{1.163660in}}%
\pgfpathlineto{\pgfqpoint{1.544211in}{1.147702in}}%
\pgfpathlineto{\pgfqpoint{1.533553in}{1.132077in}}%
\pgfpathclose%
\pgfusepath{fill}%
\end{pgfscope}%
\begin{pgfscope}%
\pgfpathrectangle{\pgfqpoint{0.050000in}{0.050000in}}{\pgfqpoint{2.081932in}{2.081932in}}%
\pgfusepath{clip}%
\pgfsetbuttcap%
\pgfsetroundjoin%
\definecolor{currentfill}{rgb}{0.267004,0.004874,0.329415}%
\pgfsetfillcolor{currentfill}%
\pgfsetlinewidth{0.000000pt}%
\definecolor{currentstroke}{rgb}{0.000000,0.000000,0.000000}%
\pgfsetstrokecolor{currentstroke}%
\pgfsetdash{}{0pt}%
\pgfpathmoveto{\pgfqpoint{1.180904in}{0.793310in}}%
\pgfpathlineto{\pgfqpoint{1.183147in}{0.784025in}}%
\pgfpathlineto{\pgfqpoint{1.185426in}{0.775138in}}%
\pgfpathlineto{\pgfqpoint{1.187732in}{0.766684in}}%
\pgfpathlineto{\pgfqpoint{1.190056in}{0.758698in}}%
\pgfpathlineto{\pgfqpoint{1.192389in}{0.751212in}}%
\pgfpathlineto{\pgfqpoint{1.217604in}{0.754560in}}%
\pgfpathlineto{\pgfqpoint{1.242390in}{0.758881in}}%
\pgfpathlineto{\pgfqpoint{1.266638in}{0.764155in}}%
\pgfpathlineto{\pgfqpoint{1.290243in}{0.770357in}}%
\pgfpathlineto{\pgfqpoint{1.313106in}{0.777457in}}%
\pgfpathlineto{\pgfqpoint{1.306945in}{0.784087in}}%
\pgfpathlineto{\pgfqpoint{1.300807in}{0.791220in}}%
\pgfpathlineto{\pgfqpoint{1.294715in}{0.798828in}}%
\pgfpathlineto{\pgfqpoint{1.288693in}{0.806878in}}%
\pgfpathlineto{\pgfqpoint{1.282765in}{0.815340in}}%
\pgfpathlineto{\pgfqpoint{1.263460in}{0.809378in}}%
\pgfpathlineto{\pgfqpoint{1.243536in}{0.804171in}}%
\pgfpathlineto{\pgfqpoint{1.223076in}{0.799745in}}%
\pgfpathlineto{\pgfqpoint{1.202169in}{0.796119in}}%
\pgfpathlineto{\pgfqpoint{1.180904in}{0.793310in}}%
\pgfpathclose%
\pgfusepath{fill}%
\end{pgfscope}%
\begin{pgfscope}%
\pgfpathrectangle{\pgfqpoint{0.050000in}{0.050000in}}{\pgfqpoint{2.081932in}{2.081932in}}%
\pgfusepath{clip}%
\pgfsetbuttcap%
\pgfsetroundjoin%
\definecolor{currentfill}{rgb}{0.876168,0.891125,0.095250}%
\pgfsetfillcolor{currentfill}%
\pgfsetlinewidth{0.000000pt}%
\definecolor{currentstroke}{rgb}{0.000000,0.000000,0.000000}%
\pgfsetstrokecolor{currentstroke}%
\pgfsetdash{}{0pt}%
\pgfpathmoveto{\pgfqpoint{1.584873in}{1.099353in}}%
\pgfpathlineto{\pgfqpoint{1.575328in}{1.106653in}}%
\pgfpathlineto{\pgfqpoint{1.565382in}{1.113604in}}%
\pgfpathlineto{\pgfqpoint{1.555075in}{1.120176in}}%
\pgfpathlineto{\pgfqpoint{1.544450in}{1.126342in}}%
\pgfpathlineto{\pgfqpoint{1.533553in}{1.132077in}}%
\pgfpathlineto{\pgfqpoint{1.544211in}{1.147702in}}%
\pgfpathlineto{\pgfqpoint{1.553100in}{1.163660in}}%
\pgfpathlineto{\pgfqpoint{1.560192in}{1.179882in}}%
\pgfpathlineto{\pgfqpoint{1.565469in}{1.196300in}}%
\pgfpathlineto{\pgfqpoint{1.568920in}{1.212846in}}%
\pgfpathlineto{\pgfqpoint{1.580628in}{1.209272in}}%
\pgfpathlineto{\pgfqpoint{1.592036in}{1.205220in}}%
\pgfpathlineto{\pgfqpoint{1.603098in}{1.200707in}}%
\pgfpathlineto{\pgfqpoint{1.613768in}{1.195753in}}%
\pgfpathlineto{\pgfqpoint{1.624002in}{1.190378in}}%
\pgfpathlineto{\pgfqpoint{1.620254in}{1.171749in}}%
\pgfpathlineto{\pgfqpoint{1.614452in}{1.153255in}}%
\pgfpathlineto{\pgfqpoint{1.606607in}{1.134972in}}%
\pgfpathlineto{\pgfqpoint{1.596738in}{1.116980in}}%
\pgfpathlineto{\pgfqpoint{1.584873in}{1.099353in}}%
\pgfpathclose%
\pgfusepath{fill}%
\end{pgfscope}%
\begin{pgfscope}%
\pgfpathrectangle{\pgfqpoint{0.050000in}{0.050000in}}{\pgfqpoint{2.081932in}{2.081932in}}%
\pgfusepath{clip}%
\pgfsetbuttcap%
\pgfsetroundjoin%
\definecolor{currentfill}{rgb}{0.267004,0.004874,0.329415}%
\pgfsetfillcolor{currentfill}%
\pgfsetlinewidth{0.000000pt}%
\definecolor{currentstroke}{rgb}{0.000000,0.000000,0.000000}%
\pgfsetstrokecolor{currentstroke}%
\pgfsetdash{}{0pt}%
\pgfpathmoveto{\pgfqpoint{0.968980in}{0.811079in}}%
\pgfpathlineto{\pgfqpoint{0.963535in}{0.802458in}}%
\pgfpathlineto{\pgfqpoint{0.958003in}{0.794247in}}%
\pgfpathlineto{\pgfqpoint{0.952408in}{0.786476in}}%
\pgfpathlineto{\pgfqpoint{0.946770in}{0.779178in}}%
\pgfpathlineto{\pgfqpoint{0.941111in}{0.772383in}}%
\pgfpathlineto{\pgfqpoint{0.964503in}{0.765910in}}%
\pgfpathlineto{\pgfqpoint{0.988568in}{0.760356in}}%
\pgfpathlineto{\pgfqpoint{1.013203in}{0.755749in}}%
\pgfpathlineto{\pgfqpoint{1.038300in}{0.752108in}}%
\pgfpathlineto{\pgfqpoint{1.063751in}{0.749452in}}%
\pgfpathlineto{\pgfqpoint{1.065517in}{0.756996in}}%
\pgfpathlineto{\pgfqpoint{1.067276in}{0.765039in}}%
\pgfpathlineto{\pgfqpoint{1.069021in}{0.773550in}}%
\pgfpathlineto{\pgfqpoint{1.070746in}{0.782493in}}%
\pgfpathlineto{\pgfqpoint{1.072444in}{0.791833in}}%
\pgfpathlineto{\pgfqpoint{1.050982in}{0.794062in}}%
\pgfpathlineto{\pgfqpoint{1.029815in}{0.797116in}}%
\pgfpathlineto{\pgfqpoint{1.009034in}{0.800983in}}%
\pgfpathlineto{\pgfqpoint{0.988726in}{0.805645in}}%
\pgfpathlineto{\pgfqpoint{0.968980in}{0.811079in}}%
\pgfpathclose%
\pgfusepath{fill}%
\end{pgfscope}%
\begin{pgfscope}%
\pgfpathrectangle{\pgfqpoint{0.050000in}{0.050000in}}{\pgfqpoint{2.081932in}{2.081932in}}%
\pgfusepath{clip}%
\pgfsetbuttcap%
\pgfsetroundjoin%
\definecolor{currentfill}{rgb}{0.124780,0.640461,0.527068}%
\pgfsetfillcolor{currentfill}%
\pgfsetlinewidth{0.000000pt}%
\definecolor{currentstroke}{rgb}{0.000000,0.000000,0.000000}%
\pgfsetstrokecolor{currentstroke}%
\pgfsetdash{}{0pt}%
\pgfpathmoveto{\pgfqpoint{0.544189in}{1.040567in}}%
\pgfpathlineto{\pgfqpoint{0.544269in}{1.048911in}}%
\pgfpathlineto{\pgfqpoint{0.545130in}{1.057336in}}%
\pgfpathlineto{\pgfqpoint{0.546767in}{1.065809in}}%
\pgfpathlineto{\pgfqpoint{0.549177in}{1.074295in}}%
\pgfpathlineto{\pgfqpoint{0.552350in}{1.082759in}}%
\pgfpathlineto{\pgfqpoint{0.560415in}{1.061635in}}%
\pgfpathlineto{\pgfqpoint{0.570805in}{1.040806in}}%
\pgfpathlineto{\pgfqpoint{0.583493in}{1.020359in}}%
\pgfpathlineto{\pgfqpoint{0.598442in}{1.000385in}}%
\pgfpathlineto{\pgfqpoint{0.615605in}{0.980972in}}%
\pgfpathlineto{\pgfqpoint{0.612752in}{0.971826in}}%
\pgfpathlineto{\pgfqpoint{0.610584in}{0.962796in}}%
\pgfpathlineto{\pgfqpoint{0.609111in}{0.953918in}}%
\pgfpathlineto{\pgfqpoint{0.608337in}{0.945230in}}%
\pgfpathlineto{\pgfqpoint{0.608264in}{0.936766in}}%
\pgfpathlineto{\pgfqpoint{0.590869in}{0.956566in}}%
\pgfpathlineto{\pgfqpoint{0.575721in}{0.976937in}}%
\pgfpathlineto{\pgfqpoint{0.562869in}{0.997788in}}%
\pgfpathlineto{\pgfqpoint{0.552350in}{1.019029in}}%
\pgfpathlineto{\pgfqpoint{0.544189in}{1.040567in}}%
\pgfpathclose%
\pgfusepath{fill}%
\end{pgfscope}%
\begin{pgfscope}%
\pgfpathrectangle{\pgfqpoint{0.050000in}{0.050000in}}{\pgfqpoint{2.081932in}{2.081932in}}%
\pgfusepath{clip}%
\pgfsetbuttcap%
\pgfsetroundjoin%
\definecolor{currentfill}{rgb}{0.606045,0.850733,0.236712}%
\pgfsetfillcolor{currentfill}%
\pgfsetlinewidth{0.000000pt}%
\definecolor{currentstroke}{rgb}{0.000000,0.000000,0.000000}%
\pgfsetstrokecolor{currentstroke}%
\pgfsetdash{}{0pt}%
\pgfpathmoveto{\pgfqpoint{0.882907in}{1.140095in}}%
\pgfpathlineto{\pgfqpoint{0.891413in}{1.137768in}}%
\pgfpathlineto{\pgfqpoint{0.899462in}{1.134883in}}%
\pgfpathlineto{\pgfqpoint{0.907022in}{1.131452in}}%
\pgfpathlineto{\pgfqpoint{0.914065in}{1.127492in}}%
\pgfpathlineto{\pgfqpoint{0.920564in}{1.123018in}}%
\pgfpathlineto{\pgfqpoint{0.928444in}{1.115754in}}%
\pgfpathlineto{\pgfqpoint{0.937113in}{1.108793in}}%
\pgfpathlineto{\pgfqpoint{0.946539in}{1.102164in}}%
\pgfpathlineto{\pgfqpoint{0.956682in}{1.095896in}}%
\pgfpathlineto{\pgfqpoint{0.967504in}{1.090016in}}%
\pgfpathlineto{\pgfqpoint{0.962521in}{1.093424in}}%
\pgfpathlineto{\pgfqpoint{0.957119in}{1.096226in}}%
\pgfpathlineto{\pgfqpoint{0.951318in}{1.098411in}}%
\pgfpathlineto{\pgfqpoint{0.945140in}{1.099967in}}%
\pgfpathlineto{\pgfqpoint{0.938610in}{1.100887in}}%
\pgfpathlineto{\pgfqpoint{0.925753in}{1.107877in}}%
\pgfpathlineto{\pgfqpoint{0.913708in}{1.115326in}}%
\pgfpathlineto{\pgfqpoint{0.902524in}{1.123202in}}%
\pgfpathlineto{\pgfqpoint{0.892243in}{1.131470in}}%
\pgfpathlineto{\pgfqpoint{0.882907in}{1.140095in}}%
\pgfpathclose%
\pgfusepath{fill}%
\end{pgfscope}%
\begin{pgfscope}%
\pgfpathrectangle{\pgfqpoint{0.050000in}{0.050000in}}{\pgfqpoint{2.081932in}{2.081932in}}%
\pgfusepath{clip}%
\pgfsetbuttcap%
\pgfsetroundjoin%
\definecolor{currentfill}{rgb}{0.268510,0.009605,0.335427}%
\pgfsetfillcolor{currentfill}%
\pgfsetlinewidth{0.000000pt}%
\definecolor{currentstroke}{rgb}{0.000000,0.000000,0.000000}%
\pgfsetstrokecolor{currentstroke}%
\pgfsetdash{}{0pt}%
\pgfpathmoveto{\pgfqpoint{0.856209in}{0.806808in}}%
\pgfpathlineto{\pgfqpoint{0.847889in}{0.801659in}}%
\pgfpathlineto{\pgfqpoint{0.839606in}{0.797060in}}%
\pgfpathlineto{\pgfqpoint{0.831393in}{0.793032in}}%
\pgfpathlineto{\pgfqpoint{0.823282in}{0.789591in}}%
\pgfpathlineto{\pgfqpoint{0.815306in}{0.786753in}}%
\pgfpathlineto{\pgfqpoint{0.838069in}{0.775388in}}%
\pgfpathlineto{\pgfqpoint{0.862054in}{0.764917in}}%
\pgfpathlineto{\pgfqpoint{0.887161in}{0.755391in}}%
\pgfpathlineto{\pgfqpoint{0.913284in}{0.746854in}}%
\pgfpathlineto{\pgfqpoint{0.918713in}{0.750757in}}%
\pgfpathlineto{\pgfqpoint{0.924232in}{0.755282in}}%
\pgfpathlineto{\pgfqpoint{0.929819in}{0.760410in}}%
\pgfpathlineto{\pgfqpoint{0.935453in}{0.766118in}}%
\pgfpathlineto{\pgfqpoint{0.941111in}{0.772383in}}%
\pgfpathlineto{\pgfqpoint{0.918491in}{0.779745in}}%
\pgfpathlineto{\pgfqpoint{0.896741in}{0.787963in}}%
\pgfpathlineto{\pgfqpoint{0.875952in}{0.796998in}}%
\pgfpathlineto{\pgfqpoint{0.856209in}{0.806808in}}%
\pgfpathclose%
\pgfusepath{fill}%
\end{pgfscope}%
\begin{pgfscope}%
\pgfpathrectangle{\pgfqpoint{0.050000in}{0.050000in}}{\pgfqpoint{2.081932in}{2.081932in}}%
\pgfusepath{clip}%
\pgfsetbuttcap%
\pgfsetroundjoin%
\definecolor{currentfill}{rgb}{0.993248,0.906157,0.143936}%
\pgfsetfillcolor{currentfill}%
\pgfsetlinewidth{0.000000pt}%
\definecolor{currentstroke}{rgb}{0.000000,0.000000,0.000000}%
\pgfsetstrokecolor{currentstroke}%
\pgfsetdash{}{0pt}%
\pgfpathmoveto{\pgfqpoint{0.736893in}{1.198899in}}%
\pgfpathlineto{\pgfqpoint{0.749190in}{1.200151in}}%
\pgfpathlineto{\pgfqpoint{0.761432in}{1.200863in}}%
\pgfpathlineto{\pgfqpoint{0.773568in}{1.201030in}}%
\pgfpathlineto{\pgfqpoint{0.785548in}{1.200653in}}%
\pgfpathlineto{\pgfqpoint{0.797324in}{1.199735in}}%
\pgfpathlineto{\pgfqpoint{0.802256in}{1.187925in}}%
\pgfpathlineto{\pgfqpoint{0.808489in}{1.176304in}}%
\pgfpathlineto{\pgfqpoint{0.816003in}{1.164923in}}%
\pgfpathlineto{\pgfqpoint{0.824772in}{1.153828in}}%
\pgfpathlineto{\pgfqpoint{0.834764in}{1.143066in}}%
\pgfpathlineto{\pgfqpoint{0.824283in}{1.141903in}}%
\pgfpathlineto{\pgfqpoint{0.813614in}{1.140157in}}%
\pgfpathlineto{\pgfqpoint{0.802801in}{1.137834in}}%
\pgfpathlineto{\pgfqpoint{0.791887in}{1.134943in}}%
\pgfpathlineto{\pgfqpoint{0.780919in}{1.131495in}}%
\pgfpathlineto{\pgfqpoint{0.769122in}{1.144308in}}%
\pgfpathlineto{\pgfqpoint{0.758788in}{1.157513in}}%
\pgfpathlineto{\pgfqpoint{0.749954in}{1.171051in}}%
\pgfpathlineto{\pgfqpoint{0.742648in}{1.184866in}}%
\pgfpathlineto{\pgfqpoint{0.736893in}{1.198899in}}%
\pgfpathclose%
\pgfusepath{fill}%
\end{pgfscope}%
\begin{pgfscope}%
\pgfpathrectangle{\pgfqpoint{0.050000in}{0.050000in}}{\pgfqpoint{2.081932in}{2.081932in}}%
\pgfusepath{clip}%
\pgfsetbuttcap%
\pgfsetroundjoin%
\definecolor{currentfill}{rgb}{0.267004,0.004874,0.329415}%
\pgfsetfillcolor{currentfill}%
\pgfsetlinewidth{0.000000pt}%
\definecolor{currentstroke}{rgb}{0.000000,0.000000,0.000000}%
\pgfsetstrokecolor{currentstroke}%
\pgfsetdash{}{0pt}%
\pgfpathmoveto{\pgfqpoint{1.072444in}{0.791833in}}%
\pgfpathlineto{\pgfqpoint{1.070746in}{0.782493in}}%
\pgfpathlineto{\pgfqpoint{1.069021in}{0.773550in}}%
\pgfpathlineto{\pgfqpoint{1.067276in}{0.765039in}}%
\pgfpathlineto{\pgfqpoint{1.065517in}{0.756996in}}%
\pgfpathlineto{\pgfqpoint{1.063751in}{0.749452in}}%
\pgfpathlineto{\pgfqpoint{1.089444in}{0.747793in}}%
\pgfpathlineto{\pgfqpoint{1.115268in}{0.747139in}}%
\pgfpathlineto{\pgfqpoint{1.141109in}{0.747492in}}%
\pgfpathlineto{\pgfqpoint{1.166853in}{0.748852in}}%
\pgfpathlineto{\pgfqpoint{1.192389in}{0.751212in}}%
\pgfpathlineto{\pgfqpoint{1.190056in}{0.758698in}}%
\pgfpathlineto{\pgfqpoint{1.187732in}{0.766684in}}%
\pgfpathlineto{\pgfqpoint{1.185426in}{0.775138in}}%
\pgfpathlineto{\pgfqpoint{1.183147in}{0.784025in}}%
\pgfpathlineto{\pgfqpoint{1.180904in}{0.793310in}}%
\pgfpathlineto{\pgfqpoint{1.159371in}{0.791330in}}%
\pgfpathlineto{\pgfqpoint{1.137665in}{0.790189in}}%
\pgfpathlineto{\pgfqpoint{1.115879in}{0.789892in}}%
\pgfpathlineto{\pgfqpoint{1.094108in}{0.790441in}}%
\pgfpathlineto{\pgfqpoint{1.072444in}{0.791833in}}%
\pgfpathclose%
\pgfusepath{fill}%
\end{pgfscope}%
\begin{pgfscope}%
\pgfpathrectangle{\pgfqpoint{0.050000in}{0.050000in}}{\pgfqpoint{2.081932in}{2.081932in}}%
\pgfusepath{clip}%
\pgfsetbuttcap%
\pgfsetroundjoin%
\definecolor{currentfill}{rgb}{0.296479,0.761561,0.424223}%
\pgfsetfillcolor{currentfill}%
\pgfsetlinewidth{0.000000pt}%
\definecolor{currentstroke}{rgb}{0.000000,0.000000,0.000000}%
\pgfsetstrokecolor{currentstroke}%
\pgfsetdash{}{0pt}%
\pgfpathmoveto{\pgfqpoint{1.161281in}{1.059105in}}%
\pgfpathlineto{\pgfqpoint{1.160014in}{1.056022in}}%
\pgfpathlineto{\pgfqpoint{1.158875in}{1.052271in}}%
\pgfpathlineto{\pgfqpoint{1.157866in}{1.047869in}}%
\pgfpathlineto{\pgfqpoint{1.156992in}{1.042836in}}%
\pgfpathlineto{\pgfqpoint{1.156257in}{1.037191in}}%
\pgfpathlineto{\pgfqpoint{1.169039in}{1.038831in}}%
\pgfpathlineto{\pgfqpoint{1.181612in}{1.040948in}}%
\pgfpathlineto{\pgfqpoint{1.193923in}{1.043533in}}%
\pgfpathlineto{\pgfqpoint{1.205920in}{1.046575in}}%
\pgfpathlineto{\pgfqpoint{1.217555in}{1.050060in}}%
\pgfpathlineto{\pgfqpoint{1.219503in}{1.055951in}}%
\pgfpathlineto{\pgfqpoint{1.221817in}{1.061279in}}%
\pgfpathlineto{\pgfqpoint{1.224487in}{1.066022in}}%
\pgfpathlineto{\pgfqpoint{1.227505in}{1.070160in}}%
\pgfpathlineto{\pgfqpoint{1.230859in}{1.073676in}}%
\pgfpathlineto{\pgfqpoint{1.217657in}{1.069730in}}%
\pgfpathlineto{\pgfqpoint{1.204041in}{1.066286in}}%
\pgfpathlineto{\pgfqpoint{1.190067in}{1.063359in}}%
\pgfpathlineto{\pgfqpoint{1.175794in}{1.060962in}}%
\pgfpathlineto{\pgfqpoint{1.161281in}{1.059105in}}%
\pgfpathclose%
\pgfusepath{fill}%
\end{pgfscope}%
\begin{pgfscope}%
\pgfpathrectangle{\pgfqpoint{0.050000in}{0.050000in}}{\pgfqpoint{2.081932in}{2.081932in}}%
\pgfusepath{clip}%
\pgfsetbuttcap%
\pgfsetroundjoin%
\definecolor{currentfill}{rgb}{0.296479,0.761561,0.424223}%
\pgfsetfillcolor{currentfill}%
\pgfsetlinewidth{0.000000pt}%
\definecolor{currentstroke}{rgb}{0.000000,0.000000,0.000000}%
\pgfsetstrokecolor{currentstroke}%
\pgfsetdash{}{0pt}%
\pgfpathmoveto{\pgfqpoint{1.016647in}{1.070856in}}%
\pgfpathlineto{\pgfqpoint{1.019726in}{1.067424in}}%
\pgfpathlineto{\pgfqpoint{1.022497in}{1.063360in}}%
\pgfpathlineto{\pgfqpoint{1.024949in}{1.058683in}}%
\pgfpathlineto{\pgfqpoint{1.027073in}{1.053412in}}%
\pgfpathlineto{\pgfqpoint{1.028862in}{1.047569in}}%
\pgfpathlineto{\pgfqpoint{1.040755in}{1.044393in}}%
\pgfpathlineto{\pgfqpoint{1.052977in}{1.041671in}}%
\pgfpathlineto{\pgfqpoint{1.065476in}{1.039413in}}%
\pgfpathlineto{\pgfqpoint{1.078201in}{1.037630in}}%
\pgfpathlineto{\pgfqpoint{1.091098in}{1.036330in}}%
\pgfpathlineto{\pgfqpoint{1.090542in}{1.041958in}}%
\pgfpathlineto{\pgfqpoint{1.089880in}{1.046972in}}%
\pgfpathlineto{\pgfqpoint{1.089117in}{1.051351in}}%
\pgfpathlineto{\pgfqpoint{1.088255in}{1.055075in}}%
\pgfpathlineto{\pgfqpoint{1.087296in}{1.058130in}}%
\pgfpathlineto{\pgfqpoint{1.072653in}{1.059602in}}%
\pgfpathlineto{\pgfqpoint{1.058206in}{1.061621in}}%
\pgfpathlineto{\pgfqpoint{1.044017in}{1.064178in}}%
\pgfpathlineto{\pgfqpoint{1.030144in}{1.067261in}}%
\pgfpathlineto{\pgfqpoint{1.016647in}{1.070856in}}%
\pgfpathclose%
\pgfusepath{fill}%
\end{pgfscope}%
\begin{pgfscope}%
\pgfpathrectangle{\pgfqpoint{0.050000in}{0.050000in}}{\pgfqpoint{2.081932in}{2.081932in}}%
\pgfusepath{clip}%
\pgfsetbuttcap%
\pgfsetroundjoin%
\definecolor{currentfill}{rgb}{0.296479,0.761561,0.424223}%
\pgfsetfillcolor{currentfill}%
\pgfsetlinewidth{0.000000pt}%
\definecolor{currentstroke}{rgb}{0.000000,0.000000,0.000000}%
\pgfsetstrokecolor{currentstroke}%
\pgfsetdash{}{0pt}%
\pgfpathmoveto{\pgfqpoint{1.087296in}{1.058130in}}%
\pgfpathlineto{\pgfqpoint{1.088255in}{1.055075in}}%
\pgfpathlineto{\pgfqpoint{1.089117in}{1.051351in}}%
\pgfpathlineto{\pgfqpoint{1.089880in}{1.046972in}}%
\pgfpathlineto{\pgfqpoint{1.090542in}{1.041958in}}%
\pgfpathlineto{\pgfqpoint{1.091098in}{1.036330in}}%
\pgfpathlineto{\pgfqpoint{1.104113in}{1.035518in}}%
\pgfpathlineto{\pgfqpoint{1.117192in}{1.035197in}}%
\pgfpathlineto{\pgfqpoint{1.130278in}{1.035370in}}%
\pgfpathlineto{\pgfqpoint{1.143318in}{1.036036in}}%
\pgfpathlineto{\pgfqpoint{1.156257in}{1.037191in}}%
\pgfpathlineto{\pgfqpoint{1.156992in}{1.042836in}}%
\pgfpathlineto{\pgfqpoint{1.157866in}{1.047869in}}%
\pgfpathlineto{\pgfqpoint{1.158875in}{1.052271in}}%
\pgfpathlineto{\pgfqpoint{1.160014in}{1.056022in}}%
\pgfpathlineto{\pgfqpoint{1.161281in}{1.059105in}}%
\pgfpathlineto{\pgfqpoint{1.146591in}{1.057797in}}%
\pgfpathlineto{\pgfqpoint{1.131784in}{1.057044in}}%
\pgfpathlineto{\pgfqpoint{1.116925in}{1.056848in}}%
\pgfpathlineto{\pgfqpoint{1.102074in}{1.057210in}}%
\pgfpathlineto{\pgfqpoint{1.087296in}{1.058130in}}%
\pgfpathclose%
\pgfusepath{fill}%
\end{pgfscope}%
\begin{pgfscope}%
\pgfpathrectangle{\pgfqpoint{0.050000in}{0.050000in}}{\pgfqpoint{2.081932in}{2.081932in}}%
\pgfusepath{clip}%
\pgfsetbuttcap%
\pgfsetroundjoin%
\definecolor{currentfill}{rgb}{0.327796,0.773980,0.406640}%
\pgfsetfillcolor{currentfill}%
\pgfsetlinewidth{0.000000pt}%
\definecolor{currentstroke}{rgb}{0.000000,0.000000,0.000000}%
\pgfsetstrokecolor{currentstroke}%
\pgfsetdash{}{0pt}%
\pgfpathmoveto{\pgfqpoint{0.552350in}{1.082759in}}%
\pgfpathlineto{\pgfqpoint{0.556274in}{1.091166in}}%
\pgfpathlineto{\pgfqpoint{0.560934in}{1.099483in}}%
\pgfpathlineto{\pgfqpoint{0.566312in}{1.107674in}}%
\pgfpathlineto{\pgfqpoint{0.572386in}{1.115706in}}%
\pgfpathlineto{\pgfqpoint{0.579133in}{1.123545in}}%
\pgfpathlineto{\pgfqpoint{0.586883in}{1.103523in}}%
\pgfpathlineto{\pgfqpoint{0.596843in}{1.083783in}}%
\pgfpathlineto{\pgfqpoint{0.608988in}{1.064411in}}%
\pgfpathlineto{\pgfqpoint{0.623283in}{1.045491in}}%
\pgfpathlineto{\pgfqpoint{0.639681in}{1.027107in}}%
\pgfpathlineto{\pgfqpoint{0.633619in}{1.017954in}}%
\pgfpathlineto{\pgfqpoint{0.628159in}{1.008726in}}%
\pgfpathlineto{\pgfqpoint{0.623325in}{0.999461in}}%
\pgfpathlineto{\pgfqpoint{0.619134in}{0.990197in}}%
\pgfpathlineto{\pgfqpoint{0.615605in}{0.980972in}}%
\pgfpathlineto{\pgfqpoint{0.598442in}{1.000385in}}%
\pgfpathlineto{\pgfqpoint{0.583493in}{1.020359in}}%
\pgfpathlineto{\pgfqpoint{0.570805in}{1.040806in}}%
\pgfpathlineto{\pgfqpoint{0.560415in}{1.061635in}}%
\pgfpathlineto{\pgfqpoint{0.552350in}{1.082759in}}%
\pgfpathclose%
\pgfusepath{fill}%
\end{pgfscope}%
\begin{pgfscope}%
\pgfpathrectangle{\pgfqpoint{0.050000in}{0.050000in}}{\pgfqpoint{2.081932in}{2.081932in}}%
\pgfusepath{clip}%
\pgfsetbuttcap%
\pgfsetroundjoin%
\definecolor{currentfill}{rgb}{0.267968,0.223549,0.512008}%
\pgfsetfillcolor{currentfill}%
\pgfsetlinewidth{0.000000pt}%
\definecolor{currentstroke}{rgb}{0.000000,0.000000,0.000000}%
\pgfsetstrokecolor{currentstroke}%
\pgfsetdash{}{0pt}%
\pgfpathmoveto{\pgfqpoint{0.675357in}{0.858483in}}%
\pgfpathlineto{\pgfqpoint{0.666946in}{0.860886in}}%
\pgfpathlineto{\pgfqpoint{0.658983in}{0.863845in}}%
\pgfpathlineto{\pgfqpoint{0.651501in}{0.867351in}}%
\pgfpathlineto{\pgfqpoint{0.644529in}{0.871391in}}%
\pgfpathlineto{\pgfqpoint{0.638099in}{0.875949in}}%
\pgfpathlineto{\pgfqpoint{0.656602in}{0.857798in}}%
\pgfpathlineto{\pgfqpoint{0.677092in}{0.840361in}}%
\pgfpathlineto{\pgfqpoint{0.699495in}{0.823717in}}%
\pgfpathlineto{\pgfqpoint{0.723722in}{0.807944in}}%
\pgfpathlineto{\pgfqpoint{0.749679in}{0.793115in}}%
\pgfpathlineto{\pgfqpoint{0.754665in}{0.789631in}}%
\pgfpathlineto{\pgfqpoint{0.760070in}{0.786765in}}%
\pgfpathlineto{\pgfqpoint{0.765870in}{0.784528in}}%
\pgfpathlineto{\pgfqpoint{0.772040in}{0.782926in}}%
\pgfpathlineto{\pgfqpoint{0.778554in}{0.781965in}}%
\pgfpathlineto{\pgfqpoint{0.754577in}{0.795655in}}%
\pgfpathlineto{\pgfqpoint{0.732183in}{0.810220in}}%
\pgfpathlineto{\pgfqpoint{0.711464in}{0.825594in}}%
\pgfpathlineto{\pgfqpoint{0.692498in}{0.841707in}}%
\pgfpathlineto{\pgfqpoint{0.675357in}{0.858483in}}%
\pgfpathclose%
\pgfusepath{fill}%
\end{pgfscope}%
\begin{pgfscope}%
\pgfpathrectangle{\pgfqpoint{0.050000in}{0.050000in}}{\pgfqpoint{2.081932in}{2.081932in}}%
\pgfusepath{clip}%
\pgfsetbuttcap%
\pgfsetroundjoin%
\definecolor{currentfill}{rgb}{0.855810,0.888601,0.097452}%
\pgfsetfillcolor{currentfill}%
\pgfsetlinewidth{0.000000pt}%
\definecolor{currentstroke}{rgb}{0.000000,0.000000,0.000000}%
\pgfsetstrokecolor{currentstroke}%
\pgfsetdash{}{0pt}%
\pgfpathmoveto{\pgfqpoint{1.362389in}{1.110509in}}%
\pgfpathlineto{\pgfqpoint{1.353588in}{1.112266in}}%
\pgfpathlineto{\pgfqpoint{1.345023in}{1.113401in}}%
\pgfpathlineto{\pgfqpoint{1.336728in}{1.113911in}}%
\pgfpathlineto{\pgfqpoint{1.328737in}{1.113795in}}%
\pgfpathlineto{\pgfqpoint{1.321079in}{1.113057in}}%
\pgfpathlineto{\pgfqpoint{1.332528in}{1.120809in}}%
\pgfpathlineto{\pgfqpoint{1.343085in}{1.128963in}}%
\pgfpathlineto{\pgfqpoint{1.352710in}{1.137485in}}%
\pgfpathlineto{\pgfqpoint{1.361364in}{1.146339in}}%
\pgfpathlineto{\pgfqpoint{1.369016in}{1.155485in}}%
\pgfpathlineto{\pgfqpoint{1.378441in}{1.157820in}}%
\pgfpathlineto{\pgfqpoint{1.388273in}{1.159604in}}%
\pgfpathlineto{\pgfqpoint{1.398473in}{1.160829in}}%
\pgfpathlineto{\pgfqpoint{1.409001in}{1.161489in}}%
\pgfpathlineto{\pgfqpoint{1.419814in}{1.161579in}}%
\pgfpathlineto{\pgfqpoint{1.410673in}{1.150579in}}%
\pgfpathlineto{\pgfqpoint{1.400319in}{1.139927in}}%
\pgfpathlineto{\pgfqpoint{1.388789in}{1.129668in}}%
\pgfpathlineto{\pgfqpoint{1.376130in}{1.119848in}}%
\pgfpathlineto{\pgfqpoint{1.362389in}{1.110509in}}%
\pgfpathclose%
\pgfusepath{fill}%
\end{pgfscope}%
\begin{pgfscope}%
\pgfpathrectangle{\pgfqpoint{0.050000in}{0.050000in}}{\pgfqpoint{2.081932in}{2.081932in}}%
\pgfusepath{clip}%
\pgfsetbuttcap%
\pgfsetroundjoin%
\definecolor{currentfill}{rgb}{0.206756,0.371758,0.553117}%
\pgfsetfillcolor{currentfill}%
\pgfsetlinewidth{0.000000pt}%
\definecolor{currentstroke}{rgb}{0.000000,0.000000,0.000000}%
\pgfsetstrokecolor{currentstroke}%
\pgfsetdash{}{0pt}%
\pgfpathmoveto{\pgfqpoint{1.531717in}{0.818915in}}%
\pgfpathlineto{\pgfqpoint{1.536779in}{0.823309in}}%
\pgfpathlineto{\pgfqpoint{1.541330in}{0.828259in}}%
\pgfpathlineto{\pgfqpoint{1.545350in}{0.833746in}}%
\pgfpathlineto{\pgfqpoint{1.548821in}{0.839749in}}%
\pgfpathlineto{\pgfqpoint{1.551728in}{0.846245in}}%
\pgfpathlineto{\pgfqpoint{1.575761in}{0.863374in}}%
\pgfpathlineto{\pgfqpoint{1.597809in}{0.881355in}}%
\pgfpathlineto{\pgfqpoint{1.617790in}{0.900101in}}%
\pgfpathlineto{\pgfqpoint{1.635632in}{0.919529in}}%
\pgfpathlineto{\pgfqpoint{1.651273in}{0.939551in}}%
\pgfpathlineto{\pgfqpoint{1.647736in}{0.932506in}}%
\pgfpathlineto{\pgfqpoint{1.643512in}{0.925827in}}%
\pgfpathlineto{\pgfqpoint{1.638620in}{0.919540in}}%
\pgfpathlineto{\pgfqpoint{1.633080in}{0.913670in}}%
\pgfpathlineto{\pgfqpoint{1.626916in}{0.908239in}}%
\pgfpathlineto{\pgfqpoint{1.611937in}{0.889064in}}%
\pgfpathlineto{\pgfqpoint{1.594863in}{0.870462in}}%
\pgfpathlineto{\pgfqpoint{1.575754in}{0.852516in}}%
\pgfpathlineto{\pgfqpoint{1.554679in}{0.835307in}}%
\pgfpathlineto{\pgfqpoint{1.531717in}{0.818915in}}%
\pgfpathclose%
\pgfusepath{fill}%
\end{pgfscope}%
\begin{pgfscope}%
\pgfpathrectangle{\pgfqpoint{0.050000in}{0.050000in}}{\pgfqpoint{2.081932in}{2.081932in}}%
\pgfusepath{clip}%
\pgfsetbuttcap%
\pgfsetroundjoin%
\definecolor{currentfill}{rgb}{0.993248,0.906157,0.143936}%
\pgfsetfillcolor{currentfill}%
\pgfsetlinewidth{0.000000pt}%
\definecolor{currentstroke}{rgb}{0.000000,0.000000,0.000000}%
\pgfsetstrokecolor{currentstroke}%
\pgfsetdash{}{0pt}%
\pgfpathmoveto{\pgfqpoint{0.676354in}{1.184764in}}%
\pgfpathlineto{\pgfqpoint{0.688168in}{1.188606in}}%
\pgfpathlineto{\pgfqpoint{0.700178in}{1.191953in}}%
\pgfpathlineto{\pgfqpoint{0.712337in}{1.194792in}}%
\pgfpathlineto{\pgfqpoint{0.724592in}{1.197110in}}%
\pgfpathlineto{\pgfqpoint{0.736893in}{1.198899in}}%
\pgfpathlineto{\pgfqpoint{0.742648in}{1.184866in}}%
\pgfpathlineto{\pgfqpoint{0.749954in}{1.171051in}}%
\pgfpathlineto{\pgfqpoint{0.758788in}{1.157513in}}%
\pgfpathlineto{\pgfqpoint{0.769122in}{1.144308in}}%
\pgfpathlineto{\pgfqpoint{0.780919in}{1.131495in}}%
\pgfpathlineto{\pgfqpoint{0.769941in}{1.127503in}}%
\pgfpathlineto{\pgfqpoint{0.758998in}{1.122983in}}%
\pgfpathlineto{\pgfqpoint{0.748136in}{1.117953in}}%
\pgfpathlineto{\pgfqpoint{0.737400in}{1.112433in}}%
\pgfpathlineto{\pgfqpoint{0.726835in}{1.106446in}}%
\pgfpathlineto{\pgfqpoint{0.713253in}{1.121350in}}%
\pgfpathlineto{\pgfqpoint{0.701378in}{1.136702in}}%
\pgfpathlineto{\pgfqpoint{0.691250in}{1.152433in}}%
\pgfpathlineto{\pgfqpoint{0.682900in}{1.168477in}}%
\pgfpathlineto{\pgfqpoint{0.676354in}{1.184764in}}%
\pgfpathclose%
\pgfusepath{fill}%
\end{pgfscope}%
\begin{pgfscope}%
\pgfpathrectangle{\pgfqpoint{0.050000in}{0.050000in}}{\pgfqpoint{2.081932in}{2.081932in}}%
\pgfusepath{clip}%
\pgfsetbuttcap%
\pgfsetroundjoin%
\definecolor{currentfill}{rgb}{0.606045,0.850733,0.236712}%
\pgfsetfillcolor{currentfill}%
\pgfsetlinewidth{0.000000pt}%
\definecolor{currentstroke}{rgb}{0.000000,0.000000,0.000000}%
\pgfsetstrokecolor{currentstroke}%
\pgfsetdash{}{0pt}%
\pgfpathmoveto{\pgfqpoint{1.252189in}{1.081452in}}%
\pgfpathlineto{\pgfqpoint{1.247367in}{1.081235in}}%
\pgfpathlineto{\pgfqpoint{1.242806in}{1.080342in}}%
\pgfpathlineto{\pgfqpoint{1.238523in}{1.078778in}}%
\pgfpathlineto{\pgfqpoint{1.234536in}{1.076553in}}%
\pgfpathlineto{\pgfqpoint{1.230859in}{1.073676in}}%
\pgfpathlineto{\pgfqpoint{1.243591in}{1.078106in}}%
\pgfpathlineto{\pgfqpoint{1.255800in}{1.083001in}}%
\pgfpathlineto{\pgfqpoint{1.267438in}{1.088340in}}%
\pgfpathlineto{\pgfqpoint{1.278455in}{1.094099in}}%
\pgfpathlineto{\pgfqpoint{1.288807in}{1.100254in}}%
\pgfpathlineto{\pgfqpoint{1.294374in}{1.103994in}}%
\pgfpathlineto{\pgfqpoint{1.300408in}{1.107157in}}%
\pgfpathlineto{\pgfqpoint{1.306888in}{1.109729in}}%
\pgfpathlineto{\pgfqpoint{1.313787in}{1.111699in}}%
\pgfpathlineto{\pgfqpoint{1.321079in}{1.113057in}}%
\pgfpathlineto{\pgfqpoint{1.308785in}{1.105741in}}%
\pgfpathlineto{\pgfqpoint{1.295693in}{1.098894in}}%
\pgfpathlineto{\pgfqpoint{1.281859in}{1.092545in}}%
\pgfpathlineto{\pgfqpoint{1.267337in}{1.086722in}}%
\pgfpathlineto{\pgfqpoint{1.252189in}{1.081452in}}%
\pgfpathclose%
\pgfusepath{fill}%
\end{pgfscope}%
\begin{pgfscope}%
\pgfpathrectangle{\pgfqpoint{0.050000in}{0.050000in}}{\pgfqpoint{2.081932in}{2.081932in}}%
\pgfusepath{clip}%
\pgfsetbuttcap%
\pgfsetroundjoin%
\definecolor{currentfill}{rgb}{0.636902,0.856542,0.216620}%
\pgfsetfillcolor{currentfill}%
\pgfsetlinewidth{0.000000pt}%
\definecolor{currentstroke}{rgb}{0.000000,0.000000,0.000000}%
\pgfsetstrokecolor{currentstroke}%
\pgfsetdash{}{0pt}%
\pgfpathmoveto{\pgfqpoint{0.579133in}{1.123545in}}%
\pgfpathlineto{\pgfqpoint{0.586526in}{1.131160in}}%
\pgfpathlineto{\pgfqpoint{0.594533in}{1.138518in}}%
\pgfpathlineto{\pgfqpoint{0.603124in}{1.145590in}}%
\pgfpathlineto{\pgfqpoint{0.612262in}{1.152345in}}%
\pgfpathlineto{\pgfqpoint{0.621910in}{1.158755in}}%
\pgfpathlineto{\pgfqpoint{0.629141in}{1.140402in}}%
\pgfpathlineto{\pgfqpoint{0.638402in}{1.122314in}}%
\pgfpathlineto{\pgfqpoint{0.649670in}{1.104571in}}%
\pgfpathlineto{\pgfqpoint{0.662908in}{1.087248in}}%
\pgfpathlineto{\pgfqpoint{0.678074in}{1.070421in}}%
\pgfpathlineto{\pgfqpoint{0.669421in}{1.062207in}}%
\pgfpathlineto{\pgfqpoint{0.661222in}{1.053732in}}%
\pgfpathlineto{\pgfqpoint{0.653511in}{1.045033in}}%
\pgfpathlineto{\pgfqpoint{0.646321in}{1.036146in}}%
\pgfpathlineto{\pgfqpoint{0.639681in}{1.027107in}}%
\pgfpathlineto{\pgfqpoint{0.623283in}{1.045491in}}%
\pgfpathlineto{\pgfqpoint{0.608988in}{1.064411in}}%
\pgfpathlineto{\pgfqpoint{0.596843in}{1.083783in}}%
\pgfpathlineto{\pgfqpoint{0.586883in}{1.103523in}}%
\pgfpathlineto{\pgfqpoint{0.579133in}{1.123545in}}%
\pgfpathclose%
\pgfusepath{fill}%
\end{pgfscope}%
\begin{pgfscope}%
\pgfpathrectangle{\pgfqpoint{0.050000in}{0.050000in}}{\pgfqpoint{2.081932in}{2.081932in}}%
\pgfusepath{clip}%
\pgfsetbuttcap%
\pgfsetroundjoin%
\definecolor{currentfill}{rgb}{0.282327,0.094955,0.417331}%
\pgfsetfillcolor{currentfill}%
\pgfsetlinewidth{0.000000pt}%
\definecolor{currentstroke}{rgb}{0.000000,0.000000,0.000000}%
\pgfsetstrokecolor{currentstroke}%
\pgfsetdash{}{0pt}%
\pgfpathmoveto{\pgfqpoint{1.343394in}{0.752738in}}%
\pgfpathlineto{\pgfqpoint{1.349180in}{0.749628in}}%
\pgfpathlineto{\pgfqpoint{1.354820in}{0.747169in}}%
\pgfpathlineto{\pgfqpoint{1.360291in}{0.745371in}}%
\pgfpathlineto{\pgfqpoint{1.365571in}{0.744244in}}%
\pgfpathlineto{\pgfqpoint{1.370637in}{0.743795in}}%
\pgfpathlineto{\pgfqpoint{1.399113in}{0.754163in}}%
\pgfpathlineto{\pgfqpoint{1.426356in}{0.765602in}}%
\pgfpathlineto{\pgfqpoint{1.452249in}{0.778059in}}%
\pgfpathlineto{\pgfqpoint{1.476685in}{0.791475in}}%
\pgfpathlineto{\pgfqpoint{1.499566in}{0.805786in}}%
\pgfpathlineto{\pgfqpoint{1.491957in}{0.804992in}}%
\pgfpathlineto{\pgfqpoint{1.484025in}{0.804818in}}%
\pgfpathlineto{\pgfqpoint{1.475803in}{0.805264in}}%
\pgfpathlineto{\pgfqpoint{1.467324in}{0.806325in}}%
\pgfpathlineto{\pgfqpoint{1.458623in}{0.807998in}}%
\pgfpathlineto{\pgfqpoint{1.438147in}{0.795233in}}%
\pgfpathlineto{\pgfqpoint{1.416294in}{0.783271in}}%
\pgfpathlineto{\pgfqpoint{1.393153in}{0.772167in}}%
\pgfpathlineto{\pgfqpoint{1.368819in}{0.761973in}}%
\pgfpathlineto{\pgfqpoint{1.343394in}{0.752738in}}%
\pgfpathclose%
\pgfusepath{fill}%
\end{pgfscope}%
\begin{pgfscope}%
\pgfpathrectangle{\pgfqpoint{0.050000in}{0.050000in}}{\pgfqpoint{2.081932in}{2.081932in}}%
\pgfusepath{clip}%
\pgfsetbuttcap%
\pgfsetroundjoin%
\definecolor{currentfill}{rgb}{0.876168,0.891125,0.095250}%
\pgfsetfillcolor{currentfill}%
\pgfsetlinewidth{0.000000pt}%
\definecolor{currentstroke}{rgb}{0.000000,0.000000,0.000000}%
\pgfsetstrokecolor{currentstroke}%
\pgfsetdash{}{0pt}%
\pgfpathmoveto{\pgfqpoint{0.621910in}{1.158755in}}%
\pgfpathlineto{\pgfqpoint{0.632030in}{1.164795in}}%
\pgfpathlineto{\pgfqpoint{0.642578in}{1.170438in}}%
\pgfpathlineto{\pgfqpoint{0.653512in}{1.175661in}}%
\pgfpathlineto{\pgfqpoint{0.664786in}{1.180444in}}%
\pgfpathlineto{\pgfqpoint{0.676354in}{1.184764in}}%
\pgfpathlineto{\pgfqpoint{0.682900in}{1.168477in}}%
\pgfpathlineto{\pgfqpoint{0.691250in}{1.152433in}}%
\pgfpathlineto{\pgfqpoint{0.701378in}{1.136702in}}%
\pgfpathlineto{\pgfqpoint{0.713253in}{1.121350in}}%
\pgfpathlineto{\pgfqpoint{0.726835in}{1.106446in}}%
\pgfpathlineto{\pgfqpoint{0.716484in}{1.100016in}}%
\pgfpathlineto{\pgfqpoint{0.706391in}{1.093170in}}%
\pgfpathlineto{\pgfqpoint{0.696598in}{1.085935in}}%
\pgfpathlineto{\pgfqpoint{0.687146in}{1.078342in}}%
\pgfpathlineto{\pgfqpoint{0.678074in}{1.070421in}}%
\pgfpathlineto{\pgfqpoint{0.662908in}{1.087248in}}%
\pgfpathlineto{\pgfqpoint{0.649670in}{1.104571in}}%
\pgfpathlineto{\pgfqpoint{0.638402in}{1.122314in}}%
\pgfpathlineto{\pgfqpoint{0.629141in}{1.140402in}}%
\pgfpathlineto{\pgfqpoint{0.621910in}{1.158755in}}%
\pgfpathclose%
\pgfusepath{fill}%
\end{pgfscope}%
\begin{pgfscope}%
\pgfpathrectangle{\pgfqpoint{0.050000in}{0.050000in}}{\pgfqpoint{2.081932in}{2.081932in}}%
\pgfusepath{clip}%
\pgfsetbuttcap%
\pgfsetroundjoin%
\definecolor{currentfill}{rgb}{0.606045,0.850733,0.236712}%
\pgfsetfillcolor{currentfill}%
\pgfsetlinewidth{0.000000pt}%
\definecolor{currentstroke}{rgb}{0.000000,0.000000,0.000000}%
\pgfsetstrokecolor{currentstroke}%
\pgfsetdash{}{0pt}%
\pgfpathmoveto{\pgfqpoint{0.938610in}{1.100887in}}%
\pgfpathlineto{\pgfqpoint{0.945140in}{1.099967in}}%
\pgfpathlineto{\pgfqpoint{0.951318in}{1.098411in}}%
\pgfpathlineto{\pgfqpoint{0.957119in}{1.096226in}}%
\pgfpathlineto{\pgfqpoint{0.962521in}{1.093424in}}%
\pgfpathlineto{\pgfqpoint{0.967504in}{1.090016in}}%
\pgfpathlineto{\pgfqpoint{0.978959in}{1.084549in}}%
\pgfpathlineto{\pgfqpoint{0.991001in}{1.079519in}}%
\pgfpathlineto{\pgfqpoint{1.003582in}{1.074948in}}%
\pgfpathlineto{\pgfqpoint{1.016647in}{1.070856in}}%
\pgfpathlineto{\pgfqpoint{1.013271in}{1.073641in}}%
\pgfpathlineto{\pgfqpoint{1.009610in}{1.075767in}}%
\pgfpathlineto{\pgfqpoint{1.005678in}{1.077223in}}%
\pgfpathlineto{\pgfqpoint{1.001490in}{1.078001in}}%
\pgfpathlineto{\pgfqpoint{0.997061in}{1.078096in}}%
\pgfpathlineto{\pgfqpoint{0.981513in}{1.082966in}}%
\pgfpathlineto{\pgfqpoint{0.966547in}{1.088403in}}%
\pgfpathlineto{\pgfqpoint{0.952226in}{1.094386in}}%
\pgfpathlineto{\pgfqpoint{0.938610in}{1.100887in}}%
\pgfpathclose%
\pgfusepath{fill}%
\end{pgfscope}%
\begin{pgfscope}%
\pgfpathrectangle{\pgfqpoint{0.050000in}{0.050000in}}{\pgfqpoint{2.081932in}{2.081932in}}%
\pgfusepath{clip}%
\pgfsetbuttcap%
\pgfsetroundjoin%
\definecolor{currentfill}{rgb}{0.268510,0.009605,0.335427}%
\pgfsetfillcolor{currentfill}%
\pgfsetlinewidth{0.000000pt}%
\definecolor{currentstroke}{rgb}{0.000000,0.000000,0.000000}%
\pgfsetstrokecolor{currentstroke}%
\pgfsetdash{}{0pt}%
\pgfpathmoveto{\pgfqpoint{1.192389in}{0.751212in}}%
\pgfpathlineto{\pgfqpoint{1.194721in}{0.744257in}}%
\pgfpathlineto{\pgfqpoint{1.197045in}{0.737861in}}%
\pgfpathlineto{\pgfqpoint{1.199349in}{0.732052in}}%
\pgfpathlineto{\pgfqpoint{1.201626in}{0.726855in}}%
\pgfpathlineto{\pgfqpoint{1.203865in}{0.722291in}}%
\pgfpathlineto{\pgfqpoint{1.233026in}{0.726177in}}%
\pgfpathlineto{\pgfqpoint{1.261683in}{0.731191in}}%
\pgfpathlineto{\pgfqpoint{1.289711in}{0.737310in}}%
\pgfpathlineto{\pgfqpoint{1.316987in}{0.744503in}}%
\pgfpathlineto{\pgfqpoint{1.343394in}{0.752738in}}%
\pgfpathlineto{\pgfqpoint{1.337486in}{0.756483in}}%
\pgfpathlineto{\pgfqpoint{1.331479in}{0.760848in}}%
\pgfpathlineto{\pgfqpoint{1.325397in}{0.765813in}}%
\pgfpathlineto{\pgfqpoint{1.319264in}{0.771357in}}%
\pgfpathlineto{\pgfqpoint{1.313106in}{0.777457in}}%
\pgfpathlineto{\pgfqpoint{1.290243in}{0.770357in}}%
\pgfpathlineto{\pgfqpoint{1.266638in}{0.764155in}}%
\pgfpathlineto{\pgfqpoint{1.242390in}{0.758881in}}%
\pgfpathlineto{\pgfqpoint{1.217604in}{0.754560in}}%
\pgfpathlineto{\pgfqpoint{1.192389in}{0.751212in}}%
\pgfpathclose%
\pgfusepath{fill}%
\end{pgfscope}%
\begin{pgfscope}%
\pgfpathrectangle{\pgfqpoint{0.050000in}{0.050000in}}{\pgfqpoint{2.081932in}{2.081932in}}%
\pgfusepath{clip}%
\pgfsetbuttcap%
\pgfsetroundjoin%
\definecolor{currentfill}{rgb}{0.855810,0.888601,0.097452}%
\pgfsetfillcolor{currentfill}%
\pgfsetlinewidth{0.000000pt}%
\definecolor{currentstroke}{rgb}{0.000000,0.000000,0.000000}%
\pgfsetstrokecolor{currentstroke}%
\pgfsetdash{}{0pt}%
\pgfpathmoveto{\pgfqpoint{0.834764in}{1.143066in}}%
\pgfpathlineto{\pgfqpoint{0.845015in}{1.143643in}}%
\pgfpathlineto{\pgfqpoint{0.854995in}{1.143632in}}%
\pgfpathlineto{\pgfqpoint{0.864662in}{1.143034in}}%
\pgfpathlineto{\pgfqpoint{0.873978in}{1.141853in}}%
\pgfpathlineto{\pgfqpoint{0.882907in}{1.140095in}}%
\pgfpathlineto{\pgfqpoint{0.892243in}{1.131470in}}%
\pgfpathlineto{\pgfqpoint{0.902524in}{1.123202in}}%
\pgfpathlineto{\pgfqpoint{0.913708in}{1.115326in}}%
\pgfpathlineto{\pgfqpoint{0.925753in}{1.107877in}}%
\pgfpathlineto{\pgfqpoint{0.938610in}{1.100887in}}%
\pgfpathlineto{\pgfqpoint{0.931752in}{1.101164in}}%
\pgfpathlineto{\pgfqpoint{0.924594in}{1.100797in}}%
\pgfpathlineto{\pgfqpoint{0.917163in}{1.099786in}}%
\pgfpathlineto{\pgfqpoint{0.909490in}{1.098131in}}%
\pgfpathlineto{\pgfqpoint{0.901604in}{1.095840in}}%
\pgfpathlineto{\pgfqpoint{0.886153in}{1.104266in}}%
\pgfpathlineto{\pgfqpoint{0.871688in}{1.113243in}}%
\pgfpathlineto{\pgfqpoint{0.858268in}{1.122730in}}%
\pgfpathlineto{\pgfqpoint{0.845944in}{1.132686in}}%
\pgfpathlineto{\pgfqpoint{0.834764in}{1.143066in}}%
\pgfpathclose%
\pgfusepath{fill}%
\end{pgfscope}%
\begin{pgfscope}%
\pgfpathrectangle{\pgfqpoint{0.050000in}{0.050000in}}{\pgfqpoint{2.081932in}{2.081932in}}%
\pgfusepath{clip}%
\pgfsetbuttcap%
\pgfsetroundjoin%
\definecolor{currentfill}{rgb}{0.268510,0.009605,0.335427}%
\pgfsetfillcolor{currentfill}%
\pgfsetlinewidth{0.000000pt}%
\definecolor{currentstroke}{rgb}{0.000000,0.000000,0.000000}%
\pgfsetstrokecolor{currentstroke}%
\pgfsetdash{}{0pt}%
\pgfpathmoveto{\pgfqpoint{0.941111in}{0.772383in}}%
\pgfpathlineto{\pgfqpoint{0.935453in}{0.766118in}}%
\pgfpathlineto{\pgfqpoint{0.929819in}{0.760410in}}%
\pgfpathlineto{\pgfqpoint{0.924232in}{0.755282in}}%
\pgfpathlineto{\pgfqpoint{0.918713in}{0.750757in}}%
\pgfpathlineto{\pgfqpoint{0.913284in}{0.746854in}}%
\pgfpathlineto{\pgfqpoint{0.940311in}{0.739346in}}%
\pgfpathlineto{\pgfqpoint{0.968125in}{0.732903in}}%
\pgfpathlineto{\pgfqpoint{0.996606in}{0.727556in}}%
\pgfpathlineto{\pgfqpoint{1.025628in}{0.723331in}}%
\pgfpathlineto{\pgfqpoint{1.055064in}{0.720249in}}%
\pgfpathlineto{\pgfqpoint{1.056759in}{0.724867in}}%
\pgfpathlineto{\pgfqpoint{1.058483in}{0.730121in}}%
\pgfpathlineto{\pgfqpoint{1.060227in}{0.735986in}}%
\pgfpathlineto{\pgfqpoint{1.061985in}{0.742439in}}%
\pgfpathlineto{\pgfqpoint{1.063751in}{0.749452in}}%
\pgfpathlineto{\pgfqpoint{1.038300in}{0.752108in}}%
\pgfpathlineto{\pgfqpoint{1.013203in}{0.755749in}}%
\pgfpathlineto{\pgfqpoint{0.988568in}{0.760356in}}%
\pgfpathlineto{\pgfqpoint{0.964503in}{0.765910in}}%
\pgfpathlineto{\pgfqpoint{0.941111in}{0.772383in}}%
\pgfpathclose%
\pgfusepath{fill}%
\end{pgfscope}%
\begin{pgfscope}%
\pgfpathrectangle{\pgfqpoint{0.050000in}{0.050000in}}{\pgfqpoint{2.081932in}{2.081932in}}%
\pgfusepath{clip}%
\pgfsetbuttcap%
\pgfsetroundjoin%
\definecolor{currentfill}{rgb}{0.282327,0.094955,0.417331}%
\pgfsetfillcolor{currentfill}%
\pgfsetlinewidth{0.000000pt}%
\definecolor{currentstroke}{rgb}{0.000000,0.000000,0.000000}%
\pgfsetstrokecolor{currentstroke}%
\pgfsetdash{}{0pt}%
\pgfpathmoveto{\pgfqpoint{0.815306in}{0.786753in}}%
\pgfpathlineto{\pgfqpoint{0.807497in}{0.784530in}}%
\pgfpathlineto{\pgfqpoint{0.799887in}{0.782934in}}%
\pgfpathlineto{\pgfqpoint{0.792507in}{0.781970in}}%
\pgfpathlineto{\pgfqpoint{0.785386in}{0.781647in}}%
\pgfpathlineto{\pgfqpoint{0.778554in}{0.781965in}}%
\pgfpathlineto{\pgfqpoint{0.804020in}{0.769216in}}%
\pgfpathlineto{\pgfqpoint{0.830867in}{0.757467in}}%
\pgfpathlineto{\pgfqpoint{0.858985in}{0.746774in}}%
\pgfpathlineto{\pgfqpoint{0.888252in}{0.737189in}}%
\pgfpathlineto{\pgfqpoint{0.892907in}{0.737772in}}%
\pgfpathlineto{\pgfqpoint{0.897759in}{0.739038in}}%
\pgfpathlineto{\pgfqpoint{0.902786in}{0.740981in}}%
\pgfpathlineto{\pgfqpoint{0.907968in}{0.743590in}}%
\pgfpathlineto{\pgfqpoint{0.913284in}{0.746854in}}%
\pgfpathlineto{\pgfqpoint{0.887161in}{0.755391in}}%
\pgfpathlineto{\pgfqpoint{0.862054in}{0.764917in}}%
\pgfpathlineto{\pgfqpoint{0.838069in}{0.775388in}}%
\pgfpathlineto{\pgfqpoint{0.815306in}{0.786753in}}%
\pgfpathclose%
\pgfusepath{fill}%
\end{pgfscope}%
\begin{pgfscope}%
\pgfpathrectangle{\pgfqpoint{0.050000in}{0.050000in}}{\pgfqpoint{2.081932in}{2.081932in}}%
\pgfusepath{clip}%
\pgfsetbuttcap%
\pgfsetroundjoin%
\definecolor{currentfill}{rgb}{0.150476,0.504369,0.557430}%
\pgfsetfillcolor{currentfill}%
\pgfsetlinewidth{0.000000pt}%
\definecolor{currentstroke}{rgb}{0.000000,0.000000,0.000000}%
\pgfsetstrokecolor{currentstroke}%
\pgfsetdash{}{0pt}%
\pgfpathmoveto{\pgfqpoint{1.551728in}{0.846245in}}%
\pgfpathlineto{\pgfqpoint{1.554058in}{0.853207in}}%
\pgfpathlineto{\pgfqpoint{1.555800in}{0.860608in}}%
\pgfpathlineto{\pgfqpoint{1.556945in}{0.868419in}}%
\pgfpathlineto{\pgfqpoint{1.557488in}{0.876608in}}%
\pgfpathlineto{\pgfqpoint{1.581828in}{0.893903in}}%
\pgfpathlineto{\pgfqpoint{1.604155in}{0.912056in}}%
\pgfpathlineto{\pgfqpoint{1.624386in}{0.930981in}}%
\pgfpathlineto{\pgfqpoint{1.642448in}{0.950593in}}%
\pgfpathlineto{\pgfqpoint{1.658278in}{0.970803in}}%
\pgfpathlineto{\pgfqpoint{1.657618in}{0.962590in}}%
\pgfpathlineto{\pgfqpoint{1.656225in}{0.954623in}}%
\pgfpathlineto{\pgfqpoint{1.654107in}{0.946933in}}%
\pgfpathlineto{\pgfqpoint{1.651273in}{0.939551in}}%
\pgfpathlineto{\pgfqpoint{1.635632in}{0.919529in}}%
\pgfpathlineto{\pgfqpoint{1.617790in}{0.900101in}}%
\pgfpathlineto{\pgfqpoint{1.597809in}{0.881355in}}%
\pgfpathlineto{\pgfqpoint{1.575761in}{0.863374in}}%
\pgfpathlineto{\pgfqpoint{1.551728in}{0.846245in}}%
\pgfpathclose%
\pgfusepath{fill}%
\end{pgfscope}%
\begin{pgfscope}%
\pgfpathrectangle{\pgfqpoint{0.050000in}{0.050000in}}{\pgfqpoint{2.081932in}{2.081932in}}%
\pgfusepath{clip}%
\pgfsetbuttcap%
\pgfsetroundjoin%
\definecolor{currentfill}{rgb}{0.268510,0.009605,0.335427}%
\pgfsetfillcolor{currentfill}%
\pgfsetlinewidth{0.000000pt}%
\definecolor{currentstroke}{rgb}{0.000000,0.000000,0.000000}%
\pgfsetstrokecolor{currentstroke}%
\pgfsetdash{}{0pt}%
\pgfpathmoveto{\pgfqpoint{1.063751in}{0.749452in}}%
\pgfpathlineto{\pgfqpoint{1.061985in}{0.742439in}}%
\pgfpathlineto{\pgfqpoint{1.060227in}{0.735986in}}%
\pgfpathlineto{\pgfqpoint{1.058483in}{0.730121in}}%
\pgfpathlineto{\pgfqpoint{1.056759in}{0.724867in}}%
\pgfpathlineto{\pgfqpoint{1.055064in}{0.720249in}}%
\pgfpathlineto{\pgfqpoint{1.084784in}{0.718323in}}%
\pgfpathlineto{\pgfqpoint{1.114657in}{0.717563in}}%
\pgfpathlineto{\pgfqpoint{1.144550in}{0.717974in}}%
\pgfpathlineto{\pgfqpoint{1.174330in}{0.719552in}}%
\pgfpathlineto{\pgfqpoint{1.203865in}{0.722291in}}%
\pgfpathlineto{\pgfqpoint{1.201626in}{0.726855in}}%
\pgfpathlineto{\pgfqpoint{1.199349in}{0.732052in}}%
\pgfpathlineto{\pgfqpoint{1.197045in}{0.737861in}}%
\pgfpathlineto{\pgfqpoint{1.194721in}{0.744257in}}%
\pgfpathlineto{\pgfqpoint{1.192389in}{0.751212in}}%
\pgfpathlineto{\pgfqpoint{1.166853in}{0.748852in}}%
\pgfpathlineto{\pgfqpoint{1.141109in}{0.747492in}}%
\pgfpathlineto{\pgfqpoint{1.115268in}{0.747139in}}%
\pgfpathlineto{\pgfqpoint{1.089444in}{0.747793in}}%
\pgfpathlineto{\pgfqpoint{1.063751in}{0.749452in}}%
\pgfpathclose%
\pgfusepath{fill}%
\end{pgfscope}%
\begin{pgfscope}%
\pgfpathrectangle{\pgfqpoint{0.050000in}{0.050000in}}{\pgfqpoint{2.081932in}{2.081932in}}%
\pgfusepath{clip}%
\pgfsetbuttcap%
\pgfsetroundjoin%
\definecolor{currentfill}{rgb}{0.993248,0.906157,0.143936}%
\pgfsetfillcolor{currentfill}%
\pgfsetlinewidth{0.000000pt}%
\definecolor{currentstroke}{rgb}{0.000000,0.000000,0.000000}%
\pgfsetstrokecolor{currentstroke}%
\pgfsetdash{}{0pt}%
\pgfpathmoveto{\pgfqpoint{1.408657in}{1.092689in}}%
\pgfpathlineto{\pgfqpoint{1.399226in}{1.097417in}}%
\pgfpathlineto{\pgfqpoint{1.389846in}{1.101578in}}%
\pgfpathlineto{\pgfqpoint{1.380554in}{1.105155in}}%
\pgfpathlineto{\pgfqpoint{1.371390in}{1.108136in}}%
\pgfpathlineto{\pgfqpoint{1.362389in}{1.110509in}}%
\pgfpathlineto{\pgfqpoint{1.376130in}{1.119848in}}%
\pgfpathlineto{\pgfqpoint{1.388789in}{1.129668in}}%
\pgfpathlineto{\pgfqpoint{1.400319in}{1.139927in}}%
\pgfpathlineto{\pgfqpoint{1.410673in}{1.150579in}}%
\pgfpathlineto{\pgfqpoint{1.419814in}{1.161579in}}%
\pgfpathlineto{\pgfqpoint{1.430868in}{1.161100in}}%
\pgfpathlineto{\pgfqpoint{1.442118in}{1.160051in}}%
\pgfpathlineto{\pgfqpoint{1.453519in}{1.158436in}}%
\pgfpathlineto{\pgfqpoint{1.465022in}{1.156261in}}%
\pgfpathlineto{\pgfqpoint{1.476582in}{1.153535in}}%
\pgfpathlineto{\pgfqpoint{1.465803in}{1.140441in}}%
\pgfpathlineto{\pgfqpoint{1.453573in}{1.127755in}}%
\pgfpathlineto{\pgfqpoint{1.439937in}{1.115532in}}%
\pgfpathlineto{\pgfqpoint{1.424946in}{1.103826in}}%
\pgfpathlineto{\pgfqpoint{1.408657in}{1.092689in}}%
\pgfpathclose%
\pgfusepath{fill}%
\end{pgfscope}%
\begin{pgfscope}%
\pgfpathrectangle{\pgfqpoint{0.050000in}{0.050000in}}{\pgfqpoint{2.081932in}{2.081932in}}%
\pgfusepath{clip}%
\pgfsetbuttcap%
\pgfsetroundjoin%
\definecolor{currentfill}{rgb}{0.606045,0.850733,0.236712}%
\pgfsetfillcolor{currentfill}%
\pgfsetlinewidth{0.000000pt}%
\definecolor{currentstroke}{rgb}{0.000000,0.000000,0.000000}%
\pgfsetstrokecolor{currentstroke}%
\pgfsetdash{}{0pt}%
\pgfpathmoveto{\pgfqpoint{1.169341in}{1.064110in}}%
\pgfpathlineto{\pgfqpoint{1.167519in}{1.064522in}}%
\pgfpathlineto{\pgfqpoint{1.165795in}{1.064223in}}%
\pgfpathlineto{\pgfqpoint{1.164177in}{1.063216in}}%
\pgfpathlineto{\pgfqpoint{1.162670in}{1.061507in}}%
\pgfpathlineto{\pgfqpoint{1.161281in}{1.059105in}}%
\pgfpathlineto{\pgfqpoint{1.175794in}{1.060962in}}%
\pgfpathlineto{\pgfqpoint{1.190067in}{1.063359in}}%
\pgfpathlineto{\pgfqpoint{1.204041in}{1.066286in}}%
\pgfpathlineto{\pgfqpoint{1.217657in}{1.069730in}}%
\pgfpathlineto{\pgfqpoint{1.230859in}{1.073676in}}%
\pgfpathlineto{\pgfqpoint{1.234536in}{1.076553in}}%
\pgfpathlineto{\pgfqpoint{1.238523in}{1.078778in}}%
\pgfpathlineto{\pgfqpoint{1.242806in}{1.080342in}}%
\pgfpathlineto{\pgfqpoint{1.247367in}{1.081235in}}%
\pgfpathlineto{\pgfqpoint{1.252189in}{1.081452in}}%
\pgfpathlineto{\pgfqpoint{1.236477in}{1.076757in}}%
\pgfpathlineto{\pgfqpoint{1.220267in}{1.072658in}}%
\pgfpathlineto{\pgfqpoint{1.203627in}{1.069174in}}%
\pgfpathlineto{\pgfqpoint{1.186628in}{1.066320in}}%
\pgfpathlineto{\pgfqpoint{1.169341in}{1.064110in}}%
\pgfpathclose%
\pgfusepath{fill}%
\end{pgfscope}%
\begin{pgfscope}%
\pgfpathrectangle{\pgfqpoint{0.050000in}{0.050000in}}{\pgfqpoint{2.081932in}{2.081932in}}%
\pgfusepath{clip}%
\pgfsetbuttcap%
\pgfsetroundjoin%
\definecolor{currentfill}{rgb}{0.606045,0.850733,0.236712}%
\pgfsetfillcolor{currentfill}%
\pgfsetlinewidth{0.000000pt}%
\definecolor{currentstroke}{rgb}{0.000000,0.000000,0.000000}%
\pgfsetstrokecolor{currentstroke}%
\pgfsetdash{}{0pt}%
\pgfpathmoveto{\pgfqpoint{0.997061in}{1.078096in}}%
\pgfpathlineto{\pgfqpoint{1.001490in}{1.078001in}}%
\pgfpathlineto{\pgfqpoint{1.005678in}{1.077223in}}%
\pgfpathlineto{\pgfqpoint{1.009610in}{1.075767in}}%
\pgfpathlineto{\pgfqpoint{1.013271in}{1.073641in}}%
\pgfpathlineto{\pgfqpoint{1.016647in}{1.070856in}}%
\pgfpathlineto{\pgfqpoint{1.030144in}{1.067261in}}%
\pgfpathlineto{\pgfqpoint{1.044017in}{1.064178in}}%
\pgfpathlineto{\pgfqpoint{1.058206in}{1.061621in}}%
\pgfpathlineto{\pgfqpoint{1.072653in}{1.059602in}}%
\pgfpathlineto{\pgfqpoint{1.087296in}{1.058130in}}%
\pgfpathlineto{\pgfqpoint{1.086245in}{1.060500in}}%
\pgfpathlineto{\pgfqpoint{1.085105in}{1.062174in}}%
\pgfpathlineto{\pgfqpoint{1.083880in}{1.063144in}}%
\pgfpathlineto{\pgfqpoint{1.082576in}{1.063403in}}%
\pgfpathlineto{\pgfqpoint{1.081197in}{1.062948in}}%
\pgfpathlineto{\pgfqpoint{1.063752in}{1.064701in}}%
\pgfpathlineto{\pgfqpoint{1.046545in}{1.067105in}}%
\pgfpathlineto{\pgfqpoint{1.029646in}{1.070148in}}%
\pgfpathlineto{\pgfqpoint{1.013128in}{1.073818in}}%
\pgfpathlineto{\pgfqpoint{0.997061in}{1.078096in}}%
\pgfpathclose%
\pgfusepath{fill}%
\end{pgfscope}%
\begin{pgfscope}%
\pgfpathrectangle{\pgfqpoint{0.050000in}{0.050000in}}{\pgfqpoint{2.081932in}{2.081932in}}%
\pgfusepath{clip}%
\pgfsetbuttcap%
\pgfsetroundjoin%
\definecolor{currentfill}{rgb}{0.206756,0.371758,0.553117}%
\pgfsetfillcolor{currentfill}%
\pgfsetlinewidth{0.000000pt}%
\definecolor{currentstroke}{rgb}{0.000000,0.000000,0.000000}%
\pgfsetstrokecolor{currentstroke}%
\pgfsetdash{}{0pt}%
\pgfpathmoveto{\pgfqpoint{0.638099in}{0.875949in}}%
\pgfpathlineto{\pgfqpoint{0.632236in}{0.881007in}}%
\pgfpathlineto{\pgfqpoint{0.626967in}{0.886547in}}%
\pgfpathlineto{\pgfqpoint{0.622313in}{0.892548in}}%
\pgfpathlineto{\pgfqpoint{0.618294in}{0.898984in}}%
\pgfpathlineto{\pgfqpoint{0.614930in}{0.905832in}}%
\pgfpathlineto{\pgfqpoint{0.634273in}{0.886873in}}%
\pgfpathlineto{\pgfqpoint{0.655707in}{0.868655in}}%
\pgfpathlineto{\pgfqpoint{0.679151in}{0.851263in}}%
\pgfpathlineto{\pgfqpoint{0.704515in}{0.834777in}}%
\pgfpathlineto{\pgfqpoint{0.731700in}{0.819276in}}%
\pgfpathlineto{\pgfqpoint{0.734312in}{0.812942in}}%
\pgfpathlineto{\pgfqpoint{0.737431in}{0.807137in}}%
\pgfpathlineto{\pgfqpoint{0.741042in}{0.801885in}}%
\pgfpathlineto{\pgfqpoint{0.745131in}{0.797205in}}%
\pgfpathlineto{\pgfqpoint{0.749679in}{0.793115in}}%
\pgfpathlineto{\pgfqpoint{0.723722in}{0.807944in}}%
\pgfpathlineto{\pgfqpoint{0.699495in}{0.823717in}}%
\pgfpathlineto{\pgfqpoint{0.677092in}{0.840361in}}%
\pgfpathlineto{\pgfqpoint{0.656602in}{0.857798in}}%
\pgfpathlineto{\pgfqpoint{0.638099in}{0.875949in}}%
\pgfpathclose%
\pgfusepath{fill}%
\end{pgfscope}%
\begin{pgfscope}%
\pgfpathrectangle{\pgfqpoint{0.050000in}{0.050000in}}{\pgfqpoint{2.081932in}{2.081932in}}%
\pgfusepath{clip}%
\pgfsetbuttcap%
\pgfsetroundjoin%
\definecolor{currentfill}{rgb}{0.606045,0.850733,0.236712}%
\pgfsetfillcolor{currentfill}%
\pgfsetlinewidth{0.000000pt}%
\definecolor{currentstroke}{rgb}{0.000000,0.000000,0.000000}%
\pgfsetstrokecolor{currentstroke}%
\pgfsetdash{}{0pt}%
\pgfpathmoveto{\pgfqpoint{1.081197in}{1.062948in}}%
\pgfpathlineto{\pgfqpoint{1.082576in}{1.063403in}}%
\pgfpathlineto{\pgfqpoint{1.083880in}{1.063144in}}%
\pgfpathlineto{\pgfqpoint{1.085105in}{1.062174in}}%
\pgfpathlineto{\pgfqpoint{1.086245in}{1.060500in}}%
\pgfpathlineto{\pgfqpoint{1.087296in}{1.058130in}}%
\pgfpathlineto{\pgfqpoint{1.102074in}{1.057210in}}%
\pgfpathlineto{\pgfqpoint{1.116925in}{1.056848in}}%
\pgfpathlineto{\pgfqpoint{1.131784in}{1.057044in}}%
\pgfpathlineto{\pgfqpoint{1.146591in}{1.057797in}}%
\pgfpathlineto{\pgfqpoint{1.161281in}{1.059105in}}%
\pgfpathlineto{\pgfqpoint{1.162670in}{1.061507in}}%
\pgfpathlineto{\pgfqpoint{1.164177in}{1.063216in}}%
\pgfpathlineto{\pgfqpoint{1.165795in}{1.064223in}}%
\pgfpathlineto{\pgfqpoint{1.167519in}{1.064522in}}%
\pgfpathlineto{\pgfqpoint{1.169341in}{1.064110in}}%
\pgfpathlineto{\pgfqpoint{1.151840in}{1.062552in}}%
\pgfpathlineto{\pgfqpoint{1.134200in}{1.061655in}}%
\pgfpathlineto{\pgfqpoint{1.116496in}{1.061421in}}%
\pgfpathlineto{\pgfqpoint{1.098803in}{1.061853in}}%
\pgfpathlineto{\pgfqpoint{1.081197in}{1.062948in}}%
\pgfpathclose%
\pgfusepath{fill}%
\end{pgfscope}%
\begin{pgfscope}%
\pgfpathrectangle{\pgfqpoint{0.050000in}{0.050000in}}{\pgfqpoint{2.081932in}{2.081932in}}%
\pgfusepath{clip}%
\pgfsetbuttcap%
\pgfsetroundjoin%
\definecolor{currentfill}{rgb}{0.855810,0.888601,0.097452}%
\pgfsetfillcolor{currentfill}%
\pgfsetlinewidth{0.000000pt}%
\definecolor{currentstroke}{rgb}{0.000000,0.000000,0.000000}%
\pgfsetstrokecolor{currentstroke}%
\pgfsetdash{}{0pt}%
\pgfpathmoveto{\pgfqpoint{1.279531in}{1.072396in}}%
\pgfpathlineto{\pgfqpoint{1.273702in}{1.075546in}}%
\pgfpathlineto{\pgfqpoint{1.268032in}{1.078033in}}%
\pgfpathlineto{\pgfqpoint{1.262542in}{1.079850in}}%
\pgfpathlineto{\pgfqpoint{1.257254in}{1.080990in}}%
\pgfpathlineto{\pgfqpoint{1.252189in}{1.081452in}}%
\pgfpathlineto{\pgfqpoint{1.267337in}{1.086722in}}%
\pgfpathlineto{\pgfqpoint{1.281859in}{1.092545in}}%
\pgfpathlineto{\pgfqpoint{1.295693in}{1.098894in}}%
\pgfpathlineto{\pgfqpoint{1.308785in}{1.105741in}}%
\pgfpathlineto{\pgfqpoint{1.321079in}{1.113057in}}%
\pgfpathlineto{\pgfqpoint{1.328737in}{1.113795in}}%
\pgfpathlineto{\pgfqpoint{1.336728in}{1.113911in}}%
\pgfpathlineto{\pgfqpoint{1.345023in}{1.113401in}}%
\pgfpathlineto{\pgfqpoint{1.353588in}{1.112266in}}%
\pgfpathlineto{\pgfqpoint{1.362389in}{1.110509in}}%
\pgfpathlineto{\pgfqpoint{1.347620in}{1.101692in}}%
\pgfpathlineto{\pgfqpoint{1.331884in}{1.093437in}}%
\pgfpathlineto{\pgfqpoint{1.315245in}{1.085780in}}%
\pgfpathlineto{\pgfqpoint{1.297769in}{1.078756in}}%
\pgfpathlineto{\pgfqpoint{1.279531in}{1.072396in}}%
\pgfpathclose%
\pgfusepath{fill}%
\end{pgfscope}%
\begin{pgfscope}%
\pgfpathrectangle{\pgfqpoint{0.050000in}{0.050000in}}{\pgfqpoint{2.081932in}{2.081932in}}%
\pgfusepath{clip}%
\pgfsetbuttcap%
\pgfsetroundjoin%
\definecolor{currentfill}{rgb}{0.124780,0.640461,0.527068}%
\pgfsetfillcolor{currentfill}%
\pgfsetlinewidth{0.000000pt}%
\definecolor{currentstroke}{rgb}{0.000000,0.000000,0.000000}%
\pgfsetstrokecolor{currentstroke}%
\pgfsetdash{}{0pt}%
\pgfpathmoveto{\pgfqpoint{1.557488in}{0.876608in}}%
\pgfpathlineto{\pgfqpoint{1.557425in}{0.885141in}}%
\pgfpathlineto{\pgfqpoint{1.556756in}{0.893983in}}%
\pgfpathlineto{\pgfqpoint{1.555483in}{0.903099in}}%
\pgfpathlineto{\pgfqpoint{1.553611in}{0.912450in}}%
\pgfpathlineto{\pgfqpoint{1.551145in}{0.921998in}}%
\pgfpathlineto{\pgfqpoint{1.575146in}{0.938952in}}%
\pgfpathlineto{\pgfqpoint{1.597166in}{0.956747in}}%
\pgfpathlineto{\pgfqpoint{1.617122in}{0.975301in}}%
\pgfpathlineto{\pgfqpoint{1.634941in}{0.994529in}}%
\pgfpathlineto{\pgfqpoint{1.650563in}{1.014345in}}%
\pgfpathlineto{\pgfqpoint{1.653562in}{1.005424in}}%
\pgfpathlineto{\pgfqpoint{1.655840in}{0.996573in}}%
\pgfpathlineto{\pgfqpoint{1.657389in}{0.987829in}}%
\pgfpathlineto{\pgfqpoint{1.658202in}{0.979227in}}%
\pgfpathlineto{\pgfqpoint{1.658278in}{0.970803in}}%
\pgfpathlineto{\pgfqpoint{1.642448in}{0.950593in}}%
\pgfpathlineto{\pgfqpoint{1.624386in}{0.930981in}}%
\pgfpathlineto{\pgfqpoint{1.604155in}{0.912056in}}%
\pgfpathlineto{\pgfqpoint{1.581828in}{0.893903in}}%
\pgfpathlineto{\pgfqpoint{1.557488in}{0.876608in}}%
\pgfpathclose%
\pgfusepath{fill}%
\end{pgfscope}%
\begin{pgfscope}%
\pgfpathrectangle{\pgfqpoint{0.050000in}{0.050000in}}{\pgfqpoint{2.081932in}{2.081932in}}%
\pgfusepath{clip}%
\pgfsetbuttcap%
\pgfsetroundjoin%
\definecolor{currentfill}{rgb}{0.267968,0.223549,0.512008}%
\pgfsetfillcolor{currentfill}%
\pgfsetlinewidth{0.000000pt}%
\definecolor{currentstroke}{rgb}{0.000000,0.000000,0.000000}%
\pgfsetstrokecolor{currentstroke}%
\pgfsetdash{}{0pt}%
\pgfpathmoveto{\pgfqpoint{1.370637in}{0.743795in}}%
\pgfpathlineto{\pgfqpoint{1.375468in}{0.744028in}}%
\pgfpathlineto{\pgfqpoint{1.380045in}{0.744942in}}%
\pgfpathlineto{\pgfqpoint{1.384348in}{0.746537in}}%
\pgfpathlineto{\pgfqpoint{1.388359in}{0.748808in}}%
\pgfpathlineto{\pgfqpoint{1.392060in}{0.751748in}}%
\pgfpathlineto{\pgfqpoint{1.422931in}{0.762987in}}%
\pgfpathlineto{\pgfqpoint{1.452453in}{0.775385in}}%
\pgfpathlineto{\pgfqpoint{1.480501in}{0.788883in}}%
\pgfpathlineto{\pgfqpoint{1.506958in}{0.803416in}}%
\pgfpathlineto{\pgfqpoint{1.531717in}{0.818915in}}%
\pgfpathlineto{\pgfqpoint{1.526166in}{0.815094in}}%
\pgfpathlineto{\pgfqpoint{1.520148in}{0.811861in}}%
\pgfpathlineto{\pgfqpoint{1.513691in}{0.809226in}}%
\pgfpathlineto{\pgfqpoint{1.506821in}{0.807198in}}%
\pgfpathlineto{\pgfqpoint{1.499566in}{0.805786in}}%
\pgfpathlineto{\pgfqpoint{1.476685in}{0.791475in}}%
\pgfpathlineto{\pgfqpoint{1.452249in}{0.778059in}}%
\pgfpathlineto{\pgfqpoint{1.426356in}{0.765602in}}%
\pgfpathlineto{\pgfqpoint{1.399113in}{0.754163in}}%
\pgfpathlineto{\pgfqpoint{1.370637in}{0.743795in}}%
\pgfpathclose%
\pgfusepath{fill}%
\end{pgfscope}%
\begin{pgfscope}%
\pgfpathrectangle{\pgfqpoint{0.050000in}{0.050000in}}{\pgfqpoint{2.081932in}{2.081932in}}%
\pgfusepath{clip}%
\pgfsetbuttcap%
\pgfsetroundjoin%
\definecolor{currentfill}{rgb}{0.993248,0.906157,0.143936}%
\pgfsetfillcolor{currentfill}%
\pgfsetlinewidth{0.000000pt}%
\definecolor{currentstroke}{rgb}{0.000000,0.000000,0.000000}%
\pgfsetstrokecolor{currentstroke}%
\pgfsetdash{}{0pt}%
\pgfpathmoveto{\pgfqpoint{0.780919in}{1.131495in}}%
\pgfpathlineto{\pgfqpoint{0.791887in}{1.134943in}}%
\pgfpathlineto{\pgfqpoint{0.802801in}{1.137834in}}%
\pgfpathlineto{\pgfqpoint{0.813614in}{1.140157in}}%
\pgfpathlineto{\pgfqpoint{0.824283in}{1.141903in}}%
\pgfpathlineto{\pgfqpoint{0.834764in}{1.143066in}}%
\pgfpathlineto{\pgfqpoint{0.845944in}{1.132686in}}%
\pgfpathlineto{\pgfqpoint{0.858268in}{1.122730in}}%
\pgfpathlineto{\pgfqpoint{0.871688in}{1.113243in}}%
\pgfpathlineto{\pgfqpoint{0.886153in}{1.104266in}}%
\pgfpathlineto{\pgfqpoint{0.901604in}{1.095840in}}%
\pgfpathlineto{\pgfqpoint{0.893538in}{1.092919in}}%
\pgfpathlineto{\pgfqpoint{0.885324in}{1.089378in}}%
\pgfpathlineto{\pgfqpoint{0.876995in}{1.085232in}}%
\pgfpathlineto{\pgfqpoint{0.868585in}{1.080495in}}%
\pgfpathlineto{\pgfqpoint{0.860129in}{1.075185in}}%
\pgfpathlineto{\pgfqpoint{0.841787in}{1.085242in}}%
\pgfpathlineto{\pgfqpoint{0.824630in}{1.095950in}}%
\pgfpathlineto{\pgfqpoint{0.808726in}{1.107262in}}%
\pgfpathlineto{\pgfqpoint{0.794137in}{1.119128in}}%
\pgfpathlineto{\pgfqpoint{0.780919in}{1.131495in}}%
\pgfpathclose%
\pgfusepath{fill}%
\end{pgfscope}%
\begin{pgfscope}%
\pgfpathrectangle{\pgfqpoint{0.050000in}{0.050000in}}{\pgfqpoint{2.081932in}{2.081932in}}%
\pgfusepath{clip}%
\pgfsetbuttcap%
\pgfsetroundjoin%
\definecolor{currentfill}{rgb}{0.993248,0.906157,0.143936}%
\pgfsetfillcolor{currentfill}%
\pgfsetlinewidth{0.000000pt}%
\definecolor{currentstroke}{rgb}{0.000000,0.000000,0.000000}%
\pgfsetstrokecolor{currentstroke}%
\pgfsetdash{}{0pt}%
\pgfpathmoveto{\pgfqpoint{1.455201in}{1.061263in}}%
\pgfpathlineto{\pgfqpoint{1.446103in}{1.068505in}}%
\pgfpathlineto{\pgfqpoint{1.436861in}{1.075296in}}%
\pgfpathlineto{\pgfqpoint{1.427513in}{1.081606in}}%
\pgfpathlineto{\pgfqpoint{1.418099in}{1.087412in}}%
\pgfpathlineto{\pgfqpoint{1.408657in}{1.092689in}}%
\pgfpathlineto{\pgfqpoint{1.424946in}{1.103826in}}%
\pgfpathlineto{\pgfqpoint{1.439937in}{1.115532in}}%
\pgfpathlineto{\pgfqpoint{1.453573in}{1.127755in}}%
\pgfpathlineto{\pgfqpoint{1.465803in}{1.140441in}}%
\pgfpathlineto{\pgfqpoint{1.476582in}{1.153535in}}%
\pgfpathlineto{\pgfqpoint{1.488150in}{1.150267in}}%
\pgfpathlineto{\pgfqpoint{1.499679in}{1.146471in}}%
\pgfpathlineto{\pgfqpoint{1.511120in}{1.142163in}}%
\pgfpathlineto{\pgfqpoint{1.522427in}{1.137358in}}%
\pgfpathlineto{\pgfqpoint{1.533553in}{1.132077in}}%
\pgfpathlineto{\pgfqpoint{1.521159in}{1.116853in}}%
\pgfpathlineto{\pgfqpoint{1.507072in}{1.102095in}}%
\pgfpathlineto{\pgfqpoint{1.491344in}{1.087868in}}%
\pgfpathlineto{\pgfqpoint{1.474031in}{1.074237in}}%
\pgfpathlineto{\pgfqpoint{1.455201in}{1.061263in}}%
\pgfpathclose%
\pgfusepath{fill}%
\end{pgfscope}%
\begin{pgfscope}%
\pgfpathrectangle{\pgfqpoint{0.050000in}{0.050000in}}{\pgfqpoint{2.081932in}{2.081932in}}%
\pgfusepath{clip}%
\pgfsetbuttcap%
\pgfsetroundjoin%
\definecolor{currentfill}{rgb}{0.866013,0.889868,0.095953}%
\pgfsetfillcolor{currentfill}%
\pgfsetlinewidth{0.000000pt}%
\definecolor{currentstroke}{rgb}{0.000000,0.000000,0.000000}%
\pgfsetstrokecolor{currentstroke}%
\pgfsetdash{}{0pt}%
\pgfpathmoveto{\pgfqpoint{0.901604in}{1.095840in}}%
\pgfpathlineto{\pgfqpoint{0.909490in}{1.098131in}}%
\pgfpathlineto{\pgfqpoint{0.917163in}{1.099786in}}%
\pgfpathlineto{\pgfqpoint{0.924594in}{1.100797in}}%
\pgfpathlineto{\pgfqpoint{0.931752in}{1.101164in}}%
\pgfpathlineto{\pgfqpoint{0.938610in}{1.100887in}}%
\pgfpathlineto{\pgfqpoint{0.952226in}{1.094386in}}%
\pgfpathlineto{\pgfqpoint{0.966547in}{1.088403in}}%
\pgfpathlineto{\pgfqpoint{0.981513in}{1.082966in}}%
\pgfpathlineto{\pgfqpoint{0.997061in}{1.078096in}}%
\pgfpathlineto{\pgfqpoint{0.992410in}{1.077507in}}%
\pgfpathlineto{\pgfqpoint{0.987555in}{1.076233in}}%
\pgfpathlineto{\pgfqpoint{0.982513in}{1.074277in}}%
\pgfpathlineto{\pgfqpoint{0.977305in}{1.071645in}}%
\pgfpathlineto{\pgfqpoint{0.971952in}{1.068346in}}%
\pgfpathlineto{\pgfqpoint{0.953226in}{1.074223in}}%
\pgfpathlineto{\pgfqpoint{0.935210in}{1.080784in}}%
\pgfpathlineto{\pgfqpoint{0.917979in}{1.088001in}}%
\pgfpathlineto{\pgfqpoint{0.901604in}{1.095840in}}%
\pgfpathclose%
\pgfusepath{fill}%
\end{pgfscope}%
\begin{pgfscope}%
\pgfpathrectangle{\pgfqpoint{0.050000in}{0.050000in}}{\pgfqpoint{2.081932in}{2.081932in}}%
\pgfusepath{clip}%
\pgfsetbuttcap%
\pgfsetroundjoin%
\definecolor{currentfill}{rgb}{0.282327,0.094955,0.417331}%
\pgfsetfillcolor{currentfill}%
\pgfsetlinewidth{0.000000pt}%
\definecolor{currentstroke}{rgb}{0.000000,0.000000,0.000000}%
\pgfsetstrokecolor{currentstroke}%
\pgfsetdash{}{0pt}%
\pgfpathmoveto{\pgfqpoint{1.203865in}{0.722291in}}%
\pgfpathlineto{\pgfqpoint{1.206059in}{0.718382in}}%
\pgfpathlineto{\pgfqpoint{1.208198in}{0.715144in}}%
\pgfpathlineto{\pgfqpoint{1.210273in}{0.712593in}}%
\pgfpathlineto{\pgfqpoint{1.212275in}{0.710742in}}%
\pgfpathlineto{\pgfqpoint{1.214197in}{0.709599in}}%
\pgfpathlineto{\pgfqpoint{1.246908in}{0.713965in}}%
\pgfpathlineto{\pgfqpoint{1.279047in}{0.719599in}}%
\pgfpathlineto{\pgfqpoint{1.310473in}{0.726471in}}%
\pgfpathlineto{\pgfqpoint{1.341047in}{0.734550in}}%
\pgfpathlineto{\pgfqpoint{1.370637in}{0.743795in}}%
\pgfpathlineto{\pgfqpoint{1.365571in}{0.744244in}}%
\pgfpathlineto{\pgfqpoint{1.360291in}{0.745371in}}%
\pgfpathlineto{\pgfqpoint{1.354820in}{0.747169in}}%
\pgfpathlineto{\pgfqpoint{1.349180in}{0.749628in}}%
\pgfpathlineto{\pgfqpoint{1.343394in}{0.752738in}}%
\pgfpathlineto{\pgfqpoint{1.316987in}{0.744503in}}%
\pgfpathlineto{\pgfqpoint{1.289711in}{0.737310in}}%
\pgfpathlineto{\pgfqpoint{1.261683in}{0.731191in}}%
\pgfpathlineto{\pgfqpoint{1.233026in}{0.726177in}}%
\pgfpathlineto{\pgfqpoint{1.203865in}{0.722291in}}%
\pgfpathclose%
\pgfusepath{fill}%
\end{pgfscope}%
\begin{pgfscope}%
\pgfpathrectangle{\pgfqpoint{0.050000in}{0.050000in}}{\pgfqpoint{2.081932in}{2.081932in}}%
\pgfusepath{clip}%
\pgfsetbuttcap%
\pgfsetroundjoin%
\definecolor{currentfill}{rgb}{0.150476,0.504369,0.557430}%
\pgfsetfillcolor{currentfill}%
\pgfsetlinewidth{0.000000pt}%
\definecolor{currentstroke}{rgb}{0.000000,0.000000,0.000000}%
\pgfsetstrokecolor{currentstroke}%
\pgfsetdash{}{0pt}%
\pgfpathmoveto{\pgfqpoint{0.614930in}{0.905832in}}%
\pgfpathlineto{\pgfqpoint{0.612233in}{0.913064in}}%
\pgfpathlineto{\pgfqpoint{0.610218in}{0.920651in}}%
\pgfpathlineto{\pgfqpoint{0.608892in}{0.928562in}}%
\pgfpathlineto{\pgfqpoint{0.608264in}{0.936766in}}%
\pgfpathlineto{\pgfqpoint{0.627849in}{0.917627in}}%
\pgfpathlineto{\pgfqpoint{0.649553in}{0.899234in}}%
\pgfpathlineto{\pgfqpoint{0.673296in}{0.881675in}}%
\pgfpathlineto{\pgfqpoint{0.698987in}{0.865030in}}%
\pgfpathlineto{\pgfqpoint{0.726524in}{0.849378in}}%
\pgfpathlineto{\pgfqpoint{0.727012in}{0.841197in}}%
\pgfpathlineto{\pgfqpoint{0.728041in}{0.833432in}}%
\pgfpathlineto{\pgfqpoint{0.729606in}{0.826115in}}%
\pgfpathlineto{\pgfqpoint{0.731700in}{0.819276in}}%
\pgfpathlineto{\pgfqpoint{0.704515in}{0.834777in}}%
\pgfpathlineto{\pgfqpoint{0.679151in}{0.851263in}}%
\pgfpathlineto{\pgfqpoint{0.655707in}{0.868655in}}%
\pgfpathlineto{\pgfqpoint{0.634273in}{0.886873in}}%
\pgfpathlineto{\pgfqpoint{0.614930in}{0.905832in}}%
\pgfpathclose%
\pgfusepath{fill}%
\end{pgfscope}%
\begin{pgfscope}%
\pgfpathrectangle{\pgfqpoint{0.050000in}{0.050000in}}{\pgfqpoint{2.081932in}{2.081932in}}%
\pgfusepath{clip}%
\pgfsetbuttcap%
\pgfsetroundjoin%
\definecolor{currentfill}{rgb}{0.282327,0.094955,0.417331}%
\pgfsetfillcolor{currentfill}%
\pgfsetlinewidth{0.000000pt}%
\definecolor{currentstroke}{rgb}{0.000000,0.000000,0.000000}%
\pgfsetstrokecolor{currentstroke}%
\pgfsetdash{}{0pt}%
\pgfpathmoveto{\pgfqpoint{0.913284in}{0.746854in}}%
\pgfpathlineto{\pgfqpoint{0.907968in}{0.743590in}}%
\pgfpathlineto{\pgfqpoint{0.902786in}{0.740981in}}%
\pgfpathlineto{\pgfqpoint{0.897759in}{0.739038in}}%
\pgfpathlineto{\pgfqpoint{0.892907in}{0.737772in}}%
\pgfpathlineto{\pgfqpoint{0.888252in}{0.737189in}}%
\pgfpathlineto{\pgfqpoint{0.918543in}{0.728758in}}%
\pgfpathlineto{\pgfqpoint{0.949727in}{0.721521in}}%
\pgfpathlineto{\pgfqpoint{0.981666in}{0.715515in}}%
\pgfpathlineto{\pgfqpoint{1.014220in}{0.710768in}}%
\pgfpathlineto{\pgfqpoint{1.047243in}{0.707304in}}%
\pgfpathlineto{\pgfqpoint{1.048698in}{0.708493in}}%
\pgfpathlineto{\pgfqpoint{1.050214in}{0.710394in}}%
\pgfpathlineto{\pgfqpoint{1.051785in}{0.712995in}}%
\pgfpathlineto{\pgfqpoint{1.053404in}{0.716285in}}%
\pgfpathlineto{\pgfqpoint{1.055064in}{0.720249in}}%
\pgfpathlineto{\pgfqpoint{1.025628in}{0.723331in}}%
\pgfpathlineto{\pgfqpoint{0.996606in}{0.727556in}}%
\pgfpathlineto{\pgfqpoint{0.968125in}{0.732903in}}%
\pgfpathlineto{\pgfqpoint{0.940311in}{0.739346in}}%
\pgfpathlineto{\pgfqpoint{0.913284in}{0.746854in}}%
\pgfpathclose%
\pgfusepath{fill}%
\end{pgfscope}%
\begin{pgfscope}%
\pgfpathrectangle{\pgfqpoint{0.050000in}{0.050000in}}{\pgfqpoint{2.081932in}{2.081932in}}%
\pgfusepath{clip}%
\pgfsetbuttcap%
\pgfsetroundjoin%
\definecolor{currentfill}{rgb}{0.327796,0.773980,0.406640}%
\pgfsetfillcolor{currentfill}%
\pgfsetlinewidth{0.000000pt}%
\definecolor{currentstroke}{rgb}{0.000000,0.000000,0.000000}%
\pgfsetstrokecolor{currentstroke}%
\pgfsetdash{}{0pt}%
\pgfpathmoveto{\pgfqpoint{1.551145in}{0.921998in}}%
\pgfpathlineto{\pgfqpoint{1.548096in}{0.931704in}}%
\pgfpathlineto{\pgfqpoint{1.544477in}{0.941526in}}%
\pgfpathlineto{\pgfqpoint{1.540301in}{0.951424in}}%
\pgfpathlineto{\pgfqpoint{1.535587in}{0.961357in}}%
\pgfpathlineto{\pgfqpoint{1.530353in}{0.971283in}}%
\pgfpathlineto{\pgfqpoint{1.553241in}{0.987327in}}%
\pgfpathlineto{\pgfqpoint{1.574250in}{1.004171in}}%
\pgfpathlineto{\pgfqpoint{1.593299in}{1.021736in}}%
\pgfpathlineto{\pgfqpoint{1.610320in}{1.039945in}}%
\pgfpathlineto{\pgfqpoint{1.625254in}{1.058713in}}%
\pgfpathlineto{\pgfqpoint{1.631628in}{1.049994in}}%
\pgfpathlineto{\pgfqpoint{1.637367in}{1.041161in}}%
\pgfpathlineto{\pgfqpoint{1.642449in}{1.032250in}}%
\pgfpathlineto{\pgfqpoint{1.646854in}{1.023299in}}%
\pgfpathlineto{\pgfqpoint{1.650563in}{1.014345in}}%
\pgfpathlineto{\pgfqpoint{1.634941in}{0.994529in}}%
\pgfpathlineto{\pgfqpoint{1.617122in}{0.975301in}}%
\pgfpathlineto{\pgfqpoint{1.597166in}{0.956747in}}%
\pgfpathlineto{\pgfqpoint{1.575146in}{0.938952in}}%
\pgfpathlineto{\pgfqpoint{1.551145in}{0.921998in}}%
\pgfpathclose%
\pgfusepath{fill}%
\end{pgfscope}%
\begin{pgfscope}%
\pgfpathrectangle{\pgfqpoint{0.050000in}{0.050000in}}{\pgfqpoint{2.081932in}{2.081932in}}%
\pgfusepath{clip}%
\pgfsetbuttcap%
\pgfsetroundjoin%
\definecolor{currentfill}{rgb}{0.267968,0.223549,0.512008}%
\pgfsetfillcolor{currentfill}%
\pgfsetlinewidth{0.000000pt}%
\definecolor{currentstroke}{rgb}{0.000000,0.000000,0.000000}%
\pgfsetstrokecolor{currentstroke}%
\pgfsetdash{}{0pt}%
\pgfpathmoveto{\pgfqpoint{0.778554in}{0.781965in}}%
\pgfpathlineto{\pgfqpoint{0.772040in}{0.782926in}}%
\pgfpathlineto{\pgfqpoint{0.765870in}{0.784528in}}%
\pgfpathlineto{\pgfqpoint{0.760070in}{0.786765in}}%
\pgfpathlineto{\pgfqpoint{0.754665in}{0.789631in}}%
\pgfpathlineto{\pgfqpoint{0.749679in}{0.793115in}}%
\pgfpathlineto{\pgfqpoint{0.777259in}{0.779301in}}%
\pgfpathlineto{\pgfqpoint{0.806350in}{0.766568in}}%
\pgfpathlineto{\pgfqpoint{0.836828in}{0.754977in}}%
\pgfpathlineto{\pgfqpoint{0.868564in}{0.744585in}}%
\pgfpathlineto{\pgfqpoint{0.871966in}{0.741740in}}%
\pgfpathlineto{\pgfqpoint{0.875651in}{0.739573in}}%
\pgfpathlineto{\pgfqpoint{0.879606in}{0.738089in}}%
\pgfpathlineto{\pgfqpoint{0.883812in}{0.737294in}}%
\pgfpathlineto{\pgfqpoint{0.888252in}{0.737189in}}%
\pgfpathlineto{\pgfqpoint{0.858985in}{0.746774in}}%
\pgfpathlineto{\pgfqpoint{0.830867in}{0.757467in}}%
\pgfpathlineto{\pgfqpoint{0.804020in}{0.769216in}}%
\pgfpathlineto{\pgfqpoint{0.778554in}{0.781965in}}%
\pgfpathclose%
\pgfusepath{fill}%
\end{pgfscope}%
\begin{pgfscope}%
\pgfpathrectangle{\pgfqpoint{0.050000in}{0.050000in}}{\pgfqpoint{2.081932in}{2.081932in}}%
\pgfusepath{clip}%
\pgfsetbuttcap%
\pgfsetroundjoin%
\definecolor{currentfill}{rgb}{0.876168,0.891125,0.095250}%
\pgfsetfillcolor{currentfill}%
\pgfsetlinewidth{0.000000pt}%
\definecolor{currentstroke}{rgb}{0.000000,0.000000,0.000000}%
\pgfsetstrokecolor{currentstroke}%
\pgfsetdash{}{0pt}%
\pgfpathmoveto{\pgfqpoint{1.497224in}{1.019364in}}%
\pgfpathlineto{\pgfqpoint{1.489401in}{1.028389in}}%
\pgfpathlineto{\pgfqpoint{1.481253in}{1.037126in}}%
\pgfpathlineto{\pgfqpoint{1.472813in}{1.045541in}}%
\pgfpathlineto{\pgfqpoint{1.464117in}{1.053598in}}%
\pgfpathlineto{\pgfqpoint{1.455201in}{1.061263in}}%
\pgfpathlineto{\pgfqpoint{1.474031in}{1.074237in}}%
\pgfpathlineto{\pgfqpoint{1.491344in}{1.087868in}}%
\pgfpathlineto{\pgfqpoint{1.507072in}{1.102095in}}%
\pgfpathlineto{\pgfqpoint{1.521159in}{1.116853in}}%
\pgfpathlineto{\pgfqpoint{1.533553in}{1.132077in}}%
\pgfpathlineto{\pgfqpoint{1.544450in}{1.126342in}}%
\pgfpathlineto{\pgfqpoint{1.555075in}{1.120176in}}%
\pgfpathlineto{\pgfqpoint{1.565382in}{1.113604in}}%
\pgfpathlineto{\pgfqpoint{1.575328in}{1.106653in}}%
\pgfpathlineto{\pgfqpoint{1.584873in}{1.099353in}}%
\pgfpathlineto{\pgfqpoint{1.571049in}{1.082171in}}%
\pgfpathlineto{\pgfqpoint{1.555312in}{1.065507in}}%
\pgfpathlineto{\pgfqpoint{1.537717in}{1.049437in}}%
\pgfpathlineto{\pgfqpoint{1.518330in}{1.034032in}}%
\pgfpathlineto{\pgfqpoint{1.497224in}{1.019364in}}%
\pgfpathclose%
\pgfusepath{fill}%
\end{pgfscope}%
\begin{pgfscope}%
\pgfpathrectangle{\pgfqpoint{0.050000in}{0.050000in}}{\pgfqpoint{2.081932in}{2.081932in}}%
\pgfusepath{clip}%
\pgfsetbuttcap%
\pgfsetroundjoin%
\definecolor{currentfill}{rgb}{0.636902,0.856542,0.216620}%
\pgfsetfillcolor{currentfill}%
\pgfsetlinewidth{0.000000pt}%
\definecolor{currentstroke}{rgb}{0.000000,0.000000,0.000000}%
\pgfsetstrokecolor{currentstroke}%
\pgfsetdash{}{0pt}%
\pgfpathmoveto{\pgfqpoint{1.530353in}{0.971283in}}%
\pgfpathlineto{\pgfqpoint{1.524621in}{0.981160in}}%
\pgfpathlineto{\pgfqpoint{1.518415in}{0.990947in}}%
\pgfpathlineto{\pgfqpoint{1.511761in}{1.000604in}}%
\pgfpathlineto{\pgfqpoint{1.504688in}{1.010089in}}%
\pgfpathlineto{\pgfqpoint{1.497224in}{1.019364in}}%
\pgfpathlineto{\pgfqpoint{1.518330in}{1.034032in}}%
\pgfpathlineto{\pgfqpoint{1.537717in}{1.049437in}}%
\pgfpathlineto{\pgfqpoint{1.555312in}{1.065507in}}%
\pgfpathlineto{\pgfqpoint{1.571049in}{1.082171in}}%
\pgfpathlineto{\pgfqpoint{1.584873in}{1.099353in}}%
\pgfpathlineto{\pgfqpoint{1.593977in}{1.091734in}}%
\pgfpathlineto{\pgfqpoint{1.602601in}{1.083827in}}%
\pgfpathlineto{\pgfqpoint{1.610711in}{1.075665in}}%
\pgfpathlineto{\pgfqpoint{1.618272in}{1.067282in}}%
\pgfpathlineto{\pgfqpoint{1.625254in}{1.058713in}}%
\pgfpathlineto{\pgfqpoint{1.610320in}{1.039945in}}%
\pgfpathlineto{\pgfqpoint{1.593299in}{1.021736in}}%
\pgfpathlineto{\pgfqpoint{1.574250in}{1.004171in}}%
\pgfpathlineto{\pgfqpoint{1.553241in}{0.987327in}}%
\pgfpathlineto{\pgfqpoint{1.530353in}{0.971283in}}%
\pgfpathclose%
\pgfusepath{fill}%
\end{pgfscope}%
\begin{pgfscope}%
\pgfpathrectangle{\pgfqpoint{0.050000in}{0.050000in}}{\pgfqpoint{2.081932in}{2.081932in}}%
\pgfusepath{clip}%
\pgfsetbuttcap%
\pgfsetroundjoin%
\definecolor{currentfill}{rgb}{0.855810,0.888601,0.097452}%
\pgfsetfillcolor{currentfill}%
\pgfsetlinewidth{0.000000pt}%
\definecolor{currentstroke}{rgb}{0.000000,0.000000,0.000000}%
\pgfsetstrokecolor{currentstroke}%
\pgfsetdash{}{0pt}%
\pgfpathmoveto{\pgfqpoint{1.179681in}{1.051457in}}%
\pgfpathlineto{\pgfqpoint{1.177476in}{1.055378in}}%
\pgfpathlineto{\pgfqpoint{1.175331in}{1.058613in}}%
\pgfpathlineto{\pgfqpoint{1.173255in}{1.061151in}}%
\pgfpathlineto{\pgfqpoint{1.171256in}{1.062985in}}%
\pgfpathlineto{\pgfqpoint{1.169341in}{1.064110in}}%
\pgfpathlineto{\pgfqpoint{1.186628in}{1.066320in}}%
\pgfpathlineto{\pgfqpoint{1.203627in}{1.069174in}}%
\pgfpathlineto{\pgfqpoint{1.220267in}{1.072658in}}%
\pgfpathlineto{\pgfqpoint{1.236477in}{1.076757in}}%
\pgfpathlineto{\pgfqpoint{1.252189in}{1.081452in}}%
\pgfpathlineto{\pgfqpoint{1.257254in}{1.080990in}}%
\pgfpathlineto{\pgfqpoint{1.262542in}{1.079850in}}%
\pgfpathlineto{\pgfqpoint{1.268032in}{1.078033in}}%
\pgfpathlineto{\pgfqpoint{1.273702in}{1.075546in}}%
\pgfpathlineto{\pgfqpoint{1.279531in}{1.072396in}}%
\pgfpathlineto{\pgfqpoint{1.260606in}{1.066729in}}%
\pgfpathlineto{\pgfqpoint{1.241075in}{1.061781in}}%
\pgfpathlineto{\pgfqpoint{1.221019in}{1.057574in}}%
\pgfpathlineto{\pgfqpoint{1.200525in}{1.054127in}}%
\pgfpathlineto{\pgfqpoint{1.179681in}{1.051457in}}%
\pgfpathclose%
\pgfusepath{fill}%
\end{pgfscope}%
\begin{pgfscope}%
\pgfpathrectangle{\pgfqpoint{0.050000in}{0.050000in}}{\pgfqpoint{2.081932in}{2.081932in}}%
\pgfusepath{clip}%
\pgfsetbuttcap%
\pgfsetroundjoin%
\definecolor{currentfill}{rgb}{0.282327,0.094955,0.417331}%
\pgfsetfillcolor{currentfill}%
\pgfsetlinewidth{0.000000pt}%
\definecolor{currentstroke}{rgb}{0.000000,0.000000,0.000000}%
\pgfsetstrokecolor{currentstroke}%
\pgfsetdash{}{0pt}%
\pgfpathmoveto{\pgfqpoint{1.055064in}{0.720249in}}%
\pgfpathlineto{\pgfqpoint{1.053404in}{0.716285in}}%
\pgfpathlineto{\pgfqpoint{1.051785in}{0.712995in}}%
\pgfpathlineto{\pgfqpoint{1.050214in}{0.710394in}}%
\pgfpathlineto{\pgfqpoint{1.048698in}{0.708493in}}%
\pgfpathlineto{\pgfqpoint{1.047243in}{0.707304in}}%
\pgfpathlineto{\pgfqpoint{1.080589in}{0.705140in}}%
\pgfpathlineto{\pgfqpoint{1.114107in}{0.704286in}}%
\pgfpathlineto{\pgfqpoint{1.147648in}{0.704748in}}%
\pgfpathlineto{\pgfqpoint{1.181061in}{0.706522in}}%
\pgfpathlineto{\pgfqpoint{1.214197in}{0.709599in}}%
\pgfpathlineto{\pgfqpoint{1.212275in}{0.710742in}}%
\pgfpathlineto{\pgfqpoint{1.210273in}{0.712593in}}%
\pgfpathlineto{\pgfqpoint{1.208198in}{0.715144in}}%
\pgfpathlineto{\pgfqpoint{1.206059in}{0.718382in}}%
\pgfpathlineto{\pgfqpoint{1.203865in}{0.722291in}}%
\pgfpathlineto{\pgfqpoint{1.174330in}{0.719552in}}%
\pgfpathlineto{\pgfqpoint{1.144550in}{0.717974in}}%
\pgfpathlineto{\pgfqpoint{1.114657in}{0.717563in}}%
\pgfpathlineto{\pgfqpoint{1.084784in}{0.718323in}}%
\pgfpathlineto{\pgfqpoint{1.055064in}{0.720249in}}%
\pgfpathclose%
\pgfusepath{fill}%
\end{pgfscope}%
\begin{pgfscope}%
\pgfpathrectangle{\pgfqpoint{0.050000in}{0.050000in}}{\pgfqpoint{2.081932in}{2.081932in}}%
\pgfusepath{clip}%
\pgfsetbuttcap%
\pgfsetroundjoin%
\definecolor{currentfill}{rgb}{0.993248,0.906157,0.143936}%
\pgfsetfillcolor{currentfill}%
\pgfsetlinewidth{0.000000pt}%
\definecolor{currentstroke}{rgb}{0.000000,0.000000,0.000000}%
\pgfsetstrokecolor{currentstroke}%
\pgfsetdash{}{0pt}%
\pgfpathmoveto{\pgfqpoint{0.726835in}{1.106446in}}%
\pgfpathlineto{\pgfqpoint{0.737400in}{1.112433in}}%
\pgfpathlineto{\pgfqpoint{0.748136in}{1.117953in}}%
\pgfpathlineto{\pgfqpoint{0.758998in}{1.122983in}}%
\pgfpathlineto{\pgfqpoint{0.769941in}{1.127503in}}%
\pgfpathlineto{\pgfqpoint{0.780919in}{1.131495in}}%
\pgfpathlineto{\pgfqpoint{0.794137in}{1.119128in}}%
\pgfpathlineto{\pgfqpoint{0.808726in}{1.107262in}}%
\pgfpathlineto{\pgfqpoint{0.824630in}{1.095950in}}%
\pgfpathlineto{\pgfqpoint{0.841787in}{1.085242in}}%
\pgfpathlineto{\pgfqpoint{0.860129in}{1.075185in}}%
\pgfpathlineto{\pgfqpoint{0.851662in}{1.069324in}}%
\pgfpathlineto{\pgfqpoint{0.843218in}{1.062935in}}%
\pgfpathlineto{\pgfqpoint{0.834833in}{1.056042in}}%
\pgfpathlineto{\pgfqpoint{0.826542in}{1.048674in}}%
\pgfpathlineto{\pgfqpoint{0.818379in}{1.040860in}}%
\pgfpathlineto{\pgfqpoint{0.797144in}{1.052584in}}%
\pgfpathlineto{\pgfqpoint{0.777297in}{1.065063in}}%
\pgfpathlineto{\pgfqpoint{0.758916in}{1.078239in}}%
\pgfpathlineto{\pgfqpoint{0.742074in}{1.092055in}}%
\pgfpathlineto{\pgfqpoint{0.726835in}{1.106446in}}%
\pgfpathclose%
\pgfusepath{fill}%
\end{pgfscope}%
\begin{pgfscope}%
\pgfpathrectangle{\pgfqpoint{0.050000in}{0.050000in}}{\pgfqpoint{2.081932in}{2.081932in}}%
\pgfusepath{clip}%
\pgfsetbuttcap%
\pgfsetroundjoin%
\definecolor{currentfill}{rgb}{0.855810,0.888601,0.097452}%
\pgfsetfillcolor{currentfill}%
\pgfsetlinewidth{0.000000pt}%
\definecolor{currentstroke}{rgb}{0.000000,0.000000,0.000000}%
\pgfsetstrokecolor{currentstroke}%
\pgfsetdash{}{0pt}%
\pgfpathmoveto{\pgfqpoint{0.971952in}{1.068346in}}%
\pgfpathlineto{\pgfqpoint{0.977305in}{1.071645in}}%
\pgfpathlineto{\pgfqpoint{0.982513in}{1.074277in}}%
\pgfpathlineto{\pgfqpoint{0.987555in}{1.076233in}}%
\pgfpathlineto{\pgfqpoint{0.992410in}{1.077507in}}%
\pgfpathlineto{\pgfqpoint{0.997061in}{1.078096in}}%
\pgfpathlineto{\pgfqpoint{1.013128in}{1.073818in}}%
\pgfpathlineto{\pgfqpoint{1.029646in}{1.070148in}}%
\pgfpathlineto{\pgfqpoint{1.046545in}{1.067105in}}%
\pgfpathlineto{\pgfqpoint{1.063752in}{1.064701in}}%
\pgfpathlineto{\pgfqpoint{1.081197in}{1.062948in}}%
\pgfpathlineto{\pgfqpoint{1.079748in}{1.061779in}}%
\pgfpathlineto{\pgfqpoint{1.078235in}{1.059899in}}%
\pgfpathlineto{\pgfqpoint{1.076663in}{1.057312in}}%
\pgfpathlineto{\pgfqpoint{1.075040in}{1.054026in}}%
\pgfpathlineto{\pgfqpoint{1.073371in}{1.050054in}}%
\pgfpathlineto{\pgfqpoint{1.052335in}{1.052172in}}%
\pgfpathlineto{\pgfqpoint{1.031587in}{1.055075in}}%
\pgfpathlineto{\pgfqpoint{1.011216in}{1.058751in}}%
\pgfpathlineto{\pgfqpoint{0.991309in}{1.063181in}}%
\pgfpathlineto{\pgfqpoint{0.971952in}{1.068346in}}%
\pgfpathclose%
\pgfusepath{fill}%
\end{pgfscope}%
\begin{pgfscope}%
\pgfpathrectangle{\pgfqpoint{0.050000in}{0.050000in}}{\pgfqpoint{2.081932in}{2.081932in}}%
\pgfusepath{clip}%
\pgfsetbuttcap%
\pgfsetroundjoin%
\definecolor{currentfill}{rgb}{0.124780,0.640461,0.527068}%
\pgfsetfillcolor{currentfill}%
\pgfsetlinewidth{0.000000pt}%
\definecolor{currentstroke}{rgb}{0.000000,0.000000,0.000000}%
\pgfsetstrokecolor{currentstroke}%
\pgfsetdash{}{0pt}%
\pgfpathmoveto{\pgfqpoint{0.608264in}{0.936766in}}%
\pgfpathlineto{\pgfqpoint{0.608337in}{0.945230in}}%
\pgfpathlineto{\pgfqpoint{0.609111in}{0.953918in}}%
\pgfpathlineto{\pgfqpoint{0.610584in}{0.962796in}}%
\pgfpathlineto{\pgfqpoint{0.612752in}{0.971826in}}%
\pgfpathlineto{\pgfqpoint{0.615605in}{0.980972in}}%
\pgfpathlineto{\pgfqpoint{0.634924in}{0.962208in}}%
\pgfpathlineto{\pgfqpoint{0.656331in}{0.944178in}}%
\pgfpathlineto{\pgfqpoint{0.679744in}{0.926965in}}%
\pgfpathlineto{\pgfqpoint{0.705076in}{0.910649in}}%
\pgfpathlineto{\pgfqpoint{0.732224in}{0.895308in}}%
\pgfpathlineto{\pgfqpoint{0.730009in}{0.885576in}}%
\pgfpathlineto{\pgfqpoint{0.728326in}{0.876079in}}%
\pgfpathlineto{\pgfqpoint{0.727182in}{0.866855in}}%
\pgfpathlineto{\pgfqpoint{0.726581in}{0.857942in}}%
\pgfpathlineto{\pgfqpoint{0.726524in}{0.849378in}}%
\pgfpathlineto{\pgfqpoint{0.698987in}{0.865030in}}%
\pgfpathlineto{\pgfqpoint{0.673296in}{0.881675in}}%
\pgfpathlineto{\pgfqpoint{0.649553in}{0.899234in}}%
\pgfpathlineto{\pgfqpoint{0.627849in}{0.917627in}}%
\pgfpathlineto{\pgfqpoint{0.608264in}{0.936766in}}%
\pgfpathclose%
\pgfusepath{fill}%
\end{pgfscope}%
\begin{pgfscope}%
\pgfpathrectangle{\pgfqpoint{0.050000in}{0.050000in}}{\pgfqpoint{2.081932in}{2.081932in}}%
\pgfusepath{clip}%
\pgfsetbuttcap%
\pgfsetroundjoin%
\definecolor{currentfill}{rgb}{0.993248,0.906157,0.143936}%
\pgfsetfillcolor{currentfill}%
\pgfsetlinewidth{0.000000pt}%
\definecolor{currentstroke}{rgb}{0.000000,0.000000,0.000000}%
\pgfsetstrokecolor{currentstroke}%
\pgfsetdash{}{0pt}%
\pgfpathmoveto{\pgfqpoint{1.310207in}{1.047187in}}%
\pgfpathlineto{\pgfqpoint{1.303950in}{1.053431in}}%
\pgfpathlineto{\pgfqpoint{1.297729in}{1.059094in}}%
\pgfpathlineto{\pgfqpoint{1.291568in}{1.064155in}}%
\pgfpathlineto{\pgfqpoint{1.285494in}{1.068595in}}%
\pgfpathlineto{\pgfqpoint{1.279531in}{1.072396in}}%
\pgfpathlineto{\pgfqpoint{1.297769in}{1.078756in}}%
\pgfpathlineto{\pgfqpoint{1.315245in}{1.085780in}}%
\pgfpathlineto{\pgfqpoint{1.331884in}{1.093437in}}%
\pgfpathlineto{\pgfqpoint{1.347620in}{1.101692in}}%
\pgfpathlineto{\pgfqpoint{1.362389in}{1.110509in}}%
\pgfpathlineto{\pgfqpoint{1.371390in}{1.108136in}}%
\pgfpathlineto{\pgfqpoint{1.380554in}{1.105155in}}%
\pgfpathlineto{\pgfqpoint{1.389846in}{1.101578in}}%
\pgfpathlineto{\pgfqpoint{1.399226in}{1.097417in}}%
\pgfpathlineto{\pgfqpoint{1.408657in}{1.092689in}}%
\pgfpathlineto{\pgfqpoint{1.391135in}{1.082170in}}%
\pgfpathlineto{\pgfqpoint{1.372450in}{1.072318in}}%
\pgfpathlineto{\pgfqpoint{1.352679in}{1.063175in}}%
\pgfpathlineto{\pgfqpoint{1.331902in}{1.054786in}}%
\pgfpathlineto{\pgfqpoint{1.310207in}{1.047187in}}%
\pgfpathclose%
\pgfusepath{fill}%
\end{pgfscope}%
\begin{pgfscope}%
\pgfpathrectangle{\pgfqpoint{0.050000in}{0.050000in}}{\pgfqpoint{2.081932in}{2.081932in}}%
\pgfusepath{clip}%
\pgfsetbuttcap%
\pgfsetroundjoin%
\definecolor{currentfill}{rgb}{0.855810,0.888601,0.097452}%
\pgfsetfillcolor{currentfill}%
\pgfsetlinewidth{0.000000pt}%
\definecolor{currentstroke}{rgb}{0.000000,0.000000,0.000000}%
\pgfsetstrokecolor{currentstroke}%
\pgfsetdash{}{0pt}%
\pgfpathmoveto{\pgfqpoint{1.073371in}{1.050054in}}%
\pgfpathlineto{\pgfqpoint{1.075040in}{1.054026in}}%
\pgfpathlineto{\pgfqpoint{1.076663in}{1.057312in}}%
\pgfpathlineto{\pgfqpoint{1.078235in}{1.059899in}}%
\pgfpathlineto{\pgfqpoint{1.079748in}{1.061779in}}%
\pgfpathlineto{\pgfqpoint{1.081197in}{1.062948in}}%
\pgfpathlineto{\pgfqpoint{1.098803in}{1.061853in}}%
\pgfpathlineto{\pgfqpoint{1.116496in}{1.061421in}}%
\pgfpathlineto{\pgfqpoint{1.134200in}{1.061655in}}%
\pgfpathlineto{\pgfqpoint{1.151840in}{1.062552in}}%
\pgfpathlineto{\pgfqpoint{1.169341in}{1.064110in}}%
\pgfpathlineto{\pgfqpoint{1.171256in}{1.062985in}}%
\pgfpathlineto{\pgfqpoint{1.173255in}{1.061151in}}%
\pgfpathlineto{\pgfqpoint{1.175331in}{1.058613in}}%
\pgfpathlineto{\pgfqpoint{1.177476in}{1.055378in}}%
\pgfpathlineto{\pgfqpoint{1.179681in}{1.051457in}}%
\pgfpathlineto{\pgfqpoint{1.158575in}{1.049576in}}%
\pgfpathlineto{\pgfqpoint{1.137300in}{1.048492in}}%
\pgfpathlineto{\pgfqpoint{1.115946in}{1.048210in}}%
\pgfpathlineto{\pgfqpoint{1.094605in}{1.048731in}}%
\pgfpathlineto{\pgfqpoint{1.073371in}{1.050054in}}%
\pgfpathclose%
\pgfusepath{fill}%
\end{pgfscope}%
\begin{pgfscope}%
\pgfpathrectangle{\pgfqpoint{0.050000in}{0.050000in}}{\pgfqpoint{2.081932in}{2.081932in}}%
\pgfusepath{clip}%
\pgfsetbuttcap%
\pgfsetroundjoin%
\definecolor{currentfill}{rgb}{0.206756,0.371758,0.553117}%
\pgfsetfillcolor{currentfill}%
\pgfsetlinewidth{0.000000pt}%
\definecolor{currentstroke}{rgb}{0.000000,0.000000,0.000000}%
\pgfsetstrokecolor{currentstroke}%
\pgfsetdash{}{0pt}%
\pgfpathmoveto{\pgfqpoint{1.392060in}{0.751748in}}%
\pgfpathlineto{\pgfqpoint{1.395435in}{0.755346in}}%
\pgfpathlineto{\pgfqpoint{1.398470in}{0.759589in}}%
\pgfpathlineto{\pgfqpoint{1.401152in}{0.764462in}}%
\pgfpathlineto{\pgfqpoint{1.403467in}{0.769946in}}%
\pgfpathlineto{\pgfqpoint{1.405407in}{0.776020in}}%
\pgfpathlineto{\pgfqpoint{1.437767in}{0.787774in}}%
\pgfpathlineto{\pgfqpoint{1.468706in}{0.800739in}}%
\pgfpathlineto{\pgfqpoint{1.498093in}{0.814852in}}%
\pgfpathlineto{\pgfqpoint{1.525805in}{0.830045in}}%
\pgfpathlineto{\pgfqpoint{1.551728in}{0.846245in}}%
\pgfpathlineto{\pgfqpoint{1.548821in}{0.839749in}}%
\pgfpathlineto{\pgfqpoint{1.545350in}{0.833746in}}%
\pgfpathlineto{\pgfqpoint{1.541330in}{0.828259in}}%
\pgfpathlineto{\pgfqpoint{1.536779in}{0.823309in}}%
\pgfpathlineto{\pgfqpoint{1.531717in}{0.818915in}}%
\pgfpathlineto{\pgfqpoint{1.506958in}{0.803416in}}%
\pgfpathlineto{\pgfqpoint{1.480501in}{0.788883in}}%
\pgfpathlineto{\pgfqpoint{1.452453in}{0.775385in}}%
\pgfpathlineto{\pgfqpoint{1.422931in}{0.762987in}}%
\pgfpathlineto{\pgfqpoint{1.392060in}{0.751748in}}%
\pgfpathclose%
\pgfusepath{fill}%
\end{pgfscope}%
\begin{pgfscope}%
\pgfpathrectangle{\pgfqpoint{0.050000in}{0.050000in}}{\pgfqpoint{2.081932in}{2.081932in}}%
\pgfusepath{clip}%
\pgfsetbuttcap%
\pgfsetroundjoin%
\definecolor{currentfill}{rgb}{0.876168,0.891125,0.095250}%
\pgfsetfillcolor{currentfill}%
\pgfsetlinewidth{0.000000pt}%
\definecolor{currentstroke}{rgb}{0.000000,0.000000,0.000000}%
\pgfsetstrokecolor{currentstroke}%
\pgfsetdash{}{0pt}%
\pgfpathmoveto{\pgfqpoint{0.678074in}{1.070421in}}%
\pgfpathlineto{\pgfqpoint{0.687146in}{1.078342in}}%
\pgfpathlineto{\pgfqpoint{0.696598in}{1.085935in}}%
\pgfpathlineto{\pgfqpoint{0.706391in}{1.093170in}}%
\pgfpathlineto{\pgfqpoint{0.716484in}{1.100016in}}%
\pgfpathlineto{\pgfqpoint{0.726835in}{1.106446in}}%
\pgfpathlineto{\pgfqpoint{0.742074in}{1.092055in}}%
\pgfpathlineto{\pgfqpoint{0.758916in}{1.078239in}}%
\pgfpathlineto{\pgfqpoint{0.777297in}{1.065063in}}%
\pgfpathlineto{\pgfqpoint{0.797144in}{1.052584in}}%
\pgfpathlineto{\pgfqpoint{0.818379in}{1.040860in}}%
\pgfpathlineto{\pgfqpoint{0.810378in}{1.032633in}}%
\pgfpathlineto{\pgfqpoint{0.802573in}{1.024026in}}%
\pgfpathlineto{\pgfqpoint{0.794997in}{1.015074in}}%
\pgfpathlineto{\pgfqpoint{0.787682in}{1.005814in}}%
\pgfpathlineto{\pgfqpoint{0.780659in}{0.996286in}}%
\pgfpathlineto{\pgfqpoint{0.756825in}{1.009548in}}%
\pgfpathlineto{\pgfqpoint{0.734566in}{1.023660in}}%
\pgfpathlineto{\pgfqpoint{0.713970in}{1.038556in}}%
\pgfpathlineto{\pgfqpoint{0.695115in}{1.054166in}}%
\pgfpathlineto{\pgfqpoint{0.678074in}{1.070421in}}%
\pgfpathclose%
\pgfusepath{fill}%
\end{pgfscope}%
\begin{pgfscope}%
\pgfpathrectangle{\pgfqpoint{0.050000in}{0.050000in}}{\pgfqpoint{2.081932in}{2.081932in}}%
\pgfusepath{clip}%
\pgfsetbuttcap%
\pgfsetroundjoin%
\definecolor{currentfill}{rgb}{0.993248,0.906157,0.143936}%
\pgfsetfillcolor{currentfill}%
\pgfsetlinewidth{0.000000pt}%
\definecolor{currentstroke}{rgb}{0.000000,0.000000,0.000000}%
\pgfsetstrokecolor{currentstroke}%
\pgfsetdash{}{0pt}%
\pgfpathmoveto{\pgfqpoint{0.860129in}{1.075185in}}%
\pgfpathlineto{\pgfqpoint{0.868585in}{1.080495in}}%
\pgfpathlineto{\pgfqpoint{0.876995in}{1.085232in}}%
\pgfpathlineto{\pgfqpoint{0.885324in}{1.089378in}}%
\pgfpathlineto{\pgfqpoint{0.893538in}{1.092919in}}%
\pgfpathlineto{\pgfqpoint{0.901604in}{1.095840in}}%
\pgfpathlineto{\pgfqpoint{0.917979in}{1.088001in}}%
\pgfpathlineto{\pgfqpoint{0.935210in}{1.080784in}}%
\pgfpathlineto{\pgfqpoint{0.953226in}{1.074223in}}%
\pgfpathlineto{\pgfqpoint{0.971952in}{1.068346in}}%
\pgfpathlineto{\pgfqpoint{0.966475in}{1.064392in}}%
\pgfpathlineto{\pgfqpoint{0.960896in}{1.059797in}}%
\pgfpathlineto{\pgfqpoint{0.955238in}{1.054577in}}%
\pgfpathlineto{\pgfqpoint{0.949523in}{1.048753in}}%
\pgfpathlineto{\pgfqpoint{0.943776in}{1.042347in}}%
\pgfpathlineto{\pgfqpoint{0.921493in}{1.049370in}}%
\pgfpathlineto{\pgfqpoint{0.900065in}{1.057208in}}%
\pgfpathlineto{\pgfqpoint{0.879582in}{1.065827in}}%
\pgfpathlineto{\pgfqpoint{0.860129in}{1.075185in}}%
\pgfpathclose%
\pgfusepath{fill}%
\end{pgfscope}%
\begin{pgfscope}%
\pgfpathrectangle{\pgfqpoint{0.050000in}{0.050000in}}{\pgfqpoint{2.081932in}{2.081932in}}%
\pgfusepath{clip}%
\pgfsetbuttcap%
\pgfsetroundjoin%
\definecolor{currentfill}{rgb}{0.327796,0.773980,0.406640}%
\pgfsetfillcolor{currentfill}%
\pgfsetlinewidth{0.000000pt}%
\definecolor{currentstroke}{rgb}{0.000000,0.000000,0.000000}%
\pgfsetstrokecolor{currentstroke}%
\pgfsetdash{}{0pt}%
\pgfpathmoveto{\pgfqpoint{0.615605in}{0.980972in}}%
\pgfpathlineto{\pgfqpoint{0.619134in}{0.990197in}}%
\pgfpathlineto{\pgfqpoint{0.623325in}{0.999461in}}%
\pgfpathlineto{\pgfqpoint{0.628159in}{1.008726in}}%
\pgfpathlineto{\pgfqpoint{0.633619in}{1.017954in}}%
\pgfpathlineto{\pgfqpoint{0.639681in}{1.027107in}}%
\pgfpathlineto{\pgfqpoint{0.658126in}{1.009341in}}%
\pgfpathlineto{\pgfqpoint{0.678552in}{0.992273in}}%
\pgfpathlineto{\pgfqpoint{0.700883in}{0.975983in}}%
\pgfpathlineto{\pgfqpoint{0.725033in}{0.960544in}}%
\pgfpathlineto{\pgfqpoint{0.750906in}{0.946031in}}%
\pgfpathlineto{\pgfqpoint{0.746204in}{0.935752in}}%
\pgfpathlineto{\pgfqpoint{0.741968in}{0.925497in}}%
\pgfpathlineto{\pgfqpoint{0.738216in}{0.915310in}}%
\pgfpathlineto{\pgfqpoint{0.734964in}{0.905233in}}%
\pgfpathlineto{\pgfqpoint{0.732224in}{0.895308in}}%
\pgfpathlineto{\pgfqpoint{0.705076in}{0.910649in}}%
\pgfpathlineto{\pgfqpoint{0.679744in}{0.926965in}}%
\pgfpathlineto{\pgfqpoint{0.656331in}{0.944178in}}%
\pgfpathlineto{\pgfqpoint{0.634924in}{0.962208in}}%
\pgfpathlineto{\pgfqpoint{0.615605in}{0.980972in}}%
\pgfpathclose%
\pgfusepath{fill}%
\end{pgfscope}%
\begin{pgfscope}%
\pgfpathrectangle{\pgfqpoint{0.050000in}{0.050000in}}{\pgfqpoint{2.081932in}{2.081932in}}%
\pgfusepath{clip}%
\pgfsetbuttcap%
\pgfsetroundjoin%
\definecolor{currentfill}{rgb}{0.636902,0.856542,0.216620}%
\pgfsetfillcolor{currentfill}%
\pgfsetlinewidth{0.000000pt}%
\definecolor{currentstroke}{rgb}{0.000000,0.000000,0.000000}%
\pgfsetstrokecolor{currentstroke}%
\pgfsetdash{}{0pt}%
\pgfpathmoveto{\pgfqpoint{0.639681in}{1.027107in}}%
\pgfpathlineto{\pgfqpoint{0.646321in}{1.036146in}}%
\pgfpathlineto{\pgfqpoint{0.653511in}{1.045033in}}%
\pgfpathlineto{\pgfqpoint{0.661222in}{1.053732in}}%
\pgfpathlineto{\pgfqpoint{0.669421in}{1.062207in}}%
\pgfpathlineto{\pgfqpoint{0.678074in}{1.070421in}}%
\pgfpathlineto{\pgfqpoint{0.695115in}{1.054166in}}%
\pgfpathlineto{\pgfqpoint{0.713970in}{1.038556in}}%
\pgfpathlineto{\pgfqpoint{0.734566in}{1.023660in}}%
\pgfpathlineto{\pgfqpoint{0.756825in}{1.009548in}}%
\pgfpathlineto{\pgfqpoint{0.780659in}{0.996286in}}%
\pgfpathlineto{\pgfqpoint{0.773957in}{0.986528in}}%
\pgfpathlineto{\pgfqpoint{0.767605in}{0.976582in}}%
\pgfpathlineto{\pgfqpoint{0.761629in}{0.966488in}}%
\pgfpathlineto{\pgfqpoint{0.756055in}{0.956290in}}%
\pgfpathlineto{\pgfqpoint{0.750906in}{0.946031in}}%
\pgfpathlineto{\pgfqpoint{0.725033in}{0.960544in}}%
\pgfpathlineto{\pgfqpoint{0.700883in}{0.975983in}}%
\pgfpathlineto{\pgfqpoint{0.678552in}{0.992273in}}%
\pgfpathlineto{\pgfqpoint{0.658126in}{1.009341in}}%
\pgfpathlineto{\pgfqpoint{0.639681in}{1.027107in}}%
\pgfpathclose%
\pgfusepath{fill}%
\end{pgfscope}%
\begin{pgfscope}%
\pgfpathrectangle{\pgfqpoint{0.050000in}{0.050000in}}{\pgfqpoint{2.081932in}{2.081932in}}%
\pgfusepath{clip}%
\pgfsetbuttcap%
\pgfsetroundjoin%
\definecolor{currentfill}{rgb}{0.267968,0.223549,0.512008}%
\pgfsetfillcolor{currentfill}%
\pgfsetlinewidth{0.000000pt}%
\definecolor{currentstroke}{rgb}{0.000000,0.000000,0.000000}%
\pgfsetstrokecolor{currentstroke}%
\pgfsetdash{}{0pt}%
\pgfpathmoveto{\pgfqpoint{1.214197in}{0.709599in}}%
\pgfpathlineto{\pgfqpoint{1.216031in}{0.709173in}}%
\pgfpathlineto{\pgfqpoint{1.217768in}{0.709466in}}%
\pgfpathlineto{\pgfqpoint{1.219401in}{0.710480in}}%
\pgfpathlineto{\pgfqpoint{1.220923in}{0.712213in}}%
\pgfpathlineto{\pgfqpoint{1.222328in}{0.714661in}}%
\pgfpathlineto{\pgfqpoint{1.257831in}{0.719397in}}%
\pgfpathlineto{\pgfqpoint{1.292709in}{0.725508in}}%
\pgfpathlineto{\pgfqpoint{1.326807in}{0.732962in}}%
\pgfpathlineto{\pgfqpoint{1.359972in}{0.741724in}}%
\pgfpathlineto{\pgfqpoint{1.392060in}{0.751748in}}%
\pgfpathlineto{\pgfqpoint{1.388359in}{0.748808in}}%
\pgfpathlineto{\pgfqpoint{1.384348in}{0.746537in}}%
\pgfpathlineto{\pgfqpoint{1.380045in}{0.744942in}}%
\pgfpathlineto{\pgfqpoint{1.375468in}{0.744028in}}%
\pgfpathlineto{\pgfqpoint{1.370637in}{0.743795in}}%
\pgfpathlineto{\pgfqpoint{1.341047in}{0.734550in}}%
\pgfpathlineto{\pgfqpoint{1.310473in}{0.726471in}}%
\pgfpathlineto{\pgfqpoint{1.279047in}{0.719599in}}%
\pgfpathlineto{\pgfqpoint{1.246908in}{0.713965in}}%
\pgfpathlineto{\pgfqpoint{1.214197in}{0.709599in}}%
\pgfpathclose%
\pgfusepath{fill}%
\end{pgfscope}%
\begin{pgfscope}%
\pgfpathrectangle{\pgfqpoint{0.050000in}{0.050000in}}{\pgfqpoint{2.081932in}{2.081932in}}%
\pgfusepath{clip}%
\pgfsetbuttcap%
\pgfsetroundjoin%
\definecolor{currentfill}{rgb}{0.206756,0.371758,0.553117}%
\pgfsetfillcolor{currentfill}%
\pgfsetlinewidth{0.000000pt}%
\definecolor{currentstroke}{rgb}{0.000000,0.000000,0.000000}%
\pgfsetstrokecolor{currentstroke}%
\pgfsetdash{}{0pt}%
\pgfpathmoveto{\pgfqpoint{0.749679in}{0.793115in}}%
\pgfpathlineto{\pgfqpoint{0.745131in}{0.797205in}}%
\pgfpathlineto{\pgfqpoint{0.741042in}{0.801885in}}%
\pgfpathlineto{\pgfqpoint{0.737431in}{0.807137in}}%
\pgfpathlineto{\pgfqpoint{0.734312in}{0.812942in}}%
\pgfpathlineto{\pgfqpoint{0.731700in}{0.819276in}}%
\pgfpathlineto{\pgfqpoint{0.760594in}{0.804834in}}%
\pgfpathlineto{\pgfqpoint{0.791079in}{0.791519in}}%
\pgfpathlineto{\pgfqpoint{0.823026in}{0.779397in}}%
\pgfpathlineto{\pgfqpoint{0.856297in}{0.768527in}}%
\pgfpathlineto{\pgfqpoint{0.858080in}{0.762499in}}%
\pgfpathlineto{\pgfqpoint{0.860208in}{0.757072in}}%
\pgfpathlineto{\pgfqpoint{0.862673in}{0.752265in}}%
\pgfpathlineto{\pgfqpoint{0.865462in}{0.748098in}}%
\pgfpathlineto{\pgfqpoint{0.868564in}{0.744585in}}%
\pgfpathlineto{\pgfqpoint{0.836828in}{0.754977in}}%
\pgfpathlineto{\pgfqpoint{0.806350in}{0.766568in}}%
\pgfpathlineto{\pgfqpoint{0.777259in}{0.779301in}}%
\pgfpathlineto{\pgfqpoint{0.749679in}{0.793115in}}%
\pgfpathclose%
\pgfusepath{fill}%
\end{pgfscope}%
\begin{pgfscope}%
\pgfpathrectangle{\pgfqpoint{0.050000in}{0.050000in}}{\pgfqpoint{2.081932in}{2.081932in}}%
\pgfusepath{clip}%
\pgfsetbuttcap%
\pgfsetroundjoin%
\definecolor{currentfill}{rgb}{0.267968,0.223549,0.512008}%
\pgfsetfillcolor{currentfill}%
\pgfsetlinewidth{0.000000pt}%
\definecolor{currentstroke}{rgb}{0.000000,0.000000,0.000000}%
\pgfsetstrokecolor{currentstroke}%
\pgfsetdash{}{0pt}%
\pgfpathmoveto{\pgfqpoint{0.888252in}{0.737189in}}%
\pgfpathlineto{\pgfqpoint{0.883812in}{0.737294in}}%
\pgfpathlineto{\pgfqpoint{0.879606in}{0.738089in}}%
\pgfpathlineto{\pgfqpoint{0.875651in}{0.739573in}}%
\pgfpathlineto{\pgfqpoint{0.871966in}{0.741740in}}%
\pgfpathlineto{\pgfqpoint{0.868564in}{0.744585in}}%
\pgfpathlineto{\pgfqpoint{0.901420in}{0.735442in}}%
\pgfpathlineto{\pgfqpoint{0.935252in}{0.727594in}}%
\pgfpathlineto{\pgfqpoint{0.969911in}{0.721078in}}%
\pgfpathlineto{\pgfqpoint{1.005242in}{0.715929in}}%
\pgfpathlineto{\pgfqpoint{1.041088in}{0.712171in}}%
\pgfpathlineto{\pgfqpoint{1.042152in}{0.709757in}}%
\pgfpathlineto{\pgfqpoint{1.043304in}{0.708060in}}%
\pgfpathlineto{\pgfqpoint{1.044541in}{0.707085in}}%
\pgfpathlineto{\pgfqpoint{1.045855in}{0.706833in}}%
\pgfpathlineto{\pgfqpoint{1.047243in}{0.707304in}}%
\pgfpathlineto{\pgfqpoint{1.014220in}{0.710768in}}%
\pgfpathlineto{\pgfqpoint{0.981666in}{0.715515in}}%
\pgfpathlineto{\pgfqpoint{0.949727in}{0.721521in}}%
\pgfpathlineto{\pgfqpoint{0.918543in}{0.728758in}}%
\pgfpathlineto{\pgfqpoint{0.888252in}{0.737189in}}%
\pgfpathclose%
\pgfusepath{fill}%
\end{pgfscope}%
\begin{pgfscope}%
\pgfpathrectangle{\pgfqpoint{0.050000in}{0.050000in}}{\pgfqpoint{2.081932in}{2.081932in}}%
\pgfusepath{clip}%
\pgfsetbuttcap%
\pgfsetroundjoin%
\definecolor{currentfill}{rgb}{0.993248,0.906157,0.143936}%
\pgfsetfillcolor{currentfill}%
\pgfsetlinewidth{0.000000pt}%
\definecolor{currentstroke}{rgb}{0.000000,0.000000,0.000000}%
\pgfsetstrokecolor{currentstroke}%
\pgfsetdash{}{0pt}%
\pgfpathmoveto{\pgfqpoint{1.341120in}{1.008196in}}%
\pgfpathlineto{\pgfqpoint{1.335073in}{1.016929in}}%
\pgfpathlineto{\pgfqpoint{1.328932in}{1.025228in}}%
\pgfpathlineto{\pgfqpoint{1.322724in}{1.033057in}}%
\pgfpathlineto{\pgfqpoint{1.316473in}{1.040387in}}%
\pgfpathlineto{\pgfqpoint{1.310207in}{1.047187in}}%
\pgfpathlineto{\pgfqpoint{1.331902in}{1.054786in}}%
\pgfpathlineto{\pgfqpoint{1.352679in}{1.063175in}}%
\pgfpathlineto{\pgfqpoint{1.372450in}{1.072318in}}%
\pgfpathlineto{\pgfqpoint{1.391135in}{1.082170in}}%
\pgfpathlineto{\pgfqpoint{1.408657in}{1.092689in}}%
\pgfpathlineto{\pgfqpoint{1.418099in}{1.087412in}}%
\pgfpathlineto{\pgfqpoint{1.427513in}{1.081606in}}%
\pgfpathlineto{\pgfqpoint{1.436861in}{1.075296in}}%
\pgfpathlineto{\pgfqpoint{1.446103in}{1.068505in}}%
\pgfpathlineto{\pgfqpoint{1.455201in}{1.061263in}}%
\pgfpathlineto{\pgfqpoint{1.434927in}{1.049004in}}%
\pgfpathlineto{\pgfqpoint{1.413290in}{1.037516in}}%
\pgfpathlineto{\pgfqpoint{1.390380in}{1.026853in}}%
\pgfpathlineto{\pgfqpoint{1.366289in}{1.017064in}}%
\pgfpathlineto{\pgfqpoint{1.341120in}{1.008196in}}%
\pgfpathclose%
\pgfusepath{fill}%
\end{pgfscope}%
\begin{pgfscope}%
\pgfpathrectangle{\pgfqpoint{0.050000in}{0.050000in}}{\pgfqpoint{2.081932in}{2.081932in}}%
\pgfusepath{clip}%
\pgfsetbuttcap%
\pgfsetroundjoin%
\definecolor{currentfill}{rgb}{0.150476,0.504369,0.557430}%
\pgfsetfillcolor{currentfill}%
\pgfsetlinewidth{0.000000pt}%
\definecolor{currentstroke}{rgb}{0.000000,0.000000,0.000000}%
\pgfsetstrokecolor{currentstroke}%
\pgfsetdash{}{0pt}%
\pgfpathmoveto{\pgfqpoint{1.405407in}{0.776020in}}%
\pgfpathlineto{\pgfqpoint{1.406961in}{0.782659in}}%
\pgfpathlineto{\pgfqpoint{1.408123in}{0.789838in}}%
\pgfpathlineto{\pgfqpoint{1.408888in}{0.797527in}}%
\pgfpathlineto{\pgfqpoint{1.409250in}{0.805696in}}%
\pgfpathlineto{\pgfqpoint{1.442039in}{0.817566in}}%
\pgfpathlineto{\pgfqpoint{1.473386in}{0.830659in}}%
\pgfpathlineto{\pgfqpoint{1.503157in}{0.844910in}}%
\pgfpathlineto{\pgfqpoint{1.531229in}{0.860251in}}%
\pgfpathlineto{\pgfqpoint{1.557488in}{0.876608in}}%
\pgfpathlineto{\pgfqpoint{1.556945in}{0.868419in}}%
\pgfpathlineto{\pgfqpoint{1.555800in}{0.860608in}}%
\pgfpathlineto{\pgfqpoint{1.554058in}{0.853207in}}%
\pgfpathlineto{\pgfqpoint{1.551728in}{0.846245in}}%
\pgfpathlineto{\pgfqpoint{1.525805in}{0.830045in}}%
\pgfpathlineto{\pgfqpoint{1.498093in}{0.814852in}}%
\pgfpathlineto{\pgfqpoint{1.468706in}{0.800739in}}%
\pgfpathlineto{\pgfqpoint{1.437767in}{0.787774in}}%
\pgfpathlineto{\pgfqpoint{1.405407in}{0.776020in}}%
\pgfpathclose%
\pgfusepath{fill}%
\end{pgfscope}%
\begin{pgfscope}%
\pgfpathrectangle{\pgfqpoint{0.050000in}{0.050000in}}{\pgfqpoint{2.081932in}{2.081932in}}%
\pgfusepath{clip}%
\pgfsetbuttcap%
\pgfsetroundjoin%
\definecolor{currentfill}{rgb}{0.993248,0.906157,0.143936}%
\pgfsetfillcolor{currentfill}%
\pgfsetlinewidth{0.000000pt}%
\definecolor{currentstroke}{rgb}{0.000000,0.000000,0.000000}%
\pgfsetstrokecolor{currentstroke}%
\pgfsetdash{}{0pt}%
\pgfpathmoveto{\pgfqpoint{1.191292in}{1.022154in}}%
\pgfpathlineto{\pgfqpoint{1.188922in}{1.029239in}}%
\pgfpathlineto{\pgfqpoint{1.186567in}{1.035735in}}%
\pgfpathlineto{\pgfqpoint{1.184236in}{1.041618in}}%
\pgfpathlineto{\pgfqpoint{1.181937in}{1.046865in}}%
\pgfpathlineto{\pgfqpoint{1.179681in}{1.051457in}}%
\pgfpathlineto{\pgfqpoint{1.200525in}{1.054127in}}%
\pgfpathlineto{\pgfqpoint{1.221019in}{1.057574in}}%
\pgfpathlineto{\pgfqpoint{1.241075in}{1.061781in}}%
\pgfpathlineto{\pgfqpoint{1.260606in}{1.066729in}}%
\pgfpathlineto{\pgfqpoint{1.279531in}{1.072396in}}%
\pgfpathlineto{\pgfqpoint{1.285494in}{1.068595in}}%
\pgfpathlineto{\pgfqpoint{1.291568in}{1.064155in}}%
\pgfpathlineto{\pgfqpoint{1.297729in}{1.059094in}}%
\pgfpathlineto{\pgfqpoint{1.303950in}{1.053431in}}%
\pgfpathlineto{\pgfqpoint{1.310207in}{1.047187in}}%
\pgfpathlineto{\pgfqpoint{1.287684in}{1.040414in}}%
\pgfpathlineto{\pgfqpoint{1.264430in}{1.034499in}}%
\pgfpathlineto{\pgfqpoint{1.240544in}{1.029469in}}%
\pgfpathlineto{\pgfqpoint{1.216129in}{1.025348in}}%
\pgfpathlineto{\pgfqpoint{1.191292in}{1.022154in}}%
\pgfpathclose%
\pgfusepath{fill}%
\end{pgfscope}%
\begin{pgfscope}%
\pgfpathrectangle{\pgfqpoint{0.050000in}{0.050000in}}{\pgfqpoint{2.081932in}{2.081932in}}%
\pgfusepath{clip}%
\pgfsetbuttcap%
\pgfsetroundjoin%
\definecolor{currentfill}{rgb}{0.267968,0.223549,0.512008}%
\pgfsetfillcolor{currentfill}%
\pgfsetlinewidth{0.000000pt}%
\definecolor{currentstroke}{rgb}{0.000000,0.000000,0.000000}%
\pgfsetstrokecolor{currentstroke}%
\pgfsetdash{}{0pt}%
\pgfpathmoveto{\pgfqpoint{1.047243in}{0.707304in}}%
\pgfpathlineto{\pgfqpoint{1.045855in}{0.706833in}}%
\pgfpathlineto{\pgfqpoint{1.044541in}{0.707085in}}%
\pgfpathlineto{\pgfqpoint{1.043304in}{0.708060in}}%
\pgfpathlineto{\pgfqpoint{1.042152in}{0.709757in}}%
\pgfpathlineto{\pgfqpoint{1.041088in}{0.712171in}}%
\pgfpathlineto{\pgfqpoint{1.077286in}{0.709822in}}%
\pgfpathlineto{\pgfqpoint{1.113674in}{0.708896in}}%
\pgfpathlineto{\pgfqpoint{1.150087in}{0.709397in}}%
\pgfpathlineto{\pgfqpoint{1.186360in}{0.711321in}}%
\pgfpathlineto{\pgfqpoint{1.222328in}{0.714661in}}%
\pgfpathlineto{\pgfqpoint{1.220923in}{0.712213in}}%
\pgfpathlineto{\pgfqpoint{1.219401in}{0.710480in}}%
\pgfpathlineto{\pgfqpoint{1.217768in}{0.709466in}}%
\pgfpathlineto{\pgfqpoint{1.216031in}{0.709173in}}%
\pgfpathlineto{\pgfqpoint{1.214197in}{0.709599in}}%
\pgfpathlineto{\pgfqpoint{1.181061in}{0.706522in}}%
\pgfpathlineto{\pgfqpoint{1.147648in}{0.704748in}}%
\pgfpathlineto{\pgfqpoint{1.114107in}{0.704286in}}%
\pgfpathlineto{\pgfqpoint{1.080589in}{0.705140in}}%
\pgfpathlineto{\pgfqpoint{1.047243in}{0.707304in}}%
\pgfpathclose%
\pgfusepath{fill}%
\end{pgfscope}%
\begin{pgfscope}%
\pgfpathrectangle{\pgfqpoint{0.050000in}{0.050000in}}{\pgfqpoint{2.081932in}{2.081932in}}%
\pgfusepath{clip}%
\pgfsetbuttcap%
\pgfsetroundjoin%
\definecolor{currentfill}{rgb}{0.993248,0.906157,0.143936}%
\pgfsetfillcolor{currentfill}%
\pgfsetlinewidth{0.000000pt}%
\definecolor{currentstroke}{rgb}{0.000000,0.000000,0.000000}%
\pgfsetstrokecolor{currentstroke}%
\pgfsetdash{}{0pt}%
\pgfpathmoveto{\pgfqpoint{0.943776in}{1.042347in}}%
\pgfpathlineto{\pgfqpoint{0.949523in}{1.048753in}}%
\pgfpathlineto{\pgfqpoint{0.955238in}{1.054577in}}%
\pgfpathlineto{\pgfqpoint{0.960896in}{1.059797in}}%
\pgfpathlineto{\pgfqpoint{0.966475in}{1.064392in}}%
\pgfpathlineto{\pgfqpoint{0.971952in}{1.068346in}}%
\pgfpathlineto{\pgfqpoint{0.991309in}{1.063181in}}%
\pgfpathlineto{\pgfqpoint{1.011216in}{1.058751in}}%
\pgfpathlineto{\pgfqpoint{1.031587in}{1.055075in}}%
\pgfpathlineto{\pgfqpoint{1.052335in}{1.052172in}}%
\pgfpathlineto{\pgfqpoint{1.073371in}{1.050054in}}%
\pgfpathlineto{\pgfqpoint{1.071664in}{1.045409in}}%
\pgfpathlineto{\pgfqpoint{1.069924in}{1.040107in}}%
\pgfpathlineto{\pgfqpoint{1.068159in}{1.034169in}}%
\pgfpathlineto{\pgfqpoint{1.066377in}{1.027617in}}%
\pgfpathlineto{\pgfqpoint{1.064583in}{1.020476in}}%
\pgfpathlineto{\pgfqpoint{1.039514in}{1.023009in}}%
\pgfpathlineto{\pgfqpoint{1.014792in}{1.026481in}}%
\pgfpathlineto{\pgfqpoint{0.990526in}{1.030876in}}%
\pgfpathlineto{\pgfqpoint{0.966820in}{1.036173in}}%
\pgfpathlineto{\pgfqpoint{0.943776in}{1.042347in}}%
\pgfpathclose%
\pgfusepath{fill}%
\end{pgfscope}%
\begin{pgfscope}%
\pgfpathrectangle{\pgfqpoint{0.050000in}{0.050000in}}{\pgfqpoint{2.081932in}{2.081932in}}%
\pgfusepath{clip}%
\pgfsetbuttcap%
\pgfsetroundjoin%
\definecolor{currentfill}{rgb}{0.993248,0.906157,0.143936}%
\pgfsetfillcolor{currentfill}%
\pgfsetlinewidth{0.000000pt}%
\definecolor{currentstroke}{rgb}{0.000000,0.000000,0.000000}%
\pgfsetstrokecolor{currentstroke}%
\pgfsetdash{}{0pt}%
\pgfpathmoveto{\pgfqpoint{0.818379in}{1.040860in}}%
\pgfpathlineto{\pgfqpoint{0.826542in}{1.048674in}}%
\pgfpathlineto{\pgfqpoint{0.834833in}{1.056042in}}%
\pgfpathlineto{\pgfqpoint{0.843218in}{1.062935in}}%
\pgfpathlineto{\pgfqpoint{0.851662in}{1.069324in}}%
\pgfpathlineto{\pgfqpoint{0.860129in}{1.075185in}}%
\pgfpathlineto{\pgfqpoint{0.879582in}{1.065827in}}%
\pgfpathlineto{\pgfqpoint{0.900065in}{1.057208in}}%
\pgfpathlineto{\pgfqpoint{0.921493in}{1.049370in}}%
\pgfpathlineto{\pgfqpoint{0.943776in}{1.042347in}}%
\pgfpathlineto{\pgfqpoint{0.938019in}{1.035384in}}%
\pgfpathlineto{\pgfqpoint{0.932277in}{1.027891in}}%
\pgfpathlineto{\pgfqpoint{0.926573in}{1.019899in}}%
\pgfpathlineto{\pgfqpoint{0.920932in}{1.011439in}}%
\pgfpathlineto{\pgfqpoint{0.915376in}{1.002545in}}%
\pgfpathlineto{\pgfqpoint{0.889516in}{1.010743in}}%
\pgfpathlineto{\pgfqpoint{0.864661in}{1.019891in}}%
\pgfpathlineto{\pgfqpoint{0.840915in}{1.029946in}}%
\pgfpathlineto{\pgfqpoint{0.818379in}{1.040860in}}%
\pgfpathclose%
\pgfusepath{fill}%
\end{pgfscope}%
\begin{pgfscope}%
\pgfpathrectangle{\pgfqpoint{0.050000in}{0.050000in}}{\pgfqpoint{2.081932in}{2.081932in}}%
\pgfusepath{clip}%
\pgfsetbuttcap%
\pgfsetroundjoin%
\definecolor{currentfill}{rgb}{0.993248,0.906157,0.143936}%
\pgfsetfillcolor{currentfill}%
\pgfsetlinewidth{0.000000pt}%
\definecolor{currentstroke}{rgb}{0.000000,0.000000,0.000000}%
\pgfsetstrokecolor{currentstroke}%
\pgfsetdash{}{0pt}%
\pgfpathmoveto{\pgfqpoint{1.064583in}{1.020476in}}%
\pgfpathlineto{\pgfqpoint{1.066377in}{1.027617in}}%
\pgfpathlineto{\pgfqpoint{1.068159in}{1.034169in}}%
\pgfpathlineto{\pgfqpoint{1.069924in}{1.040107in}}%
\pgfpathlineto{\pgfqpoint{1.071664in}{1.045409in}}%
\pgfpathlineto{\pgfqpoint{1.073371in}{1.050054in}}%
\pgfpathlineto{\pgfqpoint{1.094605in}{1.048731in}}%
\pgfpathlineto{\pgfqpoint{1.115946in}{1.048210in}}%
\pgfpathlineto{\pgfqpoint{1.137300in}{1.048492in}}%
\pgfpathlineto{\pgfqpoint{1.158575in}{1.049576in}}%
\pgfpathlineto{\pgfqpoint{1.179681in}{1.051457in}}%
\pgfpathlineto{\pgfqpoint{1.181937in}{1.046865in}}%
\pgfpathlineto{\pgfqpoint{1.184236in}{1.041618in}}%
\pgfpathlineto{\pgfqpoint{1.186567in}{1.035735in}}%
\pgfpathlineto{\pgfqpoint{1.188922in}{1.029239in}}%
\pgfpathlineto{\pgfqpoint{1.191292in}{1.022154in}}%
\pgfpathlineto{\pgfqpoint{1.166139in}{1.019904in}}%
\pgfpathlineto{\pgfqpoint{1.140780in}{1.018607in}}%
\pgfpathlineto{\pgfqpoint{1.115328in}{1.018269in}}%
\pgfpathlineto{\pgfqpoint{1.089892in}{1.018893in}}%
\pgfpathlineto{\pgfqpoint{1.064583in}{1.020476in}}%
\pgfpathclose%
\pgfusepath{fill}%
\end{pgfscope}%
\begin{pgfscope}%
\pgfpathrectangle{\pgfqpoint{0.050000in}{0.050000in}}{\pgfqpoint{2.081932in}{2.081932in}}%
\pgfusepath{clip}%
\pgfsetbuttcap%
\pgfsetroundjoin%
\definecolor{currentfill}{rgb}{0.150476,0.504369,0.557430}%
\pgfsetfillcolor{currentfill}%
\pgfsetlinewidth{0.000000pt}%
\definecolor{currentstroke}{rgb}{0.000000,0.000000,0.000000}%
\pgfsetstrokecolor{currentstroke}%
\pgfsetdash{}{0pt}%
\pgfpathmoveto{\pgfqpoint{0.731700in}{0.819276in}}%
\pgfpathlineto{\pgfqpoint{0.729606in}{0.826115in}}%
\pgfpathlineto{\pgfqpoint{0.728041in}{0.833432in}}%
\pgfpathlineto{\pgfqpoint{0.727012in}{0.841197in}}%
\pgfpathlineto{\pgfqpoint{0.726524in}{0.849378in}}%
\pgfpathlineto{\pgfqpoint{0.755796in}{0.834794in}}%
\pgfpathlineto{\pgfqpoint{0.786682in}{0.821348in}}%
\pgfpathlineto{\pgfqpoint{0.819051in}{0.809107in}}%
\pgfpathlineto{\pgfqpoint{0.852765in}{0.798129in}}%
\pgfpathlineto{\pgfqpoint{0.853098in}{0.789963in}}%
\pgfpathlineto{\pgfqpoint{0.853800in}{0.782286in}}%
\pgfpathlineto{\pgfqpoint{0.854868in}{0.775132in}}%
\pgfpathlineto{\pgfqpoint{0.856297in}{0.768527in}}%
\pgfpathlineto{\pgfqpoint{0.823026in}{0.779397in}}%
\pgfpathlineto{\pgfqpoint{0.791079in}{0.791519in}}%
\pgfpathlineto{\pgfqpoint{0.760594in}{0.804834in}}%
\pgfpathlineto{\pgfqpoint{0.731700in}{0.819276in}}%
\pgfpathclose%
\pgfusepath{fill}%
\end{pgfscope}%
\begin{pgfscope}%
\pgfpathrectangle{\pgfqpoint{0.050000in}{0.050000in}}{\pgfqpoint{2.081932in}{2.081932in}}%
\pgfusepath{clip}%
\pgfsetbuttcap%
\pgfsetroundjoin%
\definecolor{currentfill}{rgb}{0.124780,0.640461,0.527068}%
\pgfsetfillcolor{currentfill}%
\pgfsetlinewidth{0.000000pt}%
\definecolor{currentstroke}{rgb}{0.000000,0.000000,0.000000}%
\pgfsetstrokecolor{currentstroke}%
\pgfsetdash{}{0pt}%
\pgfpathmoveto{\pgfqpoint{1.409250in}{0.805696in}}%
\pgfpathlineto{\pgfqpoint{1.409208in}{0.814311in}}%
\pgfpathlineto{\pgfqpoint{1.408762in}{0.823337in}}%
\pgfpathlineto{\pgfqpoint{1.407913in}{0.832736in}}%
\pgfpathlineto{\pgfqpoint{1.406663in}{0.842470in}}%
\pgfpathlineto{\pgfqpoint{1.405018in}{0.852498in}}%
\pgfpathlineto{\pgfqpoint{1.437335in}{0.864131in}}%
\pgfpathlineto{\pgfqpoint{1.468233in}{0.876962in}}%
\pgfpathlineto{\pgfqpoint{1.497580in}{0.890929in}}%
\pgfpathlineto{\pgfqpoint{1.525255in}{0.905965in}}%
\pgfpathlineto{\pgfqpoint{1.551145in}{0.921998in}}%
\pgfpathlineto{\pgfqpoint{1.553611in}{0.912450in}}%
\pgfpathlineto{\pgfqpoint{1.555483in}{0.903099in}}%
\pgfpathlineto{\pgfqpoint{1.556756in}{0.893983in}}%
\pgfpathlineto{\pgfqpoint{1.557425in}{0.885141in}}%
\pgfpathlineto{\pgfqpoint{1.557488in}{0.876608in}}%
\pgfpathlineto{\pgfqpoint{1.531229in}{0.860251in}}%
\pgfpathlineto{\pgfqpoint{1.503157in}{0.844910in}}%
\pgfpathlineto{\pgfqpoint{1.473386in}{0.830659in}}%
\pgfpathlineto{\pgfqpoint{1.442039in}{0.817566in}}%
\pgfpathlineto{\pgfqpoint{1.409250in}{0.805696in}}%
\pgfpathclose%
\pgfusepath{fill}%
\end{pgfscope}%
\begin{pgfscope}%
\pgfpathrectangle{\pgfqpoint{0.050000in}{0.050000in}}{\pgfqpoint{2.081932in}{2.081932in}}%
\pgfusepath{clip}%
\pgfsetbuttcap%
\pgfsetroundjoin%
\definecolor{currentfill}{rgb}{0.876168,0.891125,0.095250}%
\pgfsetfillcolor{currentfill}%
\pgfsetlinewidth{0.000000pt}%
\definecolor{currentstroke}{rgb}{0.000000,0.000000,0.000000}%
\pgfsetstrokecolor{currentstroke}%
\pgfsetdash{}{0pt}%
\pgfpathmoveto{\pgfqpoint{1.369078in}{0.959308in}}%
\pgfpathlineto{\pgfqpoint{1.363870in}{0.969648in}}%
\pgfpathlineto{\pgfqpoint{1.358447in}{0.979749in}}%
\pgfpathlineto{\pgfqpoint{1.352832in}{0.989567in}}%
\pgfpathlineto{\pgfqpoint{1.347048in}{0.999062in}}%
\pgfpathlineto{\pgfqpoint{1.341120in}{1.008196in}}%
\pgfpathlineto{\pgfqpoint{1.366289in}{1.017064in}}%
\pgfpathlineto{\pgfqpoint{1.390380in}{1.026853in}}%
\pgfpathlineto{\pgfqpoint{1.413290in}{1.037516in}}%
\pgfpathlineto{\pgfqpoint{1.434927in}{1.049004in}}%
\pgfpathlineto{\pgfqpoint{1.455201in}{1.061263in}}%
\pgfpathlineto{\pgfqpoint{1.464117in}{1.053598in}}%
\pgfpathlineto{\pgfqpoint{1.472813in}{1.045541in}}%
\pgfpathlineto{\pgfqpoint{1.481253in}{1.037126in}}%
\pgfpathlineto{\pgfqpoint{1.489401in}{1.028389in}}%
\pgfpathlineto{\pgfqpoint{1.497224in}{1.019364in}}%
\pgfpathlineto{\pgfqpoint{1.474481in}{1.005499in}}%
\pgfpathlineto{\pgfqpoint{1.450192in}{0.992502in}}%
\pgfpathlineto{\pgfqpoint{1.424456in}{0.980433in}}%
\pgfpathlineto{\pgfqpoint{1.397380in}{0.969351in}}%
\pgfpathlineto{\pgfqpoint{1.369078in}{0.959308in}}%
\pgfpathclose%
\pgfusepath{fill}%
\end{pgfscope}%
\begin{pgfscope}%
\pgfpathrectangle{\pgfqpoint{0.050000in}{0.050000in}}{\pgfqpoint{2.081932in}{2.081932in}}%
\pgfusepath{clip}%
\pgfsetbuttcap%
\pgfsetroundjoin%
\definecolor{currentfill}{rgb}{0.206756,0.371758,0.553117}%
\pgfsetfillcolor{currentfill}%
\pgfsetlinewidth{0.000000pt}%
\definecolor{currentstroke}{rgb}{0.000000,0.000000,0.000000}%
\pgfsetstrokecolor{currentstroke}%
\pgfsetdash{}{0pt}%
\pgfpathmoveto{\pgfqpoint{1.222328in}{0.714661in}}%
\pgfpathlineto{\pgfqpoint{1.223610in}{0.717814in}}%
\pgfpathlineto{\pgfqpoint{1.224763in}{0.721662in}}%
\pgfpathlineto{\pgfqpoint{1.225781in}{0.726191in}}%
\pgfpathlineto{\pgfqpoint{1.226660in}{0.731384in}}%
\pgfpathlineto{\pgfqpoint{1.227397in}{0.737220in}}%
\pgfpathlineto{\pgfqpoint{1.264641in}{0.742177in}}%
\pgfpathlineto{\pgfqpoint{1.301225in}{0.748570in}}%
\pgfpathlineto{\pgfqpoint{1.336985in}{0.756369in}}%
\pgfpathlineto{\pgfqpoint{1.371764in}{0.765534in}}%
\pgfpathlineto{\pgfqpoint{1.405407in}{0.776020in}}%
\pgfpathlineto{\pgfqpoint{1.403467in}{0.769946in}}%
\pgfpathlineto{\pgfqpoint{1.401152in}{0.764462in}}%
\pgfpathlineto{\pgfqpoint{1.398470in}{0.759589in}}%
\pgfpathlineto{\pgfqpoint{1.395435in}{0.755346in}}%
\pgfpathlineto{\pgfqpoint{1.392060in}{0.751748in}}%
\pgfpathlineto{\pgfqpoint{1.359972in}{0.741724in}}%
\pgfpathlineto{\pgfqpoint{1.326807in}{0.732962in}}%
\pgfpathlineto{\pgfqpoint{1.292709in}{0.725508in}}%
\pgfpathlineto{\pgfqpoint{1.257831in}{0.719397in}}%
\pgfpathlineto{\pgfqpoint{1.222328in}{0.714661in}}%
\pgfpathclose%
\pgfusepath{fill}%
\end{pgfscope}%
\begin{pgfscope}%
\pgfpathrectangle{\pgfqpoint{0.050000in}{0.050000in}}{\pgfqpoint{2.081932in}{2.081932in}}%
\pgfusepath{clip}%
\pgfsetbuttcap%
\pgfsetroundjoin%
\definecolor{currentfill}{rgb}{0.206756,0.371758,0.553117}%
\pgfsetfillcolor{currentfill}%
\pgfsetlinewidth{0.000000pt}%
\definecolor{currentstroke}{rgb}{0.000000,0.000000,0.000000}%
\pgfsetstrokecolor{currentstroke}%
\pgfsetdash{}{0pt}%
\pgfpathmoveto{\pgfqpoint{0.868564in}{0.744585in}}%
\pgfpathlineto{\pgfqpoint{0.865462in}{0.748098in}}%
\pgfpathlineto{\pgfqpoint{0.862673in}{0.752265in}}%
\pgfpathlineto{\pgfqpoint{0.860208in}{0.757072in}}%
\pgfpathlineto{\pgfqpoint{0.858080in}{0.762499in}}%
\pgfpathlineto{\pgfqpoint{0.856297in}{0.768527in}}%
\pgfpathlineto{\pgfqpoint{0.890749in}{0.758963in}}%
\pgfpathlineto{\pgfqpoint{0.926231in}{0.750752in}}%
\pgfpathlineto{\pgfqpoint{0.962584in}{0.743936in}}%
\pgfpathlineto{\pgfqpoint{0.999646in}{0.738547in}}%
\pgfpathlineto{\pgfqpoint{1.037251in}{0.734615in}}%
\pgfpathlineto{\pgfqpoint{1.037809in}{0.728794in}}%
\pgfpathlineto{\pgfqpoint{1.038475in}{0.723621in}}%
\pgfpathlineto{\pgfqpoint{1.039245in}{0.719115in}}%
\pgfpathlineto{\pgfqpoint{1.040118in}{0.715294in}}%
\pgfpathlineto{\pgfqpoint{1.041088in}{0.712171in}}%
\pgfpathlineto{\pgfqpoint{1.005242in}{0.715929in}}%
\pgfpathlineto{\pgfqpoint{0.969911in}{0.721078in}}%
\pgfpathlineto{\pgfqpoint{0.935252in}{0.727594in}}%
\pgfpathlineto{\pgfqpoint{0.901420in}{0.735442in}}%
\pgfpathlineto{\pgfqpoint{0.868564in}{0.744585in}}%
\pgfpathclose%
\pgfusepath{fill}%
\end{pgfscope}%
\begin{pgfscope}%
\pgfpathrectangle{\pgfqpoint{0.050000in}{0.050000in}}{\pgfqpoint{2.081932in}{2.081932in}}%
\pgfusepath{clip}%
\pgfsetbuttcap%
\pgfsetroundjoin%
\definecolor{currentfill}{rgb}{0.327796,0.773980,0.406640}%
\pgfsetfillcolor{currentfill}%
\pgfsetlinewidth{0.000000pt}%
\definecolor{currentstroke}{rgb}{0.000000,0.000000,0.000000}%
\pgfsetstrokecolor{currentstroke}%
\pgfsetdash{}{0pt}%
\pgfpathmoveto{\pgfqpoint{1.405018in}{0.852498in}}%
\pgfpathlineto{\pgfqpoint{1.402984in}{0.862778in}}%
\pgfpathlineto{\pgfqpoint{1.400569in}{0.873266in}}%
\pgfpathlineto{\pgfqpoint{1.397784in}{0.883920in}}%
\pgfpathlineto{\pgfqpoint{1.394640in}{0.894694in}}%
\pgfpathlineto{\pgfqpoint{1.391150in}{0.905543in}}%
\pgfpathlineto{\pgfqpoint{1.421920in}{0.916543in}}%
\pgfpathlineto{\pgfqpoint{1.451345in}{0.928677in}}%
\pgfpathlineto{\pgfqpoint{1.479302in}{0.941889in}}%
\pgfpathlineto{\pgfqpoint{1.505673in}{0.956113in}}%
\pgfpathlineto{\pgfqpoint{1.530353in}{0.971283in}}%
\pgfpathlineto{\pgfqpoint{1.535587in}{0.961357in}}%
\pgfpathlineto{\pgfqpoint{1.540301in}{0.951424in}}%
\pgfpathlineto{\pgfqpoint{1.544477in}{0.941526in}}%
\pgfpathlineto{\pgfqpoint{1.548096in}{0.931704in}}%
\pgfpathlineto{\pgfqpoint{1.551145in}{0.921998in}}%
\pgfpathlineto{\pgfqpoint{1.525255in}{0.905965in}}%
\pgfpathlineto{\pgfqpoint{1.497580in}{0.890929in}}%
\pgfpathlineto{\pgfqpoint{1.468233in}{0.876962in}}%
\pgfpathlineto{\pgfqpoint{1.437335in}{0.864131in}}%
\pgfpathlineto{\pgfqpoint{1.405018in}{0.852498in}}%
\pgfpathclose%
\pgfusepath{fill}%
\end{pgfscope}%
\begin{pgfscope}%
\pgfpathrectangle{\pgfqpoint{0.050000in}{0.050000in}}{\pgfqpoint{2.081932in}{2.081932in}}%
\pgfusepath{clip}%
\pgfsetbuttcap%
\pgfsetroundjoin%
\definecolor{currentfill}{rgb}{0.636902,0.856542,0.216620}%
\pgfsetfillcolor{currentfill}%
\pgfsetlinewidth{0.000000pt}%
\definecolor{currentstroke}{rgb}{0.000000,0.000000,0.000000}%
\pgfsetstrokecolor{currentstroke}%
\pgfsetdash{}{0pt}%
\pgfpathmoveto{\pgfqpoint{1.391150in}{0.905543in}}%
\pgfpathlineto{\pgfqpoint{1.387329in}{0.916421in}}%
\pgfpathlineto{\pgfqpoint{1.383193in}{0.927282in}}%
\pgfpathlineto{\pgfqpoint{1.378760in}{0.938080in}}%
\pgfpathlineto{\pgfqpoint{1.374048in}{0.948770in}}%
\pgfpathlineto{\pgfqpoint{1.369078in}{0.959308in}}%
\pgfpathlineto{\pgfqpoint{1.397380in}{0.969351in}}%
\pgfpathlineto{\pgfqpoint{1.424456in}{0.980433in}}%
\pgfpathlineto{\pgfqpoint{1.450192in}{0.992502in}}%
\pgfpathlineto{\pgfqpoint{1.474481in}{1.005499in}}%
\pgfpathlineto{\pgfqpoint{1.497224in}{1.019364in}}%
\pgfpathlineto{\pgfqpoint{1.504688in}{1.010089in}}%
\pgfpathlineto{\pgfqpoint{1.511761in}{1.000604in}}%
\pgfpathlineto{\pgfqpoint{1.518415in}{0.990947in}}%
\pgfpathlineto{\pgfqpoint{1.524621in}{0.981160in}}%
\pgfpathlineto{\pgfqpoint{1.530353in}{0.971283in}}%
\pgfpathlineto{\pgfqpoint{1.505673in}{0.956113in}}%
\pgfpathlineto{\pgfqpoint{1.479302in}{0.941889in}}%
\pgfpathlineto{\pgfqpoint{1.451345in}{0.928677in}}%
\pgfpathlineto{\pgfqpoint{1.421920in}{0.916543in}}%
\pgfpathlineto{\pgfqpoint{1.391150in}{0.905543in}}%
\pgfpathclose%
\pgfusepath{fill}%
\end{pgfscope}%
\begin{pgfscope}%
\pgfpathrectangle{\pgfqpoint{0.050000in}{0.050000in}}{\pgfqpoint{2.081932in}{2.081932in}}%
\pgfusepath{clip}%
\pgfsetbuttcap%
\pgfsetroundjoin%
\definecolor{currentfill}{rgb}{0.876168,0.891125,0.095250}%
\pgfsetfillcolor{currentfill}%
\pgfsetlinewidth{0.000000pt}%
\definecolor{currentstroke}{rgb}{0.000000,0.000000,0.000000}%
\pgfsetstrokecolor{currentstroke}%
\pgfsetdash{}{0pt}%
\pgfpathmoveto{\pgfqpoint{0.780659in}{0.996286in}}%
\pgfpathlineto{\pgfqpoint{0.787682in}{1.005814in}}%
\pgfpathlineto{\pgfqpoint{0.794997in}{1.015074in}}%
\pgfpathlineto{\pgfqpoint{0.802573in}{1.024026in}}%
\pgfpathlineto{\pgfqpoint{0.810378in}{1.032633in}}%
\pgfpathlineto{\pgfqpoint{0.818379in}{1.040860in}}%
\pgfpathlineto{\pgfqpoint{0.840915in}{1.029946in}}%
\pgfpathlineto{\pgfqpoint{0.864661in}{1.019891in}}%
\pgfpathlineto{\pgfqpoint{0.889516in}{1.010743in}}%
\pgfpathlineto{\pgfqpoint{0.915376in}{1.002545in}}%
\pgfpathlineto{\pgfqpoint{0.909929in}{0.993255in}}%
\pgfpathlineto{\pgfqpoint{0.904615in}{0.983605in}}%
\pgfpathlineto{\pgfqpoint{0.899455in}{0.973636in}}%
\pgfpathlineto{\pgfqpoint{0.894472in}{0.963390in}}%
\pgfpathlineto{\pgfqpoint{0.889686in}{0.952908in}}%
\pgfpathlineto{\pgfqpoint{0.860599in}{0.962193in}}%
\pgfpathlineto{\pgfqpoint{0.832654in}{0.972552in}}%
\pgfpathlineto{\pgfqpoint{0.805970in}{0.983934in}}%
\pgfpathlineto{\pgfqpoint{0.780659in}{0.996286in}}%
\pgfpathclose%
\pgfusepath{fill}%
\end{pgfscope}%
\begin{pgfscope}%
\pgfpathrectangle{\pgfqpoint{0.050000in}{0.050000in}}{\pgfqpoint{2.081932in}{2.081932in}}%
\pgfusepath{clip}%
\pgfsetbuttcap%
\pgfsetroundjoin%
\definecolor{currentfill}{rgb}{0.124780,0.640461,0.527068}%
\pgfsetfillcolor{currentfill}%
\pgfsetlinewidth{0.000000pt}%
\definecolor{currentstroke}{rgb}{0.000000,0.000000,0.000000}%
\pgfsetstrokecolor{currentstroke}%
\pgfsetdash{}{0pt}%
\pgfpathmoveto{\pgfqpoint{0.726524in}{0.849378in}}%
\pgfpathlineto{\pgfqpoint{0.726581in}{0.857942in}}%
\pgfpathlineto{\pgfqpoint{0.727182in}{0.866855in}}%
\pgfpathlineto{\pgfqpoint{0.728326in}{0.876079in}}%
\pgfpathlineto{\pgfqpoint{0.730009in}{0.885576in}}%
\pgfpathlineto{\pgfqpoint{0.732224in}{0.895308in}}%
\pgfpathlineto{\pgfqpoint{0.761081in}{0.881014in}}%
\pgfpathlineto{\pgfqpoint{0.791525in}{0.867837in}}%
\pgfpathlineto{\pgfqpoint{0.823429in}{0.855840in}}%
\pgfpathlineto{\pgfqpoint{0.856655in}{0.845083in}}%
\pgfpathlineto{\pgfqpoint{0.855143in}{0.835003in}}%
\pgfpathlineto{\pgfqpoint{0.853994in}{0.825228in}}%
\pgfpathlineto{\pgfqpoint{0.853214in}{0.815798in}}%
\pgfpathlineto{\pgfqpoint{0.852803in}{0.806753in}}%
\pgfpathlineto{\pgfqpoint{0.852765in}{0.798129in}}%
\pgfpathlineto{\pgfqpoint{0.819051in}{0.809107in}}%
\pgfpathlineto{\pgfqpoint{0.786682in}{0.821348in}}%
\pgfpathlineto{\pgfqpoint{0.755796in}{0.834794in}}%
\pgfpathlineto{\pgfqpoint{0.726524in}{0.849378in}}%
\pgfpathclose%
\pgfusepath{fill}%
\end{pgfscope}%
\begin{pgfscope}%
\pgfpathrectangle{\pgfqpoint{0.050000in}{0.050000in}}{\pgfqpoint{2.081932in}{2.081932in}}%
\pgfusepath{clip}%
\pgfsetbuttcap%
\pgfsetroundjoin%
\definecolor{currentfill}{rgb}{0.993248,0.906157,0.143936}%
\pgfsetfillcolor{currentfill}%
\pgfsetlinewidth{0.000000pt}%
\definecolor{currentstroke}{rgb}{0.000000,0.000000,0.000000}%
\pgfsetstrokecolor{currentstroke}%
\pgfsetdash{}{0pt}%
\pgfpathmoveto{\pgfqpoint{1.203004in}{0.978960in}}%
\pgfpathlineto{\pgfqpoint{1.200712in}{0.988523in}}%
\pgfpathlineto{\pgfqpoint{1.198385in}{0.997660in}}%
\pgfpathlineto{\pgfqpoint{1.196033in}{1.006333in}}%
\pgfpathlineto{\pgfqpoint{1.193665in}{1.014509in}}%
\pgfpathlineto{\pgfqpoint{1.191292in}{1.022154in}}%
\pgfpathlineto{\pgfqpoint{1.216129in}{1.025348in}}%
\pgfpathlineto{\pgfqpoint{1.240544in}{1.029469in}}%
\pgfpathlineto{\pgfqpoint{1.264430in}{1.034499in}}%
\pgfpathlineto{\pgfqpoint{1.287684in}{1.040414in}}%
\pgfpathlineto{\pgfqpoint{1.310207in}{1.047187in}}%
\pgfpathlineto{\pgfqpoint{1.316473in}{1.040387in}}%
\pgfpathlineto{\pgfqpoint{1.322724in}{1.033057in}}%
\pgfpathlineto{\pgfqpoint{1.328932in}{1.025228in}}%
\pgfpathlineto{\pgfqpoint{1.335073in}{1.016929in}}%
\pgfpathlineto{\pgfqpoint{1.341120in}{1.008196in}}%
\pgfpathlineto{\pgfqpoint{1.314979in}{1.000289in}}%
\pgfpathlineto{\pgfqpoint{1.287978in}{0.993381in}}%
\pgfpathlineto{\pgfqpoint{1.260234in}{0.987506in}}%
\pgfpathlineto{\pgfqpoint{1.231868in}{0.982691in}}%
\pgfpathlineto{\pgfqpoint{1.203004in}{0.978960in}}%
\pgfpathclose%
\pgfusepath{fill}%
\end{pgfscope}%
\begin{pgfscope}%
\pgfpathrectangle{\pgfqpoint{0.050000in}{0.050000in}}{\pgfqpoint{2.081932in}{2.081932in}}%
\pgfusepath{clip}%
\pgfsetbuttcap%
\pgfsetroundjoin%
\definecolor{currentfill}{rgb}{0.993248,0.906157,0.143936}%
\pgfsetfillcolor{currentfill}%
\pgfsetlinewidth{0.000000pt}%
\definecolor{currentstroke}{rgb}{0.000000,0.000000,0.000000}%
\pgfsetstrokecolor{currentstroke}%
\pgfsetdash{}{0pt}%
\pgfpathmoveto{\pgfqpoint{0.915376in}{1.002545in}}%
\pgfpathlineto{\pgfqpoint{0.920932in}{1.011439in}}%
\pgfpathlineto{\pgfqpoint{0.926573in}{1.019899in}}%
\pgfpathlineto{\pgfqpoint{0.932277in}{1.027891in}}%
\pgfpathlineto{\pgfqpoint{0.938019in}{1.035384in}}%
\pgfpathlineto{\pgfqpoint{0.943776in}{1.042347in}}%
\pgfpathlineto{\pgfqpoint{0.966820in}{1.036173in}}%
\pgfpathlineto{\pgfqpoint{0.990526in}{1.030876in}}%
\pgfpathlineto{\pgfqpoint{1.014792in}{1.026481in}}%
\pgfpathlineto{\pgfqpoint{1.039514in}{1.023009in}}%
\pgfpathlineto{\pgfqpoint{1.064583in}{1.020476in}}%
\pgfpathlineto{\pgfqpoint{1.062787in}{1.012774in}}%
\pgfpathlineto{\pgfqpoint{1.060995in}{1.004541in}}%
\pgfpathlineto{\pgfqpoint{1.059214in}{0.995811in}}%
\pgfpathlineto{\pgfqpoint{1.057453in}{0.986618in}}%
\pgfpathlineto{\pgfqpoint{1.055718in}{0.976999in}}%
\pgfpathlineto{\pgfqpoint{1.026581in}{0.979959in}}%
\pgfpathlineto{\pgfqpoint{0.997854in}{0.984016in}}%
\pgfpathlineto{\pgfqpoint{0.969662in}{0.989150in}}%
\pgfpathlineto{\pgfqpoint{0.942130in}{0.995336in}}%
\pgfpathlineto{\pgfqpoint{0.915376in}{1.002545in}}%
\pgfpathclose%
\pgfusepath{fill}%
\end{pgfscope}%
\begin{pgfscope}%
\pgfpathrectangle{\pgfqpoint{0.050000in}{0.050000in}}{\pgfqpoint{2.081932in}{2.081932in}}%
\pgfusepath{clip}%
\pgfsetbuttcap%
\pgfsetroundjoin%
\definecolor{currentfill}{rgb}{0.206756,0.371758,0.553117}%
\pgfsetfillcolor{currentfill}%
\pgfsetlinewidth{0.000000pt}%
\definecolor{currentstroke}{rgb}{0.000000,0.000000,0.000000}%
\pgfsetstrokecolor{currentstroke}%
\pgfsetdash{}{0pt}%
\pgfpathmoveto{\pgfqpoint{1.041088in}{0.712171in}}%
\pgfpathlineto{\pgfqpoint{1.040118in}{0.715294in}}%
\pgfpathlineto{\pgfqpoint{1.039245in}{0.719115in}}%
\pgfpathlineto{\pgfqpoint{1.038475in}{0.723621in}}%
\pgfpathlineto{\pgfqpoint{1.037809in}{0.728794in}}%
\pgfpathlineto{\pgfqpoint{1.037251in}{0.734615in}}%
\pgfpathlineto{\pgfqpoint{1.075228in}{0.732158in}}%
\pgfpathlineto{\pgfqpoint{1.113404in}{0.731189in}}%
\pgfpathlineto{\pgfqpoint{1.151607in}{0.731712in}}%
\pgfpathlineto{\pgfqpoint{1.189662in}{0.733726in}}%
\pgfpathlineto{\pgfqpoint{1.227397in}{0.737220in}}%
\pgfpathlineto{\pgfqpoint{1.226660in}{0.731384in}}%
\pgfpathlineto{\pgfqpoint{1.225781in}{0.726191in}}%
\pgfpathlineto{\pgfqpoint{1.224763in}{0.721662in}}%
\pgfpathlineto{\pgfqpoint{1.223610in}{0.717814in}}%
\pgfpathlineto{\pgfqpoint{1.222328in}{0.714661in}}%
\pgfpathlineto{\pgfqpoint{1.186360in}{0.711321in}}%
\pgfpathlineto{\pgfqpoint{1.150087in}{0.709397in}}%
\pgfpathlineto{\pgfqpoint{1.113674in}{0.708896in}}%
\pgfpathlineto{\pgfqpoint{1.077286in}{0.709822in}}%
\pgfpathlineto{\pgfqpoint{1.041088in}{0.712171in}}%
\pgfpathclose%
\pgfusepath{fill}%
\end{pgfscope}%
\begin{pgfscope}%
\pgfpathrectangle{\pgfqpoint{0.050000in}{0.050000in}}{\pgfqpoint{2.081932in}{2.081932in}}%
\pgfusepath{clip}%
\pgfsetbuttcap%
\pgfsetroundjoin%
\definecolor{currentfill}{rgb}{0.636902,0.856542,0.216620}%
\pgfsetfillcolor{currentfill}%
\pgfsetlinewidth{0.000000pt}%
\definecolor{currentstroke}{rgb}{0.000000,0.000000,0.000000}%
\pgfsetstrokecolor{currentstroke}%
\pgfsetdash{}{0pt}%
\pgfpathmoveto{\pgfqpoint{0.750906in}{0.946031in}}%
\pgfpathlineto{\pgfqpoint{0.756055in}{0.956290in}}%
\pgfpathlineto{\pgfqpoint{0.761629in}{0.966488in}}%
\pgfpathlineto{\pgfqpoint{0.767605in}{0.976582in}}%
\pgfpathlineto{\pgfqpoint{0.773957in}{0.986528in}}%
\pgfpathlineto{\pgfqpoint{0.780659in}{0.996286in}}%
\pgfpathlineto{\pgfqpoint{0.805970in}{0.983934in}}%
\pgfpathlineto{\pgfqpoint{0.832654in}{0.972552in}}%
\pgfpathlineto{\pgfqpoint{0.860599in}{0.962193in}}%
\pgfpathlineto{\pgfqpoint{0.889686in}{0.952908in}}%
\pgfpathlineto{\pgfqpoint{0.885119in}{0.942235in}}%
\pgfpathlineto{\pgfqpoint{0.880789in}{0.931415in}}%
\pgfpathlineto{\pgfqpoint{0.876714in}{0.920494in}}%
\pgfpathlineto{\pgfqpoint{0.872913in}{0.909518in}}%
\pgfpathlineto{\pgfqpoint{0.869401in}{0.898532in}}%
\pgfpathlineto{\pgfqpoint{0.837770in}{0.908704in}}%
\pgfpathlineto{\pgfqpoint{0.807392in}{0.920048in}}%
\pgfpathlineto{\pgfqpoint{0.778397in}{0.932510in}}%
\pgfpathlineto{\pgfqpoint{0.750906in}{0.946031in}}%
\pgfpathclose%
\pgfusepath{fill}%
\end{pgfscope}%
\begin{pgfscope}%
\pgfpathrectangle{\pgfqpoint{0.050000in}{0.050000in}}{\pgfqpoint{2.081932in}{2.081932in}}%
\pgfusepath{clip}%
\pgfsetbuttcap%
\pgfsetroundjoin%
\definecolor{currentfill}{rgb}{0.327796,0.773980,0.406640}%
\pgfsetfillcolor{currentfill}%
\pgfsetlinewidth{0.000000pt}%
\definecolor{currentstroke}{rgb}{0.000000,0.000000,0.000000}%
\pgfsetstrokecolor{currentstroke}%
\pgfsetdash{}{0pt}%
\pgfpathmoveto{\pgfqpoint{0.732224in}{0.895308in}}%
\pgfpathlineto{\pgfqpoint{0.734964in}{0.905233in}}%
\pgfpathlineto{\pgfqpoint{0.738216in}{0.915310in}}%
\pgfpathlineto{\pgfqpoint{0.741968in}{0.925497in}}%
\pgfpathlineto{\pgfqpoint{0.746204in}{0.935752in}}%
\pgfpathlineto{\pgfqpoint{0.750906in}{0.946031in}}%
\pgfpathlineto{\pgfqpoint{0.778397in}{0.932510in}}%
\pgfpathlineto{\pgfqpoint{0.807392in}{0.920048in}}%
\pgfpathlineto{\pgfqpoint{0.837770in}{0.908704in}}%
\pgfpathlineto{\pgfqpoint{0.869401in}{0.898532in}}%
\pgfpathlineto{\pgfqpoint{0.866194in}{0.887585in}}%
\pgfpathlineto{\pgfqpoint{0.863304in}{0.876720in}}%
\pgfpathlineto{\pgfqpoint{0.860744in}{0.865985in}}%
\pgfpathlineto{\pgfqpoint{0.858525in}{0.855424in}}%
\pgfpathlineto{\pgfqpoint{0.856655in}{0.845083in}}%
\pgfpathlineto{\pgfqpoint{0.823429in}{0.855840in}}%
\pgfpathlineto{\pgfqpoint{0.791525in}{0.867837in}}%
\pgfpathlineto{\pgfqpoint{0.761081in}{0.881014in}}%
\pgfpathlineto{\pgfqpoint{0.732224in}{0.895308in}}%
\pgfpathclose%
\pgfusepath{fill}%
\end{pgfscope}%
\begin{pgfscope}%
\pgfpathrectangle{\pgfqpoint{0.050000in}{0.050000in}}{\pgfqpoint{2.081932in}{2.081932in}}%
\pgfusepath{clip}%
\pgfsetbuttcap%
\pgfsetroundjoin%
\definecolor{currentfill}{rgb}{0.993248,0.906157,0.143936}%
\pgfsetfillcolor{currentfill}%
\pgfsetlinewidth{0.000000pt}%
\definecolor{currentstroke}{rgb}{0.000000,0.000000,0.000000}%
\pgfsetstrokecolor{currentstroke}%
\pgfsetdash{}{0pt}%
\pgfpathmoveto{\pgfqpoint{1.055718in}{0.976999in}}%
\pgfpathlineto{\pgfqpoint{1.057453in}{0.986618in}}%
\pgfpathlineto{\pgfqpoint{1.059214in}{0.995811in}}%
\pgfpathlineto{\pgfqpoint{1.060995in}{1.004541in}}%
\pgfpathlineto{\pgfqpoint{1.062787in}{1.012774in}}%
\pgfpathlineto{\pgfqpoint{1.064583in}{1.020476in}}%
\pgfpathlineto{\pgfqpoint{1.089892in}{1.018893in}}%
\pgfpathlineto{\pgfqpoint{1.115328in}{1.018269in}}%
\pgfpathlineto{\pgfqpoint{1.140780in}{1.018607in}}%
\pgfpathlineto{\pgfqpoint{1.166139in}{1.019904in}}%
\pgfpathlineto{\pgfqpoint{1.191292in}{1.022154in}}%
\pgfpathlineto{\pgfqpoint{1.193665in}{1.014509in}}%
\pgfpathlineto{\pgfqpoint{1.196033in}{1.006333in}}%
\pgfpathlineto{\pgfqpoint{1.198385in}{0.997660in}}%
\pgfpathlineto{\pgfqpoint{1.200712in}{0.988523in}}%
\pgfpathlineto{\pgfqpoint{1.203004in}{0.978960in}}%
\pgfpathlineto{\pgfqpoint{1.173769in}{0.976330in}}%
\pgfpathlineto{\pgfqpoint{1.144292in}{0.974815in}}%
\pgfpathlineto{\pgfqpoint{1.114704in}{0.974421in}}%
\pgfpathlineto{\pgfqpoint{1.085136in}{0.975150in}}%
\pgfpathlineto{\pgfqpoint{1.055718in}{0.976999in}}%
\pgfpathclose%
\pgfusepath{fill}%
\end{pgfscope}%
\begin{pgfscope}%
\pgfpathrectangle{\pgfqpoint{0.050000in}{0.050000in}}{\pgfqpoint{2.081932in}{2.081932in}}%
\pgfusepath{clip}%
\pgfsetbuttcap%
\pgfsetroundjoin%
\definecolor{currentfill}{rgb}{0.150476,0.504369,0.557430}%
\pgfsetfillcolor{currentfill}%
\pgfsetlinewidth{0.000000pt}%
\definecolor{currentstroke}{rgb}{0.000000,0.000000,0.000000}%
\pgfsetstrokecolor{currentstroke}%
\pgfsetdash{}{0pt}%
\pgfpathmoveto{\pgfqpoint{1.227397in}{0.737220in}}%
\pgfpathlineto{\pgfqpoint{1.227988in}{0.743678in}}%
\pgfpathlineto{\pgfqpoint{1.228429in}{0.750732in}}%
\pgfpathlineto{\pgfqpoint{1.228719in}{0.758353in}}%
\pgfpathlineto{\pgfqpoint{1.228857in}{0.766510in}}%
\pgfpathlineto{\pgfqpoint{1.266602in}{0.771516in}}%
\pgfpathlineto{\pgfqpoint{1.303677in}{0.777974in}}%
\pgfpathlineto{\pgfqpoint{1.339917in}{0.785850in}}%
\pgfpathlineto{\pgfqpoint{1.375160in}{0.795106in}}%
\pgfpathlineto{\pgfqpoint{1.409250in}{0.805696in}}%
\pgfpathlineto{\pgfqpoint{1.408888in}{0.797527in}}%
\pgfpathlineto{\pgfqpoint{1.408123in}{0.789838in}}%
\pgfpathlineto{\pgfqpoint{1.406961in}{0.782659in}}%
\pgfpathlineto{\pgfqpoint{1.405407in}{0.776020in}}%
\pgfpathlineto{\pgfqpoint{1.371764in}{0.765534in}}%
\pgfpathlineto{\pgfqpoint{1.336985in}{0.756369in}}%
\pgfpathlineto{\pgfqpoint{1.301225in}{0.748570in}}%
\pgfpathlineto{\pgfqpoint{1.264641in}{0.742177in}}%
\pgfpathlineto{\pgfqpoint{1.227397in}{0.737220in}}%
\pgfpathclose%
\pgfusepath{fill}%
\end{pgfscope}%
\begin{pgfscope}%
\pgfpathrectangle{\pgfqpoint{0.050000in}{0.050000in}}{\pgfqpoint{2.081932in}{2.081932in}}%
\pgfusepath{clip}%
\pgfsetbuttcap%
\pgfsetroundjoin%
\definecolor{currentfill}{rgb}{0.150476,0.504369,0.557430}%
\pgfsetfillcolor{currentfill}%
\pgfsetlinewidth{0.000000pt}%
\definecolor{currentstroke}{rgb}{0.000000,0.000000,0.000000}%
\pgfsetstrokecolor{currentstroke}%
\pgfsetdash{}{0pt}%
\pgfpathmoveto{\pgfqpoint{0.856297in}{0.768527in}}%
\pgfpathlineto{\pgfqpoint{0.854868in}{0.775132in}}%
\pgfpathlineto{\pgfqpoint{0.853800in}{0.782286in}}%
\pgfpathlineto{\pgfqpoint{0.853098in}{0.789963in}}%
\pgfpathlineto{\pgfqpoint{0.852765in}{0.798129in}}%
\pgfpathlineto{\pgfqpoint{0.887677in}{0.788470in}}%
\pgfpathlineto{\pgfqpoint{0.923633in}{0.780177in}}%
\pgfpathlineto{\pgfqpoint{0.960474in}{0.773293in}}%
\pgfpathlineto{\pgfqpoint{0.998035in}{0.767851in}}%
\pgfpathlineto{\pgfqpoint{1.036146in}{0.763879in}}%
\pgfpathlineto{\pgfqpoint{1.036250in}{0.755722in}}%
\pgfpathlineto{\pgfqpoint{1.036470in}{0.748106in}}%
\pgfpathlineto{\pgfqpoint{1.036804in}{0.741061in}}%
\pgfpathlineto{\pgfqpoint{1.037251in}{0.734615in}}%
\pgfpathlineto{\pgfqpoint{0.999646in}{0.738547in}}%
\pgfpathlineto{\pgfqpoint{0.962584in}{0.743936in}}%
\pgfpathlineto{\pgfqpoint{0.926231in}{0.750752in}}%
\pgfpathlineto{\pgfqpoint{0.890749in}{0.758963in}}%
\pgfpathlineto{\pgfqpoint{0.856297in}{0.768527in}}%
\pgfpathclose%
\pgfusepath{fill}%
\end{pgfscope}%
\begin{pgfscope}%
\pgfpathrectangle{\pgfqpoint{0.050000in}{0.050000in}}{\pgfqpoint{2.081932in}{2.081932in}}%
\pgfusepath{clip}%
\pgfsetbuttcap%
\pgfsetroundjoin%
\definecolor{currentfill}{rgb}{0.876168,0.891125,0.095250}%
\pgfsetfillcolor{currentfill}%
\pgfsetlinewidth{0.000000pt}%
\definecolor{currentstroke}{rgb}{0.000000,0.000000,0.000000}%
\pgfsetstrokecolor{currentstroke}%
\pgfsetdash{}{0pt}%
\pgfpathmoveto{\pgfqpoint{1.213606in}{0.926181in}}%
\pgfpathlineto{\pgfqpoint{1.211631in}{0.937255in}}%
\pgfpathlineto{\pgfqpoint{1.209574in}{0.948114in}}%
\pgfpathlineto{\pgfqpoint{1.207444in}{0.958714in}}%
\pgfpathlineto{\pgfqpoint{1.205251in}{0.969010in}}%
\pgfpathlineto{\pgfqpoint{1.203004in}{0.978960in}}%
\pgfpathlineto{\pgfqpoint{1.231868in}{0.982691in}}%
\pgfpathlineto{\pgfqpoint{1.260234in}{0.987506in}}%
\pgfpathlineto{\pgfqpoint{1.287978in}{0.993381in}}%
\pgfpathlineto{\pgfqpoint{1.314979in}{1.000289in}}%
\pgfpathlineto{\pgfqpoint{1.341120in}{1.008196in}}%
\pgfpathlineto{\pgfqpoint{1.347048in}{0.999062in}}%
\pgfpathlineto{\pgfqpoint{1.352832in}{0.989567in}}%
\pgfpathlineto{\pgfqpoint{1.358447in}{0.979749in}}%
\pgfpathlineto{\pgfqpoint{1.363870in}{0.969648in}}%
\pgfpathlineto{\pgfqpoint{1.369078in}{0.959308in}}%
\pgfpathlineto{\pgfqpoint{1.339670in}{0.950351in}}%
\pgfpathlineto{\pgfqpoint{1.309285in}{0.942525in}}%
\pgfpathlineto{\pgfqpoint{1.278053in}{0.935868in}}%
\pgfpathlineto{\pgfqpoint{1.246113in}{0.930411in}}%
\pgfpathlineto{\pgfqpoint{1.213606in}{0.926181in}}%
\pgfpathclose%
\pgfusepath{fill}%
\end{pgfscope}%
\begin{pgfscope}%
\pgfpathrectangle{\pgfqpoint{0.050000in}{0.050000in}}{\pgfqpoint{2.081932in}{2.081932in}}%
\pgfusepath{clip}%
\pgfsetbuttcap%
\pgfsetroundjoin%
\definecolor{currentfill}{rgb}{0.876168,0.891125,0.095250}%
\pgfsetfillcolor{currentfill}%
\pgfsetlinewidth{0.000000pt}%
\definecolor{currentstroke}{rgb}{0.000000,0.000000,0.000000}%
\pgfsetstrokecolor{currentstroke}%
\pgfsetdash{}{0pt}%
\pgfpathmoveto{\pgfqpoint{0.889686in}{0.952908in}}%
\pgfpathlineto{\pgfqpoint{0.894472in}{0.963390in}}%
\pgfpathlineto{\pgfqpoint{0.899455in}{0.973636in}}%
\pgfpathlineto{\pgfqpoint{0.904615in}{0.983605in}}%
\pgfpathlineto{\pgfqpoint{0.909929in}{0.993255in}}%
\pgfpathlineto{\pgfqpoint{0.915376in}{1.002545in}}%
\pgfpathlineto{\pgfqpoint{0.942130in}{0.995336in}}%
\pgfpathlineto{\pgfqpoint{0.969662in}{0.989150in}}%
\pgfpathlineto{\pgfqpoint{0.997854in}{0.984016in}}%
\pgfpathlineto{\pgfqpoint{1.026581in}{0.979959in}}%
\pgfpathlineto{\pgfqpoint{1.055718in}{0.976999in}}%
\pgfpathlineto{\pgfqpoint{1.054017in}{0.966994in}}%
\pgfpathlineto{\pgfqpoint{1.052357in}{0.956644in}}%
\pgfpathlineto{\pgfqpoint{1.050745in}{0.945992in}}%
\pgfpathlineto{\pgfqpoint{1.049188in}{0.935081in}}%
\pgfpathlineto{\pgfqpoint{1.047692in}{0.923958in}}%
\pgfpathlineto{\pgfqpoint{1.014874in}{0.927313in}}%
\pgfpathlineto{\pgfqpoint{0.982523in}{0.931912in}}%
\pgfpathlineto{\pgfqpoint{0.950782in}{0.937730in}}%
\pgfpathlineto{\pgfqpoint{0.919791in}{0.944740in}}%
\pgfpathlineto{\pgfqpoint{0.889686in}{0.952908in}}%
\pgfpathclose%
\pgfusepath{fill}%
\end{pgfscope}%
\begin{pgfscope}%
\pgfpathrectangle{\pgfqpoint{0.050000in}{0.050000in}}{\pgfqpoint{2.081932in}{2.081932in}}%
\pgfusepath{clip}%
\pgfsetbuttcap%
\pgfsetroundjoin%
\definecolor{currentfill}{rgb}{0.124780,0.640461,0.527068}%
\pgfsetfillcolor{currentfill}%
\pgfsetlinewidth{0.000000pt}%
\definecolor{currentstroke}{rgb}{0.000000,0.000000,0.000000}%
\pgfsetstrokecolor{currentstroke}%
\pgfsetdash{}{0pt}%
\pgfpathmoveto{\pgfqpoint{1.228857in}{0.766510in}}%
\pgfpathlineto{\pgfqpoint{1.228841in}{0.775171in}}%
\pgfpathlineto{\pgfqpoint{1.228672in}{0.784299in}}%
\pgfpathlineto{\pgfqpoint{1.228349in}{0.793856in}}%
\pgfpathlineto{\pgfqpoint{1.227874in}{0.803804in}}%
\pgfpathlineto{\pgfqpoint{1.227250in}{0.814099in}}%
\pgfpathlineto{\pgfqpoint{1.264442in}{0.819004in}}%
\pgfpathlineto{\pgfqpoint{1.300977in}{0.825332in}}%
\pgfpathlineto{\pgfqpoint{1.336689in}{0.833050in}}%
\pgfpathlineto{\pgfqpoint{1.371420in}{0.842120in}}%
\pgfpathlineto{\pgfqpoint{1.405018in}{0.852498in}}%
\pgfpathlineto{\pgfqpoint{1.406663in}{0.842470in}}%
\pgfpathlineto{\pgfqpoint{1.407913in}{0.832736in}}%
\pgfpathlineto{\pgfqpoint{1.408762in}{0.823337in}}%
\pgfpathlineto{\pgfqpoint{1.409208in}{0.814311in}}%
\pgfpathlineto{\pgfqpoint{1.409250in}{0.805696in}}%
\pgfpathlineto{\pgfqpoint{1.375160in}{0.795106in}}%
\pgfpathlineto{\pgfqpoint{1.339917in}{0.785850in}}%
\pgfpathlineto{\pgfqpoint{1.303677in}{0.777974in}}%
\pgfpathlineto{\pgfqpoint{1.266602in}{0.771516in}}%
\pgfpathlineto{\pgfqpoint{1.228857in}{0.766510in}}%
\pgfpathclose%
\pgfusepath{fill}%
\end{pgfscope}%
\begin{pgfscope}%
\pgfpathrectangle{\pgfqpoint{0.050000in}{0.050000in}}{\pgfqpoint{2.081932in}{2.081932in}}%
\pgfusepath{clip}%
\pgfsetbuttcap%
\pgfsetroundjoin%
\definecolor{currentfill}{rgb}{0.150476,0.504369,0.557430}%
\pgfsetfillcolor{currentfill}%
\pgfsetlinewidth{0.000000pt}%
\definecolor{currentstroke}{rgb}{0.000000,0.000000,0.000000}%
\pgfsetstrokecolor{currentstroke}%
\pgfsetdash{}{0pt}%
\pgfpathmoveto{\pgfqpoint{1.037251in}{0.734615in}}%
\pgfpathlineto{\pgfqpoint{1.036804in}{0.741061in}}%
\pgfpathlineto{\pgfqpoint{1.036470in}{0.748106in}}%
\pgfpathlineto{\pgfqpoint{1.036250in}{0.755722in}}%
\pgfpathlineto{\pgfqpoint{1.036146in}{0.763879in}}%
\pgfpathlineto{\pgfqpoint{1.074635in}{0.761397in}}%
\pgfpathlineto{\pgfqpoint{1.113327in}{0.760418in}}%
\pgfpathlineto{\pgfqpoint{1.152045in}{0.760947in}}%
\pgfpathlineto{\pgfqpoint{1.190614in}{0.762981in}}%
\pgfpathlineto{\pgfqpoint{1.228857in}{0.766510in}}%
\pgfpathlineto{\pgfqpoint{1.228719in}{0.758353in}}%
\pgfpathlineto{\pgfqpoint{1.228429in}{0.750732in}}%
\pgfpathlineto{\pgfqpoint{1.227988in}{0.743678in}}%
\pgfpathlineto{\pgfqpoint{1.227397in}{0.737220in}}%
\pgfpathlineto{\pgfqpoint{1.189662in}{0.733726in}}%
\pgfpathlineto{\pgfqpoint{1.151607in}{0.731712in}}%
\pgfpathlineto{\pgfqpoint{1.113404in}{0.731189in}}%
\pgfpathlineto{\pgfqpoint{1.075228in}{0.732158in}}%
\pgfpathlineto{\pgfqpoint{1.037251in}{0.734615in}}%
\pgfpathclose%
\pgfusepath{fill}%
\end{pgfscope}%
\begin{pgfscope}%
\pgfpathrectangle{\pgfqpoint{0.050000in}{0.050000in}}{\pgfqpoint{2.081932in}{2.081932in}}%
\pgfusepath{clip}%
\pgfsetbuttcap%
\pgfsetroundjoin%
\definecolor{currentfill}{rgb}{0.124780,0.640461,0.527068}%
\pgfsetfillcolor{currentfill}%
\pgfsetlinewidth{0.000000pt}%
\definecolor{currentstroke}{rgb}{0.000000,0.000000,0.000000}%
\pgfsetstrokecolor{currentstroke}%
\pgfsetdash{}{0pt}%
\pgfpathmoveto{\pgfqpoint{0.852765in}{0.798129in}}%
\pgfpathlineto{\pgfqpoint{0.852803in}{0.806753in}}%
\pgfpathlineto{\pgfqpoint{0.853214in}{0.815798in}}%
\pgfpathlineto{\pgfqpoint{0.853994in}{0.825228in}}%
\pgfpathlineto{\pgfqpoint{0.855143in}{0.835003in}}%
\pgfpathlineto{\pgfqpoint{0.856655in}{0.845083in}}%
\pgfpathlineto{\pgfqpoint{0.891061in}{0.835617in}}%
\pgfpathlineto{\pgfqpoint{0.926494in}{0.827491in}}%
\pgfpathlineto{\pgfqpoint{0.962798in}{0.820745in}}%
\pgfpathlineto{\pgfqpoint{0.999810in}{0.815413in}}%
\pgfpathlineto{\pgfqpoint{1.037363in}{0.811521in}}%
\pgfpathlineto{\pgfqpoint{1.036890in}{0.801207in}}%
\pgfpathlineto{\pgfqpoint{1.036531in}{0.791245in}}%
\pgfpathlineto{\pgfqpoint{1.036286in}{0.781677in}}%
\pgfpathlineto{\pgfqpoint{1.036158in}{0.772542in}}%
\pgfpathlineto{\pgfqpoint{1.036146in}{0.763879in}}%
\pgfpathlineto{\pgfqpoint{0.998035in}{0.767851in}}%
\pgfpathlineto{\pgfqpoint{0.960474in}{0.773293in}}%
\pgfpathlineto{\pgfqpoint{0.923633in}{0.780177in}}%
\pgfpathlineto{\pgfqpoint{0.887677in}{0.788470in}}%
\pgfpathlineto{\pgfqpoint{0.852765in}{0.798129in}}%
\pgfpathclose%
\pgfusepath{fill}%
\end{pgfscope}%
\begin{pgfscope}%
\pgfpathrectangle{\pgfqpoint{0.050000in}{0.050000in}}{\pgfqpoint{2.081932in}{2.081932in}}%
\pgfusepath{clip}%
\pgfsetbuttcap%
\pgfsetroundjoin%
\definecolor{currentfill}{rgb}{0.636902,0.856542,0.216620}%
\pgfsetfillcolor{currentfill}%
\pgfsetlinewidth{0.000000pt}%
\definecolor{currentstroke}{rgb}{0.000000,0.000000,0.000000}%
\pgfsetstrokecolor{currentstroke}%
\pgfsetdash{}{0pt}%
\pgfpathmoveto{\pgfqpoint{1.221983in}{0.869245in}}%
\pgfpathlineto{\pgfqpoint{1.220533in}{0.880681in}}%
\pgfpathlineto{\pgfqpoint{1.218963in}{0.892141in}}%
\pgfpathlineto{\pgfqpoint{1.217280in}{0.903577in}}%
\pgfpathlineto{\pgfqpoint{1.215492in}{0.914939in}}%
\pgfpathlineto{\pgfqpoint{1.213606in}{0.926181in}}%
\pgfpathlineto{\pgfqpoint{1.246113in}{0.930411in}}%
\pgfpathlineto{\pgfqpoint{1.278053in}{0.935868in}}%
\pgfpathlineto{\pgfqpoint{1.309285in}{0.942525in}}%
\pgfpathlineto{\pgfqpoint{1.339670in}{0.950351in}}%
\pgfpathlineto{\pgfqpoint{1.369078in}{0.959308in}}%
\pgfpathlineto{\pgfqpoint{1.374048in}{0.948770in}}%
\pgfpathlineto{\pgfqpoint{1.378760in}{0.938080in}}%
\pgfpathlineto{\pgfqpoint{1.383193in}{0.927282in}}%
\pgfpathlineto{\pgfqpoint{1.387329in}{0.916421in}}%
\pgfpathlineto{\pgfqpoint{1.391150in}{0.905543in}}%
\pgfpathlineto{\pgfqpoint{1.359168in}{0.895732in}}%
\pgfpathlineto{\pgfqpoint{1.326113in}{0.887157in}}%
\pgfpathlineto{\pgfqpoint{1.292129in}{0.879862in}}%
\pgfpathlineto{\pgfqpoint{1.257368in}{0.873881in}}%
\pgfpathlineto{\pgfqpoint{1.221983in}{0.869245in}}%
\pgfpathclose%
\pgfusepath{fill}%
\end{pgfscope}%
\begin{pgfscope}%
\pgfpathrectangle{\pgfqpoint{0.050000in}{0.050000in}}{\pgfqpoint{2.081932in}{2.081932in}}%
\pgfusepath{clip}%
\pgfsetbuttcap%
\pgfsetroundjoin%
\definecolor{currentfill}{rgb}{0.876168,0.891125,0.095250}%
\pgfsetfillcolor{currentfill}%
\pgfsetlinewidth{0.000000pt}%
\definecolor{currentstroke}{rgb}{0.000000,0.000000,0.000000}%
\pgfsetstrokecolor{currentstroke}%
\pgfsetdash{}{0pt}%
\pgfpathmoveto{\pgfqpoint{1.047692in}{0.923958in}}%
\pgfpathlineto{\pgfqpoint{1.049188in}{0.935081in}}%
\pgfpathlineto{\pgfqpoint{1.050745in}{0.945992in}}%
\pgfpathlineto{\pgfqpoint{1.052357in}{0.956644in}}%
\pgfpathlineto{\pgfqpoint{1.054017in}{0.966994in}}%
\pgfpathlineto{\pgfqpoint{1.055718in}{0.976999in}}%
\pgfpathlineto{\pgfqpoint{1.085136in}{0.975150in}}%
\pgfpathlineto{\pgfqpoint{1.114704in}{0.974421in}}%
\pgfpathlineto{\pgfqpoint{1.144292in}{0.974815in}}%
\pgfpathlineto{\pgfqpoint{1.173769in}{0.976330in}}%
\pgfpathlineto{\pgfqpoint{1.203004in}{0.978960in}}%
\pgfpathlineto{\pgfqpoint{1.205251in}{0.969010in}}%
\pgfpathlineto{\pgfqpoint{1.207444in}{0.958714in}}%
\pgfpathlineto{\pgfqpoint{1.209574in}{0.948114in}}%
\pgfpathlineto{\pgfqpoint{1.211631in}{0.937255in}}%
\pgfpathlineto{\pgfqpoint{1.213606in}{0.926181in}}%
\pgfpathlineto{\pgfqpoint{1.180677in}{0.923200in}}%
\pgfpathlineto{\pgfqpoint{1.147472in}{0.921481in}}%
\pgfpathlineto{\pgfqpoint{1.114139in}{0.921035in}}%
\pgfpathlineto{\pgfqpoint{1.080830in}{0.921861in}}%
\pgfpathlineto{\pgfqpoint{1.047692in}{0.923958in}}%
\pgfpathclose%
\pgfusepath{fill}%
\end{pgfscope}%
\begin{pgfscope}%
\pgfpathrectangle{\pgfqpoint{0.050000in}{0.050000in}}{\pgfqpoint{2.081932in}{2.081932in}}%
\pgfusepath{clip}%
\pgfsetbuttcap%
\pgfsetroundjoin%
\definecolor{currentfill}{rgb}{0.327796,0.773980,0.406640}%
\pgfsetfillcolor{currentfill}%
\pgfsetlinewidth{0.000000pt}%
\definecolor{currentstroke}{rgb}{0.000000,0.000000,0.000000}%
\pgfsetstrokecolor{currentstroke}%
\pgfsetdash{}{0pt}%
\pgfpathmoveto{\pgfqpoint{1.227250in}{0.814099in}}%
\pgfpathlineto{\pgfqpoint{1.226477in}{0.824700in}}%
\pgfpathlineto{\pgfqpoint{1.225560in}{0.835561in}}%
\pgfpathlineto{\pgfqpoint{1.224502in}{0.846637in}}%
\pgfpathlineto{\pgfqpoint{1.223308in}{0.857881in}}%
\pgfpathlineto{\pgfqpoint{1.221983in}{0.869245in}}%
\pgfpathlineto{\pgfqpoint{1.257368in}{0.873881in}}%
\pgfpathlineto{\pgfqpoint{1.292129in}{0.879862in}}%
\pgfpathlineto{\pgfqpoint{1.326113in}{0.887157in}}%
\pgfpathlineto{\pgfqpoint{1.359168in}{0.895732in}}%
\pgfpathlineto{\pgfqpoint{1.391150in}{0.905543in}}%
\pgfpathlineto{\pgfqpoint{1.394640in}{0.894694in}}%
\pgfpathlineto{\pgfqpoint{1.397784in}{0.883920in}}%
\pgfpathlineto{\pgfqpoint{1.400569in}{0.873266in}}%
\pgfpathlineto{\pgfqpoint{1.402984in}{0.862778in}}%
\pgfpathlineto{\pgfqpoint{1.405018in}{0.852498in}}%
\pgfpathlineto{\pgfqpoint{1.371420in}{0.842120in}}%
\pgfpathlineto{\pgfqpoint{1.336689in}{0.833050in}}%
\pgfpathlineto{\pgfqpoint{1.300977in}{0.825332in}}%
\pgfpathlineto{\pgfqpoint{1.264442in}{0.819004in}}%
\pgfpathlineto{\pgfqpoint{1.227250in}{0.814099in}}%
\pgfpathclose%
\pgfusepath{fill}%
\end{pgfscope}%
\begin{pgfscope}%
\pgfpathrectangle{\pgfqpoint{0.050000in}{0.050000in}}{\pgfqpoint{2.081932in}{2.081932in}}%
\pgfusepath{clip}%
\pgfsetbuttcap%
\pgfsetroundjoin%
\definecolor{currentfill}{rgb}{0.636902,0.856542,0.216620}%
\pgfsetfillcolor{currentfill}%
\pgfsetlinewidth{0.000000pt}%
\definecolor{currentstroke}{rgb}{0.000000,0.000000,0.000000}%
\pgfsetstrokecolor{currentstroke}%
\pgfsetdash{}{0pt}%
\pgfpathmoveto{\pgfqpoint{0.869401in}{0.898532in}}%
\pgfpathlineto{\pgfqpoint{0.872913in}{0.909518in}}%
\pgfpathlineto{\pgfqpoint{0.876714in}{0.920494in}}%
\pgfpathlineto{\pgfqpoint{0.880789in}{0.931415in}}%
\pgfpathlineto{\pgfqpoint{0.885119in}{0.942235in}}%
\pgfpathlineto{\pgfqpoint{0.889686in}{0.952908in}}%
\pgfpathlineto{\pgfqpoint{0.919791in}{0.944740in}}%
\pgfpathlineto{\pgfqpoint{0.950782in}{0.937730in}}%
\pgfpathlineto{\pgfqpoint{0.982523in}{0.931912in}}%
\pgfpathlineto{\pgfqpoint{1.014874in}{0.927313in}}%
\pgfpathlineto{\pgfqpoint{1.047692in}{0.923958in}}%
\pgfpathlineto{\pgfqpoint{1.046264in}{0.912669in}}%
\pgfpathlineto{\pgfqpoint{1.044911in}{0.901261in}}%
\pgfpathlineto{\pgfqpoint{1.043637in}{0.889782in}}%
\pgfpathlineto{\pgfqpoint{1.042449in}{0.878282in}}%
\pgfpathlineto{\pgfqpoint{1.041350in}{0.866808in}}%
\pgfpathlineto{\pgfqpoint{1.005625in}{0.870486in}}%
\pgfpathlineto{\pgfqpoint{0.970411in}{0.875526in}}%
\pgfpathlineto{\pgfqpoint{0.935868in}{0.881903in}}%
\pgfpathlineto{\pgfqpoint{0.902148in}{0.889584in}}%
\pgfpathlineto{\pgfqpoint{0.869401in}{0.898532in}}%
\pgfpathclose%
\pgfusepath{fill}%
\end{pgfscope}%
\begin{pgfscope}%
\pgfpathrectangle{\pgfqpoint{0.050000in}{0.050000in}}{\pgfqpoint{2.081932in}{2.081932in}}%
\pgfusepath{clip}%
\pgfsetbuttcap%
\pgfsetroundjoin%
\definecolor{currentfill}{rgb}{0.327796,0.773980,0.406640}%
\pgfsetfillcolor{currentfill}%
\pgfsetlinewidth{0.000000pt}%
\definecolor{currentstroke}{rgb}{0.000000,0.000000,0.000000}%
\pgfsetstrokecolor{currentstroke}%
\pgfsetdash{}{0pt}%
\pgfpathmoveto{\pgfqpoint{0.856655in}{0.845083in}}%
\pgfpathlineto{\pgfqpoint{0.858525in}{0.855424in}}%
\pgfpathlineto{\pgfqpoint{0.860744in}{0.865985in}}%
\pgfpathlineto{\pgfqpoint{0.863304in}{0.876720in}}%
\pgfpathlineto{\pgfqpoint{0.866194in}{0.887585in}}%
\pgfpathlineto{\pgfqpoint{0.869401in}{0.898532in}}%
\pgfpathlineto{\pgfqpoint{0.902148in}{0.889584in}}%
\pgfpathlineto{\pgfqpoint{0.935868in}{0.881903in}}%
\pgfpathlineto{\pgfqpoint{0.970411in}{0.875526in}}%
\pgfpathlineto{\pgfqpoint{1.005625in}{0.870486in}}%
\pgfpathlineto{\pgfqpoint{1.041350in}{0.866808in}}%
\pgfpathlineto{\pgfqpoint{1.040347in}{0.855409in}}%
\pgfpathlineto{\pgfqpoint{1.039443in}{0.844134in}}%
\pgfpathlineto{\pgfqpoint{1.038643in}{0.833030in}}%
\pgfpathlineto{\pgfqpoint{1.037948in}{0.822143in}}%
\pgfpathlineto{\pgfqpoint{1.037363in}{0.811521in}}%
\pgfpathlineto{\pgfqpoint{0.999810in}{0.815413in}}%
\pgfpathlineto{\pgfqpoint{0.962798in}{0.820745in}}%
\pgfpathlineto{\pgfqpoint{0.926494in}{0.827491in}}%
\pgfpathlineto{\pgfqpoint{0.891061in}{0.835617in}}%
\pgfpathlineto{\pgfqpoint{0.856655in}{0.845083in}}%
\pgfpathclose%
\pgfusepath{fill}%
\end{pgfscope}%
\begin{pgfscope}%
\pgfpathrectangle{\pgfqpoint{0.050000in}{0.050000in}}{\pgfqpoint{2.081932in}{2.081932in}}%
\pgfusepath{clip}%
\pgfsetbuttcap%
\pgfsetroundjoin%
\definecolor{currentfill}{rgb}{0.124780,0.640461,0.527068}%
\pgfsetfillcolor{currentfill}%
\pgfsetlinewidth{0.000000pt}%
\definecolor{currentstroke}{rgb}{0.000000,0.000000,0.000000}%
\pgfsetstrokecolor{currentstroke}%
\pgfsetdash{}{0pt}%
\pgfpathmoveto{\pgfqpoint{1.036146in}{0.763879in}}%
\pgfpathlineto{\pgfqpoint{1.036158in}{0.772542in}}%
\pgfpathlineto{\pgfqpoint{1.036286in}{0.781677in}}%
\pgfpathlineto{\pgfqpoint{1.036531in}{0.791245in}}%
\pgfpathlineto{\pgfqpoint{1.036890in}{0.801207in}}%
\pgfpathlineto{\pgfqpoint{1.037363in}{0.811521in}}%
\pgfpathlineto{\pgfqpoint{1.075288in}{0.809089in}}%
\pgfpathlineto{\pgfqpoint{1.113413in}{0.808130in}}%
\pgfpathlineto{\pgfqpoint{1.151563in}{0.808648in}}%
\pgfpathlineto{\pgfqpoint{1.189566in}{0.810641in}}%
\pgfpathlineto{\pgfqpoint{1.227250in}{0.814099in}}%
\pgfpathlineto{\pgfqpoint{1.227874in}{0.803804in}}%
\pgfpathlineto{\pgfqpoint{1.228349in}{0.793856in}}%
\pgfpathlineto{\pgfqpoint{1.228672in}{0.784299in}}%
\pgfpathlineto{\pgfqpoint{1.228841in}{0.775171in}}%
\pgfpathlineto{\pgfqpoint{1.228857in}{0.766510in}}%
\pgfpathlineto{\pgfqpoint{1.190614in}{0.762981in}}%
\pgfpathlineto{\pgfqpoint{1.152045in}{0.760947in}}%
\pgfpathlineto{\pgfqpoint{1.113327in}{0.760418in}}%
\pgfpathlineto{\pgfqpoint{1.074635in}{0.761397in}}%
\pgfpathlineto{\pgfqpoint{1.036146in}{0.763879in}}%
\pgfpathclose%
\pgfusepath{fill}%
\end{pgfscope}%
\begin{pgfscope}%
\pgfpathrectangle{\pgfqpoint{0.050000in}{0.050000in}}{\pgfqpoint{2.081932in}{2.081932in}}%
\pgfusepath{clip}%
\pgfsetbuttcap%
\pgfsetroundjoin%
\definecolor{currentfill}{rgb}{0.636902,0.856542,0.216620}%
\pgfsetfillcolor{currentfill}%
\pgfsetlinewidth{0.000000pt}%
\definecolor{currentstroke}{rgb}{0.000000,0.000000,0.000000}%
\pgfsetstrokecolor{currentstroke}%
\pgfsetdash{}{0pt}%
\pgfpathmoveto{\pgfqpoint{1.041350in}{0.866808in}}%
\pgfpathlineto{\pgfqpoint{1.042449in}{0.878282in}}%
\pgfpathlineto{\pgfqpoint{1.043637in}{0.889782in}}%
\pgfpathlineto{\pgfqpoint{1.044911in}{0.901261in}}%
\pgfpathlineto{\pgfqpoint{1.046264in}{0.912669in}}%
\pgfpathlineto{\pgfqpoint{1.047692in}{0.923958in}}%
\pgfpathlineto{\pgfqpoint{1.080830in}{0.921861in}}%
\pgfpathlineto{\pgfqpoint{1.114139in}{0.921035in}}%
\pgfpathlineto{\pgfqpoint{1.147472in}{0.921481in}}%
\pgfpathlineto{\pgfqpoint{1.180677in}{0.923200in}}%
\pgfpathlineto{\pgfqpoint{1.213606in}{0.926181in}}%
\pgfpathlineto{\pgfqpoint{1.215492in}{0.914939in}}%
\pgfpathlineto{\pgfqpoint{1.217280in}{0.903577in}}%
\pgfpathlineto{\pgfqpoint{1.218963in}{0.892141in}}%
\pgfpathlineto{\pgfqpoint{1.220533in}{0.880681in}}%
\pgfpathlineto{\pgfqpoint{1.221983in}{0.869245in}}%
\pgfpathlineto{\pgfqpoint{1.186135in}{0.865977in}}%
\pgfpathlineto{\pgfqpoint{1.149984in}{0.864094in}}%
\pgfpathlineto{\pgfqpoint{1.113693in}{0.863604in}}%
\pgfpathlineto{\pgfqpoint{1.077427in}{0.864510in}}%
\pgfpathlineto{\pgfqpoint{1.041350in}{0.866808in}}%
\pgfpathclose%
\pgfusepath{fill}%
\end{pgfscope}%
\begin{pgfscope}%
\pgfpathrectangle{\pgfqpoint{0.050000in}{0.050000in}}{\pgfqpoint{2.081932in}{2.081932in}}%
\pgfusepath{clip}%
\pgfsetbuttcap%
\pgfsetroundjoin%
\definecolor{currentfill}{rgb}{0.327796,0.773980,0.406640}%
\pgfsetfillcolor{currentfill}%
\pgfsetlinewidth{0.000000pt}%
\definecolor{currentstroke}{rgb}{0.000000,0.000000,0.000000}%
\pgfsetstrokecolor{currentstroke}%
\pgfsetdash{}{0pt}%
\pgfpathmoveto{\pgfqpoint{1.037363in}{0.811521in}}%
\pgfpathlineto{\pgfqpoint{1.037948in}{0.822143in}}%
\pgfpathlineto{\pgfqpoint{1.038643in}{0.833030in}}%
\pgfpathlineto{\pgfqpoint{1.039443in}{0.844134in}}%
\pgfpathlineto{\pgfqpoint{1.040347in}{0.855409in}}%
\pgfpathlineto{\pgfqpoint{1.041350in}{0.866808in}}%
\pgfpathlineto{\pgfqpoint{1.077427in}{0.864510in}}%
\pgfpathlineto{\pgfqpoint{1.113693in}{0.863604in}}%
\pgfpathlineto{\pgfqpoint{1.149984in}{0.864094in}}%
\pgfpathlineto{\pgfqpoint{1.186135in}{0.865977in}}%
\pgfpathlineto{\pgfqpoint{1.221983in}{0.869245in}}%
\pgfpathlineto{\pgfqpoint{1.223308in}{0.857881in}}%
\pgfpathlineto{\pgfqpoint{1.224502in}{0.846637in}}%
\pgfpathlineto{\pgfqpoint{1.225560in}{0.835561in}}%
\pgfpathlineto{\pgfqpoint{1.226477in}{0.824700in}}%
\pgfpathlineto{\pgfqpoint{1.227250in}{0.814099in}}%
\pgfpathlineto{\pgfqpoint{1.189566in}{0.810641in}}%
\pgfpathlineto{\pgfqpoint{1.151563in}{0.808648in}}%
\pgfpathlineto{\pgfqpoint{1.113413in}{0.808130in}}%
\pgfpathlineto{\pgfqpoint{1.075288in}{0.809089in}}%
\pgfpathlineto{\pgfqpoint{1.037363in}{0.811521in}}%
\pgfpathclose%
\pgfusepath{fill}%
\end{pgfscope}%
\begin{pgfscope}%
\pgfsetbuttcap%
\pgfsetmiterjoin%
\definecolor{currentfill}{rgb}{1.000000,1.000000,1.000000}%
\pgfsetfillcolor{currentfill}%
\pgfsetlinewidth{0.000000pt}%
\definecolor{currentstroke}{rgb}{0.000000,0.000000,0.000000}%
\pgfsetstrokecolor{currentstroke}%
\pgfsetstrokeopacity{0.000000}%
\pgfsetdash{}{0pt}%
\pgfpathmoveto{\pgfqpoint{2.548318in}{0.050000in}}%
\pgfpathlineto{\pgfqpoint{4.630250in}{0.050000in}}%
\pgfpathlineto{\pgfqpoint{4.630250in}{2.131932in}}%
\pgfpathlineto{\pgfqpoint{2.548318in}{2.131932in}}%
\pgfpathlineto{\pgfqpoint{2.548318in}{0.050000in}}%
\pgfpathclose%
\pgfusepath{fill}%
\end{pgfscope}%
\begin{pgfscope}%
\pgfpathrectangle{\pgfqpoint{2.548318in}{0.050000in}}{\pgfqpoint{2.081932in}{2.081932in}}%
\pgfusepath{clip}%
\pgfsetbuttcap%
\pgfsetroundjoin%
\definecolor{currentfill}{rgb}{0.327796,0.773980,0.406640}%
\pgfsetfillcolor{currentfill}%
\pgfsetlinewidth{0.000000pt}%
\definecolor{currentstroke}{rgb}{0.000000,0.000000,0.000000}%
\pgfsetstrokecolor{currentstroke}%
\pgfsetdash{}{0pt}%
\pgfpathmoveto{\pgfqpoint{3.682801in}{0.887714in}}%
\pgfpathlineto{\pgfqpoint{3.682315in}{0.897678in}}%
\pgfpathlineto{\pgfqpoint{3.681749in}{0.907774in}}%
\pgfpathlineto{\pgfqpoint{3.681108in}{0.917963in}}%
\pgfpathlineto{\pgfqpoint{3.680392in}{0.928205in}}%
\pgfpathlineto{\pgfqpoint{3.679606in}{0.938461in}}%
\pgfpathlineto{\pgfqpoint{3.646557in}{0.937057in}}%
\pgfpathlineto{\pgfqpoint{3.613405in}{0.936674in}}%
\pgfpathlineto{\pgfqpoint{3.580268in}{0.937314in}}%
\pgfpathlineto{\pgfqpoint{3.547263in}{0.938975in}}%
\pgfpathlineto{\pgfqpoint{3.514506in}{0.941651in}}%
\pgfpathlineto{\pgfqpoint{3.513201in}{0.931431in}}%
\pgfpathlineto{\pgfqpoint{3.512013in}{0.921220in}}%
\pgfpathlineto{\pgfqpoint{3.510948in}{0.911059in}}%
\pgfpathlineto{\pgfqpoint{3.510010in}{0.900987in}}%
\pgfpathlineto{\pgfqpoint{3.509202in}{0.891042in}}%
\pgfpathlineto{\pgfqpoint{3.543646in}{0.888250in}}%
\pgfpathlineto{\pgfqpoint{3.578351in}{0.886518in}}%
\pgfpathlineto{\pgfqpoint{3.613193in}{0.885850in}}%
\pgfpathlineto{\pgfqpoint{3.648051in}{0.886249in}}%
\pgfpathlineto{\pgfqpoint{3.682801in}{0.887714in}}%
\pgfpathclose%
\pgfusepath{fill}%
\end{pgfscope}%
\begin{pgfscope}%
\pgfpathrectangle{\pgfqpoint{2.548318in}{0.050000in}}{\pgfqpoint{2.081932in}{2.081932in}}%
\pgfusepath{clip}%
\pgfsetbuttcap%
\pgfsetroundjoin%
\definecolor{currentfill}{rgb}{0.124780,0.640461,0.527068}%
\pgfsetfillcolor{currentfill}%
\pgfsetlinewidth{0.000000pt}%
\definecolor{currentstroke}{rgb}{0.000000,0.000000,0.000000}%
\pgfsetstrokecolor{currentstroke}%
\pgfsetdash{}{0pt}%
\pgfpathmoveto{\pgfqpoint{3.684001in}{0.841174in}}%
\pgfpathlineto{\pgfqpoint{3.683929in}{0.849923in}}%
\pgfpathlineto{\pgfqpoint{3.683773in}{0.858986in}}%
\pgfpathlineto{\pgfqpoint{3.683532in}{0.868330in}}%
\pgfpathlineto{\pgfqpoint{3.683207in}{0.877918in}}%
\pgfpathlineto{\pgfqpoint{3.682801in}{0.887714in}}%
\pgfpathlineto{\pgfqpoint{3.648051in}{0.886249in}}%
\pgfpathlineto{\pgfqpoint{3.613193in}{0.885850in}}%
\pgfpathlineto{\pgfqpoint{3.578351in}{0.886518in}}%
\pgfpathlineto{\pgfqpoint{3.543646in}{0.888250in}}%
\pgfpathlineto{\pgfqpoint{3.509202in}{0.891042in}}%
\pgfpathlineto{\pgfqpoint{3.508528in}{0.881262in}}%
\pgfpathlineto{\pgfqpoint{3.507989in}{0.871685in}}%
\pgfpathlineto{\pgfqpoint{3.507590in}{0.862348in}}%
\pgfpathlineto{\pgfqpoint{3.507330in}{0.853288in}}%
\pgfpathlineto{\pgfqpoint{3.507211in}{0.844539in}}%
\pgfpathlineto{\pgfqpoint{3.542289in}{0.841717in}}%
\pgfpathlineto{\pgfqpoint{3.577632in}{0.839966in}}%
\pgfpathlineto{\pgfqpoint{3.613114in}{0.839291in}}%
\pgfpathlineto{\pgfqpoint{3.648612in}{0.839694in}}%
\pgfpathlineto{\pgfqpoint{3.684001in}{0.841174in}}%
\pgfpathclose%
\pgfusepath{fill}%
\end{pgfscope}%
\begin{pgfscope}%
\pgfpathrectangle{\pgfqpoint{2.548318in}{0.050000in}}{\pgfqpoint{2.081932in}{2.081932in}}%
\pgfusepath{clip}%
\pgfsetbuttcap%
\pgfsetroundjoin%
\definecolor{currentfill}{rgb}{0.636902,0.856542,0.216620}%
\pgfsetfillcolor{currentfill}%
\pgfsetlinewidth{0.000000pt}%
\definecolor{currentstroke}{rgb}{0.000000,0.000000,0.000000}%
\pgfsetstrokecolor{currentstroke}%
\pgfsetdash{}{0pt}%
\pgfpathmoveto{\pgfqpoint{3.679606in}{0.938461in}}%
\pgfpathlineto{\pgfqpoint{3.678751in}{0.948691in}}%
\pgfpathlineto{\pgfqpoint{3.677831in}{0.958857in}}%
\pgfpathlineto{\pgfqpoint{3.676850in}{0.968920in}}%
\pgfpathlineto{\pgfqpoint{3.675811in}{0.978839in}}%
\pgfpathlineto{\pgfqpoint{3.674718in}{0.988578in}}%
\pgfpathlineto{\pgfqpoint{3.644271in}{0.987274in}}%
\pgfpathlineto{\pgfqpoint{3.613729in}{0.986919in}}%
\pgfpathlineto{\pgfqpoint{3.583200in}{0.987514in}}%
\pgfpathlineto{\pgfqpoint{3.552794in}{0.989056in}}%
\pgfpathlineto{\pgfqpoint{3.522620in}{0.991541in}}%
\pgfpathlineto{\pgfqpoint{3.520806in}{0.981854in}}%
\pgfpathlineto{\pgfqpoint{3.519081in}{0.971983in}}%
\pgfpathlineto{\pgfqpoint{3.517452in}{0.961967in}}%
\pgfpathlineto{\pgfqpoint{3.515925in}{0.951843in}}%
\pgfpathlineto{\pgfqpoint{3.514506in}{0.941651in}}%
\pgfpathlineto{\pgfqpoint{3.547263in}{0.938975in}}%
\pgfpathlineto{\pgfqpoint{3.580268in}{0.937314in}}%
\pgfpathlineto{\pgfqpoint{3.613405in}{0.936674in}}%
\pgfpathlineto{\pgfqpoint{3.646557in}{0.937057in}}%
\pgfpathlineto{\pgfqpoint{3.679606in}{0.938461in}}%
\pgfpathclose%
\pgfusepath{fill}%
\end{pgfscope}%
\begin{pgfscope}%
\pgfpathrectangle{\pgfqpoint{2.548318in}{0.050000in}}{\pgfqpoint{2.081932in}{2.081932in}}%
\pgfusepath{clip}%
\pgfsetbuttcap%
\pgfsetroundjoin%
\definecolor{currentfill}{rgb}{0.327796,0.773980,0.406640}%
\pgfsetfillcolor{currentfill}%
\pgfsetlinewidth{0.000000pt}%
\definecolor{currentstroke}{rgb}{0.000000,0.000000,0.000000}%
\pgfsetstrokecolor{currentstroke}%
\pgfsetdash{}{0pt}%
\pgfpathmoveto{\pgfqpoint{3.850678in}{0.910749in}}%
\pgfpathlineto{\pgfqpoint{3.848934in}{0.920578in}}%
\pgfpathlineto{\pgfqpoint{3.846908in}{0.930510in}}%
\pgfpathlineto{\pgfqpoint{3.844609in}{0.940507in}}%
\pgfpathlineto{\pgfqpoint{3.842045in}{0.950530in}}%
\pgfpathlineto{\pgfqpoint{3.839226in}{0.960542in}}%
\pgfpathlineto{\pgfqpoint{3.808437in}{0.954158in}}%
\pgfpathlineto{\pgfqpoint{3.776966in}{0.948742in}}%
\pgfpathlineto{\pgfqpoint{3.744927in}{0.944312in}}%
\pgfpathlineto{\pgfqpoint{3.712435in}{0.940882in}}%
\pgfpathlineto{\pgfqpoint{3.679606in}{0.938461in}}%
\pgfpathlineto{\pgfqpoint{3.680392in}{0.928205in}}%
\pgfpathlineto{\pgfqpoint{3.681108in}{0.917963in}}%
\pgfpathlineto{\pgfqpoint{3.681749in}{0.907774in}}%
\pgfpathlineto{\pgfqpoint{3.682315in}{0.897678in}}%
\pgfpathlineto{\pgfqpoint{3.682801in}{0.887714in}}%
\pgfpathlineto{\pgfqpoint{3.717322in}{0.890239in}}%
\pgfpathlineto{\pgfqpoint{3.751490in}{0.893817in}}%
\pgfpathlineto{\pgfqpoint{3.785186in}{0.898439in}}%
\pgfpathlineto{\pgfqpoint{3.818288in}{0.904088in}}%
\pgfpathlineto{\pgfqpoint{3.850678in}{0.910749in}}%
\pgfpathclose%
\pgfusepath{fill}%
\end{pgfscope}%
\begin{pgfscope}%
\pgfpathrectangle{\pgfqpoint{2.548318in}{0.050000in}}{\pgfqpoint{2.081932in}{2.081932in}}%
\pgfusepath{clip}%
\pgfsetbuttcap%
\pgfsetroundjoin%
\definecolor{currentfill}{rgb}{0.124780,0.640461,0.527068}%
\pgfsetfillcolor{currentfill}%
\pgfsetlinewidth{0.000000pt}%
\definecolor{currentstroke}{rgb}{0.000000,0.000000,0.000000}%
\pgfsetstrokecolor{currentstroke}%
\pgfsetdash{}{0pt}%
\pgfpathmoveto{\pgfqpoint{3.854978in}{0.864460in}}%
\pgfpathlineto{\pgfqpoint{3.854721in}{0.873216in}}%
\pgfpathlineto{\pgfqpoint{3.854160in}{0.882258in}}%
\pgfpathlineto{\pgfqpoint{3.853297in}{0.891552in}}%
\pgfpathlineto{\pgfqpoint{3.852135in}{0.901061in}}%
\pgfpathlineto{\pgfqpoint{3.850678in}{0.910749in}}%
\pgfpathlineto{\pgfqpoint{3.818288in}{0.904088in}}%
\pgfpathlineto{\pgfqpoint{3.785186in}{0.898439in}}%
\pgfpathlineto{\pgfqpoint{3.751490in}{0.893817in}}%
\pgfpathlineto{\pgfqpoint{3.717322in}{0.890239in}}%
\pgfpathlineto{\pgfqpoint{3.682801in}{0.887714in}}%
\pgfpathlineto{\pgfqpoint{3.683207in}{0.877918in}}%
\pgfpathlineto{\pgfqpoint{3.683532in}{0.868330in}}%
\pgfpathlineto{\pgfqpoint{3.683773in}{0.858986in}}%
\pgfpathlineto{\pgfqpoint{3.683929in}{0.849923in}}%
\pgfpathlineto{\pgfqpoint{3.684001in}{0.841174in}}%
\pgfpathlineto{\pgfqpoint{3.719156in}{0.843727in}}%
\pgfpathlineto{\pgfqpoint{3.753954in}{0.847344in}}%
\pgfpathlineto{\pgfqpoint{3.788272in}{0.852016in}}%
\pgfpathlineto{\pgfqpoint{3.821987in}{0.857727in}}%
\pgfpathlineto{\pgfqpoint{3.854978in}{0.864460in}}%
\pgfpathclose%
\pgfusepath{fill}%
\end{pgfscope}%
\begin{pgfscope}%
\pgfpathrectangle{\pgfqpoint{2.548318in}{0.050000in}}{\pgfqpoint{2.081932in}{2.081932in}}%
\pgfusepath{clip}%
\pgfsetbuttcap%
\pgfsetroundjoin%
\definecolor{currentfill}{rgb}{0.636902,0.856542,0.216620}%
\pgfsetfillcolor{currentfill}%
\pgfsetlinewidth{0.000000pt}%
\definecolor{currentstroke}{rgb}{0.000000,0.000000,0.000000}%
\pgfsetstrokecolor{currentstroke}%
\pgfsetdash{}{0pt}%
\pgfpathmoveto{\pgfqpoint{3.839226in}{0.960542in}}%
\pgfpathlineto{\pgfqpoint{3.836163in}{0.970503in}}%
\pgfpathlineto{\pgfqpoint{3.832868in}{0.980374in}}%
\pgfpathlineto{\pgfqpoint{3.829352in}{0.990119in}}%
\pgfpathlineto{\pgfqpoint{3.825631in}{0.999699in}}%
\pgfpathlineto{\pgfqpoint{3.821717in}{1.009077in}}%
\pgfpathlineto{\pgfqpoint{3.793373in}{1.003151in}}%
\pgfpathlineto{\pgfqpoint{3.764395in}{0.998124in}}%
\pgfpathlineto{\pgfqpoint{3.734888in}{0.994012in}}%
\pgfpathlineto{\pgfqpoint{3.704960in}{0.990826in}}%
\pgfpathlineto{\pgfqpoint{3.674718in}{0.988578in}}%
\pgfpathlineto{\pgfqpoint{3.675811in}{0.978839in}}%
\pgfpathlineto{\pgfqpoint{3.676850in}{0.968920in}}%
\pgfpathlineto{\pgfqpoint{3.677831in}{0.958857in}}%
\pgfpathlineto{\pgfqpoint{3.678751in}{0.948691in}}%
\pgfpathlineto{\pgfqpoint{3.679606in}{0.938461in}}%
\pgfpathlineto{\pgfqpoint{3.712435in}{0.940882in}}%
\pgfpathlineto{\pgfqpoint{3.744927in}{0.944312in}}%
\pgfpathlineto{\pgfqpoint{3.776966in}{0.948742in}}%
\pgfpathlineto{\pgfqpoint{3.808437in}{0.954158in}}%
\pgfpathlineto{\pgfqpoint{3.839226in}{0.960542in}}%
\pgfpathclose%
\pgfusepath{fill}%
\end{pgfscope}%
\begin{pgfscope}%
\pgfpathrectangle{\pgfqpoint{2.548318in}{0.050000in}}{\pgfqpoint{2.081932in}{2.081932in}}%
\pgfusepath{clip}%
\pgfsetbuttcap%
\pgfsetroundjoin%
\definecolor{currentfill}{rgb}{0.150476,0.504369,0.557430}%
\pgfsetfillcolor{currentfill}%
\pgfsetlinewidth{0.000000pt}%
\definecolor{currentstroke}{rgb}{0.000000,0.000000,0.000000}%
\pgfsetstrokecolor{currentstroke}%
\pgfsetdash{}{0pt}%
\pgfpathmoveto{\pgfqpoint{3.683434in}{0.810000in}}%
\pgfpathlineto{\pgfqpoint{3.683703in}{0.817158in}}%
\pgfpathlineto{\pgfqpoint{3.683888in}{0.824760in}}%
\pgfpathlineto{\pgfqpoint{3.683987in}{0.832776in}}%
\pgfpathlineto{\pgfqpoint{3.684001in}{0.841174in}}%
\pgfpathlineto{\pgfqpoint{3.648612in}{0.839694in}}%
\pgfpathlineto{\pgfqpoint{3.613114in}{0.839291in}}%
\pgfpathlineto{\pgfqpoint{3.577632in}{0.839966in}}%
\pgfpathlineto{\pgfqpoint{3.542289in}{0.841717in}}%
\pgfpathlineto{\pgfqpoint{3.507211in}{0.844539in}}%
\pgfpathlineto{\pgfqpoint{3.507234in}{0.836135in}}%
\pgfpathlineto{\pgfqpoint{3.507399in}{0.828109in}}%
\pgfpathlineto{\pgfqpoint{3.507706in}{0.820494in}}%
\pgfpathlineto{\pgfqpoint{3.508153in}{0.813319in}}%
\pgfpathlineto{\pgfqpoint{3.542931in}{0.810535in}}%
\pgfpathlineto{\pgfqpoint{3.577972in}{0.808808in}}%
\pgfpathlineto{\pgfqpoint{3.613152in}{0.808142in}}%
\pgfpathlineto{\pgfqpoint{3.648347in}{0.808540in}}%
\pgfpathlineto{\pgfqpoint{3.683434in}{0.810000in}}%
\pgfpathclose%
\pgfusepath{fill}%
\end{pgfscope}%
\begin{pgfscope}%
\pgfpathrectangle{\pgfqpoint{2.548318in}{0.050000in}}{\pgfqpoint{2.081932in}{2.081932in}}%
\pgfusepath{clip}%
\pgfsetbuttcap%
\pgfsetroundjoin%
\definecolor{currentfill}{rgb}{0.327796,0.773980,0.406640}%
\pgfsetfillcolor{currentfill}%
\pgfsetlinewidth{0.000000pt}%
\definecolor{currentstroke}{rgb}{0.000000,0.000000,0.000000}%
\pgfsetstrokecolor{currentstroke}%
\pgfsetdash{}{0pt}%
\pgfpathmoveto{\pgfqpoint{3.509202in}{0.891042in}}%
\pgfpathlineto{\pgfqpoint{3.510010in}{0.900987in}}%
\pgfpathlineto{\pgfqpoint{3.510948in}{0.911059in}}%
\pgfpathlineto{\pgfqpoint{3.512013in}{0.921220in}}%
\pgfpathlineto{\pgfqpoint{3.513201in}{0.931431in}}%
\pgfpathlineto{\pgfqpoint{3.514506in}{0.941651in}}%
\pgfpathlineto{\pgfqpoint{3.482116in}{0.945335in}}%
\pgfpathlineto{\pgfqpoint{3.450207in}{0.950015in}}%
\pgfpathlineto{\pgfqpoint{3.418894in}{0.955676in}}%
\pgfpathlineto{\pgfqpoint{3.388292in}{0.962301in}}%
\pgfpathlineto{\pgfqpoint{3.358511in}{0.969869in}}%
\pgfpathlineto{\pgfqpoint{3.355213in}{0.959962in}}%
\pgfpathlineto{\pgfqpoint{3.352213in}{0.950031in}}%
\pgfpathlineto{\pgfqpoint{3.349523in}{0.940115in}}%
\pgfpathlineto{\pgfqpoint{3.347152in}{0.930253in}}%
\pgfpathlineto{\pgfqpoint{3.345111in}{0.920482in}}%
\pgfpathlineto{\pgfqpoint{3.376447in}{0.912584in}}%
\pgfpathlineto{\pgfqpoint{3.408643in}{0.905672in}}%
\pgfpathlineto{\pgfqpoint{3.441579in}{0.899766in}}%
\pgfpathlineto{\pgfqpoint{3.475139in}{0.894884in}}%
\pgfpathlineto{\pgfqpoint{3.509202in}{0.891042in}}%
\pgfpathclose%
\pgfusepath{fill}%
\end{pgfscope}%
\begin{pgfscope}%
\pgfpathrectangle{\pgfqpoint{2.548318in}{0.050000in}}{\pgfqpoint{2.081932in}{2.081932in}}%
\pgfusepath{clip}%
\pgfsetbuttcap%
\pgfsetroundjoin%
\definecolor{currentfill}{rgb}{0.124780,0.640461,0.527068}%
\pgfsetfillcolor{currentfill}%
\pgfsetlinewidth{0.000000pt}%
\definecolor{currentstroke}{rgb}{0.000000,0.000000,0.000000}%
\pgfsetstrokecolor{currentstroke}%
\pgfsetdash{}{0pt}%
\pgfpathmoveto{\pgfqpoint{3.507211in}{0.844539in}}%
\pgfpathlineto{\pgfqpoint{3.507330in}{0.853288in}}%
\pgfpathlineto{\pgfqpoint{3.507590in}{0.862348in}}%
\pgfpathlineto{\pgfqpoint{3.507989in}{0.871685in}}%
\pgfpathlineto{\pgfqpoint{3.508528in}{0.881262in}}%
\pgfpathlineto{\pgfqpoint{3.509202in}{0.891042in}}%
\pgfpathlineto{\pgfqpoint{3.475139in}{0.894884in}}%
\pgfpathlineto{\pgfqpoint{3.441579in}{0.899766in}}%
\pgfpathlineto{\pgfqpoint{3.408643in}{0.905672in}}%
\pgfpathlineto{\pgfqpoint{3.376447in}{0.912584in}}%
\pgfpathlineto{\pgfqpoint{3.345111in}{0.920482in}}%
\pgfpathlineto{\pgfqpoint{3.343407in}{0.910840in}}%
\pgfpathlineto{\pgfqpoint{3.342047in}{0.901364in}}%
\pgfpathlineto{\pgfqpoint{3.341037in}{0.892092in}}%
\pgfpathlineto{\pgfqpoint{3.340380in}{0.883059in}}%
\pgfpathlineto{\pgfqpoint{3.340081in}{0.874300in}}%
\pgfpathlineto{\pgfqpoint{3.372001in}{0.866316in}}%
\pgfpathlineto{\pgfqpoint{3.404794in}{0.859328in}}%
\pgfpathlineto{\pgfqpoint{3.438341in}{0.853357in}}%
\pgfpathlineto{\pgfqpoint{3.472521in}{0.848423in}}%
\pgfpathlineto{\pgfqpoint{3.507211in}{0.844539in}}%
\pgfpathclose%
\pgfusepath{fill}%
\end{pgfscope}%
\begin{pgfscope}%
\pgfpathrectangle{\pgfqpoint{2.548318in}{0.050000in}}{\pgfqpoint{2.081932in}{2.081932in}}%
\pgfusepath{clip}%
\pgfsetbuttcap%
\pgfsetroundjoin%
\definecolor{currentfill}{rgb}{0.876168,0.891125,0.095250}%
\pgfsetfillcolor{currentfill}%
\pgfsetlinewidth{0.000000pt}%
\definecolor{currentstroke}{rgb}{0.000000,0.000000,0.000000}%
\pgfsetstrokecolor{currentstroke}%
\pgfsetdash{}{0pt}%
\pgfpathmoveto{\pgfqpoint{3.674718in}{0.988578in}}%
\pgfpathlineto{\pgfqpoint{3.673576in}{0.998098in}}%
\pgfpathlineto{\pgfqpoint{3.672388in}{1.007362in}}%
\pgfpathlineto{\pgfqpoint{3.671159in}{1.016334in}}%
\pgfpathlineto{\pgfqpoint{3.669895in}{1.024978in}}%
\pgfpathlineto{\pgfqpoint{3.668599in}{1.033260in}}%
\pgfpathlineto{\pgfqpoint{3.641410in}{1.032087in}}%
\pgfpathlineto{\pgfqpoint{3.614135in}{1.031768in}}%
\pgfpathlineto{\pgfqpoint{3.586871in}{1.032303in}}%
\pgfpathlineto{\pgfqpoint{3.559720in}{1.033689in}}%
\pgfpathlineto{\pgfqpoint{3.532777in}{1.035924in}}%
\pgfpathlineto{\pgfqpoint{3.530626in}{1.027706in}}%
\pgfpathlineto{\pgfqpoint{3.528527in}{1.019124in}}%
\pgfpathlineto{\pgfqpoint{3.526488in}{1.010212in}}%
\pgfpathlineto{\pgfqpoint{3.524516in}{1.001006in}}%
\pgfpathlineto{\pgfqpoint{3.522620in}{0.991541in}}%
\pgfpathlineto{\pgfqpoint{3.552794in}{0.989056in}}%
\pgfpathlineto{\pgfqpoint{3.583200in}{0.987514in}}%
\pgfpathlineto{\pgfqpoint{3.613729in}{0.986919in}}%
\pgfpathlineto{\pgfqpoint{3.644271in}{0.987274in}}%
\pgfpathlineto{\pgfqpoint{3.674718in}{0.988578in}}%
\pgfpathclose%
\pgfusepath{fill}%
\end{pgfscope}%
\begin{pgfscope}%
\pgfpathrectangle{\pgfqpoint{2.548318in}{0.050000in}}{\pgfqpoint{2.081932in}{2.081932in}}%
\pgfusepath{clip}%
\pgfsetbuttcap%
\pgfsetroundjoin%
\definecolor{currentfill}{rgb}{0.636902,0.856542,0.216620}%
\pgfsetfillcolor{currentfill}%
\pgfsetlinewidth{0.000000pt}%
\definecolor{currentstroke}{rgb}{0.000000,0.000000,0.000000}%
\pgfsetstrokecolor{currentstroke}%
\pgfsetdash{}{0pt}%
\pgfpathmoveto{\pgfqpoint{3.514506in}{0.941651in}}%
\pgfpathlineto{\pgfqpoint{3.515925in}{0.951843in}}%
\pgfpathlineto{\pgfqpoint{3.517452in}{0.961967in}}%
\pgfpathlineto{\pgfqpoint{3.519081in}{0.971983in}}%
\pgfpathlineto{\pgfqpoint{3.520806in}{0.981854in}}%
\pgfpathlineto{\pgfqpoint{3.522620in}{0.991541in}}%
\pgfpathlineto{\pgfqpoint{3.492786in}{0.994961in}}%
\pgfpathlineto{\pgfqpoint{3.463400in}{0.999305in}}%
\pgfpathlineto{\pgfqpoint{3.434570in}{1.004560in}}%
\pgfpathlineto{\pgfqpoint{3.406400in}{1.010709in}}%
\pgfpathlineto{\pgfqpoint{3.378994in}{1.017732in}}%
\pgfpathlineto{\pgfqpoint{3.374416in}{1.008508in}}%
\pgfpathlineto{\pgfqpoint{3.370062in}{0.999072in}}%
\pgfpathlineto{\pgfqpoint{3.365950in}{0.989462in}}%
\pgfpathlineto{\pgfqpoint{3.362095in}{0.979715in}}%
\pgfpathlineto{\pgfqpoint{3.358511in}{0.969869in}}%
\pgfpathlineto{\pgfqpoint{3.388292in}{0.962301in}}%
\pgfpathlineto{\pgfqpoint{3.418894in}{0.955676in}}%
\pgfpathlineto{\pgfqpoint{3.450207in}{0.950015in}}%
\pgfpathlineto{\pgfqpoint{3.482116in}{0.945335in}}%
\pgfpathlineto{\pgfqpoint{3.514506in}{0.941651in}}%
\pgfpathclose%
\pgfusepath{fill}%
\end{pgfscope}%
\begin{pgfscope}%
\pgfpathrectangle{\pgfqpoint{2.548318in}{0.050000in}}{\pgfqpoint{2.081932in}{2.081932in}}%
\pgfusepath{clip}%
\pgfsetbuttcap%
\pgfsetroundjoin%
\definecolor{currentfill}{rgb}{0.150476,0.504369,0.557430}%
\pgfsetfillcolor{currentfill}%
\pgfsetlinewidth{0.000000pt}%
\definecolor{currentstroke}{rgb}{0.000000,0.000000,0.000000}%
\pgfsetstrokecolor{currentstroke}%
\pgfsetdash{}{0pt}%
\pgfpathmoveto{\pgfqpoint{3.852946in}{0.832971in}}%
\pgfpathlineto{\pgfqpoint{3.853910in}{0.840249in}}%
\pgfpathlineto{\pgfqpoint{3.854572in}{0.847944in}}%
\pgfpathlineto{\pgfqpoint{3.854928in}{0.856025in}}%
\pgfpathlineto{\pgfqpoint{3.854978in}{0.864460in}}%
\pgfpathlineto{\pgfqpoint{3.821987in}{0.857727in}}%
\pgfpathlineto{\pgfqpoint{3.788272in}{0.852016in}}%
\pgfpathlineto{\pgfqpoint{3.753954in}{0.847344in}}%
\pgfpathlineto{\pgfqpoint{3.719156in}{0.843727in}}%
\pgfpathlineto{\pgfqpoint{3.684001in}{0.841174in}}%
\pgfpathlineto{\pgfqpoint{3.683987in}{0.832776in}}%
\pgfpathlineto{\pgfqpoint{3.683888in}{0.824760in}}%
\pgfpathlineto{\pgfqpoint{3.683703in}{0.817158in}}%
\pgfpathlineto{\pgfqpoint{3.683434in}{0.810000in}}%
\pgfpathlineto{\pgfqpoint{3.718289in}{0.812518in}}%
\pgfpathlineto{\pgfqpoint{3.752790in}{0.816087in}}%
\pgfpathlineto{\pgfqpoint{3.786813in}{0.820695in}}%
\pgfpathlineto{\pgfqpoint{3.820239in}{0.826328in}}%
\pgfpathlineto{\pgfqpoint{3.852946in}{0.832971in}}%
\pgfpathclose%
\pgfusepath{fill}%
\end{pgfscope}%
\begin{pgfscope}%
\pgfpathrectangle{\pgfqpoint{2.548318in}{0.050000in}}{\pgfqpoint{2.081932in}{2.081932in}}%
\pgfusepath{clip}%
\pgfsetbuttcap%
\pgfsetroundjoin%
\definecolor{currentfill}{rgb}{0.876168,0.891125,0.095250}%
\pgfsetfillcolor{currentfill}%
\pgfsetlinewidth{0.000000pt}%
\definecolor{currentstroke}{rgb}{0.000000,0.000000,0.000000}%
\pgfsetstrokecolor{currentstroke}%
\pgfsetdash{}{0pt}%
\pgfpathmoveto{\pgfqpoint{3.821717in}{1.009077in}}%
\pgfpathlineto{\pgfqpoint{3.817625in}{1.018216in}}%
\pgfpathlineto{\pgfqpoint{3.813372in}{1.027081in}}%
\pgfpathlineto{\pgfqpoint{3.808973in}{1.035637in}}%
\pgfpathlineto{\pgfqpoint{3.804445in}{1.043850in}}%
\pgfpathlineto{\pgfqpoint{3.799807in}{1.051687in}}%
\pgfpathlineto{\pgfqpoint{3.774520in}{1.046362in}}%
\pgfpathlineto{\pgfqpoint{3.748660in}{1.041843in}}%
\pgfpathlineto{\pgfqpoint{3.722321in}{1.038146in}}%
\pgfpathlineto{\pgfqpoint{3.695602in}{1.035281in}}%
\pgfpathlineto{\pgfqpoint{3.668599in}{1.033260in}}%
\pgfpathlineto{\pgfqpoint{3.669895in}{1.024978in}}%
\pgfpathlineto{\pgfqpoint{3.671159in}{1.016334in}}%
\pgfpathlineto{\pgfqpoint{3.672388in}{1.007362in}}%
\pgfpathlineto{\pgfqpoint{3.673576in}{0.998098in}}%
\pgfpathlineto{\pgfqpoint{3.674718in}{0.988578in}}%
\pgfpathlineto{\pgfqpoint{3.704960in}{0.990826in}}%
\pgfpathlineto{\pgfqpoint{3.734888in}{0.994012in}}%
\pgfpathlineto{\pgfqpoint{3.764395in}{0.998124in}}%
\pgfpathlineto{\pgfqpoint{3.793373in}{1.003151in}}%
\pgfpathlineto{\pgfqpoint{3.821717in}{1.009077in}}%
\pgfpathclose%
\pgfusepath{fill}%
\end{pgfscope}%
\begin{pgfscope}%
\pgfpathrectangle{\pgfqpoint{2.548318in}{0.050000in}}{\pgfqpoint{2.081932in}{2.081932in}}%
\pgfusepath{clip}%
\pgfsetbuttcap%
\pgfsetroundjoin%
\definecolor{currentfill}{rgb}{0.150476,0.504369,0.557430}%
\pgfsetfillcolor{currentfill}%
\pgfsetlinewidth{0.000000pt}%
\definecolor{currentstroke}{rgb}{0.000000,0.000000,0.000000}%
\pgfsetstrokecolor{currentstroke}%
\pgfsetdash{}{0pt}%
\pgfpathmoveto{\pgfqpoint{3.508153in}{0.813319in}}%
\pgfpathlineto{\pgfqpoint{3.507706in}{0.820494in}}%
\pgfpathlineto{\pgfqpoint{3.507399in}{0.828109in}}%
\pgfpathlineto{\pgfqpoint{3.507234in}{0.836135in}}%
\pgfpathlineto{\pgfqpoint{3.507211in}{0.844539in}}%
\pgfpathlineto{\pgfqpoint{3.472521in}{0.848423in}}%
\pgfpathlineto{\pgfqpoint{3.438341in}{0.853357in}}%
\pgfpathlineto{\pgfqpoint{3.404794in}{0.859328in}}%
\pgfpathlineto{\pgfqpoint{3.372001in}{0.866316in}}%
\pgfpathlineto{\pgfqpoint{3.340081in}{0.874300in}}%
\pgfpathlineto{\pgfqpoint{3.340139in}{0.865850in}}%
\pgfpathlineto{\pgfqpoint{3.340556in}{0.857741in}}%
\pgfpathlineto{\pgfqpoint{3.341330in}{0.850007in}}%
\pgfpathlineto{\pgfqpoint{3.342459in}{0.842677in}}%
\pgfpathlineto{\pgfqpoint{3.374104in}{0.834801in}}%
\pgfpathlineto{\pgfqpoint{3.406614in}{0.827908in}}%
\pgfpathlineto{\pgfqpoint{3.439872in}{0.822018in}}%
\pgfpathlineto{\pgfqpoint{3.473759in}{0.817150in}}%
\pgfpathlineto{\pgfqpoint{3.508153in}{0.813319in}}%
\pgfpathclose%
\pgfusepath{fill}%
\end{pgfscope}%
\begin{pgfscope}%
\pgfpathrectangle{\pgfqpoint{2.548318in}{0.050000in}}{\pgfqpoint{2.081932in}{2.081932in}}%
\pgfusepath{clip}%
\pgfsetbuttcap%
\pgfsetroundjoin%
\definecolor{currentfill}{rgb}{0.206756,0.371758,0.553117}%
\pgfsetfillcolor{currentfill}%
\pgfsetlinewidth{0.000000pt}%
\definecolor{currentstroke}{rgb}{0.000000,0.000000,0.000000}%
\pgfsetstrokecolor{currentstroke}%
\pgfsetdash{}{0pt}%
\pgfpathmoveto{\pgfqpoint{3.680861in}{0.781816in}}%
\pgfpathlineto{\pgfqpoint{3.681534in}{0.786358in}}%
\pgfpathlineto{\pgfqpoint{3.682129in}{0.791470in}}%
\pgfpathlineto{\pgfqpoint{3.682646in}{0.797130in}}%
\pgfpathlineto{\pgfqpoint{3.683081in}{0.803315in}}%
\pgfpathlineto{\pgfqpoint{3.683434in}{0.810000in}}%
\pgfpathlineto{\pgfqpoint{3.648347in}{0.808540in}}%
\pgfpathlineto{\pgfqpoint{3.613152in}{0.808142in}}%
\pgfpathlineto{\pgfqpoint{3.577972in}{0.808808in}}%
\pgfpathlineto{\pgfqpoint{3.542931in}{0.810535in}}%
\pgfpathlineto{\pgfqpoint{3.508153in}{0.813319in}}%
\pgfpathlineto{\pgfqpoint{3.508739in}{0.806612in}}%
\pgfpathlineto{\pgfqpoint{3.509462in}{0.800402in}}%
\pgfpathlineto{\pgfqpoint{3.510319in}{0.794713in}}%
\pgfpathlineto{\pgfqpoint{3.511308in}{0.789568in}}%
\pgfpathlineto{\pgfqpoint{3.512425in}{0.784990in}}%
\pgfpathlineto{\pgfqpoint{3.545844in}{0.782327in}}%
\pgfpathlineto{\pgfqpoint{3.579516in}{0.780675in}}%
\pgfpathlineto{\pgfqpoint{3.613323in}{0.780038in}}%
\pgfpathlineto{\pgfqpoint{3.647144in}{0.780419in}}%
\pgfpathlineto{\pgfqpoint{3.680861in}{0.781816in}}%
\pgfpathclose%
\pgfusepath{fill}%
\end{pgfscope}%
\begin{pgfscope}%
\pgfpathrectangle{\pgfqpoint{2.548318in}{0.050000in}}{\pgfqpoint{2.081932in}{2.081932in}}%
\pgfusepath{clip}%
\pgfsetbuttcap%
\pgfsetroundjoin%
\definecolor{currentfill}{rgb}{0.876168,0.891125,0.095250}%
\pgfsetfillcolor{currentfill}%
\pgfsetlinewidth{0.000000pt}%
\definecolor{currentstroke}{rgb}{0.000000,0.000000,0.000000}%
\pgfsetstrokecolor{currentstroke}%
\pgfsetdash{}{0pt}%
\pgfpathmoveto{\pgfqpoint{3.522620in}{0.991541in}}%
\pgfpathlineto{\pgfqpoint{3.524516in}{1.001006in}}%
\pgfpathlineto{\pgfqpoint{3.526488in}{1.010212in}}%
\pgfpathlineto{\pgfqpoint{3.528527in}{1.019124in}}%
\pgfpathlineto{\pgfqpoint{3.530626in}{1.027706in}}%
\pgfpathlineto{\pgfqpoint{3.532777in}{1.035924in}}%
\pgfpathlineto{\pgfqpoint{3.506143in}{1.038999in}}%
\pgfpathlineto{\pgfqpoint{3.479915in}{1.042905in}}%
\pgfpathlineto{\pgfqpoint{3.454188in}{1.047629in}}%
\pgfpathlineto{\pgfqpoint{3.429058in}{1.053155in}}%
\pgfpathlineto{\pgfqpoint{3.404619in}{1.059465in}}%
\pgfpathlineto{\pgfqpoint{3.399194in}{1.051815in}}%
\pgfpathlineto{\pgfqpoint{3.393899in}{1.043785in}}%
\pgfpathlineto{\pgfqpoint{3.388755in}{1.035406in}}%
\pgfpathlineto{\pgfqpoint{3.383780in}{1.026710in}}%
\pgfpathlineto{\pgfqpoint{3.378994in}{1.017732in}}%
\pgfpathlineto{\pgfqpoint{3.406400in}{1.010709in}}%
\pgfpathlineto{\pgfqpoint{3.434570in}{1.004560in}}%
\pgfpathlineto{\pgfqpoint{3.463400in}{0.999305in}}%
\pgfpathlineto{\pgfqpoint{3.492786in}{0.994961in}}%
\pgfpathlineto{\pgfqpoint{3.522620in}{0.991541in}}%
\pgfpathclose%
\pgfusepath{fill}%
\end{pgfscope}%
\begin{pgfscope}%
\pgfpathrectangle{\pgfqpoint{2.548318in}{0.050000in}}{\pgfqpoint{2.081932in}{2.081932in}}%
\pgfusepath{clip}%
\pgfsetbuttcap%
\pgfsetroundjoin%
\definecolor{currentfill}{rgb}{0.993248,0.906157,0.143936}%
\pgfsetfillcolor{currentfill}%
\pgfsetlinewidth{0.000000pt}%
\definecolor{currentstroke}{rgb}{0.000000,0.000000,0.000000}%
\pgfsetstrokecolor{currentstroke}%
\pgfsetdash{}{0pt}%
\pgfpathmoveto{\pgfqpoint{3.668599in}{1.033260in}}%
\pgfpathlineto{\pgfqpoint{3.667276in}{1.041147in}}%
\pgfpathlineto{\pgfqpoint{3.665933in}{1.048608in}}%
\pgfpathlineto{\pgfqpoint{3.664573in}{1.055611in}}%
\pgfpathlineto{\pgfqpoint{3.663203in}{1.062130in}}%
\pgfpathlineto{\pgfqpoint{3.661828in}{1.068136in}}%
\pgfpathlineto{\pgfqpoint{3.638244in}{1.067113in}}%
\pgfpathlineto{\pgfqpoint{3.614583in}{1.066834in}}%
\pgfpathlineto{\pgfqpoint{3.590934in}{1.067301in}}%
\pgfpathlineto{\pgfqpoint{3.567382in}{1.068510in}}%
\pgfpathlineto{\pgfqpoint{3.544015in}{1.070459in}}%
\pgfpathlineto{\pgfqpoint{3.541733in}{1.064523in}}%
\pgfpathlineto{\pgfqpoint{3.539458in}{1.058074in}}%
\pgfpathlineto{\pgfqpoint{3.537202in}{1.051139in}}%
\pgfpathlineto{\pgfqpoint{3.534972in}{1.043746in}}%
\pgfpathlineto{\pgfqpoint{3.532777in}{1.035924in}}%
\pgfpathlineto{\pgfqpoint{3.559720in}{1.033689in}}%
\pgfpathlineto{\pgfqpoint{3.586871in}{1.032303in}}%
\pgfpathlineto{\pgfqpoint{3.614135in}{1.031768in}}%
\pgfpathlineto{\pgfqpoint{3.641410in}{1.032087in}}%
\pgfpathlineto{\pgfqpoint{3.668599in}{1.033260in}}%
\pgfpathclose%
\pgfusepath{fill}%
\end{pgfscope}%
\begin{pgfscope}%
\pgfpathrectangle{\pgfqpoint{2.548318in}{0.050000in}}{\pgfqpoint{2.081932in}{2.081932in}}%
\pgfusepath{clip}%
\pgfsetbuttcap%
\pgfsetroundjoin%
\definecolor{currentfill}{rgb}{0.206756,0.371758,0.553117}%
\pgfsetfillcolor{currentfill}%
\pgfsetlinewidth{0.000000pt}%
\definecolor{currentstroke}{rgb}{0.000000,0.000000,0.000000}%
\pgfsetstrokecolor{currentstroke}%
\pgfsetdash{}{0pt}%
\pgfpathmoveto{\pgfqpoint{3.843722in}{0.803786in}}%
\pgfpathlineto{\pgfqpoint{3.846133in}{0.808579in}}%
\pgfpathlineto{\pgfqpoint{3.848268in}{0.813917in}}%
\pgfpathlineto{\pgfqpoint{3.850119in}{0.819778in}}%
\pgfpathlineto{\pgfqpoint{3.851680in}{0.826138in}}%
\pgfpathlineto{\pgfqpoint{3.852946in}{0.832971in}}%
\pgfpathlineto{\pgfqpoint{3.820239in}{0.826328in}}%
\pgfpathlineto{\pgfqpoint{3.786813in}{0.820695in}}%
\pgfpathlineto{\pgfqpoint{3.752790in}{0.816087in}}%
\pgfpathlineto{\pgfqpoint{3.718289in}{0.812518in}}%
\pgfpathlineto{\pgfqpoint{3.683434in}{0.810000in}}%
\pgfpathlineto{\pgfqpoint{3.683081in}{0.803315in}}%
\pgfpathlineto{\pgfqpoint{3.682646in}{0.797130in}}%
\pgfpathlineto{\pgfqpoint{3.682129in}{0.791470in}}%
\pgfpathlineto{\pgfqpoint{3.681534in}{0.786358in}}%
\pgfpathlineto{\pgfqpoint{3.680861in}{0.781816in}}%
\pgfpathlineto{\pgfqpoint{3.714354in}{0.784224in}}%
\pgfpathlineto{\pgfqpoint{3.747504in}{0.787638in}}%
\pgfpathlineto{\pgfqpoint{3.780193in}{0.792045in}}%
\pgfpathlineto{\pgfqpoint{3.812304in}{0.797433in}}%
\pgfpathlineto{\pgfqpoint{3.843722in}{0.803786in}}%
\pgfpathclose%
\pgfusepath{fill}%
\end{pgfscope}%
\begin{pgfscope}%
\pgfpathrectangle{\pgfqpoint{2.548318in}{0.050000in}}{\pgfqpoint{2.081932in}{2.081932in}}%
\pgfusepath{clip}%
\pgfsetbuttcap%
\pgfsetroundjoin%
\definecolor{currentfill}{rgb}{0.327796,0.773980,0.406640}%
\pgfsetfillcolor{currentfill}%
\pgfsetlinewidth{0.000000pt}%
\definecolor{currentstroke}{rgb}{0.000000,0.000000,0.000000}%
\pgfsetstrokecolor{currentstroke}%
\pgfsetdash{}{0pt}%
\pgfpathmoveto{\pgfqpoint{3.997893in}{0.958389in}}%
\pgfpathlineto{\pgfqpoint{3.995026in}{0.967932in}}%
\pgfpathlineto{\pgfqpoint{3.991697in}{0.977521in}}%
\pgfpathlineto{\pgfqpoint{3.987918in}{0.987117in}}%
\pgfpathlineto{\pgfqpoint{3.983706in}{0.996684in}}%
\pgfpathlineto{\pgfqpoint{3.979075in}{1.006183in}}%
\pgfpathlineto{\pgfqpoint{3.953343in}{0.995318in}}%
\pgfpathlineto{\pgfqpoint{3.926387in}{0.985290in}}%
\pgfpathlineto{\pgfqpoint{3.898311in}{0.976133in}}%
\pgfpathlineto{\pgfqpoint{3.869221in}{0.967874in}}%
\pgfpathlineto{\pgfqpoint{3.839226in}{0.960542in}}%
\pgfpathlineto{\pgfqpoint{3.842045in}{0.950530in}}%
\pgfpathlineto{\pgfqpoint{3.844609in}{0.940507in}}%
\pgfpathlineto{\pgfqpoint{3.846908in}{0.930510in}}%
\pgfpathlineto{\pgfqpoint{3.848934in}{0.920578in}}%
\pgfpathlineto{\pgfqpoint{3.850678in}{0.910749in}}%
\pgfpathlineto{\pgfqpoint{3.882238in}{0.918400in}}%
\pgfpathlineto{\pgfqpoint{3.912853in}{0.927018in}}%
\pgfpathlineto{\pgfqpoint{3.942407in}{0.936577in}}%
\pgfpathlineto{\pgfqpoint{3.970790in}{0.947045in}}%
\pgfpathlineto{\pgfqpoint{3.997893in}{0.958389in}}%
\pgfpathclose%
\pgfusepath{fill}%
\end{pgfscope}%
\begin{pgfscope}%
\pgfpathrectangle{\pgfqpoint{2.548318in}{0.050000in}}{\pgfqpoint{2.081932in}{2.081932in}}%
\pgfusepath{clip}%
\pgfsetbuttcap%
\pgfsetroundjoin%
\definecolor{currentfill}{rgb}{0.636902,0.856542,0.216620}%
\pgfsetfillcolor{currentfill}%
\pgfsetlinewidth{0.000000pt}%
\definecolor{currentstroke}{rgb}{0.000000,0.000000,0.000000}%
\pgfsetstrokecolor{currentstroke}%
\pgfsetdash{}{0pt}%
\pgfpathmoveto{\pgfqpoint{3.979075in}{1.006183in}}%
\pgfpathlineto{\pgfqpoint{3.974045in}{1.015580in}}%
\pgfpathlineto{\pgfqpoint{3.968633in}{1.024837in}}%
\pgfpathlineto{\pgfqpoint{3.962863in}{1.033918in}}%
\pgfpathlineto{\pgfqpoint{3.956754in}{1.042787in}}%
\pgfpathlineto{\pgfqpoint{3.950332in}{1.051412in}}%
\pgfpathlineto{\pgfqpoint{3.926687in}{1.041338in}}%
\pgfpathlineto{\pgfqpoint{3.901906in}{1.032038in}}%
\pgfpathlineto{\pgfqpoint{3.876085in}{1.023544in}}%
\pgfpathlineto{\pgfqpoint{3.849321in}{1.015881in}}%
\pgfpathlineto{\pgfqpoint{3.821717in}{1.009077in}}%
\pgfpathlineto{\pgfqpoint{3.825631in}{0.999699in}}%
\pgfpathlineto{\pgfqpoint{3.829352in}{0.990119in}}%
\pgfpathlineto{\pgfqpoint{3.832868in}{0.980374in}}%
\pgfpathlineto{\pgfqpoint{3.836163in}{0.970503in}}%
\pgfpathlineto{\pgfqpoint{3.839226in}{0.960542in}}%
\pgfpathlineto{\pgfqpoint{3.869221in}{0.967874in}}%
\pgfpathlineto{\pgfqpoint{3.898311in}{0.976133in}}%
\pgfpathlineto{\pgfqpoint{3.926387in}{0.985290in}}%
\pgfpathlineto{\pgfqpoint{3.953343in}{0.995318in}}%
\pgfpathlineto{\pgfqpoint{3.979075in}{1.006183in}}%
\pgfpathclose%
\pgfusepath{fill}%
\end{pgfscope}%
\begin{pgfscope}%
\pgfpathrectangle{\pgfqpoint{2.548318in}{0.050000in}}{\pgfqpoint{2.081932in}{2.081932in}}%
\pgfusepath{clip}%
\pgfsetbuttcap%
\pgfsetroundjoin%
\definecolor{currentfill}{rgb}{0.124780,0.640461,0.527068}%
\pgfsetfillcolor{currentfill}%
\pgfsetlinewidth{0.000000pt}%
\definecolor{currentstroke}{rgb}{0.000000,0.000000,0.000000}%
\pgfsetstrokecolor{currentstroke}%
\pgfsetdash{}{0pt}%
\pgfpathmoveto{\pgfqpoint{4.004961in}{0.912628in}}%
\pgfpathlineto{\pgfqpoint{4.004539in}{0.921399in}}%
\pgfpathlineto{\pgfqpoint{4.003617in}{0.930396in}}%
\pgfpathlineto{\pgfqpoint{4.002197in}{0.939584in}}%
\pgfpathlineto{\pgfqpoint{4.000287in}{0.948927in}}%
\pgfpathlineto{\pgfqpoint{3.997893in}{0.958389in}}%
\pgfpathlineto{\pgfqpoint{3.970790in}{0.947045in}}%
\pgfpathlineto{\pgfqpoint{3.942407in}{0.936577in}}%
\pgfpathlineto{\pgfqpoint{3.912853in}{0.927018in}}%
\pgfpathlineto{\pgfqpoint{3.882238in}{0.918400in}}%
\pgfpathlineto{\pgfqpoint{3.850678in}{0.910749in}}%
\pgfpathlineto{\pgfqpoint{3.852135in}{0.901061in}}%
\pgfpathlineto{\pgfqpoint{3.853297in}{0.891552in}}%
\pgfpathlineto{\pgfqpoint{3.854160in}{0.882258in}}%
\pgfpathlineto{\pgfqpoint{3.854721in}{0.873216in}}%
\pgfpathlineto{\pgfqpoint{3.854978in}{0.864460in}}%
\pgfpathlineto{\pgfqpoint{3.887126in}{0.872195in}}%
\pgfpathlineto{\pgfqpoint{3.918313in}{0.880909in}}%
\pgfpathlineto{\pgfqpoint{3.948423in}{0.890573in}}%
\pgfpathlineto{\pgfqpoint{3.977342in}{0.901157in}}%
\pgfpathlineto{\pgfqpoint{4.004961in}{0.912628in}}%
\pgfpathclose%
\pgfusepath{fill}%
\end{pgfscope}%
\begin{pgfscope}%
\pgfpathrectangle{\pgfqpoint{2.548318in}{0.050000in}}{\pgfqpoint{2.081932in}{2.081932in}}%
\pgfusepath{clip}%
\pgfsetbuttcap%
\pgfsetroundjoin%
\definecolor{currentfill}{rgb}{0.206756,0.371758,0.553117}%
\pgfsetfillcolor{currentfill}%
\pgfsetlinewidth{0.000000pt}%
\definecolor{currentstroke}{rgb}{0.000000,0.000000,0.000000}%
\pgfsetstrokecolor{currentstroke}%
\pgfsetdash{}{0pt}%
\pgfpathmoveto{\pgfqpoint{3.512425in}{0.784990in}}%
\pgfpathlineto{\pgfqpoint{3.511308in}{0.789568in}}%
\pgfpathlineto{\pgfqpoint{3.510319in}{0.794713in}}%
\pgfpathlineto{\pgfqpoint{3.509462in}{0.800402in}}%
\pgfpathlineto{\pgfqpoint{3.508739in}{0.806612in}}%
\pgfpathlineto{\pgfqpoint{3.508153in}{0.813319in}}%
\pgfpathlineto{\pgfqpoint{3.473759in}{0.817150in}}%
\pgfpathlineto{\pgfqpoint{3.439872in}{0.822018in}}%
\pgfpathlineto{\pgfqpoint{3.406614in}{0.827908in}}%
\pgfpathlineto{\pgfqpoint{3.374104in}{0.834801in}}%
\pgfpathlineto{\pgfqpoint{3.342459in}{0.842677in}}%
\pgfpathlineto{\pgfqpoint{3.343940in}{0.835781in}}%
\pgfpathlineto{\pgfqpoint{3.345767in}{0.829347in}}%
\pgfpathlineto{\pgfqpoint{3.347933in}{0.823400in}}%
\pgfpathlineto{\pgfqpoint{3.350432in}{0.817966in}}%
\pgfpathlineto{\pgfqpoint{3.353253in}{0.813067in}}%
\pgfpathlineto{\pgfqpoint{3.383644in}{0.805536in}}%
\pgfpathlineto{\pgfqpoint{3.414871in}{0.798944in}}%
\pgfpathlineto{\pgfqpoint{3.446821in}{0.793311in}}%
\pgfpathlineto{\pgfqpoint{3.479378in}{0.788655in}}%
\pgfpathlineto{\pgfqpoint{3.512425in}{0.784990in}}%
\pgfpathclose%
\pgfusepath{fill}%
\end{pgfscope}%
\begin{pgfscope}%
\pgfpathrectangle{\pgfqpoint{2.548318in}{0.050000in}}{\pgfqpoint{2.081932in}{2.081932in}}%
\pgfusepath{clip}%
\pgfsetbuttcap%
\pgfsetroundjoin%
\definecolor{currentfill}{rgb}{0.993248,0.906157,0.143936}%
\pgfsetfillcolor{currentfill}%
\pgfsetlinewidth{0.000000pt}%
\definecolor{currentstroke}{rgb}{0.000000,0.000000,0.000000}%
\pgfsetstrokecolor{currentstroke}%
\pgfsetdash{}{0pt}%
\pgfpathmoveto{\pgfqpoint{3.799807in}{1.051687in}}%
\pgfpathlineto{\pgfqpoint{3.795074in}{1.059118in}}%
\pgfpathlineto{\pgfqpoint{3.790267in}{1.066113in}}%
\pgfpathlineto{\pgfqpoint{3.785403in}{1.072642in}}%
\pgfpathlineto{\pgfqpoint{3.780502in}{1.078680in}}%
\pgfpathlineto{\pgfqpoint{3.775582in}{1.084201in}}%
\pgfpathlineto{\pgfqpoint{3.753671in}{1.079560in}}%
\pgfpathlineto{\pgfqpoint{3.731255in}{1.075620in}}%
\pgfpathlineto{\pgfqpoint{3.708420in}{1.072396in}}%
\pgfpathlineto{\pgfqpoint{3.685249in}{1.069899in}}%
\pgfpathlineto{\pgfqpoint{3.661828in}{1.068136in}}%
\pgfpathlineto{\pgfqpoint{3.663203in}{1.062130in}}%
\pgfpathlineto{\pgfqpoint{3.664573in}{1.055611in}}%
\pgfpathlineto{\pgfqpoint{3.665933in}{1.048608in}}%
\pgfpathlineto{\pgfqpoint{3.667276in}{1.041147in}}%
\pgfpathlineto{\pgfqpoint{3.668599in}{1.033260in}}%
\pgfpathlineto{\pgfqpoint{3.695602in}{1.035281in}}%
\pgfpathlineto{\pgfqpoint{3.722321in}{1.038146in}}%
\pgfpathlineto{\pgfqpoint{3.748660in}{1.041843in}}%
\pgfpathlineto{\pgfqpoint{3.774520in}{1.046362in}}%
\pgfpathlineto{\pgfqpoint{3.799807in}{1.051687in}}%
\pgfpathclose%
\pgfusepath{fill}%
\end{pgfscope}%
\begin{pgfscope}%
\pgfpathrectangle{\pgfqpoint{2.548318in}{0.050000in}}{\pgfqpoint{2.081932in}{2.081932in}}%
\pgfusepath{clip}%
\pgfsetbuttcap%
\pgfsetroundjoin%
\definecolor{currentfill}{rgb}{0.993248,0.906157,0.143936}%
\pgfsetfillcolor{currentfill}%
\pgfsetlinewidth{0.000000pt}%
\definecolor{currentstroke}{rgb}{0.000000,0.000000,0.000000}%
\pgfsetstrokecolor{currentstroke}%
\pgfsetdash{}{0pt}%
\pgfpathmoveto{\pgfqpoint{3.532777in}{1.035924in}}%
\pgfpathlineto{\pgfqpoint{3.534972in}{1.043746in}}%
\pgfpathlineto{\pgfqpoint{3.537202in}{1.051139in}}%
\pgfpathlineto{\pgfqpoint{3.539458in}{1.058074in}}%
\pgfpathlineto{\pgfqpoint{3.541733in}{1.064523in}}%
\pgfpathlineto{\pgfqpoint{3.544015in}{1.070459in}}%
\pgfpathlineto{\pgfqpoint{3.520919in}{1.073141in}}%
\pgfpathlineto{\pgfqpoint{3.498180in}{1.076546in}}%
\pgfpathlineto{\pgfqpoint{3.475882in}{1.080664in}}%
\pgfpathlineto{\pgfqpoint{3.454109in}{1.085480in}}%
\pgfpathlineto{\pgfqpoint{3.432942in}{1.090978in}}%
\pgfpathlineto{\pgfqpoint{3.427191in}{1.085662in}}%
\pgfpathlineto{\pgfqpoint{3.421460in}{1.079828in}}%
\pgfpathlineto{\pgfqpoint{3.415774in}{1.073499in}}%
\pgfpathlineto{\pgfqpoint{3.410153in}{1.066702in}}%
\pgfpathlineto{\pgfqpoint{3.404619in}{1.059465in}}%
\pgfpathlineto{\pgfqpoint{3.429058in}{1.053155in}}%
\pgfpathlineto{\pgfqpoint{3.454188in}{1.047629in}}%
\pgfpathlineto{\pgfqpoint{3.479915in}{1.042905in}}%
\pgfpathlineto{\pgfqpoint{3.506143in}{1.038999in}}%
\pgfpathlineto{\pgfqpoint{3.532777in}{1.035924in}}%
\pgfpathclose%
\pgfusepath{fill}%
\end{pgfscope}%
\begin{pgfscope}%
\pgfpathrectangle{\pgfqpoint{2.548318in}{0.050000in}}{\pgfqpoint{2.081932in}{2.081932in}}%
\pgfusepath{clip}%
\pgfsetbuttcap%
\pgfsetroundjoin%
\definecolor{currentfill}{rgb}{0.327796,0.773980,0.406640}%
\pgfsetfillcolor{currentfill}%
\pgfsetlinewidth{0.000000pt}%
\definecolor{currentstroke}{rgb}{0.000000,0.000000,0.000000}%
\pgfsetstrokecolor{currentstroke}%
\pgfsetdash{}{0pt}%
\pgfpathmoveto{\pgfqpoint{3.345111in}{0.920482in}}%
\pgfpathlineto{\pgfqpoint{3.347152in}{0.930253in}}%
\pgfpathlineto{\pgfqpoint{3.349523in}{0.940115in}}%
\pgfpathlineto{\pgfqpoint{3.352213in}{0.950031in}}%
\pgfpathlineto{\pgfqpoint{3.355213in}{0.959962in}}%
\pgfpathlineto{\pgfqpoint{3.358511in}{0.969869in}}%
\pgfpathlineto{\pgfqpoint{3.329663in}{0.978357in}}%
\pgfpathlineto{\pgfqpoint{3.301855in}{0.987737in}}%
\pgfpathlineto{\pgfqpoint{3.275192in}{0.997979in}}%
\pgfpathlineto{\pgfqpoint{3.249779in}{1.009049in}}%
\pgfpathlineto{\pgfqpoint{3.225715in}{1.020913in}}%
\pgfpathlineto{\pgfqpoint{3.220685in}{1.011580in}}%
\pgfpathlineto{\pgfqpoint{3.216109in}{1.002162in}}%
\pgfpathlineto{\pgfqpoint{3.212005in}{0.992697in}}%
\pgfpathlineto{\pgfqpoint{3.208388in}{0.983221in}}%
\pgfpathlineto{\pgfqpoint{3.205273in}{0.973771in}}%
\pgfpathlineto{\pgfqpoint{3.230631in}{0.961381in}}%
\pgfpathlineto{\pgfqpoint{3.257400in}{0.949822in}}%
\pgfpathlineto{\pgfqpoint{3.285475in}{0.939130in}}%
\pgfpathlineto{\pgfqpoint{3.314749in}{0.929340in}}%
\pgfpathlineto{\pgfqpoint{3.345111in}{0.920482in}}%
\pgfpathclose%
\pgfusepath{fill}%
\end{pgfscope}%
\begin{pgfscope}%
\pgfpathrectangle{\pgfqpoint{2.548318in}{0.050000in}}{\pgfqpoint{2.081932in}{2.081932in}}%
\pgfusepath{clip}%
\pgfsetbuttcap%
\pgfsetroundjoin%
\definecolor{currentfill}{rgb}{0.876168,0.891125,0.095250}%
\pgfsetfillcolor{currentfill}%
\pgfsetlinewidth{0.000000pt}%
\definecolor{currentstroke}{rgb}{0.000000,0.000000,0.000000}%
\pgfsetstrokecolor{currentstroke}%
\pgfsetdash{}{0pt}%
\pgfpathmoveto{\pgfqpoint{3.950332in}{1.051412in}}%
\pgfpathlineto{\pgfqpoint{3.943620in}{1.059756in}}%
\pgfpathlineto{\pgfqpoint{3.936644in}{1.067789in}}%
\pgfpathlineto{\pgfqpoint{3.929432in}{1.075478in}}%
\pgfpathlineto{\pgfqpoint{3.922011in}{1.082793in}}%
\pgfpathlineto{\pgfqpoint{3.914410in}{1.089705in}}%
\pgfpathlineto{\pgfqpoint{3.893364in}{1.080663in}}%
\pgfpathlineto{\pgfqpoint{3.871294in}{1.072314in}}%
\pgfpathlineto{\pgfqpoint{3.848285in}{1.064685in}}%
\pgfpathlineto{\pgfqpoint{3.824426in}{1.057802in}}%
\pgfpathlineto{\pgfqpoint{3.799807in}{1.051687in}}%
\pgfpathlineto{\pgfqpoint{3.804445in}{1.043850in}}%
\pgfpathlineto{\pgfqpoint{3.808973in}{1.035637in}}%
\pgfpathlineto{\pgfqpoint{3.813372in}{1.027081in}}%
\pgfpathlineto{\pgfqpoint{3.817625in}{1.018216in}}%
\pgfpathlineto{\pgfqpoint{3.821717in}{1.009077in}}%
\pgfpathlineto{\pgfqpoint{3.849321in}{1.015881in}}%
\pgfpathlineto{\pgfqpoint{3.876085in}{1.023544in}}%
\pgfpathlineto{\pgfqpoint{3.901906in}{1.032038in}}%
\pgfpathlineto{\pgfqpoint{3.926687in}{1.041338in}}%
\pgfpathlineto{\pgfqpoint{3.950332in}{1.051412in}}%
\pgfpathclose%
\pgfusepath{fill}%
\end{pgfscope}%
\begin{pgfscope}%
\pgfpathrectangle{\pgfqpoint{2.548318in}{0.050000in}}{\pgfqpoint{2.081932in}{2.081932in}}%
\pgfusepath{clip}%
\pgfsetbuttcap%
\pgfsetroundjoin%
\definecolor{currentfill}{rgb}{0.636902,0.856542,0.216620}%
\pgfsetfillcolor{currentfill}%
\pgfsetlinewidth{0.000000pt}%
\definecolor{currentstroke}{rgb}{0.000000,0.000000,0.000000}%
\pgfsetstrokecolor{currentstroke}%
\pgfsetdash{}{0pt}%
\pgfpathmoveto{\pgfqpoint{3.358511in}{0.969869in}}%
\pgfpathlineto{\pgfqpoint{3.362095in}{0.979715in}}%
\pgfpathlineto{\pgfqpoint{3.365950in}{0.989462in}}%
\pgfpathlineto{\pgfqpoint{3.370062in}{0.999072in}}%
\pgfpathlineto{\pgfqpoint{3.374416in}{1.008508in}}%
\pgfpathlineto{\pgfqpoint{3.378994in}{1.017732in}}%
\pgfpathlineto{\pgfqpoint{3.352455in}{1.025607in}}%
\pgfpathlineto{\pgfqpoint{3.326883in}{1.034307in}}%
\pgfpathlineto{\pgfqpoint{3.302375in}{1.043805in}}%
\pgfpathlineto{\pgfqpoint{3.279027in}{1.054068in}}%
\pgfpathlineto{\pgfqpoint{3.256929in}{1.065063in}}%
\pgfpathlineto{\pgfqpoint{3.249956in}{1.056685in}}%
\pgfpathlineto{\pgfqpoint{3.243322in}{1.048046in}}%
\pgfpathlineto{\pgfqpoint{3.237055in}{1.039181in}}%
\pgfpathlineto{\pgfqpoint{3.231179in}{1.030125in}}%
\pgfpathlineto{\pgfqpoint{3.225715in}{1.020913in}}%
\pgfpathlineto{\pgfqpoint{3.249779in}{1.009049in}}%
\pgfpathlineto{\pgfqpoint{3.275192in}{0.997979in}}%
\pgfpathlineto{\pgfqpoint{3.301855in}{0.987737in}}%
\pgfpathlineto{\pgfqpoint{3.329663in}{0.978357in}}%
\pgfpathlineto{\pgfqpoint{3.358511in}{0.969869in}}%
\pgfpathclose%
\pgfusepath{fill}%
\end{pgfscope}%
\begin{pgfscope}%
\pgfpathrectangle{\pgfqpoint{2.548318in}{0.050000in}}{\pgfqpoint{2.081932in}{2.081932in}}%
\pgfusepath{clip}%
\pgfsetbuttcap%
\pgfsetroundjoin%
\definecolor{currentfill}{rgb}{0.267968,0.223549,0.512008}%
\pgfsetfillcolor{currentfill}%
\pgfsetlinewidth{0.000000pt}%
\definecolor{currentstroke}{rgb}{0.000000,0.000000,0.000000}%
\pgfsetstrokecolor{currentstroke}%
\pgfsetdash{}{0pt}%
\pgfpathmoveto{\pgfqpoint{3.676437in}{0.768246in}}%
\pgfpathlineto{\pgfqpoint{3.677453in}{0.769695in}}%
\pgfpathlineto{\pgfqpoint{3.678406in}{0.771788in}}%
\pgfpathlineto{\pgfqpoint{3.679294in}{0.774515in}}%
\pgfpathlineto{\pgfqpoint{3.680113in}{0.777862in}}%
\pgfpathlineto{\pgfqpoint{3.680861in}{0.781816in}}%
\pgfpathlineto{\pgfqpoint{3.647144in}{0.780419in}}%
\pgfpathlineto{\pgfqpoint{3.613323in}{0.780038in}}%
\pgfpathlineto{\pgfqpoint{3.579516in}{0.780675in}}%
\pgfpathlineto{\pgfqpoint{3.545844in}{0.782327in}}%
\pgfpathlineto{\pgfqpoint{3.512425in}{0.784990in}}%
\pgfpathlineto{\pgfqpoint{3.513666in}{0.780997in}}%
\pgfpathlineto{\pgfqpoint{3.515026in}{0.777607in}}%
\pgfpathlineto{\pgfqpoint{3.516500in}{0.774835in}}%
\pgfpathlineto{\pgfqpoint{3.518083in}{0.772693in}}%
\pgfpathlineto{\pgfqpoint{3.519769in}{0.771192in}}%
\pgfpathlineto{\pgfqpoint{3.550851in}{0.768721in}}%
\pgfpathlineto{\pgfqpoint{3.582170in}{0.767187in}}%
\pgfpathlineto{\pgfqpoint{3.613616in}{0.766596in}}%
\pgfpathlineto{\pgfqpoint{3.645076in}{0.766949in}}%
\pgfpathlineto{\pgfqpoint{3.676437in}{0.768246in}}%
\pgfpathclose%
\pgfusepath{fill}%
\end{pgfscope}%
\begin{pgfscope}%
\pgfpathrectangle{\pgfqpoint{2.548318in}{0.050000in}}{\pgfqpoint{2.081932in}{2.081932in}}%
\pgfusepath{clip}%
\pgfsetbuttcap%
\pgfsetroundjoin%
\definecolor{currentfill}{rgb}{0.150476,0.504369,0.557430}%
\pgfsetfillcolor{currentfill}%
\pgfsetlinewidth{0.000000pt}%
\definecolor{currentstroke}{rgb}{0.000000,0.000000,0.000000}%
\pgfsetstrokecolor{currentstroke}%
\pgfsetdash{}{0pt}%
\pgfpathmoveto{\pgfqpoint{4.001619in}{0.880482in}}%
\pgfpathlineto{\pgfqpoint{4.003206in}{0.888013in}}%
\pgfpathlineto{\pgfqpoint{4.004294in}{0.895902in}}%
\pgfpathlineto{\pgfqpoint{4.004879in}{0.904118in}}%
\pgfpathlineto{\pgfqpoint{4.004961in}{0.912628in}}%
\pgfpathlineto{\pgfqpoint{3.977342in}{0.901157in}}%
\pgfpathlineto{\pgfqpoint{3.948423in}{0.890573in}}%
\pgfpathlineto{\pgfqpoint{3.918313in}{0.880909in}}%
\pgfpathlineto{\pgfqpoint{3.887126in}{0.872195in}}%
\pgfpathlineto{\pgfqpoint{3.854978in}{0.864460in}}%
\pgfpathlineto{\pgfqpoint{3.854928in}{0.856025in}}%
\pgfpathlineto{\pgfqpoint{3.854572in}{0.847944in}}%
\pgfpathlineto{\pgfqpoint{3.853910in}{0.840249in}}%
\pgfpathlineto{\pgfqpoint{3.852946in}{0.832971in}}%
\pgfpathlineto{\pgfqpoint{3.884815in}{0.840601in}}%
\pgfpathlineto{\pgfqpoint{3.915732in}{0.849196in}}%
\pgfpathlineto{\pgfqpoint{3.945579in}{0.858728in}}%
\pgfpathlineto{\pgfqpoint{3.974245in}{0.869168in}}%
\pgfpathlineto{\pgfqpoint{4.001619in}{0.880482in}}%
\pgfpathclose%
\pgfusepath{fill}%
\end{pgfscope}%
\begin{pgfscope}%
\pgfpathrectangle{\pgfqpoint{2.548318in}{0.050000in}}{\pgfqpoint{2.081932in}{2.081932in}}%
\pgfusepath{clip}%
\pgfsetbuttcap%
\pgfsetroundjoin%
\definecolor{currentfill}{rgb}{0.124780,0.640461,0.527068}%
\pgfsetfillcolor{currentfill}%
\pgfsetlinewidth{0.000000pt}%
\definecolor{currentstroke}{rgb}{0.000000,0.000000,0.000000}%
\pgfsetstrokecolor{currentstroke}%
\pgfsetdash{}{0pt}%
\pgfpathmoveto{\pgfqpoint{3.340081in}{0.874300in}}%
\pgfpathlineto{\pgfqpoint{3.340380in}{0.883059in}}%
\pgfpathlineto{\pgfqpoint{3.341037in}{0.892092in}}%
\pgfpathlineto{\pgfqpoint{3.342047in}{0.901364in}}%
\pgfpathlineto{\pgfqpoint{3.343407in}{0.910840in}}%
\pgfpathlineto{\pgfqpoint{3.345111in}{0.920482in}}%
\pgfpathlineto{\pgfqpoint{3.314749in}{0.929340in}}%
\pgfpathlineto{\pgfqpoint{3.285475in}{0.939130in}}%
\pgfpathlineto{\pgfqpoint{3.257400in}{0.949822in}}%
\pgfpathlineto{\pgfqpoint{3.230631in}{0.961381in}}%
\pgfpathlineto{\pgfqpoint{3.205273in}{0.973771in}}%
\pgfpathlineto{\pgfqpoint{3.202673in}{0.964383in}}%
\pgfpathlineto{\pgfqpoint{3.200597in}{0.955094in}}%
\pgfpathlineto{\pgfqpoint{3.199055in}{0.945941in}}%
\pgfpathlineto{\pgfqpoint{3.198052in}{0.936959in}}%
\pgfpathlineto{\pgfqpoint{3.197594in}{0.928183in}}%
\pgfpathlineto{\pgfqpoint{3.223438in}{0.915654in}}%
\pgfpathlineto{\pgfqpoint{3.250717in}{0.903965in}}%
\pgfpathlineto{\pgfqpoint{3.279325in}{0.893154in}}%
\pgfpathlineto{\pgfqpoint{3.309150in}{0.883255in}}%
\pgfpathlineto{\pgfqpoint{3.340081in}{0.874300in}}%
\pgfpathclose%
\pgfusepath{fill}%
\end{pgfscope}%
\begin{pgfscope}%
\pgfpathrectangle{\pgfqpoint{2.548318in}{0.050000in}}{\pgfqpoint{2.081932in}{2.081932in}}%
\pgfusepath{clip}%
\pgfsetbuttcap%
\pgfsetroundjoin%
\definecolor{currentfill}{rgb}{0.876168,0.891125,0.095250}%
\pgfsetfillcolor{currentfill}%
\pgfsetlinewidth{0.000000pt}%
\definecolor{currentstroke}{rgb}{0.000000,0.000000,0.000000}%
\pgfsetstrokecolor{currentstroke}%
\pgfsetdash{}{0pt}%
\pgfpathmoveto{\pgfqpoint{3.378994in}{1.017732in}}%
\pgfpathlineto{\pgfqpoint{3.383780in}{1.026710in}}%
\pgfpathlineto{\pgfqpoint{3.388755in}{1.035406in}}%
\pgfpathlineto{\pgfqpoint{3.393899in}{1.043785in}}%
\pgfpathlineto{\pgfqpoint{3.399194in}{1.051815in}}%
\pgfpathlineto{\pgfqpoint{3.404619in}{1.059465in}}%
\pgfpathlineto{\pgfqpoint{3.380962in}{1.066538in}}%
\pgfpathlineto{\pgfqpoint{3.358178in}{1.074351in}}%
\pgfpathlineto{\pgfqpoint{3.336354in}{1.082878in}}%
\pgfpathlineto{\pgfqpoint{3.315575in}{1.092089in}}%
\pgfpathlineto{\pgfqpoint{3.295923in}{1.101953in}}%
\pgfpathlineto{\pgfqpoint{3.287673in}{1.095342in}}%
\pgfpathlineto{\pgfqpoint{3.279618in}{1.088319in}}%
\pgfpathlineto{\pgfqpoint{3.271789in}{1.080912in}}%
\pgfpathlineto{\pgfqpoint{3.264216in}{1.073150in}}%
\pgfpathlineto{\pgfqpoint{3.256929in}{1.065063in}}%
\pgfpathlineto{\pgfqpoint{3.279027in}{1.054068in}}%
\pgfpathlineto{\pgfqpoint{3.302375in}{1.043805in}}%
\pgfpathlineto{\pgfqpoint{3.326883in}{1.034307in}}%
\pgfpathlineto{\pgfqpoint{3.352455in}{1.025607in}}%
\pgfpathlineto{\pgfqpoint{3.378994in}{1.017732in}}%
\pgfpathclose%
\pgfusepath{fill}%
\end{pgfscope}%
\begin{pgfscope}%
\pgfpathrectangle{\pgfqpoint{2.548318in}{0.050000in}}{\pgfqpoint{2.081932in}{2.081932in}}%
\pgfusepath{clip}%
\pgfsetbuttcap%
\pgfsetroundjoin%
\definecolor{currentfill}{rgb}{0.267968,0.223549,0.512008}%
\pgfsetfillcolor{currentfill}%
\pgfsetlinewidth{0.000000pt}%
\definecolor{currentstroke}{rgb}{0.000000,0.000000,0.000000}%
\pgfsetstrokecolor{currentstroke}%
\pgfsetdash{}{0pt}%
\pgfpathmoveto{\pgfqpoint{3.827872in}{0.788631in}}%
\pgfpathlineto{\pgfqpoint{3.831509in}{0.790438in}}%
\pgfpathlineto{\pgfqpoint{3.834926in}{0.792870in}}%
\pgfpathlineto{\pgfqpoint{3.838108in}{0.795914in}}%
\pgfpathlineto{\pgfqpoint{3.841043in}{0.799558in}}%
\pgfpathlineto{\pgfqpoint{3.843722in}{0.803786in}}%
\pgfpathlineto{\pgfqpoint{3.812304in}{0.797433in}}%
\pgfpathlineto{\pgfqpoint{3.780193in}{0.792045in}}%
\pgfpathlineto{\pgfqpoint{3.747504in}{0.787638in}}%
\pgfpathlineto{\pgfqpoint{3.714354in}{0.784224in}}%
\pgfpathlineto{\pgfqpoint{3.680861in}{0.781816in}}%
\pgfpathlineto{\pgfqpoint{3.680113in}{0.777862in}}%
\pgfpathlineto{\pgfqpoint{3.679294in}{0.774515in}}%
\pgfpathlineto{\pgfqpoint{3.678406in}{0.771788in}}%
\pgfpathlineto{\pgfqpoint{3.677453in}{0.769695in}}%
\pgfpathlineto{\pgfqpoint{3.676437in}{0.768246in}}%
\pgfpathlineto{\pgfqpoint{3.707589in}{0.770481in}}%
\pgfpathlineto{\pgfqpoint{3.738418in}{0.773649in}}%
\pgfpathlineto{\pgfqpoint{3.768815in}{0.777739in}}%
\pgfpathlineto{\pgfqpoint{3.798669in}{0.782738in}}%
\pgfpathlineto{\pgfqpoint{3.827872in}{0.788631in}}%
\pgfpathclose%
\pgfusepath{fill}%
\end{pgfscope}%
\begin{pgfscope}%
\pgfpathrectangle{\pgfqpoint{2.548318in}{0.050000in}}{\pgfqpoint{2.081932in}{2.081932in}}%
\pgfusepath{clip}%
\pgfsetbuttcap%
\pgfsetroundjoin%
\definecolor{currentfill}{rgb}{0.993248,0.906157,0.143936}%
\pgfsetfillcolor{currentfill}%
\pgfsetlinewidth{0.000000pt}%
\definecolor{currentstroke}{rgb}{0.000000,0.000000,0.000000}%
\pgfsetstrokecolor{currentstroke}%
\pgfsetdash{}{0pt}%
\pgfpathmoveto{\pgfqpoint{3.661828in}{1.068136in}}%
\pgfpathlineto{\pgfqpoint{3.660452in}{1.073604in}}%
\pgfpathlineto{\pgfqpoint{3.659082in}{1.078512in}}%
\pgfpathlineto{\pgfqpoint{3.657724in}{1.082837in}}%
\pgfpathlineto{\pgfqpoint{3.656381in}{1.086563in}}%
\pgfpathlineto{\pgfqpoint{3.655061in}{1.089671in}}%
\pgfpathlineto{\pgfqpoint{3.635079in}{1.088801in}}%
\pgfpathlineto{\pgfqpoint{3.615032in}{1.088565in}}%
\pgfpathlineto{\pgfqpoint{3.594994in}{1.088961in}}%
\pgfpathlineto{\pgfqpoint{3.575041in}{1.089989in}}%
\pgfpathlineto{\pgfqpoint{3.555246in}{1.091646in}}%
\pgfpathlineto{\pgfqpoint{3.553054in}{1.088606in}}%
\pgfpathlineto{\pgfqpoint{3.550827in}{1.084950in}}%
\pgfpathlineto{\pgfqpoint{3.548571in}{1.080695in}}%
\pgfpathlineto{\pgfqpoint{3.546298in}{1.075857in}}%
\pgfpathlineto{\pgfqpoint{3.544015in}{1.070459in}}%
\pgfpathlineto{\pgfqpoint{3.567382in}{1.068510in}}%
\pgfpathlineto{\pgfqpoint{3.590934in}{1.067301in}}%
\pgfpathlineto{\pgfqpoint{3.614583in}{1.066834in}}%
\pgfpathlineto{\pgfqpoint{3.638244in}{1.067113in}}%
\pgfpathlineto{\pgfqpoint{3.661828in}{1.068136in}}%
\pgfpathclose%
\pgfusepath{fill}%
\end{pgfscope}%
\begin{pgfscope}%
\pgfpathrectangle{\pgfqpoint{2.548318in}{0.050000in}}{\pgfqpoint{2.081932in}{2.081932in}}%
\pgfusepath{clip}%
\pgfsetbuttcap%
\pgfsetroundjoin%
\definecolor{currentfill}{rgb}{0.150476,0.504369,0.557430}%
\pgfsetfillcolor{currentfill}%
\pgfsetlinewidth{0.000000pt}%
\definecolor{currentstroke}{rgb}{0.000000,0.000000,0.000000}%
\pgfsetstrokecolor{currentstroke}%
\pgfsetdash{}{0pt}%
\pgfpathmoveto{\pgfqpoint{3.342459in}{0.842677in}}%
\pgfpathlineto{\pgfqpoint{3.341330in}{0.850007in}}%
\pgfpathlineto{\pgfqpoint{3.340556in}{0.857741in}}%
\pgfpathlineto{\pgfqpoint{3.340139in}{0.865850in}}%
\pgfpathlineto{\pgfqpoint{3.340081in}{0.874300in}}%
\pgfpathlineto{\pgfqpoint{3.309150in}{0.883255in}}%
\pgfpathlineto{\pgfqpoint{3.279325in}{0.893154in}}%
\pgfpathlineto{\pgfqpoint{3.250717in}{0.903965in}}%
\pgfpathlineto{\pgfqpoint{3.223438in}{0.915654in}}%
\pgfpathlineto{\pgfqpoint{3.197594in}{0.928183in}}%
\pgfpathlineto{\pgfqpoint{3.197683in}{0.919649in}}%
\pgfpathlineto{\pgfqpoint{3.198320in}{0.911389in}}%
\pgfpathlineto{\pgfqpoint{3.199501in}{0.903437in}}%
\pgfpathlineto{\pgfqpoint{3.201225in}{0.895824in}}%
\pgfpathlineto{\pgfqpoint{3.226839in}{0.883467in}}%
\pgfpathlineto{\pgfqpoint{3.253877in}{0.871938in}}%
\pgfpathlineto{\pgfqpoint{3.282233in}{0.861275in}}%
\pgfpathlineto{\pgfqpoint{3.311798in}{0.851511in}}%
\pgfpathlineto{\pgfqpoint{3.342459in}{0.842677in}}%
\pgfpathclose%
\pgfusepath{fill}%
\end{pgfscope}%
\begin{pgfscope}%
\pgfpathrectangle{\pgfqpoint{2.548318in}{0.050000in}}{\pgfqpoint{2.081932in}{2.081932in}}%
\pgfusepath{clip}%
\pgfsetbuttcap%
\pgfsetroundjoin%
\definecolor{currentfill}{rgb}{0.267968,0.223549,0.512008}%
\pgfsetfillcolor{currentfill}%
\pgfsetlinewidth{0.000000pt}%
\definecolor{currentstroke}{rgb}{0.000000,0.000000,0.000000}%
\pgfsetstrokecolor{currentstroke}%
\pgfsetdash{}{0pt}%
\pgfpathmoveto{\pgfqpoint{3.519769in}{0.771192in}}%
\pgfpathlineto{\pgfqpoint{3.518083in}{0.772693in}}%
\pgfpathlineto{\pgfqpoint{3.516500in}{0.774835in}}%
\pgfpathlineto{\pgfqpoint{3.515026in}{0.777607in}}%
\pgfpathlineto{\pgfqpoint{3.513666in}{0.780997in}}%
\pgfpathlineto{\pgfqpoint{3.512425in}{0.784990in}}%
\pgfpathlineto{\pgfqpoint{3.479378in}{0.788655in}}%
\pgfpathlineto{\pgfqpoint{3.446821in}{0.793311in}}%
\pgfpathlineto{\pgfqpoint{3.414871in}{0.798944in}}%
\pgfpathlineto{\pgfqpoint{3.383644in}{0.805536in}}%
\pgfpathlineto{\pgfqpoint{3.353253in}{0.813067in}}%
\pgfpathlineto{\pgfqpoint{3.356387in}{0.808723in}}%
\pgfpathlineto{\pgfqpoint{3.359822in}{0.804954in}}%
\pgfpathlineto{\pgfqpoint{3.363545in}{0.801774in}}%
\pgfpathlineto{\pgfqpoint{3.367541in}{0.799199in}}%
\pgfpathlineto{\pgfqpoint{3.371797in}{0.797240in}}%
\pgfpathlineto{\pgfqpoint{3.400037in}{0.790255in}}%
\pgfpathlineto{\pgfqpoint{3.429061in}{0.784140in}}%
\pgfpathlineto{\pgfqpoint{3.458764in}{0.778913in}}%
\pgfpathlineto{\pgfqpoint{3.489036in}{0.774593in}}%
\pgfpathlineto{\pgfqpoint{3.519769in}{0.771192in}}%
\pgfpathclose%
\pgfusepath{fill}%
\end{pgfscope}%
\begin{pgfscope}%
\pgfpathrectangle{\pgfqpoint{2.548318in}{0.050000in}}{\pgfqpoint{2.081932in}{2.081932in}}%
\pgfusepath{clip}%
\pgfsetbuttcap%
\pgfsetroundjoin%
\definecolor{currentfill}{rgb}{0.993248,0.906157,0.143936}%
\pgfsetfillcolor{currentfill}%
\pgfsetlinewidth{0.000000pt}%
\definecolor{currentstroke}{rgb}{0.000000,0.000000,0.000000}%
\pgfsetstrokecolor{currentstroke}%
\pgfsetdash{}{0pt}%
\pgfpathmoveto{\pgfqpoint{3.914410in}{1.089705in}}%
\pgfpathlineto{\pgfqpoint{3.906658in}{1.096186in}}%
\pgfpathlineto{\pgfqpoint{3.898786in}{1.102210in}}%
\pgfpathlineto{\pgfqpoint{3.890823in}{1.107753in}}%
\pgfpathlineto{\pgfqpoint{3.882802in}{1.112793in}}%
\pgfpathlineto{\pgfqpoint{3.874754in}{1.117307in}}%
\pgfpathlineto{\pgfqpoint{3.856563in}{1.109438in}}%
\pgfpathlineto{\pgfqpoint{3.837475in}{1.102169in}}%
\pgfpathlineto{\pgfqpoint{3.817564in}{1.095525in}}%
\pgfpathlineto{\pgfqpoint{3.796907in}{1.089529in}}%
\pgfpathlineto{\pgfqpoint{3.775582in}{1.084201in}}%
\pgfpathlineto{\pgfqpoint{3.780502in}{1.078680in}}%
\pgfpathlineto{\pgfqpoint{3.785403in}{1.072642in}}%
\pgfpathlineto{\pgfqpoint{3.790267in}{1.066113in}}%
\pgfpathlineto{\pgfqpoint{3.795074in}{1.059118in}}%
\pgfpathlineto{\pgfqpoint{3.799807in}{1.051687in}}%
\pgfpathlineto{\pgfqpoint{3.824426in}{1.057802in}}%
\pgfpathlineto{\pgfqpoint{3.848285in}{1.064685in}}%
\pgfpathlineto{\pgfqpoint{3.871294in}{1.072314in}}%
\pgfpathlineto{\pgfqpoint{3.893364in}{1.080663in}}%
\pgfpathlineto{\pgfqpoint{3.914410in}{1.089705in}}%
\pgfpathclose%
\pgfusepath{fill}%
\end{pgfscope}%
\begin{pgfscope}%
\pgfpathrectangle{\pgfqpoint{2.548318in}{0.050000in}}{\pgfqpoint{2.081932in}{2.081932in}}%
\pgfusepath{clip}%
\pgfsetbuttcap%
\pgfsetroundjoin%
\definecolor{currentfill}{rgb}{0.993248,0.906157,0.143936}%
\pgfsetfillcolor{currentfill}%
\pgfsetlinewidth{0.000000pt}%
\definecolor{currentstroke}{rgb}{0.000000,0.000000,0.000000}%
\pgfsetstrokecolor{currentstroke}%
\pgfsetdash{}{0pt}%
\pgfpathmoveto{\pgfqpoint{3.775582in}{1.084201in}}%
\pgfpathlineto{\pgfqpoint{3.770664in}{1.089183in}}%
\pgfpathlineto{\pgfqpoint{3.765766in}{1.093604in}}%
\pgfpathlineto{\pgfqpoint{3.760908in}{1.097445in}}%
\pgfpathlineto{\pgfqpoint{3.756109in}{1.100689in}}%
\pgfpathlineto{\pgfqpoint{3.751390in}{1.103323in}}%
\pgfpathlineto{\pgfqpoint{3.732845in}{1.099380in}}%
\pgfpathlineto{\pgfqpoint{3.713867in}{1.096032in}}%
\pgfpathlineto{\pgfqpoint{3.694528in}{1.093292in}}%
\pgfpathlineto{\pgfqpoint{3.674902in}{1.091169in}}%
\pgfpathlineto{\pgfqpoint{3.655061in}{1.089671in}}%
\pgfpathlineto{\pgfqpoint{3.656381in}{1.086563in}}%
\pgfpathlineto{\pgfqpoint{3.657724in}{1.082837in}}%
\pgfpathlineto{\pgfqpoint{3.659082in}{1.078512in}}%
\pgfpathlineto{\pgfqpoint{3.660452in}{1.073604in}}%
\pgfpathlineto{\pgfqpoint{3.661828in}{1.068136in}}%
\pgfpathlineto{\pgfqpoint{3.685249in}{1.069899in}}%
\pgfpathlineto{\pgfqpoint{3.708420in}{1.072396in}}%
\pgfpathlineto{\pgfqpoint{3.731255in}{1.075620in}}%
\pgfpathlineto{\pgfqpoint{3.753671in}{1.079560in}}%
\pgfpathlineto{\pgfqpoint{3.775582in}{1.084201in}}%
\pgfpathclose%
\pgfusepath{fill}%
\end{pgfscope}%
\begin{pgfscope}%
\pgfpathrectangle{\pgfqpoint{2.548318in}{0.050000in}}{\pgfqpoint{2.081932in}{2.081932in}}%
\pgfusepath{clip}%
\pgfsetbuttcap%
\pgfsetroundjoin%
\definecolor{currentfill}{rgb}{0.206756,0.371758,0.553117}%
\pgfsetfillcolor{currentfill}%
\pgfsetlinewidth{0.000000pt}%
\definecolor{currentstroke}{rgb}{0.000000,0.000000,0.000000}%
\pgfsetstrokecolor{currentstroke}%
\pgfsetdash{}{0pt}%
\pgfpathmoveto{\pgfqpoint{3.986460in}{0.849208in}}%
\pgfpathlineto{\pgfqpoint{3.990422in}{0.854524in}}%
\pgfpathlineto{\pgfqpoint{3.993930in}{0.860334in}}%
\pgfpathlineto{\pgfqpoint{3.996973in}{0.866615in}}%
\pgfpathlineto{\pgfqpoint{3.999540in}{0.873340in}}%
\pgfpathlineto{\pgfqpoint{4.001619in}{0.880482in}}%
\pgfpathlineto{\pgfqpoint{3.974245in}{0.869168in}}%
\pgfpathlineto{\pgfqpoint{3.945579in}{0.858728in}}%
\pgfpathlineto{\pgfqpoint{3.915732in}{0.849196in}}%
\pgfpathlineto{\pgfqpoint{3.884815in}{0.840601in}}%
\pgfpathlineto{\pgfqpoint{3.852946in}{0.832971in}}%
\pgfpathlineto{\pgfqpoint{3.851680in}{0.826138in}}%
\pgfpathlineto{\pgfqpoint{3.850119in}{0.819778in}}%
\pgfpathlineto{\pgfqpoint{3.848268in}{0.813917in}}%
\pgfpathlineto{\pgfqpoint{3.846133in}{0.808579in}}%
\pgfpathlineto{\pgfqpoint{3.843722in}{0.803786in}}%
\pgfpathlineto{\pgfqpoint{3.874330in}{0.811082in}}%
\pgfpathlineto{\pgfqpoint{3.904018in}{0.819300in}}%
\pgfpathlineto{\pgfqpoint{3.932674in}{0.828414in}}%
\pgfpathlineto{\pgfqpoint{3.960190in}{0.838394in}}%
\pgfpathlineto{\pgfqpoint{3.986460in}{0.849208in}}%
\pgfpathclose%
\pgfusepath{fill}%
\end{pgfscope}%
\begin{pgfscope}%
\pgfpathrectangle{\pgfqpoint{2.548318in}{0.050000in}}{\pgfqpoint{2.081932in}{2.081932in}}%
\pgfusepath{clip}%
\pgfsetbuttcap%
\pgfsetroundjoin%
\definecolor{currentfill}{rgb}{0.993248,0.906157,0.143936}%
\pgfsetfillcolor{currentfill}%
\pgfsetlinewidth{0.000000pt}%
\definecolor{currentstroke}{rgb}{0.000000,0.000000,0.000000}%
\pgfsetstrokecolor{currentstroke}%
\pgfsetdash{}{0pt}%
\pgfpathmoveto{\pgfqpoint{3.544015in}{1.070459in}}%
\pgfpathlineto{\pgfqpoint{3.546298in}{1.075857in}}%
\pgfpathlineto{\pgfqpoint{3.548571in}{1.080695in}}%
\pgfpathlineto{\pgfqpoint{3.550827in}{1.084950in}}%
\pgfpathlineto{\pgfqpoint{3.553054in}{1.088606in}}%
\pgfpathlineto{\pgfqpoint{3.555246in}{1.091646in}}%
\pgfpathlineto{\pgfqpoint{3.535684in}{1.093925in}}%
\pgfpathlineto{\pgfqpoint{3.516428in}{1.096819in}}%
\pgfpathlineto{\pgfqpoint{3.497551in}{1.100318in}}%
\pgfpathlineto{\pgfqpoint{3.479125in}{1.104409in}}%
\pgfpathlineto{\pgfqpoint{3.461219in}{1.109078in}}%
\pgfpathlineto{\pgfqpoint{3.455703in}{1.106646in}}%
\pgfpathlineto{\pgfqpoint{3.450095in}{1.103604in}}%
\pgfpathlineto{\pgfqpoint{3.444417in}{1.099968in}}%
\pgfpathlineto{\pgfqpoint{3.438691in}{1.095753in}}%
\pgfpathlineto{\pgfqpoint{3.432942in}{1.090978in}}%
\pgfpathlineto{\pgfqpoint{3.454109in}{1.085480in}}%
\pgfpathlineto{\pgfqpoint{3.475882in}{1.080664in}}%
\pgfpathlineto{\pgfqpoint{3.498180in}{1.076546in}}%
\pgfpathlineto{\pgfqpoint{3.520919in}{1.073141in}}%
\pgfpathlineto{\pgfqpoint{3.544015in}{1.070459in}}%
\pgfpathclose%
\pgfusepath{fill}%
\end{pgfscope}%
\begin{pgfscope}%
\pgfpathrectangle{\pgfqpoint{2.548318in}{0.050000in}}{\pgfqpoint{2.081932in}{2.081932in}}%
\pgfusepath{clip}%
\pgfsetbuttcap%
\pgfsetroundjoin%
\definecolor{currentfill}{rgb}{0.993248,0.906157,0.143936}%
\pgfsetfillcolor{currentfill}%
\pgfsetlinewidth{0.000000pt}%
\definecolor{currentstroke}{rgb}{0.000000,0.000000,0.000000}%
\pgfsetstrokecolor{currentstroke}%
\pgfsetdash{}{0pt}%
\pgfpathmoveto{\pgfqpoint{3.404619in}{1.059465in}}%
\pgfpathlineto{\pgfqpoint{3.410153in}{1.066702in}}%
\pgfpathlineto{\pgfqpoint{3.415774in}{1.073499in}}%
\pgfpathlineto{\pgfqpoint{3.421460in}{1.079828in}}%
\pgfpathlineto{\pgfqpoint{3.427191in}{1.085662in}}%
\pgfpathlineto{\pgfqpoint{3.432942in}{1.090978in}}%
\pgfpathlineto{\pgfqpoint{3.412462in}{1.097139in}}%
\pgfpathlineto{\pgfqpoint{3.392748in}{1.103943in}}%
\pgfpathlineto{\pgfqpoint{3.373876in}{1.111366in}}%
\pgfpathlineto{\pgfqpoint{3.355920in}{1.119381in}}%
\pgfpathlineto{\pgfqpoint{3.338950in}{1.127962in}}%
\pgfpathlineto{\pgfqpoint{3.330219in}{1.123773in}}%
\pgfpathlineto{\pgfqpoint{3.321517in}{1.119058in}}%
\pgfpathlineto{\pgfqpoint{3.312878in}{1.113835in}}%
\pgfpathlineto{\pgfqpoint{3.304335in}{1.108126in}}%
\pgfpathlineto{\pgfqpoint{3.295923in}{1.101953in}}%
\pgfpathlineto{\pgfqpoint{3.315575in}{1.092089in}}%
\pgfpathlineto{\pgfqpoint{3.336354in}{1.082878in}}%
\pgfpathlineto{\pgfqpoint{3.358178in}{1.074351in}}%
\pgfpathlineto{\pgfqpoint{3.380962in}{1.066538in}}%
\pgfpathlineto{\pgfqpoint{3.404619in}{1.059465in}}%
\pgfpathclose%
\pgfusepath{fill}%
\end{pgfscope}%
\begin{pgfscope}%
\pgfpathrectangle{\pgfqpoint{2.548318in}{0.050000in}}{\pgfqpoint{2.081932in}{2.081932in}}%
\pgfusepath{clip}%
\pgfsetbuttcap%
\pgfsetroundjoin%
\definecolor{currentfill}{rgb}{0.282327,0.094955,0.417331}%
\pgfsetfillcolor{currentfill}%
\pgfsetlinewidth{0.000000pt}%
\definecolor{currentstroke}{rgb}{0.000000,0.000000,0.000000}%
\pgfsetstrokecolor{currentstroke}%
\pgfsetdash{}{0pt}%
\pgfpathmoveto{\pgfqpoint{3.670576in}{0.770829in}}%
\pgfpathlineto{\pgfqpoint{3.671838in}{0.768998in}}%
\pgfpathlineto{\pgfqpoint{3.673060in}{0.767822in}}%
\pgfpathlineto{\pgfqpoint{3.674237in}{0.767305in}}%
\pgfpathlineto{\pgfqpoint{3.675364in}{0.767447in}}%
\pgfpathlineto{\pgfqpoint{3.676437in}{0.768246in}}%
\pgfpathlineto{\pgfqpoint{3.645076in}{0.766949in}}%
\pgfpathlineto{\pgfqpoint{3.613616in}{0.766596in}}%
\pgfpathlineto{\pgfqpoint{3.582170in}{0.767187in}}%
\pgfpathlineto{\pgfqpoint{3.550851in}{0.768721in}}%
\pgfpathlineto{\pgfqpoint{3.519769in}{0.771192in}}%
\pgfpathlineto{\pgfqpoint{3.521551in}{0.770339in}}%
\pgfpathlineto{\pgfqpoint{3.523422in}{0.770141in}}%
\pgfpathlineto{\pgfqpoint{3.525376in}{0.770599in}}%
\pgfpathlineto{\pgfqpoint{3.527405in}{0.771714in}}%
\pgfpathlineto{\pgfqpoint{3.529500in}{0.773483in}}%
\pgfpathlineto{\pgfqpoint{3.557485in}{0.771257in}}%
\pgfpathlineto{\pgfqpoint{3.585687in}{0.769876in}}%
\pgfpathlineto{\pgfqpoint{3.614005in}{0.769343in}}%
\pgfpathlineto{\pgfqpoint{3.642335in}{0.769661in}}%
\pgfpathlineto{\pgfqpoint{3.670576in}{0.770829in}}%
\pgfpathclose%
\pgfusepath{fill}%
\end{pgfscope}%
\begin{pgfscope}%
\pgfpathrectangle{\pgfqpoint{2.548318in}{0.050000in}}{\pgfqpoint{2.081932in}{2.081932in}}%
\pgfusepath{clip}%
\pgfsetbuttcap%
\pgfsetroundjoin%
\definecolor{currentfill}{rgb}{0.206756,0.371758,0.553117}%
\pgfsetfillcolor{currentfill}%
\pgfsetlinewidth{0.000000pt}%
\definecolor{currentstroke}{rgb}{0.000000,0.000000,0.000000}%
\pgfsetstrokecolor{currentstroke}%
\pgfsetdash{}{0pt}%
\pgfpathmoveto{\pgfqpoint{3.353253in}{0.813067in}}%
\pgfpathlineto{\pgfqpoint{3.350432in}{0.817966in}}%
\pgfpathlineto{\pgfqpoint{3.347933in}{0.823400in}}%
\pgfpathlineto{\pgfqpoint{3.345767in}{0.829347in}}%
\pgfpathlineto{\pgfqpoint{3.343940in}{0.835781in}}%
\pgfpathlineto{\pgfqpoint{3.342459in}{0.842677in}}%
\pgfpathlineto{\pgfqpoint{3.311798in}{0.851511in}}%
\pgfpathlineto{\pgfqpoint{3.282233in}{0.861275in}}%
\pgfpathlineto{\pgfqpoint{3.253877in}{0.871938in}}%
\pgfpathlineto{\pgfqpoint{3.226839in}{0.883467in}}%
\pgfpathlineto{\pgfqpoint{3.201225in}{0.895824in}}%
\pgfpathlineto{\pgfqpoint{3.203485in}{0.888581in}}%
\pgfpathlineto{\pgfqpoint{3.206273in}{0.881737in}}%
\pgfpathlineto{\pgfqpoint{3.209579in}{0.875320in}}%
\pgfpathlineto{\pgfqpoint{3.213391in}{0.869356in}}%
\pgfpathlineto{\pgfqpoint{3.217695in}{0.863869in}}%
\pgfpathlineto{\pgfqpoint{3.242266in}{0.852060in}}%
\pgfpathlineto{\pgfqpoint{3.268211in}{0.841041in}}%
\pgfpathlineto{\pgfqpoint{3.295428in}{0.830848in}}%
\pgfpathlineto{\pgfqpoint{3.323811in}{0.821513in}}%
\pgfpathlineto{\pgfqpoint{3.353253in}{0.813067in}}%
\pgfpathclose%
\pgfusepath{fill}%
\end{pgfscope}%
\begin{pgfscope}%
\pgfpathrectangle{\pgfqpoint{2.548318in}{0.050000in}}{\pgfqpoint{2.081932in}{2.081932in}}%
\pgfusepath{clip}%
\pgfsetbuttcap%
\pgfsetroundjoin%
\definecolor{currentfill}{rgb}{0.636902,0.856542,0.216620}%
\pgfsetfillcolor{currentfill}%
\pgfsetlinewidth{0.000000pt}%
\definecolor{currentstroke}{rgb}{0.000000,0.000000,0.000000}%
\pgfsetstrokecolor{currentstroke}%
\pgfsetdash{}{0pt}%
\pgfpathmoveto{\pgfqpoint{4.085984in}{1.071721in}}%
\pgfpathlineto{\pgfqpoint{4.079400in}{1.080292in}}%
\pgfpathlineto{\pgfqpoint{4.072320in}{1.088649in}}%
\pgfpathlineto{\pgfqpoint{4.064772in}{1.096760in}}%
\pgfpathlineto{\pgfqpoint{4.056784in}{1.104592in}}%
\pgfpathlineto{\pgfqpoint{4.048390in}{1.112117in}}%
\pgfpathlineto{\pgfqpoint{4.031746in}{1.098733in}}%
\pgfpathlineto{\pgfqpoint{4.013539in}{1.085927in}}%
\pgfpathlineto{\pgfqpoint{3.993845in}{1.073744in}}%
\pgfpathlineto{\pgfqpoint{3.972748in}{1.062226in}}%
\pgfpathlineto{\pgfqpoint{3.950332in}{1.051412in}}%
\pgfpathlineto{\pgfqpoint{3.956754in}{1.042787in}}%
\pgfpathlineto{\pgfqpoint{3.962863in}{1.033918in}}%
\pgfpathlineto{\pgfqpoint{3.968633in}{1.024837in}}%
\pgfpathlineto{\pgfqpoint{3.974045in}{1.015580in}}%
\pgfpathlineto{\pgfqpoint{3.979075in}{1.006183in}}%
\pgfpathlineto{\pgfqpoint{4.003483in}{1.017851in}}%
\pgfpathlineto{\pgfqpoint{4.026470in}{1.030281in}}%
\pgfpathlineto{\pgfqpoint{4.047941in}{1.043434in}}%
\pgfpathlineto{\pgfqpoint{4.067807in}{1.057263in}}%
\pgfpathlineto{\pgfqpoint{4.085984in}{1.071721in}}%
\pgfpathclose%
\pgfusepath{fill}%
\end{pgfscope}%
\begin{pgfscope}%
\pgfpathrectangle{\pgfqpoint{2.548318in}{0.050000in}}{\pgfqpoint{2.081932in}{2.081932in}}%
\pgfusepath{clip}%
\pgfsetbuttcap%
\pgfsetroundjoin%
\definecolor{currentfill}{rgb}{0.282327,0.094955,0.417331}%
\pgfsetfillcolor{currentfill}%
\pgfsetlinewidth{0.000000pt}%
\definecolor{currentstroke}{rgb}{0.000000,0.000000,0.000000}%
\pgfsetstrokecolor{currentstroke}%
\pgfsetdash{}{0pt}%
\pgfpathmoveto{\pgfqpoint{3.806879in}{0.789188in}}%
\pgfpathlineto{\pgfqpoint{3.811398in}{0.787789in}}%
\pgfpathlineto{\pgfqpoint{3.815774in}{0.787034in}}%
\pgfpathlineto{\pgfqpoint{3.819988in}{0.786924in}}%
\pgfpathlineto{\pgfqpoint{3.824026in}{0.787458in}}%
\pgfpathlineto{\pgfqpoint{3.827872in}{0.788631in}}%
\pgfpathlineto{\pgfqpoint{3.798669in}{0.782738in}}%
\pgfpathlineto{\pgfqpoint{3.768815in}{0.777739in}}%
\pgfpathlineto{\pgfqpoint{3.738418in}{0.773649in}}%
\pgfpathlineto{\pgfqpoint{3.707589in}{0.770481in}}%
\pgfpathlineto{\pgfqpoint{3.676437in}{0.768246in}}%
\pgfpathlineto{\pgfqpoint{3.675364in}{0.767447in}}%
\pgfpathlineto{\pgfqpoint{3.674237in}{0.767305in}}%
\pgfpathlineto{\pgfqpoint{3.673060in}{0.767822in}}%
\pgfpathlineto{\pgfqpoint{3.671838in}{0.768998in}}%
\pgfpathlineto{\pgfqpoint{3.670576in}{0.770829in}}%
\pgfpathlineto{\pgfqpoint{3.698624in}{0.772843in}}%
\pgfpathlineto{\pgfqpoint{3.726380in}{0.775696in}}%
\pgfpathlineto{\pgfqpoint{3.753740in}{0.779380in}}%
\pgfpathlineto{\pgfqpoint{3.780607in}{0.783882in}}%
\pgfpathlineto{\pgfqpoint{3.806879in}{0.789188in}}%
\pgfpathclose%
\pgfusepath{fill}%
\end{pgfscope}%
\begin{pgfscope}%
\pgfpathrectangle{\pgfqpoint{2.548318in}{0.050000in}}{\pgfqpoint{2.081932in}{2.081932in}}%
\pgfusepath{clip}%
\pgfsetbuttcap%
\pgfsetroundjoin%
\definecolor{currentfill}{rgb}{0.993248,0.906157,0.143936}%
\pgfsetfillcolor{currentfill}%
\pgfsetlinewidth{0.000000pt}%
\definecolor{currentstroke}{rgb}{0.000000,0.000000,0.000000}%
\pgfsetstrokecolor{currentstroke}%
\pgfsetdash{}{0pt}%
\pgfpathmoveto{\pgfqpoint{3.874754in}{1.117307in}}%
\pgfpathlineto{\pgfqpoint{3.866709in}{1.121278in}}%
\pgfpathlineto{\pgfqpoint{3.858701in}{1.124689in}}%
\pgfpathlineto{\pgfqpoint{3.850761in}{1.127524in}}%
\pgfpathlineto{\pgfqpoint{3.842921in}{1.129772in}}%
\pgfpathlineto{\pgfqpoint{3.835211in}{1.131422in}}%
\pgfpathlineto{\pgfqpoint{3.819855in}{1.124748in}}%
\pgfpathlineto{\pgfqpoint{3.803730in}{1.118579in}}%
\pgfpathlineto{\pgfqpoint{3.786900in}{1.112939in}}%
\pgfpathlineto{\pgfqpoint{3.769431in}{1.107848in}}%
\pgfpathlineto{\pgfqpoint{3.751390in}{1.103323in}}%
\pgfpathlineto{\pgfqpoint{3.756109in}{1.100689in}}%
\pgfpathlineto{\pgfqpoint{3.760908in}{1.097445in}}%
\pgfpathlineto{\pgfqpoint{3.765766in}{1.093604in}}%
\pgfpathlineto{\pgfqpoint{3.770664in}{1.089183in}}%
\pgfpathlineto{\pgfqpoint{3.775582in}{1.084201in}}%
\pgfpathlineto{\pgfqpoint{3.796907in}{1.089529in}}%
\pgfpathlineto{\pgfqpoint{3.817564in}{1.095525in}}%
\pgfpathlineto{\pgfqpoint{3.837475in}{1.102169in}}%
\pgfpathlineto{\pgfqpoint{3.856563in}{1.109438in}}%
\pgfpathlineto{\pgfqpoint{3.874754in}{1.117307in}}%
\pgfpathclose%
\pgfusepath{fill}%
\end{pgfscope}%
\begin{pgfscope}%
\pgfpathrectangle{\pgfqpoint{2.548318in}{0.050000in}}{\pgfqpoint{2.081932in}{2.081932in}}%
\pgfusepath{clip}%
\pgfsetbuttcap%
\pgfsetroundjoin%
\definecolor{currentfill}{rgb}{0.327796,0.773980,0.406640}%
\pgfsetfillcolor{currentfill}%
\pgfsetlinewidth{0.000000pt}%
\definecolor{currentstroke}{rgb}{0.000000,0.000000,0.000000}%
\pgfsetstrokecolor{currentstroke}%
\pgfsetdash{}{0pt}%
\pgfpathmoveto{\pgfqpoint{4.110630in}{1.026858in}}%
\pgfpathlineto{\pgfqpoint{4.106873in}{1.035982in}}%
\pgfpathlineto{\pgfqpoint{4.102511in}{1.045066in}}%
\pgfpathlineto{\pgfqpoint{4.097562in}{1.054074in}}%
\pgfpathlineto{\pgfqpoint{4.092046in}{1.062970in}}%
\pgfpathlineto{\pgfqpoint{4.085984in}{1.071721in}}%
\pgfpathlineto{\pgfqpoint{4.067807in}{1.057263in}}%
\pgfpathlineto{\pgfqpoint{4.047941in}{1.043434in}}%
\pgfpathlineto{\pgfqpoint{4.026470in}{1.030281in}}%
\pgfpathlineto{\pgfqpoint{4.003483in}{1.017851in}}%
\pgfpathlineto{\pgfqpoint{3.979075in}{1.006183in}}%
\pgfpathlineto{\pgfqpoint{3.983706in}{0.996684in}}%
\pgfpathlineto{\pgfqpoint{3.987918in}{0.987117in}}%
\pgfpathlineto{\pgfqpoint{3.991697in}{0.977521in}}%
\pgfpathlineto{\pgfqpoint{3.995026in}{0.967932in}}%
\pgfpathlineto{\pgfqpoint{3.997893in}{0.958389in}}%
\pgfpathlineto{\pgfqpoint{4.023610in}{0.970573in}}%
\pgfpathlineto{\pgfqpoint{4.047839in}{0.983556in}}%
\pgfpathlineto{\pgfqpoint{4.070481in}{0.997296in}}%
\pgfpathlineto{\pgfqpoint{4.091441in}{1.011747in}}%
\pgfpathlineto{\pgfqpoint{4.110630in}{1.026858in}}%
\pgfpathclose%
\pgfusepath{fill}%
\end{pgfscope}%
\begin{pgfscope}%
\pgfpathrectangle{\pgfqpoint{2.548318in}{0.050000in}}{\pgfqpoint{2.081932in}{2.081932in}}%
\pgfusepath{clip}%
\pgfsetbuttcap%
\pgfsetroundjoin%
\definecolor{currentfill}{rgb}{0.855810,0.888601,0.097452}%
\pgfsetfillcolor{currentfill}%
\pgfsetlinewidth{0.000000pt}%
\definecolor{currentstroke}{rgb}{0.000000,0.000000,0.000000}%
\pgfsetstrokecolor{currentstroke}%
\pgfsetdash{}{0pt}%
\pgfpathmoveto{\pgfqpoint{3.655061in}{1.089671in}}%
\pgfpathlineto{\pgfqpoint{3.653767in}{1.092147in}}%
\pgfpathlineto{\pgfqpoint{3.652506in}{1.093980in}}%
\pgfpathlineto{\pgfqpoint{3.651283in}{1.095160in}}%
\pgfpathlineto{\pgfqpoint{3.650102in}{1.095680in}}%
\pgfpathlineto{\pgfqpoint{3.648969in}{1.095537in}}%
\pgfpathlineto{\pgfqpoint{3.632231in}{1.094808in}}%
\pgfpathlineto{\pgfqpoint{3.615436in}{1.094609in}}%
\pgfpathlineto{\pgfqpoint{3.598650in}{1.094942in}}%
\pgfpathlineto{\pgfqpoint{3.581934in}{1.095804in}}%
\pgfpathlineto{\pgfqpoint{3.565354in}{1.097193in}}%
\pgfpathlineto{\pgfqpoint{3.563474in}{1.097396in}}%
\pgfpathlineto{\pgfqpoint{3.561515in}{1.096937in}}%
\pgfpathlineto{\pgfqpoint{3.559485in}{1.095822in}}%
\pgfpathlineto{\pgfqpoint{3.557392in}{1.094055in}}%
\pgfpathlineto{\pgfqpoint{3.555246in}{1.091646in}}%
\pgfpathlineto{\pgfqpoint{3.575041in}{1.089989in}}%
\pgfpathlineto{\pgfqpoint{3.594994in}{1.088961in}}%
\pgfpathlineto{\pgfqpoint{3.615032in}{1.088565in}}%
\pgfpathlineto{\pgfqpoint{3.635079in}{1.088801in}}%
\pgfpathlineto{\pgfqpoint{3.655061in}{1.089671in}}%
\pgfpathclose%
\pgfusepath{fill}%
\end{pgfscope}%
\begin{pgfscope}%
\pgfpathrectangle{\pgfqpoint{2.548318in}{0.050000in}}{\pgfqpoint{2.081932in}{2.081932in}}%
\pgfusepath{clip}%
\pgfsetbuttcap%
\pgfsetroundjoin%
\definecolor{currentfill}{rgb}{0.267968,0.223549,0.512008}%
\pgfsetfillcolor{currentfill}%
\pgfsetlinewidth{0.000000pt}%
\definecolor{currentstroke}{rgb}{0.000000,0.000000,0.000000}%
\pgfsetstrokecolor{currentstroke}%
\pgfsetdash{}{0pt}%
\pgfpathmoveto{\pgfqpoint{3.960431in}{0.830745in}}%
\pgfpathlineto{\pgfqpoint{3.966403in}{0.833298in}}%
\pgfpathlineto{\pgfqpoint{3.972012in}{0.836436in}}%
\pgfpathlineto{\pgfqpoint{3.977237in}{0.840145in}}%
\pgfpathlineto{\pgfqpoint{3.982059in}{0.844408in}}%
\pgfpathlineto{\pgfqpoint{3.986460in}{0.849208in}}%
\pgfpathlineto{\pgfqpoint{3.960190in}{0.838394in}}%
\pgfpathlineto{\pgfqpoint{3.932674in}{0.828414in}}%
\pgfpathlineto{\pgfqpoint{3.904018in}{0.819300in}}%
\pgfpathlineto{\pgfqpoint{3.874330in}{0.811082in}}%
\pgfpathlineto{\pgfqpoint{3.843722in}{0.803786in}}%
\pgfpathlineto{\pgfqpoint{3.841043in}{0.799558in}}%
\pgfpathlineto{\pgfqpoint{3.838108in}{0.795914in}}%
\pgfpathlineto{\pgfqpoint{3.834926in}{0.792870in}}%
\pgfpathlineto{\pgfqpoint{3.831509in}{0.790438in}}%
\pgfpathlineto{\pgfqpoint{3.827872in}{0.788631in}}%
\pgfpathlineto{\pgfqpoint{3.856316in}{0.795399in}}%
\pgfpathlineto{\pgfqpoint{3.883896in}{0.803021in}}%
\pgfpathlineto{\pgfqpoint{3.910509in}{0.811471in}}%
\pgfpathlineto{\pgfqpoint{3.936054in}{0.820723in}}%
\pgfpathlineto{\pgfqpoint{3.960431in}{0.830745in}}%
\pgfpathclose%
\pgfusepath{fill}%
\end{pgfscope}%
\begin{pgfscope}%
\pgfpathrectangle{\pgfqpoint{2.548318in}{0.050000in}}{\pgfqpoint{2.081932in}{2.081932in}}%
\pgfusepath{clip}%
\pgfsetbuttcap%
\pgfsetroundjoin%
\definecolor{currentfill}{rgb}{0.282327,0.094955,0.417331}%
\pgfsetfillcolor{currentfill}%
\pgfsetlinewidth{0.000000pt}%
\definecolor{currentstroke}{rgb}{0.000000,0.000000,0.000000}%
\pgfsetstrokecolor{currentstroke}%
\pgfsetdash{}{0pt}%
\pgfpathmoveto{\pgfqpoint{3.529500in}{0.773483in}}%
\pgfpathlineto{\pgfqpoint{3.527405in}{0.771714in}}%
\pgfpathlineto{\pgfqpoint{3.525376in}{0.770599in}}%
\pgfpathlineto{\pgfqpoint{3.523422in}{0.770141in}}%
\pgfpathlineto{\pgfqpoint{3.521551in}{0.770339in}}%
\pgfpathlineto{\pgfqpoint{3.519769in}{0.771192in}}%
\pgfpathlineto{\pgfqpoint{3.489036in}{0.774593in}}%
\pgfpathlineto{\pgfqpoint{3.458764in}{0.778913in}}%
\pgfpathlineto{\pgfqpoint{3.429061in}{0.784140in}}%
\pgfpathlineto{\pgfqpoint{3.400037in}{0.790255in}}%
\pgfpathlineto{\pgfqpoint{3.371797in}{0.797240in}}%
\pgfpathlineto{\pgfqpoint{3.376295in}{0.795907in}}%
\pgfpathlineto{\pgfqpoint{3.381018in}{0.795208in}}%
\pgfpathlineto{\pgfqpoint{3.385948in}{0.795145in}}%
\pgfpathlineto{\pgfqpoint{3.391066in}{0.795722in}}%
\pgfpathlineto{\pgfqpoint{3.396351in}{0.796938in}}%
\pgfpathlineto{\pgfqpoint{3.421746in}{0.790650in}}%
\pgfpathlineto{\pgfqpoint{3.447857in}{0.785144in}}%
\pgfpathlineto{\pgfqpoint{3.474585in}{0.780438in}}%
\pgfpathlineto{\pgfqpoint{3.501833in}{0.776547in}}%
\pgfpathlineto{\pgfqpoint{3.529500in}{0.773483in}}%
\pgfpathclose%
\pgfusepath{fill}%
\end{pgfscope}%
\begin{pgfscope}%
\pgfpathrectangle{\pgfqpoint{2.548318in}{0.050000in}}{\pgfqpoint{2.081932in}{2.081932in}}%
\pgfusepath{clip}%
\pgfsetbuttcap%
\pgfsetroundjoin%
\definecolor{currentfill}{rgb}{0.876168,0.891125,0.095250}%
\pgfsetfillcolor{currentfill}%
\pgfsetlinewidth{0.000000pt}%
\definecolor{currentstroke}{rgb}{0.000000,0.000000,0.000000}%
\pgfsetstrokecolor{currentstroke}%
\pgfsetdash{}{0pt}%
\pgfpathmoveto{\pgfqpoint{4.048390in}{1.112117in}}%
\pgfpathlineto{\pgfqpoint{4.039620in}{1.119303in}}%
\pgfpathlineto{\pgfqpoint{4.030509in}{1.126123in}}%
\pgfpathlineto{\pgfqpoint{4.021093in}{1.132550in}}%
\pgfpathlineto{\pgfqpoint{4.011409in}{1.138559in}}%
\pgfpathlineto{\pgfqpoint{4.001493in}{1.144125in}}%
\pgfpathlineto{\pgfqpoint{3.986744in}{1.132136in}}%
\pgfpathlineto{\pgfqpoint{3.970591in}{1.120661in}}%
\pgfpathlineto{\pgfqpoint{3.953102in}{1.109738in}}%
\pgfpathlineto{\pgfqpoint{3.934349in}{1.099408in}}%
\pgfpathlineto{\pgfqpoint{3.914410in}{1.089705in}}%
\pgfpathlineto{\pgfqpoint{3.922011in}{1.082793in}}%
\pgfpathlineto{\pgfqpoint{3.929432in}{1.075478in}}%
\pgfpathlineto{\pgfqpoint{3.936644in}{1.067789in}}%
\pgfpathlineto{\pgfqpoint{3.943620in}{1.059756in}}%
\pgfpathlineto{\pgfqpoint{3.950332in}{1.051412in}}%
\pgfpathlineto{\pgfqpoint{3.972748in}{1.062226in}}%
\pgfpathlineto{\pgfqpoint{3.993845in}{1.073744in}}%
\pgfpathlineto{\pgfqpoint{4.013539in}{1.085927in}}%
\pgfpathlineto{\pgfqpoint{4.031746in}{1.098733in}}%
\pgfpathlineto{\pgfqpoint{4.048390in}{1.112117in}}%
\pgfpathclose%
\pgfusepath{fill}%
\end{pgfscope}%
\begin{pgfscope}%
\pgfpathrectangle{\pgfqpoint{2.548318in}{0.050000in}}{\pgfqpoint{2.081932in}{2.081932in}}%
\pgfusepath{clip}%
\pgfsetbuttcap%
\pgfsetroundjoin%
\definecolor{currentfill}{rgb}{0.124780,0.640461,0.527068}%
\pgfsetfillcolor{currentfill}%
\pgfsetlinewidth{0.000000pt}%
\definecolor{currentstroke}{rgb}{0.000000,0.000000,0.000000}%
\pgfsetstrokecolor{currentstroke}%
\pgfsetdash{}{0pt}%
\pgfpathmoveto{\pgfqpoint{4.119893in}{0.981880in}}%
\pgfpathlineto{\pgfqpoint{4.119341in}{0.990671in}}%
\pgfpathlineto{\pgfqpoint{4.118131in}{0.999600in}}%
\pgfpathlineto{\pgfqpoint{4.116270in}{1.008631in}}%
\pgfpathlineto{\pgfqpoint{4.113766in}{1.017729in}}%
\pgfpathlineto{\pgfqpoint{4.110630in}{1.026858in}}%
\pgfpathlineto{\pgfqpoint{4.091441in}{1.011747in}}%
\pgfpathlineto{\pgfqpoint{4.070481in}{0.997296in}}%
\pgfpathlineto{\pgfqpoint{4.047839in}{0.983556in}}%
\pgfpathlineto{\pgfqpoint{4.023610in}{0.970573in}}%
\pgfpathlineto{\pgfqpoint{3.997893in}{0.958389in}}%
\pgfpathlineto{\pgfqpoint{4.000287in}{0.948927in}}%
\pgfpathlineto{\pgfqpoint{4.002197in}{0.939584in}}%
\pgfpathlineto{\pgfqpoint{4.003617in}{0.930396in}}%
\pgfpathlineto{\pgfqpoint{4.004539in}{0.921399in}}%
\pgfpathlineto{\pgfqpoint{4.004961in}{0.912628in}}%
\pgfpathlineto{\pgfqpoint{4.031171in}{0.924949in}}%
\pgfpathlineto{\pgfqpoint{4.055867in}{0.938080in}}%
\pgfpathlineto{\pgfqpoint{4.078950in}{0.951977in}}%
\pgfpathlineto{\pgfqpoint{4.100323in}{0.966593in}}%
\pgfpathlineto{\pgfqpoint{4.119893in}{0.981880in}}%
\pgfpathclose%
\pgfusepath{fill}%
\end{pgfscope}%
\begin{pgfscope}%
\pgfpathrectangle{\pgfqpoint{2.548318in}{0.050000in}}{\pgfqpoint{2.081932in}{2.081932in}}%
\pgfusepath{clip}%
\pgfsetbuttcap%
\pgfsetroundjoin%
\definecolor{currentfill}{rgb}{0.993248,0.906157,0.143936}%
\pgfsetfillcolor{currentfill}%
\pgfsetlinewidth{0.000000pt}%
\definecolor{currentstroke}{rgb}{0.000000,0.000000,0.000000}%
\pgfsetstrokecolor{currentstroke}%
\pgfsetdash{}{0pt}%
\pgfpathmoveto{\pgfqpoint{3.432942in}{1.090978in}}%
\pgfpathlineto{\pgfqpoint{3.438691in}{1.095753in}}%
\pgfpathlineto{\pgfqpoint{3.444417in}{1.099968in}}%
\pgfpathlineto{\pgfqpoint{3.450095in}{1.103604in}}%
\pgfpathlineto{\pgfqpoint{3.455703in}{1.106646in}}%
\pgfpathlineto{\pgfqpoint{3.461219in}{1.109078in}}%
\pgfpathlineto{\pgfqpoint{3.443902in}{1.114310in}}%
\pgfpathlineto{\pgfqpoint{3.427241in}{1.120085in}}%
\pgfpathlineto{\pgfqpoint{3.411301in}{1.126383in}}%
\pgfpathlineto{\pgfqpoint{3.396145in}{1.133181in}}%
\pgfpathlineto{\pgfqpoint{3.381833in}{1.140456in}}%
\pgfpathlineto{\pgfqpoint{3.373474in}{1.139124in}}%
\pgfpathlineto{\pgfqpoint{3.364972in}{1.137199in}}%
\pgfpathlineto{\pgfqpoint{3.356361in}{1.134689in}}%
\pgfpathlineto{\pgfqpoint{3.347676in}{1.131605in}}%
\pgfpathlineto{\pgfqpoint{3.338950in}{1.127962in}}%
\pgfpathlineto{\pgfqpoint{3.355920in}{1.119381in}}%
\pgfpathlineto{\pgfqpoint{3.373876in}{1.111366in}}%
\pgfpathlineto{\pgfqpoint{3.392748in}{1.103943in}}%
\pgfpathlineto{\pgfqpoint{3.412462in}{1.097139in}}%
\pgfpathlineto{\pgfqpoint{3.432942in}{1.090978in}}%
\pgfpathclose%
\pgfusepath{fill}%
\end{pgfscope}%
\begin{pgfscope}%
\pgfpathrectangle{\pgfqpoint{2.548318in}{0.050000in}}{\pgfqpoint{2.081932in}{2.081932in}}%
\pgfusepath{clip}%
\pgfsetbuttcap%
\pgfsetroundjoin%
\definecolor{currentfill}{rgb}{0.855810,0.888601,0.097452}%
\pgfsetfillcolor{currentfill}%
\pgfsetlinewidth{0.000000pt}%
\definecolor{currentstroke}{rgb}{0.000000,0.000000,0.000000}%
\pgfsetstrokecolor{currentstroke}%
\pgfsetdash{}{0pt}%
\pgfpathmoveto{\pgfqpoint{3.751390in}{1.103323in}}%
\pgfpathlineto{\pgfqpoint{3.746768in}{1.105333in}}%
\pgfpathlineto{\pgfqpoint{3.742262in}{1.106710in}}%
\pgfpathlineto{\pgfqpoint{3.737892in}{1.107446in}}%
\pgfpathlineto{\pgfqpoint{3.733675in}{1.107537in}}%
\pgfpathlineto{\pgfqpoint{3.729629in}{1.106980in}}%
\pgfpathlineto{\pgfqpoint{3.714108in}{1.103676in}}%
\pgfpathlineto{\pgfqpoint{3.698221in}{1.100870in}}%
\pgfpathlineto{\pgfqpoint{3.682027in}{1.098573in}}%
\pgfpathlineto{\pgfqpoint{3.665589in}{1.096793in}}%
\pgfpathlineto{\pgfqpoint{3.648969in}{1.095537in}}%
\pgfpathlineto{\pgfqpoint{3.650102in}{1.095680in}}%
\pgfpathlineto{\pgfqpoint{3.651283in}{1.095160in}}%
\pgfpathlineto{\pgfqpoint{3.652506in}{1.093980in}}%
\pgfpathlineto{\pgfqpoint{3.653767in}{1.092147in}}%
\pgfpathlineto{\pgfqpoint{3.655061in}{1.089671in}}%
\pgfpathlineto{\pgfqpoint{3.674902in}{1.091169in}}%
\pgfpathlineto{\pgfqpoint{3.694528in}{1.093292in}}%
\pgfpathlineto{\pgfqpoint{3.713867in}{1.096032in}}%
\pgfpathlineto{\pgfqpoint{3.732845in}{1.099380in}}%
\pgfpathlineto{\pgfqpoint{3.751390in}{1.103323in}}%
\pgfpathclose%
\pgfusepath{fill}%
\end{pgfscope}%
\begin{pgfscope}%
\pgfpathrectangle{\pgfqpoint{2.548318in}{0.050000in}}{\pgfqpoint{2.081932in}{2.081932in}}%
\pgfusepath{clip}%
\pgfsetbuttcap%
\pgfsetroundjoin%
\definecolor{currentfill}{rgb}{0.855810,0.888601,0.097452}%
\pgfsetfillcolor{currentfill}%
\pgfsetlinewidth{0.000000pt}%
\definecolor{currentstroke}{rgb}{0.000000,0.000000,0.000000}%
\pgfsetstrokecolor{currentstroke}%
\pgfsetdash{}{0pt}%
\pgfpathmoveto{\pgfqpoint{3.555246in}{1.091646in}}%
\pgfpathlineto{\pgfqpoint{3.557392in}{1.094055in}}%
\pgfpathlineto{\pgfqpoint{3.559485in}{1.095822in}}%
\pgfpathlineto{\pgfqpoint{3.561515in}{1.096937in}}%
\pgfpathlineto{\pgfqpoint{3.563474in}{1.097396in}}%
\pgfpathlineto{\pgfqpoint{3.565354in}{1.097193in}}%
\pgfpathlineto{\pgfqpoint{3.548971in}{1.099103in}}%
\pgfpathlineto{\pgfqpoint{3.532847in}{1.101529in}}%
\pgfpathlineto{\pgfqpoint{3.517045in}{1.104462in}}%
\pgfpathlineto{\pgfqpoint{3.501625in}{1.107890in}}%
\pgfpathlineto{\pgfqpoint{3.486645in}{1.111802in}}%
\pgfpathlineto{\pgfqpoint{3.481918in}{1.112533in}}%
\pgfpathlineto{\pgfqpoint{3.476991in}{1.112624in}}%
\pgfpathlineto{\pgfqpoint{3.471884in}{1.112075in}}%
\pgfpathlineto{\pgfqpoint{3.466620in}{1.110891in}}%
\pgfpathlineto{\pgfqpoint{3.461219in}{1.109078in}}%
\pgfpathlineto{\pgfqpoint{3.479125in}{1.104409in}}%
\pgfpathlineto{\pgfqpoint{3.497551in}{1.100318in}}%
\pgfpathlineto{\pgfqpoint{3.516428in}{1.096819in}}%
\pgfpathlineto{\pgfqpoint{3.535684in}{1.093925in}}%
\pgfpathlineto{\pgfqpoint{3.555246in}{1.091646in}}%
\pgfpathclose%
\pgfusepath{fill}%
\end{pgfscope}%
\begin{pgfscope}%
\pgfpathrectangle{\pgfqpoint{2.548318in}{0.050000in}}{\pgfqpoint{2.081932in}{2.081932in}}%
\pgfusepath{clip}%
\pgfsetbuttcap%
\pgfsetroundjoin%
\definecolor{currentfill}{rgb}{0.267968,0.223549,0.512008}%
\pgfsetfillcolor{currentfill}%
\pgfsetlinewidth{0.000000pt}%
\definecolor{currentstroke}{rgb}{0.000000,0.000000,0.000000}%
\pgfsetstrokecolor{currentstroke}%
\pgfsetdash{}{0pt}%
\pgfpathmoveto{\pgfqpoint{3.371797in}{0.797240in}}%
\pgfpathlineto{\pgfqpoint{3.367541in}{0.799199in}}%
\pgfpathlineto{\pgfqpoint{3.363545in}{0.801774in}}%
\pgfpathlineto{\pgfqpoint{3.359822in}{0.804954in}}%
\pgfpathlineto{\pgfqpoint{3.356387in}{0.808723in}}%
\pgfpathlineto{\pgfqpoint{3.353253in}{0.813067in}}%
\pgfpathlineto{\pgfqpoint{3.323811in}{0.821513in}}%
\pgfpathlineto{\pgfqpoint{3.295428in}{0.830848in}}%
\pgfpathlineto{\pgfqpoint{3.268211in}{0.841041in}}%
\pgfpathlineto{\pgfqpoint{3.242266in}{0.852060in}}%
\pgfpathlineto{\pgfqpoint{3.217695in}{0.863869in}}%
\pgfpathlineto{\pgfqpoint{3.222475in}{0.858883in}}%
\pgfpathlineto{\pgfqpoint{3.227713in}{0.854418in}}%
\pgfpathlineto{\pgfqpoint{3.233388in}{0.850493in}}%
\pgfpathlineto{\pgfqpoint{3.239480in}{0.847125in}}%
\pgfpathlineto{\pgfqpoint{3.245965in}{0.844330in}}%
\pgfpathlineto{\pgfqpoint{3.268752in}{0.833389in}}%
\pgfpathlineto{\pgfqpoint{3.292825in}{0.823177in}}%
\pgfpathlineto{\pgfqpoint{3.318090in}{0.813728in}}%
\pgfpathlineto{\pgfqpoint{3.344447in}{0.805073in}}%
\pgfpathlineto{\pgfqpoint{3.371797in}{0.797240in}}%
\pgfpathclose%
\pgfusepath{fill}%
\end{pgfscope}%
\begin{pgfscope}%
\pgfpathrectangle{\pgfqpoint{2.548318in}{0.050000in}}{\pgfqpoint{2.081932in}{2.081932in}}%
\pgfusepath{clip}%
\pgfsetbuttcap%
\pgfsetroundjoin%
\definecolor{currentfill}{rgb}{0.636902,0.856542,0.216620}%
\pgfsetfillcolor{currentfill}%
\pgfsetlinewidth{0.000000pt}%
\definecolor{currentstroke}{rgb}{0.000000,0.000000,0.000000}%
\pgfsetstrokecolor{currentstroke}%
\pgfsetdash{}{0pt}%
\pgfpathmoveto{\pgfqpoint{3.225715in}{1.020913in}}%
\pgfpathlineto{\pgfqpoint{3.231179in}{1.030125in}}%
\pgfpathlineto{\pgfqpoint{3.237055in}{1.039181in}}%
\pgfpathlineto{\pgfqpoint{3.243322in}{1.048046in}}%
\pgfpathlineto{\pgfqpoint{3.249956in}{1.056685in}}%
\pgfpathlineto{\pgfqpoint{3.256929in}{1.065063in}}%
\pgfpathlineto{\pgfqpoint{3.236172in}{1.076753in}}%
\pgfpathlineto{\pgfqpoint{3.216840in}{1.089097in}}%
\pgfpathlineto{\pgfqpoint{3.199013in}{1.102053in}}%
\pgfpathlineto{\pgfqpoint{3.182770in}{1.115575in}}%
\pgfpathlineto{\pgfqpoint{3.168180in}{1.129615in}}%
\pgfpathlineto{\pgfqpoint{3.159401in}{1.122412in}}%
\pgfpathlineto{\pgfqpoint{3.151048in}{1.114882in}}%
\pgfpathlineto{\pgfqpoint{3.143153in}{1.107055in}}%
\pgfpathlineto{\pgfqpoint{3.135747in}{1.098961in}}%
\pgfpathlineto{\pgfqpoint{3.128859in}{1.090632in}}%
\pgfpathlineto{\pgfqpoint{3.144815in}{1.075458in}}%
\pgfpathlineto{\pgfqpoint{3.162559in}{1.060849in}}%
\pgfpathlineto{\pgfqpoint{3.182014in}{1.046856in}}%
\pgfpathlineto{\pgfqpoint{3.203096in}{1.033529in}}%
\pgfpathlineto{\pgfqpoint{3.225715in}{1.020913in}}%
\pgfpathclose%
\pgfusepath{fill}%
\end{pgfscope}%
\begin{pgfscope}%
\pgfpathrectangle{\pgfqpoint{2.548318in}{0.050000in}}{\pgfqpoint{2.081932in}{2.081932in}}%
\pgfusepath{clip}%
\pgfsetbuttcap%
\pgfsetroundjoin%
\definecolor{currentfill}{rgb}{0.327796,0.773980,0.406640}%
\pgfsetfillcolor{currentfill}%
\pgfsetlinewidth{0.000000pt}%
\definecolor{currentstroke}{rgb}{0.000000,0.000000,0.000000}%
\pgfsetstrokecolor{currentstroke}%
\pgfsetdash{}{0pt}%
\pgfpathmoveto{\pgfqpoint{3.205273in}{0.973771in}}%
\pgfpathlineto{\pgfqpoint{3.208388in}{0.983221in}}%
\pgfpathlineto{\pgfqpoint{3.212005in}{0.992697in}}%
\pgfpathlineto{\pgfqpoint{3.216109in}{1.002162in}}%
\pgfpathlineto{\pgfqpoint{3.220685in}{1.011580in}}%
\pgfpathlineto{\pgfqpoint{3.225715in}{1.020913in}}%
\pgfpathlineto{\pgfqpoint{3.203096in}{1.033529in}}%
\pgfpathlineto{\pgfqpoint{3.182014in}{1.046856in}}%
\pgfpathlineto{\pgfqpoint{3.162559in}{1.060849in}}%
\pgfpathlineto{\pgfqpoint{3.144815in}{1.075458in}}%
\pgfpathlineto{\pgfqpoint{3.128859in}{1.090632in}}%
\pgfpathlineto{\pgfqpoint{3.122517in}{1.082101in}}%
\pgfpathlineto{\pgfqpoint{3.116745in}{1.073401in}}%
\pgfpathlineto{\pgfqpoint{3.111567in}{1.064566in}}%
\pgfpathlineto{\pgfqpoint{3.107003in}{1.055630in}}%
\pgfpathlineto{\pgfqpoint{3.103072in}{1.046629in}}%
\pgfpathlineto{\pgfqpoint{3.119931in}{1.030764in}}%
\pgfpathlineto{\pgfqpoint{3.138667in}{1.015494in}}%
\pgfpathlineto{\pgfqpoint{3.159196in}{1.000872in}}%
\pgfpathlineto{\pgfqpoint{3.181429in}{0.986949in}}%
\pgfpathlineto{\pgfqpoint{3.205273in}{0.973771in}}%
\pgfpathclose%
\pgfusepath{fill}%
\end{pgfscope}%
\begin{pgfscope}%
\pgfpathrectangle{\pgfqpoint{2.548318in}{0.050000in}}{\pgfqpoint{2.081932in}{2.081932in}}%
\pgfusepath{clip}%
\pgfsetbuttcap%
\pgfsetroundjoin%
\definecolor{currentfill}{rgb}{0.993248,0.906157,0.143936}%
\pgfsetfillcolor{currentfill}%
\pgfsetlinewidth{0.000000pt}%
\definecolor{currentstroke}{rgb}{0.000000,0.000000,0.000000}%
\pgfsetstrokecolor{currentstroke}%
\pgfsetdash{}{0pt}%
\pgfpathmoveto{\pgfqpoint{4.001493in}{1.144125in}}%
\pgfpathlineto{\pgfqpoint{3.991385in}{1.149227in}}%
\pgfpathlineto{\pgfqpoint{3.981125in}{1.153843in}}%
\pgfpathlineto{\pgfqpoint{3.970752in}{1.157956in}}%
\pgfpathlineto{\pgfqpoint{3.960307in}{1.161549in}}%
\pgfpathlineto{\pgfqpoint{3.949831in}{1.164606in}}%
\pgfpathlineto{\pgfqpoint{3.937146in}{1.154195in}}%
\pgfpathlineto{\pgfqpoint{3.923235in}{1.144225in}}%
\pgfpathlineto{\pgfqpoint{3.908157in}{1.134731in}}%
\pgfpathlineto{\pgfqpoint{3.891974in}{1.125748in}}%
\pgfpathlineto{\pgfqpoint{3.874754in}{1.117307in}}%
\pgfpathlineto{\pgfqpoint{3.882802in}{1.112793in}}%
\pgfpathlineto{\pgfqpoint{3.890823in}{1.107753in}}%
\pgfpathlineto{\pgfqpoint{3.898786in}{1.102210in}}%
\pgfpathlineto{\pgfqpoint{3.906658in}{1.096186in}}%
\pgfpathlineto{\pgfqpoint{3.914410in}{1.089705in}}%
\pgfpathlineto{\pgfqpoint{3.934349in}{1.099408in}}%
\pgfpathlineto{\pgfqpoint{3.953102in}{1.109738in}}%
\pgfpathlineto{\pgfqpoint{3.970591in}{1.120661in}}%
\pgfpathlineto{\pgfqpoint{3.986744in}{1.132136in}}%
\pgfpathlineto{\pgfqpoint{4.001493in}{1.144125in}}%
\pgfpathclose%
\pgfusepath{fill}%
\end{pgfscope}%
\begin{pgfscope}%
\pgfpathrectangle{\pgfqpoint{2.548318in}{0.050000in}}{\pgfqpoint{2.081932in}{2.081932in}}%
\pgfusepath{clip}%
\pgfsetbuttcap%
\pgfsetroundjoin%
\definecolor{currentfill}{rgb}{0.150476,0.504369,0.557430}%
\pgfsetfillcolor{currentfill}%
\pgfsetlinewidth{0.000000pt}%
\definecolor{currentstroke}{rgb}{0.000000,0.000000,0.000000}%
\pgfsetstrokecolor{currentstroke}%
\pgfsetdash{}{0pt}%
\pgfpathmoveto{\pgfqpoint{4.115513in}{0.948779in}}%
\pgfpathlineto{\pgfqpoint{4.117593in}{0.956678in}}%
\pgfpathlineto{\pgfqpoint{4.119018in}{0.964849in}}%
\pgfpathlineto{\pgfqpoint{4.119786in}{0.973261in}}%
\pgfpathlineto{\pgfqpoint{4.119893in}{0.981880in}}%
\pgfpathlineto{\pgfqpoint{4.100323in}{0.966593in}}%
\pgfpathlineto{\pgfqpoint{4.078950in}{0.951977in}}%
\pgfpathlineto{\pgfqpoint{4.055867in}{0.938080in}}%
\pgfpathlineto{\pgfqpoint{4.031171in}{0.924949in}}%
\pgfpathlineto{\pgfqpoint{4.004961in}{0.912628in}}%
\pgfpathlineto{\pgfqpoint{4.004879in}{0.904118in}}%
\pgfpathlineto{\pgfqpoint{4.004294in}{0.895902in}}%
\pgfpathlineto{\pgfqpoint{4.003206in}{0.888013in}}%
\pgfpathlineto{\pgfqpoint{4.001619in}{0.880482in}}%
\pgfpathlineto{\pgfqpoint{4.027596in}{0.892635in}}%
\pgfpathlineto{\pgfqpoint{4.052071in}{0.905585in}}%
\pgfpathlineto{\pgfqpoint{4.074946in}{0.919290in}}%
\pgfpathlineto{\pgfqpoint{4.096123in}{0.933705in}}%
\pgfpathlineto{\pgfqpoint{4.115513in}{0.948779in}}%
\pgfpathclose%
\pgfusepath{fill}%
\end{pgfscope}%
\begin{pgfscope}%
\pgfpathrectangle{\pgfqpoint{2.548318in}{0.050000in}}{\pgfqpoint{2.081932in}{2.081932in}}%
\pgfusepath{clip}%
\pgfsetbuttcap%
\pgfsetroundjoin%
\definecolor{currentfill}{rgb}{0.876168,0.891125,0.095250}%
\pgfsetfillcolor{currentfill}%
\pgfsetlinewidth{0.000000pt}%
\definecolor{currentstroke}{rgb}{0.000000,0.000000,0.000000}%
\pgfsetstrokecolor{currentstroke}%
\pgfsetdash{}{0pt}%
\pgfpathmoveto{\pgfqpoint{3.256929in}{1.065063in}}%
\pgfpathlineto{\pgfqpoint{3.264216in}{1.073150in}}%
\pgfpathlineto{\pgfqpoint{3.271789in}{1.080912in}}%
\pgfpathlineto{\pgfqpoint{3.279618in}{1.088319in}}%
\pgfpathlineto{\pgfqpoint{3.287673in}{1.095342in}}%
\pgfpathlineto{\pgfqpoint{3.295923in}{1.101953in}}%
\pgfpathlineto{\pgfqpoint{3.277477in}{1.112436in}}%
\pgfpathlineto{\pgfqpoint{3.260313in}{1.123502in}}%
\pgfpathlineto{\pgfqpoint{3.244501in}{1.135111in}}%
\pgfpathlineto{\pgfqpoint{3.230111in}{1.147222in}}%
\pgfpathlineto{\pgfqpoint{3.217205in}{1.159791in}}%
\pgfpathlineto{\pgfqpoint{3.206842in}{1.154617in}}%
\pgfpathlineto{\pgfqpoint{3.196719in}{1.148989in}}%
\pgfpathlineto{\pgfqpoint{3.186875in}{1.142930in}}%
\pgfpathlineto{\pgfqpoint{3.177350in}{1.136464in}}%
\pgfpathlineto{\pgfqpoint{3.168180in}{1.129615in}}%
\pgfpathlineto{\pgfqpoint{3.182770in}{1.115575in}}%
\pgfpathlineto{\pgfqpoint{3.199013in}{1.102053in}}%
\pgfpathlineto{\pgfqpoint{3.216840in}{1.089097in}}%
\pgfpathlineto{\pgfqpoint{3.236172in}{1.076753in}}%
\pgfpathlineto{\pgfqpoint{3.256929in}{1.065063in}}%
\pgfpathclose%
\pgfusepath{fill}%
\end{pgfscope}%
\begin{pgfscope}%
\pgfpathrectangle{\pgfqpoint{2.548318in}{0.050000in}}{\pgfqpoint{2.081932in}{2.081932in}}%
\pgfusepath{clip}%
\pgfsetbuttcap%
\pgfsetroundjoin%
\definecolor{currentfill}{rgb}{0.268510,0.009605,0.335427}%
\pgfsetfillcolor{currentfill}%
\pgfsetlinewidth{0.000000pt}%
\definecolor{currentstroke}{rgb}{0.000000,0.000000,0.000000}%
\pgfsetstrokecolor{currentstroke}%
\pgfsetdash{}{0pt}%
\pgfpathmoveto{\pgfqpoint{3.663842in}{0.789545in}}%
\pgfpathlineto{\pgfqpoint{3.665227in}{0.784568in}}%
\pgfpathlineto{\pgfqpoint{3.666598in}{0.780193in}}%
\pgfpathlineto{\pgfqpoint{3.667951in}{0.776436in}}%
\pgfpathlineto{\pgfqpoint{3.669278in}{0.773311in}}%
\pgfpathlineto{\pgfqpoint{3.670576in}{0.770829in}}%
\pgfpathlineto{\pgfqpoint{3.642335in}{0.769661in}}%
\pgfpathlineto{\pgfqpoint{3.614005in}{0.769343in}}%
\pgfpathlineto{\pgfqpoint{3.585687in}{0.769876in}}%
\pgfpathlineto{\pgfqpoint{3.557485in}{0.771257in}}%
\pgfpathlineto{\pgfqpoint{3.529500in}{0.773483in}}%
\pgfpathlineto{\pgfqpoint{3.531653in}{0.775902in}}%
\pgfpathlineto{\pgfqpoint{3.533857in}{0.778961in}}%
\pgfpathlineto{\pgfqpoint{3.536101in}{0.782652in}}%
\pgfpathlineto{\pgfqpoint{3.538378in}{0.786960in}}%
\pgfpathlineto{\pgfqpoint{3.540677in}{0.791869in}}%
\pgfpathlineto{\pgfqpoint{3.565106in}{0.789920in}}%
\pgfpathlineto{\pgfqpoint{3.589728in}{0.788710in}}%
\pgfpathlineto{\pgfqpoint{3.614451in}{0.788244in}}%
\pgfpathlineto{\pgfqpoint{3.639186in}{0.788523in}}%
\pgfpathlineto{\pgfqpoint{3.663842in}{0.789545in}}%
\pgfpathclose%
\pgfusepath{fill}%
\end{pgfscope}%
\begin{pgfscope}%
\pgfpathrectangle{\pgfqpoint{2.548318in}{0.050000in}}{\pgfqpoint{2.081932in}{2.081932in}}%
\pgfusepath{clip}%
\pgfsetbuttcap%
\pgfsetroundjoin%
\definecolor{currentfill}{rgb}{0.124780,0.640461,0.527068}%
\pgfsetfillcolor{currentfill}%
\pgfsetlinewidth{0.000000pt}%
\definecolor{currentstroke}{rgb}{0.000000,0.000000,0.000000}%
\pgfsetstrokecolor{currentstroke}%
\pgfsetdash{}{0pt}%
\pgfpathmoveto{\pgfqpoint{3.197594in}{0.928183in}}%
\pgfpathlineto{\pgfqpoint{3.198052in}{0.936959in}}%
\pgfpathlineto{\pgfqpoint{3.199055in}{0.945941in}}%
\pgfpathlineto{\pgfqpoint{3.200597in}{0.955094in}}%
\pgfpathlineto{\pgfqpoint{3.202673in}{0.964383in}}%
\pgfpathlineto{\pgfqpoint{3.205273in}{0.973771in}}%
\pgfpathlineto{\pgfqpoint{3.181429in}{0.986949in}}%
\pgfpathlineto{\pgfqpoint{3.159196in}{1.000872in}}%
\pgfpathlineto{\pgfqpoint{3.138667in}{1.015494in}}%
\pgfpathlineto{\pgfqpoint{3.119931in}{1.030764in}}%
\pgfpathlineto{\pgfqpoint{3.103072in}{1.046629in}}%
\pgfpathlineto{\pgfqpoint{3.099789in}{1.037598in}}%
\pgfpathlineto{\pgfqpoint{3.097169in}{1.028572in}}%
\pgfpathlineto{\pgfqpoint{3.095221in}{1.019587in}}%
\pgfpathlineto{\pgfqpoint{3.093956in}{1.010678in}}%
\pgfpathlineto{\pgfqpoint{3.093377in}{1.001882in}}%
\pgfpathlineto{\pgfqpoint{3.110578in}{0.985831in}}%
\pgfpathlineto{\pgfqpoint{3.129687in}{0.970384in}}%
\pgfpathlineto{\pgfqpoint{3.150621in}{0.955594in}}%
\pgfpathlineto{\pgfqpoint{3.173289in}{0.941511in}}%
\pgfpathlineto{\pgfqpoint{3.197594in}{0.928183in}}%
\pgfpathclose%
\pgfusepath{fill}%
\end{pgfscope}%
\begin{pgfscope}%
\pgfpathrectangle{\pgfqpoint{2.548318in}{0.050000in}}{\pgfqpoint{2.081932in}{2.081932in}}%
\pgfusepath{clip}%
\pgfsetbuttcap%
\pgfsetroundjoin%
\definecolor{currentfill}{rgb}{0.855810,0.888601,0.097452}%
\pgfsetfillcolor{currentfill}%
\pgfsetlinewidth{0.000000pt}%
\definecolor{currentstroke}{rgb}{0.000000,0.000000,0.000000}%
\pgfsetstrokecolor{currentstroke}%
\pgfsetdash{}{0pt}%
\pgfpathmoveto{\pgfqpoint{3.835211in}{1.131422in}}%
\pgfpathlineto{\pgfqpoint{3.827664in}{1.132467in}}%
\pgfpathlineto{\pgfqpoint{3.820309in}{1.132899in}}%
\pgfpathlineto{\pgfqpoint{3.813177in}{1.132717in}}%
\pgfpathlineto{\pgfqpoint{3.806297in}{1.131920in}}%
\pgfpathlineto{\pgfqpoint{3.799697in}{1.130509in}}%
\pgfpathlineto{\pgfqpoint{3.786874in}{1.124923in}}%
\pgfpathlineto{\pgfqpoint{3.773402in}{1.119759in}}%
\pgfpathlineto{\pgfqpoint{3.759334in}{1.115036in}}%
\pgfpathlineto{\pgfqpoint{3.744724in}{1.110771in}}%
\pgfpathlineto{\pgfqpoint{3.729629in}{1.106980in}}%
\pgfpathlineto{\pgfqpoint{3.733675in}{1.107537in}}%
\pgfpathlineto{\pgfqpoint{3.737892in}{1.107446in}}%
\pgfpathlineto{\pgfqpoint{3.742262in}{1.106710in}}%
\pgfpathlineto{\pgfqpoint{3.746768in}{1.105333in}}%
\pgfpathlineto{\pgfqpoint{3.751390in}{1.103323in}}%
\pgfpathlineto{\pgfqpoint{3.769431in}{1.107848in}}%
\pgfpathlineto{\pgfqpoint{3.786900in}{1.112939in}}%
\pgfpathlineto{\pgfqpoint{3.803730in}{1.118579in}}%
\pgfpathlineto{\pgfqpoint{3.819855in}{1.124748in}}%
\pgfpathlineto{\pgfqpoint{3.835211in}{1.131422in}}%
\pgfpathclose%
\pgfusepath{fill}%
\end{pgfscope}%
\begin{pgfscope}%
\pgfpathrectangle{\pgfqpoint{2.548318in}{0.050000in}}{\pgfqpoint{2.081932in}{2.081932in}}%
\pgfusepath{clip}%
\pgfsetbuttcap%
\pgfsetroundjoin%
\definecolor{currentfill}{rgb}{0.282327,0.094955,0.417331}%
\pgfsetfillcolor{currentfill}%
\pgfsetlinewidth{0.000000pt}%
\definecolor{currentstroke}{rgb}{0.000000,0.000000,0.000000}%
\pgfsetstrokecolor{currentstroke}%
\pgfsetdash{}{0pt}%
\pgfpathmoveto{\pgfqpoint{3.926000in}{0.827077in}}%
\pgfpathlineto{\pgfqpoint{3.933408in}{0.826579in}}%
\pgfpathlineto{\pgfqpoint{3.940582in}{0.826700in}}%
\pgfpathlineto{\pgfqpoint{3.947496in}{0.827438in}}%
\pgfpathlineto{\pgfqpoint{3.954121in}{0.828789in}}%
\pgfpathlineto{\pgfqpoint{3.960431in}{0.830745in}}%
\pgfpathlineto{\pgfqpoint{3.936054in}{0.820723in}}%
\pgfpathlineto{\pgfqpoint{3.910509in}{0.811471in}}%
\pgfpathlineto{\pgfqpoint{3.883896in}{0.803021in}}%
\pgfpathlineto{\pgfqpoint{3.856316in}{0.795399in}}%
\pgfpathlineto{\pgfqpoint{3.827872in}{0.788631in}}%
\pgfpathlineto{\pgfqpoint{3.824026in}{0.787458in}}%
\pgfpathlineto{\pgfqpoint{3.819988in}{0.786924in}}%
\pgfpathlineto{\pgfqpoint{3.815774in}{0.787034in}}%
\pgfpathlineto{\pgfqpoint{3.811398in}{0.787789in}}%
\pgfpathlineto{\pgfqpoint{3.806879in}{0.789188in}}%
\pgfpathlineto{\pgfqpoint{3.832462in}{0.795281in}}%
\pgfpathlineto{\pgfqpoint{3.857257in}{0.802140in}}%
\pgfpathlineto{\pgfqpoint{3.881173in}{0.809743in}}%
\pgfpathlineto{\pgfqpoint{3.904117in}{0.818064in}}%
\pgfpathlineto{\pgfqpoint{3.926000in}{0.827077in}}%
\pgfpathclose%
\pgfusepath{fill}%
\end{pgfscope}%
\begin{pgfscope}%
\pgfpathrectangle{\pgfqpoint{2.548318in}{0.050000in}}{\pgfqpoint{2.081932in}{2.081932in}}%
\pgfusepath{clip}%
\pgfsetbuttcap%
\pgfsetroundjoin%
\definecolor{currentfill}{rgb}{0.268510,0.009605,0.335427}%
\pgfsetfillcolor{currentfill}%
\pgfsetlinewidth{0.000000pt}%
\definecolor{currentstroke}{rgb}{0.000000,0.000000,0.000000}%
\pgfsetstrokecolor{currentstroke}%
\pgfsetdash{}{0pt}%
\pgfpathmoveto{\pgfqpoint{3.782782in}{0.805617in}}%
\pgfpathlineto{\pgfqpoint{3.787738in}{0.801108in}}%
\pgfpathlineto{\pgfqpoint{3.792645in}{0.797198in}}%
\pgfpathlineto{\pgfqpoint{3.797484in}{0.793900in}}%
\pgfpathlineto{\pgfqpoint{3.802235in}{0.791227in}}%
\pgfpathlineto{\pgfqpoint{3.806879in}{0.789188in}}%
\pgfpathlineto{\pgfqpoint{3.780607in}{0.783882in}}%
\pgfpathlineto{\pgfqpoint{3.753740in}{0.779380in}}%
\pgfpathlineto{\pgfqpoint{3.726380in}{0.775696in}}%
\pgfpathlineto{\pgfqpoint{3.698624in}{0.772843in}}%
\pgfpathlineto{\pgfqpoint{3.670576in}{0.770829in}}%
\pgfpathlineto{\pgfqpoint{3.669278in}{0.773311in}}%
\pgfpathlineto{\pgfqpoint{3.667951in}{0.776436in}}%
\pgfpathlineto{\pgfqpoint{3.666598in}{0.780193in}}%
\pgfpathlineto{\pgfqpoint{3.665227in}{0.784568in}}%
\pgfpathlineto{\pgfqpoint{3.663842in}{0.789545in}}%
\pgfpathlineto{\pgfqpoint{3.688327in}{0.791309in}}%
\pgfpathlineto{\pgfqpoint{3.712553in}{0.793807in}}%
\pgfpathlineto{\pgfqpoint{3.736429in}{0.797032in}}%
\pgfpathlineto{\pgfqpoint{3.759868in}{0.800973in}}%
\pgfpathlineto{\pgfqpoint{3.782782in}{0.805617in}}%
\pgfpathclose%
\pgfusepath{fill}%
\end{pgfscope}%
\begin{pgfscope}%
\pgfpathrectangle{\pgfqpoint{2.548318in}{0.050000in}}{\pgfqpoint{2.081932in}{2.081932in}}%
\pgfusepath{clip}%
\pgfsetbuttcap%
\pgfsetroundjoin%
\definecolor{currentfill}{rgb}{0.993248,0.906157,0.143936}%
\pgfsetfillcolor{currentfill}%
\pgfsetlinewidth{0.000000pt}%
\definecolor{currentstroke}{rgb}{0.000000,0.000000,0.000000}%
\pgfsetstrokecolor{currentstroke}%
\pgfsetdash{}{0pt}%
\pgfpathmoveto{\pgfqpoint{3.295923in}{1.101953in}}%
\pgfpathlineto{\pgfqpoint{3.304335in}{1.108126in}}%
\pgfpathlineto{\pgfqpoint{3.312878in}{1.113835in}}%
\pgfpathlineto{\pgfqpoint{3.321517in}{1.119058in}}%
\pgfpathlineto{\pgfqpoint{3.330219in}{1.123773in}}%
\pgfpathlineto{\pgfqpoint{3.338950in}{1.127962in}}%
\pgfpathlineto{\pgfqpoint{3.323036in}{1.137077in}}%
\pgfpathlineto{\pgfqpoint{3.308242in}{1.146694in}}%
\pgfpathlineto{\pgfqpoint{3.294630in}{1.156779in}}%
\pgfpathlineto{\pgfqpoint{3.282257in}{1.167295in}}%
\pgfpathlineto{\pgfqpoint{3.271178in}{1.178203in}}%
\pgfpathlineto{\pgfqpoint{3.260236in}{1.175568in}}%
\pgfpathlineto{\pgfqpoint{3.249325in}{1.172396in}}%
\pgfpathlineto{\pgfqpoint{3.238488in}{1.168699in}}%
\pgfpathlineto{\pgfqpoint{3.227767in}{1.164492in}}%
\pgfpathlineto{\pgfqpoint{3.217205in}{1.159791in}}%
\pgfpathlineto{\pgfqpoint{3.230111in}{1.147222in}}%
\pgfpathlineto{\pgfqpoint{3.244501in}{1.135111in}}%
\pgfpathlineto{\pgfqpoint{3.260313in}{1.123502in}}%
\pgfpathlineto{\pgfqpoint{3.277477in}{1.112436in}}%
\pgfpathlineto{\pgfqpoint{3.295923in}{1.101953in}}%
\pgfpathclose%
\pgfusepath{fill}%
\end{pgfscope}%
\begin{pgfscope}%
\pgfpathrectangle{\pgfqpoint{2.548318in}{0.050000in}}{\pgfqpoint{2.081932in}{2.081932in}}%
\pgfusepath{clip}%
\pgfsetbuttcap%
\pgfsetroundjoin%
\definecolor{currentfill}{rgb}{0.855810,0.888601,0.097452}%
\pgfsetfillcolor{currentfill}%
\pgfsetlinewidth{0.000000pt}%
\definecolor{currentstroke}{rgb}{0.000000,0.000000,0.000000}%
\pgfsetstrokecolor{currentstroke}%
\pgfsetdash{}{0pt}%
\pgfpathmoveto{\pgfqpoint{3.461219in}{1.109078in}}%
\pgfpathlineto{\pgfqpoint{3.466620in}{1.110891in}}%
\pgfpathlineto{\pgfqpoint{3.471884in}{1.112075in}}%
\pgfpathlineto{\pgfqpoint{3.476991in}{1.112624in}}%
\pgfpathlineto{\pgfqpoint{3.481918in}{1.112533in}}%
\pgfpathlineto{\pgfqpoint{3.486645in}{1.111802in}}%
\pgfpathlineto{\pgfqpoint{3.472164in}{1.116184in}}%
\pgfpathlineto{\pgfqpoint{3.458239in}{1.121019in}}%
\pgfpathlineto{\pgfqpoint{3.444923in}{1.126291in}}%
\pgfpathlineto{\pgfqpoint{3.432269in}{1.131980in}}%
\pgfpathlineto{\pgfqpoint{3.420328in}{1.138066in}}%
\pgfpathlineto{\pgfqpoint{3.413175in}{1.139753in}}%
\pgfpathlineto{\pgfqpoint{3.405718in}{1.140837in}}%
\pgfpathlineto{\pgfqpoint{3.397988in}{1.141316in}}%
\pgfpathlineto{\pgfqpoint{3.390015in}{1.141188in}}%
\pgfpathlineto{\pgfqpoint{3.381833in}{1.140456in}}%
\pgfpathlineto{\pgfqpoint{3.396145in}{1.133181in}}%
\pgfpathlineto{\pgfqpoint{3.411301in}{1.126383in}}%
\pgfpathlineto{\pgfqpoint{3.427241in}{1.120085in}}%
\pgfpathlineto{\pgfqpoint{3.443902in}{1.114310in}}%
\pgfpathlineto{\pgfqpoint{3.461219in}{1.109078in}}%
\pgfpathclose%
\pgfusepath{fill}%
\end{pgfscope}%
\begin{pgfscope}%
\pgfpathrectangle{\pgfqpoint{2.548318in}{0.050000in}}{\pgfqpoint{2.081932in}{2.081932in}}%
\pgfusepath{clip}%
\pgfsetbuttcap%
\pgfsetroundjoin%
\definecolor{currentfill}{rgb}{0.606045,0.850733,0.236712}%
\pgfsetfillcolor{currentfill}%
\pgfsetlinewidth{0.000000pt}%
\definecolor{currentstroke}{rgb}{0.000000,0.000000,0.000000}%
\pgfsetstrokecolor{currentstroke}%
\pgfsetdash{}{0pt}%
\pgfpathmoveto{\pgfqpoint{3.648969in}{1.095537in}}%
\pgfpathlineto{\pgfqpoint{3.647889in}{1.094728in}}%
\pgfpathlineto{\pgfqpoint{3.646865in}{1.093256in}}%
\pgfpathlineto{\pgfqpoint{3.645903in}{1.091124in}}%
\pgfpathlineto{\pgfqpoint{3.645006in}{1.088339in}}%
\pgfpathlineto{\pgfqpoint{3.644178in}{1.084910in}}%
\pgfpathlineto{\pgfqpoint{3.629990in}{1.084293in}}%
\pgfpathlineto{\pgfqpoint{3.615754in}{1.084125in}}%
\pgfpathlineto{\pgfqpoint{3.601525in}{1.084407in}}%
\pgfpathlineto{\pgfqpoint{3.587357in}{1.085136in}}%
\pgfpathlineto{\pgfqpoint{3.573304in}{1.086312in}}%
\pgfpathlineto{\pgfqpoint{3.571931in}{1.089784in}}%
\pgfpathlineto{\pgfqpoint{3.570442in}{1.092617in}}%
\pgfpathlineto{\pgfqpoint{3.568845in}{1.094800in}}%
\pgfpathlineto{\pgfqpoint{3.567147in}{1.096327in}}%
\pgfpathlineto{\pgfqpoint{3.565354in}{1.097193in}}%
\pgfpathlineto{\pgfqpoint{3.581934in}{1.095804in}}%
\pgfpathlineto{\pgfqpoint{3.598650in}{1.094942in}}%
\pgfpathlineto{\pgfqpoint{3.615436in}{1.094609in}}%
\pgfpathlineto{\pgfqpoint{3.632231in}{1.094808in}}%
\pgfpathlineto{\pgfqpoint{3.648969in}{1.095537in}}%
\pgfpathclose%
\pgfusepath{fill}%
\end{pgfscope}%
\begin{pgfscope}%
\pgfpathrectangle{\pgfqpoint{2.548318in}{0.050000in}}{\pgfqpoint{2.081932in}{2.081932in}}%
\pgfusepath{clip}%
\pgfsetbuttcap%
\pgfsetroundjoin%
\definecolor{currentfill}{rgb}{0.268510,0.009605,0.335427}%
\pgfsetfillcolor{currentfill}%
\pgfsetlinewidth{0.000000pt}%
\definecolor{currentstroke}{rgb}{0.000000,0.000000,0.000000}%
\pgfsetstrokecolor{currentstroke}%
\pgfsetdash{}{0pt}%
\pgfpathmoveto{\pgfqpoint{3.540677in}{0.791869in}}%
\pgfpathlineto{\pgfqpoint{3.538378in}{0.786960in}}%
\pgfpathlineto{\pgfqpoint{3.536101in}{0.782652in}}%
\pgfpathlineto{\pgfqpoint{3.533857in}{0.778961in}}%
\pgfpathlineto{\pgfqpoint{3.531653in}{0.775902in}}%
\pgfpathlineto{\pgfqpoint{3.529500in}{0.773483in}}%
\pgfpathlineto{\pgfqpoint{3.501833in}{0.776547in}}%
\pgfpathlineto{\pgfqpoint{3.474585in}{0.780438in}}%
\pgfpathlineto{\pgfqpoint{3.447857in}{0.785144in}}%
\pgfpathlineto{\pgfqpoint{3.421746in}{0.790650in}}%
\pgfpathlineto{\pgfqpoint{3.396351in}{0.796938in}}%
\pgfpathlineto{\pgfqpoint{3.401782in}{0.798789in}}%
\pgfpathlineto{\pgfqpoint{3.407338in}{0.801270in}}%
\pgfpathlineto{\pgfqpoint{3.412996in}{0.804373in}}%
\pgfpathlineto{\pgfqpoint{3.418734in}{0.808087in}}%
\pgfpathlineto{\pgfqpoint{3.424528in}{0.812397in}}%
\pgfpathlineto{\pgfqpoint{3.446667in}{0.806896in}}%
\pgfpathlineto{\pgfqpoint{3.469437in}{0.802078in}}%
\pgfpathlineto{\pgfqpoint{3.492753in}{0.797958in}}%
\pgfpathlineto{\pgfqpoint{3.516530in}{0.794552in}}%
\pgfpathlineto{\pgfqpoint{3.540677in}{0.791869in}}%
\pgfpathclose%
\pgfusepath{fill}%
\end{pgfscope}%
\begin{pgfscope}%
\pgfpathrectangle{\pgfqpoint{2.548318in}{0.050000in}}{\pgfqpoint{2.081932in}{2.081932in}}%
\pgfusepath{clip}%
\pgfsetbuttcap%
\pgfsetroundjoin%
\definecolor{currentfill}{rgb}{0.206756,0.371758,0.553117}%
\pgfsetfillcolor{currentfill}%
\pgfsetlinewidth{0.000000pt}%
\definecolor{currentstroke}{rgb}{0.000000,0.000000,0.000000}%
\pgfsetstrokecolor{currentstroke}%
\pgfsetdash{}{0pt}%
\pgfpathmoveto{\pgfqpoint{4.095652in}{0.914453in}}%
\pgfpathlineto{\pgfqpoint{4.100841in}{0.920534in}}%
\pgfpathlineto{\pgfqpoint{4.105437in}{0.927034in}}%
\pgfpathlineto{\pgfqpoint{4.109424in}{0.933927in}}%
\pgfpathlineto{\pgfqpoint{4.112787in}{0.941186in}}%
\pgfpathlineto{\pgfqpoint{4.115513in}{0.948779in}}%
\pgfpathlineto{\pgfqpoint{4.096123in}{0.933705in}}%
\pgfpathlineto{\pgfqpoint{4.074946in}{0.919290in}}%
\pgfpathlineto{\pgfqpoint{4.052071in}{0.905585in}}%
\pgfpathlineto{\pgfqpoint{4.027596in}{0.892635in}}%
\pgfpathlineto{\pgfqpoint{4.001619in}{0.880482in}}%
\pgfpathlineto{\pgfqpoint{3.999540in}{0.873340in}}%
\pgfpathlineto{\pgfqpoint{3.996973in}{0.866615in}}%
\pgfpathlineto{\pgfqpoint{3.993930in}{0.860334in}}%
\pgfpathlineto{\pgfqpoint{3.990422in}{0.854524in}}%
\pgfpathlineto{\pgfqpoint{3.986460in}{0.849208in}}%
\pgfpathlineto{\pgfqpoint{4.011381in}{0.860821in}}%
\pgfpathlineto{\pgfqpoint{4.034854in}{0.873195in}}%
\pgfpathlineto{\pgfqpoint{4.056784in}{0.886288in}}%
\pgfpathlineto{\pgfqpoint{4.077079in}{0.900057in}}%
\pgfpathlineto{\pgfqpoint{4.095652in}{0.914453in}}%
\pgfpathclose%
\pgfusepath{fill}%
\end{pgfscope}%
\begin{pgfscope}%
\pgfpathrectangle{\pgfqpoint{2.548318in}{0.050000in}}{\pgfqpoint{2.081932in}{2.081932in}}%
\pgfusepath{clip}%
\pgfsetbuttcap%
\pgfsetroundjoin%
\definecolor{currentfill}{rgb}{0.993248,0.906157,0.143936}%
\pgfsetfillcolor{currentfill}%
\pgfsetlinewidth{0.000000pt}%
\definecolor{currentstroke}{rgb}{0.000000,0.000000,0.000000}%
\pgfsetstrokecolor{currentstroke}%
\pgfsetdash{}{0pt}%
\pgfpathmoveto{\pgfqpoint{3.949831in}{1.164606in}}%
\pgfpathlineto{\pgfqpoint{3.939366in}{1.167116in}}%
\pgfpathlineto{\pgfqpoint{3.928952in}{1.169067in}}%
\pgfpathlineto{\pgfqpoint{3.918632in}{1.170450in}}%
\pgfpathlineto{\pgfqpoint{3.908445in}{1.171260in}}%
\pgfpathlineto{\pgfqpoint{3.898433in}{1.171493in}}%
\pgfpathlineto{\pgfqpoint{3.887776in}{1.162681in}}%
\pgfpathlineto{\pgfqpoint{3.876075in}{1.154238in}}%
\pgfpathlineto{\pgfqpoint{3.863378in}{1.146194in}}%
\pgfpathlineto{\pgfqpoint{3.849738in}{1.138580in}}%
\pgfpathlineto{\pgfqpoint{3.835211in}{1.131422in}}%
\pgfpathlineto{\pgfqpoint{3.842921in}{1.129772in}}%
\pgfpathlineto{\pgfqpoint{3.850761in}{1.127524in}}%
\pgfpathlineto{\pgfqpoint{3.858701in}{1.124689in}}%
\pgfpathlineto{\pgfqpoint{3.866709in}{1.121278in}}%
\pgfpathlineto{\pgfqpoint{3.874754in}{1.117307in}}%
\pgfpathlineto{\pgfqpoint{3.891974in}{1.125748in}}%
\pgfpathlineto{\pgfqpoint{3.908157in}{1.134731in}}%
\pgfpathlineto{\pgfqpoint{3.923235in}{1.144225in}}%
\pgfpathlineto{\pgfqpoint{3.937146in}{1.154195in}}%
\pgfpathlineto{\pgfqpoint{3.949831in}{1.164606in}}%
\pgfpathclose%
\pgfusepath{fill}%
\end{pgfscope}%
\begin{pgfscope}%
\pgfpathrectangle{\pgfqpoint{2.548318in}{0.050000in}}{\pgfqpoint{2.081932in}{2.081932in}}%
\pgfusepath{clip}%
\pgfsetbuttcap%
\pgfsetroundjoin%
\definecolor{currentfill}{rgb}{0.150476,0.504369,0.557430}%
\pgfsetfillcolor{currentfill}%
\pgfsetlinewidth{0.000000pt}%
\definecolor{currentstroke}{rgb}{0.000000,0.000000,0.000000}%
\pgfsetstrokecolor{currentstroke}%
\pgfsetdash{}{0pt}%
\pgfpathmoveto{\pgfqpoint{3.201225in}{0.895824in}}%
\pgfpathlineto{\pgfqpoint{3.199501in}{0.903437in}}%
\pgfpathlineto{\pgfqpoint{3.198320in}{0.911389in}}%
\pgfpathlineto{\pgfqpoint{3.197683in}{0.919649in}}%
\pgfpathlineto{\pgfqpoint{3.197594in}{0.928183in}}%
\pgfpathlineto{\pgfqpoint{3.173289in}{0.941511in}}%
\pgfpathlineto{\pgfqpoint{3.150621in}{0.955594in}}%
\pgfpathlineto{\pgfqpoint{3.129687in}{0.970384in}}%
\pgfpathlineto{\pgfqpoint{3.110578in}{0.985831in}}%
\pgfpathlineto{\pgfqpoint{3.093377in}{1.001882in}}%
\pgfpathlineto{\pgfqpoint{3.093490in}{0.993232in}}%
\pgfpathlineto{\pgfqpoint{3.094293in}{0.984763in}}%
\pgfpathlineto{\pgfqpoint{3.095786in}{0.976509in}}%
\pgfpathlineto{\pgfqpoint{3.097962in}{0.968503in}}%
\pgfpathlineto{\pgfqpoint{3.115001in}{0.952676in}}%
\pgfpathlineto{\pgfqpoint{3.133934in}{0.937443in}}%
\pgfpathlineto{\pgfqpoint{3.154676in}{0.922857in}}%
\pgfpathlineto{\pgfqpoint{3.177138in}{0.908969in}}%
\pgfpathlineto{\pgfqpoint{3.201225in}{0.895824in}}%
\pgfpathclose%
\pgfusepath{fill}%
\end{pgfscope}%
\begin{pgfscope}%
\pgfpathrectangle{\pgfqpoint{2.548318in}{0.050000in}}{\pgfqpoint{2.081932in}{2.081932in}}%
\pgfusepath{clip}%
\pgfsetbuttcap%
\pgfsetroundjoin%
\definecolor{currentfill}{rgb}{0.606045,0.850733,0.236712}%
\pgfsetfillcolor{currentfill}%
\pgfsetlinewidth{0.000000pt}%
\definecolor{currentstroke}{rgb}{0.000000,0.000000,0.000000}%
\pgfsetstrokecolor{currentstroke}%
\pgfsetdash{}{0pt}%
\pgfpathmoveto{\pgfqpoint{3.729629in}{1.106980in}}%
\pgfpathlineto{\pgfqpoint{3.725770in}{1.105776in}}%
\pgfpathlineto{\pgfqpoint{3.722115in}{1.103928in}}%
\pgfpathlineto{\pgfqpoint{3.718680in}{1.101442in}}%
\pgfpathlineto{\pgfqpoint{3.715478in}{1.098326in}}%
\pgfpathlineto{\pgfqpoint{3.712523in}{1.094592in}}%
\pgfpathlineto{\pgfqpoint{3.699377in}{1.091796in}}%
\pgfpathlineto{\pgfqpoint{3.685917in}{1.089423in}}%
\pgfpathlineto{\pgfqpoint{3.672196in}{1.087480in}}%
\pgfpathlineto{\pgfqpoint{3.658265in}{1.085974in}}%
\pgfpathlineto{\pgfqpoint{3.644178in}{1.084910in}}%
\pgfpathlineto{\pgfqpoint{3.645006in}{1.088339in}}%
\pgfpathlineto{\pgfqpoint{3.645903in}{1.091124in}}%
\pgfpathlineto{\pgfqpoint{3.646865in}{1.093256in}}%
\pgfpathlineto{\pgfqpoint{3.647889in}{1.094728in}}%
\pgfpathlineto{\pgfqpoint{3.648969in}{1.095537in}}%
\pgfpathlineto{\pgfqpoint{3.665589in}{1.096793in}}%
\pgfpathlineto{\pgfqpoint{3.682027in}{1.098573in}}%
\pgfpathlineto{\pgfqpoint{3.698221in}{1.100870in}}%
\pgfpathlineto{\pgfqpoint{3.714108in}{1.103676in}}%
\pgfpathlineto{\pgfqpoint{3.729629in}{1.106980in}}%
\pgfpathclose%
\pgfusepath{fill}%
\end{pgfscope}%
\begin{pgfscope}%
\pgfpathrectangle{\pgfqpoint{2.548318in}{0.050000in}}{\pgfqpoint{2.081932in}{2.081932in}}%
\pgfusepath{clip}%
\pgfsetbuttcap%
\pgfsetroundjoin%
\definecolor{currentfill}{rgb}{0.282327,0.094955,0.417331}%
\pgfsetfillcolor{currentfill}%
\pgfsetlinewidth{0.000000pt}%
\definecolor{currentstroke}{rgb}{0.000000,0.000000,0.000000}%
\pgfsetstrokecolor{currentstroke}%
\pgfsetdash{}{0pt}%
\pgfpathmoveto{\pgfqpoint{3.396351in}{0.796938in}}%
\pgfpathlineto{\pgfqpoint{3.391066in}{0.795722in}}%
\pgfpathlineto{\pgfqpoint{3.385948in}{0.795145in}}%
\pgfpathlineto{\pgfqpoint{3.381018in}{0.795208in}}%
\pgfpathlineto{\pgfqpoint{3.376295in}{0.795907in}}%
\pgfpathlineto{\pgfqpoint{3.371797in}{0.797240in}}%
\pgfpathlineto{\pgfqpoint{3.344447in}{0.805073in}}%
\pgfpathlineto{\pgfqpoint{3.318090in}{0.813728in}}%
\pgfpathlineto{\pgfqpoint{3.292825in}{0.823177in}}%
\pgfpathlineto{\pgfqpoint{3.268752in}{0.833389in}}%
\pgfpathlineto{\pgfqpoint{3.245965in}{0.844330in}}%
\pgfpathlineto{\pgfqpoint{3.252817in}{0.842118in}}%
\pgfpathlineto{\pgfqpoint{3.260011in}{0.840502in}}%
\pgfpathlineto{\pgfqpoint{3.267517in}{0.839488in}}%
\pgfpathlineto{\pgfqpoint{3.275306in}{0.839082in}}%
\pgfpathlineto{\pgfqpoint{3.283347in}{0.839287in}}%
\pgfpathlineto{\pgfqpoint{3.303786in}{0.829453in}}%
\pgfpathlineto{\pgfqpoint{3.325393in}{0.820272in}}%
\pgfpathlineto{\pgfqpoint{3.348082in}{0.811773in}}%
\pgfpathlineto{\pgfqpoint{3.371765in}{0.803986in}}%
\pgfpathlineto{\pgfqpoint{3.396351in}{0.796938in}}%
\pgfpathclose%
\pgfusepath{fill}%
\end{pgfscope}%
\begin{pgfscope}%
\pgfpathrectangle{\pgfqpoint{2.548318in}{0.050000in}}{\pgfqpoint{2.081932in}{2.081932in}}%
\pgfusepath{clip}%
\pgfsetbuttcap%
\pgfsetroundjoin%
\definecolor{currentfill}{rgb}{0.606045,0.850733,0.236712}%
\pgfsetfillcolor{currentfill}%
\pgfsetlinewidth{0.000000pt}%
\definecolor{currentstroke}{rgb}{0.000000,0.000000,0.000000}%
\pgfsetstrokecolor{currentstroke}%
\pgfsetdash{}{0pt}%
\pgfpathmoveto{\pgfqpoint{3.565354in}{1.097193in}}%
\pgfpathlineto{\pgfqpoint{3.567147in}{1.096327in}}%
\pgfpathlineto{\pgfqpoint{3.568845in}{1.094800in}}%
\pgfpathlineto{\pgfqpoint{3.570442in}{1.092617in}}%
\pgfpathlineto{\pgfqpoint{3.571931in}{1.089784in}}%
\pgfpathlineto{\pgfqpoint{3.573304in}{1.086312in}}%
\pgfpathlineto{\pgfqpoint{3.559420in}{1.087928in}}%
\pgfpathlineto{\pgfqpoint{3.545758in}{1.089981in}}%
\pgfpathlineto{\pgfqpoint{3.532372in}{1.092461in}}%
\pgfpathlineto{\pgfqpoint{3.519311in}{1.095361in}}%
\pgfpathlineto{\pgfqpoint{3.506628in}{1.098669in}}%
\pgfpathlineto{\pgfqpoint{3.503177in}{1.102533in}}%
\pgfpathlineto{\pgfqpoint{3.499437in}{1.105789in}}%
\pgfpathlineto{\pgfqpoint{3.495423in}{1.108425in}}%
\pgfpathlineto{\pgfqpoint{3.491153in}{1.110431in}}%
\pgfpathlineto{\pgfqpoint{3.486645in}{1.111802in}}%
\pgfpathlineto{\pgfqpoint{3.501625in}{1.107890in}}%
\pgfpathlineto{\pgfqpoint{3.517045in}{1.104462in}}%
\pgfpathlineto{\pgfqpoint{3.532847in}{1.101529in}}%
\pgfpathlineto{\pgfqpoint{3.548971in}{1.099103in}}%
\pgfpathlineto{\pgfqpoint{3.565354in}{1.097193in}}%
\pgfpathclose%
\pgfusepath{fill}%
\end{pgfscope}%
\begin{pgfscope}%
\pgfpathrectangle{\pgfqpoint{2.548318in}{0.050000in}}{\pgfqpoint{2.081932in}{2.081932in}}%
\pgfusepath{clip}%
\pgfsetbuttcap%
\pgfsetroundjoin%
\definecolor{currentfill}{rgb}{0.993248,0.906157,0.143936}%
\pgfsetfillcolor{currentfill}%
\pgfsetlinewidth{0.000000pt}%
\definecolor{currentstroke}{rgb}{0.000000,0.000000,0.000000}%
\pgfsetstrokecolor{currentstroke}%
\pgfsetdash{}{0pt}%
\pgfpathmoveto{\pgfqpoint{3.338950in}{1.127962in}}%
\pgfpathlineto{\pgfqpoint{3.347676in}{1.131605in}}%
\pgfpathlineto{\pgfqpoint{3.356361in}{1.134689in}}%
\pgfpathlineto{\pgfqpoint{3.364972in}{1.137199in}}%
\pgfpathlineto{\pgfqpoint{3.373474in}{1.139124in}}%
\pgfpathlineto{\pgfqpoint{3.381833in}{1.140456in}}%
\pgfpathlineto{\pgfqpoint{3.368422in}{1.148182in}}%
\pgfpathlineto{\pgfqpoint{3.355967in}{1.156329in}}%
\pgfpathlineto{\pgfqpoint{3.344521in}{1.164868in}}%
\pgfpathlineto{\pgfqpoint{3.334130in}{1.173768in}}%
\pgfpathlineto{\pgfqpoint{3.324841in}{1.182995in}}%
\pgfpathlineto{\pgfqpoint{3.314390in}{1.183173in}}%
\pgfpathlineto{\pgfqpoint{3.303756in}{1.182779in}}%
\pgfpathlineto{\pgfqpoint{3.292981in}{1.181816in}}%
\pgfpathlineto{\pgfqpoint{3.282107in}{1.180288in}}%
\pgfpathlineto{\pgfqpoint{3.271178in}{1.178203in}}%
\pgfpathlineto{\pgfqpoint{3.282257in}{1.167295in}}%
\pgfpathlineto{\pgfqpoint{3.294630in}{1.156779in}}%
\pgfpathlineto{\pgfqpoint{3.308242in}{1.146694in}}%
\pgfpathlineto{\pgfqpoint{3.323036in}{1.137077in}}%
\pgfpathlineto{\pgfqpoint{3.338950in}{1.127962in}}%
\pgfpathclose%
\pgfusepath{fill}%
\end{pgfscope}%
\begin{pgfscope}%
\pgfpathrectangle{\pgfqpoint{2.548318in}{0.050000in}}{\pgfqpoint{2.081932in}{2.081932in}}%
\pgfusepath{clip}%
\pgfsetbuttcap%
\pgfsetroundjoin%
\definecolor{currentfill}{rgb}{0.267004,0.004874,0.329415}%
\pgfsetfillcolor{currentfill}%
\pgfsetlinewidth{0.000000pt}%
\definecolor{currentstroke}{rgb}{0.000000,0.000000,0.000000}%
\pgfsetstrokecolor{currentstroke}%
\pgfsetdash{}{0pt}%
\pgfpathmoveto{\pgfqpoint{3.656908in}{0.822724in}}%
\pgfpathlineto{\pgfqpoint{3.658276in}{0.815066in}}%
\pgfpathlineto{\pgfqpoint{3.659660in}{0.807891in}}%
\pgfpathlineto{\pgfqpoint{3.661052in}{0.801230in}}%
\pgfpathlineto{\pgfqpoint{3.662448in}{0.795106in}}%
\pgfpathlineto{\pgfqpoint{3.663842in}{0.789545in}}%
\pgfpathlineto{\pgfqpoint{3.639186in}{0.788523in}}%
\pgfpathlineto{\pgfqpoint{3.614451in}{0.788244in}}%
\pgfpathlineto{\pgfqpoint{3.589728in}{0.788710in}}%
\pgfpathlineto{\pgfqpoint{3.565106in}{0.789920in}}%
\pgfpathlineto{\pgfqpoint{3.540677in}{0.791869in}}%
\pgfpathlineto{\pgfqpoint{3.542990in}{0.797362in}}%
\pgfpathlineto{\pgfqpoint{3.545306in}{0.803418in}}%
\pgfpathlineto{\pgfqpoint{3.547617in}{0.810012in}}%
\pgfpathlineto{\pgfqpoint{3.549913in}{0.817119in}}%
\pgfpathlineto{\pgfqpoint{3.552184in}{0.824711in}}%
\pgfpathlineto{\pgfqpoint{3.572953in}{0.823044in}}%
\pgfpathlineto{\pgfqpoint{3.593888in}{0.822010in}}%
\pgfpathlineto{\pgfqpoint{3.614911in}{0.821611in}}%
\pgfpathlineto{\pgfqpoint{3.635944in}{0.821850in}}%
\pgfpathlineto{\pgfqpoint{3.656908in}{0.822724in}}%
\pgfpathclose%
\pgfusepath{fill}%
\end{pgfscope}%
\begin{pgfscope}%
\pgfpathrectangle{\pgfqpoint{2.548318in}{0.050000in}}{\pgfqpoint{2.081932in}{2.081932in}}%
\pgfusepath{clip}%
\pgfsetbuttcap%
\pgfsetroundjoin%
\definecolor{currentfill}{rgb}{0.206756,0.371758,0.553117}%
\pgfsetfillcolor{currentfill}%
\pgfsetlinewidth{0.000000pt}%
\definecolor{currentstroke}{rgb}{0.000000,0.000000,0.000000}%
\pgfsetstrokecolor{currentstroke}%
\pgfsetdash{}{0pt}%
\pgfpathmoveto{\pgfqpoint{3.217695in}{0.863869in}}%
\pgfpathlineto{\pgfqpoint{3.213391in}{0.869356in}}%
\pgfpathlineto{\pgfqpoint{3.209579in}{0.875320in}}%
\pgfpathlineto{\pgfqpoint{3.206273in}{0.881737in}}%
\pgfpathlineto{\pgfqpoint{3.203485in}{0.888581in}}%
\pgfpathlineto{\pgfqpoint{3.201225in}{0.895824in}}%
\pgfpathlineto{\pgfqpoint{3.177138in}{0.908969in}}%
\pgfpathlineto{\pgfqpoint{3.154676in}{0.922857in}}%
\pgfpathlineto{\pgfqpoint{3.133934in}{0.937443in}}%
\pgfpathlineto{\pgfqpoint{3.115001in}{0.952676in}}%
\pgfpathlineto{\pgfqpoint{3.097962in}{0.968503in}}%
\pgfpathlineto{\pgfqpoint{3.100815in}{0.960778in}}%
\pgfpathlineto{\pgfqpoint{3.104334in}{0.953364in}}%
\pgfpathlineto{\pgfqpoint{3.108506in}{0.946291in}}%
\pgfpathlineto{\pgfqpoint{3.113316in}{0.939589in}}%
\pgfpathlineto{\pgfqpoint{3.118746in}{0.933285in}}%
\pgfpathlineto{\pgfqpoint{3.135055in}{0.918174in}}%
\pgfpathlineto{\pgfqpoint{3.153187in}{0.903627in}}%
\pgfpathlineto{\pgfqpoint{3.173063in}{0.889696in}}%
\pgfpathlineto{\pgfqpoint{3.194596in}{0.876428in}}%
\pgfpathlineto{\pgfqpoint{3.217695in}{0.863869in}}%
\pgfpathclose%
\pgfusepath{fill}%
\end{pgfscope}%
\begin{pgfscope}%
\pgfpathrectangle{\pgfqpoint{2.548318in}{0.050000in}}{\pgfqpoint{2.081932in}{2.081932in}}%
\pgfusepath{clip}%
\pgfsetbuttcap%
\pgfsetroundjoin%
\definecolor{currentfill}{rgb}{0.606045,0.850733,0.236712}%
\pgfsetfillcolor{currentfill}%
\pgfsetlinewidth{0.000000pt}%
\definecolor{currentstroke}{rgb}{0.000000,0.000000,0.000000}%
\pgfsetstrokecolor{currentstroke}%
\pgfsetdash{}{0pt}%
\pgfpathmoveto{\pgfqpoint{3.799697in}{1.130509in}}%
\pgfpathlineto{\pgfqpoint{3.793405in}{1.128488in}}%
\pgfpathlineto{\pgfqpoint{3.787446in}{1.125864in}}%
\pgfpathlineto{\pgfqpoint{3.781846in}{1.122647in}}%
\pgfpathlineto{\pgfqpoint{3.776628in}{1.118848in}}%
\pgfpathlineto{\pgfqpoint{3.771814in}{1.114481in}}%
\pgfpathlineto{\pgfqpoint{3.760973in}{1.109761in}}%
\pgfpathlineto{\pgfqpoint{3.749577in}{1.105397in}}%
\pgfpathlineto{\pgfqpoint{3.737672in}{1.101404in}}%
\pgfpathlineto{\pgfqpoint{3.725305in}{1.097798in}}%
\pgfpathlineto{\pgfqpoint{3.712523in}{1.094592in}}%
\pgfpathlineto{\pgfqpoint{3.715478in}{1.098326in}}%
\pgfpathlineto{\pgfqpoint{3.718680in}{1.101442in}}%
\pgfpathlineto{\pgfqpoint{3.722115in}{1.103928in}}%
\pgfpathlineto{\pgfqpoint{3.725770in}{1.105776in}}%
\pgfpathlineto{\pgfqpoint{3.729629in}{1.106980in}}%
\pgfpathlineto{\pgfqpoint{3.744724in}{1.110771in}}%
\pgfpathlineto{\pgfqpoint{3.759334in}{1.115036in}}%
\pgfpathlineto{\pgfqpoint{3.773402in}{1.119759in}}%
\pgfpathlineto{\pgfqpoint{3.786874in}{1.124923in}}%
\pgfpathlineto{\pgfqpoint{3.799697in}{1.130509in}}%
\pgfpathclose%
\pgfusepath{fill}%
\end{pgfscope}%
\begin{pgfscope}%
\pgfpathrectangle{\pgfqpoint{2.548318in}{0.050000in}}{\pgfqpoint{2.081932in}{2.081932in}}%
\pgfusepath{clip}%
\pgfsetbuttcap%
\pgfsetroundjoin%
\definecolor{currentfill}{rgb}{0.268510,0.009605,0.335427}%
\pgfsetfillcolor{currentfill}%
\pgfsetlinewidth{0.000000pt}%
\definecolor{currentstroke}{rgb}{0.000000,0.000000,0.000000}%
\pgfsetstrokecolor{currentstroke}%
\pgfsetdash{}{0pt}%
\pgfpathmoveto{\pgfqpoint{3.886532in}{0.838746in}}%
\pgfpathlineto{\pgfqpoint{3.894644in}{0.835211in}}%
\pgfpathlineto{\pgfqpoint{3.902679in}{0.832268in}}%
\pgfpathlineto{\pgfqpoint{3.910605in}{0.829926in}}%
\pgfpathlineto{\pgfqpoint{3.918389in}{0.828194in}}%
\pgfpathlineto{\pgfqpoint{3.926000in}{0.827077in}}%
\pgfpathlineto{\pgfqpoint{3.904117in}{0.818064in}}%
\pgfpathlineto{\pgfqpoint{3.881173in}{0.809743in}}%
\pgfpathlineto{\pgfqpoint{3.857257in}{0.802140in}}%
\pgfpathlineto{\pgfqpoint{3.832462in}{0.795281in}}%
\pgfpathlineto{\pgfqpoint{3.806879in}{0.789188in}}%
\pgfpathlineto{\pgfqpoint{3.802235in}{0.791227in}}%
\pgfpathlineto{\pgfqpoint{3.797484in}{0.793900in}}%
\pgfpathlineto{\pgfqpoint{3.792645in}{0.797198in}}%
\pgfpathlineto{\pgfqpoint{3.787738in}{0.801108in}}%
\pgfpathlineto{\pgfqpoint{3.782782in}{0.805617in}}%
\pgfpathlineto{\pgfqpoint{3.805085in}{0.810947in}}%
\pgfpathlineto{\pgfqpoint{3.826692in}{0.816947in}}%
\pgfpathlineto{\pgfqpoint{3.847523in}{0.823596in}}%
\pgfpathlineto{\pgfqpoint{3.867495in}{0.830870in}}%
\pgfpathlineto{\pgfqpoint{3.886532in}{0.838746in}}%
\pgfpathclose%
\pgfusepath{fill}%
\end{pgfscope}%
\begin{pgfscope}%
\pgfpathrectangle{\pgfqpoint{2.548318in}{0.050000in}}{\pgfqpoint{2.081932in}{2.081932in}}%
\pgfusepath{clip}%
\pgfsetbuttcap%
\pgfsetroundjoin%
\definecolor{currentfill}{rgb}{0.267968,0.223549,0.512008}%
\pgfsetfillcolor{currentfill}%
\pgfsetlinewidth{0.000000pt}%
\definecolor{currentstroke}{rgb}{0.000000,0.000000,0.000000}%
\pgfsetstrokecolor{currentstroke}%
\pgfsetdash{}{0pt}%
\pgfpathmoveto{\pgfqpoint{4.061592in}{0.891163in}}%
\pgfpathlineto{\pgfqpoint{4.069401in}{0.894804in}}%
\pgfpathlineto{\pgfqpoint{4.076740in}{0.898972in}}%
\pgfpathlineto{\pgfqpoint{4.083578in}{0.903650in}}%
\pgfpathlineto{\pgfqpoint{4.089890in}{0.908817in}}%
\pgfpathlineto{\pgfqpoint{4.095652in}{0.914453in}}%
\pgfpathlineto{\pgfqpoint{4.077079in}{0.900057in}}%
\pgfpathlineto{\pgfqpoint{4.056784in}{0.886288in}}%
\pgfpathlineto{\pgfqpoint{4.034854in}{0.873195in}}%
\pgfpathlineto{\pgfqpoint{4.011381in}{0.860821in}}%
\pgfpathlineto{\pgfqpoint{3.986460in}{0.849208in}}%
\pgfpathlineto{\pgfqpoint{3.982059in}{0.844408in}}%
\pgfpathlineto{\pgfqpoint{3.977237in}{0.840145in}}%
\pgfpathlineto{\pgfqpoint{3.972012in}{0.836436in}}%
\pgfpathlineto{\pgfqpoint{3.966403in}{0.833298in}}%
\pgfpathlineto{\pgfqpoint{3.960431in}{0.830745in}}%
\pgfpathlineto{\pgfqpoint{3.983546in}{0.841506in}}%
\pgfpathlineto{\pgfqpoint{4.005306in}{0.852968in}}%
\pgfpathlineto{\pgfqpoint{4.025623in}{0.865093in}}%
\pgfpathlineto{\pgfqpoint{4.044411in}{0.877840in}}%
\pgfpathlineto{\pgfqpoint{4.061592in}{0.891163in}}%
\pgfpathclose%
\pgfusepath{fill}%
\end{pgfscope}%
\begin{pgfscope}%
\pgfpathrectangle{\pgfqpoint{2.548318in}{0.050000in}}{\pgfqpoint{2.081932in}{2.081932in}}%
\pgfusepath{clip}%
\pgfsetbuttcap%
\pgfsetroundjoin%
\definecolor{currentfill}{rgb}{0.855810,0.888601,0.097452}%
\pgfsetfillcolor{currentfill}%
\pgfsetlinewidth{0.000000pt}%
\definecolor{currentstroke}{rgb}{0.000000,0.000000,0.000000}%
\pgfsetstrokecolor{currentstroke}%
\pgfsetdash{}{0pt}%
\pgfpathmoveto{\pgfqpoint{3.898433in}{1.171493in}}%
\pgfpathlineto{\pgfqpoint{3.888636in}{1.171146in}}%
\pgfpathlineto{\pgfqpoint{3.879093in}{1.170219in}}%
\pgfpathlineto{\pgfqpoint{3.869843in}{1.168717in}}%
\pgfpathlineto{\pgfqpoint{3.860923in}{1.166642in}}%
\pgfpathlineto{\pgfqpoint{3.852369in}{1.164003in}}%
\pgfpathlineto{\pgfqpoint{3.843509in}{1.156643in}}%
\pgfpathlineto{\pgfqpoint{3.833770in}{1.149588in}}%
\pgfpathlineto{\pgfqpoint{3.823193in}{1.142864in}}%
\pgfpathlineto{\pgfqpoint{3.811819in}{1.136496in}}%
\pgfpathlineto{\pgfqpoint{3.799697in}{1.130509in}}%
\pgfpathlineto{\pgfqpoint{3.806297in}{1.131920in}}%
\pgfpathlineto{\pgfqpoint{3.813177in}{1.132717in}}%
\pgfpathlineto{\pgfqpoint{3.820309in}{1.132899in}}%
\pgfpathlineto{\pgfqpoint{3.827664in}{1.132467in}}%
\pgfpathlineto{\pgfqpoint{3.835211in}{1.131422in}}%
\pgfpathlineto{\pgfqpoint{3.849738in}{1.138580in}}%
\pgfpathlineto{\pgfqpoint{3.863378in}{1.146194in}}%
\pgfpathlineto{\pgfqpoint{3.876075in}{1.154238in}}%
\pgfpathlineto{\pgfqpoint{3.887776in}{1.162681in}}%
\pgfpathlineto{\pgfqpoint{3.898433in}{1.171493in}}%
\pgfpathclose%
\pgfusepath{fill}%
\end{pgfscope}%
\begin{pgfscope}%
\pgfpathrectangle{\pgfqpoint{2.548318in}{0.050000in}}{\pgfqpoint{2.081932in}{2.081932in}}%
\pgfusepath{clip}%
\pgfsetbuttcap%
\pgfsetroundjoin%
\definecolor{currentfill}{rgb}{0.267004,0.004874,0.329415}%
\pgfsetfillcolor{currentfill}%
\pgfsetlinewidth{0.000000pt}%
\definecolor{currentstroke}{rgb}{0.000000,0.000000,0.000000}%
\pgfsetstrokecolor{currentstroke}%
\pgfsetdash{}{0pt}%
\pgfpathmoveto{\pgfqpoint{3.757988in}{0.836458in}}%
\pgfpathlineto{\pgfqpoint{3.762880in}{0.829260in}}%
\pgfpathlineto{\pgfqpoint{3.767826in}{0.822552in}}%
\pgfpathlineto{\pgfqpoint{3.772805in}{0.816359in}}%
\pgfpathlineto{\pgfqpoint{3.777797in}{0.810707in}}%
\pgfpathlineto{\pgfqpoint{3.782782in}{0.805617in}}%
\pgfpathlineto{\pgfqpoint{3.759868in}{0.800973in}}%
\pgfpathlineto{\pgfqpoint{3.736429in}{0.797032in}}%
\pgfpathlineto{\pgfqpoint{3.712553in}{0.793807in}}%
\pgfpathlineto{\pgfqpoint{3.688327in}{0.791309in}}%
\pgfpathlineto{\pgfqpoint{3.663842in}{0.789545in}}%
\pgfpathlineto{\pgfqpoint{3.662448in}{0.795106in}}%
\pgfpathlineto{\pgfqpoint{3.661052in}{0.801230in}}%
\pgfpathlineto{\pgfqpoint{3.659660in}{0.807891in}}%
\pgfpathlineto{\pgfqpoint{3.658276in}{0.815066in}}%
\pgfpathlineto{\pgfqpoint{3.656908in}{0.822724in}}%
\pgfpathlineto{\pgfqpoint{3.677725in}{0.824231in}}%
\pgfpathlineto{\pgfqpoint{3.698318in}{0.826367in}}%
\pgfpathlineto{\pgfqpoint{3.718610in}{0.829123in}}%
\pgfpathlineto{\pgfqpoint{3.738525in}{0.832491in}}%
\pgfpathlineto{\pgfqpoint{3.757988in}{0.836458in}}%
\pgfpathclose%
\pgfusepath{fill}%
\end{pgfscope}%
\begin{pgfscope}%
\pgfpathrectangle{\pgfqpoint{2.548318in}{0.050000in}}{\pgfqpoint{2.081932in}{2.081932in}}%
\pgfusepath{clip}%
\pgfsetbuttcap%
\pgfsetroundjoin%
\definecolor{currentfill}{rgb}{0.606045,0.850733,0.236712}%
\pgfsetfillcolor{currentfill}%
\pgfsetlinewidth{0.000000pt}%
\definecolor{currentstroke}{rgb}{0.000000,0.000000,0.000000}%
\pgfsetstrokecolor{currentstroke}%
\pgfsetdash{}{0pt}%
\pgfpathmoveto{\pgfqpoint{3.486645in}{1.111802in}}%
\pgfpathlineto{\pgfqpoint{3.491153in}{1.110431in}}%
\pgfpathlineto{\pgfqpoint{3.495423in}{1.108425in}}%
\pgfpathlineto{\pgfqpoint{3.499437in}{1.105789in}}%
\pgfpathlineto{\pgfqpoint{3.503177in}{1.102533in}}%
\pgfpathlineto{\pgfqpoint{3.506628in}{1.098669in}}%
\pgfpathlineto{\pgfqpoint{3.494371in}{1.102374in}}%
\pgfpathlineto{\pgfqpoint{3.482588in}{1.106462in}}%
\pgfpathlineto{\pgfqpoint{3.471325in}{1.110918in}}%
\pgfpathlineto{\pgfqpoint{3.460628in}{1.115724in}}%
\pgfpathlineto{\pgfqpoint{3.450539in}{1.120865in}}%
\pgfpathlineto{\pgfqpoint{3.445325in}{1.125436in}}%
\pgfpathlineto{\pgfqpoint{3.439672in}{1.129455in}}%
\pgfpathlineto{\pgfqpoint{3.433604in}{1.132908in}}%
\pgfpathlineto{\pgfqpoint{3.427147in}{1.135782in}}%
\pgfpathlineto{\pgfqpoint{3.420328in}{1.138066in}}%
\pgfpathlineto{\pgfqpoint{3.432269in}{1.131980in}}%
\pgfpathlineto{\pgfqpoint{3.444923in}{1.126291in}}%
\pgfpathlineto{\pgfqpoint{3.458239in}{1.121019in}}%
\pgfpathlineto{\pgfqpoint{3.472164in}{1.116184in}}%
\pgfpathlineto{\pgfqpoint{3.486645in}{1.111802in}}%
\pgfpathclose%
\pgfusepath{fill}%
\end{pgfscope}%
\begin{pgfscope}%
\pgfpathrectangle{\pgfqpoint{2.548318in}{0.050000in}}{\pgfqpoint{2.081932in}{2.081932in}}%
\pgfusepath{clip}%
\pgfsetbuttcap%
\pgfsetroundjoin%
\definecolor{currentfill}{rgb}{0.296479,0.761561,0.424223}%
\pgfsetfillcolor{currentfill}%
\pgfsetlinewidth{0.000000pt}%
\definecolor{currentstroke}{rgb}{0.000000,0.000000,0.000000}%
\pgfsetstrokecolor{currentstroke}%
\pgfsetdash{}{0pt}%
\pgfpathmoveto{\pgfqpoint{3.644178in}{1.084910in}}%
\pgfpathlineto{\pgfqpoint{3.643424in}{1.080851in}}%
\pgfpathlineto{\pgfqpoint{3.642745in}{1.076176in}}%
\pgfpathlineto{\pgfqpoint{3.642146in}{1.070903in}}%
\pgfpathlineto{\pgfqpoint{3.641628in}{1.065053in}}%
\pgfpathlineto{\pgfqpoint{3.641195in}{1.058648in}}%
\pgfpathlineto{\pgfqpoint{3.628595in}{1.058102in}}%
\pgfpathlineto{\pgfqpoint{3.615952in}{1.057954in}}%
\pgfpathlineto{\pgfqpoint{3.603316in}{1.058203in}}%
\pgfpathlineto{\pgfqpoint{3.590733in}{1.058848in}}%
\pgfpathlineto{\pgfqpoint{3.578254in}{1.059887in}}%
\pgfpathlineto{\pgfqpoint{3.577535in}{1.066316in}}%
\pgfpathlineto{\pgfqpoint{3.576677in}{1.072195in}}%
\pgfpathlineto{\pgfqpoint{3.575682in}{1.077500in}}%
\pgfpathlineto{\pgfqpoint{3.574556in}{1.082212in}}%
\pgfpathlineto{\pgfqpoint{3.573304in}{1.086312in}}%
\pgfpathlineto{\pgfqpoint{3.587357in}{1.085136in}}%
\pgfpathlineto{\pgfqpoint{3.601525in}{1.084407in}}%
\pgfpathlineto{\pgfqpoint{3.615754in}{1.084125in}}%
\pgfpathlineto{\pgfqpoint{3.629990in}{1.084293in}}%
\pgfpathlineto{\pgfqpoint{3.644178in}{1.084910in}}%
\pgfpathclose%
\pgfusepath{fill}%
\end{pgfscope}%
\begin{pgfscope}%
\pgfpathrectangle{\pgfqpoint{2.548318in}{0.050000in}}{\pgfqpoint{2.081932in}{2.081932in}}%
\pgfusepath{clip}%
\pgfsetbuttcap%
\pgfsetroundjoin%
\definecolor{currentfill}{rgb}{0.267004,0.004874,0.329415}%
\pgfsetfillcolor{currentfill}%
\pgfsetlinewidth{0.000000pt}%
\definecolor{currentstroke}{rgb}{0.000000,0.000000,0.000000}%
\pgfsetstrokecolor{currentstroke}%
\pgfsetdash{}{0pt}%
\pgfpathmoveto{\pgfqpoint{3.552184in}{0.824711in}}%
\pgfpathlineto{\pgfqpoint{3.549913in}{0.817119in}}%
\pgfpathlineto{\pgfqpoint{3.547617in}{0.810012in}}%
\pgfpathlineto{\pgfqpoint{3.545306in}{0.803418in}}%
\pgfpathlineto{\pgfqpoint{3.542990in}{0.797362in}}%
\pgfpathlineto{\pgfqpoint{3.540677in}{0.791869in}}%
\pgfpathlineto{\pgfqpoint{3.516530in}{0.794552in}}%
\pgfpathlineto{\pgfqpoint{3.492753in}{0.797958in}}%
\pgfpathlineto{\pgfqpoint{3.469437in}{0.802078in}}%
\pgfpathlineto{\pgfqpoint{3.446667in}{0.806896in}}%
\pgfpathlineto{\pgfqpoint{3.424528in}{0.812397in}}%
\pgfpathlineto{\pgfqpoint{3.430355in}{0.817288in}}%
\pgfpathlineto{\pgfqpoint{3.436191in}{0.822741in}}%
\pgfpathlineto{\pgfqpoint{3.442012in}{0.828735in}}%
\pgfpathlineto{\pgfqpoint{3.447793in}{0.835246in}}%
\pgfpathlineto{\pgfqpoint{3.453511in}{0.842248in}}%
\pgfpathlineto{\pgfqpoint{3.472305in}{0.837550in}}%
\pgfpathlineto{\pgfqpoint{3.491644in}{0.833434in}}%
\pgfpathlineto{\pgfqpoint{3.511453in}{0.829914in}}%
\pgfpathlineto{\pgfqpoint{3.531659in}{0.827003in}}%
\pgfpathlineto{\pgfqpoint{3.552184in}{0.824711in}}%
\pgfpathclose%
\pgfusepath{fill}%
\end{pgfscope}%
\begin{pgfscope}%
\pgfpathrectangle{\pgfqpoint{2.548318in}{0.050000in}}{\pgfqpoint{2.081932in}{2.081932in}}%
\pgfusepath{clip}%
\pgfsetbuttcap%
\pgfsetroundjoin%
\definecolor{currentfill}{rgb}{0.296479,0.761561,0.424223}%
\pgfsetfillcolor{currentfill}%
\pgfsetlinewidth{0.000000pt}%
\definecolor{currentstroke}{rgb}{0.000000,0.000000,0.000000}%
\pgfsetstrokecolor{currentstroke}%
\pgfsetdash{}{0pt}%
\pgfpathmoveto{\pgfqpoint{3.712523in}{1.094592in}}%
\pgfpathlineto{\pgfqpoint{3.709829in}{1.090252in}}%
\pgfpathlineto{\pgfqpoint{3.707408in}{1.085324in}}%
\pgfpathlineto{\pgfqpoint{3.705269in}{1.079826in}}%
\pgfpathlineto{\pgfqpoint{3.703422in}{1.073780in}}%
\pgfpathlineto{\pgfqpoint{3.701876in}{1.067209in}}%
\pgfpathlineto{\pgfqpoint{3.690207in}{1.064738in}}%
\pgfpathlineto{\pgfqpoint{3.678257in}{1.062639in}}%
\pgfpathlineto{\pgfqpoint{3.666074in}{1.060921in}}%
\pgfpathlineto{\pgfqpoint{3.653704in}{1.059589in}}%
\pgfpathlineto{\pgfqpoint{3.641195in}{1.058648in}}%
\pgfpathlineto{\pgfqpoint{3.641628in}{1.065053in}}%
\pgfpathlineto{\pgfqpoint{3.642146in}{1.070903in}}%
\pgfpathlineto{\pgfqpoint{3.642745in}{1.076176in}}%
\pgfpathlineto{\pgfqpoint{3.643424in}{1.080851in}}%
\pgfpathlineto{\pgfqpoint{3.644178in}{1.084910in}}%
\pgfpathlineto{\pgfqpoint{3.658265in}{1.085974in}}%
\pgfpathlineto{\pgfqpoint{3.672196in}{1.087480in}}%
\pgfpathlineto{\pgfqpoint{3.685917in}{1.089423in}}%
\pgfpathlineto{\pgfqpoint{3.699377in}{1.091796in}}%
\pgfpathlineto{\pgfqpoint{3.712523in}{1.094592in}}%
\pgfpathclose%
\pgfusepath{fill}%
\end{pgfscope}%
\begin{pgfscope}%
\pgfpathrectangle{\pgfqpoint{2.548318in}{0.050000in}}{\pgfqpoint{2.081932in}{2.081932in}}%
\pgfusepath{clip}%
\pgfsetbuttcap%
\pgfsetroundjoin%
\definecolor{currentfill}{rgb}{0.855810,0.888601,0.097452}%
\pgfsetfillcolor{currentfill}%
\pgfsetlinewidth{0.000000pt}%
\definecolor{currentstroke}{rgb}{0.000000,0.000000,0.000000}%
\pgfsetstrokecolor{currentstroke}%
\pgfsetdash{}{0pt}%
\pgfpathmoveto{\pgfqpoint{3.381833in}{1.140456in}}%
\pgfpathlineto{\pgfqpoint{3.390015in}{1.141188in}}%
\pgfpathlineto{\pgfqpoint{3.397988in}{1.141316in}}%
\pgfpathlineto{\pgfqpoint{3.405718in}{1.140837in}}%
\pgfpathlineto{\pgfqpoint{3.413175in}{1.139753in}}%
\pgfpathlineto{\pgfqpoint{3.420328in}{1.138066in}}%
\pgfpathlineto{\pgfqpoint{3.409148in}{1.144526in}}%
\pgfpathlineto{\pgfqpoint{3.398774in}{1.151336in}}%
\pgfpathlineto{\pgfqpoint{3.389250in}{1.158470in}}%
\pgfpathlineto{\pgfqpoint{3.380615in}{1.165903in}}%
\pgfpathlineto{\pgfqpoint{3.372905in}{1.173606in}}%
\pgfpathlineto{\pgfqpoint{3.363982in}{1.176599in}}%
\pgfpathlineto{\pgfqpoint{3.354676in}{1.179042in}}%
\pgfpathlineto{\pgfqpoint{3.345024in}{1.180926in}}%
\pgfpathlineto{\pgfqpoint{3.335066in}{1.182245in}}%
\pgfpathlineto{\pgfqpoint{3.324841in}{1.182995in}}%
\pgfpathlineto{\pgfqpoint{3.334130in}{1.173768in}}%
\pgfpathlineto{\pgfqpoint{3.344521in}{1.164868in}}%
\pgfpathlineto{\pgfqpoint{3.355967in}{1.156329in}}%
\pgfpathlineto{\pgfqpoint{3.368422in}{1.148182in}}%
\pgfpathlineto{\pgfqpoint{3.381833in}{1.140456in}}%
\pgfpathclose%
\pgfusepath{fill}%
\end{pgfscope}%
\begin{pgfscope}%
\pgfpathrectangle{\pgfqpoint{2.548318in}{0.050000in}}{\pgfqpoint{2.081932in}{2.081932in}}%
\pgfusepath{clip}%
\pgfsetbuttcap%
\pgfsetroundjoin%
\definecolor{currentfill}{rgb}{0.268510,0.009605,0.335427}%
\pgfsetfillcolor{currentfill}%
\pgfsetlinewidth{0.000000pt}%
\definecolor{currentstroke}{rgb}{0.000000,0.000000,0.000000}%
\pgfsetstrokecolor{currentstroke}%
\pgfsetdash{}{0pt}%
\pgfpathmoveto{\pgfqpoint{3.424528in}{0.812397in}}%
\pgfpathlineto{\pgfqpoint{3.418734in}{0.808087in}}%
\pgfpathlineto{\pgfqpoint{3.412996in}{0.804373in}}%
\pgfpathlineto{\pgfqpoint{3.407338in}{0.801270in}}%
\pgfpathlineto{\pgfqpoint{3.401782in}{0.798789in}}%
\pgfpathlineto{\pgfqpoint{3.396351in}{0.796938in}}%
\pgfpathlineto{\pgfqpoint{3.371765in}{0.803986in}}%
\pgfpathlineto{\pgfqpoint{3.348082in}{0.811773in}}%
\pgfpathlineto{\pgfqpoint{3.325393in}{0.820272in}}%
\pgfpathlineto{\pgfqpoint{3.303786in}{0.829453in}}%
\pgfpathlineto{\pgfqpoint{3.283347in}{0.839287in}}%
\pgfpathlineto{\pgfqpoint{3.291608in}{0.840104in}}%
\pgfpathlineto{\pgfqpoint{3.300056in}{0.841531in}}%
\pgfpathlineto{\pgfqpoint{3.308656in}{0.843563in}}%
\pgfpathlineto{\pgfqpoint{3.317375in}{0.846193in}}%
\pgfpathlineto{\pgfqpoint{3.326176in}{0.849411in}}%
\pgfpathlineto{\pgfqpoint{3.343940in}{0.840822in}}%
\pgfpathlineto{\pgfqpoint{3.362733in}{0.832799in}}%
\pgfpathlineto{\pgfqpoint{3.382481in}{0.825371in}}%
\pgfpathlineto{\pgfqpoint{3.403106in}{0.818562in}}%
\pgfpathlineto{\pgfqpoint{3.424528in}{0.812397in}}%
\pgfpathclose%
\pgfusepath{fill}%
\end{pgfscope}%
\begin{pgfscope}%
\pgfpathrectangle{\pgfqpoint{2.548318in}{0.050000in}}{\pgfqpoint{2.081932in}{2.081932in}}%
\pgfusepath{clip}%
\pgfsetbuttcap%
\pgfsetroundjoin%
\definecolor{currentfill}{rgb}{0.876168,0.891125,0.095250}%
\pgfsetfillcolor{currentfill}%
\pgfsetlinewidth{0.000000pt}%
\definecolor{currentstroke}{rgb}{0.000000,0.000000,0.000000}%
\pgfsetstrokecolor{currentstroke}%
\pgfsetdash{}{0pt}%
\pgfpathmoveto{\pgfqpoint{4.105772in}{1.185961in}}%
\pgfpathlineto{\pgfqpoint{4.095724in}{1.191708in}}%
\pgfpathlineto{\pgfqpoint{4.085290in}{1.197023in}}%
\pgfpathlineto{\pgfqpoint{4.074511in}{1.201884in}}%
\pgfpathlineto{\pgfqpoint{4.063431in}{1.206274in}}%
\pgfpathlineto{\pgfqpoint{4.052092in}{1.210175in}}%
\pgfpathlineto{\pgfqpoint{4.045231in}{1.196312in}}%
\pgfpathlineto{\pgfqpoint{4.036694in}{1.182725in}}%
\pgfpathlineto{\pgfqpoint{4.026525in}{1.169466in}}%
\pgfpathlineto{\pgfqpoint{4.014773in}{1.156583in}}%
\pgfpathlineto{\pgfqpoint{4.001493in}{1.144125in}}%
\pgfpathlineto{\pgfqpoint{4.011409in}{1.138559in}}%
\pgfpathlineto{\pgfqpoint{4.021093in}{1.132550in}}%
\pgfpathlineto{\pgfqpoint{4.030509in}{1.126123in}}%
\pgfpathlineto{\pgfqpoint{4.039620in}{1.119303in}}%
\pgfpathlineto{\pgfqpoint{4.048390in}{1.112117in}}%
\pgfpathlineto{\pgfqpoint{4.063396in}{1.126031in}}%
\pgfpathlineto{\pgfqpoint{4.076697in}{1.140427in}}%
\pgfpathlineto{\pgfqpoint{4.088230in}{1.155250in}}%
\pgfpathlineto{\pgfqpoint{4.097939in}{1.170447in}}%
\pgfpathlineto{\pgfqpoint{4.105772in}{1.185961in}}%
\pgfpathclose%
\pgfusepath{fill}%
\end{pgfscope}%
\begin{pgfscope}%
\pgfpathrectangle{\pgfqpoint{2.548318in}{0.050000in}}{\pgfqpoint{2.081932in}{2.081932in}}%
\pgfusepath{clip}%
\pgfsetbuttcap%
\pgfsetroundjoin%
\definecolor{currentfill}{rgb}{0.636902,0.856542,0.216620}%
\pgfsetfillcolor{currentfill}%
\pgfsetlinewidth{0.000000pt}%
\definecolor{currentstroke}{rgb}{0.000000,0.000000,0.000000}%
\pgfsetstrokecolor{currentstroke}%
\pgfsetdash{}{0pt}%
\pgfpathmoveto{\pgfqpoint{4.148901in}{1.151588in}}%
\pgfpathlineto{\pgfqpoint{4.141342in}{1.159127in}}%
\pgfpathlineto{\pgfqpoint{4.133216in}{1.166362in}}%
\pgfpathlineto{\pgfqpoint{4.124556in}{1.173263in}}%
\pgfpathlineto{\pgfqpoint{4.115396in}{1.179805in}}%
\pgfpathlineto{\pgfqpoint{4.105772in}{1.185961in}}%
\pgfpathlineto{\pgfqpoint{4.097939in}{1.170447in}}%
\pgfpathlineto{\pgfqpoint{4.088230in}{1.155250in}}%
\pgfpathlineto{\pgfqpoint{4.076697in}{1.140427in}}%
\pgfpathlineto{\pgfqpoint{4.063396in}{1.126031in}}%
\pgfpathlineto{\pgfqpoint{4.048390in}{1.112117in}}%
\pgfpathlineto{\pgfqpoint{4.056784in}{1.104592in}}%
\pgfpathlineto{\pgfqpoint{4.064772in}{1.096760in}}%
\pgfpathlineto{\pgfqpoint{4.072320in}{1.088649in}}%
\pgfpathlineto{\pgfqpoint{4.079400in}{1.080292in}}%
\pgfpathlineto{\pgfqpoint{4.085984in}{1.071721in}}%
\pgfpathlineto{\pgfqpoint{4.102390in}{1.086758in}}%
\pgfpathlineto{\pgfqpoint{4.116951in}{1.102321in}}%
\pgfpathlineto{\pgfqpoint{4.129598in}{1.118353in}}%
\pgfpathlineto{\pgfqpoint{4.140267in}{1.134796in}}%
\pgfpathlineto{\pgfqpoint{4.148901in}{1.151588in}}%
\pgfpathclose%
\pgfusepath{fill}%
\end{pgfscope}%
\begin{pgfscope}%
\pgfpathrectangle{\pgfqpoint{2.548318in}{0.050000in}}{\pgfqpoint{2.081932in}{2.081932in}}%
\pgfusepath{clip}%
\pgfsetbuttcap%
\pgfsetroundjoin%
\definecolor{currentfill}{rgb}{0.296479,0.761561,0.424223}%
\pgfsetfillcolor{currentfill}%
\pgfsetlinewidth{0.000000pt}%
\definecolor{currentstroke}{rgb}{0.000000,0.000000,0.000000}%
\pgfsetstrokecolor{currentstroke}%
\pgfsetdash{}{0pt}%
\pgfpathmoveto{\pgfqpoint{3.573304in}{1.086312in}}%
\pgfpathlineto{\pgfqpoint{3.574556in}{1.082212in}}%
\pgfpathlineto{\pgfqpoint{3.575682in}{1.077500in}}%
\pgfpathlineto{\pgfqpoint{3.576677in}{1.072195in}}%
\pgfpathlineto{\pgfqpoint{3.577535in}{1.066316in}}%
\pgfpathlineto{\pgfqpoint{3.578254in}{1.059887in}}%
\pgfpathlineto{\pgfqpoint{3.565926in}{1.061317in}}%
\pgfpathlineto{\pgfqpoint{3.553797in}{1.063132in}}%
\pgfpathlineto{\pgfqpoint{3.541912in}{1.065326in}}%
\pgfpathlineto{\pgfqpoint{3.530320in}{1.067890in}}%
\pgfpathlineto{\pgfqpoint{3.519064in}{1.070815in}}%
\pgfpathlineto{\pgfqpoint{3.517258in}{1.077455in}}%
\pgfpathlineto{\pgfqpoint{3.515101in}{1.083584in}}%
\pgfpathlineto{\pgfqpoint{3.512603in}{1.089176in}}%
\pgfpathlineto{\pgfqpoint{3.509774in}{1.094211in}}%
\pgfpathlineto{\pgfqpoint{3.506628in}{1.098669in}}%
\pgfpathlineto{\pgfqpoint{3.519311in}{1.095361in}}%
\pgfpathlineto{\pgfqpoint{3.532372in}{1.092461in}}%
\pgfpathlineto{\pgfqpoint{3.545758in}{1.089981in}}%
\pgfpathlineto{\pgfqpoint{3.559420in}{1.087928in}}%
\pgfpathlineto{\pgfqpoint{3.573304in}{1.086312in}}%
\pgfpathclose%
\pgfusepath{fill}%
\end{pgfscope}%
\begin{pgfscope}%
\pgfpathrectangle{\pgfqpoint{2.548318in}{0.050000in}}{\pgfqpoint{2.081932in}{2.081932in}}%
\pgfusepath{clip}%
\pgfsetbuttcap%
\pgfsetroundjoin%
\definecolor{currentfill}{rgb}{0.993248,0.906157,0.143936}%
\pgfsetfillcolor{currentfill}%
\pgfsetlinewidth{0.000000pt}%
\definecolor{currentstroke}{rgb}{0.000000,0.000000,0.000000}%
\pgfsetstrokecolor{currentstroke}%
\pgfsetdash{}{0pt}%
\pgfpathmoveto{\pgfqpoint{4.052092in}{1.210175in}}%
\pgfpathlineto{\pgfqpoint{4.040540in}{1.213571in}}%
\pgfpathlineto{\pgfqpoint{4.028819in}{1.216449in}}%
\pgfpathlineto{\pgfqpoint{4.016976in}{1.218798in}}%
\pgfpathlineto{\pgfqpoint{4.005058in}{1.220607in}}%
\pgfpathlineto{\pgfqpoint{3.993111in}{1.221871in}}%
\pgfpathlineto{\pgfqpoint{3.987286in}{1.209865in}}%
\pgfpathlineto{\pgfqpoint{3.980008in}{1.198091in}}%
\pgfpathlineto{\pgfqpoint{3.971310in}{1.186595in}}%
\pgfpathlineto{\pgfqpoint{3.961236in}{1.175419in}}%
\pgfpathlineto{\pgfqpoint{3.949831in}{1.164606in}}%
\pgfpathlineto{\pgfqpoint{3.960307in}{1.161549in}}%
\pgfpathlineto{\pgfqpoint{3.970752in}{1.157956in}}%
\pgfpathlineto{\pgfqpoint{3.981125in}{1.153843in}}%
\pgfpathlineto{\pgfqpoint{3.991385in}{1.149227in}}%
\pgfpathlineto{\pgfqpoint{4.001493in}{1.144125in}}%
\pgfpathlineto{\pgfqpoint{4.014773in}{1.156583in}}%
\pgfpathlineto{\pgfqpoint{4.026525in}{1.169466in}}%
\pgfpathlineto{\pgfqpoint{4.036694in}{1.182725in}}%
\pgfpathlineto{\pgfqpoint{4.045231in}{1.196312in}}%
\pgfpathlineto{\pgfqpoint{4.052092in}{1.210175in}}%
\pgfpathclose%
\pgfusepath{fill}%
\end{pgfscope}%
\begin{pgfscope}%
\pgfpathrectangle{\pgfqpoint{2.548318in}{0.050000in}}{\pgfqpoint{2.081932in}{2.081932in}}%
\pgfusepath{clip}%
\pgfsetbuttcap%
\pgfsetroundjoin%
\definecolor{currentfill}{rgb}{0.327796,0.773980,0.406640}%
\pgfsetfillcolor{currentfill}%
\pgfsetlinewidth{0.000000pt}%
\definecolor{currentstroke}{rgb}{0.000000,0.000000,0.000000}%
\pgfsetstrokecolor{currentstroke}%
\pgfsetdash{}{0pt}%
\pgfpathmoveto{\pgfqpoint{4.177221in}{1.110396in}}%
\pgfpathlineto{\pgfqpoint{4.172902in}{1.118994in}}%
\pgfpathlineto{\pgfqpoint{4.167888in}{1.127445in}}%
\pgfpathlineto{\pgfqpoint{4.162201in}{1.135716in}}%
\pgfpathlineto{\pgfqpoint{4.155864in}{1.143774in}}%
\pgfpathlineto{\pgfqpoint{4.148901in}{1.151588in}}%
\pgfpathlineto{\pgfqpoint{4.140267in}{1.134796in}}%
\pgfpathlineto{\pgfqpoint{4.129598in}{1.118353in}}%
\pgfpathlineto{\pgfqpoint{4.116951in}{1.102321in}}%
\pgfpathlineto{\pgfqpoint{4.102390in}{1.086758in}}%
\pgfpathlineto{\pgfqpoint{4.085984in}{1.071721in}}%
\pgfpathlineto{\pgfqpoint{4.092046in}{1.062970in}}%
\pgfpathlineto{\pgfqpoint{4.097562in}{1.054074in}}%
\pgfpathlineto{\pgfqpoint{4.102511in}{1.045066in}}%
\pgfpathlineto{\pgfqpoint{4.106873in}{1.035982in}}%
\pgfpathlineto{\pgfqpoint{4.110630in}{1.026858in}}%
\pgfpathlineto{\pgfqpoint{4.127962in}{1.042578in}}%
\pgfpathlineto{\pgfqpoint{4.143357in}{1.058852in}}%
\pgfpathlineto{\pgfqpoint{4.156743in}{1.075620in}}%
\pgfpathlineto{\pgfqpoint{4.168051in}{1.092823in}}%
\pgfpathlineto{\pgfqpoint{4.177221in}{1.110396in}}%
\pgfpathclose%
\pgfusepath{fill}%
\end{pgfscope}%
\begin{pgfscope}%
\pgfpathrectangle{\pgfqpoint{2.548318in}{0.050000in}}{\pgfqpoint{2.081932in}{2.081932in}}%
\pgfusepath{clip}%
\pgfsetbuttcap%
\pgfsetroundjoin%
\definecolor{currentfill}{rgb}{0.267968,0.223549,0.512008}%
\pgfsetfillcolor{currentfill}%
\pgfsetlinewidth{0.000000pt}%
\definecolor{currentstroke}{rgb}{0.000000,0.000000,0.000000}%
\pgfsetstrokecolor{currentstroke}%
\pgfsetdash{}{0pt}%
\pgfpathmoveto{\pgfqpoint{3.245965in}{0.844330in}}%
\pgfpathlineto{\pgfqpoint{3.239480in}{0.847125in}}%
\pgfpathlineto{\pgfqpoint{3.233388in}{0.850493in}}%
\pgfpathlineto{\pgfqpoint{3.227713in}{0.854418in}}%
\pgfpathlineto{\pgfqpoint{3.222475in}{0.858883in}}%
\pgfpathlineto{\pgfqpoint{3.217695in}{0.863869in}}%
\pgfpathlineto{\pgfqpoint{3.194596in}{0.876428in}}%
\pgfpathlineto{\pgfqpoint{3.173063in}{0.889696in}}%
\pgfpathlineto{\pgfqpoint{3.153187in}{0.903627in}}%
\pgfpathlineto{\pgfqpoint{3.135055in}{0.918174in}}%
\pgfpathlineto{\pgfqpoint{3.118746in}{0.933285in}}%
\pgfpathlineto{\pgfqpoint{3.124774in}{0.927405in}}%
\pgfpathlineto{\pgfqpoint{3.131378in}{0.921974in}}%
\pgfpathlineto{\pgfqpoint{3.138532in}{0.917013in}}%
\pgfpathlineto{\pgfqpoint{3.146208in}{0.912544in}}%
\pgfpathlineto{\pgfqpoint{3.154376in}{0.908585in}}%
\pgfpathlineto{\pgfqpoint{3.169443in}{0.894606in}}%
\pgfpathlineto{\pgfqpoint{3.186213in}{0.881145in}}%
\pgfpathlineto{\pgfqpoint{3.204610in}{0.868248in}}%
\pgfpathlineto{\pgfqpoint{3.224555in}{0.855963in}}%
\pgfpathlineto{\pgfqpoint{3.245965in}{0.844330in}}%
\pgfpathclose%
\pgfusepath{fill}%
\end{pgfscope}%
\begin{pgfscope}%
\pgfpathrectangle{\pgfqpoint{2.548318in}{0.050000in}}{\pgfqpoint{2.081932in}{2.081932in}}%
\pgfusepath{clip}%
\pgfsetbuttcap%
\pgfsetroundjoin%
\definecolor{currentfill}{rgb}{0.278791,0.062145,0.386592}%
\pgfsetfillcolor{currentfill}%
\pgfsetlinewidth{0.000000pt}%
\definecolor{currentstroke}{rgb}{0.000000,0.000000,0.000000}%
\pgfsetstrokecolor{currentstroke}%
\pgfsetdash{}{0pt}%
\pgfpathmoveto{\pgfqpoint{3.650485in}{0.867132in}}%
\pgfpathlineto{\pgfqpoint{3.651695in}{0.857553in}}%
\pgfpathlineto{\pgfqpoint{3.652948in}{0.848286in}}%
\pgfpathlineto{\pgfqpoint{3.654238in}{0.839368in}}%
\pgfpathlineto{\pgfqpoint{3.655560in}{0.830836in}}%
\pgfpathlineto{\pgfqpoint{3.656908in}{0.822724in}}%
\pgfpathlineto{\pgfqpoint{3.635944in}{0.821850in}}%
\pgfpathlineto{\pgfqpoint{3.614911in}{0.821611in}}%
\pgfpathlineto{\pgfqpoint{3.593888in}{0.822010in}}%
\pgfpathlineto{\pgfqpoint{3.572953in}{0.823044in}}%
\pgfpathlineto{\pgfqpoint{3.552184in}{0.824711in}}%
\pgfpathlineto{\pgfqpoint{3.554421in}{0.832757in}}%
\pgfpathlineto{\pgfqpoint{3.556615in}{0.841225in}}%
\pgfpathlineto{\pgfqpoint{3.558755in}{0.850080in}}%
\pgfpathlineto{\pgfqpoint{3.560834in}{0.859287in}}%
\pgfpathlineto{\pgfqpoint{3.562842in}{0.868806in}}%
\pgfpathlineto{\pgfqpoint{3.580222in}{0.867402in}}%
\pgfpathlineto{\pgfqpoint{3.597742in}{0.866530in}}%
\pgfpathlineto{\pgfqpoint{3.615337in}{0.866194in}}%
\pgfpathlineto{\pgfqpoint{3.632940in}{0.866395in}}%
\pgfpathlineto{\pgfqpoint{3.650485in}{0.867132in}}%
\pgfpathclose%
\pgfusepath{fill}%
\end{pgfscope}%
\begin{pgfscope}%
\pgfpathrectangle{\pgfqpoint{2.548318in}{0.050000in}}{\pgfqpoint{2.081932in}{2.081932in}}%
\pgfusepath{clip}%
\pgfsetbuttcap%
\pgfsetroundjoin%
\definecolor{currentfill}{rgb}{0.296479,0.761561,0.424223}%
\pgfsetfillcolor{currentfill}%
\pgfsetlinewidth{0.000000pt}%
\definecolor{currentstroke}{rgb}{0.000000,0.000000,0.000000}%
\pgfsetstrokecolor{currentstroke}%
\pgfsetdash{}{0pt}%
\pgfpathmoveto{\pgfqpoint{3.771814in}{1.114481in}}%
\pgfpathlineto{\pgfqpoint{3.767426in}{1.109563in}}%
\pgfpathlineto{\pgfqpoint{3.763482in}{1.104112in}}%
\pgfpathlineto{\pgfqpoint{3.759999in}{1.098150in}}%
\pgfpathlineto{\pgfqpoint{3.756992in}{1.091701in}}%
\pgfpathlineto{\pgfqpoint{3.754475in}{1.084789in}}%
\pgfpathlineto{\pgfqpoint{3.744862in}{1.080618in}}%
\pgfpathlineto{\pgfqpoint{3.734755in}{1.076761in}}%
\pgfpathlineto{\pgfqpoint{3.724194in}{1.073232in}}%
\pgfpathlineto{\pgfqpoint{3.713220in}{1.070044in}}%
\pgfpathlineto{\pgfqpoint{3.701876in}{1.067209in}}%
\pgfpathlineto{\pgfqpoint{3.703422in}{1.073780in}}%
\pgfpathlineto{\pgfqpoint{3.705269in}{1.079826in}}%
\pgfpathlineto{\pgfqpoint{3.707408in}{1.085324in}}%
\pgfpathlineto{\pgfqpoint{3.709829in}{1.090252in}}%
\pgfpathlineto{\pgfqpoint{3.712523in}{1.094592in}}%
\pgfpathlineto{\pgfqpoint{3.725305in}{1.097798in}}%
\pgfpathlineto{\pgfqpoint{3.737672in}{1.101404in}}%
\pgfpathlineto{\pgfqpoint{3.749577in}{1.105397in}}%
\pgfpathlineto{\pgfqpoint{3.760973in}{1.109761in}}%
\pgfpathlineto{\pgfqpoint{3.771814in}{1.114481in}}%
\pgfpathclose%
\pgfusepath{fill}%
\end{pgfscope}%
\begin{pgfscope}%
\pgfpathrectangle{\pgfqpoint{2.548318in}{0.050000in}}{\pgfqpoint{2.081932in}{2.081932in}}%
\pgfusepath{clip}%
\pgfsetbuttcap%
\pgfsetroundjoin%
\definecolor{currentfill}{rgb}{0.282327,0.094955,0.417331}%
\pgfsetfillcolor{currentfill}%
\pgfsetlinewidth{0.000000pt}%
\definecolor{currentstroke}{rgb}{0.000000,0.000000,0.000000}%
\pgfsetstrokecolor{currentstroke}%
\pgfsetdash{}{0pt}%
\pgfpathmoveto{\pgfqpoint{4.016613in}{0.881343in}}%
\pgfpathlineto{\pgfqpoint{4.026282in}{0.882155in}}%
\pgfpathlineto{\pgfqpoint{4.035652in}{0.883551in}}%
\pgfpathlineto{\pgfqpoint{4.044684in}{0.885525in}}%
\pgfpathlineto{\pgfqpoint{4.053342in}{0.888066in}}%
\pgfpathlineto{\pgfqpoint{4.061592in}{0.891163in}}%
\pgfpathlineto{\pgfqpoint{4.044411in}{0.877840in}}%
\pgfpathlineto{\pgfqpoint{4.025623in}{0.865093in}}%
\pgfpathlineto{\pgfqpoint{4.005306in}{0.852968in}}%
\pgfpathlineto{\pgfqpoint{3.983546in}{0.841506in}}%
\pgfpathlineto{\pgfqpoint{3.960431in}{0.830745in}}%
\pgfpathlineto{\pgfqpoint{3.954121in}{0.828789in}}%
\pgfpathlineto{\pgfqpoint{3.947496in}{0.827438in}}%
\pgfpathlineto{\pgfqpoint{3.940582in}{0.826700in}}%
\pgfpathlineto{\pgfqpoint{3.933408in}{0.826579in}}%
\pgfpathlineto{\pgfqpoint{3.926000in}{0.827077in}}%
\pgfpathlineto{\pgfqpoint{3.946736in}{0.836750in}}%
\pgfpathlineto{\pgfqpoint{3.966244in}{0.847049in}}%
\pgfpathlineto{\pgfqpoint{3.984442in}{0.857941in}}%
\pgfpathlineto{\pgfqpoint{4.001255in}{0.869385in}}%
\pgfpathlineto{\pgfqpoint{4.016613in}{0.881343in}}%
\pgfpathclose%
\pgfusepath{fill}%
\end{pgfscope}%
\begin{pgfscope}%
\pgfpathrectangle{\pgfqpoint{2.548318in}{0.050000in}}{\pgfqpoint{2.081932in}{2.081932in}}%
\pgfusepath{clip}%
\pgfsetbuttcap%
\pgfsetroundjoin%
\definecolor{currentfill}{rgb}{0.606045,0.850733,0.236712}%
\pgfsetfillcolor{currentfill}%
\pgfsetlinewidth{0.000000pt}%
\definecolor{currentstroke}{rgb}{0.000000,0.000000,0.000000}%
\pgfsetstrokecolor{currentstroke}%
\pgfsetdash{}{0pt}%
\pgfpathmoveto{\pgfqpoint{3.852369in}{1.164003in}}%
\pgfpathlineto{\pgfqpoint{3.844217in}{1.160810in}}%
\pgfpathlineto{\pgfqpoint{3.836500in}{1.157074in}}%
\pgfpathlineto{\pgfqpoint{3.829249in}{1.152808in}}%
\pgfpathlineto{\pgfqpoint{3.822496in}{1.148030in}}%
\pgfpathlineto{\pgfqpoint{3.816268in}{1.142757in}}%
\pgfpathlineto{\pgfqpoint{3.808804in}{1.136548in}}%
\pgfpathlineto{\pgfqpoint{3.800591in}{1.130593in}}%
\pgfpathlineto{\pgfqpoint{3.791663in}{1.124917in}}%
\pgfpathlineto{\pgfqpoint{3.782058in}{1.119539in}}%
\pgfpathlineto{\pgfqpoint{3.771814in}{1.114481in}}%
\pgfpathlineto{\pgfqpoint{3.776628in}{1.118848in}}%
\pgfpathlineto{\pgfqpoint{3.781846in}{1.122647in}}%
\pgfpathlineto{\pgfqpoint{3.787446in}{1.125864in}}%
\pgfpathlineto{\pgfqpoint{3.793405in}{1.128488in}}%
\pgfpathlineto{\pgfqpoint{3.799697in}{1.130509in}}%
\pgfpathlineto{\pgfqpoint{3.811819in}{1.136496in}}%
\pgfpathlineto{\pgfqpoint{3.823193in}{1.142864in}}%
\pgfpathlineto{\pgfqpoint{3.833770in}{1.149588in}}%
\pgfpathlineto{\pgfqpoint{3.843509in}{1.156643in}}%
\pgfpathlineto{\pgfqpoint{3.852369in}{1.164003in}}%
\pgfpathclose%
\pgfusepath{fill}%
\end{pgfscope}%
\begin{pgfscope}%
\pgfpathrectangle{\pgfqpoint{2.548318in}{0.050000in}}{\pgfqpoint{2.081932in}{2.081932in}}%
\pgfusepath{clip}%
\pgfsetbuttcap%
\pgfsetroundjoin%
\definecolor{currentfill}{rgb}{0.278791,0.062145,0.386592}%
\pgfsetfillcolor{currentfill}%
\pgfsetlinewidth{0.000000pt}%
\definecolor{currentstroke}{rgb}{0.000000,0.000000,0.000000}%
\pgfsetstrokecolor{currentstroke}%
\pgfsetdash{}{0pt}%
\pgfpathmoveto{\pgfqpoint{3.735039in}{0.878704in}}%
\pgfpathlineto{\pgfqpoint{3.739362in}{0.869533in}}%
\pgfpathlineto{\pgfqpoint{3.743837in}{0.860687in}}%
\pgfpathlineto{\pgfqpoint{3.748446in}{0.852203in}}%
\pgfpathlineto{\pgfqpoint{3.753170in}{0.844115in}}%
\pgfpathlineto{\pgfqpoint{3.757988in}{0.836458in}}%
\pgfpathlineto{\pgfqpoint{3.738525in}{0.832491in}}%
\pgfpathlineto{\pgfqpoint{3.718610in}{0.829123in}}%
\pgfpathlineto{\pgfqpoint{3.698318in}{0.826367in}}%
\pgfpathlineto{\pgfqpoint{3.677725in}{0.824231in}}%
\pgfpathlineto{\pgfqpoint{3.656908in}{0.822724in}}%
\pgfpathlineto{\pgfqpoint{3.655560in}{0.830836in}}%
\pgfpathlineto{\pgfqpoint{3.654238in}{0.839368in}}%
\pgfpathlineto{\pgfqpoint{3.652948in}{0.848286in}}%
\pgfpathlineto{\pgfqpoint{3.651695in}{0.857553in}}%
\pgfpathlineto{\pgfqpoint{3.650485in}{0.867132in}}%
\pgfpathlineto{\pgfqpoint{3.667906in}{0.868402in}}%
\pgfpathlineto{\pgfqpoint{3.685137in}{0.870202in}}%
\pgfpathlineto{\pgfqpoint{3.702112in}{0.872525in}}%
\pgfpathlineto{\pgfqpoint{3.718767in}{0.875362in}}%
\pgfpathlineto{\pgfqpoint{3.735039in}{0.878704in}}%
\pgfpathclose%
\pgfusepath{fill}%
\end{pgfscope}%
\begin{pgfscope}%
\pgfpathrectangle{\pgfqpoint{2.548318in}{0.050000in}}{\pgfqpoint{2.081932in}{2.081932in}}%
\pgfusepath{clip}%
\pgfsetbuttcap%
\pgfsetroundjoin%
\definecolor{currentfill}{rgb}{0.267004,0.004874,0.329415}%
\pgfsetfillcolor{currentfill}%
\pgfsetlinewidth{0.000000pt}%
\definecolor{currentstroke}{rgb}{0.000000,0.000000,0.000000}%
\pgfsetstrokecolor{currentstroke}%
\pgfsetdash{}{0pt}%
\pgfpathmoveto{\pgfqpoint{3.845989in}{0.864734in}}%
\pgfpathlineto{\pgfqpoint{3.853983in}{0.858492in}}%
\pgfpathlineto{\pgfqpoint{3.862068in}{0.852751in}}%
\pgfpathlineto{\pgfqpoint{3.870211in}{0.847532in}}%
\pgfpathlineto{\pgfqpoint{3.878377in}{0.842858in}}%
\pgfpathlineto{\pgfqpoint{3.886532in}{0.838746in}}%
\pgfpathlineto{\pgfqpoint{3.867495in}{0.830870in}}%
\pgfpathlineto{\pgfqpoint{3.847523in}{0.823596in}}%
\pgfpathlineto{\pgfqpoint{3.826692in}{0.816947in}}%
\pgfpathlineto{\pgfqpoint{3.805085in}{0.810947in}}%
\pgfpathlineto{\pgfqpoint{3.782782in}{0.805617in}}%
\pgfpathlineto{\pgfqpoint{3.777797in}{0.810707in}}%
\pgfpathlineto{\pgfqpoint{3.772805in}{0.816359in}}%
\pgfpathlineto{\pgfqpoint{3.767826in}{0.822552in}}%
\pgfpathlineto{\pgfqpoint{3.762880in}{0.829260in}}%
\pgfpathlineto{\pgfqpoint{3.757988in}{0.836458in}}%
\pgfpathlineto{\pgfqpoint{3.776924in}{0.841010in}}%
\pgfpathlineto{\pgfqpoint{3.795261in}{0.846133in}}%
\pgfpathlineto{\pgfqpoint{3.812930in}{0.851809in}}%
\pgfpathlineto{\pgfqpoint{3.829861in}{0.858016in}}%
\pgfpathlineto{\pgfqpoint{3.845989in}{0.864734in}}%
\pgfpathclose%
\pgfusepath{fill}%
\end{pgfscope}%
\begin{pgfscope}%
\pgfpathrectangle{\pgfqpoint{2.548318in}{0.050000in}}{\pgfqpoint{2.081932in}{2.081932in}}%
\pgfusepath{clip}%
\pgfsetbuttcap%
\pgfsetroundjoin%
\definecolor{currentfill}{rgb}{0.993248,0.906157,0.143936}%
\pgfsetfillcolor{currentfill}%
\pgfsetlinewidth{0.000000pt}%
\definecolor{currentstroke}{rgb}{0.000000,0.000000,0.000000}%
\pgfsetstrokecolor{currentstroke}%
\pgfsetdash{}{0pt}%
\pgfpathmoveto{\pgfqpoint{3.993111in}{1.221871in}}%
\pgfpathlineto{\pgfqpoint{3.981182in}{1.222583in}}%
\pgfpathlineto{\pgfqpoint{3.969319in}{1.222740in}}%
\pgfpathlineto{\pgfqpoint{3.957568in}{1.222342in}}%
\pgfpathlineto{\pgfqpoint{3.945976in}{1.221389in}}%
\pgfpathlineto{\pgfqpoint{3.934589in}{1.219886in}}%
\pgfpathlineto{\pgfqpoint{3.929761in}{1.209750in}}%
\pgfpathlineto{\pgfqpoint{3.923700in}{1.199806in}}%
\pgfpathlineto{\pgfqpoint{3.916435in}{1.190090in}}%
\pgfpathlineto{\pgfqpoint{3.908000in}{1.180641in}}%
\pgfpathlineto{\pgfqpoint{3.898433in}{1.171493in}}%
\pgfpathlineto{\pgfqpoint{3.908445in}{1.171260in}}%
\pgfpathlineto{\pgfqpoint{3.918632in}{1.170450in}}%
\pgfpathlineto{\pgfqpoint{3.928952in}{1.169067in}}%
\pgfpathlineto{\pgfqpoint{3.939366in}{1.167116in}}%
\pgfpathlineto{\pgfqpoint{3.949831in}{1.164606in}}%
\pgfpathlineto{\pgfqpoint{3.961236in}{1.175419in}}%
\pgfpathlineto{\pgfqpoint{3.971310in}{1.186595in}}%
\pgfpathlineto{\pgfqpoint{3.980008in}{1.198091in}}%
\pgfpathlineto{\pgfqpoint{3.987286in}{1.209865in}}%
\pgfpathlineto{\pgfqpoint{3.993111in}{1.221871in}}%
\pgfpathclose%
\pgfusepath{fill}%
\end{pgfscope}%
\begin{pgfscope}%
\pgfpathrectangle{\pgfqpoint{2.548318in}{0.050000in}}{\pgfqpoint{2.081932in}{2.081932in}}%
\pgfusepath{clip}%
\pgfsetbuttcap%
\pgfsetroundjoin%
\definecolor{currentfill}{rgb}{0.120638,0.625828,0.533488}%
\pgfsetfillcolor{currentfill}%
\pgfsetlinewidth{0.000000pt}%
\definecolor{currentstroke}{rgb}{0.000000,0.000000,0.000000}%
\pgfsetstrokecolor{currentstroke}%
\pgfsetdash{}{0pt}%
\pgfpathmoveto{\pgfqpoint{3.641195in}{1.058648in}}%
\pgfpathlineto{\pgfqpoint{3.640848in}{1.051714in}}%
\pgfpathlineto{\pgfqpoint{3.640589in}{1.044280in}}%
\pgfpathlineto{\pgfqpoint{3.640419in}{1.036374in}}%
\pgfpathlineto{\pgfqpoint{3.640338in}{1.028031in}}%
\pgfpathlineto{\pgfqpoint{3.640349in}{1.019283in}}%
\pgfpathlineto{\pgfqpoint{3.628199in}{1.018760in}}%
\pgfpathlineto{\pgfqpoint{3.616008in}{1.018618in}}%
\pgfpathlineto{\pgfqpoint{3.603824in}{1.018856in}}%
\pgfpathlineto{\pgfqpoint{3.591692in}{1.019475in}}%
\pgfpathlineto{\pgfqpoint{3.579659in}{1.020471in}}%
\pgfpathlineto{\pgfqpoint{3.579676in}{1.029220in}}%
\pgfpathlineto{\pgfqpoint{3.579543in}{1.037569in}}%
\pgfpathlineto{\pgfqpoint{3.579260in}{1.045485in}}%
\pgfpathlineto{\pgfqpoint{3.578830in}{1.052934in}}%
\pgfpathlineto{\pgfqpoint{3.578254in}{1.059887in}}%
\pgfpathlineto{\pgfqpoint{3.590733in}{1.058848in}}%
\pgfpathlineto{\pgfqpoint{3.603316in}{1.058203in}}%
\pgfpathlineto{\pgfqpoint{3.615952in}{1.057954in}}%
\pgfpathlineto{\pgfqpoint{3.628595in}{1.058102in}}%
\pgfpathlineto{\pgfqpoint{3.641195in}{1.058648in}}%
\pgfpathclose%
\pgfusepath{fill}%
\end{pgfscope}%
\begin{pgfscope}%
\pgfpathrectangle{\pgfqpoint{2.548318in}{0.050000in}}{\pgfqpoint{2.081932in}{2.081932in}}%
\pgfusepath{clip}%
\pgfsetbuttcap%
\pgfsetroundjoin%
\definecolor{currentfill}{rgb}{0.124780,0.640461,0.527068}%
\pgfsetfillcolor{currentfill}%
\pgfsetlinewidth{0.000000pt}%
\definecolor{currentstroke}{rgb}{0.000000,0.000000,0.000000}%
\pgfsetstrokecolor{currentstroke}%
\pgfsetdash{}{0pt}%
\pgfpathmoveto{\pgfqpoint{4.187876in}{1.066411in}}%
\pgfpathlineto{\pgfqpoint{4.187240in}{1.075224in}}%
\pgfpathlineto{\pgfqpoint{4.185849in}{1.084065in}}%
\pgfpathlineto{\pgfqpoint{4.183708in}{1.092897in}}%
\pgfpathlineto{\pgfqpoint{4.180829in}{1.101686in}}%
\pgfpathlineto{\pgfqpoint{4.177221in}{1.110396in}}%
\pgfpathlineto{\pgfqpoint{4.168051in}{1.092823in}}%
\pgfpathlineto{\pgfqpoint{4.156743in}{1.075620in}}%
\pgfpathlineto{\pgfqpoint{4.143357in}{1.058852in}}%
\pgfpathlineto{\pgfqpoint{4.127962in}{1.042578in}}%
\pgfpathlineto{\pgfqpoint{4.110630in}{1.026858in}}%
\pgfpathlineto{\pgfqpoint{4.113766in}{1.017729in}}%
\pgfpathlineto{\pgfqpoint{4.116270in}{1.008631in}}%
\pgfpathlineto{\pgfqpoint{4.118131in}{0.999600in}}%
\pgfpathlineto{\pgfqpoint{4.119341in}{0.990671in}}%
\pgfpathlineto{\pgfqpoint{4.119893in}{0.981880in}}%
\pgfpathlineto{\pgfqpoint{4.137575in}{0.997784in}}%
\pgfpathlineto{\pgfqpoint{4.153286in}{1.014249in}}%
\pgfpathlineto{\pgfqpoint{4.166952in}{1.031216in}}%
\pgfpathlineto{\pgfqpoint{4.178502in}{1.048625in}}%
\pgfpathlineto{\pgfqpoint{4.187876in}{1.066411in}}%
\pgfpathclose%
\pgfusepath{fill}%
\end{pgfscope}%
\begin{pgfscope}%
\pgfpathrectangle{\pgfqpoint{2.548318in}{0.050000in}}{\pgfqpoint{2.081932in}{2.081932in}}%
\pgfusepath{clip}%
\pgfsetbuttcap%
\pgfsetroundjoin%
\definecolor{currentfill}{rgb}{0.296479,0.761561,0.424223}%
\pgfsetfillcolor{currentfill}%
\pgfsetlinewidth{0.000000pt}%
\definecolor{currentstroke}{rgb}{0.000000,0.000000,0.000000}%
\pgfsetstrokecolor{currentstroke}%
\pgfsetdash{}{0pt}%
\pgfpathmoveto{\pgfqpoint{3.506628in}{1.098669in}}%
\pgfpathlineto{\pgfqpoint{3.509774in}{1.094211in}}%
\pgfpathlineto{\pgfqpoint{3.512603in}{1.089176in}}%
\pgfpathlineto{\pgfqpoint{3.515101in}{1.083584in}}%
\pgfpathlineto{\pgfqpoint{3.517258in}{1.077455in}}%
\pgfpathlineto{\pgfqpoint{3.519064in}{1.070815in}}%
\pgfpathlineto{\pgfqpoint{3.508188in}{1.074090in}}%
\pgfpathlineto{\pgfqpoint{3.497736in}{1.077703in}}%
\pgfpathlineto{\pgfqpoint{3.487748in}{1.081640in}}%
\pgfpathlineto{\pgfqpoint{3.478264in}{1.085888in}}%
\pgfpathlineto{\pgfqpoint{3.469322in}{1.090429in}}%
\pgfpathlineto{\pgfqpoint{3.466596in}{1.097450in}}%
\pgfpathlineto{\pgfqpoint{3.463339in}{1.104030in}}%
\pgfpathlineto{\pgfqpoint{3.459566in}{1.110141in}}%
\pgfpathlineto{\pgfqpoint{3.455293in}{1.115760in}}%
\pgfpathlineto{\pgfqpoint{3.450539in}{1.120865in}}%
\pgfpathlineto{\pgfqpoint{3.460628in}{1.115724in}}%
\pgfpathlineto{\pgfqpoint{3.471325in}{1.110918in}}%
\pgfpathlineto{\pgfqpoint{3.482588in}{1.106462in}}%
\pgfpathlineto{\pgfqpoint{3.494371in}{1.102374in}}%
\pgfpathlineto{\pgfqpoint{3.506628in}{1.098669in}}%
\pgfpathclose%
\pgfusepath{fill}%
\end{pgfscope}%
\begin{pgfscope}%
\pgfpathrectangle{\pgfqpoint{2.548318in}{0.050000in}}{\pgfqpoint{2.081932in}{2.081932in}}%
\pgfusepath{clip}%
\pgfsetbuttcap%
\pgfsetroundjoin%
\definecolor{currentfill}{rgb}{0.876168,0.891125,0.095250}%
\pgfsetfillcolor{currentfill}%
\pgfsetlinewidth{0.000000pt}%
\definecolor{currentstroke}{rgb}{0.000000,0.000000,0.000000}%
\pgfsetstrokecolor{currentstroke}%
\pgfsetdash{}{0pt}%
\pgfpathmoveto{\pgfqpoint{3.168180in}{1.129615in}}%
\pgfpathlineto{\pgfqpoint{3.177350in}{1.136464in}}%
\pgfpathlineto{\pgfqpoint{3.186875in}{1.142930in}}%
\pgfpathlineto{\pgfqpoint{3.196719in}{1.148989in}}%
\pgfpathlineto{\pgfqpoint{3.206842in}{1.154617in}}%
\pgfpathlineto{\pgfqpoint{3.217205in}{1.159791in}}%
\pgfpathlineto{\pgfqpoint{3.205841in}{1.172773in}}%
\pgfpathlineto{\pgfqpoint{3.196072in}{1.186120in}}%
\pgfpathlineto{\pgfqpoint{3.187947in}{1.199781in}}%
\pgfpathlineto{\pgfqpoint{3.181509in}{1.213705in}}%
\pgfpathlineto{\pgfqpoint{3.176792in}{1.227839in}}%
\pgfpathlineto{\pgfqpoint{3.165261in}{1.224390in}}%
\pgfpathlineto{\pgfqpoint{3.153991in}{1.220439in}}%
\pgfpathlineto{\pgfqpoint{3.143026in}{1.216002in}}%
\pgfpathlineto{\pgfqpoint{3.132411in}{1.211096in}}%
\pgfpathlineto{\pgfqpoint{3.122187in}{1.205741in}}%
\pgfpathlineto{\pgfqpoint{3.127618in}{1.189914in}}%
\pgfpathlineto{\pgfqpoint{3.134979in}{1.174329in}}%
\pgfpathlineto{\pgfqpoint{3.144226in}{1.159046in}}%
\pgfpathlineto{\pgfqpoint{3.155312in}{1.144124in}}%
\pgfpathlineto{\pgfqpoint{3.168180in}{1.129615in}}%
\pgfpathclose%
\pgfusepath{fill}%
\end{pgfscope}%
\begin{pgfscope}%
\pgfpathrectangle{\pgfqpoint{2.548318in}{0.050000in}}{\pgfqpoint{2.081932in}{2.081932in}}%
\pgfusepath{clip}%
\pgfsetbuttcap%
\pgfsetroundjoin%
\definecolor{currentfill}{rgb}{0.278791,0.062145,0.386592}%
\pgfsetfillcolor{currentfill}%
\pgfsetlinewidth{0.000000pt}%
\definecolor{currentstroke}{rgb}{0.000000,0.000000,0.000000}%
\pgfsetstrokecolor{currentstroke}%
\pgfsetdash{}{0pt}%
\pgfpathmoveto{\pgfqpoint{3.562842in}{0.868806in}}%
\pgfpathlineto{\pgfqpoint{3.560834in}{0.859287in}}%
\pgfpathlineto{\pgfqpoint{3.558755in}{0.850080in}}%
\pgfpathlineto{\pgfqpoint{3.556615in}{0.841225in}}%
\pgfpathlineto{\pgfqpoint{3.554421in}{0.832757in}}%
\pgfpathlineto{\pgfqpoint{3.552184in}{0.824711in}}%
\pgfpathlineto{\pgfqpoint{3.531659in}{0.827003in}}%
\pgfpathlineto{\pgfqpoint{3.511453in}{0.829914in}}%
\pgfpathlineto{\pgfqpoint{3.491644in}{0.833434in}}%
\pgfpathlineto{\pgfqpoint{3.472305in}{0.837550in}}%
\pgfpathlineto{\pgfqpoint{3.453511in}{0.842248in}}%
\pgfpathlineto{\pgfqpoint{3.459141in}{0.849714in}}%
\pgfpathlineto{\pgfqpoint{3.464661in}{0.857614in}}%
\pgfpathlineto{\pgfqpoint{3.470047in}{0.865914in}}%
\pgfpathlineto{\pgfqpoint{3.475276in}{0.874582in}}%
\pgfpathlineto{\pgfqpoint{3.480327in}{0.883581in}}%
\pgfpathlineto{\pgfqpoint{3.496033in}{0.879624in}}%
\pgfpathlineto{\pgfqpoint{3.512200in}{0.876157in}}%
\pgfpathlineto{\pgfqpoint{3.528767in}{0.873192in}}%
\pgfpathlineto{\pgfqpoint{3.545669in}{0.870739in}}%
\pgfpathlineto{\pgfqpoint{3.562842in}{0.868806in}}%
\pgfpathclose%
\pgfusepath{fill}%
\end{pgfscope}%
\begin{pgfscope}%
\pgfpathrectangle{\pgfqpoint{2.548318in}{0.050000in}}{\pgfqpoint{2.081932in}{2.081932in}}%
\pgfusepath{clip}%
\pgfsetbuttcap%
\pgfsetroundjoin%
\definecolor{currentfill}{rgb}{0.120638,0.625828,0.533488}%
\pgfsetfillcolor{currentfill}%
\pgfsetlinewidth{0.000000pt}%
\definecolor{currentstroke}{rgb}{0.000000,0.000000,0.000000}%
\pgfsetstrokecolor{currentstroke}%
\pgfsetdash{}{0pt}%
\pgfpathmoveto{\pgfqpoint{3.701876in}{1.067209in}}%
\pgfpathlineto{\pgfqpoint{3.700638in}{1.060141in}}%
\pgfpathlineto{\pgfqpoint{3.699713in}{1.052603in}}%
\pgfpathlineto{\pgfqpoint{3.699106in}{1.044626in}}%
\pgfpathlineto{\pgfqpoint{3.698820in}{1.036242in}}%
\pgfpathlineto{\pgfqpoint{3.698856in}{1.027487in}}%
\pgfpathlineto{\pgfqpoint{3.687605in}{1.025119in}}%
\pgfpathlineto{\pgfqpoint{3.676084in}{1.023108in}}%
\pgfpathlineto{\pgfqpoint{3.664338in}{1.021461in}}%
\pgfpathlineto{\pgfqpoint{3.652410in}{1.020184in}}%
\pgfpathlineto{\pgfqpoint{3.640349in}{1.019283in}}%
\pgfpathlineto{\pgfqpoint{3.640338in}{1.028031in}}%
\pgfpathlineto{\pgfqpoint{3.640419in}{1.036374in}}%
\pgfpathlineto{\pgfqpoint{3.640589in}{1.044280in}}%
\pgfpathlineto{\pgfqpoint{3.640848in}{1.051714in}}%
\pgfpathlineto{\pgfqpoint{3.641195in}{1.058648in}}%
\pgfpathlineto{\pgfqpoint{3.653704in}{1.059589in}}%
\pgfpathlineto{\pgfqpoint{3.666074in}{1.060921in}}%
\pgfpathlineto{\pgfqpoint{3.678257in}{1.062639in}}%
\pgfpathlineto{\pgfqpoint{3.690207in}{1.064738in}}%
\pgfpathlineto{\pgfqpoint{3.701876in}{1.067209in}}%
\pgfpathclose%
\pgfusepath{fill}%
\end{pgfscope}%
\begin{pgfscope}%
\pgfpathrectangle{\pgfqpoint{2.548318in}{0.050000in}}{\pgfqpoint{2.081932in}{2.081932in}}%
\pgfusepath{clip}%
\pgfsetbuttcap%
\pgfsetroundjoin%
\definecolor{currentfill}{rgb}{0.636902,0.856542,0.216620}%
\pgfsetfillcolor{currentfill}%
\pgfsetlinewidth{0.000000pt}%
\definecolor{currentstroke}{rgb}{0.000000,0.000000,0.000000}%
\pgfsetstrokecolor{currentstroke}%
\pgfsetdash{}{0pt}%
\pgfpathmoveto{\pgfqpoint{3.128859in}{1.090632in}}%
\pgfpathlineto{\pgfqpoint{3.135747in}{1.098961in}}%
\pgfpathlineto{\pgfqpoint{3.143153in}{1.107055in}}%
\pgfpathlineto{\pgfqpoint{3.151048in}{1.114882in}}%
\pgfpathlineto{\pgfqpoint{3.159401in}{1.122412in}}%
\pgfpathlineto{\pgfqpoint{3.168180in}{1.129615in}}%
\pgfpathlineto{\pgfqpoint{3.155312in}{1.144124in}}%
\pgfpathlineto{\pgfqpoint{3.144226in}{1.159046in}}%
\pgfpathlineto{\pgfqpoint{3.134979in}{1.174329in}}%
\pgfpathlineto{\pgfqpoint{3.127618in}{1.189914in}}%
\pgfpathlineto{\pgfqpoint{3.122187in}{1.205741in}}%
\pgfpathlineto{\pgfqpoint{3.112394in}{1.199957in}}%
\pgfpathlineto{\pgfqpoint{3.103072in}{1.193766in}}%
\pgfpathlineto{\pgfqpoint{3.094257in}{1.187194in}}%
\pgfpathlineto{\pgfqpoint{3.085985in}{1.180266in}}%
\pgfpathlineto{\pgfqpoint{3.078289in}{1.173008in}}%
\pgfpathlineto{\pgfqpoint{3.084315in}{1.155867in}}%
\pgfpathlineto{\pgfqpoint{3.092437in}{1.138997in}}%
\pgfpathlineto{\pgfqpoint{3.102606in}{1.122460in}}%
\pgfpathlineto{\pgfqpoint{3.114767in}{1.106319in}}%
\pgfpathlineto{\pgfqpoint{3.128859in}{1.090632in}}%
\pgfpathclose%
\pgfusepath{fill}%
\end{pgfscope}%
\begin{pgfscope}%
\pgfpathrectangle{\pgfqpoint{2.548318in}{0.050000in}}{\pgfqpoint{2.081932in}{2.081932in}}%
\pgfusepath{clip}%
\pgfsetbuttcap%
\pgfsetroundjoin%
\definecolor{currentfill}{rgb}{0.606045,0.850733,0.236712}%
\pgfsetfillcolor{currentfill}%
\pgfsetlinewidth{0.000000pt}%
\definecolor{currentstroke}{rgb}{0.000000,0.000000,0.000000}%
\pgfsetstrokecolor{currentstroke}%
\pgfsetdash{}{0pt}%
\pgfpathmoveto{\pgfqpoint{3.420328in}{1.138066in}}%
\pgfpathlineto{\pgfqpoint{3.427147in}{1.135782in}}%
\pgfpathlineto{\pgfqpoint{3.433604in}{1.132908in}}%
\pgfpathlineto{\pgfqpoint{3.439672in}{1.129455in}}%
\pgfpathlineto{\pgfqpoint{3.445325in}{1.125436in}}%
\pgfpathlineto{\pgfqpoint{3.450539in}{1.120865in}}%
\pgfpathlineto{\pgfqpoint{3.441099in}{1.126320in}}%
\pgfpathlineto{\pgfqpoint{3.432346in}{1.132068in}}%
\pgfpathlineto{\pgfqpoint{3.424315in}{1.138089in}}%
\pgfpathlineto{\pgfqpoint{3.417042in}{1.144359in}}%
\pgfpathlineto{\pgfqpoint{3.410555in}{1.150855in}}%
\pgfpathlineto{\pgfqpoint{3.404061in}{1.156389in}}%
\pgfpathlineto{\pgfqpoint{3.397019in}{1.161449in}}%
\pgfpathlineto{\pgfqpoint{3.389458in}{1.166017in}}%
\pgfpathlineto{\pgfqpoint{3.381409in}{1.170074in}}%
\pgfpathlineto{\pgfqpoint{3.372905in}{1.173606in}}%
\pgfpathlineto{\pgfqpoint{3.380615in}{1.165903in}}%
\pgfpathlineto{\pgfqpoint{3.389250in}{1.158470in}}%
\pgfpathlineto{\pgfqpoint{3.398774in}{1.151336in}}%
\pgfpathlineto{\pgfqpoint{3.409148in}{1.144526in}}%
\pgfpathlineto{\pgfqpoint{3.420328in}{1.138066in}}%
\pgfpathclose%
\pgfusepath{fill}%
\end{pgfscope}%
\begin{pgfscope}%
\pgfpathrectangle{\pgfqpoint{2.548318in}{0.050000in}}{\pgfqpoint{2.081932in}{2.081932in}}%
\pgfusepath{clip}%
\pgfsetbuttcap%
\pgfsetroundjoin%
\definecolor{currentfill}{rgb}{0.993248,0.906157,0.143936}%
\pgfsetfillcolor{currentfill}%
\pgfsetlinewidth{0.000000pt}%
\definecolor{currentstroke}{rgb}{0.000000,0.000000,0.000000}%
\pgfsetstrokecolor{currentstroke}%
\pgfsetdash{}{0pt}%
\pgfpathmoveto{\pgfqpoint{3.217205in}{1.159791in}}%
\pgfpathlineto{\pgfqpoint{3.227767in}{1.164492in}}%
\pgfpathlineto{\pgfqpoint{3.238488in}{1.168699in}}%
\pgfpathlineto{\pgfqpoint{3.249325in}{1.172396in}}%
\pgfpathlineto{\pgfqpoint{3.260236in}{1.175568in}}%
\pgfpathlineto{\pgfqpoint{3.271178in}{1.178203in}}%
\pgfpathlineto{\pgfqpoint{3.261441in}{1.189463in}}%
\pgfpathlineto{\pgfqpoint{3.253092in}{1.201033in}}%
\pgfpathlineto{\pgfqpoint{3.246171in}{1.212870in}}%
\pgfpathlineto{\pgfqpoint{3.240713in}{1.224927in}}%
\pgfpathlineto{\pgfqpoint{3.236748in}{1.237159in}}%
\pgfpathlineto{\pgfqpoint{3.224607in}{1.236380in}}%
\pgfpathlineto{\pgfqpoint{3.212493in}{1.235052in}}%
\pgfpathlineto{\pgfqpoint{3.200455in}{1.233181in}}%
\pgfpathlineto{\pgfqpoint{3.188538in}{1.230773in}}%
\pgfpathlineto{\pgfqpoint{3.176792in}{1.227839in}}%
\pgfpathlineto{\pgfqpoint{3.181509in}{1.213705in}}%
\pgfpathlineto{\pgfqpoint{3.187947in}{1.199781in}}%
\pgfpathlineto{\pgfqpoint{3.196072in}{1.186120in}}%
\pgfpathlineto{\pgfqpoint{3.205841in}{1.172773in}}%
\pgfpathlineto{\pgfqpoint{3.217205in}{1.159791in}}%
\pgfpathclose%
\pgfusepath{fill}%
\end{pgfscope}%
\begin{pgfscope}%
\pgfpathrectangle{\pgfqpoint{2.548318in}{0.050000in}}{\pgfqpoint{2.081932in}{2.081932in}}%
\pgfusepath{clip}%
\pgfsetbuttcap%
\pgfsetroundjoin%
\definecolor{currentfill}{rgb}{0.120638,0.625828,0.533488}%
\pgfsetfillcolor{currentfill}%
\pgfsetlinewidth{0.000000pt}%
\definecolor{currentstroke}{rgb}{0.000000,0.000000,0.000000}%
\pgfsetstrokecolor{currentstroke}%
\pgfsetdash{}{0pt}%
\pgfpathmoveto{\pgfqpoint{3.578254in}{1.059887in}}%
\pgfpathlineto{\pgfqpoint{3.578830in}{1.052934in}}%
\pgfpathlineto{\pgfqpoint{3.579260in}{1.045485in}}%
\pgfpathlineto{\pgfqpoint{3.579543in}{1.037569in}}%
\pgfpathlineto{\pgfqpoint{3.579676in}{1.029220in}}%
\pgfpathlineto{\pgfqpoint{3.579659in}{1.020471in}}%
\pgfpathlineto{\pgfqpoint{3.567772in}{1.021841in}}%
\pgfpathlineto{\pgfqpoint{3.556077in}{1.023580in}}%
\pgfpathlineto{\pgfqpoint{3.544619in}{1.025682in}}%
\pgfpathlineto{\pgfqpoint{3.533443in}{1.028139in}}%
\pgfpathlineto{\pgfqpoint{3.522591in}{1.030942in}}%
\pgfpathlineto{\pgfqpoint{3.522634in}{1.039700in}}%
\pgfpathlineto{\pgfqpoint{3.522299in}{1.048100in}}%
\pgfpathlineto{\pgfqpoint{3.521590in}{1.056108in}}%
\pgfpathlineto{\pgfqpoint{3.520510in}{1.063689in}}%
\pgfpathlineto{\pgfqpoint{3.519064in}{1.070815in}}%
\pgfpathlineto{\pgfqpoint{3.530320in}{1.067890in}}%
\pgfpathlineto{\pgfqpoint{3.541912in}{1.065326in}}%
\pgfpathlineto{\pgfqpoint{3.553797in}{1.063132in}}%
\pgfpathlineto{\pgfqpoint{3.565926in}{1.061317in}}%
\pgfpathlineto{\pgfqpoint{3.578254in}{1.059887in}}%
\pgfpathclose%
\pgfusepath{fill}%
\end{pgfscope}%
\begin{pgfscope}%
\pgfpathrectangle{\pgfqpoint{2.548318in}{0.050000in}}{\pgfqpoint{2.081932in}{2.081932in}}%
\pgfusepath{clip}%
\pgfsetbuttcap%
\pgfsetroundjoin%
\definecolor{currentfill}{rgb}{0.267004,0.004874,0.329415}%
\pgfsetfillcolor{currentfill}%
\pgfsetlinewidth{0.000000pt}%
\definecolor{currentstroke}{rgb}{0.000000,0.000000,0.000000}%
\pgfsetstrokecolor{currentstroke}%
\pgfsetdash{}{0pt}%
\pgfpathmoveto{\pgfqpoint{3.453511in}{0.842248in}}%
\pgfpathlineto{\pgfqpoint{3.447793in}{0.835246in}}%
\pgfpathlineto{\pgfqpoint{3.442012in}{0.828735in}}%
\pgfpathlineto{\pgfqpoint{3.436191in}{0.822741in}}%
\pgfpathlineto{\pgfqpoint{3.430355in}{0.817288in}}%
\pgfpathlineto{\pgfqpoint{3.424528in}{0.812397in}}%
\pgfpathlineto{\pgfqpoint{3.403106in}{0.818562in}}%
\pgfpathlineto{\pgfqpoint{3.382481in}{0.825371in}}%
\pgfpathlineto{\pgfqpoint{3.362733in}{0.832799in}}%
\pgfpathlineto{\pgfqpoint{3.343940in}{0.840822in}}%
\pgfpathlineto{\pgfqpoint{3.326176in}{0.849411in}}%
\pgfpathlineto{\pgfqpoint{3.335023in}{0.853207in}}%
\pgfpathlineto{\pgfqpoint{3.343881in}{0.857564in}}%
\pgfpathlineto{\pgfqpoint{3.352712in}{0.862466in}}%
\pgfpathlineto{\pgfqpoint{3.361481in}{0.867895in}}%
\pgfpathlineto{\pgfqpoint{3.370150in}{0.873827in}}%
\pgfpathlineto{\pgfqpoint{3.385185in}{0.866504in}}%
\pgfpathlineto{\pgfqpoint{3.401103in}{0.859662in}}%
\pgfpathlineto{\pgfqpoint{3.417840in}{0.853323in}}%
\pgfpathlineto{\pgfqpoint{3.435332in}{0.847512in}}%
\pgfpathlineto{\pgfqpoint{3.453511in}{0.842248in}}%
\pgfpathclose%
\pgfusepath{fill}%
\end{pgfscope}%
\begin{pgfscope}%
\pgfpathrectangle{\pgfqpoint{2.548318in}{0.050000in}}{\pgfqpoint{2.081932in}{2.081932in}}%
\pgfusepath{clip}%
\pgfsetbuttcap%
\pgfsetroundjoin%
\definecolor{currentfill}{rgb}{0.278012,0.180367,0.486697}%
\pgfsetfillcolor{currentfill}%
\pgfsetlinewidth{0.000000pt}%
\definecolor{currentstroke}{rgb}{0.000000,0.000000,0.000000}%
\pgfsetstrokecolor{currentstroke}%
\pgfsetdash{}{0pt}%
\pgfpathmoveto{\pgfqpoint{3.645246in}{0.918262in}}%
\pgfpathlineto{\pgfqpoint{3.646170in}{0.907746in}}%
\pgfpathlineto{\pgfqpoint{3.647161in}{0.897331in}}%
\pgfpathlineto{\pgfqpoint{3.648213in}{0.887062in}}%
\pgfpathlineto{\pgfqpoint{3.649323in}{0.876982in}}%
\pgfpathlineto{\pgfqpoint{3.650485in}{0.867132in}}%
\pgfpathlineto{\pgfqpoint{3.632940in}{0.866395in}}%
\pgfpathlineto{\pgfqpoint{3.615337in}{0.866194in}}%
\pgfpathlineto{\pgfqpoint{3.597742in}{0.866530in}}%
\pgfpathlineto{\pgfqpoint{3.580222in}{0.867402in}}%
\pgfpathlineto{\pgfqpoint{3.562842in}{0.868806in}}%
\pgfpathlineto{\pgfqpoint{3.564771in}{0.878600in}}%
\pgfpathlineto{\pgfqpoint{3.566612in}{0.888627in}}%
\pgfpathlineto{\pgfqpoint{3.568358in}{0.898844in}}%
\pgfpathlineto{\pgfqpoint{3.570001in}{0.909211in}}%
\pgfpathlineto{\pgfqpoint{3.571534in}{0.919682in}}%
\pgfpathlineto{\pgfqpoint{3.586150in}{0.918491in}}%
\pgfpathlineto{\pgfqpoint{3.600885in}{0.917752in}}%
\pgfpathlineto{\pgfqpoint{3.615684in}{0.917467in}}%
\pgfpathlineto{\pgfqpoint{3.630490in}{0.917637in}}%
\pgfpathlineto{\pgfqpoint{3.645246in}{0.918262in}}%
\pgfpathclose%
\pgfusepath{fill}%
\end{pgfscope}%
\begin{pgfscope}%
\pgfpathrectangle{\pgfqpoint{2.548318in}{0.050000in}}{\pgfqpoint{2.081932in}{2.081932in}}%
\pgfusepath{clip}%
\pgfsetbuttcap%
\pgfsetroundjoin%
\definecolor{currentfill}{rgb}{0.327796,0.773980,0.406640}%
\pgfsetfillcolor{currentfill}%
\pgfsetlinewidth{0.000000pt}%
\definecolor{currentstroke}{rgb}{0.000000,0.000000,0.000000}%
\pgfsetstrokecolor{currentstroke}%
\pgfsetdash{}{0pt}%
\pgfpathmoveto{\pgfqpoint{3.103072in}{1.046629in}}%
\pgfpathlineto{\pgfqpoint{3.107003in}{1.055630in}}%
\pgfpathlineto{\pgfqpoint{3.111567in}{1.064566in}}%
\pgfpathlineto{\pgfqpoint{3.116745in}{1.073401in}}%
\pgfpathlineto{\pgfqpoint{3.122517in}{1.082101in}}%
\pgfpathlineto{\pgfqpoint{3.128859in}{1.090632in}}%
\pgfpathlineto{\pgfqpoint{3.114767in}{1.106319in}}%
\pgfpathlineto{\pgfqpoint{3.102606in}{1.122460in}}%
\pgfpathlineto{\pgfqpoint{3.092437in}{1.138997in}}%
\pgfpathlineto{\pgfqpoint{3.084315in}{1.155867in}}%
\pgfpathlineto{\pgfqpoint{3.078289in}{1.173008in}}%
\pgfpathlineto{\pgfqpoint{3.071200in}{1.165450in}}%
\pgfpathlineto{\pgfqpoint{3.064747in}{1.157620in}}%
\pgfpathlineto{\pgfqpoint{3.058956in}{1.149551in}}%
\pgfpathlineto{\pgfqpoint{3.053850in}{1.141273in}}%
\pgfpathlineto{\pgfqpoint{3.049451in}{1.132819in}}%
\pgfpathlineto{\pgfqpoint{3.055878in}{1.114875in}}%
\pgfpathlineto{\pgfqpoint{3.064509in}{1.097219in}}%
\pgfpathlineto{\pgfqpoint{3.075292in}{1.079917in}}%
\pgfpathlineto{\pgfqpoint{3.088168in}{1.063033in}}%
\pgfpathlineto{\pgfqpoint{3.103072in}{1.046629in}}%
\pgfpathclose%
\pgfusepath{fill}%
\end{pgfscope}%
\begin{pgfscope}%
\pgfpathrectangle{\pgfqpoint{2.548318in}{0.050000in}}{\pgfqpoint{2.081932in}{2.081932in}}%
\pgfusepath{clip}%
\pgfsetbuttcap%
\pgfsetroundjoin%
\definecolor{currentfill}{rgb}{0.282327,0.094955,0.417331}%
\pgfsetfillcolor{currentfill}%
\pgfsetlinewidth{0.000000pt}%
\definecolor{currentstroke}{rgb}{0.000000,0.000000,0.000000}%
\pgfsetstrokecolor{currentstroke}%
\pgfsetdash{}{0pt}%
\pgfpathmoveto{\pgfqpoint{3.283347in}{0.839287in}}%
\pgfpathlineto{\pgfqpoint{3.275306in}{0.839082in}}%
\pgfpathlineto{\pgfqpoint{3.267517in}{0.839488in}}%
\pgfpathlineto{\pgfqpoint{3.260011in}{0.840502in}}%
\pgfpathlineto{\pgfqpoint{3.252817in}{0.842118in}}%
\pgfpathlineto{\pgfqpoint{3.245965in}{0.844330in}}%
\pgfpathlineto{\pgfqpoint{3.224555in}{0.855963in}}%
\pgfpathlineto{\pgfqpoint{3.204610in}{0.868248in}}%
\pgfpathlineto{\pgfqpoint{3.186213in}{0.881145in}}%
\pgfpathlineto{\pgfqpoint{3.169443in}{0.894606in}}%
\pgfpathlineto{\pgfqpoint{3.154376in}{0.908585in}}%
\pgfpathlineto{\pgfqpoint{3.163004in}{0.905155in}}%
\pgfpathlineto{\pgfqpoint{3.172058in}{0.902266in}}%
\pgfpathlineto{\pgfqpoint{3.181501in}{0.899933in}}%
\pgfpathlineto{\pgfqpoint{3.191297in}{0.898166in}}%
\pgfpathlineto{\pgfqpoint{3.201405in}{0.896972in}}%
\pgfpathlineto{\pgfqpoint{3.214851in}{0.884432in}}%
\pgfpathlineto{\pgfqpoint{3.229837in}{0.872352in}}%
\pgfpathlineto{\pgfqpoint{3.246295in}{0.860774in}}%
\pgfpathlineto{\pgfqpoint{3.264157in}{0.849739in}}%
\pgfpathlineto{\pgfqpoint{3.283347in}{0.839287in}}%
\pgfpathclose%
\pgfusepath{fill}%
\end{pgfscope}%
\begin{pgfscope}%
\pgfpathrectangle{\pgfqpoint{2.548318in}{0.050000in}}{\pgfqpoint{2.081932in}{2.081932in}}%
\pgfusepath{clip}%
\pgfsetbuttcap%
\pgfsetroundjoin%
\definecolor{currentfill}{rgb}{0.993248,0.906157,0.143936}%
\pgfsetfillcolor{currentfill}%
\pgfsetlinewidth{0.000000pt}%
\definecolor{currentstroke}{rgb}{0.000000,0.000000,0.000000}%
\pgfsetstrokecolor{currentstroke}%
\pgfsetdash{}{0pt}%
\pgfpathmoveto{\pgfqpoint{3.271178in}{1.178203in}}%
\pgfpathlineto{\pgfqpoint{3.282107in}{1.180288in}}%
\pgfpathlineto{\pgfqpoint{3.292981in}{1.181816in}}%
\pgfpathlineto{\pgfqpoint{3.303756in}{1.182779in}}%
\pgfpathlineto{\pgfqpoint{3.314390in}{1.183173in}}%
\pgfpathlineto{\pgfqpoint{3.324841in}{1.182995in}}%
\pgfpathlineto{\pgfqpoint{3.316693in}{1.192515in}}%
\pgfpathlineto{\pgfqpoint{3.309723in}{1.202292in}}%
\pgfpathlineto{\pgfqpoint{3.303966in}{1.212288in}}%
\pgfpathlineto{\pgfqpoint{3.299448in}{1.222465in}}%
\pgfpathlineto{\pgfqpoint{3.296193in}{1.232783in}}%
\pgfpathlineto{\pgfqpoint{3.284630in}{1.234754in}}%
\pgfpathlineto{\pgfqpoint{3.272857in}{1.236181in}}%
\pgfpathlineto{\pgfqpoint{3.260920in}{1.237060in}}%
\pgfpathlineto{\pgfqpoint{3.248868in}{1.237386in}}%
\pgfpathlineto{\pgfqpoint{3.236748in}{1.237159in}}%
\pgfpathlineto{\pgfqpoint{3.240713in}{1.224927in}}%
\pgfpathlineto{\pgfqpoint{3.246171in}{1.212870in}}%
\pgfpathlineto{\pgfqpoint{3.253092in}{1.201033in}}%
\pgfpathlineto{\pgfqpoint{3.261441in}{1.189463in}}%
\pgfpathlineto{\pgfqpoint{3.271178in}{1.178203in}}%
\pgfpathclose%
\pgfusepath{fill}%
\end{pgfscope}%
\begin{pgfscope}%
\pgfpathrectangle{\pgfqpoint{2.548318in}{0.050000in}}{\pgfqpoint{2.081932in}{2.081932in}}%
\pgfusepath{clip}%
\pgfsetbuttcap%
\pgfsetroundjoin%
\definecolor{currentfill}{rgb}{0.150476,0.504369,0.557430}%
\pgfsetfillcolor{currentfill}%
\pgfsetlinewidth{0.000000pt}%
\definecolor{currentstroke}{rgb}{0.000000,0.000000,0.000000}%
\pgfsetstrokecolor{currentstroke}%
\pgfsetdash{}{0pt}%
\pgfpathmoveto{\pgfqpoint{4.182837in}{1.032127in}}%
\pgfpathlineto{\pgfqpoint{4.185229in}{1.040483in}}%
\pgfpathlineto{\pgfqpoint{4.186869in}{1.049005in}}%
\pgfpathlineto{\pgfqpoint{4.187752in}{1.057659in}}%
\pgfpathlineto{\pgfqpoint{4.187876in}{1.066411in}}%
\pgfpathlineto{\pgfqpoint{4.178502in}{1.048625in}}%
\pgfpathlineto{\pgfqpoint{4.166952in}{1.031216in}}%
\pgfpathlineto{\pgfqpoint{4.153286in}{1.014249in}}%
\pgfpathlineto{\pgfqpoint{4.137575in}{0.997784in}}%
\pgfpathlineto{\pgfqpoint{4.119893in}{0.981880in}}%
\pgfpathlineto{\pgfqpoint{4.119786in}{0.973261in}}%
\pgfpathlineto{\pgfqpoint{4.119018in}{0.964849in}}%
\pgfpathlineto{\pgfqpoint{4.117593in}{0.956678in}}%
\pgfpathlineto{\pgfqpoint{4.115513in}{0.948779in}}%
\pgfpathlineto{\pgfqpoint{4.133029in}{0.964462in}}%
\pgfpathlineto{\pgfqpoint{4.148591in}{0.980698in}}%
\pgfpathlineto{\pgfqpoint{4.162124in}{0.997428in}}%
\pgfpathlineto{\pgfqpoint{4.173560in}{1.014592in}}%
\pgfpathlineto{\pgfqpoint{4.182837in}{1.032127in}}%
\pgfpathclose%
\pgfusepath{fill}%
\end{pgfscope}%
\begin{pgfscope}%
\pgfpathrectangle{\pgfqpoint{2.548318in}{0.050000in}}{\pgfqpoint{2.081932in}{2.081932in}}%
\pgfusepath{clip}%
\pgfsetbuttcap%
\pgfsetroundjoin%
\definecolor{currentfill}{rgb}{0.162142,0.474838,0.558140}%
\pgfsetfillcolor{currentfill}%
\pgfsetlinewidth{0.000000pt}%
\definecolor{currentstroke}{rgb}{0.000000,0.000000,0.000000}%
\pgfsetstrokecolor{currentstroke}%
\pgfsetdash{}{0pt}%
\pgfpathmoveto{\pgfqpoint{3.640349in}{1.019283in}}%
\pgfpathlineto{\pgfqpoint{3.640449in}{1.010168in}}%
\pgfpathlineto{\pgfqpoint{3.640640in}{1.000722in}}%
\pgfpathlineto{\pgfqpoint{3.640921in}{0.990987in}}%
\pgfpathlineto{\pgfqpoint{3.641289in}{0.981002in}}%
\pgfpathlineto{\pgfqpoint{3.641745in}{0.970810in}}%
\pgfpathlineto{\pgfqpoint{3.628852in}{0.970259in}}%
\pgfpathlineto{\pgfqpoint{3.615916in}{0.970109in}}%
\pgfpathlineto{\pgfqpoint{3.602986in}{0.970360in}}%
\pgfpathlineto{\pgfqpoint{3.590112in}{0.971011in}}%
\pgfpathlineto{\pgfqpoint{3.577343in}{0.972061in}}%
\pgfpathlineto{\pgfqpoint{3.578099in}{0.982231in}}%
\pgfpathlineto{\pgfqpoint{3.578710in}{0.992199in}}%
\pgfpathlineto{\pgfqpoint{3.579175in}{1.001922in}}%
\pgfpathlineto{\pgfqpoint{3.579492in}{1.011359in}}%
\pgfpathlineto{\pgfqpoint{3.579659in}{1.020471in}}%
\pgfpathlineto{\pgfqpoint{3.591692in}{1.019475in}}%
\pgfpathlineto{\pgfqpoint{3.603824in}{1.018856in}}%
\pgfpathlineto{\pgfqpoint{3.616008in}{1.018618in}}%
\pgfpathlineto{\pgfqpoint{3.628199in}{1.018760in}}%
\pgfpathlineto{\pgfqpoint{3.640349in}{1.019283in}}%
\pgfpathclose%
\pgfusepath{fill}%
\end{pgfscope}%
\begin{pgfscope}%
\pgfpathrectangle{\pgfqpoint{2.548318in}{0.050000in}}{\pgfqpoint{2.081932in}{2.081932in}}%
\pgfusepath{clip}%
\pgfsetbuttcap%
\pgfsetroundjoin%
\definecolor{currentfill}{rgb}{0.855810,0.888601,0.097452}%
\pgfsetfillcolor{currentfill}%
\pgfsetlinewidth{0.000000pt}%
\definecolor{currentstroke}{rgb}{0.000000,0.000000,0.000000}%
\pgfsetstrokecolor{currentstroke}%
\pgfsetdash{}{0pt}%
\pgfpathmoveto{\pgfqpoint{3.934589in}{1.219886in}}%
\pgfpathlineto{\pgfqpoint{3.923452in}{1.217836in}}%
\pgfpathlineto{\pgfqpoint{3.912610in}{1.215249in}}%
\pgfpathlineto{\pgfqpoint{3.902105in}{1.212133in}}%
\pgfpathlineto{\pgfqpoint{3.891980in}{1.208501in}}%
\pgfpathlineto{\pgfqpoint{3.882276in}{1.204366in}}%
\pgfpathlineto{\pgfqpoint{3.878310in}{1.195920in}}%
\pgfpathlineto{\pgfqpoint{3.873311in}{1.187630in}}%
\pgfpathlineto{\pgfqpoint{3.867303in}{1.179526in}}%
\pgfpathlineto{\pgfqpoint{3.860311in}{1.171641in}}%
\pgfpathlineto{\pgfqpoint{3.852369in}{1.164003in}}%
\pgfpathlineto{\pgfqpoint{3.860923in}{1.166642in}}%
\pgfpathlineto{\pgfqpoint{3.869843in}{1.168717in}}%
\pgfpathlineto{\pgfqpoint{3.879093in}{1.170219in}}%
\pgfpathlineto{\pgfqpoint{3.888636in}{1.171146in}}%
\pgfpathlineto{\pgfqpoint{3.898433in}{1.171493in}}%
\pgfpathlineto{\pgfqpoint{3.908000in}{1.180641in}}%
\pgfpathlineto{\pgfqpoint{3.916435in}{1.190090in}}%
\pgfpathlineto{\pgfqpoint{3.923700in}{1.199806in}}%
\pgfpathlineto{\pgfqpoint{3.929761in}{1.209750in}}%
\pgfpathlineto{\pgfqpoint{3.934589in}{1.219886in}}%
\pgfpathclose%
\pgfusepath{fill}%
\end{pgfscope}%
\begin{pgfscope}%
\pgfpathrectangle{\pgfqpoint{2.548318in}{0.050000in}}{\pgfqpoint{2.081932in}{2.081932in}}%
\pgfusepath{clip}%
\pgfsetbuttcap%
\pgfsetroundjoin%
\definecolor{currentfill}{rgb}{0.278012,0.180367,0.486697}%
\pgfsetfillcolor{currentfill}%
\pgfsetlinewidth{0.000000pt}%
\definecolor{currentstroke}{rgb}{0.000000,0.000000,0.000000}%
\pgfsetstrokecolor{currentstroke}%
\pgfsetdash{}{0pt}%
\pgfpathmoveto{\pgfqpoint{3.716333in}{0.928072in}}%
\pgfpathlineto{\pgfqpoint{3.719632in}{0.917866in}}%
\pgfpathlineto{\pgfqpoint{3.723168in}{0.907785in}}%
\pgfpathlineto{\pgfqpoint{3.726925in}{0.897870in}}%
\pgfpathlineto{\pgfqpoint{3.730888in}{0.888163in}}%
\pgfpathlineto{\pgfqpoint{3.735039in}{0.878704in}}%
\pgfpathlineto{\pgfqpoint{3.718767in}{0.875362in}}%
\pgfpathlineto{\pgfqpoint{3.702112in}{0.872525in}}%
\pgfpathlineto{\pgfqpoint{3.685137in}{0.870202in}}%
\pgfpathlineto{\pgfqpoint{3.667906in}{0.868402in}}%
\pgfpathlineto{\pgfqpoint{3.650485in}{0.867132in}}%
\pgfpathlineto{\pgfqpoint{3.649323in}{0.876982in}}%
\pgfpathlineto{\pgfqpoint{3.648213in}{0.887062in}}%
\pgfpathlineto{\pgfqpoint{3.647161in}{0.897331in}}%
\pgfpathlineto{\pgfqpoint{3.646170in}{0.907746in}}%
\pgfpathlineto{\pgfqpoint{3.645246in}{0.918262in}}%
\pgfpathlineto{\pgfqpoint{3.659897in}{0.919339in}}%
\pgfpathlineto{\pgfqpoint{3.674386in}{0.920865in}}%
\pgfpathlineto{\pgfqpoint{3.688658in}{0.922834in}}%
\pgfpathlineto{\pgfqpoint{3.702658in}{0.925239in}}%
\pgfpathlineto{\pgfqpoint{3.716333in}{0.928072in}}%
\pgfpathclose%
\pgfusepath{fill}%
\end{pgfscope}%
\begin{pgfscope}%
\pgfpathrectangle{\pgfqpoint{2.548318in}{0.050000in}}{\pgfqpoint{2.081932in}{2.081932in}}%
\pgfusepath{clip}%
\pgfsetbuttcap%
\pgfsetroundjoin%
\definecolor{currentfill}{rgb}{0.120638,0.625828,0.533488}%
\pgfsetfillcolor{currentfill}%
\pgfsetlinewidth{0.000000pt}%
\definecolor{currentstroke}{rgb}{0.000000,0.000000,0.000000}%
\pgfsetstrokecolor{currentstroke}%
\pgfsetdash{}{0pt}%
\pgfpathmoveto{\pgfqpoint{3.754475in}{1.084789in}}%
\pgfpathlineto{\pgfqpoint{3.752459in}{1.077443in}}%
\pgfpathlineto{\pgfqpoint{3.750953in}{1.069692in}}%
\pgfpathlineto{\pgfqpoint{3.749965in}{1.061567in}}%
\pgfpathlineto{\pgfqpoint{3.749499in}{1.053102in}}%
\pgfpathlineto{\pgfqpoint{3.749558in}{1.044331in}}%
\pgfpathlineto{\pgfqpoint{3.740294in}{1.040335in}}%
\pgfpathlineto{\pgfqpoint{3.730552in}{1.036640in}}%
\pgfpathlineto{\pgfqpoint{3.720371in}{1.033258in}}%
\pgfpathlineto{\pgfqpoint{3.709792in}{1.030204in}}%
\pgfpathlineto{\pgfqpoint{3.698856in}{1.027487in}}%
\pgfpathlineto{\pgfqpoint{3.698820in}{1.036242in}}%
\pgfpathlineto{\pgfqpoint{3.699106in}{1.044626in}}%
\pgfpathlineto{\pgfqpoint{3.699713in}{1.052603in}}%
\pgfpathlineto{\pgfqpoint{3.700638in}{1.060141in}}%
\pgfpathlineto{\pgfqpoint{3.701876in}{1.067209in}}%
\pgfpathlineto{\pgfqpoint{3.713220in}{1.070044in}}%
\pgfpathlineto{\pgfqpoint{3.724194in}{1.073232in}}%
\pgfpathlineto{\pgfqpoint{3.734755in}{1.076761in}}%
\pgfpathlineto{\pgfqpoint{3.744862in}{1.080618in}}%
\pgfpathlineto{\pgfqpoint{3.754475in}{1.084789in}}%
\pgfpathclose%
\pgfusepath{fill}%
\end{pgfscope}%
\begin{pgfscope}%
\pgfpathrectangle{\pgfqpoint{2.548318in}{0.050000in}}{\pgfqpoint{2.081932in}{2.081932in}}%
\pgfusepath{clip}%
\pgfsetbuttcap%
\pgfsetroundjoin%
\definecolor{currentfill}{rgb}{0.227802,0.326594,0.546532}%
\pgfsetfillcolor{currentfill}%
\pgfsetlinewidth{0.000000pt}%
\definecolor{currentstroke}{rgb}{0.000000,0.000000,0.000000}%
\pgfsetstrokecolor{currentstroke}%
\pgfsetdash{}{0pt}%
\pgfpathmoveto{\pgfqpoint{3.641745in}{0.970810in}}%
\pgfpathlineto{\pgfqpoint{3.642285in}{0.960453in}}%
\pgfpathlineto{\pgfqpoint{3.642909in}{0.949975in}}%
\pgfpathlineto{\pgfqpoint{3.643612in}{0.939421in}}%
\pgfpathlineto{\pgfqpoint{3.644392in}{0.928835in}}%
\pgfpathlineto{\pgfqpoint{3.645246in}{0.918262in}}%
\pgfpathlineto{\pgfqpoint{3.630490in}{0.917637in}}%
\pgfpathlineto{\pgfqpoint{3.615684in}{0.917467in}}%
\pgfpathlineto{\pgfqpoint{3.600885in}{0.917752in}}%
\pgfpathlineto{\pgfqpoint{3.586150in}{0.918491in}}%
\pgfpathlineto{\pgfqpoint{3.571534in}{0.919682in}}%
\pgfpathlineto{\pgfqpoint{3.572951in}{0.930214in}}%
\pgfpathlineto{\pgfqpoint{3.574246in}{0.940762in}}%
\pgfpathlineto{\pgfqpoint{3.575413in}{0.951282in}}%
\pgfpathlineto{\pgfqpoint{3.576446in}{0.961730in}}%
\pgfpathlineto{\pgfqpoint{3.577343in}{0.972061in}}%
\pgfpathlineto{\pgfqpoint{3.590112in}{0.971011in}}%
\pgfpathlineto{\pgfqpoint{3.602986in}{0.970360in}}%
\pgfpathlineto{\pgfqpoint{3.615916in}{0.970109in}}%
\pgfpathlineto{\pgfqpoint{3.628852in}{0.970259in}}%
\pgfpathlineto{\pgfqpoint{3.641745in}{0.970810in}}%
\pgfpathclose%
\pgfusepath{fill}%
\end{pgfscope}%
\begin{pgfscope}%
\pgfpathrectangle{\pgfqpoint{2.548318in}{0.050000in}}{\pgfqpoint{2.081932in}{2.081932in}}%
\pgfusepath{clip}%
\pgfsetbuttcap%
\pgfsetroundjoin%
\definecolor{currentfill}{rgb}{0.162142,0.474838,0.558140}%
\pgfsetfillcolor{currentfill}%
\pgfsetlinewidth{0.000000pt}%
\definecolor{currentstroke}{rgb}{0.000000,0.000000,0.000000}%
\pgfsetstrokecolor{currentstroke}%
\pgfsetdash{}{0pt}%
\pgfpathmoveto{\pgfqpoint{3.698856in}{1.027487in}}%
\pgfpathlineto{\pgfqpoint{3.699215in}{1.018396in}}%
\pgfpathlineto{\pgfqpoint{3.699896in}{1.009007in}}%
\pgfpathlineto{\pgfqpoint{3.700896in}{0.999360in}}%
\pgfpathlineto{\pgfqpoint{3.702212in}{0.989494in}}%
\pgfpathlineto{\pgfqpoint{3.703837in}{0.979450in}}%
\pgfpathlineto{\pgfqpoint{3.691896in}{0.976956in}}%
\pgfpathlineto{\pgfqpoint{3.679669in}{0.974838in}}%
\pgfpathlineto{\pgfqpoint{3.667202in}{0.973103in}}%
\pgfpathlineto{\pgfqpoint{3.654545in}{0.971759in}}%
\pgfpathlineto{\pgfqpoint{3.641745in}{0.970810in}}%
\pgfpathlineto{\pgfqpoint{3.641289in}{0.981002in}}%
\pgfpathlineto{\pgfqpoint{3.640921in}{0.990987in}}%
\pgfpathlineto{\pgfqpoint{3.640640in}{1.000722in}}%
\pgfpathlineto{\pgfqpoint{3.640449in}{1.010168in}}%
\pgfpathlineto{\pgfqpoint{3.640349in}{1.019283in}}%
\pgfpathlineto{\pgfqpoint{3.652410in}{1.020184in}}%
\pgfpathlineto{\pgfqpoint{3.664338in}{1.021461in}}%
\pgfpathlineto{\pgfqpoint{3.676084in}{1.023108in}}%
\pgfpathlineto{\pgfqpoint{3.687605in}{1.025119in}}%
\pgfpathlineto{\pgfqpoint{3.698856in}{1.027487in}}%
\pgfpathclose%
\pgfusepath{fill}%
\end{pgfscope}%
\begin{pgfscope}%
\pgfpathrectangle{\pgfqpoint{2.548318in}{0.050000in}}{\pgfqpoint{2.081932in}{2.081932in}}%
\pgfusepath{clip}%
\pgfsetbuttcap%
\pgfsetroundjoin%
\definecolor{currentfill}{rgb}{0.278012,0.180367,0.486697}%
\pgfsetfillcolor{currentfill}%
\pgfsetlinewidth{0.000000pt}%
\definecolor{currentstroke}{rgb}{0.000000,0.000000,0.000000}%
\pgfsetstrokecolor{currentstroke}%
\pgfsetdash{}{0pt}%
\pgfpathmoveto{\pgfqpoint{3.571534in}{0.919682in}}%
\pgfpathlineto{\pgfqpoint{3.570001in}{0.909211in}}%
\pgfpathlineto{\pgfqpoint{3.568358in}{0.898844in}}%
\pgfpathlineto{\pgfqpoint{3.566612in}{0.888627in}}%
\pgfpathlineto{\pgfqpoint{3.564771in}{0.878600in}}%
\pgfpathlineto{\pgfqpoint{3.562842in}{0.868806in}}%
\pgfpathlineto{\pgfqpoint{3.545669in}{0.870739in}}%
\pgfpathlineto{\pgfqpoint{3.528767in}{0.873192in}}%
\pgfpathlineto{\pgfqpoint{3.512200in}{0.876157in}}%
\pgfpathlineto{\pgfqpoint{3.496033in}{0.879624in}}%
\pgfpathlineto{\pgfqpoint{3.480327in}{0.883581in}}%
\pgfpathlineto{\pgfqpoint{3.485177in}{0.892874in}}%
\pgfpathlineto{\pgfqpoint{3.489806in}{0.902424in}}%
\pgfpathlineto{\pgfqpoint{3.494196in}{0.912189in}}%
\pgfpathlineto{\pgfqpoint{3.498326in}{0.922130in}}%
\pgfpathlineto{\pgfqpoint{3.502179in}{0.932204in}}%
\pgfpathlineto{\pgfqpoint{3.515374in}{0.928852in}}%
\pgfpathlineto{\pgfqpoint{3.528960in}{0.925913in}}%
\pgfpathlineto{\pgfqpoint{3.542884in}{0.923400in}}%
\pgfpathlineto{\pgfqpoint{3.557094in}{0.921320in}}%
\pgfpathlineto{\pgfqpoint{3.571534in}{0.919682in}}%
\pgfpathclose%
\pgfusepath{fill}%
\end{pgfscope}%
\begin{pgfscope}%
\pgfpathrectangle{\pgfqpoint{2.548318in}{0.050000in}}{\pgfqpoint{2.081932in}{2.081932in}}%
\pgfusepath{clip}%
\pgfsetbuttcap%
\pgfsetroundjoin%
\definecolor{currentfill}{rgb}{0.278791,0.062145,0.386592}%
\pgfsetfillcolor{currentfill}%
\pgfsetlinewidth{0.000000pt}%
\definecolor{currentstroke}{rgb}{0.000000,0.000000,0.000000}%
\pgfsetstrokecolor{currentstroke}%
\pgfsetdash{}{0pt}%
\pgfpathmoveto{\pgfqpoint{3.808522in}{0.902504in}}%
\pgfpathlineto{\pgfqpoint{3.815575in}{0.894175in}}%
\pgfpathlineto{\pgfqpoint{3.822879in}{0.886202in}}%
\pgfpathlineto{\pgfqpoint{3.830404in}{0.878617in}}%
\pgfpathlineto{\pgfqpoint{3.838118in}{0.871451in}}%
\pgfpathlineto{\pgfqpoint{3.845989in}{0.864734in}}%
\pgfpathlineto{\pgfqpoint{3.829861in}{0.858016in}}%
\pgfpathlineto{\pgfqpoint{3.812930in}{0.851809in}}%
\pgfpathlineto{\pgfqpoint{3.795261in}{0.846133in}}%
\pgfpathlineto{\pgfqpoint{3.776924in}{0.841010in}}%
\pgfpathlineto{\pgfqpoint{3.757988in}{0.836458in}}%
\pgfpathlineto{\pgfqpoint{3.753170in}{0.844115in}}%
\pgfpathlineto{\pgfqpoint{3.748446in}{0.852203in}}%
\pgfpathlineto{\pgfqpoint{3.743837in}{0.860687in}}%
\pgfpathlineto{\pgfqpoint{3.739362in}{0.869533in}}%
\pgfpathlineto{\pgfqpoint{3.735039in}{0.878704in}}%
\pgfpathlineto{\pgfqpoint{3.750866in}{0.882538in}}%
\pgfpathlineto{\pgfqpoint{3.766186in}{0.886852in}}%
\pgfpathlineto{\pgfqpoint{3.780940in}{0.891629in}}%
\pgfpathlineto{\pgfqpoint{3.795070in}{0.896853in}}%
\pgfpathlineto{\pgfqpoint{3.808522in}{0.902504in}}%
\pgfpathclose%
\pgfusepath{fill}%
\end{pgfscope}%
\begin{pgfscope}%
\pgfpathrectangle{\pgfqpoint{2.548318in}{0.050000in}}{\pgfqpoint{2.081932in}{2.081932in}}%
\pgfusepath{clip}%
\pgfsetbuttcap%
\pgfsetroundjoin%
\definecolor{currentfill}{rgb}{0.268510,0.009605,0.335427}%
\pgfsetfillcolor{currentfill}%
\pgfsetlinewidth{0.000000pt}%
\definecolor{currentstroke}{rgb}{0.000000,0.000000,0.000000}%
\pgfsetstrokecolor{currentstroke}%
\pgfsetdash{}{0pt}%
\pgfpathmoveto{\pgfqpoint{3.965163in}{0.886105in}}%
\pgfpathlineto{\pgfqpoint{3.975728in}{0.883982in}}%
\pgfpathlineto{\pgfqpoint{3.986198in}{0.882440in}}%
\pgfpathlineto{\pgfqpoint{3.996530in}{0.881485in}}%
\pgfpathlineto{\pgfqpoint{4.006682in}{0.881119in}}%
\pgfpathlineto{\pgfqpoint{4.016613in}{0.881343in}}%
\pgfpathlineto{\pgfqpoint{4.001255in}{0.869385in}}%
\pgfpathlineto{\pgfqpoint{3.984442in}{0.857941in}}%
\pgfpathlineto{\pgfqpoint{3.966244in}{0.847049in}}%
\pgfpathlineto{\pgfqpoint{3.946736in}{0.836750in}}%
\pgfpathlineto{\pgfqpoint{3.926000in}{0.827077in}}%
\pgfpathlineto{\pgfqpoint{3.918389in}{0.828194in}}%
\pgfpathlineto{\pgfqpoint{3.910605in}{0.829926in}}%
\pgfpathlineto{\pgfqpoint{3.902679in}{0.832268in}}%
\pgfpathlineto{\pgfqpoint{3.894644in}{0.835211in}}%
\pgfpathlineto{\pgfqpoint{3.886532in}{0.838746in}}%
\pgfpathlineto{\pgfqpoint{3.904558in}{0.847195in}}%
\pgfpathlineto{\pgfqpoint{3.921502in}{0.856188in}}%
\pgfpathlineto{\pgfqpoint{3.937293in}{0.865694in}}%
\pgfpathlineto{\pgfqpoint{3.951868in}{0.875678in}}%
\pgfpathlineto{\pgfqpoint{3.965163in}{0.886105in}}%
\pgfpathclose%
\pgfusepath{fill}%
\end{pgfscope}%
\begin{pgfscope}%
\pgfpathrectangle{\pgfqpoint{2.548318in}{0.050000in}}{\pgfqpoint{2.081932in}{2.081932in}}%
\pgfusepath{clip}%
\pgfsetbuttcap%
\pgfsetroundjoin%
\definecolor{currentfill}{rgb}{0.296479,0.761561,0.424223}%
\pgfsetfillcolor{currentfill}%
\pgfsetlinewidth{0.000000pt}%
\definecolor{currentstroke}{rgb}{0.000000,0.000000,0.000000}%
\pgfsetstrokecolor{currentstroke}%
\pgfsetdash{}{0pt}%
\pgfpathmoveto{\pgfqpoint{3.816268in}{1.142757in}}%
\pgfpathlineto{\pgfqpoint{3.810592in}{1.137010in}}%
\pgfpathlineto{\pgfqpoint{3.805491in}{1.130812in}}%
\pgfpathlineto{\pgfqpoint{3.800987in}{1.124187in}}%
\pgfpathlineto{\pgfqpoint{3.797100in}{1.117160in}}%
\pgfpathlineto{\pgfqpoint{3.793846in}{1.109760in}}%
\pgfpathlineto{\pgfqpoint{3.787242in}{1.104279in}}%
\pgfpathlineto{\pgfqpoint{3.779972in}{1.099021in}}%
\pgfpathlineto{\pgfqpoint{3.772065in}{1.094008in}}%
\pgfpathlineto{\pgfqpoint{3.763555in}{1.089258in}}%
\pgfpathlineto{\pgfqpoint{3.754475in}{1.084789in}}%
\pgfpathlineto{\pgfqpoint{3.756992in}{1.091701in}}%
\pgfpathlineto{\pgfqpoint{3.759999in}{1.098150in}}%
\pgfpathlineto{\pgfqpoint{3.763482in}{1.104112in}}%
\pgfpathlineto{\pgfqpoint{3.767426in}{1.109563in}}%
\pgfpathlineto{\pgfqpoint{3.771814in}{1.114481in}}%
\pgfpathlineto{\pgfqpoint{3.782058in}{1.119539in}}%
\pgfpathlineto{\pgfqpoint{3.791663in}{1.124917in}}%
\pgfpathlineto{\pgfqpoint{3.800591in}{1.130593in}}%
\pgfpathlineto{\pgfqpoint{3.808804in}{1.136548in}}%
\pgfpathlineto{\pgfqpoint{3.816268in}{1.142757in}}%
\pgfpathclose%
\pgfusepath{fill}%
\end{pgfscope}%
\begin{pgfscope}%
\pgfpathrectangle{\pgfqpoint{2.548318in}{0.050000in}}{\pgfqpoint{2.081932in}{2.081932in}}%
\pgfusepath{clip}%
\pgfsetbuttcap%
\pgfsetroundjoin%
\definecolor{currentfill}{rgb}{0.162142,0.474838,0.558140}%
\pgfsetfillcolor{currentfill}%
\pgfsetlinewidth{0.000000pt}%
\definecolor{currentstroke}{rgb}{0.000000,0.000000,0.000000}%
\pgfsetstrokecolor{currentstroke}%
\pgfsetdash{}{0pt}%
\pgfpathmoveto{\pgfqpoint{3.579659in}{1.020471in}}%
\pgfpathlineto{\pgfqpoint{3.579492in}{1.011359in}}%
\pgfpathlineto{\pgfqpoint{3.579175in}{1.001922in}}%
\pgfpathlineto{\pgfqpoint{3.578710in}{0.992199in}}%
\pgfpathlineto{\pgfqpoint{3.578099in}{0.982231in}}%
\pgfpathlineto{\pgfqpoint{3.577343in}{0.972061in}}%
\pgfpathlineto{\pgfqpoint{3.564728in}{0.973504in}}%
\pgfpathlineto{\pgfqpoint{3.552317in}{0.975336in}}%
\pgfpathlineto{\pgfqpoint{3.540156in}{0.977550in}}%
\pgfpathlineto{\pgfqpoint{3.528293in}{0.980137in}}%
\pgfpathlineto{\pgfqpoint{3.516774in}{0.983089in}}%
\pgfpathlineto{\pgfqpoint{3.518672in}{0.993070in}}%
\pgfpathlineto{\pgfqpoint{3.520208in}{1.002886in}}%
\pgfpathlineto{\pgfqpoint{3.521376in}{1.012496in}}%
\pgfpathlineto{\pgfqpoint{3.522172in}{1.021861in}}%
\pgfpathlineto{\pgfqpoint{3.522591in}{1.030942in}}%
\pgfpathlineto{\pgfqpoint{3.533443in}{1.028139in}}%
\pgfpathlineto{\pgfqpoint{3.544619in}{1.025682in}}%
\pgfpathlineto{\pgfqpoint{3.556077in}{1.023580in}}%
\pgfpathlineto{\pgfqpoint{3.567772in}{1.021841in}}%
\pgfpathlineto{\pgfqpoint{3.579659in}{1.020471in}}%
\pgfpathclose%
\pgfusepath{fill}%
\end{pgfscope}%
\begin{pgfscope}%
\pgfpathrectangle{\pgfqpoint{2.548318in}{0.050000in}}{\pgfqpoint{2.081932in}{2.081932in}}%
\pgfusepath{clip}%
\pgfsetbuttcap%
\pgfsetroundjoin%
\definecolor{currentfill}{rgb}{0.227802,0.326594,0.546532}%
\pgfsetfillcolor{currentfill}%
\pgfsetlinewidth{0.000000pt}%
\definecolor{currentstroke}{rgb}{0.000000,0.000000,0.000000}%
\pgfsetstrokecolor{currentstroke}%
\pgfsetdash{}{0pt}%
\pgfpathmoveto{\pgfqpoint{3.703837in}{0.979450in}}%
\pgfpathlineto{\pgfqpoint{3.705766in}{0.969272in}}%
\pgfpathlineto{\pgfqpoint{3.707990in}{0.959002in}}%
\pgfpathlineto{\pgfqpoint{3.710500in}{0.948683in}}%
\pgfpathlineto{\pgfqpoint{3.713285in}{0.938358in}}%
\pgfpathlineto{\pgfqpoint{3.716333in}{0.928072in}}%
\pgfpathlineto{\pgfqpoint{3.702658in}{0.925239in}}%
\pgfpathlineto{\pgfqpoint{3.688658in}{0.922834in}}%
\pgfpathlineto{\pgfqpoint{3.674386in}{0.920865in}}%
\pgfpathlineto{\pgfqpoint{3.659897in}{0.919339in}}%
\pgfpathlineto{\pgfqpoint{3.645246in}{0.918262in}}%
\pgfpathlineto{\pgfqpoint{3.644392in}{0.928835in}}%
\pgfpathlineto{\pgfqpoint{3.643612in}{0.939421in}}%
\pgfpathlineto{\pgfqpoint{3.642909in}{0.949975in}}%
\pgfpathlineto{\pgfqpoint{3.642285in}{0.960453in}}%
\pgfpathlineto{\pgfqpoint{3.641745in}{0.970810in}}%
\pgfpathlineto{\pgfqpoint{3.654545in}{0.971759in}}%
\pgfpathlineto{\pgfqpoint{3.667202in}{0.973103in}}%
\pgfpathlineto{\pgfqpoint{3.679669in}{0.974838in}}%
\pgfpathlineto{\pgfqpoint{3.691896in}{0.976956in}}%
\pgfpathlineto{\pgfqpoint{3.703837in}{0.979450in}}%
\pgfpathclose%
\pgfusepath{fill}%
\end{pgfscope}%
\begin{pgfscope}%
\pgfpathrectangle{\pgfqpoint{2.548318in}{0.050000in}}{\pgfqpoint{2.081932in}{2.081932in}}%
\pgfusepath{clip}%
\pgfsetbuttcap%
\pgfsetroundjoin%
\definecolor{currentfill}{rgb}{0.120638,0.625828,0.533488}%
\pgfsetfillcolor{currentfill}%
\pgfsetlinewidth{0.000000pt}%
\definecolor{currentstroke}{rgb}{0.000000,0.000000,0.000000}%
\pgfsetstrokecolor{currentstroke}%
\pgfsetdash{}{0pt}%
\pgfpathmoveto{\pgfqpoint{3.519064in}{1.070815in}}%
\pgfpathlineto{\pgfqpoint{3.520510in}{1.063689in}}%
\pgfpathlineto{\pgfqpoint{3.521590in}{1.056108in}}%
\pgfpathlineto{\pgfqpoint{3.522299in}{1.048100in}}%
\pgfpathlineto{\pgfqpoint{3.522634in}{1.039700in}}%
\pgfpathlineto{\pgfqpoint{3.522591in}{1.030942in}}%
\pgfpathlineto{\pgfqpoint{3.512107in}{1.034080in}}%
\pgfpathlineto{\pgfqpoint{3.502032in}{1.037542in}}%
\pgfpathlineto{\pgfqpoint{3.492405in}{1.041315in}}%
\pgfpathlineto{\pgfqpoint{3.483264in}{1.045384in}}%
\pgfpathlineto{\pgfqpoint{3.474647in}{1.049734in}}%
\pgfpathlineto{\pgfqpoint{3.474711in}{1.058510in}}%
\pgfpathlineto{\pgfqpoint{3.474207in}{1.067001in}}%
\pgfpathlineto{\pgfqpoint{3.473136in}{1.075173in}}%
\pgfpathlineto{\pgfqpoint{3.471505in}{1.082993in}}%
\pgfpathlineto{\pgfqpoint{3.469322in}{1.090429in}}%
\pgfpathlineto{\pgfqpoint{3.478264in}{1.085888in}}%
\pgfpathlineto{\pgfqpoint{3.487748in}{1.081640in}}%
\pgfpathlineto{\pgfqpoint{3.497736in}{1.077703in}}%
\pgfpathlineto{\pgfqpoint{3.508188in}{1.074090in}}%
\pgfpathlineto{\pgfqpoint{3.519064in}{1.070815in}}%
\pgfpathclose%
\pgfusepath{fill}%
\end{pgfscope}%
\begin{pgfscope}%
\pgfpathrectangle{\pgfqpoint{2.548318in}{0.050000in}}{\pgfqpoint{2.081932in}{2.081932in}}%
\pgfusepath{clip}%
\pgfsetbuttcap%
\pgfsetroundjoin%
\definecolor{currentfill}{rgb}{0.124780,0.640461,0.527068}%
\pgfsetfillcolor{currentfill}%
\pgfsetlinewidth{0.000000pt}%
\definecolor{currentstroke}{rgb}{0.000000,0.000000,0.000000}%
\pgfsetstrokecolor{currentstroke}%
\pgfsetdash{}{0pt}%
\pgfpathmoveto{\pgfqpoint{3.093377in}{1.001882in}}%
\pgfpathlineto{\pgfqpoint{3.093956in}{1.010678in}}%
\pgfpathlineto{\pgfqpoint{3.095221in}{1.019587in}}%
\pgfpathlineto{\pgfqpoint{3.097169in}{1.028572in}}%
\pgfpathlineto{\pgfqpoint{3.099789in}{1.037598in}}%
\pgfpathlineto{\pgfqpoint{3.103072in}{1.046629in}}%
\pgfpathlineto{\pgfqpoint{3.088168in}{1.063033in}}%
\pgfpathlineto{\pgfqpoint{3.075292in}{1.079917in}}%
\pgfpathlineto{\pgfqpoint{3.064509in}{1.097219in}}%
\pgfpathlineto{\pgfqpoint{3.055878in}{1.114875in}}%
\pgfpathlineto{\pgfqpoint{3.049451in}{1.132819in}}%
\pgfpathlineto{\pgfqpoint{3.045778in}{1.124224in}}%
\pgfpathlineto{\pgfqpoint{3.042844in}{1.115520in}}%
\pgfpathlineto{\pgfqpoint{3.040664in}{1.106742in}}%
\pgfpathlineto{\pgfqpoint{3.039247in}{1.097926in}}%
\pgfpathlineto{\pgfqpoint{3.038600in}{1.089107in}}%
\pgfpathlineto{\pgfqpoint{3.045180in}{1.070944in}}%
\pgfpathlineto{\pgfqpoint{3.054004in}{1.053074in}}%
\pgfpathlineto{\pgfqpoint{3.065020in}{1.035564in}}%
\pgfpathlineto{\pgfqpoint{3.078167in}{1.018479in}}%
\pgfpathlineto{\pgfqpoint{3.093377in}{1.001882in}}%
\pgfpathclose%
\pgfusepath{fill}%
\end{pgfscope}%
\begin{pgfscope}%
\pgfpathrectangle{\pgfqpoint{2.548318in}{0.050000in}}{\pgfqpoint{2.081932in}{2.081932in}}%
\pgfusepath{clip}%
\pgfsetbuttcap%
\pgfsetroundjoin%
\definecolor{currentfill}{rgb}{0.227802,0.326594,0.546532}%
\pgfsetfillcolor{currentfill}%
\pgfsetlinewidth{0.000000pt}%
\definecolor{currentstroke}{rgb}{0.000000,0.000000,0.000000}%
\pgfsetstrokecolor{currentstroke}%
\pgfsetdash{}{0pt}%
\pgfpathmoveto{\pgfqpoint{3.577343in}{0.972061in}}%
\pgfpathlineto{\pgfqpoint{3.576446in}{0.961730in}}%
\pgfpathlineto{\pgfqpoint{3.575413in}{0.951282in}}%
\pgfpathlineto{\pgfqpoint{3.574246in}{0.940762in}}%
\pgfpathlineto{\pgfqpoint{3.572951in}{0.930214in}}%
\pgfpathlineto{\pgfqpoint{3.571534in}{0.919682in}}%
\pgfpathlineto{\pgfqpoint{3.557094in}{0.921320in}}%
\pgfpathlineto{\pgfqpoint{3.542884in}{0.923400in}}%
\pgfpathlineto{\pgfqpoint{3.528960in}{0.925913in}}%
\pgfpathlineto{\pgfqpoint{3.515374in}{0.928852in}}%
\pgfpathlineto{\pgfqpoint{3.502179in}{0.932204in}}%
\pgfpathlineto{\pgfqpoint{3.505740in}{0.942370in}}%
\pgfpathlineto{\pgfqpoint{3.508993in}{0.952584in}}%
\pgfpathlineto{\pgfqpoint{3.511924in}{0.962804in}}%
\pgfpathlineto{\pgfqpoint{3.514522in}{0.972987in}}%
\pgfpathlineto{\pgfqpoint{3.516774in}{0.983089in}}%
\pgfpathlineto{\pgfqpoint{3.528293in}{0.980137in}}%
\pgfpathlineto{\pgfqpoint{3.540156in}{0.977550in}}%
\pgfpathlineto{\pgfqpoint{3.552317in}{0.975336in}}%
\pgfpathlineto{\pgfqpoint{3.564728in}{0.973504in}}%
\pgfpathlineto{\pgfqpoint{3.577343in}{0.972061in}}%
\pgfpathclose%
\pgfusepath{fill}%
\end{pgfscope}%
\begin{pgfscope}%
\pgfpathrectangle{\pgfqpoint{2.548318in}{0.050000in}}{\pgfqpoint{2.081932in}{2.081932in}}%
\pgfusepath{clip}%
\pgfsetbuttcap%
\pgfsetroundjoin%
\definecolor{currentfill}{rgb}{0.855810,0.888601,0.097452}%
\pgfsetfillcolor{currentfill}%
\pgfsetlinewidth{0.000000pt}%
\definecolor{currentstroke}{rgb}{0.000000,0.000000,0.000000}%
\pgfsetstrokecolor{currentstroke}%
\pgfsetdash{}{0pt}%
\pgfpathmoveto{\pgfqpoint{3.324841in}{1.182995in}}%
\pgfpathlineto{\pgfqpoint{3.335066in}{1.182245in}}%
\pgfpathlineto{\pgfqpoint{3.345024in}{1.180926in}}%
\pgfpathlineto{\pgfqpoint{3.354676in}{1.179042in}}%
\pgfpathlineto{\pgfqpoint{3.363982in}{1.176599in}}%
\pgfpathlineto{\pgfqpoint{3.372905in}{1.173606in}}%
\pgfpathlineto{\pgfqpoint{3.366155in}{1.181549in}}%
\pgfpathlineto{\pgfqpoint{3.360395in}{1.189703in}}%
\pgfpathlineto{\pgfqpoint{3.355651in}{1.198035in}}%
\pgfpathlineto{\pgfqpoint{3.351945in}{1.206514in}}%
\pgfpathlineto{\pgfqpoint{3.349297in}{1.215106in}}%
\pgfpathlineto{\pgfqpoint{3.339448in}{1.219643in}}%
\pgfpathlineto{\pgfqpoint{3.329172in}{1.223693in}}%
\pgfpathlineto{\pgfqpoint{3.318508in}{1.227242in}}%
\pgfpathlineto{\pgfqpoint{3.307501in}{1.230276in}}%
\pgfpathlineto{\pgfqpoint{3.296193in}{1.232783in}}%
\pgfpathlineto{\pgfqpoint{3.299448in}{1.222465in}}%
\pgfpathlineto{\pgfqpoint{3.303966in}{1.212288in}}%
\pgfpathlineto{\pgfqpoint{3.309723in}{1.202292in}}%
\pgfpathlineto{\pgfqpoint{3.316693in}{1.192515in}}%
\pgfpathlineto{\pgfqpoint{3.324841in}{1.182995in}}%
\pgfpathclose%
\pgfusepath{fill}%
\end{pgfscope}%
\begin{pgfscope}%
\pgfpathrectangle{\pgfqpoint{2.548318in}{0.050000in}}{\pgfqpoint{2.081932in}{2.081932in}}%
\pgfusepath{clip}%
\pgfsetbuttcap%
\pgfsetroundjoin%
\definecolor{currentfill}{rgb}{0.296479,0.761561,0.424223}%
\pgfsetfillcolor{currentfill}%
\pgfsetlinewidth{0.000000pt}%
\definecolor{currentstroke}{rgb}{0.000000,0.000000,0.000000}%
\pgfsetstrokecolor{currentstroke}%
\pgfsetdash{}{0pt}%
\pgfpathmoveto{\pgfqpoint{3.450539in}{1.120865in}}%
\pgfpathlineto{\pgfqpoint{3.455293in}{1.115760in}}%
\pgfpathlineto{\pgfqpoint{3.459566in}{1.110141in}}%
\pgfpathlineto{\pgfqpoint{3.463339in}{1.104030in}}%
\pgfpathlineto{\pgfqpoint{3.466596in}{1.097450in}}%
\pgfpathlineto{\pgfqpoint{3.469322in}{1.090429in}}%
\pgfpathlineto{\pgfqpoint{3.460958in}{1.095247in}}%
\pgfpathlineto{\pgfqpoint{3.453207in}{1.100324in}}%
\pgfpathlineto{\pgfqpoint{3.446099in}{1.105640in}}%
\pgfpathlineto{\pgfqpoint{3.439664in}{1.111175in}}%
\pgfpathlineto{\pgfqpoint{3.433930in}{1.116907in}}%
\pgfpathlineto{\pgfqpoint{3.430539in}{1.124447in}}%
\pgfpathlineto{\pgfqpoint{3.426487in}{1.131640in}}%
\pgfpathlineto{\pgfqpoint{3.421792in}{1.138456in}}%
\pgfpathlineto{\pgfqpoint{3.416474in}{1.144870in}}%
\pgfpathlineto{\pgfqpoint{3.410555in}{1.150855in}}%
\pgfpathlineto{\pgfqpoint{3.417042in}{1.144359in}}%
\pgfpathlineto{\pgfqpoint{3.424315in}{1.138089in}}%
\pgfpathlineto{\pgfqpoint{3.432346in}{1.132068in}}%
\pgfpathlineto{\pgfqpoint{3.441099in}{1.126320in}}%
\pgfpathlineto{\pgfqpoint{3.450539in}{1.120865in}}%
\pgfpathclose%
\pgfusepath{fill}%
\end{pgfscope}%
\begin{pgfscope}%
\pgfpathrectangle{\pgfqpoint{2.548318in}{0.050000in}}{\pgfqpoint{2.081932in}{2.081932in}}%
\pgfusepath{clip}%
\pgfsetbuttcap%
\pgfsetroundjoin%
\definecolor{currentfill}{rgb}{0.278791,0.062145,0.386592}%
\pgfsetfillcolor{currentfill}%
\pgfsetlinewidth{0.000000pt}%
\definecolor{currentstroke}{rgb}{0.000000,0.000000,0.000000}%
\pgfsetstrokecolor{currentstroke}%
\pgfsetdash{}{0pt}%
\pgfpathmoveto{\pgfqpoint{3.480327in}{0.883581in}}%
\pgfpathlineto{\pgfqpoint{3.475276in}{0.874582in}}%
\pgfpathlineto{\pgfqpoint{3.470047in}{0.865914in}}%
\pgfpathlineto{\pgfqpoint{3.464661in}{0.857614in}}%
\pgfpathlineto{\pgfqpoint{3.459141in}{0.849714in}}%
\pgfpathlineto{\pgfqpoint{3.453511in}{0.842248in}}%
\pgfpathlineto{\pgfqpoint{3.435332in}{0.847512in}}%
\pgfpathlineto{\pgfqpoint{3.417840in}{0.853323in}}%
\pgfpathlineto{\pgfqpoint{3.401103in}{0.859662in}}%
\pgfpathlineto{\pgfqpoint{3.385185in}{0.866504in}}%
\pgfpathlineto{\pgfqpoint{3.370150in}{0.873827in}}%
\pgfpathlineto{\pgfqpoint{3.378684in}{0.880241in}}%
\pgfpathlineto{\pgfqpoint{3.387047in}{0.887108in}}%
\pgfpathlineto{\pgfqpoint{3.395205in}{0.894403in}}%
\pgfpathlineto{\pgfqpoint{3.403123in}{0.902094in}}%
\pgfpathlineto{\pgfqpoint{3.410767in}{0.910150in}}%
\pgfpathlineto{\pgfqpoint{3.423296in}{0.903993in}}%
\pgfpathlineto{\pgfqpoint{3.436570in}{0.898237in}}%
\pgfpathlineto{\pgfqpoint{3.450537in}{0.892904in}}%
\pgfpathlineto{\pgfqpoint{3.465142in}{0.888013in}}%
\pgfpathlineto{\pgfqpoint{3.480327in}{0.883581in}}%
\pgfpathclose%
\pgfusepath{fill}%
\end{pgfscope}%
\begin{pgfscope}%
\pgfpathrectangle{\pgfqpoint{2.548318in}{0.050000in}}{\pgfqpoint{2.081932in}{2.081932in}}%
\pgfusepath{clip}%
\pgfsetbuttcap%
\pgfsetroundjoin%
\definecolor{currentfill}{rgb}{0.206756,0.371758,0.553117}%
\pgfsetfillcolor{currentfill}%
\pgfsetlinewidth{0.000000pt}%
\definecolor{currentstroke}{rgb}{0.000000,0.000000,0.000000}%
\pgfsetstrokecolor{currentstroke}%
\pgfsetdash{}{0pt}%
\pgfpathmoveto{\pgfqpoint{4.160006in}{0.994001in}}%
\pgfpathlineto{\pgfqpoint{4.165969in}{1.001034in}}%
\pgfpathlineto{\pgfqpoint{4.171252in}{1.008393in}}%
\pgfpathlineto{\pgfqpoint{4.175836in}{1.016050in}}%
\pgfpathlineto{\pgfqpoint{4.179702in}{1.023972in}}%
\pgfpathlineto{\pgfqpoint{4.182837in}{1.032127in}}%
\pgfpathlineto{\pgfqpoint{4.173560in}{1.014592in}}%
\pgfpathlineto{\pgfqpoint{4.162124in}{0.997428in}}%
\pgfpathlineto{\pgfqpoint{4.148591in}{0.980698in}}%
\pgfpathlineto{\pgfqpoint{4.133029in}{0.964462in}}%
\pgfpathlineto{\pgfqpoint{4.115513in}{0.948779in}}%
\pgfpathlineto{\pgfqpoint{4.112787in}{0.941186in}}%
\pgfpathlineto{\pgfqpoint{4.109424in}{0.933927in}}%
\pgfpathlineto{\pgfqpoint{4.105437in}{0.927034in}}%
\pgfpathlineto{\pgfqpoint{4.100841in}{0.920534in}}%
\pgfpathlineto{\pgfqpoint{4.095652in}{0.914453in}}%
\pgfpathlineto{\pgfqpoint{4.112421in}{0.929427in}}%
\pgfpathlineto{\pgfqpoint{4.127308in}{0.944926in}}%
\pgfpathlineto{\pgfqpoint{4.140244in}{0.960894in}}%
\pgfpathlineto{\pgfqpoint{4.151163in}{0.977272in}}%
\pgfpathlineto{\pgfqpoint{4.160006in}{0.994001in}}%
\pgfpathclose%
\pgfusepath{fill}%
\end{pgfscope}%
\begin{pgfscope}%
\pgfpathrectangle{\pgfqpoint{2.548318in}{0.050000in}}{\pgfqpoint{2.081932in}{2.081932in}}%
\pgfusepath{clip}%
\pgfsetbuttcap%
\pgfsetroundjoin%
\definecolor{currentfill}{rgb}{0.162142,0.474838,0.558140}%
\pgfsetfillcolor{currentfill}%
\pgfsetlinewidth{0.000000pt}%
\definecolor{currentstroke}{rgb}{0.000000,0.000000,0.000000}%
\pgfsetstrokecolor{currentstroke}%
\pgfsetdash{}{0pt}%
\pgfpathmoveto{\pgfqpoint{3.749558in}{1.044331in}}%
\pgfpathlineto{\pgfqpoint{3.750143in}{1.035291in}}%
\pgfpathlineto{\pgfqpoint{3.751252in}{1.026018in}}%
\pgfpathlineto{\pgfqpoint{3.752880in}{1.016552in}}%
\pgfpathlineto{\pgfqpoint{3.755021in}{1.006931in}}%
\pgfpathlineto{\pgfqpoint{3.757668in}{0.997195in}}%
\pgfpathlineto{\pgfqpoint{3.747829in}{0.992985in}}%
\pgfpathlineto{\pgfqpoint{3.737485in}{0.989092in}}%
\pgfpathlineto{\pgfqpoint{3.726676in}{0.985529in}}%
\pgfpathlineto{\pgfqpoint{3.715446in}{0.982312in}}%
\pgfpathlineto{\pgfqpoint{3.703837in}{0.979450in}}%
\pgfpathlineto{\pgfqpoint{3.702212in}{0.989494in}}%
\pgfpathlineto{\pgfqpoint{3.700896in}{0.999360in}}%
\pgfpathlineto{\pgfqpoint{3.699896in}{1.009007in}}%
\pgfpathlineto{\pgfqpoint{3.699215in}{1.018396in}}%
\pgfpathlineto{\pgfqpoint{3.698856in}{1.027487in}}%
\pgfpathlineto{\pgfqpoint{3.709792in}{1.030204in}}%
\pgfpathlineto{\pgfqpoint{3.720371in}{1.033258in}}%
\pgfpathlineto{\pgfqpoint{3.730552in}{1.036640in}}%
\pgfpathlineto{\pgfqpoint{3.740294in}{1.040335in}}%
\pgfpathlineto{\pgfqpoint{3.749558in}{1.044331in}}%
\pgfpathclose%
\pgfusepath{fill}%
\end{pgfscope}%
\begin{pgfscope}%
\pgfpathrectangle{\pgfqpoint{2.548318in}{0.050000in}}{\pgfqpoint{2.081932in}{2.081932in}}%
\pgfusepath{clip}%
\pgfsetbuttcap%
\pgfsetroundjoin%
\definecolor{currentfill}{rgb}{0.606045,0.850733,0.236712}%
\pgfsetfillcolor{currentfill}%
\pgfsetlinewidth{0.000000pt}%
\definecolor{currentstroke}{rgb}{0.000000,0.000000,0.000000}%
\pgfsetstrokecolor{currentstroke}%
\pgfsetdash{}{0pt}%
\pgfpathmoveto{\pgfqpoint{3.882276in}{1.204366in}}%
\pgfpathlineto{\pgfqpoint{3.873031in}{1.199744in}}%
\pgfpathlineto{\pgfqpoint{3.864282in}{1.194653in}}%
\pgfpathlineto{\pgfqpoint{3.856067in}{1.189112in}}%
\pgfpathlineto{\pgfqpoint{3.848417in}{1.183144in}}%
\pgfpathlineto{\pgfqpoint{3.841365in}{1.176772in}}%
\pgfpathlineto{\pgfqpoint{3.838056in}{1.169660in}}%
\pgfpathlineto{\pgfqpoint{3.833871in}{1.162676in}}%
\pgfpathlineto{\pgfqpoint{3.828829in}{1.155846in}}%
\pgfpathlineto{\pgfqpoint{3.822953in}{1.149198in}}%
\pgfpathlineto{\pgfqpoint{3.816268in}{1.142757in}}%
\pgfpathlineto{\pgfqpoint{3.822496in}{1.148030in}}%
\pgfpathlineto{\pgfqpoint{3.829249in}{1.152808in}}%
\pgfpathlineto{\pgfqpoint{3.836500in}{1.157074in}}%
\pgfpathlineto{\pgfqpoint{3.844217in}{1.160810in}}%
\pgfpathlineto{\pgfqpoint{3.852369in}{1.164003in}}%
\pgfpathlineto{\pgfqpoint{3.860311in}{1.171641in}}%
\pgfpathlineto{\pgfqpoint{3.867303in}{1.179526in}}%
\pgfpathlineto{\pgfqpoint{3.873311in}{1.187630in}}%
\pgfpathlineto{\pgfqpoint{3.878310in}{1.195920in}}%
\pgfpathlineto{\pgfqpoint{3.882276in}{1.204366in}}%
\pgfpathclose%
\pgfusepath{fill}%
\end{pgfscope}%
\begin{pgfscope}%
\pgfpathrectangle{\pgfqpoint{2.548318in}{0.050000in}}{\pgfqpoint{2.081932in}{2.081932in}}%
\pgfusepath{clip}%
\pgfsetbuttcap%
\pgfsetroundjoin%
\definecolor{currentfill}{rgb}{0.278012,0.180367,0.486697}%
\pgfsetfillcolor{currentfill}%
\pgfsetlinewidth{0.000000pt}%
\definecolor{currentstroke}{rgb}{0.000000,0.000000,0.000000}%
\pgfsetstrokecolor{currentstroke}%
\pgfsetdash{}{0pt}%
\pgfpathmoveto{\pgfqpoint{3.778021in}{0.948229in}}%
\pgfpathlineto{\pgfqpoint{3.783398in}{0.938665in}}%
\pgfpathlineto{\pgfqpoint{3.789161in}{0.929272in}}%
\pgfpathlineto{\pgfqpoint{3.795287in}{0.920089in}}%
\pgfpathlineto{\pgfqpoint{3.801750in}{0.911154in}}%
\pgfpathlineto{\pgfqpoint{3.808522in}{0.902504in}}%
\pgfpathlineto{\pgfqpoint{3.795070in}{0.896853in}}%
\pgfpathlineto{\pgfqpoint{3.780940in}{0.891629in}}%
\pgfpathlineto{\pgfqpoint{3.766186in}{0.886852in}}%
\pgfpathlineto{\pgfqpoint{3.750866in}{0.882538in}}%
\pgfpathlineto{\pgfqpoint{3.735039in}{0.878704in}}%
\pgfpathlineto{\pgfqpoint{3.730888in}{0.888163in}}%
\pgfpathlineto{\pgfqpoint{3.726925in}{0.897870in}}%
\pgfpathlineto{\pgfqpoint{3.723168in}{0.907785in}}%
\pgfpathlineto{\pgfqpoint{3.719632in}{0.917866in}}%
\pgfpathlineto{\pgfqpoint{3.716333in}{0.928072in}}%
\pgfpathlineto{\pgfqpoint{3.729630in}{0.931321in}}%
\pgfpathlineto{\pgfqpoint{3.742496in}{0.934975in}}%
\pgfpathlineto{\pgfqpoint{3.754882in}{0.939022in}}%
\pgfpathlineto{\pgfqpoint{3.766740in}{0.943445in}}%
\pgfpathlineto{\pgfqpoint{3.778021in}{0.948229in}}%
\pgfpathclose%
\pgfusepath{fill}%
\end{pgfscope}%
\begin{pgfscope}%
\pgfpathrectangle{\pgfqpoint{2.548318in}{0.050000in}}{\pgfqpoint{2.081932in}{2.081932in}}%
\pgfusepath{clip}%
\pgfsetbuttcap%
\pgfsetroundjoin%
\definecolor{currentfill}{rgb}{0.268510,0.009605,0.335427}%
\pgfsetfillcolor{currentfill}%
\pgfsetlinewidth{0.000000pt}%
\definecolor{currentstroke}{rgb}{0.000000,0.000000,0.000000}%
\pgfsetstrokecolor{currentstroke}%
\pgfsetdash{}{0pt}%
\pgfpathmoveto{\pgfqpoint{3.326176in}{0.849411in}}%
\pgfpathlineto{\pgfqpoint{3.317375in}{0.846193in}}%
\pgfpathlineto{\pgfqpoint{3.308656in}{0.843563in}}%
\pgfpathlineto{\pgfqpoint{3.300056in}{0.841531in}}%
\pgfpathlineto{\pgfqpoint{3.291608in}{0.840104in}}%
\pgfpathlineto{\pgfqpoint{3.283347in}{0.839287in}}%
\pgfpathlineto{\pgfqpoint{3.264157in}{0.849739in}}%
\pgfpathlineto{\pgfqpoint{3.246295in}{0.860774in}}%
\pgfpathlineto{\pgfqpoint{3.229837in}{0.872352in}}%
\pgfpathlineto{\pgfqpoint{3.214851in}{0.884432in}}%
\pgfpathlineto{\pgfqpoint{3.201405in}{0.896972in}}%
\pgfpathlineto{\pgfqpoint{3.211784in}{0.896357in}}%
\pgfpathlineto{\pgfqpoint{3.222394in}{0.896325in}}%
\pgfpathlineto{\pgfqpoint{3.233190in}{0.896877in}}%
\pgfpathlineto{\pgfqpoint{3.244129in}{0.898012in}}%
\pgfpathlineto{\pgfqpoint{3.255167in}{0.899725in}}%
\pgfpathlineto{\pgfqpoint{3.266785in}{0.888798in}}%
\pgfpathlineto{\pgfqpoint{3.279754in}{0.878265in}}%
\pgfpathlineto{\pgfqpoint{3.294017in}{0.868166in}}%
\pgfpathlineto{\pgfqpoint{3.309512in}{0.858537in}}%
\pgfpathlineto{\pgfqpoint{3.326176in}{0.849411in}}%
\pgfpathclose%
\pgfusepath{fill}%
\end{pgfscope}%
\begin{pgfscope}%
\pgfpathrectangle{\pgfqpoint{2.548318in}{0.050000in}}{\pgfqpoint{2.081932in}{2.081932in}}%
\pgfusepath{clip}%
\pgfsetbuttcap%
\pgfsetroundjoin%
\definecolor{currentfill}{rgb}{0.162142,0.474838,0.558140}%
\pgfsetfillcolor{currentfill}%
\pgfsetlinewidth{0.000000pt}%
\definecolor{currentstroke}{rgb}{0.000000,0.000000,0.000000}%
\pgfsetstrokecolor{currentstroke}%
\pgfsetdash{}{0pt}%
\pgfpathmoveto{\pgfqpoint{3.522591in}{1.030942in}}%
\pgfpathlineto{\pgfqpoint{3.522172in}{1.021861in}}%
\pgfpathlineto{\pgfqpoint{3.521376in}{1.012496in}}%
\pgfpathlineto{\pgfqpoint{3.520208in}{1.002886in}}%
\pgfpathlineto{\pgfqpoint{3.518672in}{0.993070in}}%
\pgfpathlineto{\pgfqpoint{3.516774in}{0.983089in}}%
\pgfpathlineto{\pgfqpoint{3.505644in}{0.986395in}}%
\pgfpathlineto{\pgfqpoint{3.494947in}{0.990042in}}%
\pgfpathlineto{\pgfqpoint{3.484725in}{0.994016in}}%
\pgfpathlineto{\pgfqpoint{3.475018in}{0.998304in}}%
\pgfpathlineto{\pgfqpoint{3.465865in}{1.002888in}}%
\pgfpathlineto{\pgfqpoint{3.468731in}{1.012525in}}%
\pgfpathlineto{\pgfqpoint{3.471050in}{1.022067in}}%
\pgfpathlineto{\pgfqpoint{3.472813in}{1.031475in}}%
\pgfpathlineto{\pgfqpoint{3.474014in}{1.040710in}}%
\pgfpathlineto{\pgfqpoint{3.474647in}{1.049734in}}%
\pgfpathlineto{\pgfqpoint{3.483264in}{1.045384in}}%
\pgfpathlineto{\pgfqpoint{3.492405in}{1.041315in}}%
\pgfpathlineto{\pgfqpoint{3.502032in}{1.037542in}}%
\pgfpathlineto{\pgfqpoint{3.512107in}{1.034080in}}%
\pgfpathlineto{\pgfqpoint{3.522591in}{1.030942in}}%
\pgfpathclose%
\pgfusepath{fill}%
\end{pgfscope}%
\begin{pgfscope}%
\pgfpathrectangle{\pgfqpoint{2.548318in}{0.050000in}}{\pgfqpoint{2.081932in}{2.081932in}}%
\pgfusepath{clip}%
\pgfsetbuttcap%
\pgfsetroundjoin%
\definecolor{currentfill}{rgb}{0.150476,0.504369,0.557430}%
\pgfsetfillcolor{currentfill}%
\pgfsetlinewidth{0.000000pt}%
\definecolor{currentstroke}{rgb}{0.000000,0.000000,0.000000}%
\pgfsetstrokecolor{currentstroke}%
\pgfsetdash{}{0pt}%
\pgfpathmoveto{\pgfqpoint{3.097962in}{0.968503in}}%
\pgfpathlineto{\pgfqpoint{3.095786in}{0.976509in}}%
\pgfpathlineto{\pgfqpoint{3.094293in}{0.984763in}}%
\pgfpathlineto{\pgfqpoint{3.093490in}{0.993232in}}%
\pgfpathlineto{\pgfqpoint{3.093377in}{1.001882in}}%
\pgfpathlineto{\pgfqpoint{3.078167in}{1.018479in}}%
\pgfpathlineto{\pgfqpoint{3.065020in}{1.035564in}}%
\pgfpathlineto{\pgfqpoint{3.054004in}{1.053074in}}%
\pgfpathlineto{\pgfqpoint{3.045180in}{1.070944in}}%
\pgfpathlineto{\pgfqpoint{3.038600in}{1.089107in}}%
\pgfpathlineto{\pgfqpoint{3.038726in}{1.080320in}}%
\pgfpathlineto{\pgfqpoint{3.039625in}{1.071600in}}%
\pgfpathlineto{\pgfqpoint{3.041296in}{1.062983in}}%
\pgfpathlineto{\pgfqpoint{3.043732in}{1.054503in}}%
\pgfpathlineto{\pgfqpoint{3.050240in}{1.036597in}}%
\pgfpathlineto{\pgfqpoint{3.058972in}{1.018978in}}%
\pgfpathlineto{\pgfqpoint{3.069878in}{1.001715in}}%
\pgfpathlineto{\pgfqpoint{3.082897in}{0.984869in}}%
\pgfpathlineto{\pgfqpoint{3.097962in}{0.968503in}}%
\pgfpathclose%
\pgfusepath{fill}%
\end{pgfscope}%
\begin{pgfscope}%
\pgfpathrectangle{\pgfqpoint{2.548318in}{0.050000in}}{\pgfqpoint{2.081932in}{2.081932in}}%
\pgfusepath{clip}%
\pgfsetbuttcap%
\pgfsetroundjoin%
\definecolor{currentfill}{rgb}{0.227802,0.326594,0.546532}%
\pgfsetfillcolor{currentfill}%
\pgfsetlinewidth{0.000000pt}%
\definecolor{currentstroke}{rgb}{0.000000,0.000000,0.000000}%
\pgfsetstrokecolor{currentstroke}%
\pgfsetdash{}{0pt}%
\pgfpathmoveto{\pgfqpoint{3.757668in}{0.997195in}}%
\pgfpathlineto{\pgfqpoint{3.760808in}{0.987385in}}%
\pgfpathlineto{\pgfqpoint{3.764430in}{0.977543in}}%
\pgfpathlineto{\pgfqpoint{3.768517in}{0.967708in}}%
\pgfpathlineto{\pgfqpoint{3.773054in}{0.957924in}}%
\pgfpathlineto{\pgfqpoint{3.778021in}{0.948229in}}%
\pgfpathlineto{\pgfqpoint{3.766740in}{0.943445in}}%
\pgfpathlineto{\pgfqpoint{3.754882in}{0.939022in}}%
\pgfpathlineto{\pgfqpoint{3.742496in}{0.934975in}}%
\pgfpathlineto{\pgfqpoint{3.729630in}{0.931321in}}%
\pgfpathlineto{\pgfqpoint{3.716333in}{0.928072in}}%
\pgfpathlineto{\pgfqpoint{3.713285in}{0.938358in}}%
\pgfpathlineto{\pgfqpoint{3.710500in}{0.948683in}}%
\pgfpathlineto{\pgfqpoint{3.707990in}{0.959002in}}%
\pgfpathlineto{\pgfqpoint{3.705766in}{0.969272in}}%
\pgfpathlineto{\pgfqpoint{3.703837in}{0.979450in}}%
\pgfpathlineto{\pgfqpoint{3.715446in}{0.982312in}}%
\pgfpathlineto{\pgfqpoint{3.726676in}{0.985529in}}%
\pgfpathlineto{\pgfqpoint{3.737485in}{0.989092in}}%
\pgfpathlineto{\pgfqpoint{3.747829in}{0.992985in}}%
\pgfpathlineto{\pgfqpoint{3.757668in}{0.997195in}}%
\pgfpathclose%
\pgfusepath{fill}%
\end{pgfscope}%
\begin{pgfscope}%
\pgfpathrectangle{\pgfqpoint{2.548318in}{0.050000in}}{\pgfqpoint{2.081932in}{2.081932in}}%
\pgfusepath{clip}%
\pgfsetbuttcap%
\pgfsetroundjoin%
\definecolor{currentfill}{rgb}{0.278012,0.180367,0.486697}%
\pgfsetfillcolor{currentfill}%
\pgfsetlinewidth{0.000000pt}%
\definecolor{currentstroke}{rgb}{0.000000,0.000000,0.000000}%
\pgfsetstrokecolor{currentstroke}%
\pgfsetdash{}{0pt}%
\pgfpathmoveto{\pgfqpoint{3.502179in}{0.932204in}}%
\pgfpathlineto{\pgfqpoint{3.498326in}{0.922130in}}%
\pgfpathlineto{\pgfqpoint{3.494196in}{0.912189in}}%
\pgfpathlineto{\pgfqpoint{3.489806in}{0.902424in}}%
\pgfpathlineto{\pgfqpoint{3.485177in}{0.892874in}}%
\pgfpathlineto{\pgfqpoint{3.480327in}{0.883581in}}%
\pgfpathlineto{\pgfqpoint{3.465142in}{0.888013in}}%
\pgfpathlineto{\pgfqpoint{3.450537in}{0.892904in}}%
\pgfpathlineto{\pgfqpoint{3.436570in}{0.898237in}}%
\pgfpathlineto{\pgfqpoint{3.423296in}{0.903993in}}%
\pgfpathlineto{\pgfqpoint{3.410767in}{0.910150in}}%
\pgfpathlineto{\pgfqpoint{3.418106in}{0.918539in}}%
\pgfpathlineto{\pgfqpoint{3.425110in}{0.927225in}}%
\pgfpathlineto{\pgfqpoint{3.431747in}{0.936172in}}%
\pgfpathlineto{\pgfqpoint{3.437992in}{0.945343in}}%
\pgfpathlineto{\pgfqpoint{3.443817in}{0.954700in}}%
\pgfpathlineto{\pgfqpoint{3.454317in}{0.949489in}}%
\pgfpathlineto{\pgfqpoint{3.465449in}{0.944617in}}%
\pgfpathlineto{\pgfqpoint{3.477168in}{0.940101in}}%
\pgfpathlineto{\pgfqpoint{3.489428in}{0.935959in}}%
\pgfpathlineto{\pgfqpoint{3.502179in}{0.932204in}}%
\pgfpathclose%
\pgfusepath{fill}%
\end{pgfscope}%
\begin{pgfscope}%
\pgfpathrectangle{\pgfqpoint{2.548318in}{0.050000in}}{\pgfqpoint{2.081932in}{2.081932in}}%
\pgfusepath{clip}%
\pgfsetbuttcap%
\pgfsetroundjoin%
\definecolor{currentfill}{rgb}{0.267004,0.004874,0.329415}%
\pgfsetfillcolor{currentfill}%
\pgfsetlinewidth{0.000000pt}%
\definecolor{currentstroke}{rgb}{0.000000,0.000000,0.000000}%
\pgfsetstrokecolor{currentstroke}%
\pgfsetdash{}{0pt}%
\pgfpathmoveto{\pgfqpoint{3.912430in}{0.905077in}}%
\pgfpathlineto{\pgfqpoint{3.922819in}{0.900215in}}%
\pgfpathlineto{\pgfqpoint{3.933330in}{0.895870in}}%
\pgfpathlineto{\pgfqpoint{3.943920in}{0.892061in}}%
\pgfpathlineto{\pgfqpoint{3.954546in}{0.888801in}}%
\pgfpathlineto{\pgfqpoint{3.965163in}{0.886105in}}%
\pgfpathlineto{\pgfqpoint{3.951868in}{0.875678in}}%
\pgfpathlineto{\pgfqpoint{3.937293in}{0.865694in}}%
\pgfpathlineto{\pgfqpoint{3.921502in}{0.856188in}}%
\pgfpathlineto{\pgfqpoint{3.904558in}{0.847195in}}%
\pgfpathlineto{\pgfqpoint{3.886532in}{0.838746in}}%
\pgfpathlineto{\pgfqpoint{3.878377in}{0.842858in}}%
\pgfpathlineto{\pgfqpoint{3.870211in}{0.847532in}}%
\pgfpathlineto{\pgfqpoint{3.862068in}{0.852751in}}%
\pgfpathlineto{\pgfqpoint{3.853983in}{0.858492in}}%
\pgfpathlineto{\pgfqpoint{3.845989in}{0.864734in}}%
\pgfpathlineto{\pgfqpoint{3.861248in}{0.871938in}}%
\pgfpathlineto{\pgfqpoint{3.875579in}{0.879603in}}%
\pgfpathlineto{\pgfqpoint{3.888922in}{0.887701in}}%
\pgfpathlineto{\pgfqpoint{3.901223in}{0.896203in}}%
\pgfpathlineto{\pgfqpoint{3.912430in}{0.905077in}}%
\pgfpathclose%
\pgfusepath{fill}%
\end{pgfscope}%
\begin{pgfscope}%
\pgfpathrectangle{\pgfqpoint{2.548318in}{0.050000in}}{\pgfqpoint{2.081932in}{2.081932in}}%
\pgfusepath{clip}%
\pgfsetbuttcap%
\pgfsetroundjoin%
\definecolor{currentfill}{rgb}{0.120638,0.625828,0.533488}%
\pgfsetfillcolor{currentfill}%
\pgfsetlinewidth{0.000000pt}%
\definecolor{currentstroke}{rgb}{0.000000,0.000000,0.000000}%
\pgfsetstrokecolor{currentstroke}%
\pgfsetdash{}{0pt}%
\pgfpathmoveto{\pgfqpoint{3.793846in}{1.109760in}}%
\pgfpathlineto{\pgfqpoint{3.791241in}{1.102017in}}%
\pgfpathlineto{\pgfqpoint{3.789295in}{1.093962in}}%
\pgfpathlineto{\pgfqpoint{3.788018in}{1.085626in}}%
\pgfpathlineto{\pgfqpoint{3.787417in}{1.077045in}}%
\pgfpathlineto{\pgfqpoint{3.787493in}{1.068252in}}%
\pgfpathlineto{\pgfqpoint{3.781132in}{1.063002in}}%
\pgfpathlineto{\pgfqpoint{3.774128in}{1.057965in}}%
\pgfpathlineto{\pgfqpoint{3.766510in}{1.053163in}}%
\pgfpathlineto{\pgfqpoint{3.758309in}{1.048613in}}%
\pgfpathlineto{\pgfqpoint{3.749558in}{1.044331in}}%
\pgfpathlineto{\pgfqpoint{3.749499in}{1.053102in}}%
\pgfpathlineto{\pgfqpoint{3.749965in}{1.061567in}}%
\pgfpathlineto{\pgfqpoint{3.750953in}{1.069692in}}%
\pgfpathlineto{\pgfqpoint{3.752459in}{1.077443in}}%
\pgfpathlineto{\pgfqpoint{3.754475in}{1.084789in}}%
\pgfpathlineto{\pgfqpoint{3.763555in}{1.089258in}}%
\pgfpathlineto{\pgfqpoint{3.772065in}{1.094008in}}%
\pgfpathlineto{\pgfqpoint{3.779972in}{1.099021in}}%
\pgfpathlineto{\pgfqpoint{3.787242in}{1.104279in}}%
\pgfpathlineto{\pgfqpoint{3.793846in}{1.109760in}}%
\pgfpathclose%
\pgfusepath{fill}%
\end{pgfscope}%
\begin{pgfscope}%
\pgfpathrectangle{\pgfqpoint{2.548318in}{0.050000in}}{\pgfqpoint{2.081932in}{2.081932in}}%
\pgfusepath{clip}%
\pgfsetbuttcap%
\pgfsetroundjoin%
\definecolor{currentfill}{rgb}{0.606045,0.850733,0.236712}%
\pgfsetfillcolor{currentfill}%
\pgfsetlinewidth{0.000000pt}%
\definecolor{currentstroke}{rgb}{0.000000,0.000000,0.000000}%
\pgfsetstrokecolor{currentstroke}%
\pgfsetdash{}{0pt}%
\pgfpathmoveto{\pgfqpoint{3.372905in}{1.173606in}}%
\pgfpathlineto{\pgfqpoint{3.381409in}{1.170074in}}%
\pgfpathlineto{\pgfqpoint{3.389458in}{1.166017in}}%
\pgfpathlineto{\pgfqpoint{3.397019in}{1.161449in}}%
\pgfpathlineto{\pgfqpoint{3.404061in}{1.156389in}}%
\pgfpathlineto{\pgfqpoint{3.410555in}{1.150855in}}%
\pgfpathlineto{\pgfqpoint{3.404884in}{1.157551in}}%
\pgfpathlineto{\pgfqpoint{3.400052in}{1.164422in}}%
\pgfpathlineto{\pgfqpoint{3.396083in}{1.171441in}}%
\pgfpathlineto{\pgfqpoint{3.392994in}{1.178580in}}%
\pgfpathlineto{\pgfqpoint{3.390801in}{1.185812in}}%
\pgfpathlineto{\pgfqpoint{3.383648in}{1.192479in}}%
\pgfpathlineto{\pgfqpoint{3.375888in}{1.198766in}}%
\pgfpathlineto{\pgfqpoint{3.367553in}{1.204648in}}%
\pgfpathlineto{\pgfqpoint{3.358678in}{1.210102in}}%
\pgfpathlineto{\pgfqpoint{3.349297in}{1.215106in}}%
\pgfpathlineto{\pgfqpoint{3.351945in}{1.206514in}}%
\pgfpathlineto{\pgfqpoint{3.355651in}{1.198035in}}%
\pgfpathlineto{\pgfqpoint{3.360395in}{1.189703in}}%
\pgfpathlineto{\pgfqpoint{3.366155in}{1.181549in}}%
\pgfpathlineto{\pgfqpoint{3.372905in}{1.173606in}}%
\pgfpathclose%
\pgfusepath{fill}%
\end{pgfscope}%
\begin{pgfscope}%
\pgfpathrectangle{\pgfqpoint{2.548318in}{0.050000in}}{\pgfqpoint{2.081932in}{2.081932in}}%
\pgfusepath{clip}%
\pgfsetbuttcap%
\pgfsetroundjoin%
\definecolor{currentfill}{rgb}{0.227802,0.326594,0.546532}%
\pgfsetfillcolor{currentfill}%
\pgfsetlinewidth{0.000000pt}%
\definecolor{currentstroke}{rgb}{0.000000,0.000000,0.000000}%
\pgfsetstrokecolor{currentstroke}%
\pgfsetdash{}{0pt}%
\pgfpathmoveto{\pgfqpoint{3.516774in}{0.983089in}}%
\pgfpathlineto{\pgfqpoint{3.514522in}{0.972987in}}%
\pgfpathlineto{\pgfqpoint{3.511924in}{0.962804in}}%
\pgfpathlineto{\pgfqpoint{3.508993in}{0.952584in}}%
\pgfpathlineto{\pgfqpoint{3.505740in}{0.942370in}}%
\pgfpathlineto{\pgfqpoint{3.502179in}{0.932204in}}%
\pgfpathlineto{\pgfqpoint{3.489428in}{0.935959in}}%
\pgfpathlineto{\pgfqpoint{3.477168in}{0.940101in}}%
\pgfpathlineto{\pgfqpoint{3.465449in}{0.944617in}}%
\pgfpathlineto{\pgfqpoint{3.454317in}{0.949489in}}%
\pgfpathlineto{\pgfqpoint{3.443817in}{0.954700in}}%
\pgfpathlineto{\pgfqpoint{3.449198in}{0.964204in}}%
\pgfpathlineto{\pgfqpoint{3.454113in}{0.973815in}}%
\pgfpathlineto{\pgfqpoint{3.458541in}{0.983493in}}%
\pgfpathlineto{\pgfqpoint{3.462463in}{0.993197in}}%
\pgfpathlineto{\pgfqpoint{3.465865in}{1.002888in}}%
\pgfpathlineto{\pgfqpoint{3.475018in}{0.998304in}}%
\pgfpathlineto{\pgfqpoint{3.484725in}{0.994016in}}%
\pgfpathlineto{\pgfqpoint{3.494947in}{0.990042in}}%
\pgfpathlineto{\pgfqpoint{3.505644in}{0.986395in}}%
\pgfpathlineto{\pgfqpoint{3.516774in}{0.983089in}}%
\pgfpathclose%
\pgfusepath{fill}%
\end{pgfscope}%
\begin{pgfscope}%
\pgfpathrectangle{\pgfqpoint{2.548318in}{0.050000in}}{\pgfqpoint{2.081932in}{2.081932in}}%
\pgfusepath{clip}%
\pgfsetbuttcap%
\pgfsetroundjoin%
\definecolor{currentfill}{rgb}{0.120638,0.625828,0.533488}%
\pgfsetfillcolor{currentfill}%
\pgfsetlinewidth{0.000000pt}%
\definecolor{currentstroke}{rgb}{0.000000,0.000000,0.000000}%
\pgfsetstrokecolor{currentstroke}%
\pgfsetdash{}{0pt}%
\pgfpathmoveto{\pgfqpoint{3.469322in}{1.090429in}}%
\pgfpathlineto{\pgfqpoint{3.471505in}{1.082993in}}%
\pgfpathlineto{\pgfqpoint{3.473136in}{1.075173in}}%
\pgfpathlineto{\pgfqpoint{3.474207in}{1.067001in}}%
\pgfpathlineto{\pgfqpoint{3.474711in}{1.058510in}}%
\pgfpathlineto{\pgfqpoint{3.474647in}{1.049734in}}%
\pgfpathlineto{\pgfqpoint{3.466588in}{1.054350in}}%
\pgfpathlineto{\pgfqpoint{3.459119in}{1.059213in}}%
\pgfpathlineto{\pgfqpoint{3.452272in}{1.064305in}}%
\pgfpathlineto{\pgfqpoint{3.446075in}{1.069607in}}%
\pgfpathlineto{\pgfqpoint{3.440553in}{1.075097in}}%
\pgfpathlineto{\pgfqpoint{3.440633in}{1.083896in}}%
\pgfpathlineto{\pgfqpoint{3.440005in}{1.092511in}}%
\pgfpathlineto{\pgfqpoint{3.438674in}{1.100907in}}%
\pgfpathlineto{\pgfqpoint{3.436646in}{1.109050in}}%
\pgfpathlineto{\pgfqpoint{3.433930in}{1.116907in}}%
\pgfpathlineto{\pgfqpoint{3.439664in}{1.111175in}}%
\pgfpathlineto{\pgfqpoint{3.446099in}{1.105640in}}%
\pgfpathlineto{\pgfqpoint{3.453207in}{1.100324in}}%
\pgfpathlineto{\pgfqpoint{3.460958in}{1.095247in}}%
\pgfpathlineto{\pgfqpoint{3.469322in}{1.090429in}}%
\pgfpathclose%
\pgfusepath{fill}%
\end{pgfscope}%
\begin{pgfscope}%
\pgfpathrectangle{\pgfqpoint{2.548318in}{0.050000in}}{\pgfqpoint{2.081932in}{2.081932in}}%
\pgfusepath{clip}%
\pgfsetbuttcap%
\pgfsetroundjoin%
\definecolor{currentfill}{rgb}{0.267968,0.223549,0.512008}%
\pgfsetfillcolor{currentfill}%
\pgfsetlinewidth{0.000000pt}%
\definecolor{currentstroke}{rgb}{0.000000,0.000000,0.000000}%
\pgfsetstrokecolor{currentstroke}%
\pgfsetdash{}{0pt}%
\pgfpathmoveto{\pgfqpoint{4.120908in}{0.964705in}}%
\pgfpathlineto{\pgfqpoint{4.129867in}{0.969698in}}%
\pgfpathlineto{\pgfqpoint{4.138288in}{0.975147in}}%
\pgfpathlineto{\pgfqpoint{4.146139in}{0.981031in}}%
\pgfpathlineto{\pgfqpoint{4.153387in}{0.987324in}}%
\pgfpathlineto{\pgfqpoint{4.160006in}{0.994001in}}%
\pgfpathlineto{\pgfqpoint{4.151163in}{0.977272in}}%
\pgfpathlineto{\pgfqpoint{4.140244in}{0.960894in}}%
\pgfpathlineto{\pgfqpoint{4.127308in}{0.944926in}}%
\pgfpathlineto{\pgfqpoint{4.112421in}{0.929427in}}%
\pgfpathlineto{\pgfqpoint{4.095652in}{0.914453in}}%
\pgfpathlineto{\pgfqpoint{4.089890in}{0.908817in}}%
\pgfpathlineto{\pgfqpoint{4.083578in}{0.903650in}}%
\pgfpathlineto{\pgfqpoint{4.076740in}{0.898972in}}%
\pgfpathlineto{\pgfqpoint{4.069401in}{0.894804in}}%
\pgfpathlineto{\pgfqpoint{4.061592in}{0.891163in}}%
\pgfpathlineto{\pgfqpoint{4.077088in}{0.905017in}}%
\pgfpathlineto{\pgfqpoint{4.090830in}{0.919351in}}%
\pgfpathlineto{\pgfqpoint{4.102752in}{0.934113in}}%
\pgfpathlineto{\pgfqpoint{4.112796in}{0.949250in}}%
\pgfpathlineto{\pgfqpoint{4.120908in}{0.964705in}}%
\pgfpathclose%
\pgfusepath{fill}%
\end{pgfscope}%
\begin{pgfscope}%
\pgfpathrectangle{\pgfqpoint{2.548318in}{0.050000in}}{\pgfqpoint{2.081932in}{2.081932in}}%
\pgfusepath{clip}%
\pgfsetbuttcap%
\pgfsetroundjoin%
\definecolor{currentfill}{rgb}{0.296479,0.761561,0.424223}%
\pgfsetfillcolor{currentfill}%
\pgfsetlinewidth{0.000000pt}%
\definecolor{currentstroke}{rgb}{0.000000,0.000000,0.000000}%
\pgfsetstrokecolor{currentstroke}%
\pgfsetdash{}{0pt}%
\pgfpathmoveto{\pgfqpoint{3.841365in}{1.176772in}}%
\pgfpathlineto{\pgfqpoint{3.834939in}{1.170019in}}%
\pgfpathlineto{\pgfqpoint{3.829167in}{1.162914in}}%
\pgfpathlineto{\pgfqpoint{3.824071in}{1.155484in}}%
\pgfpathlineto{\pgfqpoint{3.819674in}{1.147758in}}%
\pgfpathlineto{\pgfqpoint{3.815995in}{1.139767in}}%
\pgfpathlineto{\pgfqpoint{3.813085in}{1.133496in}}%
\pgfpathlineto{\pgfqpoint{3.809397in}{1.127336in}}%
\pgfpathlineto{\pgfqpoint{3.804948in}{1.121311in}}%
\pgfpathlineto{\pgfqpoint{3.799757in}{1.115445in}}%
\pgfpathlineto{\pgfqpoint{3.793846in}{1.109760in}}%
\pgfpathlineto{\pgfqpoint{3.797100in}{1.117160in}}%
\pgfpathlineto{\pgfqpoint{3.800987in}{1.124187in}}%
\pgfpathlineto{\pgfqpoint{3.805491in}{1.130812in}}%
\pgfpathlineto{\pgfqpoint{3.810592in}{1.137010in}}%
\pgfpathlineto{\pgfqpoint{3.816268in}{1.142757in}}%
\pgfpathlineto{\pgfqpoint{3.822953in}{1.149198in}}%
\pgfpathlineto{\pgfqpoint{3.828829in}{1.155846in}}%
\pgfpathlineto{\pgfqpoint{3.833871in}{1.162676in}}%
\pgfpathlineto{\pgfqpoint{3.838056in}{1.169660in}}%
\pgfpathlineto{\pgfqpoint{3.841365in}{1.176772in}}%
\pgfpathclose%
\pgfusepath{fill}%
\end{pgfscope}%
\begin{pgfscope}%
\pgfpathrectangle{\pgfqpoint{2.548318in}{0.050000in}}{\pgfqpoint{2.081932in}{2.081932in}}%
\pgfusepath{clip}%
\pgfsetbuttcap%
\pgfsetroundjoin%
\definecolor{currentfill}{rgb}{0.267004,0.004874,0.329415}%
\pgfsetfillcolor{currentfill}%
\pgfsetlinewidth{0.000000pt}%
\definecolor{currentstroke}{rgb}{0.000000,0.000000,0.000000}%
\pgfsetstrokecolor{currentstroke}%
\pgfsetdash{}{0pt}%
\pgfpathmoveto{\pgfqpoint{3.370150in}{0.873827in}}%
\pgfpathlineto{\pgfqpoint{3.361481in}{0.867895in}}%
\pgfpathlineto{\pgfqpoint{3.352712in}{0.862466in}}%
\pgfpathlineto{\pgfqpoint{3.343881in}{0.857564in}}%
\pgfpathlineto{\pgfqpoint{3.335023in}{0.853207in}}%
\pgfpathlineto{\pgfqpoint{3.326176in}{0.849411in}}%
\pgfpathlineto{\pgfqpoint{3.309512in}{0.858537in}}%
\pgfpathlineto{\pgfqpoint{3.294017in}{0.868166in}}%
\pgfpathlineto{\pgfqpoint{3.279754in}{0.878265in}}%
\pgfpathlineto{\pgfqpoint{3.266785in}{0.888798in}}%
\pgfpathlineto{\pgfqpoint{3.255167in}{0.899725in}}%
\pgfpathlineto{\pgfqpoint{3.266257in}{0.902010in}}%
\pgfpathlineto{\pgfqpoint{3.277354in}{0.904858in}}%
\pgfpathlineto{\pgfqpoint{3.288413in}{0.908259in}}%
\pgfpathlineto{\pgfqpoint{3.299388in}{0.912199in}}%
\pgfpathlineto{\pgfqpoint{3.310233in}{0.916662in}}%
\pgfpathlineto{\pgfqpoint{3.320008in}{0.907368in}}%
\pgfpathlineto{\pgfqpoint{3.330935in}{0.898405in}}%
\pgfpathlineto{\pgfqpoint{3.342970in}{0.889807in}}%
\pgfpathlineto{\pgfqpoint{3.356059in}{0.881604in}}%
\pgfpathlineto{\pgfqpoint{3.370150in}{0.873827in}}%
\pgfpathclose%
\pgfusepath{fill}%
\end{pgfscope}%
\begin{pgfscope}%
\pgfpathrectangle{\pgfqpoint{2.548318in}{0.050000in}}{\pgfqpoint{2.081932in}{2.081932in}}%
\pgfusepath{clip}%
\pgfsetbuttcap%
\pgfsetroundjoin%
\definecolor{currentfill}{rgb}{0.206756,0.371758,0.553117}%
\pgfsetfillcolor{currentfill}%
\pgfsetlinewidth{0.000000pt}%
\definecolor{currentstroke}{rgb}{0.000000,0.000000,0.000000}%
\pgfsetstrokecolor{currentstroke}%
\pgfsetdash{}{0pt}%
\pgfpathmoveto{\pgfqpoint{3.118746in}{0.933285in}}%
\pgfpathlineto{\pgfqpoint{3.113316in}{0.939589in}}%
\pgfpathlineto{\pgfqpoint{3.108506in}{0.946291in}}%
\pgfpathlineto{\pgfqpoint{3.104334in}{0.953364in}}%
\pgfpathlineto{\pgfqpoint{3.100815in}{0.960778in}}%
\pgfpathlineto{\pgfqpoint{3.097962in}{0.968503in}}%
\pgfpathlineto{\pgfqpoint{3.082897in}{0.984869in}}%
\pgfpathlineto{\pgfqpoint{3.069878in}{1.001715in}}%
\pgfpathlineto{\pgfqpoint{3.058972in}{1.018978in}}%
\pgfpathlineto{\pgfqpoint{3.050240in}{1.036597in}}%
\pgfpathlineto{\pgfqpoint{3.043732in}{1.054503in}}%
\pgfpathlineto{\pgfqpoint{3.046926in}{1.046195in}}%
\pgfpathlineto{\pgfqpoint{3.050864in}{1.038092in}}%
\pgfpathlineto{\pgfqpoint{3.055532in}{1.030228in}}%
\pgfpathlineto{\pgfqpoint{3.060912in}{1.022634in}}%
\pgfpathlineto{\pgfqpoint{3.066984in}{1.015342in}}%
\pgfpathlineto{\pgfqpoint{3.073166in}{0.998264in}}%
\pgfpathlineto{\pgfqpoint{3.081486in}{0.981457in}}%
\pgfpathlineto{\pgfqpoint{3.091895in}{0.964985in}}%
\pgfpathlineto{\pgfqpoint{3.104336in}{0.948908in}}%
\pgfpathlineto{\pgfqpoint{3.118746in}{0.933285in}}%
\pgfpathclose%
\pgfusepath{fill}%
\end{pgfscope}%
\begin{pgfscope}%
\pgfpathrectangle{\pgfqpoint{2.548318in}{0.050000in}}{\pgfqpoint{2.081932in}{2.081932in}}%
\pgfusepath{clip}%
\pgfsetbuttcap%
\pgfsetroundjoin%
\definecolor{currentfill}{rgb}{0.278791,0.062145,0.386592}%
\pgfsetfillcolor{currentfill}%
\pgfsetlinewidth{0.000000pt}%
\definecolor{currentstroke}{rgb}{0.000000,0.000000,0.000000}%
\pgfsetstrokecolor{currentstroke}%
\pgfsetdash{}{0pt}%
\pgfpathmoveto{\pgfqpoint{3.863806in}{0.936399in}}%
\pgfpathlineto{\pgfqpoint{3.872951in}{0.929283in}}%
\pgfpathlineto{\pgfqpoint{3.882426in}{0.922566in}}%
\pgfpathlineto{\pgfqpoint{3.892192in}{0.916275in}}%
\pgfpathlineto{\pgfqpoint{3.902207in}{0.910438in}}%
\pgfpathlineto{\pgfqpoint{3.912430in}{0.905077in}}%
\pgfpathlineto{\pgfqpoint{3.901223in}{0.896203in}}%
\pgfpathlineto{\pgfqpoint{3.888922in}{0.887701in}}%
\pgfpathlineto{\pgfqpoint{3.875579in}{0.879603in}}%
\pgfpathlineto{\pgfqpoint{3.861248in}{0.871938in}}%
\pgfpathlineto{\pgfqpoint{3.845989in}{0.864734in}}%
\pgfpathlineto{\pgfqpoint{3.838118in}{0.871451in}}%
\pgfpathlineto{\pgfqpoint{3.830404in}{0.878617in}}%
\pgfpathlineto{\pgfqpoint{3.822879in}{0.886202in}}%
\pgfpathlineto{\pgfqpoint{3.815575in}{0.894175in}}%
\pgfpathlineto{\pgfqpoint{3.808522in}{0.902504in}}%
\pgfpathlineto{\pgfqpoint{3.821240in}{0.908562in}}%
\pgfpathlineto{\pgfqpoint{3.833175in}{0.915005in}}%
\pgfpathlineto{\pgfqpoint{3.844277in}{0.921809in}}%
\pgfpathlineto{\pgfqpoint{3.854502in}{0.928950in}}%
\pgfpathlineto{\pgfqpoint{3.863806in}{0.936399in}}%
\pgfpathclose%
\pgfusepath{fill}%
\end{pgfscope}%
\begin{pgfscope}%
\pgfpathrectangle{\pgfqpoint{2.548318in}{0.050000in}}{\pgfqpoint{2.081932in}{2.081932in}}%
\pgfusepath{clip}%
\pgfsetbuttcap%
\pgfsetroundjoin%
\definecolor{currentfill}{rgb}{0.162142,0.474838,0.558140}%
\pgfsetfillcolor{currentfill}%
\pgfsetlinewidth{0.000000pt}%
\definecolor{currentstroke}{rgb}{0.000000,0.000000,0.000000}%
\pgfsetstrokecolor{currentstroke}%
\pgfsetdash{}{0pt}%
\pgfpathmoveto{\pgfqpoint{3.787493in}{1.068252in}}%
\pgfpathlineto{\pgfqpoint{3.788249in}{1.059284in}}%
\pgfpathlineto{\pgfqpoint{3.789681in}{1.050178in}}%
\pgfpathlineto{\pgfqpoint{3.791785in}{1.040971in}}%
\pgfpathlineto{\pgfqpoint{3.794553in}{1.031700in}}%
\pgfpathlineto{\pgfqpoint{3.797973in}{1.022404in}}%
\pgfpathlineto{\pgfqpoint{3.791211in}{1.016870in}}%
\pgfpathlineto{\pgfqpoint{3.783767in}{1.011562in}}%
\pgfpathlineto{\pgfqpoint{3.775673in}{1.006501in}}%
\pgfpathlineto{\pgfqpoint{3.766961in}{1.001706in}}%
\pgfpathlineto{\pgfqpoint{3.757668in}{0.997195in}}%
\pgfpathlineto{\pgfqpoint{3.755021in}{1.006931in}}%
\pgfpathlineto{\pgfqpoint{3.752880in}{1.016552in}}%
\pgfpathlineto{\pgfqpoint{3.751252in}{1.026018in}}%
\pgfpathlineto{\pgfqpoint{3.750143in}{1.035291in}}%
\pgfpathlineto{\pgfqpoint{3.749558in}{1.044331in}}%
\pgfpathlineto{\pgfqpoint{3.758309in}{1.048613in}}%
\pgfpathlineto{\pgfqpoint{3.766510in}{1.053163in}}%
\pgfpathlineto{\pgfqpoint{3.774128in}{1.057965in}}%
\pgfpathlineto{\pgfqpoint{3.781132in}{1.063002in}}%
\pgfpathlineto{\pgfqpoint{3.787493in}{1.068252in}}%
\pgfpathclose%
\pgfusepath{fill}%
\end{pgfscope}%
\begin{pgfscope}%
\pgfpathrectangle{\pgfqpoint{2.548318in}{0.050000in}}{\pgfqpoint{2.081932in}{2.081932in}}%
\pgfusepath{clip}%
\pgfsetbuttcap%
\pgfsetroundjoin%
\definecolor{currentfill}{rgb}{0.993248,0.906157,0.143936}%
\pgfsetfillcolor{currentfill}%
\pgfsetlinewidth{0.000000pt}%
\definecolor{currentstroke}{rgb}{0.000000,0.000000,0.000000}%
\pgfsetstrokecolor{currentstroke}%
\pgfsetdash{}{0pt}%
\pgfpathmoveto{\pgfqpoint{4.060163in}{1.281723in}}%
\pgfpathlineto{\pgfqpoint{4.048272in}{1.283233in}}%
\pgfpathlineto{\pgfqpoint{4.036214in}{1.284191in}}%
\pgfpathlineto{\pgfqpoint{4.024038in}{1.284593in}}%
\pgfpathlineto{\pgfqpoint{4.011792in}{1.284438in}}%
\pgfpathlineto{\pgfqpoint{3.999523in}{1.283727in}}%
\pgfpathlineto{\pgfqpoint{4.001324in}{1.271276in}}%
\pgfpathlineto{\pgfqpoint{4.001573in}{1.258814in}}%
\pgfpathlineto{\pgfqpoint{4.000276in}{1.246394in}}%
\pgfpathlineto{\pgfqpoint{3.997449in}{1.234063in}}%
\pgfpathlineto{\pgfqpoint{3.993111in}{1.221871in}}%
\pgfpathlineto{\pgfqpoint{4.005058in}{1.220607in}}%
\pgfpathlineto{\pgfqpoint{4.016976in}{1.218798in}}%
\pgfpathlineto{\pgfqpoint{4.028819in}{1.216449in}}%
\pgfpathlineto{\pgfqpoint{4.040540in}{1.213571in}}%
\pgfpathlineto{\pgfqpoint{4.052092in}{1.210175in}}%
\pgfpathlineto{\pgfqpoint{4.057240in}{1.224261in}}%
\pgfpathlineto{\pgfqpoint{4.060642in}{1.238515in}}%
\pgfpathlineto{\pgfqpoint{4.062274in}{1.252882in}}%
\pgfpathlineto{\pgfqpoint{4.062118in}{1.267304in}}%
\pgfpathlineto{\pgfqpoint{4.060163in}{1.281723in}}%
\pgfpathclose%
\pgfusepath{fill}%
\end{pgfscope}%
\begin{pgfscope}%
\pgfpathrectangle{\pgfqpoint{2.548318in}{0.050000in}}{\pgfqpoint{2.081932in}{2.081932in}}%
\pgfusepath{clip}%
\pgfsetbuttcap%
\pgfsetroundjoin%
\definecolor{currentfill}{rgb}{0.993248,0.906157,0.143936}%
\pgfsetfillcolor{currentfill}%
\pgfsetlinewidth{0.000000pt}%
\definecolor{currentstroke}{rgb}{0.000000,0.000000,0.000000}%
\pgfsetstrokecolor{currentstroke}%
\pgfsetdash{}{0pt}%
\pgfpathmoveto{\pgfqpoint{3.999523in}{1.283727in}}%
\pgfpathlineto{\pgfqpoint{3.987281in}{1.282463in}}%
\pgfpathlineto{\pgfqpoint{3.975113in}{1.280651in}}%
\pgfpathlineto{\pgfqpoint{3.963067in}{1.278299in}}%
\pgfpathlineto{\pgfqpoint{3.951192in}{1.275415in}}%
\pgfpathlineto{\pgfqpoint{3.939532in}{1.272012in}}%
\pgfpathlineto{\pgfqpoint{3.941146in}{1.261531in}}%
\pgfpathlineto{\pgfqpoint{3.941450in}{1.251036in}}%
\pgfpathlineto{\pgfqpoint{3.940452in}{1.240569in}}%
\pgfpathlineto{\pgfqpoint{3.938160in}{1.230172in}}%
\pgfpathlineto{\pgfqpoint{3.934589in}{1.219886in}}%
\pgfpathlineto{\pgfqpoint{3.945976in}{1.221389in}}%
\pgfpathlineto{\pgfqpoint{3.957568in}{1.222342in}}%
\pgfpathlineto{\pgfqpoint{3.969319in}{1.222740in}}%
\pgfpathlineto{\pgfqpoint{3.981182in}{1.222583in}}%
\pgfpathlineto{\pgfqpoint{3.993111in}{1.221871in}}%
\pgfpathlineto{\pgfqpoint{3.997449in}{1.234063in}}%
\pgfpathlineto{\pgfqpoint{4.000276in}{1.246394in}}%
\pgfpathlineto{\pgfqpoint{4.001573in}{1.258814in}}%
\pgfpathlineto{\pgfqpoint{4.001324in}{1.271276in}}%
\pgfpathlineto{\pgfqpoint{3.999523in}{1.283727in}}%
\pgfpathclose%
\pgfusepath{fill}%
\end{pgfscope}%
\begin{pgfscope}%
\pgfpathrectangle{\pgfqpoint{2.548318in}{0.050000in}}{\pgfqpoint{2.081932in}{2.081932in}}%
\pgfusepath{clip}%
\pgfsetbuttcap%
\pgfsetroundjoin%
\definecolor{currentfill}{rgb}{0.296479,0.761561,0.424223}%
\pgfsetfillcolor{currentfill}%
\pgfsetlinewidth{0.000000pt}%
\definecolor{currentstroke}{rgb}{0.000000,0.000000,0.000000}%
\pgfsetstrokecolor{currentstroke}%
\pgfsetdash{}{0pt}%
\pgfpathmoveto{\pgfqpoint{3.410555in}{1.150855in}}%
\pgfpathlineto{\pgfqpoint{3.416474in}{1.144870in}}%
\pgfpathlineto{\pgfqpoint{3.421792in}{1.138456in}}%
\pgfpathlineto{\pgfqpoint{3.426487in}{1.131640in}}%
\pgfpathlineto{\pgfqpoint{3.430539in}{1.124447in}}%
\pgfpathlineto{\pgfqpoint{3.433930in}{1.116907in}}%
\pgfpathlineto{\pgfqpoint{3.428921in}{1.122816in}}%
\pgfpathlineto{\pgfqpoint{3.424659in}{1.128877in}}%
\pgfpathlineto{\pgfqpoint{3.421163in}{1.135067in}}%
\pgfpathlineto{\pgfqpoint{3.418448in}{1.141362in}}%
\pgfpathlineto{\pgfqpoint{3.416529in}{1.147737in}}%
\pgfpathlineto{\pgfqpoint{3.412798in}{1.155885in}}%
\pgfpathlineto{\pgfqpoint{3.408340in}{1.163798in}}%
\pgfpathlineto{\pgfqpoint{3.403172in}{1.171443in}}%
\pgfpathlineto{\pgfqpoint{3.397318in}{1.178791in}}%
\pgfpathlineto{\pgfqpoint{3.390801in}{1.185812in}}%
\pgfpathlineto{\pgfqpoint{3.392994in}{1.178580in}}%
\pgfpathlineto{\pgfqpoint{3.396083in}{1.171441in}}%
\pgfpathlineto{\pgfqpoint{3.400052in}{1.164422in}}%
\pgfpathlineto{\pgfqpoint{3.404884in}{1.157551in}}%
\pgfpathlineto{\pgfqpoint{3.410555in}{1.150855in}}%
\pgfpathclose%
\pgfusepath{fill}%
\end{pgfscope}%
\begin{pgfscope}%
\pgfpathrectangle{\pgfqpoint{2.548318in}{0.050000in}}{\pgfqpoint{2.081932in}{2.081932in}}%
\pgfusepath{clip}%
\pgfsetbuttcap%
\pgfsetroundjoin%
\definecolor{currentfill}{rgb}{0.876168,0.891125,0.095250}%
\pgfsetfillcolor{currentfill}%
\pgfsetlinewidth{0.000000pt}%
\definecolor{currentstroke}{rgb}{0.000000,0.000000,0.000000}%
\pgfsetstrokecolor{currentstroke}%
\pgfsetdash{}{0pt}%
\pgfpathmoveto{\pgfqpoint{4.115508in}{1.266160in}}%
\pgfpathlineto{\pgfqpoint{4.105137in}{1.270305in}}%
\pgfpathlineto{\pgfqpoint{4.094373in}{1.273947in}}%
\pgfpathlineto{\pgfqpoint{4.083260in}{1.277071in}}%
\pgfpathlineto{\pgfqpoint{4.071841in}{1.279667in}}%
\pgfpathlineto{\pgfqpoint{4.060163in}{1.281723in}}%
\pgfpathlineto{\pgfqpoint{4.062118in}{1.267304in}}%
\pgfpathlineto{\pgfqpoint{4.062274in}{1.252882in}}%
\pgfpathlineto{\pgfqpoint{4.060642in}{1.238515in}}%
\pgfpathlineto{\pgfqpoint{4.057240in}{1.224261in}}%
\pgfpathlineto{\pgfqpoint{4.052092in}{1.210175in}}%
\pgfpathlineto{\pgfqpoint{4.063431in}{1.206274in}}%
\pgfpathlineto{\pgfqpoint{4.074511in}{1.201884in}}%
\pgfpathlineto{\pgfqpoint{4.085290in}{1.197023in}}%
\pgfpathlineto{\pgfqpoint{4.095724in}{1.191708in}}%
\pgfpathlineto{\pgfqpoint{4.105772in}{1.185961in}}%
\pgfpathlineto{\pgfqpoint{4.111686in}{1.201733in}}%
\pgfpathlineto{\pgfqpoint{4.115643in}{1.217702in}}%
\pgfpathlineto{\pgfqpoint{4.117612in}{1.233806in}}%
\pgfpathlineto{\pgfqpoint{4.117572in}{1.249980in}}%
\pgfpathlineto{\pgfqpoint{4.115508in}{1.266160in}}%
\pgfpathclose%
\pgfusepath{fill}%
\end{pgfscope}%
\begin{pgfscope}%
\pgfpathrectangle{\pgfqpoint{2.548318in}{0.050000in}}{\pgfqpoint{2.081932in}{2.081932in}}%
\pgfusepath{clip}%
\pgfsetbuttcap%
\pgfsetroundjoin%
\definecolor{currentfill}{rgb}{0.162142,0.474838,0.558140}%
\pgfsetfillcolor{currentfill}%
\pgfsetlinewidth{0.000000pt}%
\definecolor{currentstroke}{rgb}{0.000000,0.000000,0.000000}%
\pgfsetstrokecolor{currentstroke}%
\pgfsetdash{}{0pt}%
\pgfpathmoveto{\pgfqpoint{3.474647in}{1.049734in}}%
\pgfpathlineto{\pgfqpoint{3.474014in}{1.040710in}}%
\pgfpathlineto{\pgfqpoint{3.472813in}{1.031475in}}%
\pgfpathlineto{\pgfqpoint{3.471050in}{1.022067in}}%
\pgfpathlineto{\pgfqpoint{3.468731in}{1.012525in}}%
\pgfpathlineto{\pgfqpoint{3.465865in}{1.002888in}}%
\pgfpathlineto{\pgfqpoint{3.457303in}{1.007752in}}%
\pgfpathlineto{\pgfqpoint{3.449367in}{1.012877in}}%
\pgfpathlineto{\pgfqpoint{3.442090in}{1.018244in}}%
\pgfpathlineto{\pgfqpoint{3.435502in}{1.023832in}}%
\pgfpathlineto{\pgfqpoint{3.429630in}{1.029620in}}%
\pgfpathlineto{\pgfqpoint{3.433195in}{1.038789in}}%
\pgfpathlineto{\pgfqpoint{3.436080in}{1.047959in}}%
\pgfpathlineto{\pgfqpoint{3.438273in}{1.057092in}}%
\pgfpathlineto{\pgfqpoint{3.439766in}{1.066150in}}%
\pgfpathlineto{\pgfqpoint{3.440553in}{1.075097in}}%
\pgfpathlineto{\pgfqpoint{3.446075in}{1.069607in}}%
\pgfpathlineto{\pgfqpoint{3.452272in}{1.064305in}}%
\pgfpathlineto{\pgfqpoint{3.459119in}{1.059213in}}%
\pgfpathlineto{\pgfqpoint{3.466588in}{1.054350in}}%
\pgfpathlineto{\pgfqpoint{3.474647in}{1.049734in}}%
\pgfpathclose%
\pgfusepath{fill}%
\end{pgfscope}%
\begin{pgfscope}%
\pgfpathrectangle{\pgfqpoint{2.548318in}{0.050000in}}{\pgfqpoint{2.081932in}{2.081932in}}%
\pgfusepath{clip}%
\pgfsetbuttcap%
\pgfsetroundjoin%
\definecolor{currentfill}{rgb}{0.278012,0.180367,0.486697}%
\pgfsetfillcolor{currentfill}%
\pgfsetlinewidth{0.000000pt}%
\definecolor{currentstroke}{rgb}{0.000000,0.000000,0.000000}%
\pgfsetstrokecolor{currentstroke}%
\pgfsetdash{}{0pt}%
\pgfpathmoveto{\pgfqpoint{3.824299in}{0.976894in}}%
\pgfpathlineto{\pgfqpoint{3.831259in}{0.968250in}}%
\pgfpathlineto{\pgfqpoint{3.838721in}{0.959845in}}%
\pgfpathlineto{\pgfqpoint{3.846655in}{0.951712in}}%
\pgfpathlineto{\pgfqpoint{3.855028in}{0.943886in}}%
\pgfpathlineto{\pgfqpoint{3.863806in}{0.936399in}}%
\pgfpathlineto{\pgfqpoint{3.854502in}{0.928950in}}%
\pgfpathlineto{\pgfqpoint{3.844277in}{0.921809in}}%
\pgfpathlineto{\pgfqpoint{3.833175in}{0.915005in}}%
\pgfpathlineto{\pgfqpoint{3.821240in}{0.908562in}}%
\pgfpathlineto{\pgfqpoint{3.808522in}{0.902504in}}%
\pgfpathlineto{\pgfqpoint{3.801750in}{0.911154in}}%
\pgfpathlineto{\pgfqpoint{3.795287in}{0.920089in}}%
\pgfpathlineto{\pgfqpoint{3.789161in}{0.929272in}}%
\pgfpathlineto{\pgfqpoint{3.783398in}{0.938665in}}%
\pgfpathlineto{\pgfqpoint{3.778021in}{0.948229in}}%
\pgfpathlineto{\pgfqpoint{3.788683in}{0.953356in}}%
\pgfpathlineto{\pgfqpoint{3.798680in}{0.958807in}}%
\pgfpathlineto{\pgfqpoint{3.807974in}{0.964562in}}%
\pgfpathlineto{\pgfqpoint{3.816526in}{0.970598in}}%
\pgfpathlineto{\pgfqpoint{3.824299in}{0.976894in}}%
\pgfpathclose%
\pgfusepath{fill}%
\end{pgfscope}%
\begin{pgfscope}%
\pgfpathrectangle{\pgfqpoint{2.548318in}{0.050000in}}{\pgfqpoint{2.081932in}{2.081932in}}%
\pgfusepath{clip}%
\pgfsetbuttcap%
\pgfsetroundjoin%
\definecolor{currentfill}{rgb}{0.227802,0.326594,0.546532}%
\pgfsetfillcolor{currentfill}%
\pgfsetlinewidth{0.000000pt}%
\definecolor{currentstroke}{rgb}{0.000000,0.000000,0.000000}%
\pgfsetstrokecolor{currentstroke}%
\pgfsetdash{}{0pt}%
\pgfpathmoveto{\pgfqpoint{3.797973in}{1.022404in}}%
\pgfpathlineto{\pgfqpoint{3.802033in}{1.013122in}}%
\pgfpathlineto{\pgfqpoint{3.806716in}{1.003892in}}%
\pgfpathlineto{\pgfqpoint{3.812003in}{0.994752in}}%
\pgfpathlineto{\pgfqpoint{3.817872in}{0.985740in}}%
\pgfpathlineto{\pgfqpoint{3.824299in}{0.976894in}}%
\pgfpathlineto{\pgfqpoint{3.816526in}{0.970598in}}%
\pgfpathlineto{\pgfqpoint{3.807974in}{0.964562in}}%
\pgfpathlineto{\pgfqpoint{3.798680in}{0.958807in}}%
\pgfpathlineto{\pgfqpoint{3.788683in}{0.953356in}}%
\pgfpathlineto{\pgfqpoint{3.778021in}{0.948229in}}%
\pgfpathlineto{\pgfqpoint{3.773054in}{0.957924in}}%
\pgfpathlineto{\pgfqpoint{3.768517in}{0.967708in}}%
\pgfpathlineto{\pgfqpoint{3.764430in}{0.977543in}}%
\pgfpathlineto{\pgfqpoint{3.760808in}{0.987385in}}%
\pgfpathlineto{\pgfqpoint{3.757668in}{0.997195in}}%
\pgfpathlineto{\pgfqpoint{3.766961in}{1.001706in}}%
\pgfpathlineto{\pgfqpoint{3.775673in}{1.006501in}}%
\pgfpathlineto{\pgfqpoint{3.783767in}{1.011562in}}%
\pgfpathlineto{\pgfqpoint{3.791211in}{1.016870in}}%
\pgfpathlineto{\pgfqpoint{3.797973in}{1.022404in}}%
\pgfpathclose%
\pgfusepath{fill}%
\end{pgfscope}%
\begin{pgfscope}%
\pgfpathrectangle{\pgfqpoint{2.548318in}{0.050000in}}{\pgfqpoint{2.081932in}{2.081932in}}%
\pgfusepath{clip}%
\pgfsetbuttcap%
\pgfsetroundjoin%
\definecolor{currentfill}{rgb}{0.855810,0.888601,0.097452}%
\pgfsetfillcolor{currentfill}%
\pgfsetlinewidth{0.000000pt}%
\definecolor{currentstroke}{rgb}{0.000000,0.000000,0.000000}%
\pgfsetstrokecolor{currentstroke}%
\pgfsetdash{}{0pt}%
\pgfpathmoveto{\pgfqpoint{3.939532in}{1.272012in}}%
\pgfpathlineto{\pgfqpoint{3.928136in}{1.268102in}}%
\pgfpathlineto{\pgfqpoint{3.917046in}{1.263701in}}%
\pgfpathlineto{\pgfqpoint{3.906308in}{1.258826in}}%
\pgfpathlineto{\pgfqpoint{3.895963in}{1.253496in}}%
\pgfpathlineto{\pgfqpoint{3.886053in}{1.247733in}}%
\pgfpathlineto{\pgfqpoint{3.887469in}{1.239022in}}%
\pgfpathlineto{\pgfqpoint{3.887794in}{1.230295in}}%
\pgfpathlineto{\pgfqpoint{3.887031in}{1.221587in}}%
\pgfpathlineto{\pgfqpoint{3.885188in}{1.212932in}}%
\pgfpathlineto{\pgfqpoint{3.882276in}{1.204366in}}%
\pgfpathlineto{\pgfqpoint{3.891980in}{1.208501in}}%
\pgfpathlineto{\pgfqpoint{3.902105in}{1.212133in}}%
\pgfpathlineto{\pgfqpoint{3.912610in}{1.215249in}}%
\pgfpathlineto{\pgfqpoint{3.923452in}{1.217836in}}%
\pgfpathlineto{\pgfqpoint{3.934589in}{1.219886in}}%
\pgfpathlineto{\pgfqpoint{3.938160in}{1.230172in}}%
\pgfpathlineto{\pgfqpoint{3.940452in}{1.240569in}}%
\pgfpathlineto{\pgfqpoint{3.941450in}{1.251036in}}%
\pgfpathlineto{\pgfqpoint{3.941146in}{1.261531in}}%
\pgfpathlineto{\pgfqpoint{3.939532in}{1.272012in}}%
\pgfpathclose%
\pgfusepath{fill}%
\end{pgfscope}%
\begin{pgfscope}%
\pgfpathrectangle{\pgfqpoint{2.548318in}{0.050000in}}{\pgfqpoint{2.081932in}{2.081932in}}%
\pgfusepath{clip}%
\pgfsetbuttcap%
\pgfsetroundjoin%
\definecolor{currentfill}{rgb}{0.278791,0.062145,0.386592}%
\pgfsetfillcolor{currentfill}%
\pgfsetlinewidth{0.000000pt}%
\definecolor{currentstroke}{rgb}{0.000000,0.000000,0.000000}%
\pgfsetstrokecolor{currentstroke}%
\pgfsetdash{}{0pt}%
\pgfpathmoveto{\pgfqpoint{3.410767in}{0.910150in}}%
\pgfpathlineto{\pgfqpoint{3.403123in}{0.902094in}}%
\pgfpathlineto{\pgfqpoint{3.395205in}{0.894403in}}%
\pgfpathlineto{\pgfqpoint{3.387047in}{0.887108in}}%
\pgfpathlineto{\pgfqpoint{3.378684in}{0.880241in}}%
\pgfpathlineto{\pgfqpoint{3.370150in}{0.873827in}}%
\pgfpathlineto{\pgfqpoint{3.356059in}{0.881604in}}%
\pgfpathlineto{\pgfqpoint{3.342970in}{0.889807in}}%
\pgfpathlineto{\pgfqpoint{3.330935in}{0.898405in}}%
\pgfpathlineto{\pgfqpoint{3.320008in}{0.907368in}}%
\pgfpathlineto{\pgfqpoint{3.310233in}{0.916662in}}%
\pgfpathlineto{\pgfqpoint{3.320905in}{0.921630in}}%
\pgfpathlineto{\pgfqpoint{3.331357in}{0.927083in}}%
\pgfpathlineto{\pgfqpoint{3.341548in}{0.932999in}}%
\pgfpathlineto{\pgfqpoint{3.351435in}{0.939353in}}%
\pgfpathlineto{\pgfqpoint{3.360977in}{0.946119in}}%
\pgfpathlineto{\pgfqpoint{3.369076in}{0.938322in}}%
\pgfpathlineto{\pgfqpoint{3.378146in}{0.930799in}}%
\pgfpathlineto{\pgfqpoint{3.388145in}{0.923578in}}%
\pgfpathlineto{\pgfqpoint{3.399034in}{0.916687in}}%
\pgfpathlineto{\pgfqpoint{3.410767in}{0.910150in}}%
\pgfpathclose%
\pgfusepath{fill}%
\end{pgfscope}%
\begin{pgfscope}%
\pgfpathrectangle{\pgfqpoint{2.548318in}{0.050000in}}{\pgfqpoint{2.081932in}{2.081932in}}%
\pgfusepath{clip}%
\pgfsetbuttcap%
\pgfsetroundjoin%
\definecolor{currentfill}{rgb}{0.282327,0.094955,0.417331}%
\pgfsetfillcolor{currentfill}%
\pgfsetlinewidth{0.000000pt}%
\definecolor{currentstroke}{rgb}{0.000000,0.000000,0.000000}%
\pgfsetstrokecolor{currentstroke}%
\pgfsetdash{}{0pt}%
\pgfpathmoveto{\pgfqpoint{4.069384in}{0.947254in}}%
\pgfpathlineto{\pgfqpoint{4.080451in}{0.949689in}}%
\pgfpathlineto{\pgfqpoint{4.091179in}{0.952666in}}%
\pgfpathlineto{\pgfqpoint{4.101525in}{0.956171in}}%
\pgfpathlineto{\pgfqpoint{4.111448in}{0.960189in}}%
\pgfpathlineto{\pgfqpoint{4.120908in}{0.964705in}}%
\pgfpathlineto{\pgfqpoint{4.112796in}{0.949250in}}%
\pgfpathlineto{\pgfqpoint{4.102752in}{0.934113in}}%
\pgfpathlineto{\pgfqpoint{4.090830in}{0.919351in}}%
\pgfpathlineto{\pgfqpoint{4.077088in}{0.905017in}}%
\pgfpathlineto{\pgfqpoint{4.061592in}{0.891163in}}%
\pgfpathlineto{\pgfqpoint{4.053342in}{0.888066in}}%
\pgfpathlineto{\pgfqpoint{4.044684in}{0.885525in}}%
\pgfpathlineto{\pgfqpoint{4.035652in}{0.883551in}}%
\pgfpathlineto{\pgfqpoint{4.026282in}{0.882155in}}%
\pgfpathlineto{\pgfqpoint{4.016613in}{0.881343in}}%
\pgfpathlineto{\pgfqpoint{4.030447in}{0.893771in}}%
\pgfpathlineto{\pgfqpoint{4.042696in}{0.906624in}}%
\pgfpathlineto{\pgfqpoint{4.053302in}{0.919856in}}%
\pgfpathlineto{\pgfqpoint{4.062213in}{0.933416in}}%
\pgfpathlineto{\pgfqpoint{4.069384in}{0.947254in}}%
\pgfpathclose%
\pgfusepath{fill}%
\end{pgfscope}%
\begin{pgfscope}%
\pgfpathrectangle{\pgfqpoint{2.548318in}{0.050000in}}{\pgfqpoint{2.081932in}{2.081932in}}%
\pgfusepath{clip}%
\pgfsetbuttcap%
\pgfsetroundjoin%
\definecolor{currentfill}{rgb}{0.636902,0.856542,0.216620}%
\pgfsetfillcolor{currentfill}%
\pgfsetlinewidth{0.000000pt}%
\definecolor{currentstroke}{rgb}{0.000000,0.000000,0.000000}%
\pgfsetstrokecolor{currentstroke}%
\pgfsetdash{}{0pt}%
\pgfpathmoveto{\pgfqpoint{4.160082in}{1.238507in}}%
\pgfpathlineto{\pgfqpoint{4.152263in}{1.244893in}}%
\pgfpathlineto{\pgfqpoint{4.143860in}{1.250873in}}%
\pgfpathlineto{\pgfqpoint{4.134909in}{1.256425in}}%
\pgfpathlineto{\pgfqpoint{4.125446in}{1.261527in}}%
\pgfpathlineto{\pgfqpoint{4.115508in}{1.266160in}}%
\pgfpathlineto{\pgfqpoint{4.117572in}{1.249980in}}%
\pgfpathlineto{\pgfqpoint{4.117612in}{1.233806in}}%
\pgfpathlineto{\pgfqpoint{4.115643in}{1.217702in}}%
\pgfpathlineto{\pgfqpoint{4.111686in}{1.201733in}}%
\pgfpathlineto{\pgfqpoint{4.105772in}{1.185961in}}%
\pgfpathlineto{\pgfqpoint{4.115396in}{1.179805in}}%
\pgfpathlineto{\pgfqpoint{4.124556in}{1.173263in}}%
\pgfpathlineto{\pgfqpoint{4.133216in}{1.166362in}}%
\pgfpathlineto{\pgfqpoint{4.141342in}{1.159127in}}%
\pgfpathlineto{\pgfqpoint{4.148901in}{1.151588in}}%
\pgfpathlineto{\pgfqpoint{4.155451in}{1.168667in}}%
\pgfpathlineto{\pgfqpoint{4.159875in}{1.185966in}}%
\pgfpathlineto{\pgfqpoint{4.162137in}{1.203419in}}%
\pgfpathlineto{\pgfqpoint{4.162213in}{1.220957in}}%
\pgfpathlineto{\pgfqpoint{4.160082in}{1.238507in}}%
\pgfpathclose%
\pgfusepath{fill}%
\end{pgfscope}%
\begin{pgfscope}%
\pgfpathrectangle{\pgfqpoint{2.548318in}{0.050000in}}{\pgfqpoint{2.081932in}{2.081932in}}%
\pgfusepath{clip}%
\pgfsetbuttcap%
\pgfsetroundjoin%
\definecolor{currentfill}{rgb}{0.120638,0.625828,0.533488}%
\pgfsetfillcolor{currentfill}%
\pgfsetlinewidth{0.000000pt}%
\definecolor{currentstroke}{rgb}{0.000000,0.000000,0.000000}%
\pgfsetstrokecolor{currentstroke}%
\pgfsetdash{}{0pt}%
\pgfpathmoveto{\pgfqpoint{3.815995in}{1.139767in}}%
\pgfpathlineto{\pgfqpoint{3.813049in}{1.131543in}}%
\pgfpathlineto{\pgfqpoint{3.810849in}{1.123119in}}%
\pgfpathlineto{\pgfqpoint{3.809406in}{1.114529in}}%
\pgfpathlineto{\pgfqpoint{3.808725in}{1.105807in}}%
\pgfpathlineto{\pgfqpoint{3.808812in}{1.096988in}}%
\pgfpathlineto{\pgfqpoint{3.806014in}{1.090983in}}%
\pgfpathlineto{\pgfqpoint{3.802465in}{1.085085in}}%
\pgfpathlineto{\pgfqpoint{3.798183in}{1.079315in}}%
\pgfpathlineto{\pgfqpoint{3.793185in}{1.073697in}}%
\pgfpathlineto{\pgfqpoint{3.787493in}{1.068252in}}%
\pgfpathlineto{\pgfqpoint{3.787417in}{1.077045in}}%
\pgfpathlineto{\pgfqpoint{3.788018in}{1.085626in}}%
\pgfpathlineto{\pgfqpoint{3.789295in}{1.093962in}}%
\pgfpathlineto{\pgfqpoint{3.791241in}{1.102017in}}%
\pgfpathlineto{\pgfqpoint{3.793846in}{1.109760in}}%
\pgfpathlineto{\pgfqpoint{3.799757in}{1.115445in}}%
\pgfpathlineto{\pgfqpoint{3.804948in}{1.121311in}}%
\pgfpathlineto{\pgfqpoint{3.809397in}{1.127336in}}%
\pgfpathlineto{\pgfqpoint{3.813085in}{1.133496in}}%
\pgfpathlineto{\pgfqpoint{3.815995in}{1.139767in}}%
\pgfpathclose%
\pgfusepath{fill}%
\end{pgfscope}%
\begin{pgfscope}%
\pgfpathrectangle{\pgfqpoint{2.548318in}{0.050000in}}{\pgfqpoint{2.081932in}{2.081932in}}%
\pgfusepath{clip}%
\pgfsetbuttcap%
\pgfsetroundjoin%
\definecolor{currentfill}{rgb}{0.267968,0.223549,0.512008}%
\pgfsetfillcolor{currentfill}%
\pgfsetlinewidth{0.000000pt}%
\definecolor{currentstroke}{rgb}{0.000000,0.000000,0.000000}%
\pgfsetstrokecolor{currentstroke}%
\pgfsetdash{}{0pt}%
\pgfpathmoveto{\pgfqpoint{3.154376in}{0.908585in}}%
\pgfpathlineto{\pgfqpoint{3.146208in}{0.912544in}}%
\pgfpathlineto{\pgfqpoint{3.138532in}{0.917013in}}%
\pgfpathlineto{\pgfqpoint{3.131378in}{0.921974in}}%
\pgfpathlineto{\pgfqpoint{3.124774in}{0.927405in}}%
\pgfpathlineto{\pgfqpoint{3.118746in}{0.933285in}}%
\pgfpathlineto{\pgfqpoint{3.104336in}{0.948908in}}%
\pgfpathlineto{\pgfqpoint{3.091895in}{0.964985in}}%
\pgfpathlineto{\pgfqpoint{3.081486in}{0.981457in}}%
\pgfpathlineto{\pgfqpoint{3.073166in}{0.998264in}}%
\pgfpathlineto{\pgfqpoint{3.066984in}{1.015342in}}%
\pgfpathlineto{\pgfqpoint{3.073723in}{1.008382in}}%
\pgfpathlineto{\pgfqpoint{3.081104in}{1.001783in}}%
\pgfpathlineto{\pgfqpoint{3.089096in}{0.995571in}}%
\pgfpathlineto{\pgfqpoint{3.097668in}{0.989773in}}%
\pgfpathlineto{\pgfqpoint{3.106786in}{0.984412in}}%
\pgfpathlineto{\pgfqpoint{3.112424in}{0.968642in}}%
\pgfpathlineto{\pgfqpoint{3.120049in}{0.953117in}}%
\pgfpathlineto{\pgfqpoint{3.129619in}{0.937894in}}%
\pgfpathlineto{\pgfqpoint{3.141080in}{0.923032in}}%
\pgfpathlineto{\pgfqpoint{3.154376in}{0.908585in}}%
\pgfpathclose%
\pgfusepath{fill}%
\end{pgfscope}%
\begin{pgfscope}%
\pgfpathrectangle{\pgfqpoint{2.548318in}{0.050000in}}{\pgfqpoint{2.081932in}{2.081932in}}%
\pgfusepath{clip}%
\pgfsetbuttcap%
\pgfsetroundjoin%
\definecolor{currentfill}{rgb}{0.227802,0.326594,0.546532}%
\pgfsetfillcolor{currentfill}%
\pgfsetlinewidth{0.000000pt}%
\definecolor{currentstroke}{rgb}{0.000000,0.000000,0.000000}%
\pgfsetstrokecolor{currentstroke}%
\pgfsetdash{}{0pt}%
\pgfpathmoveto{\pgfqpoint{3.465865in}{1.002888in}}%
\pgfpathlineto{\pgfqpoint{3.462463in}{0.993197in}}%
\pgfpathlineto{\pgfqpoint{3.458541in}{0.983493in}}%
\pgfpathlineto{\pgfqpoint{3.454113in}{0.973815in}}%
\pgfpathlineto{\pgfqpoint{3.449198in}{0.964204in}}%
\pgfpathlineto{\pgfqpoint{3.443817in}{0.954700in}}%
\pgfpathlineto{\pgfqpoint{3.433990in}{0.960229in}}%
\pgfpathlineto{\pgfqpoint{3.424877in}{0.966057in}}%
\pgfpathlineto{\pgfqpoint{3.416516in}{0.972161in}}%
\pgfpathlineto{\pgfqpoint{3.408940in}{0.978519in}}%
\pgfpathlineto{\pgfqpoint{3.402183in}{0.985105in}}%
\pgfpathlineto{\pgfqpoint{3.408885in}{0.993707in}}%
\pgfpathlineto{\pgfqpoint{3.415004in}{1.002496in}}%
\pgfpathlineto{\pgfqpoint{3.420516in}{1.011436in}}%
\pgfpathlineto{\pgfqpoint{3.425397in}{1.020490in}}%
\pgfpathlineto{\pgfqpoint{3.429630in}{1.029620in}}%
\pgfpathlineto{\pgfqpoint{3.435502in}{1.023832in}}%
\pgfpathlineto{\pgfqpoint{3.442090in}{1.018244in}}%
\pgfpathlineto{\pgfqpoint{3.449367in}{1.012877in}}%
\pgfpathlineto{\pgfqpoint{3.457303in}{1.007752in}}%
\pgfpathlineto{\pgfqpoint{3.465865in}{1.002888in}}%
\pgfpathclose%
\pgfusepath{fill}%
\end{pgfscope}%
\begin{pgfscope}%
\pgfpathrectangle{\pgfqpoint{2.548318in}{0.050000in}}{\pgfqpoint{2.081932in}{2.081932in}}%
\pgfusepath{clip}%
\pgfsetbuttcap%
\pgfsetroundjoin%
\definecolor{currentfill}{rgb}{0.278012,0.180367,0.486697}%
\pgfsetfillcolor{currentfill}%
\pgfsetlinewidth{0.000000pt}%
\definecolor{currentstroke}{rgb}{0.000000,0.000000,0.000000}%
\pgfsetstrokecolor{currentstroke}%
\pgfsetdash{}{0pt}%
\pgfpathmoveto{\pgfqpoint{3.443817in}{0.954700in}}%
\pgfpathlineto{\pgfqpoint{3.437992in}{0.945343in}}%
\pgfpathlineto{\pgfqpoint{3.431747in}{0.936172in}}%
\pgfpathlineto{\pgfqpoint{3.425110in}{0.927225in}}%
\pgfpathlineto{\pgfqpoint{3.418106in}{0.918539in}}%
\pgfpathlineto{\pgfqpoint{3.410767in}{0.910150in}}%
\pgfpathlineto{\pgfqpoint{3.399034in}{0.916687in}}%
\pgfpathlineto{\pgfqpoint{3.388145in}{0.923578in}}%
\pgfpathlineto{\pgfqpoint{3.378146in}{0.930799in}}%
\pgfpathlineto{\pgfqpoint{3.369076in}{0.938322in}}%
\pgfpathlineto{\pgfqpoint{3.360977in}{0.946119in}}%
\pgfpathlineto{\pgfqpoint{3.370134in}{0.953270in}}%
\pgfpathlineto{\pgfqpoint{3.378868in}{0.960776in}}%
\pgfpathlineto{\pgfqpoint{3.387143in}{0.968606in}}%
\pgfpathlineto{\pgfqpoint{3.394926in}{0.976727in}}%
\pgfpathlineto{\pgfqpoint{3.402183in}{0.985105in}}%
\pgfpathlineto{\pgfqpoint{3.408940in}{0.978519in}}%
\pgfpathlineto{\pgfqpoint{3.416516in}{0.972161in}}%
\pgfpathlineto{\pgfqpoint{3.424877in}{0.966057in}}%
\pgfpathlineto{\pgfqpoint{3.433990in}{0.960229in}}%
\pgfpathlineto{\pgfqpoint{3.443817in}{0.954700in}}%
\pgfpathclose%
\pgfusepath{fill}%
\end{pgfscope}%
\begin{pgfscope}%
\pgfpathrectangle{\pgfqpoint{2.548318in}{0.050000in}}{\pgfqpoint{2.081932in}{2.081932in}}%
\pgfusepath{clip}%
\pgfsetbuttcap%
\pgfsetroundjoin%
\definecolor{currentfill}{rgb}{0.993248,0.906157,0.143936}%
\pgfsetfillcolor{currentfill}%
\pgfsetlinewidth{0.000000pt}%
\definecolor{currentstroke}{rgb}{0.000000,0.000000,0.000000}%
\pgfsetstrokecolor{currentstroke}%
\pgfsetdash{}{0pt}%
\pgfpathmoveto{\pgfqpoint{3.236748in}{1.237159in}}%
\pgfpathlineto{\pgfqpoint{3.248868in}{1.237386in}}%
\pgfpathlineto{\pgfqpoint{3.260920in}{1.237060in}}%
\pgfpathlineto{\pgfqpoint{3.272857in}{1.236181in}}%
\pgfpathlineto{\pgfqpoint{3.284630in}{1.234754in}}%
\pgfpathlineto{\pgfqpoint{3.296193in}{1.232783in}}%
\pgfpathlineto{\pgfqpoint{3.294221in}{1.243202in}}%
\pgfpathlineto{\pgfqpoint{3.293545in}{1.253680in}}%
\pgfpathlineto{\pgfqpoint{3.294174in}{1.264175in}}%
\pgfpathlineto{\pgfqpoint{3.296113in}{1.274646in}}%
\pgfpathlineto{\pgfqpoint{3.299358in}{1.285048in}}%
\pgfpathlineto{\pgfqpoint{3.287815in}{1.288931in}}%
\pgfpathlineto{\pgfqpoint{3.276055in}{1.292303in}}%
\pgfpathlineto{\pgfqpoint{3.264126in}{1.295149in}}%
\pgfpathlineto{\pgfqpoint{3.252074in}{1.297460in}}%
\pgfpathlineto{\pgfqpoint{3.239946in}{1.299224in}}%
\pgfpathlineto{\pgfqpoint{3.236206in}{1.286857in}}%
\pgfpathlineto{\pgfqpoint{3.234019in}{1.274416in}}%
\pgfpathlineto{\pgfqpoint{3.233385in}{1.261953in}}%
\pgfpathlineto{\pgfqpoint{3.234299in}{1.249517in}}%
\pgfpathlineto{\pgfqpoint{3.236748in}{1.237159in}}%
\pgfpathclose%
\pgfusepath{fill}%
\end{pgfscope}%
\begin{pgfscope}%
\pgfpathrectangle{\pgfqpoint{2.548318in}{0.050000in}}{\pgfqpoint{2.081932in}{2.081932in}}%
\pgfusepath{clip}%
\pgfsetbuttcap%
\pgfsetroundjoin%
\definecolor{currentfill}{rgb}{0.606045,0.850733,0.236712}%
\pgfsetfillcolor{currentfill}%
\pgfsetlinewidth{0.000000pt}%
\definecolor{currentstroke}{rgb}{0.000000,0.000000,0.000000}%
\pgfsetstrokecolor{currentstroke}%
\pgfsetdash{}{0pt}%
\pgfpathmoveto{\pgfqpoint{3.886053in}{1.247733in}}%
\pgfpathlineto{\pgfqpoint{3.876616in}{1.241557in}}%
\pgfpathlineto{\pgfqpoint{3.867691in}{1.234995in}}%
\pgfpathlineto{\pgfqpoint{3.859313in}{1.228070in}}%
\pgfpathlineto{\pgfqpoint{3.851514in}{1.220812in}}%
\pgfpathlineto{\pgfqpoint{3.844328in}{1.213247in}}%
\pgfpathlineto{\pgfqpoint{3.845570in}{1.205926in}}%
\pgfpathlineto{\pgfqpoint{3.845892in}{1.198589in}}%
\pgfpathlineto{\pgfqpoint{3.845294in}{1.191265in}}%
\pgfpathlineto{\pgfqpoint{3.843782in}{1.183983in}}%
\pgfpathlineto{\pgfqpoint{3.841365in}{1.176772in}}%
\pgfpathlineto{\pgfqpoint{3.848417in}{1.183144in}}%
\pgfpathlineto{\pgfqpoint{3.856067in}{1.189112in}}%
\pgfpathlineto{\pgfqpoint{3.864282in}{1.194653in}}%
\pgfpathlineto{\pgfqpoint{3.873031in}{1.199744in}}%
\pgfpathlineto{\pgfqpoint{3.882276in}{1.204366in}}%
\pgfpathlineto{\pgfqpoint{3.885188in}{1.212932in}}%
\pgfpathlineto{\pgfqpoint{3.887031in}{1.221587in}}%
\pgfpathlineto{\pgfqpoint{3.887794in}{1.230295in}}%
\pgfpathlineto{\pgfqpoint{3.887469in}{1.239022in}}%
\pgfpathlineto{\pgfqpoint{3.886053in}{1.247733in}}%
\pgfpathclose%
\pgfusepath{fill}%
\end{pgfscope}%
\begin{pgfscope}%
\pgfpathrectangle{\pgfqpoint{2.548318in}{0.050000in}}{\pgfqpoint{2.081932in}{2.081932in}}%
\pgfusepath{clip}%
\pgfsetbuttcap%
\pgfsetroundjoin%
\definecolor{currentfill}{rgb}{0.993248,0.906157,0.143936}%
\pgfsetfillcolor{currentfill}%
\pgfsetlinewidth{0.000000pt}%
\definecolor{currentstroke}{rgb}{0.000000,0.000000,0.000000}%
\pgfsetstrokecolor{currentstroke}%
\pgfsetdash{}{0pt}%
\pgfpathmoveto{\pgfqpoint{3.176792in}{1.227839in}}%
\pgfpathlineto{\pgfqpoint{3.188538in}{1.230773in}}%
\pgfpathlineto{\pgfqpoint{3.200455in}{1.233181in}}%
\pgfpathlineto{\pgfqpoint{3.212493in}{1.235052in}}%
\pgfpathlineto{\pgfqpoint{3.224607in}{1.236380in}}%
\pgfpathlineto{\pgfqpoint{3.236748in}{1.237159in}}%
\pgfpathlineto{\pgfqpoint{3.234299in}{1.249517in}}%
\pgfpathlineto{\pgfqpoint{3.233385in}{1.261953in}}%
\pgfpathlineto{\pgfqpoint{3.234019in}{1.274416in}}%
\pgfpathlineto{\pgfqpoint{3.236206in}{1.286857in}}%
\pgfpathlineto{\pgfqpoint{3.239946in}{1.299224in}}%
\pgfpathlineto{\pgfqpoint{3.227791in}{1.300436in}}%
\pgfpathlineto{\pgfqpoint{3.215655in}{1.301089in}}%
\pgfpathlineto{\pgfqpoint{3.203588in}{1.301181in}}%
\pgfpathlineto{\pgfqpoint{3.191636in}{1.300712in}}%
\pgfpathlineto{\pgfqpoint{3.179847in}{1.299682in}}%
\pgfpathlineto{\pgfqpoint{3.175642in}{1.285349in}}%
\pgfpathlineto{\pgfqpoint{3.173239in}{1.270940in}}%
\pgfpathlineto{\pgfqpoint{3.172637in}{1.256513in}}%
\pgfpathlineto{\pgfqpoint{3.173827in}{1.242127in}}%
\pgfpathlineto{\pgfqpoint{3.176792in}{1.227839in}}%
\pgfpathclose%
\pgfusepath{fill}%
\end{pgfscope}%
\begin{pgfscope}%
\pgfpathrectangle{\pgfqpoint{2.548318in}{0.050000in}}{\pgfqpoint{2.081932in}{2.081932in}}%
\pgfusepath{clip}%
\pgfsetbuttcap%
\pgfsetroundjoin%
\definecolor{currentfill}{rgb}{0.855810,0.888601,0.097452}%
\pgfsetfillcolor{currentfill}%
\pgfsetlinewidth{0.000000pt}%
\definecolor{currentstroke}{rgb}{0.000000,0.000000,0.000000}%
\pgfsetstrokecolor{currentstroke}%
\pgfsetdash{}{0pt}%
\pgfpathmoveto{\pgfqpoint{3.296193in}{1.232783in}}%
\pgfpathlineto{\pgfqpoint{3.307501in}{1.230276in}}%
\pgfpathlineto{\pgfqpoint{3.318508in}{1.227242in}}%
\pgfpathlineto{\pgfqpoint{3.329172in}{1.223693in}}%
\pgfpathlineto{\pgfqpoint{3.339448in}{1.219643in}}%
\pgfpathlineto{\pgfqpoint{3.349297in}{1.215106in}}%
\pgfpathlineto{\pgfqpoint{3.347721in}{1.223778in}}%
\pgfpathlineto{\pgfqpoint{3.347227in}{1.232494in}}%
\pgfpathlineto{\pgfqpoint{3.347823in}{1.241220in}}%
\pgfpathlineto{\pgfqpoint{3.349510in}{1.249921in}}%
\pgfpathlineto{\pgfqpoint{3.352285in}{1.258560in}}%
\pgfpathlineto{\pgfqpoint{3.342480in}{1.264734in}}%
\pgfpathlineto{\pgfqpoint{3.332243in}{1.270491in}}%
\pgfpathlineto{\pgfqpoint{3.321616in}{1.275810in}}%
\pgfpathlineto{\pgfqpoint{3.310640in}{1.280668in}}%
\pgfpathlineto{\pgfqpoint{3.299358in}{1.285048in}}%
\pgfpathlineto{\pgfqpoint{3.296113in}{1.274646in}}%
\pgfpathlineto{\pgfqpoint{3.294174in}{1.264175in}}%
\pgfpathlineto{\pgfqpoint{3.293545in}{1.253680in}}%
\pgfpathlineto{\pgfqpoint{3.294221in}{1.243202in}}%
\pgfpathlineto{\pgfqpoint{3.296193in}{1.232783in}}%
\pgfpathclose%
\pgfusepath{fill}%
\end{pgfscope}%
\begin{pgfscope}%
\pgfpathrectangle{\pgfqpoint{2.548318in}{0.050000in}}{\pgfqpoint{2.081932in}{2.081932in}}%
\pgfusepath{clip}%
\pgfsetbuttcap%
\pgfsetroundjoin%
\definecolor{currentfill}{rgb}{0.120638,0.625828,0.533488}%
\pgfsetfillcolor{currentfill}%
\pgfsetlinewidth{0.000000pt}%
\definecolor{currentstroke}{rgb}{0.000000,0.000000,0.000000}%
\pgfsetstrokecolor{currentstroke}%
\pgfsetdash{}{0pt}%
\pgfpathmoveto{\pgfqpoint{3.433930in}{1.116907in}}%
\pgfpathlineto{\pgfqpoint{3.436646in}{1.109050in}}%
\pgfpathlineto{\pgfqpoint{3.438674in}{1.100907in}}%
\pgfpathlineto{\pgfqpoint{3.440005in}{1.092511in}}%
\pgfpathlineto{\pgfqpoint{3.440633in}{1.083896in}}%
\pgfpathlineto{\pgfqpoint{3.440553in}{1.075097in}}%
\pgfpathlineto{\pgfqpoint{3.435731in}{1.080756in}}%
\pgfpathlineto{\pgfqpoint{3.431629in}{1.086560in}}%
\pgfpathlineto{\pgfqpoint{3.428265in}{1.092487in}}%
\pgfpathlineto{\pgfqpoint{3.425656in}{1.098515in}}%
\pgfpathlineto{\pgfqpoint{3.423812in}{1.104618in}}%
\pgfpathlineto{\pgfqpoint{3.423900in}{1.113444in}}%
\pgfpathlineto{\pgfqpoint{3.423210in}{1.122204in}}%
\pgfpathlineto{\pgfqpoint{3.421746in}{1.130862in}}%
\pgfpathlineto{\pgfqpoint{3.419516in}{1.139384in}}%
\pgfpathlineto{\pgfqpoint{3.416529in}{1.147737in}}%
\pgfpathlineto{\pgfqpoint{3.418448in}{1.141362in}}%
\pgfpathlineto{\pgfqpoint{3.421163in}{1.135067in}}%
\pgfpathlineto{\pgfqpoint{3.424659in}{1.128877in}}%
\pgfpathlineto{\pgfqpoint{3.428921in}{1.122816in}}%
\pgfpathlineto{\pgfqpoint{3.433930in}{1.116907in}}%
\pgfpathclose%
\pgfusepath{fill}%
\end{pgfscope}%
\begin{pgfscope}%
\pgfpathrectangle{\pgfqpoint{2.548318in}{0.050000in}}{\pgfqpoint{2.081932in}{2.081932in}}%
\pgfusepath{clip}%
\pgfsetbuttcap%
\pgfsetroundjoin%
\definecolor{currentfill}{rgb}{0.876168,0.891125,0.095250}%
\pgfsetfillcolor{currentfill}%
\pgfsetlinewidth{0.000000pt}%
\definecolor{currentstroke}{rgb}{0.000000,0.000000,0.000000}%
\pgfsetstrokecolor{currentstroke}%
\pgfsetdash{}{0pt}%
\pgfpathmoveto{\pgfqpoint{3.122187in}{1.205741in}}%
\pgfpathlineto{\pgfqpoint{3.132411in}{1.211096in}}%
\pgfpathlineto{\pgfqpoint{3.143026in}{1.216002in}}%
\pgfpathlineto{\pgfqpoint{3.153991in}{1.220439in}}%
\pgfpathlineto{\pgfqpoint{3.165261in}{1.224390in}}%
\pgfpathlineto{\pgfqpoint{3.176792in}{1.227839in}}%
\pgfpathlineto{\pgfqpoint{3.173827in}{1.242127in}}%
\pgfpathlineto{\pgfqpoint{3.172637in}{1.256513in}}%
\pgfpathlineto{\pgfqpoint{3.173239in}{1.270940in}}%
\pgfpathlineto{\pgfqpoint{3.175642in}{1.285349in}}%
\pgfpathlineto{\pgfqpoint{3.179847in}{1.299682in}}%
\pgfpathlineto{\pgfqpoint{3.168268in}{1.298095in}}%
\pgfpathlineto{\pgfqpoint{3.156945in}{1.295956in}}%
\pgfpathlineto{\pgfqpoint{3.145923in}{1.293275in}}%
\pgfpathlineto{\pgfqpoint{3.135246in}{1.290060in}}%
\pgfpathlineto{\pgfqpoint{3.124957in}{1.286323in}}%
\pgfpathlineto{\pgfqpoint{3.120358in}{1.270230in}}%
\pgfpathlineto{\pgfqpoint{3.117790in}{1.254059in}}%
\pgfpathlineto{\pgfqpoint{3.117248in}{1.237877in}}%
\pgfpathlineto{\pgfqpoint{3.118721in}{1.221750in}}%
\pgfpathlineto{\pgfqpoint{3.122187in}{1.205741in}}%
\pgfpathclose%
\pgfusepath{fill}%
\end{pgfscope}%
\begin{pgfscope}%
\pgfpathrectangle{\pgfqpoint{2.548318in}{0.050000in}}{\pgfqpoint{2.081932in}{2.081932in}}%
\pgfusepath{clip}%
\pgfsetbuttcap%
\pgfsetroundjoin%
\definecolor{currentfill}{rgb}{0.327796,0.773980,0.406640}%
\pgfsetfillcolor{currentfill}%
\pgfsetlinewidth{0.000000pt}%
\definecolor{currentstroke}{rgb}{0.000000,0.000000,0.000000}%
\pgfsetstrokecolor{currentstroke}%
\pgfsetdash{}{0pt}%
\pgfpathmoveto{\pgfqpoint{4.189404in}{1.201435in}}%
\pgfpathlineto{\pgfqpoint{4.184930in}{1.209440in}}%
\pgfpathlineto{\pgfqpoint{4.179737in}{1.217180in}}%
\pgfpathlineto{\pgfqpoint{4.173848in}{1.224624in}}%
\pgfpathlineto{\pgfqpoint{4.167287in}{1.231742in}}%
\pgfpathlineto{\pgfqpoint{4.160082in}{1.238507in}}%
\pgfpathlineto{\pgfqpoint{4.162213in}{1.220957in}}%
\pgfpathlineto{\pgfqpoint{4.162137in}{1.203419in}}%
\pgfpathlineto{\pgfqpoint{4.159875in}{1.185966in}}%
\pgfpathlineto{\pgfqpoint{4.155451in}{1.168667in}}%
\pgfpathlineto{\pgfqpoint{4.148901in}{1.151588in}}%
\pgfpathlineto{\pgfqpoint{4.155864in}{1.143774in}}%
\pgfpathlineto{\pgfqpoint{4.162201in}{1.135716in}}%
\pgfpathlineto{\pgfqpoint{4.167888in}{1.127445in}}%
\pgfpathlineto{\pgfqpoint{4.172902in}{1.118994in}}%
\pgfpathlineto{\pgfqpoint{4.177221in}{1.110396in}}%
\pgfpathlineto{\pgfqpoint{4.184200in}{1.128274in}}%
\pgfpathlineto{\pgfqpoint{4.188940in}{1.146389in}}%
\pgfpathlineto{\pgfqpoint{4.191406in}{1.164669in}}%
\pgfpathlineto{\pgfqpoint{4.191567in}{1.183042in}}%
\pgfpathlineto{\pgfqpoint{4.189404in}{1.201435in}}%
\pgfpathclose%
\pgfusepath{fill}%
\end{pgfscope}%
\begin{pgfscope}%
\pgfpathrectangle{\pgfqpoint{2.548318in}{0.050000in}}{\pgfqpoint{2.081932in}{2.081932in}}%
\pgfusepath{clip}%
\pgfsetbuttcap%
\pgfsetroundjoin%
\definecolor{currentfill}{rgb}{0.268510,0.009605,0.335427}%
\pgfsetfillcolor{currentfill}%
\pgfsetlinewidth{0.000000pt}%
\definecolor{currentstroke}{rgb}{0.000000,0.000000,0.000000}%
\pgfsetstrokecolor{currentstroke}%
\pgfsetdash{}{0pt}%
\pgfpathmoveto{\pgfqpoint{4.010598in}{0.943484in}}%
\pgfpathlineto{\pgfqpoint{4.022657in}{0.943102in}}%
\pgfpathlineto{\pgfqpoint{4.034613in}{0.943290in}}%
\pgfpathlineto{\pgfqpoint{4.046419in}{0.944047in}}%
\pgfpathlineto{\pgfqpoint{4.058025in}{0.945370in}}%
\pgfpathlineto{\pgfqpoint{4.069384in}{0.947254in}}%
\pgfpathlineto{\pgfqpoint{4.062213in}{0.933416in}}%
\pgfpathlineto{\pgfqpoint{4.053302in}{0.919856in}}%
\pgfpathlineto{\pgfqpoint{4.042696in}{0.906624in}}%
\pgfpathlineto{\pgfqpoint{4.030447in}{0.893771in}}%
\pgfpathlineto{\pgfqpoint{4.016613in}{0.881343in}}%
\pgfpathlineto{\pgfqpoint{4.006682in}{0.881119in}}%
\pgfpathlineto{\pgfqpoint{3.996530in}{0.881485in}}%
\pgfpathlineto{\pgfqpoint{3.986198in}{0.882440in}}%
\pgfpathlineto{\pgfqpoint{3.975728in}{0.883982in}}%
\pgfpathlineto{\pgfqpoint{3.965163in}{0.886105in}}%
\pgfpathlineto{\pgfqpoint{3.977122in}{0.896936in}}%
\pgfpathlineto{\pgfqpoint{3.987691in}{0.908132in}}%
\pgfpathlineto{\pgfqpoint{3.996821in}{0.919652in}}%
\pgfpathlineto{\pgfqpoint{4.004470in}{0.931451in}}%
\pgfpathlineto{\pgfqpoint{4.010598in}{0.943484in}}%
\pgfpathclose%
\pgfusepath{fill}%
\end{pgfscope}%
\begin{pgfscope}%
\pgfpathrectangle{\pgfqpoint{2.548318in}{0.050000in}}{\pgfqpoint{2.081932in}{2.081932in}}%
\pgfusepath{clip}%
\pgfsetbuttcap%
\pgfsetroundjoin%
\definecolor{currentfill}{rgb}{0.606045,0.850733,0.236712}%
\pgfsetfillcolor{currentfill}%
\pgfsetlinewidth{0.000000pt}%
\definecolor{currentstroke}{rgb}{0.000000,0.000000,0.000000}%
\pgfsetstrokecolor{currentstroke}%
\pgfsetdash{}{0pt}%
\pgfpathmoveto{\pgfqpoint{3.349297in}{1.215106in}}%
\pgfpathlineto{\pgfqpoint{3.358678in}{1.210102in}}%
\pgfpathlineto{\pgfqpoint{3.367553in}{1.204648in}}%
\pgfpathlineto{\pgfqpoint{3.375888in}{1.198766in}}%
\pgfpathlineto{\pgfqpoint{3.383648in}{1.192479in}}%
\pgfpathlineto{\pgfqpoint{3.390801in}{1.185812in}}%
\pgfpathlineto{\pgfqpoint{3.389515in}{1.193108in}}%
\pgfpathlineto{\pgfqpoint{3.389145in}{1.200438in}}%
\pgfpathlineto{\pgfqpoint{3.389695in}{1.207774in}}%
\pgfpathlineto{\pgfqpoint{3.391166in}{1.215085in}}%
\pgfpathlineto{\pgfqpoint{3.393555in}{1.222342in}}%
\pgfpathlineto{\pgfqpoint{3.386449in}{1.230206in}}%
\pgfpathlineto{\pgfqpoint{3.378736in}{1.237789in}}%
\pgfpathlineto{\pgfqpoint{3.370449in}{1.245062in}}%
\pgfpathlineto{\pgfqpoint{3.361621in}{1.251994in}}%
\pgfpathlineto{\pgfqpoint{3.352285in}{1.258560in}}%
\pgfpathlineto{\pgfqpoint{3.349510in}{1.249921in}}%
\pgfpathlineto{\pgfqpoint{3.347823in}{1.241220in}}%
\pgfpathlineto{\pgfqpoint{3.347227in}{1.232494in}}%
\pgfpathlineto{\pgfqpoint{3.347721in}{1.223778in}}%
\pgfpathlineto{\pgfqpoint{3.349297in}{1.215106in}}%
\pgfpathclose%
\pgfusepath{fill}%
\end{pgfscope}%
\begin{pgfscope}%
\pgfpathrectangle{\pgfqpoint{2.548318in}{0.050000in}}{\pgfqpoint{2.081932in}{2.081932in}}%
\pgfusepath{clip}%
\pgfsetbuttcap%
\pgfsetroundjoin%
\definecolor{currentfill}{rgb}{0.162142,0.474838,0.558140}%
\pgfsetfillcolor{currentfill}%
\pgfsetlinewidth{0.000000pt}%
\definecolor{currentstroke}{rgb}{0.000000,0.000000,0.000000}%
\pgfsetstrokecolor{currentstroke}%
\pgfsetdash{}{0pt}%
\pgfpathmoveto{\pgfqpoint{3.808812in}{1.096988in}}%
\pgfpathlineto{\pgfqpoint{3.809666in}{1.088108in}}%
\pgfpathlineto{\pgfqpoint{3.811285in}{1.079204in}}%
\pgfpathlineto{\pgfqpoint{3.813664in}{1.070310in}}%
\pgfpathlineto{\pgfqpoint{3.816794in}{1.061465in}}%
\pgfpathlineto{\pgfqpoint{3.820662in}{1.052703in}}%
\pgfpathlineto{\pgfqpoint{3.817679in}{1.046371in}}%
\pgfpathlineto{\pgfqpoint{3.813900in}{1.040151in}}%
\pgfpathlineto{\pgfqpoint{3.809342in}{1.034067in}}%
\pgfpathlineto{\pgfqpoint{3.804026in}{1.028144in}}%
\pgfpathlineto{\pgfqpoint{3.797973in}{1.022404in}}%
\pgfpathlineto{\pgfqpoint{3.794553in}{1.031700in}}%
\pgfpathlineto{\pgfqpoint{3.791785in}{1.040971in}}%
\pgfpathlineto{\pgfqpoint{3.789681in}{1.050178in}}%
\pgfpathlineto{\pgfqpoint{3.788249in}{1.059284in}}%
\pgfpathlineto{\pgfqpoint{3.787493in}{1.068252in}}%
\pgfpathlineto{\pgfqpoint{3.793185in}{1.073697in}}%
\pgfpathlineto{\pgfqpoint{3.798183in}{1.079315in}}%
\pgfpathlineto{\pgfqpoint{3.802465in}{1.085085in}}%
\pgfpathlineto{\pgfqpoint{3.806014in}{1.090983in}}%
\pgfpathlineto{\pgfqpoint{3.808812in}{1.096988in}}%
\pgfpathclose%
\pgfusepath{fill}%
\end{pgfscope}%
\begin{pgfscope}%
\pgfpathrectangle{\pgfqpoint{2.548318in}{0.050000in}}{\pgfqpoint{2.081932in}{2.081932in}}%
\pgfusepath{clip}%
\pgfsetbuttcap%
\pgfsetroundjoin%
\definecolor{currentfill}{rgb}{0.282327,0.094955,0.417331}%
\pgfsetfillcolor{currentfill}%
\pgfsetlinewidth{0.000000pt}%
\definecolor{currentstroke}{rgb}{0.000000,0.000000,0.000000}%
\pgfsetstrokecolor{currentstroke}%
\pgfsetdash{}{0pt}%
\pgfpathmoveto{\pgfqpoint{3.201405in}{0.896972in}}%
\pgfpathlineto{\pgfqpoint{3.191297in}{0.898166in}}%
\pgfpathlineto{\pgfqpoint{3.181501in}{0.899933in}}%
\pgfpathlineto{\pgfqpoint{3.172058in}{0.902266in}}%
\pgfpathlineto{\pgfqpoint{3.163004in}{0.905155in}}%
\pgfpathlineto{\pgfqpoint{3.154376in}{0.908585in}}%
\pgfpathlineto{\pgfqpoint{3.141080in}{0.923032in}}%
\pgfpathlineto{\pgfqpoint{3.129619in}{0.937894in}}%
\pgfpathlineto{\pgfqpoint{3.120049in}{0.953117in}}%
\pgfpathlineto{\pgfqpoint{3.112424in}{0.968642in}}%
\pgfpathlineto{\pgfqpoint{3.106786in}{0.984412in}}%
\pgfpathlineto{\pgfqpoint{3.116413in}{0.979511in}}%
\pgfpathlineto{\pgfqpoint{3.126511in}{0.975091in}}%
\pgfpathlineto{\pgfqpoint{3.137038in}{0.971170in}}%
\pgfpathlineto{\pgfqpoint{3.147952in}{0.967765in}}%
\pgfpathlineto{\pgfqpoint{3.159208in}{0.964890in}}%
\pgfpathlineto{\pgfqpoint{3.164152in}{0.950778in}}%
\pgfpathlineto{\pgfqpoint{3.170885in}{0.936879in}}%
\pgfpathlineto{\pgfqpoint{3.179368in}{0.923244in}}%
\pgfpathlineto{\pgfqpoint{3.189558in}{0.909925in}}%
\pgfpathlineto{\pgfqpoint{3.201405in}{0.896972in}}%
\pgfpathclose%
\pgfusepath{fill}%
\end{pgfscope}%
\begin{pgfscope}%
\pgfpathrectangle{\pgfqpoint{2.548318in}{0.050000in}}{\pgfqpoint{2.081932in}{2.081932in}}%
\pgfusepath{clip}%
\pgfsetbuttcap%
\pgfsetroundjoin%
\definecolor{currentfill}{rgb}{0.296479,0.761561,0.424223}%
\pgfsetfillcolor{currentfill}%
\pgfsetlinewidth{0.000000pt}%
\definecolor{currentstroke}{rgb}{0.000000,0.000000,0.000000}%
\pgfsetstrokecolor{currentstroke}%
\pgfsetdash{}{0pt}%
\pgfpathmoveto{\pgfqpoint{3.844328in}{1.213247in}}%
\pgfpathlineto{\pgfqpoint{3.837782in}{1.205405in}}%
\pgfpathlineto{\pgfqpoint{3.831903in}{1.197318in}}%
\pgfpathlineto{\pgfqpoint{3.826715in}{1.189017in}}%
\pgfpathlineto{\pgfqpoint{3.822240in}{1.180535in}}%
\pgfpathlineto{\pgfqpoint{3.818495in}{1.171905in}}%
\pgfpathlineto{\pgfqpoint{3.819622in}{1.165459in}}%
\pgfpathlineto{\pgfqpoint{3.819932in}{1.158996in}}%
\pgfpathlineto{\pgfqpoint{3.819428in}{1.152542in}}%
\pgfpathlineto{\pgfqpoint{3.818113in}{1.146124in}}%
\pgfpathlineto{\pgfqpoint{3.815995in}{1.139767in}}%
\pgfpathlineto{\pgfqpoint{3.819674in}{1.147758in}}%
\pgfpathlineto{\pgfqpoint{3.824071in}{1.155484in}}%
\pgfpathlineto{\pgfqpoint{3.829167in}{1.162914in}}%
\pgfpathlineto{\pgfqpoint{3.834939in}{1.170019in}}%
\pgfpathlineto{\pgfqpoint{3.841365in}{1.176772in}}%
\pgfpathlineto{\pgfqpoint{3.843782in}{1.183983in}}%
\pgfpathlineto{\pgfqpoint{3.845294in}{1.191265in}}%
\pgfpathlineto{\pgfqpoint{3.845892in}{1.198589in}}%
\pgfpathlineto{\pgfqpoint{3.845570in}{1.205926in}}%
\pgfpathlineto{\pgfqpoint{3.844328in}{1.213247in}}%
\pgfpathclose%
\pgfusepath{fill}%
\end{pgfscope}%
\begin{pgfscope}%
\pgfpathrectangle{\pgfqpoint{2.548318in}{0.050000in}}{\pgfqpoint{2.081932in}{2.081932in}}%
\pgfusepath{clip}%
\pgfsetbuttcap%
\pgfsetroundjoin%
\definecolor{currentfill}{rgb}{0.636902,0.856542,0.216620}%
\pgfsetfillcolor{currentfill}%
\pgfsetlinewidth{0.000000pt}%
\definecolor{currentstroke}{rgb}{0.000000,0.000000,0.000000}%
\pgfsetstrokecolor{currentstroke}%
\pgfsetdash{}{0pt}%
\pgfpathmoveto{\pgfqpoint{3.078289in}{1.173008in}}%
\pgfpathlineto{\pgfqpoint{3.085985in}{1.180266in}}%
\pgfpathlineto{\pgfqpoint{3.094257in}{1.187194in}}%
\pgfpathlineto{\pgfqpoint{3.103072in}{1.193766in}}%
\pgfpathlineto{\pgfqpoint{3.112394in}{1.199957in}}%
\pgfpathlineto{\pgfqpoint{3.122187in}{1.205741in}}%
\pgfpathlineto{\pgfqpoint{3.118721in}{1.221750in}}%
\pgfpathlineto{\pgfqpoint{3.117248in}{1.237877in}}%
\pgfpathlineto{\pgfqpoint{3.117790in}{1.254059in}}%
\pgfpathlineto{\pgfqpoint{3.120358in}{1.270230in}}%
\pgfpathlineto{\pgfqpoint{3.124957in}{1.286323in}}%
\pgfpathlineto{\pgfqpoint{3.115097in}{1.282080in}}%
\pgfpathlineto{\pgfqpoint{3.105707in}{1.277346in}}%
\pgfpathlineto{\pgfqpoint{3.096824in}{1.272139in}}%
\pgfpathlineto{\pgfqpoint{3.088484in}{1.266480in}}%
\pgfpathlineto{\pgfqpoint{3.080723in}{1.260390in}}%
\pgfpathlineto{\pgfqpoint{3.075828in}{1.242924in}}%
\pgfpathlineto{\pgfqpoint{3.073149in}{1.225381in}}%
\pgfpathlineto{\pgfqpoint{3.072677in}{1.207833in}}%
\pgfpathlineto{\pgfqpoint{3.074399in}{1.190353in}}%
\pgfpathlineto{\pgfqpoint{3.078289in}{1.173008in}}%
\pgfpathclose%
\pgfusepath{fill}%
\end{pgfscope}%
\begin{pgfscope}%
\pgfpathrectangle{\pgfqpoint{2.548318in}{0.050000in}}{\pgfqpoint{2.081932in}{2.081932in}}%
\pgfusepath{clip}%
\pgfsetbuttcap%
\pgfsetroundjoin%
\definecolor{currentfill}{rgb}{0.162142,0.474838,0.558140}%
\pgfsetfillcolor{currentfill}%
\pgfsetlinewidth{0.000000pt}%
\definecolor{currentstroke}{rgb}{0.000000,0.000000,0.000000}%
\pgfsetstrokecolor{currentstroke}%
\pgfsetdash{}{0pt}%
\pgfpathmoveto{\pgfqpoint{3.440553in}{1.075097in}}%
\pgfpathlineto{\pgfqpoint{3.439766in}{1.066150in}}%
\pgfpathlineto{\pgfqpoint{3.438273in}{1.057092in}}%
\pgfpathlineto{\pgfqpoint{3.436080in}{1.047959in}}%
\pgfpathlineto{\pgfqpoint{3.433195in}{1.038789in}}%
\pgfpathlineto{\pgfqpoint{3.429630in}{1.029620in}}%
\pgfpathlineto{\pgfqpoint{3.424499in}{1.035586in}}%
\pgfpathlineto{\pgfqpoint{3.420133in}{1.041706in}}%
\pgfpathlineto{\pgfqpoint{3.416550in}{1.047957in}}%
\pgfpathlineto{\pgfqpoint{3.413767in}{1.054313in}}%
\pgfpathlineto{\pgfqpoint{3.411798in}{1.060751in}}%
\pgfpathlineto{\pgfqpoint{3.415720in}{1.069370in}}%
\pgfpathlineto{\pgfqpoint{3.418893in}{1.078102in}}%
\pgfpathlineto{\pgfqpoint{3.421305in}{1.086911in}}%
\pgfpathlineto{\pgfqpoint{3.422947in}{1.095762in}}%
\pgfpathlineto{\pgfqpoint{3.423812in}{1.104618in}}%
\pgfpathlineto{\pgfqpoint{3.425656in}{1.098515in}}%
\pgfpathlineto{\pgfqpoint{3.428265in}{1.092487in}}%
\pgfpathlineto{\pgfqpoint{3.431629in}{1.086560in}}%
\pgfpathlineto{\pgfqpoint{3.435731in}{1.080756in}}%
\pgfpathlineto{\pgfqpoint{3.440553in}{1.075097in}}%
\pgfpathclose%
\pgfusepath{fill}%
\end{pgfscope}%
\begin{pgfscope}%
\pgfpathrectangle{\pgfqpoint{2.548318in}{0.050000in}}{\pgfqpoint{2.081932in}{2.081932in}}%
\pgfusepath{clip}%
\pgfsetbuttcap%
\pgfsetroundjoin%
\definecolor{currentfill}{rgb}{0.267004,0.004874,0.329415}%
\pgfsetfillcolor{currentfill}%
\pgfsetlinewidth{0.000000pt}%
\definecolor{currentstroke}{rgb}{0.000000,0.000000,0.000000}%
\pgfsetstrokecolor{currentstroke}%
\pgfsetdash{}{0pt}%
\pgfpathmoveto{\pgfqpoint{3.950510in}{0.953833in}}%
\pgfpathlineto{\pgfqpoint{3.962334in}{0.950662in}}%
\pgfpathlineto{\pgfqpoint{3.974305in}{0.948033in}}%
\pgfpathlineto{\pgfqpoint{3.986372in}{0.945955in}}%
\pgfpathlineto{\pgfqpoint{3.998486in}{0.944437in}}%
\pgfpathlineto{\pgfqpoint{4.010598in}{0.943484in}}%
\pgfpathlineto{\pgfqpoint{4.004470in}{0.931451in}}%
\pgfpathlineto{\pgfqpoint{3.996821in}{0.919652in}}%
\pgfpathlineto{\pgfqpoint{3.987691in}{0.908132in}}%
\pgfpathlineto{\pgfqpoint{3.977122in}{0.896936in}}%
\pgfpathlineto{\pgfqpoint{3.965163in}{0.886105in}}%
\pgfpathlineto{\pgfqpoint{3.954546in}{0.888801in}}%
\pgfpathlineto{\pgfqpoint{3.943920in}{0.892061in}}%
\pgfpathlineto{\pgfqpoint{3.933330in}{0.895870in}}%
\pgfpathlineto{\pgfqpoint{3.922819in}{0.900215in}}%
\pgfpathlineto{\pgfqpoint{3.912430in}{0.905077in}}%
\pgfpathlineto{\pgfqpoint{3.922495in}{0.914291in}}%
\pgfpathlineto{\pgfqpoint{3.931373in}{0.923810in}}%
\pgfpathlineto{\pgfqpoint{3.939025in}{0.933598in}}%
\pgfpathlineto{\pgfqpoint{3.945414in}{0.943618in}}%
\pgfpathlineto{\pgfqpoint{3.950510in}{0.953833in}}%
\pgfpathclose%
\pgfusepath{fill}%
\end{pgfscope}%
\begin{pgfscope}%
\pgfpathrectangle{\pgfqpoint{2.548318in}{0.050000in}}{\pgfqpoint{2.081932in}{2.081932in}}%
\pgfusepath{clip}%
\pgfsetbuttcap%
\pgfsetroundjoin%
\definecolor{currentfill}{rgb}{0.227802,0.326594,0.546532}%
\pgfsetfillcolor{currentfill}%
\pgfsetlinewidth{0.000000pt}%
\definecolor{currentstroke}{rgb}{0.000000,0.000000,0.000000}%
\pgfsetstrokecolor{currentstroke}%
\pgfsetdash{}{0pt}%
\pgfpathmoveto{\pgfqpoint{3.820662in}{1.052703in}}%
\pgfpathlineto{\pgfqpoint{3.825255in}{1.044061in}}%
\pgfpathlineto{\pgfqpoint{3.830553in}{1.035574in}}%
\pgfpathlineto{\pgfqpoint{3.836536in}{1.027277in}}%
\pgfpathlineto{\pgfqpoint{3.843181in}{1.019205in}}%
\pgfpathlineto{\pgfqpoint{3.850459in}{1.011390in}}%
\pgfpathlineto{\pgfqpoint{3.847006in}{1.004177in}}%
\pgfpathlineto{\pgfqpoint{3.842641in}{0.997093in}}%
\pgfpathlineto{\pgfqpoint{3.837386in}{0.990167in}}%
\pgfpathlineto{\pgfqpoint{3.831263in}{0.983425in}}%
\pgfpathlineto{\pgfqpoint{3.824299in}{0.976894in}}%
\pgfpathlineto{\pgfqpoint{3.817872in}{0.985740in}}%
\pgfpathlineto{\pgfqpoint{3.812003in}{0.994752in}}%
\pgfpathlineto{\pgfqpoint{3.806716in}{1.003892in}}%
\pgfpathlineto{\pgfqpoint{3.802033in}{1.013122in}}%
\pgfpathlineto{\pgfqpoint{3.797973in}{1.022404in}}%
\pgfpathlineto{\pgfqpoint{3.804026in}{1.028144in}}%
\pgfpathlineto{\pgfqpoint{3.809342in}{1.034067in}}%
\pgfpathlineto{\pgfqpoint{3.813900in}{1.040151in}}%
\pgfpathlineto{\pgfqpoint{3.817679in}{1.046371in}}%
\pgfpathlineto{\pgfqpoint{3.820662in}{1.052703in}}%
\pgfpathclose%
\pgfusepath{fill}%
\end{pgfscope}%
\begin{pgfscope}%
\pgfpathrectangle{\pgfqpoint{2.548318in}{0.050000in}}{\pgfqpoint{2.081932in}{2.081932in}}%
\pgfusepath{clip}%
\pgfsetbuttcap%
\pgfsetroundjoin%
\definecolor{currentfill}{rgb}{0.124780,0.640461,0.527068}%
\pgfsetfillcolor{currentfill}%
\pgfsetlinewidth{0.000000pt}%
\definecolor{currentstroke}{rgb}{0.000000,0.000000,0.000000}%
\pgfsetstrokecolor{currentstroke}%
\pgfsetdash{}{0pt}%
\pgfpathmoveto{\pgfqpoint{4.200446in}{1.158578in}}%
\pgfpathlineto{\pgfqpoint{4.199787in}{1.167414in}}%
\pgfpathlineto{\pgfqpoint{4.198345in}{1.176152in}}%
\pgfpathlineto{\pgfqpoint{4.196127in}{1.184758in}}%
\pgfpathlineto{\pgfqpoint{4.193142in}{1.193197in}}%
\pgfpathlineto{\pgfqpoint{4.189404in}{1.201435in}}%
\pgfpathlineto{\pgfqpoint{4.191567in}{1.183042in}}%
\pgfpathlineto{\pgfqpoint{4.191406in}{1.164669in}}%
\pgfpathlineto{\pgfqpoint{4.188940in}{1.146389in}}%
\pgfpathlineto{\pgfqpoint{4.184200in}{1.128274in}}%
\pgfpathlineto{\pgfqpoint{4.177221in}{1.110396in}}%
\pgfpathlineto{\pgfqpoint{4.180829in}{1.101686in}}%
\pgfpathlineto{\pgfqpoint{4.183708in}{1.092897in}}%
\pgfpathlineto{\pgfqpoint{4.185849in}{1.084065in}}%
\pgfpathlineto{\pgfqpoint{4.187240in}{1.075224in}}%
\pgfpathlineto{\pgfqpoint{4.187876in}{1.066411in}}%
\pgfpathlineto{\pgfqpoint{4.195017in}{1.084506in}}%
\pgfpathlineto{\pgfqpoint{4.199879in}{1.102843in}}%
\pgfpathlineto{\pgfqpoint{4.202424in}{1.121350in}}%
\pgfpathlineto{\pgfqpoint{4.202620in}{1.139953in}}%
\pgfpathlineto{\pgfqpoint{4.200446in}{1.158578in}}%
\pgfpathclose%
\pgfusepath{fill}%
\end{pgfscope}%
\begin{pgfscope}%
\pgfpathrectangle{\pgfqpoint{2.548318in}{0.050000in}}{\pgfqpoint{2.081932in}{2.081932in}}%
\pgfusepath{clip}%
\pgfsetbuttcap%
\pgfsetroundjoin%
\definecolor{currentfill}{rgb}{0.296479,0.761561,0.424223}%
\pgfsetfillcolor{currentfill}%
\pgfsetlinewidth{0.000000pt}%
\definecolor{currentstroke}{rgb}{0.000000,0.000000,0.000000}%
\pgfsetstrokecolor{currentstroke}%
\pgfsetdash{}{0pt}%
\pgfpathmoveto{\pgfqpoint{3.390801in}{1.185812in}}%
\pgfpathlineto{\pgfqpoint{3.397318in}{1.178791in}}%
\pgfpathlineto{\pgfqpoint{3.403172in}{1.171443in}}%
\pgfpathlineto{\pgfqpoint{3.408340in}{1.163798in}}%
\pgfpathlineto{\pgfqpoint{3.412798in}{1.155885in}}%
\pgfpathlineto{\pgfqpoint{3.416529in}{1.147737in}}%
\pgfpathlineto{\pgfqpoint{3.415415in}{1.154166in}}%
\pgfpathlineto{\pgfqpoint{3.415112in}{1.160624in}}%
\pgfpathlineto{\pgfqpoint{3.415625in}{1.167086in}}%
\pgfpathlineto{\pgfqpoint{3.416954in}{1.173524in}}%
\pgfpathlineto{\pgfqpoint{3.419096in}{1.179913in}}%
\pgfpathlineto{\pgfqpoint{3.415394in}{1.188702in}}%
\pgfpathlineto{\pgfqpoint{3.410970in}{1.197374in}}%
\pgfpathlineto{\pgfqpoint{3.405841in}{1.205894in}}%
\pgfpathlineto{\pgfqpoint{3.400028in}{1.214227in}}%
\pgfpathlineto{\pgfqpoint{3.393555in}{1.222342in}}%
\pgfpathlineto{\pgfqpoint{3.391166in}{1.215085in}}%
\pgfpathlineto{\pgfqpoint{3.389695in}{1.207774in}}%
\pgfpathlineto{\pgfqpoint{3.389145in}{1.200438in}}%
\pgfpathlineto{\pgfqpoint{3.389515in}{1.193108in}}%
\pgfpathlineto{\pgfqpoint{3.390801in}{1.185812in}}%
\pgfpathclose%
\pgfusepath{fill}%
\end{pgfscope}%
\begin{pgfscope}%
\pgfpathrectangle{\pgfqpoint{2.548318in}{0.050000in}}{\pgfqpoint{2.081932in}{2.081932in}}%
\pgfusepath{clip}%
\pgfsetbuttcap%
\pgfsetroundjoin%
\definecolor{currentfill}{rgb}{0.268510,0.009605,0.335427}%
\pgfsetfillcolor{currentfill}%
\pgfsetlinewidth{0.000000pt}%
\definecolor{currentstroke}{rgb}{0.000000,0.000000,0.000000}%
\pgfsetstrokecolor{currentstroke}%
\pgfsetdash{}{0pt}%
\pgfpathmoveto{\pgfqpoint{3.255167in}{0.899725in}}%
\pgfpathlineto{\pgfqpoint{3.244129in}{0.898012in}}%
\pgfpathlineto{\pgfqpoint{3.233190in}{0.896877in}}%
\pgfpathlineto{\pgfqpoint{3.222394in}{0.896325in}}%
\pgfpathlineto{\pgfqpoint{3.211784in}{0.896357in}}%
\pgfpathlineto{\pgfqpoint{3.201405in}{0.896972in}}%
\pgfpathlineto{\pgfqpoint{3.189558in}{0.909925in}}%
\pgfpathlineto{\pgfqpoint{3.179368in}{0.923244in}}%
\pgfpathlineto{\pgfqpoint{3.170885in}{0.936879in}}%
\pgfpathlineto{\pgfqpoint{3.164152in}{0.950778in}}%
\pgfpathlineto{\pgfqpoint{3.159208in}{0.964890in}}%
\pgfpathlineto{\pgfqpoint{3.170761in}{0.962557in}}%
\pgfpathlineto{\pgfqpoint{3.182564in}{0.960776in}}%
\pgfpathlineto{\pgfqpoint{3.194567in}{0.959555in}}%
\pgfpathlineto{\pgfqpoint{3.206722in}{0.958899in}}%
\pgfpathlineto{\pgfqpoint{3.218979in}{0.958811in}}%
\pgfpathlineto{\pgfqpoint{3.223164in}{0.946548in}}%
\pgfpathlineto{\pgfqpoint{3.228909in}{0.934463in}}%
\pgfpathlineto{\pgfqpoint{3.236184in}{0.922600in}}%
\pgfpathlineto{\pgfqpoint{3.244950in}{0.911006in}}%
\pgfpathlineto{\pgfqpoint{3.255167in}{0.899725in}}%
\pgfpathclose%
\pgfusepath{fill}%
\end{pgfscope}%
\begin{pgfscope}%
\pgfpathrectangle{\pgfqpoint{2.548318in}{0.050000in}}{\pgfqpoint{2.081932in}{2.081932in}}%
\pgfusepath{clip}%
\pgfsetbuttcap%
\pgfsetroundjoin%
\definecolor{currentfill}{rgb}{0.278791,0.062145,0.386592}%
\pgfsetfillcolor{currentfill}%
\pgfsetlinewidth{0.000000pt}%
\definecolor{currentstroke}{rgb}{0.000000,0.000000,0.000000}%
\pgfsetstrokecolor{currentstroke}%
\pgfsetdash{}{0pt}%
\pgfpathmoveto{\pgfqpoint{3.895252in}{0.977267in}}%
\pgfpathlineto{\pgfqpoint{3.905635in}{0.971630in}}%
\pgfpathlineto{\pgfqpoint{3.916396in}{0.966448in}}%
\pgfpathlineto{\pgfqpoint{3.927494in}{0.961742in}}%
\pgfpathlineto{\pgfqpoint{3.938880in}{0.957531in}}%
\pgfpathlineto{\pgfqpoint{3.950510in}{0.953833in}}%
\pgfpathlineto{\pgfqpoint{3.945414in}{0.943618in}}%
\pgfpathlineto{\pgfqpoint{3.939025in}{0.933598in}}%
\pgfpathlineto{\pgfqpoint{3.931373in}{0.923810in}}%
\pgfpathlineto{\pgfqpoint{3.922495in}{0.914291in}}%
\pgfpathlineto{\pgfqpoint{3.912430in}{0.905077in}}%
\pgfpathlineto{\pgfqpoint{3.902207in}{0.910438in}}%
\pgfpathlineto{\pgfqpoint{3.892192in}{0.916275in}}%
\pgfpathlineto{\pgfqpoint{3.882426in}{0.922566in}}%
\pgfpathlineto{\pgfqpoint{3.872951in}{0.929283in}}%
\pgfpathlineto{\pgfqpoint{3.863806in}{0.936399in}}%
\pgfpathlineto{\pgfqpoint{3.872150in}{0.944130in}}%
\pgfpathlineto{\pgfqpoint{3.879497in}{0.952114in}}%
\pgfpathlineto{\pgfqpoint{3.885815in}{0.960319in}}%
\pgfpathlineto{\pgfqpoint{3.891075in}{0.968714in}}%
\pgfpathlineto{\pgfqpoint{3.895252in}{0.977267in}}%
\pgfpathclose%
\pgfusepath{fill}%
\end{pgfscope}%
\begin{pgfscope}%
\pgfpathrectangle{\pgfqpoint{2.548318in}{0.050000in}}{\pgfqpoint{2.081932in}{2.081932in}}%
\pgfusepath{clip}%
\pgfsetbuttcap%
\pgfsetroundjoin%
\definecolor{currentfill}{rgb}{0.278012,0.180367,0.486697}%
\pgfsetfillcolor{currentfill}%
\pgfsetlinewidth{0.000000pt}%
\definecolor{currentstroke}{rgb}{0.000000,0.000000,0.000000}%
\pgfsetstrokecolor{currentstroke}%
\pgfsetdash{}{0pt}%
\pgfpathmoveto{\pgfqpoint{3.850459in}{1.011390in}}%
\pgfpathlineto{\pgfqpoint{3.858343in}{1.003865in}}%
\pgfpathlineto{\pgfqpoint{3.866800in}{0.996661in}}%
\pgfpathlineto{\pgfqpoint{3.875795in}{0.989808in}}%
\pgfpathlineto{\pgfqpoint{3.885292in}{0.983334in}}%
\pgfpathlineto{\pgfqpoint{3.895252in}{0.977267in}}%
\pgfpathlineto{\pgfqpoint{3.891075in}{0.968714in}}%
\pgfpathlineto{\pgfqpoint{3.885815in}{0.960319in}}%
\pgfpathlineto{\pgfqpoint{3.879497in}{0.952114in}}%
\pgfpathlineto{\pgfqpoint{3.872150in}{0.944130in}}%
\pgfpathlineto{\pgfqpoint{3.863806in}{0.936399in}}%
\pgfpathlineto{\pgfqpoint{3.855028in}{0.943886in}}%
\pgfpathlineto{\pgfqpoint{3.846655in}{0.951712in}}%
\pgfpathlineto{\pgfqpoint{3.838721in}{0.959845in}}%
\pgfpathlineto{\pgfqpoint{3.831259in}{0.968250in}}%
\pgfpathlineto{\pgfqpoint{3.824299in}{0.976894in}}%
\pgfpathlineto{\pgfqpoint{3.831263in}{0.983425in}}%
\pgfpathlineto{\pgfqpoint{3.837386in}{0.990167in}}%
\pgfpathlineto{\pgfqpoint{3.842641in}{0.997093in}}%
\pgfpathlineto{\pgfqpoint{3.847006in}{1.004177in}}%
\pgfpathlineto{\pgfqpoint{3.850459in}{1.011390in}}%
\pgfpathclose%
\pgfusepath{fill}%
\end{pgfscope}%
\begin{pgfscope}%
\pgfpathrectangle{\pgfqpoint{2.548318in}{0.050000in}}{\pgfqpoint{2.081932in}{2.081932in}}%
\pgfusepath{clip}%
\pgfsetbuttcap%
\pgfsetroundjoin%
\definecolor{currentfill}{rgb}{0.120638,0.625828,0.533488}%
\pgfsetfillcolor{currentfill}%
\pgfsetlinewidth{0.000000pt}%
\definecolor{currentstroke}{rgb}{0.000000,0.000000,0.000000}%
\pgfsetstrokecolor{currentstroke}%
\pgfsetdash{}{0pt}%
\pgfpathmoveto{\pgfqpoint{3.818495in}{1.171905in}}%
\pgfpathlineto{\pgfqpoint{3.815498in}{1.163162in}}%
\pgfpathlineto{\pgfqpoint{3.813260in}{1.154340in}}%
\pgfpathlineto{\pgfqpoint{3.811791in}{1.145474in}}%
\pgfpathlineto{\pgfqpoint{3.811099in}{1.136601in}}%
\pgfpathlineto{\pgfqpoint{3.811187in}{1.127754in}}%
\pgfpathlineto{\pgfqpoint{3.812280in}{1.121583in}}%
\pgfpathlineto{\pgfqpoint{3.812586in}{1.115397in}}%
\pgfpathlineto{\pgfqpoint{3.812107in}{1.109219in}}%
\pgfpathlineto{\pgfqpoint{3.810846in}{1.103074in}}%
\pgfpathlineto{\pgfqpoint{3.808812in}{1.096988in}}%
\pgfpathlineto{\pgfqpoint{3.808725in}{1.105807in}}%
\pgfpathlineto{\pgfqpoint{3.809406in}{1.114529in}}%
\pgfpathlineto{\pgfqpoint{3.810849in}{1.123119in}}%
\pgfpathlineto{\pgfqpoint{3.813049in}{1.131543in}}%
\pgfpathlineto{\pgfqpoint{3.815995in}{1.139767in}}%
\pgfpathlineto{\pgfqpoint{3.818113in}{1.146124in}}%
\pgfpathlineto{\pgfqpoint{3.819428in}{1.152542in}}%
\pgfpathlineto{\pgfqpoint{3.819932in}{1.158996in}}%
\pgfpathlineto{\pgfqpoint{3.819622in}{1.165459in}}%
\pgfpathlineto{\pgfqpoint{3.818495in}{1.171905in}}%
\pgfpathclose%
\pgfusepath{fill}%
\end{pgfscope}%
\begin{pgfscope}%
\pgfpathrectangle{\pgfqpoint{2.548318in}{0.050000in}}{\pgfqpoint{2.081932in}{2.081932in}}%
\pgfusepath{clip}%
\pgfsetbuttcap%
\pgfsetroundjoin%
\definecolor{currentfill}{rgb}{0.227802,0.326594,0.546532}%
\pgfsetfillcolor{currentfill}%
\pgfsetlinewidth{0.000000pt}%
\definecolor{currentstroke}{rgb}{0.000000,0.000000,0.000000}%
\pgfsetstrokecolor{currentstroke}%
\pgfsetdash{}{0pt}%
\pgfpathmoveto{\pgfqpoint{3.429630in}{1.029620in}}%
\pgfpathlineto{\pgfqpoint{3.425397in}{1.020490in}}%
\pgfpathlineto{\pgfqpoint{3.420516in}{1.011436in}}%
\pgfpathlineto{\pgfqpoint{3.415004in}{1.002496in}}%
\pgfpathlineto{\pgfqpoint{3.408885in}{0.993707in}}%
\pgfpathlineto{\pgfqpoint{3.402183in}{0.985105in}}%
\pgfpathlineto{\pgfqpoint{3.396273in}{0.991896in}}%
\pgfpathlineto{\pgfqpoint{3.391236in}{0.998864in}}%
\pgfpathlineto{\pgfqpoint{3.387096in}{1.005983in}}%
\pgfpathlineto{\pgfqpoint{3.383871in}{1.013225in}}%
\pgfpathlineto{\pgfqpoint{3.381578in}{1.020561in}}%
\pgfpathlineto{\pgfqpoint{3.388961in}{1.028100in}}%
\pgfpathlineto{\pgfqpoint{3.395700in}{1.035921in}}%
\pgfpathlineto{\pgfqpoint{3.401768in}{1.043992in}}%
\pgfpathlineto{\pgfqpoint{3.407141in}{1.052280in}}%
\pgfpathlineto{\pgfqpoint{3.411798in}{1.060751in}}%
\pgfpathlineto{\pgfqpoint{3.413767in}{1.054313in}}%
\pgfpathlineto{\pgfqpoint{3.416550in}{1.047957in}}%
\pgfpathlineto{\pgfqpoint{3.420133in}{1.041706in}}%
\pgfpathlineto{\pgfqpoint{3.424499in}{1.035586in}}%
\pgfpathlineto{\pgfqpoint{3.429630in}{1.029620in}}%
\pgfpathclose%
\pgfusepath{fill}%
\end{pgfscope}%
\begin{pgfscope}%
\pgfpathrectangle{\pgfqpoint{2.548318in}{0.050000in}}{\pgfqpoint{2.081932in}{2.081932in}}%
\pgfusepath{clip}%
\pgfsetbuttcap%
\pgfsetroundjoin%
\definecolor{currentfill}{rgb}{0.327796,0.773980,0.406640}%
\pgfsetfillcolor{currentfill}%
\pgfsetlinewidth{0.000000pt}%
\definecolor{currentstroke}{rgb}{0.000000,0.000000,0.000000}%
\pgfsetstrokecolor{currentstroke}%
\pgfsetdash{}{0pt}%
\pgfpathmoveto{\pgfqpoint{3.049451in}{1.132819in}}%
\pgfpathlineto{\pgfqpoint{3.053850in}{1.141273in}}%
\pgfpathlineto{\pgfqpoint{3.058956in}{1.149551in}}%
\pgfpathlineto{\pgfqpoint{3.064747in}{1.157620in}}%
\pgfpathlineto{\pgfqpoint{3.071200in}{1.165450in}}%
\pgfpathlineto{\pgfqpoint{3.078289in}{1.173008in}}%
\pgfpathlineto{\pgfqpoint{3.074399in}{1.190353in}}%
\pgfpathlineto{\pgfqpoint{3.072677in}{1.207833in}}%
\pgfpathlineto{\pgfqpoint{3.073149in}{1.225381in}}%
\pgfpathlineto{\pgfqpoint{3.075828in}{1.242924in}}%
\pgfpathlineto{\pgfqpoint{3.080723in}{1.260390in}}%
\pgfpathlineto{\pgfqpoint{3.073570in}{1.253894in}}%
\pgfpathlineto{\pgfqpoint{3.067057in}{1.247017in}}%
\pgfpathlineto{\pgfqpoint{3.061210in}{1.239785in}}%
\pgfpathlineto{\pgfqpoint{3.056055in}{1.232228in}}%
\pgfpathlineto{\pgfqpoint{3.051611in}{1.224376in}}%
\pgfpathlineto{\pgfqpoint{3.046533in}{1.206064in}}%
\pgfpathlineto{\pgfqpoint{3.043791in}{1.187678in}}%
\pgfpathlineto{\pgfqpoint{3.043376in}{1.169292in}}%
\pgfpathlineto{\pgfqpoint{3.045272in}{1.150982in}}%
\pgfpathlineto{\pgfqpoint{3.049451in}{1.132819in}}%
\pgfpathclose%
\pgfusepath{fill}%
\end{pgfscope}%
\begin{pgfscope}%
\pgfpathrectangle{\pgfqpoint{2.548318in}{0.050000in}}{\pgfqpoint{2.081932in}{2.081932in}}%
\pgfusepath{clip}%
\pgfsetbuttcap%
\pgfsetroundjoin%
\definecolor{currentfill}{rgb}{0.267004,0.004874,0.329415}%
\pgfsetfillcolor{currentfill}%
\pgfsetlinewidth{0.000000pt}%
\definecolor{currentstroke}{rgb}{0.000000,0.000000,0.000000}%
\pgfsetstrokecolor{currentstroke}%
\pgfsetdash{}{0pt}%
\pgfpathmoveto{\pgfqpoint{3.310233in}{0.916662in}}%
\pgfpathlineto{\pgfqpoint{3.299388in}{0.912199in}}%
\pgfpathlineto{\pgfqpoint{3.288413in}{0.908259in}}%
\pgfpathlineto{\pgfqpoint{3.277354in}{0.904858in}}%
\pgfpathlineto{\pgfqpoint{3.266257in}{0.902010in}}%
\pgfpathlineto{\pgfqpoint{3.255167in}{0.899725in}}%
\pgfpathlineto{\pgfqpoint{3.244950in}{0.911006in}}%
\pgfpathlineto{\pgfqpoint{3.236184in}{0.922600in}}%
\pgfpathlineto{\pgfqpoint{3.228909in}{0.934463in}}%
\pgfpathlineto{\pgfqpoint{3.223164in}{0.946548in}}%
\pgfpathlineto{\pgfqpoint{3.218979in}{0.958811in}}%
\pgfpathlineto{\pgfqpoint{3.231288in}{0.959292in}}%
\pgfpathlineto{\pgfqpoint{3.243598in}{0.960340in}}%
\pgfpathlineto{\pgfqpoint{3.255858in}{0.961951in}}%
\pgfpathlineto{\pgfqpoint{3.268018in}{0.964118in}}%
\pgfpathlineto{\pgfqpoint{3.280028in}{0.966833in}}%
\pgfpathlineto{\pgfqpoint{3.283473in}{0.956432in}}%
\pgfpathlineto{\pgfqpoint{3.288243in}{0.946176in}}%
\pgfpathlineto{\pgfqpoint{3.294314in}{0.936103in}}%
\pgfpathlineto{\pgfqpoint{3.301656in}{0.926252in}}%
\pgfpathlineto{\pgfqpoint{3.310233in}{0.916662in}}%
\pgfpathclose%
\pgfusepath{fill}%
\end{pgfscope}%
\begin{pgfscope}%
\pgfpathrectangle{\pgfqpoint{2.548318in}{0.050000in}}{\pgfqpoint{2.081932in}{2.081932in}}%
\pgfusepath{clip}%
\pgfsetbuttcap%
\pgfsetroundjoin%
\definecolor{currentfill}{rgb}{0.150476,0.504369,0.557430}%
\pgfsetfillcolor{currentfill}%
\pgfsetlinewidth{0.000000pt}%
\definecolor{currentstroke}{rgb}{0.000000,0.000000,0.000000}%
\pgfsetstrokecolor{currentstroke}%
\pgfsetdash{}{0pt}%
\pgfpathmoveto{\pgfqpoint{4.195224in}{1.122983in}}%
\pgfpathlineto{\pgfqpoint{4.197703in}{1.131848in}}%
\pgfpathlineto{\pgfqpoint{4.199403in}{1.140759in}}%
\pgfpathlineto{\pgfqpoint{4.200318in}{1.149681in}}%
\pgfpathlineto{\pgfqpoint{4.200446in}{1.158578in}}%
\pgfpathlineto{\pgfqpoint{4.202620in}{1.139953in}}%
\pgfpathlineto{\pgfqpoint{4.202424in}{1.121350in}}%
\pgfpathlineto{\pgfqpoint{4.199879in}{1.102843in}}%
\pgfpathlineto{\pgfqpoint{4.195017in}{1.084506in}}%
\pgfpathlineto{\pgfqpoint{4.187876in}{1.066411in}}%
\pgfpathlineto{\pgfqpoint{4.187752in}{1.057659in}}%
\pgfpathlineto{\pgfqpoint{4.186869in}{1.049005in}}%
\pgfpathlineto{\pgfqpoint{4.185229in}{1.040483in}}%
\pgfpathlineto{\pgfqpoint{4.182837in}{1.032127in}}%
\pgfpathlineto{\pgfqpoint{4.189901in}{1.049967in}}%
\pgfpathlineto{\pgfqpoint{4.194706in}{1.068044in}}%
\pgfpathlineto{\pgfqpoint{4.197213in}{1.086287in}}%
\pgfpathlineto{\pgfqpoint{4.197392in}{1.104625in}}%
\pgfpathlineto{\pgfqpoint{4.195224in}{1.122983in}}%
\pgfpathclose%
\pgfusepath{fill}%
\end{pgfscope}%
\begin{pgfscope}%
\pgfpathrectangle{\pgfqpoint{2.548318in}{0.050000in}}{\pgfqpoint{2.081932in}{2.081932in}}%
\pgfusepath{clip}%
\pgfsetbuttcap%
\pgfsetroundjoin%
\definecolor{currentfill}{rgb}{0.120638,0.625828,0.533488}%
\pgfsetfillcolor{currentfill}%
\pgfsetlinewidth{0.000000pt}%
\definecolor{currentstroke}{rgb}{0.000000,0.000000,0.000000}%
\pgfsetstrokecolor{currentstroke}%
\pgfsetdash{}{0pt}%
\pgfpathmoveto{\pgfqpoint{3.416529in}{1.147737in}}%
\pgfpathlineto{\pgfqpoint{3.419516in}{1.139384in}}%
\pgfpathlineto{\pgfqpoint{3.421746in}{1.130862in}}%
\pgfpathlineto{\pgfqpoint{3.423210in}{1.122204in}}%
\pgfpathlineto{\pgfqpoint{3.423900in}{1.113444in}}%
\pgfpathlineto{\pgfqpoint{3.423812in}{1.104618in}}%
\pgfpathlineto{\pgfqpoint{3.422745in}{1.110773in}}%
\pgfpathlineto{\pgfqpoint{3.422461in}{1.116956in}}%
\pgfpathlineto{\pgfqpoint{3.422962in}{1.123141in}}%
\pgfpathlineto{\pgfqpoint{3.424250in}{1.129303in}}%
\pgfpathlineto{\pgfqpoint{3.426321in}{1.135418in}}%
\pgfpathlineto{\pgfqpoint{3.426408in}{1.144271in}}%
\pgfpathlineto{\pgfqpoint{3.425723in}{1.153183in}}%
\pgfpathlineto{\pgfqpoint{3.424272in}{1.162118in}}%
\pgfpathlineto{\pgfqpoint{3.422059in}{1.171039in}}%
\pgfpathlineto{\pgfqpoint{3.419096in}{1.179913in}}%
\pgfpathlineto{\pgfqpoint{3.416954in}{1.173524in}}%
\pgfpathlineto{\pgfqpoint{3.415625in}{1.167086in}}%
\pgfpathlineto{\pgfqpoint{3.415112in}{1.160624in}}%
\pgfpathlineto{\pgfqpoint{3.415415in}{1.154166in}}%
\pgfpathlineto{\pgfqpoint{3.416529in}{1.147737in}}%
\pgfpathclose%
\pgfusepath{fill}%
\end{pgfscope}%
\begin{pgfscope}%
\pgfpathrectangle{\pgfqpoint{2.548318in}{0.050000in}}{\pgfqpoint{2.081932in}{2.081932in}}%
\pgfusepath{clip}%
\pgfsetbuttcap%
\pgfsetroundjoin%
\definecolor{currentfill}{rgb}{0.278012,0.180367,0.486697}%
\pgfsetfillcolor{currentfill}%
\pgfsetlinewidth{0.000000pt}%
\definecolor{currentstroke}{rgb}{0.000000,0.000000,0.000000}%
\pgfsetstrokecolor{currentstroke}%
\pgfsetdash{}{0pt}%
\pgfpathmoveto{\pgfqpoint{3.402183in}{0.985105in}}%
\pgfpathlineto{\pgfqpoint{3.394926in}{0.976727in}}%
\pgfpathlineto{\pgfqpoint{3.387143in}{0.968606in}}%
\pgfpathlineto{\pgfqpoint{3.378868in}{0.960776in}}%
\pgfpathlineto{\pgfqpoint{3.370134in}{0.953270in}}%
\pgfpathlineto{\pgfqpoint{3.360977in}{0.946119in}}%
\pgfpathlineto{\pgfqpoint{3.353882in}{0.954161in}}%
\pgfpathlineto{\pgfqpoint{3.347824in}{0.962417in}}%
\pgfpathlineto{\pgfqpoint{3.342830in}{0.970855in}}%
\pgfpathlineto{\pgfqpoint{3.338926in}{0.979443in}}%
\pgfpathlineto{\pgfqpoint{3.336130in}{0.988146in}}%
\pgfpathlineto{\pgfqpoint{3.346238in}{0.993833in}}%
\pgfpathlineto{\pgfqpoint{3.355875in}{0.999944in}}%
\pgfpathlineto{\pgfqpoint{3.365002in}{1.006454in}}%
\pgfpathlineto{\pgfqpoint{3.373581in}{1.013336in}}%
\pgfpathlineto{\pgfqpoint{3.381578in}{1.020561in}}%
\pgfpathlineto{\pgfqpoint{3.383871in}{1.013225in}}%
\pgfpathlineto{\pgfqpoint{3.387096in}{1.005983in}}%
\pgfpathlineto{\pgfqpoint{3.391236in}{0.998864in}}%
\pgfpathlineto{\pgfqpoint{3.396273in}{0.991896in}}%
\pgfpathlineto{\pgfqpoint{3.402183in}{0.985105in}}%
\pgfpathclose%
\pgfusepath{fill}%
\end{pgfscope}%
\begin{pgfscope}%
\pgfpathrectangle{\pgfqpoint{2.548318in}{0.050000in}}{\pgfqpoint{2.081932in}{2.081932in}}%
\pgfusepath{clip}%
\pgfsetbuttcap%
\pgfsetroundjoin%
\definecolor{currentfill}{rgb}{0.278791,0.062145,0.386592}%
\pgfsetfillcolor{currentfill}%
\pgfsetlinewidth{0.000000pt}%
\definecolor{currentstroke}{rgb}{0.000000,0.000000,0.000000}%
\pgfsetstrokecolor{currentstroke}%
\pgfsetdash{}{0pt}%
\pgfpathmoveto{\pgfqpoint{3.360977in}{0.946119in}}%
\pgfpathlineto{\pgfqpoint{3.351435in}{0.939353in}}%
\pgfpathlineto{\pgfqpoint{3.341548in}{0.932999in}}%
\pgfpathlineto{\pgfqpoint{3.331357in}{0.927083in}}%
\pgfpathlineto{\pgfqpoint{3.320905in}{0.921630in}}%
\pgfpathlineto{\pgfqpoint{3.310233in}{0.916662in}}%
\pgfpathlineto{\pgfqpoint{3.301656in}{0.926252in}}%
\pgfpathlineto{\pgfqpoint{3.294314in}{0.936103in}}%
\pgfpathlineto{\pgfqpoint{3.288243in}{0.946176in}}%
\pgfpathlineto{\pgfqpoint{3.283473in}{0.956432in}}%
\pgfpathlineto{\pgfqpoint{3.280028in}{0.966833in}}%
\pgfpathlineto{\pgfqpoint{3.291838in}{0.970083in}}%
\pgfpathlineto{\pgfqpoint{3.303401in}{0.973857in}}%
\pgfpathlineto{\pgfqpoint{3.314667in}{0.978137in}}%
\pgfpathlineto{\pgfqpoint{3.325592in}{0.982907in}}%
\pgfpathlineto{\pgfqpoint{3.336130in}{0.988146in}}%
\pgfpathlineto{\pgfqpoint{3.338926in}{0.979443in}}%
\pgfpathlineto{\pgfqpoint{3.342830in}{0.970855in}}%
\pgfpathlineto{\pgfqpoint{3.347824in}{0.962417in}}%
\pgfpathlineto{\pgfqpoint{3.353882in}{0.954161in}}%
\pgfpathlineto{\pgfqpoint{3.360977in}{0.946119in}}%
\pgfpathclose%
\pgfusepath{fill}%
\end{pgfscope}%
\begin{pgfscope}%
\pgfpathrectangle{\pgfqpoint{2.548318in}{0.050000in}}{\pgfqpoint{2.081932in}{2.081932in}}%
\pgfusepath{clip}%
\pgfsetbuttcap%
\pgfsetroundjoin%
\definecolor{currentfill}{rgb}{0.855810,0.888601,0.097452}%
\pgfsetfillcolor{currentfill}%
\pgfsetlinewidth{0.000000pt}%
\definecolor{currentstroke}{rgb}{0.000000,0.000000,0.000000}%
\pgfsetstrokecolor{currentstroke}%
\pgfsetdash{}{0pt}%
\pgfpathmoveto{\pgfqpoint{3.912043in}{1.322702in}}%
\pgfpathlineto{\pgfqpoint{3.901543in}{1.316956in}}%
\pgfpathlineto{\pgfqpoint{3.891331in}{1.310767in}}%
\pgfpathlineto{\pgfqpoint{3.881447in}{1.304160in}}%
\pgfpathlineto{\pgfqpoint{3.871931in}{1.297161in}}%
\pgfpathlineto{\pgfqpoint{3.862819in}{1.289796in}}%
\pgfpathlineto{\pgfqpoint{3.869577in}{1.281699in}}%
\pgfpathlineto{\pgfqpoint{3.875297in}{1.273410in}}%
\pgfpathlineto{\pgfqpoint{3.879959in}{1.264962in}}%
\pgfpathlineto{\pgfqpoint{3.883548in}{1.256391in}}%
\pgfpathlineto{\pgfqpoint{3.886053in}{1.247733in}}%
\pgfpathlineto{\pgfqpoint{3.895963in}{1.253496in}}%
\pgfpathlineto{\pgfqpoint{3.906308in}{1.258826in}}%
\pgfpathlineto{\pgfqpoint{3.917046in}{1.263701in}}%
\pgfpathlineto{\pgfqpoint{3.928136in}{1.268102in}}%
\pgfpathlineto{\pgfqpoint{3.939532in}{1.272012in}}%
\pgfpathlineto{\pgfqpoint{3.936611in}{1.282435in}}%
\pgfpathlineto{\pgfqpoint{3.932387in}{1.292759in}}%
\pgfpathlineto{\pgfqpoint{3.926873in}{1.302939in}}%
\pgfpathlineto{\pgfqpoint{3.920084in}{1.312934in}}%
\pgfpathlineto{\pgfqpoint{3.912043in}{1.322702in}}%
\pgfpathclose%
\pgfusepath{fill}%
\end{pgfscope}%
\begin{pgfscope}%
\pgfpathrectangle{\pgfqpoint{2.548318in}{0.050000in}}{\pgfqpoint{2.081932in}{2.081932in}}%
\pgfusepath{clip}%
\pgfsetbuttcap%
\pgfsetroundjoin%
\definecolor{currentfill}{rgb}{0.606045,0.850733,0.236712}%
\pgfsetfillcolor{currentfill}%
\pgfsetlinewidth{0.000000pt}%
\definecolor{currentstroke}{rgb}{0.000000,0.000000,0.000000}%
\pgfsetstrokecolor{currentstroke}%
\pgfsetdash{}{0pt}%
\pgfpathmoveto{\pgfqpoint{3.862819in}{1.289796in}}%
\pgfpathlineto{\pgfqpoint{3.854146in}{1.282096in}}%
\pgfpathlineto{\pgfqpoint{3.845946in}{1.274090in}}%
\pgfpathlineto{\pgfqpoint{3.838252in}{1.265809in}}%
\pgfpathlineto{\pgfqpoint{3.831094in}{1.257286in}}%
\pgfpathlineto{\pgfqpoint{3.824500in}{1.248554in}}%
\pgfpathlineto{\pgfqpoint{3.830242in}{1.241763in}}%
\pgfpathlineto{\pgfqpoint{3.835112in}{1.234808in}}%
\pgfpathlineto{\pgfqpoint{3.839091in}{1.227717in}}%
\pgfpathlineto{\pgfqpoint{3.842166in}{1.220520in}}%
\pgfpathlineto{\pgfqpoint{3.844328in}{1.213247in}}%
\pgfpathlineto{\pgfqpoint{3.851514in}{1.220812in}}%
\pgfpathlineto{\pgfqpoint{3.859313in}{1.228070in}}%
\pgfpathlineto{\pgfqpoint{3.867691in}{1.234995in}}%
\pgfpathlineto{\pgfqpoint{3.876616in}{1.241557in}}%
\pgfpathlineto{\pgfqpoint{3.886053in}{1.247733in}}%
\pgfpathlineto{\pgfqpoint{3.883548in}{1.256391in}}%
\pgfpathlineto{\pgfqpoint{3.879959in}{1.264962in}}%
\pgfpathlineto{\pgfqpoint{3.875297in}{1.273410in}}%
\pgfpathlineto{\pgfqpoint{3.869577in}{1.281699in}}%
\pgfpathlineto{\pgfqpoint{3.862819in}{1.289796in}}%
\pgfpathclose%
\pgfusepath{fill}%
\end{pgfscope}%
\begin{pgfscope}%
\pgfpathrectangle{\pgfqpoint{2.548318in}{0.050000in}}{\pgfqpoint{2.081932in}{2.081932in}}%
\pgfusepath{clip}%
\pgfsetbuttcap%
\pgfsetroundjoin%
\definecolor{currentfill}{rgb}{0.993248,0.906157,0.143936}%
\pgfsetfillcolor{currentfill}%
\pgfsetlinewidth{0.000000pt}%
\definecolor{currentstroke}{rgb}{0.000000,0.000000,0.000000}%
\pgfsetstrokecolor{currentstroke}%
\pgfsetdash{}{0pt}%
\pgfpathmoveto{\pgfqpoint{3.967409in}{1.344052in}}%
\pgfpathlineto{\pgfqpoint{3.956098in}{1.340826in}}%
\pgfpathlineto{\pgfqpoint{3.944862in}{1.337062in}}%
\pgfpathlineto{\pgfqpoint{3.933745in}{1.332775in}}%
\pgfpathlineto{\pgfqpoint{3.922791in}{1.327982in}}%
\pgfpathlineto{\pgfqpoint{3.912043in}{1.322702in}}%
\pgfpathlineto{\pgfqpoint{3.920084in}{1.312934in}}%
\pgfpathlineto{\pgfqpoint{3.926873in}{1.302939in}}%
\pgfpathlineto{\pgfqpoint{3.932387in}{1.292759in}}%
\pgfpathlineto{\pgfqpoint{3.936611in}{1.282435in}}%
\pgfpathlineto{\pgfqpoint{3.939532in}{1.272012in}}%
\pgfpathlineto{\pgfqpoint{3.951192in}{1.275415in}}%
\pgfpathlineto{\pgfqpoint{3.963067in}{1.278299in}}%
\pgfpathlineto{\pgfqpoint{3.975113in}{1.280651in}}%
\pgfpathlineto{\pgfqpoint{3.987281in}{1.282463in}}%
\pgfpathlineto{\pgfqpoint{3.999523in}{1.283727in}}%
\pgfpathlineto{\pgfqpoint{3.996168in}{1.296118in}}%
\pgfpathlineto{\pgfqpoint{3.991264in}{1.308396in}}%
\pgfpathlineto{\pgfqpoint{3.984823in}{1.320512in}}%
\pgfpathlineto{\pgfqpoint{3.976863in}{1.332414in}}%
\pgfpathlineto{\pgfqpoint{3.967409in}{1.344052in}}%
\pgfpathclose%
\pgfusepath{fill}%
\end{pgfscope}%
\begin{pgfscope}%
\pgfpathrectangle{\pgfqpoint{2.548318in}{0.050000in}}{\pgfqpoint{2.081932in}{2.081932in}}%
\pgfusepath{clip}%
\pgfsetbuttcap%
\pgfsetroundjoin%
\definecolor{currentfill}{rgb}{0.162142,0.474838,0.558140}%
\pgfsetfillcolor{currentfill}%
\pgfsetlinewidth{0.000000pt}%
\definecolor{currentstroke}{rgb}{0.000000,0.000000,0.000000}%
\pgfsetstrokecolor{currentstroke}%
\pgfsetdash{}{0pt}%
\pgfpathmoveto{\pgfqpoint{3.811187in}{1.127754in}}%
\pgfpathlineto{\pgfqpoint{3.812056in}{1.118970in}}%
\pgfpathlineto{\pgfqpoint{3.813703in}{1.110284in}}%
\pgfpathlineto{\pgfqpoint{3.816124in}{1.101731in}}%
\pgfpathlineto{\pgfqpoint{3.819308in}{1.093345in}}%
\pgfpathlineto{\pgfqpoint{3.823245in}{1.085161in}}%
\pgfpathlineto{\pgfqpoint{3.824394in}{1.078650in}}%
\pgfpathlineto{\pgfqpoint{3.824707in}{1.072122in}}%
\pgfpathlineto{\pgfqpoint{3.824185in}{1.065604in}}%
\pgfpathlineto{\pgfqpoint{3.822835in}{1.059123in}}%
\pgfpathlineto{\pgfqpoint{3.820662in}{1.052703in}}%
\pgfpathlineto{\pgfqpoint{3.816794in}{1.061465in}}%
\pgfpathlineto{\pgfqpoint{3.813664in}{1.070310in}}%
\pgfpathlineto{\pgfqpoint{3.811285in}{1.079204in}}%
\pgfpathlineto{\pgfqpoint{3.809666in}{1.088108in}}%
\pgfpathlineto{\pgfqpoint{3.808812in}{1.096988in}}%
\pgfpathlineto{\pgfqpoint{3.810846in}{1.103074in}}%
\pgfpathlineto{\pgfqpoint{3.812107in}{1.109219in}}%
\pgfpathlineto{\pgfqpoint{3.812586in}{1.115397in}}%
\pgfpathlineto{\pgfqpoint{3.812280in}{1.121583in}}%
\pgfpathlineto{\pgfqpoint{3.811187in}{1.127754in}}%
\pgfpathclose%
\pgfusepath{fill}%
\end{pgfscope}%
\begin{pgfscope}%
\pgfpathrectangle{\pgfqpoint{2.548318in}{0.050000in}}{\pgfqpoint{2.081932in}{2.081932in}}%
\pgfusepath{clip}%
\pgfsetbuttcap%
\pgfsetroundjoin%
\definecolor{currentfill}{rgb}{0.124780,0.640461,0.527068}%
\pgfsetfillcolor{currentfill}%
\pgfsetlinewidth{0.000000pt}%
\definecolor{currentstroke}{rgb}{0.000000,0.000000,0.000000}%
\pgfsetstrokecolor{currentstroke}%
\pgfsetdash{}{0pt}%
\pgfpathmoveto{\pgfqpoint{3.038600in}{1.089107in}}%
\pgfpathlineto{\pgfqpoint{3.039247in}{1.097926in}}%
\pgfpathlineto{\pgfqpoint{3.040664in}{1.106742in}}%
\pgfpathlineto{\pgfqpoint{3.042844in}{1.115520in}}%
\pgfpathlineto{\pgfqpoint{3.045778in}{1.124224in}}%
\pgfpathlineto{\pgfqpoint{3.049451in}{1.132819in}}%
\pgfpathlineto{\pgfqpoint{3.045272in}{1.150982in}}%
\pgfpathlineto{\pgfqpoint{3.043376in}{1.169292in}}%
\pgfpathlineto{\pgfqpoint{3.043791in}{1.187678in}}%
\pgfpathlineto{\pgfqpoint{3.046533in}{1.206064in}}%
\pgfpathlineto{\pgfqpoint{3.051611in}{1.224376in}}%
\pgfpathlineto{\pgfqpoint{3.047900in}{1.216259in}}%
\pgfpathlineto{\pgfqpoint{3.044936in}{1.207910in}}%
\pgfpathlineto{\pgfqpoint{3.042733in}{1.199363in}}%
\pgfpathlineto{\pgfqpoint{3.041300in}{1.190651in}}%
\pgfpathlineto{\pgfqpoint{3.040646in}{1.181811in}}%
\pgfpathlineto{\pgfqpoint{3.035500in}{1.163266in}}%
\pgfpathlineto{\pgfqpoint{3.032737in}{1.144647in}}%
\pgfpathlineto{\pgfqpoint{3.032346in}{1.126031in}}%
\pgfpathlineto{\pgfqpoint{3.034310in}{1.107493in}}%
\pgfpathlineto{\pgfqpoint{3.038600in}{1.089107in}}%
\pgfpathclose%
\pgfusepath{fill}%
\end{pgfscope}%
\begin{pgfscope}%
\pgfpathrectangle{\pgfqpoint{2.548318in}{0.050000in}}{\pgfqpoint{2.081932in}{2.081932in}}%
\pgfusepath{clip}%
\pgfsetbuttcap%
\pgfsetroundjoin%
\definecolor{currentfill}{rgb}{0.296479,0.761561,0.424223}%
\pgfsetfillcolor{currentfill}%
\pgfsetlinewidth{0.000000pt}%
\definecolor{currentstroke}{rgb}{0.000000,0.000000,0.000000}%
\pgfsetstrokecolor{currentstroke}%
\pgfsetdash{}{0pt}%
\pgfpathmoveto{\pgfqpoint{3.824500in}{1.248554in}}%
\pgfpathlineto{\pgfqpoint{3.818495in}{1.239648in}}%
\pgfpathlineto{\pgfqpoint{3.813104in}{1.230601in}}%
\pgfpathlineto{\pgfqpoint{3.808347in}{1.221450in}}%
\pgfpathlineto{\pgfqpoint{3.804245in}{1.212229in}}%
\pgfpathlineto{\pgfqpoint{3.800814in}{1.202976in}}%
\pgfpathlineto{\pgfqpoint{3.805921in}{1.197003in}}%
\pgfpathlineto{\pgfqpoint{3.810257in}{1.190884in}}%
\pgfpathlineto{\pgfqpoint{3.813806in}{1.184644in}}%
\pgfpathlineto{\pgfqpoint{3.816555in}{1.178309in}}%
\pgfpathlineto{\pgfqpoint{3.818495in}{1.171905in}}%
\pgfpathlineto{\pgfqpoint{3.822240in}{1.180535in}}%
\pgfpathlineto{\pgfqpoint{3.826715in}{1.189017in}}%
\pgfpathlineto{\pgfqpoint{3.831903in}{1.197318in}}%
\pgfpathlineto{\pgfqpoint{3.837782in}{1.205405in}}%
\pgfpathlineto{\pgfqpoint{3.844328in}{1.213247in}}%
\pgfpathlineto{\pgfqpoint{3.842166in}{1.220520in}}%
\pgfpathlineto{\pgfqpoint{3.839091in}{1.227717in}}%
\pgfpathlineto{\pgfqpoint{3.835112in}{1.234808in}}%
\pgfpathlineto{\pgfqpoint{3.830242in}{1.241763in}}%
\pgfpathlineto{\pgfqpoint{3.824500in}{1.248554in}}%
\pgfpathclose%
\pgfusepath{fill}%
\end{pgfscope}%
\begin{pgfscope}%
\pgfpathrectangle{\pgfqpoint{2.548318in}{0.050000in}}{\pgfqpoint{2.081932in}{2.081932in}}%
\pgfusepath{clip}%
\pgfsetbuttcap%
\pgfsetroundjoin%
\definecolor{currentfill}{rgb}{0.206756,0.371758,0.553117}%
\pgfsetfillcolor{currentfill}%
\pgfsetlinewidth{0.000000pt}%
\definecolor{currentstroke}{rgb}{0.000000,0.000000,0.000000}%
\pgfsetstrokecolor{currentstroke}%
\pgfsetdash{}{0pt}%
\pgfpathmoveto{\pgfqpoint{4.171575in}{1.080619in}}%
\pgfpathlineto{\pgfqpoint{4.177749in}{1.088714in}}%
\pgfpathlineto{\pgfqpoint{4.183220in}{1.097032in}}%
\pgfpathlineto{\pgfqpoint{4.187969in}{1.105540in}}%
\pgfpathlineto{\pgfqpoint{4.191975in}{1.114202in}}%
\pgfpathlineto{\pgfqpoint{4.195224in}{1.122983in}}%
\pgfpathlineto{\pgfqpoint{4.197392in}{1.104625in}}%
\pgfpathlineto{\pgfqpoint{4.197213in}{1.086287in}}%
\pgfpathlineto{\pgfqpoint{4.194706in}{1.068044in}}%
\pgfpathlineto{\pgfqpoint{4.189901in}{1.049967in}}%
\pgfpathlineto{\pgfqpoint{4.182837in}{1.032127in}}%
\pgfpathlineto{\pgfqpoint{4.179702in}{1.023972in}}%
\pgfpathlineto{\pgfqpoint{4.175836in}{1.016050in}}%
\pgfpathlineto{\pgfqpoint{4.171252in}{1.008393in}}%
\pgfpathlineto{\pgfqpoint{4.165969in}{1.001034in}}%
\pgfpathlineto{\pgfqpoint{4.160006in}{0.994001in}}%
\pgfpathlineto{\pgfqpoint{4.166723in}{1.011017in}}%
\pgfpathlineto{\pgfqpoint{4.171270in}{1.028255in}}%
\pgfpathlineto{\pgfqpoint{4.173611in}{1.045647in}}%
\pgfpathlineto{\pgfqpoint{4.173719in}{1.063125in}}%
\pgfpathlineto{\pgfqpoint{4.171575in}{1.080619in}}%
\pgfpathclose%
\pgfusepath{fill}%
\end{pgfscope}%
\begin{pgfscope}%
\pgfpathrectangle{\pgfqpoint{2.548318in}{0.050000in}}{\pgfqpoint{2.081932in}{2.081932in}}%
\pgfusepath{clip}%
\pgfsetbuttcap%
\pgfsetroundjoin%
\definecolor{currentfill}{rgb}{0.855810,0.888601,0.097452}%
\pgfsetfillcolor{currentfill}%
\pgfsetlinewidth{0.000000pt}%
\definecolor{currentstroke}{rgb}{0.000000,0.000000,0.000000}%
\pgfsetstrokecolor{currentstroke}%
\pgfsetdash{}{0pt}%
\pgfpathmoveto{\pgfqpoint{3.299358in}{1.285048in}}%
\pgfpathlineto{\pgfqpoint{3.310640in}{1.280668in}}%
\pgfpathlineto{\pgfqpoint{3.321616in}{1.275810in}}%
\pgfpathlineto{\pgfqpoint{3.332243in}{1.270491in}}%
\pgfpathlineto{\pgfqpoint{3.342480in}{1.264734in}}%
\pgfpathlineto{\pgfqpoint{3.352285in}{1.258560in}}%
\pgfpathlineto{\pgfqpoint{3.356141in}{1.267103in}}%
\pgfpathlineto{\pgfqpoint{3.361068in}{1.275515in}}%
\pgfpathlineto{\pgfqpoint{3.367047in}{1.283759in}}%
\pgfpathlineto{\pgfqpoint{3.374060in}{1.291802in}}%
\pgfpathlineto{\pgfqpoint{3.382080in}{1.299609in}}%
\pgfpathlineto{\pgfqpoint{3.373320in}{1.307350in}}%
\pgfpathlineto{\pgfqpoint{3.364171in}{1.314742in}}%
\pgfpathlineto{\pgfqpoint{3.354668in}{1.321757in}}%
\pgfpathlineto{\pgfqpoint{3.344848in}{1.328367in}}%
\pgfpathlineto{\pgfqpoint{3.334749in}{1.334546in}}%
\pgfpathlineto{\pgfqpoint{3.325186in}{1.325122in}}%
\pgfpathlineto{\pgfqpoint{3.316838in}{1.315419in}}%
\pgfpathlineto{\pgfqpoint{3.309736in}{1.305477in}}%
\pgfpathlineto{\pgfqpoint{3.303903in}{1.295339in}}%
\pgfpathlineto{\pgfqpoint{3.299358in}{1.285048in}}%
\pgfpathclose%
\pgfusepath{fill}%
\end{pgfscope}%
\begin{pgfscope}%
\pgfpathrectangle{\pgfqpoint{2.548318in}{0.050000in}}{\pgfqpoint{2.081932in}{2.081932in}}%
\pgfusepath{clip}%
\pgfsetbuttcap%
\pgfsetroundjoin%
\definecolor{currentfill}{rgb}{0.606045,0.850733,0.236712}%
\pgfsetfillcolor{currentfill}%
\pgfsetlinewidth{0.000000pt}%
\definecolor{currentstroke}{rgb}{0.000000,0.000000,0.000000}%
\pgfsetstrokecolor{currentstroke}%
\pgfsetdash{}{0pt}%
\pgfpathmoveto{\pgfqpoint{3.352285in}{1.258560in}}%
\pgfpathlineto{\pgfqpoint{3.361621in}{1.251994in}}%
\pgfpathlineto{\pgfqpoint{3.370449in}{1.245062in}}%
\pgfpathlineto{\pgfqpoint{3.378736in}{1.237789in}}%
\pgfpathlineto{\pgfqpoint{3.386449in}{1.230206in}}%
\pgfpathlineto{\pgfqpoint{3.393555in}{1.222342in}}%
\pgfpathlineto{\pgfqpoint{3.396856in}{1.229515in}}%
\pgfpathlineto{\pgfqpoint{3.401058in}{1.236574in}}%
\pgfpathlineto{\pgfqpoint{3.406146in}{1.243491in}}%
\pgfpathlineto{\pgfqpoint{3.412102in}{1.250236in}}%
\pgfpathlineto{\pgfqpoint{3.418905in}{1.256782in}}%
\pgfpathlineto{\pgfqpoint{3.412569in}{1.265787in}}%
\pgfpathlineto{\pgfqpoint{3.405690in}{1.274606in}}%
\pgfpathlineto{\pgfqpoint{3.398297in}{1.283206in}}%
\pgfpathlineto{\pgfqpoint{3.390416in}{1.291551in}}%
\pgfpathlineto{\pgfqpoint{3.382080in}{1.299609in}}%
\pgfpathlineto{\pgfqpoint{3.374060in}{1.291802in}}%
\pgfpathlineto{\pgfqpoint{3.367047in}{1.283759in}}%
\pgfpathlineto{\pgfqpoint{3.361068in}{1.275515in}}%
\pgfpathlineto{\pgfqpoint{3.356141in}{1.267103in}}%
\pgfpathlineto{\pgfqpoint{3.352285in}{1.258560in}}%
\pgfpathclose%
\pgfusepath{fill}%
\end{pgfscope}%
\begin{pgfscope}%
\pgfpathrectangle{\pgfqpoint{2.548318in}{0.050000in}}{\pgfqpoint{2.081932in}{2.081932in}}%
\pgfusepath{clip}%
\pgfsetbuttcap%
\pgfsetroundjoin%
\definecolor{currentfill}{rgb}{0.162142,0.474838,0.558140}%
\pgfsetfillcolor{currentfill}%
\pgfsetlinewidth{0.000000pt}%
\definecolor{currentstroke}{rgb}{0.000000,0.000000,0.000000}%
\pgfsetstrokecolor{currentstroke}%
\pgfsetdash{}{0pt}%
\pgfpathmoveto{\pgfqpoint{3.423812in}{1.104618in}}%
\pgfpathlineto{\pgfqpoint{3.422947in}{1.095762in}}%
\pgfpathlineto{\pgfqpoint{3.421305in}{1.086911in}}%
\pgfpathlineto{\pgfqpoint{3.418893in}{1.078102in}}%
\pgfpathlineto{\pgfqpoint{3.415720in}{1.069370in}}%
\pgfpathlineto{\pgfqpoint{3.411798in}{1.060751in}}%
\pgfpathlineto{\pgfqpoint{3.410652in}{1.067244in}}%
\pgfpathlineto{\pgfqpoint{3.410338in}{1.073767in}}%
\pgfpathlineto{\pgfqpoint{3.410858in}{1.080293in}}%
\pgfpathlineto{\pgfqpoint{3.412214in}{1.086796in}}%
\pgfpathlineto{\pgfqpoint{3.414402in}{1.093250in}}%
\pgfpathlineto{\pgfqpoint{3.418294in}{1.101289in}}%
\pgfpathlineto{\pgfqpoint{3.421441in}{1.109559in}}%
\pgfpathlineto{\pgfqpoint{3.423834in}{1.118027in}}%
\pgfpathlineto{\pgfqpoint{3.425462in}{1.126658in}}%
\pgfpathlineto{\pgfqpoint{3.426321in}{1.135418in}}%
\pgfpathlineto{\pgfqpoint{3.424250in}{1.129303in}}%
\pgfpathlineto{\pgfqpoint{3.422962in}{1.123141in}}%
\pgfpathlineto{\pgfqpoint{3.422461in}{1.116956in}}%
\pgfpathlineto{\pgfqpoint{3.422745in}{1.110773in}}%
\pgfpathlineto{\pgfqpoint{3.423812in}{1.104618in}}%
\pgfpathclose%
\pgfusepath{fill}%
\end{pgfscope}%
\begin{pgfscope}%
\pgfpathrectangle{\pgfqpoint{2.548318in}{0.050000in}}{\pgfqpoint{2.081932in}{2.081932in}}%
\pgfusepath{clip}%
\pgfsetbuttcap%
\pgfsetroundjoin%
\definecolor{currentfill}{rgb}{0.993248,0.906157,0.143936}%
\pgfsetfillcolor{currentfill}%
\pgfsetlinewidth{0.000000pt}%
\definecolor{currentstroke}{rgb}{0.000000,0.000000,0.000000}%
\pgfsetstrokecolor{currentstroke}%
\pgfsetdash{}{0pt}%
\pgfpathmoveto{\pgfqpoint{4.023535in}{1.351704in}}%
\pgfpathlineto{\pgfqpoint{4.012516in}{1.351330in}}%
\pgfpathlineto{\pgfqpoint{4.001350in}{1.350372in}}%
\pgfpathlineto{\pgfqpoint{3.990080in}{1.348835in}}%
\pgfpathlineto{\pgfqpoint{3.978752in}{1.346725in}}%
\pgfpathlineto{\pgfqpoint{3.967409in}{1.344052in}}%
\pgfpathlineto{\pgfqpoint{3.976863in}{1.332414in}}%
\pgfpathlineto{\pgfqpoint{3.984823in}{1.320512in}}%
\pgfpathlineto{\pgfqpoint{3.991264in}{1.308396in}}%
\pgfpathlineto{\pgfqpoint{3.996168in}{1.296118in}}%
\pgfpathlineto{\pgfqpoint{3.999523in}{1.283727in}}%
\pgfpathlineto{\pgfqpoint{4.011792in}{1.284438in}}%
\pgfpathlineto{\pgfqpoint{4.024038in}{1.284593in}}%
\pgfpathlineto{\pgfqpoint{4.036214in}{1.284191in}}%
\pgfpathlineto{\pgfqpoint{4.048272in}{1.283233in}}%
\pgfpathlineto{\pgfqpoint{4.060163in}{1.281723in}}%
\pgfpathlineto{\pgfqpoint{4.056405in}{1.296081in}}%
\pgfpathlineto{\pgfqpoint{4.050848in}{1.310317in}}%
\pgfpathlineto{\pgfqpoint{4.043503in}{1.324373in}}%
\pgfpathlineto{\pgfqpoint{4.034390in}{1.338188in}}%
\pgfpathlineto{\pgfqpoint{4.023535in}{1.351704in}}%
\pgfpathclose%
\pgfusepath{fill}%
\end{pgfscope}%
\begin{pgfscope}%
\pgfpathrectangle{\pgfqpoint{2.548318in}{0.050000in}}{\pgfqpoint{2.081932in}{2.081932in}}%
\pgfusepath{clip}%
\pgfsetbuttcap%
\pgfsetroundjoin%
\definecolor{currentfill}{rgb}{0.993248,0.906157,0.143936}%
\pgfsetfillcolor{currentfill}%
\pgfsetlinewidth{0.000000pt}%
\definecolor{currentstroke}{rgb}{0.000000,0.000000,0.000000}%
\pgfsetstrokecolor{currentstroke}%
\pgfsetdash{}{0pt}%
\pgfpathmoveto{\pgfqpoint{3.239946in}{1.299224in}}%
\pgfpathlineto{\pgfqpoint{3.252074in}{1.297460in}}%
\pgfpathlineto{\pgfqpoint{3.264126in}{1.295149in}}%
\pgfpathlineto{\pgfqpoint{3.276055in}{1.292303in}}%
\pgfpathlineto{\pgfqpoint{3.287815in}{1.288931in}}%
\pgfpathlineto{\pgfqpoint{3.299358in}{1.285048in}}%
\pgfpathlineto{\pgfqpoint{3.303903in}{1.295339in}}%
\pgfpathlineto{\pgfqpoint{3.309736in}{1.305477in}}%
\pgfpathlineto{\pgfqpoint{3.316838in}{1.315419in}}%
\pgfpathlineto{\pgfqpoint{3.325186in}{1.325122in}}%
\pgfpathlineto{\pgfqpoint{3.334749in}{1.334546in}}%
\pgfpathlineto{\pgfqpoint{3.324411in}{1.340269in}}%
\pgfpathlineto{\pgfqpoint{3.313873in}{1.345513in}}%
\pgfpathlineto{\pgfqpoint{3.303178in}{1.350257in}}%
\pgfpathlineto{\pgfqpoint{3.292366in}{1.354483in}}%
\pgfpathlineto{\pgfqpoint{3.281480in}{1.358172in}}%
\pgfpathlineto{\pgfqpoint{3.270207in}{1.346936in}}%
\pgfpathlineto{\pgfqpoint{3.260387in}{1.335374in}}%
\pgfpathlineto{\pgfqpoint{3.252053in}{1.323534in}}%
\pgfpathlineto{\pgfqpoint{3.245233in}{1.311467in}}%
\pgfpathlineto{\pgfqpoint{3.239946in}{1.299224in}}%
\pgfpathclose%
\pgfusepath{fill}%
\end{pgfscope}%
\begin{pgfscope}%
\pgfpathrectangle{\pgfqpoint{2.548318in}{0.050000in}}{\pgfqpoint{2.081932in}{2.081932in}}%
\pgfusepath{clip}%
\pgfsetbuttcap%
\pgfsetroundjoin%
\definecolor{currentfill}{rgb}{0.296479,0.761561,0.424223}%
\pgfsetfillcolor{currentfill}%
\pgfsetlinewidth{0.000000pt}%
\definecolor{currentstroke}{rgb}{0.000000,0.000000,0.000000}%
\pgfsetstrokecolor{currentstroke}%
\pgfsetdash{}{0pt}%
\pgfpathmoveto{\pgfqpoint{3.393555in}{1.222342in}}%
\pgfpathlineto{\pgfqpoint{3.400028in}{1.214227in}}%
\pgfpathlineto{\pgfqpoint{3.405841in}{1.205894in}}%
\pgfpathlineto{\pgfqpoint{3.410970in}{1.197374in}}%
\pgfpathlineto{\pgfqpoint{3.415394in}{1.188702in}}%
\pgfpathlineto{\pgfqpoint{3.419096in}{1.179913in}}%
\pgfpathlineto{\pgfqpoint{3.422044in}{1.186226in}}%
\pgfpathlineto{\pgfqpoint{3.425790in}{1.192438in}}%
\pgfpathlineto{\pgfqpoint{3.430319in}{1.198523in}}%
\pgfpathlineto{\pgfqpoint{3.435616in}{1.204455in}}%
\pgfpathlineto{\pgfqpoint{3.441660in}{1.210211in}}%
\pgfpathlineto{\pgfqpoint{3.438364in}{1.219610in}}%
\pgfpathlineto{\pgfqpoint{3.434423in}{1.229003in}}%
\pgfpathlineto{\pgfqpoint{3.429854in}{1.238354in}}%
\pgfpathlineto{\pgfqpoint{3.424675in}{1.247625in}}%
\pgfpathlineto{\pgfqpoint{3.418905in}{1.256782in}}%
\pgfpathlineto{\pgfqpoint{3.412102in}{1.250236in}}%
\pgfpathlineto{\pgfqpoint{3.406146in}{1.243491in}}%
\pgfpathlineto{\pgfqpoint{3.401058in}{1.236574in}}%
\pgfpathlineto{\pgfqpoint{3.396856in}{1.229515in}}%
\pgfpathlineto{\pgfqpoint{3.393555in}{1.222342in}}%
\pgfpathclose%
\pgfusepath{fill}%
\end{pgfscope}%
\begin{pgfscope}%
\pgfpathrectangle{\pgfqpoint{2.548318in}{0.050000in}}{\pgfqpoint{2.081932in}{2.081932in}}%
\pgfusepath{clip}%
\pgfsetbuttcap%
\pgfsetroundjoin%
\definecolor{currentfill}{rgb}{0.227802,0.326594,0.546532}%
\pgfsetfillcolor{currentfill}%
\pgfsetlinewidth{0.000000pt}%
\definecolor{currentstroke}{rgb}{0.000000,0.000000,0.000000}%
\pgfsetstrokecolor{currentstroke}%
\pgfsetdash{}{0pt}%
\pgfpathmoveto{\pgfqpoint{3.823245in}{1.085161in}}%
\pgfpathlineto{\pgfqpoint{3.827920in}{1.077212in}}%
\pgfpathlineto{\pgfqpoint{3.833315in}{1.069530in}}%
\pgfpathlineto{\pgfqpoint{3.839409in}{1.062147in}}%
\pgfpathlineto{\pgfqpoint{3.846178in}{1.055093in}}%
\pgfpathlineto{\pgfqpoint{3.853596in}{1.048398in}}%
\pgfpathlineto{\pgfqpoint{3.854879in}{1.040969in}}%
\pgfpathlineto{\pgfqpoint{3.855202in}{1.033525in}}%
\pgfpathlineto{\pgfqpoint{3.854569in}{1.026093in}}%
\pgfpathlineto{\pgfqpoint{3.852985in}{1.018705in}}%
\pgfpathlineto{\pgfqpoint{3.850459in}{1.011390in}}%
\pgfpathlineto{\pgfqpoint{3.843181in}{1.019205in}}%
\pgfpathlineto{\pgfqpoint{3.836536in}{1.027277in}}%
\pgfpathlineto{\pgfqpoint{3.830553in}{1.035574in}}%
\pgfpathlineto{\pgfqpoint{3.825255in}{1.044061in}}%
\pgfpathlineto{\pgfqpoint{3.820662in}{1.052703in}}%
\pgfpathlineto{\pgfqpoint{3.822835in}{1.059123in}}%
\pgfpathlineto{\pgfqpoint{3.824185in}{1.065604in}}%
\pgfpathlineto{\pgfqpoint{3.824707in}{1.072122in}}%
\pgfpathlineto{\pgfqpoint{3.824394in}{1.078650in}}%
\pgfpathlineto{\pgfqpoint{3.823245in}{1.085161in}}%
\pgfpathclose%
\pgfusepath{fill}%
\end{pgfscope}%
\begin{pgfscope}%
\pgfpathrectangle{\pgfqpoint{2.548318in}{0.050000in}}{\pgfqpoint{2.081932in}{2.081932in}}%
\pgfusepath{clip}%
\pgfsetbuttcap%
\pgfsetroundjoin%
\definecolor{currentfill}{rgb}{0.150476,0.504369,0.557430}%
\pgfsetfillcolor{currentfill}%
\pgfsetlinewidth{0.000000pt}%
\definecolor{currentstroke}{rgb}{0.000000,0.000000,0.000000}%
\pgfsetstrokecolor{currentstroke}%
\pgfsetdash{}{0pt}%
\pgfpathmoveto{\pgfqpoint{3.043732in}{1.054503in}}%
\pgfpathlineto{\pgfqpoint{3.041296in}{1.062983in}}%
\pgfpathlineto{\pgfqpoint{3.039625in}{1.071600in}}%
\pgfpathlineto{\pgfqpoint{3.038726in}{1.080320in}}%
\pgfpathlineto{\pgfqpoint{3.038600in}{1.089107in}}%
\pgfpathlineto{\pgfqpoint{3.034310in}{1.107493in}}%
\pgfpathlineto{\pgfqpoint{3.032346in}{1.126031in}}%
\pgfpathlineto{\pgfqpoint{3.032737in}{1.144647in}}%
\pgfpathlineto{\pgfqpoint{3.035500in}{1.163266in}}%
\pgfpathlineto{\pgfqpoint{3.040646in}{1.181811in}}%
\pgfpathlineto{\pgfqpoint{3.040773in}{1.172877in}}%
\pgfpathlineto{\pgfqpoint{3.041682in}{1.163887in}}%
\pgfpathlineto{\pgfqpoint{3.043371in}{1.154876in}}%
\pgfpathlineto{\pgfqpoint{3.045833in}{1.145882in}}%
\pgfpathlineto{\pgfqpoint{3.040719in}{1.127604in}}%
\pgfpathlineto{\pgfqpoint{3.037965in}{1.109252in}}%
\pgfpathlineto{\pgfqpoint{3.037563in}{1.090902in}}%
\pgfpathlineto{\pgfqpoint{3.039495in}{1.072628in}}%
\pgfpathlineto{\pgfqpoint{3.043732in}{1.054503in}}%
\pgfpathclose%
\pgfusepath{fill}%
\end{pgfscope}%
\begin{pgfscope}%
\pgfpathrectangle{\pgfqpoint{2.548318in}{0.050000in}}{\pgfqpoint{2.081932in}{2.081932in}}%
\pgfusepath{clip}%
\pgfsetbuttcap%
\pgfsetroundjoin%
\definecolor{currentfill}{rgb}{0.120638,0.625828,0.533488}%
\pgfsetfillcolor{currentfill}%
\pgfsetlinewidth{0.000000pt}%
\definecolor{currentstroke}{rgb}{0.000000,0.000000,0.000000}%
\pgfsetstrokecolor{currentstroke}%
\pgfsetdash{}{0pt}%
\pgfpathmoveto{\pgfqpoint{3.800814in}{1.202976in}}%
\pgfpathlineto{\pgfqpoint{3.798067in}{1.193727in}}%
\pgfpathlineto{\pgfqpoint{3.796017in}{1.184517in}}%
\pgfpathlineto{\pgfqpoint{3.794671in}{1.175383in}}%
\pgfpathlineto{\pgfqpoint{3.794037in}{1.166362in}}%
\pgfpathlineto{\pgfqpoint{3.794118in}{1.157488in}}%
\pgfpathlineto{\pgfqpoint{3.799045in}{1.151773in}}%
\pgfpathlineto{\pgfqpoint{3.803229in}{1.145917in}}%
\pgfpathlineto{\pgfqpoint{3.806655in}{1.139946in}}%
\pgfpathlineto{\pgfqpoint{3.809310in}{1.133883in}}%
\pgfpathlineto{\pgfqpoint{3.811187in}{1.127754in}}%
\pgfpathlineto{\pgfqpoint{3.811099in}{1.136601in}}%
\pgfpathlineto{\pgfqpoint{3.811791in}{1.145474in}}%
\pgfpathlineto{\pgfqpoint{3.813260in}{1.154340in}}%
\pgfpathlineto{\pgfqpoint{3.815498in}{1.163162in}}%
\pgfpathlineto{\pgfqpoint{3.818495in}{1.171905in}}%
\pgfpathlineto{\pgfqpoint{3.816555in}{1.178309in}}%
\pgfpathlineto{\pgfqpoint{3.813806in}{1.184644in}}%
\pgfpathlineto{\pgfqpoint{3.810257in}{1.190884in}}%
\pgfpathlineto{\pgfqpoint{3.805921in}{1.197003in}}%
\pgfpathlineto{\pgfqpoint{3.800814in}{1.202976in}}%
\pgfpathclose%
\pgfusepath{fill}%
\end{pgfscope}%
\begin{pgfscope}%
\pgfpathrectangle{\pgfqpoint{2.548318in}{0.050000in}}{\pgfqpoint{2.081932in}{2.081932in}}%
\pgfusepath{clip}%
\pgfsetbuttcap%
\pgfsetroundjoin%
\definecolor{currentfill}{rgb}{0.267968,0.223549,0.512008}%
\pgfsetfillcolor{currentfill}%
\pgfsetlinewidth{0.000000pt}%
\definecolor{currentstroke}{rgb}{0.000000,0.000000,0.000000}%
\pgfsetstrokecolor{currentstroke}%
\pgfsetdash{}{0pt}%
\pgfpathmoveto{\pgfqpoint{4.131140in}{1.044632in}}%
\pgfpathlineto{\pgfqpoint{4.140398in}{1.051130in}}%
\pgfpathlineto{\pgfqpoint{4.149105in}{1.058007in}}%
\pgfpathlineto{\pgfqpoint{4.157224in}{1.065234in}}%
\pgfpathlineto{\pgfqpoint{4.164725in}{1.072782in}}%
\pgfpathlineto{\pgfqpoint{4.171575in}{1.080619in}}%
\pgfpathlineto{\pgfqpoint{4.173719in}{1.063125in}}%
\pgfpathlineto{\pgfqpoint{4.173611in}{1.045647in}}%
\pgfpathlineto{\pgfqpoint{4.171270in}{1.028255in}}%
\pgfpathlineto{\pgfqpoint{4.166723in}{1.011017in}}%
\pgfpathlineto{\pgfqpoint{4.160006in}{0.994001in}}%
\pgfpathlineto{\pgfqpoint{4.153387in}{0.987324in}}%
\pgfpathlineto{\pgfqpoint{4.146139in}{0.981031in}}%
\pgfpathlineto{\pgfqpoint{4.138288in}{0.975147in}}%
\pgfpathlineto{\pgfqpoint{4.129867in}{0.969698in}}%
\pgfpathlineto{\pgfqpoint{4.120908in}{0.964705in}}%
\pgfpathlineto{\pgfqpoint{4.127043in}{0.980419in}}%
\pgfpathlineto{\pgfqpoint{4.131161in}{0.996331in}}%
\pgfpathlineto{\pgfqpoint{4.133232in}{1.012380in}}%
\pgfpathlineto{\pgfqpoint{4.133230in}{1.028502in}}%
\pgfpathlineto{\pgfqpoint{4.131140in}{1.044632in}}%
\pgfpathclose%
\pgfusepath{fill}%
\end{pgfscope}%
\begin{pgfscope}%
\pgfpathrectangle{\pgfqpoint{2.548318in}{0.050000in}}{\pgfqpoint{2.081932in}{2.081932in}}%
\pgfusepath{clip}%
\pgfsetbuttcap%
\pgfsetroundjoin%
\definecolor{currentfill}{rgb}{0.876168,0.891125,0.095250}%
\pgfsetfillcolor{currentfill}%
\pgfsetlinewidth{0.000000pt}%
\definecolor{currentstroke}{rgb}{0.000000,0.000000,0.000000}%
\pgfsetstrokecolor{currentstroke}%
\pgfsetdash{}{0pt}%
\pgfpathmoveto{\pgfqpoint{4.074902in}{1.344809in}}%
\pgfpathlineto{\pgfqpoint{4.065266in}{1.347346in}}%
\pgfpathlineto{\pgfqpoint{4.055270in}{1.349309in}}%
\pgfpathlineto{\pgfqpoint{4.044955in}{1.350692in}}%
\pgfpathlineto{\pgfqpoint{4.034363in}{1.351491in}}%
\pgfpathlineto{\pgfqpoint{4.023535in}{1.351704in}}%
\pgfpathlineto{\pgfqpoint{4.034390in}{1.338188in}}%
\pgfpathlineto{\pgfqpoint{4.043503in}{1.324373in}}%
\pgfpathlineto{\pgfqpoint{4.050848in}{1.310317in}}%
\pgfpathlineto{\pgfqpoint{4.056405in}{1.296081in}}%
\pgfpathlineto{\pgfqpoint{4.060163in}{1.281723in}}%
\pgfpathlineto{\pgfqpoint{4.071841in}{1.279667in}}%
\pgfpathlineto{\pgfqpoint{4.083260in}{1.277071in}}%
\pgfpathlineto{\pgfqpoint{4.094373in}{1.273947in}}%
\pgfpathlineto{\pgfqpoint{4.105137in}{1.270305in}}%
\pgfpathlineto{\pgfqpoint{4.115508in}{1.266160in}}%
\pgfpathlineto{\pgfqpoint{4.111413in}{1.282279in}}%
\pgfpathlineto{\pgfqpoint{4.105290in}{1.298271in}}%
\pgfpathlineto{\pgfqpoint{4.097149in}{1.314068in}}%
\pgfpathlineto{\pgfqpoint{4.087010in}{1.329603in}}%
\pgfpathlineto{\pgfqpoint{4.074902in}{1.344809in}}%
\pgfpathclose%
\pgfusepath{fill}%
\end{pgfscope}%
\begin{pgfscope}%
\pgfpathrectangle{\pgfqpoint{2.548318in}{0.050000in}}{\pgfqpoint{2.081932in}{2.081932in}}%
\pgfusepath{clip}%
\pgfsetbuttcap%
\pgfsetroundjoin%
\definecolor{currentfill}{rgb}{0.227802,0.326594,0.546532}%
\pgfsetfillcolor{currentfill}%
\pgfsetlinewidth{0.000000pt}%
\definecolor{currentstroke}{rgb}{0.000000,0.000000,0.000000}%
\pgfsetstrokecolor{currentstroke}%
\pgfsetdash{}{0pt}%
\pgfpathmoveto{\pgfqpoint{3.411798in}{1.060751in}}%
\pgfpathlineto{\pgfqpoint{3.407141in}{1.052280in}}%
\pgfpathlineto{\pgfqpoint{3.401768in}{1.043992in}}%
\pgfpathlineto{\pgfqpoint{3.395700in}{1.035921in}}%
\pgfpathlineto{\pgfqpoint{3.388961in}{1.028100in}}%
\pgfpathlineto{\pgfqpoint{3.381578in}{1.020561in}}%
\pgfpathlineto{\pgfqpoint{3.380229in}{1.027963in}}%
\pgfpathlineto{\pgfqpoint{3.379833in}{1.035400in}}%
\pgfpathlineto{\pgfqpoint{3.380395in}{1.042844in}}%
\pgfpathlineto{\pgfqpoint{3.381916in}{1.050263in}}%
\pgfpathlineto{\pgfqpoint{3.384392in}{1.057628in}}%
\pgfpathlineto{\pgfqpoint{3.391728in}{1.064043in}}%
\pgfpathlineto{\pgfqpoint{3.398421in}{1.070841in}}%
\pgfpathlineto{\pgfqpoint{3.404446in}{1.077995in}}%
\pgfpathlineto{\pgfqpoint{3.409780in}{1.085474in}}%
\pgfpathlineto{\pgfqpoint{3.414402in}{1.093250in}}%
\pgfpathlineto{\pgfqpoint{3.412214in}{1.086796in}}%
\pgfpathlineto{\pgfqpoint{3.410858in}{1.080293in}}%
\pgfpathlineto{\pgfqpoint{3.410338in}{1.073767in}}%
\pgfpathlineto{\pgfqpoint{3.410652in}{1.067244in}}%
\pgfpathlineto{\pgfqpoint{3.411798in}{1.060751in}}%
\pgfpathclose%
\pgfusepath{fill}%
\end{pgfscope}%
\begin{pgfscope}%
\pgfpathrectangle{\pgfqpoint{2.548318in}{0.050000in}}{\pgfqpoint{2.081932in}{2.081932in}}%
\pgfusepath{clip}%
\pgfsetbuttcap%
\pgfsetroundjoin%
\definecolor{currentfill}{rgb}{0.993248,0.906157,0.143936}%
\pgfsetfillcolor{currentfill}%
\pgfsetlinewidth{0.000000pt}%
\definecolor{currentstroke}{rgb}{0.000000,0.000000,0.000000}%
\pgfsetstrokecolor{currentstroke}%
\pgfsetdash{}{0pt}%
\pgfpathmoveto{\pgfqpoint{3.179847in}{1.299682in}}%
\pgfpathlineto{\pgfqpoint{3.191636in}{1.300712in}}%
\pgfpathlineto{\pgfqpoint{3.203588in}{1.301181in}}%
\pgfpathlineto{\pgfqpoint{3.215655in}{1.301089in}}%
\pgfpathlineto{\pgfqpoint{3.227791in}{1.300436in}}%
\pgfpathlineto{\pgfqpoint{3.239946in}{1.299224in}}%
\pgfpathlineto{\pgfqpoint{3.245233in}{1.311467in}}%
\pgfpathlineto{\pgfqpoint{3.252053in}{1.323534in}}%
\pgfpathlineto{\pgfqpoint{3.260387in}{1.335374in}}%
\pgfpathlineto{\pgfqpoint{3.270207in}{1.346936in}}%
\pgfpathlineto{\pgfqpoint{3.281480in}{1.358172in}}%
\pgfpathlineto{\pgfqpoint{3.270563in}{1.361309in}}%
\pgfpathlineto{\pgfqpoint{3.259658in}{1.363882in}}%
\pgfpathlineto{\pgfqpoint{3.248808in}{1.365879in}}%
\pgfpathlineto{\pgfqpoint{3.238055in}{1.367292in}}%
\pgfpathlineto{\pgfqpoint{3.227444in}{1.368113in}}%
\pgfpathlineto{\pgfqpoint{3.214466in}{1.355055in}}%
\pgfpathlineto{\pgfqpoint{3.203184in}{1.341624in}}%
\pgfpathlineto{\pgfqpoint{3.193636in}{1.327879in}}%
\pgfpathlineto{\pgfqpoint{3.185849in}{1.313878in}}%
\pgfpathlineto{\pgfqpoint{3.179847in}{1.299682in}}%
\pgfpathclose%
\pgfusepath{fill}%
\end{pgfscope}%
\begin{pgfscope}%
\pgfpathrectangle{\pgfqpoint{2.548318in}{0.050000in}}{\pgfqpoint{2.081932in}{2.081932in}}%
\pgfusepath{clip}%
\pgfsetbuttcap%
\pgfsetroundjoin%
\definecolor{currentfill}{rgb}{0.120638,0.625828,0.533488}%
\pgfsetfillcolor{currentfill}%
\pgfsetlinewidth{0.000000pt}%
\definecolor{currentstroke}{rgb}{0.000000,0.000000,0.000000}%
\pgfsetstrokecolor{currentstroke}%
\pgfsetdash{}{0pt}%
\pgfpathmoveto{\pgfqpoint{3.419096in}{1.179913in}}%
\pgfpathlineto{\pgfqpoint{3.422059in}{1.171039in}}%
\pgfpathlineto{\pgfqpoint{3.424272in}{1.162118in}}%
\pgfpathlineto{\pgfqpoint{3.425723in}{1.153183in}}%
\pgfpathlineto{\pgfqpoint{3.426408in}{1.144271in}}%
\pgfpathlineto{\pgfqpoint{3.426321in}{1.135418in}}%
\pgfpathlineto{\pgfqpoint{3.429168in}{1.141460in}}%
\pgfpathlineto{\pgfqpoint{3.432784in}{1.147404in}}%
\pgfpathlineto{\pgfqpoint{3.437154in}{1.153227in}}%
\pgfpathlineto{\pgfqpoint{3.442263in}{1.158903in}}%
\pgfpathlineto{\pgfqpoint{3.448092in}{1.164410in}}%
\pgfpathlineto{\pgfqpoint{3.448169in}{1.173290in}}%
\pgfpathlineto{\pgfqpoint{3.447560in}{1.182346in}}%
\pgfpathlineto{\pgfqpoint{3.446268in}{1.191542in}}%
\pgfpathlineto{\pgfqpoint{3.444298in}{1.200843in}}%
\pgfpathlineto{\pgfqpoint{3.441660in}{1.210211in}}%
\pgfpathlineto{\pgfqpoint{3.435616in}{1.204455in}}%
\pgfpathlineto{\pgfqpoint{3.430319in}{1.198523in}}%
\pgfpathlineto{\pgfqpoint{3.425790in}{1.192438in}}%
\pgfpathlineto{\pgfqpoint{3.422044in}{1.186226in}}%
\pgfpathlineto{\pgfqpoint{3.419096in}{1.179913in}}%
\pgfpathclose%
\pgfusepath{fill}%
\end{pgfscope}%
\begin{pgfscope}%
\pgfpathrectangle{\pgfqpoint{2.548318in}{0.050000in}}{\pgfqpoint{2.081932in}{2.081932in}}%
\pgfusepath{clip}%
\pgfsetbuttcap%
\pgfsetroundjoin%
\definecolor{currentfill}{rgb}{0.278012,0.180367,0.486697}%
\pgfsetfillcolor{currentfill}%
\pgfsetlinewidth{0.000000pt}%
\definecolor{currentstroke}{rgb}{0.000000,0.000000,0.000000}%
\pgfsetstrokecolor{currentstroke}%
\pgfsetdash{}{0pt}%
\pgfpathmoveto{\pgfqpoint{3.853596in}{1.048398in}}%
\pgfpathlineto{\pgfqpoint{3.861634in}{1.042088in}}%
\pgfpathlineto{\pgfqpoint{3.870259in}{1.036190in}}%
\pgfpathlineto{\pgfqpoint{3.879438in}{1.030728in}}%
\pgfpathlineto{\pgfqpoint{3.889133in}{1.025726in}}%
\pgfpathlineto{\pgfqpoint{3.899306in}{1.021204in}}%
\pgfpathlineto{\pgfqpoint{3.900773in}{1.012377in}}%
\pgfpathlineto{\pgfqpoint{3.901096in}{1.003534in}}%
\pgfpathlineto{\pgfqpoint{3.900277in}{0.994711in}}%
\pgfpathlineto{\pgfqpoint{3.898325in}{0.985944in}}%
\pgfpathlineto{\pgfqpoint{3.895252in}{0.977267in}}%
\pgfpathlineto{\pgfqpoint{3.885292in}{0.983334in}}%
\pgfpathlineto{\pgfqpoint{3.875795in}{0.989808in}}%
\pgfpathlineto{\pgfqpoint{3.866800in}{0.996661in}}%
\pgfpathlineto{\pgfqpoint{3.858343in}{1.003865in}}%
\pgfpathlineto{\pgfqpoint{3.850459in}{1.011390in}}%
\pgfpathlineto{\pgfqpoint{3.852985in}{1.018705in}}%
\pgfpathlineto{\pgfqpoint{3.854569in}{1.026093in}}%
\pgfpathlineto{\pgfqpoint{3.855202in}{1.033525in}}%
\pgfpathlineto{\pgfqpoint{3.854879in}{1.040969in}}%
\pgfpathlineto{\pgfqpoint{3.853596in}{1.048398in}}%
\pgfpathclose%
\pgfusepath{fill}%
\end{pgfscope}%
\begin{pgfscope}%
\pgfpathrectangle{\pgfqpoint{2.548318in}{0.050000in}}{\pgfqpoint{2.081932in}{2.081932in}}%
\pgfusepath{clip}%
\pgfsetbuttcap%
\pgfsetroundjoin%
\definecolor{currentfill}{rgb}{0.282327,0.094955,0.417331}%
\pgfsetfillcolor{currentfill}%
\pgfsetlinewidth{0.000000pt}%
\definecolor{currentstroke}{rgb}{0.000000,0.000000,0.000000}%
\pgfsetstrokecolor{currentstroke}%
\pgfsetdash{}{0pt}%
\pgfpathmoveto{\pgfqpoint{4.077975in}{1.018711in}}%
\pgfpathlineto{\pgfqpoint{4.089383in}{1.022945in}}%
\pgfpathlineto{\pgfqpoint{4.100447in}{1.027674in}}%
\pgfpathlineto{\pgfqpoint{4.111124in}{1.032880in}}%
\pgfpathlineto{\pgfqpoint{4.121369in}{1.038541in}}%
\pgfpathlineto{\pgfqpoint{4.131140in}{1.044632in}}%
\pgfpathlineto{\pgfqpoint{4.133230in}{1.028502in}}%
\pgfpathlineto{\pgfqpoint{4.133232in}{1.012380in}}%
\pgfpathlineto{\pgfqpoint{4.131161in}{0.996331in}}%
\pgfpathlineto{\pgfqpoint{4.127043in}{0.980419in}}%
\pgfpathlineto{\pgfqpoint{4.120908in}{0.964705in}}%
\pgfpathlineto{\pgfqpoint{4.111448in}{0.960189in}}%
\pgfpathlineto{\pgfqpoint{4.101525in}{0.956171in}}%
\pgfpathlineto{\pgfqpoint{4.091179in}{0.952666in}}%
\pgfpathlineto{\pgfqpoint{4.080451in}{0.949689in}}%
\pgfpathlineto{\pgfqpoint{4.069384in}{0.947254in}}%
\pgfpathlineto{\pgfqpoint{4.074776in}{0.961317in}}%
\pgfpathlineto{\pgfqpoint{4.078353in}{0.975551in}}%
\pgfpathlineto{\pgfqpoint{4.080091in}{0.989899in}}%
\pgfpathlineto{\pgfqpoint{4.079969in}{1.004305in}}%
\pgfpathlineto{\pgfqpoint{4.077975in}{1.018711in}}%
\pgfpathclose%
\pgfusepath{fill}%
\end{pgfscope}%
\begin{pgfscope}%
\pgfpathrectangle{\pgfqpoint{2.548318in}{0.050000in}}{\pgfqpoint{2.081932in}{2.081932in}}%
\pgfusepath{clip}%
\pgfsetbuttcap%
\pgfsetroundjoin%
\definecolor{currentfill}{rgb}{0.206756,0.371758,0.553117}%
\pgfsetfillcolor{currentfill}%
\pgfsetlinewidth{0.000000pt}%
\definecolor{currentstroke}{rgb}{0.000000,0.000000,0.000000}%
\pgfsetstrokecolor{currentstroke}%
\pgfsetdash{}{0pt}%
\pgfpathmoveto{\pgfqpoint{3.066984in}{1.015342in}}%
\pgfpathlineto{\pgfqpoint{3.060912in}{1.022634in}}%
\pgfpathlineto{\pgfqpoint{3.055532in}{1.030228in}}%
\pgfpathlineto{\pgfqpoint{3.050864in}{1.038092in}}%
\pgfpathlineto{\pgfqpoint{3.046926in}{1.046195in}}%
\pgfpathlineto{\pgfqpoint{3.043732in}{1.054503in}}%
\pgfpathlineto{\pgfqpoint{3.039495in}{1.072628in}}%
\pgfpathlineto{\pgfqpoint{3.037563in}{1.090902in}}%
\pgfpathlineto{\pgfqpoint{3.037965in}{1.109252in}}%
\pgfpathlineto{\pgfqpoint{3.040719in}{1.127604in}}%
\pgfpathlineto{\pgfqpoint{3.045833in}{1.145882in}}%
\pgfpathlineto{\pgfqpoint{3.049059in}{1.136940in}}%
\pgfpathlineto{\pgfqpoint{3.053038in}{1.128089in}}%
\pgfpathlineto{\pgfqpoint{3.057753in}{1.119364in}}%
\pgfpathlineto{\pgfqpoint{3.063186in}{1.110800in}}%
\pgfpathlineto{\pgfqpoint{3.069315in}{1.102434in}}%
\pgfpathlineto{\pgfqpoint{3.064347in}{1.085022in}}%
\pgfpathlineto{\pgfqpoint{3.061642in}{1.067535in}}%
\pgfpathlineto{\pgfqpoint{3.061192in}{1.050046in}}%
\pgfpathlineto{\pgfqpoint{3.062981in}{1.032625in}}%
\pgfpathlineto{\pgfqpoint{3.066984in}{1.015342in}}%
\pgfpathclose%
\pgfusepath{fill}%
\end{pgfscope}%
\begin{pgfscope}%
\pgfpathrectangle{\pgfqpoint{2.548318in}{0.050000in}}{\pgfqpoint{2.081932in}{2.081932in}}%
\pgfusepath{clip}%
\pgfsetbuttcap%
\pgfsetroundjoin%
\definecolor{currentfill}{rgb}{0.162142,0.474838,0.558140}%
\pgfsetfillcolor{currentfill}%
\pgfsetlinewidth{0.000000pt}%
\definecolor{currentstroke}{rgb}{0.000000,0.000000,0.000000}%
\pgfsetstrokecolor{currentstroke}%
\pgfsetdash{}{0pt}%
\pgfpathmoveto{\pgfqpoint{3.794118in}{1.157488in}}%
\pgfpathlineto{\pgfqpoint{3.794914in}{1.148798in}}%
\pgfpathlineto{\pgfqpoint{3.796423in}{1.140325in}}%
\pgfpathlineto{\pgfqpoint{3.798640in}{1.132104in}}%
\pgfpathlineto{\pgfqpoint{3.801558in}{1.124168in}}%
\pgfpathlineto{\pgfqpoint{3.805167in}{1.116549in}}%
\pgfpathlineto{\pgfqpoint{3.810392in}{1.110514in}}%
\pgfpathlineto{\pgfqpoint{3.814826in}{1.104332in}}%
\pgfpathlineto{\pgfqpoint{3.818454in}{1.098029in}}%
\pgfpathlineto{\pgfqpoint{3.821264in}{1.091630in}}%
\pgfpathlineto{\pgfqpoint{3.823245in}{1.085161in}}%
\pgfpathlineto{\pgfqpoint{3.819308in}{1.093345in}}%
\pgfpathlineto{\pgfqpoint{3.816124in}{1.101731in}}%
\pgfpathlineto{\pgfqpoint{3.813703in}{1.110284in}}%
\pgfpathlineto{\pgfqpoint{3.812056in}{1.118970in}}%
\pgfpathlineto{\pgfqpoint{3.811187in}{1.127754in}}%
\pgfpathlineto{\pgfqpoint{3.809310in}{1.133883in}}%
\pgfpathlineto{\pgfqpoint{3.806655in}{1.139946in}}%
\pgfpathlineto{\pgfqpoint{3.803229in}{1.145917in}}%
\pgfpathlineto{\pgfqpoint{3.799045in}{1.151773in}}%
\pgfpathlineto{\pgfqpoint{3.794118in}{1.157488in}}%
\pgfpathclose%
\pgfusepath{fill}%
\end{pgfscope}%
\begin{pgfscope}%
\pgfpathrectangle{\pgfqpoint{2.548318in}{0.050000in}}{\pgfqpoint{2.081932in}{2.081932in}}%
\pgfusepath{clip}%
\pgfsetbuttcap%
\pgfsetroundjoin%
\definecolor{currentfill}{rgb}{0.278791,0.062145,0.386592}%
\pgfsetfillcolor{currentfill}%
\pgfsetlinewidth{0.000000pt}%
\definecolor{currentstroke}{rgb}{0.000000,0.000000,0.000000}%
\pgfsetstrokecolor{currentstroke}%
\pgfsetdash{}{0pt}%
\pgfpathmoveto{\pgfqpoint{3.899306in}{1.021204in}}%
\pgfpathlineto{\pgfqpoint{3.909915in}{1.017181in}}%
\pgfpathlineto{\pgfqpoint{3.920918in}{1.013674in}}%
\pgfpathlineto{\pgfqpoint{3.932270in}{1.010699in}}%
\pgfpathlineto{\pgfqpoint{3.943925in}{1.008267in}}%
\pgfpathlineto{\pgfqpoint{3.955835in}{1.006389in}}%
\pgfpathlineto{\pgfqpoint{3.957503in}{0.995819in}}%
\pgfpathlineto{\pgfqpoint{3.957796in}{0.985236in}}%
\pgfpathlineto{\pgfqpoint{3.956720in}{0.974682in}}%
\pgfpathlineto{\pgfqpoint{3.954286in}{0.964201in}}%
\pgfpathlineto{\pgfqpoint{3.950510in}{0.953833in}}%
\pgfpathlineto{\pgfqpoint{3.938880in}{0.957531in}}%
\pgfpathlineto{\pgfqpoint{3.927494in}{0.961742in}}%
\pgfpathlineto{\pgfqpoint{3.916396in}{0.966448in}}%
\pgfpathlineto{\pgfqpoint{3.905635in}{0.971630in}}%
\pgfpathlineto{\pgfqpoint{3.895252in}{0.977267in}}%
\pgfpathlineto{\pgfqpoint{3.898325in}{0.985944in}}%
\pgfpathlineto{\pgfqpoint{3.900277in}{0.994711in}}%
\pgfpathlineto{\pgfqpoint{3.901096in}{1.003534in}}%
\pgfpathlineto{\pgfqpoint{3.900773in}{1.012377in}}%
\pgfpathlineto{\pgfqpoint{3.899306in}{1.021204in}}%
\pgfpathclose%
\pgfusepath{fill}%
\end{pgfscope}%
\begin{pgfscope}%
\pgfpathrectangle{\pgfqpoint{2.548318in}{0.050000in}}{\pgfqpoint{2.081932in}{2.081932in}}%
\pgfusepath{clip}%
\pgfsetbuttcap%
\pgfsetroundjoin%
\definecolor{currentfill}{rgb}{0.606045,0.850733,0.236712}%
\pgfsetfillcolor{currentfill}%
\pgfsetlinewidth{0.000000pt}%
\definecolor{currentstroke}{rgb}{0.000000,0.000000,0.000000}%
\pgfsetstrokecolor{currentstroke}%
\pgfsetdash{}{0pt}%
\pgfpathmoveto{\pgfqpoint{3.814455in}{1.326220in}}%
\pgfpathlineto{\pgfqpoint{3.807455in}{1.317184in}}%
\pgfpathlineto{\pgfqpoint{3.800840in}{1.307915in}}%
\pgfpathlineto{\pgfqpoint{3.794635in}{1.298449in}}%
\pgfpathlineto{\pgfqpoint{3.788864in}{1.288822in}}%
\pgfpathlineto{\pgfqpoint{3.783549in}{1.279072in}}%
\pgfpathlineto{\pgfqpoint{3.793270in}{1.273515in}}%
\pgfpathlineto{\pgfqpoint{3.802260in}{1.267660in}}%
\pgfpathlineto{\pgfqpoint{3.810482in}{1.261530in}}%
\pgfpathlineto{\pgfqpoint{3.817905in}{1.255153in}}%
\pgfpathlineto{\pgfqpoint{3.824500in}{1.248554in}}%
\pgfpathlineto{\pgfqpoint{3.831094in}{1.257286in}}%
\pgfpathlineto{\pgfqpoint{3.838252in}{1.265809in}}%
\pgfpathlineto{\pgfqpoint{3.845946in}{1.274090in}}%
\pgfpathlineto{\pgfqpoint{3.854146in}{1.282096in}}%
\pgfpathlineto{\pgfqpoint{3.862819in}{1.289796in}}%
\pgfpathlineto{\pgfqpoint{3.855046in}{1.297666in}}%
\pgfpathlineto{\pgfqpoint{3.846288in}{1.305275in}}%
\pgfpathlineto{\pgfqpoint{3.836578in}{1.312592in}}%
\pgfpathlineto{\pgfqpoint{3.825953in}{1.319583in}}%
\pgfpathlineto{\pgfqpoint{3.814455in}{1.326220in}}%
\pgfpathclose%
\pgfusepath{fill}%
\end{pgfscope}%
\begin{pgfscope}%
\pgfpathrectangle{\pgfqpoint{2.548318in}{0.050000in}}{\pgfqpoint{2.081932in}{2.081932in}}%
\pgfusepath{clip}%
\pgfsetbuttcap%
\pgfsetroundjoin%
\definecolor{currentfill}{rgb}{0.268510,0.009605,0.335427}%
\pgfsetfillcolor{currentfill}%
\pgfsetlinewidth{0.000000pt}%
\definecolor{currentstroke}{rgb}{0.000000,0.000000,0.000000}%
\pgfsetstrokecolor{currentstroke}%
\pgfsetdash{}{0pt}%
\pgfpathmoveto{\pgfqpoint{4.017483in}{1.005517in}}%
\pgfpathlineto{\pgfqpoint{4.029877in}{1.007053in}}%
\pgfpathlineto{\pgfqpoint{4.042174in}{1.009149in}}%
\pgfpathlineto{\pgfqpoint{4.054322in}{1.011799in}}%
\pgfpathlineto{\pgfqpoint{4.066273in}{1.014991in}}%
\pgfpathlineto{\pgfqpoint{4.077975in}{1.018711in}}%
\pgfpathlineto{\pgfqpoint{4.079969in}{1.004305in}}%
\pgfpathlineto{\pgfqpoint{4.080091in}{0.989899in}}%
\pgfpathlineto{\pgfqpoint{4.078353in}{0.975551in}}%
\pgfpathlineto{\pgfqpoint{4.074776in}{0.961317in}}%
\pgfpathlineto{\pgfqpoint{4.069384in}{0.947254in}}%
\pgfpathlineto{\pgfqpoint{4.058025in}{0.945370in}}%
\pgfpathlineto{\pgfqpoint{4.046419in}{0.944047in}}%
\pgfpathlineto{\pgfqpoint{4.034613in}{0.943290in}}%
\pgfpathlineto{\pgfqpoint{4.022657in}{0.943102in}}%
\pgfpathlineto{\pgfqpoint{4.010598in}{0.943484in}}%
\pgfpathlineto{\pgfqpoint{4.015172in}{0.955707in}}%
\pgfpathlineto{\pgfqpoint{4.018166in}{0.968071in}}%
\pgfpathlineto{\pgfqpoint{4.019558in}{0.980527in}}%
\pgfpathlineto{\pgfqpoint{4.019334in}{0.993026in}}%
\pgfpathlineto{\pgfqpoint{4.017483in}{1.005517in}}%
\pgfpathclose%
\pgfusepath{fill}%
\end{pgfscope}%
\begin{pgfscope}%
\pgfpathrectangle{\pgfqpoint{2.548318in}{0.050000in}}{\pgfqpoint{2.081932in}{2.081932in}}%
\pgfusepath{clip}%
\pgfsetbuttcap%
\pgfsetroundjoin%
\definecolor{currentfill}{rgb}{0.278012,0.180367,0.486697}%
\pgfsetfillcolor{currentfill}%
\pgfsetlinewidth{0.000000pt}%
\definecolor{currentstroke}{rgb}{0.000000,0.000000,0.000000}%
\pgfsetstrokecolor{currentstroke}%
\pgfsetdash{}{0pt}%
\pgfpathmoveto{\pgfqpoint{3.381578in}{1.020561in}}%
\pgfpathlineto{\pgfqpoint{3.373581in}{1.013336in}}%
\pgfpathlineto{\pgfqpoint{3.365002in}{1.006454in}}%
\pgfpathlineto{\pgfqpoint{3.355875in}{0.999944in}}%
\pgfpathlineto{\pgfqpoint{3.346238in}{0.993833in}}%
\pgfpathlineto{\pgfqpoint{3.336130in}{0.988146in}}%
\pgfpathlineto{\pgfqpoint{3.334458in}{0.996931in}}%
\pgfpathlineto{\pgfqpoint{3.333922in}{1.005762in}}%
\pgfpathlineto{\pgfqpoint{3.334529in}{1.014604in}}%
\pgfpathlineto{\pgfqpoint{3.336280in}{1.023421in}}%
\pgfpathlineto{\pgfqpoint{3.339175in}{1.032178in}}%
\pgfpathlineto{\pgfqpoint{3.349240in}{1.036311in}}%
\pgfpathlineto{\pgfqpoint{3.358832in}{1.040943in}}%
\pgfpathlineto{\pgfqpoint{3.367911in}{1.046054in}}%
\pgfpathlineto{\pgfqpoint{3.376443in}{1.051624in}}%
\pgfpathlineto{\pgfqpoint{3.384392in}{1.057628in}}%
\pgfpathlineto{\pgfqpoint{3.381916in}{1.050263in}}%
\pgfpathlineto{\pgfqpoint{3.380395in}{1.042844in}}%
\pgfpathlineto{\pgfqpoint{3.379833in}{1.035400in}}%
\pgfpathlineto{\pgfqpoint{3.380229in}{1.027963in}}%
\pgfpathlineto{\pgfqpoint{3.381578in}{1.020561in}}%
\pgfpathclose%
\pgfusepath{fill}%
\end{pgfscope}%
\begin{pgfscope}%
\pgfpathrectangle{\pgfqpoint{2.548318in}{0.050000in}}{\pgfqpoint{2.081932in}{2.081932in}}%
\pgfusepath{clip}%
\pgfsetbuttcap%
\pgfsetroundjoin%
\definecolor{currentfill}{rgb}{0.267004,0.004874,0.329415}%
\pgfsetfillcolor{currentfill}%
\pgfsetlinewidth{0.000000pt}%
\definecolor{currentstroke}{rgb}{0.000000,0.000000,0.000000}%
\pgfsetstrokecolor{currentstroke}%
\pgfsetdash{}{0pt}%
\pgfpathmoveto{\pgfqpoint{3.955835in}{1.006389in}}%
\pgfpathlineto{\pgfqpoint{3.967952in}{1.005074in}}%
\pgfpathlineto{\pgfqpoint{3.980226in}{1.004327in}}%
\pgfpathlineto{\pgfqpoint{3.992606in}{1.004152in}}%
\pgfpathlineto{\pgfqpoint{4.005042in}{1.004549in}}%
\pgfpathlineto{\pgfqpoint{4.017483in}{1.005517in}}%
\pgfpathlineto{\pgfqpoint{4.019334in}{0.993026in}}%
\pgfpathlineto{\pgfqpoint{4.019558in}{0.980527in}}%
\pgfpathlineto{\pgfqpoint{4.018166in}{0.968071in}}%
\pgfpathlineto{\pgfqpoint{4.015172in}{0.955707in}}%
\pgfpathlineto{\pgfqpoint{4.010598in}{0.943484in}}%
\pgfpathlineto{\pgfqpoint{3.998486in}{0.944437in}}%
\pgfpathlineto{\pgfqpoint{3.986372in}{0.945955in}}%
\pgfpathlineto{\pgfqpoint{3.974305in}{0.948033in}}%
\pgfpathlineto{\pgfqpoint{3.962334in}{0.950662in}}%
\pgfpathlineto{\pgfqpoint{3.950510in}{0.953833in}}%
\pgfpathlineto{\pgfqpoint{3.954286in}{0.964201in}}%
\pgfpathlineto{\pgfqpoint{3.956720in}{0.974682in}}%
\pgfpathlineto{\pgfqpoint{3.957796in}{0.985236in}}%
\pgfpathlineto{\pgfqpoint{3.957503in}{0.995819in}}%
\pgfpathlineto{\pgfqpoint{3.955835in}{1.006389in}}%
\pgfpathclose%
\pgfusepath{fill}%
\end{pgfscope}%
\begin{pgfscope}%
\pgfpathrectangle{\pgfqpoint{2.548318in}{0.050000in}}{\pgfqpoint{2.081932in}{2.081932in}}%
\pgfusepath{clip}%
\pgfsetbuttcap%
\pgfsetroundjoin%
\definecolor{currentfill}{rgb}{0.296479,0.761561,0.424223}%
\pgfsetfillcolor{currentfill}%
\pgfsetlinewidth{0.000000pt}%
\definecolor{currentstroke}{rgb}{0.000000,0.000000,0.000000}%
\pgfsetstrokecolor{currentstroke}%
\pgfsetdash{}{0pt}%
\pgfpathmoveto{\pgfqpoint{3.783549in}{1.279072in}}%
\pgfpathlineto{\pgfqpoint{3.778711in}{1.269236in}}%
\pgfpathlineto{\pgfqpoint{3.774368in}{1.259353in}}%
\pgfpathlineto{\pgfqpoint{3.770538in}{1.249460in}}%
\pgfpathlineto{\pgfqpoint{3.767235in}{1.239597in}}%
\pgfpathlineto{\pgfqpoint{3.764472in}{1.229801in}}%
\pgfpathlineto{\pgfqpoint{3.773091in}{1.224919in}}%
\pgfpathlineto{\pgfqpoint{3.781065in}{1.219773in}}%
\pgfpathlineto{\pgfqpoint{3.788362in}{1.214385in}}%
\pgfpathlineto{\pgfqpoint{3.794953in}{1.208779in}}%
\pgfpathlineto{\pgfqpoint{3.800814in}{1.202976in}}%
\pgfpathlineto{\pgfqpoint{3.804245in}{1.212229in}}%
\pgfpathlineto{\pgfqpoint{3.808347in}{1.221450in}}%
\pgfpathlineto{\pgfqpoint{3.813104in}{1.230601in}}%
\pgfpathlineto{\pgfqpoint{3.818495in}{1.239648in}}%
\pgfpathlineto{\pgfqpoint{3.824500in}{1.248554in}}%
\pgfpathlineto{\pgfqpoint{3.817905in}{1.255153in}}%
\pgfpathlineto{\pgfqpoint{3.810482in}{1.261530in}}%
\pgfpathlineto{\pgfqpoint{3.802260in}{1.267660in}}%
\pgfpathlineto{\pgfqpoint{3.793270in}{1.273515in}}%
\pgfpathlineto{\pgfqpoint{3.783549in}{1.279072in}}%
\pgfpathclose%
\pgfusepath{fill}%
\end{pgfscope}%
\begin{pgfscope}%
\pgfpathrectangle{\pgfqpoint{2.548318in}{0.050000in}}{\pgfqpoint{2.081932in}{2.081932in}}%
\pgfusepath{clip}%
\pgfsetbuttcap%
\pgfsetroundjoin%
\definecolor{currentfill}{rgb}{0.636902,0.856542,0.216620}%
\pgfsetfillcolor{currentfill}%
\pgfsetlinewidth{0.000000pt}%
\definecolor{currentstroke}{rgb}{0.000000,0.000000,0.000000}%
\pgfsetstrokecolor{currentstroke}%
\pgfsetdash{}{0pt}%
\pgfpathmoveto{\pgfqpoint{4.116370in}{1.323931in}}%
\pgfpathlineto{\pgfqpoint{4.109089in}{1.329151in}}%
\pgfpathlineto{\pgfqpoint{4.101268in}{1.333865in}}%
\pgfpathlineto{\pgfqpoint{4.092940in}{1.338057in}}%
\pgfpathlineto{\pgfqpoint{4.084139in}{1.341709in}}%
\pgfpathlineto{\pgfqpoint{4.074902in}{1.344809in}}%
\pgfpathlineto{\pgfqpoint{4.087010in}{1.329603in}}%
\pgfpathlineto{\pgfqpoint{4.097149in}{1.314068in}}%
\pgfpathlineto{\pgfqpoint{4.105290in}{1.298271in}}%
\pgfpathlineto{\pgfqpoint{4.111413in}{1.282279in}}%
\pgfpathlineto{\pgfqpoint{4.115508in}{1.266160in}}%
\pgfpathlineto{\pgfqpoint{4.125446in}{1.261527in}}%
\pgfpathlineto{\pgfqpoint{4.134909in}{1.256425in}}%
\pgfpathlineto{\pgfqpoint{4.143860in}{1.250873in}}%
\pgfpathlineto{\pgfqpoint{4.152263in}{1.244893in}}%
\pgfpathlineto{\pgfqpoint{4.160082in}{1.238507in}}%
\pgfpathlineto{\pgfqpoint{4.155738in}{1.256000in}}%
\pgfpathlineto{\pgfqpoint{4.149180in}{1.273362in}}%
\pgfpathlineto{\pgfqpoint{4.140418in}{1.290521in}}%
\pgfpathlineto{\pgfqpoint{4.129471in}{1.307401in}}%
\pgfpathlineto{\pgfqpoint{4.116370in}{1.323931in}}%
\pgfpathclose%
\pgfusepath{fill}%
\end{pgfscope}%
\begin{pgfscope}%
\pgfpathrectangle{\pgfqpoint{2.548318in}{0.050000in}}{\pgfqpoint{2.081932in}{2.081932in}}%
\pgfusepath{clip}%
\pgfsetbuttcap%
\pgfsetroundjoin%
\definecolor{currentfill}{rgb}{0.606045,0.850733,0.236712}%
\pgfsetfillcolor{currentfill}%
\pgfsetlinewidth{0.000000pt}%
\definecolor{currentstroke}{rgb}{0.000000,0.000000,0.000000}%
\pgfsetstrokecolor{currentstroke}%
\pgfsetdash{}{0pt}%
\pgfpathmoveto{\pgfqpoint{3.382080in}{1.299609in}}%
\pgfpathlineto{\pgfqpoint{3.390416in}{1.291551in}}%
\pgfpathlineto{\pgfqpoint{3.398297in}{1.283206in}}%
\pgfpathlineto{\pgfqpoint{3.405690in}{1.274606in}}%
\pgfpathlineto{\pgfqpoint{3.412569in}{1.265787in}}%
\pgfpathlineto{\pgfqpoint{3.418905in}{1.256782in}}%
\pgfpathlineto{\pgfqpoint{3.426529in}{1.263099in}}%
\pgfpathlineto{\pgfqpoint{3.434945in}{1.269162in}}%
\pgfpathlineto{\pgfqpoint{3.444120in}{1.274945in}}%
\pgfpathlineto{\pgfqpoint{3.454017in}{1.280422in}}%
\pgfpathlineto{\pgfqpoint{3.448781in}{1.290217in}}%
\pgfpathlineto{\pgfqpoint{3.443097in}{1.299894in}}%
\pgfpathlineto{\pgfqpoint{3.436984in}{1.309413in}}%
\pgfpathlineto{\pgfqpoint{3.430468in}{1.318738in}}%
\pgfpathlineto{\pgfqpoint{3.423573in}{1.327833in}}%
\pgfpathlineto{\pgfqpoint{3.411865in}{1.321290in}}%
\pgfpathlineto{\pgfqpoint{3.401019in}{1.314385in}}%
\pgfpathlineto{\pgfqpoint{3.391077in}{1.307148in}}%
\pgfpathlineto{\pgfqpoint{3.382080in}{1.299609in}}%
\pgfpathclose%
\pgfusepath{fill}%
\end{pgfscope}%
\begin{pgfscope}%
\pgfpathrectangle{\pgfqpoint{2.548318in}{0.050000in}}{\pgfqpoint{2.081932in}{2.081932in}}%
\pgfusepath{clip}%
\pgfsetbuttcap%
\pgfsetroundjoin%
\definecolor{currentfill}{rgb}{0.855810,0.888601,0.097452}%
\pgfsetfillcolor{currentfill}%
\pgfsetlinewidth{0.000000pt}%
\definecolor{currentstroke}{rgb}{0.000000,0.000000,0.000000}%
\pgfsetstrokecolor{currentstroke}%
\pgfsetdash{}{0pt}%
\pgfpathmoveto{\pgfqpoint{3.854233in}{1.366698in}}%
\pgfpathlineto{\pgfqpoint{3.845740in}{1.359338in}}%
\pgfpathlineto{\pgfqpoint{3.837485in}{1.351578in}}%
\pgfpathlineto{\pgfqpoint{3.829499in}{1.343450in}}%
\pgfpathlineto{\pgfqpoint{3.821812in}{1.334986in}}%
\pgfpathlineto{\pgfqpoint{3.814455in}{1.326220in}}%
\pgfpathlineto{\pgfqpoint{3.825953in}{1.319583in}}%
\pgfpathlineto{\pgfqpoint{3.836578in}{1.312592in}}%
\pgfpathlineto{\pgfqpoint{3.846288in}{1.305275in}}%
\pgfpathlineto{\pgfqpoint{3.855046in}{1.297666in}}%
\pgfpathlineto{\pgfqpoint{3.862819in}{1.289796in}}%
\pgfpathlineto{\pgfqpoint{3.871931in}{1.297161in}}%
\pgfpathlineto{\pgfqpoint{3.881447in}{1.304160in}}%
\pgfpathlineto{\pgfqpoint{3.891331in}{1.310767in}}%
\pgfpathlineto{\pgfqpoint{3.901543in}{1.316956in}}%
\pgfpathlineto{\pgfqpoint{3.912043in}{1.322702in}}%
\pgfpathlineto{\pgfqpoint{3.902779in}{1.332200in}}%
\pgfpathlineto{\pgfqpoint{3.892324in}{1.341388in}}%
\pgfpathlineto{\pgfqpoint{3.880717in}{1.350225in}}%
\pgfpathlineto{\pgfqpoint{3.868004in}{1.358675in}}%
\pgfpathlineto{\pgfqpoint{3.854233in}{1.366698in}}%
\pgfpathclose%
\pgfusepath{fill}%
\end{pgfscope}%
\begin{pgfscope}%
\pgfpathrectangle{\pgfqpoint{2.548318in}{0.050000in}}{\pgfqpoint{2.081932in}{2.081932in}}%
\pgfusepath{clip}%
\pgfsetbuttcap%
\pgfsetroundjoin%
\definecolor{currentfill}{rgb}{0.876168,0.891125,0.095250}%
\pgfsetfillcolor{currentfill}%
\pgfsetlinewidth{0.000000pt}%
\definecolor{currentstroke}{rgb}{0.000000,0.000000,0.000000}%
\pgfsetstrokecolor{currentstroke}%
\pgfsetdash{}{0pt}%
\pgfpathmoveto{\pgfqpoint{3.124957in}{1.286323in}}%
\pgfpathlineto{\pgfqpoint{3.135246in}{1.290060in}}%
\pgfpathlineto{\pgfqpoint{3.145923in}{1.293275in}}%
\pgfpathlineto{\pgfqpoint{3.156945in}{1.295956in}}%
\pgfpathlineto{\pgfqpoint{3.168268in}{1.298095in}}%
\pgfpathlineto{\pgfqpoint{3.179847in}{1.299682in}}%
\pgfpathlineto{\pgfqpoint{3.185849in}{1.313878in}}%
\pgfpathlineto{\pgfqpoint{3.193636in}{1.327879in}}%
\pgfpathlineto{\pgfqpoint{3.203184in}{1.341624in}}%
\pgfpathlineto{\pgfqpoint{3.214466in}{1.355055in}}%
\pgfpathlineto{\pgfqpoint{3.227444in}{1.368113in}}%
\pgfpathlineto{\pgfqpoint{3.217015in}{1.368339in}}%
\pgfpathlineto{\pgfqpoint{3.206812in}{1.367967in}}%
\pgfpathlineto{\pgfqpoint{3.196875in}{1.366998in}}%
\pgfpathlineto{\pgfqpoint{3.187244in}{1.365434in}}%
\pgfpathlineto{\pgfqpoint{3.177958in}{1.363281in}}%
\pgfpathlineto{\pgfqpoint{3.163445in}{1.348581in}}%
\pgfpathlineto{\pgfqpoint{3.150853in}{1.333469in}}%
\pgfpathlineto{\pgfqpoint{3.140222in}{1.318010in}}%
\pgfpathlineto{\pgfqpoint{3.131583in}{1.302273in}}%
\pgfpathlineto{\pgfqpoint{3.124957in}{1.286323in}}%
\pgfpathclose%
\pgfusepath{fill}%
\end{pgfscope}%
\begin{pgfscope}%
\pgfpathrectangle{\pgfqpoint{2.548318in}{0.050000in}}{\pgfqpoint{2.081932in}{2.081932in}}%
\pgfusepath{clip}%
\pgfsetbuttcap%
\pgfsetroundjoin%
\definecolor{currentfill}{rgb}{0.162142,0.474838,0.558140}%
\pgfsetfillcolor{currentfill}%
\pgfsetlinewidth{0.000000pt}%
\definecolor{currentstroke}{rgb}{0.000000,0.000000,0.000000}%
\pgfsetstrokecolor{currentstroke}%
\pgfsetdash{}{0pt}%
\pgfpathmoveto{\pgfqpoint{3.426321in}{1.135418in}}%
\pgfpathlineto{\pgfqpoint{3.425462in}{1.126658in}}%
\pgfpathlineto{\pgfqpoint{3.423834in}{1.118027in}}%
\pgfpathlineto{\pgfqpoint{3.421441in}{1.109559in}}%
\pgfpathlineto{\pgfqpoint{3.418294in}{1.101289in}}%
\pgfpathlineto{\pgfqpoint{3.414402in}{1.093250in}}%
\pgfpathlineto{\pgfqpoint{3.417415in}{1.099627in}}%
\pgfpathlineto{\pgfqpoint{3.421245in}{1.105903in}}%
\pgfpathlineto{\pgfqpoint{3.425877in}{1.112050in}}%
\pgfpathlineto{\pgfqpoint{3.431295in}{1.118043in}}%
\pgfpathlineto{\pgfqpoint{3.437480in}{1.123858in}}%
\pgfpathlineto{\pgfqpoint{3.440945in}{1.131345in}}%
\pgfpathlineto{\pgfqpoint{3.443748in}{1.139176in}}%
\pgfpathlineto{\pgfqpoint{3.445878in}{1.147319in}}%
\pgfpathlineto{\pgfqpoint{3.447327in}{1.155741in}}%
\pgfpathlineto{\pgfqpoint{3.448092in}{1.164410in}}%
\pgfpathlineto{\pgfqpoint{3.442263in}{1.158903in}}%
\pgfpathlineto{\pgfqpoint{3.437154in}{1.153227in}}%
\pgfpathlineto{\pgfqpoint{3.432784in}{1.147404in}}%
\pgfpathlineto{\pgfqpoint{3.429168in}{1.141460in}}%
\pgfpathlineto{\pgfqpoint{3.426321in}{1.135418in}}%
\pgfpathclose%
\pgfusepath{fill}%
\end{pgfscope}%
\begin{pgfscope}%
\pgfpathrectangle{\pgfqpoint{2.548318in}{0.050000in}}{\pgfqpoint{2.081932in}{2.081932in}}%
\pgfusepath{clip}%
\pgfsetbuttcap%
\pgfsetroundjoin%
\definecolor{currentfill}{rgb}{0.267968,0.223549,0.512008}%
\pgfsetfillcolor{currentfill}%
\pgfsetlinewidth{0.000000pt}%
\definecolor{currentstroke}{rgb}{0.000000,0.000000,0.000000}%
\pgfsetstrokecolor{currentstroke}%
\pgfsetdash{}{0pt}%
\pgfpathmoveto{\pgfqpoint{3.106786in}{0.984412in}}%
\pgfpathlineto{\pgfqpoint{3.097668in}{0.989773in}}%
\pgfpathlineto{\pgfqpoint{3.089096in}{0.995571in}}%
\pgfpathlineto{\pgfqpoint{3.081104in}{1.001783in}}%
\pgfpathlineto{\pgfqpoint{3.073723in}{1.008382in}}%
\pgfpathlineto{\pgfqpoint{3.066984in}{1.015342in}}%
\pgfpathlineto{\pgfqpoint{3.062981in}{1.032625in}}%
\pgfpathlineto{\pgfqpoint{3.061192in}{1.050046in}}%
\pgfpathlineto{\pgfqpoint{3.061642in}{1.067535in}}%
\pgfpathlineto{\pgfqpoint{3.064347in}{1.085022in}}%
\pgfpathlineto{\pgfqpoint{3.069315in}{1.102434in}}%
\pgfpathlineto{\pgfqpoint{3.076116in}{1.094300in}}%
\pgfpathlineto{\pgfqpoint{3.083561in}{1.086431in}}%
\pgfpathlineto{\pgfqpoint{3.091621in}{1.078861in}}%
\pgfpathlineto{\pgfqpoint{3.100262in}{1.071619in}}%
\pgfpathlineto{\pgfqpoint{3.109449in}{1.064737in}}%
\pgfpathlineto{\pgfqpoint{3.104744in}{1.048690in}}%
\pgfpathlineto{\pgfqpoint{3.102135in}{1.032569in}}%
\pgfpathlineto{\pgfqpoint{3.101616in}{1.016438in}}%
\pgfpathlineto{\pgfqpoint{3.103174in}{1.000365in}}%
\pgfpathlineto{\pgfqpoint{3.106786in}{0.984412in}}%
\pgfpathclose%
\pgfusepath{fill}%
\end{pgfscope}%
\begin{pgfscope}%
\pgfpathrectangle{\pgfqpoint{2.548318in}{0.050000in}}{\pgfqpoint{2.081932in}{2.081932in}}%
\pgfusepath{clip}%
\pgfsetbuttcap%
\pgfsetroundjoin%
\definecolor{currentfill}{rgb}{0.296479,0.761561,0.424223}%
\pgfsetfillcolor{currentfill}%
\pgfsetlinewidth{0.000000pt}%
\definecolor{currentstroke}{rgb}{0.000000,0.000000,0.000000}%
\pgfsetstrokecolor{currentstroke}%
\pgfsetdash{}{0pt}%
\pgfpathmoveto{\pgfqpoint{3.418905in}{1.256782in}}%
\pgfpathlineto{\pgfqpoint{3.424675in}{1.247625in}}%
\pgfpathlineto{\pgfqpoint{3.429854in}{1.238354in}}%
\pgfpathlineto{\pgfqpoint{3.434423in}{1.229003in}}%
\pgfpathlineto{\pgfqpoint{3.438364in}{1.219610in}}%
\pgfpathlineto{\pgfqpoint{3.441660in}{1.210211in}}%
\pgfpathlineto{\pgfqpoint{3.448429in}{1.215765in}}%
\pgfpathlineto{\pgfqpoint{3.455897in}{1.221094in}}%
\pgfpathlineto{\pgfqpoint{3.464034in}{1.226175in}}%
\pgfpathlineto{\pgfqpoint{3.472808in}{1.230987in}}%
\pgfpathlineto{\pgfqpoint{3.470087in}{1.240807in}}%
\pgfpathlineto{\pgfqpoint{3.466833in}{1.250699in}}%
\pgfpathlineto{\pgfqpoint{3.463060in}{1.260624in}}%
\pgfpathlineto{\pgfqpoint{3.458782in}{1.270545in}}%
\pgfpathlineto{\pgfqpoint{3.454017in}{1.280422in}}%
\pgfpathlineto{\pgfqpoint{3.444120in}{1.274945in}}%
\pgfpathlineto{\pgfqpoint{3.434945in}{1.269162in}}%
\pgfpathlineto{\pgfqpoint{3.426529in}{1.263099in}}%
\pgfpathlineto{\pgfqpoint{3.418905in}{1.256782in}}%
\pgfpathclose%
\pgfusepath{fill}%
\end{pgfscope}%
\begin{pgfscope}%
\pgfpathrectangle{\pgfqpoint{2.548318in}{0.050000in}}{\pgfqpoint{2.081932in}{2.081932in}}%
\pgfusepath{clip}%
\pgfsetbuttcap%
\pgfsetroundjoin%
\definecolor{currentfill}{rgb}{0.866013,0.889868,0.095953}%
\pgfsetfillcolor{currentfill}%
\pgfsetlinewidth{0.000000pt}%
\definecolor{currentstroke}{rgb}{0.000000,0.000000,0.000000}%
\pgfsetstrokecolor{currentstroke}%
\pgfsetdash{}{0pt}%
\pgfpathmoveto{\pgfqpoint{3.334749in}{1.334546in}}%
\pgfpathlineto{\pgfqpoint{3.344848in}{1.328367in}}%
\pgfpathlineto{\pgfqpoint{3.354668in}{1.321757in}}%
\pgfpathlineto{\pgfqpoint{3.364171in}{1.314742in}}%
\pgfpathlineto{\pgfqpoint{3.373320in}{1.307350in}}%
\pgfpathlineto{\pgfqpoint{3.382080in}{1.299609in}}%
\pgfpathlineto{\pgfqpoint{3.391077in}{1.307148in}}%
\pgfpathlineto{\pgfqpoint{3.401019in}{1.314385in}}%
\pgfpathlineto{\pgfqpoint{3.411865in}{1.321290in}}%
\pgfpathlineto{\pgfqpoint{3.423573in}{1.327833in}}%
\pgfpathlineto{\pgfqpoint{3.416325in}{1.336662in}}%
\pgfpathlineto{\pgfqpoint{3.408753in}{1.345191in}}%
\pgfpathlineto{\pgfqpoint{3.400885in}{1.353386in}}%
\pgfpathlineto{\pgfqpoint{3.392752in}{1.361216in}}%
\pgfpathlineto{\pgfqpoint{3.384386in}{1.368649in}}%
\pgfpathlineto{\pgfqpoint{3.370360in}{1.360738in}}%
\pgfpathlineto{\pgfqpoint{3.357380in}{1.352393in}}%
\pgfpathlineto{\pgfqpoint{3.345494in}{1.343649in}}%
\pgfpathlineto{\pgfqpoint{3.334749in}{1.334546in}}%
\pgfpathclose%
\pgfusepath{fill}%
\end{pgfscope}%
\begin{pgfscope}%
\pgfpathrectangle{\pgfqpoint{2.548318in}{0.050000in}}{\pgfqpoint{2.081932in}{2.081932in}}%
\pgfusepath{clip}%
\pgfsetbuttcap%
\pgfsetroundjoin%
\definecolor{currentfill}{rgb}{0.278791,0.062145,0.386592}%
\pgfsetfillcolor{currentfill}%
\pgfsetlinewidth{0.000000pt}%
\definecolor{currentstroke}{rgb}{0.000000,0.000000,0.000000}%
\pgfsetstrokecolor{currentstroke}%
\pgfsetdash{}{0pt}%
\pgfpathmoveto{\pgfqpoint{3.336130in}{0.988146in}}%
\pgfpathlineto{\pgfqpoint{3.325592in}{0.982907in}}%
\pgfpathlineto{\pgfqpoint{3.314667in}{0.978137in}}%
\pgfpathlineto{\pgfqpoint{3.303401in}{0.973857in}}%
\pgfpathlineto{\pgfqpoint{3.291838in}{0.970083in}}%
\pgfpathlineto{\pgfqpoint{3.280028in}{0.966833in}}%
\pgfpathlineto{\pgfqpoint{3.277930in}{0.977336in}}%
\pgfpathlineto{\pgfqpoint{3.277193in}{0.987901in}}%
\pgfpathlineto{\pgfqpoint{3.277827in}{0.998486in}}%
\pgfpathlineto{\pgfqpoint{3.279836in}{1.009046in}}%
\pgfpathlineto{\pgfqpoint{3.283219in}{1.019540in}}%
\pgfpathlineto{\pgfqpoint{3.295012in}{1.020957in}}%
\pgfpathlineto{\pgfqpoint{3.306550in}{1.022939in}}%
\pgfpathlineto{\pgfqpoint{3.317786in}{1.025478in}}%
\pgfpathlineto{\pgfqpoint{3.328676in}{1.028562in}}%
\pgfpathlineto{\pgfqpoint{3.339175in}{1.032178in}}%
\pgfpathlineto{\pgfqpoint{3.336280in}{1.023421in}}%
\pgfpathlineto{\pgfqpoint{3.334529in}{1.014604in}}%
\pgfpathlineto{\pgfqpoint{3.333922in}{1.005762in}}%
\pgfpathlineto{\pgfqpoint{3.334458in}{0.996931in}}%
\pgfpathlineto{\pgfqpoint{3.336130in}{0.988146in}}%
\pgfpathclose%
\pgfusepath{fill}%
\end{pgfscope}%
\begin{pgfscope}%
\pgfpathrectangle{\pgfqpoint{2.548318in}{0.050000in}}{\pgfqpoint{2.081932in}{2.081932in}}%
\pgfusepath{clip}%
\pgfsetbuttcap%
\pgfsetroundjoin%
\definecolor{currentfill}{rgb}{0.120638,0.625828,0.533488}%
\pgfsetfillcolor{currentfill}%
\pgfsetlinewidth{0.000000pt}%
\definecolor{currentstroke}{rgb}{0.000000,0.000000,0.000000}%
\pgfsetstrokecolor{currentstroke}%
\pgfsetdash{}{0pt}%
\pgfpathmoveto{\pgfqpoint{3.764472in}{1.229801in}}%
\pgfpathlineto{\pgfqpoint{3.762261in}{1.220111in}}%
\pgfpathlineto{\pgfqpoint{3.760611in}{1.210564in}}%
\pgfpathlineto{\pgfqpoint{3.759528in}{1.201197in}}%
\pgfpathlineto{\pgfqpoint{3.759018in}{1.192047in}}%
\pgfpathlineto{\pgfqpoint{3.759083in}{1.183151in}}%
\pgfpathlineto{\pgfqpoint{3.767390in}{1.178481in}}%
\pgfpathlineto{\pgfqpoint{3.775076in}{1.173558in}}%
\pgfpathlineto{\pgfqpoint{3.782110in}{1.168404in}}%
\pgfpathlineto{\pgfqpoint{3.788466in}{1.163040in}}%
\pgfpathlineto{\pgfqpoint{3.794118in}{1.157488in}}%
\pgfpathlineto{\pgfqpoint{3.794037in}{1.166362in}}%
\pgfpathlineto{\pgfqpoint{3.794671in}{1.175383in}}%
\pgfpathlineto{\pgfqpoint{3.796017in}{1.184517in}}%
\pgfpathlineto{\pgfqpoint{3.798067in}{1.193727in}}%
\pgfpathlineto{\pgfqpoint{3.800814in}{1.202976in}}%
\pgfpathlineto{\pgfqpoint{3.794953in}{1.208779in}}%
\pgfpathlineto{\pgfqpoint{3.788362in}{1.214385in}}%
\pgfpathlineto{\pgfqpoint{3.781065in}{1.219773in}}%
\pgfpathlineto{\pgfqpoint{3.773091in}{1.224919in}}%
\pgfpathlineto{\pgfqpoint{3.764472in}{1.229801in}}%
\pgfpathclose%
\pgfusepath{fill}%
\end{pgfscope}%
\begin{pgfscope}%
\pgfpathrectangle{\pgfqpoint{2.548318in}{0.050000in}}{\pgfqpoint{2.081932in}{2.081932in}}%
\pgfusepath{clip}%
\pgfsetbuttcap%
\pgfsetroundjoin%
\definecolor{currentfill}{rgb}{0.282327,0.094955,0.417331}%
\pgfsetfillcolor{currentfill}%
\pgfsetlinewidth{0.000000pt}%
\definecolor{currentstroke}{rgb}{0.000000,0.000000,0.000000}%
\pgfsetstrokecolor{currentstroke}%
\pgfsetdash{}{0pt}%
\pgfpathmoveto{\pgfqpoint{3.159208in}{0.964890in}}%
\pgfpathlineto{\pgfqpoint{3.147952in}{0.967765in}}%
\pgfpathlineto{\pgfqpoint{3.137038in}{0.971170in}}%
\pgfpathlineto{\pgfqpoint{3.126511in}{0.975091in}}%
\pgfpathlineto{\pgfqpoint{3.116413in}{0.979511in}}%
\pgfpathlineto{\pgfqpoint{3.106786in}{0.984412in}}%
\pgfpathlineto{\pgfqpoint{3.103174in}{1.000365in}}%
\pgfpathlineto{\pgfqpoint{3.101616in}{1.016438in}}%
\pgfpathlineto{\pgfqpoint{3.102135in}{1.032569in}}%
\pgfpathlineto{\pgfqpoint{3.104744in}{1.048690in}}%
\pgfpathlineto{\pgfqpoint{3.109449in}{1.064737in}}%
\pgfpathlineto{\pgfqpoint{3.119145in}{1.058243in}}%
\pgfpathlineto{\pgfqpoint{3.129309in}{1.052164in}}%
\pgfpathlineto{\pgfqpoint{3.139900in}{1.046524in}}%
\pgfpathlineto{\pgfqpoint{3.150875in}{1.041348in}}%
\pgfpathlineto{\pgfqpoint{3.162188in}{1.036656in}}%
\pgfpathlineto{\pgfqpoint{3.157853in}{1.022334in}}%
\pgfpathlineto{\pgfqpoint{3.155393in}{1.007938in}}%
\pgfpathlineto{\pgfqpoint{3.154808in}{0.993526in}}%
\pgfpathlineto{\pgfqpoint{3.156085in}{0.979158in}}%
\pgfpathlineto{\pgfqpoint{3.159208in}{0.964890in}}%
\pgfpathclose%
\pgfusepath{fill}%
\end{pgfscope}%
\begin{pgfscope}%
\pgfpathrectangle{\pgfqpoint{2.548318in}{0.050000in}}{\pgfqpoint{2.081932in}{2.081932in}}%
\pgfusepath{clip}%
\pgfsetbuttcap%
\pgfsetroundjoin%
\definecolor{currentfill}{rgb}{0.227802,0.326594,0.546532}%
\pgfsetfillcolor{currentfill}%
\pgfsetlinewidth{0.000000pt}%
\definecolor{currentstroke}{rgb}{0.000000,0.000000,0.000000}%
\pgfsetstrokecolor{currentstroke}%
\pgfsetdash{}{0pt}%
\pgfpathmoveto{\pgfqpoint{3.805167in}{1.116549in}}%
\pgfpathlineto{\pgfqpoint{3.809452in}{1.109277in}}%
\pgfpathlineto{\pgfqpoint{3.814398in}{1.102383in}}%
\pgfpathlineto{\pgfqpoint{3.819987in}{1.095894in}}%
\pgfpathlineto{\pgfqpoint{3.826197in}{1.089837in}}%
\pgfpathlineto{\pgfqpoint{3.833005in}{1.084237in}}%
\pgfpathlineto{\pgfqpoint{3.838974in}{1.077342in}}%
\pgfpathlineto{\pgfqpoint{3.844034in}{1.070282in}}%
\pgfpathlineto{\pgfqpoint{3.848166in}{1.063084in}}%
\pgfpathlineto{\pgfqpoint{3.851357in}{1.055779in}}%
\pgfpathlineto{\pgfqpoint{3.853596in}{1.048398in}}%
\pgfpathlineto{\pgfqpoint{3.846178in}{1.055093in}}%
\pgfpathlineto{\pgfqpoint{3.839409in}{1.062147in}}%
\pgfpathlineto{\pgfqpoint{3.833315in}{1.069530in}}%
\pgfpathlineto{\pgfqpoint{3.827920in}{1.077212in}}%
\pgfpathlineto{\pgfqpoint{3.823245in}{1.085161in}}%
\pgfpathlineto{\pgfqpoint{3.821264in}{1.091630in}}%
\pgfpathlineto{\pgfqpoint{3.818454in}{1.098029in}}%
\pgfpathlineto{\pgfqpoint{3.814826in}{1.104332in}}%
\pgfpathlineto{\pgfqpoint{3.810392in}{1.110514in}}%
\pgfpathlineto{\pgfqpoint{3.805167in}{1.116549in}}%
\pgfpathclose%
\pgfusepath{fill}%
\end{pgfscope}%
\begin{pgfscope}%
\pgfpathrectangle{\pgfqpoint{2.548318in}{0.050000in}}{\pgfqpoint{2.081932in}{2.081932in}}%
\pgfusepath{clip}%
\pgfsetbuttcap%
\pgfsetroundjoin%
\definecolor{currentfill}{rgb}{0.120638,0.625828,0.533488}%
\pgfsetfillcolor{currentfill}%
\pgfsetlinewidth{0.000000pt}%
\definecolor{currentstroke}{rgb}{0.000000,0.000000,0.000000}%
\pgfsetstrokecolor{currentstroke}%
\pgfsetdash{}{0pt}%
\pgfpathmoveto{\pgfqpoint{3.441660in}{1.210211in}}%
\pgfpathlineto{\pgfqpoint{3.444298in}{1.200843in}}%
\pgfpathlineto{\pgfqpoint{3.446268in}{1.191542in}}%
\pgfpathlineto{\pgfqpoint{3.447560in}{1.182346in}}%
\pgfpathlineto{\pgfqpoint{3.448169in}{1.173290in}}%
\pgfpathlineto{\pgfqpoint{3.448092in}{1.164410in}}%
\pgfpathlineto{\pgfqpoint{3.454618in}{1.169723in}}%
\pgfpathlineto{\pgfqpoint{3.461817in}{1.174821in}}%
\pgfpathlineto{\pgfqpoint{3.469660in}{1.179682in}}%
\pgfpathlineto{\pgfqpoint{3.478116in}{1.184285in}}%
\pgfpathlineto{\pgfqpoint{3.478180in}{1.193183in}}%
\pgfpathlineto{\pgfqpoint{3.477678in}{1.202338in}}%
\pgfpathlineto{\pgfqpoint{3.476611in}{1.211715in}}%
\pgfpathlineto{\pgfqpoint{3.474985in}{1.221277in}}%
\pgfpathlineto{\pgfqpoint{3.472808in}{1.230987in}}%
\pgfpathlineto{\pgfqpoint{3.464034in}{1.226175in}}%
\pgfpathlineto{\pgfqpoint{3.455897in}{1.221094in}}%
\pgfpathlineto{\pgfqpoint{3.448429in}{1.215765in}}%
\pgfpathlineto{\pgfqpoint{3.441660in}{1.210211in}}%
\pgfpathclose%
\pgfusepath{fill}%
\end{pgfscope}%
\begin{pgfscope}%
\pgfpathrectangle{\pgfqpoint{2.548318in}{0.050000in}}{\pgfqpoint{2.081932in}{2.081932in}}%
\pgfusepath{clip}%
\pgfsetbuttcap%
\pgfsetroundjoin%
\definecolor{currentfill}{rgb}{0.267004,0.004874,0.329415}%
\pgfsetfillcolor{currentfill}%
\pgfsetlinewidth{0.000000pt}%
\definecolor{currentstroke}{rgb}{0.000000,0.000000,0.000000}%
\pgfsetstrokecolor{currentstroke}%
\pgfsetdash{}{0pt}%
\pgfpathmoveto{\pgfqpoint{3.280028in}{0.966833in}}%
\pgfpathlineto{\pgfqpoint{3.268018in}{0.964118in}}%
\pgfpathlineto{\pgfqpoint{3.255858in}{0.961951in}}%
\pgfpathlineto{\pgfqpoint{3.243598in}{0.960340in}}%
\pgfpathlineto{\pgfqpoint{3.231288in}{0.959292in}}%
\pgfpathlineto{\pgfqpoint{3.218979in}{0.958811in}}%
\pgfpathlineto{\pgfqpoint{3.216381in}{0.971203in}}%
\pgfpathlineto{\pgfqpoint{3.215389in}{0.983674in}}%
\pgfpathlineto{\pgfqpoint{3.216017in}{0.996176in}}%
\pgfpathlineto{\pgfqpoint{3.218272in}{1.008657in}}%
\pgfpathlineto{\pgfqpoint{3.222154in}{1.021067in}}%
\pgfpathlineto{\pgfqpoint{3.234481in}{1.019612in}}%
\pgfpathlineto{\pgfqpoint{3.246801in}{1.018729in}}%
\pgfpathlineto{\pgfqpoint{3.259065in}{1.018423in}}%
\pgfpathlineto{\pgfqpoint{3.271221in}{1.018693in}}%
\pgfpathlineto{\pgfqpoint{3.283219in}{1.019540in}}%
\pgfpathlineto{\pgfqpoint{3.279836in}{1.009046in}}%
\pgfpathlineto{\pgfqpoint{3.277827in}{0.998486in}}%
\pgfpathlineto{\pgfqpoint{3.277193in}{0.987901in}}%
\pgfpathlineto{\pgfqpoint{3.277930in}{0.977336in}}%
\pgfpathlineto{\pgfqpoint{3.280028in}{0.966833in}}%
\pgfpathclose%
\pgfusepath{fill}%
\end{pgfscope}%
\begin{pgfscope}%
\pgfpathrectangle{\pgfqpoint{2.548318in}{0.050000in}}{\pgfqpoint{2.081932in}{2.081932in}}%
\pgfusepath{clip}%
\pgfsetbuttcap%
\pgfsetroundjoin%
\definecolor{currentfill}{rgb}{0.993248,0.906157,0.143936}%
\pgfsetfillcolor{currentfill}%
\pgfsetlinewidth{0.000000pt}%
\definecolor{currentstroke}{rgb}{0.000000,0.000000,0.000000}%
\pgfsetstrokecolor{currentstroke}%
\pgfsetdash{}{0pt}%
\pgfpathmoveto{\pgfqpoint{3.899080in}{1.396550in}}%
\pgfpathlineto{\pgfqpoint{3.889908in}{1.391591in}}%
\pgfpathlineto{\pgfqpoint{3.880803in}{1.386102in}}%
\pgfpathlineto{\pgfqpoint{3.871799in}{1.380108in}}%
\pgfpathlineto{\pgfqpoint{3.862931in}{1.373631in}}%
\pgfpathlineto{\pgfqpoint{3.854233in}{1.366698in}}%
\pgfpathlineto{\pgfqpoint{3.868004in}{1.358675in}}%
\pgfpathlineto{\pgfqpoint{3.880717in}{1.350225in}}%
\pgfpathlineto{\pgfqpoint{3.892324in}{1.341388in}}%
\pgfpathlineto{\pgfqpoint{3.902779in}{1.332200in}}%
\pgfpathlineto{\pgfqpoint{3.912043in}{1.322702in}}%
\pgfpathlineto{\pgfqpoint{3.922791in}{1.327982in}}%
\pgfpathlineto{\pgfqpoint{3.933745in}{1.332775in}}%
\pgfpathlineto{\pgfqpoint{3.944862in}{1.337062in}}%
\pgfpathlineto{\pgfqpoint{3.956098in}{1.340826in}}%
\pgfpathlineto{\pgfqpoint{3.967409in}{1.344052in}}%
\pgfpathlineto{\pgfqpoint{3.956494in}{1.355374in}}%
\pgfpathlineto{\pgfqpoint{3.944154in}{1.366333in}}%
\pgfpathlineto{\pgfqpoint{3.930436in}{1.376879in}}%
\pgfpathlineto{\pgfqpoint{3.915392in}{1.386967in}}%
\pgfpathlineto{\pgfqpoint{3.899080in}{1.396550in}}%
\pgfpathclose%
\pgfusepath{fill}%
\end{pgfscope}%
\begin{pgfscope}%
\pgfpathrectangle{\pgfqpoint{2.548318in}{0.050000in}}{\pgfqpoint{2.081932in}{2.081932in}}%
\pgfusepath{clip}%
\pgfsetbuttcap%
\pgfsetroundjoin%
\definecolor{currentfill}{rgb}{0.268510,0.009605,0.335427}%
\pgfsetfillcolor{currentfill}%
\pgfsetlinewidth{0.000000pt}%
\definecolor{currentstroke}{rgb}{0.000000,0.000000,0.000000}%
\pgfsetstrokecolor{currentstroke}%
\pgfsetdash{}{0pt}%
\pgfpathmoveto{\pgfqpoint{3.218979in}{0.958811in}}%
\pgfpathlineto{\pgfqpoint{3.206722in}{0.958899in}}%
\pgfpathlineto{\pgfqpoint{3.194567in}{0.959555in}}%
\pgfpathlineto{\pgfqpoint{3.182564in}{0.960776in}}%
\pgfpathlineto{\pgfqpoint{3.170761in}{0.962557in}}%
\pgfpathlineto{\pgfqpoint{3.159208in}{0.964890in}}%
\pgfpathlineto{\pgfqpoint{3.156085in}{0.979158in}}%
\pgfpathlineto{\pgfqpoint{3.154808in}{0.993526in}}%
\pgfpathlineto{\pgfqpoint{3.155393in}{1.007938in}}%
\pgfpathlineto{\pgfqpoint{3.157853in}{1.022334in}}%
\pgfpathlineto{\pgfqpoint{3.162188in}{1.036656in}}%
\pgfpathlineto{\pgfqpoint{3.173792in}{1.032469in}}%
\pgfpathlineto{\pgfqpoint{3.185640in}{1.028802in}}%
\pgfpathlineto{\pgfqpoint{3.197683in}{1.025671in}}%
\pgfpathlineto{\pgfqpoint{3.209871in}{1.023089in}}%
\pgfpathlineto{\pgfqpoint{3.222154in}{1.021067in}}%
\pgfpathlineto{\pgfqpoint{3.218272in}{1.008657in}}%
\pgfpathlineto{\pgfqpoint{3.216017in}{0.996176in}}%
\pgfpathlineto{\pgfqpoint{3.215389in}{0.983674in}}%
\pgfpathlineto{\pgfqpoint{3.216381in}{0.971203in}}%
\pgfpathlineto{\pgfqpoint{3.218979in}{0.958811in}}%
\pgfpathclose%
\pgfusepath{fill}%
\end{pgfscope}%
\begin{pgfscope}%
\pgfpathrectangle{\pgfqpoint{2.548318in}{0.050000in}}{\pgfqpoint{2.081932in}{2.081932in}}%
\pgfusepath{clip}%
\pgfsetbuttcap%
\pgfsetroundjoin%
\definecolor{currentfill}{rgb}{0.327796,0.773980,0.406640}%
\pgfsetfillcolor{currentfill}%
\pgfsetlinewidth{0.000000pt}%
\definecolor{currentstroke}{rgb}{0.000000,0.000000,0.000000}%
\pgfsetstrokecolor{currentstroke}%
\pgfsetdash{}{0pt}%
\pgfpathmoveto{\pgfqpoint{4.143696in}{1.291033in}}%
\pgfpathlineto{\pgfqpoint{4.139524in}{1.298436in}}%
\pgfpathlineto{\pgfqpoint{4.134682in}{1.305454in}}%
\pgfpathlineto{\pgfqpoint{4.129194in}{1.312060in}}%
\pgfpathlineto{\pgfqpoint{4.123081in}{1.318227in}}%
\pgfpathlineto{\pgfqpoint{4.116370in}{1.323931in}}%
\pgfpathlineto{\pgfqpoint{4.129471in}{1.307401in}}%
\pgfpathlineto{\pgfqpoint{4.140418in}{1.290521in}}%
\pgfpathlineto{\pgfqpoint{4.149180in}{1.273362in}}%
\pgfpathlineto{\pgfqpoint{4.155738in}{1.256000in}}%
\pgfpathlineto{\pgfqpoint{4.160082in}{1.238507in}}%
\pgfpathlineto{\pgfqpoint{4.167287in}{1.231742in}}%
\pgfpathlineto{\pgfqpoint{4.173848in}{1.224624in}}%
\pgfpathlineto{\pgfqpoint{4.179737in}{1.217180in}}%
\pgfpathlineto{\pgfqpoint{4.184930in}{1.209440in}}%
\pgfpathlineto{\pgfqpoint{4.189404in}{1.201435in}}%
\pgfpathlineto{\pgfqpoint{4.184906in}{1.219773in}}%
\pgfpathlineto{\pgfqpoint{4.178072in}{1.237978in}}%
\pgfpathlineto{\pgfqpoint{4.168911in}{1.255975in}}%
\pgfpathlineto{\pgfqpoint{4.157442in}{1.273686in}}%
\pgfpathlineto{\pgfqpoint{4.143696in}{1.291033in}}%
\pgfpathclose%
\pgfusepath{fill}%
\end{pgfscope}%
\begin{pgfscope}%
\pgfpathrectangle{\pgfqpoint{2.548318in}{0.050000in}}{\pgfqpoint{2.081932in}{2.081932in}}%
\pgfusepath{clip}%
\pgfsetbuttcap%
\pgfsetroundjoin%
\definecolor{currentfill}{rgb}{0.636902,0.856542,0.216620}%
\pgfsetfillcolor{currentfill}%
\pgfsetlinewidth{0.000000pt}%
\definecolor{currentstroke}{rgb}{0.000000,0.000000,0.000000}%
\pgfsetstrokecolor{currentstroke}%
\pgfsetdash{}{0pt}%
\pgfpathmoveto{\pgfqpoint{3.080723in}{1.260390in}}%
\pgfpathlineto{\pgfqpoint{3.088484in}{1.266480in}}%
\pgfpathlineto{\pgfqpoint{3.096824in}{1.272139in}}%
\pgfpathlineto{\pgfqpoint{3.105707in}{1.277346in}}%
\pgfpathlineto{\pgfqpoint{3.115097in}{1.282080in}}%
\pgfpathlineto{\pgfqpoint{3.124957in}{1.286323in}}%
\pgfpathlineto{\pgfqpoint{3.131583in}{1.302273in}}%
\pgfpathlineto{\pgfqpoint{3.140222in}{1.318010in}}%
\pgfpathlineto{\pgfqpoint{3.150853in}{1.333469in}}%
\pgfpathlineto{\pgfqpoint{3.163445in}{1.348581in}}%
\pgfpathlineto{\pgfqpoint{3.177958in}{1.363281in}}%
\pgfpathlineto{\pgfqpoint{3.169056in}{1.360547in}}%
\pgfpathlineto{\pgfqpoint{3.160573in}{1.357240in}}%
\pgfpathlineto{\pgfqpoint{3.152546in}{1.353373in}}%
\pgfpathlineto{\pgfqpoint{3.145006in}{1.348961in}}%
\pgfpathlineto{\pgfqpoint{3.137986in}{1.344020in}}%
\pgfpathlineto{\pgfqpoint{3.122252in}{1.328032in}}%
\pgfpathlineto{\pgfqpoint{3.108621in}{1.311602in}}%
\pgfpathlineto{\pgfqpoint{3.097136in}{1.294803in}}%
\pgfpathlineto{\pgfqpoint{3.087829in}{1.277708in}}%
\pgfpathlineto{\pgfqpoint{3.080723in}{1.260390in}}%
\pgfpathclose%
\pgfusepath{fill}%
\end{pgfscope}%
\begin{pgfscope}%
\pgfpathrectangle{\pgfqpoint{2.548318in}{0.050000in}}{\pgfqpoint{2.081932in}{2.081932in}}%
\pgfusepath{clip}%
\pgfsetbuttcap%
\pgfsetroundjoin%
\definecolor{currentfill}{rgb}{0.227802,0.326594,0.546532}%
\pgfsetfillcolor{currentfill}%
\pgfsetlinewidth{0.000000pt}%
\definecolor{currentstroke}{rgb}{0.000000,0.000000,0.000000}%
\pgfsetstrokecolor{currentstroke}%
\pgfsetdash{}{0pt}%
\pgfpathmoveto{\pgfqpoint{3.414402in}{1.093250in}}%
\pgfpathlineto{\pgfqpoint{3.409780in}{1.085474in}}%
\pgfpathlineto{\pgfqpoint{3.404446in}{1.077995in}}%
\pgfpathlineto{\pgfqpoint{3.398421in}{1.070841in}}%
\pgfpathlineto{\pgfqpoint{3.391728in}{1.064043in}}%
\pgfpathlineto{\pgfqpoint{3.384392in}{1.057628in}}%
\pgfpathlineto{\pgfqpoint{3.387817in}{1.064909in}}%
\pgfpathlineto{\pgfqpoint{3.392181in}{1.072075in}}%
\pgfpathlineto{\pgfqpoint{3.397469in}{1.079097in}}%
\pgfpathlineto{\pgfqpoint{3.403661in}{1.085945in}}%
\pgfpathlineto{\pgfqpoint{3.410735in}{1.092590in}}%
\pgfpathlineto{\pgfqpoint{3.417276in}{1.097933in}}%
\pgfpathlineto{\pgfqpoint{3.423243in}{1.103756in}}%
\pgfpathlineto{\pgfqpoint{3.428612in}{1.110035in}}%
\pgfpathlineto{\pgfqpoint{3.433363in}{1.116745in}}%
\pgfpathlineto{\pgfqpoint{3.437480in}{1.123858in}}%
\pgfpathlineto{\pgfqpoint{3.431295in}{1.118043in}}%
\pgfpathlineto{\pgfqpoint{3.425877in}{1.112050in}}%
\pgfpathlineto{\pgfqpoint{3.421245in}{1.105903in}}%
\pgfpathlineto{\pgfqpoint{3.417415in}{1.099627in}}%
\pgfpathlineto{\pgfqpoint{3.414402in}{1.093250in}}%
\pgfpathclose%
\pgfusepath{fill}%
\end{pgfscope}%
\begin{pgfscope}%
\pgfpathrectangle{\pgfqpoint{2.548318in}{0.050000in}}{\pgfqpoint{2.081932in}{2.081932in}}%
\pgfusepath{clip}%
\pgfsetbuttcap%
\pgfsetroundjoin%
\definecolor{currentfill}{rgb}{0.993248,0.906157,0.143936}%
\pgfsetfillcolor{currentfill}%
\pgfsetlinewidth{0.000000pt}%
\definecolor{currentstroke}{rgb}{0.000000,0.000000,0.000000}%
\pgfsetstrokecolor{currentstroke}%
\pgfsetdash{}{0pt}%
\pgfpathmoveto{\pgfqpoint{3.281480in}{1.358172in}}%
\pgfpathlineto{\pgfqpoint{3.292366in}{1.354483in}}%
\pgfpathlineto{\pgfqpoint{3.303178in}{1.350257in}}%
\pgfpathlineto{\pgfqpoint{3.313873in}{1.345513in}}%
\pgfpathlineto{\pgfqpoint{3.324411in}{1.340269in}}%
\pgfpathlineto{\pgfqpoint{3.334749in}{1.334546in}}%
\pgfpathlineto{\pgfqpoint{3.345494in}{1.343649in}}%
\pgfpathlineto{\pgfqpoint{3.357380in}{1.352393in}}%
\pgfpathlineto{\pgfqpoint{3.370360in}{1.360738in}}%
\pgfpathlineto{\pgfqpoint{3.384386in}{1.368649in}}%
\pgfpathlineto{\pgfqpoint{3.375817in}{1.375655in}}%
\pgfpathlineto{\pgfqpoint{3.367080in}{1.382207in}}%
\pgfpathlineto{\pgfqpoint{3.358209in}{1.388278in}}%
\pgfpathlineto{\pgfqpoint{3.349238in}{1.393844in}}%
\pgfpathlineto{\pgfqpoint{3.340202in}{1.398881in}}%
\pgfpathlineto{\pgfqpoint{3.323583in}{1.389431in}}%
\pgfpathlineto{\pgfqpoint{3.308218in}{1.379466in}}%
\pgfpathlineto{\pgfqpoint{3.294166in}{1.369031in}}%
\pgfpathlineto{\pgfqpoint{3.281480in}{1.358172in}}%
\pgfpathclose%
\pgfusepath{fill}%
\end{pgfscope}%
\begin{pgfscope}%
\pgfpathrectangle{\pgfqpoint{2.548318in}{0.050000in}}{\pgfqpoint{2.081932in}{2.081932in}}%
\pgfusepath{clip}%
\pgfsetbuttcap%
\pgfsetroundjoin%
\definecolor{currentfill}{rgb}{0.162142,0.474838,0.558140}%
\pgfsetfillcolor{currentfill}%
\pgfsetlinewidth{0.000000pt}%
\definecolor{currentstroke}{rgb}{0.000000,0.000000,0.000000}%
\pgfsetstrokecolor{currentstroke}%
\pgfsetdash{}{0pt}%
\pgfpathmoveto{\pgfqpoint{3.759083in}{1.183151in}}%
\pgfpathlineto{\pgfqpoint{3.759723in}{1.174542in}}%
\pgfpathlineto{\pgfqpoint{3.760938in}{1.166256in}}%
\pgfpathlineto{\pgfqpoint{3.762723in}{1.158325in}}%
\pgfpathlineto{\pgfqpoint{3.765072in}{1.150780in}}%
\pgfpathlineto{\pgfqpoint{3.767977in}{1.143653in}}%
\pgfpathlineto{\pgfqpoint{3.776799in}{1.138720in}}%
\pgfpathlineto{\pgfqpoint{3.784959in}{1.133520in}}%
\pgfpathlineto{\pgfqpoint{3.792426in}{1.128076in}}%
\pgfpathlineto{\pgfqpoint{3.799171in}{1.122411in}}%
\pgfpathlineto{\pgfqpoint{3.805167in}{1.116549in}}%
\pgfpathlineto{\pgfqpoint{3.801558in}{1.124168in}}%
\pgfpathlineto{\pgfqpoint{3.798640in}{1.132104in}}%
\pgfpathlineto{\pgfqpoint{3.796423in}{1.140325in}}%
\pgfpathlineto{\pgfqpoint{3.794914in}{1.148798in}}%
\pgfpathlineto{\pgfqpoint{3.794118in}{1.157488in}}%
\pgfpathlineto{\pgfqpoint{3.788466in}{1.163040in}}%
\pgfpathlineto{\pgfqpoint{3.782110in}{1.168404in}}%
\pgfpathlineto{\pgfqpoint{3.775076in}{1.173558in}}%
\pgfpathlineto{\pgfqpoint{3.767390in}{1.178481in}}%
\pgfpathlineto{\pgfqpoint{3.759083in}{1.183151in}}%
\pgfpathclose%
\pgfusepath{fill}%
\end{pgfscope}%
\begin{pgfscope}%
\pgfpathrectangle{\pgfqpoint{2.548318in}{0.050000in}}{\pgfqpoint{2.081932in}{2.081932in}}%
\pgfusepath{clip}%
\pgfsetbuttcap%
\pgfsetroundjoin%
\definecolor{currentfill}{rgb}{0.296479,0.761561,0.424223}%
\pgfsetfillcolor{currentfill}%
\pgfsetlinewidth{0.000000pt}%
\definecolor{currentstroke}{rgb}{0.000000,0.000000,0.000000}%
\pgfsetstrokecolor{currentstroke}%
\pgfsetdash{}{0pt}%
\pgfpathmoveto{\pgfqpoint{3.725430in}{1.301577in}}%
\pgfpathlineto{\pgfqpoint{3.722275in}{1.291051in}}%
\pgfpathlineto{\pgfqpoint{3.719444in}{1.280547in}}%
\pgfpathlineto{\pgfqpoint{3.716947in}{1.270104in}}%
\pgfpathlineto{\pgfqpoint{3.714794in}{1.259764in}}%
\pgfpathlineto{\pgfqpoint{3.712994in}{1.249566in}}%
\pgfpathlineto{\pgfqpoint{3.724271in}{1.246292in}}%
\pgfpathlineto{\pgfqpoint{3.735100in}{1.242662in}}%
\pgfpathlineto{\pgfqpoint{3.745438in}{1.238692in}}%
\pgfpathlineto{\pgfqpoint{3.755242in}{1.234399in}}%
\pgfpathlineto{\pgfqpoint{3.764472in}{1.229801in}}%
\pgfpathlineto{\pgfqpoint{3.767235in}{1.239597in}}%
\pgfpathlineto{\pgfqpoint{3.770538in}{1.249460in}}%
\pgfpathlineto{\pgfqpoint{3.774368in}{1.259353in}}%
\pgfpathlineto{\pgfqpoint{3.778711in}{1.269236in}}%
\pgfpathlineto{\pgfqpoint{3.783549in}{1.279072in}}%
\pgfpathlineto{\pgfqpoint{3.773135in}{1.284305in}}%
\pgfpathlineto{\pgfqpoint{3.762070in}{1.289193in}}%
\pgfpathlineto{\pgfqpoint{3.750398in}{1.293714in}}%
\pgfpathlineto{\pgfqpoint{3.738168in}{1.297848in}}%
\pgfpathlineto{\pgfqpoint{3.725430in}{1.301577in}}%
\pgfpathclose%
\pgfusepath{fill}%
\end{pgfscope}%
\begin{pgfscope}%
\pgfpathrectangle{\pgfqpoint{2.548318in}{0.050000in}}{\pgfqpoint{2.081932in}{2.081932in}}%
\pgfusepath{clip}%
\pgfsetbuttcap%
\pgfsetroundjoin%
\definecolor{currentfill}{rgb}{0.296479,0.761561,0.424223}%
\pgfsetfillcolor{currentfill}%
\pgfsetlinewidth{0.000000pt}%
\definecolor{currentstroke}{rgb}{0.000000,0.000000,0.000000}%
\pgfsetstrokecolor{currentstroke}%
\pgfsetdash{}{0pt}%
\pgfpathmoveto{\pgfqpoint{3.454017in}{1.280422in}}%
\pgfpathlineto{\pgfqpoint{3.458782in}{1.270545in}}%
\pgfpathlineto{\pgfqpoint{3.463060in}{1.260624in}}%
\pgfpathlineto{\pgfqpoint{3.466833in}{1.250699in}}%
\pgfpathlineto{\pgfqpoint{3.470087in}{1.240807in}}%
\pgfpathlineto{\pgfqpoint{3.472808in}{1.230987in}}%
\pgfpathlineto{\pgfqpoint{3.482184in}{1.235510in}}%
\pgfpathlineto{\pgfqpoint{3.492124in}{1.239723in}}%
\pgfpathlineto{\pgfqpoint{3.502588in}{1.243609in}}%
\pgfpathlineto{\pgfqpoint{3.513533in}{1.247151in}}%
\pgfpathlineto{\pgfqpoint{3.524914in}{1.250333in}}%
\pgfpathlineto{\pgfqpoint{3.523167in}{1.260547in}}%
\pgfpathlineto{\pgfqpoint{3.521079in}{1.270906in}}%
\pgfpathlineto{\pgfqpoint{3.518656in}{1.281370in}}%
\pgfpathlineto{\pgfqpoint{3.515909in}{1.291898in}}%
\pgfpathlineto{\pgfqpoint{3.512849in}{1.302451in}}%
\pgfpathlineto{\pgfqpoint{3.499993in}{1.298827in}}%
\pgfpathlineto{\pgfqpoint{3.487632in}{1.294793in}}%
\pgfpathlineto{\pgfqpoint{3.475817in}{1.290368in}}%
\pgfpathlineto{\pgfqpoint{3.464597in}{1.285570in}}%
\pgfpathlineto{\pgfqpoint{3.454017in}{1.280422in}}%
\pgfpathclose%
\pgfusepath{fill}%
\end{pgfscope}%
\begin{pgfscope}%
\pgfpathrectangle{\pgfqpoint{2.548318in}{0.050000in}}{\pgfqpoint{2.081932in}{2.081932in}}%
\pgfusepath{clip}%
\pgfsetbuttcap%
\pgfsetroundjoin%
\definecolor{currentfill}{rgb}{0.606045,0.850733,0.236712}%
\pgfsetfillcolor{currentfill}%
\pgfsetlinewidth{0.000000pt}%
\definecolor{currentstroke}{rgb}{0.000000,0.000000,0.000000}%
\pgfsetstrokecolor{currentstroke}%
\pgfsetdash{}{0pt}%
\pgfpathmoveto{\pgfqpoint{3.745597in}{1.353121in}}%
\pgfpathlineto{\pgfqpoint{3.741028in}{1.343090in}}%
\pgfpathlineto{\pgfqpoint{3.736710in}{1.332881in}}%
\pgfpathlineto{\pgfqpoint{3.732661in}{1.322533in}}%
\pgfpathlineto{\pgfqpoint{3.728897in}{1.312085in}}%
\pgfpathlineto{\pgfqpoint{3.725430in}{1.301577in}}%
\pgfpathlineto{\pgfqpoint{3.738168in}{1.297848in}}%
\pgfpathlineto{\pgfqpoint{3.750398in}{1.293714in}}%
\pgfpathlineto{\pgfqpoint{3.762070in}{1.289193in}}%
\pgfpathlineto{\pgfqpoint{3.773135in}{1.284305in}}%
\pgfpathlineto{\pgfqpoint{3.783549in}{1.279072in}}%
\pgfpathlineto{\pgfqpoint{3.788864in}{1.288822in}}%
\pgfpathlineto{\pgfqpoint{3.794635in}{1.298449in}}%
\pgfpathlineto{\pgfqpoint{3.800840in}{1.307915in}}%
\pgfpathlineto{\pgfqpoint{3.807455in}{1.317184in}}%
\pgfpathlineto{\pgfqpoint{3.814455in}{1.326220in}}%
\pgfpathlineto{\pgfqpoint{3.802129in}{1.332472in}}%
\pgfpathlineto{\pgfqpoint{3.789025in}{1.338314in}}%
\pgfpathlineto{\pgfqpoint{3.775197in}{1.343718in}}%
\pgfpathlineto{\pgfqpoint{3.760701in}{1.348661in}}%
\pgfpathlineto{\pgfqpoint{3.745597in}{1.353121in}}%
\pgfpathclose%
\pgfusepath{fill}%
\end{pgfscope}%
\begin{pgfscope}%
\pgfpathrectangle{\pgfqpoint{2.548318in}{0.050000in}}{\pgfqpoint{2.081932in}{2.081932in}}%
\pgfusepath{clip}%
\pgfsetbuttcap%
\pgfsetroundjoin%
\definecolor{currentfill}{rgb}{0.606045,0.850733,0.236712}%
\pgfsetfillcolor{currentfill}%
\pgfsetlinewidth{0.000000pt}%
\definecolor{currentstroke}{rgb}{0.000000,0.000000,0.000000}%
\pgfsetstrokecolor{currentstroke}%
\pgfsetdash{}{0pt}%
\pgfpathmoveto{\pgfqpoint{3.423573in}{1.327833in}}%
\pgfpathlineto{\pgfqpoint{3.430468in}{1.318738in}}%
\pgfpathlineto{\pgfqpoint{3.436984in}{1.309413in}}%
\pgfpathlineto{\pgfqpoint{3.443097in}{1.299894in}}%
\pgfpathlineto{\pgfqpoint{3.448781in}{1.290217in}}%
\pgfpathlineto{\pgfqpoint{3.454017in}{1.280422in}}%
\pgfpathlineto{\pgfqpoint{3.464597in}{1.285570in}}%
\pgfpathlineto{\pgfqpoint{3.475817in}{1.290368in}}%
\pgfpathlineto{\pgfqpoint{3.487632in}{1.294793in}}%
\pgfpathlineto{\pgfqpoint{3.499993in}{1.298827in}}%
\pgfpathlineto{\pgfqpoint{3.512849in}{1.302451in}}%
\pgfpathlineto{\pgfqpoint{3.509486in}{1.312988in}}%
\pgfpathlineto{\pgfqpoint{3.505833in}{1.323468in}}%
\pgfpathlineto{\pgfqpoint{3.501905in}{1.333851in}}%
\pgfpathlineto{\pgfqpoint{3.497716in}{1.344097in}}%
\pgfpathlineto{\pgfqpoint{3.493283in}{1.354166in}}%
\pgfpathlineto{\pgfqpoint{3.478038in}{1.349831in}}%
\pgfpathlineto{\pgfqpoint{3.463385in}{1.345008in}}%
\pgfpathlineto{\pgfqpoint{3.449386in}{1.339717in}}%
\pgfpathlineto{\pgfqpoint{3.436097in}{1.333984in}}%
\pgfpathlineto{\pgfqpoint{3.423573in}{1.327833in}}%
\pgfpathclose%
\pgfusepath{fill}%
\end{pgfscope}%
\begin{pgfscope}%
\pgfpathrectangle{\pgfqpoint{2.548318in}{0.050000in}}{\pgfqpoint{2.081932in}{2.081932in}}%
\pgfusepath{clip}%
\pgfsetbuttcap%
\pgfsetroundjoin%
\definecolor{currentfill}{rgb}{0.162142,0.474838,0.558140}%
\pgfsetfillcolor{currentfill}%
\pgfsetlinewidth{0.000000pt}%
\definecolor{currentstroke}{rgb}{0.000000,0.000000,0.000000}%
\pgfsetstrokecolor{currentstroke}%
\pgfsetdash{}{0pt}%
\pgfpathmoveto{\pgfqpoint{3.448092in}{1.164410in}}%
\pgfpathlineto{\pgfqpoint{3.447327in}{1.155741in}}%
\pgfpathlineto{\pgfqpoint{3.445878in}{1.147319in}}%
\pgfpathlineto{\pgfqpoint{3.443748in}{1.139176in}}%
\pgfpathlineto{\pgfqpoint{3.440945in}{1.131345in}}%
\pgfpathlineto{\pgfqpoint{3.437480in}{1.123858in}}%
\pgfpathlineto{\pgfqpoint{3.444406in}{1.129470in}}%
\pgfpathlineto{\pgfqpoint{3.452048in}{1.134854in}}%
\pgfpathlineto{\pgfqpoint{3.460376in}{1.139989in}}%
\pgfpathlineto{\pgfqpoint{3.469357in}{1.144851in}}%
\pgfpathlineto{\pgfqpoint{3.472218in}{1.151957in}}%
\pgfpathlineto{\pgfqpoint{3.474532in}{1.159484in}}%
\pgfpathlineto{\pgfqpoint{3.476289in}{1.167402in}}%
\pgfpathlineto{\pgfqpoint{3.477486in}{1.175680in}}%
\pgfpathlineto{\pgfqpoint{3.478116in}{1.184285in}}%
\pgfpathlineto{\pgfqpoint{3.469660in}{1.179682in}}%
\pgfpathlineto{\pgfqpoint{3.461817in}{1.174821in}}%
\pgfpathlineto{\pgfqpoint{3.454618in}{1.169723in}}%
\pgfpathlineto{\pgfqpoint{3.448092in}{1.164410in}}%
\pgfpathclose%
\pgfusepath{fill}%
\end{pgfscope}%
\begin{pgfscope}%
\pgfpathrectangle{\pgfqpoint{2.548318in}{0.050000in}}{\pgfqpoint{2.081932in}{2.081932in}}%
\pgfusepath{clip}%
\pgfsetbuttcap%
\pgfsetroundjoin%
\definecolor{currentfill}{rgb}{0.278012,0.180367,0.486697}%
\pgfsetfillcolor{currentfill}%
\pgfsetlinewidth{0.000000pt}%
\definecolor{currentstroke}{rgb}{0.000000,0.000000,0.000000}%
\pgfsetstrokecolor{currentstroke}%
\pgfsetdash{}{0pt}%
\pgfpathmoveto{\pgfqpoint{3.833005in}{1.084237in}}%
\pgfpathlineto{\pgfqpoint{3.840384in}{1.079118in}}%
\pgfpathlineto{\pgfqpoint{3.848305in}{1.074501in}}%
\pgfpathlineto{\pgfqpoint{3.856738in}{1.070405in}}%
\pgfpathlineto{\pgfqpoint{3.865650in}{1.066849in}}%
\pgfpathlineto{\pgfqpoint{3.875006in}{1.063847in}}%
\pgfpathlineto{\pgfqpoint{3.882084in}{1.055637in}}%
\pgfpathlineto{\pgfqpoint{3.888071in}{1.047232in}}%
\pgfpathlineto{\pgfqpoint{3.892946in}{1.038667in}}%
\pgfpathlineto{\pgfqpoint{3.896695in}{1.029979in}}%
\pgfpathlineto{\pgfqpoint{3.899306in}{1.021204in}}%
\pgfpathlineto{\pgfqpoint{3.889133in}{1.025726in}}%
\pgfpathlineto{\pgfqpoint{3.879438in}{1.030728in}}%
\pgfpathlineto{\pgfqpoint{3.870259in}{1.036190in}}%
\pgfpathlineto{\pgfqpoint{3.861634in}{1.042088in}}%
\pgfpathlineto{\pgfqpoint{3.853596in}{1.048398in}}%
\pgfpathlineto{\pgfqpoint{3.851357in}{1.055779in}}%
\pgfpathlineto{\pgfqpoint{3.848166in}{1.063084in}}%
\pgfpathlineto{\pgfqpoint{3.844034in}{1.070282in}}%
\pgfpathlineto{\pgfqpoint{3.838974in}{1.077342in}}%
\pgfpathlineto{\pgfqpoint{3.833005in}{1.084237in}}%
\pgfpathclose%
\pgfusepath{fill}%
\end{pgfscope}%
\begin{pgfscope}%
\pgfpathrectangle{\pgfqpoint{2.548318in}{0.050000in}}{\pgfqpoint{2.081932in}{2.081932in}}%
\pgfusepath{clip}%
\pgfsetbuttcap%
\pgfsetroundjoin%
\definecolor{currentfill}{rgb}{0.120638,0.625828,0.533488}%
\pgfsetfillcolor{currentfill}%
\pgfsetlinewidth{0.000000pt}%
\definecolor{currentstroke}{rgb}{0.000000,0.000000,0.000000}%
\pgfsetstrokecolor{currentstroke}%
\pgfsetdash{}{0pt}%
\pgfpathmoveto{\pgfqpoint{3.712994in}{1.249566in}}%
\pgfpathlineto{\pgfqpoint{3.711553in}{1.239549in}}%
\pgfpathlineto{\pgfqpoint{3.710478in}{1.229751in}}%
\pgfpathlineto{\pgfqpoint{3.709773in}{1.220212in}}%
\pgfpathlineto{\pgfqpoint{3.709440in}{1.210967in}}%
\pgfpathlineto{\pgfqpoint{3.709483in}{1.202054in}}%
\pgfpathlineto{\pgfqpoint{3.720347in}{1.198923in}}%
\pgfpathlineto{\pgfqpoint{3.730780in}{1.195451in}}%
\pgfpathlineto{\pgfqpoint{3.740741in}{1.191654in}}%
\pgfpathlineto{\pgfqpoint{3.750188in}{1.187548in}}%
\pgfpathlineto{\pgfqpoint{3.759083in}{1.183151in}}%
\pgfpathlineto{\pgfqpoint{3.759018in}{1.192047in}}%
\pgfpathlineto{\pgfqpoint{3.759528in}{1.201197in}}%
\pgfpathlineto{\pgfqpoint{3.760611in}{1.210564in}}%
\pgfpathlineto{\pgfqpoint{3.762261in}{1.220111in}}%
\pgfpathlineto{\pgfqpoint{3.764472in}{1.229801in}}%
\pgfpathlineto{\pgfqpoint{3.755242in}{1.234399in}}%
\pgfpathlineto{\pgfqpoint{3.745438in}{1.238692in}}%
\pgfpathlineto{\pgfqpoint{3.735100in}{1.242662in}}%
\pgfpathlineto{\pgfqpoint{3.724271in}{1.246292in}}%
\pgfpathlineto{\pgfqpoint{3.712994in}{1.249566in}}%
\pgfpathclose%
\pgfusepath{fill}%
\end{pgfscope}%
\begin{pgfscope}%
\pgfpathrectangle{\pgfqpoint{2.548318in}{0.050000in}}{\pgfqpoint{2.081932in}{2.081932in}}%
\pgfusepath{clip}%
\pgfsetbuttcap%
\pgfsetroundjoin%
\definecolor{currentfill}{rgb}{0.993248,0.906157,0.143936}%
\pgfsetfillcolor{currentfill}%
\pgfsetlinewidth{0.000000pt}%
\definecolor{currentstroke}{rgb}{0.000000,0.000000,0.000000}%
\pgfsetstrokecolor{currentstroke}%
\pgfsetdash{}{0pt}%
\pgfpathmoveto{\pgfqpoint{3.944654in}{1.412770in}}%
\pgfpathlineto{\pgfqpoint{3.935698in}{1.410721in}}%
\pgfpathlineto{\pgfqpoint{3.926626in}{1.408061in}}%
\pgfpathlineto{\pgfqpoint{3.917475in}{1.404803in}}%
\pgfpathlineto{\pgfqpoint{3.908280in}{1.400961in}}%
\pgfpathlineto{\pgfqpoint{3.899080in}{1.396550in}}%
\pgfpathlineto{\pgfqpoint{3.915392in}{1.386967in}}%
\pgfpathlineto{\pgfqpoint{3.930436in}{1.376879in}}%
\pgfpathlineto{\pgfqpoint{3.944154in}{1.366333in}}%
\pgfpathlineto{\pgfqpoint{3.956494in}{1.355374in}}%
\pgfpathlineto{\pgfqpoint{3.967409in}{1.344052in}}%
\pgfpathlineto{\pgfqpoint{3.978752in}{1.346725in}}%
\pgfpathlineto{\pgfqpoint{3.990080in}{1.348835in}}%
\pgfpathlineto{\pgfqpoint{4.001350in}{1.350372in}}%
\pgfpathlineto{\pgfqpoint{4.012516in}{1.351330in}}%
\pgfpathlineto{\pgfqpoint{4.023535in}{1.351704in}}%
\pgfpathlineto{\pgfqpoint{4.010975in}{1.364861in}}%
\pgfpathlineto{\pgfqpoint{3.996751in}{1.377602in}}%
\pgfpathlineto{\pgfqpoint{3.980916in}{1.389870in}}%
\pgfpathlineto{\pgfqpoint{3.963528in}{1.401611in}}%
\pgfpathlineto{\pgfqpoint{3.944654in}{1.412770in}}%
\pgfpathclose%
\pgfusepath{fill}%
\end{pgfscope}%
\begin{pgfscope}%
\pgfpathrectangle{\pgfqpoint{2.548318in}{0.050000in}}{\pgfqpoint{2.081932in}{2.081932in}}%
\pgfusepath{clip}%
\pgfsetbuttcap%
\pgfsetroundjoin%
\definecolor{currentfill}{rgb}{0.120638,0.625828,0.533488}%
\pgfsetfillcolor{currentfill}%
\pgfsetlinewidth{0.000000pt}%
\definecolor{currentstroke}{rgb}{0.000000,0.000000,0.000000}%
\pgfsetstrokecolor{currentstroke}%
\pgfsetdash{}{0pt}%
\pgfpathmoveto{\pgfqpoint{3.472808in}{1.230987in}}%
\pgfpathlineto{\pgfqpoint{3.474985in}{1.221277in}}%
\pgfpathlineto{\pgfqpoint{3.476611in}{1.211715in}}%
\pgfpathlineto{\pgfqpoint{3.477678in}{1.202338in}}%
\pgfpathlineto{\pgfqpoint{3.478180in}{1.193183in}}%
\pgfpathlineto{\pgfqpoint{3.478116in}{1.184285in}}%
\pgfpathlineto{\pgfqpoint{3.487152in}{1.188611in}}%
\pgfpathlineto{\pgfqpoint{3.496730in}{1.192641in}}%
\pgfpathlineto{\pgfqpoint{3.506812in}{1.196357in}}%
\pgfpathlineto{\pgfqpoint{3.517357in}{1.199744in}}%
\pgfpathlineto{\pgfqpoint{3.528321in}{1.202787in}}%
\pgfpathlineto{\pgfqpoint{3.528361in}{1.211702in}}%
\pgfpathlineto{\pgfqpoint{3.528039in}{1.220950in}}%
\pgfpathlineto{\pgfqpoint{3.527355in}{1.230496in}}%
\pgfpathlineto{\pgfqpoint{3.526311in}{1.240303in}}%
\pgfpathlineto{\pgfqpoint{3.524914in}{1.250333in}}%
\pgfpathlineto{\pgfqpoint{3.513533in}{1.247151in}}%
\pgfpathlineto{\pgfqpoint{3.502588in}{1.243609in}}%
\pgfpathlineto{\pgfqpoint{3.492124in}{1.239723in}}%
\pgfpathlineto{\pgfqpoint{3.482184in}{1.235510in}}%
\pgfpathlineto{\pgfqpoint{3.472808in}{1.230987in}}%
\pgfpathclose%
\pgfusepath{fill}%
\end{pgfscope}%
\begin{pgfscope}%
\pgfpathrectangle{\pgfqpoint{2.548318in}{0.050000in}}{\pgfqpoint{2.081932in}{2.081932in}}%
\pgfusepath{clip}%
\pgfsetbuttcap%
\pgfsetroundjoin%
\definecolor{currentfill}{rgb}{0.124780,0.640461,0.527068}%
\pgfsetfillcolor{currentfill}%
\pgfsetlinewidth{0.000000pt}%
\definecolor{currentstroke}{rgb}{0.000000,0.000000,0.000000}%
\pgfsetstrokecolor{currentstroke}%
\pgfsetdash{}{0pt}%
\pgfpathmoveto{\pgfqpoint{4.153997in}{1.249335in}}%
\pgfpathlineto{\pgfqpoint{4.153382in}{1.258189in}}%
\pgfpathlineto{\pgfqpoint{4.152036in}{1.266820in}}%
\pgfpathlineto{\pgfqpoint{4.149967in}{1.275194in}}%
\pgfpathlineto{\pgfqpoint{4.147183in}{1.283275in}}%
\pgfpathlineto{\pgfqpoint{4.143696in}{1.291033in}}%
\pgfpathlineto{\pgfqpoint{4.157442in}{1.273686in}}%
\pgfpathlineto{\pgfqpoint{4.168911in}{1.255975in}}%
\pgfpathlineto{\pgfqpoint{4.178072in}{1.237978in}}%
\pgfpathlineto{\pgfqpoint{4.184906in}{1.219773in}}%
\pgfpathlineto{\pgfqpoint{4.189404in}{1.201435in}}%
\pgfpathlineto{\pgfqpoint{4.193142in}{1.193197in}}%
\pgfpathlineto{\pgfqpoint{4.196127in}{1.184758in}}%
\pgfpathlineto{\pgfqpoint{4.198345in}{1.176152in}}%
\pgfpathlineto{\pgfqpoint{4.199787in}{1.167414in}}%
\pgfpathlineto{\pgfqpoint{4.200446in}{1.158578in}}%
\pgfpathlineto{\pgfqpoint{4.195893in}{1.177149in}}%
\pgfpathlineto{\pgfqpoint{4.188957in}{1.195588in}}%
\pgfpathlineto{\pgfqpoint{4.179647in}{1.213818in}}%
\pgfpathlineto{\pgfqpoint{4.167984in}{1.231760in}}%
\pgfpathlineto{\pgfqpoint{4.153997in}{1.249335in}}%
\pgfpathclose%
\pgfusepath{fill}%
\end{pgfscope}%
\begin{pgfscope}%
\pgfpathrectangle{\pgfqpoint{2.548318in}{0.050000in}}{\pgfqpoint{2.081932in}{2.081932in}}%
\pgfusepath{clip}%
\pgfsetbuttcap%
\pgfsetroundjoin%
\definecolor{currentfill}{rgb}{0.855810,0.888601,0.097452}%
\pgfsetfillcolor{currentfill}%
\pgfsetlinewidth{0.000000pt}%
\definecolor{currentstroke}{rgb}{0.000000,0.000000,0.000000}%
\pgfsetstrokecolor{currentstroke}%
\pgfsetdash{}{0pt}%
\pgfpathmoveto{\pgfqpoint{3.771592in}{1.399255in}}%
\pgfpathlineto{\pgfqpoint{3.766039in}{1.390686in}}%
\pgfpathlineto{\pgfqpoint{3.760642in}{1.381753in}}%
\pgfpathlineto{\pgfqpoint{3.755423in}{1.372490in}}%
\pgfpathlineto{\pgfqpoint{3.750402in}{1.362933in}}%
\pgfpathlineto{\pgfqpoint{3.745597in}{1.353121in}}%
\pgfpathlineto{\pgfqpoint{3.760701in}{1.348661in}}%
\pgfpathlineto{\pgfqpoint{3.775197in}{1.343718in}}%
\pgfpathlineto{\pgfqpoint{3.789025in}{1.338314in}}%
\pgfpathlineto{\pgfqpoint{3.802129in}{1.332472in}}%
\pgfpathlineto{\pgfqpoint{3.814455in}{1.326220in}}%
\pgfpathlineto{\pgfqpoint{3.821812in}{1.334986in}}%
\pgfpathlineto{\pgfqpoint{3.829499in}{1.343450in}}%
\pgfpathlineto{\pgfqpoint{3.837485in}{1.351578in}}%
\pgfpathlineto{\pgfqpoint{3.845740in}{1.359338in}}%
\pgfpathlineto{\pgfqpoint{3.854233in}{1.366698in}}%
\pgfpathlineto{\pgfqpoint{3.839460in}{1.374261in}}%
\pgfpathlineto{\pgfqpoint{3.823742in}{1.381328in}}%
\pgfpathlineto{\pgfqpoint{3.807146in}{1.387869in}}%
\pgfpathlineto{\pgfqpoint{3.789738in}{1.393853in}}%
\pgfpathlineto{\pgfqpoint{3.771592in}{1.399255in}}%
\pgfpathclose%
\pgfusepath{fill}%
\end{pgfscope}%
\begin{pgfscope}%
\pgfpathrectangle{\pgfqpoint{2.548318in}{0.050000in}}{\pgfqpoint{2.081932in}{2.081932in}}%
\pgfusepath{clip}%
\pgfsetbuttcap%
\pgfsetroundjoin%
\definecolor{currentfill}{rgb}{0.855810,0.888601,0.097452}%
\pgfsetfillcolor{currentfill}%
\pgfsetlinewidth{0.000000pt}%
\definecolor{currentstroke}{rgb}{0.000000,0.000000,0.000000}%
\pgfsetstrokecolor{currentstroke}%
\pgfsetdash{}{0pt}%
\pgfpathmoveto{\pgfqpoint{3.384386in}{1.368649in}}%
\pgfpathlineto{\pgfqpoint{3.392752in}{1.361216in}}%
\pgfpathlineto{\pgfqpoint{3.400885in}{1.353386in}}%
\pgfpathlineto{\pgfqpoint{3.408753in}{1.345191in}}%
\pgfpathlineto{\pgfqpoint{3.416325in}{1.336662in}}%
\pgfpathlineto{\pgfqpoint{3.423573in}{1.327833in}}%
\pgfpathlineto{\pgfqpoint{3.436097in}{1.333984in}}%
\pgfpathlineto{\pgfqpoint{3.449386in}{1.339717in}}%
\pgfpathlineto{\pgfqpoint{3.463385in}{1.345008in}}%
\pgfpathlineto{\pgfqpoint{3.478038in}{1.349831in}}%
\pgfpathlineto{\pgfqpoint{3.493283in}{1.354166in}}%
\pgfpathlineto{\pgfqpoint{3.488621in}{1.364019in}}%
\pgfpathlineto{\pgfqpoint{3.483749in}{1.373618in}}%
\pgfpathlineto{\pgfqpoint{3.478686in}{1.382926in}}%
\pgfpathlineto{\pgfqpoint{3.473450in}{1.391905in}}%
\pgfpathlineto{\pgfqpoint{3.468061in}{1.400521in}}%
\pgfpathlineto{\pgfqpoint{3.449744in}{1.395270in}}%
\pgfpathlineto{\pgfqpoint{3.432146in}{1.389430in}}%
\pgfpathlineto{\pgfqpoint{3.415341in}{1.383027in}}%
\pgfpathlineto{\pgfqpoint{3.399399in}{1.376090in}}%
\pgfpathlineto{\pgfqpoint{3.384386in}{1.368649in}}%
\pgfpathclose%
\pgfusepath{fill}%
\end{pgfscope}%
\begin{pgfscope}%
\pgfpathrectangle{\pgfqpoint{2.548318in}{0.050000in}}{\pgfqpoint{2.081932in}{2.081932in}}%
\pgfusepath{clip}%
\pgfsetbuttcap%
\pgfsetroundjoin%
\definecolor{currentfill}{rgb}{0.327796,0.773980,0.406640}%
\pgfsetfillcolor{currentfill}%
\pgfsetlinewidth{0.000000pt}%
\definecolor{currentstroke}{rgb}{0.000000,0.000000,0.000000}%
\pgfsetstrokecolor{currentstroke}%
\pgfsetdash{}{0pt}%
\pgfpathmoveto{\pgfqpoint{3.051611in}{1.224376in}}%
\pgfpathlineto{\pgfqpoint{3.056055in}{1.232228in}}%
\pgfpathlineto{\pgfqpoint{3.061210in}{1.239785in}}%
\pgfpathlineto{\pgfqpoint{3.067057in}{1.247017in}}%
\pgfpathlineto{\pgfqpoint{3.073570in}{1.253894in}}%
\pgfpathlineto{\pgfqpoint{3.080723in}{1.260390in}}%
\pgfpathlineto{\pgfqpoint{3.087829in}{1.277708in}}%
\pgfpathlineto{\pgfqpoint{3.097136in}{1.294803in}}%
\pgfpathlineto{\pgfqpoint{3.108621in}{1.311602in}}%
\pgfpathlineto{\pgfqpoint{3.122252in}{1.328032in}}%
\pgfpathlineto{\pgfqpoint{3.137986in}{1.344020in}}%
\pgfpathlineto{\pgfqpoint{3.131515in}{1.338570in}}%
\pgfpathlineto{\pgfqpoint{3.125621in}{1.332630in}}%
\pgfpathlineto{\pgfqpoint{3.120328in}{1.326225in}}%
\pgfpathlineto{\pgfqpoint{3.115659in}{1.319381in}}%
\pgfpathlineto{\pgfqpoint{3.111635in}{1.312122in}}%
\pgfpathlineto{\pgfqpoint{3.095104in}{1.295338in}}%
\pgfpathlineto{\pgfqpoint{3.080798in}{1.278094in}}%
\pgfpathlineto{\pgfqpoint{3.068761in}{1.260468in}}%
\pgfpathlineto{\pgfqpoint{3.059025in}{1.242536in}}%
\pgfpathlineto{\pgfqpoint{3.051611in}{1.224376in}}%
\pgfpathclose%
\pgfusepath{fill}%
\end{pgfscope}%
\begin{pgfscope}%
\pgfpathrectangle{\pgfqpoint{2.548318in}{0.050000in}}{\pgfqpoint{2.081932in}{2.081932in}}%
\pgfusepath{clip}%
\pgfsetbuttcap%
\pgfsetroundjoin%
\definecolor{currentfill}{rgb}{0.278012,0.180367,0.486697}%
\pgfsetfillcolor{currentfill}%
\pgfsetlinewidth{0.000000pt}%
\definecolor{currentstroke}{rgb}{0.000000,0.000000,0.000000}%
\pgfsetstrokecolor{currentstroke}%
\pgfsetdash{}{0pt}%
\pgfpathmoveto{\pgfqpoint{3.384392in}{1.057628in}}%
\pgfpathlineto{\pgfqpoint{3.376443in}{1.051624in}}%
\pgfpathlineto{\pgfqpoint{3.367911in}{1.046054in}}%
\pgfpathlineto{\pgfqpoint{3.358832in}{1.040943in}}%
\pgfpathlineto{\pgfqpoint{3.349240in}{1.036311in}}%
\pgfpathlineto{\pgfqpoint{3.339175in}{1.032178in}}%
\pgfpathlineto{\pgfqpoint{3.343204in}{1.040838in}}%
\pgfpathlineto{\pgfqpoint{3.348358in}{1.049366in}}%
\pgfpathlineto{\pgfqpoint{3.354618in}{1.057725in}}%
\pgfpathlineto{\pgfqpoint{3.361963in}{1.065882in}}%
\pgfpathlineto{\pgfqpoint{3.370367in}{1.073800in}}%
\pgfpathlineto{\pgfqpoint{3.379360in}{1.076444in}}%
\pgfpathlineto{\pgfqpoint{3.387926in}{1.079660in}}%
\pgfpathlineto{\pgfqpoint{3.396032in}{1.083434in}}%
\pgfpathlineto{\pgfqpoint{3.403644in}{1.087750in}}%
\pgfpathlineto{\pgfqpoint{3.410735in}{1.092590in}}%
\pgfpathlineto{\pgfqpoint{3.403661in}{1.085945in}}%
\pgfpathlineto{\pgfqpoint{3.397469in}{1.079097in}}%
\pgfpathlineto{\pgfqpoint{3.392181in}{1.072075in}}%
\pgfpathlineto{\pgfqpoint{3.387817in}{1.064909in}}%
\pgfpathlineto{\pgfqpoint{3.384392in}{1.057628in}}%
\pgfpathclose%
\pgfusepath{fill}%
\end{pgfscope}%
\begin{pgfscope}%
\pgfpathrectangle{\pgfqpoint{2.548318in}{0.050000in}}{\pgfqpoint{2.081932in}{2.081932in}}%
\pgfusepath{clip}%
\pgfsetbuttcap%
\pgfsetroundjoin%
\definecolor{currentfill}{rgb}{0.296479,0.761561,0.424223}%
\pgfsetfillcolor{currentfill}%
\pgfsetlinewidth{0.000000pt}%
\definecolor{currentstroke}{rgb}{0.000000,0.000000,0.000000}%
\pgfsetstrokecolor{currentstroke}%
\pgfsetdash{}{0pt}%
\pgfpathmoveto{\pgfqpoint{3.656044in}{1.313642in}}%
\pgfpathlineto{\pgfqpoint{3.654915in}{1.302744in}}%
\pgfpathlineto{\pgfqpoint{3.653903in}{1.291905in}}%
\pgfpathlineto{\pgfqpoint{3.653010in}{1.281167in}}%
\pgfpathlineto{\pgfqpoint{3.652241in}{1.270569in}}%
\pgfpathlineto{\pgfqpoint{3.651597in}{1.260155in}}%
\pgfpathlineto{\pgfqpoint{3.664376in}{1.258841in}}%
\pgfpathlineto{\pgfqpoint{3.676957in}{1.257118in}}%
\pgfpathlineto{\pgfqpoint{3.689288in}{1.254991in}}%
\pgfpathlineto{\pgfqpoint{3.701317in}{1.252470in}}%
\pgfpathlineto{\pgfqpoint{3.712994in}{1.249566in}}%
\pgfpathlineto{\pgfqpoint{3.714794in}{1.259764in}}%
\pgfpathlineto{\pgfqpoint{3.716947in}{1.270104in}}%
\pgfpathlineto{\pgfqpoint{3.719444in}{1.280547in}}%
\pgfpathlineto{\pgfqpoint{3.722275in}{1.291051in}}%
\pgfpathlineto{\pgfqpoint{3.725430in}{1.301577in}}%
\pgfpathlineto{\pgfqpoint{3.712237in}{1.304885in}}%
\pgfpathlineto{\pgfqpoint{3.698644in}{1.307758in}}%
\pgfpathlineto{\pgfqpoint{3.684709in}{1.310181in}}%
\pgfpathlineto{\pgfqpoint{3.670488in}{1.312146in}}%
\pgfpathlineto{\pgfqpoint{3.656044in}{1.313642in}}%
\pgfpathclose%
\pgfusepath{fill}%
\end{pgfscope}%
\begin{pgfscope}%
\pgfpathrectangle{\pgfqpoint{2.548318in}{0.050000in}}{\pgfqpoint{2.081932in}{2.081932in}}%
\pgfusepath{clip}%
\pgfsetbuttcap%
\pgfsetroundjoin%
\definecolor{currentfill}{rgb}{0.993248,0.906157,0.143936}%
\pgfsetfillcolor{currentfill}%
\pgfsetlinewidth{0.000000pt}%
\definecolor{currentstroke}{rgb}{0.000000,0.000000,0.000000}%
\pgfsetstrokecolor{currentstroke}%
\pgfsetdash{}{0pt}%
\pgfpathmoveto{\pgfqpoint{3.227444in}{1.368113in}}%
\pgfpathlineto{\pgfqpoint{3.238055in}{1.367292in}}%
\pgfpathlineto{\pgfqpoint{3.248808in}{1.365879in}}%
\pgfpathlineto{\pgfqpoint{3.259658in}{1.363882in}}%
\pgfpathlineto{\pgfqpoint{3.270563in}{1.361309in}}%
\pgfpathlineto{\pgfqpoint{3.281480in}{1.358172in}}%
\pgfpathlineto{\pgfqpoint{3.294166in}{1.369031in}}%
\pgfpathlineto{\pgfqpoint{3.308218in}{1.379466in}}%
\pgfpathlineto{\pgfqpoint{3.323583in}{1.389431in}}%
\pgfpathlineto{\pgfqpoint{3.340202in}{1.398881in}}%
\pgfpathlineto{\pgfqpoint{3.331136in}{1.403369in}}%
\pgfpathlineto{\pgfqpoint{3.322077in}{1.407289in}}%
\pgfpathlineto{\pgfqpoint{3.313059in}{1.410624in}}%
\pgfpathlineto{\pgfqpoint{3.304120in}{1.413360in}}%
\pgfpathlineto{\pgfqpoint{3.295295in}{1.415484in}}%
\pgfpathlineto{\pgfqpoint{3.276063in}{1.404479in}}%
\pgfpathlineto{\pgfqpoint{3.258299in}{1.392880in}}%
\pgfpathlineto{\pgfqpoint{3.242073in}{1.380740in}}%
\pgfpathlineto{\pgfqpoint{3.227444in}{1.368113in}}%
\pgfpathclose%
\pgfusepath{fill}%
\end{pgfscope}%
\begin{pgfscope}%
\pgfpathrectangle{\pgfqpoint{2.548318in}{0.050000in}}{\pgfqpoint{2.081932in}{2.081932in}}%
\pgfusepath{clip}%
\pgfsetbuttcap%
\pgfsetroundjoin%
\definecolor{currentfill}{rgb}{0.296479,0.761561,0.424223}%
\pgfsetfillcolor{currentfill}%
\pgfsetlinewidth{0.000000pt}%
\definecolor{currentstroke}{rgb}{0.000000,0.000000,0.000000}%
\pgfsetstrokecolor{currentstroke}%
\pgfsetdash{}{0pt}%
\pgfpathmoveto{\pgfqpoint{3.512849in}{1.302451in}}%
\pgfpathlineto{\pgfqpoint{3.515909in}{1.291898in}}%
\pgfpathlineto{\pgfqpoint{3.518656in}{1.281370in}}%
\pgfpathlineto{\pgfqpoint{3.521079in}{1.270906in}}%
\pgfpathlineto{\pgfqpoint{3.523167in}{1.260547in}}%
\pgfpathlineto{\pgfqpoint{3.524914in}{1.250333in}}%
\pgfpathlineto{\pgfqpoint{3.536683in}{1.253141in}}%
\pgfpathlineto{\pgfqpoint{3.548792in}{1.255564in}}%
\pgfpathlineto{\pgfqpoint{3.561190in}{1.257590in}}%
\pgfpathlineto{\pgfqpoint{3.573825in}{1.259211in}}%
\pgfpathlineto{\pgfqpoint{3.586644in}{1.260420in}}%
\pgfpathlineto{\pgfqpoint{3.586061in}{1.270840in}}%
\pgfpathlineto{\pgfqpoint{3.585363in}{1.281444in}}%
\pgfpathlineto{\pgfqpoint{3.584554in}{1.292190in}}%
\pgfpathlineto{\pgfqpoint{3.583637in}{1.303037in}}%
\pgfpathlineto{\pgfqpoint{3.582614in}{1.313944in}}%
\pgfpathlineto{\pgfqpoint{3.568123in}{1.312567in}}%
\pgfpathlineto{\pgfqpoint{3.553842in}{1.310720in}}%
\pgfpathlineto{\pgfqpoint{3.539830in}{1.308411in}}%
\pgfpathlineto{\pgfqpoint{3.526146in}{1.305651in}}%
\pgfpathlineto{\pgfqpoint{3.512849in}{1.302451in}}%
\pgfpathclose%
\pgfusepath{fill}%
\end{pgfscope}%
\begin{pgfscope}%
\pgfpathrectangle{\pgfqpoint{2.548318in}{0.050000in}}{\pgfqpoint{2.081932in}{2.081932in}}%
\pgfusepath{clip}%
\pgfsetbuttcap%
\pgfsetroundjoin%
\definecolor{currentfill}{rgb}{0.227802,0.326594,0.546532}%
\pgfsetfillcolor{currentfill}%
\pgfsetlinewidth{0.000000pt}%
\definecolor{currentstroke}{rgb}{0.000000,0.000000,0.000000}%
\pgfsetstrokecolor{currentstroke}%
\pgfsetdash{}{0pt}%
\pgfpathmoveto{\pgfqpoint{3.767977in}{1.143653in}}%
\pgfpathlineto{\pgfqpoint{3.771427in}{1.136972in}}%
\pgfpathlineto{\pgfqpoint{3.775411in}{1.130764in}}%
\pgfpathlineto{\pgfqpoint{3.779913in}{1.125055in}}%
\pgfpathlineto{\pgfqpoint{3.784917in}{1.119868in}}%
\pgfpathlineto{\pgfqpoint{3.790404in}{1.115226in}}%
\pgfpathlineto{\pgfqpoint{3.800521in}{1.109583in}}%
\pgfpathlineto{\pgfqpoint{3.809874in}{1.103636in}}%
\pgfpathlineto{\pgfqpoint{3.818427in}{1.097412in}}%
\pgfpathlineto{\pgfqpoint{3.826147in}{1.090936in}}%
\pgfpathlineto{\pgfqpoint{3.833005in}{1.084237in}}%
\pgfpathlineto{\pgfqpoint{3.826197in}{1.089837in}}%
\pgfpathlineto{\pgfqpoint{3.819987in}{1.095894in}}%
\pgfpathlineto{\pgfqpoint{3.814398in}{1.102383in}}%
\pgfpathlineto{\pgfqpoint{3.809452in}{1.109277in}}%
\pgfpathlineto{\pgfqpoint{3.805167in}{1.116549in}}%
\pgfpathlineto{\pgfqpoint{3.799171in}{1.122411in}}%
\pgfpathlineto{\pgfqpoint{3.792426in}{1.128076in}}%
\pgfpathlineto{\pgfqpoint{3.784959in}{1.133520in}}%
\pgfpathlineto{\pgfqpoint{3.776799in}{1.138720in}}%
\pgfpathlineto{\pgfqpoint{3.767977in}{1.143653in}}%
\pgfpathclose%
\pgfusepath{fill}%
\end{pgfscope}%
\begin{pgfscope}%
\pgfpathrectangle{\pgfqpoint{2.548318in}{0.050000in}}{\pgfqpoint{2.081932in}{2.081932in}}%
\pgfusepath{clip}%
\pgfsetbuttcap%
\pgfsetroundjoin%
\definecolor{currentfill}{rgb}{0.606045,0.850733,0.236712}%
\pgfsetfillcolor{currentfill}%
\pgfsetlinewidth{0.000000pt}%
\definecolor{currentstroke}{rgb}{0.000000,0.000000,0.000000}%
\pgfsetstrokecolor{currentstroke}%
\pgfsetdash{}{0pt}%
\pgfpathmoveto{\pgfqpoint{3.663258in}{1.367555in}}%
\pgfpathlineto{\pgfqpoint{3.661623in}{1.356988in}}%
\pgfpathlineto{\pgfqpoint{3.660078in}{1.346272in}}%
\pgfpathlineto{\pgfqpoint{3.658630in}{1.335448in}}%
\pgfpathlineto{\pgfqpoint{3.657283in}{1.324557in}}%
\pgfpathlineto{\pgfqpoint{3.656044in}{1.313642in}}%
\pgfpathlineto{\pgfqpoint{3.670488in}{1.312146in}}%
\pgfpathlineto{\pgfqpoint{3.684709in}{1.310181in}}%
\pgfpathlineto{\pgfqpoint{3.698644in}{1.307758in}}%
\pgfpathlineto{\pgfqpoint{3.712237in}{1.304885in}}%
\pgfpathlineto{\pgfqpoint{3.725430in}{1.301577in}}%
\pgfpathlineto{\pgfqpoint{3.728897in}{1.312085in}}%
\pgfpathlineto{\pgfqpoint{3.732661in}{1.322533in}}%
\pgfpathlineto{\pgfqpoint{3.736710in}{1.332881in}}%
\pgfpathlineto{\pgfqpoint{3.741028in}{1.343090in}}%
\pgfpathlineto{\pgfqpoint{3.745597in}{1.353121in}}%
\pgfpathlineto{\pgfqpoint{3.729949in}{1.357078in}}%
\pgfpathlineto{\pgfqpoint{3.713822in}{1.360514in}}%
\pgfpathlineto{\pgfqpoint{3.697284in}{1.363414in}}%
\pgfpathlineto{\pgfqpoint{3.680406in}{1.365764in}}%
\pgfpathlineto{\pgfqpoint{3.663258in}{1.367555in}}%
\pgfpathclose%
\pgfusepath{fill}%
\end{pgfscope}%
\begin{pgfscope}%
\pgfpathrectangle{\pgfqpoint{2.548318in}{0.050000in}}{\pgfqpoint{2.081932in}{2.081932in}}%
\pgfusepath{clip}%
\pgfsetbuttcap%
\pgfsetroundjoin%
\definecolor{currentfill}{rgb}{0.606045,0.850733,0.236712}%
\pgfsetfillcolor{currentfill}%
\pgfsetlinewidth{0.000000pt}%
\definecolor{currentstroke}{rgb}{0.000000,0.000000,0.000000}%
\pgfsetstrokecolor{currentstroke}%
\pgfsetdash{}{0pt}%
\pgfpathmoveto{\pgfqpoint{3.493283in}{1.354166in}}%
\pgfpathlineto{\pgfqpoint{3.497716in}{1.344097in}}%
\pgfpathlineto{\pgfqpoint{3.501905in}{1.333851in}}%
\pgfpathlineto{\pgfqpoint{3.505833in}{1.323468in}}%
\pgfpathlineto{\pgfqpoint{3.509486in}{1.312988in}}%
\pgfpathlineto{\pgfqpoint{3.512849in}{1.302451in}}%
\pgfpathlineto{\pgfqpoint{3.526146in}{1.305651in}}%
\pgfpathlineto{\pgfqpoint{3.539830in}{1.308411in}}%
\pgfpathlineto{\pgfqpoint{3.553842in}{1.310720in}}%
\pgfpathlineto{\pgfqpoint{3.568123in}{1.312567in}}%
\pgfpathlineto{\pgfqpoint{3.582614in}{1.313944in}}%
\pgfpathlineto{\pgfqpoint{3.581491in}{1.324870in}}%
\pgfpathlineto{\pgfqpoint{3.580270in}{1.335771in}}%
\pgfpathlineto{\pgfqpoint{3.578957in}{1.346607in}}%
\pgfpathlineto{\pgfqpoint{3.577557in}{1.357337in}}%
\pgfpathlineto{\pgfqpoint{3.576075in}{1.367917in}}%
\pgfpathlineto{\pgfqpoint{3.558873in}{1.366269in}}%
\pgfpathlineto{\pgfqpoint{3.541920in}{1.364058in}}%
\pgfpathlineto{\pgfqpoint{3.525291in}{1.361296in}}%
\pgfpathlineto{\pgfqpoint{3.509056in}{1.357993in}}%
\pgfpathlineto{\pgfqpoint{3.493283in}{1.354166in}}%
\pgfpathclose%
\pgfusepath{fill}%
\end{pgfscope}%
\begin{pgfscope}%
\pgfpathrectangle{\pgfqpoint{2.548318in}{0.050000in}}{\pgfqpoint{2.081932in}{2.081932in}}%
\pgfusepath{clip}%
\pgfsetbuttcap%
\pgfsetroundjoin%
\definecolor{currentfill}{rgb}{0.278791,0.062145,0.386592}%
\pgfsetfillcolor{currentfill}%
\pgfsetlinewidth{0.000000pt}%
\definecolor{currentstroke}{rgb}{0.000000,0.000000,0.000000}%
\pgfsetstrokecolor{currentstroke}%
\pgfsetdash{}{0pt}%
\pgfpathmoveto{\pgfqpoint{3.875006in}{1.063847in}}%
\pgfpathlineto{\pgfqpoint{3.884767in}{1.061413in}}%
\pgfpathlineto{\pgfqpoint{3.894896in}{1.059558in}}%
\pgfpathlineto{\pgfqpoint{3.905352in}{1.058290in}}%
\pgfpathlineto{\pgfqpoint{3.916092in}{1.057615in}}%
\pgfpathlineto{\pgfqpoint{3.927074in}{1.057538in}}%
\pgfpathlineto{\pgfqpoint{3.935501in}{1.047679in}}%
\pgfpathlineto{\pgfqpoint{3.942611in}{1.037592in}}%
\pgfpathlineto{\pgfqpoint{3.948381in}{1.027319in}}%
\pgfpathlineto{\pgfqpoint{3.952793in}{1.016904in}}%
\pgfpathlineto{\pgfqpoint{3.955835in}{1.006389in}}%
\pgfpathlineto{\pgfqpoint{3.943925in}{1.008267in}}%
\pgfpathlineto{\pgfqpoint{3.932270in}{1.010699in}}%
\pgfpathlineto{\pgfqpoint{3.920918in}{1.013674in}}%
\pgfpathlineto{\pgfqpoint{3.909915in}{1.017181in}}%
\pgfpathlineto{\pgfqpoint{3.899306in}{1.021204in}}%
\pgfpathlineto{\pgfqpoint{3.896695in}{1.029979in}}%
\pgfpathlineto{\pgfqpoint{3.892946in}{1.038667in}}%
\pgfpathlineto{\pgfqpoint{3.888071in}{1.047232in}}%
\pgfpathlineto{\pgfqpoint{3.882084in}{1.055637in}}%
\pgfpathlineto{\pgfqpoint{3.875006in}{1.063847in}}%
\pgfpathclose%
\pgfusepath{fill}%
\end{pgfscope}%
\begin{pgfscope}%
\pgfpathrectangle{\pgfqpoint{2.548318in}{0.050000in}}{\pgfqpoint{2.081932in}{2.081932in}}%
\pgfusepath{clip}%
\pgfsetbuttcap%
\pgfsetroundjoin%
\definecolor{currentfill}{rgb}{0.296479,0.761561,0.424223}%
\pgfsetfillcolor{currentfill}%
\pgfsetlinewidth{0.000000pt}%
\definecolor{currentstroke}{rgb}{0.000000,0.000000,0.000000}%
\pgfsetstrokecolor{currentstroke}%
\pgfsetdash{}{0pt}%
\pgfpathmoveto{\pgfqpoint{3.582614in}{1.313944in}}%
\pgfpathlineto{\pgfqpoint{3.583637in}{1.303037in}}%
\pgfpathlineto{\pgfqpoint{3.584554in}{1.292190in}}%
\pgfpathlineto{\pgfqpoint{3.585363in}{1.281444in}}%
\pgfpathlineto{\pgfqpoint{3.586061in}{1.270840in}}%
\pgfpathlineto{\pgfqpoint{3.586644in}{1.260420in}}%
\pgfpathlineto{\pgfqpoint{3.599593in}{1.261211in}}%
\pgfpathlineto{\pgfqpoint{3.612618in}{1.261580in}}%
\pgfpathlineto{\pgfqpoint{3.625664in}{1.261527in}}%
\pgfpathlineto{\pgfqpoint{3.638675in}{1.261051in}}%
\pgfpathlineto{\pgfqpoint{3.651597in}{1.260155in}}%
\pgfpathlineto{\pgfqpoint{3.652241in}{1.270569in}}%
\pgfpathlineto{\pgfqpoint{3.653010in}{1.281167in}}%
\pgfpathlineto{\pgfqpoint{3.653903in}{1.291905in}}%
\pgfpathlineto{\pgfqpoint{3.654915in}{1.302744in}}%
\pgfpathlineto{\pgfqpoint{3.656044in}{1.313642in}}%
\pgfpathlineto{\pgfqpoint{3.641436in}{1.314664in}}%
\pgfpathlineto{\pgfqpoint{3.626726in}{1.315206in}}%
\pgfpathlineto{\pgfqpoint{3.611978in}{1.315267in}}%
\pgfpathlineto{\pgfqpoint{3.597253in}{1.314845in}}%
\pgfpathlineto{\pgfqpoint{3.582614in}{1.313944in}}%
\pgfpathclose%
\pgfusepath{fill}%
\end{pgfscope}%
\begin{pgfscope}%
\pgfpathrectangle{\pgfqpoint{2.548318in}{0.050000in}}{\pgfqpoint{2.081932in}{2.081932in}}%
\pgfusepath{clip}%
\pgfsetbuttcap%
\pgfsetroundjoin%
\definecolor{currentfill}{rgb}{0.120638,0.625828,0.533488}%
\pgfsetfillcolor{currentfill}%
\pgfsetlinewidth{0.000000pt}%
\definecolor{currentstroke}{rgb}{0.000000,0.000000,0.000000}%
\pgfsetstrokecolor{currentstroke}%
\pgfsetdash{}{0pt}%
\pgfpathmoveto{\pgfqpoint{3.651597in}{1.260155in}}%
\pgfpathlineto{\pgfqpoint{3.651082in}{1.249962in}}%
\pgfpathlineto{\pgfqpoint{3.650698in}{1.240030in}}%
\pgfpathlineto{\pgfqpoint{3.650446in}{1.230398in}}%
\pgfpathlineto{\pgfqpoint{3.650327in}{1.221102in}}%
\pgfpathlineto{\pgfqpoint{3.650342in}{1.212179in}}%
\pgfpathlineto{\pgfqpoint{3.662650in}{1.210924in}}%
\pgfpathlineto{\pgfqpoint{3.674769in}{1.209275in}}%
\pgfpathlineto{\pgfqpoint{3.686646in}{1.207241in}}%
\pgfpathlineto{\pgfqpoint{3.698234in}{1.204831in}}%
\pgfpathlineto{\pgfqpoint{3.709483in}{1.202054in}}%
\pgfpathlineto{\pgfqpoint{3.709440in}{1.210967in}}%
\pgfpathlineto{\pgfqpoint{3.709773in}{1.220212in}}%
\pgfpathlineto{\pgfqpoint{3.710478in}{1.229751in}}%
\pgfpathlineto{\pgfqpoint{3.711553in}{1.239549in}}%
\pgfpathlineto{\pgfqpoint{3.712994in}{1.249566in}}%
\pgfpathlineto{\pgfqpoint{3.701317in}{1.252470in}}%
\pgfpathlineto{\pgfqpoint{3.689288in}{1.254991in}}%
\pgfpathlineto{\pgfqpoint{3.676957in}{1.257118in}}%
\pgfpathlineto{\pgfqpoint{3.664376in}{1.258841in}}%
\pgfpathlineto{\pgfqpoint{3.651597in}{1.260155in}}%
\pgfpathclose%
\pgfusepath{fill}%
\end{pgfscope}%
\begin{pgfscope}%
\pgfpathrectangle{\pgfqpoint{2.548318in}{0.050000in}}{\pgfqpoint{2.081932in}{2.081932in}}%
\pgfusepath{clip}%
\pgfsetbuttcap%
\pgfsetroundjoin%
\definecolor{currentfill}{rgb}{0.120638,0.625828,0.533488}%
\pgfsetfillcolor{currentfill}%
\pgfsetlinewidth{0.000000pt}%
\definecolor{currentstroke}{rgb}{0.000000,0.000000,0.000000}%
\pgfsetstrokecolor{currentstroke}%
\pgfsetdash{}{0pt}%
\pgfpathmoveto{\pgfqpoint{3.524914in}{1.250333in}}%
\pgfpathlineto{\pgfqpoint{3.526311in}{1.240303in}}%
\pgfpathlineto{\pgfqpoint{3.527355in}{1.230496in}}%
\pgfpathlineto{\pgfqpoint{3.528039in}{1.220950in}}%
\pgfpathlineto{\pgfqpoint{3.528361in}{1.211702in}}%
\pgfpathlineto{\pgfqpoint{3.528321in}{1.202787in}}%
\pgfpathlineto{\pgfqpoint{3.539658in}{1.205473in}}%
\pgfpathlineto{\pgfqpoint{3.551322in}{1.207790in}}%
\pgfpathlineto{\pgfqpoint{3.563264in}{1.209728in}}%
\pgfpathlineto{\pgfqpoint{3.575435in}{1.211278in}}%
\pgfpathlineto{\pgfqpoint{3.587782in}{1.212433in}}%
\pgfpathlineto{\pgfqpoint{3.587795in}{1.221356in}}%
\pgfpathlineto{\pgfqpoint{3.587688in}{1.230653in}}%
\pgfpathlineto{\pgfqpoint{3.587459in}{1.240287in}}%
\pgfpathlineto{\pgfqpoint{3.587111in}{1.250222in}}%
\pgfpathlineto{\pgfqpoint{3.586644in}{1.260420in}}%
\pgfpathlineto{\pgfqpoint{3.573825in}{1.259211in}}%
\pgfpathlineto{\pgfqpoint{3.561190in}{1.257590in}}%
\pgfpathlineto{\pgfqpoint{3.548792in}{1.255564in}}%
\pgfpathlineto{\pgfqpoint{3.536683in}{1.253141in}}%
\pgfpathlineto{\pgfqpoint{3.524914in}{1.250333in}}%
\pgfpathclose%
\pgfusepath{fill}%
\end{pgfscope}%
\begin{pgfscope}%
\pgfpathrectangle{\pgfqpoint{2.548318in}{0.050000in}}{\pgfqpoint{2.081932in}{2.081932in}}%
\pgfusepath{clip}%
\pgfsetbuttcap%
\pgfsetroundjoin%
\definecolor{currentfill}{rgb}{0.162142,0.474838,0.558140}%
\pgfsetfillcolor{currentfill}%
\pgfsetlinewidth{0.000000pt}%
\definecolor{currentstroke}{rgb}{0.000000,0.000000,0.000000}%
\pgfsetstrokecolor{currentstroke}%
\pgfsetdash{}{0pt}%
\pgfpathmoveto{\pgfqpoint{3.709483in}{1.202054in}}%
\pgfpathlineto{\pgfqpoint{3.709900in}{1.193506in}}%
\pgfpathlineto{\pgfqpoint{3.710691in}{1.185358in}}%
\pgfpathlineto{\pgfqpoint{3.711854in}{1.177642in}}%
\pgfpathlineto{\pgfqpoint{3.713385in}{1.170388in}}%
\pgfpathlineto{\pgfqpoint{3.715278in}{1.163626in}}%
\pgfpathlineto{\pgfqpoint{3.726823in}{1.160317in}}%
\pgfpathlineto{\pgfqpoint{3.737910in}{1.156649in}}%
\pgfpathlineto{\pgfqpoint{3.748493in}{1.152637in}}%
\pgfpathlineto{\pgfqpoint{3.758529in}{1.148299in}}%
\pgfpathlineto{\pgfqpoint{3.767977in}{1.143653in}}%
\pgfpathlineto{\pgfqpoint{3.765072in}{1.150780in}}%
\pgfpathlineto{\pgfqpoint{3.762723in}{1.158325in}}%
\pgfpathlineto{\pgfqpoint{3.760938in}{1.166256in}}%
\pgfpathlineto{\pgfqpoint{3.759723in}{1.174542in}}%
\pgfpathlineto{\pgfqpoint{3.759083in}{1.183151in}}%
\pgfpathlineto{\pgfqpoint{3.750188in}{1.187548in}}%
\pgfpathlineto{\pgfqpoint{3.740741in}{1.191654in}}%
\pgfpathlineto{\pgfqpoint{3.730780in}{1.195451in}}%
\pgfpathlineto{\pgfqpoint{3.720347in}{1.198923in}}%
\pgfpathlineto{\pgfqpoint{3.709483in}{1.202054in}}%
\pgfpathclose%
\pgfusepath{fill}%
\end{pgfscope}%
\begin{pgfscope}%
\pgfpathrectangle{\pgfqpoint{2.548318in}{0.050000in}}{\pgfqpoint{2.081932in}{2.081932in}}%
\pgfusepath{clip}%
\pgfsetbuttcap%
\pgfsetroundjoin%
\definecolor{currentfill}{rgb}{0.162142,0.474838,0.558140}%
\pgfsetfillcolor{currentfill}%
\pgfsetlinewidth{0.000000pt}%
\definecolor{currentstroke}{rgb}{0.000000,0.000000,0.000000}%
\pgfsetstrokecolor{currentstroke}%
\pgfsetdash{}{0pt}%
\pgfpathmoveto{\pgfqpoint{3.478116in}{1.184285in}}%
\pgfpathlineto{\pgfqpoint{3.477486in}{1.175680in}}%
\pgfpathlineto{\pgfqpoint{3.476289in}{1.167402in}}%
\pgfpathlineto{\pgfqpoint{3.474532in}{1.159484in}}%
\pgfpathlineto{\pgfqpoint{3.472218in}{1.151957in}}%
\pgfpathlineto{\pgfqpoint{3.469357in}{1.144851in}}%
\pgfpathlineto{\pgfqpoint{3.478954in}{1.149422in}}%
\pgfpathlineto{\pgfqpoint{3.489130in}{1.153679in}}%
\pgfpathlineto{\pgfqpoint{3.499842in}{1.157606in}}%
\pgfpathlineto{\pgfqpoint{3.511047in}{1.161186in}}%
\pgfpathlineto{\pgfqpoint{3.522699in}{1.164402in}}%
\pgfpathlineto{\pgfqpoint{3.524535in}{1.171149in}}%
\pgfpathlineto{\pgfqpoint{3.526020in}{1.178392in}}%
\pgfpathlineto{\pgfqpoint{3.527148in}{1.186099in}}%
\pgfpathlineto{\pgfqpoint{3.527916in}{1.194242in}}%
\pgfpathlineto{\pgfqpoint{3.528321in}{1.202787in}}%
\pgfpathlineto{\pgfqpoint{3.517357in}{1.199744in}}%
\pgfpathlineto{\pgfqpoint{3.506812in}{1.196357in}}%
\pgfpathlineto{\pgfqpoint{3.496730in}{1.192641in}}%
\pgfpathlineto{\pgfqpoint{3.487152in}{1.188611in}}%
\pgfpathlineto{\pgfqpoint{3.478116in}{1.184285in}}%
\pgfpathclose%
\pgfusepath{fill}%
\end{pgfscope}%
\begin{pgfscope}%
\pgfpathrectangle{\pgfqpoint{2.548318in}{0.050000in}}{\pgfqpoint{2.081932in}{2.081932in}}%
\pgfusepath{clip}%
\pgfsetbuttcap%
\pgfsetroundjoin%
\definecolor{currentfill}{rgb}{0.150476,0.504369,0.557430}%
\pgfsetfillcolor{currentfill}%
\pgfsetlinewidth{0.000000pt}%
\definecolor{currentstroke}{rgb}{0.000000,0.000000,0.000000}%
\pgfsetstrokecolor{currentstroke}%
\pgfsetdash{}{0pt}%
\pgfpathmoveto{\pgfqpoint{4.149124in}{1.212426in}}%
\pgfpathlineto{\pgfqpoint{4.151437in}{1.221802in}}%
\pgfpathlineto{\pgfqpoint{4.153024in}{1.231105in}}%
\pgfpathlineto{\pgfqpoint{4.153878in}{1.240295in}}%
\pgfpathlineto{\pgfqpoint{4.153997in}{1.249335in}}%
\pgfpathlineto{\pgfqpoint{4.167984in}{1.231760in}}%
\pgfpathlineto{\pgfqpoint{4.179647in}{1.213818in}}%
\pgfpathlineto{\pgfqpoint{4.188957in}{1.195588in}}%
\pgfpathlineto{\pgfqpoint{4.195893in}{1.177149in}}%
\pgfpathlineto{\pgfqpoint{4.200446in}{1.158578in}}%
\pgfpathlineto{\pgfqpoint{4.200318in}{1.149681in}}%
\pgfpathlineto{\pgfqpoint{4.199403in}{1.140759in}}%
\pgfpathlineto{\pgfqpoint{4.197703in}{1.131848in}}%
\pgfpathlineto{\pgfqpoint{4.195224in}{1.122983in}}%
\pgfpathlineto{\pgfqpoint{4.190696in}{1.141287in}}%
\pgfpathlineto{\pgfqpoint{4.183808in}{1.159460in}}%
\pgfpathlineto{\pgfqpoint{4.174569in}{1.177426in}}%
\pgfpathlineto{\pgfqpoint{4.162998in}{1.195107in}}%
\pgfpathlineto{\pgfqpoint{4.149124in}{1.212426in}}%
\pgfpathclose%
\pgfusepath{fill}%
\end{pgfscope}%
\begin{pgfscope}%
\pgfpathrectangle{\pgfqpoint{2.548318in}{0.050000in}}{\pgfqpoint{2.081932in}{2.081932in}}%
\pgfusepath{clip}%
\pgfsetbuttcap%
\pgfsetroundjoin%
\definecolor{currentfill}{rgb}{0.227802,0.326594,0.546532}%
\pgfsetfillcolor{currentfill}%
\pgfsetlinewidth{0.000000pt}%
\definecolor{currentstroke}{rgb}{0.000000,0.000000,0.000000}%
\pgfsetstrokecolor{currentstroke}%
\pgfsetdash{}{0pt}%
\pgfpathmoveto{\pgfqpoint{3.437480in}{1.123858in}}%
\pgfpathlineto{\pgfqpoint{3.433363in}{1.116745in}}%
\pgfpathlineto{\pgfqpoint{3.428612in}{1.110035in}}%
\pgfpathlineto{\pgfqpoint{3.423243in}{1.103756in}}%
\pgfpathlineto{\pgfqpoint{3.417276in}{1.097933in}}%
\pgfpathlineto{\pgfqpoint{3.410735in}{1.092590in}}%
\pgfpathlineto{\pgfqpoint{3.418665in}{1.099005in}}%
\pgfpathlineto{\pgfqpoint{3.427420in}{1.105162in}}%
\pgfpathlineto{\pgfqpoint{3.436966in}{1.111034in}}%
\pgfpathlineto{\pgfqpoint{3.447266in}{1.116598in}}%
\pgfpathlineto{\pgfqpoint{3.452671in}{1.121197in}}%
\pgfpathlineto{\pgfqpoint{3.457600in}{1.126345in}}%
\pgfpathlineto{\pgfqpoint{3.462034in}{1.132019in}}%
\pgfpathlineto{\pgfqpoint{3.465958in}{1.138197in}}%
\pgfpathlineto{\pgfqpoint{3.469357in}{1.144851in}}%
\pgfpathlineto{\pgfqpoint{3.460376in}{1.139989in}}%
\pgfpathlineto{\pgfqpoint{3.452048in}{1.134854in}}%
\pgfpathlineto{\pgfqpoint{3.444406in}{1.129470in}}%
\pgfpathlineto{\pgfqpoint{3.437480in}{1.123858in}}%
\pgfpathclose%
\pgfusepath{fill}%
\end{pgfscope}%
\begin{pgfscope}%
\pgfpathrectangle{\pgfqpoint{2.548318in}{0.050000in}}{\pgfqpoint{2.081932in}{2.081932in}}%
\pgfusepath{clip}%
\pgfsetbuttcap%
\pgfsetroundjoin%
\definecolor{currentfill}{rgb}{0.606045,0.850733,0.236712}%
\pgfsetfillcolor{currentfill}%
\pgfsetlinewidth{0.000000pt}%
\definecolor{currentstroke}{rgb}{0.000000,0.000000,0.000000}%
\pgfsetstrokecolor{currentstroke}%
\pgfsetdash{}{0pt}%
\pgfpathmoveto{\pgfqpoint{3.576075in}{1.367917in}}%
\pgfpathlineto{\pgfqpoint{3.577557in}{1.357337in}}%
\pgfpathlineto{\pgfqpoint{3.578957in}{1.346607in}}%
\pgfpathlineto{\pgfqpoint{3.580270in}{1.335771in}}%
\pgfpathlineto{\pgfqpoint{3.581491in}{1.324870in}}%
\pgfpathlineto{\pgfqpoint{3.582614in}{1.313944in}}%
\pgfpathlineto{\pgfqpoint{3.597253in}{1.314845in}}%
\pgfpathlineto{\pgfqpoint{3.611978in}{1.315267in}}%
\pgfpathlineto{\pgfqpoint{3.626726in}{1.315206in}}%
\pgfpathlineto{\pgfqpoint{3.641436in}{1.314664in}}%
\pgfpathlineto{\pgfqpoint{3.656044in}{1.313642in}}%
\pgfpathlineto{\pgfqpoint{3.657283in}{1.324557in}}%
\pgfpathlineto{\pgfqpoint{3.658630in}{1.335448in}}%
\pgfpathlineto{\pgfqpoint{3.660078in}{1.346272in}}%
\pgfpathlineto{\pgfqpoint{3.661623in}{1.356988in}}%
\pgfpathlineto{\pgfqpoint{3.663258in}{1.367555in}}%
\pgfpathlineto{\pgfqpoint{3.645915in}{1.368778in}}%
\pgfpathlineto{\pgfqpoint{3.628451in}{1.369427in}}%
\pgfpathlineto{\pgfqpoint{3.610939in}{1.369500in}}%
\pgfpathlineto{\pgfqpoint{3.593456in}{1.368996in}}%
\pgfpathlineto{\pgfqpoint{3.576075in}{1.367917in}}%
\pgfpathclose%
\pgfusepath{fill}%
\end{pgfscope}%
\begin{pgfscope}%
\pgfpathrectangle{\pgfqpoint{2.548318in}{0.050000in}}{\pgfqpoint{2.081932in}{2.081932in}}%
\pgfusepath{clip}%
\pgfsetbuttcap%
\pgfsetroundjoin%
\definecolor{currentfill}{rgb}{0.120638,0.625828,0.533488}%
\pgfsetfillcolor{currentfill}%
\pgfsetlinewidth{0.000000pt}%
\definecolor{currentstroke}{rgb}{0.000000,0.000000,0.000000}%
\pgfsetstrokecolor{currentstroke}%
\pgfsetdash{}{0pt}%
\pgfpathmoveto{\pgfqpoint{3.586644in}{1.260420in}}%
\pgfpathlineto{\pgfqpoint{3.587111in}{1.250222in}}%
\pgfpathlineto{\pgfqpoint{3.587459in}{1.240287in}}%
\pgfpathlineto{\pgfqpoint{3.587688in}{1.230653in}}%
\pgfpathlineto{\pgfqpoint{3.587795in}{1.221356in}}%
\pgfpathlineto{\pgfqpoint{3.587782in}{1.212433in}}%
\pgfpathlineto{\pgfqpoint{3.600254in}{1.213189in}}%
\pgfpathlineto{\pgfqpoint{3.612799in}{1.213543in}}%
\pgfpathlineto{\pgfqpoint{3.625364in}{1.213492in}}%
\pgfpathlineto{\pgfqpoint{3.637896in}{1.213037in}}%
\pgfpathlineto{\pgfqpoint{3.650342in}{1.212179in}}%
\pgfpathlineto{\pgfqpoint{3.650327in}{1.221102in}}%
\pgfpathlineto{\pgfqpoint{3.650446in}{1.230398in}}%
\pgfpathlineto{\pgfqpoint{3.650698in}{1.240030in}}%
\pgfpathlineto{\pgfqpoint{3.651082in}{1.249962in}}%
\pgfpathlineto{\pgfqpoint{3.651597in}{1.260155in}}%
\pgfpathlineto{\pgfqpoint{3.638675in}{1.261051in}}%
\pgfpathlineto{\pgfqpoint{3.625664in}{1.261527in}}%
\pgfpathlineto{\pgfqpoint{3.612618in}{1.261580in}}%
\pgfpathlineto{\pgfqpoint{3.599593in}{1.261211in}}%
\pgfpathlineto{\pgfqpoint{3.586644in}{1.260420in}}%
\pgfpathclose%
\pgfusepath{fill}%
\end{pgfscope}%
\begin{pgfscope}%
\pgfpathrectangle{\pgfqpoint{2.548318in}{0.050000in}}{\pgfqpoint{2.081932in}{2.081932in}}%
\pgfusepath{clip}%
\pgfsetbuttcap%
\pgfsetroundjoin%
\definecolor{currentfill}{rgb}{0.876168,0.891125,0.095250}%
\pgfsetfillcolor{currentfill}%
\pgfsetlinewidth{0.000000pt}%
\definecolor{currentstroke}{rgb}{0.000000,0.000000,0.000000}%
\pgfsetstrokecolor{currentstroke}%
\pgfsetdash{}{0pt}%
\pgfpathmoveto{\pgfqpoint{3.986465in}{1.413610in}}%
\pgfpathlineto{\pgfqpoint{3.978614in}{1.414707in}}%
\pgfpathlineto{\pgfqpoint{3.970474in}{1.415171in}}%
\pgfpathlineto{\pgfqpoint{3.962078in}{1.415001in}}%
\pgfpathlineto{\pgfqpoint{3.953459in}{1.414199in}}%
\pgfpathlineto{\pgfqpoint{3.944654in}{1.412770in}}%
\pgfpathlineto{\pgfqpoint{3.963528in}{1.401611in}}%
\pgfpathlineto{\pgfqpoint{3.980916in}{1.389870in}}%
\pgfpathlineto{\pgfqpoint{3.996751in}{1.377602in}}%
\pgfpathlineto{\pgfqpoint{4.010975in}{1.364861in}}%
\pgfpathlineto{\pgfqpoint{4.023535in}{1.351704in}}%
\pgfpathlineto{\pgfqpoint{4.034363in}{1.351491in}}%
\pgfpathlineto{\pgfqpoint{4.044955in}{1.350692in}}%
\pgfpathlineto{\pgfqpoint{4.055270in}{1.349309in}}%
\pgfpathlineto{\pgfqpoint{4.065266in}{1.347346in}}%
\pgfpathlineto{\pgfqpoint{4.074902in}{1.344809in}}%
\pgfpathlineto{\pgfqpoint{4.060862in}{1.359620in}}%
\pgfpathlineto{\pgfqpoint{4.044937in}{1.373969in}}%
\pgfpathlineto{\pgfqpoint{4.027185in}{1.387791in}}%
\pgfpathlineto{\pgfqpoint{4.007669in}{1.401025in}}%
\pgfpathlineto{\pgfqpoint{3.986465in}{1.413610in}}%
\pgfpathclose%
\pgfusepath{fill}%
\end{pgfscope}%
\begin{pgfscope}%
\pgfpathrectangle{\pgfqpoint{2.548318in}{0.050000in}}{\pgfqpoint{2.081932in}{2.081932in}}%
\pgfusepath{clip}%
\pgfsetbuttcap%
\pgfsetroundjoin%
\definecolor{currentfill}{rgb}{0.993248,0.906157,0.143936}%
\pgfsetfillcolor{currentfill}%
\pgfsetlinewidth{0.000000pt}%
\definecolor{currentstroke}{rgb}{0.000000,0.000000,0.000000}%
\pgfsetstrokecolor{currentstroke}%
\pgfsetdash{}{0pt}%
\pgfpathmoveto{\pgfqpoint{3.800949in}{1.435482in}}%
\pgfpathlineto{\pgfqpoint{3.794941in}{1.429219in}}%
\pgfpathlineto{\pgfqpoint{3.788979in}{1.422437in}}%
\pgfpathlineto{\pgfqpoint{3.783084in}{1.415162in}}%
\pgfpathlineto{\pgfqpoint{3.777281in}{1.407424in}}%
\pgfpathlineto{\pgfqpoint{3.771592in}{1.399255in}}%
\pgfpathlineto{\pgfqpoint{3.789738in}{1.393853in}}%
\pgfpathlineto{\pgfqpoint{3.807146in}{1.387869in}}%
\pgfpathlineto{\pgfqpoint{3.823742in}{1.381328in}}%
\pgfpathlineto{\pgfqpoint{3.839460in}{1.374261in}}%
\pgfpathlineto{\pgfqpoint{3.854233in}{1.366698in}}%
\pgfpathlineto{\pgfqpoint{3.862931in}{1.373631in}}%
\pgfpathlineto{\pgfqpoint{3.871799in}{1.380108in}}%
\pgfpathlineto{\pgfqpoint{3.880803in}{1.386102in}}%
\pgfpathlineto{\pgfqpoint{3.889908in}{1.391591in}}%
\pgfpathlineto{\pgfqpoint{3.899080in}{1.396550in}}%
\pgfpathlineto{\pgfqpoint{3.881563in}{1.405587in}}%
\pgfpathlineto{\pgfqpoint{3.862913in}{1.414036in}}%
\pgfpathlineto{\pgfqpoint{3.843205in}{1.421858in}}%
\pgfpathlineto{\pgfqpoint{3.822522in}{1.429018in}}%
\pgfpathlineto{\pgfqpoint{3.800949in}{1.435482in}}%
\pgfpathclose%
\pgfusepath{fill}%
\end{pgfscope}%
\begin{pgfscope}%
\pgfpathrectangle{\pgfqpoint{2.548318in}{0.050000in}}{\pgfqpoint{2.081932in}{2.081932in}}%
\pgfusepath{clip}%
\pgfsetbuttcap%
\pgfsetroundjoin%
\definecolor{currentfill}{rgb}{0.993248,0.906157,0.143936}%
\pgfsetfillcolor{currentfill}%
\pgfsetlinewidth{0.000000pt}%
\definecolor{currentstroke}{rgb}{0.000000,0.000000,0.000000}%
\pgfsetstrokecolor{currentstroke}%
\pgfsetdash{}{0pt}%
\pgfpathmoveto{\pgfqpoint{3.340202in}{1.398881in}}%
\pgfpathlineto{\pgfqpoint{3.349238in}{1.393844in}}%
\pgfpathlineto{\pgfqpoint{3.358209in}{1.388278in}}%
\pgfpathlineto{\pgfqpoint{3.367080in}{1.382207in}}%
\pgfpathlineto{\pgfqpoint{3.375817in}{1.375655in}}%
\pgfpathlineto{\pgfqpoint{3.384386in}{1.368649in}}%
\pgfpathlineto{\pgfqpoint{3.399399in}{1.376090in}}%
\pgfpathlineto{\pgfqpoint{3.415341in}{1.383027in}}%
\pgfpathlineto{\pgfqpoint{3.432146in}{1.389430in}}%
\pgfpathlineto{\pgfqpoint{3.449744in}{1.395270in}}%
\pgfpathlineto{\pgfqpoint{3.468061in}{1.400521in}}%
\pgfpathlineto{\pgfqpoint{3.462541in}{1.408738in}}%
\pgfpathlineto{\pgfqpoint{3.456911in}{1.416526in}}%
\pgfpathlineto{\pgfqpoint{3.451192in}{1.423851in}}%
\pgfpathlineto{\pgfqpoint{3.445406in}{1.430684in}}%
\pgfpathlineto{\pgfqpoint{3.439576in}{1.436997in}}%
\pgfpathlineto{\pgfqpoint{3.417797in}{1.430713in}}%
\pgfpathlineto{\pgfqpoint{3.396885in}{1.423726in}}%
\pgfpathlineto{\pgfqpoint{3.376927in}{1.416068in}}%
\pgfpathlineto{\pgfqpoint{3.358007in}{1.407773in}}%
\pgfpathlineto{\pgfqpoint{3.340202in}{1.398881in}}%
\pgfpathclose%
\pgfusepath{fill}%
\end{pgfscope}%
\begin{pgfscope}%
\pgfpathrectangle{\pgfqpoint{2.548318in}{0.050000in}}{\pgfqpoint{2.081932in}{2.081932in}}%
\pgfusepath{clip}%
\pgfsetbuttcap%
\pgfsetroundjoin%
\definecolor{currentfill}{rgb}{0.124780,0.640461,0.527068}%
\pgfsetfillcolor{currentfill}%
\pgfsetlinewidth{0.000000pt}%
\definecolor{currentstroke}{rgb}{0.000000,0.000000,0.000000}%
\pgfsetstrokecolor{currentstroke}%
\pgfsetdash{}{0pt}%
\pgfpathmoveto{\pgfqpoint{3.040646in}{1.181811in}}%
\pgfpathlineto{\pgfqpoint{3.041300in}{1.190651in}}%
\pgfpathlineto{\pgfqpoint{3.042733in}{1.199363in}}%
\pgfpathlineto{\pgfqpoint{3.044936in}{1.207910in}}%
\pgfpathlineto{\pgfqpoint{3.047900in}{1.216259in}}%
\pgfpathlineto{\pgfqpoint{3.051611in}{1.224376in}}%
\pgfpathlineto{\pgfqpoint{3.059025in}{1.242536in}}%
\pgfpathlineto{\pgfqpoint{3.068761in}{1.260468in}}%
\pgfpathlineto{\pgfqpoint{3.080798in}{1.278094in}}%
\pgfpathlineto{\pgfqpoint{3.095104in}{1.295338in}}%
\pgfpathlineto{\pgfqpoint{3.111635in}{1.312122in}}%
\pgfpathlineto{\pgfqpoint{3.108272in}{1.304480in}}%
\pgfpathlineto{\pgfqpoint{3.105587in}{1.296484in}}%
\pgfpathlineto{\pgfqpoint{3.103591in}{1.288167in}}%
\pgfpathlineto{\pgfqpoint{3.102293in}{1.279562in}}%
\pgfpathlineto{\pgfqpoint{3.101700in}{1.270704in}}%
\pgfpathlineto{\pgfqpoint{3.084871in}{1.253697in}}%
\pgfpathlineto{\pgfqpoint{3.070312in}{1.236226in}}%
\pgfpathlineto{\pgfqpoint{3.058069in}{1.218369in}}%
\pgfpathlineto{\pgfqpoint{3.048173in}{1.200205in}}%
\pgfpathlineto{\pgfqpoint{3.040646in}{1.181811in}}%
\pgfpathclose%
\pgfusepath{fill}%
\end{pgfscope}%
\begin{pgfscope}%
\pgfpathrectangle{\pgfqpoint{2.548318in}{0.050000in}}{\pgfqpoint{2.081932in}{2.081932in}}%
\pgfusepath{clip}%
\pgfsetbuttcap%
\pgfsetroundjoin%
\definecolor{currentfill}{rgb}{0.278791,0.062145,0.386592}%
\pgfsetfillcolor{currentfill}%
\pgfsetlinewidth{0.000000pt}%
\definecolor{currentstroke}{rgb}{0.000000,0.000000,0.000000}%
\pgfsetstrokecolor{currentstroke}%
\pgfsetdash{}{0pt}%
\pgfpathmoveto{\pgfqpoint{3.339175in}{1.032178in}}%
\pgfpathlineto{\pgfqpoint{3.328676in}{1.028562in}}%
\pgfpathlineto{\pgfqpoint{3.317786in}{1.025478in}}%
\pgfpathlineto{\pgfqpoint{3.306550in}{1.022939in}}%
\pgfpathlineto{\pgfqpoint{3.295012in}{1.020957in}}%
\pgfpathlineto{\pgfqpoint{3.283219in}{1.019540in}}%
\pgfpathlineto{\pgfqpoint{3.287970in}{1.029923in}}%
\pgfpathlineto{\pgfqpoint{3.294074in}{1.040153in}}%
\pgfpathlineto{\pgfqpoint{3.301513in}{1.050187in}}%
\pgfpathlineto{\pgfqpoint{3.310264in}{1.059981in}}%
\pgfpathlineto{\pgfqpoint{3.320294in}{1.069495in}}%
\pgfpathlineto{\pgfqpoint{3.330857in}{1.069148in}}%
\pgfpathlineto{\pgfqpoint{3.341188in}{1.069408in}}%
\pgfpathlineto{\pgfqpoint{3.351242in}{1.070273in}}%
\pgfpathlineto{\pgfqpoint{3.360982in}{1.071740in}}%
\pgfpathlineto{\pgfqpoint{3.370367in}{1.073800in}}%
\pgfpathlineto{\pgfqpoint{3.361963in}{1.065882in}}%
\pgfpathlineto{\pgfqpoint{3.354618in}{1.057725in}}%
\pgfpathlineto{\pgfqpoint{3.348358in}{1.049366in}}%
\pgfpathlineto{\pgfqpoint{3.343204in}{1.040838in}}%
\pgfpathlineto{\pgfqpoint{3.339175in}{1.032178in}}%
\pgfpathclose%
\pgfusepath{fill}%
\end{pgfscope}%
\begin{pgfscope}%
\pgfpathrectangle{\pgfqpoint{2.548318in}{0.050000in}}{\pgfqpoint{2.081932in}{2.081932in}}%
\pgfusepath{clip}%
\pgfsetbuttcap%
\pgfsetroundjoin%
\definecolor{currentfill}{rgb}{0.267004,0.004874,0.329415}%
\pgfsetfillcolor{currentfill}%
\pgfsetlinewidth{0.000000pt}%
\definecolor{currentstroke}{rgb}{0.000000,0.000000,0.000000}%
\pgfsetstrokecolor{currentstroke}%
\pgfsetdash{}{0pt}%
\pgfpathmoveto{\pgfqpoint{3.927074in}{1.057538in}}%
\pgfpathlineto{\pgfqpoint{3.938253in}{1.058059in}}%
\pgfpathlineto{\pgfqpoint{3.949583in}{1.059177in}}%
\pgfpathlineto{\pgfqpoint{3.961018in}{1.060889in}}%
\pgfpathlineto{\pgfqpoint{3.972511in}{1.063188in}}%
\pgfpathlineto{\pgfqpoint{3.984016in}{1.066066in}}%
\pgfpathlineto{\pgfqpoint{3.993887in}{1.054381in}}%
\pgfpathlineto{\pgfqpoint{4.002192in}{1.042433in}}%
\pgfpathlineto{\pgfqpoint{4.008904in}{1.030272in}}%
\pgfpathlineto{\pgfqpoint{4.014005in}{1.017949in}}%
\pgfpathlineto{\pgfqpoint{4.017483in}{1.005517in}}%
\pgfpathlineto{\pgfqpoint{4.005042in}{1.004549in}}%
\pgfpathlineto{\pgfqpoint{3.992606in}{1.004152in}}%
\pgfpathlineto{\pgfqpoint{3.980226in}{1.004327in}}%
\pgfpathlineto{\pgfqpoint{3.967952in}{1.005074in}}%
\pgfpathlineto{\pgfqpoint{3.955835in}{1.006389in}}%
\pgfpathlineto{\pgfqpoint{3.952793in}{1.016904in}}%
\pgfpathlineto{\pgfqpoint{3.948381in}{1.027319in}}%
\pgfpathlineto{\pgfqpoint{3.942611in}{1.037592in}}%
\pgfpathlineto{\pgfqpoint{3.935501in}{1.047679in}}%
\pgfpathlineto{\pgfqpoint{3.927074in}{1.057538in}}%
\pgfpathclose%
\pgfusepath{fill}%
\end{pgfscope}%
\begin{pgfscope}%
\pgfpathrectangle{\pgfqpoint{2.548318in}{0.050000in}}{\pgfqpoint{2.081932in}{2.081932in}}%
\pgfusepath{clip}%
\pgfsetbuttcap%
\pgfsetroundjoin%
\definecolor{currentfill}{rgb}{0.855810,0.888601,0.097452}%
\pgfsetfillcolor{currentfill}%
\pgfsetlinewidth{0.000000pt}%
\definecolor{currentstroke}{rgb}{0.000000,0.000000,0.000000}%
\pgfsetstrokecolor{currentstroke}%
\pgfsetdash{}{0pt}%
\pgfpathmoveto{\pgfqpoint{3.672565in}{1.416746in}}%
\pgfpathlineto{\pgfqpoint{3.670576in}{1.407524in}}%
\pgfpathlineto{\pgfqpoint{3.668644in}{1.397956in}}%
\pgfpathlineto{\pgfqpoint{3.666775in}{1.388079in}}%
\pgfpathlineto{\pgfqpoint{3.664978in}{1.377932in}}%
\pgfpathlineto{\pgfqpoint{3.663258in}{1.367555in}}%
\pgfpathlineto{\pgfqpoint{3.680406in}{1.365764in}}%
\pgfpathlineto{\pgfqpoint{3.697284in}{1.363414in}}%
\pgfpathlineto{\pgfqpoint{3.713822in}{1.360514in}}%
\pgfpathlineto{\pgfqpoint{3.729949in}{1.357078in}}%
\pgfpathlineto{\pgfqpoint{3.745597in}{1.353121in}}%
\pgfpathlineto{\pgfqpoint{3.750402in}{1.362933in}}%
\pgfpathlineto{\pgfqpoint{3.755423in}{1.372490in}}%
\pgfpathlineto{\pgfqpoint{3.760642in}{1.381753in}}%
\pgfpathlineto{\pgfqpoint{3.766039in}{1.390686in}}%
\pgfpathlineto{\pgfqpoint{3.771592in}{1.399255in}}%
\pgfpathlineto{\pgfqpoint{3.752784in}{1.404048in}}%
\pgfpathlineto{\pgfqpoint{3.733394in}{1.408211in}}%
\pgfpathlineto{\pgfqpoint{3.713503in}{1.411726in}}%
\pgfpathlineto{\pgfqpoint{3.693198in}{1.414575in}}%
\pgfpathlineto{\pgfqpoint{3.672565in}{1.416746in}}%
\pgfpathclose%
\pgfusepath{fill}%
\end{pgfscope}%
\begin{pgfscope}%
\pgfpathrectangle{\pgfqpoint{2.548318in}{0.050000in}}{\pgfqpoint{2.081932in}{2.081932in}}%
\pgfusepath{clip}%
\pgfsetbuttcap%
\pgfsetroundjoin%
\definecolor{currentfill}{rgb}{0.206756,0.371758,0.553117}%
\pgfsetfillcolor{currentfill}%
\pgfsetlinewidth{0.000000pt}%
\definecolor{currentstroke}{rgb}{0.000000,0.000000,0.000000}%
\pgfsetstrokecolor{currentstroke}%
\pgfsetdash{}{0pt}%
\pgfpathmoveto{\pgfqpoint{4.127076in}{1.165794in}}%
\pgfpathlineto{\pgfqpoint{4.132829in}{1.174958in}}%
\pgfpathlineto{\pgfqpoint{4.137930in}{1.184242in}}%
\pgfpathlineto{\pgfqpoint{4.142358in}{1.193607in}}%
\pgfpathlineto{\pgfqpoint{4.146094in}{1.203015in}}%
\pgfpathlineto{\pgfqpoint{4.149124in}{1.212426in}}%
\pgfpathlineto{\pgfqpoint{4.162998in}{1.195107in}}%
\pgfpathlineto{\pgfqpoint{4.174569in}{1.177426in}}%
\pgfpathlineto{\pgfqpoint{4.183808in}{1.159460in}}%
\pgfpathlineto{\pgfqpoint{4.190696in}{1.141287in}}%
\pgfpathlineto{\pgfqpoint{4.195224in}{1.122983in}}%
\pgfpathlineto{\pgfqpoint{4.191975in}{1.114202in}}%
\pgfpathlineto{\pgfqpoint{4.187969in}{1.105540in}}%
\pgfpathlineto{\pgfqpoint{4.183220in}{1.097032in}}%
\pgfpathlineto{\pgfqpoint{4.177749in}{1.088714in}}%
\pgfpathlineto{\pgfqpoint{4.171575in}{1.080619in}}%
\pgfpathlineto{\pgfqpoint{4.167170in}{1.098058in}}%
\pgfpathlineto{\pgfqpoint{4.160502in}{1.115367in}}%
\pgfpathlineto{\pgfqpoint{4.151583in}{1.132475in}}%
\pgfpathlineto{\pgfqpoint{4.140431in}{1.149309in}}%
\pgfpathlineto{\pgfqpoint{4.127076in}{1.165794in}}%
\pgfpathclose%
\pgfusepath{fill}%
\end{pgfscope}%
\begin{pgfscope}%
\pgfpathrectangle{\pgfqpoint{2.548318in}{0.050000in}}{\pgfqpoint{2.081932in}{2.081932in}}%
\pgfusepath{clip}%
\pgfsetbuttcap%
\pgfsetroundjoin%
\definecolor{currentfill}{rgb}{0.855810,0.888601,0.097452}%
\pgfsetfillcolor{currentfill}%
\pgfsetlinewidth{0.000000pt}%
\definecolor{currentstroke}{rgb}{0.000000,0.000000,0.000000}%
\pgfsetstrokecolor{currentstroke}%
\pgfsetdash{}{0pt}%
\pgfpathmoveto{\pgfqpoint{3.468061in}{1.400521in}}%
\pgfpathlineto{\pgfqpoint{3.473450in}{1.391905in}}%
\pgfpathlineto{\pgfqpoint{3.478686in}{1.382926in}}%
\pgfpathlineto{\pgfqpoint{3.483749in}{1.373618in}}%
\pgfpathlineto{\pgfqpoint{3.488621in}{1.364019in}}%
\pgfpathlineto{\pgfqpoint{3.493283in}{1.354166in}}%
\pgfpathlineto{\pgfqpoint{3.509056in}{1.357993in}}%
\pgfpathlineto{\pgfqpoint{3.525291in}{1.361296in}}%
\pgfpathlineto{\pgfqpoint{3.541920in}{1.364058in}}%
\pgfpathlineto{\pgfqpoint{3.558873in}{1.366269in}}%
\pgfpathlineto{\pgfqpoint{3.576075in}{1.367917in}}%
\pgfpathlineto{\pgfqpoint{3.574517in}{1.378308in}}%
\pgfpathlineto{\pgfqpoint{3.572888in}{1.388470in}}%
\pgfpathlineto{\pgfqpoint{3.571195in}{1.398362in}}%
\pgfpathlineto{\pgfqpoint{3.569443in}{1.407946in}}%
\pgfpathlineto{\pgfqpoint{3.567641in}{1.417184in}}%
\pgfpathlineto{\pgfqpoint{3.546941in}{1.415186in}}%
\pgfpathlineto{\pgfqpoint{3.526546in}{1.412507in}}%
\pgfpathlineto{\pgfqpoint{3.506544in}{1.409159in}}%
\pgfpathlineto{\pgfqpoint{3.487022in}{1.405157in}}%
\pgfpathlineto{\pgfqpoint{3.468061in}{1.400521in}}%
\pgfpathclose%
\pgfusepath{fill}%
\end{pgfscope}%
\begin{pgfscope}%
\pgfpathrectangle{\pgfqpoint{2.548318in}{0.050000in}}{\pgfqpoint{2.081932in}{2.081932in}}%
\pgfusepath{clip}%
\pgfsetbuttcap%
\pgfsetroundjoin%
\definecolor{currentfill}{rgb}{0.162142,0.474838,0.558140}%
\pgfsetfillcolor{currentfill}%
\pgfsetlinewidth{0.000000pt}%
\definecolor{currentstroke}{rgb}{0.000000,0.000000,0.000000}%
\pgfsetstrokecolor{currentstroke}%
\pgfsetdash{}{0pt}%
\pgfpathmoveto{\pgfqpoint{3.650342in}{1.212179in}}%
\pgfpathlineto{\pgfqpoint{3.650491in}{1.203665in}}%
\pgfpathlineto{\pgfqpoint{3.650774in}{1.195591in}}%
\pgfpathlineto{\pgfqpoint{3.651190in}{1.187990in}}%
\pgfpathlineto{\pgfqpoint{3.651737in}{1.180893in}}%
\pgfpathlineto{\pgfqpoint{3.652414in}{1.174328in}}%
\pgfpathlineto{\pgfqpoint{3.665498in}{1.173001in}}%
\pgfpathlineto{\pgfqpoint{3.678381in}{1.171259in}}%
\pgfpathlineto{\pgfqpoint{3.691006in}{1.169109in}}%
\pgfpathlineto{\pgfqpoint{3.703323in}{1.166561in}}%
\pgfpathlineto{\pgfqpoint{3.715278in}{1.163626in}}%
\pgfpathlineto{\pgfqpoint{3.713385in}{1.170388in}}%
\pgfpathlineto{\pgfqpoint{3.711854in}{1.177642in}}%
\pgfpathlineto{\pgfqpoint{3.710691in}{1.185358in}}%
\pgfpathlineto{\pgfqpoint{3.709900in}{1.193506in}}%
\pgfpathlineto{\pgfqpoint{3.709483in}{1.202054in}}%
\pgfpathlineto{\pgfqpoint{3.698234in}{1.204831in}}%
\pgfpathlineto{\pgfqpoint{3.686646in}{1.207241in}}%
\pgfpathlineto{\pgfqpoint{3.674769in}{1.209275in}}%
\pgfpathlineto{\pgfqpoint{3.662650in}{1.210924in}}%
\pgfpathlineto{\pgfqpoint{3.650342in}{1.212179in}}%
\pgfpathclose%
\pgfusepath{fill}%
\end{pgfscope}%
\begin{pgfscope}%
\pgfpathrectangle{\pgfqpoint{2.548318in}{0.050000in}}{\pgfqpoint{2.081932in}{2.081932in}}%
\pgfusepath{clip}%
\pgfsetbuttcap%
\pgfsetroundjoin%
\definecolor{currentfill}{rgb}{0.162142,0.474838,0.558140}%
\pgfsetfillcolor{currentfill}%
\pgfsetlinewidth{0.000000pt}%
\definecolor{currentstroke}{rgb}{0.000000,0.000000,0.000000}%
\pgfsetstrokecolor{currentstroke}%
\pgfsetdash{}{0pt}%
\pgfpathmoveto{\pgfqpoint{3.528321in}{1.202787in}}%
\pgfpathlineto{\pgfqpoint{3.527916in}{1.194242in}}%
\pgfpathlineto{\pgfqpoint{3.527148in}{1.186099in}}%
\pgfpathlineto{\pgfqpoint{3.526020in}{1.178392in}}%
\pgfpathlineto{\pgfqpoint{3.524535in}{1.171149in}}%
\pgfpathlineto{\pgfqpoint{3.522699in}{1.164402in}}%
\pgfpathlineto{\pgfqpoint{3.534749in}{1.167240in}}%
\pgfpathlineto{\pgfqpoint{3.547147in}{1.169689in}}%
\pgfpathlineto{\pgfqpoint{3.559841in}{1.171736in}}%
\pgfpathlineto{\pgfqpoint{3.572779in}{1.173375in}}%
\pgfpathlineto{\pgfqpoint{3.585905in}{1.174596in}}%
\pgfpathlineto{\pgfqpoint{3.586518in}{1.181156in}}%
\pgfpathlineto{\pgfqpoint{3.587014in}{1.188250in}}%
\pgfpathlineto{\pgfqpoint{3.587390in}{1.195847in}}%
\pgfpathlineto{\pgfqpoint{3.587647in}{1.203919in}}%
\pgfpathlineto{\pgfqpoint{3.587782in}{1.212433in}}%
\pgfpathlineto{\pgfqpoint{3.575435in}{1.211278in}}%
\pgfpathlineto{\pgfqpoint{3.563264in}{1.209728in}}%
\pgfpathlineto{\pgfqpoint{3.551322in}{1.207790in}}%
\pgfpathlineto{\pgfqpoint{3.539658in}{1.205473in}}%
\pgfpathlineto{\pgfqpoint{3.528321in}{1.202787in}}%
\pgfpathclose%
\pgfusepath{fill}%
\end{pgfscope}%
\begin{pgfscope}%
\pgfpathrectangle{\pgfqpoint{2.548318in}{0.050000in}}{\pgfqpoint{2.081932in}{2.081932in}}%
\pgfusepath{clip}%
\pgfsetbuttcap%
\pgfsetroundjoin%
\definecolor{currentfill}{rgb}{0.876168,0.891125,0.095250}%
\pgfsetfillcolor{currentfill}%
\pgfsetlinewidth{0.000000pt}%
\definecolor{currentstroke}{rgb}{0.000000,0.000000,0.000000}%
\pgfsetstrokecolor{currentstroke}%
\pgfsetdash{}{0pt}%
\pgfpathmoveto{\pgfqpoint{3.177958in}{1.363281in}}%
\pgfpathlineto{\pgfqpoint{3.187244in}{1.365434in}}%
\pgfpathlineto{\pgfqpoint{3.196875in}{1.366998in}}%
\pgfpathlineto{\pgfqpoint{3.206812in}{1.367967in}}%
\pgfpathlineto{\pgfqpoint{3.217015in}{1.368339in}}%
\pgfpathlineto{\pgfqpoint{3.227444in}{1.368113in}}%
\pgfpathlineto{\pgfqpoint{3.242073in}{1.380740in}}%
\pgfpathlineto{\pgfqpoint{3.258299in}{1.392880in}}%
\pgfpathlineto{\pgfqpoint{3.276063in}{1.404479in}}%
\pgfpathlineto{\pgfqpoint{3.295295in}{1.415484in}}%
\pgfpathlineto{\pgfqpoint{3.286618in}{1.416987in}}%
\pgfpathlineto{\pgfqpoint{3.278126in}{1.417861in}}%
\pgfpathlineto{\pgfqpoint{3.269852in}{1.418101in}}%
\pgfpathlineto{\pgfqpoint{3.261830in}{1.417704in}}%
\pgfpathlineto{\pgfqpoint{3.254093in}{1.416671in}}%
\pgfpathlineto{\pgfqpoint{3.232482in}{1.404260in}}%
\pgfpathlineto{\pgfqpoint{3.212539in}{1.391184in}}%
\pgfpathlineto{\pgfqpoint{3.194342in}{1.377504in}}%
\pgfpathlineto{\pgfqpoint{3.177958in}{1.363281in}}%
\pgfpathclose%
\pgfusepath{fill}%
\end{pgfscope}%
\begin{pgfscope}%
\pgfpathrectangle{\pgfqpoint{2.548318in}{0.050000in}}{\pgfqpoint{2.081932in}{2.081932in}}%
\pgfusepath{clip}%
\pgfsetbuttcap%
\pgfsetroundjoin%
\definecolor{currentfill}{rgb}{0.268510,0.009605,0.335427}%
\pgfsetfillcolor{currentfill}%
\pgfsetlinewidth{0.000000pt}%
\definecolor{currentstroke}{rgb}{0.000000,0.000000,0.000000}%
\pgfsetstrokecolor{currentstroke}%
\pgfsetdash{}{0pt}%
\pgfpathmoveto{\pgfqpoint{3.984016in}{1.066066in}}%
\pgfpathlineto{\pgfqpoint{3.995484in}{1.069511in}}%
\pgfpathlineto{\pgfqpoint{4.006869in}{1.073511in}}%
\pgfpathlineto{\pgfqpoint{4.018123in}{1.078048in}}%
\pgfpathlineto{\pgfqpoint{4.029199in}{1.083105in}}%
\pgfpathlineto{\pgfqpoint{4.040052in}{1.088662in}}%
\pgfpathlineto{\pgfqpoint{4.051313in}{1.075148in}}%
\pgfpathlineto{\pgfqpoint{4.060759in}{1.061336in}}%
\pgfpathlineto{\pgfqpoint{4.068363in}{1.047286in}}%
\pgfpathlineto{\pgfqpoint{4.074106in}{1.033058in}}%
\pgfpathlineto{\pgfqpoint{4.077975in}{1.018711in}}%
\pgfpathlineto{\pgfqpoint{4.066273in}{1.014991in}}%
\pgfpathlineto{\pgfqpoint{4.054322in}{1.011799in}}%
\pgfpathlineto{\pgfqpoint{4.042174in}{1.009149in}}%
\pgfpathlineto{\pgfqpoint{4.029877in}{1.007053in}}%
\pgfpathlineto{\pgfqpoint{4.017483in}{1.005517in}}%
\pgfpathlineto{\pgfqpoint{4.014005in}{1.017949in}}%
\pgfpathlineto{\pgfqpoint{4.008904in}{1.030272in}}%
\pgfpathlineto{\pgfqpoint{4.002192in}{1.042433in}}%
\pgfpathlineto{\pgfqpoint{3.993887in}{1.054381in}}%
\pgfpathlineto{\pgfqpoint{3.984016in}{1.066066in}}%
\pgfpathclose%
\pgfusepath{fill}%
\end{pgfscope}%
\begin{pgfscope}%
\pgfpathrectangle{\pgfqpoint{2.548318in}{0.050000in}}{\pgfqpoint{2.081932in}{2.081932in}}%
\pgfusepath{clip}%
\pgfsetbuttcap%
\pgfsetroundjoin%
\definecolor{currentfill}{rgb}{0.278012,0.180367,0.486697}%
\pgfsetfillcolor{currentfill}%
\pgfsetlinewidth{0.000000pt}%
\definecolor{currentstroke}{rgb}{0.000000,0.000000,0.000000}%
\pgfsetstrokecolor{currentstroke}%
\pgfsetdash{}{0pt}%
\pgfpathmoveto{\pgfqpoint{3.790404in}{1.115226in}}%
\pgfpathlineto{\pgfqpoint{3.796354in}{1.111148in}}%
\pgfpathlineto{\pgfqpoint{3.802743in}{1.107652in}}%
\pgfpathlineto{\pgfqpoint{3.809548in}{1.104752in}}%
\pgfpathlineto{\pgfqpoint{3.816741in}{1.102463in}}%
\pgfpathlineto{\pgfqpoint{3.824295in}{1.100795in}}%
\pgfpathlineto{\pgfqpoint{3.836357in}{1.094061in}}%
\pgfpathlineto{\pgfqpoint{3.847501in}{1.086968in}}%
\pgfpathlineto{\pgfqpoint{3.857682in}{1.079547in}}%
\pgfpathlineto{\pgfqpoint{3.866862in}{1.071829in}}%
\pgfpathlineto{\pgfqpoint{3.875006in}{1.063847in}}%
\pgfpathlineto{\pgfqpoint{3.865650in}{1.066849in}}%
\pgfpathlineto{\pgfqpoint{3.856738in}{1.070405in}}%
\pgfpathlineto{\pgfqpoint{3.848305in}{1.074501in}}%
\pgfpathlineto{\pgfqpoint{3.840384in}{1.079118in}}%
\pgfpathlineto{\pgfqpoint{3.833005in}{1.084237in}}%
\pgfpathlineto{\pgfqpoint{3.826147in}{1.090936in}}%
\pgfpathlineto{\pgfqpoint{3.818427in}{1.097412in}}%
\pgfpathlineto{\pgfqpoint{3.809874in}{1.103636in}}%
\pgfpathlineto{\pgfqpoint{3.800521in}{1.109583in}}%
\pgfpathlineto{\pgfqpoint{3.790404in}{1.115226in}}%
\pgfpathclose%
\pgfusepath{fill}%
\end{pgfscope}%
\begin{pgfscope}%
\pgfpathrectangle{\pgfqpoint{2.548318in}{0.050000in}}{\pgfqpoint{2.081932in}{2.081932in}}%
\pgfusepath{clip}%
\pgfsetbuttcap%
\pgfsetroundjoin%
\definecolor{currentfill}{rgb}{0.267968,0.223549,0.512008}%
\pgfsetfillcolor{currentfill}%
\pgfsetlinewidth{0.000000pt}%
\definecolor{currentstroke}{rgb}{0.000000,0.000000,0.000000}%
\pgfsetstrokecolor{currentstroke}%
\pgfsetdash{}{0pt}%
\pgfpathmoveto{\pgfqpoint{4.089434in}{1.123075in}}%
\pgfpathlineto{\pgfqpoint{4.098046in}{1.131086in}}%
\pgfpathlineto{\pgfqpoint{4.106149in}{1.139397in}}%
\pgfpathlineto{\pgfqpoint{4.113708in}{1.147977in}}%
\pgfpathlineto{\pgfqpoint{4.120694in}{1.156788in}}%
\pgfpathlineto{\pgfqpoint{4.127076in}{1.165794in}}%
\pgfpathlineto{\pgfqpoint{4.140431in}{1.149309in}}%
\pgfpathlineto{\pgfqpoint{4.151583in}{1.132475in}}%
\pgfpathlineto{\pgfqpoint{4.160502in}{1.115367in}}%
\pgfpathlineto{\pgfqpoint{4.167170in}{1.098058in}}%
\pgfpathlineto{\pgfqpoint{4.171575in}{1.080619in}}%
\pgfpathlineto{\pgfqpoint{4.164725in}{1.072782in}}%
\pgfpathlineto{\pgfqpoint{4.157224in}{1.065234in}}%
\pgfpathlineto{\pgfqpoint{4.149105in}{1.058007in}}%
\pgfpathlineto{\pgfqpoint{4.140398in}{1.051130in}}%
\pgfpathlineto{\pgfqpoint{4.131140in}{1.044632in}}%
\pgfpathlineto{\pgfqpoint{4.126956in}{1.060705in}}%
\pgfpathlineto{\pgfqpoint{4.120678in}{1.076652in}}%
\pgfpathlineto{\pgfqpoint{4.112317in}{1.092408in}}%
\pgfpathlineto{\pgfqpoint{4.101893in}{1.107905in}}%
\pgfpathlineto{\pgfqpoint{4.089434in}{1.123075in}}%
\pgfpathclose%
\pgfusepath{fill}%
\end{pgfscope}%
\begin{pgfscope}%
\pgfpathrectangle{\pgfqpoint{2.548318in}{0.050000in}}{\pgfqpoint{2.081932in}{2.081932in}}%
\pgfusepath{clip}%
\pgfsetbuttcap%
\pgfsetroundjoin%
\definecolor{currentfill}{rgb}{0.855810,0.888601,0.097452}%
\pgfsetfillcolor{currentfill}%
\pgfsetlinewidth{0.000000pt}%
\definecolor{currentstroke}{rgb}{0.000000,0.000000,0.000000}%
\pgfsetstrokecolor{currentstroke}%
\pgfsetdash{}{0pt}%
\pgfpathmoveto{\pgfqpoint{3.567641in}{1.417184in}}%
\pgfpathlineto{\pgfqpoint{3.569443in}{1.407946in}}%
\pgfpathlineto{\pgfqpoint{3.571195in}{1.398362in}}%
\pgfpathlineto{\pgfqpoint{3.572888in}{1.388470in}}%
\pgfpathlineto{\pgfqpoint{3.574517in}{1.378308in}}%
\pgfpathlineto{\pgfqpoint{3.576075in}{1.367917in}}%
\pgfpathlineto{\pgfqpoint{3.593456in}{1.368996in}}%
\pgfpathlineto{\pgfqpoint{3.610939in}{1.369500in}}%
\pgfpathlineto{\pgfqpoint{3.628451in}{1.369427in}}%
\pgfpathlineto{\pgfqpoint{3.645915in}{1.368778in}}%
\pgfpathlineto{\pgfqpoint{3.663258in}{1.367555in}}%
\pgfpathlineto{\pgfqpoint{3.664978in}{1.377932in}}%
\pgfpathlineto{\pgfqpoint{3.666775in}{1.388079in}}%
\pgfpathlineto{\pgfqpoint{3.668644in}{1.397956in}}%
\pgfpathlineto{\pgfqpoint{3.670576in}{1.407524in}}%
\pgfpathlineto{\pgfqpoint{3.672565in}{1.416746in}}%
\pgfpathlineto{\pgfqpoint{3.651694in}{1.418228in}}%
\pgfpathlineto{\pgfqpoint{3.630675in}{1.419015in}}%
\pgfpathlineto{\pgfqpoint{3.609599in}{1.419103in}}%
\pgfpathlineto{\pgfqpoint{3.588557in}{1.418492in}}%
\pgfpathlineto{\pgfqpoint{3.567641in}{1.417184in}}%
\pgfpathclose%
\pgfusepath{fill}%
\end{pgfscope}%
\begin{pgfscope}%
\pgfpathrectangle{\pgfqpoint{2.548318in}{0.050000in}}{\pgfqpoint{2.081932in}{2.081932in}}%
\pgfusepath{clip}%
\pgfsetbuttcap%
\pgfsetroundjoin%
\definecolor{currentfill}{rgb}{0.162142,0.474838,0.558140}%
\pgfsetfillcolor{currentfill}%
\pgfsetlinewidth{0.000000pt}%
\definecolor{currentstroke}{rgb}{0.000000,0.000000,0.000000}%
\pgfsetstrokecolor{currentstroke}%
\pgfsetdash{}{0pt}%
\pgfpathmoveto{\pgfqpoint{3.587782in}{1.212433in}}%
\pgfpathlineto{\pgfqpoint{3.587647in}{1.203919in}}%
\pgfpathlineto{\pgfqpoint{3.587390in}{1.195847in}}%
\pgfpathlineto{\pgfqpoint{3.587014in}{1.188250in}}%
\pgfpathlineto{\pgfqpoint{3.586518in}{1.181156in}}%
\pgfpathlineto{\pgfqpoint{3.585905in}{1.174596in}}%
\pgfpathlineto{\pgfqpoint{3.599164in}{1.175396in}}%
\pgfpathlineto{\pgfqpoint{3.612501in}{1.175769in}}%
\pgfpathlineto{\pgfqpoint{3.625859in}{1.175715in}}%
\pgfpathlineto{\pgfqpoint{3.639182in}{1.175234in}}%
\pgfpathlineto{\pgfqpoint{3.652414in}{1.174328in}}%
\pgfpathlineto{\pgfqpoint{3.651737in}{1.180893in}}%
\pgfpathlineto{\pgfqpoint{3.651190in}{1.187990in}}%
\pgfpathlineto{\pgfqpoint{3.650774in}{1.195591in}}%
\pgfpathlineto{\pgfqpoint{3.650491in}{1.203665in}}%
\pgfpathlineto{\pgfqpoint{3.650342in}{1.212179in}}%
\pgfpathlineto{\pgfqpoint{3.637896in}{1.213037in}}%
\pgfpathlineto{\pgfqpoint{3.625364in}{1.213492in}}%
\pgfpathlineto{\pgfqpoint{3.612799in}{1.213543in}}%
\pgfpathlineto{\pgfqpoint{3.600254in}{1.213189in}}%
\pgfpathlineto{\pgfqpoint{3.587782in}{1.212433in}}%
\pgfpathclose%
\pgfusepath{fill}%
\end{pgfscope}%
\begin{pgfscope}%
\pgfpathrectangle{\pgfqpoint{2.548318in}{0.050000in}}{\pgfqpoint{2.081932in}{2.081932in}}%
\pgfusepath{clip}%
\pgfsetbuttcap%
\pgfsetroundjoin%
\definecolor{currentfill}{rgb}{0.282327,0.094955,0.417331}%
\pgfsetfillcolor{currentfill}%
\pgfsetlinewidth{0.000000pt}%
\definecolor{currentstroke}{rgb}{0.000000,0.000000,0.000000}%
\pgfsetstrokecolor{currentstroke}%
\pgfsetdash{}{0pt}%
\pgfpathmoveto{\pgfqpoint{4.040052in}{1.088662in}}%
\pgfpathlineto{\pgfqpoint{4.050637in}{1.094696in}}%
\pgfpathlineto{\pgfqpoint{4.060910in}{1.101181in}}%
\pgfpathlineto{\pgfqpoint{4.070828in}{1.108092in}}%
\pgfpathlineto{\pgfqpoint{4.080349in}{1.115401in}}%
\pgfpathlineto{\pgfqpoint{4.089434in}{1.123075in}}%
\pgfpathlineto{\pgfqpoint{4.101893in}{1.107905in}}%
\pgfpathlineto{\pgfqpoint{4.112317in}{1.092408in}}%
\pgfpathlineto{\pgfqpoint{4.120678in}{1.076652in}}%
\pgfpathlineto{\pgfqpoint{4.126956in}{1.060705in}}%
\pgfpathlineto{\pgfqpoint{4.131140in}{1.044632in}}%
\pgfpathlineto{\pgfqpoint{4.121369in}{1.038541in}}%
\pgfpathlineto{\pgfqpoint{4.111124in}{1.032880in}}%
\pgfpathlineto{\pgfqpoint{4.100447in}{1.027674in}}%
\pgfpathlineto{\pgfqpoint{4.089383in}{1.022945in}}%
\pgfpathlineto{\pgfqpoint{4.077975in}{1.018711in}}%
\pgfpathlineto{\pgfqpoint{4.074106in}{1.033058in}}%
\pgfpathlineto{\pgfqpoint{4.068363in}{1.047286in}}%
\pgfpathlineto{\pgfqpoint{4.060759in}{1.061336in}}%
\pgfpathlineto{\pgfqpoint{4.051313in}{1.075148in}}%
\pgfpathlineto{\pgfqpoint{4.040052in}{1.088662in}}%
\pgfpathclose%
\pgfusepath{fill}%
\end{pgfscope}%
\begin{pgfscope}%
\pgfpathrectangle{\pgfqpoint{2.548318in}{0.050000in}}{\pgfqpoint{2.081932in}{2.081932in}}%
\pgfusepath{clip}%
\pgfsetbuttcap%
\pgfsetroundjoin%
\definecolor{currentfill}{rgb}{0.227802,0.326594,0.546532}%
\pgfsetfillcolor{currentfill}%
\pgfsetlinewidth{0.000000pt}%
\definecolor{currentstroke}{rgb}{0.000000,0.000000,0.000000}%
\pgfsetstrokecolor{currentstroke}%
\pgfsetdash{}{0pt}%
\pgfpathmoveto{\pgfqpoint{3.715278in}{1.163626in}}%
\pgfpathlineto{\pgfqpoint{3.717527in}{1.157384in}}%
\pgfpathlineto{\pgfqpoint{3.720124in}{1.151686in}}%
\pgfpathlineto{\pgfqpoint{3.723059in}{1.146557in}}%
\pgfpathlineto{\pgfqpoint{3.726323in}{1.142017in}}%
\pgfpathlineto{\pgfqpoint{3.729902in}{1.138087in}}%
\pgfpathlineto{\pgfqpoint{3.743164in}{1.134299in}}%
\pgfpathlineto{\pgfqpoint{3.755897in}{1.130099in}}%
\pgfpathlineto{\pgfqpoint{3.768047in}{1.125507in}}%
\pgfpathlineto{\pgfqpoint{3.779566in}{1.120542in}}%
\pgfpathlineto{\pgfqpoint{3.790404in}{1.115226in}}%
\pgfpathlineto{\pgfqpoint{3.784917in}{1.119868in}}%
\pgfpathlineto{\pgfqpoint{3.779913in}{1.125055in}}%
\pgfpathlineto{\pgfqpoint{3.775411in}{1.130764in}}%
\pgfpathlineto{\pgfqpoint{3.771427in}{1.136972in}}%
\pgfpathlineto{\pgfqpoint{3.767977in}{1.143653in}}%
\pgfpathlineto{\pgfqpoint{3.758529in}{1.148299in}}%
\pgfpathlineto{\pgfqpoint{3.748493in}{1.152637in}}%
\pgfpathlineto{\pgfqpoint{3.737910in}{1.156649in}}%
\pgfpathlineto{\pgfqpoint{3.726823in}{1.160317in}}%
\pgfpathlineto{\pgfqpoint{3.715278in}{1.163626in}}%
\pgfpathclose%
\pgfusepath{fill}%
\end{pgfscope}%
\begin{pgfscope}%
\pgfpathrectangle{\pgfqpoint{2.548318in}{0.050000in}}{\pgfqpoint{2.081932in}{2.081932in}}%
\pgfusepath{clip}%
\pgfsetbuttcap%
\pgfsetroundjoin%
\definecolor{currentfill}{rgb}{0.227802,0.326594,0.546532}%
\pgfsetfillcolor{currentfill}%
\pgfsetlinewidth{0.000000pt}%
\definecolor{currentstroke}{rgb}{0.000000,0.000000,0.000000}%
\pgfsetstrokecolor{currentstroke}%
\pgfsetdash{}{0pt}%
\pgfpathmoveto{\pgfqpoint{3.469357in}{1.144851in}}%
\pgfpathlineto{\pgfqpoint{3.465958in}{1.138197in}}%
\pgfpathlineto{\pgfqpoint{3.462034in}{1.132019in}}%
\pgfpathlineto{\pgfqpoint{3.457600in}{1.126345in}}%
\pgfpathlineto{\pgfqpoint{3.452671in}{1.121197in}}%
\pgfpathlineto{\pgfqpoint{3.447266in}{1.116598in}}%
\pgfpathlineto{\pgfqpoint{3.458278in}{1.121827in}}%
\pgfpathlineto{\pgfqpoint{3.469957in}{1.126700in}}%
\pgfpathlineto{\pgfqpoint{3.482257in}{1.131195in}}%
\pgfpathlineto{\pgfqpoint{3.495127in}{1.135293in}}%
\pgfpathlineto{\pgfqpoint{3.508512in}{1.138975in}}%
\pgfpathlineto{\pgfqpoint{3.511984in}{1.142877in}}%
\pgfpathlineto{\pgfqpoint{3.515150in}{1.147391in}}%
\pgfpathlineto{\pgfqpoint{3.517998in}{1.152498in}}%
\pgfpathlineto{\pgfqpoint{3.520517in}{1.158176in}}%
\pgfpathlineto{\pgfqpoint{3.522699in}{1.164402in}}%
\pgfpathlineto{\pgfqpoint{3.511047in}{1.161186in}}%
\pgfpathlineto{\pgfqpoint{3.499842in}{1.157606in}}%
\pgfpathlineto{\pgfqpoint{3.489130in}{1.153679in}}%
\pgfpathlineto{\pgfqpoint{3.478954in}{1.149422in}}%
\pgfpathlineto{\pgfqpoint{3.469357in}{1.144851in}}%
\pgfpathclose%
\pgfusepath{fill}%
\end{pgfscope}%
\begin{pgfscope}%
\pgfpathrectangle{\pgfqpoint{2.548318in}{0.050000in}}{\pgfqpoint{2.081932in}{2.081932in}}%
\pgfusepath{clip}%
\pgfsetbuttcap%
\pgfsetroundjoin%
\definecolor{currentfill}{rgb}{0.150476,0.504369,0.557430}%
\pgfsetfillcolor{currentfill}%
\pgfsetlinewidth{0.000000pt}%
\definecolor{currentstroke}{rgb}{0.000000,0.000000,0.000000}%
\pgfsetstrokecolor{currentstroke}%
\pgfsetdash{}{0pt}%
\pgfpathmoveto{\pgfqpoint{3.045833in}{1.145882in}}%
\pgfpathlineto{\pgfqpoint{3.043371in}{1.154876in}}%
\pgfpathlineto{\pgfqpoint{3.041682in}{1.163887in}}%
\pgfpathlineto{\pgfqpoint{3.040773in}{1.172877in}}%
\pgfpathlineto{\pgfqpoint{3.040646in}{1.181811in}}%
\pgfpathlineto{\pgfqpoint{3.048173in}{1.200205in}}%
\pgfpathlineto{\pgfqpoint{3.058069in}{1.218369in}}%
\pgfpathlineto{\pgfqpoint{3.070312in}{1.236226in}}%
\pgfpathlineto{\pgfqpoint{3.084871in}{1.253697in}}%
\pgfpathlineto{\pgfqpoint{3.101700in}{1.270704in}}%
\pgfpathlineto{\pgfqpoint{3.101815in}{1.261630in}}%
\pgfpathlineto{\pgfqpoint{3.102639in}{1.252376in}}%
\pgfpathlineto{\pgfqpoint{3.104169in}{1.242981in}}%
\pgfpathlineto{\pgfqpoint{3.106400in}{1.233482in}}%
\pgfpathlineto{\pgfqpoint{3.089712in}{1.216724in}}%
\pgfpathlineto{\pgfqpoint{3.075273in}{1.199508in}}%
\pgfpathlineto{\pgfqpoint{3.063127in}{1.181911in}}%
\pgfpathlineto{\pgfqpoint{3.053306in}{1.164010in}}%
\pgfpathlineto{\pgfqpoint{3.045833in}{1.145882in}}%
\pgfpathclose%
\pgfusepath{fill}%
\end{pgfscope}%
\begin{pgfscope}%
\pgfpathrectangle{\pgfqpoint{2.548318in}{0.050000in}}{\pgfqpoint{2.081932in}{2.081932in}}%
\pgfusepath{clip}%
\pgfsetbuttcap%
\pgfsetroundjoin%
\definecolor{currentfill}{rgb}{0.267004,0.004874,0.329415}%
\pgfsetfillcolor{currentfill}%
\pgfsetlinewidth{0.000000pt}%
\definecolor{currentstroke}{rgb}{0.000000,0.000000,0.000000}%
\pgfsetstrokecolor{currentstroke}%
\pgfsetdash{}{0pt}%
\pgfpathmoveto{\pgfqpoint{3.283219in}{1.019540in}}%
\pgfpathlineto{\pgfqpoint{3.271221in}{1.018693in}}%
\pgfpathlineto{\pgfqpoint{3.259065in}{1.018423in}}%
\pgfpathlineto{\pgfqpoint{3.246801in}{1.018729in}}%
\pgfpathlineto{\pgfqpoint{3.234481in}{1.019612in}}%
\pgfpathlineto{\pgfqpoint{3.222154in}{1.021067in}}%
\pgfpathlineto{\pgfqpoint{3.227656in}{1.033354in}}%
\pgfpathlineto{\pgfqpoint{3.234766in}{1.045466in}}%
\pgfpathlineto{\pgfqpoint{3.243462in}{1.057352in}}%
\pgfpathlineto{\pgfqpoint{3.253718in}{1.068963in}}%
\pgfpathlineto{\pgfqpoint{3.265499in}{1.080246in}}%
\pgfpathlineto{\pgfqpoint{3.276572in}{1.076916in}}%
\pgfpathlineto{\pgfqpoint{3.287634in}{1.074167in}}%
\pgfpathlineto{\pgfqpoint{3.298638in}{1.072008in}}%
\pgfpathlineto{\pgfqpoint{3.309540in}{1.070449in}}%
\pgfpathlineto{\pgfqpoint{3.320294in}{1.069495in}}%
\pgfpathlineto{\pgfqpoint{3.310264in}{1.059981in}}%
\pgfpathlineto{\pgfqpoint{3.301513in}{1.050187in}}%
\pgfpathlineto{\pgfqpoint{3.294074in}{1.040153in}}%
\pgfpathlineto{\pgfqpoint{3.287970in}{1.029923in}}%
\pgfpathlineto{\pgfqpoint{3.283219in}{1.019540in}}%
\pgfpathclose%
\pgfusepath{fill}%
\end{pgfscope}%
\begin{pgfscope}%
\pgfpathrectangle{\pgfqpoint{2.548318in}{0.050000in}}{\pgfqpoint{2.081932in}{2.081932in}}%
\pgfusepath{clip}%
\pgfsetbuttcap%
\pgfsetroundjoin%
\definecolor{currentfill}{rgb}{0.278012,0.180367,0.486697}%
\pgfsetfillcolor{currentfill}%
\pgfsetlinewidth{0.000000pt}%
\definecolor{currentstroke}{rgb}{0.000000,0.000000,0.000000}%
\pgfsetstrokecolor{currentstroke}%
\pgfsetdash{}{0pt}%
\pgfpathmoveto{\pgfqpoint{3.410735in}{1.092590in}}%
\pgfpathlineto{\pgfqpoint{3.403644in}{1.087750in}}%
\pgfpathlineto{\pgfqpoint{3.396032in}{1.083434in}}%
\pgfpathlineto{\pgfqpoint{3.387926in}{1.079660in}}%
\pgfpathlineto{\pgfqpoint{3.379360in}{1.076444in}}%
\pgfpathlineto{\pgfqpoint{3.370367in}{1.073800in}}%
\pgfpathlineto{\pgfqpoint{3.379799in}{1.081446in}}%
\pgfpathlineto{\pgfqpoint{3.390223in}{1.088788in}}%
\pgfpathlineto{\pgfqpoint{3.401599in}{1.095793in}}%
\pgfpathlineto{\pgfqpoint{3.413882in}{1.102431in}}%
\pgfpathlineto{\pgfqpoint{3.421323in}{1.104040in}}%
\pgfpathlineto{\pgfqpoint{3.428409in}{1.106273in}}%
\pgfpathlineto{\pgfqpoint{3.435112in}{1.109119in}}%
\pgfpathlineto{\pgfqpoint{3.441406in}{1.112566in}}%
\pgfpathlineto{\pgfqpoint{3.447266in}{1.116598in}}%
\pgfpathlineto{\pgfqpoint{3.436966in}{1.111034in}}%
\pgfpathlineto{\pgfqpoint{3.427420in}{1.105162in}}%
\pgfpathlineto{\pgfqpoint{3.418665in}{1.099005in}}%
\pgfpathlineto{\pgfqpoint{3.410735in}{1.092590in}}%
\pgfpathclose%
\pgfusepath{fill}%
\end{pgfscope}%
\begin{pgfscope}%
\pgfpathrectangle{\pgfqpoint{2.548318in}{0.050000in}}{\pgfqpoint{2.081932in}{2.081932in}}%
\pgfusepath{clip}%
\pgfsetbuttcap%
\pgfsetroundjoin%
\definecolor{currentfill}{rgb}{0.636902,0.856542,0.216620}%
\pgfsetfillcolor{currentfill}%
\pgfsetlinewidth{0.000000pt}%
\definecolor{currentstroke}{rgb}{0.000000,0.000000,0.000000}%
\pgfsetstrokecolor{currentstroke}%
\pgfsetdash{}{0pt}%
\pgfpathmoveto{\pgfqpoint{4.020289in}{1.398805in}}%
\pgfpathlineto{\pgfqpoint{4.014345in}{1.402978in}}%
\pgfpathlineto{\pgfqpoint{4.007963in}{1.406556in}}%
\pgfpathlineto{\pgfqpoint{4.001170in}{1.409527in}}%
\pgfpathlineto{\pgfqpoint{3.993994in}{1.411881in}}%
\pgfpathlineto{\pgfqpoint{3.986465in}{1.413610in}}%
\pgfpathlineto{\pgfqpoint{4.007669in}{1.401025in}}%
\pgfpathlineto{\pgfqpoint{4.027185in}{1.387791in}}%
\pgfpathlineto{\pgfqpoint{4.044937in}{1.373969in}}%
\pgfpathlineto{\pgfqpoint{4.060862in}{1.359620in}}%
\pgfpathlineto{\pgfqpoint{4.074902in}{1.344809in}}%
\pgfpathlineto{\pgfqpoint{4.084139in}{1.341709in}}%
\pgfpathlineto{\pgfqpoint{4.092940in}{1.338057in}}%
\pgfpathlineto{\pgfqpoint{4.101268in}{1.333865in}}%
\pgfpathlineto{\pgfqpoint{4.109089in}{1.329151in}}%
\pgfpathlineto{\pgfqpoint{4.116370in}{1.323931in}}%
\pgfpathlineto{\pgfqpoint{4.101153in}{1.340037in}}%
\pgfpathlineto{\pgfqpoint{4.083872in}{1.355647in}}%
\pgfpathlineto{\pgfqpoint{4.064586in}{1.370691in}}%
\pgfpathlineto{\pgfqpoint{4.043364in}{1.385100in}}%
\pgfpathlineto{\pgfqpoint{4.020289in}{1.398805in}}%
\pgfpathclose%
\pgfusepath{fill}%
\end{pgfscope}%
\begin{pgfscope}%
\pgfpathrectangle{\pgfqpoint{2.548318in}{0.050000in}}{\pgfqpoint{2.081932in}{2.081932in}}%
\pgfusepath{clip}%
\pgfsetbuttcap%
\pgfsetroundjoin%
\definecolor{currentfill}{rgb}{0.993248,0.906157,0.143936}%
\pgfsetfillcolor{currentfill}%
\pgfsetlinewidth{0.000000pt}%
\definecolor{currentstroke}{rgb}{0.000000,0.000000,0.000000}%
\pgfsetstrokecolor{currentstroke}%
\pgfsetdash{}{0pt}%
\pgfpathmoveto{\pgfqpoint{3.830839in}{1.458153in}}%
\pgfpathlineto{\pgfqpoint{3.824960in}{1.454840in}}%
\pgfpathlineto{\pgfqpoint{3.819009in}{1.450898in}}%
\pgfpathlineto{\pgfqpoint{3.813007in}{1.446344in}}%
\pgfpathlineto{\pgfqpoint{3.806979in}{1.441198in}}%
\pgfpathlineto{\pgfqpoint{3.800949in}{1.435482in}}%
\pgfpathlineto{\pgfqpoint{3.822522in}{1.429018in}}%
\pgfpathlineto{\pgfqpoint{3.843205in}{1.421858in}}%
\pgfpathlineto{\pgfqpoint{3.862913in}{1.414036in}}%
\pgfpathlineto{\pgfqpoint{3.881563in}{1.405587in}}%
\pgfpathlineto{\pgfqpoint{3.899080in}{1.396550in}}%
\pgfpathlineto{\pgfqpoint{3.908280in}{1.400961in}}%
\pgfpathlineto{\pgfqpoint{3.917475in}{1.404803in}}%
\pgfpathlineto{\pgfqpoint{3.926626in}{1.408061in}}%
\pgfpathlineto{\pgfqpoint{3.935698in}{1.410721in}}%
\pgfpathlineto{\pgfqpoint{3.944654in}{1.412770in}}%
\pgfpathlineto{\pgfqpoint{3.924369in}{1.423297in}}%
\pgfpathlineto{\pgfqpoint{3.902752in}{1.433143in}}%
\pgfpathlineto{\pgfqpoint{3.879894in}{1.442262in}}%
\pgfpathlineto{\pgfqpoint{3.855889in}{1.450612in}}%
\pgfpathlineto{\pgfqpoint{3.830839in}{1.458153in}}%
\pgfpathclose%
\pgfusepath{fill}%
\end{pgfscope}%
\begin{pgfscope}%
\pgfpathrectangle{\pgfqpoint{2.548318in}{0.050000in}}{\pgfqpoint{2.081932in}{2.081932in}}%
\pgfusepath{clip}%
\pgfsetbuttcap%
\pgfsetroundjoin%
\definecolor{currentfill}{rgb}{0.206756,0.371758,0.553117}%
\pgfsetfillcolor{currentfill}%
\pgfsetlinewidth{0.000000pt}%
\definecolor{currentstroke}{rgb}{0.000000,0.000000,0.000000}%
\pgfsetstrokecolor{currentstroke}%
\pgfsetdash{}{0pt}%
\pgfpathmoveto{\pgfqpoint{3.069315in}{1.102434in}}%
\pgfpathlineto{\pgfqpoint{3.063186in}{1.110800in}}%
\pgfpathlineto{\pgfqpoint{3.057753in}{1.119364in}}%
\pgfpathlineto{\pgfqpoint{3.053038in}{1.128089in}}%
\pgfpathlineto{\pgfqpoint{3.049059in}{1.136940in}}%
\pgfpathlineto{\pgfqpoint{3.045833in}{1.145882in}}%
\pgfpathlineto{\pgfqpoint{3.053306in}{1.164010in}}%
\pgfpathlineto{\pgfqpoint{3.063127in}{1.181911in}}%
\pgfpathlineto{\pgfqpoint{3.075273in}{1.199508in}}%
\pgfpathlineto{\pgfqpoint{3.089712in}{1.216724in}}%
\pgfpathlineto{\pgfqpoint{3.106400in}{1.233482in}}%
\pgfpathlineto{\pgfqpoint{3.109323in}{1.223921in}}%
\pgfpathlineto{\pgfqpoint{3.112927in}{1.214335in}}%
\pgfpathlineto{\pgfqpoint{3.117197in}{1.204765in}}%
\pgfpathlineto{\pgfqpoint{3.122116in}{1.195251in}}%
\pgfpathlineto{\pgfqpoint{3.127664in}{1.185832in}}%
\pgfpathlineto{\pgfqpoint{3.111617in}{1.169885in}}%
\pgfpathlineto{\pgfqpoint{3.097721in}{1.153499in}}%
\pgfpathlineto{\pgfqpoint{3.086019in}{1.136746in}}%
\pgfpathlineto{\pgfqpoint{3.076543in}{1.119700in}}%
\pgfpathlineto{\pgfqpoint{3.069315in}{1.102434in}}%
\pgfpathclose%
\pgfusepath{fill}%
\end{pgfscope}%
\begin{pgfscope}%
\pgfpathrectangle{\pgfqpoint{2.548318in}{0.050000in}}{\pgfqpoint{2.081932in}{2.081932in}}%
\pgfusepath{clip}%
\pgfsetbuttcap%
\pgfsetroundjoin%
\definecolor{currentfill}{rgb}{0.993248,0.906157,0.143936}%
\pgfsetfillcolor{currentfill}%
\pgfsetlinewidth{0.000000pt}%
\definecolor{currentstroke}{rgb}{0.000000,0.000000,0.000000}%
\pgfsetstrokecolor{currentstroke}%
\pgfsetdash{}{0pt}%
\pgfpathmoveto{\pgfqpoint{3.683086in}{1.456427in}}%
\pgfpathlineto{\pgfqpoint{3.680932in}{1.449458in}}%
\pgfpathlineto{\pgfqpoint{3.678795in}{1.441974in}}%
\pgfpathlineto{\pgfqpoint{3.676682in}{1.434005in}}%
\pgfpathlineto{\pgfqpoint{3.674603in}{1.425585in}}%
\pgfpathlineto{\pgfqpoint{3.672565in}{1.416746in}}%
\pgfpathlineto{\pgfqpoint{3.693198in}{1.414575in}}%
\pgfpathlineto{\pgfqpoint{3.713503in}{1.411726in}}%
\pgfpathlineto{\pgfqpoint{3.733394in}{1.408211in}}%
\pgfpathlineto{\pgfqpoint{3.752784in}{1.404048in}}%
\pgfpathlineto{\pgfqpoint{3.771592in}{1.399255in}}%
\pgfpathlineto{\pgfqpoint{3.777281in}{1.407424in}}%
\pgfpathlineto{\pgfqpoint{3.783084in}{1.415162in}}%
\pgfpathlineto{\pgfqpoint{3.788979in}{1.422437in}}%
\pgfpathlineto{\pgfqpoint{3.794941in}{1.429219in}}%
\pgfpathlineto{\pgfqpoint{3.800949in}{1.435482in}}%
\pgfpathlineto{\pgfqpoint{3.778579in}{1.441220in}}%
\pgfpathlineto{\pgfqpoint{3.755507in}{1.446205in}}%
\pgfpathlineto{\pgfqpoint{3.731832in}{1.450414in}}%
\pgfpathlineto{\pgfqpoint{3.707656in}{1.453826in}}%
\pgfpathlineto{\pgfqpoint{3.683086in}{1.456427in}}%
\pgfpathclose%
\pgfusepath{fill}%
\end{pgfscope}%
\begin{pgfscope}%
\pgfpathrectangle{\pgfqpoint{2.548318in}{0.050000in}}{\pgfqpoint{2.081932in}{2.081932in}}%
\pgfusepath{clip}%
\pgfsetbuttcap%
\pgfsetroundjoin%
\definecolor{currentfill}{rgb}{0.993248,0.906157,0.143936}%
\pgfsetfillcolor{currentfill}%
\pgfsetlinewidth{0.000000pt}%
\definecolor{currentstroke}{rgb}{0.000000,0.000000,0.000000}%
\pgfsetstrokecolor{currentstroke}%
\pgfsetdash{}{0pt}%
\pgfpathmoveto{\pgfqpoint{3.295295in}{1.415484in}}%
\pgfpathlineto{\pgfqpoint{3.304120in}{1.413360in}}%
\pgfpathlineto{\pgfqpoint{3.313059in}{1.410624in}}%
\pgfpathlineto{\pgfqpoint{3.322077in}{1.407289in}}%
\pgfpathlineto{\pgfqpoint{3.331136in}{1.403369in}}%
\pgfpathlineto{\pgfqpoint{3.340202in}{1.398881in}}%
\pgfpathlineto{\pgfqpoint{3.358007in}{1.407773in}}%
\pgfpathlineto{\pgfqpoint{3.376927in}{1.416068in}}%
\pgfpathlineto{\pgfqpoint{3.396885in}{1.423726in}}%
\pgfpathlineto{\pgfqpoint{3.417797in}{1.430713in}}%
\pgfpathlineto{\pgfqpoint{3.439576in}{1.436997in}}%
\pgfpathlineto{\pgfqpoint{3.433725in}{1.442764in}}%
\pgfpathlineto{\pgfqpoint{3.427877in}{1.447961in}}%
\pgfpathlineto{\pgfqpoint{3.422053in}{1.452566in}}%
\pgfpathlineto{\pgfqpoint{3.416277in}{1.456559in}}%
\pgfpathlineto{\pgfqpoint{3.410573in}{1.459921in}}%
\pgfpathlineto{\pgfqpoint{3.385279in}{1.452590in}}%
\pgfpathlineto{\pgfqpoint{3.361004in}{1.444441in}}%
\pgfpathlineto{\pgfqpoint{3.337852in}{1.435511in}}%
\pgfpathlineto{\pgfqpoint{3.315919in}{1.425844in}}%
\pgfpathlineto{\pgfqpoint{3.295295in}{1.415484in}}%
\pgfpathclose%
\pgfusepath{fill}%
\end{pgfscope}%
\begin{pgfscope}%
\pgfpathrectangle{\pgfqpoint{2.548318in}{0.050000in}}{\pgfqpoint{2.081932in}{2.081932in}}%
\pgfusepath{clip}%
\pgfsetbuttcap%
\pgfsetroundjoin%
\definecolor{currentfill}{rgb}{0.268510,0.009605,0.335427}%
\pgfsetfillcolor{currentfill}%
\pgfsetlinewidth{0.000000pt}%
\definecolor{currentstroke}{rgb}{0.000000,0.000000,0.000000}%
\pgfsetstrokecolor{currentstroke}%
\pgfsetdash{}{0pt}%
\pgfpathmoveto{\pgfqpoint{3.222154in}{1.021067in}}%
\pgfpathlineto{\pgfqpoint{3.209871in}{1.023089in}}%
\pgfpathlineto{\pgfqpoint{3.197683in}{1.025671in}}%
\pgfpathlineto{\pgfqpoint{3.185640in}{1.028802in}}%
\pgfpathlineto{\pgfqpoint{3.173792in}{1.032469in}}%
\pgfpathlineto{\pgfqpoint{3.162188in}{1.036656in}}%
\pgfpathlineto{\pgfqpoint{3.168394in}{1.050845in}}%
\pgfpathlineto{\pgfqpoint{3.176458in}{1.064841in}}%
\pgfpathlineto{\pgfqpoint{3.186357in}{1.078583in}}%
\pgfpathlineto{\pgfqpoint{3.198063in}{1.092014in}}%
\pgfpathlineto{\pgfqpoint{3.211537in}{1.105073in}}%
\pgfpathlineto{\pgfqpoint{3.221992in}{1.099080in}}%
\pgfpathlineto{\pgfqpoint{3.232659in}{1.093580in}}%
\pgfpathlineto{\pgfqpoint{3.243497in}{1.088594in}}%
\pgfpathlineto{\pgfqpoint{3.254458in}{1.084144in}}%
\pgfpathlineto{\pgfqpoint{3.265499in}{1.080246in}}%
\pgfpathlineto{\pgfqpoint{3.253718in}{1.068963in}}%
\pgfpathlineto{\pgfqpoint{3.243462in}{1.057352in}}%
\pgfpathlineto{\pgfqpoint{3.234766in}{1.045466in}}%
\pgfpathlineto{\pgfqpoint{3.227656in}{1.033354in}}%
\pgfpathlineto{\pgfqpoint{3.222154in}{1.021067in}}%
\pgfpathclose%
\pgfusepath{fill}%
\end{pgfscope}%
\begin{pgfscope}%
\pgfpathrectangle{\pgfqpoint{2.548318in}{0.050000in}}{\pgfqpoint{2.081932in}{2.081932in}}%
\pgfusepath{clip}%
\pgfsetbuttcap%
\pgfsetroundjoin%
\definecolor{currentfill}{rgb}{0.993248,0.906157,0.143936}%
\pgfsetfillcolor{currentfill}%
\pgfsetlinewidth{0.000000pt}%
\definecolor{currentstroke}{rgb}{0.000000,0.000000,0.000000}%
\pgfsetstrokecolor{currentstroke}%
\pgfsetdash{}{0pt}%
\pgfpathmoveto{\pgfqpoint{3.439576in}{1.436997in}}%
\pgfpathlineto{\pgfqpoint{3.445406in}{1.430684in}}%
\pgfpathlineto{\pgfqpoint{3.451192in}{1.423851in}}%
\pgfpathlineto{\pgfqpoint{3.456911in}{1.416526in}}%
\pgfpathlineto{\pgfqpoint{3.462541in}{1.408738in}}%
\pgfpathlineto{\pgfqpoint{3.468061in}{1.400521in}}%
\pgfpathlineto{\pgfqpoint{3.487022in}{1.405157in}}%
\pgfpathlineto{\pgfqpoint{3.506544in}{1.409159in}}%
\pgfpathlineto{\pgfqpoint{3.526546in}{1.412507in}}%
\pgfpathlineto{\pgfqpoint{3.546941in}{1.415186in}}%
\pgfpathlineto{\pgfqpoint{3.567641in}{1.417184in}}%
\pgfpathlineto{\pgfqpoint{3.565794in}{1.426040in}}%
\pgfpathlineto{\pgfqpoint{3.563910in}{1.434478in}}%
\pgfpathlineto{\pgfqpoint{3.561995in}{1.442464in}}%
\pgfpathlineto{\pgfqpoint{3.560058in}{1.449966in}}%
\pgfpathlineto{\pgfqpoint{3.558106in}{1.456953in}}%
\pgfpathlineto{\pgfqpoint{3.533454in}{1.454559in}}%
\pgfpathlineto{\pgfqpoint{3.509171in}{1.451349in}}%
\pgfpathlineto{\pgfqpoint{3.485361in}{1.447339in}}%
\pgfpathlineto{\pgfqpoint{3.462130in}{1.442548in}}%
\pgfpathlineto{\pgfqpoint{3.439576in}{1.436997in}}%
\pgfpathclose%
\pgfusepath{fill}%
\end{pgfscope}%
\begin{pgfscope}%
\pgfpathrectangle{\pgfqpoint{2.548318in}{0.050000in}}{\pgfqpoint{2.081932in}{2.081932in}}%
\pgfusepath{clip}%
\pgfsetbuttcap%
\pgfsetroundjoin%
\definecolor{currentfill}{rgb}{0.227802,0.326594,0.546532}%
\pgfsetfillcolor{currentfill}%
\pgfsetlinewidth{0.000000pt}%
\definecolor{currentstroke}{rgb}{0.000000,0.000000,0.000000}%
\pgfsetstrokecolor{currentstroke}%
\pgfsetdash{}{0pt}%
\pgfpathmoveto{\pgfqpoint{3.652414in}{1.174328in}}%
\pgfpathlineto{\pgfqpoint{3.653218in}{1.168322in}}%
\pgfpathlineto{\pgfqpoint{3.654147in}{1.162899in}}%
\pgfpathlineto{\pgfqpoint{3.655196in}{1.158082in}}%
\pgfpathlineto{\pgfqpoint{3.656363in}{1.153891in}}%
\pgfpathlineto{\pgfqpoint{3.657644in}{1.150345in}}%
\pgfpathlineto{\pgfqpoint{3.672687in}{1.148825in}}%
\pgfpathlineto{\pgfqpoint{3.687497in}{1.146829in}}%
\pgfpathlineto{\pgfqpoint{3.702009in}{1.144366in}}%
\pgfpathlineto{\pgfqpoint{3.716164in}{1.141448in}}%
\pgfpathlineto{\pgfqpoint{3.729902in}{1.138087in}}%
\pgfpathlineto{\pgfqpoint{3.726323in}{1.142017in}}%
\pgfpathlineto{\pgfqpoint{3.723059in}{1.146557in}}%
\pgfpathlineto{\pgfqpoint{3.720124in}{1.151686in}}%
\pgfpathlineto{\pgfqpoint{3.717527in}{1.157384in}}%
\pgfpathlineto{\pgfqpoint{3.715278in}{1.163626in}}%
\pgfpathlineto{\pgfqpoint{3.703323in}{1.166561in}}%
\pgfpathlineto{\pgfqpoint{3.691006in}{1.169109in}}%
\pgfpathlineto{\pgfqpoint{3.678381in}{1.171259in}}%
\pgfpathlineto{\pgfqpoint{3.665498in}{1.173001in}}%
\pgfpathlineto{\pgfqpoint{3.652414in}{1.174328in}}%
\pgfpathclose%
\pgfusepath{fill}%
\end{pgfscope}%
\begin{pgfscope}%
\pgfpathrectangle{\pgfqpoint{2.548318in}{0.050000in}}{\pgfqpoint{2.081932in}{2.081932in}}%
\pgfusepath{clip}%
\pgfsetbuttcap%
\pgfsetroundjoin%
\definecolor{currentfill}{rgb}{0.227802,0.326594,0.546532}%
\pgfsetfillcolor{currentfill}%
\pgfsetlinewidth{0.000000pt}%
\definecolor{currentstroke}{rgb}{0.000000,0.000000,0.000000}%
\pgfsetstrokecolor{currentstroke}%
\pgfsetdash{}{0pt}%
\pgfpathmoveto{\pgfqpoint{3.522699in}{1.164402in}}%
\pgfpathlineto{\pgfqpoint{3.520517in}{1.158176in}}%
\pgfpathlineto{\pgfqpoint{3.517998in}{1.152498in}}%
\pgfpathlineto{\pgfqpoint{3.515150in}{1.147391in}}%
\pgfpathlineto{\pgfqpoint{3.511984in}{1.142877in}}%
\pgfpathlineto{\pgfqpoint{3.508512in}{1.138975in}}%
\pgfpathlineto{\pgfqpoint{3.522359in}{1.142226in}}%
\pgfpathlineto{\pgfqpoint{3.536608in}{1.145030in}}%
\pgfpathlineto{\pgfqpoint{3.551200in}{1.147376in}}%
\pgfpathlineto{\pgfqpoint{3.566074in}{1.149253in}}%
\pgfpathlineto{\pgfqpoint{3.581166in}{1.150652in}}%
\pgfpathlineto{\pgfqpoint{3.582326in}{1.154189in}}%
\pgfpathlineto{\pgfqpoint{3.583383in}{1.158370in}}%
\pgfpathlineto{\pgfqpoint{3.584335in}{1.163180in}}%
\pgfpathlineto{\pgfqpoint{3.585176in}{1.168596in}}%
\pgfpathlineto{\pgfqpoint{3.585905in}{1.174596in}}%
\pgfpathlineto{\pgfqpoint{3.572779in}{1.173375in}}%
\pgfpathlineto{\pgfqpoint{3.559841in}{1.171736in}}%
\pgfpathlineto{\pgfqpoint{3.547147in}{1.169689in}}%
\pgfpathlineto{\pgfqpoint{3.534749in}{1.167240in}}%
\pgfpathlineto{\pgfqpoint{3.522699in}{1.164402in}}%
\pgfpathclose%
\pgfusepath{fill}%
\end{pgfscope}%
\begin{pgfscope}%
\pgfpathrectangle{\pgfqpoint{2.548318in}{0.050000in}}{\pgfqpoint{2.081932in}{2.081932in}}%
\pgfusepath{clip}%
\pgfsetbuttcap%
\pgfsetroundjoin%
\definecolor{currentfill}{rgb}{0.636902,0.856542,0.216620}%
\pgfsetfillcolor{currentfill}%
\pgfsetlinewidth{0.000000pt}%
\definecolor{currentstroke}{rgb}{0.000000,0.000000,0.000000}%
\pgfsetstrokecolor{currentstroke}%
\pgfsetdash{}{0pt}%
\pgfpathmoveto{\pgfqpoint{3.137986in}{1.344020in}}%
\pgfpathlineto{\pgfqpoint{3.145006in}{1.348961in}}%
\pgfpathlineto{\pgfqpoint{3.152546in}{1.353373in}}%
\pgfpathlineto{\pgfqpoint{3.160573in}{1.357240in}}%
\pgfpathlineto{\pgfqpoint{3.169056in}{1.360547in}}%
\pgfpathlineto{\pgfqpoint{3.177958in}{1.363281in}}%
\pgfpathlineto{\pgfqpoint{3.194342in}{1.377504in}}%
\pgfpathlineto{\pgfqpoint{3.212539in}{1.391184in}}%
\pgfpathlineto{\pgfqpoint{3.232482in}{1.404260in}}%
\pgfpathlineto{\pgfqpoint{3.254093in}{1.416671in}}%
\pgfpathlineto{\pgfqpoint{3.246673in}{1.415005in}}%
\pgfpathlineto{\pgfqpoint{3.239601in}{1.412709in}}%
\pgfpathlineto{\pgfqpoint{3.232906in}{1.409793in}}%
\pgfpathlineto{\pgfqpoint{3.226616in}{1.406265in}}%
\pgfpathlineto{\pgfqpoint{3.220759in}{1.402141in}}%
\pgfpathlineto{\pgfqpoint{3.197236in}{1.388622in}}%
\pgfpathlineto{\pgfqpoint{3.175546in}{1.374385in}}%
\pgfpathlineto{\pgfqpoint{3.155772in}{1.359494in}}%
\pgfpathlineto{\pgfqpoint{3.137986in}{1.344020in}}%
\pgfpathclose%
\pgfusepath{fill}%
\end{pgfscope}%
\begin{pgfscope}%
\pgfpathrectangle{\pgfqpoint{2.548318in}{0.050000in}}{\pgfqpoint{2.081932in}{2.081932in}}%
\pgfusepath{clip}%
\pgfsetbuttcap%
\pgfsetroundjoin%
\definecolor{currentfill}{rgb}{0.267968,0.223549,0.512008}%
\pgfsetfillcolor{currentfill}%
\pgfsetlinewidth{0.000000pt}%
\definecolor{currentstroke}{rgb}{0.000000,0.000000,0.000000}%
\pgfsetstrokecolor{currentstroke}%
\pgfsetdash{}{0pt}%
\pgfpathmoveto{\pgfqpoint{3.109449in}{1.064737in}}%
\pgfpathlineto{\pgfqpoint{3.100262in}{1.071619in}}%
\pgfpathlineto{\pgfqpoint{3.091621in}{1.078861in}}%
\pgfpathlineto{\pgfqpoint{3.083561in}{1.086431in}}%
\pgfpathlineto{\pgfqpoint{3.076116in}{1.094300in}}%
\pgfpathlineto{\pgfqpoint{3.069315in}{1.102434in}}%
\pgfpathlineto{\pgfqpoint{3.076543in}{1.119700in}}%
\pgfpathlineto{\pgfqpoint{3.086019in}{1.136746in}}%
\pgfpathlineto{\pgfqpoint{3.097721in}{1.153499in}}%
\pgfpathlineto{\pgfqpoint{3.111617in}{1.169885in}}%
\pgfpathlineto{\pgfqpoint{3.127664in}{1.185832in}}%
\pgfpathlineto{\pgfqpoint{3.133819in}{1.176546in}}%
\pgfpathlineto{\pgfqpoint{3.140554in}{1.167434in}}%
\pgfpathlineto{\pgfqpoint{3.147842in}{1.158532in}}%
\pgfpathlineto{\pgfqpoint{3.155653in}{1.149878in}}%
\pgfpathlineto{\pgfqpoint{3.163954in}{1.141507in}}%
\pgfpathlineto{\pgfqpoint{3.149011in}{1.126839in}}%
\pgfpathlineto{\pgfqpoint{3.136053in}{1.111761in}}%
\pgfpathlineto{\pgfqpoint{3.125121in}{1.096340in}}%
\pgfpathlineto{\pgfqpoint{3.116246in}{1.080643in}}%
\pgfpathlineto{\pgfqpoint{3.109449in}{1.064737in}}%
\pgfpathclose%
\pgfusepath{fill}%
\end{pgfscope}%
\begin{pgfscope}%
\pgfpathrectangle{\pgfqpoint{2.548318in}{0.050000in}}{\pgfqpoint{2.081932in}{2.081932in}}%
\pgfusepath{clip}%
\pgfsetbuttcap%
\pgfsetroundjoin%
\definecolor{currentfill}{rgb}{0.282327,0.094955,0.417331}%
\pgfsetfillcolor{currentfill}%
\pgfsetlinewidth{0.000000pt}%
\definecolor{currentstroke}{rgb}{0.000000,0.000000,0.000000}%
\pgfsetstrokecolor{currentstroke}%
\pgfsetdash{}{0pt}%
\pgfpathmoveto{\pgfqpoint{3.162188in}{1.036656in}}%
\pgfpathlineto{\pgfqpoint{3.150875in}{1.041348in}}%
\pgfpathlineto{\pgfqpoint{3.139900in}{1.046524in}}%
\pgfpathlineto{\pgfqpoint{3.129309in}{1.052164in}}%
\pgfpathlineto{\pgfqpoint{3.119145in}{1.058243in}}%
\pgfpathlineto{\pgfqpoint{3.109449in}{1.064737in}}%
\pgfpathlineto{\pgfqpoint{3.116246in}{1.080643in}}%
\pgfpathlineto{\pgfqpoint{3.125121in}{1.096340in}}%
\pgfpathlineto{\pgfqpoint{3.136053in}{1.111761in}}%
\pgfpathlineto{\pgfqpoint{3.149011in}{1.126839in}}%
\pgfpathlineto{\pgfqpoint{3.163954in}{1.141507in}}%
\pgfpathlineto{\pgfqpoint{3.172711in}{1.133455in}}%
\pgfpathlineto{\pgfqpoint{3.181886in}{1.125755in}}%
\pgfpathlineto{\pgfqpoint{3.191443in}{1.118438in}}%
\pgfpathlineto{\pgfqpoint{3.201340in}{1.111534in}}%
\pgfpathlineto{\pgfqpoint{3.211537in}{1.105073in}}%
\pgfpathlineto{\pgfqpoint{3.198063in}{1.092014in}}%
\pgfpathlineto{\pgfqpoint{3.186357in}{1.078583in}}%
\pgfpathlineto{\pgfqpoint{3.176458in}{1.064841in}}%
\pgfpathlineto{\pgfqpoint{3.168394in}{1.050845in}}%
\pgfpathlineto{\pgfqpoint{3.162188in}{1.036656in}}%
\pgfpathclose%
\pgfusepath{fill}%
\end{pgfscope}%
\begin{pgfscope}%
\pgfpathrectangle{\pgfqpoint{2.548318in}{0.050000in}}{\pgfqpoint{2.081932in}{2.081932in}}%
\pgfusepath{clip}%
\pgfsetbuttcap%
\pgfsetroundjoin%
\definecolor{currentfill}{rgb}{0.278791,0.062145,0.386592}%
\pgfsetfillcolor{currentfill}%
\pgfsetlinewidth{0.000000pt}%
\definecolor{currentstroke}{rgb}{0.000000,0.000000,0.000000}%
\pgfsetstrokecolor{currentstroke}%
\pgfsetdash{}{0pt}%
\pgfpathmoveto{\pgfqpoint{3.824295in}{1.100795in}}%
\pgfpathlineto{\pgfqpoint{3.832181in}{1.099755in}}%
\pgfpathlineto{\pgfqpoint{3.840367in}{1.099351in}}%
\pgfpathlineto{\pgfqpoint{3.848821in}{1.099584in}}%
\pgfpathlineto{\pgfqpoint{3.857510in}{1.100456in}}%
\pgfpathlineto{\pgfqpoint{3.866398in}{1.101965in}}%
\pgfpathlineto{\pgfqpoint{3.880860in}{1.093861in}}%
\pgfpathlineto{\pgfqpoint{3.894208in}{1.085327in}}%
\pgfpathlineto{\pgfqpoint{3.906390in}{1.076403in}}%
\pgfpathlineto{\pgfqpoint{3.917358in}{1.067126in}}%
\pgfpathlineto{\pgfqpoint{3.927074in}{1.057538in}}%
\pgfpathlineto{\pgfqpoint{3.916092in}{1.057615in}}%
\pgfpathlineto{\pgfqpoint{3.905352in}{1.058290in}}%
\pgfpathlineto{\pgfqpoint{3.894896in}{1.059558in}}%
\pgfpathlineto{\pgfqpoint{3.884767in}{1.061413in}}%
\pgfpathlineto{\pgfqpoint{3.875006in}{1.063847in}}%
\pgfpathlineto{\pgfqpoint{3.866862in}{1.071829in}}%
\pgfpathlineto{\pgfqpoint{3.857682in}{1.079547in}}%
\pgfpathlineto{\pgfqpoint{3.847501in}{1.086968in}}%
\pgfpathlineto{\pgfqpoint{3.836357in}{1.094061in}}%
\pgfpathlineto{\pgfqpoint{3.824295in}{1.100795in}}%
\pgfpathclose%
\pgfusepath{fill}%
\end{pgfscope}%
\begin{pgfscope}%
\pgfpathrectangle{\pgfqpoint{2.548318in}{0.050000in}}{\pgfqpoint{2.081932in}{2.081932in}}%
\pgfusepath{clip}%
\pgfsetbuttcap%
\pgfsetroundjoin%
\definecolor{currentfill}{rgb}{0.993248,0.906157,0.143936}%
\pgfsetfillcolor{currentfill}%
\pgfsetlinewidth{0.000000pt}%
\definecolor{currentstroke}{rgb}{0.000000,0.000000,0.000000}%
\pgfsetstrokecolor{currentstroke}%
\pgfsetdash{}{0pt}%
\pgfpathmoveto{\pgfqpoint{3.558106in}{1.456953in}}%
\pgfpathlineto{\pgfqpoint{3.560058in}{1.449966in}}%
\pgfpathlineto{\pgfqpoint{3.561995in}{1.442464in}}%
\pgfpathlineto{\pgfqpoint{3.563910in}{1.434478in}}%
\pgfpathlineto{\pgfqpoint{3.565794in}{1.426040in}}%
\pgfpathlineto{\pgfqpoint{3.567641in}{1.417184in}}%
\pgfpathlineto{\pgfqpoint{3.588557in}{1.418492in}}%
\pgfpathlineto{\pgfqpoint{3.609599in}{1.419103in}}%
\pgfpathlineto{\pgfqpoint{3.630675in}{1.419015in}}%
\pgfpathlineto{\pgfqpoint{3.651694in}{1.418228in}}%
\pgfpathlineto{\pgfqpoint{3.672565in}{1.416746in}}%
\pgfpathlineto{\pgfqpoint{3.674603in}{1.425585in}}%
\pgfpathlineto{\pgfqpoint{3.676682in}{1.434005in}}%
\pgfpathlineto{\pgfqpoint{3.678795in}{1.441974in}}%
\pgfpathlineto{\pgfqpoint{3.680932in}{1.449458in}}%
\pgfpathlineto{\pgfqpoint{3.683086in}{1.456427in}}%
\pgfpathlineto{\pgfqpoint{3.658227in}{1.458203in}}%
\pgfpathlineto{\pgfqpoint{3.633190in}{1.459146in}}%
\pgfpathlineto{\pgfqpoint{3.608084in}{1.459252in}}%
\pgfpathlineto{\pgfqpoint{3.583019in}{1.458520in}}%
\pgfpathlineto{\pgfqpoint{3.558106in}{1.456953in}}%
\pgfpathclose%
\pgfusepath{fill}%
\end{pgfscope}%
\begin{pgfscope}%
\pgfpathrectangle{\pgfqpoint{2.548318in}{0.050000in}}{\pgfqpoint{2.081932in}{2.081932in}}%
\pgfusepath{clip}%
\pgfsetbuttcap%
\pgfsetroundjoin%
\definecolor{currentfill}{rgb}{0.227802,0.326594,0.546532}%
\pgfsetfillcolor{currentfill}%
\pgfsetlinewidth{0.000000pt}%
\definecolor{currentstroke}{rgb}{0.000000,0.000000,0.000000}%
\pgfsetstrokecolor{currentstroke}%
\pgfsetdash{}{0pt}%
\pgfpathmoveto{\pgfqpoint{3.585905in}{1.174596in}}%
\pgfpathlineto{\pgfqpoint{3.585176in}{1.168596in}}%
\pgfpathlineto{\pgfqpoint{3.584335in}{1.163180in}}%
\pgfpathlineto{\pgfqpoint{3.583383in}{1.158370in}}%
\pgfpathlineto{\pgfqpoint{3.582326in}{1.154189in}}%
\pgfpathlineto{\pgfqpoint{3.581166in}{1.150652in}}%
\pgfpathlineto{\pgfqpoint{3.596412in}{1.151568in}}%
\pgfpathlineto{\pgfqpoint{3.611749in}{1.151996in}}%
\pgfpathlineto{\pgfqpoint{3.627109in}{1.151934in}}%
\pgfpathlineto{\pgfqpoint{3.642429in}{1.151383in}}%
\pgfpathlineto{\pgfqpoint{3.657644in}{1.150345in}}%
\pgfpathlineto{\pgfqpoint{3.656363in}{1.153891in}}%
\pgfpathlineto{\pgfqpoint{3.655196in}{1.158082in}}%
\pgfpathlineto{\pgfqpoint{3.654147in}{1.162899in}}%
\pgfpathlineto{\pgfqpoint{3.653218in}{1.168322in}}%
\pgfpathlineto{\pgfqpoint{3.652414in}{1.174328in}}%
\pgfpathlineto{\pgfqpoint{3.639182in}{1.175234in}}%
\pgfpathlineto{\pgfqpoint{3.625859in}{1.175715in}}%
\pgfpathlineto{\pgfqpoint{3.612501in}{1.175769in}}%
\pgfpathlineto{\pgfqpoint{3.599164in}{1.175396in}}%
\pgfpathlineto{\pgfqpoint{3.585905in}{1.174596in}}%
\pgfpathclose%
\pgfusepath{fill}%
\end{pgfscope}%
\begin{pgfscope}%
\pgfpathrectangle{\pgfqpoint{2.548318in}{0.050000in}}{\pgfqpoint{2.081932in}{2.081932in}}%
\pgfusepath{clip}%
\pgfsetbuttcap%
\pgfsetroundjoin%
\definecolor{currentfill}{rgb}{0.278791,0.062145,0.386592}%
\pgfsetfillcolor{currentfill}%
\pgfsetlinewidth{0.000000pt}%
\definecolor{currentstroke}{rgb}{0.000000,0.000000,0.000000}%
\pgfsetstrokecolor{currentstroke}%
\pgfsetdash{}{0pt}%
\pgfpathmoveto{\pgfqpoint{3.370367in}{1.073800in}}%
\pgfpathlineto{\pgfqpoint{3.360982in}{1.071740in}}%
\pgfpathlineto{\pgfqpoint{3.351242in}{1.070273in}}%
\pgfpathlineto{\pgfqpoint{3.341188in}{1.069408in}}%
\pgfpathlineto{\pgfqpoint{3.330857in}{1.069148in}}%
\pgfpathlineto{\pgfqpoint{3.320294in}{1.069495in}}%
\pgfpathlineto{\pgfqpoint{3.331568in}{1.078686in}}%
\pgfpathlineto{\pgfqpoint{3.344044in}{1.087516in}}%
\pgfpathlineto{\pgfqpoint{3.357674in}{1.095945in}}%
\pgfpathlineto{\pgfqpoint{3.372404in}{1.103935in}}%
\pgfpathlineto{\pgfqpoint{3.381161in}{1.102356in}}%
\pgfpathlineto{\pgfqpoint{3.389720in}{1.101415in}}%
\pgfpathlineto{\pgfqpoint{3.398049in}{1.101114in}}%
\pgfpathlineto{\pgfqpoint{3.406113in}{1.101454in}}%
\pgfpathlineto{\pgfqpoint{3.413882in}{1.102431in}}%
\pgfpathlineto{\pgfqpoint{3.401599in}{1.095793in}}%
\pgfpathlineto{\pgfqpoint{3.390223in}{1.088788in}}%
\pgfpathlineto{\pgfqpoint{3.379799in}{1.081446in}}%
\pgfpathlineto{\pgfqpoint{3.370367in}{1.073800in}}%
\pgfpathclose%
\pgfusepath{fill}%
\end{pgfscope}%
\begin{pgfscope}%
\pgfpathrectangle{\pgfqpoint{2.548318in}{0.050000in}}{\pgfqpoint{2.081932in}{2.081932in}}%
\pgfusepath{clip}%
\pgfsetbuttcap%
\pgfsetroundjoin%
\definecolor{currentfill}{rgb}{0.278012,0.180367,0.486697}%
\pgfsetfillcolor{currentfill}%
\pgfsetlinewidth{0.000000pt}%
\definecolor{currentstroke}{rgb}{0.000000,0.000000,0.000000}%
\pgfsetstrokecolor{currentstroke}%
\pgfsetdash{}{0pt}%
\pgfpathmoveto{\pgfqpoint{3.729902in}{1.138087in}}%
\pgfpathlineto{\pgfqpoint{3.733783in}{1.134784in}}%
\pgfpathlineto{\pgfqpoint{3.737952in}{1.132122in}}%
\pgfpathlineto{\pgfqpoint{3.742394in}{1.130114in}}%
\pgfpathlineto{\pgfqpoint{3.747091in}{1.128769in}}%
\pgfpathlineto{\pgfqpoint{3.752025in}{1.128095in}}%
\pgfpathlineto{\pgfqpoint{3.767881in}{1.123568in}}%
\pgfpathlineto{\pgfqpoint{3.783097in}{1.118551in}}%
\pgfpathlineto{\pgfqpoint{3.797611in}{1.113067in}}%
\pgfpathlineto{\pgfqpoint{3.811362in}{1.107139in}}%
\pgfpathlineto{\pgfqpoint{3.824295in}{1.100795in}}%
\pgfpathlineto{\pgfqpoint{3.816741in}{1.102463in}}%
\pgfpathlineto{\pgfqpoint{3.809548in}{1.104752in}}%
\pgfpathlineto{\pgfqpoint{3.802743in}{1.107652in}}%
\pgfpathlineto{\pgfqpoint{3.796354in}{1.111148in}}%
\pgfpathlineto{\pgfqpoint{3.790404in}{1.115226in}}%
\pgfpathlineto{\pgfqpoint{3.779566in}{1.120542in}}%
\pgfpathlineto{\pgfqpoint{3.768047in}{1.125507in}}%
\pgfpathlineto{\pgfqpoint{3.755897in}{1.130099in}}%
\pgfpathlineto{\pgfqpoint{3.743164in}{1.134299in}}%
\pgfpathlineto{\pgfqpoint{3.729902in}{1.138087in}}%
\pgfpathclose%
\pgfusepath{fill}%
\end{pgfscope}%
\begin{pgfscope}%
\pgfpathrectangle{\pgfqpoint{2.548318in}{0.050000in}}{\pgfqpoint{2.081932in}{2.081932in}}%
\pgfusepath{clip}%
\pgfsetbuttcap%
\pgfsetroundjoin%
\definecolor{currentfill}{rgb}{0.278012,0.180367,0.486697}%
\pgfsetfillcolor{currentfill}%
\pgfsetlinewidth{0.000000pt}%
\definecolor{currentstroke}{rgb}{0.000000,0.000000,0.000000}%
\pgfsetstrokecolor{currentstroke}%
\pgfsetdash{}{0pt}%
\pgfpathmoveto{\pgfqpoint{3.447266in}{1.116598in}}%
\pgfpathlineto{\pgfqpoint{3.441406in}{1.112566in}}%
\pgfpathlineto{\pgfqpoint{3.435112in}{1.109119in}}%
\pgfpathlineto{\pgfqpoint{3.428409in}{1.106273in}}%
\pgfpathlineto{\pgfqpoint{3.421323in}{1.104040in}}%
\pgfpathlineto{\pgfqpoint{3.413882in}{1.102431in}}%
\pgfpathlineto{\pgfqpoint{3.427023in}{1.108673in}}%
\pgfpathlineto{\pgfqpoint{3.440969in}{1.114492in}}%
\pgfpathlineto{\pgfqpoint{3.455663in}{1.119861in}}%
\pgfpathlineto{\pgfqpoint{3.471044in}{1.124756in}}%
\pgfpathlineto{\pgfqpoint{3.487049in}{1.129156in}}%
\pgfpathlineto{\pgfqpoint{3.491836in}{1.129791in}}%
\pgfpathlineto{\pgfqpoint{3.496393in}{1.131099in}}%
\pgfpathlineto{\pgfqpoint{3.500702in}{1.133073in}}%
\pgfpathlineto{\pgfqpoint{3.504747in}{1.135702in}}%
\pgfpathlineto{\pgfqpoint{3.508512in}{1.138975in}}%
\pgfpathlineto{\pgfqpoint{3.495127in}{1.135293in}}%
\pgfpathlineto{\pgfqpoint{3.482257in}{1.131195in}}%
\pgfpathlineto{\pgfqpoint{3.469957in}{1.126700in}}%
\pgfpathlineto{\pgfqpoint{3.458278in}{1.121827in}}%
\pgfpathlineto{\pgfqpoint{3.447266in}{1.116598in}}%
\pgfpathclose%
\pgfusepath{fill}%
\end{pgfscope}%
\begin{pgfscope}%
\pgfpathrectangle{\pgfqpoint{2.548318in}{0.050000in}}{\pgfqpoint{2.081932in}{2.081932in}}%
\pgfusepath{clip}%
\pgfsetbuttcap%
\pgfsetroundjoin%
\definecolor{currentfill}{rgb}{0.327796,0.773980,0.406640}%
\pgfsetfillcolor{currentfill}%
\pgfsetlinewidth{0.000000pt}%
\definecolor{currentstroke}{rgb}{0.000000,0.000000,0.000000}%
\pgfsetstrokecolor{currentstroke}%
\pgfsetdash{}{0pt}%
\pgfpathmoveto{\pgfqpoint{4.042612in}{1.369670in}}%
\pgfpathlineto{\pgfqpoint{4.039202in}{1.376528in}}%
\pgfpathlineto{\pgfqpoint{4.035245in}{1.382895in}}%
\pgfpathlineto{\pgfqpoint{4.030761in}{1.388745in}}%
\pgfpathlineto{\pgfqpoint{4.025768in}{1.394055in}}%
\pgfpathlineto{\pgfqpoint{4.020289in}{1.398805in}}%
\pgfpathlineto{\pgfqpoint{4.043364in}{1.385100in}}%
\pgfpathlineto{\pgfqpoint{4.064586in}{1.370691in}}%
\pgfpathlineto{\pgfqpoint{4.083872in}{1.355647in}}%
\pgfpathlineto{\pgfqpoint{4.101153in}{1.340037in}}%
\pgfpathlineto{\pgfqpoint{4.116370in}{1.323931in}}%
\pgfpathlineto{\pgfqpoint{4.123081in}{1.318227in}}%
\pgfpathlineto{\pgfqpoint{4.129194in}{1.312060in}}%
\pgfpathlineto{\pgfqpoint{4.134682in}{1.305454in}}%
\pgfpathlineto{\pgfqpoint{4.139524in}{1.298436in}}%
\pgfpathlineto{\pgfqpoint{4.143696in}{1.291033in}}%
\pgfpathlineto{\pgfqpoint{4.127714in}{1.307941in}}%
\pgfpathlineto{\pgfqpoint{4.109546in}{1.324332in}}%
\pgfpathlineto{\pgfqpoint{4.089256in}{1.340132in}}%
\pgfpathlineto{\pgfqpoint{4.066917in}{1.355268in}}%
\pgfpathlineto{\pgfqpoint{4.042612in}{1.369670in}}%
\pgfpathclose%
\pgfusepath{fill}%
\end{pgfscope}%
\begin{pgfscope}%
\pgfpathrectangle{\pgfqpoint{2.548318in}{0.050000in}}{\pgfqpoint{2.081932in}{2.081932in}}%
\pgfusepath{clip}%
\pgfsetbuttcap%
\pgfsetroundjoin%
\definecolor{currentfill}{rgb}{0.876168,0.891125,0.095250}%
\pgfsetfillcolor{currentfill}%
\pgfsetlinewidth{0.000000pt}%
\definecolor{currentstroke}{rgb}{0.000000,0.000000,0.000000}%
\pgfsetstrokecolor{currentstroke}%
\pgfsetdash{}{0pt}%
\pgfpathmoveto{\pgfqpoint{3.858308in}{1.464844in}}%
\pgfpathlineto{\pgfqpoint{3.853147in}{1.464851in}}%
\pgfpathlineto{\pgfqpoint{3.847797in}{1.464180in}}%
\pgfpathlineto{\pgfqpoint{3.842280in}{1.462834in}}%
\pgfpathlineto{\pgfqpoint{3.836620in}{1.460822in}}%
\pgfpathlineto{\pgfqpoint{3.830839in}{1.458153in}}%
\pgfpathlineto{\pgfqpoint{3.855889in}{1.450612in}}%
\pgfpathlineto{\pgfqpoint{3.879894in}{1.442262in}}%
\pgfpathlineto{\pgfqpoint{3.902752in}{1.433143in}}%
\pgfpathlineto{\pgfqpoint{3.924369in}{1.423297in}}%
\pgfpathlineto{\pgfqpoint{3.944654in}{1.412770in}}%
\pgfpathlineto{\pgfqpoint{3.953459in}{1.414199in}}%
\pgfpathlineto{\pgfqpoint{3.962078in}{1.415001in}}%
\pgfpathlineto{\pgfqpoint{3.970474in}{1.415171in}}%
\pgfpathlineto{\pgfqpoint{3.978614in}{1.414707in}}%
\pgfpathlineto{\pgfqpoint{3.986465in}{1.413610in}}%
\pgfpathlineto{\pgfqpoint{3.963655in}{1.425486in}}%
\pgfpathlineto{\pgfqpoint{3.939331in}{1.436598in}}%
\pgfpathlineto{\pgfqpoint{3.913592in}{1.446894in}}%
\pgfpathlineto{\pgfqpoint{3.886547in}{1.456325in}}%
\pgfpathlineto{\pgfqpoint{3.858308in}{1.464844in}}%
\pgfpathclose%
\pgfusepath{fill}%
\end{pgfscope}%
\begin{pgfscope}%
\pgfpathrectangle{\pgfqpoint{2.548318in}{0.050000in}}{\pgfqpoint{2.081932in}{2.081932in}}%
\pgfusepath{clip}%
\pgfsetbuttcap%
\pgfsetroundjoin%
\definecolor{currentfill}{rgb}{0.876168,0.891125,0.095250}%
\pgfsetfillcolor{currentfill}%
\pgfsetlinewidth{0.000000pt}%
\definecolor{currentstroke}{rgb}{0.000000,0.000000,0.000000}%
\pgfsetstrokecolor{currentstroke}%
\pgfsetdash{}{0pt}%
\pgfpathmoveto{\pgfqpoint{3.254093in}{1.416671in}}%
\pgfpathlineto{\pgfqpoint{3.261830in}{1.417704in}}%
\pgfpathlineto{\pgfqpoint{3.269852in}{1.418101in}}%
\pgfpathlineto{\pgfqpoint{3.278126in}{1.417861in}}%
\pgfpathlineto{\pgfqpoint{3.286618in}{1.416987in}}%
\pgfpathlineto{\pgfqpoint{3.295295in}{1.415484in}}%
\pgfpathlineto{\pgfqpoint{3.315919in}{1.425844in}}%
\pgfpathlineto{\pgfqpoint{3.337852in}{1.435511in}}%
\pgfpathlineto{\pgfqpoint{3.361004in}{1.444441in}}%
\pgfpathlineto{\pgfqpoint{3.385279in}{1.452590in}}%
\pgfpathlineto{\pgfqpoint{3.410573in}{1.459921in}}%
\pgfpathlineto{\pgfqpoint{3.404963in}{1.462639in}}%
\pgfpathlineto{\pgfqpoint{3.399470in}{1.464699in}}%
\pgfpathlineto{\pgfqpoint{3.394117in}{1.466090in}}%
\pgfpathlineto{\pgfqpoint{3.388925in}{1.466806in}}%
\pgfpathlineto{\pgfqpoint{3.383916in}{1.466842in}}%
\pgfpathlineto{\pgfqpoint{3.355400in}{1.458559in}}%
\pgfpathlineto{\pgfqpoint{3.328047in}{1.449354in}}%
\pgfpathlineto{\pgfqpoint{3.301973in}{1.439272in}}%
\pgfpathlineto{\pgfqpoint{3.277288in}{1.428360in}}%
\pgfpathlineto{\pgfqpoint{3.254093in}{1.416671in}}%
\pgfpathclose%
\pgfusepath{fill}%
\end{pgfscope}%
\begin{pgfscope}%
\pgfpathrectangle{\pgfqpoint{2.548318in}{0.050000in}}{\pgfqpoint{2.081932in}{2.081932in}}%
\pgfusepath{clip}%
\pgfsetbuttcap%
\pgfsetroundjoin%
\definecolor{currentfill}{rgb}{0.327796,0.773980,0.406640}%
\pgfsetfillcolor{currentfill}%
\pgfsetlinewidth{0.000000pt}%
\definecolor{currentstroke}{rgb}{0.000000,0.000000,0.000000}%
\pgfsetstrokecolor{currentstroke}%
\pgfsetdash{}{0pt}%
\pgfpathmoveto{\pgfqpoint{3.111635in}{1.312122in}}%
\pgfpathlineto{\pgfqpoint{3.115659in}{1.319381in}}%
\pgfpathlineto{\pgfqpoint{3.120328in}{1.326225in}}%
\pgfpathlineto{\pgfqpoint{3.125621in}{1.332630in}}%
\pgfpathlineto{\pgfqpoint{3.131515in}{1.338570in}}%
\pgfpathlineto{\pgfqpoint{3.137986in}{1.344020in}}%
\pgfpathlineto{\pgfqpoint{3.155772in}{1.359494in}}%
\pgfpathlineto{\pgfqpoint{3.175546in}{1.374385in}}%
\pgfpathlineto{\pgfqpoint{3.197236in}{1.388622in}}%
\pgfpathlineto{\pgfqpoint{3.220759in}{1.402141in}}%
\pgfpathlineto{\pgfqpoint{3.215358in}{1.397433in}}%
\pgfpathlineto{\pgfqpoint{3.210437in}{1.392162in}}%
\pgfpathlineto{\pgfqpoint{3.206017in}{1.386346in}}%
\pgfpathlineto{\pgfqpoint{3.202118in}{1.380008in}}%
\pgfpathlineto{\pgfqpoint{3.198757in}{1.373175in}}%
\pgfpathlineto{\pgfqpoint{3.173978in}{1.358969in}}%
\pgfpathlineto{\pgfqpoint{3.151143in}{1.344011in}}%
\pgfpathlineto{\pgfqpoint{3.130336in}{1.328372in}}%
\pgfpathlineto{\pgfqpoint{3.111635in}{1.312122in}}%
\pgfpathclose%
\pgfusepath{fill}%
\end{pgfscope}%
\begin{pgfscope}%
\pgfpathrectangle{\pgfqpoint{2.548318in}{0.050000in}}{\pgfqpoint{2.081932in}{2.081932in}}%
\pgfusepath{clip}%
\pgfsetbuttcap%
\pgfsetroundjoin%
\definecolor{currentfill}{rgb}{0.993248,0.906157,0.143936}%
\pgfsetfillcolor{currentfill}%
\pgfsetlinewidth{0.000000pt}%
\definecolor{currentstroke}{rgb}{0.000000,0.000000,0.000000}%
\pgfsetstrokecolor{currentstroke}%
\pgfsetdash{}{0pt}%
\pgfpathmoveto{\pgfqpoint{3.693808in}{1.482605in}}%
\pgfpathlineto{\pgfqpoint{3.691698in}{1.478604in}}%
\pgfpathlineto{\pgfqpoint{3.689563in}{1.473964in}}%
\pgfpathlineto{\pgfqpoint{3.687410in}{1.468706in}}%
\pgfpathlineto{\pgfqpoint{3.685248in}{1.462852in}}%
\pgfpathlineto{\pgfqpoint{3.683086in}{1.456427in}}%
\pgfpathlineto{\pgfqpoint{3.707656in}{1.453826in}}%
\pgfpathlineto{\pgfqpoint{3.731832in}{1.450414in}}%
\pgfpathlineto{\pgfqpoint{3.755507in}{1.446205in}}%
\pgfpathlineto{\pgfqpoint{3.778579in}{1.441220in}}%
\pgfpathlineto{\pgfqpoint{3.800949in}{1.435482in}}%
\pgfpathlineto{\pgfqpoint{3.806979in}{1.441198in}}%
\pgfpathlineto{\pgfqpoint{3.813007in}{1.446344in}}%
\pgfpathlineto{\pgfqpoint{3.819009in}{1.450898in}}%
\pgfpathlineto{\pgfqpoint{3.824960in}{1.454840in}}%
\pgfpathlineto{\pgfqpoint{3.830839in}{1.458153in}}%
\pgfpathlineto{\pgfqpoint{3.804849in}{1.464849in}}%
\pgfpathlineto{\pgfqpoint{3.778033in}{1.470668in}}%
\pgfpathlineto{\pgfqpoint{3.750506in}{1.475582in}}%
\pgfpathlineto{\pgfqpoint{3.722390in}{1.479568in}}%
\pgfpathlineto{\pgfqpoint{3.693808in}{1.482605in}}%
\pgfpathclose%
\pgfusepath{fill}%
\end{pgfscope}%
\begin{pgfscope}%
\pgfpathrectangle{\pgfqpoint{2.548318in}{0.050000in}}{\pgfqpoint{2.081932in}{2.081932in}}%
\pgfusepath{clip}%
\pgfsetbuttcap%
\pgfsetroundjoin%
\definecolor{currentfill}{rgb}{0.267004,0.004874,0.329415}%
\pgfsetfillcolor{currentfill}%
\pgfsetlinewidth{0.000000pt}%
\definecolor{currentstroke}{rgb}{0.000000,0.000000,0.000000}%
\pgfsetstrokecolor{currentstroke}%
\pgfsetdash{}{0pt}%
\pgfpathmoveto{\pgfqpoint{3.866398in}{1.101965in}}%
\pgfpathlineto{\pgfqpoint{3.875449in}{1.104105in}}%
\pgfpathlineto{\pgfqpoint{3.884628in}{1.106871in}}%
\pgfpathlineto{\pgfqpoint{3.893897in}{1.110250in}}%
\pgfpathlineto{\pgfqpoint{3.903218in}{1.114233in}}%
\pgfpathlineto{\pgfqpoint{3.912553in}{1.118802in}}%
\pgfpathlineto{\pgfqpoint{3.929625in}{1.109172in}}%
\pgfpathlineto{\pgfqpoint{3.945364in}{1.099037in}}%
\pgfpathlineto{\pgfqpoint{3.959711in}{1.088443in}}%
\pgfpathlineto{\pgfqpoint{3.972610in}{1.077436in}}%
\pgfpathlineto{\pgfqpoint{3.984016in}{1.066066in}}%
\pgfpathlineto{\pgfqpoint{3.972511in}{1.063188in}}%
\pgfpathlineto{\pgfqpoint{3.961018in}{1.060889in}}%
\pgfpathlineto{\pgfqpoint{3.949583in}{1.059177in}}%
\pgfpathlineto{\pgfqpoint{3.938253in}{1.058059in}}%
\pgfpathlineto{\pgfqpoint{3.927074in}{1.057538in}}%
\pgfpathlineto{\pgfqpoint{3.917358in}{1.067126in}}%
\pgfpathlineto{\pgfqpoint{3.906390in}{1.076403in}}%
\pgfpathlineto{\pgfqpoint{3.894208in}{1.085327in}}%
\pgfpathlineto{\pgfqpoint{3.880860in}{1.093861in}}%
\pgfpathlineto{\pgfqpoint{3.866398in}{1.101965in}}%
\pgfpathclose%
\pgfusepath{fill}%
\end{pgfscope}%
\begin{pgfscope}%
\pgfpathrectangle{\pgfqpoint{2.548318in}{0.050000in}}{\pgfqpoint{2.081932in}{2.081932in}}%
\pgfusepath{clip}%
\pgfsetbuttcap%
\pgfsetroundjoin%
\definecolor{currentfill}{rgb}{0.993248,0.906157,0.143936}%
\pgfsetfillcolor{currentfill}%
\pgfsetlinewidth{0.000000pt}%
\definecolor{currentstroke}{rgb}{0.000000,0.000000,0.000000}%
\pgfsetstrokecolor{currentstroke}%
\pgfsetdash{}{0pt}%
\pgfpathmoveto{\pgfqpoint{3.410573in}{1.459921in}}%
\pgfpathlineto{\pgfqpoint{3.416277in}{1.456559in}}%
\pgfpathlineto{\pgfqpoint{3.422053in}{1.452566in}}%
\pgfpathlineto{\pgfqpoint{3.427877in}{1.447961in}}%
\pgfpathlineto{\pgfqpoint{3.433725in}{1.442764in}}%
\pgfpathlineto{\pgfqpoint{3.439576in}{1.436997in}}%
\pgfpathlineto{\pgfqpoint{3.462130in}{1.442548in}}%
\pgfpathlineto{\pgfqpoint{3.485361in}{1.447339in}}%
\pgfpathlineto{\pgfqpoint{3.509171in}{1.451349in}}%
\pgfpathlineto{\pgfqpoint{3.533454in}{1.454559in}}%
\pgfpathlineto{\pgfqpoint{3.558106in}{1.456953in}}%
\pgfpathlineto{\pgfqpoint{3.556147in}{1.463396in}}%
\pgfpathlineto{\pgfqpoint{3.554188in}{1.469268in}}%
\pgfpathlineto{\pgfqpoint{3.552237in}{1.474543in}}%
\pgfpathlineto{\pgfqpoint{3.550301in}{1.479201in}}%
\pgfpathlineto{\pgfqpoint{3.548389in}{1.483219in}}%
\pgfpathlineto{\pgfqpoint{3.519711in}{1.480423in}}%
\pgfpathlineto{\pgfqpoint{3.491467in}{1.476675in}}%
\pgfpathlineto{\pgfqpoint{3.463782in}{1.471993in}}%
\pgfpathlineto{\pgfqpoint{3.436778in}{1.466399in}}%
\pgfpathlineto{\pgfqpoint{3.410573in}{1.459921in}}%
\pgfpathclose%
\pgfusepath{fill}%
\end{pgfscope}%
\begin{pgfscope}%
\pgfpathrectangle{\pgfqpoint{2.548318in}{0.050000in}}{\pgfqpoint{2.081932in}{2.081932in}}%
\pgfusepath{clip}%
\pgfsetbuttcap%
\pgfsetroundjoin%
\definecolor{currentfill}{rgb}{0.278012,0.180367,0.486697}%
\pgfsetfillcolor{currentfill}%
\pgfsetlinewidth{0.000000pt}%
\definecolor{currentstroke}{rgb}{0.000000,0.000000,0.000000}%
\pgfsetstrokecolor{currentstroke}%
\pgfsetdash{}{0pt}%
\pgfpathmoveto{\pgfqpoint{3.657644in}{1.150345in}}%
\pgfpathlineto{\pgfqpoint{3.659032in}{1.147459in}}%
\pgfpathlineto{\pgfqpoint{3.660524in}{1.145248in}}%
\pgfpathlineto{\pgfqpoint{3.662113in}{1.143720in}}%
\pgfpathlineto{\pgfqpoint{3.663794in}{1.142886in}}%
\pgfpathlineto{\pgfqpoint{3.665560in}{1.142749in}}%
\pgfpathlineto{\pgfqpoint{3.683568in}{1.140931in}}%
\pgfpathlineto{\pgfqpoint{3.701294in}{1.138544in}}%
\pgfpathlineto{\pgfqpoint{3.718661in}{1.135600in}}%
\pgfpathlineto{\pgfqpoint{3.735595in}{1.132112in}}%
\pgfpathlineto{\pgfqpoint{3.752025in}{1.128095in}}%
\pgfpathlineto{\pgfqpoint{3.747091in}{1.128769in}}%
\pgfpathlineto{\pgfqpoint{3.742394in}{1.130114in}}%
\pgfpathlineto{\pgfqpoint{3.737952in}{1.132122in}}%
\pgfpathlineto{\pgfqpoint{3.733783in}{1.134784in}}%
\pgfpathlineto{\pgfqpoint{3.729902in}{1.138087in}}%
\pgfpathlineto{\pgfqpoint{3.716164in}{1.141448in}}%
\pgfpathlineto{\pgfqpoint{3.702009in}{1.144366in}}%
\pgfpathlineto{\pgfqpoint{3.687497in}{1.146829in}}%
\pgfpathlineto{\pgfqpoint{3.672687in}{1.148825in}}%
\pgfpathlineto{\pgfqpoint{3.657644in}{1.150345in}}%
\pgfpathclose%
\pgfusepath{fill}%
\end{pgfscope}%
\begin{pgfscope}%
\pgfpathrectangle{\pgfqpoint{2.548318in}{0.050000in}}{\pgfqpoint{2.081932in}{2.081932in}}%
\pgfusepath{clip}%
\pgfsetbuttcap%
\pgfsetroundjoin%
\definecolor{currentfill}{rgb}{0.278012,0.180367,0.486697}%
\pgfsetfillcolor{currentfill}%
\pgfsetlinewidth{0.000000pt}%
\definecolor{currentstroke}{rgb}{0.000000,0.000000,0.000000}%
\pgfsetstrokecolor{currentstroke}%
\pgfsetdash{}{0pt}%
\pgfpathmoveto{\pgfqpoint{3.508512in}{1.138975in}}%
\pgfpathlineto{\pgfqpoint{3.504747in}{1.135702in}}%
\pgfpathlineto{\pgfqpoint{3.500702in}{1.133073in}}%
\pgfpathlineto{\pgfqpoint{3.496393in}{1.131099in}}%
\pgfpathlineto{\pgfqpoint{3.491836in}{1.129791in}}%
\pgfpathlineto{\pgfqpoint{3.487049in}{1.129156in}}%
\pgfpathlineto{\pgfqpoint{3.503611in}{1.133042in}}%
\pgfpathlineto{\pgfqpoint{3.520659in}{1.136394in}}%
\pgfpathlineto{\pgfqpoint{3.538122in}{1.139199in}}%
\pgfpathlineto{\pgfqpoint{3.555925in}{1.141443in}}%
\pgfpathlineto{\pgfqpoint{3.573992in}{1.143117in}}%
\pgfpathlineto{\pgfqpoint{3.575593in}{1.143240in}}%
\pgfpathlineto{\pgfqpoint{3.577116in}{1.144061in}}%
\pgfpathlineto{\pgfqpoint{3.578556in}{1.145577in}}%
\pgfpathlineto{\pgfqpoint{3.579908in}{1.147777in}}%
\pgfpathlineto{\pgfqpoint{3.581166in}{1.150652in}}%
\pgfpathlineto{\pgfqpoint{3.566074in}{1.149253in}}%
\pgfpathlineto{\pgfqpoint{3.551200in}{1.147376in}}%
\pgfpathlineto{\pgfqpoint{3.536608in}{1.145030in}}%
\pgfpathlineto{\pgfqpoint{3.522359in}{1.142226in}}%
\pgfpathlineto{\pgfqpoint{3.508512in}{1.138975in}}%
\pgfpathclose%
\pgfusepath{fill}%
\end{pgfscope}%
\begin{pgfscope}%
\pgfpathrectangle{\pgfqpoint{2.548318in}{0.050000in}}{\pgfqpoint{2.081932in}{2.081932in}}%
\pgfusepath{clip}%
\pgfsetbuttcap%
\pgfsetroundjoin%
\definecolor{currentfill}{rgb}{0.267004,0.004874,0.329415}%
\pgfsetfillcolor{currentfill}%
\pgfsetlinewidth{0.000000pt}%
\definecolor{currentstroke}{rgb}{0.000000,0.000000,0.000000}%
\pgfsetstrokecolor{currentstroke}%
\pgfsetdash{}{0pt}%
\pgfpathmoveto{\pgfqpoint{3.320294in}{1.069495in}}%
\pgfpathlineto{\pgfqpoint{3.309540in}{1.070449in}}%
\pgfpathlineto{\pgfqpoint{3.298638in}{1.072008in}}%
\pgfpathlineto{\pgfqpoint{3.287634in}{1.074167in}}%
\pgfpathlineto{\pgfqpoint{3.276572in}{1.076916in}}%
\pgfpathlineto{\pgfqpoint{3.265499in}{1.080246in}}%
\pgfpathlineto{\pgfqpoint{3.278762in}{1.091153in}}%
\pgfpathlineto{\pgfqpoint{3.293460in}{1.101636in}}%
\pgfpathlineto{\pgfqpoint{3.309537in}{1.111648in}}%
\pgfpathlineto{\pgfqpoint{3.326929in}{1.121143in}}%
\pgfpathlineto{\pgfqpoint{3.336127in}{1.116499in}}%
\pgfpathlineto{\pgfqpoint{3.345311in}{1.112441in}}%
\pgfpathlineto{\pgfqpoint{3.354443in}{1.108987in}}%
\pgfpathlineto{\pgfqpoint{3.363486in}{1.106148in}}%
\pgfpathlineto{\pgfqpoint{3.372404in}{1.103935in}}%
\pgfpathlineto{\pgfqpoint{3.357674in}{1.095945in}}%
\pgfpathlineto{\pgfqpoint{3.344044in}{1.087516in}}%
\pgfpathlineto{\pgfqpoint{3.331568in}{1.078686in}}%
\pgfpathlineto{\pgfqpoint{3.320294in}{1.069495in}}%
\pgfpathclose%
\pgfusepath{fill}%
\end{pgfscope}%
\begin{pgfscope}%
\pgfpathrectangle{\pgfqpoint{2.548318in}{0.050000in}}{\pgfqpoint{2.081932in}{2.081932in}}%
\pgfusepath{clip}%
\pgfsetbuttcap%
\pgfsetroundjoin%
\definecolor{currentfill}{rgb}{0.124780,0.640461,0.527068}%
\pgfsetfillcolor{currentfill}%
\pgfsetlinewidth{0.000000pt}%
\definecolor{currentstroke}{rgb}{0.000000,0.000000,0.000000}%
\pgfsetstrokecolor{currentstroke}%
\pgfsetdash{}{0pt}%
\pgfpathmoveto{\pgfqpoint{4.051034in}{1.329028in}}%
\pgfpathlineto{\pgfqpoint{4.050531in}{1.337896in}}%
\pgfpathlineto{\pgfqpoint{4.049431in}{1.346428in}}%
\pgfpathlineto{\pgfqpoint{4.047738in}{1.354589in}}%
\pgfpathlineto{\pgfqpoint{4.045462in}{1.362347in}}%
\pgfpathlineto{\pgfqpoint{4.042612in}{1.369670in}}%
\pgfpathlineto{\pgfqpoint{4.066917in}{1.355268in}}%
\pgfpathlineto{\pgfqpoint{4.089256in}{1.340132in}}%
\pgfpathlineto{\pgfqpoint{4.109546in}{1.324332in}}%
\pgfpathlineto{\pgfqpoint{4.127714in}{1.307941in}}%
\pgfpathlineto{\pgfqpoint{4.143696in}{1.291033in}}%
\pgfpathlineto{\pgfqpoint{4.147183in}{1.283275in}}%
\pgfpathlineto{\pgfqpoint{4.149967in}{1.275194in}}%
\pgfpathlineto{\pgfqpoint{4.152036in}{1.266820in}}%
\pgfpathlineto{\pgfqpoint{4.153382in}{1.258189in}}%
\pgfpathlineto{\pgfqpoint{4.153997in}{1.249335in}}%
\pgfpathlineto{\pgfqpoint{4.137727in}{1.266467in}}%
\pgfpathlineto{\pgfqpoint{4.119227in}{1.283076in}}%
\pgfpathlineto{\pgfqpoint{4.098560in}{1.299089in}}%
\pgfpathlineto{\pgfqpoint{4.075801in}{1.314430in}}%
\pgfpathlineto{\pgfqpoint{4.051034in}{1.329028in}}%
\pgfpathclose%
\pgfusepath{fill}%
\end{pgfscope}%
\begin{pgfscope}%
\pgfpathrectangle{\pgfqpoint{2.548318in}{0.050000in}}{\pgfqpoint{2.081932in}{2.081932in}}%
\pgfusepath{clip}%
\pgfsetbuttcap%
\pgfsetroundjoin%
\definecolor{currentfill}{rgb}{0.993248,0.906157,0.143936}%
\pgfsetfillcolor{currentfill}%
\pgfsetlinewidth{0.000000pt}%
\definecolor{currentstroke}{rgb}{0.000000,0.000000,0.000000}%
\pgfsetstrokecolor{currentstroke}%
\pgfsetdash{}{0pt}%
\pgfpathmoveto{\pgfqpoint{3.548389in}{1.483219in}}%
\pgfpathlineto{\pgfqpoint{3.550301in}{1.479201in}}%
\pgfpathlineto{\pgfqpoint{3.552237in}{1.474543in}}%
\pgfpathlineto{\pgfqpoint{3.554188in}{1.469268in}}%
\pgfpathlineto{\pgfqpoint{3.556147in}{1.463396in}}%
\pgfpathlineto{\pgfqpoint{3.558106in}{1.456953in}}%
\pgfpathlineto{\pgfqpoint{3.583019in}{1.458520in}}%
\pgfpathlineto{\pgfqpoint{3.608084in}{1.459252in}}%
\pgfpathlineto{\pgfqpoint{3.633190in}{1.459146in}}%
\pgfpathlineto{\pgfqpoint{3.658227in}{1.458203in}}%
\pgfpathlineto{\pgfqpoint{3.683086in}{1.456427in}}%
\pgfpathlineto{\pgfqpoint{3.685248in}{1.462852in}}%
\pgfpathlineto{\pgfqpoint{3.687410in}{1.468706in}}%
\pgfpathlineto{\pgfqpoint{3.689563in}{1.473964in}}%
\pgfpathlineto{\pgfqpoint{3.691698in}{1.478604in}}%
\pgfpathlineto{\pgfqpoint{3.693808in}{1.482605in}}%
\pgfpathlineto{\pgfqpoint{3.664886in}{1.484679in}}%
\pgfpathlineto{\pgfqpoint{3.635754in}{1.485781in}}%
\pgfpathlineto{\pgfqpoint{3.606540in}{1.485905in}}%
\pgfpathlineto{\pgfqpoint{3.577375in}{1.485049in}}%
\pgfpathlineto{\pgfqpoint{3.548389in}{1.483219in}}%
\pgfpathclose%
\pgfusepath{fill}%
\end{pgfscope}%
\begin{pgfscope}%
\pgfpathrectangle{\pgfqpoint{2.548318in}{0.050000in}}{\pgfqpoint{2.081932in}{2.081932in}}%
\pgfusepath{clip}%
\pgfsetbuttcap%
\pgfsetroundjoin%
\definecolor{currentfill}{rgb}{0.278012,0.180367,0.486697}%
\pgfsetfillcolor{currentfill}%
\pgfsetlinewidth{0.000000pt}%
\definecolor{currentstroke}{rgb}{0.000000,0.000000,0.000000}%
\pgfsetstrokecolor{currentstroke}%
\pgfsetdash{}{0pt}%
\pgfpathmoveto{\pgfqpoint{3.581166in}{1.150652in}}%
\pgfpathlineto{\pgfqpoint{3.579908in}{1.147777in}}%
\pgfpathlineto{\pgfqpoint{3.578556in}{1.145577in}}%
\pgfpathlineto{\pgfqpoint{3.577116in}{1.144061in}}%
\pgfpathlineto{\pgfqpoint{3.575593in}{1.143240in}}%
\pgfpathlineto{\pgfqpoint{3.573992in}{1.143117in}}%
\pgfpathlineto{\pgfqpoint{3.592247in}{1.144212in}}%
\pgfpathlineto{\pgfqpoint{3.610609in}{1.144724in}}%
\pgfpathlineto{\pgfqpoint{3.629002in}{1.144650in}}%
\pgfpathlineto{\pgfqpoint{3.647345in}{1.143991in}}%
\pgfpathlineto{\pgfqpoint{3.665560in}{1.142749in}}%
\pgfpathlineto{\pgfqpoint{3.663794in}{1.142886in}}%
\pgfpathlineto{\pgfqpoint{3.662113in}{1.143720in}}%
\pgfpathlineto{\pgfqpoint{3.660524in}{1.145248in}}%
\pgfpathlineto{\pgfqpoint{3.659032in}{1.147459in}}%
\pgfpathlineto{\pgfqpoint{3.657644in}{1.150345in}}%
\pgfpathlineto{\pgfqpoint{3.642429in}{1.151383in}}%
\pgfpathlineto{\pgfqpoint{3.627109in}{1.151934in}}%
\pgfpathlineto{\pgfqpoint{3.611749in}{1.151996in}}%
\pgfpathlineto{\pgfqpoint{3.596412in}{1.151568in}}%
\pgfpathlineto{\pgfqpoint{3.581166in}{1.150652in}}%
\pgfpathclose%
\pgfusepath{fill}%
\end{pgfscope}%
\begin{pgfscope}%
\pgfpathrectangle{\pgfqpoint{2.548318in}{0.050000in}}{\pgfqpoint{2.081932in}{2.081932in}}%
\pgfusepath{clip}%
\pgfsetbuttcap%
\pgfsetroundjoin%
\definecolor{currentfill}{rgb}{0.278791,0.062145,0.386592}%
\pgfsetfillcolor{currentfill}%
\pgfsetlinewidth{0.000000pt}%
\definecolor{currentstroke}{rgb}{0.000000,0.000000,0.000000}%
\pgfsetstrokecolor{currentstroke}%
\pgfsetdash{}{0pt}%
\pgfpathmoveto{\pgfqpoint{3.752025in}{1.128095in}}%
\pgfpathlineto{\pgfqpoint{3.757177in}{1.128097in}}%
\pgfpathlineto{\pgfqpoint{3.762527in}{1.128777in}}%
\pgfpathlineto{\pgfqpoint{3.768054in}{1.130133in}}%
\pgfpathlineto{\pgfqpoint{3.773736in}{1.132162in}}%
\pgfpathlineto{\pgfqpoint{3.779550in}{1.134859in}}%
\pgfpathlineto{\pgfqpoint{3.798626in}{1.129400in}}%
\pgfpathlineto{\pgfqpoint{3.816923in}{1.123353in}}%
\pgfpathlineto{\pgfqpoint{3.834364in}{1.116744in}}%
\pgfpathlineto{\pgfqpoint{3.850878in}{1.109604in}}%
\pgfpathlineto{\pgfqpoint{3.866398in}{1.101965in}}%
\pgfpathlineto{\pgfqpoint{3.857510in}{1.100456in}}%
\pgfpathlineto{\pgfqpoint{3.848821in}{1.099584in}}%
\pgfpathlineto{\pgfqpoint{3.840367in}{1.099351in}}%
\pgfpathlineto{\pgfqpoint{3.832181in}{1.099755in}}%
\pgfpathlineto{\pgfqpoint{3.824295in}{1.100795in}}%
\pgfpathlineto{\pgfqpoint{3.811362in}{1.107139in}}%
\pgfpathlineto{\pgfqpoint{3.797611in}{1.113067in}}%
\pgfpathlineto{\pgfqpoint{3.783097in}{1.118551in}}%
\pgfpathlineto{\pgfqpoint{3.767881in}{1.123568in}}%
\pgfpathlineto{\pgfqpoint{3.752025in}{1.128095in}}%
\pgfpathclose%
\pgfusepath{fill}%
\end{pgfscope}%
\begin{pgfscope}%
\pgfpathrectangle{\pgfqpoint{2.548318in}{0.050000in}}{\pgfqpoint{2.081932in}{2.081932in}}%
\pgfusepath{clip}%
\pgfsetbuttcap%
\pgfsetroundjoin%
\definecolor{currentfill}{rgb}{0.278791,0.062145,0.386592}%
\pgfsetfillcolor{currentfill}%
\pgfsetlinewidth{0.000000pt}%
\definecolor{currentstroke}{rgb}{0.000000,0.000000,0.000000}%
\pgfsetstrokecolor{currentstroke}%
\pgfsetdash{}{0pt}%
\pgfpathmoveto{\pgfqpoint{3.413882in}{1.102431in}}%
\pgfpathlineto{\pgfqpoint{3.406113in}{1.101454in}}%
\pgfpathlineto{\pgfqpoint{3.398049in}{1.101114in}}%
\pgfpathlineto{\pgfqpoint{3.389720in}{1.101415in}}%
\pgfpathlineto{\pgfqpoint{3.381161in}{1.102356in}}%
\pgfpathlineto{\pgfqpoint{3.372404in}{1.103935in}}%
\pgfpathlineto{\pgfqpoint{3.388177in}{1.111452in}}%
\pgfpathlineto{\pgfqpoint{3.404928in}{1.118461in}}%
\pgfpathlineto{\pgfqpoint{3.422588in}{1.124931in}}%
\pgfpathlineto{\pgfqpoint{3.441086in}{1.130832in}}%
\pgfpathlineto{\pgfqpoint{3.460343in}{1.136138in}}%
\pgfpathlineto{\pgfqpoint{3.465985in}{1.133395in}}%
\pgfpathlineto{\pgfqpoint{3.471498in}{1.131321in}}%
\pgfpathlineto{\pgfqpoint{3.476860in}{1.129921in}}%
\pgfpathlineto{\pgfqpoint{3.482051in}{1.129199in}}%
\pgfpathlineto{\pgfqpoint{3.487049in}{1.129156in}}%
\pgfpathlineto{\pgfqpoint{3.471044in}{1.124756in}}%
\pgfpathlineto{\pgfqpoint{3.455663in}{1.119861in}}%
\pgfpathlineto{\pgfqpoint{3.440969in}{1.114492in}}%
\pgfpathlineto{\pgfqpoint{3.427023in}{1.108673in}}%
\pgfpathlineto{\pgfqpoint{3.413882in}{1.102431in}}%
\pgfpathclose%
\pgfusepath{fill}%
\end{pgfscope}%
\begin{pgfscope}%
\pgfpathrectangle{\pgfqpoint{2.548318in}{0.050000in}}{\pgfqpoint{2.081932in}{2.081932in}}%
\pgfusepath{clip}%
\pgfsetbuttcap%
\pgfsetroundjoin%
\definecolor{currentfill}{rgb}{0.268510,0.009605,0.335427}%
\pgfsetfillcolor{currentfill}%
\pgfsetlinewidth{0.000000pt}%
\definecolor{currentstroke}{rgb}{0.000000,0.000000,0.000000}%
\pgfsetstrokecolor{currentstroke}%
\pgfsetdash{}{0pt}%
\pgfpathmoveto{\pgfqpoint{3.912553in}{1.118802in}}%
\pgfpathlineto{\pgfqpoint{3.921862in}{1.123940in}}%
\pgfpathlineto{\pgfqpoint{3.931109in}{1.129628in}}%
\pgfpathlineto{\pgfqpoint{3.940254in}{1.135843in}}%
\pgfpathlineto{\pgfqpoint{3.949260in}{1.142559in}}%
\pgfpathlineto{\pgfqpoint{3.958089in}{1.149751in}}%
\pgfpathlineto{\pgfqpoint{3.977713in}{1.138584in}}%
\pgfpathlineto{\pgfqpoint{3.995787in}{1.126838in}}%
\pgfpathlineto{\pgfqpoint{4.012241in}{1.114565in}}%
\pgfpathlineto{\pgfqpoint{4.027014in}{1.101820in}}%
\pgfpathlineto{\pgfqpoint{4.040052in}{1.088662in}}%
\pgfpathlineto{\pgfqpoint{4.029199in}{1.083105in}}%
\pgfpathlineto{\pgfqpoint{4.018123in}{1.078048in}}%
\pgfpathlineto{\pgfqpoint{4.006869in}{1.073511in}}%
\pgfpathlineto{\pgfqpoint{3.995484in}{1.069511in}}%
\pgfpathlineto{\pgfqpoint{3.984016in}{1.066066in}}%
\pgfpathlineto{\pgfqpoint{3.972610in}{1.077436in}}%
\pgfpathlineto{\pgfqpoint{3.959711in}{1.088443in}}%
\pgfpathlineto{\pgfqpoint{3.945364in}{1.099037in}}%
\pgfpathlineto{\pgfqpoint{3.929625in}{1.109172in}}%
\pgfpathlineto{\pgfqpoint{3.912553in}{1.118802in}}%
\pgfpathclose%
\pgfusepath{fill}%
\end{pgfscope}%
\begin{pgfscope}%
\pgfpathrectangle{\pgfqpoint{2.548318in}{0.050000in}}{\pgfqpoint{2.081932in}{2.081932in}}%
\pgfusepath{clip}%
\pgfsetbuttcap%
\pgfsetroundjoin%
\definecolor{currentfill}{rgb}{0.124780,0.640461,0.527068}%
\pgfsetfillcolor{currentfill}%
\pgfsetlinewidth{0.000000pt}%
\definecolor{currentstroke}{rgb}{0.000000,0.000000,0.000000}%
\pgfsetstrokecolor{currentstroke}%
\pgfsetdash{}{0pt}%
\pgfpathmoveto{\pgfqpoint{3.101700in}{1.270704in}}%
\pgfpathlineto{\pgfqpoint{3.102293in}{1.279562in}}%
\pgfpathlineto{\pgfqpoint{3.103591in}{1.288167in}}%
\pgfpathlineto{\pgfqpoint{3.105587in}{1.296484in}}%
\pgfpathlineto{\pgfqpoint{3.108272in}{1.304480in}}%
\pgfpathlineto{\pgfqpoint{3.111635in}{1.312122in}}%
\pgfpathlineto{\pgfqpoint{3.130336in}{1.328372in}}%
\pgfpathlineto{\pgfqpoint{3.151143in}{1.344011in}}%
\pgfpathlineto{\pgfqpoint{3.173978in}{1.358969in}}%
\pgfpathlineto{\pgfqpoint{3.198757in}{1.373175in}}%
\pgfpathlineto{\pgfqpoint{3.195948in}{1.365872in}}%
\pgfpathlineto{\pgfqpoint{3.193704in}{1.358129in}}%
\pgfpathlineto{\pgfqpoint{3.192036in}{1.349977in}}%
\pgfpathlineto{\pgfqpoint{3.190952in}{1.341449in}}%
\pgfpathlineto{\pgfqpoint{3.190456in}{1.332581in}}%
\pgfpathlineto{\pgfqpoint{3.165206in}{1.318182in}}%
\pgfpathlineto{\pgfqpoint{3.141939in}{1.303021in}}%
\pgfpathlineto{\pgfqpoint{3.120745in}{1.287171in}}%
\pgfpathlineto{\pgfqpoint{3.101700in}{1.270704in}}%
\pgfpathclose%
\pgfusepath{fill}%
\end{pgfscope}%
\begin{pgfscope}%
\pgfpathrectangle{\pgfqpoint{2.548318in}{0.050000in}}{\pgfqpoint{2.081932in}{2.081932in}}%
\pgfusepath{clip}%
\pgfsetbuttcap%
\pgfsetroundjoin%
\definecolor{currentfill}{rgb}{0.150476,0.504369,0.557430}%
\pgfsetfillcolor{currentfill}%
\pgfsetlinewidth{0.000000pt}%
\definecolor{currentstroke}{rgb}{0.000000,0.000000,0.000000}%
\pgfsetstrokecolor{currentstroke}%
\pgfsetdash{}{0pt}%
\pgfpathmoveto{\pgfqpoint{4.047050in}{1.290947in}}%
\pgfpathlineto{\pgfqpoint{4.048941in}{1.300781in}}%
\pgfpathlineto{\pgfqpoint{4.050238in}{1.310432in}}%
\pgfpathlineto{\pgfqpoint{4.050936in}{1.319861in}}%
\pgfpathlineto{\pgfqpoint{4.051034in}{1.329028in}}%
\pgfpathlineto{\pgfqpoint{4.075801in}{1.314430in}}%
\pgfpathlineto{\pgfqpoint{4.098560in}{1.299089in}}%
\pgfpathlineto{\pgfqpoint{4.119227in}{1.283076in}}%
\pgfpathlineto{\pgfqpoint{4.137727in}{1.266467in}}%
\pgfpathlineto{\pgfqpoint{4.153997in}{1.249335in}}%
\pgfpathlineto{\pgfqpoint{4.153878in}{1.240295in}}%
\pgfpathlineto{\pgfqpoint{4.153024in}{1.231105in}}%
\pgfpathlineto{\pgfqpoint{4.151437in}{1.221802in}}%
\pgfpathlineto{\pgfqpoint{4.149124in}{1.212426in}}%
\pgfpathlineto{\pgfqpoint{4.132990in}{1.229307in}}%
\pgfpathlineto{\pgfqpoint{4.114647in}{1.245673in}}%
\pgfpathlineto{\pgfqpoint{4.094159in}{1.261450in}}%
\pgfpathlineto{\pgfqpoint{4.071598in}{1.276565in}}%
\pgfpathlineto{\pgfqpoint{4.047050in}{1.290947in}}%
\pgfpathclose%
\pgfusepath{fill}%
\end{pgfscope}%
\begin{pgfscope}%
\pgfpathrectangle{\pgfqpoint{2.548318in}{0.050000in}}{\pgfqpoint{2.081932in}{2.081932in}}%
\pgfusepath{clip}%
\pgfsetbuttcap%
\pgfsetroundjoin%
\definecolor{currentfill}{rgb}{0.268510,0.009605,0.335427}%
\pgfsetfillcolor{currentfill}%
\pgfsetlinewidth{0.000000pt}%
\definecolor{currentstroke}{rgb}{0.000000,0.000000,0.000000}%
\pgfsetstrokecolor{currentstroke}%
\pgfsetdash{}{0pt}%
\pgfpathmoveto{\pgfqpoint{3.265499in}{1.080246in}}%
\pgfpathlineto{\pgfqpoint{3.254458in}{1.084144in}}%
\pgfpathlineto{\pgfqpoint{3.243497in}{1.088594in}}%
\pgfpathlineto{\pgfqpoint{3.232659in}{1.093580in}}%
\pgfpathlineto{\pgfqpoint{3.221992in}{1.099080in}}%
\pgfpathlineto{\pgfqpoint{3.211537in}{1.105073in}}%
\pgfpathlineto{\pgfqpoint{3.226733in}{1.117704in}}%
\pgfpathlineto{\pgfqpoint{3.243595in}{1.129849in}}%
\pgfpathlineto{\pgfqpoint{3.262061in}{1.141455in}}%
\pgfpathlineto{\pgfqpoint{3.282059in}{1.152467in}}%
\pgfpathlineto{\pgfqpoint{3.290759in}{1.145202in}}%
\pgfpathlineto{\pgfqpoint{3.299633in}{1.138411in}}%
\pgfpathlineto{\pgfqpoint{3.308645in}{1.132121in}}%
\pgfpathlineto{\pgfqpoint{3.317756in}{1.126358in}}%
\pgfpathlineto{\pgfqpoint{3.326929in}{1.121143in}}%
\pgfpathlineto{\pgfqpoint{3.309537in}{1.111648in}}%
\pgfpathlineto{\pgfqpoint{3.293460in}{1.101636in}}%
\pgfpathlineto{\pgfqpoint{3.278762in}{1.091153in}}%
\pgfpathlineto{\pgfqpoint{3.265499in}{1.080246in}}%
\pgfpathclose%
\pgfusepath{fill}%
\end{pgfscope}%
\begin{pgfscope}%
\pgfpathrectangle{\pgfqpoint{2.548318in}{0.050000in}}{\pgfqpoint{2.081932in}{2.081932in}}%
\pgfusepath{clip}%
\pgfsetbuttcap%
\pgfsetroundjoin%
\definecolor{currentfill}{rgb}{0.636902,0.856542,0.216620}%
\pgfsetfillcolor{currentfill}%
\pgfsetlinewidth{0.000000pt}%
\definecolor{currentstroke}{rgb}{0.000000,0.000000,0.000000}%
\pgfsetstrokecolor{currentstroke}%
\pgfsetdash{}{0pt}%
\pgfpathmoveto{\pgfqpoint{3.880565in}{1.454653in}}%
\pgfpathlineto{\pgfqpoint{3.876652in}{1.458028in}}%
\pgfpathlineto{\pgfqpoint{3.872451in}{1.460743in}}%
\pgfpathlineto{\pgfqpoint{3.867981in}{1.462787in}}%
\pgfpathlineto{\pgfqpoint{3.863260in}{1.464155in}}%
\pgfpathlineto{\pgfqpoint{3.858308in}{1.464844in}}%
\pgfpathlineto{\pgfqpoint{3.886547in}{1.456325in}}%
\pgfpathlineto{\pgfqpoint{3.913592in}{1.446894in}}%
\pgfpathlineto{\pgfqpoint{3.939331in}{1.436598in}}%
\pgfpathlineto{\pgfqpoint{3.963655in}{1.425486in}}%
\pgfpathlineto{\pgfqpoint{3.986465in}{1.413610in}}%
\pgfpathlineto{\pgfqpoint{3.993994in}{1.411881in}}%
\pgfpathlineto{\pgfqpoint{4.001170in}{1.409527in}}%
\pgfpathlineto{\pgfqpoint{4.007963in}{1.406556in}}%
\pgfpathlineto{\pgfqpoint{4.014345in}{1.402978in}}%
\pgfpathlineto{\pgfqpoint{4.020289in}{1.398805in}}%
\pgfpathlineto{\pgfqpoint{3.995448in}{1.411744in}}%
\pgfpathlineto{\pgfqpoint{3.968943in}{1.423855in}}%
\pgfpathlineto{\pgfqpoint{3.940881in}{1.435078in}}%
\pgfpathlineto{\pgfqpoint{3.911380in}{1.445361in}}%
\pgfpathlineto{\pgfqpoint{3.880565in}{1.454653in}}%
\pgfpathclose%
\pgfusepath{fill}%
\end{pgfscope}%
\begin{pgfscope}%
\pgfpathrectangle{\pgfqpoint{2.548318in}{0.050000in}}{\pgfqpoint{2.081932in}{2.081932in}}%
\pgfusepath{clip}%
\pgfsetbuttcap%
\pgfsetroundjoin%
\definecolor{currentfill}{rgb}{0.636902,0.856542,0.216620}%
\pgfsetfillcolor{currentfill}%
\pgfsetlinewidth{0.000000pt}%
\definecolor{currentstroke}{rgb}{0.000000,0.000000,0.000000}%
\pgfsetstrokecolor{currentstroke}%
\pgfsetdash{}{0pt}%
\pgfpathmoveto{\pgfqpoint{3.220759in}{1.402141in}}%
\pgfpathlineto{\pgfqpoint{3.226616in}{1.406265in}}%
\pgfpathlineto{\pgfqpoint{3.232906in}{1.409793in}}%
\pgfpathlineto{\pgfqpoint{3.239601in}{1.412709in}}%
\pgfpathlineto{\pgfqpoint{3.246673in}{1.415005in}}%
\pgfpathlineto{\pgfqpoint{3.254093in}{1.416671in}}%
\pgfpathlineto{\pgfqpoint{3.277288in}{1.428360in}}%
\pgfpathlineto{\pgfqpoint{3.301973in}{1.439272in}}%
\pgfpathlineto{\pgfqpoint{3.328047in}{1.449354in}}%
\pgfpathlineto{\pgfqpoint{3.355400in}{1.458559in}}%
\pgfpathlineto{\pgfqpoint{3.383916in}{1.466842in}}%
\pgfpathlineto{\pgfqpoint{3.379111in}{1.466194in}}%
\pgfpathlineto{\pgfqpoint{3.374529in}{1.464865in}}%
\pgfpathlineto{\pgfqpoint{3.370191in}{1.462857in}}%
\pgfpathlineto{\pgfqpoint{3.366114in}{1.460176in}}%
\pgfpathlineto{\pgfqpoint{3.362316in}{1.456832in}}%
\pgfpathlineto{\pgfqpoint{3.331196in}{1.447798in}}%
\pgfpathlineto{\pgfqpoint{3.301356in}{1.437760in}}%
\pgfpathlineto{\pgfqpoint{3.272925in}{1.426769in}}%
\pgfpathlineto{\pgfqpoint{3.246023in}{1.414876in}}%
\pgfpathlineto{\pgfqpoint{3.220759in}{1.402141in}}%
\pgfpathclose%
\pgfusepath{fill}%
\end{pgfscope}%
\begin{pgfscope}%
\pgfpathrectangle{\pgfqpoint{2.548318in}{0.050000in}}{\pgfqpoint{2.081932in}{2.081932in}}%
\pgfusepath{clip}%
\pgfsetbuttcap%
\pgfsetroundjoin%
\definecolor{currentfill}{rgb}{0.282327,0.094955,0.417331}%
\pgfsetfillcolor{currentfill}%
\pgfsetlinewidth{0.000000pt}%
\definecolor{currentstroke}{rgb}{0.000000,0.000000,0.000000}%
\pgfsetstrokecolor{currentstroke}%
\pgfsetdash{}{0pt}%
\pgfpathmoveto{\pgfqpoint{3.958089in}{1.149751in}}%
\pgfpathlineto{\pgfqpoint{3.966703in}{1.157388in}}%
\pgfpathlineto{\pgfqpoint{3.975067in}{1.165440in}}%
\pgfpathlineto{\pgfqpoint{3.983146in}{1.173873in}}%
\pgfpathlineto{\pgfqpoint{3.990905in}{1.182653in}}%
\pgfpathlineto{\pgfqpoint{3.998312in}{1.191743in}}%
\pgfpathlineto{\pgfqpoint{4.020173in}{1.179179in}}%
\pgfpathlineto{\pgfqpoint{4.040287in}{1.165969in}}%
\pgfpathlineto{\pgfqpoint{4.058579in}{1.152173in}}%
\pgfpathlineto{\pgfqpoint{4.074981in}{1.137853in}}%
\pgfpathlineto{\pgfqpoint{4.089434in}{1.123075in}}%
\pgfpathlineto{\pgfqpoint{4.080349in}{1.115401in}}%
\pgfpathlineto{\pgfqpoint{4.070828in}{1.108092in}}%
\pgfpathlineto{\pgfqpoint{4.060910in}{1.101181in}}%
\pgfpathlineto{\pgfqpoint{4.050637in}{1.094696in}}%
\pgfpathlineto{\pgfqpoint{4.040052in}{1.088662in}}%
\pgfpathlineto{\pgfqpoint{4.027014in}{1.101820in}}%
\pgfpathlineto{\pgfqpoint{4.012241in}{1.114565in}}%
\pgfpathlineto{\pgfqpoint{3.995787in}{1.126838in}}%
\pgfpathlineto{\pgfqpoint{3.977713in}{1.138584in}}%
\pgfpathlineto{\pgfqpoint{3.958089in}{1.149751in}}%
\pgfpathclose%
\pgfusepath{fill}%
\end{pgfscope}%
\begin{pgfscope}%
\pgfpathrectangle{\pgfqpoint{2.548318in}{0.050000in}}{\pgfqpoint{2.081932in}{2.081932in}}%
\pgfusepath{clip}%
\pgfsetbuttcap%
\pgfsetroundjoin%
\definecolor{currentfill}{rgb}{0.206756,0.371758,0.553117}%
\pgfsetfillcolor{currentfill}%
\pgfsetlinewidth{0.000000pt}%
\definecolor{currentstroke}{rgb}{0.000000,0.000000,0.000000}%
\pgfsetstrokecolor{currentstroke}%
\pgfsetdash{}{0pt}%
\pgfpathmoveto{\pgfqpoint{4.029032in}{1.240489in}}%
\pgfpathlineto{\pgfqpoint{4.033732in}{1.250612in}}%
\pgfpathlineto{\pgfqpoint{4.037899in}{1.260761in}}%
\pgfpathlineto{\pgfqpoint{4.041518in}{1.270895in}}%
\pgfpathlineto{\pgfqpoint{4.044572in}{1.280971in}}%
\pgfpathlineto{\pgfqpoint{4.047050in}{1.290947in}}%
\pgfpathlineto{\pgfqpoint{4.071598in}{1.276565in}}%
\pgfpathlineto{\pgfqpoint{4.094159in}{1.261450in}}%
\pgfpathlineto{\pgfqpoint{4.114647in}{1.245673in}}%
\pgfpathlineto{\pgfqpoint{4.132990in}{1.229307in}}%
\pgfpathlineto{\pgfqpoint{4.149124in}{1.212426in}}%
\pgfpathlineto{\pgfqpoint{4.146094in}{1.203015in}}%
\pgfpathlineto{\pgfqpoint{4.142358in}{1.193607in}}%
\pgfpathlineto{\pgfqpoint{4.137930in}{1.184242in}}%
\pgfpathlineto{\pgfqpoint{4.132829in}{1.174958in}}%
\pgfpathlineto{\pgfqpoint{4.127076in}{1.165794in}}%
\pgfpathlineto{\pgfqpoint{4.111559in}{1.181859in}}%
\pgfpathlineto{\pgfqpoint{4.093929in}{1.197430in}}%
\pgfpathlineto{\pgfqpoint{4.074249in}{1.212438in}}%
\pgfpathlineto{\pgfqpoint{4.052589in}{1.226813in}}%
\pgfpathlineto{\pgfqpoint{4.029032in}{1.240489in}}%
\pgfpathclose%
\pgfusepath{fill}%
\end{pgfscope}%
\begin{pgfscope}%
\pgfpathrectangle{\pgfqpoint{2.548318in}{0.050000in}}{\pgfqpoint{2.081932in}{2.081932in}}%
\pgfusepath{clip}%
\pgfsetbuttcap%
\pgfsetroundjoin%
\definecolor{currentfill}{rgb}{0.876168,0.891125,0.095250}%
\pgfsetfillcolor{currentfill}%
\pgfsetlinewidth{0.000000pt}%
\definecolor{currentstroke}{rgb}{0.000000,0.000000,0.000000}%
\pgfsetstrokecolor{currentstroke}%
\pgfsetdash{}{0pt}%
\pgfpathmoveto{\pgfqpoint{3.703671in}{1.492484in}}%
\pgfpathlineto{\pgfqpoint{3.701817in}{1.491896in}}%
\pgfpathlineto{\pgfqpoint{3.699895in}{1.490606in}}%
\pgfpathlineto{\pgfqpoint{3.697915in}{1.488620in}}%
\pgfpathlineto{\pgfqpoint{3.695883in}{1.485949in}}%
\pgfpathlineto{\pgfqpoint{3.693808in}{1.482605in}}%
\pgfpathlineto{\pgfqpoint{3.722390in}{1.479568in}}%
\pgfpathlineto{\pgfqpoint{3.750506in}{1.475582in}}%
\pgfpathlineto{\pgfqpoint{3.778033in}{1.470668in}}%
\pgfpathlineto{\pgfqpoint{3.804849in}{1.464849in}}%
\pgfpathlineto{\pgfqpoint{3.830839in}{1.458153in}}%
\pgfpathlineto{\pgfqpoint{3.836620in}{1.460822in}}%
\pgfpathlineto{\pgfqpoint{3.842280in}{1.462834in}}%
\pgfpathlineto{\pgfqpoint{3.847797in}{1.464180in}}%
\pgfpathlineto{\pgfqpoint{3.853147in}{1.464851in}}%
\pgfpathlineto{\pgfqpoint{3.858308in}{1.464844in}}%
\pgfpathlineto{\pgfqpoint{3.828999in}{1.472410in}}%
\pgfpathlineto{\pgfqpoint{3.798745in}{1.478987in}}%
\pgfpathlineto{\pgfqpoint{3.767680in}{1.484543in}}%
\pgfpathlineto{\pgfqpoint{3.735942in}{1.489049in}}%
\pgfpathlineto{\pgfqpoint{3.703671in}{1.492484in}}%
\pgfpathclose%
\pgfusepath{fill}%
\end{pgfscope}%
\begin{pgfscope}%
\pgfpathrectangle{\pgfqpoint{2.548318in}{0.050000in}}{\pgfqpoint{2.081932in}{2.081932in}}%
\pgfusepath{clip}%
\pgfsetbuttcap%
\pgfsetroundjoin%
\definecolor{currentfill}{rgb}{0.267968,0.223549,0.512008}%
\pgfsetfillcolor{currentfill}%
\pgfsetlinewidth{0.000000pt}%
\definecolor{currentstroke}{rgb}{0.000000,0.000000,0.000000}%
\pgfsetstrokecolor{currentstroke}%
\pgfsetdash{}{0pt}%
\pgfpathmoveto{\pgfqpoint{3.998312in}{1.191743in}}%
\pgfpathlineto{\pgfqpoint{4.005335in}{1.201106in}}%
\pgfpathlineto{\pgfqpoint{4.011946in}{1.210702in}}%
\pgfpathlineto{\pgfqpoint{4.018116in}{1.220492in}}%
\pgfpathlineto{\pgfqpoint{4.023819in}{1.230435in}}%
\pgfpathlineto{\pgfqpoint{4.029032in}{1.240489in}}%
\pgfpathlineto{\pgfqpoint{4.052589in}{1.226813in}}%
\pgfpathlineto{\pgfqpoint{4.074249in}{1.212438in}}%
\pgfpathlineto{\pgfqpoint{4.093929in}{1.197430in}}%
\pgfpathlineto{\pgfqpoint{4.111559in}{1.181859in}}%
\pgfpathlineto{\pgfqpoint{4.127076in}{1.165794in}}%
\pgfpathlineto{\pgfqpoint{4.120694in}{1.156788in}}%
\pgfpathlineto{\pgfqpoint{4.113708in}{1.147977in}}%
\pgfpathlineto{\pgfqpoint{4.106149in}{1.139397in}}%
\pgfpathlineto{\pgfqpoint{4.098046in}{1.131086in}}%
\pgfpathlineto{\pgfqpoint{4.089434in}{1.123075in}}%
\pgfpathlineto{\pgfqpoint{4.074981in}{1.137853in}}%
\pgfpathlineto{\pgfqpoint{4.058579in}{1.152173in}}%
\pgfpathlineto{\pgfqpoint{4.040287in}{1.165969in}}%
\pgfpathlineto{\pgfqpoint{4.020173in}{1.179179in}}%
\pgfpathlineto{\pgfqpoint{3.998312in}{1.191743in}}%
\pgfpathclose%
\pgfusepath{fill}%
\end{pgfscope}%
\begin{pgfscope}%
\pgfpathrectangle{\pgfqpoint{2.548318in}{0.050000in}}{\pgfqpoint{2.081932in}{2.081932in}}%
\pgfusepath{clip}%
\pgfsetbuttcap%
\pgfsetroundjoin%
\definecolor{currentfill}{rgb}{0.150476,0.504369,0.557430}%
\pgfsetfillcolor{currentfill}%
\pgfsetlinewidth{0.000000pt}%
\definecolor{currentstroke}{rgb}{0.000000,0.000000,0.000000}%
\pgfsetstrokecolor{currentstroke}%
\pgfsetdash{}{0pt}%
\pgfpathmoveto{\pgfqpoint{3.106400in}{1.233482in}}%
\pgfpathlineto{\pgfqpoint{3.104169in}{1.242981in}}%
\pgfpathlineto{\pgfqpoint{3.102639in}{1.252376in}}%
\pgfpathlineto{\pgfqpoint{3.101815in}{1.261630in}}%
\pgfpathlineto{\pgfqpoint{3.101700in}{1.270704in}}%
\pgfpathlineto{\pgfqpoint{3.120745in}{1.287171in}}%
\pgfpathlineto{\pgfqpoint{3.141939in}{1.303021in}}%
\pgfpathlineto{\pgfqpoint{3.165206in}{1.318182in}}%
\pgfpathlineto{\pgfqpoint{3.190456in}{1.332581in}}%
\pgfpathlineto{\pgfqpoint{3.190553in}{1.323408in}}%
\pgfpathlineto{\pgfqpoint{3.191241in}{1.313969in}}%
\pgfpathlineto{\pgfqpoint{3.192520in}{1.304302in}}%
\pgfpathlineto{\pgfqpoint{3.194384in}{1.294447in}}%
\pgfpathlineto{\pgfqpoint{3.169356in}{1.280261in}}%
\pgfpathlineto{\pgfqpoint{3.146293in}{1.265324in}}%
\pgfpathlineto{\pgfqpoint{3.125282in}{1.249707in}}%
\pgfpathlineto{\pgfqpoint{3.106400in}{1.233482in}}%
\pgfpathclose%
\pgfusepath{fill}%
\end{pgfscope}%
\begin{pgfscope}%
\pgfpathrectangle{\pgfqpoint{2.548318in}{0.050000in}}{\pgfqpoint{2.081932in}{2.081932in}}%
\pgfusepath{clip}%
\pgfsetbuttcap%
\pgfsetroundjoin%
\definecolor{currentfill}{rgb}{0.876168,0.891125,0.095250}%
\pgfsetfillcolor{currentfill}%
\pgfsetlinewidth{0.000000pt}%
\definecolor{currentstroke}{rgb}{0.000000,0.000000,0.000000}%
\pgfsetstrokecolor{currentstroke}%
\pgfsetdash{}{0pt}%
\pgfpathmoveto{\pgfqpoint{3.383916in}{1.466842in}}%
\pgfpathlineto{\pgfqpoint{3.388925in}{1.466806in}}%
\pgfpathlineto{\pgfqpoint{3.394117in}{1.466090in}}%
\pgfpathlineto{\pgfqpoint{3.399470in}{1.464699in}}%
\pgfpathlineto{\pgfqpoint{3.404963in}{1.462639in}}%
\pgfpathlineto{\pgfqpoint{3.410573in}{1.459921in}}%
\pgfpathlineto{\pgfqpoint{3.436778in}{1.466399in}}%
\pgfpathlineto{\pgfqpoint{3.463782in}{1.471993in}}%
\pgfpathlineto{\pgfqpoint{3.491467in}{1.476675in}}%
\pgfpathlineto{\pgfqpoint{3.519711in}{1.480423in}}%
\pgfpathlineto{\pgfqpoint{3.548389in}{1.483219in}}%
\pgfpathlineto{\pgfqpoint{3.546509in}{1.486580in}}%
\pgfpathlineto{\pgfqpoint{3.544667in}{1.489268in}}%
\pgfpathlineto{\pgfqpoint{3.542872in}{1.491270in}}%
\pgfpathlineto{\pgfqpoint{3.541131in}{1.492576in}}%
\pgfpathlineto{\pgfqpoint{3.539450in}{1.493178in}}%
\pgfpathlineto{\pgfqpoint{3.507069in}{1.490016in}}%
\pgfpathlineto{\pgfqpoint{3.475185in}{1.485778in}}%
\pgfpathlineto{\pgfqpoint{3.443940in}{1.480485in}}%
\pgfpathlineto{\pgfqpoint{3.413472in}{1.474162in}}%
\pgfpathlineto{\pgfqpoint{3.383916in}{1.466842in}}%
\pgfpathclose%
\pgfusepath{fill}%
\end{pgfscope}%
\begin{pgfscope}%
\pgfpathrectangle{\pgfqpoint{2.548318in}{0.050000in}}{\pgfqpoint{2.081932in}{2.081932in}}%
\pgfusepath{clip}%
\pgfsetbuttcap%
\pgfsetroundjoin%
\definecolor{currentfill}{rgb}{0.278791,0.062145,0.386592}%
\pgfsetfillcolor{currentfill}%
\pgfsetlinewidth{0.000000pt}%
\definecolor{currentstroke}{rgb}{0.000000,0.000000,0.000000}%
\pgfsetstrokecolor{currentstroke}%
\pgfsetdash{}{0pt}%
\pgfpathmoveto{\pgfqpoint{3.665560in}{1.142749in}}%
\pgfpathlineto{\pgfqpoint{3.667404in}{1.143314in}}%
\pgfpathlineto{\pgfqpoint{3.669319in}{1.144579in}}%
\pgfpathlineto{\pgfqpoint{3.671299in}{1.146542in}}%
\pgfpathlineto{\pgfqpoint{3.673334in}{1.149198in}}%
\pgfpathlineto{\pgfqpoint{3.675417in}{1.152538in}}%
\pgfpathlineto{\pgfqpoint{3.697117in}{1.150343in}}%
\pgfpathlineto{\pgfqpoint{3.718471in}{1.147463in}}%
\pgfpathlineto{\pgfqpoint{3.739387in}{1.143911in}}%
\pgfpathlineto{\pgfqpoint{3.759776in}{1.139703in}}%
\pgfpathlineto{\pgfqpoint{3.779550in}{1.134859in}}%
\pgfpathlineto{\pgfqpoint{3.773736in}{1.132162in}}%
\pgfpathlineto{\pgfqpoint{3.768054in}{1.130133in}}%
\pgfpathlineto{\pgfqpoint{3.762527in}{1.128777in}}%
\pgfpathlineto{\pgfqpoint{3.757177in}{1.128097in}}%
\pgfpathlineto{\pgfqpoint{3.752025in}{1.128095in}}%
\pgfpathlineto{\pgfqpoint{3.735595in}{1.132112in}}%
\pgfpathlineto{\pgfqpoint{3.718661in}{1.135600in}}%
\pgfpathlineto{\pgfqpoint{3.701294in}{1.138544in}}%
\pgfpathlineto{\pgfqpoint{3.683568in}{1.140931in}}%
\pgfpathlineto{\pgfqpoint{3.665560in}{1.142749in}}%
\pgfpathclose%
\pgfusepath{fill}%
\end{pgfscope}%
\begin{pgfscope}%
\pgfpathrectangle{\pgfqpoint{2.548318in}{0.050000in}}{\pgfqpoint{2.081932in}{2.081932in}}%
\pgfusepath{clip}%
\pgfsetbuttcap%
\pgfsetroundjoin%
\definecolor{currentfill}{rgb}{0.282327,0.094955,0.417331}%
\pgfsetfillcolor{currentfill}%
\pgfsetlinewidth{0.000000pt}%
\definecolor{currentstroke}{rgb}{0.000000,0.000000,0.000000}%
\pgfsetstrokecolor{currentstroke}%
\pgfsetdash{}{0pt}%
\pgfpathmoveto{\pgfqpoint{3.211537in}{1.105073in}}%
\pgfpathlineto{\pgfqpoint{3.201340in}{1.111534in}}%
\pgfpathlineto{\pgfqpoint{3.191443in}{1.118438in}}%
\pgfpathlineto{\pgfqpoint{3.181886in}{1.125755in}}%
\pgfpathlineto{\pgfqpoint{3.172711in}{1.133455in}}%
\pgfpathlineto{\pgfqpoint{3.163954in}{1.141507in}}%
\pgfpathlineto{\pgfqpoint{3.180831in}{1.155701in}}%
\pgfpathlineto{\pgfqpoint{3.199582in}{1.169355in}}%
\pgfpathlineto{\pgfqpoint{3.220138in}{1.182408in}}%
\pgfpathlineto{\pgfqpoint{3.242420in}{1.194799in}}%
\pgfpathlineto{\pgfqpoint{3.249720in}{1.185646in}}%
\pgfpathlineto{\pgfqpoint{3.257367in}{1.176800in}}%
\pgfpathlineto{\pgfqpoint{3.265328in}{1.168298in}}%
\pgfpathlineto{\pgfqpoint{3.273570in}{1.160176in}}%
\pgfpathlineto{\pgfqpoint{3.282059in}{1.152467in}}%
\pgfpathlineto{\pgfqpoint{3.262061in}{1.141455in}}%
\pgfpathlineto{\pgfqpoint{3.243595in}{1.129849in}}%
\pgfpathlineto{\pgfqpoint{3.226733in}{1.117704in}}%
\pgfpathlineto{\pgfqpoint{3.211537in}{1.105073in}}%
\pgfpathclose%
\pgfusepath{fill}%
\end{pgfscope}%
\begin{pgfscope}%
\pgfpathrectangle{\pgfqpoint{2.548318in}{0.050000in}}{\pgfqpoint{2.081932in}{2.081932in}}%
\pgfusepath{clip}%
\pgfsetbuttcap%
\pgfsetroundjoin%
\definecolor{currentfill}{rgb}{0.278791,0.062145,0.386592}%
\pgfsetfillcolor{currentfill}%
\pgfsetlinewidth{0.000000pt}%
\definecolor{currentstroke}{rgb}{0.000000,0.000000,0.000000}%
\pgfsetstrokecolor{currentstroke}%
\pgfsetdash{}{0pt}%
\pgfpathmoveto{\pgfqpoint{3.487049in}{1.129156in}}%
\pgfpathlineto{\pgfqpoint{3.482051in}{1.129199in}}%
\pgfpathlineto{\pgfqpoint{3.476860in}{1.129921in}}%
\pgfpathlineto{\pgfqpoint{3.471498in}{1.131321in}}%
\pgfpathlineto{\pgfqpoint{3.465985in}{1.133395in}}%
\pgfpathlineto{\pgfqpoint{3.460343in}{1.136138in}}%
\pgfpathlineto{\pgfqpoint{3.480277in}{1.140824in}}%
\pgfpathlineto{\pgfqpoint{3.500806in}{1.144869in}}%
\pgfpathlineto{\pgfqpoint{3.521839in}{1.148253in}}%
\pgfpathlineto{\pgfqpoint{3.543288in}{1.150961in}}%
\pgfpathlineto{\pgfqpoint{3.565059in}{1.152981in}}%
\pgfpathlineto{\pgfqpoint{3.566947in}{1.149625in}}%
\pgfpathlineto{\pgfqpoint{3.568791in}{1.146954in}}%
\pgfpathlineto{\pgfqpoint{3.570585in}{1.144975in}}%
\pgfpathlineto{\pgfqpoint{3.572321in}{1.143695in}}%
\pgfpathlineto{\pgfqpoint{3.573992in}{1.143117in}}%
\pgfpathlineto{\pgfqpoint{3.555925in}{1.141443in}}%
\pgfpathlineto{\pgfqpoint{3.538122in}{1.139199in}}%
\pgfpathlineto{\pgfqpoint{3.520659in}{1.136394in}}%
\pgfpathlineto{\pgfqpoint{3.503611in}{1.133042in}}%
\pgfpathlineto{\pgfqpoint{3.487049in}{1.129156in}}%
\pgfpathclose%
\pgfusepath{fill}%
\end{pgfscope}%
\begin{pgfscope}%
\pgfpathrectangle{\pgfqpoint{2.548318in}{0.050000in}}{\pgfqpoint{2.081932in}{2.081932in}}%
\pgfusepath{clip}%
\pgfsetbuttcap%
\pgfsetroundjoin%
\definecolor{currentfill}{rgb}{0.206756,0.371758,0.553117}%
\pgfsetfillcolor{currentfill}%
\pgfsetlinewidth{0.000000pt}%
\definecolor{currentstroke}{rgb}{0.000000,0.000000,0.000000}%
\pgfsetstrokecolor{currentstroke}%
\pgfsetdash{}{0pt}%
\pgfpathmoveto{\pgfqpoint{3.127664in}{1.185832in}}%
\pgfpathlineto{\pgfqpoint{3.122116in}{1.195251in}}%
\pgfpathlineto{\pgfqpoint{3.117197in}{1.204765in}}%
\pgfpathlineto{\pgfqpoint{3.112927in}{1.214335in}}%
\pgfpathlineto{\pgfqpoint{3.109323in}{1.223921in}}%
\pgfpathlineto{\pgfqpoint{3.106400in}{1.233482in}}%
\pgfpathlineto{\pgfqpoint{3.125282in}{1.249707in}}%
\pgfpathlineto{\pgfqpoint{3.146293in}{1.265324in}}%
\pgfpathlineto{\pgfqpoint{3.169356in}{1.280261in}}%
\pgfpathlineto{\pgfqpoint{3.194384in}{1.294447in}}%
\pgfpathlineto{\pgfqpoint{3.196826in}{1.284446in}}%
\pgfpathlineto{\pgfqpoint{3.199836in}{1.274340in}}%
\pgfpathlineto{\pgfqpoint{3.203403in}{1.264172in}}%
\pgfpathlineto{\pgfqpoint{3.207511in}{1.253983in}}%
\pgfpathlineto{\pgfqpoint{3.212143in}{1.243817in}}%
\pgfpathlineto{\pgfqpoint{3.188128in}{1.230328in}}%
\pgfpathlineto{\pgfqpoint{3.165989in}{1.216123in}}%
\pgfpathlineto{\pgfqpoint{3.145809in}{1.201268in}}%
\pgfpathlineto{\pgfqpoint{3.127664in}{1.185832in}}%
\pgfpathclose%
\pgfusepath{fill}%
\end{pgfscope}%
\begin{pgfscope}%
\pgfpathrectangle{\pgfqpoint{2.548318in}{0.050000in}}{\pgfqpoint{2.081932in}{2.081932in}}%
\pgfusepath{clip}%
\pgfsetbuttcap%
\pgfsetroundjoin%
\definecolor{currentfill}{rgb}{0.267004,0.004874,0.329415}%
\pgfsetfillcolor{currentfill}%
\pgfsetlinewidth{0.000000pt}%
\definecolor{currentstroke}{rgb}{0.000000,0.000000,0.000000}%
\pgfsetstrokecolor{currentstroke}%
\pgfsetdash{}{0pt}%
\pgfpathmoveto{\pgfqpoint{3.779550in}{1.134859in}}%
\pgfpathlineto{\pgfqpoint{3.785475in}{1.138213in}}%
\pgfpathlineto{\pgfqpoint{3.791484in}{1.142214in}}%
\pgfpathlineto{\pgfqpoint{3.797555in}{1.146846in}}%
\pgfpathlineto{\pgfqpoint{3.803662in}{1.152093in}}%
\pgfpathlineto{\pgfqpoint{3.809780in}{1.157934in}}%
\pgfpathlineto{\pgfqpoint{3.832381in}{1.151435in}}%
\pgfpathlineto{\pgfqpoint{3.854047in}{1.144238in}}%
\pgfpathlineto{\pgfqpoint{3.874687in}{1.136375in}}%
\pgfpathlineto{\pgfqpoint{3.894215in}{1.127883in}}%
\pgfpathlineto{\pgfqpoint{3.912553in}{1.118802in}}%
\pgfpathlineto{\pgfqpoint{3.903218in}{1.114233in}}%
\pgfpathlineto{\pgfqpoint{3.893897in}{1.110250in}}%
\pgfpathlineto{\pgfqpoint{3.884628in}{1.106871in}}%
\pgfpathlineto{\pgfqpoint{3.875449in}{1.104105in}}%
\pgfpathlineto{\pgfqpoint{3.866398in}{1.101965in}}%
\pgfpathlineto{\pgfqpoint{3.850878in}{1.109604in}}%
\pgfpathlineto{\pgfqpoint{3.834364in}{1.116744in}}%
\pgfpathlineto{\pgfqpoint{3.816923in}{1.123353in}}%
\pgfpathlineto{\pgfqpoint{3.798626in}{1.129400in}}%
\pgfpathlineto{\pgfqpoint{3.779550in}{1.134859in}}%
\pgfpathclose%
\pgfusepath{fill}%
\end{pgfscope}%
\begin{pgfscope}%
\pgfpathrectangle{\pgfqpoint{2.548318in}{0.050000in}}{\pgfqpoint{2.081932in}{2.081932in}}%
\pgfusepath{clip}%
\pgfsetbuttcap%
\pgfsetroundjoin%
\definecolor{currentfill}{rgb}{0.267968,0.223549,0.512008}%
\pgfsetfillcolor{currentfill}%
\pgfsetlinewidth{0.000000pt}%
\definecolor{currentstroke}{rgb}{0.000000,0.000000,0.000000}%
\pgfsetstrokecolor{currentstroke}%
\pgfsetdash{}{0pt}%
\pgfpathmoveto{\pgfqpoint{3.163954in}{1.141507in}}%
\pgfpathlineto{\pgfqpoint{3.155653in}{1.149878in}}%
\pgfpathlineto{\pgfqpoint{3.147842in}{1.158532in}}%
\pgfpathlineto{\pgfqpoint{3.140554in}{1.167434in}}%
\pgfpathlineto{\pgfqpoint{3.133819in}{1.176546in}}%
\pgfpathlineto{\pgfqpoint{3.127664in}{1.185832in}}%
\pgfpathlineto{\pgfqpoint{3.145809in}{1.201268in}}%
\pgfpathlineto{\pgfqpoint{3.165989in}{1.216123in}}%
\pgfpathlineto{\pgfqpoint{3.188128in}{1.230328in}}%
\pgfpathlineto{\pgfqpoint{3.212143in}{1.243817in}}%
\pgfpathlineto{\pgfqpoint{3.217281in}{1.233716in}}%
\pgfpathlineto{\pgfqpoint{3.222902in}{1.223722in}}%
\pgfpathlineto{\pgfqpoint{3.228983in}{1.213877in}}%
\pgfpathlineto{\pgfqpoint{3.235498in}{1.204223in}}%
\pgfpathlineto{\pgfqpoint{3.242420in}{1.194799in}}%
\pgfpathlineto{\pgfqpoint{3.220138in}{1.182408in}}%
\pgfpathlineto{\pgfqpoint{3.199582in}{1.169355in}}%
\pgfpathlineto{\pgfqpoint{3.180831in}{1.155701in}}%
\pgfpathlineto{\pgfqpoint{3.163954in}{1.141507in}}%
\pgfpathclose%
\pgfusepath{fill}%
\end{pgfscope}%
\begin{pgfscope}%
\pgfpathrectangle{\pgfqpoint{2.548318in}{0.050000in}}{\pgfqpoint{2.081932in}{2.081932in}}%
\pgfusepath{clip}%
\pgfsetbuttcap%
\pgfsetroundjoin%
\definecolor{currentfill}{rgb}{0.267004,0.004874,0.329415}%
\pgfsetfillcolor{currentfill}%
\pgfsetlinewidth{0.000000pt}%
\definecolor{currentstroke}{rgb}{0.000000,0.000000,0.000000}%
\pgfsetstrokecolor{currentstroke}%
\pgfsetdash{}{0pt}%
\pgfpathmoveto{\pgfqpoint{3.372404in}{1.103935in}}%
\pgfpathlineto{\pgfqpoint{3.363486in}{1.106148in}}%
\pgfpathlineto{\pgfqpoint{3.354443in}{1.108987in}}%
\pgfpathlineto{\pgfqpoint{3.345311in}{1.112441in}}%
\pgfpathlineto{\pgfqpoint{3.336127in}{1.116499in}}%
\pgfpathlineto{\pgfqpoint{3.326929in}{1.121143in}}%
\pgfpathlineto{\pgfqpoint{3.345570in}{1.130080in}}%
\pgfpathlineto{\pgfqpoint{3.365382in}{1.138417in}}%
\pgfpathlineto{\pgfqpoint{3.386285in}{1.146115in}}%
\pgfpathlineto{\pgfqpoint{3.408192in}{1.153140in}}%
\pgfpathlineto{\pgfqpoint{3.431010in}{1.159457in}}%
\pgfpathlineto{\pgfqpoint{3.436947in}{1.153567in}}%
\pgfpathlineto{\pgfqpoint{3.442873in}{1.148270in}}%
\pgfpathlineto{\pgfqpoint{3.448764in}{1.143589in}}%
\pgfpathlineto{\pgfqpoint{3.454595in}{1.139540in}}%
\pgfpathlineto{\pgfqpoint{3.460343in}{1.136138in}}%
\pgfpathlineto{\pgfqpoint{3.441086in}{1.130832in}}%
\pgfpathlineto{\pgfqpoint{3.422588in}{1.124931in}}%
\pgfpathlineto{\pgfqpoint{3.404928in}{1.118461in}}%
\pgfpathlineto{\pgfqpoint{3.388177in}{1.111452in}}%
\pgfpathlineto{\pgfqpoint{3.372404in}{1.103935in}}%
\pgfpathclose%
\pgfusepath{fill}%
\end{pgfscope}%
\begin{pgfscope}%
\pgfpathrectangle{\pgfqpoint{2.548318in}{0.050000in}}{\pgfqpoint{2.081932in}{2.081932in}}%
\pgfusepath{clip}%
\pgfsetbuttcap%
\pgfsetroundjoin%
\definecolor{currentfill}{rgb}{0.876168,0.891125,0.095250}%
\pgfsetfillcolor{currentfill}%
\pgfsetlinewidth{0.000000pt}%
\definecolor{currentstroke}{rgb}{0.000000,0.000000,0.000000}%
\pgfsetstrokecolor{currentstroke}%
\pgfsetdash{}{0pt}%
\pgfpathmoveto{\pgfqpoint{3.539450in}{1.493178in}}%
\pgfpathlineto{\pgfqpoint{3.541131in}{1.492576in}}%
\pgfpathlineto{\pgfqpoint{3.542872in}{1.491270in}}%
\pgfpathlineto{\pgfqpoint{3.544667in}{1.489268in}}%
\pgfpathlineto{\pgfqpoint{3.546509in}{1.486580in}}%
\pgfpathlineto{\pgfqpoint{3.548389in}{1.483219in}}%
\pgfpathlineto{\pgfqpoint{3.577375in}{1.485049in}}%
\pgfpathlineto{\pgfqpoint{3.606540in}{1.485905in}}%
\pgfpathlineto{\pgfqpoint{3.635754in}{1.485781in}}%
\pgfpathlineto{\pgfqpoint{3.664886in}{1.484679in}}%
\pgfpathlineto{\pgfqpoint{3.693808in}{1.482605in}}%
\pgfpathlineto{\pgfqpoint{3.695883in}{1.485949in}}%
\pgfpathlineto{\pgfqpoint{3.697915in}{1.488620in}}%
\pgfpathlineto{\pgfqpoint{3.699895in}{1.490606in}}%
\pgfpathlineto{\pgfqpoint{3.701817in}{1.491896in}}%
\pgfpathlineto{\pgfqpoint{3.703671in}{1.492484in}}%
\pgfpathlineto{\pgfqpoint{3.671012in}{1.494830in}}%
\pgfpathlineto{\pgfqpoint{3.638112in}{1.496076in}}%
\pgfpathlineto{\pgfqpoint{3.605119in}{1.496216in}}%
\pgfpathlineto{\pgfqpoint{3.572183in}{1.495248in}}%
\pgfpathlineto{\pgfqpoint{3.539450in}{1.493178in}}%
\pgfpathclose%
\pgfusepath{fill}%
\end{pgfscope}%
\begin{pgfscope}%
\pgfpathrectangle{\pgfqpoint{2.548318in}{0.050000in}}{\pgfqpoint{2.081932in}{2.081932in}}%
\pgfusepath{clip}%
\pgfsetbuttcap%
\pgfsetroundjoin%
\definecolor{currentfill}{rgb}{0.278791,0.062145,0.386592}%
\pgfsetfillcolor{currentfill}%
\pgfsetlinewidth{0.000000pt}%
\definecolor{currentstroke}{rgb}{0.000000,0.000000,0.000000}%
\pgfsetstrokecolor{currentstroke}%
\pgfsetdash{}{0pt}%
\pgfpathmoveto{\pgfqpoint{3.573992in}{1.143117in}}%
\pgfpathlineto{\pgfqpoint{3.572321in}{1.143695in}}%
\pgfpathlineto{\pgfqpoint{3.570585in}{1.144975in}}%
\pgfpathlineto{\pgfqpoint{3.568791in}{1.146954in}}%
\pgfpathlineto{\pgfqpoint{3.566947in}{1.149625in}}%
\pgfpathlineto{\pgfqpoint{3.565059in}{1.152981in}}%
\pgfpathlineto{\pgfqpoint{3.587058in}{1.154303in}}%
\pgfpathlineto{\pgfqpoint{3.609190in}{1.154921in}}%
\pgfpathlineto{\pgfqpoint{3.631358in}{1.154832in}}%
\pgfpathlineto{\pgfqpoint{3.653466in}{1.154036in}}%
\pgfpathlineto{\pgfqpoint{3.675417in}{1.152538in}}%
\pgfpathlineto{\pgfqpoint{3.673334in}{1.149198in}}%
\pgfpathlineto{\pgfqpoint{3.671299in}{1.146542in}}%
\pgfpathlineto{\pgfqpoint{3.669319in}{1.144579in}}%
\pgfpathlineto{\pgfqpoint{3.667404in}{1.143314in}}%
\pgfpathlineto{\pgfqpoint{3.665560in}{1.142749in}}%
\pgfpathlineto{\pgfqpoint{3.647345in}{1.143991in}}%
\pgfpathlineto{\pgfqpoint{3.629002in}{1.144650in}}%
\pgfpathlineto{\pgfqpoint{3.610609in}{1.144724in}}%
\pgfpathlineto{\pgfqpoint{3.592247in}{1.144212in}}%
\pgfpathlineto{\pgfqpoint{3.573992in}{1.143117in}}%
\pgfpathclose%
\pgfusepath{fill}%
\end{pgfscope}%
\begin{pgfscope}%
\pgfpathrectangle{\pgfqpoint{2.548318in}{0.050000in}}{\pgfqpoint{2.081932in}{2.081932in}}%
\pgfusepath{clip}%
\pgfsetbuttcap%
\pgfsetroundjoin%
\definecolor{currentfill}{rgb}{0.327796,0.773980,0.406640}%
\pgfsetfillcolor{currentfill}%
\pgfsetlinewidth{0.000000pt}%
\definecolor{currentstroke}{rgb}{0.000000,0.000000,0.000000}%
\pgfsetstrokecolor{currentstroke}%
\pgfsetdash{}{0pt}%
\pgfpathmoveto{\pgfqpoint{3.895271in}{1.428386in}}%
\pgfpathlineto{\pgfqpoint{3.893023in}{1.434828in}}%
\pgfpathlineto{\pgfqpoint{3.890417in}{1.440698in}}%
\pgfpathlineto{\pgfqpoint{3.887462in}{1.445972in}}%
\pgfpathlineto{\pgfqpoint{3.884174in}{1.450629in}}%
\pgfpathlineto{\pgfqpoint{3.880565in}{1.454653in}}%
\pgfpathlineto{\pgfqpoint{3.911380in}{1.445361in}}%
\pgfpathlineto{\pgfqpoint{3.940881in}{1.435078in}}%
\pgfpathlineto{\pgfqpoint{3.968943in}{1.423855in}}%
\pgfpathlineto{\pgfqpoint{3.995448in}{1.411744in}}%
\pgfpathlineto{\pgfqpoint{4.020289in}{1.398805in}}%
\pgfpathlineto{\pgfqpoint{4.025768in}{1.394055in}}%
\pgfpathlineto{\pgfqpoint{4.030761in}{1.388745in}}%
\pgfpathlineto{\pgfqpoint{4.035245in}{1.382895in}}%
\pgfpathlineto{\pgfqpoint{4.039202in}{1.376528in}}%
\pgfpathlineto{\pgfqpoint{4.042612in}{1.369670in}}%
\pgfpathlineto{\pgfqpoint{4.016437in}{1.383268in}}%
\pgfpathlineto{\pgfqpoint{3.988496in}{1.395999in}}%
\pgfpathlineto{\pgfqpoint{3.958904in}{1.407800in}}%
\pgfpathlineto{\pgfqpoint{3.927785in}{1.418613in}}%
\pgfpathlineto{\pgfqpoint{3.895271in}{1.428386in}}%
\pgfpathclose%
\pgfusepath{fill}%
\end{pgfscope}%
\begin{pgfscope}%
\pgfpathrectangle{\pgfqpoint{2.548318in}{0.050000in}}{\pgfqpoint{2.081932in}{2.081932in}}%
\pgfusepath{clip}%
\pgfsetbuttcap%
\pgfsetroundjoin%
\definecolor{currentfill}{rgb}{0.327796,0.773980,0.406640}%
\pgfsetfillcolor{currentfill}%
\pgfsetlinewidth{0.000000pt}%
\definecolor{currentstroke}{rgb}{0.000000,0.000000,0.000000}%
\pgfsetstrokecolor{currentstroke}%
\pgfsetdash{}{0pt}%
\pgfpathmoveto{\pgfqpoint{3.198757in}{1.373175in}}%
\pgfpathlineto{\pgfqpoint{3.202118in}{1.380008in}}%
\pgfpathlineto{\pgfqpoint{3.206017in}{1.386346in}}%
\pgfpathlineto{\pgfqpoint{3.210437in}{1.392162in}}%
\pgfpathlineto{\pgfqpoint{3.215358in}{1.397433in}}%
\pgfpathlineto{\pgfqpoint{3.220759in}{1.402141in}}%
\pgfpathlineto{\pgfqpoint{3.246023in}{1.414876in}}%
\pgfpathlineto{\pgfqpoint{3.272925in}{1.426769in}}%
\pgfpathlineto{\pgfqpoint{3.301356in}{1.437760in}}%
\pgfpathlineto{\pgfqpoint{3.331196in}{1.447798in}}%
\pgfpathlineto{\pgfqpoint{3.362316in}{1.456832in}}%
\pgfpathlineto{\pgfqpoint{3.358814in}{1.452837in}}%
\pgfpathlineto{\pgfqpoint{3.355622in}{1.448205in}}%
\pgfpathlineto{\pgfqpoint{3.352755in}{1.442954in}}%
\pgfpathlineto{\pgfqpoint{3.350225in}{1.437104in}}%
\pgfpathlineto{\pgfqpoint{3.348044in}{1.430678in}}%
\pgfpathlineto{\pgfqpoint{3.315205in}{1.421176in}}%
\pgfpathlineto{\pgfqpoint{3.283727in}{1.410620in}}%
\pgfpathlineto{\pgfqpoint{3.253743in}{1.399063in}}%
\pgfpathlineto{\pgfqpoint{3.225381in}{1.386561in}}%
\pgfpathlineto{\pgfqpoint{3.198757in}{1.373175in}}%
\pgfpathclose%
\pgfusepath{fill}%
\end{pgfscope}%
\begin{pgfscope}%
\pgfpathrectangle{\pgfqpoint{2.548318in}{0.050000in}}{\pgfqpoint{2.081932in}{2.081932in}}%
\pgfusepath{clip}%
\pgfsetbuttcap%
\pgfsetroundjoin%
\definecolor{currentfill}{rgb}{0.636902,0.856542,0.216620}%
\pgfsetfillcolor{currentfill}%
\pgfsetlinewidth{0.000000pt}%
\definecolor{currentstroke}{rgb}{0.000000,0.000000,0.000000}%
\pgfsetstrokecolor{currentstroke}%
\pgfsetdash{}{0pt}%
\pgfpathmoveto{\pgfqpoint{3.711669in}{1.484814in}}%
\pgfpathlineto{\pgfqpoint{3.710262in}{1.487753in}}%
\pgfpathlineto{\pgfqpoint{3.708753in}{1.489996in}}%
\pgfpathlineto{\pgfqpoint{3.707146in}{1.491534in}}%
\pgfpathlineto{\pgfqpoint{3.705450in}{1.492363in}}%
\pgfpathlineto{\pgfqpoint{3.703671in}{1.492484in}}%
\pgfpathlineto{\pgfqpoint{3.735942in}{1.489049in}}%
\pgfpathlineto{\pgfqpoint{3.767680in}{1.484543in}}%
\pgfpathlineto{\pgfqpoint{3.798745in}{1.478987in}}%
\pgfpathlineto{\pgfqpoint{3.828999in}{1.472410in}}%
\pgfpathlineto{\pgfqpoint{3.858308in}{1.464844in}}%
\pgfpathlineto{\pgfqpoint{3.863260in}{1.464155in}}%
\pgfpathlineto{\pgfqpoint{3.867981in}{1.462787in}}%
\pgfpathlineto{\pgfqpoint{3.872451in}{1.460743in}}%
\pgfpathlineto{\pgfqpoint{3.876652in}{1.458028in}}%
\pgfpathlineto{\pgfqpoint{3.880565in}{1.454653in}}%
\pgfpathlineto{\pgfqpoint{3.848570in}{1.462907in}}%
\pgfpathlineto{\pgfqpoint{3.815534in}{1.470084in}}%
\pgfpathlineto{\pgfqpoint{3.781604in}{1.476147in}}%
\pgfpathlineto{\pgfqpoint{3.746930in}{1.481065in}}%
\pgfpathlineto{\pgfqpoint{3.711669in}{1.484814in}}%
\pgfpathclose%
\pgfusepath{fill}%
\end{pgfscope}%
\begin{pgfscope}%
\pgfpathrectangle{\pgfqpoint{2.548318in}{0.050000in}}{\pgfqpoint{2.081932in}{2.081932in}}%
\pgfusepath{clip}%
\pgfsetbuttcap%
\pgfsetroundjoin%
\definecolor{currentfill}{rgb}{0.636902,0.856542,0.216620}%
\pgfsetfillcolor{currentfill}%
\pgfsetlinewidth{0.000000pt}%
\definecolor{currentstroke}{rgb}{0.000000,0.000000,0.000000}%
\pgfsetstrokecolor{currentstroke}%
\pgfsetdash{}{0pt}%
\pgfpathmoveto{\pgfqpoint{3.362316in}{1.456832in}}%
\pgfpathlineto{\pgfqpoint{3.366114in}{1.460176in}}%
\pgfpathlineto{\pgfqpoint{3.370191in}{1.462857in}}%
\pgfpathlineto{\pgfqpoint{3.374529in}{1.464865in}}%
\pgfpathlineto{\pgfqpoint{3.379111in}{1.466194in}}%
\pgfpathlineto{\pgfqpoint{3.383916in}{1.466842in}}%
\pgfpathlineto{\pgfqpoint{3.413472in}{1.474162in}}%
\pgfpathlineto{\pgfqpoint{3.443940in}{1.480485in}}%
\pgfpathlineto{\pgfqpoint{3.475185in}{1.485778in}}%
\pgfpathlineto{\pgfqpoint{3.507069in}{1.490016in}}%
\pgfpathlineto{\pgfqpoint{3.539450in}{1.493178in}}%
\pgfpathlineto{\pgfqpoint{3.537838in}{1.493072in}}%
\pgfpathlineto{\pgfqpoint{3.536301in}{1.492256in}}%
\pgfpathlineto{\pgfqpoint{3.534845in}{1.490731in}}%
\pgfpathlineto{\pgfqpoint{3.533476in}{1.488500in}}%
\pgfpathlineto{\pgfqpoint{3.532201in}{1.485572in}}%
\pgfpathlineto{\pgfqpoint{3.496819in}{1.482121in}}%
\pgfpathlineto{\pgfqpoint{3.461985in}{1.477495in}}%
\pgfpathlineto{\pgfqpoint{3.427855in}{1.471718in}}%
\pgfpathlineto{\pgfqpoint{3.394583in}{1.464819in}}%
\pgfpathlineto{\pgfqpoint{3.362316in}{1.456832in}}%
\pgfpathclose%
\pgfusepath{fill}%
\end{pgfscope}%
\begin{pgfscope}%
\pgfpathrectangle{\pgfqpoint{2.548318in}{0.050000in}}{\pgfqpoint{2.081932in}{2.081932in}}%
\pgfusepath{clip}%
\pgfsetbuttcap%
\pgfsetroundjoin%
\definecolor{currentfill}{rgb}{0.268510,0.009605,0.335427}%
\pgfsetfillcolor{currentfill}%
\pgfsetlinewidth{0.000000pt}%
\definecolor{currentstroke}{rgb}{0.000000,0.000000,0.000000}%
\pgfsetstrokecolor{currentstroke}%
\pgfsetdash{}{0pt}%
\pgfpathmoveto{\pgfqpoint{3.809780in}{1.157934in}}%
\pgfpathlineto{\pgfqpoint{3.815885in}{1.164347in}}%
\pgfpathlineto{\pgfqpoint{3.821950in}{1.171307in}}%
\pgfpathlineto{\pgfqpoint{3.827951in}{1.178786in}}%
\pgfpathlineto{\pgfqpoint{3.833863in}{1.186754in}}%
\pgfpathlineto{\pgfqpoint{3.839660in}{1.195180in}}%
\pgfpathlineto{\pgfqpoint{3.865736in}{1.187630in}}%
\pgfpathlineto{\pgfqpoint{3.890718in}{1.179271in}}%
\pgfpathlineto{\pgfqpoint{3.914503in}{1.170142in}}%
\pgfpathlineto{\pgfqpoint{3.936990in}{1.160286in}}%
\pgfpathlineto{\pgfqpoint{3.958089in}{1.149751in}}%
\pgfpathlineto{\pgfqpoint{3.949260in}{1.142559in}}%
\pgfpathlineto{\pgfqpoint{3.940254in}{1.135843in}}%
\pgfpathlineto{\pgfqpoint{3.931109in}{1.129628in}}%
\pgfpathlineto{\pgfqpoint{3.921862in}{1.123940in}}%
\pgfpathlineto{\pgfqpoint{3.912553in}{1.118802in}}%
\pgfpathlineto{\pgfqpoint{3.894215in}{1.127883in}}%
\pgfpathlineto{\pgfqpoint{3.874687in}{1.136375in}}%
\pgfpathlineto{\pgfqpoint{3.854047in}{1.144238in}}%
\pgfpathlineto{\pgfqpoint{3.832381in}{1.151435in}}%
\pgfpathlineto{\pgfqpoint{3.809780in}{1.157934in}}%
\pgfpathclose%
\pgfusepath{fill}%
\end{pgfscope}%
\begin{pgfscope}%
\pgfpathrectangle{\pgfqpoint{2.548318in}{0.050000in}}{\pgfqpoint{2.081932in}{2.081932in}}%
\pgfusepath{clip}%
\pgfsetbuttcap%
\pgfsetroundjoin%
\definecolor{currentfill}{rgb}{0.268510,0.009605,0.335427}%
\pgfsetfillcolor{currentfill}%
\pgfsetlinewidth{0.000000pt}%
\definecolor{currentstroke}{rgb}{0.000000,0.000000,0.000000}%
\pgfsetstrokecolor{currentstroke}%
\pgfsetdash{}{0pt}%
\pgfpathmoveto{\pgfqpoint{3.326929in}{1.121143in}}%
\pgfpathlineto{\pgfqpoint{3.317756in}{1.126358in}}%
\pgfpathlineto{\pgfqpoint{3.308645in}{1.132121in}}%
\pgfpathlineto{\pgfqpoint{3.299633in}{1.138411in}}%
\pgfpathlineto{\pgfqpoint{3.290759in}{1.145202in}}%
\pgfpathlineto{\pgfqpoint{3.282059in}{1.152467in}}%
\pgfpathlineto{\pgfqpoint{3.303511in}{1.162836in}}%
\pgfpathlineto{\pgfqpoint{3.326330in}{1.172512in}}%
\pgfpathlineto{\pgfqpoint{3.350421in}{1.181451in}}%
\pgfpathlineto{\pgfqpoint{3.375685in}{1.189610in}}%
\pgfpathlineto{\pgfqpoint{3.402015in}{1.196951in}}%
\pgfpathlineto{\pgfqpoint{3.407641in}{1.188477in}}%
\pgfpathlineto{\pgfqpoint{3.413378in}{1.180459in}}%
\pgfpathlineto{\pgfqpoint{3.419201in}{1.172930in}}%
\pgfpathlineto{\pgfqpoint{3.425087in}{1.165920in}}%
\pgfpathlineto{\pgfqpoint{3.431010in}{1.159457in}}%
\pgfpathlineto{\pgfqpoint{3.408192in}{1.153140in}}%
\pgfpathlineto{\pgfqpoint{3.386285in}{1.146115in}}%
\pgfpathlineto{\pgfqpoint{3.365382in}{1.138417in}}%
\pgfpathlineto{\pgfqpoint{3.345570in}{1.130080in}}%
\pgfpathlineto{\pgfqpoint{3.326929in}{1.121143in}}%
\pgfpathclose%
\pgfusepath{fill}%
\end{pgfscope}%
\begin{pgfscope}%
\pgfpathrectangle{\pgfqpoint{2.548318in}{0.050000in}}{\pgfqpoint{2.081932in}{2.081932in}}%
\pgfusepath{clip}%
\pgfsetbuttcap%
\pgfsetroundjoin%
\definecolor{currentfill}{rgb}{0.267004,0.004874,0.329415}%
\pgfsetfillcolor{currentfill}%
\pgfsetlinewidth{0.000000pt}%
\definecolor{currentstroke}{rgb}{0.000000,0.000000,0.000000}%
\pgfsetstrokecolor{currentstroke}%
\pgfsetdash{}{0pt}%
\pgfpathmoveto{\pgfqpoint{3.675417in}{1.152538in}}%
\pgfpathlineto{\pgfqpoint{3.677540in}{1.156550in}}%
\pgfpathlineto{\pgfqpoint{3.679694in}{1.161220in}}%
\pgfpathlineto{\pgfqpoint{3.681870in}{1.166532in}}%
\pgfpathlineto{\pgfqpoint{3.684059in}{1.172465in}}%
\pgfpathlineto{\pgfqpoint{3.686253in}{1.178996in}}%
\pgfpathlineto{\pgfqpoint{3.712009in}{1.176381in}}%
\pgfpathlineto{\pgfqpoint{3.737348in}{1.172949in}}%
\pgfpathlineto{\pgfqpoint{3.762161in}{1.168716in}}%
\pgfpathlineto{\pgfqpoint{3.786340in}{1.163703in}}%
\pgfpathlineto{\pgfqpoint{3.809780in}{1.157934in}}%
\pgfpathlineto{\pgfqpoint{3.803662in}{1.152093in}}%
\pgfpathlineto{\pgfqpoint{3.797555in}{1.146846in}}%
\pgfpathlineto{\pgfqpoint{3.791484in}{1.142214in}}%
\pgfpathlineto{\pgfqpoint{3.785475in}{1.138213in}}%
\pgfpathlineto{\pgfqpoint{3.779550in}{1.134859in}}%
\pgfpathlineto{\pgfqpoint{3.759776in}{1.139703in}}%
\pgfpathlineto{\pgfqpoint{3.739387in}{1.143911in}}%
\pgfpathlineto{\pgfqpoint{3.718471in}{1.147463in}}%
\pgfpathlineto{\pgfqpoint{3.697117in}{1.150343in}}%
\pgfpathlineto{\pgfqpoint{3.675417in}{1.152538in}}%
\pgfpathclose%
\pgfusepath{fill}%
\end{pgfscope}%
\begin{pgfscope}%
\pgfpathrectangle{\pgfqpoint{2.548318in}{0.050000in}}{\pgfqpoint{2.081932in}{2.081932in}}%
\pgfusepath{clip}%
\pgfsetbuttcap%
\pgfsetroundjoin%
\definecolor{currentfill}{rgb}{0.267004,0.004874,0.329415}%
\pgfsetfillcolor{currentfill}%
\pgfsetlinewidth{0.000000pt}%
\definecolor{currentstroke}{rgb}{0.000000,0.000000,0.000000}%
\pgfsetstrokecolor{currentstroke}%
\pgfsetdash{}{0pt}%
\pgfpathmoveto{\pgfqpoint{3.460343in}{1.136138in}}%
\pgfpathlineto{\pgfqpoint{3.454595in}{1.139540in}}%
\pgfpathlineto{\pgfqpoint{3.448764in}{1.143589in}}%
\pgfpathlineto{\pgfqpoint{3.442873in}{1.148270in}}%
\pgfpathlineto{\pgfqpoint{3.436947in}{1.153567in}}%
\pgfpathlineto{\pgfqpoint{3.431010in}{1.159457in}}%
\pgfpathlineto{\pgfqpoint{3.454644in}{1.165039in}}%
\pgfpathlineto{\pgfqpoint{3.478990in}{1.169857in}}%
\pgfpathlineto{\pgfqpoint{3.503944in}{1.173890in}}%
\pgfpathlineto{\pgfqpoint{3.529397in}{1.177117in}}%
\pgfpathlineto{\pgfqpoint{3.555239in}{1.179525in}}%
\pgfpathlineto{\pgfqpoint{3.557227in}{1.172976in}}%
\pgfpathlineto{\pgfqpoint{3.559211in}{1.167026in}}%
\pgfpathlineto{\pgfqpoint{3.561184in}{1.161697in}}%
\pgfpathlineto{\pgfqpoint{3.563135in}{1.157010in}}%
\pgfpathlineto{\pgfqpoint{3.565059in}{1.152981in}}%
\pgfpathlineto{\pgfqpoint{3.543288in}{1.150961in}}%
\pgfpathlineto{\pgfqpoint{3.521839in}{1.148253in}}%
\pgfpathlineto{\pgfqpoint{3.500806in}{1.144869in}}%
\pgfpathlineto{\pgfqpoint{3.480277in}{1.140824in}}%
\pgfpathlineto{\pgfqpoint{3.460343in}{1.136138in}}%
\pgfpathclose%
\pgfusepath{fill}%
\end{pgfscope}%
\begin{pgfscope}%
\pgfpathrectangle{\pgfqpoint{2.548318in}{0.050000in}}{\pgfqpoint{2.081932in}{2.081932in}}%
\pgfusepath{clip}%
\pgfsetbuttcap%
\pgfsetroundjoin%
\definecolor{currentfill}{rgb}{0.124780,0.640461,0.527068}%
\pgfsetfillcolor{currentfill}%
\pgfsetlinewidth{0.000000pt}%
\definecolor{currentstroke}{rgb}{0.000000,0.000000,0.000000}%
\pgfsetstrokecolor{currentstroke}%
\pgfsetdash{}{0pt}%
\pgfpathmoveto{\pgfqpoint{3.900823in}{1.388557in}}%
\pgfpathlineto{\pgfqpoint{3.900491in}{1.397434in}}%
\pgfpathlineto{\pgfqpoint{3.899766in}{1.405889in}}%
\pgfpathlineto{\pgfqpoint{3.898650in}{1.413887in}}%
\pgfpathlineto{\pgfqpoint{3.897150in}{1.421396in}}%
\pgfpathlineto{\pgfqpoint{3.895271in}{1.428386in}}%
\pgfpathlineto{\pgfqpoint{3.927785in}{1.418613in}}%
\pgfpathlineto{\pgfqpoint{3.958904in}{1.407800in}}%
\pgfpathlineto{\pgfqpoint{3.988496in}{1.395999in}}%
\pgfpathlineto{\pgfqpoint{4.016437in}{1.383268in}}%
\pgfpathlineto{\pgfqpoint{4.042612in}{1.369670in}}%
\pgfpathlineto{\pgfqpoint{4.045462in}{1.362347in}}%
\pgfpathlineto{\pgfqpoint{4.047738in}{1.354589in}}%
\pgfpathlineto{\pgfqpoint{4.049431in}{1.346428in}}%
\pgfpathlineto{\pgfqpoint{4.050531in}{1.337896in}}%
\pgfpathlineto{\pgfqpoint{4.051034in}{1.329028in}}%
\pgfpathlineto{\pgfqpoint{4.024357in}{1.342813in}}%
\pgfpathlineto{\pgfqpoint{3.995875in}{1.355720in}}%
\pgfpathlineto{\pgfqpoint{3.965707in}{1.367684in}}%
\pgfpathlineto{\pgfqpoint{3.933977in}{1.378648in}}%
\pgfpathlineto{\pgfqpoint{3.900823in}{1.388557in}}%
\pgfpathclose%
\pgfusepath{fill}%
\end{pgfscope}%
\begin{pgfscope}%
\pgfpathrectangle{\pgfqpoint{2.548318in}{0.050000in}}{\pgfqpoint{2.081932in}{2.081932in}}%
\pgfusepath{clip}%
\pgfsetbuttcap%
\pgfsetroundjoin%
\definecolor{currentfill}{rgb}{0.124780,0.640461,0.527068}%
\pgfsetfillcolor{currentfill}%
\pgfsetlinewidth{0.000000pt}%
\definecolor{currentstroke}{rgb}{0.000000,0.000000,0.000000}%
\pgfsetstrokecolor{currentstroke}%
\pgfsetdash{}{0pt}%
\pgfpathmoveto{\pgfqpoint{3.190456in}{1.332581in}}%
\pgfpathlineto{\pgfqpoint{3.190952in}{1.341449in}}%
\pgfpathlineto{\pgfqpoint{3.192036in}{1.349977in}}%
\pgfpathlineto{\pgfqpoint{3.193704in}{1.358129in}}%
\pgfpathlineto{\pgfqpoint{3.195948in}{1.365872in}}%
\pgfpathlineto{\pgfqpoint{3.198757in}{1.373175in}}%
\pgfpathlineto{\pgfqpoint{3.225381in}{1.386561in}}%
\pgfpathlineto{\pgfqpoint{3.253743in}{1.399063in}}%
\pgfpathlineto{\pgfqpoint{3.283727in}{1.410620in}}%
\pgfpathlineto{\pgfqpoint{3.315205in}{1.421176in}}%
\pgfpathlineto{\pgfqpoint{3.348044in}{1.430678in}}%
\pgfpathlineto{\pgfqpoint{3.346221in}{1.423702in}}%
\pgfpathlineto{\pgfqpoint{3.344765in}{1.416202in}}%
\pgfpathlineto{\pgfqpoint{3.343682in}{1.408211in}}%
\pgfpathlineto{\pgfqpoint{3.342978in}{1.399759in}}%
\pgfpathlineto{\pgfqpoint{3.342656in}{1.390882in}}%
\pgfpathlineto{\pgfqpoint{3.309169in}{1.381246in}}%
\pgfpathlineto{\pgfqpoint{3.277073in}{1.370543in}}%
\pgfpathlineto{\pgfqpoint{3.246504in}{1.358825in}}%
\pgfpathlineto{\pgfqpoint{3.217592in}{1.346151in}}%
\pgfpathlineto{\pgfqpoint{3.190456in}{1.332581in}}%
\pgfpathclose%
\pgfusepath{fill}%
\end{pgfscope}%
\begin{pgfscope}%
\pgfpathrectangle{\pgfqpoint{2.548318in}{0.050000in}}{\pgfqpoint{2.081932in}{2.081932in}}%
\pgfusepath{clip}%
\pgfsetbuttcap%
\pgfsetroundjoin%
\definecolor{currentfill}{rgb}{0.636902,0.856542,0.216620}%
\pgfsetfillcolor{currentfill}%
\pgfsetlinewidth{0.000000pt}%
\definecolor{currentstroke}{rgb}{0.000000,0.000000,0.000000}%
\pgfsetstrokecolor{currentstroke}%
\pgfsetdash{}{0pt}%
\pgfpathmoveto{\pgfqpoint{3.532201in}{1.485572in}}%
\pgfpathlineto{\pgfqpoint{3.533476in}{1.488500in}}%
\pgfpathlineto{\pgfqpoint{3.534845in}{1.490731in}}%
\pgfpathlineto{\pgfqpoint{3.536301in}{1.492256in}}%
\pgfpathlineto{\pgfqpoint{3.537838in}{1.493072in}}%
\pgfpathlineto{\pgfqpoint{3.539450in}{1.493178in}}%
\pgfpathlineto{\pgfqpoint{3.572183in}{1.495248in}}%
\pgfpathlineto{\pgfqpoint{3.605119in}{1.496216in}}%
\pgfpathlineto{\pgfqpoint{3.638112in}{1.496076in}}%
\pgfpathlineto{\pgfqpoint{3.671012in}{1.494830in}}%
\pgfpathlineto{\pgfqpoint{3.703671in}{1.492484in}}%
\pgfpathlineto{\pgfqpoint{3.705450in}{1.492363in}}%
\pgfpathlineto{\pgfqpoint{3.707146in}{1.491534in}}%
\pgfpathlineto{\pgfqpoint{3.708753in}{1.489996in}}%
\pgfpathlineto{\pgfqpoint{3.710262in}{1.487753in}}%
\pgfpathlineto{\pgfqpoint{3.711669in}{1.484814in}}%
\pgfpathlineto{\pgfqpoint{3.675980in}{1.487376in}}%
\pgfpathlineto{\pgfqpoint{3.640025in}{1.488736in}}%
\pgfpathlineto{\pgfqpoint{3.603968in}{1.488888in}}%
\pgfpathlineto{\pgfqpoint{3.567972in}{1.487832in}}%
\pgfpathlineto{\pgfqpoint{3.532201in}{1.485572in}}%
\pgfpathclose%
\pgfusepath{fill}%
\end{pgfscope}%
\begin{pgfscope}%
\pgfpathrectangle{\pgfqpoint{2.548318in}{0.050000in}}{\pgfqpoint{2.081932in}{2.081932in}}%
\pgfusepath{clip}%
\pgfsetbuttcap%
\pgfsetroundjoin%
\definecolor{currentfill}{rgb}{0.267004,0.004874,0.329415}%
\pgfsetfillcolor{currentfill}%
\pgfsetlinewidth{0.000000pt}%
\definecolor{currentstroke}{rgb}{0.000000,0.000000,0.000000}%
\pgfsetstrokecolor{currentstroke}%
\pgfsetdash{}{0pt}%
\pgfpathmoveto{\pgfqpoint{3.565059in}{1.152981in}}%
\pgfpathlineto{\pgfqpoint{3.563135in}{1.157010in}}%
\pgfpathlineto{\pgfqpoint{3.561184in}{1.161697in}}%
\pgfpathlineto{\pgfqpoint{3.559211in}{1.167026in}}%
\pgfpathlineto{\pgfqpoint{3.557227in}{1.172976in}}%
\pgfpathlineto{\pgfqpoint{3.555239in}{1.179525in}}%
\pgfpathlineto{\pgfqpoint{3.581354in}{1.181101in}}%
\pgfpathlineto{\pgfqpoint{3.607630in}{1.181837in}}%
\pgfpathlineto{\pgfqpoint{3.633949in}{1.181731in}}%
\pgfpathlineto{\pgfqpoint{3.660195in}{1.180783in}}%
\pgfpathlineto{\pgfqpoint{3.686253in}{1.178996in}}%
\pgfpathlineto{\pgfqpoint{3.684059in}{1.172465in}}%
\pgfpathlineto{\pgfqpoint{3.681870in}{1.166532in}}%
\pgfpathlineto{\pgfqpoint{3.679694in}{1.161220in}}%
\pgfpathlineto{\pgfqpoint{3.677540in}{1.156550in}}%
\pgfpathlineto{\pgfqpoint{3.675417in}{1.152538in}}%
\pgfpathlineto{\pgfqpoint{3.653466in}{1.154036in}}%
\pgfpathlineto{\pgfqpoint{3.631358in}{1.154832in}}%
\pgfpathlineto{\pgfqpoint{3.609190in}{1.154921in}}%
\pgfpathlineto{\pgfqpoint{3.587058in}{1.154303in}}%
\pgfpathlineto{\pgfqpoint{3.565059in}{1.152981in}}%
\pgfpathclose%
\pgfusepath{fill}%
\end{pgfscope}%
\begin{pgfscope}%
\pgfpathrectangle{\pgfqpoint{2.548318in}{0.050000in}}{\pgfqpoint{2.081932in}{2.081932in}}%
\pgfusepath{clip}%
\pgfsetbuttcap%
\pgfsetroundjoin%
\definecolor{currentfill}{rgb}{0.282327,0.094955,0.417331}%
\pgfsetfillcolor{currentfill}%
\pgfsetlinewidth{0.000000pt}%
\definecolor{currentstroke}{rgb}{0.000000,0.000000,0.000000}%
\pgfsetstrokecolor{currentstroke}%
\pgfsetdash{}{0pt}%
\pgfpathmoveto{\pgfqpoint{3.839660in}{1.195180in}}%
\pgfpathlineto{\pgfqpoint{3.845319in}{1.204029in}}%
\pgfpathlineto{\pgfqpoint{3.850816in}{1.213265in}}%
\pgfpathlineto{\pgfqpoint{3.856126in}{1.222850in}}%
\pgfpathlineto{\pgfqpoint{3.861228in}{1.232744in}}%
\pgfpathlineto{\pgfqpoint{3.866101in}{1.242907in}}%
\pgfpathlineto{\pgfqpoint{3.895241in}{1.234397in}}%
\pgfpathlineto{\pgfqpoint{3.923148in}{1.224979in}}%
\pgfpathlineto{\pgfqpoint{3.949701in}{1.214697in}}%
\pgfpathlineto{\pgfqpoint{3.974789in}{1.203601in}}%
\pgfpathlineto{\pgfqpoint{3.998312in}{1.191743in}}%
\pgfpathlineto{\pgfqpoint{3.990905in}{1.182653in}}%
\pgfpathlineto{\pgfqpoint{3.983146in}{1.173873in}}%
\pgfpathlineto{\pgfqpoint{3.975067in}{1.165440in}}%
\pgfpathlineto{\pgfqpoint{3.966703in}{1.157388in}}%
\pgfpathlineto{\pgfqpoint{3.958089in}{1.149751in}}%
\pgfpathlineto{\pgfqpoint{3.936990in}{1.160286in}}%
\pgfpathlineto{\pgfqpoint{3.914503in}{1.170142in}}%
\pgfpathlineto{\pgfqpoint{3.890718in}{1.179271in}}%
\pgfpathlineto{\pgfqpoint{3.865736in}{1.187630in}}%
\pgfpathlineto{\pgfqpoint{3.839660in}{1.195180in}}%
\pgfpathclose%
\pgfusepath{fill}%
\end{pgfscope}%
\begin{pgfscope}%
\pgfpathrectangle{\pgfqpoint{2.548318in}{0.050000in}}{\pgfqpoint{2.081932in}{2.081932in}}%
\pgfusepath{clip}%
\pgfsetbuttcap%
\pgfsetroundjoin%
\definecolor{currentfill}{rgb}{0.282327,0.094955,0.417331}%
\pgfsetfillcolor{currentfill}%
\pgfsetlinewidth{0.000000pt}%
\definecolor{currentstroke}{rgb}{0.000000,0.000000,0.000000}%
\pgfsetstrokecolor{currentstroke}%
\pgfsetdash{}{0pt}%
\pgfpathmoveto{\pgfqpoint{3.282059in}{1.152467in}}%
\pgfpathlineto{\pgfqpoint{3.273570in}{1.160176in}}%
\pgfpathlineto{\pgfqpoint{3.265328in}{1.168298in}}%
\pgfpathlineto{\pgfqpoint{3.257367in}{1.176800in}}%
\pgfpathlineto{\pgfqpoint{3.249720in}{1.185646in}}%
\pgfpathlineto{\pgfqpoint{3.242420in}{1.194799in}}%
\pgfpathlineto{\pgfqpoint{3.266341in}{1.206471in}}%
\pgfpathlineto{\pgfqpoint{3.291803in}{1.217367in}}%
\pgfpathlineto{\pgfqpoint{3.318703in}{1.227436in}}%
\pgfpathlineto{\pgfqpoint{3.346927in}{1.236629in}}%
\pgfpathlineto{\pgfqpoint{3.376356in}{1.244902in}}%
\pgfpathlineto{\pgfqpoint{3.381084in}{1.234698in}}%
\pgfpathlineto{\pgfqpoint{3.386036in}{1.224760in}}%
\pgfpathlineto{\pgfqpoint{3.391190in}{1.215130in}}%
\pgfpathlineto{\pgfqpoint{3.396524in}{1.205847in}}%
\pgfpathlineto{\pgfqpoint{3.402015in}{1.196951in}}%
\pgfpathlineto{\pgfqpoint{3.375685in}{1.189610in}}%
\pgfpathlineto{\pgfqpoint{3.350421in}{1.181451in}}%
\pgfpathlineto{\pgfqpoint{3.326330in}{1.172512in}}%
\pgfpathlineto{\pgfqpoint{3.303511in}{1.162836in}}%
\pgfpathlineto{\pgfqpoint{3.282059in}{1.152467in}}%
\pgfpathclose%
\pgfusepath{fill}%
\end{pgfscope}%
\begin{pgfscope}%
\pgfpathrectangle{\pgfqpoint{2.548318in}{0.050000in}}{\pgfqpoint{2.081932in}{2.081932in}}%
\pgfusepath{clip}%
\pgfsetbuttcap%
\pgfsetroundjoin%
\definecolor{currentfill}{rgb}{0.150476,0.504369,0.557430}%
\pgfsetfillcolor{currentfill}%
\pgfsetlinewidth{0.000000pt}%
\definecolor{currentstroke}{rgb}{0.000000,0.000000,0.000000}%
\pgfsetstrokecolor{currentstroke}%
\pgfsetdash{}{0pt}%
\pgfpathmoveto{\pgfqpoint{3.898196in}{1.349589in}}%
\pgfpathlineto{\pgfqpoint{3.899443in}{1.359771in}}%
\pgfpathlineto{\pgfqpoint{3.900298in}{1.369686in}}%
\pgfpathlineto{\pgfqpoint{3.900758in}{1.379295in}}%
\pgfpathlineto{\pgfqpoint{3.900823in}{1.388557in}}%
\pgfpathlineto{\pgfqpoint{3.933977in}{1.378648in}}%
\pgfpathlineto{\pgfqpoint{3.965707in}{1.367684in}}%
\pgfpathlineto{\pgfqpoint{3.995875in}{1.355720in}}%
\pgfpathlineto{\pgfqpoint{4.024357in}{1.342813in}}%
\pgfpathlineto{\pgfqpoint{4.051034in}{1.329028in}}%
\pgfpathlineto{\pgfqpoint{4.050936in}{1.319861in}}%
\pgfpathlineto{\pgfqpoint{4.050238in}{1.310432in}}%
\pgfpathlineto{\pgfqpoint{4.048941in}{1.300781in}}%
\pgfpathlineto{\pgfqpoint{4.047050in}{1.290947in}}%
\pgfpathlineto{\pgfqpoint{4.020610in}{1.304528in}}%
\pgfpathlineto{\pgfqpoint{3.992384in}{1.317242in}}%
\pgfpathlineto{\pgfqpoint{3.962488in}{1.329028in}}%
\pgfpathlineto{\pgfqpoint{3.931048in}{1.339828in}}%
\pgfpathlineto{\pgfqpoint{3.898196in}{1.349589in}}%
\pgfpathclose%
\pgfusepath{fill}%
\end{pgfscope}%
\begin{pgfscope}%
\pgfpathrectangle{\pgfqpoint{2.548318in}{0.050000in}}{\pgfqpoint{2.081932in}{2.081932in}}%
\pgfusepath{clip}%
\pgfsetbuttcap%
\pgfsetroundjoin%
\definecolor{currentfill}{rgb}{0.150476,0.504369,0.557430}%
\pgfsetfillcolor{currentfill}%
\pgfsetlinewidth{0.000000pt}%
\definecolor{currentstroke}{rgb}{0.000000,0.000000,0.000000}%
\pgfsetstrokecolor{currentstroke}%
\pgfsetdash{}{0pt}%
\pgfpathmoveto{\pgfqpoint{3.194384in}{1.294447in}}%
\pgfpathlineto{\pgfqpoint{3.192520in}{1.304302in}}%
\pgfpathlineto{\pgfqpoint{3.191241in}{1.313969in}}%
\pgfpathlineto{\pgfqpoint{3.190553in}{1.323408in}}%
\pgfpathlineto{\pgfqpoint{3.190456in}{1.332581in}}%
\pgfpathlineto{\pgfqpoint{3.217592in}{1.346151in}}%
\pgfpathlineto{\pgfqpoint{3.246504in}{1.358825in}}%
\pgfpathlineto{\pgfqpoint{3.277073in}{1.370543in}}%
\pgfpathlineto{\pgfqpoint{3.309169in}{1.381246in}}%
\pgfpathlineto{\pgfqpoint{3.342656in}{1.390882in}}%
\pgfpathlineto{\pgfqpoint{3.342719in}{1.381616in}}%
\pgfpathlineto{\pgfqpoint{3.343166in}{1.372000in}}%
\pgfpathlineto{\pgfqpoint{3.343996in}{1.362074in}}%
\pgfpathlineto{\pgfqpoint{3.345206in}{1.351879in}}%
\pgfpathlineto{\pgfqpoint{3.312026in}{1.342388in}}%
\pgfpathlineto{\pgfqpoint{3.280221in}{1.331845in}}%
\pgfpathlineto{\pgfqpoint{3.249930in}{1.320302in}}%
\pgfpathlineto{\pgfqpoint{3.221278in}{1.307816in}}%
\pgfpathlineto{\pgfqpoint{3.194384in}{1.294447in}}%
\pgfpathclose%
\pgfusepath{fill}%
\end{pgfscope}%
\begin{pgfscope}%
\pgfpathrectangle{\pgfqpoint{2.548318in}{0.050000in}}{\pgfqpoint{2.081932in}{2.081932in}}%
\pgfusepath{clip}%
\pgfsetbuttcap%
\pgfsetroundjoin%
\definecolor{currentfill}{rgb}{0.267968,0.223549,0.512008}%
\pgfsetfillcolor{currentfill}%
\pgfsetlinewidth{0.000000pt}%
\definecolor{currentstroke}{rgb}{0.000000,0.000000,0.000000}%
\pgfsetstrokecolor{currentstroke}%
\pgfsetdash{}{0pt}%
\pgfpathmoveto{\pgfqpoint{3.866101in}{1.242907in}}%
\pgfpathlineto{\pgfqpoint{3.870722in}{1.253295in}}%
\pgfpathlineto{\pgfqpoint{3.875073in}{1.263865in}}%
\pgfpathlineto{\pgfqpoint{3.879135in}{1.274574in}}%
\pgfpathlineto{\pgfqpoint{3.882890in}{1.285376in}}%
\pgfpathlineto{\pgfqpoint{3.886323in}{1.296225in}}%
\pgfpathlineto{\pgfqpoint{3.917803in}{1.286951in}}%
\pgfpathlineto{\pgfqpoint{3.947938in}{1.276688in}}%
\pgfpathlineto{\pgfqpoint{3.976600in}{1.265486in}}%
\pgfpathlineto{\pgfqpoint{4.003668in}{1.253400in}}%
\pgfpathlineto{\pgfqpoint{4.029032in}{1.240489in}}%
\pgfpathlineto{\pgfqpoint{4.023819in}{1.230435in}}%
\pgfpathlineto{\pgfqpoint{4.018116in}{1.220492in}}%
\pgfpathlineto{\pgfqpoint{4.011946in}{1.210702in}}%
\pgfpathlineto{\pgfqpoint{4.005335in}{1.201106in}}%
\pgfpathlineto{\pgfqpoint{3.998312in}{1.191743in}}%
\pgfpathlineto{\pgfqpoint{3.974789in}{1.203601in}}%
\pgfpathlineto{\pgfqpoint{3.949701in}{1.214697in}}%
\pgfpathlineto{\pgfqpoint{3.923148in}{1.224979in}}%
\pgfpathlineto{\pgfqpoint{3.895241in}{1.234397in}}%
\pgfpathlineto{\pgfqpoint{3.866101in}{1.242907in}}%
\pgfpathclose%
\pgfusepath{fill}%
\end{pgfscope}%
\begin{pgfscope}%
\pgfpathrectangle{\pgfqpoint{2.548318in}{0.050000in}}{\pgfqpoint{2.081932in}{2.081932in}}%
\pgfusepath{clip}%
\pgfsetbuttcap%
\pgfsetroundjoin%
\definecolor{currentfill}{rgb}{0.327796,0.773980,0.406640}%
\pgfsetfillcolor{currentfill}%
\pgfsetlinewidth{0.000000pt}%
\definecolor{currentstroke}{rgb}{0.000000,0.000000,0.000000}%
\pgfsetstrokecolor{currentstroke}%
\pgfsetdash{}{0pt}%
\pgfpathmoveto{\pgfqpoint{3.716957in}{1.460119in}}%
\pgfpathlineto{\pgfqpoint{3.716149in}{1.466333in}}%
\pgfpathlineto{\pgfqpoint{3.715211in}{1.471930in}}%
\pgfpathlineto{\pgfqpoint{3.714149in}{1.476888in}}%
\pgfpathlineto{\pgfqpoint{3.712967in}{1.481188in}}%
\pgfpathlineto{\pgfqpoint{3.711669in}{1.484814in}}%
\pgfpathlineto{\pgfqpoint{3.746930in}{1.481065in}}%
\pgfpathlineto{\pgfqpoint{3.781604in}{1.476147in}}%
\pgfpathlineto{\pgfqpoint{3.815534in}{1.470084in}}%
\pgfpathlineto{\pgfqpoint{3.848570in}{1.462907in}}%
\pgfpathlineto{\pgfqpoint{3.880565in}{1.454653in}}%
\pgfpathlineto{\pgfqpoint{3.884174in}{1.450629in}}%
\pgfpathlineto{\pgfqpoint{3.887462in}{1.445972in}}%
\pgfpathlineto{\pgfqpoint{3.890417in}{1.440698in}}%
\pgfpathlineto{\pgfqpoint{3.893023in}{1.434828in}}%
\pgfpathlineto{\pgfqpoint{3.895271in}{1.428386in}}%
\pgfpathlineto{\pgfqpoint{3.861504in}{1.437069in}}%
\pgfpathlineto{\pgfqpoint{3.826631in}{1.444619in}}%
\pgfpathlineto{\pgfqpoint{3.790808in}{1.450998in}}%
\pgfpathlineto{\pgfqpoint{3.754195in}{1.456173in}}%
\pgfpathlineto{\pgfqpoint{3.716957in}{1.460119in}}%
\pgfpathclose%
\pgfusepath{fill}%
\end{pgfscope}%
\begin{pgfscope}%
\pgfpathrectangle{\pgfqpoint{2.548318in}{0.050000in}}{\pgfqpoint{2.081932in}{2.081932in}}%
\pgfusepath{clip}%
\pgfsetbuttcap%
\pgfsetroundjoin%
\definecolor{currentfill}{rgb}{0.206756,0.371758,0.553117}%
\pgfsetfillcolor{currentfill}%
\pgfsetlinewidth{0.000000pt}%
\definecolor{currentstroke}{rgb}{0.000000,0.000000,0.000000}%
\pgfsetstrokecolor{currentstroke}%
\pgfsetdash{}{0pt}%
\pgfpathmoveto{\pgfqpoint{3.886323in}{1.296225in}}%
\pgfpathlineto{\pgfqpoint{3.889419in}{1.307077in}}%
\pgfpathlineto{\pgfqpoint{3.892166in}{1.317884in}}%
\pgfpathlineto{\pgfqpoint{3.894550in}{1.328602in}}%
\pgfpathlineto{\pgfqpoint{3.896563in}{1.339185in}}%
\pgfpathlineto{\pgfqpoint{3.898196in}{1.349589in}}%
\pgfpathlineto{\pgfqpoint{3.931048in}{1.339828in}}%
\pgfpathlineto{\pgfqpoint{3.962488in}{1.329028in}}%
\pgfpathlineto{\pgfqpoint{3.992384in}{1.317242in}}%
\pgfpathlineto{\pgfqpoint{4.020610in}{1.304528in}}%
\pgfpathlineto{\pgfqpoint{4.047050in}{1.290947in}}%
\pgfpathlineto{\pgfqpoint{4.044572in}{1.280971in}}%
\pgfpathlineto{\pgfqpoint{4.041518in}{1.270895in}}%
\pgfpathlineto{\pgfqpoint{4.037899in}{1.260761in}}%
\pgfpathlineto{\pgfqpoint{4.033732in}{1.250612in}}%
\pgfpathlineto{\pgfqpoint{4.029032in}{1.240489in}}%
\pgfpathlineto{\pgfqpoint{4.003668in}{1.253400in}}%
\pgfpathlineto{\pgfqpoint{3.976600in}{1.265486in}}%
\pgfpathlineto{\pgfqpoint{3.947938in}{1.276688in}}%
\pgfpathlineto{\pgfqpoint{3.917803in}{1.286951in}}%
\pgfpathlineto{\pgfqpoint{3.886323in}{1.296225in}}%
\pgfpathclose%
\pgfusepath{fill}%
\end{pgfscope}%
\begin{pgfscope}%
\pgfpathrectangle{\pgfqpoint{2.548318in}{0.050000in}}{\pgfqpoint{2.081932in}{2.081932in}}%
\pgfusepath{clip}%
\pgfsetbuttcap%
\pgfsetroundjoin%
\definecolor{currentfill}{rgb}{0.327796,0.773980,0.406640}%
\pgfsetfillcolor{currentfill}%
\pgfsetlinewidth{0.000000pt}%
\definecolor{currentstroke}{rgb}{0.000000,0.000000,0.000000}%
\pgfsetstrokecolor{currentstroke}%
\pgfsetdash{}{0pt}%
\pgfpathmoveto{\pgfqpoint{3.348044in}{1.430678in}}%
\pgfpathlineto{\pgfqpoint{3.350225in}{1.437104in}}%
\pgfpathlineto{\pgfqpoint{3.352755in}{1.442954in}}%
\pgfpathlineto{\pgfqpoint{3.355622in}{1.448205in}}%
\pgfpathlineto{\pgfqpoint{3.358814in}{1.452837in}}%
\pgfpathlineto{\pgfqpoint{3.362316in}{1.456832in}}%
\pgfpathlineto{\pgfqpoint{3.394583in}{1.464819in}}%
\pgfpathlineto{\pgfqpoint{3.427855in}{1.471718in}}%
\pgfpathlineto{\pgfqpoint{3.461985in}{1.477495in}}%
\pgfpathlineto{\pgfqpoint{3.496819in}{1.482121in}}%
\pgfpathlineto{\pgfqpoint{3.532201in}{1.485572in}}%
\pgfpathlineto{\pgfqpoint{3.531026in}{1.481956in}}%
\pgfpathlineto{\pgfqpoint{3.529954in}{1.477665in}}%
\pgfpathlineto{\pgfqpoint{3.528991in}{1.472715in}}%
\pgfpathlineto{\pgfqpoint{3.528141in}{1.467125in}}%
\pgfpathlineto{\pgfqpoint{3.527409in}{1.460917in}}%
\pgfpathlineto{\pgfqpoint{3.490042in}{1.457284in}}%
\pgfpathlineto{\pgfqpoint{3.453259in}{1.452417in}}%
\pgfpathlineto{\pgfqpoint{3.417224in}{1.446338in}}%
\pgfpathlineto{\pgfqpoint{3.382100in}{1.439080in}}%
\pgfpathlineto{\pgfqpoint{3.348044in}{1.430678in}}%
\pgfpathclose%
\pgfusepath{fill}%
\end{pgfscope}%
\begin{pgfscope}%
\pgfpathrectangle{\pgfqpoint{2.548318in}{0.050000in}}{\pgfqpoint{2.081932in}{2.081932in}}%
\pgfusepath{clip}%
\pgfsetbuttcap%
\pgfsetroundjoin%
\definecolor{currentfill}{rgb}{0.267968,0.223549,0.512008}%
\pgfsetfillcolor{currentfill}%
\pgfsetlinewidth{0.000000pt}%
\definecolor{currentstroke}{rgb}{0.000000,0.000000,0.000000}%
\pgfsetstrokecolor{currentstroke}%
\pgfsetdash{}{0pt}%
\pgfpathmoveto{\pgfqpoint{3.242420in}{1.194799in}}%
\pgfpathlineto{\pgfqpoint{3.235498in}{1.204223in}}%
\pgfpathlineto{\pgfqpoint{3.228983in}{1.213877in}}%
\pgfpathlineto{\pgfqpoint{3.222902in}{1.223722in}}%
\pgfpathlineto{\pgfqpoint{3.217281in}{1.233716in}}%
\pgfpathlineto{\pgfqpoint{3.212143in}{1.243817in}}%
\pgfpathlineto{\pgfqpoint{3.237940in}{1.256526in}}%
\pgfpathlineto{\pgfqpoint{3.265415in}{1.268394in}}%
\pgfpathlineto{\pgfqpoint{3.294454in}{1.279364in}}%
\pgfpathlineto{\pgfqpoint{3.324936in}{1.289383in}}%
\pgfpathlineto{\pgfqpoint{3.356729in}{1.298401in}}%
\pgfpathlineto{\pgfqpoint{3.360061in}{1.287520in}}%
\pgfpathlineto{\pgfqpoint{3.363706in}{1.276685in}}%
\pgfpathlineto{\pgfqpoint{3.367648in}{1.265940in}}%
\pgfpathlineto{\pgfqpoint{3.371871in}{1.255331in}}%
\pgfpathlineto{\pgfqpoint{3.376356in}{1.244902in}}%
\pgfpathlineto{\pgfqpoint{3.346927in}{1.236629in}}%
\pgfpathlineto{\pgfqpoint{3.318703in}{1.227436in}}%
\pgfpathlineto{\pgfqpoint{3.291803in}{1.217367in}}%
\pgfpathlineto{\pgfqpoint{3.266341in}{1.206471in}}%
\pgfpathlineto{\pgfqpoint{3.242420in}{1.194799in}}%
\pgfpathclose%
\pgfusepath{fill}%
\end{pgfscope}%
\begin{pgfscope}%
\pgfpathrectangle{\pgfqpoint{2.548318in}{0.050000in}}{\pgfqpoint{2.081932in}{2.081932in}}%
\pgfusepath{clip}%
\pgfsetbuttcap%
\pgfsetroundjoin%
\definecolor{currentfill}{rgb}{0.206756,0.371758,0.553117}%
\pgfsetfillcolor{currentfill}%
\pgfsetlinewidth{0.000000pt}%
\definecolor{currentstroke}{rgb}{0.000000,0.000000,0.000000}%
\pgfsetstrokecolor{currentstroke}%
\pgfsetdash{}{0pt}%
\pgfpathmoveto{\pgfqpoint{3.212143in}{1.243817in}}%
\pgfpathlineto{\pgfqpoint{3.207511in}{1.253983in}}%
\pgfpathlineto{\pgfqpoint{3.203403in}{1.264172in}}%
\pgfpathlineto{\pgfqpoint{3.199836in}{1.274340in}}%
\pgfpathlineto{\pgfqpoint{3.196826in}{1.284446in}}%
\pgfpathlineto{\pgfqpoint{3.194384in}{1.294447in}}%
\pgfpathlineto{\pgfqpoint{3.221278in}{1.307816in}}%
\pgfpathlineto{\pgfqpoint{3.249930in}{1.320302in}}%
\pgfpathlineto{\pgfqpoint{3.280221in}{1.331845in}}%
\pgfpathlineto{\pgfqpoint{3.312026in}{1.342388in}}%
\pgfpathlineto{\pgfqpoint{3.345206in}{1.351879in}}%
\pgfpathlineto{\pgfqpoint{3.346791in}{1.341458in}}%
\pgfpathlineto{\pgfqpoint{3.348745in}{1.330855in}}%
\pgfpathlineto{\pgfqpoint{3.351059in}{1.320114in}}%
\pgfpathlineto{\pgfqpoint{3.353724in}{1.309281in}}%
\pgfpathlineto{\pgfqpoint{3.356729in}{1.298401in}}%
\pgfpathlineto{\pgfqpoint{3.324936in}{1.289383in}}%
\pgfpathlineto{\pgfqpoint{3.294454in}{1.279364in}}%
\pgfpathlineto{\pgfqpoint{3.265415in}{1.268394in}}%
\pgfpathlineto{\pgfqpoint{3.237940in}{1.256526in}}%
\pgfpathlineto{\pgfqpoint{3.212143in}{1.243817in}}%
\pgfpathclose%
\pgfusepath{fill}%
\end{pgfscope}%
\begin{pgfscope}%
\pgfpathrectangle{\pgfqpoint{2.548318in}{0.050000in}}{\pgfqpoint{2.081932in}{2.081932in}}%
\pgfusepath{clip}%
\pgfsetbuttcap%
\pgfsetroundjoin%
\definecolor{currentfill}{rgb}{0.268510,0.009605,0.335427}%
\pgfsetfillcolor{currentfill}%
\pgfsetlinewidth{0.000000pt}%
\definecolor{currentstroke}{rgb}{0.000000,0.000000,0.000000}%
\pgfsetstrokecolor{currentstroke}%
\pgfsetdash{}{0pt}%
\pgfpathmoveto{\pgfqpoint{3.686253in}{1.178996in}}%
\pgfpathlineto{\pgfqpoint{3.688443in}{1.186102in}}%
\pgfpathlineto{\pgfqpoint{3.690619in}{1.193753in}}%
\pgfpathlineto{\pgfqpoint{3.692772in}{1.201920in}}%
\pgfpathlineto{\pgfqpoint{3.694894in}{1.210569in}}%
\pgfpathlineto{\pgfqpoint{3.696975in}{1.219667in}}%
\pgfpathlineto{\pgfqpoint{3.726741in}{1.216625in}}%
\pgfpathlineto{\pgfqpoint{3.756021in}{1.212634in}}%
\pgfpathlineto{\pgfqpoint{3.784684in}{1.207712in}}%
\pgfpathlineto{\pgfqpoint{3.812604in}{1.201885in}}%
\pgfpathlineto{\pgfqpoint{3.839660in}{1.195180in}}%
\pgfpathlineto{\pgfqpoint{3.833863in}{1.186754in}}%
\pgfpathlineto{\pgfqpoint{3.827951in}{1.178786in}}%
\pgfpathlineto{\pgfqpoint{3.821950in}{1.171307in}}%
\pgfpathlineto{\pgfqpoint{3.815885in}{1.164347in}}%
\pgfpathlineto{\pgfqpoint{3.809780in}{1.157934in}}%
\pgfpathlineto{\pgfqpoint{3.786340in}{1.163703in}}%
\pgfpathlineto{\pgfqpoint{3.762161in}{1.168716in}}%
\pgfpathlineto{\pgfqpoint{3.737348in}{1.172949in}}%
\pgfpathlineto{\pgfqpoint{3.712009in}{1.176381in}}%
\pgfpathlineto{\pgfqpoint{3.686253in}{1.178996in}}%
\pgfpathclose%
\pgfusepath{fill}%
\end{pgfscope}%
\begin{pgfscope}%
\pgfpathrectangle{\pgfqpoint{2.548318in}{0.050000in}}{\pgfqpoint{2.081932in}{2.081932in}}%
\pgfusepath{clip}%
\pgfsetbuttcap%
\pgfsetroundjoin%
\definecolor{currentfill}{rgb}{0.268510,0.009605,0.335427}%
\pgfsetfillcolor{currentfill}%
\pgfsetlinewidth{0.000000pt}%
\definecolor{currentstroke}{rgb}{0.000000,0.000000,0.000000}%
\pgfsetstrokecolor{currentstroke}%
\pgfsetdash{}{0pt}%
\pgfpathmoveto{\pgfqpoint{3.431010in}{1.159457in}}%
\pgfpathlineto{\pgfqpoint{3.425087in}{1.165920in}}%
\pgfpathlineto{\pgfqpoint{3.419201in}{1.172930in}}%
\pgfpathlineto{\pgfqpoint{3.413378in}{1.180459in}}%
\pgfpathlineto{\pgfqpoint{3.407641in}{1.188477in}}%
\pgfpathlineto{\pgfqpoint{3.402015in}{1.196951in}}%
\pgfpathlineto{\pgfqpoint{3.429297in}{1.203437in}}%
\pgfpathlineto{\pgfqpoint{3.457414in}{1.209039in}}%
\pgfpathlineto{\pgfqpoint{3.486242in}{1.213728in}}%
\pgfpathlineto{\pgfqpoint{3.515654in}{1.217482in}}%
\pgfpathlineto{\pgfqpoint{3.545521in}{1.220282in}}%
\pgfpathlineto{\pgfqpoint{3.547408in}{1.211167in}}%
\pgfpathlineto{\pgfqpoint{3.549330in}{1.202501in}}%
\pgfpathlineto{\pgfqpoint{3.551282in}{1.194316in}}%
\pgfpathlineto{\pgfqpoint{3.553254in}{1.186648in}}%
\pgfpathlineto{\pgfqpoint{3.555239in}{1.179525in}}%
\pgfpathlineto{\pgfqpoint{3.529397in}{1.177117in}}%
\pgfpathlineto{\pgfqpoint{3.503944in}{1.173890in}}%
\pgfpathlineto{\pgfqpoint{3.478990in}{1.169857in}}%
\pgfpathlineto{\pgfqpoint{3.454644in}{1.165039in}}%
\pgfpathlineto{\pgfqpoint{3.431010in}{1.159457in}}%
\pgfpathclose%
\pgfusepath{fill}%
\end{pgfscope}%
\begin{pgfscope}%
\pgfpathrectangle{\pgfqpoint{2.548318in}{0.050000in}}{\pgfqpoint{2.081932in}{2.081932in}}%
\pgfusepath{clip}%
\pgfsetbuttcap%
\pgfsetroundjoin%
\definecolor{currentfill}{rgb}{0.327796,0.773980,0.406640}%
\pgfsetfillcolor{currentfill}%
\pgfsetlinewidth{0.000000pt}%
\definecolor{currentstroke}{rgb}{0.000000,0.000000,0.000000}%
\pgfsetstrokecolor{currentstroke}%
\pgfsetdash{}{0pt}%
\pgfpathmoveto{\pgfqpoint{3.527409in}{1.460917in}}%
\pgfpathlineto{\pgfqpoint{3.528141in}{1.467125in}}%
\pgfpathlineto{\pgfqpoint{3.528991in}{1.472715in}}%
\pgfpathlineto{\pgfqpoint{3.529954in}{1.477665in}}%
\pgfpathlineto{\pgfqpoint{3.531026in}{1.481956in}}%
\pgfpathlineto{\pgfqpoint{3.532201in}{1.485572in}}%
\pgfpathlineto{\pgfqpoint{3.567972in}{1.487832in}}%
\pgfpathlineto{\pgfqpoint{3.603968in}{1.488888in}}%
\pgfpathlineto{\pgfqpoint{3.640025in}{1.488736in}}%
\pgfpathlineto{\pgfqpoint{3.675980in}{1.487376in}}%
\pgfpathlineto{\pgfqpoint{3.711669in}{1.484814in}}%
\pgfpathlineto{\pgfqpoint{3.712967in}{1.481188in}}%
\pgfpathlineto{\pgfqpoint{3.714149in}{1.476888in}}%
\pgfpathlineto{\pgfqpoint{3.715211in}{1.471930in}}%
\pgfpathlineto{\pgfqpoint{3.716149in}{1.466333in}}%
\pgfpathlineto{\pgfqpoint{3.716957in}{1.460119in}}%
\pgfpathlineto{\pgfqpoint{3.679265in}{1.462814in}}%
\pgfpathlineto{\pgfqpoint{3.641290in}{1.464246in}}%
\pgfpathlineto{\pgfqpoint{3.603206in}{1.464406in}}%
\pgfpathlineto{\pgfqpoint{3.565188in}{1.463295in}}%
\pgfpathlineto{\pgfqpoint{3.527409in}{1.460917in}}%
\pgfpathclose%
\pgfusepath{fill}%
\end{pgfscope}%
\begin{pgfscope}%
\pgfpathrectangle{\pgfqpoint{2.548318in}{0.050000in}}{\pgfqpoint{2.081932in}{2.081932in}}%
\pgfusepath{clip}%
\pgfsetbuttcap%
\pgfsetroundjoin%
\definecolor{currentfill}{rgb}{0.268510,0.009605,0.335427}%
\pgfsetfillcolor{currentfill}%
\pgfsetlinewidth{0.000000pt}%
\definecolor{currentstroke}{rgb}{0.000000,0.000000,0.000000}%
\pgfsetstrokecolor{currentstroke}%
\pgfsetdash{}{0pt}%
\pgfpathmoveto{\pgfqpoint{3.555239in}{1.179525in}}%
\pgfpathlineto{\pgfqpoint{3.553254in}{1.186648in}}%
\pgfpathlineto{\pgfqpoint{3.551282in}{1.194316in}}%
\pgfpathlineto{\pgfqpoint{3.549330in}{1.202501in}}%
\pgfpathlineto{\pgfqpoint{3.547408in}{1.211167in}}%
\pgfpathlineto{\pgfqpoint{3.545521in}{1.220282in}}%
\pgfpathlineto{\pgfqpoint{3.575710in}{1.222115in}}%
\pgfpathlineto{\pgfqpoint{3.606085in}{1.222972in}}%
\pgfpathlineto{\pgfqpoint{3.636512in}{1.222849in}}%
\pgfpathlineto{\pgfqpoint{3.666854in}{1.221745in}}%
\pgfpathlineto{\pgfqpoint{3.696975in}{1.219667in}}%
\pgfpathlineto{\pgfqpoint{3.694894in}{1.210569in}}%
\pgfpathlineto{\pgfqpoint{3.692772in}{1.201920in}}%
\pgfpathlineto{\pgfqpoint{3.690619in}{1.193753in}}%
\pgfpathlineto{\pgfqpoint{3.688443in}{1.186102in}}%
\pgfpathlineto{\pgfqpoint{3.686253in}{1.178996in}}%
\pgfpathlineto{\pgfqpoint{3.660195in}{1.180783in}}%
\pgfpathlineto{\pgfqpoint{3.633949in}{1.181731in}}%
\pgfpathlineto{\pgfqpoint{3.607630in}{1.181837in}}%
\pgfpathlineto{\pgfqpoint{3.581354in}{1.181101in}}%
\pgfpathlineto{\pgfqpoint{3.555239in}{1.179525in}}%
\pgfpathclose%
\pgfusepath{fill}%
\end{pgfscope}%
\begin{pgfscope}%
\pgfpathrectangle{\pgfqpoint{2.548318in}{0.050000in}}{\pgfqpoint{2.081932in}{2.081932in}}%
\pgfusepath{clip}%
\pgfsetbuttcap%
\pgfsetroundjoin%
\definecolor{currentfill}{rgb}{0.124780,0.640461,0.527068}%
\pgfsetfillcolor{currentfill}%
\pgfsetlinewidth{0.000000pt}%
\definecolor{currentstroke}{rgb}{0.000000,0.000000,0.000000}%
\pgfsetstrokecolor{currentstroke}%
\pgfsetdash{}{0pt}%
\pgfpathmoveto{\pgfqpoint{3.718954in}{1.420738in}}%
\pgfpathlineto{\pgfqpoint{3.718835in}{1.429619in}}%
\pgfpathlineto{\pgfqpoint{3.718574in}{1.438031in}}%
\pgfpathlineto{\pgfqpoint{3.718173in}{1.445939in}}%
\pgfpathlineto{\pgfqpoint{3.717633in}{1.453312in}}%
\pgfpathlineto{\pgfqpoint{3.716957in}{1.460119in}}%
\pgfpathlineto{\pgfqpoint{3.754195in}{1.456173in}}%
\pgfpathlineto{\pgfqpoint{3.790808in}{1.450998in}}%
\pgfpathlineto{\pgfqpoint{3.826631in}{1.444619in}}%
\pgfpathlineto{\pgfqpoint{3.861504in}{1.437069in}}%
\pgfpathlineto{\pgfqpoint{3.895271in}{1.428386in}}%
\pgfpathlineto{\pgfqpoint{3.897150in}{1.421396in}}%
\pgfpathlineto{\pgfqpoint{3.898650in}{1.413887in}}%
\pgfpathlineto{\pgfqpoint{3.899766in}{1.405889in}}%
\pgfpathlineto{\pgfqpoint{3.900491in}{1.397434in}}%
\pgfpathlineto{\pgfqpoint{3.900823in}{1.388557in}}%
\pgfpathlineto{\pgfqpoint{3.866387in}{1.397362in}}%
\pgfpathlineto{\pgfqpoint{3.830821in}{1.405019in}}%
\pgfpathlineto{\pgfqpoint{3.794283in}{1.411488in}}%
\pgfpathlineto{\pgfqpoint{3.756938in}{1.416737in}}%
\pgfpathlineto{\pgfqpoint{3.718954in}{1.420738in}}%
\pgfpathclose%
\pgfusepath{fill}%
\end{pgfscope}%
\begin{pgfscope}%
\pgfpathrectangle{\pgfqpoint{2.548318in}{0.050000in}}{\pgfqpoint{2.081932in}{2.081932in}}%
\pgfusepath{clip}%
\pgfsetbuttcap%
\pgfsetroundjoin%
\definecolor{currentfill}{rgb}{0.124780,0.640461,0.527068}%
\pgfsetfillcolor{currentfill}%
\pgfsetlinewidth{0.000000pt}%
\definecolor{currentstroke}{rgb}{0.000000,0.000000,0.000000}%
\pgfsetstrokecolor{currentstroke}%
\pgfsetdash{}{0pt}%
\pgfpathmoveto{\pgfqpoint{3.342656in}{1.390882in}}%
\pgfpathlineto{\pgfqpoint{3.342978in}{1.399759in}}%
\pgfpathlineto{\pgfqpoint{3.343682in}{1.408211in}}%
\pgfpathlineto{\pgfqpoint{3.344765in}{1.416202in}}%
\pgfpathlineto{\pgfqpoint{3.346221in}{1.423702in}}%
\pgfpathlineto{\pgfqpoint{3.348044in}{1.430678in}}%
\pgfpathlineto{\pgfqpoint{3.382100in}{1.439080in}}%
\pgfpathlineto{\pgfqpoint{3.417224in}{1.446338in}}%
\pgfpathlineto{\pgfqpoint{3.453259in}{1.452417in}}%
\pgfpathlineto{\pgfqpoint{3.490042in}{1.457284in}}%
\pgfpathlineto{\pgfqpoint{3.527409in}{1.460917in}}%
\pgfpathlineto{\pgfqpoint{3.526796in}{1.454114in}}%
\pgfpathlineto{\pgfqpoint{3.526307in}{1.446745in}}%
\pgfpathlineto{\pgfqpoint{3.525944in}{1.438839in}}%
\pgfpathlineto{\pgfqpoint{3.525707in}{1.430428in}}%
\pgfpathlineto{\pgfqpoint{3.525599in}{1.421547in}}%
\pgfpathlineto{\pgfqpoint{3.487484in}{1.417863in}}%
\pgfpathlineto{\pgfqpoint{3.449964in}{1.412927in}}%
\pgfpathlineto{\pgfqpoint{3.413210in}{1.406762in}}%
\pgfpathlineto{\pgfqpoint{3.377387in}{1.399401in}}%
\pgfpathlineto{\pgfqpoint{3.342656in}{1.390882in}}%
\pgfpathclose%
\pgfusepath{fill}%
\end{pgfscope}%
\begin{pgfscope}%
\pgfpathrectangle{\pgfqpoint{2.548318in}{0.050000in}}{\pgfqpoint{2.081932in}{2.081932in}}%
\pgfusepath{clip}%
\pgfsetbuttcap%
\pgfsetroundjoin%
\definecolor{currentfill}{rgb}{0.282327,0.094955,0.417331}%
\pgfsetfillcolor{currentfill}%
\pgfsetlinewidth{0.000000pt}%
\definecolor{currentstroke}{rgb}{0.000000,0.000000,0.000000}%
\pgfsetstrokecolor{currentstroke}%
\pgfsetdash{}{0pt}%
\pgfpathmoveto{\pgfqpoint{3.696975in}{1.219667in}}%
\pgfpathlineto{\pgfqpoint{3.699007in}{1.229176in}}%
\pgfpathlineto{\pgfqpoint{3.700980in}{1.239057in}}%
\pgfpathlineto{\pgfqpoint{3.702888in}{1.249270in}}%
\pgfpathlineto{\pgfqpoint{3.704721in}{1.259772in}}%
\pgfpathlineto{\pgfqpoint{3.706471in}{1.270519in}}%
\pgfpathlineto{\pgfqpoint{3.739789in}{1.267087in}}%
\pgfpathlineto{\pgfqpoint{3.772555in}{1.262585in}}%
\pgfpathlineto{\pgfqpoint{3.804623in}{1.257035in}}%
\pgfpathlineto{\pgfqpoint{3.835850in}{1.250465in}}%
\pgfpathlineto{\pgfqpoint{3.866101in}{1.242907in}}%
\pgfpathlineto{\pgfqpoint{3.861228in}{1.232744in}}%
\pgfpathlineto{\pgfqpoint{3.856126in}{1.222850in}}%
\pgfpathlineto{\pgfqpoint{3.850816in}{1.213265in}}%
\pgfpathlineto{\pgfqpoint{3.845319in}{1.204029in}}%
\pgfpathlineto{\pgfqpoint{3.839660in}{1.195180in}}%
\pgfpathlineto{\pgfqpoint{3.812604in}{1.201885in}}%
\pgfpathlineto{\pgfqpoint{3.784684in}{1.207712in}}%
\pgfpathlineto{\pgfqpoint{3.756021in}{1.212634in}}%
\pgfpathlineto{\pgfqpoint{3.726741in}{1.216625in}}%
\pgfpathlineto{\pgfqpoint{3.696975in}{1.219667in}}%
\pgfpathclose%
\pgfusepath{fill}%
\end{pgfscope}%
\begin{pgfscope}%
\pgfpathrectangle{\pgfqpoint{2.548318in}{0.050000in}}{\pgfqpoint{2.081932in}{2.081932in}}%
\pgfusepath{clip}%
\pgfsetbuttcap%
\pgfsetroundjoin%
\definecolor{currentfill}{rgb}{0.282327,0.094955,0.417331}%
\pgfsetfillcolor{currentfill}%
\pgfsetlinewidth{0.000000pt}%
\definecolor{currentstroke}{rgb}{0.000000,0.000000,0.000000}%
\pgfsetstrokecolor{currentstroke}%
\pgfsetdash{}{0pt}%
\pgfpathmoveto{\pgfqpoint{3.402015in}{1.196951in}}%
\pgfpathlineto{\pgfqpoint{3.396524in}{1.205847in}}%
\pgfpathlineto{\pgfqpoint{3.391190in}{1.215130in}}%
\pgfpathlineto{\pgfqpoint{3.386036in}{1.224760in}}%
\pgfpathlineto{\pgfqpoint{3.381084in}{1.234698in}}%
\pgfpathlineto{\pgfqpoint{3.376356in}{1.244902in}}%
\pgfpathlineto{\pgfqpoint{3.406862in}{1.252215in}}%
\pgfpathlineto{\pgfqpoint{3.438311in}{1.258531in}}%
\pgfpathlineto{\pgfqpoint{3.470566in}{1.263820in}}%
\pgfpathlineto{\pgfqpoint{3.503483in}{1.268054in}}%
\pgfpathlineto{\pgfqpoint{3.536915in}{1.271213in}}%
\pgfpathlineto{\pgfqpoint{3.538501in}{1.260451in}}%
\pgfpathlineto{\pgfqpoint{3.540162in}{1.249934in}}%
\pgfpathlineto{\pgfqpoint{3.541891in}{1.239705in}}%
\pgfpathlineto{\pgfqpoint{3.543680in}{1.229808in}}%
\pgfpathlineto{\pgfqpoint{3.545521in}{1.220282in}}%
\pgfpathlineto{\pgfqpoint{3.515654in}{1.217482in}}%
\pgfpathlineto{\pgfqpoint{3.486242in}{1.213728in}}%
\pgfpathlineto{\pgfqpoint{3.457414in}{1.209039in}}%
\pgfpathlineto{\pgfqpoint{3.429297in}{1.203437in}}%
\pgfpathlineto{\pgfqpoint{3.402015in}{1.196951in}}%
\pgfpathclose%
\pgfusepath{fill}%
\end{pgfscope}%
\begin{pgfscope}%
\pgfpathrectangle{\pgfqpoint{2.548318in}{0.050000in}}{\pgfqpoint{2.081932in}{2.081932in}}%
\pgfusepath{clip}%
\pgfsetbuttcap%
\pgfsetroundjoin%
\definecolor{currentfill}{rgb}{0.150476,0.504369,0.557430}%
\pgfsetfillcolor{currentfill}%
\pgfsetlinewidth{0.000000pt}%
\definecolor{currentstroke}{rgb}{0.000000,0.000000,0.000000}%
\pgfsetstrokecolor{currentstroke}%
\pgfsetdash{}{0pt}%
\pgfpathmoveto{\pgfqpoint{3.718010in}{1.381287in}}%
\pgfpathlineto{\pgfqpoint{3.718458in}{1.391658in}}%
\pgfpathlineto{\pgfqpoint{3.718766in}{1.401718in}}%
\pgfpathlineto{\pgfqpoint{3.718931in}{1.411425in}}%
\pgfpathlineto{\pgfqpoint{3.718954in}{1.420738in}}%
\pgfpathlineto{\pgfqpoint{3.756938in}{1.416737in}}%
\pgfpathlineto{\pgfqpoint{3.794283in}{1.411488in}}%
\pgfpathlineto{\pgfqpoint{3.830821in}{1.405019in}}%
\pgfpathlineto{\pgfqpoint{3.866387in}{1.397362in}}%
\pgfpathlineto{\pgfqpoint{3.900823in}{1.388557in}}%
\pgfpathlineto{\pgfqpoint{3.900758in}{1.379295in}}%
\pgfpathlineto{\pgfqpoint{3.900298in}{1.369686in}}%
\pgfpathlineto{\pgfqpoint{3.899443in}{1.359771in}}%
\pgfpathlineto{\pgfqpoint{3.898196in}{1.349589in}}%
\pgfpathlineto{\pgfqpoint{3.864077in}{1.358262in}}%
\pgfpathlineto{\pgfqpoint{3.828839in}{1.365804in}}%
\pgfpathlineto{\pgfqpoint{3.792639in}{1.372176in}}%
\pgfpathlineto{\pgfqpoint{3.755640in}{1.377345in}}%
\pgfpathlineto{\pgfqpoint{3.718010in}{1.381287in}}%
\pgfpathclose%
\pgfusepath{fill}%
\end{pgfscope}%
\begin{pgfscope}%
\pgfpathrectangle{\pgfqpoint{2.548318in}{0.050000in}}{\pgfqpoint{2.081932in}{2.081932in}}%
\pgfusepath{clip}%
\pgfsetbuttcap%
\pgfsetroundjoin%
\definecolor{currentfill}{rgb}{0.124780,0.640461,0.527068}%
\pgfsetfillcolor{currentfill}%
\pgfsetlinewidth{0.000000pt}%
\definecolor{currentstroke}{rgb}{0.000000,0.000000,0.000000}%
\pgfsetstrokecolor{currentstroke}%
\pgfsetdash{}{0pt}%
\pgfpathmoveto{\pgfqpoint{3.525599in}{1.421547in}}%
\pgfpathlineto{\pgfqpoint{3.525707in}{1.430428in}}%
\pgfpathlineto{\pgfqpoint{3.525944in}{1.438839in}}%
\pgfpathlineto{\pgfqpoint{3.526307in}{1.446745in}}%
\pgfpathlineto{\pgfqpoint{3.526796in}{1.454114in}}%
\pgfpathlineto{\pgfqpoint{3.527409in}{1.460917in}}%
\pgfpathlineto{\pgfqpoint{3.565188in}{1.463295in}}%
\pgfpathlineto{\pgfqpoint{3.603206in}{1.464406in}}%
\pgfpathlineto{\pgfqpoint{3.641290in}{1.464246in}}%
\pgfpathlineto{\pgfqpoint{3.679265in}{1.462814in}}%
\pgfpathlineto{\pgfqpoint{3.716957in}{1.460119in}}%
\pgfpathlineto{\pgfqpoint{3.717633in}{1.453312in}}%
\pgfpathlineto{\pgfqpoint{3.718173in}{1.445939in}}%
\pgfpathlineto{\pgfqpoint{3.718574in}{1.438031in}}%
\pgfpathlineto{\pgfqpoint{3.718835in}{1.429619in}}%
\pgfpathlineto{\pgfqpoint{3.718954in}{1.420738in}}%
\pgfpathlineto{\pgfqpoint{3.680506in}{1.423472in}}%
\pgfpathlineto{\pgfqpoint{3.641768in}{1.424924in}}%
\pgfpathlineto{\pgfqpoint{3.602919in}{1.425087in}}%
\pgfpathlineto{\pgfqpoint{3.564136in}{1.423959in}}%
\pgfpathlineto{\pgfqpoint{3.525599in}{1.421547in}}%
\pgfpathclose%
\pgfusepath{fill}%
\end{pgfscope}%
\begin{pgfscope}%
\pgfpathrectangle{\pgfqpoint{2.548318in}{0.050000in}}{\pgfqpoint{2.081932in}{2.081932in}}%
\pgfusepath{clip}%
\pgfsetbuttcap%
\pgfsetroundjoin%
\definecolor{currentfill}{rgb}{0.150476,0.504369,0.557430}%
\pgfsetfillcolor{currentfill}%
\pgfsetlinewidth{0.000000pt}%
\definecolor{currentstroke}{rgb}{0.000000,0.000000,0.000000}%
\pgfsetstrokecolor{currentstroke}%
\pgfsetdash{}{0pt}%
\pgfpathmoveto{\pgfqpoint{3.345206in}{1.351879in}}%
\pgfpathlineto{\pgfqpoint{3.343996in}{1.362074in}}%
\pgfpathlineto{\pgfqpoint{3.343166in}{1.372000in}}%
\pgfpathlineto{\pgfqpoint{3.342719in}{1.381616in}}%
\pgfpathlineto{\pgfqpoint{3.342656in}{1.390882in}}%
\pgfpathlineto{\pgfqpoint{3.377387in}{1.399401in}}%
\pgfpathlineto{\pgfqpoint{3.413210in}{1.406762in}}%
\pgfpathlineto{\pgfqpoint{3.449964in}{1.412927in}}%
\pgfpathlineto{\pgfqpoint{3.487484in}{1.417863in}}%
\pgfpathlineto{\pgfqpoint{3.525599in}{1.421547in}}%
\pgfpathlineto{\pgfqpoint{3.525620in}{1.412232in}}%
\pgfpathlineto{\pgfqpoint{3.525770in}{1.402523in}}%
\pgfpathlineto{\pgfqpoint{3.526049in}{1.392459in}}%
\pgfpathlineto{\pgfqpoint{3.526456in}{1.382084in}}%
\pgfpathlineto{\pgfqpoint{3.488695in}{1.378455in}}%
\pgfpathlineto{\pgfqpoint{3.451523in}{1.373593in}}%
\pgfpathlineto{\pgfqpoint{3.415110in}{1.367521in}}%
\pgfpathlineto{\pgfqpoint{3.379618in}{1.360271in}}%
\pgfpathlineto{\pgfqpoint{3.345206in}{1.351879in}}%
\pgfpathclose%
\pgfusepath{fill}%
\end{pgfscope}%
\begin{pgfscope}%
\pgfpathrectangle{\pgfqpoint{2.548318in}{0.050000in}}{\pgfqpoint{2.081932in}{2.081932in}}%
\pgfusepath{clip}%
\pgfsetbuttcap%
\pgfsetroundjoin%
\definecolor{currentfill}{rgb}{0.282327,0.094955,0.417331}%
\pgfsetfillcolor{currentfill}%
\pgfsetlinewidth{0.000000pt}%
\definecolor{currentstroke}{rgb}{0.000000,0.000000,0.000000}%
\pgfsetstrokecolor{currentstroke}%
\pgfsetdash{}{0pt}%
\pgfpathmoveto{\pgfqpoint{3.545521in}{1.220282in}}%
\pgfpathlineto{\pgfqpoint{3.543680in}{1.229808in}}%
\pgfpathlineto{\pgfqpoint{3.541891in}{1.239705in}}%
\pgfpathlineto{\pgfqpoint{3.540162in}{1.249934in}}%
\pgfpathlineto{\pgfqpoint{3.538501in}{1.260451in}}%
\pgfpathlineto{\pgfqpoint{3.536915in}{1.271213in}}%
\pgfpathlineto{\pgfqpoint{3.570710in}{1.273281in}}%
\pgfpathlineto{\pgfqpoint{3.604718in}{1.274248in}}%
\pgfpathlineto{\pgfqpoint{3.638783in}{1.274108in}}%
\pgfpathlineto{\pgfqpoint{3.672752in}{1.272864in}}%
\pgfpathlineto{\pgfqpoint{3.706471in}{1.270519in}}%
\pgfpathlineto{\pgfqpoint{3.704721in}{1.259772in}}%
\pgfpathlineto{\pgfqpoint{3.702888in}{1.249270in}}%
\pgfpathlineto{\pgfqpoint{3.700980in}{1.239057in}}%
\pgfpathlineto{\pgfqpoint{3.699007in}{1.229176in}}%
\pgfpathlineto{\pgfqpoint{3.696975in}{1.219667in}}%
\pgfpathlineto{\pgfqpoint{3.666854in}{1.221745in}}%
\pgfpathlineto{\pgfqpoint{3.636512in}{1.222849in}}%
\pgfpathlineto{\pgfqpoint{3.606085in}{1.222972in}}%
\pgfpathlineto{\pgfqpoint{3.575710in}{1.222115in}}%
\pgfpathlineto{\pgfqpoint{3.545521in}{1.220282in}}%
\pgfpathclose%
\pgfusepath{fill}%
\end{pgfscope}%
\begin{pgfscope}%
\pgfpathrectangle{\pgfqpoint{2.548318in}{0.050000in}}{\pgfqpoint{2.081932in}{2.081932in}}%
\pgfusepath{clip}%
\pgfsetbuttcap%
\pgfsetroundjoin%
\definecolor{currentfill}{rgb}{0.267968,0.223549,0.512008}%
\pgfsetfillcolor{currentfill}%
\pgfsetlinewidth{0.000000pt}%
\definecolor{currentstroke}{rgb}{0.000000,0.000000,0.000000}%
\pgfsetstrokecolor{currentstroke}%
\pgfsetdash{}{0pt}%
\pgfpathmoveto{\pgfqpoint{3.706471in}{1.270519in}}%
\pgfpathlineto{\pgfqpoint{3.708132in}{1.281467in}}%
\pgfpathlineto{\pgfqpoint{3.709696in}{1.292570in}}%
\pgfpathlineto{\pgfqpoint{3.711156in}{1.303780in}}%
\pgfpathlineto{\pgfqpoint{3.712506in}{1.315051in}}%
\pgfpathlineto{\pgfqpoint{3.713740in}{1.326335in}}%
\pgfpathlineto{\pgfqpoint{3.749775in}{1.322592in}}%
\pgfpathlineto{\pgfqpoint{3.785207in}{1.317682in}}%
\pgfpathlineto{\pgfqpoint{3.819879in}{1.311629in}}%
\pgfpathlineto{\pgfqpoint{3.853634in}{1.304465in}}%
\pgfpathlineto{\pgfqpoint{3.886323in}{1.296225in}}%
\pgfpathlineto{\pgfqpoint{3.882890in}{1.285376in}}%
\pgfpathlineto{\pgfqpoint{3.879135in}{1.274574in}}%
\pgfpathlineto{\pgfqpoint{3.875073in}{1.263865in}}%
\pgfpathlineto{\pgfqpoint{3.870722in}{1.253295in}}%
\pgfpathlineto{\pgfqpoint{3.866101in}{1.242907in}}%
\pgfpathlineto{\pgfqpoint{3.835850in}{1.250465in}}%
\pgfpathlineto{\pgfqpoint{3.804623in}{1.257035in}}%
\pgfpathlineto{\pgfqpoint{3.772555in}{1.262585in}}%
\pgfpathlineto{\pgfqpoint{3.739789in}{1.267087in}}%
\pgfpathlineto{\pgfqpoint{3.706471in}{1.270519in}}%
\pgfpathclose%
\pgfusepath{fill}%
\end{pgfscope}%
\begin{pgfscope}%
\pgfpathrectangle{\pgfqpoint{2.548318in}{0.050000in}}{\pgfqpoint{2.081932in}{2.081932in}}%
\pgfusepath{clip}%
\pgfsetbuttcap%
\pgfsetroundjoin%
\definecolor{currentfill}{rgb}{0.267968,0.223549,0.512008}%
\pgfsetfillcolor{currentfill}%
\pgfsetlinewidth{0.000000pt}%
\definecolor{currentstroke}{rgb}{0.000000,0.000000,0.000000}%
\pgfsetstrokecolor{currentstroke}%
\pgfsetdash{}{0pt}%
\pgfpathmoveto{\pgfqpoint{3.376356in}{1.244902in}}%
\pgfpathlineto{\pgfqpoint{3.371871in}{1.255331in}}%
\pgfpathlineto{\pgfqpoint{3.367648in}{1.265940in}}%
\pgfpathlineto{\pgfqpoint{3.363706in}{1.276685in}}%
\pgfpathlineto{\pgfqpoint{3.360061in}{1.287520in}}%
\pgfpathlineto{\pgfqpoint{3.356729in}{1.298401in}}%
\pgfpathlineto{\pgfqpoint{3.389697in}{1.306373in}}%
\pgfpathlineto{\pgfqpoint{3.423694in}{1.313260in}}%
\pgfpathlineto{\pgfqpoint{3.458570in}{1.319028in}}%
\pgfpathlineto{\pgfqpoint{3.494167in}{1.323646in}}%
\pgfpathlineto{\pgfqpoint{3.530326in}{1.327092in}}%
\pgfpathlineto{\pgfqpoint{3.531445in}{1.315797in}}%
\pgfpathlineto{\pgfqpoint{3.532669in}{1.304514in}}%
\pgfpathlineto{\pgfqpoint{3.533992in}{1.293291in}}%
\pgfpathlineto{\pgfqpoint{3.535409in}{1.282175in}}%
\pgfpathlineto{\pgfqpoint{3.536915in}{1.271213in}}%
\pgfpathlineto{\pgfqpoint{3.503483in}{1.268054in}}%
\pgfpathlineto{\pgfqpoint{3.470566in}{1.263820in}}%
\pgfpathlineto{\pgfqpoint{3.438311in}{1.258531in}}%
\pgfpathlineto{\pgfqpoint{3.406862in}{1.252215in}}%
\pgfpathlineto{\pgfqpoint{3.376356in}{1.244902in}}%
\pgfpathclose%
\pgfusepath{fill}%
\end{pgfscope}%
\begin{pgfscope}%
\pgfpathrectangle{\pgfqpoint{2.548318in}{0.050000in}}{\pgfqpoint{2.081932in}{2.081932in}}%
\pgfusepath{clip}%
\pgfsetbuttcap%
\pgfsetroundjoin%
\definecolor{currentfill}{rgb}{0.206756,0.371758,0.553117}%
\pgfsetfillcolor{currentfill}%
\pgfsetlinewidth{0.000000pt}%
\definecolor{currentstroke}{rgb}{0.000000,0.000000,0.000000}%
\pgfsetstrokecolor{currentstroke}%
\pgfsetdash{}{0pt}%
\pgfpathmoveto{\pgfqpoint{3.713740in}{1.326335in}}%
\pgfpathlineto{\pgfqpoint{3.714853in}{1.337584in}}%
\pgfpathlineto{\pgfqpoint{3.715841in}{1.348751in}}%
\pgfpathlineto{\pgfqpoint{3.716698in}{1.359788in}}%
\pgfpathlineto{\pgfqpoint{3.717422in}{1.370649in}}%
\pgfpathlineto{\pgfqpoint{3.718010in}{1.381287in}}%
\pgfpathlineto{\pgfqpoint{3.755640in}{1.377345in}}%
\pgfpathlineto{\pgfqpoint{3.792639in}{1.372176in}}%
\pgfpathlineto{\pgfqpoint{3.828839in}{1.365804in}}%
\pgfpathlineto{\pgfqpoint{3.864077in}{1.358262in}}%
\pgfpathlineto{\pgfqpoint{3.898196in}{1.349589in}}%
\pgfpathlineto{\pgfqpoint{3.896563in}{1.339185in}}%
\pgfpathlineto{\pgfqpoint{3.894550in}{1.328602in}}%
\pgfpathlineto{\pgfqpoint{3.892166in}{1.317884in}}%
\pgfpathlineto{\pgfqpoint{3.889419in}{1.307077in}}%
\pgfpathlineto{\pgfqpoint{3.886323in}{1.296225in}}%
\pgfpathlineto{\pgfqpoint{3.853634in}{1.304465in}}%
\pgfpathlineto{\pgfqpoint{3.819879in}{1.311629in}}%
\pgfpathlineto{\pgfqpoint{3.785207in}{1.317682in}}%
\pgfpathlineto{\pgfqpoint{3.749775in}{1.322592in}}%
\pgfpathlineto{\pgfqpoint{3.713740in}{1.326335in}}%
\pgfpathclose%
\pgfusepath{fill}%
\end{pgfscope}%
\begin{pgfscope}%
\pgfpathrectangle{\pgfqpoint{2.548318in}{0.050000in}}{\pgfqpoint{2.081932in}{2.081932in}}%
\pgfusepath{clip}%
\pgfsetbuttcap%
\pgfsetroundjoin%
\definecolor{currentfill}{rgb}{0.206756,0.371758,0.553117}%
\pgfsetfillcolor{currentfill}%
\pgfsetlinewidth{0.000000pt}%
\definecolor{currentstroke}{rgb}{0.000000,0.000000,0.000000}%
\pgfsetstrokecolor{currentstroke}%
\pgfsetdash{}{0pt}%
\pgfpathmoveto{\pgfqpoint{3.356729in}{1.298401in}}%
\pgfpathlineto{\pgfqpoint{3.353724in}{1.309281in}}%
\pgfpathlineto{\pgfqpoint{3.351059in}{1.320114in}}%
\pgfpathlineto{\pgfqpoint{3.348745in}{1.330855in}}%
\pgfpathlineto{\pgfqpoint{3.346791in}{1.341458in}}%
\pgfpathlineto{\pgfqpoint{3.345206in}{1.351879in}}%
\pgfpathlineto{\pgfqpoint{3.379618in}{1.360271in}}%
\pgfpathlineto{\pgfqpoint{3.415110in}{1.367521in}}%
\pgfpathlineto{\pgfqpoint{3.451523in}{1.373593in}}%
\pgfpathlineto{\pgfqpoint{3.488695in}{1.378455in}}%
\pgfpathlineto{\pgfqpoint{3.526456in}{1.382084in}}%
\pgfpathlineto{\pgfqpoint{3.526988in}{1.371440in}}%
\pgfpathlineto{\pgfqpoint{3.527645in}{1.360572in}}%
\pgfpathlineto{\pgfqpoint{3.528422in}{1.349527in}}%
\pgfpathlineto{\pgfqpoint{3.529317in}{1.338351in}}%
\pgfpathlineto{\pgfqpoint{3.530326in}{1.327092in}}%
\pgfpathlineto{\pgfqpoint{3.494167in}{1.323646in}}%
\pgfpathlineto{\pgfqpoint{3.458570in}{1.319028in}}%
\pgfpathlineto{\pgfqpoint{3.423694in}{1.313260in}}%
\pgfpathlineto{\pgfqpoint{3.389697in}{1.306373in}}%
\pgfpathlineto{\pgfqpoint{3.356729in}{1.298401in}}%
\pgfpathclose%
\pgfusepath{fill}%
\end{pgfscope}%
\begin{pgfscope}%
\pgfpathrectangle{\pgfqpoint{2.548318in}{0.050000in}}{\pgfqpoint{2.081932in}{2.081932in}}%
\pgfusepath{clip}%
\pgfsetbuttcap%
\pgfsetroundjoin%
\definecolor{currentfill}{rgb}{0.150476,0.504369,0.557430}%
\pgfsetfillcolor{currentfill}%
\pgfsetlinewidth{0.000000pt}%
\definecolor{currentstroke}{rgb}{0.000000,0.000000,0.000000}%
\pgfsetstrokecolor{currentstroke}%
\pgfsetdash{}{0pt}%
\pgfpathmoveto{\pgfqpoint{3.526456in}{1.382084in}}%
\pgfpathlineto{\pgfqpoint{3.526049in}{1.392459in}}%
\pgfpathlineto{\pgfqpoint{3.525770in}{1.402523in}}%
\pgfpathlineto{\pgfqpoint{3.525620in}{1.412232in}}%
\pgfpathlineto{\pgfqpoint{3.525599in}{1.421547in}}%
\pgfpathlineto{\pgfqpoint{3.564136in}{1.423959in}}%
\pgfpathlineto{\pgfqpoint{3.602919in}{1.425087in}}%
\pgfpathlineto{\pgfqpoint{3.641768in}{1.424924in}}%
\pgfpathlineto{\pgfqpoint{3.680506in}{1.423472in}}%
\pgfpathlineto{\pgfqpoint{3.718954in}{1.420738in}}%
\pgfpathlineto{\pgfqpoint{3.718931in}{1.411425in}}%
\pgfpathlineto{\pgfqpoint{3.718766in}{1.401718in}}%
\pgfpathlineto{\pgfqpoint{3.718458in}{1.391658in}}%
\pgfpathlineto{\pgfqpoint{3.718010in}{1.381287in}}%
\pgfpathlineto{\pgfqpoint{3.679919in}{1.383979in}}%
\pgfpathlineto{\pgfqpoint{3.641542in}{1.385409in}}%
\pgfpathlineto{\pgfqpoint{3.603055in}{1.385570in}}%
\pgfpathlineto{\pgfqpoint{3.564634in}{1.384459in}}%
\pgfpathlineto{\pgfqpoint{3.526456in}{1.382084in}}%
\pgfpathclose%
\pgfusepath{fill}%
\end{pgfscope}%
\begin{pgfscope}%
\pgfpathrectangle{\pgfqpoint{2.548318in}{0.050000in}}{\pgfqpoint{2.081932in}{2.081932in}}%
\pgfusepath{clip}%
\pgfsetbuttcap%
\pgfsetroundjoin%
\definecolor{currentfill}{rgb}{0.267968,0.223549,0.512008}%
\pgfsetfillcolor{currentfill}%
\pgfsetlinewidth{0.000000pt}%
\definecolor{currentstroke}{rgb}{0.000000,0.000000,0.000000}%
\pgfsetstrokecolor{currentstroke}%
\pgfsetdash{}{0pt}%
\pgfpathmoveto{\pgfqpoint{3.536915in}{1.271213in}}%
\pgfpathlineto{\pgfqpoint{3.535409in}{1.282175in}}%
\pgfpathlineto{\pgfqpoint{3.533992in}{1.293291in}}%
\pgfpathlineto{\pgfqpoint{3.532669in}{1.304514in}}%
\pgfpathlineto{\pgfqpoint{3.531445in}{1.315797in}}%
\pgfpathlineto{\pgfqpoint{3.530326in}{1.327092in}}%
\pgfpathlineto{\pgfqpoint{3.566883in}{1.329348in}}%
\pgfpathlineto{\pgfqpoint{3.603670in}{1.330403in}}%
\pgfpathlineto{\pgfqpoint{3.640521in}{1.330250in}}%
\pgfpathlineto{\pgfqpoint{3.677267in}{1.328892in}}%
\pgfpathlineto{\pgfqpoint{3.713740in}{1.326335in}}%
\pgfpathlineto{\pgfqpoint{3.712506in}{1.315051in}}%
\pgfpathlineto{\pgfqpoint{3.711156in}{1.303780in}}%
\pgfpathlineto{\pgfqpoint{3.709696in}{1.292570in}}%
\pgfpathlineto{\pgfqpoint{3.708132in}{1.281467in}}%
\pgfpathlineto{\pgfqpoint{3.706471in}{1.270519in}}%
\pgfpathlineto{\pgfqpoint{3.672752in}{1.272864in}}%
\pgfpathlineto{\pgfqpoint{3.638783in}{1.274108in}}%
\pgfpathlineto{\pgfqpoint{3.604718in}{1.274248in}}%
\pgfpathlineto{\pgfqpoint{3.570710in}{1.273281in}}%
\pgfpathlineto{\pgfqpoint{3.536915in}{1.271213in}}%
\pgfpathclose%
\pgfusepath{fill}%
\end{pgfscope}%
\begin{pgfscope}%
\pgfpathrectangle{\pgfqpoint{2.548318in}{0.050000in}}{\pgfqpoint{2.081932in}{2.081932in}}%
\pgfusepath{clip}%
\pgfsetbuttcap%
\pgfsetroundjoin%
\definecolor{currentfill}{rgb}{0.206756,0.371758,0.553117}%
\pgfsetfillcolor{currentfill}%
\pgfsetlinewidth{0.000000pt}%
\definecolor{currentstroke}{rgb}{0.000000,0.000000,0.000000}%
\pgfsetstrokecolor{currentstroke}%
\pgfsetdash{}{0pt}%
\pgfpathmoveto{\pgfqpoint{3.530326in}{1.327092in}}%
\pgfpathlineto{\pgfqpoint{3.529317in}{1.338351in}}%
\pgfpathlineto{\pgfqpoint{3.528422in}{1.349527in}}%
\pgfpathlineto{\pgfqpoint{3.527645in}{1.360572in}}%
\pgfpathlineto{\pgfqpoint{3.526988in}{1.371440in}}%
\pgfpathlineto{\pgfqpoint{3.526456in}{1.382084in}}%
\pgfpathlineto{\pgfqpoint{3.564634in}{1.384459in}}%
\pgfpathlineto{\pgfqpoint{3.603055in}{1.385570in}}%
\pgfpathlineto{\pgfqpoint{3.641542in}{1.385409in}}%
\pgfpathlineto{\pgfqpoint{3.679919in}{1.383979in}}%
\pgfpathlineto{\pgfqpoint{3.718010in}{1.381287in}}%
\pgfpathlineto{\pgfqpoint{3.717422in}{1.370649in}}%
\pgfpathlineto{\pgfqpoint{3.716698in}{1.359788in}}%
\pgfpathlineto{\pgfqpoint{3.715841in}{1.348751in}}%
\pgfpathlineto{\pgfqpoint{3.714853in}{1.337584in}}%
\pgfpathlineto{\pgfqpoint{3.713740in}{1.326335in}}%
\pgfpathlineto{\pgfqpoint{3.677267in}{1.328892in}}%
\pgfpathlineto{\pgfqpoint{3.640521in}{1.330250in}}%
\pgfpathlineto{\pgfqpoint{3.603670in}{1.330403in}}%
\pgfpathlineto{\pgfqpoint{3.566883in}{1.329348in}}%
\pgfpathlineto{\pgfqpoint{3.530326in}{1.327092in}}%
\pgfpathclose%
\pgfusepath{fill}%
\end{pgfscope}%
\end{pgfpicture}%
\makeatother%
\endgroup%

	\caption{Gezeigt ist ein Beispiel für eine zweidimensionale Mannigfaltigkeit eingebettet im $\real^3$, der sogenannte Torus. Bewegt man sich entlang der Oberfläche des Torus, so erscheint sie lokal flach und ähnelt damit dem $\real^2$. Aus diesem Grund ist der abgebildete Torus eine 2-Mannigfaltigkeit -- trotz der Tatsache, dass das Objekt als Ganzes nicht in einem zweidimensionalen Raum dargestellt werden kann.}
	\label{fig:Torus}
\end{figure}
Lokal erscheint die Oberfläche des Torus flach, das heißt sie ähnelt dem $\real^2$ und nicht dem $\real^3$, in dem sie eingebettet ist. Die intrinsische Dimension im Kontext der Topologie ist also zwei. Eine Karte kann nun informell wie eine Landkarte eines Teils der Oberfläche beschrieben werden. Bewegt man sich entlang
dieser Oberfläche, so muss die Karte irgendwann gewechselt werden. Dieser Übergang von zwei sich
überlappenden Karten wird durch den Homöomorphismus $\phi$ formalisiert. Betrachtet man alle Karten
von allen Teilen der Oberfläche ergibt dies den Atlas des Torus.

Eine Mannigfaltigkeit kann darüberhinaus noch mit weiteren Eigenschaften versehen werden. Dazu
gehören glatte Mannigfaltigkeiten, auf denen Differentialrechnung möglich ist und ein sogenannter
Tangentialraum definiert werden kann. Außerdem kann zwischen zusammenhängenden und
nicht-zusammenhängenden Mannigfaltigkeiten unterschieden werden, wobei letztere aus mehreren
nicht-verbundenen Untermannigfaltigkeiten bestehen. Im restlichen Teil dieser Arbeit wird mit einer
Mannigfaltigkeit Bezug auf eine im $\real^D$ eingebettete $d$-dimensionale Mannigfaltigkeit ohne
weitere Annahmen über die Struktur (glatt, zusammenhängend) genommen. Die intrinsische Dimension
von Daten kann also auch als die topologische Dimension der im $\real^D$ eingebetteten
Mannigfaltigkeit $\mathcal{M}$, auf der die Daten liegen, definiert werden. Auf eine umfassende
formale Definition wird an dieser Stelle jedoch verzichtet. Für eine mathematisch korrekte
Definition der hier genannten Begriffe aus der Topologie wird auf \textcites{Lee.2011}{Lee.2012}
verwiesen.

\section{Ansätze der Dimensionsreduktion}
\label{ch:Dimensionsreduktion:Ansaetze}
Methoden der Dimensionsreduktion können auf unterschiedlichste Weise in Kategorien eingeteilt werden. In diesem Abschnitt werden kurz die gängigsten Kategorisierungsmöglichkeiten vorgestellt. Dazu gehört die Unterteilung in (i) lineare und nichtlineare Methoden, (ii) konvexe und nicht-konvexe Methoden, sowie (iii) Distanz- und Topologie-erhaltende Methoden.

Eine simple und weitverbreitete Unterteilung ist die in lineare und nichtlineare Methoden. Während
lineare Methoden wie die Hauptkomponentenanalyse (\secref{ch:MethodenDerDimRed:statistisch:PCA})
nur lineare Zusammenhänge in den Daten erkennen können, sind nichtlineare Methoden in dieser
Hinsicht deutlich flexibler. Zu letzteren gehören beispielsweise die Kernel PCA
(\secref{ch:MethodenDerDimRed:statistisch:kPCA}) und die Autoencoder
(\secref{ch:MethodenDerDimRed:ML:AE}), welche eine bestimmte Klasse von neuronalen Netzen
darstellen.

\textbf{Konvexe} Methoden optimieren eine Zielfunktion, bei der jedes lokale Optimum gleichzeitig das globale Optimum ist. Die Optimierung ist daher leichter, jedoch sind die möglichen Zielfunktionen auch deutlich eingeschränkter. \textbf{Nicht-konvexe} Methoden wie der Autoencoder, das Sammon Mapping oder das Manifold Charting\addref optimieren nicht-konvexe Zielfunktionen und können damit bei der Optimierung in suboptimalen lokalen Extrempunkten \enquote{steckenbleiben}. Die Optimierung erfolgt meist mittels des Gradientenabstiegsverfahrens (engl. \textit{gradient descent}) oder anderen mathmatischen Optimierungsverfahren \parencite[siehe z.B.][]{Guler.2010}.

\textbf{Distanz-basierte Methoden} versuchen die paarweisen Distanzen in den Daten zu erhalten, womit die zugrundeliegende Struktur erhalten werden soll \parencite[3]{Gracia.2014}. Vertreter dieser Kategorie sind die (klassische) Multidimensionale
Skalierung \parencites{Kruskal.1964}{Cox.2008}, das Sammon Mapping \addref und die Curvilinear Component Analysis
\addref. Hierbei muss eine Distanz jedoch nicht die euklidische Distanz zwischen zwei Punkten sein.
Beispielweise ist die \newterm{geodätische Distanz} als die Distanz entlang der Mannigfaltigkeit
definiert, auf der die Punkte liegen. Mit dem Hintergrundwissen zu Mannigfaltigkeiten wird jedoch
deutlich, dass euklidische Distanzen vor allem bei nichtlinearen Mannigfaltigkeiten mit einer hohen
Krümmung irreführend sein können. Die geodätische Distanz wirkt diesem Problem entgegen, ist aber
nicht immer leicht zu berechnen. Daneben gibt es \textbf{Topologie-erhaltende Methoden}, die
gezielt die Topologie der Mannigfaltigkeit erhalten wollen und dies durch geometrische Überlegungen
erzielen \parencite[4]{Gracia.2014}. Zu dieser Kategorie gehört beispielsweise das Locally Linear Embedding
(\secref{ch:MethodenDerDimRed:statistisch:LLE}), die Self-Organizing Map \parencite{Kohonen.1990} oder die Uniform Manifold Approximation and Projection (UMAP) \parencite{McInnes.2018}.

Andere Algorithmen, die in dieser Arbeit nicht vorgestellt werden können sind Isomap \parencite{Tenenbaum.2000}, Maximum Variance Unfolding (MVU) \parencite{Weinberger.2006}, Laplacian Eigenmaps, Diffusion Maps, Varianten von LLE wie Hessian LLE
oder Local Tangent Space Alignment (LTSA), sowie Manifold Charting und Local Linear
Charting.\addref

\section{Relation zur Merkmalsextrahierung}
\label{ch:Dimensionsreduktion:Merkmalsextrahierung}

Die Merkmalsextrahierung ist ein sehr verwandtes Gebiet zur Dimensionsreduktion, da oftmals
Methoden der Dimensionsreduktion für die Merkmalsextrahierung eingesetzt werden \parencite[3]{Guyon.2006b}. Das Ziel hierbei ist es, in einem Vorverarbeitungsschritt
\enquote{irrelevante} Merkmale durch eine Transformation oder Selektion der Merkmale zu entfernen
und dann den eigentlichen Schätzer (Klassifikator oder Regressor) auf den neuen Merkmalen zu
trainieren. Insgesamt erhofft man sich durch eine Merkmalsextrahierung, dass der Klassifikator
(oder Regressor) durch robuster gegenüber Rauschen (engl. \textit{noise}) ist und eine ähnliche
oder bessere Performance bezüglich gewählter Gütemaße aufweist.

Wie wir in der Einleitung in \chapref{ch:Enleitung} gesehen haben, gibt es viele Gründe, wieso man
eine Reduktion der Dimension eines Datensatzes erreichen möchte. Unter anderem können so sehr große
Datenmengen effizienter gespeichert werden, ohne dabei einen zu großen Informationsverlust zu
erleiden. Deswegen ist das Gebiet der Dimensionsreduktion ein immer noch aktuelles Forschungsthema
mit einer großen Bandbreite an unterschiedlichen Methoden, wie in wir in
\secref{ch:Dimensionsreduktion:Ansaetze} gesehen haben. Dem Anwender kann es mitunter schwerfallen,
den richtigen Algorithmus auszuwählen, da es laut dem dem \newterm{No Free Lunch Theorem} \parencite{Wolpert.1997} keinen \enquote{besten} Algorithmus für jede Situation gibt. Im nächsten
Schritt (\chapref{ch:MethodenDerDimRed}) werden einige dieser Algorithmen vorgestellt, die dann in
\chapref{ch:Vergleich} miteinander verglichen werden, um dem Anwender einen Anhaltspunkt für die
Auswahl einer Dimensionsreduktionsmethode zu geben.