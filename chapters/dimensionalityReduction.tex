%% ==============================
\chapter{Dimensionsreduktion}
\label{ch:Dimensionsreduktion}
%% ==============================

Die Dimensionsreduktion hat im Kern das Ziel der Abbildung eines hochdimensionalen Datensatzes auf
eine niedrigdimensionale, sogenannte \newterm{latente} Repräsentation, wobei möglichst wenig
Information über die Daten verloren gehen soll.\addref Dies ist darin begründet, dass Daten oft nur
künstlich hochdimensional, also \textit{redundant} sind. Dies bedeutet, dass die Daten effizienter
über eine kleinere Menge von Merkmalen $\rvect{y}_1,\ldots,\rvect{y}_d$ ausgedrückt werden kann,
als über die ursprüngliche Repräsentation durch die Merkmale $\rvect{x}_1,\ldots,\rvect{x}_D$ mit
$d < D$. Hierbei bezeichnet $D$ die \newterm{extrinsische Dimension} des zugehörigen
Ursprungsraumes $\mathcal{X}$ (welcher üblicherweise dem $\real^D$ entspricht) und $d$ bezeichnet
die \newterm{intrinsische Dimension} der Daten. Die intrinsische Dimension wird teilweise auch als
latente Dimension bezeichnet.

Wir beziehen uns also auf die ursprüngliche (hochdimensionale) Repräsentation durch den
$D$-dimensionalen Zufallsvektor $\rvect{x} = \tr{(\rv{x_1}, \ldots, \rv{x_D})}$ und auf die latente
(niedrigdimensionale) Repräsentation durch den $d$-dimensionalen Zufallsvektor $\rvect{y}$. Liegt
eine konkrete Stichprobe vor, so werden die einzelnen Stichproben $\vect{x}^{(i)}, i = 1,\ldots,n$
als Spaltenvektoren in der $n \times D$ Datenmatrix $\mat{X}$ angeordnet. Analog dazu werden die
transformierten Daten in der Matrix $\mat{Y} \in \real^{n \times d}$ angeordnet. Wir werden
annehmen, dass die Datenmatrix $\mat{X}$ \textit{zentriert} ist. Dies kann jederzeit durch
Subtraktion des Erwartungswertes einer Variable $x_i$ von Spalte $i$ sichergestellt werden.

Nachdem wir nun grundlegende Terminologie und das Ziel der Dimensionsreduktion geklärt haben, wird
nun noch auf einige weitere wichtige Ideen der Dimensionsreduktion eingegangen. Dazu wird in
\secref{ch:Dimensionsreduktion:FluchDerDim} der Fluch der Dimensionalität sowie in
\secref{ch:Dimensionsreduktion:MannigfaltigkeitenIntrinsDim} die Idee von Mannigfaltigkeiten und
der intrinsischen Dimension behandelt. In \secref{ch:Dimensionsreduktion:Ansaetze} werden kurz
verschiedene Ansätze zur Dimensionsreduktion vorgestellt und in
\secref{ch:Dimensionsreduktion:Merkmalsextrahierung} wird die Relation zum Gebiet der
Merksmalsextrahierung besprochen.
\section{Der Fluch der Dimensionalität}
\label{ch:Dimensionsreduktion:FluchDerDim}

\section{Mannigfaltigkeiten und intrinsische Dimension}
\label{ch:Dimensionsreduktion:MannigfaltigkeitenIntrinsDim}

Wie eingangs besprochen geht man bei hochdimensionalen Daten oft von Redundanz oder
Abhängigkeitsstrukturen in den Merkmalen aus. Eng damit verbunden ist die Idee, dass Daten auf
einer sogenannten \newterm{Mannigfaltigkeit} liegen, welche der Grundbaustein eines wichtiges
Teilgebietes der Dimensionsreduktion, nämlich dem Erlernen von Mannigfaltigkeiten \parencite{Cayton.2005} ist. Dieses Teilgebiet wird durch die \newterm{Mannigfaltigkeits-Hypothese}
(engl. \textit{manifold hypothesis}) motiviert. Dabei wird hier mehr der Fokus auf das intuitive
Verständnis einer Mannigfaltigkeit gelegt. Für eine mathematisch korrekte Definition aus der
Topologie, einem Teilgebiet der Mathematik, wird auf ..\addref verwiesen.

Die Mannigfaltigkeits-Hypothese behauptet, dass reale hochdimensionale Daten in sehr vielen Fällen
auf einer in diesem hochdimensionalen Raum \textit{eingebetteten} Mannigfaltigkeit $\mathcal{M}$
der Dimensionalität $d$ < $D$ liegen \parencite[vgl.][1]{Cayton.2005}. \missingfigure{Ekläre Mannigfaltigkeit mit einer Abbildung}. Die
intrinsische Dimension $d$ entspricht in diesem Kontext der Dimension der Mannigfaltigkeit
$\mathcal{M}$.

\section{Ansätze der Dimensionsreduktion}
\label{ch:Dimensionsreduktion:Ansaetze}
\idea{generelle Ansätze grob formulieren, also nichtlinear vs. linear, topology vs. distance preservation, convex vs. non-convex}

\section{Relation zur Merkmalsextrahierung}
\label{ch:Dimensionsreduktion:Merkmalsextrahierung}

Wie wir in der Einleitung in \chapref{ch:Enleitung} gesehen haben, gibt es viele Gründe, wieso man
eine Reduktion der Dimension eines Datensatzes erreichen möchte. Unter anderem können so sehr große
Datenmengen effizienter gespeichert werden, ohne dabei einen zu großen Informationsverlust zu
erleiden. Deswegen ist das Gebiet der Dimensionsreduktion ein immer noch aktuelles Forschungsthema,
weshalb mittlerweile auch eine ganze Reihe an verschiedensten Methoden zur Dimensionsreduktion zur
Verfügung stehen. Dem Anwender kann es mitunter schwerfallen, den richtigen Algorithmus
auszuwählen, da es laut dem dem \newterm{No Free Lunch Theorem} \parencite{Wolpert.1997} keinen \enquote{besten} Algorithmus für jede Situation gibt. Im nächsten
Schritt (\chapref{ch:MethodenDerDimRed}) werden einige dieser Algorithmen vorgestellt.