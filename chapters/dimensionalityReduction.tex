%% ==============================
\chapter{Dimensionsreduktion}
\label{ch:Dimensionsreduktion}
%% ==============================

Die Dimensionsreduktion hat im Kern das Ziel der Abbildung eines hochdimensionalen Datensatzes auf
eine niedrigdimensionale, sogenannte \newterm{latente} Repräsentation, wobei möglichst wenig
Information über die Daten verloren gehen soll \parencite[2]{Lee.2007}. Dies ist darin begründet, dass Daten oft nur künstlich hochdimensional, also
\textit{redundant} sind. Dies bedeutet, dass die Daten effizienter über eine kleinere Menge von
Merkmalen $y_1,\ldots,y_d$ ausgedrückt werden können, als über die ursprüngliche Repräsentation
durch die Merkmale $x_1,\ldots,x_D$ mit $d < D$ und oft $d \ll D$. Hierbei bezeichnet $D$ die
\newterm{extrinsische Dimension} des zugehörigen Ursprungsraumes $\mathcal{X}$ (welcher
üblicherweise dem $\real^D$ entspricht) und $d$ bezeichnet die \newterm{intrinsische Dimension} der
Daten. Die intrinsische Dimension wird teilweise auch als latente Dimension bezeichnet und
beschreibt die minimale Anzahl an Merkmalsvariablen $y_i$, die für die Generierung der Daten
benötigt werden \parencite[47]{Lee.2007}. Die intrinsische Dimension kann neben dieser intuitiven Sichtweise auch über
topologische Überlegungen der zugrundeliegenden Verteilung der Daten definiert werden. Diese Idee
wird in \secref{ch:Dimensionsreduktion:MannigfaltigkeitenIntrinsDim} durch das Konzept von
Mannigfaltigkeiten erläutert.

Formell wird die ursprüngliche (hochdimensionale) Repräsentation mit dem $D$-dimensionalen
Zufallsvektor $\rvect{x} = \tr{(x_1, \ldots, x_D)}$ und die latente (niedrigdimensionale)
Repräsentation mit dem $d$-dimensionalen Zufallsvektor $\rvect{y}$ gekennzeichnet. Hierbei
bezeichnet $\tr{\,(\cdot)\,}$ die Transponierte. Liegt eine konkrete Stichprobe vor, so werden die
einzelnen Stichproben $\vect{x}_i, i = 1,\ldots,n$ als Spaltenvektoren in der $n \times D$
Datenmatrix $\mat{X}$ angeordnet. Analog dazu werden die transformierten (auch: projizierten) Daten
in der Matrix $\mat{Y} \in \real^{n \times d}$ angeordnet. Es wird angenommen, dass die Datenmatrix
$\mat{X}$ \textit{zentriert} ist. Dies kann jederzeit durch Subtraktion des Erwartungswertes
$\Exp[x_i]$ einer Variable $x_i$ von Spalte $i$ sichergestellt werden.

Nachdem nun die grundlegende Terminologie und das Ziel der Dimensionsreduktion geklärt wurde,
werden im Folgenden einige weitere wichtige Ideen und Konzepte erläutert. Dazu wird in
\secref{ch:Dimensionsreduktion:FluchDerDim} der Fluch der Dimensionalität sowie in
\secref{ch:Dimensionsreduktion:MannigfaltigkeitenIntrinsDim} die Idee von Mannigfaltigkeiten und
der intrinsischen Dimension behandelt. Letztlich werden in \secref{ch:Dimensionsreduktion:Ansaetze}
kurz verschiedene Möglichkeiten der Einordnung von Dimensionsreduktionsmethoden vorgestellt.

\section{Der Fluch der Dimensionalität}
\label{ch:Dimensionsreduktion:FluchDerDim}

Der Term \enquote{Fluch der Dimensionalität} (engl. \textit{Curse of Dimensionality}) wurde von
\textcite{Bellman.1957} im Kontext der dynamischen Programmierung hervorgebracht und ist seitdem zu
einem Sammelbegriff für die statistischen Probleme und Herausforderungen in hochdimensionalen
Räumen geworden. Diese Räume weisen Phänomene auf, die in niedrigeren Dimensionen nicht anzutreffen
sind. Die zwei Phänomene, die den Begriff dabei hauptsächlich prägen, sind das \newterm{Phänomen
	der leeren Räume} und die \newterm{Konzentration von Normen} \parencite[6 -- 9]{Lee.2007}. Im Folgenden werden die zwei Phänomene kurz erläutert.

% Die Konzentration von Normen geht auf das mathematische Gebiet der Konzentrationsungleichungen zurück und bezeichnet die Beobachtung, dass sich die Norm eines Vektors
% $\vect{x} \in \real^D$ um den Mittelwert der Verteilung der Normen konzentriert.
% Dies folgt aus dem Zentralen Grenzwertsatz, der besagt, dass eine Summe von unabhängigen und identisch verteilten Zufallsvariablen (approximativ) normalverteilt ist. Betrachtet man den Vektor $\vect{x}$ als einen Zufallsvektor, wobei jede Komponente unabhängig und identisch verteilt ist, kann man den Zentralen Grenzwertsatz auf auf die euklidische Norm
% \begin{equation}
% 	\norm{\vect{x}}^2 = x_1^2 + x_2^2 + \cdots + x_D^2
% \end{equation}
% anwenden: Die Verteilung der Norm konvergiert für $D \rightarrow \infty$ also gegen eine Normalverteilung

Dies führt dazu, dass Distanzmaße wie die euklidische Norm in hohen Dimensionen ihre Bedeutung
verlieren. Man beobachtet, dass mit steigender Dimension $D$ alle Punkte mehr oder weniger gleich
weit weg vom Ursprung liegen und sich die Verteilung der Normen daher konzentriert. Aufgrund dessen
wird die Interpretation der $K$ nächsten Nachbarn fragwürdig, weshalb insbesondere die Performance
von Clustering-Algorithmen wie z.B. K-Means in hohen Dimensionen stark abfällt.
\textcite{Aggarwal.2001} untersuchen diesen Effekt für unterschiedliche Distanzmaße wie der $L_1$-,
$L_2$- oder allgemein der $L_p$-Norm und stellen dabei fest, dass die Aussagekraft einer $L_p$-Norm
in höheren Dimensionen für große $p$ schneller abnimmt als z.B. für $p = 2$.

Des Weiteren tritt das Phänomen der leeren Räume auf. Hier stellt man fest, dass hochdimensionale
Räume dünn besetzt sind, da das Volumen im Raum exponentiell mit der Dimension $D$ ansteigt. Möchte
man daher eine Schätzung einer Funktion zu einem gewissen Genauigkeitsgrad durchführen, so steigt
die Anzahl der dazu benötigten Datenpunkte exponentiell mit $D$ an \parencite[6]{Lee.2007}. Dieses Phänomen führt zu interessanten Beobachtungen, wie z.B. das Verhältnis
des Volumens einer Hypersphäre und eines Hyperwürfels im $\real^D$ zeigt. Dazu betrachtet man eine
Hypersphäre mit Radius $r$ und den Hyperwürfel, der die Sphäre umschließt, d.h. die Kantenlänge
entspricht dem Durchmesser $2r$ der Sphäre. Das Verhältnis des Volumens dieser Sphäre zum Volumen
des Würfels konvergiert für $D \rightarrow \infty$ gegen Null. Das bedeutet, dass das Volumen in
den Ecken konzentriert \parencite[6 -- 7]{Lee.2007} und damit die meisten Punkte weit weg vom Ursprung liegen.

Zusammenfassend besteht der Fluch der Dimensionalität also darin, dass die Performance von
Klassifikatoren und Regressoren in hochdimensionalen Räumen aufgrund von Schwierigkeiten in der
Distanzmessung und aufgrund der dünn besetzten Räume degradiert und die Gefahr des Overfittings
höher ist. Die Dimensionsreduktion ist nun neben dem verwandten Gebiet der Merkmalsextrahierung ein
Werkzeug, um die statistischen und algorithmischen Probleme in hochdimensionalen Räumen zu
reduzieren, indem

\section{Mannigfaltigkeiten und intrinsische Dimension}
\label{ch:Dimensionsreduktion:MannigfaltigkeitenIntrinsDim}

Wie eingangs besprochen wird bei hochdimensionalen Daten oft von Redundanz oder
Abhängigkeitsstrukturen in den Merkmalen ausgegangen. Eng damit verbunden ist die Idee, dass Daten
auf einer sogenannten \newterm{Mannigfaltigkeit} (engl. \textit{manifold}) liegen.
Dimensionsreduktionsmethoden, die auf dieser Idee basieren, gehören zu einem wichtigen Teilgebiet
der Dimensionreduktion: dem Erlernen von Mannigfaltigkeiten \parencite{Cayton.2005}. Ein Vertreter dieser Methoden ist Locally Linear Embedding, das in
\subsecref{ch:MethodenDerDimRed:statistisch:LLE} noch eingehend behandelt wird. Motiviert werden
diese Ansätze durch die Hypothese, dass reale hochdimensionale Daten in vielen Fällen auf einer in
diesem hochdimensionalen Raum \textit{eingebetteten} Mannigfaltigkeit $\mathcal{M}$ der
Dimensionalität $d$ < $D$ liegen \parencite[vgl.][1]{Cayton.2005}. In diesem Abschnitt wird ein kurzer Überblick über den abstrakten
Begriff einer Mannigfaltigkeit gegeben, wodurch ein geometrischer Bezug der intrinsischen Dimension
hergestellt werden kann.

Eine $d$-dimensionale Mannigfaltigkeit $\mathcal{M}$ ist lokal \textit{homöomorph} zum $\real^d$,
das heißt $\mathcal{M}$ ähnelt \textit{lokal} dem $\real^d$ \parencite[3]{Lee.2011}. Das bedeutet, dass es für jeden Punkt $\vect{z} \in \mathcal{M}$ eine stetige
Abbildung $\phi: B_\epsilon(\vect{z}) \rightarrow \real^d$ gibt, deren Inverse ebenfalls stetig
ist. Hierbei ist $B_\epsilon(\vect{z})$ ein Ball mit Radius $\epsilon > 0$ um $\vect{z}$. Die
Abbildung $\phi$ heißt \textit{Karte} und die Gesamtheit aller Karten ergibt den \textit{Atlas} von
$\mathcal{M}$ \parencite[4]{Cayton.2005}. Intuitiv können diese Begriffe besser anhand eines anschaulichen Beispiels
erklärt werden, weshalb in \figref{fig:Torus} ein Torus dargestellt ist. Dieses Objekt ist eine
zweidimensionale Mannigfaltigkeit eingebettet im $\real^3$.
\begin{figure}[ht]
	\centering
	\includegraphics{torus.pdf}
	\caption[Ein Beispiel für eine zweidimensionale Mannigfaltigkeit: ein Torus]{Gezeigt ist ein Beispiel für eine zweidimensionale Mannigfaltigkeit eingebettet im $\real^3$, der sogenannte Torus. Bewegt man sich entlang der Oberfläche des Torus, so erscheint sie lokal flach und ähnelt damit dem $\real^2$. Aus diesem Grund ist der abgebildete Torus eine 2-Mannigfaltigkeit -- trotz der Tatsache, dass das Objekt als Ganzes nicht in einem zweidimensionalen Raum dargestellt werden kann. Eigene Darstellung\protect\footnotemark}
	\label{fig:Torus}
\end{figure}
\footnotetext{angelehnt an \url{https://scipython.com/book/chapter-7-matplotlib/examples/a-torus/}}
Lokal erscheint die Oberfläche des Torus flach, das heißt sie ähnelt dem $\real^2$ und nicht dem $\real^3$, in dem sie eingebettet ist. Die intrinsische Dimension im Kontext der Topologie ist also zwei. Eine Karte kann nun informell wie eine Landkarte eines Teils der Oberfläche beschrieben werden. Bewegt man sich entlang
dieser Oberfläche, so muss die Karte irgendwann gewechselt werden. Dieser Übergang von zwei sich
überlappenden Karten wird durch den Homöomorphismus $\phi$ formalisiert. Betrachtet man alle Karten
von allen Teilen der Oberfläche ergibt dies den Atlas des Torus.

Eine Mannigfaltigkeit kann darüber hinaus noch mit weiteren Eigenschaften versehen werden. Dazu
gehören glatte Mannigfaltigkeiten, auf denen Differentialrechnung möglich ist und ein sogenannter
Tangentialraum definiert werden kann. Außerdem kann zwischen zusammenhängenden und
nicht-zusammenhängenden Mannigfaltigkeiten unterschieden werden, wobei letztere aus mehreren
nicht-verbundenen Untermannigfaltigkeiten bestehen. Im restlichen Teil dieser Arbeit wird mit einer
Mannigfaltigkeit Bezug auf eine im $\real^D$ eingebettete $d$-dimensionale Mannigfaltigkeit ohne
weitere Annahmen über die Struktur (glatt, zusammenhängend) genommen. Die intrinsische Dimension
von Daten kann also auch als die topologische Dimension der im $\real^D$ eingebetteten
Mannigfaltigkeit $\mathcal{M}$, auf der die Daten liegen, definiert werden. Auf eine umfassende
formale Definition wird an dieser Stelle jedoch verzichtet. Für eine mathematisch korrekte
Definition der hier genannten Begriffe aus der Topologie wird auf \textcites{Lee.2011}{Lee.2012}
verwiesen.

\section{Einordnung der Dimensionsreduktionsmethoden}
\label{ch:Dimensionsreduktion:Ansaetze}
Methoden der Dimensionsreduktion können auf unterschiedliche Weisen in Kategorien eingeteilt werden. Die in dieser Arbeit verwendete Kategorisierung ist in \figref{fig:Kategorisierung} dargestellt und weicht von anderen Taxonomien ab, da der Fokus hier auf dem Vergleich statistischer Methoden und Machine Learning Ansätzen liegt.
% code adapted from https://tex.stackexchange.com/questions/255159/help-on-drawing-a-tree-in-latex?rq=1
\tikzset{
	my node/.style={
			draw=gray,
			inner color=gray!5,
			outer color=gray!10,
			thick,
			minimum width=1cm,
			rounded corners=3,
			text height=1.5ex,
			text depth=0ex,
		}
}
\begin{figure}[h]
	\centering
	\begin{forest}
		for tree={%
		my node,
		l sep+=5pt,
		grow'=south,
		edge={gray, thick},
		parent anchor=south,
		child anchor=north,
		if n children=0{tier=last}{},
		if={isodd(n_children())}{
				for children={
						if={equal(n,(n_children("!u")+1)/2)}{calign with current}{}
					}
			}{}
		}
		[Dimensionsreduktion, s sep=15mm,
		[Machine Learning
					[CAE] [AE]]
			[Statistik
					[LLE] [Kernel PCA] [PCA]]
		]
	\end{forest}
	\caption[Kategorisierung der Dimensionsreduktionsmethoden]{Die hier verwendete Kategorisierung der fünf Methoden.}
	\label{fig:Kategorisierung}
\end{figure}

Statistische Methoden haben eine solide theoretische Fundierung und sind meist etablierte
Algorithmen auf einem jeweiligen Gebiet, wie es zum Beispiel mit der Principal Component Analysis
(\subsecref{ch:MethodenDerDimRed:statistisch:PCA}) der Fall ist. Moderne Machine Learning Methoden
basieren auf neuronalen Netzen und versuchen, aus einer großen Menge an Trainingsdaten eine
Approximation der gesuchten Funktion zu lernen. Autoencoder sind hierfür ein Paradebeispiel und
werden deshalb in \secref{ch:MethodenDerDimRed:modern} eingehend behandelt.

Für einen Überblick und zur besseren Einschätzung werden in diesem Abschnitt weitere gängige
Kategorisierungsmöglichkeiten vorgestellt. Die von \textcite{vanderMaaten.2009} verwendete
Taxonomie ist in \figref{fig:KategorisierungMaaten}
abgebildet.% code adapted from https://tex.stackexchange.com/questions/255159/help-on-drawing-a-tree-in-latex?rq=1
\tikzset{
	my node/.style={
			draw=gray,
			inner color=gray!5,
			outer color=gray!10,
			thick,
			minimum width=1cm,
			rounded corners=3,
			text height=1.5ex,
			text depth=0ex,
		}
}
\begin{figure}[h]
	\centering
	\begin{forest}
		for tree={%
		my node,
		l sep+=5pt,
		grow'=south,
		edge={gray, thick},
		parent anchor=south,
		child anchor=north,
		fit=band,
		if n children=0{tier=last}{},
		if={isodd(n_children())}{
				for children={
						if={equal(n,(n_children("!u")+1)/2)}{calign with current}{}
					}
			}{}
		}
		[Dimensionsreduktion, s sep=15mm,
		[Konvex
			[nicht-vollwertige EWZ [LLE]] [vollwertige EWZ [Kernel PCA] [PCA]]]
		[Nicht-konvex
		[CAE] [AE]]
		]
	\end{forest}
	\caption[Alternative Kategorisierung der Dimensionsreduktionsmethoden]{Ein Auszug aus der Kategorisierung nach \textcite{vanderMaaten.2009}. Eine vollwertige Eigenwertszerlegung (EWZ) zerlegt eine (dichte) Matrix in ihre Eigenwerte und -vektoren. Eine nicht-vollwertige Eigenwertszerlegung meint die Zerlegung einer dünn besetzten Matrix \parencite[3,7]{vanderMaaten.2009}.} \label{fig:KategorisierungMaaten}
\end{figure} Hier werden die
Dimensionsreduktionsmethoden zunächst in konvexe und nicht-konvexe Methoden eingeordnet.
\textbf{Konvexe} Methoden optimieren eine Zielfunktion, bei der jedes lokale Optimum gleichzeitig
das globale Optimum ist. Die Optimierung ist daher leichter, jedoch sind die möglichen
Zielfunktionen auch deutlich eingeschränkter. Die Lösung dieser Zielfunktionen reduziert sich
meistens auf ein Eigenwertsproblem, weswegen die konvexen Methoden weiter in Methoden mit einer
vollwertigen und nicht-vollwertigen Eigenwertzerlegung (EWZ)eingegliedert werden können \parencite[3]{vanderMaaten.2009}. \textbf{Nicht-konvexe} Methoden wie der Autoencoder optimieren
nicht-konvexe Zielfunktionen und können damit bei der Optimierung in suboptimalen lokalen
Extrempunkten \enquote{steckenbleiben}. Die Optimierung erfolgt meist mittels des
Gradientenabstiegsverfahrens (engl. \textit{gradient descent}) oder anderer mathmatischer
Optimierungsverfahren \parencite[siehe z.B.][]{Guler.2010}.

% \textbf{Distanz-basierte Methoden} versuchen die paarweisen Distanzen in den Daten zu erhalten, womit die zugrundeliegende Struktur erhalten werden soll \parencite[3]{Gracia.2014}. Vertreter dieser Kategorie sind die (klassische) Multidimensionale
% Skalierung \parencites{Kruskal.1964}{Cox.2008}, das Sammon Mapping \addref und die Curvilinear Component Analysis
% \addref. Hierbei muss eine Distanz jedoch nicht die euklidische Distanz zwischen zwei Punkten sein.
% Beispielweise ist die \newterm{geodätische Distanz} als die Distanz entlang der Mannigfaltigkeit
% definiert, auf der die Punkte liegen. Mit dem Hintergrundwissen zu Mannigfaltigkeiten wird
% deutlich, dass euklidische Distanzen vor allem bei nichtlinearen Mannigfaltigkeiten mit einer hohen
% Krümmung irreführend sein können. Die geodätische Distanz wirkt diesem Problem entgegen, ist aber
% nicht immer leicht zu berechnen. Daneben gibt es \textbf{Topologie-erhaltende Methoden}, die
% gezielt die Topologie der Mannigfaltigkeit erhalten wollen und dies durch geometrische Überlegungen
% erzielen \parencite[4]{Gracia.2014}. Zu dieser Kategorie gehört beispielsweise das Locally Linear Embedding
% (\secref{ch:MethodenDerDimRed:statistisch:LLE}), die Self-Organizing Map \parencite{Kohonen.1990} oder die Uniform Manifold Approximation and Projection (UMAP) \parencite{McInnes.2018}.

Neben den in dieser Arbeit vorgestellten Methoden gibt es noch viele weitere Algorithmen. In der
Kategorie der nicht-konvexen Methoden sind dabei das \textit{Sammon Mapping} \parencite{Sammon.1969} und eine weitere Variante von neuronalen Netzen, den \textit{Self-Organizing
	Maps} \parencite{Kohonen.1990} zu nennen. Eine weitere wichtige konvexe Methode ist das
\textit{Multidimensional Scaling} \parencites{Kruskal.1964}{Cox.2008}, welches in der klassischen Form im Ergebnis jedoch sehr ähnlich
zur Principal Component Analysis ist. Des Weiteren sind in dieser Kategorie noch \textit{Isomap} \parencite{Tenenbaum.2000} und Erweiterungen von Locally Linear Embedding wie \textit{Hessian Locally
	Linear Embedding} \parencite{Donoho.2003} oder \textit{Local Tangent Space Alignment} \parencite{Zhang.2002} zu nennen. \textit{Maximum Variance Unfolding} \parencite{Weinberger.2006}, auch bekannt als \textit{Semidefinite Embedding}, erweitert die Kernel
Principal Component Analysis, indem die Kernel-Matrix gelernt wird.

Wie zu erkennen ist, gibt es eine große Auswahl an verschieden Methoden der Dimensionsreduktion.
Nach dem \newterm{No Free Lunch Theorem} \parencite{Wolpert.1997} gibt es jedoch keinen \enquote{besten} Algorithmus für jede Situation,
weswegen eine Methode domänenspezifisch ausgewählt werden muss. Zur Reduktion von Speicher- und
Rechenkapazitäten und aufgrund des Fluchs der Dimensionalität kann dies trotzdem vorteilhaft sein.
Mit dem Hintergrundwissen zu wichtigen Konzepten wie der Mannigfaltigkeit und der intrinsischen
Dimension (\secref{ch:Dimensionsreduktion:MannigfaltigkeitenIntrinsDim}) können im Folgenden die
einzelnen Methoden vorgestellt werden.