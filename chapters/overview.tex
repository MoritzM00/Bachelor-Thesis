%% ==============================
\section{Übersicht über die Methoden}
\label{sec:Übersicht}
%% ==============================
	Wir können sehen, dass die drei Methoden auf diesem Datensatz bis auf kleine Abweichungen sehr ähnliche Ergebnisse liefern.
	In Abbildung \ref{fig:Residualmatrizen} sind die absoluten Abweichungen zwischen der reproduzierten und empirischen Korrelationsmatrix dargestellt. Hier fällt auf, dass die iterierte Hauptachsen-Faktorisierung deutlich besser abschneidet als die Hauptkomponenten-Methode.
	\begin{figure}[h]
		%% Creator: Matplotlib, PGF backend
%%
%% To include the figure in your LaTeX document, write
%%   \input{<filename>.pgf}
%%
%% Make sure the required packages are loaded in your preamble
%%   \usepackage{pgf}
%%
%% Also ensure that all the required font packages are loaded; for instance,
%% the lmodern package is sometimes necessary when using math font.
%%   \usepackage{lmodern}
%%
%% Figures using additional raster images can only be included by \input if
%% they are in the same directory as the main LaTeX file. For loading figures
%% from other directories you can use the `import` package
%%   \usepackage{import}
%%
%% and then include the figures with
%%   \import{<path to file>}{<filename>.pgf}
%%
%% Matplotlib used the following preamble
%%
\begingroup%
\makeatletter%
\begin{pgfpicture}%
\pgfpathrectangle{\pgfpointorigin}{\pgfqpoint{6.694230in}{4.500000in}}%
\pgfusepath{use as bounding box, clip}%
\begin{pgfscope}%
\pgfsetbuttcap%
\pgfsetmiterjoin%
\pgfsetlinewidth{0.000000pt}%
\definecolor{currentstroke}{rgb}{1.000000,1.000000,1.000000}%
\pgfsetstrokecolor{currentstroke}%
\pgfsetstrokeopacity{0.000000}%
\pgfsetdash{}{0pt}%
\pgfpathmoveto{\pgfqpoint{0.000000in}{0.000000in}}%
\pgfpathlineto{\pgfqpoint{6.694230in}{0.000000in}}%
\pgfpathlineto{\pgfqpoint{6.694230in}{4.500000in}}%
\pgfpathlineto{\pgfqpoint{0.000000in}{4.500000in}}%
\pgfpathlineto{\pgfqpoint{0.000000in}{0.000000in}}%
\pgfpathclose%
\pgfusepath{}%
\end{pgfscope}%
\begin{pgfscope}%
\pgfsetbuttcap%
\pgfsetmiterjoin%
\definecolor{currentfill}{rgb}{1.000000,1.000000,1.000000}%
\pgfsetfillcolor{currentfill}%
\pgfsetlinewidth{0.000000pt}%
\definecolor{currentstroke}{rgb}{0.000000,0.000000,0.000000}%
\pgfsetstrokecolor{currentstroke}%
\pgfsetstrokeopacity{0.000000}%
\pgfsetdash{}{0pt}%
\pgfpathmoveto{\pgfqpoint{0.316667in}{0.772276in}}%
\pgfpathlineto{\pgfqpoint{3.272115in}{0.772276in}}%
\pgfpathlineto{\pgfqpoint{3.272115in}{3.727724in}}%
\pgfpathlineto{\pgfqpoint{0.316667in}{3.727724in}}%
\pgfpathlineto{\pgfqpoint{0.316667in}{0.772276in}}%
\pgfpathclose%
\pgfusepath{fill}%
\end{pgfscope}%
\begin{pgfscope}%
\pgfpathrectangle{\pgfqpoint{0.316667in}{0.772276in}}{\pgfqpoint{2.955448in}{2.955448in}}%
\pgfusepath{clip}%
\pgfsetbuttcap%
\pgfsetroundjoin%
\pgfsetlinewidth{0.000000pt}%
\definecolor{currentstroke}{rgb}{1.000000,1.000000,1.000000}%
\pgfsetstrokecolor{currentstroke}%
\pgfsetdash{}{0pt}%
\pgfpathmoveto{\pgfqpoint{0.316667in}{3.727724in}}%
\pgfpathlineto{\pgfqpoint{0.645050in}{3.727724in}}%
\pgfpathlineto{\pgfqpoint{0.645050in}{3.399341in}}%
\pgfpathlineto{\pgfqpoint{0.316667in}{3.399341in}}%
\pgfpathlineto{\pgfqpoint{0.316667in}{3.727724in}}%
\pgfusepath{}%
\end{pgfscope}%
\begin{pgfscope}%
\pgfpathrectangle{\pgfqpoint{0.316667in}{0.772276in}}{\pgfqpoint{2.955448in}{2.955448in}}%
\pgfusepath{clip}%
\pgfsetbuttcap%
\pgfsetroundjoin%
\pgfsetlinewidth{0.000000pt}%
\definecolor{currentstroke}{rgb}{1.000000,1.000000,1.000000}%
\pgfsetstrokecolor{currentstroke}%
\pgfsetdash{}{0pt}%
\pgfpathmoveto{\pgfqpoint{0.645050in}{3.727724in}}%
\pgfpathlineto{\pgfqpoint{0.973433in}{3.727724in}}%
\pgfpathlineto{\pgfqpoint{0.973433in}{3.399341in}}%
\pgfpathlineto{\pgfqpoint{0.645050in}{3.399341in}}%
\pgfpathlineto{\pgfqpoint{0.645050in}{3.727724in}}%
\pgfusepath{}%
\end{pgfscope}%
\begin{pgfscope}%
\pgfpathrectangle{\pgfqpoint{0.316667in}{0.772276in}}{\pgfqpoint{2.955448in}{2.955448in}}%
\pgfusepath{clip}%
\pgfsetbuttcap%
\pgfsetroundjoin%
\pgfsetlinewidth{0.000000pt}%
\definecolor{currentstroke}{rgb}{1.000000,1.000000,1.000000}%
\pgfsetstrokecolor{currentstroke}%
\pgfsetdash{}{0pt}%
\pgfpathmoveto{\pgfqpoint{0.973433in}{3.727724in}}%
\pgfpathlineto{\pgfqpoint{1.301816in}{3.727724in}}%
\pgfpathlineto{\pgfqpoint{1.301816in}{3.399341in}}%
\pgfpathlineto{\pgfqpoint{0.973433in}{3.399341in}}%
\pgfpathlineto{\pgfqpoint{0.973433in}{3.727724in}}%
\pgfusepath{}%
\end{pgfscope}%
\begin{pgfscope}%
\pgfpathrectangle{\pgfqpoint{0.316667in}{0.772276in}}{\pgfqpoint{2.955448in}{2.955448in}}%
\pgfusepath{clip}%
\pgfsetbuttcap%
\pgfsetroundjoin%
\pgfsetlinewidth{0.000000pt}%
\definecolor{currentstroke}{rgb}{1.000000,1.000000,1.000000}%
\pgfsetstrokecolor{currentstroke}%
\pgfsetdash{}{0pt}%
\pgfpathmoveto{\pgfqpoint{1.301816in}{3.727724in}}%
\pgfpathlineto{\pgfqpoint{1.630199in}{3.727724in}}%
\pgfpathlineto{\pgfqpoint{1.630199in}{3.399341in}}%
\pgfpathlineto{\pgfqpoint{1.301816in}{3.399341in}}%
\pgfpathlineto{\pgfqpoint{1.301816in}{3.727724in}}%
\pgfusepath{}%
\end{pgfscope}%
\begin{pgfscope}%
\pgfpathrectangle{\pgfqpoint{0.316667in}{0.772276in}}{\pgfqpoint{2.955448in}{2.955448in}}%
\pgfusepath{clip}%
\pgfsetbuttcap%
\pgfsetroundjoin%
\pgfsetlinewidth{0.000000pt}%
\definecolor{currentstroke}{rgb}{1.000000,1.000000,1.000000}%
\pgfsetstrokecolor{currentstroke}%
\pgfsetdash{}{0pt}%
\pgfpathmoveto{\pgfqpoint{1.630199in}{3.727724in}}%
\pgfpathlineto{\pgfqpoint{1.958583in}{3.727724in}}%
\pgfpathlineto{\pgfqpoint{1.958583in}{3.399341in}}%
\pgfpathlineto{\pgfqpoint{1.630199in}{3.399341in}}%
\pgfpathlineto{\pgfqpoint{1.630199in}{3.727724in}}%
\pgfusepath{}%
\end{pgfscope}%
\begin{pgfscope}%
\pgfpathrectangle{\pgfqpoint{0.316667in}{0.772276in}}{\pgfqpoint{2.955448in}{2.955448in}}%
\pgfusepath{clip}%
\pgfsetbuttcap%
\pgfsetroundjoin%
\pgfsetlinewidth{0.000000pt}%
\definecolor{currentstroke}{rgb}{1.000000,1.000000,1.000000}%
\pgfsetstrokecolor{currentstroke}%
\pgfsetdash{}{0pt}%
\pgfpathmoveto{\pgfqpoint{1.958583in}{3.727724in}}%
\pgfpathlineto{\pgfqpoint{2.286966in}{3.727724in}}%
\pgfpathlineto{\pgfqpoint{2.286966in}{3.399341in}}%
\pgfpathlineto{\pgfqpoint{1.958583in}{3.399341in}}%
\pgfpathlineto{\pgfqpoint{1.958583in}{3.727724in}}%
\pgfusepath{}%
\end{pgfscope}%
\begin{pgfscope}%
\pgfpathrectangle{\pgfqpoint{0.316667in}{0.772276in}}{\pgfqpoint{2.955448in}{2.955448in}}%
\pgfusepath{clip}%
\pgfsetbuttcap%
\pgfsetroundjoin%
\pgfsetlinewidth{0.000000pt}%
\definecolor{currentstroke}{rgb}{1.000000,1.000000,1.000000}%
\pgfsetstrokecolor{currentstroke}%
\pgfsetdash{}{0pt}%
\pgfpathmoveto{\pgfqpoint{2.286966in}{3.727724in}}%
\pgfpathlineto{\pgfqpoint{2.615349in}{3.727724in}}%
\pgfpathlineto{\pgfqpoint{2.615349in}{3.399341in}}%
\pgfpathlineto{\pgfqpoint{2.286966in}{3.399341in}}%
\pgfpathlineto{\pgfqpoint{2.286966in}{3.727724in}}%
\pgfusepath{}%
\end{pgfscope}%
\begin{pgfscope}%
\pgfpathrectangle{\pgfqpoint{0.316667in}{0.772276in}}{\pgfqpoint{2.955448in}{2.955448in}}%
\pgfusepath{clip}%
\pgfsetbuttcap%
\pgfsetroundjoin%
\pgfsetlinewidth{0.000000pt}%
\definecolor{currentstroke}{rgb}{1.000000,1.000000,1.000000}%
\pgfsetstrokecolor{currentstroke}%
\pgfsetdash{}{0pt}%
\pgfpathmoveto{\pgfqpoint{2.615349in}{3.727724in}}%
\pgfpathlineto{\pgfqpoint{2.943732in}{3.727724in}}%
\pgfpathlineto{\pgfqpoint{2.943732in}{3.399341in}}%
\pgfpathlineto{\pgfqpoint{2.615349in}{3.399341in}}%
\pgfpathlineto{\pgfqpoint{2.615349in}{3.727724in}}%
\pgfusepath{}%
\end{pgfscope}%
\begin{pgfscope}%
\pgfpathrectangle{\pgfqpoint{0.316667in}{0.772276in}}{\pgfqpoint{2.955448in}{2.955448in}}%
\pgfusepath{clip}%
\pgfsetbuttcap%
\pgfsetroundjoin%
\pgfsetlinewidth{0.000000pt}%
\definecolor{currentstroke}{rgb}{1.000000,1.000000,1.000000}%
\pgfsetstrokecolor{currentstroke}%
\pgfsetdash{}{0pt}%
\pgfpathmoveto{\pgfqpoint{2.943732in}{3.727724in}}%
\pgfpathlineto{\pgfqpoint{3.272115in}{3.727724in}}%
\pgfpathlineto{\pgfqpoint{3.272115in}{3.399341in}}%
\pgfpathlineto{\pgfqpoint{2.943732in}{3.399341in}}%
\pgfpathlineto{\pgfqpoint{2.943732in}{3.727724in}}%
\pgfusepath{}%
\end{pgfscope}%
\begin{pgfscope}%
\pgfpathrectangle{\pgfqpoint{0.316667in}{0.772276in}}{\pgfqpoint{2.955448in}{2.955448in}}%
\pgfusepath{clip}%
\pgfsetbuttcap%
\pgfsetroundjoin%
\definecolor{currentfill}{rgb}{0.839216,0.939608,0.931765}%
\pgfsetfillcolor{currentfill}%
\pgfsetlinewidth{0.000000pt}%
\definecolor{currentstroke}{rgb}{1.000000,1.000000,1.000000}%
\pgfsetstrokecolor{currentstroke}%
\pgfsetdash{}{0pt}%
\pgfpathmoveto{\pgfqpoint{0.316667in}{3.399341in}}%
\pgfpathlineto{\pgfqpoint{0.645050in}{3.399341in}}%
\pgfpathlineto{\pgfqpoint{0.645050in}{3.070958in}}%
\pgfpathlineto{\pgfqpoint{0.316667in}{3.070958in}}%
\pgfpathlineto{\pgfqpoint{0.316667in}{3.399341in}}%
\pgfusepath{fill}%
\end{pgfscope}%
\begin{pgfscope}%
\pgfpathrectangle{\pgfqpoint{0.316667in}{0.772276in}}{\pgfqpoint{2.955448in}{2.955448in}}%
\pgfusepath{clip}%
\pgfsetbuttcap%
\pgfsetroundjoin%
\pgfsetlinewidth{0.000000pt}%
\definecolor{currentstroke}{rgb}{1.000000,1.000000,1.000000}%
\pgfsetstrokecolor{currentstroke}%
\pgfsetdash{}{0pt}%
\pgfpathmoveto{\pgfqpoint{0.645050in}{3.399341in}}%
\pgfpathlineto{\pgfqpoint{0.973433in}{3.399341in}}%
\pgfpathlineto{\pgfqpoint{0.973433in}{3.070958in}}%
\pgfpathlineto{\pgfqpoint{0.645050in}{3.070958in}}%
\pgfpathlineto{\pgfqpoint{0.645050in}{3.399341in}}%
\pgfusepath{}%
\end{pgfscope}%
\begin{pgfscope}%
\pgfpathrectangle{\pgfqpoint{0.316667in}{0.772276in}}{\pgfqpoint{2.955448in}{2.955448in}}%
\pgfusepath{clip}%
\pgfsetbuttcap%
\pgfsetroundjoin%
\pgfsetlinewidth{0.000000pt}%
\definecolor{currentstroke}{rgb}{1.000000,1.000000,1.000000}%
\pgfsetstrokecolor{currentstroke}%
\pgfsetdash{}{0pt}%
\pgfpathmoveto{\pgfqpoint{0.973433in}{3.399341in}}%
\pgfpathlineto{\pgfqpoint{1.301816in}{3.399341in}}%
\pgfpathlineto{\pgfqpoint{1.301816in}{3.070958in}}%
\pgfpathlineto{\pgfqpoint{0.973433in}{3.070958in}}%
\pgfpathlineto{\pgfqpoint{0.973433in}{3.399341in}}%
\pgfusepath{}%
\end{pgfscope}%
\begin{pgfscope}%
\pgfpathrectangle{\pgfqpoint{0.316667in}{0.772276in}}{\pgfqpoint{2.955448in}{2.955448in}}%
\pgfusepath{clip}%
\pgfsetbuttcap%
\pgfsetroundjoin%
\pgfsetlinewidth{0.000000pt}%
\definecolor{currentstroke}{rgb}{1.000000,1.000000,1.000000}%
\pgfsetstrokecolor{currentstroke}%
\pgfsetdash{}{0pt}%
\pgfpathmoveto{\pgfqpoint{1.301816in}{3.399341in}}%
\pgfpathlineto{\pgfqpoint{1.630199in}{3.399341in}}%
\pgfpathlineto{\pgfqpoint{1.630199in}{3.070958in}}%
\pgfpathlineto{\pgfqpoint{1.301816in}{3.070958in}}%
\pgfpathlineto{\pgfqpoint{1.301816in}{3.399341in}}%
\pgfusepath{}%
\end{pgfscope}%
\begin{pgfscope}%
\pgfpathrectangle{\pgfqpoint{0.316667in}{0.772276in}}{\pgfqpoint{2.955448in}{2.955448in}}%
\pgfusepath{clip}%
\pgfsetbuttcap%
\pgfsetroundjoin%
\pgfsetlinewidth{0.000000pt}%
\definecolor{currentstroke}{rgb}{1.000000,1.000000,1.000000}%
\pgfsetstrokecolor{currentstroke}%
\pgfsetdash{}{0pt}%
\pgfpathmoveto{\pgfqpoint{1.630199in}{3.399341in}}%
\pgfpathlineto{\pgfqpoint{1.958583in}{3.399341in}}%
\pgfpathlineto{\pgfqpoint{1.958583in}{3.070958in}}%
\pgfpathlineto{\pgfqpoint{1.630199in}{3.070958in}}%
\pgfpathlineto{\pgfqpoint{1.630199in}{3.399341in}}%
\pgfusepath{}%
\end{pgfscope}%
\begin{pgfscope}%
\pgfpathrectangle{\pgfqpoint{0.316667in}{0.772276in}}{\pgfqpoint{2.955448in}{2.955448in}}%
\pgfusepath{clip}%
\pgfsetbuttcap%
\pgfsetroundjoin%
\pgfsetlinewidth{0.000000pt}%
\definecolor{currentstroke}{rgb}{1.000000,1.000000,1.000000}%
\pgfsetstrokecolor{currentstroke}%
\pgfsetdash{}{0pt}%
\pgfpathmoveto{\pgfqpoint{1.958583in}{3.399341in}}%
\pgfpathlineto{\pgfqpoint{2.286966in}{3.399341in}}%
\pgfpathlineto{\pgfqpoint{2.286966in}{3.070958in}}%
\pgfpathlineto{\pgfqpoint{1.958583in}{3.070958in}}%
\pgfpathlineto{\pgfqpoint{1.958583in}{3.399341in}}%
\pgfusepath{}%
\end{pgfscope}%
\begin{pgfscope}%
\pgfpathrectangle{\pgfqpoint{0.316667in}{0.772276in}}{\pgfqpoint{2.955448in}{2.955448in}}%
\pgfusepath{clip}%
\pgfsetbuttcap%
\pgfsetroundjoin%
\pgfsetlinewidth{0.000000pt}%
\definecolor{currentstroke}{rgb}{1.000000,1.000000,1.000000}%
\pgfsetstrokecolor{currentstroke}%
\pgfsetdash{}{0pt}%
\pgfpathmoveto{\pgfqpoint{2.286966in}{3.399341in}}%
\pgfpathlineto{\pgfqpoint{2.615349in}{3.399341in}}%
\pgfpathlineto{\pgfqpoint{2.615349in}{3.070958in}}%
\pgfpathlineto{\pgfqpoint{2.286966in}{3.070958in}}%
\pgfpathlineto{\pgfqpoint{2.286966in}{3.399341in}}%
\pgfusepath{}%
\end{pgfscope}%
\begin{pgfscope}%
\pgfpathrectangle{\pgfqpoint{0.316667in}{0.772276in}}{\pgfqpoint{2.955448in}{2.955448in}}%
\pgfusepath{clip}%
\pgfsetbuttcap%
\pgfsetroundjoin%
\pgfsetlinewidth{0.000000pt}%
\definecolor{currentstroke}{rgb}{1.000000,1.000000,1.000000}%
\pgfsetstrokecolor{currentstroke}%
\pgfsetdash{}{0pt}%
\pgfpathmoveto{\pgfqpoint{2.615349in}{3.399341in}}%
\pgfpathlineto{\pgfqpoint{2.943732in}{3.399341in}}%
\pgfpathlineto{\pgfqpoint{2.943732in}{3.070958in}}%
\pgfpathlineto{\pgfqpoint{2.615349in}{3.070958in}}%
\pgfpathlineto{\pgfqpoint{2.615349in}{3.399341in}}%
\pgfusepath{}%
\end{pgfscope}%
\begin{pgfscope}%
\pgfpathrectangle{\pgfqpoint{0.316667in}{0.772276in}}{\pgfqpoint{2.955448in}{2.955448in}}%
\pgfusepath{clip}%
\pgfsetbuttcap%
\pgfsetroundjoin%
\pgfsetlinewidth{0.000000pt}%
\definecolor{currentstroke}{rgb}{1.000000,1.000000,1.000000}%
\pgfsetstrokecolor{currentstroke}%
\pgfsetdash{}{0pt}%
\pgfpathmoveto{\pgfqpoint{2.943732in}{3.399341in}}%
\pgfpathlineto{\pgfqpoint{3.272115in}{3.399341in}}%
\pgfpathlineto{\pgfqpoint{3.272115in}{3.070958in}}%
\pgfpathlineto{\pgfqpoint{2.943732in}{3.070958in}}%
\pgfpathlineto{\pgfqpoint{2.943732in}{3.399341in}}%
\pgfusepath{}%
\end{pgfscope}%
\begin{pgfscope}%
\pgfpathrectangle{\pgfqpoint{0.316667in}{0.772276in}}{\pgfqpoint{2.955448in}{2.955448in}}%
\pgfusepath{clip}%
\pgfsetbuttcap%
\pgfsetroundjoin%
\definecolor{currentfill}{rgb}{0.839216,0.939608,0.931765}%
\pgfsetfillcolor{currentfill}%
\pgfsetlinewidth{0.000000pt}%
\definecolor{currentstroke}{rgb}{1.000000,1.000000,1.000000}%
\pgfsetstrokecolor{currentstroke}%
\pgfsetdash{}{0pt}%
\pgfpathmoveto{\pgfqpoint{0.316667in}{3.070958in}}%
\pgfpathlineto{\pgfqpoint{0.645050in}{3.070958in}}%
\pgfpathlineto{\pgfqpoint{0.645050in}{2.742575in}}%
\pgfpathlineto{\pgfqpoint{0.316667in}{2.742575in}}%
\pgfpathlineto{\pgfqpoint{0.316667in}{3.070958in}}%
\pgfusepath{fill}%
\end{pgfscope}%
\begin{pgfscope}%
\pgfpathrectangle{\pgfqpoint{0.316667in}{0.772276in}}{\pgfqpoint{2.955448in}{2.955448in}}%
\pgfusepath{clip}%
\pgfsetbuttcap%
\pgfsetroundjoin%
\definecolor{currentfill}{rgb}{0.560000,0.829804,0.759216}%
\pgfsetfillcolor{currentfill}%
\pgfsetlinewidth{0.000000pt}%
\definecolor{currentstroke}{rgb}{1.000000,1.000000,1.000000}%
\pgfsetstrokecolor{currentstroke}%
\pgfsetdash{}{0pt}%
\pgfpathmoveto{\pgfqpoint{0.645050in}{3.070958in}}%
\pgfpathlineto{\pgfqpoint{0.973433in}{3.070958in}}%
\pgfpathlineto{\pgfqpoint{0.973433in}{2.742575in}}%
\pgfpathlineto{\pgfqpoint{0.645050in}{2.742575in}}%
\pgfpathlineto{\pgfqpoint{0.645050in}{3.070958in}}%
\pgfusepath{fill}%
\end{pgfscope}%
\begin{pgfscope}%
\pgfpathrectangle{\pgfqpoint{0.316667in}{0.772276in}}{\pgfqpoint{2.955448in}{2.955448in}}%
\pgfusepath{clip}%
\pgfsetbuttcap%
\pgfsetroundjoin%
\pgfsetlinewidth{0.000000pt}%
\definecolor{currentstroke}{rgb}{1.000000,1.000000,1.000000}%
\pgfsetstrokecolor{currentstroke}%
\pgfsetdash{}{0pt}%
\pgfpathmoveto{\pgfqpoint{0.973433in}{3.070958in}}%
\pgfpathlineto{\pgfqpoint{1.301816in}{3.070958in}}%
\pgfpathlineto{\pgfqpoint{1.301816in}{2.742575in}}%
\pgfpathlineto{\pgfqpoint{0.973433in}{2.742575in}}%
\pgfpathlineto{\pgfqpoint{0.973433in}{3.070958in}}%
\pgfusepath{}%
\end{pgfscope}%
\begin{pgfscope}%
\pgfpathrectangle{\pgfqpoint{0.316667in}{0.772276in}}{\pgfqpoint{2.955448in}{2.955448in}}%
\pgfusepath{clip}%
\pgfsetbuttcap%
\pgfsetroundjoin%
\pgfsetlinewidth{0.000000pt}%
\definecolor{currentstroke}{rgb}{1.000000,1.000000,1.000000}%
\pgfsetstrokecolor{currentstroke}%
\pgfsetdash{}{0pt}%
\pgfpathmoveto{\pgfqpoint{1.301816in}{3.070958in}}%
\pgfpathlineto{\pgfqpoint{1.630199in}{3.070958in}}%
\pgfpathlineto{\pgfqpoint{1.630199in}{2.742575in}}%
\pgfpathlineto{\pgfqpoint{1.301816in}{2.742575in}}%
\pgfpathlineto{\pgfqpoint{1.301816in}{3.070958in}}%
\pgfusepath{}%
\end{pgfscope}%
\begin{pgfscope}%
\pgfpathrectangle{\pgfqpoint{0.316667in}{0.772276in}}{\pgfqpoint{2.955448in}{2.955448in}}%
\pgfusepath{clip}%
\pgfsetbuttcap%
\pgfsetroundjoin%
\pgfsetlinewidth{0.000000pt}%
\definecolor{currentstroke}{rgb}{1.000000,1.000000,1.000000}%
\pgfsetstrokecolor{currentstroke}%
\pgfsetdash{}{0pt}%
\pgfpathmoveto{\pgfqpoint{1.630199in}{3.070958in}}%
\pgfpathlineto{\pgfqpoint{1.958583in}{3.070958in}}%
\pgfpathlineto{\pgfqpoint{1.958583in}{2.742575in}}%
\pgfpathlineto{\pgfqpoint{1.630199in}{2.742575in}}%
\pgfpathlineto{\pgfqpoint{1.630199in}{3.070958in}}%
\pgfusepath{}%
\end{pgfscope}%
\begin{pgfscope}%
\pgfpathrectangle{\pgfqpoint{0.316667in}{0.772276in}}{\pgfqpoint{2.955448in}{2.955448in}}%
\pgfusepath{clip}%
\pgfsetbuttcap%
\pgfsetroundjoin%
\pgfsetlinewidth{0.000000pt}%
\definecolor{currentstroke}{rgb}{1.000000,1.000000,1.000000}%
\pgfsetstrokecolor{currentstroke}%
\pgfsetdash{}{0pt}%
\pgfpathmoveto{\pgfqpoint{1.958583in}{3.070958in}}%
\pgfpathlineto{\pgfqpoint{2.286966in}{3.070958in}}%
\pgfpathlineto{\pgfqpoint{2.286966in}{2.742575in}}%
\pgfpathlineto{\pgfqpoint{1.958583in}{2.742575in}}%
\pgfpathlineto{\pgfqpoint{1.958583in}{3.070958in}}%
\pgfusepath{}%
\end{pgfscope}%
\begin{pgfscope}%
\pgfpathrectangle{\pgfqpoint{0.316667in}{0.772276in}}{\pgfqpoint{2.955448in}{2.955448in}}%
\pgfusepath{clip}%
\pgfsetbuttcap%
\pgfsetroundjoin%
\pgfsetlinewidth{0.000000pt}%
\definecolor{currentstroke}{rgb}{1.000000,1.000000,1.000000}%
\pgfsetstrokecolor{currentstroke}%
\pgfsetdash{}{0pt}%
\pgfpathmoveto{\pgfqpoint{2.286966in}{3.070958in}}%
\pgfpathlineto{\pgfqpoint{2.615349in}{3.070958in}}%
\pgfpathlineto{\pgfqpoint{2.615349in}{2.742575in}}%
\pgfpathlineto{\pgfqpoint{2.286966in}{2.742575in}}%
\pgfpathlineto{\pgfqpoint{2.286966in}{3.070958in}}%
\pgfusepath{}%
\end{pgfscope}%
\begin{pgfscope}%
\pgfpathrectangle{\pgfqpoint{0.316667in}{0.772276in}}{\pgfqpoint{2.955448in}{2.955448in}}%
\pgfusepath{clip}%
\pgfsetbuttcap%
\pgfsetroundjoin%
\pgfsetlinewidth{0.000000pt}%
\definecolor{currentstroke}{rgb}{1.000000,1.000000,1.000000}%
\pgfsetstrokecolor{currentstroke}%
\pgfsetdash{}{0pt}%
\pgfpathmoveto{\pgfqpoint{2.615349in}{3.070958in}}%
\pgfpathlineto{\pgfqpoint{2.943732in}{3.070958in}}%
\pgfpathlineto{\pgfqpoint{2.943732in}{2.742575in}}%
\pgfpathlineto{\pgfqpoint{2.615349in}{2.742575in}}%
\pgfpathlineto{\pgfqpoint{2.615349in}{3.070958in}}%
\pgfusepath{}%
\end{pgfscope}%
\begin{pgfscope}%
\pgfpathrectangle{\pgfqpoint{0.316667in}{0.772276in}}{\pgfqpoint{2.955448in}{2.955448in}}%
\pgfusepath{clip}%
\pgfsetbuttcap%
\pgfsetroundjoin%
\pgfsetlinewidth{0.000000pt}%
\definecolor{currentstroke}{rgb}{1.000000,1.000000,1.000000}%
\pgfsetstrokecolor{currentstroke}%
\pgfsetdash{}{0pt}%
\pgfpathmoveto{\pgfqpoint{2.943732in}{3.070958in}}%
\pgfpathlineto{\pgfqpoint{3.272115in}{3.070958in}}%
\pgfpathlineto{\pgfqpoint{3.272115in}{2.742575in}}%
\pgfpathlineto{\pgfqpoint{2.943732in}{2.742575in}}%
\pgfpathlineto{\pgfqpoint{2.943732in}{3.070958in}}%
\pgfusepath{}%
\end{pgfscope}%
\begin{pgfscope}%
\pgfpathrectangle{\pgfqpoint{0.316667in}{0.772276in}}{\pgfqpoint{2.955448in}{2.955448in}}%
\pgfusepath{clip}%
\pgfsetbuttcap%
\pgfsetroundjoin%
\definecolor{currentfill}{rgb}{0.891503,0.958431,0.971503}%
\pgfsetfillcolor{currentfill}%
\pgfsetlinewidth{0.000000pt}%
\definecolor{currentstroke}{rgb}{1.000000,1.000000,1.000000}%
\pgfsetstrokecolor{currentstroke}%
\pgfsetdash{}{0pt}%
\pgfpathmoveto{\pgfqpoint{0.316667in}{2.742575in}}%
\pgfpathlineto{\pgfqpoint{0.645050in}{2.742575in}}%
\pgfpathlineto{\pgfqpoint{0.645050in}{2.414192in}}%
\pgfpathlineto{\pgfqpoint{0.316667in}{2.414192in}}%
\pgfpathlineto{\pgfqpoint{0.316667in}{2.742575in}}%
\pgfusepath{fill}%
\end{pgfscope}%
\begin{pgfscope}%
\pgfpathrectangle{\pgfqpoint{0.316667in}{0.772276in}}{\pgfqpoint{2.955448in}{2.955448in}}%
\pgfusepath{clip}%
\pgfsetbuttcap%
\pgfsetroundjoin%
\definecolor{currentfill}{rgb}{0.930980,0.973595,0.983791}%
\pgfsetfillcolor{currentfill}%
\pgfsetlinewidth{0.000000pt}%
\definecolor{currentstroke}{rgb}{1.000000,1.000000,1.000000}%
\pgfsetstrokecolor{currentstroke}%
\pgfsetdash{}{0pt}%
\pgfpathmoveto{\pgfqpoint{0.645050in}{2.742575in}}%
\pgfpathlineto{\pgfqpoint{0.973433in}{2.742575in}}%
\pgfpathlineto{\pgfqpoint{0.973433in}{2.414192in}}%
\pgfpathlineto{\pgfqpoint{0.645050in}{2.414192in}}%
\pgfpathlineto{\pgfqpoint{0.645050in}{2.742575in}}%
\pgfusepath{fill}%
\end{pgfscope}%
\begin{pgfscope}%
\pgfpathrectangle{\pgfqpoint{0.316667in}{0.772276in}}{\pgfqpoint{2.955448in}{2.955448in}}%
\pgfusepath{clip}%
\pgfsetbuttcap%
\pgfsetroundjoin%
\definecolor{currentfill}{rgb}{0.839216,0.939608,0.931765}%
\pgfsetfillcolor{currentfill}%
\pgfsetlinewidth{0.000000pt}%
\definecolor{currentstroke}{rgb}{1.000000,1.000000,1.000000}%
\pgfsetstrokecolor{currentstroke}%
\pgfsetdash{}{0pt}%
\pgfpathmoveto{\pgfqpoint{0.973433in}{2.742575in}}%
\pgfpathlineto{\pgfqpoint{1.301816in}{2.742575in}}%
\pgfpathlineto{\pgfqpoint{1.301816in}{2.414192in}}%
\pgfpathlineto{\pgfqpoint{0.973433in}{2.414192in}}%
\pgfpathlineto{\pgfqpoint{0.973433in}{2.742575in}}%
\pgfusepath{fill}%
\end{pgfscope}%
\begin{pgfscope}%
\pgfpathrectangle{\pgfqpoint{0.316667in}{0.772276in}}{\pgfqpoint{2.955448in}{2.955448in}}%
\pgfusepath{clip}%
\pgfsetbuttcap%
\pgfsetroundjoin%
\pgfsetlinewidth{0.000000pt}%
\definecolor{currentstroke}{rgb}{1.000000,1.000000,1.000000}%
\pgfsetstrokecolor{currentstroke}%
\pgfsetdash{}{0pt}%
\pgfpathmoveto{\pgfqpoint{1.301816in}{2.742575in}}%
\pgfpathlineto{\pgfqpoint{1.630199in}{2.742575in}}%
\pgfpathlineto{\pgfqpoint{1.630199in}{2.414192in}}%
\pgfpathlineto{\pgfqpoint{1.301816in}{2.414192in}}%
\pgfpathlineto{\pgfqpoint{1.301816in}{2.742575in}}%
\pgfusepath{}%
\end{pgfscope}%
\begin{pgfscope}%
\pgfpathrectangle{\pgfqpoint{0.316667in}{0.772276in}}{\pgfqpoint{2.955448in}{2.955448in}}%
\pgfusepath{clip}%
\pgfsetbuttcap%
\pgfsetroundjoin%
\pgfsetlinewidth{0.000000pt}%
\definecolor{currentstroke}{rgb}{1.000000,1.000000,1.000000}%
\pgfsetstrokecolor{currentstroke}%
\pgfsetdash{}{0pt}%
\pgfpathmoveto{\pgfqpoint{1.630199in}{2.742575in}}%
\pgfpathlineto{\pgfqpoint{1.958583in}{2.742575in}}%
\pgfpathlineto{\pgfqpoint{1.958583in}{2.414192in}}%
\pgfpathlineto{\pgfqpoint{1.630199in}{2.414192in}}%
\pgfpathlineto{\pgfqpoint{1.630199in}{2.742575in}}%
\pgfusepath{}%
\end{pgfscope}%
\begin{pgfscope}%
\pgfpathrectangle{\pgfqpoint{0.316667in}{0.772276in}}{\pgfqpoint{2.955448in}{2.955448in}}%
\pgfusepath{clip}%
\pgfsetbuttcap%
\pgfsetroundjoin%
\pgfsetlinewidth{0.000000pt}%
\definecolor{currentstroke}{rgb}{1.000000,1.000000,1.000000}%
\pgfsetstrokecolor{currentstroke}%
\pgfsetdash{}{0pt}%
\pgfpathmoveto{\pgfqpoint{1.958583in}{2.742575in}}%
\pgfpathlineto{\pgfqpoint{2.286966in}{2.742575in}}%
\pgfpathlineto{\pgfqpoint{2.286966in}{2.414192in}}%
\pgfpathlineto{\pgfqpoint{1.958583in}{2.414192in}}%
\pgfpathlineto{\pgfqpoint{1.958583in}{2.742575in}}%
\pgfusepath{}%
\end{pgfscope}%
\begin{pgfscope}%
\pgfpathrectangle{\pgfqpoint{0.316667in}{0.772276in}}{\pgfqpoint{2.955448in}{2.955448in}}%
\pgfusepath{clip}%
\pgfsetbuttcap%
\pgfsetroundjoin%
\pgfsetlinewidth{0.000000pt}%
\definecolor{currentstroke}{rgb}{1.000000,1.000000,1.000000}%
\pgfsetstrokecolor{currentstroke}%
\pgfsetdash{}{0pt}%
\pgfpathmoveto{\pgfqpoint{2.286966in}{2.742575in}}%
\pgfpathlineto{\pgfqpoint{2.615349in}{2.742575in}}%
\pgfpathlineto{\pgfqpoint{2.615349in}{2.414192in}}%
\pgfpathlineto{\pgfqpoint{2.286966in}{2.414192in}}%
\pgfpathlineto{\pgfqpoint{2.286966in}{2.742575in}}%
\pgfusepath{}%
\end{pgfscope}%
\begin{pgfscope}%
\pgfpathrectangle{\pgfqpoint{0.316667in}{0.772276in}}{\pgfqpoint{2.955448in}{2.955448in}}%
\pgfusepath{clip}%
\pgfsetbuttcap%
\pgfsetroundjoin%
\pgfsetlinewidth{0.000000pt}%
\definecolor{currentstroke}{rgb}{1.000000,1.000000,1.000000}%
\pgfsetstrokecolor{currentstroke}%
\pgfsetdash{}{0pt}%
\pgfpathmoveto{\pgfqpoint{2.615349in}{2.742575in}}%
\pgfpathlineto{\pgfqpoint{2.943732in}{2.742575in}}%
\pgfpathlineto{\pgfqpoint{2.943732in}{2.414192in}}%
\pgfpathlineto{\pgfqpoint{2.615349in}{2.414192in}}%
\pgfpathlineto{\pgfqpoint{2.615349in}{2.742575in}}%
\pgfusepath{}%
\end{pgfscope}%
\begin{pgfscope}%
\pgfpathrectangle{\pgfqpoint{0.316667in}{0.772276in}}{\pgfqpoint{2.955448in}{2.955448in}}%
\pgfusepath{clip}%
\pgfsetbuttcap%
\pgfsetroundjoin%
\pgfsetlinewidth{0.000000pt}%
\definecolor{currentstroke}{rgb}{1.000000,1.000000,1.000000}%
\pgfsetstrokecolor{currentstroke}%
\pgfsetdash{}{0pt}%
\pgfpathmoveto{\pgfqpoint{2.943732in}{2.742575in}}%
\pgfpathlineto{\pgfqpoint{3.272115in}{2.742575in}}%
\pgfpathlineto{\pgfqpoint{3.272115in}{2.414192in}}%
\pgfpathlineto{\pgfqpoint{2.943732in}{2.414192in}}%
\pgfpathlineto{\pgfqpoint{2.943732in}{2.742575in}}%
\pgfusepath{}%
\end{pgfscope}%
\begin{pgfscope}%
\pgfpathrectangle{\pgfqpoint{0.316667in}{0.772276in}}{\pgfqpoint{2.955448in}{2.955448in}}%
\pgfusepath{clip}%
\pgfsetbuttcap%
\pgfsetroundjoin%
\definecolor{currentfill}{rgb}{0.968627,0.988235,0.992157}%
\pgfsetfillcolor{currentfill}%
\pgfsetlinewidth{0.000000pt}%
\definecolor{currentstroke}{rgb}{1.000000,1.000000,1.000000}%
\pgfsetstrokecolor{currentstroke}%
\pgfsetdash{}{0pt}%
\pgfpathmoveto{\pgfqpoint{0.316667in}{2.414192in}}%
\pgfpathlineto{\pgfqpoint{0.645050in}{2.414192in}}%
\pgfpathlineto{\pgfqpoint{0.645050in}{2.085808in}}%
\pgfpathlineto{\pgfqpoint{0.316667in}{2.085808in}}%
\pgfpathlineto{\pgfqpoint{0.316667in}{2.414192in}}%
\pgfusepath{fill}%
\end{pgfscope}%
\begin{pgfscope}%
\pgfpathrectangle{\pgfqpoint{0.316667in}{0.772276in}}{\pgfqpoint{2.955448in}{2.955448in}}%
\pgfusepath{clip}%
\pgfsetbuttcap%
\pgfsetroundjoin%
\definecolor{currentfill}{rgb}{0.930980,0.973595,0.983791}%
\pgfsetfillcolor{currentfill}%
\pgfsetlinewidth{0.000000pt}%
\definecolor{currentstroke}{rgb}{1.000000,1.000000,1.000000}%
\pgfsetstrokecolor{currentstroke}%
\pgfsetdash{}{0pt}%
\pgfpathmoveto{\pgfqpoint{0.645050in}{2.414192in}}%
\pgfpathlineto{\pgfqpoint{0.973433in}{2.414192in}}%
\pgfpathlineto{\pgfqpoint{0.973433in}{2.085808in}}%
\pgfpathlineto{\pgfqpoint{0.645050in}{2.085808in}}%
\pgfpathlineto{\pgfqpoint{0.645050in}{2.414192in}}%
\pgfusepath{fill}%
\end{pgfscope}%
\begin{pgfscope}%
\pgfpathrectangle{\pgfqpoint{0.316667in}{0.772276in}}{\pgfqpoint{2.955448in}{2.955448in}}%
\pgfusepath{clip}%
\pgfsetbuttcap%
\pgfsetroundjoin%
\definecolor{currentfill}{rgb}{0.215686,0.636601,0.398693}%
\pgfsetfillcolor{currentfill}%
\pgfsetlinewidth{0.000000pt}%
\definecolor{currentstroke}{rgb}{1.000000,1.000000,1.000000}%
\pgfsetstrokecolor{currentstroke}%
\pgfsetdash{}{0pt}%
\pgfpathmoveto{\pgfqpoint{0.973433in}{2.414192in}}%
\pgfpathlineto{\pgfqpoint{1.301816in}{2.414192in}}%
\pgfpathlineto{\pgfqpoint{1.301816in}{2.085808in}}%
\pgfpathlineto{\pgfqpoint{0.973433in}{2.085808in}}%
\pgfpathlineto{\pgfqpoint{0.973433in}{2.414192in}}%
\pgfusepath{fill}%
\end{pgfscope}%
\begin{pgfscope}%
\pgfpathrectangle{\pgfqpoint{0.316667in}{0.772276in}}{\pgfqpoint{2.955448in}{2.955448in}}%
\pgfusepath{clip}%
\pgfsetbuttcap%
\pgfsetroundjoin%
\definecolor{currentfill}{rgb}{0.930980,0.973595,0.983791}%
\pgfsetfillcolor{currentfill}%
\pgfsetlinewidth{0.000000pt}%
\definecolor{currentstroke}{rgb}{1.000000,1.000000,1.000000}%
\pgfsetstrokecolor{currentstroke}%
\pgfsetdash{}{0pt}%
\pgfpathmoveto{\pgfqpoint{1.301816in}{2.414192in}}%
\pgfpathlineto{\pgfqpoint{1.630199in}{2.414192in}}%
\pgfpathlineto{\pgfqpoint{1.630199in}{2.085808in}}%
\pgfpathlineto{\pgfqpoint{1.301816in}{2.085808in}}%
\pgfpathlineto{\pgfqpoint{1.301816in}{2.414192in}}%
\pgfusepath{fill}%
\end{pgfscope}%
\begin{pgfscope}%
\pgfpathrectangle{\pgfqpoint{0.316667in}{0.772276in}}{\pgfqpoint{2.955448in}{2.955448in}}%
\pgfusepath{clip}%
\pgfsetbuttcap%
\pgfsetroundjoin%
\pgfsetlinewidth{0.000000pt}%
\definecolor{currentstroke}{rgb}{1.000000,1.000000,1.000000}%
\pgfsetstrokecolor{currentstroke}%
\pgfsetdash{}{0pt}%
\pgfpathmoveto{\pgfqpoint{1.630199in}{2.414192in}}%
\pgfpathlineto{\pgfqpoint{1.958583in}{2.414192in}}%
\pgfpathlineto{\pgfqpoint{1.958583in}{2.085808in}}%
\pgfpathlineto{\pgfqpoint{1.630199in}{2.085808in}}%
\pgfpathlineto{\pgfqpoint{1.630199in}{2.414192in}}%
\pgfusepath{}%
\end{pgfscope}%
\begin{pgfscope}%
\pgfpathrectangle{\pgfqpoint{0.316667in}{0.772276in}}{\pgfqpoint{2.955448in}{2.955448in}}%
\pgfusepath{clip}%
\pgfsetbuttcap%
\pgfsetroundjoin%
\pgfsetlinewidth{0.000000pt}%
\definecolor{currentstroke}{rgb}{1.000000,1.000000,1.000000}%
\pgfsetstrokecolor{currentstroke}%
\pgfsetdash{}{0pt}%
\pgfpathmoveto{\pgfqpoint{1.958583in}{2.414192in}}%
\pgfpathlineto{\pgfqpoint{2.286966in}{2.414192in}}%
\pgfpathlineto{\pgfqpoint{2.286966in}{2.085808in}}%
\pgfpathlineto{\pgfqpoint{1.958583in}{2.085808in}}%
\pgfpathlineto{\pgfqpoint{1.958583in}{2.414192in}}%
\pgfusepath{}%
\end{pgfscope}%
\begin{pgfscope}%
\pgfpathrectangle{\pgfqpoint{0.316667in}{0.772276in}}{\pgfqpoint{2.955448in}{2.955448in}}%
\pgfusepath{clip}%
\pgfsetbuttcap%
\pgfsetroundjoin%
\pgfsetlinewidth{0.000000pt}%
\definecolor{currentstroke}{rgb}{1.000000,1.000000,1.000000}%
\pgfsetstrokecolor{currentstroke}%
\pgfsetdash{}{0pt}%
\pgfpathmoveto{\pgfqpoint{2.286966in}{2.414192in}}%
\pgfpathlineto{\pgfqpoint{2.615349in}{2.414192in}}%
\pgfpathlineto{\pgfqpoint{2.615349in}{2.085808in}}%
\pgfpathlineto{\pgfqpoint{2.286966in}{2.085808in}}%
\pgfpathlineto{\pgfqpoint{2.286966in}{2.414192in}}%
\pgfusepath{}%
\end{pgfscope}%
\begin{pgfscope}%
\pgfpathrectangle{\pgfqpoint{0.316667in}{0.772276in}}{\pgfqpoint{2.955448in}{2.955448in}}%
\pgfusepath{clip}%
\pgfsetbuttcap%
\pgfsetroundjoin%
\pgfsetlinewidth{0.000000pt}%
\definecolor{currentstroke}{rgb}{1.000000,1.000000,1.000000}%
\pgfsetstrokecolor{currentstroke}%
\pgfsetdash{}{0pt}%
\pgfpathmoveto{\pgfqpoint{2.615349in}{2.414192in}}%
\pgfpathlineto{\pgfqpoint{2.943732in}{2.414192in}}%
\pgfpathlineto{\pgfqpoint{2.943732in}{2.085808in}}%
\pgfpathlineto{\pgfqpoint{2.615349in}{2.085808in}}%
\pgfpathlineto{\pgfqpoint{2.615349in}{2.414192in}}%
\pgfusepath{}%
\end{pgfscope}%
\begin{pgfscope}%
\pgfpathrectangle{\pgfqpoint{0.316667in}{0.772276in}}{\pgfqpoint{2.955448in}{2.955448in}}%
\pgfusepath{clip}%
\pgfsetbuttcap%
\pgfsetroundjoin%
\pgfsetlinewidth{0.000000pt}%
\definecolor{currentstroke}{rgb}{1.000000,1.000000,1.000000}%
\pgfsetstrokecolor{currentstroke}%
\pgfsetdash{}{0pt}%
\pgfpathmoveto{\pgfqpoint{2.943732in}{2.414192in}}%
\pgfpathlineto{\pgfqpoint{3.272115in}{2.414192in}}%
\pgfpathlineto{\pgfqpoint{3.272115in}{2.085808in}}%
\pgfpathlineto{\pgfqpoint{2.943732in}{2.085808in}}%
\pgfpathlineto{\pgfqpoint{2.943732in}{2.414192in}}%
\pgfusepath{}%
\end{pgfscope}%
\begin{pgfscope}%
\pgfpathrectangle{\pgfqpoint{0.316667in}{0.772276in}}{\pgfqpoint{2.955448in}{2.955448in}}%
\pgfusepath{clip}%
\pgfsetbuttcap%
\pgfsetroundjoin%
\definecolor{currentfill}{rgb}{0.891503,0.958431,0.971503}%
\pgfsetfillcolor{currentfill}%
\pgfsetlinewidth{0.000000pt}%
\definecolor{currentstroke}{rgb}{1.000000,1.000000,1.000000}%
\pgfsetstrokecolor{currentstroke}%
\pgfsetdash{}{0pt}%
\pgfpathmoveto{\pgfqpoint{0.316667in}{2.085808in}}%
\pgfpathlineto{\pgfqpoint{0.645050in}{2.085808in}}%
\pgfpathlineto{\pgfqpoint{0.645050in}{1.757425in}}%
\pgfpathlineto{\pgfqpoint{0.316667in}{1.757425in}}%
\pgfpathlineto{\pgfqpoint{0.316667in}{2.085808in}}%
\pgfusepath{fill}%
\end{pgfscope}%
\begin{pgfscope}%
\pgfpathrectangle{\pgfqpoint{0.316667in}{0.772276in}}{\pgfqpoint{2.955448in}{2.955448in}}%
\pgfusepath{clip}%
\pgfsetbuttcap%
\pgfsetroundjoin%
\definecolor{currentfill}{rgb}{0.930980,0.973595,0.983791}%
\pgfsetfillcolor{currentfill}%
\pgfsetlinewidth{0.000000pt}%
\definecolor{currentstroke}{rgb}{1.000000,1.000000,1.000000}%
\pgfsetstrokecolor{currentstroke}%
\pgfsetdash{}{0pt}%
\pgfpathmoveto{\pgfqpoint{0.645050in}{2.085808in}}%
\pgfpathlineto{\pgfqpoint{0.973433in}{2.085808in}}%
\pgfpathlineto{\pgfqpoint{0.973433in}{1.757425in}}%
\pgfpathlineto{\pgfqpoint{0.645050in}{1.757425in}}%
\pgfpathlineto{\pgfqpoint{0.645050in}{2.085808in}}%
\pgfusepath{fill}%
\end{pgfscope}%
\begin{pgfscope}%
\pgfpathrectangle{\pgfqpoint{0.316667in}{0.772276in}}{\pgfqpoint{2.955448in}{2.955448in}}%
\pgfusepath{clip}%
\pgfsetbuttcap%
\pgfsetroundjoin%
\definecolor{currentfill}{rgb}{0.215686,0.636601,0.398693}%
\pgfsetfillcolor{currentfill}%
\pgfsetlinewidth{0.000000pt}%
\definecolor{currentstroke}{rgb}{1.000000,1.000000,1.000000}%
\pgfsetstrokecolor{currentstroke}%
\pgfsetdash{}{0pt}%
\pgfpathmoveto{\pgfqpoint{0.973433in}{2.085808in}}%
\pgfpathlineto{\pgfqpoint{1.301816in}{2.085808in}}%
\pgfpathlineto{\pgfqpoint{1.301816in}{1.757425in}}%
\pgfpathlineto{\pgfqpoint{0.973433in}{1.757425in}}%
\pgfpathlineto{\pgfqpoint{0.973433in}{2.085808in}}%
\pgfusepath{fill}%
\end{pgfscope}%
\begin{pgfscope}%
\pgfpathrectangle{\pgfqpoint{0.316667in}{0.772276in}}{\pgfqpoint{2.955448in}{2.955448in}}%
\pgfusepath{clip}%
\pgfsetbuttcap%
\pgfsetroundjoin%
\definecolor{currentfill}{rgb}{0.839216,0.939608,0.931765}%
\pgfsetfillcolor{currentfill}%
\pgfsetlinewidth{0.000000pt}%
\definecolor{currentstroke}{rgb}{1.000000,1.000000,1.000000}%
\pgfsetstrokecolor{currentstroke}%
\pgfsetdash{}{0pt}%
\pgfpathmoveto{\pgfqpoint{1.301816in}{2.085808in}}%
\pgfpathlineto{\pgfqpoint{1.630199in}{2.085808in}}%
\pgfpathlineto{\pgfqpoint{1.630199in}{1.757425in}}%
\pgfpathlineto{\pgfqpoint{1.301816in}{1.757425in}}%
\pgfpathlineto{\pgfqpoint{1.301816in}{2.085808in}}%
\pgfusepath{fill}%
\end{pgfscope}%
\begin{pgfscope}%
\pgfpathrectangle{\pgfqpoint{0.316667in}{0.772276in}}{\pgfqpoint{2.955448in}{2.955448in}}%
\pgfusepath{clip}%
\pgfsetbuttcap%
\pgfsetroundjoin%
\definecolor{currentfill}{rgb}{0.773333,0.915033,0.886797}%
\pgfsetfillcolor{currentfill}%
\pgfsetlinewidth{0.000000pt}%
\definecolor{currentstroke}{rgb}{1.000000,1.000000,1.000000}%
\pgfsetstrokecolor{currentstroke}%
\pgfsetdash{}{0pt}%
\pgfpathmoveto{\pgfqpoint{1.630199in}{2.085808in}}%
\pgfpathlineto{\pgfqpoint{1.958583in}{2.085808in}}%
\pgfpathlineto{\pgfqpoint{1.958583in}{1.757425in}}%
\pgfpathlineto{\pgfqpoint{1.630199in}{1.757425in}}%
\pgfpathlineto{\pgfqpoint{1.630199in}{2.085808in}}%
\pgfusepath{fill}%
\end{pgfscope}%
\begin{pgfscope}%
\pgfpathrectangle{\pgfqpoint{0.316667in}{0.772276in}}{\pgfqpoint{2.955448in}{2.955448in}}%
\pgfusepath{clip}%
\pgfsetbuttcap%
\pgfsetroundjoin%
\pgfsetlinewidth{0.000000pt}%
\definecolor{currentstroke}{rgb}{1.000000,1.000000,1.000000}%
\pgfsetstrokecolor{currentstroke}%
\pgfsetdash{}{0pt}%
\pgfpathmoveto{\pgfqpoint{1.958583in}{2.085808in}}%
\pgfpathlineto{\pgfqpoint{2.286966in}{2.085808in}}%
\pgfpathlineto{\pgfqpoint{2.286966in}{1.757425in}}%
\pgfpathlineto{\pgfqpoint{1.958583in}{1.757425in}}%
\pgfpathlineto{\pgfqpoint{1.958583in}{2.085808in}}%
\pgfusepath{}%
\end{pgfscope}%
\begin{pgfscope}%
\pgfpathrectangle{\pgfqpoint{0.316667in}{0.772276in}}{\pgfqpoint{2.955448in}{2.955448in}}%
\pgfusepath{clip}%
\pgfsetbuttcap%
\pgfsetroundjoin%
\pgfsetlinewidth{0.000000pt}%
\definecolor{currentstroke}{rgb}{1.000000,1.000000,1.000000}%
\pgfsetstrokecolor{currentstroke}%
\pgfsetdash{}{0pt}%
\pgfpathmoveto{\pgfqpoint{2.286966in}{2.085808in}}%
\pgfpathlineto{\pgfqpoint{2.615349in}{2.085808in}}%
\pgfpathlineto{\pgfqpoint{2.615349in}{1.757425in}}%
\pgfpathlineto{\pgfqpoint{2.286966in}{1.757425in}}%
\pgfpathlineto{\pgfqpoint{2.286966in}{2.085808in}}%
\pgfusepath{}%
\end{pgfscope}%
\begin{pgfscope}%
\pgfpathrectangle{\pgfqpoint{0.316667in}{0.772276in}}{\pgfqpoint{2.955448in}{2.955448in}}%
\pgfusepath{clip}%
\pgfsetbuttcap%
\pgfsetroundjoin%
\pgfsetlinewidth{0.000000pt}%
\definecolor{currentstroke}{rgb}{1.000000,1.000000,1.000000}%
\pgfsetstrokecolor{currentstroke}%
\pgfsetdash{}{0pt}%
\pgfpathmoveto{\pgfqpoint{2.615349in}{2.085808in}}%
\pgfpathlineto{\pgfqpoint{2.943732in}{2.085808in}}%
\pgfpathlineto{\pgfqpoint{2.943732in}{1.757425in}}%
\pgfpathlineto{\pgfqpoint{2.615349in}{1.757425in}}%
\pgfpathlineto{\pgfqpoint{2.615349in}{2.085808in}}%
\pgfusepath{}%
\end{pgfscope}%
\begin{pgfscope}%
\pgfpathrectangle{\pgfqpoint{0.316667in}{0.772276in}}{\pgfqpoint{2.955448in}{2.955448in}}%
\pgfusepath{clip}%
\pgfsetbuttcap%
\pgfsetroundjoin%
\pgfsetlinewidth{0.000000pt}%
\definecolor{currentstroke}{rgb}{1.000000,1.000000,1.000000}%
\pgfsetstrokecolor{currentstroke}%
\pgfsetdash{}{0pt}%
\pgfpathmoveto{\pgfqpoint{2.943732in}{2.085808in}}%
\pgfpathlineto{\pgfqpoint{3.272115in}{2.085808in}}%
\pgfpathlineto{\pgfqpoint{3.272115in}{1.757425in}}%
\pgfpathlineto{\pgfqpoint{2.943732in}{1.757425in}}%
\pgfpathlineto{\pgfqpoint{2.943732in}{2.085808in}}%
\pgfusepath{}%
\end{pgfscope}%
\begin{pgfscope}%
\pgfpathrectangle{\pgfqpoint{0.316667in}{0.772276in}}{\pgfqpoint{2.955448in}{2.955448in}}%
\pgfusepath{clip}%
\pgfsetbuttcap%
\pgfsetroundjoin%
\definecolor{currentfill}{rgb}{0.930980,0.973595,0.983791}%
\pgfsetfillcolor{currentfill}%
\pgfsetlinewidth{0.000000pt}%
\definecolor{currentstroke}{rgb}{1.000000,1.000000,1.000000}%
\pgfsetstrokecolor{currentstroke}%
\pgfsetdash{}{0pt}%
\pgfpathmoveto{\pgfqpoint{0.316667in}{1.757425in}}%
\pgfpathlineto{\pgfqpoint{0.645050in}{1.757425in}}%
\pgfpathlineto{\pgfqpoint{0.645050in}{1.429042in}}%
\pgfpathlineto{\pgfqpoint{0.316667in}{1.429042in}}%
\pgfpathlineto{\pgfqpoint{0.316667in}{1.757425in}}%
\pgfusepath{fill}%
\end{pgfscope}%
\begin{pgfscope}%
\pgfpathrectangle{\pgfqpoint{0.316667in}{0.772276in}}{\pgfqpoint{2.955448in}{2.955448in}}%
\pgfusepath{clip}%
\pgfsetbuttcap%
\pgfsetroundjoin%
\definecolor{currentfill}{rgb}{0.891503,0.958431,0.971503}%
\pgfsetfillcolor{currentfill}%
\pgfsetlinewidth{0.000000pt}%
\definecolor{currentstroke}{rgb}{1.000000,1.000000,1.000000}%
\pgfsetstrokecolor{currentstroke}%
\pgfsetdash{}{0pt}%
\pgfpathmoveto{\pgfqpoint{0.645050in}{1.757425in}}%
\pgfpathlineto{\pgfqpoint{0.973433in}{1.757425in}}%
\pgfpathlineto{\pgfqpoint{0.973433in}{1.429042in}}%
\pgfpathlineto{\pgfqpoint{0.645050in}{1.429042in}}%
\pgfpathlineto{\pgfqpoint{0.645050in}{1.757425in}}%
\pgfusepath{fill}%
\end{pgfscope}%
\begin{pgfscope}%
\pgfpathrectangle{\pgfqpoint{0.316667in}{0.772276in}}{\pgfqpoint{2.955448in}{2.955448in}}%
\pgfusepath{clip}%
\pgfsetbuttcap%
\pgfsetroundjoin%
\definecolor{currentfill}{rgb}{0.082353,0.498039,0.231373}%
\pgfsetfillcolor{currentfill}%
\pgfsetlinewidth{0.000000pt}%
\definecolor{currentstroke}{rgb}{1.000000,1.000000,1.000000}%
\pgfsetstrokecolor{currentstroke}%
\pgfsetdash{}{0pt}%
\pgfpathmoveto{\pgfqpoint{0.973433in}{1.757425in}}%
\pgfpathlineto{\pgfqpoint{1.301816in}{1.757425in}}%
\pgfpathlineto{\pgfqpoint{1.301816in}{1.429042in}}%
\pgfpathlineto{\pgfqpoint{0.973433in}{1.429042in}}%
\pgfpathlineto{\pgfqpoint{0.973433in}{1.757425in}}%
\pgfusepath{fill}%
\end{pgfscope}%
\begin{pgfscope}%
\pgfpathrectangle{\pgfqpoint{0.316667in}{0.772276in}}{\pgfqpoint{2.955448in}{2.955448in}}%
\pgfusepath{clip}%
\pgfsetbuttcap%
\pgfsetroundjoin%
\definecolor{currentfill}{rgb}{0.891503,0.958431,0.971503}%
\pgfsetfillcolor{currentfill}%
\pgfsetlinewidth{0.000000pt}%
\definecolor{currentstroke}{rgb}{1.000000,1.000000,1.000000}%
\pgfsetstrokecolor{currentstroke}%
\pgfsetdash{}{0pt}%
\pgfpathmoveto{\pgfqpoint{1.301816in}{1.757425in}}%
\pgfpathlineto{\pgfqpoint{1.630199in}{1.757425in}}%
\pgfpathlineto{\pgfqpoint{1.630199in}{1.429042in}}%
\pgfpathlineto{\pgfqpoint{1.301816in}{1.429042in}}%
\pgfpathlineto{\pgfqpoint{1.301816in}{1.757425in}}%
\pgfusepath{fill}%
\end{pgfscope}%
\begin{pgfscope}%
\pgfpathrectangle{\pgfqpoint{0.316667in}{0.772276in}}{\pgfqpoint{2.955448in}{2.955448in}}%
\pgfusepath{clip}%
\pgfsetbuttcap%
\pgfsetroundjoin%
\definecolor{currentfill}{rgb}{0.839216,0.939608,0.931765}%
\pgfsetfillcolor{currentfill}%
\pgfsetlinewidth{0.000000pt}%
\definecolor{currentstroke}{rgb}{1.000000,1.000000,1.000000}%
\pgfsetstrokecolor{currentstroke}%
\pgfsetdash{}{0pt}%
\pgfpathmoveto{\pgfqpoint{1.630199in}{1.757425in}}%
\pgfpathlineto{\pgfqpoint{1.958583in}{1.757425in}}%
\pgfpathlineto{\pgfqpoint{1.958583in}{1.429042in}}%
\pgfpathlineto{\pgfqpoint{1.630199in}{1.429042in}}%
\pgfpathlineto{\pgfqpoint{1.630199in}{1.757425in}}%
\pgfusepath{fill}%
\end{pgfscope}%
\begin{pgfscope}%
\pgfpathrectangle{\pgfqpoint{0.316667in}{0.772276in}}{\pgfqpoint{2.955448in}{2.955448in}}%
\pgfusepath{clip}%
\pgfsetbuttcap%
\pgfsetroundjoin%
\definecolor{currentfill}{rgb}{0.930980,0.973595,0.983791}%
\pgfsetfillcolor{currentfill}%
\pgfsetlinewidth{0.000000pt}%
\definecolor{currentstroke}{rgb}{1.000000,1.000000,1.000000}%
\pgfsetstrokecolor{currentstroke}%
\pgfsetdash{}{0pt}%
\pgfpathmoveto{\pgfqpoint{1.958583in}{1.757425in}}%
\pgfpathlineto{\pgfqpoint{2.286966in}{1.757425in}}%
\pgfpathlineto{\pgfqpoint{2.286966in}{1.429042in}}%
\pgfpathlineto{\pgfqpoint{1.958583in}{1.429042in}}%
\pgfpathlineto{\pgfqpoint{1.958583in}{1.757425in}}%
\pgfusepath{fill}%
\end{pgfscope}%
\begin{pgfscope}%
\pgfpathrectangle{\pgfqpoint{0.316667in}{0.772276in}}{\pgfqpoint{2.955448in}{2.955448in}}%
\pgfusepath{clip}%
\pgfsetbuttcap%
\pgfsetroundjoin%
\pgfsetlinewidth{0.000000pt}%
\definecolor{currentstroke}{rgb}{1.000000,1.000000,1.000000}%
\pgfsetstrokecolor{currentstroke}%
\pgfsetdash{}{0pt}%
\pgfpathmoveto{\pgfqpoint{2.286966in}{1.757425in}}%
\pgfpathlineto{\pgfqpoint{2.615349in}{1.757425in}}%
\pgfpathlineto{\pgfqpoint{2.615349in}{1.429042in}}%
\pgfpathlineto{\pgfqpoint{2.286966in}{1.429042in}}%
\pgfpathlineto{\pgfqpoint{2.286966in}{1.757425in}}%
\pgfusepath{}%
\end{pgfscope}%
\begin{pgfscope}%
\pgfpathrectangle{\pgfqpoint{0.316667in}{0.772276in}}{\pgfqpoint{2.955448in}{2.955448in}}%
\pgfusepath{clip}%
\pgfsetbuttcap%
\pgfsetroundjoin%
\pgfsetlinewidth{0.000000pt}%
\definecolor{currentstroke}{rgb}{1.000000,1.000000,1.000000}%
\pgfsetstrokecolor{currentstroke}%
\pgfsetdash{}{0pt}%
\pgfpathmoveto{\pgfqpoint{2.615349in}{1.757425in}}%
\pgfpathlineto{\pgfqpoint{2.943732in}{1.757425in}}%
\pgfpathlineto{\pgfqpoint{2.943732in}{1.429042in}}%
\pgfpathlineto{\pgfqpoint{2.615349in}{1.429042in}}%
\pgfpathlineto{\pgfqpoint{2.615349in}{1.757425in}}%
\pgfusepath{}%
\end{pgfscope}%
\begin{pgfscope}%
\pgfpathrectangle{\pgfqpoint{0.316667in}{0.772276in}}{\pgfqpoint{2.955448in}{2.955448in}}%
\pgfusepath{clip}%
\pgfsetbuttcap%
\pgfsetroundjoin%
\pgfsetlinewidth{0.000000pt}%
\definecolor{currentstroke}{rgb}{1.000000,1.000000,1.000000}%
\pgfsetstrokecolor{currentstroke}%
\pgfsetdash{}{0pt}%
\pgfpathmoveto{\pgfqpoint{2.943732in}{1.757425in}}%
\pgfpathlineto{\pgfqpoint{3.272115in}{1.757425in}}%
\pgfpathlineto{\pgfqpoint{3.272115in}{1.429042in}}%
\pgfpathlineto{\pgfqpoint{2.943732in}{1.429042in}}%
\pgfpathlineto{\pgfqpoint{2.943732in}{1.757425in}}%
\pgfusepath{}%
\end{pgfscope}%
\begin{pgfscope}%
\pgfpathrectangle{\pgfqpoint{0.316667in}{0.772276in}}{\pgfqpoint{2.955448in}{2.955448in}}%
\pgfusepath{clip}%
\pgfsetbuttcap%
\pgfsetroundjoin%
\definecolor{currentfill}{rgb}{0.891503,0.958431,0.971503}%
\pgfsetfillcolor{currentfill}%
\pgfsetlinewidth{0.000000pt}%
\definecolor{currentstroke}{rgb}{1.000000,1.000000,1.000000}%
\pgfsetstrokecolor{currentstroke}%
\pgfsetdash{}{0pt}%
\pgfpathmoveto{\pgfqpoint{0.316667in}{1.429042in}}%
\pgfpathlineto{\pgfqpoint{0.645050in}{1.429042in}}%
\pgfpathlineto{\pgfqpoint{0.645050in}{1.100659in}}%
\pgfpathlineto{\pgfqpoint{0.316667in}{1.100659in}}%
\pgfpathlineto{\pgfqpoint{0.316667in}{1.429042in}}%
\pgfusepath{fill}%
\end{pgfscope}%
\begin{pgfscope}%
\pgfpathrectangle{\pgfqpoint{0.316667in}{0.772276in}}{\pgfqpoint{2.955448in}{2.955448in}}%
\pgfusepath{clip}%
\pgfsetbuttcap%
\pgfsetroundjoin%
\definecolor{currentfill}{rgb}{0.839216,0.939608,0.931765}%
\pgfsetfillcolor{currentfill}%
\pgfsetlinewidth{0.000000pt}%
\definecolor{currentstroke}{rgb}{1.000000,1.000000,1.000000}%
\pgfsetstrokecolor{currentstroke}%
\pgfsetdash{}{0pt}%
\pgfpathmoveto{\pgfqpoint{0.645050in}{1.429042in}}%
\pgfpathlineto{\pgfqpoint{0.973433in}{1.429042in}}%
\pgfpathlineto{\pgfqpoint{0.973433in}{1.100659in}}%
\pgfpathlineto{\pgfqpoint{0.645050in}{1.100659in}}%
\pgfpathlineto{\pgfqpoint{0.645050in}{1.429042in}}%
\pgfusepath{fill}%
\end{pgfscope}%
\begin{pgfscope}%
\pgfpathrectangle{\pgfqpoint{0.316667in}{0.772276in}}{\pgfqpoint{2.955448in}{2.955448in}}%
\pgfusepath{clip}%
\pgfsetbuttcap%
\pgfsetroundjoin%
\definecolor{currentfill}{rgb}{0.009150,0.435294,0.179085}%
\pgfsetfillcolor{currentfill}%
\pgfsetlinewidth{0.000000pt}%
\definecolor{currentstroke}{rgb}{1.000000,1.000000,1.000000}%
\pgfsetstrokecolor{currentstroke}%
\pgfsetdash{}{0pt}%
\pgfpathmoveto{\pgfqpoint{0.973433in}{1.429042in}}%
\pgfpathlineto{\pgfqpoint{1.301816in}{1.429042in}}%
\pgfpathlineto{\pgfqpoint{1.301816in}{1.100659in}}%
\pgfpathlineto{\pgfqpoint{0.973433in}{1.100659in}}%
\pgfpathlineto{\pgfqpoint{0.973433in}{1.429042in}}%
\pgfusepath{fill}%
\end{pgfscope}%
\begin{pgfscope}%
\pgfpathrectangle{\pgfqpoint{0.316667in}{0.772276in}}{\pgfqpoint{2.955448in}{2.955448in}}%
\pgfusepath{clip}%
\pgfsetbuttcap%
\pgfsetroundjoin%
\definecolor{currentfill}{rgb}{0.891503,0.958431,0.971503}%
\pgfsetfillcolor{currentfill}%
\pgfsetlinewidth{0.000000pt}%
\definecolor{currentstroke}{rgb}{1.000000,1.000000,1.000000}%
\pgfsetstrokecolor{currentstroke}%
\pgfsetdash{}{0pt}%
\pgfpathmoveto{\pgfqpoint{1.301816in}{1.429042in}}%
\pgfpathlineto{\pgfqpoint{1.630199in}{1.429042in}}%
\pgfpathlineto{\pgfqpoint{1.630199in}{1.100659in}}%
\pgfpathlineto{\pgfqpoint{1.301816in}{1.100659in}}%
\pgfpathlineto{\pgfqpoint{1.301816in}{1.429042in}}%
\pgfusepath{fill}%
\end{pgfscope}%
\begin{pgfscope}%
\pgfpathrectangle{\pgfqpoint{0.316667in}{0.772276in}}{\pgfqpoint{2.955448in}{2.955448in}}%
\pgfusepath{clip}%
\pgfsetbuttcap%
\pgfsetroundjoin%
\definecolor{currentfill}{rgb}{0.666667,0.873203,0.826144}%
\pgfsetfillcolor{currentfill}%
\pgfsetlinewidth{0.000000pt}%
\definecolor{currentstroke}{rgb}{1.000000,1.000000,1.000000}%
\pgfsetstrokecolor{currentstroke}%
\pgfsetdash{}{0pt}%
\pgfpathmoveto{\pgfqpoint{1.630199in}{1.429042in}}%
\pgfpathlineto{\pgfqpoint{1.958583in}{1.429042in}}%
\pgfpathlineto{\pgfqpoint{1.958583in}{1.100659in}}%
\pgfpathlineto{\pgfqpoint{1.630199in}{1.100659in}}%
\pgfpathlineto{\pgfqpoint{1.630199in}{1.429042in}}%
\pgfusepath{fill}%
\end{pgfscope}%
\begin{pgfscope}%
\pgfpathrectangle{\pgfqpoint{0.316667in}{0.772276in}}{\pgfqpoint{2.955448in}{2.955448in}}%
\pgfusepath{clip}%
\pgfsetbuttcap%
\pgfsetroundjoin%
\definecolor{currentfill}{rgb}{0.891503,0.958431,0.971503}%
\pgfsetfillcolor{currentfill}%
\pgfsetlinewidth{0.000000pt}%
\definecolor{currentstroke}{rgb}{1.000000,1.000000,1.000000}%
\pgfsetstrokecolor{currentstroke}%
\pgfsetdash{}{0pt}%
\pgfpathmoveto{\pgfqpoint{1.958583in}{1.429042in}}%
\pgfpathlineto{\pgfqpoint{2.286966in}{1.429042in}}%
\pgfpathlineto{\pgfqpoint{2.286966in}{1.100659in}}%
\pgfpathlineto{\pgfqpoint{1.958583in}{1.100659in}}%
\pgfpathlineto{\pgfqpoint{1.958583in}{1.429042in}}%
\pgfusepath{fill}%
\end{pgfscope}%
\begin{pgfscope}%
\pgfpathrectangle{\pgfqpoint{0.316667in}{0.772276in}}{\pgfqpoint{2.955448in}{2.955448in}}%
\pgfusepath{clip}%
\pgfsetbuttcap%
\pgfsetroundjoin%
\definecolor{currentfill}{rgb}{0.773333,0.915033,0.886797}%
\pgfsetfillcolor{currentfill}%
\pgfsetlinewidth{0.000000pt}%
\definecolor{currentstroke}{rgb}{1.000000,1.000000,1.000000}%
\pgfsetstrokecolor{currentstroke}%
\pgfsetdash{}{0pt}%
\pgfpathmoveto{\pgfqpoint{2.286966in}{1.429042in}}%
\pgfpathlineto{\pgfqpoint{2.615349in}{1.429042in}}%
\pgfpathlineto{\pgfqpoint{2.615349in}{1.100659in}}%
\pgfpathlineto{\pgfqpoint{2.286966in}{1.100659in}}%
\pgfpathlineto{\pgfqpoint{2.286966in}{1.429042in}}%
\pgfusepath{fill}%
\end{pgfscope}%
\begin{pgfscope}%
\pgfpathrectangle{\pgfqpoint{0.316667in}{0.772276in}}{\pgfqpoint{2.955448in}{2.955448in}}%
\pgfusepath{clip}%
\pgfsetbuttcap%
\pgfsetroundjoin%
\pgfsetlinewidth{0.000000pt}%
\definecolor{currentstroke}{rgb}{1.000000,1.000000,1.000000}%
\pgfsetstrokecolor{currentstroke}%
\pgfsetdash{}{0pt}%
\pgfpathmoveto{\pgfqpoint{2.615349in}{1.429042in}}%
\pgfpathlineto{\pgfqpoint{2.943732in}{1.429042in}}%
\pgfpathlineto{\pgfqpoint{2.943732in}{1.100659in}}%
\pgfpathlineto{\pgfqpoint{2.615349in}{1.100659in}}%
\pgfpathlineto{\pgfqpoint{2.615349in}{1.429042in}}%
\pgfusepath{}%
\end{pgfscope}%
\begin{pgfscope}%
\pgfpathrectangle{\pgfqpoint{0.316667in}{0.772276in}}{\pgfqpoint{2.955448in}{2.955448in}}%
\pgfusepath{clip}%
\pgfsetbuttcap%
\pgfsetroundjoin%
\pgfsetlinewidth{0.000000pt}%
\definecolor{currentstroke}{rgb}{1.000000,1.000000,1.000000}%
\pgfsetstrokecolor{currentstroke}%
\pgfsetdash{}{0pt}%
\pgfpathmoveto{\pgfqpoint{2.943732in}{1.429042in}}%
\pgfpathlineto{\pgfqpoint{3.272115in}{1.429042in}}%
\pgfpathlineto{\pgfqpoint{3.272115in}{1.100659in}}%
\pgfpathlineto{\pgfqpoint{2.943732in}{1.100659in}}%
\pgfpathlineto{\pgfqpoint{2.943732in}{1.429042in}}%
\pgfusepath{}%
\end{pgfscope}%
\begin{pgfscope}%
\pgfpathrectangle{\pgfqpoint{0.316667in}{0.772276in}}{\pgfqpoint{2.955448in}{2.955448in}}%
\pgfusepath{clip}%
\pgfsetbuttcap%
\pgfsetroundjoin%
\definecolor{currentfill}{rgb}{0.891503,0.958431,0.971503}%
\pgfsetfillcolor{currentfill}%
\pgfsetlinewidth{0.000000pt}%
\definecolor{currentstroke}{rgb}{1.000000,1.000000,1.000000}%
\pgfsetstrokecolor{currentstroke}%
\pgfsetdash{}{0pt}%
\pgfpathmoveto{\pgfqpoint{0.316667in}{1.100659in}}%
\pgfpathlineto{\pgfqpoint{0.645050in}{1.100659in}}%
\pgfpathlineto{\pgfqpoint{0.645050in}{0.772276in}}%
\pgfpathlineto{\pgfqpoint{0.316667in}{0.772276in}}%
\pgfpathlineto{\pgfqpoint{0.316667in}{1.100659in}}%
\pgfusepath{fill}%
\end{pgfscope}%
\begin{pgfscope}%
\pgfpathrectangle{\pgfqpoint{0.316667in}{0.772276in}}{\pgfqpoint{2.955448in}{2.955448in}}%
\pgfusepath{clip}%
\pgfsetbuttcap%
\pgfsetroundjoin%
\definecolor{currentfill}{rgb}{0.891503,0.958431,0.971503}%
\pgfsetfillcolor{currentfill}%
\pgfsetlinewidth{0.000000pt}%
\definecolor{currentstroke}{rgb}{1.000000,1.000000,1.000000}%
\pgfsetstrokecolor{currentstroke}%
\pgfsetdash{}{0pt}%
\pgfpathmoveto{\pgfqpoint{0.645050in}{1.100659in}}%
\pgfpathlineto{\pgfqpoint{0.973433in}{1.100659in}}%
\pgfpathlineto{\pgfqpoint{0.973433in}{0.772276in}}%
\pgfpathlineto{\pgfqpoint{0.645050in}{0.772276in}}%
\pgfpathlineto{\pgfqpoint{0.645050in}{1.100659in}}%
\pgfusepath{fill}%
\end{pgfscope}%
\begin{pgfscope}%
\pgfpathrectangle{\pgfqpoint{0.316667in}{0.772276in}}{\pgfqpoint{2.955448in}{2.955448in}}%
\pgfusepath{clip}%
\pgfsetbuttcap%
\pgfsetroundjoin%
\definecolor{currentfill}{rgb}{0.283922,0.698039,0.498824}%
\pgfsetfillcolor{currentfill}%
\pgfsetlinewidth{0.000000pt}%
\definecolor{currentstroke}{rgb}{1.000000,1.000000,1.000000}%
\pgfsetstrokecolor{currentstroke}%
\pgfsetdash{}{0pt}%
\pgfpathmoveto{\pgfqpoint{0.973433in}{1.100659in}}%
\pgfpathlineto{\pgfqpoint{1.301816in}{1.100659in}}%
\pgfpathlineto{\pgfqpoint{1.301816in}{0.772276in}}%
\pgfpathlineto{\pgfqpoint{0.973433in}{0.772276in}}%
\pgfpathlineto{\pgfqpoint{0.973433in}{1.100659in}}%
\pgfusepath{fill}%
\end{pgfscope}%
\begin{pgfscope}%
\pgfpathrectangle{\pgfqpoint{0.316667in}{0.772276in}}{\pgfqpoint{2.955448in}{2.955448in}}%
\pgfusepath{clip}%
\pgfsetbuttcap%
\pgfsetroundjoin%
\definecolor{currentfill}{rgb}{0.839216,0.939608,0.931765}%
\pgfsetfillcolor{currentfill}%
\pgfsetlinewidth{0.000000pt}%
\definecolor{currentstroke}{rgb}{1.000000,1.000000,1.000000}%
\pgfsetstrokecolor{currentstroke}%
\pgfsetdash{}{0pt}%
\pgfpathmoveto{\pgfqpoint{1.301816in}{1.100659in}}%
\pgfpathlineto{\pgfqpoint{1.630199in}{1.100659in}}%
\pgfpathlineto{\pgfqpoint{1.630199in}{0.772276in}}%
\pgfpathlineto{\pgfqpoint{1.301816in}{0.772276in}}%
\pgfpathlineto{\pgfqpoint{1.301816in}{1.100659in}}%
\pgfusepath{fill}%
\end{pgfscope}%
\begin{pgfscope}%
\pgfpathrectangle{\pgfqpoint{0.316667in}{0.772276in}}{\pgfqpoint{2.955448in}{2.955448in}}%
\pgfusepath{clip}%
\pgfsetbuttcap%
\pgfsetroundjoin%
\definecolor{currentfill}{rgb}{0.773333,0.915033,0.886797}%
\pgfsetfillcolor{currentfill}%
\pgfsetlinewidth{0.000000pt}%
\definecolor{currentstroke}{rgb}{1.000000,1.000000,1.000000}%
\pgfsetstrokecolor{currentstroke}%
\pgfsetdash{}{0pt}%
\pgfpathmoveto{\pgfqpoint{1.630199in}{1.100659in}}%
\pgfpathlineto{\pgfqpoint{1.958583in}{1.100659in}}%
\pgfpathlineto{\pgfqpoint{1.958583in}{0.772276in}}%
\pgfpathlineto{\pgfqpoint{1.630199in}{0.772276in}}%
\pgfpathlineto{\pgfqpoint{1.630199in}{1.100659in}}%
\pgfusepath{fill}%
\end{pgfscope}%
\begin{pgfscope}%
\pgfpathrectangle{\pgfqpoint{0.316667in}{0.772276in}}{\pgfqpoint{2.955448in}{2.955448in}}%
\pgfusepath{clip}%
\pgfsetbuttcap%
\pgfsetroundjoin%
\definecolor{currentfill}{rgb}{0.930980,0.973595,0.983791}%
\pgfsetfillcolor{currentfill}%
\pgfsetlinewidth{0.000000pt}%
\definecolor{currentstroke}{rgb}{1.000000,1.000000,1.000000}%
\pgfsetstrokecolor{currentstroke}%
\pgfsetdash{}{0pt}%
\pgfpathmoveto{\pgfqpoint{1.958583in}{1.100659in}}%
\pgfpathlineto{\pgfqpoint{2.286966in}{1.100659in}}%
\pgfpathlineto{\pgfqpoint{2.286966in}{0.772276in}}%
\pgfpathlineto{\pgfqpoint{1.958583in}{0.772276in}}%
\pgfpathlineto{\pgfqpoint{1.958583in}{1.100659in}}%
\pgfusepath{fill}%
\end{pgfscope}%
\begin{pgfscope}%
\pgfpathrectangle{\pgfqpoint{0.316667in}{0.772276in}}{\pgfqpoint{2.955448in}{2.955448in}}%
\pgfusepath{clip}%
\pgfsetbuttcap%
\pgfsetroundjoin%
\definecolor{currentfill}{rgb}{0.773333,0.915033,0.886797}%
\pgfsetfillcolor{currentfill}%
\pgfsetlinewidth{0.000000pt}%
\definecolor{currentstroke}{rgb}{1.000000,1.000000,1.000000}%
\pgfsetstrokecolor{currentstroke}%
\pgfsetdash{}{0pt}%
\pgfpathmoveto{\pgfqpoint{2.286966in}{1.100659in}}%
\pgfpathlineto{\pgfqpoint{2.615349in}{1.100659in}}%
\pgfpathlineto{\pgfqpoint{2.615349in}{0.772276in}}%
\pgfpathlineto{\pgfqpoint{2.286966in}{0.772276in}}%
\pgfpathlineto{\pgfqpoint{2.286966in}{1.100659in}}%
\pgfusepath{fill}%
\end{pgfscope}%
\begin{pgfscope}%
\pgfpathrectangle{\pgfqpoint{0.316667in}{0.772276in}}{\pgfqpoint{2.955448in}{2.955448in}}%
\pgfusepath{clip}%
\pgfsetbuttcap%
\pgfsetroundjoin%
\definecolor{currentfill}{rgb}{0.000000,0.266667,0.105882}%
\pgfsetfillcolor{currentfill}%
\pgfsetlinewidth{0.000000pt}%
\definecolor{currentstroke}{rgb}{1.000000,1.000000,1.000000}%
\pgfsetstrokecolor{currentstroke}%
\pgfsetdash{}{0pt}%
\pgfpathmoveto{\pgfqpoint{2.615349in}{1.100659in}}%
\pgfpathlineto{\pgfqpoint{2.943732in}{1.100659in}}%
\pgfpathlineto{\pgfqpoint{2.943732in}{0.772276in}}%
\pgfpathlineto{\pgfqpoint{2.615349in}{0.772276in}}%
\pgfpathlineto{\pgfqpoint{2.615349in}{1.100659in}}%
\pgfusepath{fill}%
\end{pgfscope}%
\begin{pgfscope}%
\pgfpathrectangle{\pgfqpoint{0.316667in}{0.772276in}}{\pgfqpoint{2.955448in}{2.955448in}}%
\pgfusepath{clip}%
\pgfsetbuttcap%
\pgfsetroundjoin%
\pgfsetlinewidth{0.000000pt}%
\definecolor{currentstroke}{rgb}{1.000000,1.000000,1.000000}%
\pgfsetstrokecolor{currentstroke}%
\pgfsetdash{}{0pt}%
\pgfpathmoveto{\pgfqpoint{2.943732in}{1.100659in}}%
\pgfpathlineto{\pgfqpoint{3.272115in}{1.100659in}}%
\pgfpathlineto{\pgfqpoint{3.272115in}{0.772276in}}%
\pgfpathlineto{\pgfqpoint{2.943732in}{0.772276in}}%
\pgfpathlineto{\pgfqpoint{2.943732in}{1.100659in}}%
\pgfusepath{}%
\end{pgfscope}%
\begin{pgfscope}%
\pgfsetbuttcap%
\pgfsetroundjoin%
\definecolor{currentfill}{rgb}{0.000000,0.000000,0.000000}%
\pgfsetfillcolor{currentfill}%
\pgfsetlinewidth{0.803000pt}%
\definecolor{currentstroke}{rgb}{0.000000,0.000000,0.000000}%
\pgfsetstrokecolor{currentstroke}%
\pgfsetdash{}{0pt}%
\pgfsys@defobject{currentmarker}{\pgfqpoint{0.000000in}{-0.048611in}}{\pgfqpoint{0.000000in}{0.000000in}}{%
\pgfpathmoveto{\pgfqpoint{0.000000in}{0.000000in}}%
\pgfpathlineto{\pgfqpoint{0.000000in}{-0.048611in}}%
\pgfusepath{stroke,fill}%
}%
\begin{pgfscope}%
\pgfsys@transformshift{0.480859in}{0.772276in}%
\pgfsys@useobject{currentmarker}{}%
\end{pgfscope}%
\end{pgfscope}%
\begin{pgfscope}%
\definecolor{textcolor}{rgb}{0.000000,0.000000,0.000000}%
\pgfsetstrokecolor{textcolor}%
\pgfsetfillcolor{textcolor}%
\pgftext[x=0.480859in,y=0.675054in,,top]{\color{textcolor}\rmfamily\fontsize{10.000000}{12.000000}\selectfont 1}%
\end{pgfscope}%
\begin{pgfscope}%
\pgfsetbuttcap%
\pgfsetroundjoin%
\definecolor{currentfill}{rgb}{0.000000,0.000000,0.000000}%
\pgfsetfillcolor{currentfill}%
\pgfsetlinewidth{0.803000pt}%
\definecolor{currentstroke}{rgb}{0.000000,0.000000,0.000000}%
\pgfsetstrokecolor{currentstroke}%
\pgfsetdash{}{0pt}%
\pgfsys@defobject{currentmarker}{\pgfqpoint{0.000000in}{-0.048611in}}{\pgfqpoint{0.000000in}{0.000000in}}{%
\pgfpathmoveto{\pgfqpoint{0.000000in}{0.000000in}}%
\pgfpathlineto{\pgfqpoint{0.000000in}{-0.048611in}}%
\pgfusepath{stroke,fill}%
}%
\begin{pgfscope}%
\pgfsys@transformshift{0.809242in}{0.772276in}%
\pgfsys@useobject{currentmarker}{}%
\end{pgfscope}%
\end{pgfscope}%
\begin{pgfscope}%
\definecolor{textcolor}{rgb}{0.000000,0.000000,0.000000}%
\pgfsetstrokecolor{textcolor}%
\pgfsetfillcolor{textcolor}%
\pgftext[x=0.809242in,y=0.675054in,,top]{\color{textcolor}\rmfamily\fontsize{10.000000}{12.000000}\selectfont 2}%
\end{pgfscope}%
\begin{pgfscope}%
\pgfsetbuttcap%
\pgfsetroundjoin%
\definecolor{currentfill}{rgb}{0.000000,0.000000,0.000000}%
\pgfsetfillcolor{currentfill}%
\pgfsetlinewidth{0.803000pt}%
\definecolor{currentstroke}{rgb}{0.000000,0.000000,0.000000}%
\pgfsetstrokecolor{currentstroke}%
\pgfsetdash{}{0pt}%
\pgfsys@defobject{currentmarker}{\pgfqpoint{0.000000in}{-0.048611in}}{\pgfqpoint{0.000000in}{0.000000in}}{%
\pgfpathmoveto{\pgfqpoint{0.000000in}{0.000000in}}%
\pgfpathlineto{\pgfqpoint{0.000000in}{-0.048611in}}%
\pgfusepath{stroke,fill}%
}%
\begin{pgfscope}%
\pgfsys@transformshift{1.137625in}{0.772276in}%
\pgfsys@useobject{currentmarker}{}%
\end{pgfscope}%
\end{pgfscope}%
\begin{pgfscope}%
\definecolor{textcolor}{rgb}{0.000000,0.000000,0.000000}%
\pgfsetstrokecolor{textcolor}%
\pgfsetfillcolor{textcolor}%
\pgftext[x=1.137625in,y=0.675054in,,top]{\color{textcolor}\rmfamily\fontsize{10.000000}{12.000000}\selectfont 3}%
\end{pgfscope}%
\begin{pgfscope}%
\pgfsetbuttcap%
\pgfsetroundjoin%
\definecolor{currentfill}{rgb}{0.000000,0.000000,0.000000}%
\pgfsetfillcolor{currentfill}%
\pgfsetlinewidth{0.803000pt}%
\definecolor{currentstroke}{rgb}{0.000000,0.000000,0.000000}%
\pgfsetstrokecolor{currentstroke}%
\pgfsetdash{}{0pt}%
\pgfsys@defobject{currentmarker}{\pgfqpoint{0.000000in}{-0.048611in}}{\pgfqpoint{0.000000in}{0.000000in}}{%
\pgfpathmoveto{\pgfqpoint{0.000000in}{0.000000in}}%
\pgfpathlineto{\pgfqpoint{0.000000in}{-0.048611in}}%
\pgfusepath{stroke,fill}%
}%
\begin{pgfscope}%
\pgfsys@transformshift{1.466008in}{0.772276in}%
\pgfsys@useobject{currentmarker}{}%
\end{pgfscope}%
\end{pgfscope}%
\begin{pgfscope}%
\definecolor{textcolor}{rgb}{0.000000,0.000000,0.000000}%
\pgfsetstrokecolor{textcolor}%
\pgfsetfillcolor{textcolor}%
\pgftext[x=1.466008in,y=0.675054in,,top]{\color{textcolor}\rmfamily\fontsize{10.000000}{12.000000}\selectfont 4}%
\end{pgfscope}%
\begin{pgfscope}%
\pgfsetbuttcap%
\pgfsetroundjoin%
\definecolor{currentfill}{rgb}{0.000000,0.000000,0.000000}%
\pgfsetfillcolor{currentfill}%
\pgfsetlinewidth{0.803000pt}%
\definecolor{currentstroke}{rgb}{0.000000,0.000000,0.000000}%
\pgfsetstrokecolor{currentstroke}%
\pgfsetdash{}{0pt}%
\pgfsys@defobject{currentmarker}{\pgfqpoint{0.000000in}{-0.048611in}}{\pgfqpoint{0.000000in}{0.000000in}}{%
\pgfpathmoveto{\pgfqpoint{0.000000in}{0.000000in}}%
\pgfpathlineto{\pgfqpoint{0.000000in}{-0.048611in}}%
\pgfusepath{stroke,fill}%
}%
\begin{pgfscope}%
\pgfsys@transformshift{1.794391in}{0.772276in}%
\pgfsys@useobject{currentmarker}{}%
\end{pgfscope}%
\end{pgfscope}%
\begin{pgfscope}%
\definecolor{textcolor}{rgb}{0.000000,0.000000,0.000000}%
\pgfsetstrokecolor{textcolor}%
\pgfsetfillcolor{textcolor}%
\pgftext[x=1.794391in,y=0.675054in,,top]{\color{textcolor}\rmfamily\fontsize{10.000000}{12.000000}\selectfont 5}%
\end{pgfscope}%
\begin{pgfscope}%
\pgfsetbuttcap%
\pgfsetroundjoin%
\definecolor{currentfill}{rgb}{0.000000,0.000000,0.000000}%
\pgfsetfillcolor{currentfill}%
\pgfsetlinewidth{0.803000pt}%
\definecolor{currentstroke}{rgb}{0.000000,0.000000,0.000000}%
\pgfsetstrokecolor{currentstroke}%
\pgfsetdash{}{0pt}%
\pgfsys@defobject{currentmarker}{\pgfqpoint{0.000000in}{-0.048611in}}{\pgfqpoint{0.000000in}{0.000000in}}{%
\pgfpathmoveto{\pgfqpoint{0.000000in}{0.000000in}}%
\pgfpathlineto{\pgfqpoint{0.000000in}{-0.048611in}}%
\pgfusepath{stroke,fill}%
}%
\begin{pgfscope}%
\pgfsys@transformshift{2.122774in}{0.772276in}%
\pgfsys@useobject{currentmarker}{}%
\end{pgfscope}%
\end{pgfscope}%
\begin{pgfscope}%
\definecolor{textcolor}{rgb}{0.000000,0.000000,0.000000}%
\pgfsetstrokecolor{textcolor}%
\pgfsetfillcolor{textcolor}%
\pgftext[x=2.122774in,y=0.675054in,,top]{\color{textcolor}\rmfamily\fontsize{10.000000}{12.000000}\selectfont 6}%
\end{pgfscope}%
\begin{pgfscope}%
\pgfsetbuttcap%
\pgfsetroundjoin%
\definecolor{currentfill}{rgb}{0.000000,0.000000,0.000000}%
\pgfsetfillcolor{currentfill}%
\pgfsetlinewidth{0.803000pt}%
\definecolor{currentstroke}{rgb}{0.000000,0.000000,0.000000}%
\pgfsetstrokecolor{currentstroke}%
\pgfsetdash{}{0pt}%
\pgfsys@defobject{currentmarker}{\pgfqpoint{0.000000in}{-0.048611in}}{\pgfqpoint{0.000000in}{0.000000in}}{%
\pgfpathmoveto{\pgfqpoint{0.000000in}{0.000000in}}%
\pgfpathlineto{\pgfqpoint{0.000000in}{-0.048611in}}%
\pgfusepath{stroke,fill}%
}%
\begin{pgfscope}%
\pgfsys@transformshift{2.451157in}{0.772276in}%
\pgfsys@useobject{currentmarker}{}%
\end{pgfscope}%
\end{pgfscope}%
\begin{pgfscope}%
\definecolor{textcolor}{rgb}{0.000000,0.000000,0.000000}%
\pgfsetstrokecolor{textcolor}%
\pgfsetfillcolor{textcolor}%
\pgftext[x=2.451157in,y=0.675054in,,top]{\color{textcolor}\rmfamily\fontsize{10.000000}{12.000000}\selectfont 7}%
\end{pgfscope}%
\begin{pgfscope}%
\pgfsetbuttcap%
\pgfsetroundjoin%
\definecolor{currentfill}{rgb}{0.000000,0.000000,0.000000}%
\pgfsetfillcolor{currentfill}%
\pgfsetlinewidth{0.803000pt}%
\definecolor{currentstroke}{rgb}{0.000000,0.000000,0.000000}%
\pgfsetstrokecolor{currentstroke}%
\pgfsetdash{}{0pt}%
\pgfsys@defobject{currentmarker}{\pgfqpoint{0.000000in}{-0.048611in}}{\pgfqpoint{0.000000in}{0.000000in}}{%
\pgfpathmoveto{\pgfqpoint{0.000000in}{0.000000in}}%
\pgfpathlineto{\pgfqpoint{0.000000in}{-0.048611in}}%
\pgfusepath{stroke,fill}%
}%
\begin{pgfscope}%
\pgfsys@transformshift{2.779540in}{0.772276in}%
\pgfsys@useobject{currentmarker}{}%
\end{pgfscope}%
\end{pgfscope}%
\begin{pgfscope}%
\definecolor{textcolor}{rgb}{0.000000,0.000000,0.000000}%
\pgfsetstrokecolor{textcolor}%
\pgfsetfillcolor{textcolor}%
\pgftext[x=2.779540in,y=0.675054in,,top]{\color{textcolor}\rmfamily\fontsize{10.000000}{12.000000}\selectfont 8}%
\end{pgfscope}%
\begin{pgfscope}%
\pgfsetbuttcap%
\pgfsetroundjoin%
\definecolor{currentfill}{rgb}{0.000000,0.000000,0.000000}%
\pgfsetfillcolor{currentfill}%
\pgfsetlinewidth{0.803000pt}%
\definecolor{currentstroke}{rgb}{0.000000,0.000000,0.000000}%
\pgfsetstrokecolor{currentstroke}%
\pgfsetdash{}{0pt}%
\pgfsys@defobject{currentmarker}{\pgfqpoint{0.000000in}{-0.048611in}}{\pgfqpoint{0.000000in}{0.000000in}}{%
\pgfpathmoveto{\pgfqpoint{0.000000in}{0.000000in}}%
\pgfpathlineto{\pgfqpoint{0.000000in}{-0.048611in}}%
\pgfusepath{stroke,fill}%
}%
\begin{pgfscope}%
\pgfsys@transformshift{3.107923in}{0.772276in}%
\pgfsys@useobject{currentmarker}{}%
\end{pgfscope}%
\end{pgfscope}%
\begin{pgfscope}%
\definecolor{textcolor}{rgb}{0.000000,0.000000,0.000000}%
\pgfsetstrokecolor{textcolor}%
\pgfsetfillcolor{textcolor}%
\pgftext[x=3.107923in,y=0.675054in,,top]{\color{textcolor}\rmfamily\fontsize{10.000000}{12.000000}\selectfont 9}%
\end{pgfscope}%
\begin{pgfscope}%
\pgfsetbuttcap%
\pgfsetroundjoin%
\definecolor{currentfill}{rgb}{0.000000,0.000000,0.000000}%
\pgfsetfillcolor{currentfill}%
\pgfsetlinewidth{0.803000pt}%
\definecolor{currentstroke}{rgb}{0.000000,0.000000,0.000000}%
\pgfsetstrokecolor{currentstroke}%
\pgfsetdash{}{0pt}%
\pgfsys@defobject{currentmarker}{\pgfqpoint{-0.048611in}{0.000000in}}{\pgfqpoint{-0.000000in}{0.000000in}}{%
\pgfpathmoveto{\pgfqpoint{-0.000000in}{0.000000in}}%
\pgfpathlineto{\pgfqpoint{-0.048611in}{0.000000in}}%
\pgfusepath{stroke,fill}%
}%
\begin{pgfscope}%
\pgfsys@transformshift{0.316667in}{3.563532in}%
\pgfsys@useobject{currentmarker}{}%
\end{pgfscope}%
\end{pgfscope}%
\begin{pgfscope}%
\definecolor{textcolor}{rgb}{0.000000,0.000000,0.000000}%
\pgfsetstrokecolor{textcolor}%
\pgfsetfillcolor{textcolor}%
\pgftext[x=0.219445in,y=3.563532in,right,]{\color{textcolor}\rmfamily\fontsize{10.000000}{12.000000}\selectfont 1}%
\end{pgfscope}%
\begin{pgfscope}%
\pgfsetbuttcap%
\pgfsetroundjoin%
\definecolor{currentfill}{rgb}{0.000000,0.000000,0.000000}%
\pgfsetfillcolor{currentfill}%
\pgfsetlinewidth{0.803000pt}%
\definecolor{currentstroke}{rgb}{0.000000,0.000000,0.000000}%
\pgfsetstrokecolor{currentstroke}%
\pgfsetdash{}{0pt}%
\pgfsys@defobject{currentmarker}{\pgfqpoint{-0.048611in}{0.000000in}}{\pgfqpoint{-0.000000in}{0.000000in}}{%
\pgfpathmoveto{\pgfqpoint{-0.000000in}{0.000000in}}%
\pgfpathlineto{\pgfqpoint{-0.048611in}{0.000000in}}%
\pgfusepath{stroke,fill}%
}%
\begin{pgfscope}%
\pgfsys@transformshift{0.316667in}{3.235149in}%
\pgfsys@useobject{currentmarker}{}%
\end{pgfscope}%
\end{pgfscope}%
\begin{pgfscope}%
\definecolor{textcolor}{rgb}{0.000000,0.000000,0.000000}%
\pgfsetstrokecolor{textcolor}%
\pgfsetfillcolor{textcolor}%
\pgftext[x=0.219445in,y=3.235149in,right,]{\color{textcolor}\rmfamily\fontsize{10.000000}{12.000000}\selectfont 2}%
\end{pgfscope}%
\begin{pgfscope}%
\pgfsetbuttcap%
\pgfsetroundjoin%
\definecolor{currentfill}{rgb}{0.000000,0.000000,0.000000}%
\pgfsetfillcolor{currentfill}%
\pgfsetlinewidth{0.803000pt}%
\definecolor{currentstroke}{rgb}{0.000000,0.000000,0.000000}%
\pgfsetstrokecolor{currentstroke}%
\pgfsetdash{}{0pt}%
\pgfsys@defobject{currentmarker}{\pgfqpoint{-0.048611in}{0.000000in}}{\pgfqpoint{-0.000000in}{0.000000in}}{%
\pgfpathmoveto{\pgfqpoint{-0.000000in}{0.000000in}}%
\pgfpathlineto{\pgfqpoint{-0.048611in}{0.000000in}}%
\pgfusepath{stroke,fill}%
}%
\begin{pgfscope}%
\pgfsys@transformshift{0.316667in}{2.906766in}%
\pgfsys@useobject{currentmarker}{}%
\end{pgfscope}%
\end{pgfscope}%
\begin{pgfscope}%
\definecolor{textcolor}{rgb}{0.000000,0.000000,0.000000}%
\pgfsetstrokecolor{textcolor}%
\pgfsetfillcolor{textcolor}%
\pgftext[x=0.219445in,y=2.906766in,right,]{\color{textcolor}\rmfamily\fontsize{10.000000}{12.000000}\selectfont 3}%
\end{pgfscope}%
\begin{pgfscope}%
\pgfsetbuttcap%
\pgfsetroundjoin%
\definecolor{currentfill}{rgb}{0.000000,0.000000,0.000000}%
\pgfsetfillcolor{currentfill}%
\pgfsetlinewidth{0.803000pt}%
\definecolor{currentstroke}{rgb}{0.000000,0.000000,0.000000}%
\pgfsetstrokecolor{currentstroke}%
\pgfsetdash{}{0pt}%
\pgfsys@defobject{currentmarker}{\pgfqpoint{-0.048611in}{0.000000in}}{\pgfqpoint{-0.000000in}{0.000000in}}{%
\pgfpathmoveto{\pgfqpoint{-0.000000in}{0.000000in}}%
\pgfpathlineto{\pgfqpoint{-0.048611in}{0.000000in}}%
\pgfusepath{stroke,fill}%
}%
\begin{pgfscope}%
\pgfsys@transformshift{0.316667in}{2.578383in}%
\pgfsys@useobject{currentmarker}{}%
\end{pgfscope}%
\end{pgfscope}%
\begin{pgfscope}%
\definecolor{textcolor}{rgb}{0.000000,0.000000,0.000000}%
\pgfsetstrokecolor{textcolor}%
\pgfsetfillcolor{textcolor}%
\pgftext[x=0.219445in,y=2.578383in,right,]{\color{textcolor}\rmfamily\fontsize{10.000000}{12.000000}\selectfont 4}%
\end{pgfscope}%
\begin{pgfscope}%
\pgfsetbuttcap%
\pgfsetroundjoin%
\definecolor{currentfill}{rgb}{0.000000,0.000000,0.000000}%
\pgfsetfillcolor{currentfill}%
\pgfsetlinewidth{0.803000pt}%
\definecolor{currentstroke}{rgb}{0.000000,0.000000,0.000000}%
\pgfsetstrokecolor{currentstroke}%
\pgfsetdash{}{0pt}%
\pgfsys@defobject{currentmarker}{\pgfqpoint{-0.048611in}{0.000000in}}{\pgfqpoint{-0.000000in}{0.000000in}}{%
\pgfpathmoveto{\pgfqpoint{-0.000000in}{0.000000in}}%
\pgfpathlineto{\pgfqpoint{-0.048611in}{0.000000in}}%
\pgfusepath{stroke,fill}%
}%
\begin{pgfscope}%
\pgfsys@transformshift{0.316667in}{2.250000in}%
\pgfsys@useobject{currentmarker}{}%
\end{pgfscope}%
\end{pgfscope}%
\begin{pgfscope}%
\definecolor{textcolor}{rgb}{0.000000,0.000000,0.000000}%
\pgfsetstrokecolor{textcolor}%
\pgfsetfillcolor{textcolor}%
\pgftext[x=0.219445in,y=2.250000in,right,]{\color{textcolor}\rmfamily\fontsize{10.000000}{12.000000}\selectfont 5}%
\end{pgfscope}%
\begin{pgfscope}%
\pgfsetbuttcap%
\pgfsetroundjoin%
\definecolor{currentfill}{rgb}{0.000000,0.000000,0.000000}%
\pgfsetfillcolor{currentfill}%
\pgfsetlinewidth{0.803000pt}%
\definecolor{currentstroke}{rgb}{0.000000,0.000000,0.000000}%
\pgfsetstrokecolor{currentstroke}%
\pgfsetdash{}{0pt}%
\pgfsys@defobject{currentmarker}{\pgfqpoint{-0.048611in}{0.000000in}}{\pgfqpoint{-0.000000in}{0.000000in}}{%
\pgfpathmoveto{\pgfqpoint{-0.000000in}{0.000000in}}%
\pgfpathlineto{\pgfqpoint{-0.048611in}{0.000000in}}%
\pgfusepath{stroke,fill}%
}%
\begin{pgfscope}%
\pgfsys@transformshift{0.316667in}{1.921617in}%
\pgfsys@useobject{currentmarker}{}%
\end{pgfscope}%
\end{pgfscope}%
\begin{pgfscope}%
\definecolor{textcolor}{rgb}{0.000000,0.000000,0.000000}%
\pgfsetstrokecolor{textcolor}%
\pgfsetfillcolor{textcolor}%
\pgftext[x=0.219445in,y=1.921617in,right,]{\color{textcolor}\rmfamily\fontsize{10.000000}{12.000000}\selectfont 6}%
\end{pgfscope}%
\begin{pgfscope}%
\pgfsetbuttcap%
\pgfsetroundjoin%
\definecolor{currentfill}{rgb}{0.000000,0.000000,0.000000}%
\pgfsetfillcolor{currentfill}%
\pgfsetlinewidth{0.803000pt}%
\definecolor{currentstroke}{rgb}{0.000000,0.000000,0.000000}%
\pgfsetstrokecolor{currentstroke}%
\pgfsetdash{}{0pt}%
\pgfsys@defobject{currentmarker}{\pgfqpoint{-0.048611in}{0.000000in}}{\pgfqpoint{-0.000000in}{0.000000in}}{%
\pgfpathmoveto{\pgfqpoint{-0.000000in}{0.000000in}}%
\pgfpathlineto{\pgfqpoint{-0.048611in}{0.000000in}}%
\pgfusepath{stroke,fill}%
}%
\begin{pgfscope}%
\pgfsys@transformshift{0.316667in}{1.593234in}%
\pgfsys@useobject{currentmarker}{}%
\end{pgfscope}%
\end{pgfscope}%
\begin{pgfscope}%
\definecolor{textcolor}{rgb}{0.000000,0.000000,0.000000}%
\pgfsetstrokecolor{textcolor}%
\pgfsetfillcolor{textcolor}%
\pgftext[x=0.219445in,y=1.593234in,right,]{\color{textcolor}\rmfamily\fontsize{10.000000}{12.000000}\selectfont 7}%
\end{pgfscope}%
\begin{pgfscope}%
\pgfsetbuttcap%
\pgfsetroundjoin%
\definecolor{currentfill}{rgb}{0.000000,0.000000,0.000000}%
\pgfsetfillcolor{currentfill}%
\pgfsetlinewidth{0.803000pt}%
\definecolor{currentstroke}{rgb}{0.000000,0.000000,0.000000}%
\pgfsetstrokecolor{currentstroke}%
\pgfsetdash{}{0pt}%
\pgfsys@defobject{currentmarker}{\pgfqpoint{-0.048611in}{0.000000in}}{\pgfqpoint{-0.000000in}{0.000000in}}{%
\pgfpathmoveto{\pgfqpoint{-0.000000in}{0.000000in}}%
\pgfpathlineto{\pgfqpoint{-0.048611in}{0.000000in}}%
\pgfusepath{stroke,fill}%
}%
\begin{pgfscope}%
\pgfsys@transformshift{0.316667in}{1.264851in}%
\pgfsys@useobject{currentmarker}{}%
\end{pgfscope}%
\end{pgfscope}%
\begin{pgfscope}%
\definecolor{textcolor}{rgb}{0.000000,0.000000,0.000000}%
\pgfsetstrokecolor{textcolor}%
\pgfsetfillcolor{textcolor}%
\pgftext[x=0.219445in,y=1.264851in,right,]{\color{textcolor}\rmfamily\fontsize{10.000000}{12.000000}\selectfont 8}%
\end{pgfscope}%
\begin{pgfscope}%
\pgfsetbuttcap%
\pgfsetroundjoin%
\definecolor{currentfill}{rgb}{0.000000,0.000000,0.000000}%
\pgfsetfillcolor{currentfill}%
\pgfsetlinewidth{0.803000pt}%
\definecolor{currentstroke}{rgb}{0.000000,0.000000,0.000000}%
\pgfsetstrokecolor{currentstroke}%
\pgfsetdash{}{0pt}%
\pgfsys@defobject{currentmarker}{\pgfqpoint{-0.048611in}{0.000000in}}{\pgfqpoint{-0.000000in}{0.000000in}}{%
\pgfpathmoveto{\pgfqpoint{-0.000000in}{0.000000in}}%
\pgfpathlineto{\pgfqpoint{-0.048611in}{0.000000in}}%
\pgfusepath{stroke,fill}%
}%
\begin{pgfscope}%
\pgfsys@transformshift{0.316667in}{0.936468in}%
\pgfsys@useobject{currentmarker}{}%
\end{pgfscope}%
\end{pgfscope}%
\begin{pgfscope}%
\definecolor{textcolor}{rgb}{0.000000,0.000000,0.000000}%
\pgfsetstrokecolor{textcolor}%
\pgfsetfillcolor{textcolor}%
\pgftext[x=0.219445in,y=0.936468in,right,]{\color{textcolor}\rmfamily\fontsize{10.000000}{12.000000}\selectfont 9}%
\end{pgfscope}%
\begin{pgfscope}%
\definecolor{textcolor}{rgb}{0.150000,0.150000,0.150000}%
\pgfsetstrokecolor{textcolor}%
\pgfsetfillcolor{textcolor}%
\pgftext[x=0.480859in,y=3.235149in,,]{\color{textcolor}\rmfamily\fontsize{10.000000}{12.000000}\selectfont 0.03}%
\end{pgfscope}%
\begin{pgfscope}%
\definecolor{textcolor}{rgb}{0.150000,0.150000,0.150000}%
\pgfsetstrokecolor{textcolor}%
\pgfsetfillcolor{textcolor}%
\pgftext[x=0.480859in,y=2.906766in,,]{\color{textcolor}\rmfamily\fontsize{10.000000}{12.000000}\selectfont 0.03}%
\end{pgfscope}%
\begin{pgfscope}%
\definecolor{textcolor}{rgb}{0.150000,0.150000,0.150000}%
\pgfsetstrokecolor{textcolor}%
\pgfsetfillcolor{textcolor}%
\pgftext[x=0.809242in,y=2.906766in,,]{\color{textcolor}\rmfamily\fontsize{10.000000}{12.000000}\selectfont 0.06}%
\end{pgfscope}%
\begin{pgfscope}%
\definecolor{textcolor}{rgb}{0.150000,0.150000,0.150000}%
\pgfsetstrokecolor{textcolor}%
\pgfsetfillcolor{textcolor}%
\pgftext[x=0.480859in,y=2.578383in,,]{\color{textcolor}\rmfamily\fontsize{10.000000}{12.000000}\selectfont 0.02}%
\end{pgfscope}%
\begin{pgfscope}%
\definecolor{textcolor}{rgb}{0.150000,0.150000,0.150000}%
\pgfsetstrokecolor{textcolor}%
\pgfsetfillcolor{textcolor}%
\pgftext[x=0.809242in,y=2.578383in,,]{\color{textcolor}\rmfamily\fontsize{10.000000}{12.000000}\selectfont 0.01}%
\end{pgfscope}%
\begin{pgfscope}%
\definecolor{textcolor}{rgb}{0.150000,0.150000,0.150000}%
\pgfsetstrokecolor{textcolor}%
\pgfsetfillcolor{textcolor}%
\pgftext[x=1.137625in,y=2.578383in,,]{\color{textcolor}\rmfamily\fontsize{10.000000}{12.000000}\selectfont 0.03}%
\end{pgfscope}%
\begin{pgfscope}%
\definecolor{textcolor}{rgb}{0.150000,0.150000,0.150000}%
\pgfsetstrokecolor{textcolor}%
\pgfsetfillcolor{textcolor}%
\pgftext[x=0.480859in,y=2.250000in,,]{\color{textcolor}\rmfamily\fontsize{10.000000}{12.000000}\selectfont 0}%
\end{pgfscope}%
\begin{pgfscope}%
\definecolor{textcolor}{rgb}{0.150000,0.150000,0.150000}%
\pgfsetstrokecolor{textcolor}%
\pgfsetfillcolor{textcolor}%
\pgftext[x=0.809242in,y=2.250000in,,]{\color{textcolor}\rmfamily\fontsize{10.000000}{12.000000}\selectfont 0.01}%
\end{pgfscope}%
\begin{pgfscope}%
\definecolor{textcolor}{rgb}{1.000000,1.000000,1.000000}%
\pgfsetstrokecolor{textcolor}%
\pgfsetfillcolor{textcolor}%
\pgftext[x=1.137625in,y=2.250000in,,]{\color{textcolor}\rmfamily\fontsize{10.000000}{12.000000}\selectfont 0.1}%
\end{pgfscope}%
\begin{pgfscope}%
\definecolor{textcolor}{rgb}{0.150000,0.150000,0.150000}%
\pgfsetstrokecolor{textcolor}%
\pgfsetfillcolor{textcolor}%
\pgftext[x=1.466008in,y=2.250000in,,]{\color{textcolor}\rmfamily\fontsize{10.000000}{12.000000}\selectfont 0.01}%
\end{pgfscope}%
\begin{pgfscope}%
\definecolor{textcolor}{rgb}{0.150000,0.150000,0.150000}%
\pgfsetstrokecolor{textcolor}%
\pgfsetfillcolor{textcolor}%
\pgftext[x=0.480859in,y=1.921617in,,]{\color{textcolor}\rmfamily\fontsize{10.000000}{12.000000}\selectfont 0.02}%
\end{pgfscope}%
\begin{pgfscope}%
\definecolor{textcolor}{rgb}{0.150000,0.150000,0.150000}%
\pgfsetstrokecolor{textcolor}%
\pgfsetfillcolor{textcolor}%
\pgftext[x=0.809242in,y=1.921617in,,]{\color{textcolor}\rmfamily\fontsize{10.000000}{12.000000}\selectfont 0.01}%
\end{pgfscope}%
\begin{pgfscope}%
\definecolor{textcolor}{rgb}{1.000000,1.000000,1.000000}%
\pgfsetstrokecolor{textcolor}%
\pgfsetfillcolor{textcolor}%
\pgftext[x=1.137625in,y=1.921617in,,]{\color{textcolor}\rmfamily\fontsize{10.000000}{12.000000}\selectfont 0.1}%
\end{pgfscope}%
\begin{pgfscope}%
\definecolor{textcolor}{rgb}{0.150000,0.150000,0.150000}%
\pgfsetstrokecolor{textcolor}%
\pgfsetfillcolor{textcolor}%
\pgftext[x=1.466008in,y=1.921617in,,]{\color{textcolor}\rmfamily\fontsize{10.000000}{12.000000}\selectfont 0.03}%
\end{pgfscope}%
\begin{pgfscope}%
\definecolor{textcolor}{rgb}{0.150000,0.150000,0.150000}%
\pgfsetstrokecolor{textcolor}%
\pgfsetfillcolor{textcolor}%
\pgftext[x=1.794391in,y=1.921617in,,]{\color{textcolor}\rmfamily\fontsize{10.000000}{12.000000}\selectfont 0.04}%
\end{pgfscope}%
\begin{pgfscope}%
\definecolor{textcolor}{rgb}{0.150000,0.150000,0.150000}%
\pgfsetstrokecolor{textcolor}%
\pgfsetfillcolor{textcolor}%
\pgftext[x=0.480859in,y=1.593234in,,]{\color{textcolor}\rmfamily\fontsize{10.000000}{12.000000}\selectfont 0.01}%
\end{pgfscope}%
\begin{pgfscope}%
\definecolor{textcolor}{rgb}{0.150000,0.150000,0.150000}%
\pgfsetstrokecolor{textcolor}%
\pgfsetfillcolor{textcolor}%
\pgftext[x=0.809242in,y=1.593234in,,]{\color{textcolor}\rmfamily\fontsize{10.000000}{12.000000}\selectfont 0.02}%
\end{pgfscope}%
\begin{pgfscope}%
\definecolor{textcolor}{rgb}{1.000000,1.000000,1.000000}%
\pgfsetstrokecolor{textcolor}%
\pgfsetfillcolor{textcolor}%
\pgftext[x=1.137625in,y=1.593234in,,]{\color{textcolor}\rmfamily\fontsize{10.000000}{12.000000}\selectfont 0.12}%
\end{pgfscope}%
\begin{pgfscope}%
\definecolor{textcolor}{rgb}{0.150000,0.150000,0.150000}%
\pgfsetstrokecolor{textcolor}%
\pgfsetfillcolor{textcolor}%
\pgftext[x=1.466008in,y=1.593234in,,]{\color{textcolor}\rmfamily\fontsize{10.000000}{12.000000}\selectfont 0.02}%
\end{pgfscope}%
\begin{pgfscope}%
\definecolor{textcolor}{rgb}{0.150000,0.150000,0.150000}%
\pgfsetstrokecolor{textcolor}%
\pgfsetfillcolor{textcolor}%
\pgftext[x=1.794391in,y=1.593234in,,]{\color{textcolor}\rmfamily\fontsize{10.000000}{12.000000}\selectfont 0.03}%
\end{pgfscope}%
\begin{pgfscope}%
\definecolor{textcolor}{rgb}{0.150000,0.150000,0.150000}%
\pgfsetstrokecolor{textcolor}%
\pgfsetfillcolor{textcolor}%
\pgftext[x=2.122774in,y=1.593234in,,]{\color{textcolor}\rmfamily\fontsize{10.000000}{12.000000}\selectfont 0.01}%
\end{pgfscope}%
\begin{pgfscope}%
\definecolor{textcolor}{rgb}{0.150000,0.150000,0.150000}%
\pgfsetstrokecolor{textcolor}%
\pgfsetfillcolor{textcolor}%
\pgftext[x=0.480859in,y=1.264851in,,]{\color{textcolor}\rmfamily\fontsize{10.000000}{12.000000}\selectfont 0.02}%
\end{pgfscope}%
\begin{pgfscope}%
\definecolor{textcolor}{rgb}{0.150000,0.150000,0.150000}%
\pgfsetstrokecolor{textcolor}%
\pgfsetfillcolor{textcolor}%
\pgftext[x=0.809242in,y=1.264851in,,]{\color{textcolor}\rmfamily\fontsize{10.000000}{12.000000}\selectfont 0.03}%
\end{pgfscope}%
\begin{pgfscope}%
\definecolor{textcolor}{rgb}{1.000000,1.000000,1.000000}%
\pgfsetstrokecolor{textcolor}%
\pgfsetfillcolor{textcolor}%
\pgftext[x=1.137625in,y=1.264851in,,]{\color{textcolor}\rmfamily\fontsize{10.000000}{12.000000}\selectfont 0.13}%
\end{pgfscope}%
\begin{pgfscope}%
\definecolor{textcolor}{rgb}{0.150000,0.150000,0.150000}%
\pgfsetstrokecolor{textcolor}%
\pgfsetfillcolor{textcolor}%
\pgftext[x=1.466008in,y=1.264851in,,]{\color{textcolor}\rmfamily\fontsize{10.000000}{12.000000}\selectfont 0.02}%
\end{pgfscope}%
\begin{pgfscope}%
\definecolor{textcolor}{rgb}{0.150000,0.150000,0.150000}%
\pgfsetstrokecolor{textcolor}%
\pgfsetfillcolor{textcolor}%
\pgftext[x=1.794391in,y=1.264851in,,]{\color{textcolor}\rmfamily\fontsize{10.000000}{12.000000}\selectfont 0.05}%
\end{pgfscope}%
\begin{pgfscope}%
\definecolor{textcolor}{rgb}{0.150000,0.150000,0.150000}%
\pgfsetstrokecolor{textcolor}%
\pgfsetfillcolor{textcolor}%
\pgftext[x=2.122774in,y=1.264851in,,]{\color{textcolor}\rmfamily\fontsize{10.000000}{12.000000}\selectfont 0.02}%
\end{pgfscope}%
\begin{pgfscope}%
\definecolor{textcolor}{rgb}{0.150000,0.150000,0.150000}%
\pgfsetstrokecolor{textcolor}%
\pgfsetfillcolor{textcolor}%
\pgftext[x=2.451157in,y=1.264851in,,]{\color{textcolor}\rmfamily\fontsize{10.000000}{12.000000}\selectfont 0.04}%
\end{pgfscope}%
\begin{pgfscope}%
\definecolor{textcolor}{rgb}{0.150000,0.150000,0.150000}%
\pgfsetstrokecolor{textcolor}%
\pgfsetfillcolor{textcolor}%
\pgftext[x=0.480859in,y=0.936468in,,]{\color{textcolor}\rmfamily\fontsize{10.000000}{12.000000}\selectfont 0.02}%
\end{pgfscope}%
\begin{pgfscope}%
\definecolor{textcolor}{rgb}{0.150000,0.150000,0.150000}%
\pgfsetstrokecolor{textcolor}%
\pgfsetfillcolor{textcolor}%
\pgftext[x=0.809242in,y=0.936468in,,]{\color{textcolor}\rmfamily\fontsize{10.000000}{12.000000}\selectfont 0.02}%
\end{pgfscope}%
\begin{pgfscope}%
\definecolor{textcolor}{rgb}{1.000000,1.000000,1.000000}%
\pgfsetstrokecolor{textcolor}%
\pgfsetfillcolor{textcolor}%
\pgftext[x=1.137625in,y=0.936468in,,]{\color{textcolor}\rmfamily\fontsize{10.000000}{12.000000}\selectfont 0.09}%
\end{pgfscope}%
\begin{pgfscope}%
\definecolor{textcolor}{rgb}{0.150000,0.150000,0.150000}%
\pgfsetstrokecolor{textcolor}%
\pgfsetfillcolor{textcolor}%
\pgftext[x=1.466008in,y=0.936468in,,]{\color{textcolor}\rmfamily\fontsize{10.000000}{12.000000}\selectfont 0.03}%
\end{pgfscope}%
\begin{pgfscope}%
\definecolor{textcolor}{rgb}{0.150000,0.150000,0.150000}%
\pgfsetstrokecolor{textcolor}%
\pgfsetfillcolor{textcolor}%
\pgftext[x=1.794391in,y=0.936468in,,]{\color{textcolor}\rmfamily\fontsize{10.000000}{12.000000}\selectfont 0.04}%
\end{pgfscope}%
\begin{pgfscope}%
\definecolor{textcolor}{rgb}{0.150000,0.150000,0.150000}%
\pgfsetstrokecolor{textcolor}%
\pgfsetfillcolor{textcolor}%
\pgftext[x=2.122774in,y=0.936468in,,]{\color{textcolor}\rmfamily\fontsize{10.000000}{12.000000}\selectfont 0.01}%
\end{pgfscope}%
\begin{pgfscope}%
\definecolor{textcolor}{rgb}{0.150000,0.150000,0.150000}%
\pgfsetstrokecolor{textcolor}%
\pgfsetfillcolor{textcolor}%
\pgftext[x=2.451157in,y=0.936468in,,]{\color{textcolor}\rmfamily\fontsize{10.000000}{12.000000}\selectfont 0.04}%
\end{pgfscope}%
\begin{pgfscope}%
\definecolor{textcolor}{rgb}{1.000000,1.000000,1.000000}%
\pgfsetstrokecolor{textcolor}%
\pgfsetfillcolor{textcolor}%
\pgftext[x=2.779540in,y=0.936468in,,]{\color{textcolor}\rmfamily\fontsize{10.000000}{12.000000}\selectfont 0.15}%
\end{pgfscope}%
\begin{pgfscope}%
\definecolor{textcolor}{rgb}{0.000000,0.000000,0.000000}%
\pgfsetstrokecolor{textcolor}%
\pgfsetfillcolor{textcolor}%
\pgftext[x=1.794391in,y=3.811057in,,base]{\color{textcolor}\rmfamily\fontsize{11.000000}{13.200000}\selectfont PC (RMSE = 0.0545)}%
\end{pgfscope}%
\begin{pgfscope}%
\pgfsetbuttcap%
\pgfsetmiterjoin%
\definecolor{currentfill}{rgb}{1.000000,1.000000,1.000000}%
\pgfsetfillcolor{currentfill}%
\pgfsetlinewidth{0.000000pt}%
\definecolor{currentstroke}{rgb}{0.000000,0.000000,0.000000}%
\pgfsetstrokecolor{currentstroke}%
\pgfsetstrokeopacity{0.000000}%
\pgfsetdash{}{0pt}%
\pgfpathmoveto{\pgfqpoint{3.588782in}{0.772276in}}%
\pgfpathlineto{\pgfqpoint{6.544230in}{0.772276in}}%
\pgfpathlineto{\pgfqpoint{6.544230in}{3.727724in}}%
\pgfpathlineto{\pgfqpoint{3.588782in}{3.727724in}}%
\pgfpathlineto{\pgfqpoint{3.588782in}{0.772276in}}%
\pgfpathclose%
\pgfusepath{fill}%
\end{pgfscope}%
\begin{pgfscope}%
\pgfpathrectangle{\pgfqpoint{3.588782in}{0.772276in}}{\pgfqpoint{2.955448in}{2.955448in}}%
\pgfusepath{clip}%
\pgfsetbuttcap%
\pgfsetroundjoin%
\pgfsetlinewidth{0.000000pt}%
\definecolor{currentstroke}{rgb}{1.000000,1.000000,1.000000}%
\pgfsetstrokecolor{currentstroke}%
\pgfsetdash{}{0pt}%
\pgfpathmoveto{\pgfqpoint{3.588782in}{3.727724in}}%
\pgfpathlineto{\pgfqpoint{3.917165in}{3.727724in}}%
\pgfpathlineto{\pgfqpoint{3.917165in}{3.399341in}}%
\pgfpathlineto{\pgfqpoint{3.588782in}{3.399341in}}%
\pgfpathlineto{\pgfqpoint{3.588782in}{3.727724in}}%
\pgfusepath{}%
\end{pgfscope}%
\begin{pgfscope}%
\pgfpathrectangle{\pgfqpoint{3.588782in}{0.772276in}}{\pgfqpoint{2.955448in}{2.955448in}}%
\pgfusepath{clip}%
\pgfsetbuttcap%
\pgfsetroundjoin%
\pgfsetlinewidth{0.000000pt}%
\definecolor{currentstroke}{rgb}{1.000000,1.000000,1.000000}%
\pgfsetstrokecolor{currentstroke}%
\pgfsetdash{}{0pt}%
\pgfpathmoveto{\pgfqpoint{3.917165in}{3.727724in}}%
\pgfpathlineto{\pgfqpoint{4.245548in}{3.727724in}}%
\pgfpathlineto{\pgfqpoint{4.245548in}{3.399341in}}%
\pgfpathlineto{\pgfqpoint{3.917165in}{3.399341in}}%
\pgfpathlineto{\pgfqpoint{3.917165in}{3.727724in}}%
\pgfusepath{}%
\end{pgfscope}%
\begin{pgfscope}%
\pgfpathrectangle{\pgfqpoint{3.588782in}{0.772276in}}{\pgfqpoint{2.955448in}{2.955448in}}%
\pgfusepath{clip}%
\pgfsetbuttcap%
\pgfsetroundjoin%
\pgfsetlinewidth{0.000000pt}%
\definecolor{currentstroke}{rgb}{1.000000,1.000000,1.000000}%
\pgfsetstrokecolor{currentstroke}%
\pgfsetdash{}{0pt}%
\pgfpathmoveto{\pgfqpoint{4.245548in}{3.727724in}}%
\pgfpathlineto{\pgfqpoint{4.573931in}{3.727724in}}%
\pgfpathlineto{\pgfqpoint{4.573931in}{3.399341in}}%
\pgfpathlineto{\pgfqpoint{4.245548in}{3.399341in}}%
\pgfpathlineto{\pgfqpoint{4.245548in}{3.727724in}}%
\pgfusepath{}%
\end{pgfscope}%
\begin{pgfscope}%
\pgfpathrectangle{\pgfqpoint{3.588782in}{0.772276in}}{\pgfqpoint{2.955448in}{2.955448in}}%
\pgfusepath{clip}%
\pgfsetbuttcap%
\pgfsetroundjoin%
\pgfsetlinewidth{0.000000pt}%
\definecolor{currentstroke}{rgb}{1.000000,1.000000,1.000000}%
\pgfsetstrokecolor{currentstroke}%
\pgfsetdash{}{0pt}%
\pgfpathmoveto{\pgfqpoint{4.573931in}{3.727724in}}%
\pgfpathlineto{\pgfqpoint{4.902314in}{3.727724in}}%
\pgfpathlineto{\pgfqpoint{4.902314in}{3.399341in}}%
\pgfpathlineto{\pgfqpoint{4.573931in}{3.399341in}}%
\pgfpathlineto{\pgfqpoint{4.573931in}{3.727724in}}%
\pgfusepath{}%
\end{pgfscope}%
\begin{pgfscope}%
\pgfpathrectangle{\pgfqpoint{3.588782in}{0.772276in}}{\pgfqpoint{2.955448in}{2.955448in}}%
\pgfusepath{clip}%
\pgfsetbuttcap%
\pgfsetroundjoin%
\pgfsetlinewidth{0.000000pt}%
\definecolor{currentstroke}{rgb}{1.000000,1.000000,1.000000}%
\pgfsetstrokecolor{currentstroke}%
\pgfsetdash{}{0pt}%
\pgfpathmoveto{\pgfqpoint{4.902314in}{3.727724in}}%
\pgfpathlineto{\pgfqpoint{5.230698in}{3.727724in}}%
\pgfpathlineto{\pgfqpoint{5.230698in}{3.399341in}}%
\pgfpathlineto{\pgfqpoint{4.902314in}{3.399341in}}%
\pgfpathlineto{\pgfqpoint{4.902314in}{3.727724in}}%
\pgfusepath{}%
\end{pgfscope}%
\begin{pgfscope}%
\pgfpathrectangle{\pgfqpoint{3.588782in}{0.772276in}}{\pgfqpoint{2.955448in}{2.955448in}}%
\pgfusepath{clip}%
\pgfsetbuttcap%
\pgfsetroundjoin%
\pgfsetlinewidth{0.000000pt}%
\definecolor{currentstroke}{rgb}{1.000000,1.000000,1.000000}%
\pgfsetstrokecolor{currentstroke}%
\pgfsetdash{}{0pt}%
\pgfpathmoveto{\pgfqpoint{5.230698in}{3.727724in}}%
\pgfpathlineto{\pgfqpoint{5.559081in}{3.727724in}}%
\pgfpathlineto{\pgfqpoint{5.559081in}{3.399341in}}%
\pgfpathlineto{\pgfqpoint{5.230698in}{3.399341in}}%
\pgfpathlineto{\pgfqpoint{5.230698in}{3.727724in}}%
\pgfusepath{}%
\end{pgfscope}%
\begin{pgfscope}%
\pgfpathrectangle{\pgfqpoint{3.588782in}{0.772276in}}{\pgfqpoint{2.955448in}{2.955448in}}%
\pgfusepath{clip}%
\pgfsetbuttcap%
\pgfsetroundjoin%
\pgfsetlinewidth{0.000000pt}%
\definecolor{currentstroke}{rgb}{1.000000,1.000000,1.000000}%
\pgfsetstrokecolor{currentstroke}%
\pgfsetdash{}{0pt}%
\pgfpathmoveto{\pgfqpoint{5.559081in}{3.727724in}}%
\pgfpathlineto{\pgfqpoint{5.887464in}{3.727724in}}%
\pgfpathlineto{\pgfqpoint{5.887464in}{3.399341in}}%
\pgfpathlineto{\pgfqpoint{5.559081in}{3.399341in}}%
\pgfpathlineto{\pgfqpoint{5.559081in}{3.727724in}}%
\pgfusepath{}%
\end{pgfscope}%
\begin{pgfscope}%
\pgfpathrectangle{\pgfqpoint{3.588782in}{0.772276in}}{\pgfqpoint{2.955448in}{2.955448in}}%
\pgfusepath{clip}%
\pgfsetbuttcap%
\pgfsetroundjoin%
\pgfsetlinewidth{0.000000pt}%
\definecolor{currentstroke}{rgb}{1.000000,1.000000,1.000000}%
\pgfsetstrokecolor{currentstroke}%
\pgfsetdash{}{0pt}%
\pgfpathmoveto{\pgfqpoint{5.887464in}{3.727724in}}%
\pgfpathlineto{\pgfqpoint{6.215847in}{3.727724in}}%
\pgfpathlineto{\pgfqpoint{6.215847in}{3.399341in}}%
\pgfpathlineto{\pgfqpoint{5.887464in}{3.399341in}}%
\pgfpathlineto{\pgfqpoint{5.887464in}{3.727724in}}%
\pgfusepath{}%
\end{pgfscope}%
\begin{pgfscope}%
\pgfpathrectangle{\pgfqpoint{3.588782in}{0.772276in}}{\pgfqpoint{2.955448in}{2.955448in}}%
\pgfusepath{clip}%
\pgfsetbuttcap%
\pgfsetroundjoin%
\pgfsetlinewidth{0.000000pt}%
\definecolor{currentstroke}{rgb}{1.000000,1.000000,1.000000}%
\pgfsetstrokecolor{currentstroke}%
\pgfsetdash{}{0pt}%
\pgfpathmoveto{\pgfqpoint{6.215847in}{3.727724in}}%
\pgfpathlineto{\pgfqpoint{6.544230in}{3.727724in}}%
\pgfpathlineto{\pgfqpoint{6.544230in}{3.399341in}}%
\pgfpathlineto{\pgfqpoint{6.215847in}{3.399341in}}%
\pgfpathlineto{\pgfqpoint{6.215847in}{3.727724in}}%
\pgfusepath{}%
\end{pgfscope}%
\begin{pgfscope}%
\pgfpathrectangle{\pgfqpoint{3.588782in}{0.772276in}}{\pgfqpoint{2.955448in}{2.955448in}}%
\pgfusepath{clip}%
\pgfsetbuttcap%
\pgfsetroundjoin%
\definecolor{currentfill}{rgb}{0.968627,0.988235,0.992157}%
\pgfsetfillcolor{currentfill}%
\pgfsetlinewidth{0.000000pt}%
\definecolor{currentstroke}{rgb}{1.000000,1.000000,1.000000}%
\pgfsetstrokecolor{currentstroke}%
\pgfsetdash{}{0pt}%
\pgfpathmoveto{\pgfqpoint{3.588782in}{3.399341in}}%
\pgfpathlineto{\pgfqpoint{3.917165in}{3.399341in}}%
\pgfpathlineto{\pgfqpoint{3.917165in}{3.070958in}}%
\pgfpathlineto{\pgfqpoint{3.588782in}{3.070958in}}%
\pgfpathlineto{\pgfqpoint{3.588782in}{3.399341in}}%
\pgfusepath{fill}%
\end{pgfscope}%
\begin{pgfscope}%
\pgfpathrectangle{\pgfqpoint{3.588782in}{0.772276in}}{\pgfqpoint{2.955448in}{2.955448in}}%
\pgfusepath{clip}%
\pgfsetbuttcap%
\pgfsetroundjoin%
\pgfsetlinewidth{0.000000pt}%
\definecolor{currentstroke}{rgb}{1.000000,1.000000,1.000000}%
\pgfsetstrokecolor{currentstroke}%
\pgfsetdash{}{0pt}%
\pgfpathmoveto{\pgfqpoint{3.917165in}{3.399341in}}%
\pgfpathlineto{\pgfqpoint{4.245548in}{3.399341in}}%
\pgfpathlineto{\pgfqpoint{4.245548in}{3.070958in}}%
\pgfpathlineto{\pgfqpoint{3.917165in}{3.070958in}}%
\pgfpathlineto{\pgfqpoint{3.917165in}{3.399341in}}%
\pgfusepath{}%
\end{pgfscope}%
\begin{pgfscope}%
\pgfpathrectangle{\pgfqpoint{3.588782in}{0.772276in}}{\pgfqpoint{2.955448in}{2.955448in}}%
\pgfusepath{clip}%
\pgfsetbuttcap%
\pgfsetroundjoin%
\pgfsetlinewidth{0.000000pt}%
\definecolor{currentstroke}{rgb}{1.000000,1.000000,1.000000}%
\pgfsetstrokecolor{currentstroke}%
\pgfsetdash{}{0pt}%
\pgfpathmoveto{\pgfqpoint{4.245548in}{3.399341in}}%
\pgfpathlineto{\pgfqpoint{4.573931in}{3.399341in}}%
\pgfpathlineto{\pgfqpoint{4.573931in}{3.070958in}}%
\pgfpathlineto{\pgfqpoint{4.245548in}{3.070958in}}%
\pgfpathlineto{\pgfqpoint{4.245548in}{3.399341in}}%
\pgfusepath{}%
\end{pgfscope}%
\begin{pgfscope}%
\pgfpathrectangle{\pgfqpoint{3.588782in}{0.772276in}}{\pgfqpoint{2.955448in}{2.955448in}}%
\pgfusepath{clip}%
\pgfsetbuttcap%
\pgfsetroundjoin%
\pgfsetlinewidth{0.000000pt}%
\definecolor{currentstroke}{rgb}{1.000000,1.000000,1.000000}%
\pgfsetstrokecolor{currentstroke}%
\pgfsetdash{}{0pt}%
\pgfpathmoveto{\pgfqpoint{4.573931in}{3.399341in}}%
\pgfpathlineto{\pgfqpoint{4.902314in}{3.399341in}}%
\pgfpathlineto{\pgfqpoint{4.902314in}{3.070958in}}%
\pgfpathlineto{\pgfqpoint{4.573931in}{3.070958in}}%
\pgfpathlineto{\pgfqpoint{4.573931in}{3.399341in}}%
\pgfusepath{}%
\end{pgfscope}%
\begin{pgfscope}%
\pgfpathrectangle{\pgfqpoint{3.588782in}{0.772276in}}{\pgfqpoint{2.955448in}{2.955448in}}%
\pgfusepath{clip}%
\pgfsetbuttcap%
\pgfsetroundjoin%
\pgfsetlinewidth{0.000000pt}%
\definecolor{currentstroke}{rgb}{1.000000,1.000000,1.000000}%
\pgfsetstrokecolor{currentstroke}%
\pgfsetdash{}{0pt}%
\pgfpathmoveto{\pgfqpoint{4.902314in}{3.399341in}}%
\pgfpathlineto{\pgfqpoint{5.230698in}{3.399341in}}%
\pgfpathlineto{\pgfqpoint{5.230698in}{3.070958in}}%
\pgfpathlineto{\pgfqpoint{4.902314in}{3.070958in}}%
\pgfpathlineto{\pgfqpoint{4.902314in}{3.399341in}}%
\pgfusepath{}%
\end{pgfscope}%
\begin{pgfscope}%
\pgfpathrectangle{\pgfqpoint{3.588782in}{0.772276in}}{\pgfqpoint{2.955448in}{2.955448in}}%
\pgfusepath{clip}%
\pgfsetbuttcap%
\pgfsetroundjoin%
\pgfsetlinewidth{0.000000pt}%
\definecolor{currentstroke}{rgb}{1.000000,1.000000,1.000000}%
\pgfsetstrokecolor{currentstroke}%
\pgfsetdash{}{0pt}%
\pgfpathmoveto{\pgfqpoint{5.230698in}{3.399341in}}%
\pgfpathlineto{\pgfqpoint{5.559081in}{3.399341in}}%
\pgfpathlineto{\pgfqpoint{5.559081in}{3.070958in}}%
\pgfpathlineto{\pgfqpoint{5.230698in}{3.070958in}}%
\pgfpathlineto{\pgfqpoint{5.230698in}{3.399341in}}%
\pgfusepath{}%
\end{pgfscope}%
\begin{pgfscope}%
\pgfpathrectangle{\pgfqpoint{3.588782in}{0.772276in}}{\pgfqpoint{2.955448in}{2.955448in}}%
\pgfusepath{clip}%
\pgfsetbuttcap%
\pgfsetroundjoin%
\pgfsetlinewidth{0.000000pt}%
\definecolor{currentstroke}{rgb}{1.000000,1.000000,1.000000}%
\pgfsetstrokecolor{currentstroke}%
\pgfsetdash{}{0pt}%
\pgfpathmoveto{\pgfqpoint{5.559081in}{3.399341in}}%
\pgfpathlineto{\pgfqpoint{5.887464in}{3.399341in}}%
\pgfpathlineto{\pgfqpoint{5.887464in}{3.070958in}}%
\pgfpathlineto{\pgfqpoint{5.559081in}{3.070958in}}%
\pgfpathlineto{\pgfqpoint{5.559081in}{3.399341in}}%
\pgfusepath{}%
\end{pgfscope}%
\begin{pgfscope}%
\pgfpathrectangle{\pgfqpoint{3.588782in}{0.772276in}}{\pgfqpoint{2.955448in}{2.955448in}}%
\pgfusepath{clip}%
\pgfsetbuttcap%
\pgfsetroundjoin%
\pgfsetlinewidth{0.000000pt}%
\definecolor{currentstroke}{rgb}{1.000000,1.000000,1.000000}%
\pgfsetstrokecolor{currentstroke}%
\pgfsetdash{}{0pt}%
\pgfpathmoveto{\pgfqpoint{5.887464in}{3.399341in}}%
\pgfpathlineto{\pgfqpoint{6.215847in}{3.399341in}}%
\pgfpathlineto{\pgfqpoint{6.215847in}{3.070958in}}%
\pgfpathlineto{\pgfqpoint{5.887464in}{3.070958in}}%
\pgfpathlineto{\pgfqpoint{5.887464in}{3.399341in}}%
\pgfusepath{}%
\end{pgfscope}%
\begin{pgfscope}%
\pgfpathrectangle{\pgfqpoint{3.588782in}{0.772276in}}{\pgfqpoint{2.955448in}{2.955448in}}%
\pgfusepath{clip}%
\pgfsetbuttcap%
\pgfsetroundjoin%
\pgfsetlinewidth{0.000000pt}%
\definecolor{currentstroke}{rgb}{1.000000,1.000000,1.000000}%
\pgfsetstrokecolor{currentstroke}%
\pgfsetdash{}{0pt}%
\pgfpathmoveto{\pgfqpoint{6.215847in}{3.399341in}}%
\pgfpathlineto{\pgfqpoint{6.544230in}{3.399341in}}%
\pgfpathlineto{\pgfqpoint{6.544230in}{3.070958in}}%
\pgfpathlineto{\pgfqpoint{6.215847in}{3.070958in}}%
\pgfpathlineto{\pgfqpoint{6.215847in}{3.399341in}}%
\pgfusepath{}%
\end{pgfscope}%
\begin{pgfscope}%
\pgfpathrectangle{\pgfqpoint{3.588782in}{0.772276in}}{\pgfqpoint{2.955448in}{2.955448in}}%
\pgfusepath{clip}%
\pgfsetbuttcap%
\pgfsetroundjoin%
\definecolor{currentfill}{rgb}{0.534902,0.818977,0.741007}%
\pgfsetfillcolor{currentfill}%
\pgfsetlinewidth{0.000000pt}%
\definecolor{currentstroke}{rgb}{1.000000,1.000000,1.000000}%
\pgfsetstrokecolor{currentstroke}%
\pgfsetdash{}{0pt}%
\pgfpathmoveto{\pgfqpoint{3.588782in}{3.070958in}}%
\pgfpathlineto{\pgfqpoint{3.917165in}{3.070958in}}%
\pgfpathlineto{\pgfqpoint{3.917165in}{2.742575in}}%
\pgfpathlineto{\pgfqpoint{3.588782in}{2.742575in}}%
\pgfpathlineto{\pgfqpoint{3.588782in}{3.070958in}}%
\pgfusepath{fill}%
\end{pgfscope}%
\begin{pgfscope}%
\pgfpathrectangle{\pgfqpoint{3.588782in}{0.772276in}}{\pgfqpoint{2.955448in}{2.955448in}}%
\pgfusepath{clip}%
\pgfsetbuttcap%
\pgfsetroundjoin%
\definecolor{currentfill}{rgb}{0.534902,0.818977,0.741007}%
\pgfsetfillcolor{currentfill}%
\pgfsetlinewidth{0.000000pt}%
\definecolor{currentstroke}{rgb}{1.000000,1.000000,1.000000}%
\pgfsetstrokecolor{currentstroke}%
\pgfsetdash{}{0pt}%
\pgfpathmoveto{\pgfqpoint{3.917165in}{3.070958in}}%
\pgfpathlineto{\pgfqpoint{4.245548in}{3.070958in}}%
\pgfpathlineto{\pgfqpoint{4.245548in}{2.742575in}}%
\pgfpathlineto{\pgfqpoint{3.917165in}{2.742575in}}%
\pgfpathlineto{\pgfqpoint{3.917165in}{3.070958in}}%
\pgfusepath{fill}%
\end{pgfscope}%
\begin{pgfscope}%
\pgfpathrectangle{\pgfqpoint{3.588782in}{0.772276in}}{\pgfqpoint{2.955448in}{2.955448in}}%
\pgfusepath{clip}%
\pgfsetbuttcap%
\pgfsetroundjoin%
\pgfsetlinewidth{0.000000pt}%
\definecolor{currentstroke}{rgb}{1.000000,1.000000,1.000000}%
\pgfsetstrokecolor{currentstroke}%
\pgfsetdash{}{0pt}%
\pgfpathmoveto{\pgfqpoint{4.245548in}{3.070958in}}%
\pgfpathlineto{\pgfqpoint{4.573931in}{3.070958in}}%
\pgfpathlineto{\pgfqpoint{4.573931in}{2.742575in}}%
\pgfpathlineto{\pgfqpoint{4.245548in}{2.742575in}}%
\pgfpathlineto{\pgfqpoint{4.245548in}{3.070958in}}%
\pgfusepath{}%
\end{pgfscope}%
\begin{pgfscope}%
\pgfpathrectangle{\pgfqpoint{3.588782in}{0.772276in}}{\pgfqpoint{2.955448in}{2.955448in}}%
\pgfusepath{clip}%
\pgfsetbuttcap%
\pgfsetroundjoin%
\pgfsetlinewidth{0.000000pt}%
\definecolor{currentstroke}{rgb}{1.000000,1.000000,1.000000}%
\pgfsetstrokecolor{currentstroke}%
\pgfsetdash{}{0pt}%
\pgfpathmoveto{\pgfqpoint{4.573931in}{3.070958in}}%
\pgfpathlineto{\pgfqpoint{4.902314in}{3.070958in}}%
\pgfpathlineto{\pgfqpoint{4.902314in}{2.742575in}}%
\pgfpathlineto{\pgfqpoint{4.573931in}{2.742575in}}%
\pgfpathlineto{\pgfqpoint{4.573931in}{3.070958in}}%
\pgfusepath{}%
\end{pgfscope}%
\begin{pgfscope}%
\pgfpathrectangle{\pgfqpoint{3.588782in}{0.772276in}}{\pgfqpoint{2.955448in}{2.955448in}}%
\pgfusepath{clip}%
\pgfsetbuttcap%
\pgfsetroundjoin%
\pgfsetlinewidth{0.000000pt}%
\definecolor{currentstroke}{rgb}{1.000000,1.000000,1.000000}%
\pgfsetstrokecolor{currentstroke}%
\pgfsetdash{}{0pt}%
\pgfpathmoveto{\pgfqpoint{4.902314in}{3.070958in}}%
\pgfpathlineto{\pgfqpoint{5.230698in}{3.070958in}}%
\pgfpathlineto{\pgfqpoint{5.230698in}{2.742575in}}%
\pgfpathlineto{\pgfqpoint{4.902314in}{2.742575in}}%
\pgfpathlineto{\pgfqpoint{4.902314in}{3.070958in}}%
\pgfusepath{}%
\end{pgfscope}%
\begin{pgfscope}%
\pgfpathrectangle{\pgfqpoint{3.588782in}{0.772276in}}{\pgfqpoint{2.955448in}{2.955448in}}%
\pgfusepath{clip}%
\pgfsetbuttcap%
\pgfsetroundjoin%
\pgfsetlinewidth{0.000000pt}%
\definecolor{currentstroke}{rgb}{1.000000,1.000000,1.000000}%
\pgfsetstrokecolor{currentstroke}%
\pgfsetdash{}{0pt}%
\pgfpathmoveto{\pgfqpoint{5.230698in}{3.070958in}}%
\pgfpathlineto{\pgfqpoint{5.559081in}{3.070958in}}%
\pgfpathlineto{\pgfqpoint{5.559081in}{2.742575in}}%
\pgfpathlineto{\pgfqpoint{5.230698in}{2.742575in}}%
\pgfpathlineto{\pgfqpoint{5.230698in}{3.070958in}}%
\pgfusepath{}%
\end{pgfscope}%
\begin{pgfscope}%
\pgfpathrectangle{\pgfqpoint{3.588782in}{0.772276in}}{\pgfqpoint{2.955448in}{2.955448in}}%
\pgfusepath{clip}%
\pgfsetbuttcap%
\pgfsetroundjoin%
\pgfsetlinewidth{0.000000pt}%
\definecolor{currentstroke}{rgb}{1.000000,1.000000,1.000000}%
\pgfsetstrokecolor{currentstroke}%
\pgfsetdash{}{0pt}%
\pgfpathmoveto{\pgfqpoint{5.559081in}{3.070958in}}%
\pgfpathlineto{\pgfqpoint{5.887464in}{3.070958in}}%
\pgfpathlineto{\pgfqpoint{5.887464in}{2.742575in}}%
\pgfpathlineto{\pgfqpoint{5.559081in}{2.742575in}}%
\pgfpathlineto{\pgfqpoint{5.559081in}{3.070958in}}%
\pgfusepath{}%
\end{pgfscope}%
\begin{pgfscope}%
\pgfpathrectangle{\pgfqpoint{3.588782in}{0.772276in}}{\pgfqpoint{2.955448in}{2.955448in}}%
\pgfusepath{clip}%
\pgfsetbuttcap%
\pgfsetroundjoin%
\pgfsetlinewidth{0.000000pt}%
\definecolor{currentstroke}{rgb}{1.000000,1.000000,1.000000}%
\pgfsetstrokecolor{currentstroke}%
\pgfsetdash{}{0pt}%
\pgfpathmoveto{\pgfqpoint{5.887464in}{3.070958in}}%
\pgfpathlineto{\pgfqpoint{6.215847in}{3.070958in}}%
\pgfpathlineto{\pgfqpoint{6.215847in}{2.742575in}}%
\pgfpathlineto{\pgfqpoint{5.887464in}{2.742575in}}%
\pgfpathlineto{\pgfqpoint{5.887464in}{3.070958in}}%
\pgfusepath{}%
\end{pgfscope}%
\begin{pgfscope}%
\pgfpathrectangle{\pgfqpoint{3.588782in}{0.772276in}}{\pgfqpoint{2.955448in}{2.955448in}}%
\pgfusepath{clip}%
\pgfsetbuttcap%
\pgfsetroundjoin%
\pgfsetlinewidth{0.000000pt}%
\definecolor{currentstroke}{rgb}{1.000000,1.000000,1.000000}%
\pgfsetstrokecolor{currentstroke}%
\pgfsetdash{}{0pt}%
\pgfpathmoveto{\pgfqpoint{6.215847in}{3.070958in}}%
\pgfpathlineto{\pgfqpoint{6.544230in}{3.070958in}}%
\pgfpathlineto{\pgfqpoint{6.544230in}{2.742575in}}%
\pgfpathlineto{\pgfqpoint{6.215847in}{2.742575in}}%
\pgfpathlineto{\pgfqpoint{6.215847in}{3.070958in}}%
\pgfusepath{}%
\end{pgfscope}%
\begin{pgfscope}%
\pgfpathrectangle{\pgfqpoint{3.588782in}{0.772276in}}{\pgfqpoint{2.955448in}{2.955448in}}%
\pgfusepath{clip}%
\pgfsetbuttcap%
\pgfsetroundjoin%
\definecolor{currentfill}{rgb}{0.922122,0.970150,0.981822}%
\pgfsetfillcolor{currentfill}%
\pgfsetlinewidth{0.000000pt}%
\definecolor{currentstroke}{rgb}{1.000000,1.000000,1.000000}%
\pgfsetstrokecolor{currentstroke}%
\pgfsetdash{}{0pt}%
\pgfpathmoveto{\pgfqpoint{3.588782in}{2.742575in}}%
\pgfpathlineto{\pgfqpoint{3.917165in}{2.742575in}}%
\pgfpathlineto{\pgfqpoint{3.917165in}{2.414192in}}%
\pgfpathlineto{\pgfqpoint{3.588782in}{2.414192in}}%
\pgfpathlineto{\pgfqpoint{3.588782in}{2.742575in}}%
\pgfusepath{fill}%
\end{pgfscope}%
\begin{pgfscope}%
\pgfpathrectangle{\pgfqpoint{3.588782in}{0.772276in}}{\pgfqpoint{2.955448in}{2.955448in}}%
\pgfusepath{clip}%
\pgfsetbuttcap%
\pgfsetroundjoin%
\definecolor{currentfill}{rgb}{0.866897,0.949573,0.952803}%
\pgfsetfillcolor{currentfill}%
\pgfsetlinewidth{0.000000pt}%
\definecolor{currentstroke}{rgb}{1.000000,1.000000,1.000000}%
\pgfsetstrokecolor{currentstroke}%
\pgfsetdash{}{0pt}%
\pgfpathmoveto{\pgfqpoint{3.917165in}{2.742575in}}%
\pgfpathlineto{\pgfqpoint{4.245548in}{2.742575in}}%
\pgfpathlineto{\pgfqpoint{4.245548in}{2.414192in}}%
\pgfpathlineto{\pgfqpoint{3.917165in}{2.414192in}}%
\pgfpathlineto{\pgfqpoint{3.917165in}{2.742575in}}%
\pgfusepath{fill}%
\end{pgfscope}%
\begin{pgfscope}%
\pgfpathrectangle{\pgfqpoint{3.588782in}{0.772276in}}{\pgfqpoint{2.955448in}{2.955448in}}%
\pgfusepath{clip}%
\pgfsetbuttcap%
\pgfsetroundjoin%
\definecolor{currentfill}{rgb}{0.397724,0.759554,0.640308}%
\pgfsetfillcolor{currentfill}%
\pgfsetlinewidth{0.000000pt}%
\definecolor{currentstroke}{rgb}{1.000000,1.000000,1.000000}%
\pgfsetstrokecolor{currentstroke}%
\pgfsetdash{}{0pt}%
\pgfpathmoveto{\pgfqpoint{4.245548in}{2.742575in}}%
\pgfpathlineto{\pgfqpoint{4.573931in}{2.742575in}}%
\pgfpathlineto{\pgfqpoint{4.573931in}{2.414192in}}%
\pgfpathlineto{\pgfqpoint{4.245548in}{2.414192in}}%
\pgfpathlineto{\pgfqpoint{4.245548in}{2.742575in}}%
\pgfusepath{fill}%
\end{pgfscope}%
\begin{pgfscope}%
\pgfpathrectangle{\pgfqpoint{3.588782in}{0.772276in}}{\pgfqpoint{2.955448in}{2.955448in}}%
\pgfusepath{clip}%
\pgfsetbuttcap%
\pgfsetroundjoin%
\pgfsetlinewidth{0.000000pt}%
\definecolor{currentstroke}{rgb}{1.000000,1.000000,1.000000}%
\pgfsetstrokecolor{currentstroke}%
\pgfsetdash{}{0pt}%
\pgfpathmoveto{\pgfqpoint{4.573931in}{2.742575in}}%
\pgfpathlineto{\pgfqpoint{4.902314in}{2.742575in}}%
\pgfpathlineto{\pgfqpoint{4.902314in}{2.414192in}}%
\pgfpathlineto{\pgfqpoint{4.573931in}{2.414192in}}%
\pgfpathlineto{\pgfqpoint{4.573931in}{2.742575in}}%
\pgfusepath{}%
\end{pgfscope}%
\begin{pgfscope}%
\pgfpathrectangle{\pgfqpoint{3.588782in}{0.772276in}}{\pgfqpoint{2.955448in}{2.955448in}}%
\pgfusepath{clip}%
\pgfsetbuttcap%
\pgfsetroundjoin%
\pgfsetlinewidth{0.000000pt}%
\definecolor{currentstroke}{rgb}{1.000000,1.000000,1.000000}%
\pgfsetstrokecolor{currentstroke}%
\pgfsetdash{}{0pt}%
\pgfpathmoveto{\pgfqpoint{4.902314in}{2.742575in}}%
\pgfpathlineto{\pgfqpoint{5.230698in}{2.742575in}}%
\pgfpathlineto{\pgfqpoint{5.230698in}{2.414192in}}%
\pgfpathlineto{\pgfqpoint{4.902314in}{2.414192in}}%
\pgfpathlineto{\pgfqpoint{4.902314in}{2.742575in}}%
\pgfusepath{}%
\end{pgfscope}%
\begin{pgfscope}%
\pgfpathrectangle{\pgfqpoint{3.588782in}{0.772276in}}{\pgfqpoint{2.955448in}{2.955448in}}%
\pgfusepath{clip}%
\pgfsetbuttcap%
\pgfsetroundjoin%
\pgfsetlinewidth{0.000000pt}%
\definecolor{currentstroke}{rgb}{1.000000,1.000000,1.000000}%
\pgfsetstrokecolor{currentstroke}%
\pgfsetdash{}{0pt}%
\pgfpathmoveto{\pgfqpoint{5.230698in}{2.742575in}}%
\pgfpathlineto{\pgfqpoint{5.559081in}{2.742575in}}%
\pgfpathlineto{\pgfqpoint{5.559081in}{2.414192in}}%
\pgfpathlineto{\pgfqpoint{5.230698in}{2.414192in}}%
\pgfpathlineto{\pgfqpoint{5.230698in}{2.742575in}}%
\pgfusepath{}%
\end{pgfscope}%
\begin{pgfscope}%
\pgfpathrectangle{\pgfqpoint{3.588782in}{0.772276in}}{\pgfqpoint{2.955448in}{2.955448in}}%
\pgfusepath{clip}%
\pgfsetbuttcap%
\pgfsetroundjoin%
\pgfsetlinewidth{0.000000pt}%
\definecolor{currentstroke}{rgb}{1.000000,1.000000,1.000000}%
\pgfsetstrokecolor{currentstroke}%
\pgfsetdash{}{0pt}%
\pgfpathmoveto{\pgfqpoint{5.559081in}{2.742575in}}%
\pgfpathlineto{\pgfqpoint{5.887464in}{2.742575in}}%
\pgfpathlineto{\pgfqpoint{5.887464in}{2.414192in}}%
\pgfpathlineto{\pgfqpoint{5.559081in}{2.414192in}}%
\pgfpathlineto{\pgfqpoint{5.559081in}{2.742575in}}%
\pgfusepath{}%
\end{pgfscope}%
\begin{pgfscope}%
\pgfpathrectangle{\pgfqpoint{3.588782in}{0.772276in}}{\pgfqpoint{2.955448in}{2.955448in}}%
\pgfusepath{clip}%
\pgfsetbuttcap%
\pgfsetroundjoin%
\pgfsetlinewidth{0.000000pt}%
\definecolor{currentstroke}{rgb}{1.000000,1.000000,1.000000}%
\pgfsetstrokecolor{currentstroke}%
\pgfsetdash{}{0pt}%
\pgfpathmoveto{\pgfqpoint{5.887464in}{2.742575in}}%
\pgfpathlineto{\pgfqpoint{6.215847in}{2.742575in}}%
\pgfpathlineto{\pgfqpoint{6.215847in}{2.414192in}}%
\pgfpathlineto{\pgfqpoint{5.887464in}{2.414192in}}%
\pgfpathlineto{\pgfqpoint{5.887464in}{2.742575in}}%
\pgfusepath{}%
\end{pgfscope}%
\begin{pgfscope}%
\pgfpathrectangle{\pgfqpoint{3.588782in}{0.772276in}}{\pgfqpoint{2.955448in}{2.955448in}}%
\pgfusepath{clip}%
\pgfsetbuttcap%
\pgfsetroundjoin%
\pgfsetlinewidth{0.000000pt}%
\definecolor{currentstroke}{rgb}{1.000000,1.000000,1.000000}%
\pgfsetstrokecolor{currentstroke}%
\pgfsetdash{}{0pt}%
\pgfpathmoveto{\pgfqpoint{6.215847in}{2.742575in}}%
\pgfpathlineto{\pgfqpoint{6.544230in}{2.742575in}}%
\pgfpathlineto{\pgfqpoint{6.544230in}{2.414192in}}%
\pgfpathlineto{\pgfqpoint{6.215847in}{2.414192in}}%
\pgfpathlineto{\pgfqpoint{6.215847in}{2.742575in}}%
\pgfusepath{}%
\end{pgfscope}%
\begin{pgfscope}%
\pgfpathrectangle{\pgfqpoint{3.588782in}{0.772276in}}{\pgfqpoint{2.955448in}{2.955448in}}%
\pgfusepath{clip}%
\pgfsetbuttcap%
\pgfsetroundjoin%
\definecolor{currentfill}{rgb}{0.922122,0.970150,0.981822}%
\pgfsetfillcolor{currentfill}%
\pgfsetlinewidth{0.000000pt}%
\definecolor{currentstroke}{rgb}{1.000000,1.000000,1.000000}%
\pgfsetstrokecolor{currentstroke}%
\pgfsetdash{}{0pt}%
\pgfpathmoveto{\pgfqpoint{3.588782in}{2.414192in}}%
\pgfpathlineto{\pgfqpoint{3.917165in}{2.414192in}}%
\pgfpathlineto{\pgfqpoint{3.917165in}{2.085808in}}%
\pgfpathlineto{\pgfqpoint{3.588782in}{2.085808in}}%
\pgfpathlineto{\pgfqpoint{3.588782in}{2.414192in}}%
\pgfusepath{fill}%
\end{pgfscope}%
\begin{pgfscope}%
\pgfpathrectangle{\pgfqpoint{3.588782in}{0.772276in}}{\pgfqpoint{2.955448in}{2.955448in}}%
\pgfusepath{clip}%
\pgfsetbuttcap%
\pgfsetroundjoin%
\definecolor{currentfill}{rgb}{0.922122,0.970150,0.981822}%
\pgfsetfillcolor{currentfill}%
\pgfsetlinewidth{0.000000pt}%
\definecolor{currentstroke}{rgb}{1.000000,1.000000,1.000000}%
\pgfsetstrokecolor{currentstroke}%
\pgfsetdash{}{0pt}%
\pgfpathmoveto{\pgfqpoint{3.917165in}{2.414192in}}%
\pgfpathlineto{\pgfqpoint{4.245548in}{2.414192in}}%
\pgfpathlineto{\pgfqpoint{4.245548in}{2.085808in}}%
\pgfpathlineto{\pgfqpoint{3.917165in}{2.085808in}}%
\pgfpathlineto{\pgfqpoint{3.917165in}{2.414192in}}%
\pgfusepath{fill}%
\end{pgfscope}%
\begin{pgfscope}%
\pgfpathrectangle{\pgfqpoint{3.588782in}{0.772276in}}{\pgfqpoint{2.955448in}{2.955448in}}%
\pgfusepath{clip}%
\pgfsetbuttcap%
\pgfsetroundjoin%
\definecolor{currentfill}{rgb}{0.922122,0.970150,0.981822}%
\pgfsetfillcolor{currentfill}%
\pgfsetlinewidth{0.000000pt}%
\definecolor{currentstroke}{rgb}{1.000000,1.000000,1.000000}%
\pgfsetstrokecolor{currentstroke}%
\pgfsetdash{}{0pt}%
\pgfpathmoveto{\pgfqpoint{4.245548in}{2.414192in}}%
\pgfpathlineto{\pgfqpoint{4.573931in}{2.414192in}}%
\pgfpathlineto{\pgfqpoint{4.573931in}{2.085808in}}%
\pgfpathlineto{\pgfqpoint{4.245548in}{2.085808in}}%
\pgfpathlineto{\pgfqpoint{4.245548in}{2.414192in}}%
\pgfusepath{fill}%
\end{pgfscope}%
\begin{pgfscope}%
\pgfpathrectangle{\pgfqpoint{3.588782in}{0.772276in}}{\pgfqpoint{2.955448in}{2.955448in}}%
\pgfusepath{clip}%
\pgfsetbuttcap%
\pgfsetroundjoin%
\definecolor{currentfill}{rgb}{0.968627,0.988235,0.992157}%
\pgfsetfillcolor{currentfill}%
\pgfsetlinewidth{0.000000pt}%
\definecolor{currentstroke}{rgb}{1.000000,1.000000,1.000000}%
\pgfsetstrokecolor{currentstroke}%
\pgfsetdash{}{0pt}%
\pgfpathmoveto{\pgfqpoint{4.573931in}{2.414192in}}%
\pgfpathlineto{\pgfqpoint{4.902314in}{2.414192in}}%
\pgfpathlineto{\pgfqpoint{4.902314in}{2.085808in}}%
\pgfpathlineto{\pgfqpoint{4.573931in}{2.085808in}}%
\pgfpathlineto{\pgfqpoint{4.573931in}{2.414192in}}%
\pgfusepath{fill}%
\end{pgfscope}%
\begin{pgfscope}%
\pgfpathrectangle{\pgfqpoint{3.588782in}{0.772276in}}{\pgfqpoint{2.955448in}{2.955448in}}%
\pgfusepath{clip}%
\pgfsetbuttcap%
\pgfsetroundjoin%
\pgfsetlinewidth{0.000000pt}%
\definecolor{currentstroke}{rgb}{1.000000,1.000000,1.000000}%
\pgfsetstrokecolor{currentstroke}%
\pgfsetdash{}{0pt}%
\pgfpathmoveto{\pgfqpoint{4.902314in}{2.414192in}}%
\pgfpathlineto{\pgfqpoint{5.230698in}{2.414192in}}%
\pgfpathlineto{\pgfqpoint{5.230698in}{2.085808in}}%
\pgfpathlineto{\pgfqpoint{4.902314in}{2.085808in}}%
\pgfpathlineto{\pgfqpoint{4.902314in}{2.414192in}}%
\pgfusepath{}%
\end{pgfscope}%
\begin{pgfscope}%
\pgfpathrectangle{\pgfqpoint{3.588782in}{0.772276in}}{\pgfqpoint{2.955448in}{2.955448in}}%
\pgfusepath{clip}%
\pgfsetbuttcap%
\pgfsetroundjoin%
\pgfsetlinewidth{0.000000pt}%
\definecolor{currentstroke}{rgb}{1.000000,1.000000,1.000000}%
\pgfsetstrokecolor{currentstroke}%
\pgfsetdash{}{0pt}%
\pgfpathmoveto{\pgfqpoint{5.230698in}{2.414192in}}%
\pgfpathlineto{\pgfqpoint{5.559081in}{2.414192in}}%
\pgfpathlineto{\pgfqpoint{5.559081in}{2.085808in}}%
\pgfpathlineto{\pgfqpoint{5.230698in}{2.085808in}}%
\pgfpathlineto{\pgfqpoint{5.230698in}{2.414192in}}%
\pgfusepath{}%
\end{pgfscope}%
\begin{pgfscope}%
\pgfpathrectangle{\pgfqpoint{3.588782in}{0.772276in}}{\pgfqpoint{2.955448in}{2.955448in}}%
\pgfusepath{clip}%
\pgfsetbuttcap%
\pgfsetroundjoin%
\pgfsetlinewidth{0.000000pt}%
\definecolor{currentstroke}{rgb}{1.000000,1.000000,1.000000}%
\pgfsetstrokecolor{currentstroke}%
\pgfsetdash{}{0pt}%
\pgfpathmoveto{\pgfqpoint{5.559081in}{2.414192in}}%
\pgfpathlineto{\pgfqpoint{5.887464in}{2.414192in}}%
\pgfpathlineto{\pgfqpoint{5.887464in}{2.085808in}}%
\pgfpathlineto{\pgfqpoint{5.559081in}{2.085808in}}%
\pgfpathlineto{\pgfqpoint{5.559081in}{2.414192in}}%
\pgfusepath{}%
\end{pgfscope}%
\begin{pgfscope}%
\pgfpathrectangle{\pgfqpoint{3.588782in}{0.772276in}}{\pgfqpoint{2.955448in}{2.955448in}}%
\pgfusepath{clip}%
\pgfsetbuttcap%
\pgfsetroundjoin%
\pgfsetlinewidth{0.000000pt}%
\definecolor{currentstroke}{rgb}{1.000000,1.000000,1.000000}%
\pgfsetstrokecolor{currentstroke}%
\pgfsetdash{}{0pt}%
\pgfpathmoveto{\pgfqpoint{5.887464in}{2.414192in}}%
\pgfpathlineto{\pgfqpoint{6.215847in}{2.414192in}}%
\pgfpathlineto{\pgfqpoint{6.215847in}{2.085808in}}%
\pgfpathlineto{\pgfqpoint{5.887464in}{2.085808in}}%
\pgfpathlineto{\pgfqpoint{5.887464in}{2.414192in}}%
\pgfusepath{}%
\end{pgfscope}%
\begin{pgfscope}%
\pgfpathrectangle{\pgfqpoint{3.588782in}{0.772276in}}{\pgfqpoint{2.955448in}{2.955448in}}%
\pgfusepath{clip}%
\pgfsetbuttcap%
\pgfsetroundjoin%
\pgfsetlinewidth{0.000000pt}%
\definecolor{currentstroke}{rgb}{1.000000,1.000000,1.000000}%
\pgfsetstrokecolor{currentstroke}%
\pgfsetdash{}{0pt}%
\pgfpathmoveto{\pgfqpoint{6.215847in}{2.414192in}}%
\pgfpathlineto{\pgfqpoint{6.544230in}{2.414192in}}%
\pgfpathlineto{\pgfqpoint{6.544230in}{2.085808in}}%
\pgfpathlineto{\pgfqpoint{6.215847in}{2.085808in}}%
\pgfpathlineto{\pgfqpoint{6.215847in}{2.414192in}}%
\pgfusepath{}%
\end{pgfscope}%
\begin{pgfscope}%
\pgfpathrectangle{\pgfqpoint{3.588782in}{0.772276in}}{\pgfqpoint{2.955448in}{2.955448in}}%
\pgfusepath{clip}%
\pgfsetbuttcap%
\pgfsetroundjoin%
\definecolor{currentfill}{rgb}{0.922122,0.970150,0.981822}%
\pgfsetfillcolor{currentfill}%
\pgfsetlinewidth{0.000000pt}%
\definecolor{currentstroke}{rgb}{1.000000,1.000000,1.000000}%
\pgfsetstrokecolor{currentstroke}%
\pgfsetdash{}{0pt}%
\pgfpathmoveto{\pgfqpoint{3.588782in}{2.085808in}}%
\pgfpathlineto{\pgfqpoint{3.917165in}{2.085808in}}%
\pgfpathlineto{\pgfqpoint{3.917165in}{1.757425in}}%
\pgfpathlineto{\pgfqpoint{3.588782in}{1.757425in}}%
\pgfpathlineto{\pgfqpoint{3.588782in}{2.085808in}}%
\pgfusepath{fill}%
\end{pgfscope}%
\begin{pgfscope}%
\pgfpathrectangle{\pgfqpoint{3.588782in}{0.772276in}}{\pgfqpoint{2.955448in}{2.955448in}}%
\pgfusepath{clip}%
\pgfsetbuttcap%
\pgfsetroundjoin%
\definecolor{currentfill}{rgb}{0.922122,0.970150,0.981822}%
\pgfsetfillcolor{currentfill}%
\pgfsetlinewidth{0.000000pt}%
\definecolor{currentstroke}{rgb}{1.000000,1.000000,1.000000}%
\pgfsetstrokecolor{currentstroke}%
\pgfsetdash{}{0pt}%
\pgfpathmoveto{\pgfqpoint{3.917165in}{2.085808in}}%
\pgfpathlineto{\pgfqpoint{4.245548in}{2.085808in}}%
\pgfpathlineto{\pgfqpoint{4.245548in}{1.757425in}}%
\pgfpathlineto{\pgfqpoint{3.917165in}{1.757425in}}%
\pgfpathlineto{\pgfqpoint{3.917165in}{2.085808in}}%
\pgfusepath{fill}%
\end{pgfscope}%
\begin{pgfscope}%
\pgfpathrectangle{\pgfqpoint{3.588782in}{0.772276in}}{\pgfqpoint{2.955448in}{2.955448in}}%
\pgfusepath{clip}%
\pgfsetbuttcap%
\pgfsetroundjoin%
\definecolor{currentfill}{rgb}{0.922122,0.970150,0.981822}%
\pgfsetfillcolor{currentfill}%
\pgfsetlinewidth{0.000000pt}%
\definecolor{currentstroke}{rgb}{1.000000,1.000000,1.000000}%
\pgfsetstrokecolor{currentstroke}%
\pgfsetdash{}{0pt}%
\pgfpathmoveto{\pgfqpoint{4.245548in}{2.085808in}}%
\pgfpathlineto{\pgfqpoint{4.573931in}{2.085808in}}%
\pgfpathlineto{\pgfqpoint{4.573931in}{1.757425in}}%
\pgfpathlineto{\pgfqpoint{4.245548in}{1.757425in}}%
\pgfpathlineto{\pgfqpoint{4.245548in}{2.085808in}}%
\pgfusepath{fill}%
\end{pgfscope}%
\begin{pgfscope}%
\pgfpathrectangle{\pgfqpoint{3.588782in}{0.772276in}}{\pgfqpoint{2.955448in}{2.955448in}}%
\pgfusepath{clip}%
\pgfsetbuttcap%
\pgfsetroundjoin%
\definecolor{currentfill}{rgb}{0.922122,0.970150,0.981822}%
\pgfsetfillcolor{currentfill}%
\pgfsetlinewidth{0.000000pt}%
\definecolor{currentstroke}{rgb}{1.000000,1.000000,1.000000}%
\pgfsetstrokecolor{currentstroke}%
\pgfsetdash{}{0pt}%
\pgfpathmoveto{\pgfqpoint{4.573931in}{2.085808in}}%
\pgfpathlineto{\pgfqpoint{4.902314in}{2.085808in}}%
\pgfpathlineto{\pgfqpoint{4.902314in}{1.757425in}}%
\pgfpathlineto{\pgfqpoint{4.573931in}{1.757425in}}%
\pgfpathlineto{\pgfqpoint{4.573931in}{2.085808in}}%
\pgfusepath{fill}%
\end{pgfscope}%
\begin{pgfscope}%
\pgfpathrectangle{\pgfqpoint{3.588782in}{0.772276in}}{\pgfqpoint{2.955448in}{2.955448in}}%
\pgfusepath{clip}%
\pgfsetbuttcap%
\pgfsetroundjoin%
\definecolor{currentfill}{rgb}{0.922122,0.970150,0.981822}%
\pgfsetfillcolor{currentfill}%
\pgfsetlinewidth{0.000000pt}%
\definecolor{currentstroke}{rgb}{1.000000,1.000000,1.000000}%
\pgfsetstrokecolor{currentstroke}%
\pgfsetdash{}{0pt}%
\pgfpathmoveto{\pgfqpoint{4.902314in}{2.085808in}}%
\pgfpathlineto{\pgfqpoint{5.230698in}{2.085808in}}%
\pgfpathlineto{\pgfqpoint{5.230698in}{1.757425in}}%
\pgfpathlineto{\pgfqpoint{4.902314in}{1.757425in}}%
\pgfpathlineto{\pgfqpoint{4.902314in}{2.085808in}}%
\pgfusepath{fill}%
\end{pgfscope}%
\begin{pgfscope}%
\pgfpathrectangle{\pgfqpoint{3.588782in}{0.772276in}}{\pgfqpoint{2.955448in}{2.955448in}}%
\pgfusepath{clip}%
\pgfsetbuttcap%
\pgfsetroundjoin%
\pgfsetlinewidth{0.000000pt}%
\definecolor{currentstroke}{rgb}{1.000000,1.000000,1.000000}%
\pgfsetstrokecolor{currentstroke}%
\pgfsetdash{}{0pt}%
\pgfpathmoveto{\pgfqpoint{5.230698in}{2.085808in}}%
\pgfpathlineto{\pgfqpoint{5.559081in}{2.085808in}}%
\pgfpathlineto{\pgfqpoint{5.559081in}{1.757425in}}%
\pgfpathlineto{\pgfqpoint{5.230698in}{1.757425in}}%
\pgfpathlineto{\pgfqpoint{5.230698in}{2.085808in}}%
\pgfusepath{}%
\end{pgfscope}%
\begin{pgfscope}%
\pgfpathrectangle{\pgfqpoint{3.588782in}{0.772276in}}{\pgfqpoint{2.955448in}{2.955448in}}%
\pgfusepath{clip}%
\pgfsetbuttcap%
\pgfsetroundjoin%
\pgfsetlinewidth{0.000000pt}%
\definecolor{currentstroke}{rgb}{1.000000,1.000000,1.000000}%
\pgfsetstrokecolor{currentstroke}%
\pgfsetdash{}{0pt}%
\pgfpathmoveto{\pgfqpoint{5.559081in}{2.085808in}}%
\pgfpathlineto{\pgfqpoint{5.887464in}{2.085808in}}%
\pgfpathlineto{\pgfqpoint{5.887464in}{1.757425in}}%
\pgfpathlineto{\pgfqpoint{5.559081in}{1.757425in}}%
\pgfpathlineto{\pgfqpoint{5.559081in}{2.085808in}}%
\pgfusepath{}%
\end{pgfscope}%
\begin{pgfscope}%
\pgfpathrectangle{\pgfqpoint{3.588782in}{0.772276in}}{\pgfqpoint{2.955448in}{2.955448in}}%
\pgfusepath{clip}%
\pgfsetbuttcap%
\pgfsetroundjoin%
\pgfsetlinewidth{0.000000pt}%
\definecolor{currentstroke}{rgb}{1.000000,1.000000,1.000000}%
\pgfsetstrokecolor{currentstroke}%
\pgfsetdash{}{0pt}%
\pgfpathmoveto{\pgfqpoint{5.887464in}{2.085808in}}%
\pgfpathlineto{\pgfqpoint{6.215847in}{2.085808in}}%
\pgfpathlineto{\pgfqpoint{6.215847in}{1.757425in}}%
\pgfpathlineto{\pgfqpoint{5.887464in}{1.757425in}}%
\pgfpathlineto{\pgfqpoint{5.887464in}{2.085808in}}%
\pgfusepath{}%
\end{pgfscope}%
\begin{pgfscope}%
\pgfpathrectangle{\pgfqpoint{3.588782in}{0.772276in}}{\pgfqpoint{2.955448in}{2.955448in}}%
\pgfusepath{clip}%
\pgfsetbuttcap%
\pgfsetroundjoin%
\pgfsetlinewidth{0.000000pt}%
\definecolor{currentstroke}{rgb}{1.000000,1.000000,1.000000}%
\pgfsetstrokecolor{currentstroke}%
\pgfsetdash{}{0pt}%
\pgfpathmoveto{\pgfqpoint{6.215847in}{2.085808in}}%
\pgfpathlineto{\pgfqpoint{6.544230in}{2.085808in}}%
\pgfpathlineto{\pgfqpoint{6.544230in}{1.757425in}}%
\pgfpathlineto{\pgfqpoint{6.215847in}{1.757425in}}%
\pgfpathlineto{\pgfqpoint{6.215847in}{2.085808in}}%
\pgfusepath{}%
\end{pgfscope}%
\begin{pgfscope}%
\pgfpathrectangle{\pgfqpoint{3.588782in}{0.772276in}}{\pgfqpoint{2.955448in}{2.955448in}}%
\pgfusepath{clip}%
\pgfsetbuttcap%
\pgfsetroundjoin%
\definecolor{currentfill}{rgb}{0.922122,0.970150,0.981822}%
\pgfsetfillcolor{currentfill}%
\pgfsetlinewidth{0.000000pt}%
\definecolor{currentstroke}{rgb}{1.000000,1.000000,1.000000}%
\pgfsetstrokecolor{currentstroke}%
\pgfsetdash{}{0pt}%
\pgfpathmoveto{\pgfqpoint{3.588782in}{1.757425in}}%
\pgfpathlineto{\pgfqpoint{3.917165in}{1.757425in}}%
\pgfpathlineto{\pgfqpoint{3.917165in}{1.429042in}}%
\pgfpathlineto{\pgfqpoint{3.588782in}{1.429042in}}%
\pgfpathlineto{\pgfqpoint{3.588782in}{1.757425in}}%
\pgfusepath{fill}%
\end{pgfscope}%
\begin{pgfscope}%
\pgfpathrectangle{\pgfqpoint{3.588782in}{0.772276in}}{\pgfqpoint{2.955448in}{2.955448in}}%
\pgfusepath{clip}%
\pgfsetbuttcap%
\pgfsetroundjoin%
\definecolor{currentfill}{rgb}{0.922122,0.970150,0.981822}%
\pgfsetfillcolor{currentfill}%
\pgfsetlinewidth{0.000000pt}%
\definecolor{currentstroke}{rgb}{1.000000,1.000000,1.000000}%
\pgfsetstrokecolor{currentstroke}%
\pgfsetdash{}{0pt}%
\pgfpathmoveto{\pgfqpoint{3.917165in}{1.757425in}}%
\pgfpathlineto{\pgfqpoint{4.245548in}{1.757425in}}%
\pgfpathlineto{\pgfqpoint{4.245548in}{1.429042in}}%
\pgfpathlineto{\pgfqpoint{3.917165in}{1.429042in}}%
\pgfpathlineto{\pgfqpoint{3.917165in}{1.757425in}}%
\pgfusepath{fill}%
\end{pgfscope}%
\begin{pgfscope}%
\pgfpathrectangle{\pgfqpoint{3.588782in}{0.772276in}}{\pgfqpoint{2.955448in}{2.955448in}}%
\pgfusepath{clip}%
\pgfsetbuttcap%
\pgfsetroundjoin%
\definecolor{currentfill}{rgb}{0.798431,0.924875,0.901069}%
\pgfsetfillcolor{currentfill}%
\pgfsetlinewidth{0.000000pt}%
\definecolor{currentstroke}{rgb}{1.000000,1.000000,1.000000}%
\pgfsetstrokecolor{currentstroke}%
\pgfsetdash{}{0pt}%
\pgfpathmoveto{\pgfqpoint{4.245548in}{1.757425in}}%
\pgfpathlineto{\pgfqpoint{4.573931in}{1.757425in}}%
\pgfpathlineto{\pgfqpoint{4.573931in}{1.429042in}}%
\pgfpathlineto{\pgfqpoint{4.245548in}{1.429042in}}%
\pgfpathlineto{\pgfqpoint{4.245548in}{1.757425in}}%
\pgfusepath{fill}%
\end{pgfscope}%
\begin{pgfscope}%
\pgfpathrectangle{\pgfqpoint{3.588782in}{0.772276in}}{\pgfqpoint{2.955448in}{2.955448in}}%
\pgfusepath{clip}%
\pgfsetbuttcap%
\pgfsetroundjoin%
\definecolor{currentfill}{rgb}{0.866897,0.949573,0.952803}%
\pgfsetfillcolor{currentfill}%
\pgfsetlinewidth{0.000000pt}%
\definecolor{currentstroke}{rgb}{1.000000,1.000000,1.000000}%
\pgfsetstrokecolor{currentstroke}%
\pgfsetdash{}{0pt}%
\pgfpathmoveto{\pgfqpoint{4.573931in}{1.757425in}}%
\pgfpathlineto{\pgfqpoint{4.902314in}{1.757425in}}%
\pgfpathlineto{\pgfqpoint{4.902314in}{1.429042in}}%
\pgfpathlineto{\pgfqpoint{4.573931in}{1.429042in}}%
\pgfpathlineto{\pgfqpoint{4.573931in}{1.757425in}}%
\pgfusepath{fill}%
\end{pgfscope}%
\begin{pgfscope}%
\pgfpathrectangle{\pgfqpoint{3.588782in}{0.772276in}}{\pgfqpoint{2.955448in}{2.955448in}}%
\pgfusepath{clip}%
\pgfsetbuttcap%
\pgfsetroundjoin%
\definecolor{currentfill}{rgb}{0.866897,0.949573,0.952803}%
\pgfsetfillcolor{currentfill}%
\pgfsetlinewidth{0.000000pt}%
\definecolor{currentstroke}{rgb}{1.000000,1.000000,1.000000}%
\pgfsetstrokecolor{currentstroke}%
\pgfsetdash{}{0pt}%
\pgfpathmoveto{\pgfqpoint{4.902314in}{1.757425in}}%
\pgfpathlineto{\pgfqpoint{5.230698in}{1.757425in}}%
\pgfpathlineto{\pgfqpoint{5.230698in}{1.429042in}}%
\pgfpathlineto{\pgfqpoint{4.902314in}{1.429042in}}%
\pgfpathlineto{\pgfqpoint{4.902314in}{1.757425in}}%
\pgfusepath{fill}%
\end{pgfscope}%
\begin{pgfscope}%
\pgfpathrectangle{\pgfqpoint{3.588782in}{0.772276in}}{\pgfqpoint{2.955448in}{2.955448in}}%
\pgfusepath{clip}%
\pgfsetbuttcap%
\pgfsetroundjoin%
\definecolor{currentfill}{rgb}{0.922122,0.970150,0.981822}%
\pgfsetfillcolor{currentfill}%
\pgfsetlinewidth{0.000000pt}%
\definecolor{currentstroke}{rgb}{1.000000,1.000000,1.000000}%
\pgfsetstrokecolor{currentstroke}%
\pgfsetdash{}{0pt}%
\pgfpathmoveto{\pgfqpoint{5.230698in}{1.757425in}}%
\pgfpathlineto{\pgfqpoint{5.559081in}{1.757425in}}%
\pgfpathlineto{\pgfqpoint{5.559081in}{1.429042in}}%
\pgfpathlineto{\pgfqpoint{5.230698in}{1.429042in}}%
\pgfpathlineto{\pgfqpoint{5.230698in}{1.757425in}}%
\pgfusepath{fill}%
\end{pgfscope}%
\begin{pgfscope}%
\pgfpathrectangle{\pgfqpoint{3.588782in}{0.772276in}}{\pgfqpoint{2.955448in}{2.955448in}}%
\pgfusepath{clip}%
\pgfsetbuttcap%
\pgfsetroundjoin%
\pgfsetlinewidth{0.000000pt}%
\definecolor{currentstroke}{rgb}{1.000000,1.000000,1.000000}%
\pgfsetstrokecolor{currentstroke}%
\pgfsetdash{}{0pt}%
\pgfpathmoveto{\pgfqpoint{5.559081in}{1.757425in}}%
\pgfpathlineto{\pgfqpoint{5.887464in}{1.757425in}}%
\pgfpathlineto{\pgfqpoint{5.887464in}{1.429042in}}%
\pgfpathlineto{\pgfqpoint{5.559081in}{1.429042in}}%
\pgfpathlineto{\pgfqpoint{5.559081in}{1.757425in}}%
\pgfusepath{}%
\end{pgfscope}%
\begin{pgfscope}%
\pgfpathrectangle{\pgfqpoint{3.588782in}{0.772276in}}{\pgfqpoint{2.955448in}{2.955448in}}%
\pgfusepath{clip}%
\pgfsetbuttcap%
\pgfsetroundjoin%
\pgfsetlinewidth{0.000000pt}%
\definecolor{currentstroke}{rgb}{1.000000,1.000000,1.000000}%
\pgfsetstrokecolor{currentstroke}%
\pgfsetdash{}{0pt}%
\pgfpathmoveto{\pgfqpoint{5.887464in}{1.757425in}}%
\pgfpathlineto{\pgfqpoint{6.215847in}{1.757425in}}%
\pgfpathlineto{\pgfqpoint{6.215847in}{1.429042in}}%
\pgfpathlineto{\pgfqpoint{5.887464in}{1.429042in}}%
\pgfpathlineto{\pgfqpoint{5.887464in}{1.757425in}}%
\pgfusepath{}%
\end{pgfscope}%
\begin{pgfscope}%
\pgfpathrectangle{\pgfqpoint{3.588782in}{0.772276in}}{\pgfqpoint{2.955448in}{2.955448in}}%
\pgfusepath{clip}%
\pgfsetbuttcap%
\pgfsetroundjoin%
\pgfsetlinewidth{0.000000pt}%
\definecolor{currentstroke}{rgb}{1.000000,1.000000,1.000000}%
\pgfsetstrokecolor{currentstroke}%
\pgfsetdash{}{0pt}%
\pgfpathmoveto{\pgfqpoint{6.215847in}{1.757425in}}%
\pgfpathlineto{\pgfqpoint{6.544230in}{1.757425in}}%
\pgfpathlineto{\pgfqpoint{6.544230in}{1.429042in}}%
\pgfpathlineto{\pgfqpoint{6.215847in}{1.429042in}}%
\pgfpathlineto{\pgfqpoint{6.215847in}{1.757425in}}%
\pgfusepath{}%
\end{pgfscope}%
\begin{pgfscope}%
\pgfpathrectangle{\pgfqpoint{3.588782in}{0.772276in}}{\pgfqpoint{2.955448in}{2.955448in}}%
\pgfusepath{clip}%
\pgfsetbuttcap%
\pgfsetroundjoin%
\definecolor{currentfill}{rgb}{0.866897,0.949573,0.952803}%
\pgfsetfillcolor{currentfill}%
\pgfsetlinewidth{0.000000pt}%
\definecolor{currentstroke}{rgb}{1.000000,1.000000,1.000000}%
\pgfsetstrokecolor{currentstroke}%
\pgfsetdash{}{0pt}%
\pgfpathmoveto{\pgfqpoint{3.588782in}{1.429042in}}%
\pgfpathlineto{\pgfqpoint{3.917165in}{1.429042in}}%
\pgfpathlineto{\pgfqpoint{3.917165in}{1.100659in}}%
\pgfpathlineto{\pgfqpoint{3.588782in}{1.100659in}}%
\pgfpathlineto{\pgfqpoint{3.588782in}{1.429042in}}%
\pgfusepath{fill}%
\end{pgfscope}%
\begin{pgfscope}%
\pgfpathrectangle{\pgfqpoint{3.588782in}{0.772276in}}{\pgfqpoint{2.955448in}{2.955448in}}%
\pgfusepath{clip}%
\pgfsetbuttcap%
\pgfsetroundjoin%
\definecolor{currentfill}{rgb}{0.866897,0.949573,0.952803}%
\pgfsetfillcolor{currentfill}%
\pgfsetlinewidth{0.000000pt}%
\definecolor{currentstroke}{rgb}{1.000000,1.000000,1.000000}%
\pgfsetstrokecolor{currentstroke}%
\pgfsetdash{}{0pt}%
\pgfpathmoveto{\pgfqpoint{3.917165in}{1.429042in}}%
\pgfpathlineto{\pgfqpoint{4.245548in}{1.429042in}}%
\pgfpathlineto{\pgfqpoint{4.245548in}{1.100659in}}%
\pgfpathlineto{\pgfqpoint{3.917165in}{1.100659in}}%
\pgfpathlineto{\pgfqpoint{3.917165in}{1.429042in}}%
\pgfusepath{fill}%
\end{pgfscope}%
\begin{pgfscope}%
\pgfpathrectangle{\pgfqpoint{3.588782in}{0.772276in}}{\pgfqpoint{2.955448in}{2.955448in}}%
\pgfusepath{clip}%
\pgfsetbuttcap%
\pgfsetroundjoin%
\definecolor{currentfill}{rgb}{0.000000,0.372595,0.149804}%
\pgfsetfillcolor{currentfill}%
\pgfsetlinewidth{0.000000pt}%
\definecolor{currentstroke}{rgb}{1.000000,1.000000,1.000000}%
\pgfsetstrokecolor{currentstroke}%
\pgfsetdash{}{0pt}%
\pgfpathmoveto{\pgfqpoint{4.245548in}{1.429042in}}%
\pgfpathlineto{\pgfqpoint{4.573931in}{1.429042in}}%
\pgfpathlineto{\pgfqpoint{4.573931in}{1.100659in}}%
\pgfpathlineto{\pgfqpoint{4.245548in}{1.100659in}}%
\pgfpathlineto{\pgfqpoint{4.245548in}{1.429042in}}%
\pgfusepath{fill}%
\end{pgfscope}%
\begin{pgfscope}%
\pgfpathrectangle{\pgfqpoint{3.588782in}{0.772276in}}{\pgfqpoint{2.955448in}{2.955448in}}%
\pgfusepath{clip}%
\pgfsetbuttcap%
\pgfsetroundjoin%
\definecolor{currentfill}{rgb}{0.798431,0.924875,0.901069}%
\pgfsetfillcolor{currentfill}%
\pgfsetlinewidth{0.000000pt}%
\definecolor{currentstroke}{rgb}{1.000000,1.000000,1.000000}%
\pgfsetstrokecolor{currentstroke}%
\pgfsetdash{}{0pt}%
\pgfpathmoveto{\pgfqpoint{4.573931in}{1.429042in}}%
\pgfpathlineto{\pgfqpoint{4.902314in}{1.429042in}}%
\pgfpathlineto{\pgfqpoint{4.902314in}{1.100659in}}%
\pgfpathlineto{\pgfqpoint{4.573931in}{1.100659in}}%
\pgfpathlineto{\pgfqpoint{4.573931in}{1.429042in}}%
\pgfusepath{fill}%
\end{pgfscope}%
\begin{pgfscope}%
\pgfpathrectangle{\pgfqpoint{3.588782in}{0.772276in}}{\pgfqpoint{2.955448in}{2.955448in}}%
\pgfusepath{clip}%
\pgfsetbuttcap%
\pgfsetroundjoin%
\definecolor{currentfill}{rgb}{0.666667,0.873203,0.826144}%
\pgfsetfillcolor{currentfill}%
\pgfsetlinewidth{0.000000pt}%
\definecolor{currentstroke}{rgb}{1.000000,1.000000,1.000000}%
\pgfsetstrokecolor{currentstroke}%
\pgfsetdash{}{0pt}%
\pgfpathmoveto{\pgfqpoint{4.902314in}{1.429042in}}%
\pgfpathlineto{\pgfqpoint{5.230698in}{1.429042in}}%
\pgfpathlineto{\pgfqpoint{5.230698in}{1.100659in}}%
\pgfpathlineto{\pgfqpoint{4.902314in}{1.100659in}}%
\pgfpathlineto{\pgfqpoint{4.902314in}{1.429042in}}%
\pgfusepath{fill}%
\end{pgfscope}%
\begin{pgfscope}%
\pgfpathrectangle{\pgfqpoint{3.588782in}{0.772276in}}{\pgfqpoint{2.955448in}{2.955448in}}%
\pgfusepath{clip}%
\pgfsetbuttcap%
\pgfsetroundjoin%
\definecolor{currentfill}{rgb}{0.922122,0.970150,0.981822}%
\pgfsetfillcolor{currentfill}%
\pgfsetlinewidth{0.000000pt}%
\definecolor{currentstroke}{rgb}{1.000000,1.000000,1.000000}%
\pgfsetstrokecolor{currentstroke}%
\pgfsetdash{}{0pt}%
\pgfpathmoveto{\pgfqpoint{5.230698in}{1.429042in}}%
\pgfpathlineto{\pgfqpoint{5.559081in}{1.429042in}}%
\pgfpathlineto{\pgfqpoint{5.559081in}{1.100659in}}%
\pgfpathlineto{\pgfqpoint{5.230698in}{1.100659in}}%
\pgfpathlineto{\pgfqpoint{5.230698in}{1.429042in}}%
\pgfusepath{fill}%
\end{pgfscope}%
\begin{pgfscope}%
\pgfpathrectangle{\pgfqpoint{3.588782in}{0.772276in}}{\pgfqpoint{2.955448in}{2.955448in}}%
\pgfusepath{clip}%
\pgfsetbuttcap%
\pgfsetroundjoin%
\definecolor{currentfill}{rgb}{0.798431,0.924875,0.901069}%
\pgfsetfillcolor{currentfill}%
\pgfsetlinewidth{0.000000pt}%
\definecolor{currentstroke}{rgb}{1.000000,1.000000,1.000000}%
\pgfsetstrokecolor{currentstroke}%
\pgfsetdash{}{0pt}%
\pgfpathmoveto{\pgfqpoint{5.559081in}{1.429042in}}%
\pgfpathlineto{\pgfqpoint{5.887464in}{1.429042in}}%
\pgfpathlineto{\pgfqpoint{5.887464in}{1.100659in}}%
\pgfpathlineto{\pgfqpoint{5.559081in}{1.100659in}}%
\pgfpathlineto{\pgfqpoint{5.559081in}{1.429042in}}%
\pgfusepath{fill}%
\end{pgfscope}%
\begin{pgfscope}%
\pgfpathrectangle{\pgfqpoint{3.588782in}{0.772276in}}{\pgfqpoint{2.955448in}{2.955448in}}%
\pgfusepath{clip}%
\pgfsetbuttcap%
\pgfsetroundjoin%
\pgfsetlinewidth{0.000000pt}%
\definecolor{currentstroke}{rgb}{1.000000,1.000000,1.000000}%
\pgfsetstrokecolor{currentstroke}%
\pgfsetdash{}{0pt}%
\pgfpathmoveto{\pgfqpoint{5.887464in}{1.429042in}}%
\pgfpathlineto{\pgfqpoint{6.215847in}{1.429042in}}%
\pgfpathlineto{\pgfqpoint{6.215847in}{1.100659in}}%
\pgfpathlineto{\pgfqpoint{5.887464in}{1.100659in}}%
\pgfpathlineto{\pgfqpoint{5.887464in}{1.429042in}}%
\pgfusepath{}%
\end{pgfscope}%
\begin{pgfscope}%
\pgfpathrectangle{\pgfqpoint{3.588782in}{0.772276in}}{\pgfqpoint{2.955448in}{2.955448in}}%
\pgfusepath{clip}%
\pgfsetbuttcap%
\pgfsetroundjoin%
\pgfsetlinewidth{0.000000pt}%
\definecolor{currentstroke}{rgb}{1.000000,1.000000,1.000000}%
\pgfsetstrokecolor{currentstroke}%
\pgfsetdash{}{0pt}%
\pgfpathmoveto{\pgfqpoint{6.215847in}{1.429042in}}%
\pgfpathlineto{\pgfqpoint{6.544230in}{1.429042in}}%
\pgfpathlineto{\pgfqpoint{6.544230in}{1.100659in}}%
\pgfpathlineto{\pgfqpoint{6.215847in}{1.100659in}}%
\pgfpathlineto{\pgfqpoint{6.215847in}{1.429042in}}%
\pgfusepath{}%
\end{pgfscope}%
\begin{pgfscope}%
\pgfpathrectangle{\pgfqpoint{3.588782in}{0.772276in}}{\pgfqpoint{2.955448in}{2.955448in}}%
\pgfusepath{clip}%
\pgfsetbuttcap%
\pgfsetroundjoin%
\definecolor{currentfill}{rgb}{0.866897,0.949573,0.952803}%
\pgfsetfillcolor{currentfill}%
\pgfsetlinewidth{0.000000pt}%
\definecolor{currentstroke}{rgb}{1.000000,1.000000,1.000000}%
\pgfsetstrokecolor{currentstroke}%
\pgfsetdash{}{0pt}%
\pgfpathmoveto{\pgfqpoint{3.588782in}{1.100659in}}%
\pgfpathlineto{\pgfqpoint{3.917165in}{1.100659in}}%
\pgfpathlineto{\pgfqpoint{3.917165in}{0.772276in}}%
\pgfpathlineto{\pgfqpoint{3.588782in}{0.772276in}}%
\pgfpathlineto{\pgfqpoint{3.588782in}{1.100659in}}%
\pgfusepath{fill}%
\end{pgfscope}%
\begin{pgfscope}%
\pgfpathrectangle{\pgfqpoint{3.588782in}{0.772276in}}{\pgfqpoint{2.955448in}{2.955448in}}%
\pgfusepath{clip}%
\pgfsetbuttcap%
\pgfsetroundjoin%
\definecolor{currentfill}{rgb}{0.798431,0.924875,0.901069}%
\pgfsetfillcolor{currentfill}%
\pgfsetlinewidth{0.000000pt}%
\definecolor{currentstroke}{rgb}{1.000000,1.000000,1.000000}%
\pgfsetstrokecolor{currentstroke}%
\pgfsetdash{}{0pt}%
\pgfpathmoveto{\pgfqpoint{3.917165in}{1.100659in}}%
\pgfpathlineto{\pgfqpoint{4.245548in}{1.100659in}}%
\pgfpathlineto{\pgfqpoint{4.245548in}{0.772276in}}%
\pgfpathlineto{\pgfqpoint{3.917165in}{0.772276in}}%
\pgfpathlineto{\pgfqpoint{3.917165in}{1.100659in}}%
\pgfusepath{fill}%
\end{pgfscope}%
\begin{pgfscope}%
\pgfpathrectangle{\pgfqpoint{3.588782in}{0.772276in}}{\pgfqpoint{2.955448in}{2.955448in}}%
\pgfusepath{clip}%
\pgfsetbuttcap%
\pgfsetroundjoin%
\definecolor{currentfill}{rgb}{0.000000,0.266667,0.105882}%
\pgfsetfillcolor{currentfill}%
\pgfsetlinewidth{0.000000pt}%
\definecolor{currentstroke}{rgb}{1.000000,1.000000,1.000000}%
\pgfsetstrokecolor{currentstroke}%
\pgfsetdash{}{0pt}%
\pgfpathmoveto{\pgfqpoint{4.245548in}{1.100659in}}%
\pgfpathlineto{\pgfqpoint{4.573931in}{1.100659in}}%
\pgfpathlineto{\pgfqpoint{4.573931in}{0.772276in}}%
\pgfpathlineto{\pgfqpoint{4.245548in}{0.772276in}}%
\pgfpathlineto{\pgfqpoint{4.245548in}{1.100659in}}%
\pgfusepath{fill}%
\end{pgfscope}%
\begin{pgfscope}%
\pgfpathrectangle{\pgfqpoint{3.588782in}{0.772276in}}{\pgfqpoint{2.955448in}{2.955448in}}%
\pgfusepath{clip}%
\pgfsetbuttcap%
\pgfsetroundjoin%
\definecolor{currentfill}{rgb}{0.798431,0.924875,0.901069}%
\pgfsetfillcolor{currentfill}%
\pgfsetlinewidth{0.000000pt}%
\definecolor{currentstroke}{rgb}{1.000000,1.000000,1.000000}%
\pgfsetstrokecolor{currentstroke}%
\pgfsetdash{}{0pt}%
\pgfpathmoveto{\pgfqpoint{4.573931in}{1.100659in}}%
\pgfpathlineto{\pgfqpoint{4.902314in}{1.100659in}}%
\pgfpathlineto{\pgfqpoint{4.902314in}{0.772276in}}%
\pgfpathlineto{\pgfqpoint{4.573931in}{0.772276in}}%
\pgfpathlineto{\pgfqpoint{4.573931in}{1.100659in}}%
\pgfusepath{fill}%
\end{pgfscope}%
\begin{pgfscope}%
\pgfpathrectangle{\pgfqpoint{3.588782in}{0.772276in}}{\pgfqpoint{2.955448in}{2.955448in}}%
\pgfusepath{clip}%
\pgfsetbuttcap%
\pgfsetroundjoin%
\definecolor{currentfill}{rgb}{0.666667,0.873203,0.826144}%
\pgfsetfillcolor{currentfill}%
\pgfsetlinewidth{0.000000pt}%
\definecolor{currentstroke}{rgb}{1.000000,1.000000,1.000000}%
\pgfsetstrokecolor{currentstroke}%
\pgfsetdash{}{0pt}%
\pgfpathmoveto{\pgfqpoint{4.902314in}{1.100659in}}%
\pgfpathlineto{\pgfqpoint{5.230698in}{1.100659in}}%
\pgfpathlineto{\pgfqpoint{5.230698in}{0.772276in}}%
\pgfpathlineto{\pgfqpoint{4.902314in}{0.772276in}}%
\pgfpathlineto{\pgfqpoint{4.902314in}{1.100659in}}%
\pgfusepath{fill}%
\end{pgfscope}%
\begin{pgfscope}%
\pgfpathrectangle{\pgfqpoint{3.588782in}{0.772276in}}{\pgfqpoint{2.955448in}{2.955448in}}%
\pgfusepath{clip}%
\pgfsetbuttcap%
\pgfsetroundjoin%
\definecolor{currentfill}{rgb}{0.866897,0.949573,0.952803}%
\pgfsetfillcolor{currentfill}%
\pgfsetlinewidth{0.000000pt}%
\definecolor{currentstroke}{rgb}{1.000000,1.000000,1.000000}%
\pgfsetstrokecolor{currentstroke}%
\pgfsetdash{}{0pt}%
\pgfpathmoveto{\pgfqpoint{5.230698in}{1.100659in}}%
\pgfpathlineto{\pgfqpoint{5.559081in}{1.100659in}}%
\pgfpathlineto{\pgfqpoint{5.559081in}{0.772276in}}%
\pgfpathlineto{\pgfqpoint{5.230698in}{0.772276in}}%
\pgfpathlineto{\pgfqpoint{5.230698in}{1.100659in}}%
\pgfusepath{fill}%
\end{pgfscope}%
\begin{pgfscope}%
\pgfpathrectangle{\pgfqpoint{3.588782in}{0.772276in}}{\pgfqpoint{2.955448in}{2.955448in}}%
\pgfusepath{clip}%
\pgfsetbuttcap%
\pgfsetroundjoin%
\definecolor{currentfill}{rgb}{0.666667,0.873203,0.826144}%
\pgfsetfillcolor{currentfill}%
\pgfsetlinewidth{0.000000pt}%
\definecolor{currentstroke}{rgb}{1.000000,1.000000,1.000000}%
\pgfsetstrokecolor{currentstroke}%
\pgfsetdash{}{0pt}%
\pgfpathmoveto{\pgfqpoint{5.559081in}{1.100659in}}%
\pgfpathlineto{\pgfqpoint{5.887464in}{1.100659in}}%
\pgfpathlineto{\pgfqpoint{5.887464in}{0.772276in}}%
\pgfpathlineto{\pgfqpoint{5.559081in}{0.772276in}}%
\pgfpathlineto{\pgfqpoint{5.559081in}{1.100659in}}%
\pgfusepath{fill}%
\end{pgfscope}%
\begin{pgfscope}%
\pgfpathrectangle{\pgfqpoint{3.588782in}{0.772276in}}{\pgfqpoint{2.955448in}{2.955448in}}%
\pgfusepath{clip}%
\pgfsetbuttcap%
\pgfsetroundjoin%
\definecolor{currentfill}{rgb}{0.968627,0.988235,0.992157}%
\pgfsetfillcolor{currentfill}%
\pgfsetlinewidth{0.000000pt}%
\definecolor{currentstroke}{rgb}{1.000000,1.000000,1.000000}%
\pgfsetstrokecolor{currentstroke}%
\pgfsetdash{}{0pt}%
\pgfpathmoveto{\pgfqpoint{5.887464in}{1.100659in}}%
\pgfpathlineto{\pgfqpoint{6.215847in}{1.100659in}}%
\pgfpathlineto{\pgfqpoint{6.215847in}{0.772276in}}%
\pgfpathlineto{\pgfqpoint{5.887464in}{0.772276in}}%
\pgfpathlineto{\pgfqpoint{5.887464in}{1.100659in}}%
\pgfusepath{fill}%
\end{pgfscope}%
\begin{pgfscope}%
\pgfpathrectangle{\pgfqpoint{3.588782in}{0.772276in}}{\pgfqpoint{2.955448in}{2.955448in}}%
\pgfusepath{clip}%
\pgfsetbuttcap%
\pgfsetroundjoin%
\pgfsetlinewidth{0.000000pt}%
\definecolor{currentstroke}{rgb}{1.000000,1.000000,1.000000}%
\pgfsetstrokecolor{currentstroke}%
\pgfsetdash{}{0pt}%
\pgfpathmoveto{\pgfqpoint{6.215847in}{1.100659in}}%
\pgfpathlineto{\pgfqpoint{6.544230in}{1.100659in}}%
\pgfpathlineto{\pgfqpoint{6.544230in}{0.772276in}}%
\pgfpathlineto{\pgfqpoint{6.215847in}{0.772276in}}%
\pgfpathlineto{\pgfqpoint{6.215847in}{1.100659in}}%
\pgfusepath{}%
\end{pgfscope}%
\begin{pgfscope}%
\pgfsetbuttcap%
\pgfsetroundjoin%
\definecolor{currentfill}{rgb}{0.000000,0.000000,0.000000}%
\pgfsetfillcolor{currentfill}%
\pgfsetlinewidth{0.803000pt}%
\definecolor{currentstroke}{rgb}{0.000000,0.000000,0.000000}%
\pgfsetstrokecolor{currentstroke}%
\pgfsetdash{}{0pt}%
\pgfsys@defobject{currentmarker}{\pgfqpoint{0.000000in}{-0.048611in}}{\pgfqpoint{0.000000in}{0.000000in}}{%
\pgfpathmoveto{\pgfqpoint{0.000000in}{0.000000in}}%
\pgfpathlineto{\pgfqpoint{0.000000in}{-0.048611in}}%
\pgfusepath{stroke,fill}%
}%
\begin{pgfscope}%
\pgfsys@transformshift{3.752974in}{0.772276in}%
\pgfsys@useobject{currentmarker}{}%
\end{pgfscope}%
\end{pgfscope}%
\begin{pgfscope}%
\definecolor{textcolor}{rgb}{0.000000,0.000000,0.000000}%
\pgfsetstrokecolor{textcolor}%
\pgfsetfillcolor{textcolor}%
\pgftext[x=3.752974in,y=0.675054in,,top]{\color{textcolor}\rmfamily\fontsize{10.000000}{12.000000}\selectfont 1}%
\end{pgfscope}%
\begin{pgfscope}%
\pgfsetbuttcap%
\pgfsetroundjoin%
\definecolor{currentfill}{rgb}{0.000000,0.000000,0.000000}%
\pgfsetfillcolor{currentfill}%
\pgfsetlinewidth{0.803000pt}%
\definecolor{currentstroke}{rgb}{0.000000,0.000000,0.000000}%
\pgfsetstrokecolor{currentstroke}%
\pgfsetdash{}{0pt}%
\pgfsys@defobject{currentmarker}{\pgfqpoint{0.000000in}{-0.048611in}}{\pgfqpoint{0.000000in}{0.000000in}}{%
\pgfpathmoveto{\pgfqpoint{0.000000in}{0.000000in}}%
\pgfpathlineto{\pgfqpoint{0.000000in}{-0.048611in}}%
\pgfusepath{stroke,fill}%
}%
\begin{pgfscope}%
\pgfsys@transformshift{4.081357in}{0.772276in}%
\pgfsys@useobject{currentmarker}{}%
\end{pgfscope}%
\end{pgfscope}%
\begin{pgfscope}%
\definecolor{textcolor}{rgb}{0.000000,0.000000,0.000000}%
\pgfsetstrokecolor{textcolor}%
\pgfsetfillcolor{textcolor}%
\pgftext[x=4.081357in,y=0.675054in,,top]{\color{textcolor}\rmfamily\fontsize{10.000000}{12.000000}\selectfont 2}%
\end{pgfscope}%
\begin{pgfscope}%
\pgfsetbuttcap%
\pgfsetroundjoin%
\definecolor{currentfill}{rgb}{0.000000,0.000000,0.000000}%
\pgfsetfillcolor{currentfill}%
\pgfsetlinewidth{0.803000pt}%
\definecolor{currentstroke}{rgb}{0.000000,0.000000,0.000000}%
\pgfsetstrokecolor{currentstroke}%
\pgfsetdash{}{0pt}%
\pgfsys@defobject{currentmarker}{\pgfqpoint{0.000000in}{-0.048611in}}{\pgfqpoint{0.000000in}{0.000000in}}{%
\pgfpathmoveto{\pgfqpoint{0.000000in}{0.000000in}}%
\pgfpathlineto{\pgfqpoint{0.000000in}{-0.048611in}}%
\pgfusepath{stroke,fill}%
}%
\begin{pgfscope}%
\pgfsys@transformshift{4.409740in}{0.772276in}%
\pgfsys@useobject{currentmarker}{}%
\end{pgfscope}%
\end{pgfscope}%
\begin{pgfscope}%
\definecolor{textcolor}{rgb}{0.000000,0.000000,0.000000}%
\pgfsetstrokecolor{textcolor}%
\pgfsetfillcolor{textcolor}%
\pgftext[x=4.409740in,y=0.675054in,,top]{\color{textcolor}\rmfamily\fontsize{10.000000}{12.000000}\selectfont 3}%
\end{pgfscope}%
\begin{pgfscope}%
\pgfsetbuttcap%
\pgfsetroundjoin%
\definecolor{currentfill}{rgb}{0.000000,0.000000,0.000000}%
\pgfsetfillcolor{currentfill}%
\pgfsetlinewidth{0.803000pt}%
\definecolor{currentstroke}{rgb}{0.000000,0.000000,0.000000}%
\pgfsetstrokecolor{currentstroke}%
\pgfsetdash{}{0pt}%
\pgfsys@defobject{currentmarker}{\pgfqpoint{0.000000in}{-0.048611in}}{\pgfqpoint{0.000000in}{0.000000in}}{%
\pgfpathmoveto{\pgfqpoint{0.000000in}{0.000000in}}%
\pgfpathlineto{\pgfqpoint{0.000000in}{-0.048611in}}%
\pgfusepath{stroke,fill}%
}%
\begin{pgfscope}%
\pgfsys@transformshift{4.738123in}{0.772276in}%
\pgfsys@useobject{currentmarker}{}%
\end{pgfscope}%
\end{pgfscope}%
\begin{pgfscope}%
\definecolor{textcolor}{rgb}{0.000000,0.000000,0.000000}%
\pgfsetstrokecolor{textcolor}%
\pgfsetfillcolor{textcolor}%
\pgftext[x=4.738123in,y=0.675054in,,top]{\color{textcolor}\rmfamily\fontsize{10.000000}{12.000000}\selectfont 4}%
\end{pgfscope}%
\begin{pgfscope}%
\pgfsetbuttcap%
\pgfsetroundjoin%
\definecolor{currentfill}{rgb}{0.000000,0.000000,0.000000}%
\pgfsetfillcolor{currentfill}%
\pgfsetlinewidth{0.803000pt}%
\definecolor{currentstroke}{rgb}{0.000000,0.000000,0.000000}%
\pgfsetstrokecolor{currentstroke}%
\pgfsetdash{}{0pt}%
\pgfsys@defobject{currentmarker}{\pgfqpoint{0.000000in}{-0.048611in}}{\pgfqpoint{0.000000in}{0.000000in}}{%
\pgfpathmoveto{\pgfqpoint{0.000000in}{0.000000in}}%
\pgfpathlineto{\pgfqpoint{0.000000in}{-0.048611in}}%
\pgfusepath{stroke,fill}%
}%
\begin{pgfscope}%
\pgfsys@transformshift{5.066506in}{0.772276in}%
\pgfsys@useobject{currentmarker}{}%
\end{pgfscope}%
\end{pgfscope}%
\begin{pgfscope}%
\definecolor{textcolor}{rgb}{0.000000,0.000000,0.000000}%
\pgfsetstrokecolor{textcolor}%
\pgfsetfillcolor{textcolor}%
\pgftext[x=5.066506in,y=0.675054in,,top]{\color{textcolor}\rmfamily\fontsize{10.000000}{12.000000}\selectfont 5}%
\end{pgfscope}%
\begin{pgfscope}%
\pgfsetbuttcap%
\pgfsetroundjoin%
\definecolor{currentfill}{rgb}{0.000000,0.000000,0.000000}%
\pgfsetfillcolor{currentfill}%
\pgfsetlinewidth{0.803000pt}%
\definecolor{currentstroke}{rgb}{0.000000,0.000000,0.000000}%
\pgfsetstrokecolor{currentstroke}%
\pgfsetdash{}{0pt}%
\pgfsys@defobject{currentmarker}{\pgfqpoint{0.000000in}{-0.048611in}}{\pgfqpoint{0.000000in}{0.000000in}}{%
\pgfpathmoveto{\pgfqpoint{0.000000in}{0.000000in}}%
\pgfpathlineto{\pgfqpoint{0.000000in}{-0.048611in}}%
\pgfusepath{stroke,fill}%
}%
\begin{pgfscope}%
\pgfsys@transformshift{5.394889in}{0.772276in}%
\pgfsys@useobject{currentmarker}{}%
\end{pgfscope}%
\end{pgfscope}%
\begin{pgfscope}%
\definecolor{textcolor}{rgb}{0.000000,0.000000,0.000000}%
\pgfsetstrokecolor{textcolor}%
\pgfsetfillcolor{textcolor}%
\pgftext[x=5.394889in,y=0.675054in,,top]{\color{textcolor}\rmfamily\fontsize{10.000000}{12.000000}\selectfont 6}%
\end{pgfscope}%
\begin{pgfscope}%
\pgfsetbuttcap%
\pgfsetroundjoin%
\definecolor{currentfill}{rgb}{0.000000,0.000000,0.000000}%
\pgfsetfillcolor{currentfill}%
\pgfsetlinewidth{0.803000pt}%
\definecolor{currentstroke}{rgb}{0.000000,0.000000,0.000000}%
\pgfsetstrokecolor{currentstroke}%
\pgfsetdash{}{0pt}%
\pgfsys@defobject{currentmarker}{\pgfqpoint{0.000000in}{-0.048611in}}{\pgfqpoint{0.000000in}{0.000000in}}{%
\pgfpathmoveto{\pgfqpoint{0.000000in}{0.000000in}}%
\pgfpathlineto{\pgfqpoint{0.000000in}{-0.048611in}}%
\pgfusepath{stroke,fill}%
}%
\begin{pgfscope}%
\pgfsys@transformshift{5.723272in}{0.772276in}%
\pgfsys@useobject{currentmarker}{}%
\end{pgfscope}%
\end{pgfscope}%
\begin{pgfscope}%
\definecolor{textcolor}{rgb}{0.000000,0.000000,0.000000}%
\pgfsetstrokecolor{textcolor}%
\pgfsetfillcolor{textcolor}%
\pgftext[x=5.723272in,y=0.675054in,,top]{\color{textcolor}\rmfamily\fontsize{10.000000}{12.000000}\selectfont 7}%
\end{pgfscope}%
\begin{pgfscope}%
\pgfsetbuttcap%
\pgfsetroundjoin%
\definecolor{currentfill}{rgb}{0.000000,0.000000,0.000000}%
\pgfsetfillcolor{currentfill}%
\pgfsetlinewidth{0.803000pt}%
\definecolor{currentstroke}{rgb}{0.000000,0.000000,0.000000}%
\pgfsetstrokecolor{currentstroke}%
\pgfsetdash{}{0pt}%
\pgfsys@defobject{currentmarker}{\pgfqpoint{0.000000in}{-0.048611in}}{\pgfqpoint{0.000000in}{0.000000in}}{%
\pgfpathmoveto{\pgfqpoint{0.000000in}{0.000000in}}%
\pgfpathlineto{\pgfqpoint{0.000000in}{-0.048611in}}%
\pgfusepath{stroke,fill}%
}%
\begin{pgfscope}%
\pgfsys@transformshift{6.051655in}{0.772276in}%
\pgfsys@useobject{currentmarker}{}%
\end{pgfscope}%
\end{pgfscope}%
\begin{pgfscope}%
\definecolor{textcolor}{rgb}{0.000000,0.000000,0.000000}%
\pgfsetstrokecolor{textcolor}%
\pgfsetfillcolor{textcolor}%
\pgftext[x=6.051655in,y=0.675054in,,top]{\color{textcolor}\rmfamily\fontsize{10.000000}{12.000000}\selectfont 8}%
\end{pgfscope}%
\begin{pgfscope}%
\pgfsetbuttcap%
\pgfsetroundjoin%
\definecolor{currentfill}{rgb}{0.000000,0.000000,0.000000}%
\pgfsetfillcolor{currentfill}%
\pgfsetlinewidth{0.803000pt}%
\definecolor{currentstroke}{rgb}{0.000000,0.000000,0.000000}%
\pgfsetstrokecolor{currentstroke}%
\pgfsetdash{}{0pt}%
\pgfsys@defobject{currentmarker}{\pgfqpoint{0.000000in}{-0.048611in}}{\pgfqpoint{0.000000in}{0.000000in}}{%
\pgfpathmoveto{\pgfqpoint{0.000000in}{0.000000in}}%
\pgfpathlineto{\pgfqpoint{0.000000in}{-0.048611in}}%
\pgfusepath{stroke,fill}%
}%
\begin{pgfscope}%
\pgfsys@transformshift{6.380038in}{0.772276in}%
\pgfsys@useobject{currentmarker}{}%
\end{pgfscope}%
\end{pgfscope}%
\begin{pgfscope}%
\definecolor{textcolor}{rgb}{0.000000,0.000000,0.000000}%
\pgfsetstrokecolor{textcolor}%
\pgfsetfillcolor{textcolor}%
\pgftext[x=6.380038in,y=0.675054in,,top]{\color{textcolor}\rmfamily\fontsize{10.000000}{12.000000}\selectfont 9}%
\end{pgfscope}%
\begin{pgfscope}%
\pgfsetbuttcap%
\pgfsetroundjoin%
\definecolor{currentfill}{rgb}{0.000000,0.000000,0.000000}%
\pgfsetfillcolor{currentfill}%
\pgfsetlinewidth{0.803000pt}%
\definecolor{currentstroke}{rgb}{0.000000,0.000000,0.000000}%
\pgfsetstrokecolor{currentstroke}%
\pgfsetdash{}{0pt}%
\pgfsys@defobject{currentmarker}{\pgfqpoint{-0.048611in}{0.000000in}}{\pgfqpoint{-0.000000in}{0.000000in}}{%
\pgfpathmoveto{\pgfqpoint{-0.000000in}{0.000000in}}%
\pgfpathlineto{\pgfqpoint{-0.048611in}{0.000000in}}%
\pgfusepath{stroke,fill}%
}%
\begin{pgfscope}%
\pgfsys@transformshift{3.588782in}{3.563532in}%
\pgfsys@useobject{currentmarker}{}%
\end{pgfscope}%
\end{pgfscope}%
\begin{pgfscope}%
\definecolor{textcolor}{rgb}{0.000000,0.000000,0.000000}%
\pgfsetstrokecolor{textcolor}%
\pgfsetfillcolor{textcolor}%
\pgftext[x=3.491560in,y=3.563532in,right,]{\color{textcolor}\rmfamily\fontsize{10.000000}{12.000000}\selectfont 1}%
\end{pgfscope}%
\begin{pgfscope}%
\pgfsetbuttcap%
\pgfsetroundjoin%
\definecolor{currentfill}{rgb}{0.000000,0.000000,0.000000}%
\pgfsetfillcolor{currentfill}%
\pgfsetlinewidth{0.803000pt}%
\definecolor{currentstroke}{rgb}{0.000000,0.000000,0.000000}%
\pgfsetstrokecolor{currentstroke}%
\pgfsetdash{}{0pt}%
\pgfsys@defobject{currentmarker}{\pgfqpoint{-0.048611in}{0.000000in}}{\pgfqpoint{-0.000000in}{0.000000in}}{%
\pgfpathmoveto{\pgfqpoint{-0.000000in}{0.000000in}}%
\pgfpathlineto{\pgfqpoint{-0.048611in}{0.000000in}}%
\pgfusepath{stroke,fill}%
}%
\begin{pgfscope}%
\pgfsys@transformshift{3.588782in}{3.235149in}%
\pgfsys@useobject{currentmarker}{}%
\end{pgfscope}%
\end{pgfscope}%
\begin{pgfscope}%
\definecolor{textcolor}{rgb}{0.000000,0.000000,0.000000}%
\pgfsetstrokecolor{textcolor}%
\pgfsetfillcolor{textcolor}%
\pgftext[x=3.491560in,y=3.235149in,right,]{\color{textcolor}\rmfamily\fontsize{10.000000}{12.000000}\selectfont 2}%
\end{pgfscope}%
\begin{pgfscope}%
\pgfsetbuttcap%
\pgfsetroundjoin%
\definecolor{currentfill}{rgb}{0.000000,0.000000,0.000000}%
\pgfsetfillcolor{currentfill}%
\pgfsetlinewidth{0.803000pt}%
\definecolor{currentstroke}{rgb}{0.000000,0.000000,0.000000}%
\pgfsetstrokecolor{currentstroke}%
\pgfsetdash{}{0pt}%
\pgfsys@defobject{currentmarker}{\pgfqpoint{-0.048611in}{0.000000in}}{\pgfqpoint{-0.000000in}{0.000000in}}{%
\pgfpathmoveto{\pgfqpoint{-0.000000in}{0.000000in}}%
\pgfpathlineto{\pgfqpoint{-0.048611in}{0.000000in}}%
\pgfusepath{stroke,fill}%
}%
\begin{pgfscope}%
\pgfsys@transformshift{3.588782in}{2.906766in}%
\pgfsys@useobject{currentmarker}{}%
\end{pgfscope}%
\end{pgfscope}%
\begin{pgfscope}%
\definecolor{textcolor}{rgb}{0.000000,0.000000,0.000000}%
\pgfsetstrokecolor{textcolor}%
\pgfsetfillcolor{textcolor}%
\pgftext[x=3.491560in,y=2.906766in,right,]{\color{textcolor}\rmfamily\fontsize{10.000000}{12.000000}\selectfont 3}%
\end{pgfscope}%
\begin{pgfscope}%
\pgfsetbuttcap%
\pgfsetroundjoin%
\definecolor{currentfill}{rgb}{0.000000,0.000000,0.000000}%
\pgfsetfillcolor{currentfill}%
\pgfsetlinewidth{0.803000pt}%
\definecolor{currentstroke}{rgb}{0.000000,0.000000,0.000000}%
\pgfsetstrokecolor{currentstroke}%
\pgfsetdash{}{0pt}%
\pgfsys@defobject{currentmarker}{\pgfqpoint{-0.048611in}{0.000000in}}{\pgfqpoint{-0.000000in}{0.000000in}}{%
\pgfpathmoveto{\pgfqpoint{-0.000000in}{0.000000in}}%
\pgfpathlineto{\pgfqpoint{-0.048611in}{0.000000in}}%
\pgfusepath{stroke,fill}%
}%
\begin{pgfscope}%
\pgfsys@transformshift{3.588782in}{2.578383in}%
\pgfsys@useobject{currentmarker}{}%
\end{pgfscope}%
\end{pgfscope}%
\begin{pgfscope}%
\definecolor{textcolor}{rgb}{0.000000,0.000000,0.000000}%
\pgfsetstrokecolor{textcolor}%
\pgfsetfillcolor{textcolor}%
\pgftext[x=3.491560in,y=2.578383in,right,]{\color{textcolor}\rmfamily\fontsize{10.000000}{12.000000}\selectfont 4}%
\end{pgfscope}%
\begin{pgfscope}%
\pgfsetbuttcap%
\pgfsetroundjoin%
\definecolor{currentfill}{rgb}{0.000000,0.000000,0.000000}%
\pgfsetfillcolor{currentfill}%
\pgfsetlinewidth{0.803000pt}%
\definecolor{currentstroke}{rgb}{0.000000,0.000000,0.000000}%
\pgfsetstrokecolor{currentstroke}%
\pgfsetdash{}{0pt}%
\pgfsys@defobject{currentmarker}{\pgfqpoint{-0.048611in}{0.000000in}}{\pgfqpoint{-0.000000in}{0.000000in}}{%
\pgfpathmoveto{\pgfqpoint{-0.000000in}{0.000000in}}%
\pgfpathlineto{\pgfqpoint{-0.048611in}{0.000000in}}%
\pgfusepath{stroke,fill}%
}%
\begin{pgfscope}%
\pgfsys@transformshift{3.588782in}{2.250000in}%
\pgfsys@useobject{currentmarker}{}%
\end{pgfscope}%
\end{pgfscope}%
\begin{pgfscope}%
\definecolor{textcolor}{rgb}{0.000000,0.000000,0.000000}%
\pgfsetstrokecolor{textcolor}%
\pgfsetfillcolor{textcolor}%
\pgftext[x=3.491560in,y=2.250000in,right,]{\color{textcolor}\rmfamily\fontsize{10.000000}{12.000000}\selectfont 5}%
\end{pgfscope}%
\begin{pgfscope}%
\pgfsetbuttcap%
\pgfsetroundjoin%
\definecolor{currentfill}{rgb}{0.000000,0.000000,0.000000}%
\pgfsetfillcolor{currentfill}%
\pgfsetlinewidth{0.803000pt}%
\definecolor{currentstroke}{rgb}{0.000000,0.000000,0.000000}%
\pgfsetstrokecolor{currentstroke}%
\pgfsetdash{}{0pt}%
\pgfsys@defobject{currentmarker}{\pgfqpoint{-0.048611in}{0.000000in}}{\pgfqpoint{-0.000000in}{0.000000in}}{%
\pgfpathmoveto{\pgfqpoint{-0.000000in}{0.000000in}}%
\pgfpathlineto{\pgfqpoint{-0.048611in}{0.000000in}}%
\pgfusepath{stroke,fill}%
}%
\begin{pgfscope}%
\pgfsys@transformshift{3.588782in}{1.921617in}%
\pgfsys@useobject{currentmarker}{}%
\end{pgfscope}%
\end{pgfscope}%
\begin{pgfscope}%
\definecolor{textcolor}{rgb}{0.000000,0.000000,0.000000}%
\pgfsetstrokecolor{textcolor}%
\pgfsetfillcolor{textcolor}%
\pgftext[x=3.491560in,y=1.921617in,right,]{\color{textcolor}\rmfamily\fontsize{10.000000}{12.000000}\selectfont 6}%
\end{pgfscope}%
\begin{pgfscope}%
\pgfsetbuttcap%
\pgfsetroundjoin%
\definecolor{currentfill}{rgb}{0.000000,0.000000,0.000000}%
\pgfsetfillcolor{currentfill}%
\pgfsetlinewidth{0.803000pt}%
\definecolor{currentstroke}{rgb}{0.000000,0.000000,0.000000}%
\pgfsetstrokecolor{currentstroke}%
\pgfsetdash{}{0pt}%
\pgfsys@defobject{currentmarker}{\pgfqpoint{-0.048611in}{0.000000in}}{\pgfqpoint{-0.000000in}{0.000000in}}{%
\pgfpathmoveto{\pgfqpoint{-0.000000in}{0.000000in}}%
\pgfpathlineto{\pgfqpoint{-0.048611in}{0.000000in}}%
\pgfusepath{stroke,fill}%
}%
\begin{pgfscope}%
\pgfsys@transformshift{3.588782in}{1.593234in}%
\pgfsys@useobject{currentmarker}{}%
\end{pgfscope}%
\end{pgfscope}%
\begin{pgfscope}%
\definecolor{textcolor}{rgb}{0.000000,0.000000,0.000000}%
\pgfsetstrokecolor{textcolor}%
\pgfsetfillcolor{textcolor}%
\pgftext[x=3.491560in,y=1.593234in,right,]{\color{textcolor}\rmfamily\fontsize{10.000000}{12.000000}\selectfont 7}%
\end{pgfscope}%
\begin{pgfscope}%
\pgfsetbuttcap%
\pgfsetroundjoin%
\definecolor{currentfill}{rgb}{0.000000,0.000000,0.000000}%
\pgfsetfillcolor{currentfill}%
\pgfsetlinewidth{0.803000pt}%
\definecolor{currentstroke}{rgb}{0.000000,0.000000,0.000000}%
\pgfsetstrokecolor{currentstroke}%
\pgfsetdash{}{0pt}%
\pgfsys@defobject{currentmarker}{\pgfqpoint{-0.048611in}{0.000000in}}{\pgfqpoint{-0.000000in}{0.000000in}}{%
\pgfpathmoveto{\pgfqpoint{-0.000000in}{0.000000in}}%
\pgfpathlineto{\pgfqpoint{-0.048611in}{0.000000in}}%
\pgfusepath{stroke,fill}%
}%
\begin{pgfscope}%
\pgfsys@transformshift{3.588782in}{1.264851in}%
\pgfsys@useobject{currentmarker}{}%
\end{pgfscope}%
\end{pgfscope}%
\begin{pgfscope}%
\definecolor{textcolor}{rgb}{0.000000,0.000000,0.000000}%
\pgfsetstrokecolor{textcolor}%
\pgfsetfillcolor{textcolor}%
\pgftext[x=3.491560in,y=1.264851in,right,]{\color{textcolor}\rmfamily\fontsize{10.000000}{12.000000}\selectfont 8}%
\end{pgfscope}%
\begin{pgfscope}%
\pgfsetbuttcap%
\pgfsetroundjoin%
\definecolor{currentfill}{rgb}{0.000000,0.000000,0.000000}%
\pgfsetfillcolor{currentfill}%
\pgfsetlinewidth{0.803000pt}%
\definecolor{currentstroke}{rgb}{0.000000,0.000000,0.000000}%
\pgfsetstrokecolor{currentstroke}%
\pgfsetdash{}{0pt}%
\pgfsys@defobject{currentmarker}{\pgfqpoint{-0.048611in}{0.000000in}}{\pgfqpoint{-0.000000in}{0.000000in}}{%
\pgfpathmoveto{\pgfqpoint{-0.000000in}{0.000000in}}%
\pgfpathlineto{\pgfqpoint{-0.048611in}{0.000000in}}%
\pgfusepath{stroke,fill}%
}%
\begin{pgfscope}%
\pgfsys@transformshift{3.588782in}{0.936468in}%
\pgfsys@useobject{currentmarker}{}%
\end{pgfscope}%
\end{pgfscope}%
\begin{pgfscope}%
\definecolor{textcolor}{rgb}{0.000000,0.000000,0.000000}%
\pgfsetstrokecolor{textcolor}%
\pgfsetfillcolor{textcolor}%
\pgftext[x=3.491560in,y=0.936468in,right,]{\color{textcolor}\rmfamily\fontsize{10.000000}{12.000000}\selectfont 9}%
\end{pgfscope}%
\begin{pgfscope}%
\definecolor{textcolor}{rgb}{0.150000,0.150000,0.150000}%
\pgfsetstrokecolor{textcolor}%
\pgfsetfillcolor{textcolor}%
\pgftext[x=3.752974in,y=3.235149in,,]{\color{textcolor}\rmfamily\fontsize{10.000000}{12.000000}\selectfont 0}%
\end{pgfscope}%
\begin{pgfscope}%
\definecolor{textcolor}{rgb}{0.150000,0.150000,0.150000}%
\pgfsetstrokecolor{textcolor}%
\pgfsetfillcolor{textcolor}%
\pgftext[x=3.752974in,y=2.906766in,,]{\color{textcolor}\rmfamily\fontsize{10.000000}{12.000000}\selectfont 0.05}%
\end{pgfscope}%
\begin{pgfscope}%
\definecolor{textcolor}{rgb}{0.150000,0.150000,0.150000}%
\pgfsetstrokecolor{textcolor}%
\pgfsetfillcolor{textcolor}%
\pgftext[x=4.081357in,y=2.906766in,,]{\color{textcolor}\rmfamily\fontsize{10.000000}{12.000000}\selectfont 0.05}%
\end{pgfscope}%
\begin{pgfscope}%
\definecolor{textcolor}{rgb}{0.150000,0.150000,0.150000}%
\pgfsetstrokecolor{textcolor}%
\pgfsetfillcolor{textcolor}%
\pgftext[x=3.752974in,y=2.578383in,,]{\color{textcolor}\rmfamily\fontsize{10.000000}{12.000000}\selectfont 0.01}%
\end{pgfscope}%
\begin{pgfscope}%
\definecolor{textcolor}{rgb}{0.150000,0.150000,0.150000}%
\pgfsetstrokecolor{textcolor}%
\pgfsetfillcolor{textcolor}%
\pgftext[x=4.081357in,y=2.578383in,,]{\color{textcolor}\rmfamily\fontsize{10.000000}{12.000000}\selectfont 0.02}%
\end{pgfscope}%
\begin{pgfscope}%
\definecolor{textcolor}{rgb}{0.150000,0.150000,0.150000}%
\pgfsetstrokecolor{textcolor}%
\pgfsetfillcolor{textcolor}%
\pgftext[x=4.409740in,y=2.578383in,,]{\color{textcolor}\rmfamily\fontsize{10.000000}{12.000000}\selectfont 0.06}%
\end{pgfscope}%
\begin{pgfscope}%
\definecolor{textcolor}{rgb}{0.150000,0.150000,0.150000}%
\pgfsetstrokecolor{textcolor}%
\pgfsetfillcolor{textcolor}%
\pgftext[x=3.752974in,y=2.250000in,,]{\color{textcolor}\rmfamily\fontsize{10.000000}{12.000000}\selectfont 0.01}%
\end{pgfscope}%
\begin{pgfscope}%
\definecolor{textcolor}{rgb}{0.150000,0.150000,0.150000}%
\pgfsetstrokecolor{textcolor}%
\pgfsetfillcolor{textcolor}%
\pgftext[x=4.081357in,y=2.250000in,,]{\color{textcolor}\rmfamily\fontsize{10.000000}{12.000000}\selectfont 0.01}%
\end{pgfscope}%
\begin{pgfscope}%
\definecolor{textcolor}{rgb}{0.150000,0.150000,0.150000}%
\pgfsetstrokecolor{textcolor}%
\pgfsetfillcolor{textcolor}%
\pgftext[x=4.409740in,y=2.250000in,,]{\color{textcolor}\rmfamily\fontsize{10.000000}{12.000000}\selectfont 0.01}%
\end{pgfscope}%
\begin{pgfscope}%
\definecolor{textcolor}{rgb}{0.150000,0.150000,0.150000}%
\pgfsetstrokecolor{textcolor}%
\pgfsetfillcolor{textcolor}%
\pgftext[x=4.738123in,y=2.250000in,,]{\color{textcolor}\rmfamily\fontsize{10.000000}{12.000000}\selectfont 0}%
\end{pgfscope}%
\begin{pgfscope}%
\definecolor{textcolor}{rgb}{0.150000,0.150000,0.150000}%
\pgfsetstrokecolor{textcolor}%
\pgfsetfillcolor{textcolor}%
\pgftext[x=3.752974in,y=1.921617in,,]{\color{textcolor}\rmfamily\fontsize{10.000000}{12.000000}\selectfont 0.01}%
\end{pgfscope}%
\begin{pgfscope}%
\definecolor{textcolor}{rgb}{0.150000,0.150000,0.150000}%
\pgfsetstrokecolor{textcolor}%
\pgfsetfillcolor{textcolor}%
\pgftext[x=4.081357in,y=1.921617in,,]{\color{textcolor}\rmfamily\fontsize{10.000000}{12.000000}\selectfont 0.01}%
\end{pgfscope}%
\begin{pgfscope}%
\definecolor{textcolor}{rgb}{0.150000,0.150000,0.150000}%
\pgfsetstrokecolor{textcolor}%
\pgfsetfillcolor{textcolor}%
\pgftext[x=4.409740in,y=1.921617in,,]{\color{textcolor}\rmfamily\fontsize{10.000000}{12.000000}\selectfont 0.01}%
\end{pgfscope}%
\begin{pgfscope}%
\definecolor{textcolor}{rgb}{0.150000,0.150000,0.150000}%
\pgfsetstrokecolor{textcolor}%
\pgfsetfillcolor{textcolor}%
\pgftext[x=4.738123in,y=1.921617in,,]{\color{textcolor}\rmfamily\fontsize{10.000000}{12.000000}\selectfont 0.01}%
\end{pgfscope}%
\begin{pgfscope}%
\definecolor{textcolor}{rgb}{0.150000,0.150000,0.150000}%
\pgfsetstrokecolor{textcolor}%
\pgfsetfillcolor{textcolor}%
\pgftext[x=5.066506in,y=1.921617in,,]{\color{textcolor}\rmfamily\fontsize{10.000000}{12.000000}\selectfont 0.01}%
\end{pgfscope}%
\begin{pgfscope}%
\definecolor{textcolor}{rgb}{0.150000,0.150000,0.150000}%
\pgfsetstrokecolor{textcolor}%
\pgfsetfillcolor{textcolor}%
\pgftext[x=3.752974in,y=1.593234in,,]{\color{textcolor}\rmfamily\fontsize{10.000000}{12.000000}\selectfont 0.01}%
\end{pgfscope}%
\begin{pgfscope}%
\definecolor{textcolor}{rgb}{0.150000,0.150000,0.150000}%
\pgfsetstrokecolor{textcolor}%
\pgfsetfillcolor{textcolor}%
\pgftext[x=4.081357in,y=1.593234in,,]{\color{textcolor}\rmfamily\fontsize{10.000000}{12.000000}\selectfont 0.01}%
\end{pgfscope}%
\begin{pgfscope}%
\definecolor{textcolor}{rgb}{0.150000,0.150000,0.150000}%
\pgfsetstrokecolor{textcolor}%
\pgfsetfillcolor{textcolor}%
\pgftext[x=4.409740in,y=1.593234in,,]{\color{textcolor}\rmfamily\fontsize{10.000000}{12.000000}\selectfont 0.03}%
\end{pgfscope}%
\begin{pgfscope}%
\definecolor{textcolor}{rgb}{0.150000,0.150000,0.150000}%
\pgfsetstrokecolor{textcolor}%
\pgfsetfillcolor{textcolor}%
\pgftext[x=4.738123in,y=1.593234in,,]{\color{textcolor}\rmfamily\fontsize{10.000000}{12.000000}\selectfont 0.02}%
\end{pgfscope}%
\begin{pgfscope}%
\definecolor{textcolor}{rgb}{0.150000,0.150000,0.150000}%
\pgfsetstrokecolor{textcolor}%
\pgfsetfillcolor{textcolor}%
\pgftext[x=5.066506in,y=1.593234in,,]{\color{textcolor}\rmfamily\fontsize{10.000000}{12.000000}\selectfont 0.02}%
\end{pgfscope}%
\begin{pgfscope}%
\definecolor{textcolor}{rgb}{0.150000,0.150000,0.150000}%
\pgfsetstrokecolor{textcolor}%
\pgfsetfillcolor{textcolor}%
\pgftext[x=5.394889in,y=1.593234in,,]{\color{textcolor}\rmfamily\fontsize{10.000000}{12.000000}\selectfont 0.01}%
\end{pgfscope}%
\begin{pgfscope}%
\definecolor{textcolor}{rgb}{0.150000,0.150000,0.150000}%
\pgfsetstrokecolor{textcolor}%
\pgfsetfillcolor{textcolor}%
\pgftext[x=3.752974in,y=1.264851in,,]{\color{textcolor}\rmfamily\fontsize{10.000000}{12.000000}\selectfont 0.02}%
\end{pgfscope}%
\begin{pgfscope}%
\definecolor{textcolor}{rgb}{0.150000,0.150000,0.150000}%
\pgfsetstrokecolor{textcolor}%
\pgfsetfillcolor{textcolor}%
\pgftext[x=4.081357in,y=1.264851in,,]{\color{textcolor}\rmfamily\fontsize{10.000000}{12.000000}\selectfont 0.02}%
\end{pgfscope}%
\begin{pgfscope}%
\definecolor{textcolor}{rgb}{1.000000,1.000000,1.000000}%
\pgfsetstrokecolor{textcolor}%
\pgfsetfillcolor{textcolor}%
\pgftext[x=4.409740in,y=1.264851in,,]{\color{textcolor}\rmfamily\fontsize{10.000000}{12.000000}\selectfont 0.11}%
\end{pgfscope}%
\begin{pgfscope}%
\definecolor{textcolor}{rgb}{0.150000,0.150000,0.150000}%
\pgfsetstrokecolor{textcolor}%
\pgfsetfillcolor{textcolor}%
\pgftext[x=4.738123in,y=1.264851in,,]{\color{textcolor}\rmfamily\fontsize{10.000000}{12.000000}\selectfont 0.03}%
\end{pgfscope}%
\begin{pgfscope}%
\definecolor{textcolor}{rgb}{0.150000,0.150000,0.150000}%
\pgfsetstrokecolor{textcolor}%
\pgfsetfillcolor{textcolor}%
\pgftext[x=5.066506in,y=1.264851in,,]{\color{textcolor}\rmfamily\fontsize{10.000000}{12.000000}\selectfont 0.04}%
\end{pgfscope}%
\begin{pgfscope}%
\definecolor{textcolor}{rgb}{0.150000,0.150000,0.150000}%
\pgfsetstrokecolor{textcolor}%
\pgfsetfillcolor{textcolor}%
\pgftext[x=5.394889in,y=1.264851in,,]{\color{textcolor}\rmfamily\fontsize{10.000000}{12.000000}\selectfont 0.01}%
\end{pgfscope}%
\begin{pgfscope}%
\definecolor{textcolor}{rgb}{0.150000,0.150000,0.150000}%
\pgfsetstrokecolor{textcolor}%
\pgfsetfillcolor{textcolor}%
\pgftext[x=5.723272in,y=1.264851in,,]{\color{textcolor}\rmfamily\fontsize{10.000000}{12.000000}\selectfont 0.03}%
\end{pgfscope}%
\begin{pgfscope}%
\definecolor{textcolor}{rgb}{0.150000,0.150000,0.150000}%
\pgfsetstrokecolor{textcolor}%
\pgfsetfillcolor{textcolor}%
\pgftext[x=3.752974in,y=0.936468in,,]{\color{textcolor}\rmfamily\fontsize{10.000000}{12.000000}\selectfont 0.02}%
\end{pgfscope}%
\begin{pgfscope}%
\definecolor{textcolor}{rgb}{0.150000,0.150000,0.150000}%
\pgfsetstrokecolor{textcolor}%
\pgfsetfillcolor{textcolor}%
\pgftext[x=4.081357in,y=0.936468in,,]{\color{textcolor}\rmfamily\fontsize{10.000000}{12.000000}\selectfont 0.03}%
\end{pgfscope}%
\begin{pgfscope}%
\definecolor{textcolor}{rgb}{1.000000,1.000000,1.000000}%
\pgfsetstrokecolor{textcolor}%
\pgfsetfillcolor{textcolor}%
\pgftext[x=4.409740in,y=0.936468in,,]{\color{textcolor}\rmfamily\fontsize{10.000000}{12.000000}\selectfont 0.12}%
\end{pgfscope}%
\begin{pgfscope}%
\definecolor{textcolor}{rgb}{0.150000,0.150000,0.150000}%
\pgfsetstrokecolor{textcolor}%
\pgfsetfillcolor{textcolor}%
\pgftext[x=4.738123in,y=0.936468in,,]{\color{textcolor}\rmfamily\fontsize{10.000000}{12.000000}\selectfont 0.03}%
\end{pgfscope}%
\begin{pgfscope}%
\definecolor{textcolor}{rgb}{0.150000,0.150000,0.150000}%
\pgfsetstrokecolor{textcolor}%
\pgfsetfillcolor{textcolor}%
\pgftext[x=5.066506in,y=0.936468in,,]{\color{textcolor}\rmfamily\fontsize{10.000000}{12.000000}\selectfont 0.04}%
\end{pgfscope}%
\begin{pgfscope}%
\definecolor{textcolor}{rgb}{0.150000,0.150000,0.150000}%
\pgfsetstrokecolor{textcolor}%
\pgfsetfillcolor{textcolor}%
\pgftext[x=5.394889in,y=0.936468in,,]{\color{textcolor}\rmfamily\fontsize{10.000000}{12.000000}\selectfont 0.02}%
\end{pgfscope}%
\begin{pgfscope}%
\definecolor{textcolor}{rgb}{0.150000,0.150000,0.150000}%
\pgfsetstrokecolor{textcolor}%
\pgfsetfillcolor{textcolor}%
\pgftext[x=5.723272in,y=0.936468in,,]{\color{textcolor}\rmfamily\fontsize{10.000000}{12.000000}\selectfont 0.04}%
\end{pgfscope}%
\begin{pgfscope}%
\definecolor{textcolor}{rgb}{0.150000,0.150000,0.150000}%
\pgfsetstrokecolor{textcolor}%
\pgfsetfillcolor{textcolor}%
\pgftext[x=6.051655in,y=0.936468in,,]{\color{textcolor}\rmfamily\fontsize{10.000000}{12.000000}\selectfont 0}%
\end{pgfscope}%
\begin{pgfscope}%
\definecolor{textcolor}{rgb}{0.000000,0.000000,0.000000}%
\pgfsetstrokecolor{textcolor}%
\pgfsetfillcolor{textcolor}%
\pgftext[x=5.066506in,y=3.811057in,,base]{\color{textcolor}\rmfamily\fontsize{11.000000}{13.200000}\selectfont Iterierte PAF (RMSE = 0.0364)}%
\end{pgfscope}%
\end{pgfpicture}%
\makeatother%
\endgroup%

		\caption[(Absolute) Residualmatrizen als Heatmap]{Hier sieht man die absoluten Abweichungen zwischen der empirischen und der reproduzierten Korrelationsmatrix bei zwei verschiedenen Methoden. Vor allem die Korrelationen zwischen Merkmal 3 (\textit{housing\_median\_age}) und
			den restlichen Merkmalen können nicht gut reproduziert werden. Die Abweichungen sind bei der iterierten Hauptachsen-Faktorisierung
			jedoch im Durchschnitt geringer. Diese Extraktionsmethode kann die empirische Korrelationsmatrix also besser reproduzieren.
			(Eigene Darstellung)}
		\label{fig:Residualmatrizen}
	\end{figure}
	
