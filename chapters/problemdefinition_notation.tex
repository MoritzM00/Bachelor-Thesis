%% ==============================
\chapter{Problemdefinition und Notation}
\label{ch:ProblemdefinitionUndNotation}
%% ==============================

Oftmals liegen hochdimensionale Daten in einem niedrigdimensionalem Unterraum (engl. \textit{manifold}). Wir beziehen uns auf die hochdimensionalen Daten $\vect{x_1},\ldots,\vect{x_n}$ mit der Datenmatrix $\mat{X} \in \mathbb{R}^{n \times D}$. Jeder Datenpunkt $\vect{x_i}$ ist also ein $D$-dimensionaler Vektor. Die niedrigdimensionale Repräsentation bezeichnen wir mit $\vect{y_1},\ldots,\vect{y_n}$, die in einer $n \times d$-Datenmatrix $\mat{Y}$ angeordnet sind. Hierbei bezeichnet $d$ die
\newterm{intrinsische Dimension} des Datensatzes, wobei $d < D$ gilt, um eine Dimensionsreduktion zu erreichen.


Das Ziel der Dimensionsreduktion ist es nun, diesen Unterraum zu finden und die Daten in diese niedrigdimensionale Repräsentation zu projizieren, wobei möglichst wenig Informationen gegenüber der ursprünglichen (hochdimensionalen) Repräsentation verloren gehen sollen.