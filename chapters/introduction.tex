%% ==============================
\chapter{Einleitung}
\label{ch:Enleitung}
%% ==============================

\idea{
	\begin{enumerate}
		\item Motivation der Dimensionsreduktion darstellen (s. Kasten eins weiter unten)
		\item Was ist Dimensionsreduktion in verständlichen Worten?
		\item Anwendungsgebiete, zB Stable Diffusion
	\end{enumerate}
}
\idea{Wofür braucht man DimRed?
	\begin{enumerate}
		\item Curse of Dimensionality
		\item Reduzierung des Speicheraufwands
		\item Reduzierung des Rechenaufwands beim Training
		\item Reduzierung von Overfitting
		\item Datenvisualisierung
		\item Anwendungen v.a. auch in Computer Vision
	\end{enumerate}
	Punkte zwei und drei (und eins eig. auch) treffen auf Stable Diffusion zu
}
\idea{Related Work: Welche Arbeiten wurden auf diesem Bereich schon veröffentlicht

	Wie grenze ich meine Arbeit davon ab?}

Das Ziel dieser Arbeit ist es herauszufinden, ob modernere Methoden der Dimensionsreduktion wie zum
Beispiel \newterm{Autoencoder} gegenüber traditionelleren Methoden wie der
\newterm{Hauptkomponentenanalyse} (PCA) einen echten Vorteil durch die erhöhte Flexibilität
erreichen können. Die hier vorgestellten Methoden sind die PCA, Kernel PCA sowie t-SNE \unsure{wie
	nenne ich die Methoden hier? mit Abkürzungen oder ausgeschrieben, englisch/deutsch?}als Vertreter
der traditionellen Methoden (\secref{ch:MethodenDerDimRed:traditionell}) und der Autoencoder,
Variational Autoencoder sowie die Self-Organizing Maps als Vertreter der modernen Methoden
(\secref{ch:MethodenDerDimRed:modern}). Darüber hinaus soll herausgearbeitet werden, wann sich
bestimmte Algorithmen besser eignen als andere und dies begründet darlegt werden. Außerdem wird
untersucht, inwiefern ein Autoencoder mit ausschließlich linearen Aktivierungsfunktionen
tatsächlich mit der Hauptkomponentenanalyse übereinstimmt und was passiert, wenn man schrittweise
Nichtlinearität hinzunimmt.

Die Arbeit gliedert sich wie folgt in drei Teile ein: Zuerst wird in
\chapref{ch:Dimensionsreduktion} die Grundlage für das Verständnis der Dimensionsreduktionsmethoden
gelegt, sowie wichtige Begrifflichkeiten erläutert. Im darauffolgenden
\chapref{ch:MethodenDerDimRed} werden dann die einzelnen traditionellen und modernen Methoden
vorgestellt. Anschließend werden die zwei Gruppen in einem systematischen empirischen Vergleich in
\chapref{ch:Vergleich} gegenübergestellt und abgeschlossen wird die Arbeit mit einem Fazit in
\chapref{ch:Schluss}.