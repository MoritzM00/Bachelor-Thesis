%% ==============================
\chapter{Mathematische Notation}
\label{ch:MathematischeNotation}
%% ==============================

In diesem Kapital wird kurz auf die in dieser Arbeit benutzte mathematische Notation eingegangen.

Skalare werden klein und kursiv geschrieben.
Vektoren werden durch ein fettgedruckte kleine Buchstaben gekennzeichnet.
Matrizen werden dann durch

Wir bezeichnen ein Skalar mit $a$\nomenclature[A, 01]{$a$}{Ein Skalar} und einen Vektor mit $\vect{a}$\nomenclature[A, 02]{$\vect{a}$}{Ein Vektor}. Eine Matrix wird mit einem fettgedruckten Großbuchstaben $\mat{A}$\nomenclature[A, 03]{$\mat{A}$}{Eine Matrix} gekennzeichnet.
Zufallszahlen sind die kursiven Äquivalente zu Skalaren, Vektoren und Matrizen.

\nomenclature[Z, 01]{$\rv{a}$}{Eine eindimensionale Zufallsvariable}
\nomenclature[Z, 02]{$\rvect{a}$}{Ein Zufallsvektor}
\nomenclature[Z, 03]{$\rmat{A}$}{Eine Zufallsmatrix}
