
\nomenclature[A, 02]{$\vect{a}$}{Ein Vektor}
\nomenclature[A, 03]{$\mat{A}$}{Eine Matrix}
\nomenclature[A, 04]{$\tr{\mat{A}}$}{Die Transponierte einer Matrix}
\nomenclature[A, 05]{$\det(\mat{A})$}{Die Determinante einer Matrix}
\nomenclature[A, 06]{$\set{A}$}{Eine Menge}
\nomenclature[A, 06]{$|\set{A}|$}{Die Kardinalität (auch: Mächtigkeit) einer Menge $\set{A}$}
\nomenclature[A, 06]{$\set{A} \cap \set{B}$}{Der Durschnitt zweier Mengen $\set{A}$ und $\set{B}$}
\nomenclature[A, 07]{$\N$}{Die natürlichen Zahlen}
\nomenclature[A, 07]{$\real$}{Die reellen Zahlen}
\nomenclature[A, 08]{$f: \set{A} \rightarrow \set{B}$}{Eine Funktion $f$, die von der Menge $\set{A}$ auf die Menge $\set{B}$ abbildet.}

\nomenclature[A, 09]{$\diag(\vect{a})$}{Eine Diagonalmatrix mit den Diagonalelementen impliziert durch den Vektor $\vect{a}$}
\nomenclature[A, 10]{$\diag(\mat{A})$}{Ein Vektor mit den Diagonalelementen der Matrix $\mat{A}$}
\nomenclature[A, 11]{$\norm{\vect{a}}$}{Die euklidische Norm (auch L2-Norm) des Vektors $\vect{a}$}
\nomenclature[A, 12]{$\norm{\mat{A}}_F$}{Die Frobenius-Norm der Matrix $\mat{A}$}
\nomenclature[A, 13]{$\identity_n$}{Eine $n \times n$ Identitätsmatrix}

% =============================
%\nomenclature[P, 01]{$\rv{a}$}{Eine eindimensionale Zufallsvariable}
%\nomenclature[P, 02]{$\rvect{a}$}{Ein vektorwertige Zufallsvariable, auch Zufallsvektor}
%\nomenclature[P, 03]{$\rmat{A}$}{Eine matrixwertige Zufallsvariable, auch Zufallsmatrix}

\nomenclature[P, 04]{$\rv{x} \sim P$}{$\rv{x}$ folgt einer $P$-Verteilung}
\nomenclature[P, 05]{$\Exp_{\rv{x} \sim P}[f(x)]$ oder $\Exp{[f(x)]}$}{Erwartungswert von $f(x)$ bezüglich $P$}

\nomenclature[P, 06]{$\Var(f(x))$}{Varianz von $f(x)$}
\nomenclature[P, 07]{$\Cov(f(x), g(x))$}{Kovarianz von $f(x)$ und $g(x)$}

%\nomenclature[P, 08]{$\KLDiv{P}{Q}$}{Kullback-Leibler Divergenz von zwei Verteilungen $P$ und $Q$}
%\nomenclature[P, 09]{$\gaussian(\vect{x} | \vect{\mu}, \mat{\Sigma})$}{Normalverteilung über $\vect{x}$ mit Erwartungswertvektor $\vect{\mu}$ und Kovarianzmatrix $\mat{\Sigma}$}

\nomenclature[D, 01]{$d$}{Die intrinsische Dimension der Daten}
\nomenclature[D, 02]{$D$}{Die extrinsische Dimension der Daten}
\nomenclature[D, 03]{$\mathcal{X}$}{Der Ursprungsraum der Daten, üblicherweise gilt $\mathcal{X} = \real^D$}
\nomenclature[D, 04]{$\vect{x}_1,\ldots,\vect{x}_n \in \mathcal{X}$}{Eine Stichprobe vom Umfang n}
\nomenclature[D, 05]{$\mat{X}$}{Die $n \times D$ Datenmatrix in der ursprünglichen (hochdimensionalen) Repräsentation mit Stichprobe $\vect{x}_i$ in Zeile $i$}
\nomenclature[D, 06]{$\mat{Y}$}{Die $n \times d$ Datenmatrix in der latenten (niedrigdimensionalen) Repräsentation}
\nomenclature[D, 07]{$\mathcal{M}$}{Eine Mannigfaltigkeit $\mathcal{M}$}