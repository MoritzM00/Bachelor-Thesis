
\nomenclature[L, 010]{$\vect{a}$}{Ein Vektor}
\nomenclature[L, 020]{$\mat{A}$}{Eine Matrix}
\nomenclature[L, 030]{$\tr{\mat{A}}$}{Die Transponierte einer Matrix}
\nomenclature[L, 040]{$\det(\mat{A})$}{Die Determinante einer Matrix}
\nomenclature[L, 050]{$\diag(\vect{a})$}{Eine Diagonalmatrix mit den Diagonalelementen impliziert durch den Vektor $\vect{a}$}
\nomenclature[L, 060]{$\diag(\mat{A})$}{Ein Vektor mit den Diagonalelementen der Matrix $\mat{A}$}
\nomenclature[L, 061]{$\Spur(\mat{A})$}{Die Spur einer Matrix entspricht der Summe der Diagonalelemente von $\mat{A}$}
\nomenclature[L, 070]{$\norm{\vect{a}}$}{Die euklidische Norm (auch L2-Norm) eines Vektors $\vect{a}$}
\nomenclature[L, 080]{$\norm{\mat{A}}_F$}{Die Frobenius-Norm einer Matrix $\mat{A}$}
\nomenclature[L, 090]{$\identity_n$}{Eine $n \times n$ Identitätsmatrix}

\nomenclature[M, 010]{$\set{A}$}{Eine Menge}
\nomenclature[M, 020]{$\left| \set{A} \right|$}{Die Kardinalität (auch: Mächtigkeit) einer Menge}
\nomenclature[M, 030]{$\set{A} \cap \set{B}$}{Der Durchschnitt zweier Mengen $\set{A}$ und $\set{B}$}
\nomenclature[M, 040]{$\N$}{Die natürlichen Zahlen}
\nomenclature[M, 050]{$\real$}{Die reellen Zahlen}

% =============================
\nomenclature[P, 010]{$\rv{x} \sim P$}{$\rv{x}$ folgt einer $P$-Verteilung}
\nomenclature[P, 020]{$\Exp_{\rv{x} \sim P}[f(x)]$ oder $\Exp{[f(x)]}$}{Erwartungswert von $f(x)$ bezüglich $P$}

\nomenclature[P, 030]{$\Var(f(x))$}{Varianz von $f(x)$}
\nomenclature[P, 040]{$\Cov(f(x), g(x))$}{Kovarianz von $f(x)$ und $g(x)$}

%\nomenclature[P, 08]{$\KLDiv{P}{Q}$}{Kullback-Leibler Divergenz von zwei Verteilungen $P$ und $Q$}
%\nomenclature[P, 09]{$\gaussian(\vect{x} | \vect{\mu}, \mat{\Sigma})$}{Normalverteilung über $\vect{x}$ mit Erwartungswertvektor $\vect{\mu}$ und Kovarianzmatrix $\mat{\Sigma}$}

\nomenclature[A, 009]{$f: \set{A} \rightarrow \set{B}$}{Eine Funktion $f$, die von der Menge $\set{A}$ auf die Menge $\set{B}$ abbildet.}
\nomenclature[A, 010]{$d$}{Die intrinsische Dimension der Daten}
\nomenclature[A, 020]{$D$}{Die extrinsische Dimension der Daten}
\nomenclature[A, 030]{$\mathcal{X}$}{Der Ursprungsraum der Daten, üblicherweise gilt $\mathcal{X} = \real^D$}
\nomenclature[A, 040]{$\vect{x}_1,\ldots,\vect{x}_n \in \mathcal{X}$}{Eine Stichprobe vom Umfang n}
\nomenclature[A, 050]{$\mat{X}$}{Die $n \times D$ Datenmatrix in der ursprünglichen (hochdimensionalen) Repräsentation mit Stichprobe $\vect{x}_i$ in Zeile $i$}
\nomenclature[A, 060]{$\mat{Y}$}{Die $n \times d$ Datenmatrix in der latenten (niedrigdimensionalen) Repräsentation}
\nomenclature[A, 070]{$\mathcal{M}$}{Eine Mannigfaltigkeit $\mathcal{M}$}

\nomenclature[A, 080]{$\Klle$}{Die Nachbarschaftsgröße $K$ für Locally Linear Embedding}
\nomenclature[A, 081]{$\Kid$}{Die Nachbarschaftsgröße $K$ für den Schätzer der intrinsischen Dimension (ID)}
\nomenclature[A, 082]{$\Kqk$}{Die Nachbarschaftsgröße $K$ für die Qualitätskriterien (QK)}

\nomenclature[Z]{ConvAE}{Convolutional Autoencoder}