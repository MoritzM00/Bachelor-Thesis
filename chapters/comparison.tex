%% ==============================
\chapter{Vergleich der Methoden}
\label{ch:Vergleich}
%% ==============================

In \chapref{ch:Dimensionsreduktion} haben wir grundlegende Begriffe geklärt und in
\chapref{ch:MethodenDerDimRed} sechs Methoden der Dimensionsreduktion näher betrachtet. Im jetzigen
Kapitel möchten wir die traditionellen Methoden aus \secref{ch:MethodenDerDimRed:traditionell} mit
den modernen Methoden aus \secref{ch:MethodenDerDimRed:modern} vergleichen. Dazu gehen wir in
\secref{ch:Vergleich:sec:Methodik} auf die Methodik ein, wobei wir uns unter anderem kurz
anschauen, wie man die intrinsische Dimension schätzen kann. In
\secref{ch:Vergleich:sec:VerwendeteDatensaetze} werden die beim Vergleich verwendeten Datensätze
vorgestellt und anschließend werden in \secref{ch:Vergleich:sec:Resultate} die Ergebnisse des
empirischen Vergleichs vorgestellt.

\section{Methodik}
\label{ch:Vergleich:sec:Methodik}

\subsection{Allgemeines Vorgehen}
\label{ch:Vergleich:sec:Methodik:subsec:AllgemeinesVorgehen}

\idea{Wie vergleicht man die unterschiedlichen Methoden?
	Rekonstruktionsfehler alleine ist wenig aussagekräftig

	Generalisierungsfehler eines 1-NN Klassifizieres wie Maaten 2009 }

\idea{Vergleich von PCA mit Autoencoder

	Hier könnte man vor allem untersuchen, inwiefern ein Autoencoder mit linearen Akt.Funktionen PCA
	identisch ist, sowie schrittweise Nichtlinearität hinzunehmen}

\subsection{Schätzen der intrinsischen Dimension}
\label{ch:Vergleich:sec:Methodik:subsec:SchaetzenDerIntrinsischenDim}

\section{Verwendete Datensätze}
\label{ch:Vergleich:sec:VerwendeteDatensaetze}

\section{Resultate}
\label{ch:Vergleich:sec:Resultate}
