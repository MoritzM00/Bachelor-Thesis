%% ==============================
\chapter{Vergleich der Methoden}
\label{ch:Vergleich}
%% ==============================

In \chapref{ch:Dimensionsreduktion} haben wir grundlegende Begriffe geklärt und in
\chapref{ch:MethodenDerDimRed} sechs Methoden der Dimensionsreduktion näher betrachtet. Im jetzigen
Kapitel möchten wir die traditionellen Methoden aus \secref{ch:MethodenDerDimRed:traditionell} mit
den modernen Methoden aus \secref{ch:MethodenDerDimRed:modern} vergleichen. Dazu gehen wir in
\secref{ch:Vergleich:sec:Methodik} auf die Methodik ein, wobei hier unterschiedliche
Qualitätskriterien und Methoden zur Schätzung der intrinsischen Dimension betrachtet werden. In
\secref{ch:Vergleich:sec:VerwendeteDatensaetze} wird auf die im Vergleich verwendeten Datensätzen
eingegangen und anschließend werden in \secref{ch:Vergleich:sec:Resultate} die Ergebnisse des
empirischen Vergleichs vorgestellt.

\section{Methodik}
\label{ch:Vergleich:sec:Methodik}

In diesem Abschnitt wird auf die Methodik des Vergleichs der Dimensionsreduktionsmethoden
eingegangen. Dazu werden in \subsecref{ch:Vergleich:sec:Methodik:subsec:Qualitaetskriterien} die
hier verwendeten Qualitätskriterien einer Dimensionsreduktion genauer betrachtet. Des Weiteren wird
in \subsecref{ch:Vergleich:sec:Methodik:subsec:SchaetzenDerIntrinsischenDim} kurz die
nicht-triviale Schätzung der intrinsischen Dimension behandelt.\todo{hier noch kurz auf das
	allgemeine Setup eingehen und PCA-AE Vergleich erwähnen} \idea{Vergleich von PCA mit Autoencoder

	Hier könnte man vor allem untersuchen, inwiefern ein Autoencoder mit linearen Akt.Funktionen PCA
	identisch ist, sowie schrittweise Nichtlinearität hinzunehmen }
\subsection{Qualitätskriterien der Dimensionsreduktion}
\label{ch:Vergleich:sec:Methodik:subsec:Qualitaetskriterien}
Trotz des immensen Forschungsinteresses für Methoden der Dimensionsreduktion, so sind Qualitätskriterien, die die Güte einer Dimensionsreduktion beschreiben, vergleichsweise wenig erforscht. Deshalb gibt es keine eindeutige Kennzahl, die bei einem Vergleich von Dimensionsreduktionsmethoden eingesetzt wird.\todo{hier noch auf den naheliegenden Rekonstruktionsfehler eingehen + erklären wieso er nicht sehr aussagekräftig ist}

Im ausführlichen Benchmark von \textcite{vanderMaaten.2009} wird auf den Generalisierungsfehler
eines 1-Nächste-Nachbar Klassifikators, sowie auf die zwei Kennzahlen \newterm{Trustworthiness} und
\newterm{Continuity} gesetzt. Daneben gibt es noch viele weitere Kennzahlen, die auf der
sogenannten \newterm{Co-Ranking Matrix} basieren \parencite{Lee.2009}.

\ldots

\subsubsection{Die Co-Ranking Matrix}
\subsubsection{Trustworthiness und Continuity}

\idea{
	Mögliche Gütemaße sind
	\begin{enumerate}
		\item Generalisierungsfehler eines 1-NN Klassifikators wie \textcite{vanderMaaten.2009}
		\item Trustworthiness und Continuity
		\item Co-Ranking Matrix + viele darauf basierende Maße \parencite{Lee.2009}
		\item Correlation Dimension
	\end{enumerate}
}

\subsection{Schätzen der intrinsischen Dimension}
\label{ch:Vergleich:sec:Methodik:subsec:SchaetzenDerIntrinsischenDim}

\idea{sehr mathematisch im Detail, Maaten.2009 benutzt den MLE-Schätzer
	\parencite{Levina.2004}. Dazu tendiere ich momentan auch }

\section{Verwendete Datensätze}
\label{ch:Vergleich:sec:VerwendeteDatensaetze}
\idea{Es gibt viele Standard-Benchmark Datensätze die in anderen Arbeiten verwendet werden. Vlt ist es sinnvoll, 1-2 davon auch hier reinzunehmen für die Vergleichbarkeit, und dann 1-2 neue Datensätze,

	insgesamt vlt 2 künstliche Datensätze und 4 real-world Datensätze }
\section{Resultate}
\label{ch:Vergleich:sec:Resultate}
