%% ==============================
\chapter{Vergleich der Methoden}
\label{ch:Vergleich}
%% ==============================

In \chapref{ch:Dimensionsreduktion} wurden grundlegende Begriffe geklärt und in
\chapref{ch:MethodenDerDimRed} fünf Methoden der Dimensionsreduktion näher betrachtet. Im jetzigen
Kapitel werden die statistischen Methoden aus \secref{ch:MethodenDerDimRed:statistisch} mit den
Machine Learning Methoden aus \secref{ch:MethodenDerDimRed:modern} empirisch auf künstlichen und
natürlichen Datensätzen verglichen. Dazu wird in \secref{ch:Vergleich:sec:Methodik} auf die
Methodik des Vergleichs eingegangen, indem die Schätzung der intrinsischen Dimension und die
eingesetzten Qualitätskriterien erläutert werden. Außerdem werden in
\secref{ch:Vergleich:sec:VerwendeteDatensaetze} die verwendeten Datensätze und in
\secref{ch:Vergleich:sec:ParameterwahlTrainingsdetails} die Parameterwahl der Methoden und
insbesondere die Architektur der Autoencoder vorgestellt. Letztlich werden in
\secref{ch:Vergleich:sec:Resultate} die Resultate des empirischen Vergleichs diskutiert.

\section{Methodik}
\label{ch:Vergleich:sec:Methodik}

Um die statistischen Methoden mit den Machine Learning Ansätzen zu vergleichen, wird die
Performance der Methoden auf mehreren künstlichen sowie natürlichen Datensätzen mit der
Vertrauenswürdigkeit, der Kontinuität und dem $\lcmc$-Kriterium gemessen. Diese Kriterien werden
jeweils für mehrere Nachbarschaftsgrößen berechnet, um eine größere Aussagekraft über die Güte der
gefundenen niedrigdimensionalen Repräsentation zu erreichen. Die Qualitätskriterien werden in
\subsecref{ch:Vergleich:sec:Methodik:subsec:Qualitaetskriterien} eingehend erläutert. Für den
Vergleich werden zwei künstliche und vier natürliche Datensätze verwendet, die in
\secref{ch:Vergleich:sec:VerwendeteDatensaetze} genauer vorgestellt werden. Insgesamt soll damit
ein Überblick über die Stärken und Schwächen der statistischen und der Machine Learning Methoden
geschafft werden. Wie bereits erwähnt, besteht zwischen Autoencodern und der
Hauptkomponentenanalyse eine enge Verbindung. Neben dem Vergleich der zwei Gruppen wird daher in
\secref{ch:Vergleich:sec:Resultate:PCA_AE} der Zusammenhang zwischen der Hauptkomponentenanalyse
und Autoencodern genauer untersucht.

\subsection{Qualitätskriterien der Dimensionsreduktion}
\label{ch:Vergleich:sec:Methodik:subsec:Qualitaetskriterien}
Wie bereits in \chapref{ch:Dimensionsreduktion} erläutert, hat die Dimensionsreduktion das Ziel einer möglichst \enquote{verlustfreien} Transformation der ursprünglichen Repräsentation in eine latente Repräsentation von geringerer Dimension. Für Autoencoder bedeutet \enquote{verlustfrei} die Minimierung des Rekonstruktionsfehlers wie z.B. die quadratische Abweichung (siehe \eqref{eq:MSE_loss}). Dieser Rekonstruktionsfehler ist jedoch wie \textcite[18]{vanderMaaten.2009} hervorhebt, nicht sehr aussagekräftig für die Güte einer Dimensionsreduktion. Hinzu kommt, dass der Rekonstruktionsfehler nicht für alle Methoden berechnet werden kann, da eine inverse Transformation der niedrigdimensionalen in die ursprüngliche Repräsentation benötigt wird. Damit fällt der Rekonstruktionsfehler als geeignetes Qualitätskriterium heraus. Das immense Forschungsinteresse für Methoden der Dimensionsreduktion hat daher mit der Zeit dafür gesorgt, dass immer mehr Qualitätskriterien entwickelt wurden. \textcite{Gracia.2014} stellen einige Qualitätskriterien vor und vergleichen diese miteinander. Trotzdem gibt es in der Literatur keine eindeutige Kennzahl, die bei einem Vergleich von Dimensionsreduktionsmethoden standardmäßig eingesetzt wird \parencite[vgl.][1 -- 2]{Lee.2009}. Stattdessen bedient man sich mehrerer Kennzahlen, die
unterschiedliche Dinge bestrafen und versucht so die Stärken und Schwächen einer
Dimensionsreduktionsmethode zu erkennen \parencite[486]{Venna.2001}. Die im Folgenden vorgestellten Qualitätskriterien versuchen die Güte
mithilfe von Rängen zu quantifizieren.

Im ausführlichen Benchmark von \textcite{vanderMaaten.2009} wird auf den Generalisierungsfehler
eines 1-Nächste-Nachbar Klassifikators, sowie auf die zwei Kennzahlen
\newterm{Vertrauenswürdigkeit} (engl. \textit{Trustworthiness}) und \newterm{Kontinuität} (engl.
\textit{Continuity}) \parencites{Venna.2001}{Venna.2006} gesetzt. Die Vertrauenswürdigkeit und die Kontinuität sind
rangbasierte Qualitätskriterien und werden auch in diesem Vergleich eingesetzt. Daneben gibt es
noch viele weitere rangbasierte Qualitätskriterien, welche einheitlich durch die sogenannte
\newterm{Co-Ranking Matrix} \parencite[1432]{Lee.2009} ausgedrückt werden können. Ebenso können die Vertrauenswürdigkeit und
Kontinuität über die Co-Ranking Matrix berechnet werden \parencite[1433]{Lee.2009}. Für eine ausführliche Behandlung dessen wird auf \textcite{Lee.2009}
verwiesen. In dieser Arbeit werden drei Kriterien verwendet: (1) Die Vertrauenswürdigkeit und (2)
die Kontinuität einer Dimensionsreduktion, sowie (3) das \newterm{Local Continuity Meta-Criterion}
(LCMC). Diese werden im Folgenden genauer betrachtet. \nomenclature[Z]{LCMC}{Local Continuity
	Meta-Criterion}.
\subsubsection{Vertrauenswürdigkeit und Kontinuität}
\label{ch:Vergleich:sec:Methodik:subsec:Qualitaetskriterien:TC}
Diese beiden Kennzahlen basieren auf der Idee des Erhalts von Nachbarschaften (engl.
\textit{neighborhood preservation}) einer Dimensionsreduktion. Sie bilden also ab, wie gut die
lokale Struktur erhalten wird. Eine $K$-Nachbarschaft $\set{M}_i(K)$ eines Punktes $\vect{x}_i$ ist
definiert als die Menge der Indizes der $K$-nächsten Punkte zu $\vect{x}_i$ ($i = 1, \ldots, n$).
Analog kann die $K$-Nachbarschaft $\widetilde{\set{M}}_i(K)$ des dazugehörigen niedrigdimensionalen
Punktes $\vect{y}_i$ definiert werden. Diese Nachbarschaft wird in einer Dimensionsreduktion
erhalten, wenn $\set{M}_i(K) = \widetilde{\set{M}}_i(K)$, das heißt die Nachbarschaften bleiben von
der Dimensionsreduktion unverändert.

Zum einen kann es nun passieren, dass Punkte, die \textit{vor} der Projektion weit weg voneinander
lagen, \textit{nach} der Projektion aber nah beieinander sind. Mit anderen Worten können Punkte,
die eigentlich unterschiedlich sind, nun ähnlich erscheinen. Aus diesem Grund sagt man, dass die
Vertrauenswürdigkeit der Dimensionsreduktion niedrig ist. Zum anderen ist der gegenteilige Fall
möglich. Nah beieinander liegende Punkte sind nach der Projektion weit weg voneinander. Dies
reduziert die Kontinuität einer Dimensionsreduktion \parencite[486 -- 487]{Venna.2001}.

Formal definiert man zusätzlich zu den Nachbarschaftsmengen von oben die beiden Mengen
$\set{U}_i(K)$ und $\set{V}_i(K)$ wie folgt:
\begin{gather}
	\set{U}_i(K) =  \left\{ j \in \N \mid j \notin \set{M}_i(K) \land j \in \widetilde{\set{M}}_i(K) \right\} \, , \\
	\set{V}_i(K) =  \left\{ j \in \N \mid j \in \set{M}_i(K) \land j \notin \widetilde{\set{M}}_i(K) \right\} \, .
\end{gather}
Diese beiden Mengen bilden lediglich die zwei intuitiv besprochenen Fälle im vorherigen Absatz mathematisch ab. Hierbei entspricht $\set{U}_i(K)$ dem ersten und $\set{V}_i(K)$ dem zweiten Fall.
Damit kann die Vertrauenswürdigkeit $T(K)$ als
\begin{equation}
	T(K) = 1 - \frac{2}{nK(2n - 3K - 1)} \sum_{i = 1}^{n}\sum_{j \in \set{U}_i(K) } \left( r­_{\vect{y}}(i, j) - K \right)
\end{equation}
definiert werden \parencite[487]{Venna.2001}, wobei $r_{\vect{y}}(i, j)$ den Rang von des niedrigdimensionalen Vektors
$\vect{y}_j$ bezeichnet, wenn die Datenpunkte absteigend nach der euklidischen Distanz von
$\vect{y}_i$ geordnet sind. Der Term vor der Summation skaliert das Qualitätskriterium so, dass $0
	\leq T(K) \leq 1$ gilt.\footnote{Dies gilt nur für den Fall, dass $K < n/2$ gilt.} Ein Wert von
$T(K) = 1­$ spricht für eine hohe Vertrauenswürdigkeit.

Analog wird die Kontinuität $C(K)$ über $\set{V}_i(K)$ wie folgt definiert
\begin{equation}
	C(K) = 1 - \frac{2}{nK(2n - 3K - 1)} \sum_{i = 1}^{n}\sum_{j \in \set{V}_i(K) } \left( r_{\vect{x}}(i, j) - K \right) \, ,
\end{equation}
wobei $r_{\vect{x}}(i, j)$ nun den Rang zwischen den Datenpunkten in der hochdimensionalen Repräsentation bezeichnet \parencite[487]{Venna.2001}. Auch hier gilt $0 \leq C(K) \leq 1$ und höher ist besser. Die Kontinuität
misst also, wie gut die ursprünglichen Nachbarschaften erhalten werden.

Üblicherweise werden die beiden Kriterien für mehrere Werte der Nachbarschaftsgröße $K$ berechnet und in einer Abbildung dargestellt. So entfällt die etwas willkürliche Wahl einer spezifischen Nachbarschaftsgröße und ermöglicht die Betrachtung von sowohl sehr kleinen als auch größeren Nachbarschaften.
% \subsubsection{Die Co-Ranking Matrix}
% Ein Eintrag $q_{kl}$ der Co-Ranking Matrix $\mat{Q}$ ist die Anzahl der Paare von Datenpunkten $(i,
% 	j)$, die den Rang $r_{\vect{x}}(i, j) = k$ in der hochdimensionalen und den Rang $r_{\vect{y}}(i,
% 	j) = l$ in der niedrigdimensionalen Repräsentation haben.

\subsubsection{Local Continuity Meta-Criterion}
\label{ch:Vergleich:sec:Methodik:subsec:Qualitaetskriterien:LCMC}
Das Local Continuity Meta-Criterion (LCMC), entwickelt von \textcite{Chen.2009}, ist ebenso wie die Vertrauenswürdigkeit und Kontinuität ein rang-basiertes Qualitätskriterium. LCMC betrachtet die Überschneidung der $K$-Nachbarschaften in der ursprünglichen und latenten Repräsentation. Nimmt man wieder die $K$-Nachbarschaften aös $M_i(K)$ und $\widetilde{M}_i(K)$ an, das heißt die Indexmengen der $K$-Nachbarschaften in der ursprünglichen beziehungsweise latenten Repräsentation, dann lässt sich das LCMC wie folgt berechnen \parencite[212]{Chen.2009}:
\begin{equation}
	\label{eq:LCMC}
	\text{LCMC}(K) = \frac{1}{nK} \sum_{i=1}^{n} \left( \left| \set{M}_i(K) \cap \widetilde{\set{M}}_i(K) \right| \right) - \frac{K}{n - 1} \,,
\end{equation}
wobei $­|\cdot|$ die Kardinalität einer Menge und $\set{A} \cap \set{B}$ den Durchschnitt zweier Mengen $\set{A}$ und $\set{B}$ bezeichnet. Der Term vor der Summation skaliert auch dieses Kriterium auf den Wertebereich zwischen Null und Eins, wobei LCMC$(K) = 1$ den besten Wert darstellt. Die durschnittliche Überschneidung der beiden Indexmengen wird wie in \eqref{eq:LCMC} ersichtlich noch um einen additiven Term korrigiert. Dieser Term entspricht dem Erwartungswert einer zufälligen Überschneidung und kann durch eine hypergeometrische Verteilung mit $K$ Defekten aus $n - 1$ modelliert werden \parencite[213]{Chen.2009}. Diese sogenannte Basislinie (engl. \textit{Baseline}) wird mit steigendem
$K$ größer, so dass das Kriterium für den maximalen Wert der Nachbarschftsgröße $K = n - 1$ den
Wert Null annimmt. Aus diesem Grund besitzt das LCMC ein wohldefiniertes Maximum. Allerdings sind
kleinere Werte von $K$ oftmals wichtiger, um beispielsweise die Erhaltung der Struktur einer
Mannigfaltigkeit zu messen.

\subsection{Schätzen der intrinsischen Dimension}
\label{ch:Vergleich:sec:Methodik:subsec:SchaetzenDerIntrinsischenDim}

Bis jetzt wurde immer angenommen, dass die intrinsische Dimension $d$ der Daten bekannt ist, da die
meisten Dimensionsreduktionsmethoden die intrinsische Dimension nicht selbst berechnen. Das
Schätzen der intrinsischen Dimension ist also ein nicht unwichtiges Teilproblem der
Dimensionsreduktion, da eine Unterschätzung von $d$ dazu führt, dass relevante Strukturen
zwangsweise verloren gehen \parencite[1]{Levina.2004}. Erschwert wird dieses Problem durch die Tatsache, dass es sehr viele
Definitionen der intrinsischen Dimension und damit auch sehr viele unterschiedliche Schätzer gibt.
Im Folgenden soll ein kurzer Überblick verschafft werden, jedoch geht eine detaillierte Behandlung
der Schätzung über den Rahmen dieser Arbeit hinaus. Daher wird für einen Überblick und Vergleich
dieser Schätzer auf \textcites{Campadelli.2015}{Bac.2021}{Verveer.1995} verwiesen.

In \secref{ch:Dimensionsreduktion:MannigfaltigkeitenIntrinsDim} haben wir die \textit{topologische
	Dimension} kennengelernt, welche sich in der Literatur zur Strukturerkennung durchgesetzt hat \parencite[1]{Campadelli.2015}. Allerdings bringt diese Definition praktische Schwierigkeiten mit sich,
weswegen die meisten Schätzer der intrinsischen Dimension auf der damit verwandten
\textit{fraktalen Dimension} wie zum Beispiel der Schätzer der Korrelationsdimension \parencite{Camastra.2002} basieren. Daneben gibt es noch Nächste-Nachbar-basierte Schätzer \parencite[1]{Campadelli.2015}. Zu dieser Kategorie gehört auch der weitverbreitete und in dieser
Arbeit verwendete \newterm{Maximum Likelihood Schätzer} von \textcite{Levina.2004}. Diese Schätzer
betrachten die Verteilung der Nachbarschaft eines Punktes $\rvect{x}$ als Funktion der
intrinsischen Dimension -- üblicherweise innerhalb einer kleinen Kugel um $\rvect{x}$
\parencite[8]{Campadelli.2015}. Der Maximum Likelihood Schätzer nimmt an, dass die Beobachtungen, die
in einer solchen Kugel liegen, einem homogenen Poisson-(Zähl-)Prozess folgen
\parencite[2]{Levina.2004}. Dieser Prozess hängt von $d$ ab, weshalb mittels der Maximum Likelihood
Methode ein Schätzwert $\estNormal{d}$ für einen fixen Punkt $\rvect{x}_i$ als
\begin{equation}
	\estNormal{d}_K(\vect{x}_i) = \left( \frac{1}{K - 1} \sum_{j=1}^{K - 1} \log \frac{T_K(\vect{x}_i)}{T_j(\vect{x}_i)} \right)^{-1}
\end{equation}
berechnet werden kann \parencite[4]{Levina.2004}. Hierbei ist $K$ die Anzahl der nächste Nachbarn und $T_{K}(\vect{x}_i)$ die
euklidische Distanz von $\vect{x}_i$ zu seinem $K$-ten Nachbar. Dies ist jedoch nur eine lokale
Schätzung für einen fixen Punkt $\vect{x}_i$. Um eine globale Schätzung zu erhalten, wird der
Mittelwert über alle Beobachtungen für mehrere Werte von $K$ gebildet.
\section{Verwendete Datensätze}
\label{ch:Vergleich:sec:VerwendeteDatensaetze}
Es werden sowohl künstliche als auch natürliche Datensätze eingesetzt, um Eigenschaften der
verschiedenen Methoden miteinander zu vergleichen. Eine Übersicht über die Dimensionen und Stichprobengrößen der verwendeten Datensätze befindet sich in Tabelle \ref{tab:uebersicht-datensaetze}.

\subsection{Künstliche Datensätze}
\label{ch:Vergleich:sec:VerwendeteDatensaetze:kuenstlich}
Zu den künstlichen Datensätzen gehört die weitverbreitete \textit{Swiss Roll} und der \textit{Twin Peaks} Datensatz,
\begin{figure}[ht]
	\begin{center}
		\includegraphics{artificial_datasets.pdf}
	\end{center}
	\caption[Künstliche Datensätze]{\figleft Die Swiss Roll. Die intrinsische Dimension beträgt zwei, weil die Daten auf einer \enquote{eingerollten Ebene} liegen. Die Dimensionsreduktionsmethoden müssen die Swiss Roll \enquote{entfalten}, um eine zweidimensionale Repräsentation zu erhalten. \figright Der Twin Peaks Datensatz. Dieser besteht aus je zwei spitzen Bergen, die nach oben und unten zeigen. Auch dieser Datensatz hat eine intrinsische Dimension von zwei. Intuitiv gesehen kann man sich lokal nur nach rechts oder links bewegen, d.h. die Twin Peaks homöomorph zum $\real^2$ sind.}
	\label{fig:ArtificialDatasets}
\end{figure}
die in \figref{fig:ArtificialDatasets} dargestellt sind.
Beide künstlichen Datensätze bestehen aus 5000 Datenpunkten und haben eine extrinsische Dimension von drei und eine intrinsische
Dimension von zwei. Dies erlaubt eine visuelle Betrachtung der Datensätze und der gefundenen
latenten Räume, schränkt aber gleichzeitig die Architektur eines (unterbestimmten) Autoencoders
sehr ein. Künstliche Datensätze sind aber nicht zwangsweise aussagekräftig für die Performance auf echten Datensätzen, weswegen zusätzlich fünf natürliche Datensätze hinzugezogen werden.

\subsection{Natürliche Datensätze}
\label{ch:Vergleich:sec:VerwendeteDatensaetze:natuerlich}
Natürliche
Datensätze weisen laut empirischen Ergebnissen \addref oft komplexe nichtlineare Zusammenhänge
auf und sind daher für die Dimensionsreduktion anspruchsvoller als kleine, künstlich generierte
Datensätze. Der Nachteil besteht darin, dass die intrinsische Dimension in der Regel deutlich über
zwei liegt und daher der latente Raum nicht visualisiert werden kann. Nichtsdestotrotz liefern die
Qualitätskriterien hinreichende gute Indizien für die Performance der Methoden. Bei den natürlichen
Datensätzen wurden größtenteils Bilddatensätze, aber auch ein Datensatz mit Genausprägungen ausgewählt, da diese eine
sehr hohe extrinsische Dimension aufweisen und daher für die Dimensionsreduktion gut geeignet sind.
Konkret wurde (1) der \textit{MNIST}-Datensatz \parencite{LeCun.2010}, (2) der \textit{Olivetti Faces}-Datensatz
\footnote{\url{https://cam-orl.co.uk/facedatabase.html}}, (3) der \textit{Labeled Faces in the
	Wild} (LFW) Datensatz \parencite{GaryB.Huang.2007}, (4) der \textit{Facial Emotion Recognition} (FER) Datensatz \parencite{DumitruIanGoodfellowWillCukierskiYoshuaBengio.2013} und (5) der
\textit{ICMR}-Datensatz\footnote{\url{https://www.kaggle.com/datasets/shibumohapatra/icmr-data?select=labels.csv}}
ausgewählt, um die Performance der Dimensionsreduktionsmethoden auf natürlichen Datensätzen zu
evaluieren. Der weitverbreitete MNIST-Datensatz besteht aus 60 000 Grauton-Bildern von
handgeschriebenen Zahlen in der Auflösung $28 \times 28$. Die Anzahl der Pixel und damit die
extrinsische Dimension beträgt 784. Der \textit{Olivetti Faces}-Datensatz enthält Bilder von
Gesichtern aus zehn unterschiedlichen Positionen von 40 Personen und ist damit der kleinste
natürliche Datensatz in diesem Vergleich mit 400 Bildern. Die Bilder haben eine Auflösung von $64
	\times 64$, was einer extrinsischen Dimension von 4096 entspricht. Der LFW-Datensatz enthält
insgesamt über 13 000 Bilder von Gesichtern, jedoch wurde hier eine Teilmenge ausgewählt, sodass
jede Person mindestens 30 mal vorkommt. Dies resultiert in einer Stichprobengröße von 2370. Die
Bilder haben eine Auflösung von $62 \times 47$ und damit eine extrinsische Dimension von 2914. Der
FER-Datensatz besteht aus 28 709 Bildern von Gesichtern mit sieben unterschiedlichen Emotionen. Die
Bilder haben eine Auflösung von $48 \times 48$ und sind damit 2304-dimensional. Beispiele der
Bild-Datensätze sind in \figref{fig:Dataset_samples} zu finden.
\begin{figure}
	\begin{center}
		%% Creator: Matplotlib, PGF backend
%%
%% To include the figure in your LaTeX document, write
%%   \input{<filename>.pgf}
%%
%% Make sure the required packages are loaded in your preamble
%%   \usepackage{pgf}
%%
%% Also ensure that all the required font packages are loaded; for instance,
%% the lmodern package is sometimes necessary when using math font.
%%   \usepackage{lmodern}
%%
%% Figures using additional raster images can only be included by \input if
%% they are in the same directory as the main LaTeX file. For loading figures
%% from other directories you can use the `import` package
%%   \usepackage{import}
%%
%% and then include the figures with
%%   \import{<path to file>}{<filename>.pgf}
%%
%% Matplotlib used the following preamble
%%   
%%   \usepackage{fontspec}
%%   \setmainfont{DejaVuSerif.ttf}[Path=\detokenize{/Users/moritzmistol/.pyenv/versions/3.9.13/envs/thesis/lib/python3.9/site-packages/matplotlib/mpl-data/fonts/ttf/}]
%%   \setsansfont{DejaVuSans.ttf}[Path=\detokenize{/Users/moritzmistol/.pyenv/versions/3.9.13/envs/thesis/lib/python3.9/site-packages/matplotlib/mpl-data/fonts/ttf/}]
%%   \setmonofont{DejaVuSansMono.ttf}[Path=\detokenize{/Users/moritzmistol/.pyenv/versions/3.9.13/envs/thesis/lib/python3.9/site-packages/matplotlib/mpl-data/fonts/ttf/}]
%%   \makeatletter\@ifpackageloaded{underscore}{}{\usepackage[strings]{underscore}}\makeatother
%%
\begingroup%
\makeatletter%
\begin{pgfpicture}%
\pgfpathrectangle{\pgfpointorigin}{\pgfqpoint{5.474492in}{5.811320in}}%
\pgfusepath{use as bounding box, clip}%
\begin{pgfscope}%
\pgfsetbuttcap%
\pgfsetmiterjoin%
\definecolor{currentfill}{rgb}{1.000000,1.000000,1.000000}%
\pgfsetfillcolor{currentfill}%
\pgfsetlinewidth{0.000000pt}%
\definecolor{currentstroke}{rgb}{1.000000,1.000000,1.000000}%
\pgfsetstrokecolor{currentstroke}%
\pgfsetdash{}{0pt}%
\pgfpathmoveto{\pgfqpoint{0.000000in}{0.000000in}}%
\pgfpathlineto{\pgfqpoint{5.474492in}{0.000000in}}%
\pgfpathlineto{\pgfqpoint{5.474492in}{5.811320in}}%
\pgfpathlineto{\pgfqpoint{0.000000in}{5.811320in}}%
\pgfpathlineto{\pgfqpoint{0.000000in}{0.000000in}}%
\pgfpathclose%
\pgfusepath{fill}%
\end{pgfscope}%
\begin{pgfscope}%
\pgfpathrectangle{\pgfqpoint{0.050000in}{2.975000in}}{\pgfqpoint{2.576359in}{2.576359in}}%
\pgfusepath{clip}%
\pgfsys@transformshift{0.050000in}{2.975000in}%
\pgftext[left,bottom]{\includegraphics[interpolate=true,width=2.580000in,height=2.580000in]{dataset_samples-img0.png}}%
\end{pgfscope}%
\begin{pgfscope}%
\definecolor{textcolor}{rgb}{0.000000,0.000000,0.000000}%
\pgfsetstrokecolor{textcolor}%
\pgfsetfillcolor{textcolor}%
\pgftext[x=1.338179in,y=5.634692in,,base]{\color{textcolor}\rmfamily\fontsize{12.000000}{14.400000}\selectfont MNIST Zahlen}%
\end{pgfscope}%
\begin{pgfscope}%
\pgfpathrectangle{\pgfqpoint{2.848134in}{2.975000in}}{\pgfqpoint{2.576359in}{2.576359in}}%
\pgfusepath{clip}%
\pgfsys@transformshift{2.848134in}{2.975000in}%
\pgftext[left,bottom]{\includegraphics[interpolate=true,width=2.580000in,height=2.580000in]{dataset_samples-img1.png}}%
\end{pgfscope}%
\begin{pgfscope}%
\definecolor{textcolor}{rgb}{0.000000,0.000000,0.000000}%
\pgfsetstrokecolor{textcolor}%
\pgfsetfillcolor{textcolor}%
\pgftext[x=4.136313in,y=5.634692in,,base]{\color{textcolor}\rmfamily\fontsize{12.000000}{14.400000}\selectfont Facial Emotion Recognition}%
\end{pgfscope}%
\begin{pgfscope}%
\pgfpathrectangle{\pgfqpoint{0.361656in}{0.050000in}}{\pgfqpoint{1.953046in}{2.576359in}}%
\pgfusepath{clip}%
\pgfsys@transformshift{0.361656in}{0.050000in}%
\pgftext[left,bottom]{\includegraphics[interpolate=true,width=1.960000in,height=2.580000in]{dataset_samples-img2.png}}%
\end{pgfscope}%
\begin{pgfscope}%
\definecolor{textcolor}{rgb}{0.000000,0.000000,0.000000}%
\pgfsetstrokecolor{textcolor}%
\pgfsetfillcolor{textcolor}%
\pgftext[x=1.338179in,y=2.709692in,,base]{\color{textcolor}\rmfamily\fontsize{12.000000}{14.400000}\selectfont Labeled Faces in the Wild}%
\end{pgfscope}%
\begin{pgfscope}%
\pgfpathrectangle{\pgfqpoint{2.848134in}{0.050000in}}{\pgfqpoint{2.576359in}{2.576359in}}%
\pgfusepath{clip}%
\pgfsys@transformshift{2.848134in}{0.050000in}%
\pgftext[left,bottom]{\includegraphics[interpolate=true,width=2.580000in,height=2.580000in]{dataset_samples-img3.png}}%
\end{pgfscope}%
\begin{pgfscope}%
\definecolor{textcolor}{rgb}{0.000000,0.000000,0.000000}%
\pgfsetstrokecolor{textcolor}%
\pgfsetfillcolor{textcolor}%
\pgftext[x=4.136313in,y=2.709692in,,base]{\color{textcolor}\rmfamily\fontsize{12.000000}{14.400000}\selectfont Olivetti Faces}%
\end{pgfscope}%
\end{pgfpicture}%
\makeatother%
\endgroup%

	\end{center}
	\caption[Beispielbilder der natürlichen Datensätze]{Zu sehen sind je vier Beispielbilder der natürlichen Bilddatensätze aus dem Vergleich. \captiona Handgeschriebene Zahlen aus dem MNIST-Datensatz \captionb Beispielbilder aus dem FER-Datensatz \captionc Beispielbilder aus dem LFW-Datensatz \captiond Der Olivetti Faces Datensatz }
	\label{fig:Dataset_samples}
\end{figure}

Letztlich enthält der ICMR-Datensatz Genausprägungen von 801 Personen, die mit fünf
unterschiedlichen Arten von Krebs diagnostiziert wurden. Der Datensatz enthält Genausprägungen von
über 20 000 Genen, womit dieser Datensatz die höchste extrinsische Dimension im Vergleich besitzt.
In Tabelle \ref{tab:uebersicht-datensaetze} sind die extrinsischen und intrinsischen Dimensionen,
sowie die Stichprobengrößen der Datensätze zusammengefasst. Die intrinsischen Dimensionen sind die
Schätzungen des in \subsecref{ch:Vergleich:sec:Methodik:subsec:SchaetzenDerIntrinsischenDim}
erläuterten Maximum Likelihood Schätzers für den jeweiligen Datensatz.

\begin{table}[]
	\centering
	\begin{tabular}{@{}lrrr@{}}
		\toprule
		Datensatz      & extrinsische Dimension $D$ & intrinsische Dimension $d$ & Stichprobengröße $n$ \\ \midrule
		Swiss Roll     & 3                          & 2                          & 5 000                \\
		Twin Peaks     & 3                          & 2                          & 5 000                \\
		MNIST          & 784                        & 15                         & 60 000               \\
		Olivetti Faces & 4 096                      & 9                          & 400                  \\
		LFW            & 2 914                      & 21                         & 2370                 \\
		FER            & 2 304                      & 26                         & 28 709               \\
		ICMR           & 20 531                     & 22                         & 801                  \\
		\bottomrule
	\end{tabular}
	\caption[Übersicht über die extrinsischen und intrinsischen Dimensionen, sowie die Stichprobengröße der in diesem Vergleich verwendeten Datensätze]{Übersicht über die extrinsischen und intrinsischen Dimensionen, sowie die Stichprobengröße der in diesem Vergleich verwendeten Datensätze. Bei Bilddatensätzen entspricht die extrinsische Dimension der Anzahl der Pixel im Bild. Die intrinsische Dimension wurde mit dem Maximum Likelihood Schätzer aus \subsecref{ch:Vergleich:sec:Methodik:subsec:SchaetzenDerIntrinsischenDim} mit einer Nachbarschaftsgröße $K=5$ geschätzt.}
	\label{tab:uebersicht-datensaetze}
\end{table}

\section{Trainingsdetails und Parameterwahl}
\label{ch:Vergleich:sec:ParameterwahlTrainingsdetails}

In diesem Abschnitt werden die gewählten (Hyper-)Parameter und Details des Trainierens der
Dimensionsreduktionsmethoden vorgestellt. Eine Übersicht ist in
Tabelle~\ref{tab:uebersicht-parameter} zu finden. Alle Datensätze wurde vor dem Anwenden der
Dimensionsreduktionsmethoden standardisiert, das heißt $\mat{X}$ weist einen spaltenweisen
Erwartungswert von Null und eine spaltenweise Varianz von Eins auf. Für die Evaluierung der
Methoden wurde aufgrund von Restriktionen der Rechenleistung (insbesondere des Arbeitsspeichers)
ein Downsampling auf maximal 25 000 Stichproben durchgeführt.\footnote{Die Evaluierung ist aufgrund
	der Berechnung der Co-Ranking-Matrix speicherintensiv, da diese maximal $(n-1) \times (n-1)$ groß
	ist und damit quadratisch mit $n$ ansteigt. Um die Qualitätskriterien für den MNIST-Datensatz zu
	berechnen, werden circa 25 GB Arbeitsspeicher benötigt.} Dies ist allerdings nur für den MNIST- und
FER-Datensatz von Relevanz. Alle anderen Datensätze haben weniger Stichproben.

Für die Hauptkomponentenanalyse gibt es neben der Anzahl der zu behaltenden Hauptkomponenten keine
Parameter. Die Anzahl der Hauptkomponenten entspricht der intrinsischen Dimension, welche vom
Datensatz abhängt und für alle Methoden identisch ist. Dies trifft analog auf Kernel PCA und auf
Locally Linear Embedding zu. Für Autoencoder bestimmt die intrinsische Dimension die Größe der
Bottleneck-Schicht. Für Kernel PCA wird ein RBF-Kernel (Gauß-Kernel) eingesetzt, wofür ein
Hyperparameter $\gamma$ gewählt werden kann. Dieser wird Datensatz-spezifisch auf $\gamma =
	\sfrac{1}{D}$ gesetzt. Für Locally Linear Embedding gibt es zwei Parameter: (1) einen Parameter
$K$, der die Größe der verwendeten Nachbarschaft kontrolliert und (2) eine
Regularisierungs-Konstante $\epsilon$, die auf die Diagonale der lokalen Kovarianzmatrix addiert
wird.

Den statistischen Methoden gegenüber haben Autoencoder deutlich mehr Freiheitsgrade, um die
Architektur zu bestimmen. Die wichtigsten Parameter sind die Anzahl der Schichten $m$, sowie die
Anzahl an Neuronen pro Schicht und die Wahl der Aktivierungsfunktionen. Daneben gibt es noch viele
weitere Freiheitsgrade für das Trainieren des Autoencoders, welche teilweise vom Datensatz
abhängen. Diese wurden für alle Autoencoder nahezu identisch gewählt und werden in
Appendix~\ref{ch:Appendix:Architektur-Details} genauer erläutert. Für beide künstlichen Datensätze
und für den Olivetti Face Datensatz wird ein dreischichtiger Autoencoder eingesetzt. Für alle
anderen Datensätze wird ein fünfschichtiger Autoencoder trainiert. Alle vollvernetzten Autoencoder
und der Contractive Autoencoder verwenden Sigmoid-Aktivierungsfunktionen im Encoder und Decoder.
Auf Bilddatensätzen wurden zusätzlich Convolutional Autoencoder trainiert, deren Architekturen an
\textcite[14]{Ghosh.2019} angelehnt und ebenfalls in Appendix~\ref{ch:Appendix:Architektur-Details}
spezifiziert sind.

\begin{table}[ht]
	\tymax=300pt
	\centering
	\begin{tabulary}{\linewidth}{CC}
		\toprule
		Methode                            & Parameter                                                            \\ \midrule
		Hauptkomponentenanalyse (PCA)      & --                                                                   \\ \midrule
		Kernel PCA                         & $\kappa(\vect{x}_1, \vect{x}_2) = \exp(- \norm{\vect{x}_1 - \vect{x}_2}^2/(2\gamma^2))$, $\gamma=\sfrac{1}{D}$ \\ \midrule
		Locally Linear Embedding (LLE)     & $K=10$, $\epsilon=1\mathrm{e}{-3}$                                   \\ \midrule
		Autoencoder (AE)                   & $m \in \{3, 5\}$, Sigmoid-Aktivierungsfunktion \newline (siehe
		Appendix~\ref{ch:Appendix:Architektur-Details})                                                           \\ \midrule Contractive Autoencoder (CAE) & $m = 3$, $\lambda=1\mathrm{e}{-4}$,
		Sigmoid-Aktivierungsfunktion (siehe Appendix~\ref{ch:Appendix:Architektur-Details})                       \\ \midrule
		Convolutional Autoencoder (ConvAE) & $m > 5$, ReLU-Aktivierungsfunktion \newline (siehe
		Appendix~\ref{ch:Appendix:Architektur-Details})                                                           \\ \bottomrule
	\end{tabulary}
	\caption[Übersicht über die verwendeten Parameter der Methoden]{Übersicht über die verwendeten Parameter. Hierbei ist $\kappa$ die Kernel-Funktion, $D$ die extrinsische Dimension des Datensatzes, $K$ die Nachbarschaftsgröße, $\epsilon$ eine Regularisierungskonstante für LLE, $m$ die Anzahl der Schichten im Autoencoder und $\lambda$ eine multiplikative Konstante für den kontrahierenden Fehlerterm des CAE.}
	\label{tab:uebersicht-parameter}
\end{table}
\section{Resultate}
\label{ch:Vergleich:sec:Resultate}

In diesem Abschnitt werden die Resultate des empirischen Vergleichs vorgestellt. Dazu werden die
Werte der verschiedenen Qualitätskriterien auf den künstlichen und natürlichen Datensätzen in
Abhängigkeit der Nachbarschaftsgröße $K$ abgebildet. Die verschiedenen Methoden wurden mit den im
vorhergehenden \secref{ch:Vergleich:sec:ParameterwahlTrainingsdetails} erläuterten Parametern auf
den künstlichen und natürlichen Datensätzen trainiert und hinsichtlich der in
\secref{ch:Vergleich:sec:Methodik:subsec:Qualitaetskriterien} vorgestellten Qualitätskriterien für
eine Nachbarschaftsgröße von $1 \leq K \leq 100$ evaluiert. Einige der Abbildungen für die
Qualitätskriterien sind im Appendix \ref{ch:Appendix:Qualitaetskriterien} zu finden.\unsure{vlt
	eine Tabelle mit Mittelwert von T und C und Maximum von LCMC einfügen für die Übersicht} In
\secref{ch:Vergleich:sec:Resultate:kuenstlich} werden die Ergebnisse auf den künstlichen
Datensätzen und in \secref{ch:Vergleich:sec:Resultate:natuerlich} die Ergebnisse auf den
natürlichen Datensätzen diskutiert. Letztlich wird in \secref{ch:Vergleich:sec:Resultate:PCA_AE}
der Zusammenhang zwischen der Hauptkomponentenanalyse und Autoencodern empirisch untersucht.

\subsection{Resultate auf künstlichen Datensätzen}
\label{ch:Vergleich:sec:Resultate:kuenstlich}

Die Qualitätskriterien für den Swiss Roll Datensatz sind in \figref{fig:SwissRollMetrics}
\begin{figure}[ht]
	\begin{center}
		%% Creator: Matplotlib, PGF backend
%%
%% To include the figure in your LaTeX document, write
%%   \input{<filename>.pgf}
%%
%% Make sure the required packages are loaded in your preamble
%%   \usepackage{pgf}
%%
%% Also ensure that all the required font packages are loaded; for instance,
%% the lmodern package is sometimes necessary when using math font.
%%   \usepackage{lmodern}
%%
%% Figures using additional raster images can only be included by \input if
%% they are in the same directory as the main LaTeX file. For loading figures
%% from other directories you can use the `import` package
%%   \usepackage{import}
%%
%% and then include the figures with
%%   \import{<path to file>}{<filename>.pgf}
%%
%% Matplotlib used the following preamble
%%   
%%   \usepackage{fontspec}
%%   \setmainfont{DejaVuSerif.ttf}[Path=\detokenize{/Users/moritzmistol/.pyenv/versions/3.9.13/envs/thesis/lib/python3.9/site-packages/matplotlib/mpl-data/fonts/ttf/}]
%%   \setsansfont{DejaVuSans.ttf}[Path=\detokenize{/Users/moritzmistol/.pyenv/versions/3.9.13/envs/thesis/lib/python3.9/site-packages/matplotlib/mpl-data/fonts/ttf/}]
%%   \setmonofont{DejaVuSansMono.ttf}[Path=\detokenize{/Users/moritzmistol/.pyenv/versions/3.9.13/envs/thesis/lib/python3.9/site-packages/matplotlib/mpl-data/fonts/ttf/}]
%%   \makeatletter\@ifpackageloaded{underscore}{}{\usepackage[strings]{underscore}}\makeatother
%%
\begingroup%
\makeatletter%
\begin{pgfpicture}%
\pgfpathrectangle{\pgfpointorigin}{\pgfqpoint{5.642256in}{3.835212in}}%
\pgfusepath{use as bounding box, clip}%
\begin{pgfscope}%
\pgfsetbuttcap%
\pgfsetmiterjoin%
\definecolor{currentfill}{rgb}{1.000000,1.000000,1.000000}%
\pgfsetfillcolor{currentfill}%
\pgfsetlinewidth{0.000000pt}%
\definecolor{currentstroke}{rgb}{1.000000,1.000000,1.000000}%
\pgfsetstrokecolor{currentstroke}%
\pgfsetdash{}{0pt}%
\pgfpathmoveto{\pgfqpoint{0.000000in}{0.000000in}}%
\pgfpathlineto{\pgfqpoint{5.642256in}{0.000000in}}%
\pgfpathlineto{\pgfqpoint{5.642256in}{3.835212in}}%
\pgfpathlineto{\pgfqpoint{0.000000in}{3.835212in}}%
\pgfpathlineto{\pgfqpoint{0.000000in}{0.000000in}}%
\pgfpathclose%
\pgfusepath{fill}%
\end{pgfscope}%
\begin{pgfscope}%
\pgfsetbuttcap%
\pgfsetmiterjoin%
\definecolor{currentfill}{rgb}{1.000000,1.000000,1.000000}%
\pgfsetfillcolor{currentfill}%
\pgfsetlinewidth{0.000000pt}%
\definecolor{currentstroke}{rgb}{0.000000,0.000000,0.000000}%
\pgfsetstrokecolor{currentstroke}%
\pgfsetstrokeopacity{0.000000}%
\pgfsetdash{}{0pt}%
\pgfpathmoveto{\pgfqpoint{0.539970in}{2.347992in}}%
\pgfpathlineto{\pgfqpoint{2.746849in}{2.347992in}}%
\pgfpathlineto{\pgfqpoint{2.746849in}{3.574193in}}%
\pgfpathlineto{\pgfqpoint{0.539970in}{3.574193in}}%
\pgfpathlineto{\pgfqpoint{0.539970in}{2.347992in}}%
\pgfpathclose%
\pgfusepath{fill}%
\end{pgfscope}%
\begin{pgfscope}%
\pgfsetbuttcap%
\pgfsetroundjoin%
\definecolor{currentfill}{rgb}{0.000000,0.000000,0.000000}%
\pgfsetfillcolor{currentfill}%
\pgfsetlinewidth{0.501875pt}%
\definecolor{currentstroke}{rgb}{0.000000,0.000000,0.000000}%
\pgfsetstrokecolor{currentstroke}%
\pgfsetdash{}{0pt}%
\pgfsys@defobject{currentmarker}{\pgfqpoint{0.000000in}{0.000000in}}{\pgfqpoint{0.000000in}{0.041667in}}{%
\pgfpathmoveto{\pgfqpoint{0.000000in}{0.000000in}}%
\pgfpathlineto{\pgfqpoint{0.000000in}{0.041667in}}%
\pgfusepath{stroke,fill}%
}%
\begin{pgfscope}%
\pgfsys@transformshift{0.539970in}{2.347992in}%
\pgfsys@useobject{currentmarker}{}%
\end{pgfscope}%
\end{pgfscope}%
\begin{pgfscope}%
\pgfsetbuttcap%
\pgfsetroundjoin%
\definecolor{currentfill}{rgb}{0.000000,0.000000,0.000000}%
\pgfsetfillcolor{currentfill}%
\pgfsetlinewidth{0.501875pt}%
\definecolor{currentstroke}{rgb}{0.000000,0.000000,0.000000}%
\pgfsetstrokecolor{currentstroke}%
\pgfsetdash{}{0pt}%
\pgfsys@defobject{currentmarker}{\pgfqpoint{0.000000in}{-0.041667in}}{\pgfqpoint{0.000000in}{0.000000in}}{%
\pgfpathmoveto{\pgfqpoint{0.000000in}{0.000000in}}%
\pgfpathlineto{\pgfqpoint{0.000000in}{-0.041667in}}%
\pgfusepath{stroke,fill}%
}%
\begin{pgfscope}%
\pgfsys@transformshift{0.539970in}{3.574193in}%
\pgfsys@useobject{currentmarker}{}%
\end{pgfscope}%
\end{pgfscope}%
\begin{pgfscope}%
\definecolor{textcolor}{rgb}{0.000000,0.000000,0.000000}%
\pgfsetstrokecolor{textcolor}%
\pgfsetfillcolor{textcolor}%
\pgftext[x=0.539970in,y=2.299381in,,top]{\color{textcolor}\rmfamily\fontsize{10.000000}{12.000000}\selectfont \(\displaystyle {0}\)}%
\end{pgfscope}%
\begin{pgfscope}%
\pgfsetbuttcap%
\pgfsetroundjoin%
\definecolor{currentfill}{rgb}{0.000000,0.000000,0.000000}%
\pgfsetfillcolor{currentfill}%
\pgfsetlinewidth{0.501875pt}%
\definecolor{currentstroke}{rgb}{0.000000,0.000000,0.000000}%
\pgfsetstrokecolor{currentstroke}%
\pgfsetdash{}{0pt}%
\pgfsys@defobject{currentmarker}{\pgfqpoint{0.000000in}{0.000000in}}{\pgfqpoint{0.000000in}{0.041667in}}{%
\pgfpathmoveto{\pgfqpoint{0.000000in}{0.000000in}}%
\pgfpathlineto{\pgfqpoint{0.000000in}{0.041667in}}%
\pgfusepath{stroke,fill}%
}%
\begin{pgfscope}%
\pgfsys@transformshift{0.976976in}{2.347992in}%
\pgfsys@useobject{currentmarker}{}%
\end{pgfscope}%
\end{pgfscope}%
\begin{pgfscope}%
\pgfsetbuttcap%
\pgfsetroundjoin%
\definecolor{currentfill}{rgb}{0.000000,0.000000,0.000000}%
\pgfsetfillcolor{currentfill}%
\pgfsetlinewidth{0.501875pt}%
\definecolor{currentstroke}{rgb}{0.000000,0.000000,0.000000}%
\pgfsetstrokecolor{currentstroke}%
\pgfsetdash{}{0pt}%
\pgfsys@defobject{currentmarker}{\pgfqpoint{0.000000in}{-0.041667in}}{\pgfqpoint{0.000000in}{0.000000in}}{%
\pgfpathmoveto{\pgfqpoint{0.000000in}{0.000000in}}%
\pgfpathlineto{\pgfqpoint{0.000000in}{-0.041667in}}%
\pgfusepath{stroke,fill}%
}%
\begin{pgfscope}%
\pgfsys@transformshift{0.976976in}{3.574193in}%
\pgfsys@useobject{currentmarker}{}%
\end{pgfscope}%
\end{pgfscope}%
\begin{pgfscope}%
\definecolor{textcolor}{rgb}{0.000000,0.000000,0.000000}%
\pgfsetstrokecolor{textcolor}%
\pgfsetfillcolor{textcolor}%
\pgftext[x=0.976976in,y=2.299381in,,top]{\color{textcolor}\rmfamily\fontsize{10.000000}{12.000000}\selectfont \(\displaystyle {20}\)}%
\end{pgfscope}%
\begin{pgfscope}%
\pgfsetbuttcap%
\pgfsetroundjoin%
\definecolor{currentfill}{rgb}{0.000000,0.000000,0.000000}%
\pgfsetfillcolor{currentfill}%
\pgfsetlinewidth{0.501875pt}%
\definecolor{currentstroke}{rgb}{0.000000,0.000000,0.000000}%
\pgfsetstrokecolor{currentstroke}%
\pgfsetdash{}{0pt}%
\pgfsys@defobject{currentmarker}{\pgfqpoint{0.000000in}{0.000000in}}{\pgfqpoint{0.000000in}{0.041667in}}{%
\pgfpathmoveto{\pgfqpoint{0.000000in}{0.000000in}}%
\pgfpathlineto{\pgfqpoint{0.000000in}{0.041667in}}%
\pgfusepath{stroke,fill}%
}%
\begin{pgfscope}%
\pgfsys@transformshift{1.413981in}{2.347992in}%
\pgfsys@useobject{currentmarker}{}%
\end{pgfscope}%
\end{pgfscope}%
\begin{pgfscope}%
\pgfsetbuttcap%
\pgfsetroundjoin%
\definecolor{currentfill}{rgb}{0.000000,0.000000,0.000000}%
\pgfsetfillcolor{currentfill}%
\pgfsetlinewidth{0.501875pt}%
\definecolor{currentstroke}{rgb}{0.000000,0.000000,0.000000}%
\pgfsetstrokecolor{currentstroke}%
\pgfsetdash{}{0pt}%
\pgfsys@defobject{currentmarker}{\pgfqpoint{0.000000in}{-0.041667in}}{\pgfqpoint{0.000000in}{0.000000in}}{%
\pgfpathmoveto{\pgfqpoint{0.000000in}{0.000000in}}%
\pgfpathlineto{\pgfqpoint{0.000000in}{-0.041667in}}%
\pgfusepath{stroke,fill}%
}%
\begin{pgfscope}%
\pgfsys@transformshift{1.413981in}{3.574193in}%
\pgfsys@useobject{currentmarker}{}%
\end{pgfscope}%
\end{pgfscope}%
\begin{pgfscope}%
\definecolor{textcolor}{rgb}{0.000000,0.000000,0.000000}%
\pgfsetstrokecolor{textcolor}%
\pgfsetfillcolor{textcolor}%
\pgftext[x=1.413981in,y=2.299381in,,top]{\color{textcolor}\rmfamily\fontsize{10.000000}{12.000000}\selectfont \(\displaystyle {40}\)}%
\end{pgfscope}%
\begin{pgfscope}%
\pgfsetbuttcap%
\pgfsetroundjoin%
\definecolor{currentfill}{rgb}{0.000000,0.000000,0.000000}%
\pgfsetfillcolor{currentfill}%
\pgfsetlinewidth{0.501875pt}%
\definecolor{currentstroke}{rgb}{0.000000,0.000000,0.000000}%
\pgfsetstrokecolor{currentstroke}%
\pgfsetdash{}{0pt}%
\pgfsys@defobject{currentmarker}{\pgfqpoint{0.000000in}{0.000000in}}{\pgfqpoint{0.000000in}{0.041667in}}{%
\pgfpathmoveto{\pgfqpoint{0.000000in}{0.000000in}}%
\pgfpathlineto{\pgfqpoint{0.000000in}{0.041667in}}%
\pgfusepath{stroke,fill}%
}%
\begin{pgfscope}%
\pgfsys@transformshift{1.850987in}{2.347992in}%
\pgfsys@useobject{currentmarker}{}%
\end{pgfscope}%
\end{pgfscope}%
\begin{pgfscope}%
\pgfsetbuttcap%
\pgfsetroundjoin%
\definecolor{currentfill}{rgb}{0.000000,0.000000,0.000000}%
\pgfsetfillcolor{currentfill}%
\pgfsetlinewidth{0.501875pt}%
\definecolor{currentstroke}{rgb}{0.000000,0.000000,0.000000}%
\pgfsetstrokecolor{currentstroke}%
\pgfsetdash{}{0pt}%
\pgfsys@defobject{currentmarker}{\pgfqpoint{0.000000in}{-0.041667in}}{\pgfqpoint{0.000000in}{0.000000in}}{%
\pgfpathmoveto{\pgfqpoint{0.000000in}{0.000000in}}%
\pgfpathlineto{\pgfqpoint{0.000000in}{-0.041667in}}%
\pgfusepath{stroke,fill}%
}%
\begin{pgfscope}%
\pgfsys@transformshift{1.850987in}{3.574193in}%
\pgfsys@useobject{currentmarker}{}%
\end{pgfscope}%
\end{pgfscope}%
\begin{pgfscope}%
\definecolor{textcolor}{rgb}{0.000000,0.000000,0.000000}%
\pgfsetstrokecolor{textcolor}%
\pgfsetfillcolor{textcolor}%
\pgftext[x=1.850987in,y=2.299381in,,top]{\color{textcolor}\rmfamily\fontsize{10.000000}{12.000000}\selectfont \(\displaystyle {60}\)}%
\end{pgfscope}%
\begin{pgfscope}%
\pgfsetbuttcap%
\pgfsetroundjoin%
\definecolor{currentfill}{rgb}{0.000000,0.000000,0.000000}%
\pgfsetfillcolor{currentfill}%
\pgfsetlinewidth{0.501875pt}%
\definecolor{currentstroke}{rgb}{0.000000,0.000000,0.000000}%
\pgfsetstrokecolor{currentstroke}%
\pgfsetdash{}{0pt}%
\pgfsys@defobject{currentmarker}{\pgfqpoint{0.000000in}{0.000000in}}{\pgfqpoint{0.000000in}{0.041667in}}{%
\pgfpathmoveto{\pgfqpoint{0.000000in}{0.000000in}}%
\pgfpathlineto{\pgfqpoint{0.000000in}{0.041667in}}%
\pgfusepath{stroke,fill}%
}%
\begin{pgfscope}%
\pgfsys@transformshift{2.287993in}{2.347992in}%
\pgfsys@useobject{currentmarker}{}%
\end{pgfscope}%
\end{pgfscope}%
\begin{pgfscope}%
\pgfsetbuttcap%
\pgfsetroundjoin%
\definecolor{currentfill}{rgb}{0.000000,0.000000,0.000000}%
\pgfsetfillcolor{currentfill}%
\pgfsetlinewidth{0.501875pt}%
\definecolor{currentstroke}{rgb}{0.000000,0.000000,0.000000}%
\pgfsetstrokecolor{currentstroke}%
\pgfsetdash{}{0pt}%
\pgfsys@defobject{currentmarker}{\pgfqpoint{0.000000in}{-0.041667in}}{\pgfqpoint{0.000000in}{0.000000in}}{%
\pgfpathmoveto{\pgfqpoint{0.000000in}{0.000000in}}%
\pgfpathlineto{\pgfqpoint{0.000000in}{-0.041667in}}%
\pgfusepath{stroke,fill}%
}%
\begin{pgfscope}%
\pgfsys@transformshift{2.287993in}{3.574193in}%
\pgfsys@useobject{currentmarker}{}%
\end{pgfscope}%
\end{pgfscope}%
\begin{pgfscope}%
\definecolor{textcolor}{rgb}{0.000000,0.000000,0.000000}%
\pgfsetstrokecolor{textcolor}%
\pgfsetfillcolor{textcolor}%
\pgftext[x=2.287993in,y=2.299381in,,top]{\color{textcolor}\rmfamily\fontsize{10.000000}{12.000000}\selectfont \(\displaystyle {80}\)}%
\end{pgfscope}%
\begin{pgfscope}%
\pgfsetbuttcap%
\pgfsetroundjoin%
\definecolor{currentfill}{rgb}{0.000000,0.000000,0.000000}%
\pgfsetfillcolor{currentfill}%
\pgfsetlinewidth{0.501875pt}%
\definecolor{currentstroke}{rgb}{0.000000,0.000000,0.000000}%
\pgfsetstrokecolor{currentstroke}%
\pgfsetdash{}{0pt}%
\pgfsys@defobject{currentmarker}{\pgfqpoint{0.000000in}{0.000000in}}{\pgfqpoint{0.000000in}{0.020833in}}{%
\pgfpathmoveto{\pgfqpoint{0.000000in}{0.000000in}}%
\pgfpathlineto{\pgfqpoint{0.000000in}{0.020833in}}%
\pgfusepath{stroke,fill}%
}%
\begin{pgfscope}%
\pgfsys@transformshift{0.649221in}{2.347992in}%
\pgfsys@useobject{currentmarker}{}%
\end{pgfscope}%
\end{pgfscope}%
\begin{pgfscope}%
\pgfsetbuttcap%
\pgfsetroundjoin%
\definecolor{currentfill}{rgb}{0.000000,0.000000,0.000000}%
\pgfsetfillcolor{currentfill}%
\pgfsetlinewidth{0.501875pt}%
\definecolor{currentstroke}{rgb}{0.000000,0.000000,0.000000}%
\pgfsetstrokecolor{currentstroke}%
\pgfsetdash{}{0pt}%
\pgfsys@defobject{currentmarker}{\pgfqpoint{0.000000in}{-0.020833in}}{\pgfqpoint{0.000000in}{0.000000in}}{%
\pgfpathmoveto{\pgfqpoint{0.000000in}{0.000000in}}%
\pgfpathlineto{\pgfqpoint{0.000000in}{-0.020833in}}%
\pgfusepath{stroke,fill}%
}%
\begin{pgfscope}%
\pgfsys@transformshift{0.649221in}{3.574193in}%
\pgfsys@useobject{currentmarker}{}%
\end{pgfscope}%
\end{pgfscope}%
\begin{pgfscope}%
\pgfsetbuttcap%
\pgfsetroundjoin%
\definecolor{currentfill}{rgb}{0.000000,0.000000,0.000000}%
\pgfsetfillcolor{currentfill}%
\pgfsetlinewidth{0.501875pt}%
\definecolor{currentstroke}{rgb}{0.000000,0.000000,0.000000}%
\pgfsetstrokecolor{currentstroke}%
\pgfsetdash{}{0pt}%
\pgfsys@defobject{currentmarker}{\pgfqpoint{0.000000in}{0.000000in}}{\pgfqpoint{0.000000in}{0.020833in}}{%
\pgfpathmoveto{\pgfqpoint{0.000000in}{0.000000in}}%
\pgfpathlineto{\pgfqpoint{0.000000in}{0.020833in}}%
\pgfusepath{stroke,fill}%
}%
\begin{pgfscope}%
\pgfsys@transformshift{0.758473in}{2.347992in}%
\pgfsys@useobject{currentmarker}{}%
\end{pgfscope}%
\end{pgfscope}%
\begin{pgfscope}%
\pgfsetbuttcap%
\pgfsetroundjoin%
\definecolor{currentfill}{rgb}{0.000000,0.000000,0.000000}%
\pgfsetfillcolor{currentfill}%
\pgfsetlinewidth{0.501875pt}%
\definecolor{currentstroke}{rgb}{0.000000,0.000000,0.000000}%
\pgfsetstrokecolor{currentstroke}%
\pgfsetdash{}{0pt}%
\pgfsys@defobject{currentmarker}{\pgfqpoint{0.000000in}{-0.020833in}}{\pgfqpoint{0.000000in}{0.000000in}}{%
\pgfpathmoveto{\pgfqpoint{0.000000in}{0.000000in}}%
\pgfpathlineto{\pgfqpoint{0.000000in}{-0.020833in}}%
\pgfusepath{stroke,fill}%
}%
\begin{pgfscope}%
\pgfsys@transformshift{0.758473in}{3.574193in}%
\pgfsys@useobject{currentmarker}{}%
\end{pgfscope}%
\end{pgfscope}%
\begin{pgfscope}%
\pgfsetbuttcap%
\pgfsetroundjoin%
\definecolor{currentfill}{rgb}{0.000000,0.000000,0.000000}%
\pgfsetfillcolor{currentfill}%
\pgfsetlinewidth{0.501875pt}%
\definecolor{currentstroke}{rgb}{0.000000,0.000000,0.000000}%
\pgfsetstrokecolor{currentstroke}%
\pgfsetdash{}{0pt}%
\pgfsys@defobject{currentmarker}{\pgfqpoint{0.000000in}{0.000000in}}{\pgfqpoint{0.000000in}{0.020833in}}{%
\pgfpathmoveto{\pgfqpoint{0.000000in}{0.000000in}}%
\pgfpathlineto{\pgfqpoint{0.000000in}{0.020833in}}%
\pgfusepath{stroke,fill}%
}%
\begin{pgfscope}%
\pgfsys@transformshift{0.867724in}{2.347992in}%
\pgfsys@useobject{currentmarker}{}%
\end{pgfscope}%
\end{pgfscope}%
\begin{pgfscope}%
\pgfsetbuttcap%
\pgfsetroundjoin%
\definecolor{currentfill}{rgb}{0.000000,0.000000,0.000000}%
\pgfsetfillcolor{currentfill}%
\pgfsetlinewidth{0.501875pt}%
\definecolor{currentstroke}{rgb}{0.000000,0.000000,0.000000}%
\pgfsetstrokecolor{currentstroke}%
\pgfsetdash{}{0pt}%
\pgfsys@defobject{currentmarker}{\pgfqpoint{0.000000in}{-0.020833in}}{\pgfqpoint{0.000000in}{0.000000in}}{%
\pgfpathmoveto{\pgfqpoint{0.000000in}{0.000000in}}%
\pgfpathlineto{\pgfqpoint{0.000000in}{-0.020833in}}%
\pgfusepath{stroke,fill}%
}%
\begin{pgfscope}%
\pgfsys@transformshift{0.867724in}{3.574193in}%
\pgfsys@useobject{currentmarker}{}%
\end{pgfscope}%
\end{pgfscope}%
\begin{pgfscope}%
\pgfsetbuttcap%
\pgfsetroundjoin%
\definecolor{currentfill}{rgb}{0.000000,0.000000,0.000000}%
\pgfsetfillcolor{currentfill}%
\pgfsetlinewidth{0.501875pt}%
\definecolor{currentstroke}{rgb}{0.000000,0.000000,0.000000}%
\pgfsetstrokecolor{currentstroke}%
\pgfsetdash{}{0pt}%
\pgfsys@defobject{currentmarker}{\pgfqpoint{0.000000in}{0.000000in}}{\pgfqpoint{0.000000in}{0.020833in}}{%
\pgfpathmoveto{\pgfqpoint{0.000000in}{0.000000in}}%
\pgfpathlineto{\pgfqpoint{0.000000in}{0.020833in}}%
\pgfusepath{stroke,fill}%
}%
\begin{pgfscope}%
\pgfsys@transformshift{1.086227in}{2.347992in}%
\pgfsys@useobject{currentmarker}{}%
\end{pgfscope}%
\end{pgfscope}%
\begin{pgfscope}%
\pgfsetbuttcap%
\pgfsetroundjoin%
\definecolor{currentfill}{rgb}{0.000000,0.000000,0.000000}%
\pgfsetfillcolor{currentfill}%
\pgfsetlinewidth{0.501875pt}%
\definecolor{currentstroke}{rgb}{0.000000,0.000000,0.000000}%
\pgfsetstrokecolor{currentstroke}%
\pgfsetdash{}{0pt}%
\pgfsys@defobject{currentmarker}{\pgfqpoint{0.000000in}{-0.020833in}}{\pgfqpoint{0.000000in}{0.000000in}}{%
\pgfpathmoveto{\pgfqpoint{0.000000in}{0.000000in}}%
\pgfpathlineto{\pgfqpoint{0.000000in}{-0.020833in}}%
\pgfusepath{stroke,fill}%
}%
\begin{pgfscope}%
\pgfsys@transformshift{1.086227in}{3.574193in}%
\pgfsys@useobject{currentmarker}{}%
\end{pgfscope}%
\end{pgfscope}%
\begin{pgfscope}%
\pgfsetbuttcap%
\pgfsetroundjoin%
\definecolor{currentfill}{rgb}{0.000000,0.000000,0.000000}%
\pgfsetfillcolor{currentfill}%
\pgfsetlinewidth{0.501875pt}%
\definecolor{currentstroke}{rgb}{0.000000,0.000000,0.000000}%
\pgfsetstrokecolor{currentstroke}%
\pgfsetdash{}{0pt}%
\pgfsys@defobject{currentmarker}{\pgfqpoint{0.000000in}{0.000000in}}{\pgfqpoint{0.000000in}{0.020833in}}{%
\pgfpathmoveto{\pgfqpoint{0.000000in}{0.000000in}}%
\pgfpathlineto{\pgfqpoint{0.000000in}{0.020833in}}%
\pgfusepath{stroke,fill}%
}%
\begin{pgfscope}%
\pgfsys@transformshift{1.195478in}{2.347992in}%
\pgfsys@useobject{currentmarker}{}%
\end{pgfscope}%
\end{pgfscope}%
\begin{pgfscope}%
\pgfsetbuttcap%
\pgfsetroundjoin%
\definecolor{currentfill}{rgb}{0.000000,0.000000,0.000000}%
\pgfsetfillcolor{currentfill}%
\pgfsetlinewidth{0.501875pt}%
\definecolor{currentstroke}{rgb}{0.000000,0.000000,0.000000}%
\pgfsetstrokecolor{currentstroke}%
\pgfsetdash{}{0pt}%
\pgfsys@defobject{currentmarker}{\pgfqpoint{0.000000in}{-0.020833in}}{\pgfqpoint{0.000000in}{0.000000in}}{%
\pgfpathmoveto{\pgfqpoint{0.000000in}{0.000000in}}%
\pgfpathlineto{\pgfqpoint{0.000000in}{-0.020833in}}%
\pgfusepath{stroke,fill}%
}%
\begin{pgfscope}%
\pgfsys@transformshift{1.195478in}{3.574193in}%
\pgfsys@useobject{currentmarker}{}%
\end{pgfscope}%
\end{pgfscope}%
\begin{pgfscope}%
\pgfsetbuttcap%
\pgfsetroundjoin%
\definecolor{currentfill}{rgb}{0.000000,0.000000,0.000000}%
\pgfsetfillcolor{currentfill}%
\pgfsetlinewidth{0.501875pt}%
\definecolor{currentstroke}{rgb}{0.000000,0.000000,0.000000}%
\pgfsetstrokecolor{currentstroke}%
\pgfsetdash{}{0pt}%
\pgfsys@defobject{currentmarker}{\pgfqpoint{0.000000in}{0.000000in}}{\pgfqpoint{0.000000in}{0.020833in}}{%
\pgfpathmoveto{\pgfqpoint{0.000000in}{0.000000in}}%
\pgfpathlineto{\pgfqpoint{0.000000in}{0.020833in}}%
\pgfusepath{stroke,fill}%
}%
\begin{pgfscope}%
\pgfsys@transformshift{1.304730in}{2.347992in}%
\pgfsys@useobject{currentmarker}{}%
\end{pgfscope}%
\end{pgfscope}%
\begin{pgfscope}%
\pgfsetbuttcap%
\pgfsetroundjoin%
\definecolor{currentfill}{rgb}{0.000000,0.000000,0.000000}%
\pgfsetfillcolor{currentfill}%
\pgfsetlinewidth{0.501875pt}%
\definecolor{currentstroke}{rgb}{0.000000,0.000000,0.000000}%
\pgfsetstrokecolor{currentstroke}%
\pgfsetdash{}{0pt}%
\pgfsys@defobject{currentmarker}{\pgfqpoint{0.000000in}{-0.020833in}}{\pgfqpoint{0.000000in}{0.000000in}}{%
\pgfpathmoveto{\pgfqpoint{0.000000in}{0.000000in}}%
\pgfpathlineto{\pgfqpoint{0.000000in}{-0.020833in}}%
\pgfusepath{stroke,fill}%
}%
\begin{pgfscope}%
\pgfsys@transformshift{1.304730in}{3.574193in}%
\pgfsys@useobject{currentmarker}{}%
\end{pgfscope}%
\end{pgfscope}%
\begin{pgfscope}%
\pgfsetbuttcap%
\pgfsetroundjoin%
\definecolor{currentfill}{rgb}{0.000000,0.000000,0.000000}%
\pgfsetfillcolor{currentfill}%
\pgfsetlinewidth{0.501875pt}%
\definecolor{currentstroke}{rgb}{0.000000,0.000000,0.000000}%
\pgfsetstrokecolor{currentstroke}%
\pgfsetdash{}{0pt}%
\pgfsys@defobject{currentmarker}{\pgfqpoint{0.000000in}{0.000000in}}{\pgfqpoint{0.000000in}{0.020833in}}{%
\pgfpathmoveto{\pgfqpoint{0.000000in}{0.000000in}}%
\pgfpathlineto{\pgfqpoint{0.000000in}{0.020833in}}%
\pgfusepath{stroke,fill}%
}%
\begin{pgfscope}%
\pgfsys@transformshift{1.523233in}{2.347992in}%
\pgfsys@useobject{currentmarker}{}%
\end{pgfscope}%
\end{pgfscope}%
\begin{pgfscope}%
\pgfsetbuttcap%
\pgfsetroundjoin%
\definecolor{currentfill}{rgb}{0.000000,0.000000,0.000000}%
\pgfsetfillcolor{currentfill}%
\pgfsetlinewidth{0.501875pt}%
\definecolor{currentstroke}{rgb}{0.000000,0.000000,0.000000}%
\pgfsetstrokecolor{currentstroke}%
\pgfsetdash{}{0pt}%
\pgfsys@defobject{currentmarker}{\pgfqpoint{0.000000in}{-0.020833in}}{\pgfqpoint{0.000000in}{0.000000in}}{%
\pgfpathmoveto{\pgfqpoint{0.000000in}{0.000000in}}%
\pgfpathlineto{\pgfqpoint{0.000000in}{-0.020833in}}%
\pgfusepath{stroke,fill}%
}%
\begin{pgfscope}%
\pgfsys@transformshift{1.523233in}{3.574193in}%
\pgfsys@useobject{currentmarker}{}%
\end{pgfscope}%
\end{pgfscope}%
\begin{pgfscope}%
\pgfsetbuttcap%
\pgfsetroundjoin%
\definecolor{currentfill}{rgb}{0.000000,0.000000,0.000000}%
\pgfsetfillcolor{currentfill}%
\pgfsetlinewidth{0.501875pt}%
\definecolor{currentstroke}{rgb}{0.000000,0.000000,0.000000}%
\pgfsetstrokecolor{currentstroke}%
\pgfsetdash{}{0pt}%
\pgfsys@defobject{currentmarker}{\pgfqpoint{0.000000in}{0.000000in}}{\pgfqpoint{0.000000in}{0.020833in}}{%
\pgfpathmoveto{\pgfqpoint{0.000000in}{0.000000in}}%
\pgfpathlineto{\pgfqpoint{0.000000in}{0.020833in}}%
\pgfusepath{stroke,fill}%
}%
\begin{pgfscope}%
\pgfsys@transformshift{1.632484in}{2.347992in}%
\pgfsys@useobject{currentmarker}{}%
\end{pgfscope}%
\end{pgfscope}%
\begin{pgfscope}%
\pgfsetbuttcap%
\pgfsetroundjoin%
\definecolor{currentfill}{rgb}{0.000000,0.000000,0.000000}%
\pgfsetfillcolor{currentfill}%
\pgfsetlinewidth{0.501875pt}%
\definecolor{currentstroke}{rgb}{0.000000,0.000000,0.000000}%
\pgfsetstrokecolor{currentstroke}%
\pgfsetdash{}{0pt}%
\pgfsys@defobject{currentmarker}{\pgfqpoint{0.000000in}{-0.020833in}}{\pgfqpoint{0.000000in}{0.000000in}}{%
\pgfpathmoveto{\pgfqpoint{0.000000in}{0.000000in}}%
\pgfpathlineto{\pgfqpoint{0.000000in}{-0.020833in}}%
\pgfusepath{stroke,fill}%
}%
\begin{pgfscope}%
\pgfsys@transformshift{1.632484in}{3.574193in}%
\pgfsys@useobject{currentmarker}{}%
\end{pgfscope}%
\end{pgfscope}%
\begin{pgfscope}%
\pgfsetbuttcap%
\pgfsetroundjoin%
\definecolor{currentfill}{rgb}{0.000000,0.000000,0.000000}%
\pgfsetfillcolor{currentfill}%
\pgfsetlinewidth{0.501875pt}%
\definecolor{currentstroke}{rgb}{0.000000,0.000000,0.000000}%
\pgfsetstrokecolor{currentstroke}%
\pgfsetdash{}{0pt}%
\pgfsys@defobject{currentmarker}{\pgfqpoint{0.000000in}{0.000000in}}{\pgfqpoint{0.000000in}{0.020833in}}{%
\pgfpathmoveto{\pgfqpoint{0.000000in}{0.000000in}}%
\pgfpathlineto{\pgfqpoint{0.000000in}{0.020833in}}%
\pgfusepath{stroke,fill}%
}%
\begin{pgfscope}%
\pgfsys@transformshift{1.741736in}{2.347992in}%
\pgfsys@useobject{currentmarker}{}%
\end{pgfscope}%
\end{pgfscope}%
\begin{pgfscope}%
\pgfsetbuttcap%
\pgfsetroundjoin%
\definecolor{currentfill}{rgb}{0.000000,0.000000,0.000000}%
\pgfsetfillcolor{currentfill}%
\pgfsetlinewidth{0.501875pt}%
\definecolor{currentstroke}{rgb}{0.000000,0.000000,0.000000}%
\pgfsetstrokecolor{currentstroke}%
\pgfsetdash{}{0pt}%
\pgfsys@defobject{currentmarker}{\pgfqpoint{0.000000in}{-0.020833in}}{\pgfqpoint{0.000000in}{0.000000in}}{%
\pgfpathmoveto{\pgfqpoint{0.000000in}{0.000000in}}%
\pgfpathlineto{\pgfqpoint{0.000000in}{-0.020833in}}%
\pgfusepath{stroke,fill}%
}%
\begin{pgfscope}%
\pgfsys@transformshift{1.741736in}{3.574193in}%
\pgfsys@useobject{currentmarker}{}%
\end{pgfscope}%
\end{pgfscope}%
\begin{pgfscope}%
\pgfsetbuttcap%
\pgfsetroundjoin%
\definecolor{currentfill}{rgb}{0.000000,0.000000,0.000000}%
\pgfsetfillcolor{currentfill}%
\pgfsetlinewidth{0.501875pt}%
\definecolor{currentstroke}{rgb}{0.000000,0.000000,0.000000}%
\pgfsetstrokecolor{currentstroke}%
\pgfsetdash{}{0pt}%
\pgfsys@defobject{currentmarker}{\pgfqpoint{0.000000in}{0.000000in}}{\pgfqpoint{0.000000in}{0.020833in}}{%
\pgfpathmoveto{\pgfqpoint{0.000000in}{0.000000in}}%
\pgfpathlineto{\pgfqpoint{0.000000in}{0.020833in}}%
\pgfusepath{stroke,fill}%
}%
\begin{pgfscope}%
\pgfsys@transformshift{1.960238in}{2.347992in}%
\pgfsys@useobject{currentmarker}{}%
\end{pgfscope}%
\end{pgfscope}%
\begin{pgfscope}%
\pgfsetbuttcap%
\pgfsetroundjoin%
\definecolor{currentfill}{rgb}{0.000000,0.000000,0.000000}%
\pgfsetfillcolor{currentfill}%
\pgfsetlinewidth{0.501875pt}%
\definecolor{currentstroke}{rgb}{0.000000,0.000000,0.000000}%
\pgfsetstrokecolor{currentstroke}%
\pgfsetdash{}{0pt}%
\pgfsys@defobject{currentmarker}{\pgfqpoint{0.000000in}{-0.020833in}}{\pgfqpoint{0.000000in}{0.000000in}}{%
\pgfpathmoveto{\pgfqpoint{0.000000in}{0.000000in}}%
\pgfpathlineto{\pgfqpoint{0.000000in}{-0.020833in}}%
\pgfusepath{stroke,fill}%
}%
\begin{pgfscope}%
\pgfsys@transformshift{1.960238in}{3.574193in}%
\pgfsys@useobject{currentmarker}{}%
\end{pgfscope}%
\end{pgfscope}%
\begin{pgfscope}%
\pgfsetbuttcap%
\pgfsetroundjoin%
\definecolor{currentfill}{rgb}{0.000000,0.000000,0.000000}%
\pgfsetfillcolor{currentfill}%
\pgfsetlinewidth{0.501875pt}%
\definecolor{currentstroke}{rgb}{0.000000,0.000000,0.000000}%
\pgfsetstrokecolor{currentstroke}%
\pgfsetdash{}{0pt}%
\pgfsys@defobject{currentmarker}{\pgfqpoint{0.000000in}{0.000000in}}{\pgfqpoint{0.000000in}{0.020833in}}{%
\pgfpathmoveto{\pgfqpoint{0.000000in}{0.000000in}}%
\pgfpathlineto{\pgfqpoint{0.000000in}{0.020833in}}%
\pgfusepath{stroke,fill}%
}%
\begin{pgfscope}%
\pgfsys@transformshift{2.069490in}{2.347992in}%
\pgfsys@useobject{currentmarker}{}%
\end{pgfscope}%
\end{pgfscope}%
\begin{pgfscope}%
\pgfsetbuttcap%
\pgfsetroundjoin%
\definecolor{currentfill}{rgb}{0.000000,0.000000,0.000000}%
\pgfsetfillcolor{currentfill}%
\pgfsetlinewidth{0.501875pt}%
\definecolor{currentstroke}{rgb}{0.000000,0.000000,0.000000}%
\pgfsetstrokecolor{currentstroke}%
\pgfsetdash{}{0pt}%
\pgfsys@defobject{currentmarker}{\pgfqpoint{0.000000in}{-0.020833in}}{\pgfqpoint{0.000000in}{0.000000in}}{%
\pgfpathmoveto{\pgfqpoint{0.000000in}{0.000000in}}%
\pgfpathlineto{\pgfqpoint{0.000000in}{-0.020833in}}%
\pgfusepath{stroke,fill}%
}%
\begin{pgfscope}%
\pgfsys@transformshift{2.069490in}{3.574193in}%
\pgfsys@useobject{currentmarker}{}%
\end{pgfscope}%
\end{pgfscope}%
\begin{pgfscope}%
\pgfsetbuttcap%
\pgfsetroundjoin%
\definecolor{currentfill}{rgb}{0.000000,0.000000,0.000000}%
\pgfsetfillcolor{currentfill}%
\pgfsetlinewidth{0.501875pt}%
\definecolor{currentstroke}{rgb}{0.000000,0.000000,0.000000}%
\pgfsetstrokecolor{currentstroke}%
\pgfsetdash{}{0pt}%
\pgfsys@defobject{currentmarker}{\pgfqpoint{0.000000in}{0.000000in}}{\pgfqpoint{0.000000in}{0.020833in}}{%
\pgfpathmoveto{\pgfqpoint{0.000000in}{0.000000in}}%
\pgfpathlineto{\pgfqpoint{0.000000in}{0.020833in}}%
\pgfusepath{stroke,fill}%
}%
\begin{pgfscope}%
\pgfsys@transformshift{2.178741in}{2.347992in}%
\pgfsys@useobject{currentmarker}{}%
\end{pgfscope}%
\end{pgfscope}%
\begin{pgfscope}%
\pgfsetbuttcap%
\pgfsetroundjoin%
\definecolor{currentfill}{rgb}{0.000000,0.000000,0.000000}%
\pgfsetfillcolor{currentfill}%
\pgfsetlinewidth{0.501875pt}%
\definecolor{currentstroke}{rgb}{0.000000,0.000000,0.000000}%
\pgfsetstrokecolor{currentstroke}%
\pgfsetdash{}{0pt}%
\pgfsys@defobject{currentmarker}{\pgfqpoint{0.000000in}{-0.020833in}}{\pgfqpoint{0.000000in}{0.000000in}}{%
\pgfpathmoveto{\pgfqpoint{0.000000in}{0.000000in}}%
\pgfpathlineto{\pgfqpoint{0.000000in}{-0.020833in}}%
\pgfusepath{stroke,fill}%
}%
\begin{pgfscope}%
\pgfsys@transformshift{2.178741in}{3.574193in}%
\pgfsys@useobject{currentmarker}{}%
\end{pgfscope}%
\end{pgfscope}%
\begin{pgfscope}%
\pgfsetbuttcap%
\pgfsetroundjoin%
\definecolor{currentfill}{rgb}{0.000000,0.000000,0.000000}%
\pgfsetfillcolor{currentfill}%
\pgfsetlinewidth{0.501875pt}%
\definecolor{currentstroke}{rgb}{0.000000,0.000000,0.000000}%
\pgfsetstrokecolor{currentstroke}%
\pgfsetdash{}{0pt}%
\pgfsys@defobject{currentmarker}{\pgfqpoint{0.000000in}{0.000000in}}{\pgfqpoint{0.000000in}{0.020833in}}{%
\pgfpathmoveto{\pgfqpoint{0.000000in}{0.000000in}}%
\pgfpathlineto{\pgfqpoint{0.000000in}{0.020833in}}%
\pgfusepath{stroke,fill}%
}%
\begin{pgfscope}%
\pgfsys@transformshift{2.397244in}{2.347992in}%
\pgfsys@useobject{currentmarker}{}%
\end{pgfscope}%
\end{pgfscope}%
\begin{pgfscope}%
\pgfsetbuttcap%
\pgfsetroundjoin%
\definecolor{currentfill}{rgb}{0.000000,0.000000,0.000000}%
\pgfsetfillcolor{currentfill}%
\pgfsetlinewidth{0.501875pt}%
\definecolor{currentstroke}{rgb}{0.000000,0.000000,0.000000}%
\pgfsetstrokecolor{currentstroke}%
\pgfsetdash{}{0pt}%
\pgfsys@defobject{currentmarker}{\pgfqpoint{0.000000in}{-0.020833in}}{\pgfqpoint{0.000000in}{0.000000in}}{%
\pgfpathmoveto{\pgfqpoint{0.000000in}{0.000000in}}%
\pgfpathlineto{\pgfqpoint{0.000000in}{-0.020833in}}%
\pgfusepath{stroke,fill}%
}%
\begin{pgfscope}%
\pgfsys@transformshift{2.397244in}{3.574193in}%
\pgfsys@useobject{currentmarker}{}%
\end{pgfscope}%
\end{pgfscope}%
\begin{pgfscope}%
\pgfsetbuttcap%
\pgfsetroundjoin%
\definecolor{currentfill}{rgb}{0.000000,0.000000,0.000000}%
\pgfsetfillcolor{currentfill}%
\pgfsetlinewidth{0.501875pt}%
\definecolor{currentstroke}{rgb}{0.000000,0.000000,0.000000}%
\pgfsetstrokecolor{currentstroke}%
\pgfsetdash{}{0pt}%
\pgfsys@defobject{currentmarker}{\pgfqpoint{0.000000in}{0.000000in}}{\pgfqpoint{0.000000in}{0.020833in}}{%
\pgfpathmoveto{\pgfqpoint{0.000000in}{0.000000in}}%
\pgfpathlineto{\pgfqpoint{0.000000in}{0.020833in}}%
\pgfusepath{stroke,fill}%
}%
\begin{pgfscope}%
\pgfsys@transformshift{2.506496in}{2.347992in}%
\pgfsys@useobject{currentmarker}{}%
\end{pgfscope}%
\end{pgfscope}%
\begin{pgfscope}%
\pgfsetbuttcap%
\pgfsetroundjoin%
\definecolor{currentfill}{rgb}{0.000000,0.000000,0.000000}%
\pgfsetfillcolor{currentfill}%
\pgfsetlinewidth{0.501875pt}%
\definecolor{currentstroke}{rgb}{0.000000,0.000000,0.000000}%
\pgfsetstrokecolor{currentstroke}%
\pgfsetdash{}{0pt}%
\pgfsys@defobject{currentmarker}{\pgfqpoint{0.000000in}{-0.020833in}}{\pgfqpoint{0.000000in}{0.000000in}}{%
\pgfpathmoveto{\pgfqpoint{0.000000in}{0.000000in}}%
\pgfpathlineto{\pgfqpoint{0.000000in}{-0.020833in}}%
\pgfusepath{stroke,fill}%
}%
\begin{pgfscope}%
\pgfsys@transformshift{2.506496in}{3.574193in}%
\pgfsys@useobject{currentmarker}{}%
\end{pgfscope}%
\end{pgfscope}%
\begin{pgfscope}%
\pgfsetbuttcap%
\pgfsetroundjoin%
\definecolor{currentfill}{rgb}{0.000000,0.000000,0.000000}%
\pgfsetfillcolor{currentfill}%
\pgfsetlinewidth{0.501875pt}%
\definecolor{currentstroke}{rgb}{0.000000,0.000000,0.000000}%
\pgfsetstrokecolor{currentstroke}%
\pgfsetdash{}{0pt}%
\pgfsys@defobject{currentmarker}{\pgfqpoint{0.000000in}{0.000000in}}{\pgfqpoint{0.000000in}{0.020833in}}{%
\pgfpathmoveto{\pgfqpoint{0.000000in}{0.000000in}}%
\pgfpathlineto{\pgfqpoint{0.000000in}{0.020833in}}%
\pgfusepath{stroke,fill}%
}%
\begin{pgfscope}%
\pgfsys@transformshift{2.615747in}{2.347992in}%
\pgfsys@useobject{currentmarker}{}%
\end{pgfscope}%
\end{pgfscope}%
\begin{pgfscope}%
\pgfsetbuttcap%
\pgfsetroundjoin%
\definecolor{currentfill}{rgb}{0.000000,0.000000,0.000000}%
\pgfsetfillcolor{currentfill}%
\pgfsetlinewidth{0.501875pt}%
\definecolor{currentstroke}{rgb}{0.000000,0.000000,0.000000}%
\pgfsetstrokecolor{currentstroke}%
\pgfsetdash{}{0pt}%
\pgfsys@defobject{currentmarker}{\pgfqpoint{0.000000in}{-0.020833in}}{\pgfqpoint{0.000000in}{0.000000in}}{%
\pgfpathmoveto{\pgfqpoint{0.000000in}{0.000000in}}%
\pgfpathlineto{\pgfqpoint{0.000000in}{-0.020833in}}%
\pgfusepath{stroke,fill}%
}%
\begin{pgfscope}%
\pgfsys@transformshift{2.615747in}{3.574193in}%
\pgfsys@useobject{currentmarker}{}%
\end{pgfscope}%
\end{pgfscope}%
\begin{pgfscope}%
\pgfsetbuttcap%
\pgfsetroundjoin%
\definecolor{currentfill}{rgb}{0.000000,0.000000,0.000000}%
\pgfsetfillcolor{currentfill}%
\pgfsetlinewidth{0.501875pt}%
\definecolor{currentstroke}{rgb}{0.000000,0.000000,0.000000}%
\pgfsetstrokecolor{currentstroke}%
\pgfsetdash{}{0pt}%
\pgfsys@defobject{currentmarker}{\pgfqpoint{0.000000in}{0.000000in}}{\pgfqpoint{0.000000in}{0.020833in}}{%
\pgfpathmoveto{\pgfqpoint{0.000000in}{0.000000in}}%
\pgfpathlineto{\pgfqpoint{0.000000in}{0.020833in}}%
\pgfusepath{stroke,fill}%
}%
\begin{pgfscope}%
\pgfsys@transformshift{2.724998in}{2.347992in}%
\pgfsys@useobject{currentmarker}{}%
\end{pgfscope}%
\end{pgfscope}%
\begin{pgfscope}%
\pgfsetbuttcap%
\pgfsetroundjoin%
\definecolor{currentfill}{rgb}{0.000000,0.000000,0.000000}%
\pgfsetfillcolor{currentfill}%
\pgfsetlinewidth{0.501875pt}%
\definecolor{currentstroke}{rgb}{0.000000,0.000000,0.000000}%
\pgfsetstrokecolor{currentstroke}%
\pgfsetdash{}{0pt}%
\pgfsys@defobject{currentmarker}{\pgfqpoint{0.000000in}{-0.020833in}}{\pgfqpoint{0.000000in}{0.000000in}}{%
\pgfpathmoveto{\pgfqpoint{0.000000in}{0.000000in}}%
\pgfpathlineto{\pgfqpoint{0.000000in}{-0.020833in}}%
\pgfusepath{stroke,fill}%
}%
\begin{pgfscope}%
\pgfsys@transformshift{2.724998in}{3.574193in}%
\pgfsys@useobject{currentmarker}{}%
\end{pgfscope}%
\end{pgfscope}%
\begin{pgfscope}%
\definecolor{textcolor}{rgb}{0.000000,0.000000,0.000000}%
\pgfsetstrokecolor{textcolor}%
\pgfsetfillcolor{textcolor}%
\pgftext[x=1.643409in,y=2.109413in,,top]{\color{textcolor}\rmfamily\fontsize{10.000000}{12.000000}\selectfont \(\displaystyle K\)}%
\end{pgfscope}%
\begin{pgfscope}%
\pgfsetbuttcap%
\pgfsetroundjoin%
\definecolor{currentfill}{rgb}{0.000000,0.000000,0.000000}%
\pgfsetfillcolor{currentfill}%
\pgfsetlinewidth{0.501875pt}%
\definecolor{currentstroke}{rgb}{0.000000,0.000000,0.000000}%
\pgfsetstrokecolor{currentstroke}%
\pgfsetdash{}{0pt}%
\pgfsys@defobject{currentmarker}{\pgfqpoint{0.000000in}{0.000000in}}{\pgfqpoint{0.041667in}{0.000000in}}{%
\pgfpathmoveto{\pgfqpoint{0.000000in}{0.000000in}}%
\pgfpathlineto{\pgfqpoint{0.041667in}{0.000000in}}%
\pgfusepath{stroke,fill}%
}%
\begin{pgfscope}%
\pgfsys@transformshift{0.539970in}{2.393470in}%
\pgfsys@useobject{currentmarker}{}%
\end{pgfscope}%
\end{pgfscope}%
\begin{pgfscope}%
\pgfsetbuttcap%
\pgfsetroundjoin%
\definecolor{currentfill}{rgb}{0.000000,0.000000,0.000000}%
\pgfsetfillcolor{currentfill}%
\pgfsetlinewidth{0.501875pt}%
\definecolor{currentstroke}{rgb}{0.000000,0.000000,0.000000}%
\pgfsetstrokecolor{currentstroke}%
\pgfsetdash{}{0pt}%
\pgfsys@defobject{currentmarker}{\pgfqpoint{-0.041667in}{0.000000in}}{\pgfqpoint{-0.000000in}{0.000000in}}{%
\pgfpathmoveto{\pgfqpoint{-0.000000in}{0.000000in}}%
\pgfpathlineto{\pgfqpoint{-0.041667in}{0.000000in}}%
\pgfusepath{stroke,fill}%
}%
\begin{pgfscope}%
\pgfsys@transformshift{2.746849in}{2.393470in}%
\pgfsys@useobject{currentmarker}{}%
\end{pgfscope}%
\end{pgfscope}%
\begin{pgfscope}%
\definecolor{textcolor}{rgb}{0.000000,0.000000,0.000000}%
\pgfsetstrokecolor{textcolor}%
\pgfsetfillcolor{textcolor}%
\pgftext[x=0.244444in, y=2.340708in, left, base]{\color{textcolor}\rmfamily\fontsize{10.000000}{12.000000}\selectfont \(\displaystyle {0.85}\)}%
\end{pgfscope}%
\begin{pgfscope}%
\pgfsetbuttcap%
\pgfsetroundjoin%
\definecolor{currentfill}{rgb}{0.000000,0.000000,0.000000}%
\pgfsetfillcolor{currentfill}%
\pgfsetlinewidth{0.501875pt}%
\definecolor{currentstroke}{rgb}{0.000000,0.000000,0.000000}%
\pgfsetstrokecolor{currentstroke}%
\pgfsetdash{}{0pt}%
\pgfsys@defobject{currentmarker}{\pgfqpoint{0.000000in}{0.000000in}}{\pgfqpoint{0.041667in}{0.000000in}}{%
\pgfpathmoveto{\pgfqpoint{0.000000in}{0.000000in}}%
\pgfpathlineto{\pgfqpoint{0.041667in}{0.000000in}}%
\pgfusepath{stroke,fill}%
}%
\begin{pgfscope}%
\pgfsys@transformshift{0.539970in}{2.769106in}%
\pgfsys@useobject{currentmarker}{}%
\end{pgfscope}%
\end{pgfscope}%
\begin{pgfscope}%
\pgfsetbuttcap%
\pgfsetroundjoin%
\definecolor{currentfill}{rgb}{0.000000,0.000000,0.000000}%
\pgfsetfillcolor{currentfill}%
\pgfsetlinewidth{0.501875pt}%
\definecolor{currentstroke}{rgb}{0.000000,0.000000,0.000000}%
\pgfsetstrokecolor{currentstroke}%
\pgfsetdash{}{0pt}%
\pgfsys@defobject{currentmarker}{\pgfqpoint{-0.041667in}{0.000000in}}{\pgfqpoint{-0.000000in}{0.000000in}}{%
\pgfpathmoveto{\pgfqpoint{-0.000000in}{0.000000in}}%
\pgfpathlineto{\pgfqpoint{-0.041667in}{0.000000in}}%
\pgfusepath{stroke,fill}%
}%
\begin{pgfscope}%
\pgfsys@transformshift{2.746849in}{2.769106in}%
\pgfsys@useobject{currentmarker}{}%
\end{pgfscope}%
\end{pgfscope}%
\begin{pgfscope}%
\definecolor{textcolor}{rgb}{0.000000,0.000000,0.000000}%
\pgfsetstrokecolor{textcolor}%
\pgfsetfillcolor{textcolor}%
\pgftext[x=0.244444in, y=2.716344in, left, base]{\color{textcolor}\rmfamily\fontsize{10.000000}{12.000000}\selectfont \(\displaystyle {0.90}\)}%
\end{pgfscope}%
\begin{pgfscope}%
\pgfsetbuttcap%
\pgfsetroundjoin%
\definecolor{currentfill}{rgb}{0.000000,0.000000,0.000000}%
\pgfsetfillcolor{currentfill}%
\pgfsetlinewidth{0.501875pt}%
\definecolor{currentstroke}{rgb}{0.000000,0.000000,0.000000}%
\pgfsetstrokecolor{currentstroke}%
\pgfsetdash{}{0pt}%
\pgfsys@defobject{currentmarker}{\pgfqpoint{0.000000in}{0.000000in}}{\pgfqpoint{0.041667in}{0.000000in}}{%
\pgfpathmoveto{\pgfqpoint{0.000000in}{0.000000in}}%
\pgfpathlineto{\pgfqpoint{0.041667in}{0.000000in}}%
\pgfusepath{stroke,fill}%
}%
\begin{pgfscope}%
\pgfsys@transformshift{0.539970in}{3.144742in}%
\pgfsys@useobject{currentmarker}{}%
\end{pgfscope}%
\end{pgfscope}%
\begin{pgfscope}%
\pgfsetbuttcap%
\pgfsetroundjoin%
\definecolor{currentfill}{rgb}{0.000000,0.000000,0.000000}%
\pgfsetfillcolor{currentfill}%
\pgfsetlinewidth{0.501875pt}%
\definecolor{currentstroke}{rgb}{0.000000,0.000000,0.000000}%
\pgfsetstrokecolor{currentstroke}%
\pgfsetdash{}{0pt}%
\pgfsys@defobject{currentmarker}{\pgfqpoint{-0.041667in}{0.000000in}}{\pgfqpoint{-0.000000in}{0.000000in}}{%
\pgfpathmoveto{\pgfqpoint{-0.000000in}{0.000000in}}%
\pgfpathlineto{\pgfqpoint{-0.041667in}{0.000000in}}%
\pgfusepath{stroke,fill}%
}%
\begin{pgfscope}%
\pgfsys@transformshift{2.746849in}{3.144742in}%
\pgfsys@useobject{currentmarker}{}%
\end{pgfscope}%
\end{pgfscope}%
\begin{pgfscope}%
\definecolor{textcolor}{rgb}{0.000000,0.000000,0.000000}%
\pgfsetstrokecolor{textcolor}%
\pgfsetfillcolor{textcolor}%
\pgftext[x=0.244444in, y=3.091980in, left, base]{\color{textcolor}\rmfamily\fontsize{10.000000}{12.000000}\selectfont \(\displaystyle {0.95}\)}%
\end{pgfscope}%
\begin{pgfscope}%
\pgfsetbuttcap%
\pgfsetroundjoin%
\definecolor{currentfill}{rgb}{0.000000,0.000000,0.000000}%
\pgfsetfillcolor{currentfill}%
\pgfsetlinewidth{0.501875pt}%
\definecolor{currentstroke}{rgb}{0.000000,0.000000,0.000000}%
\pgfsetstrokecolor{currentstroke}%
\pgfsetdash{}{0pt}%
\pgfsys@defobject{currentmarker}{\pgfqpoint{0.000000in}{0.000000in}}{\pgfqpoint{0.041667in}{0.000000in}}{%
\pgfpathmoveto{\pgfqpoint{0.000000in}{0.000000in}}%
\pgfpathlineto{\pgfqpoint{0.041667in}{0.000000in}}%
\pgfusepath{stroke,fill}%
}%
\begin{pgfscope}%
\pgfsys@transformshift{0.539970in}{3.520378in}%
\pgfsys@useobject{currentmarker}{}%
\end{pgfscope}%
\end{pgfscope}%
\begin{pgfscope}%
\pgfsetbuttcap%
\pgfsetroundjoin%
\definecolor{currentfill}{rgb}{0.000000,0.000000,0.000000}%
\pgfsetfillcolor{currentfill}%
\pgfsetlinewidth{0.501875pt}%
\definecolor{currentstroke}{rgb}{0.000000,0.000000,0.000000}%
\pgfsetstrokecolor{currentstroke}%
\pgfsetdash{}{0pt}%
\pgfsys@defobject{currentmarker}{\pgfqpoint{-0.041667in}{0.000000in}}{\pgfqpoint{-0.000000in}{0.000000in}}{%
\pgfpathmoveto{\pgfqpoint{-0.000000in}{0.000000in}}%
\pgfpathlineto{\pgfqpoint{-0.041667in}{0.000000in}}%
\pgfusepath{stroke,fill}%
}%
\begin{pgfscope}%
\pgfsys@transformshift{2.746849in}{3.520378in}%
\pgfsys@useobject{currentmarker}{}%
\end{pgfscope}%
\end{pgfscope}%
\begin{pgfscope}%
\definecolor{textcolor}{rgb}{0.000000,0.000000,0.000000}%
\pgfsetstrokecolor{textcolor}%
\pgfsetfillcolor{textcolor}%
\pgftext[x=0.244444in, y=3.467616in, left, base]{\color{textcolor}\rmfamily\fontsize{10.000000}{12.000000}\selectfont \(\displaystyle {1.00}\)}%
\end{pgfscope}%
\begin{pgfscope}%
\pgfsetbuttcap%
\pgfsetroundjoin%
\definecolor{currentfill}{rgb}{0.000000,0.000000,0.000000}%
\pgfsetfillcolor{currentfill}%
\pgfsetlinewidth{0.501875pt}%
\definecolor{currentstroke}{rgb}{0.000000,0.000000,0.000000}%
\pgfsetstrokecolor{currentstroke}%
\pgfsetdash{}{0pt}%
\pgfsys@defobject{currentmarker}{\pgfqpoint{0.000000in}{0.000000in}}{\pgfqpoint{0.020833in}{0.000000in}}{%
\pgfpathmoveto{\pgfqpoint{0.000000in}{0.000000in}}%
\pgfpathlineto{\pgfqpoint{0.020833in}{0.000000in}}%
\pgfusepath{stroke,fill}%
}%
\begin{pgfscope}%
\pgfsys@transformshift{0.539970in}{2.468597in}%
\pgfsys@useobject{currentmarker}{}%
\end{pgfscope}%
\end{pgfscope}%
\begin{pgfscope}%
\pgfsetbuttcap%
\pgfsetroundjoin%
\definecolor{currentfill}{rgb}{0.000000,0.000000,0.000000}%
\pgfsetfillcolor{currentfill}%
\pgfsetlinewidth{0.501875pt}%
\definecolor{currentstroke}{rgb}{0.000000,0.000000,0.000000}%
\pgfsetstrokecolor{currentstroke}%
\pgfsetdash{}{0pt}%
\pgfsys@defobject{currentmarker}{\pgfqpoint{-0.020833in}{0.000000in}}{\pgfqpoint{-0.000000in}{0.000000in}}{%
\pgfpathmoveto{\pgfqpoint{-0.000000in}{0.000000in}}%
\pgfpathlineto{\pgfqpoint{-0.020833in}{0.000000in}}%
\pgfusepath{stroke,fill}%
}%
\begin{pgfscope}%
\pgfsys@transformshift{2.746849in}{2.468597in}%
\pgfsys@useobject{currentmarker}{}%
\end{pgfscope}%
\end{pgfscope}%
\begin{pgfscope}%
\pgfsetbuttcap%
\pgfsetroundjoin%
\definecolor{currentfill}{rgb}{0.000000,0.000000,0.000000}%
\pgfsetfillcolor{currentfill}%
\pgfsetlinewidth{0.501875pt}%
\definecolor{currentstroke}{rgb}{0.000000,0.000000,0.000000}%
\pgfsetstrokecolor{currentstroke}%
\pgfsetdash{}{0pt}%
\pgfsys@defobject{currentmarker}{\pgfqpoint{0.000000in}{0.000000in}}{\pgfqpoint{0.020833in}{0.000000in}}{%
\pgfpathmoveto{\pgfqpoint{0.000000in}{0.000000in}}%
\pgfpathlineto{\pgfqpoint{0.020833in}{0.000000in}}%
\pgfusepath{stroke,fill}%
}%
\begin{pgfscope}%
\pgfsys@transformshift{0.539970in}{2.543724in}%
\pgfsys@useobject{currentmarker}{}%
\end{pgfscope}%
\end{pgfscope}%
\begin{pgfscope}%
\pgfsetbuttcap%
\pgfsetroundjoin%
\definecolor{currentfill}{rgb}{0.000000,0.000000,0.000000}%
\pgfsetfillcolor{currentfill}%
\pgfsetlinewidth{0.501875pt}%
\definecolor{currentstroke}{rgb}{0.000000,0.000000,0.000000}%
\pgfsetstrokecolor{currentstroke}%
\pgfsetdash{}{0pt}%
\pgfsys@defobject{currentmarker}{\pgfqpoint{-0.020833in}{0.000000in}}{\pgfqpoint{-0.000000in}{0.000000in}}{%
\pgfpathmoveto{\pgfqpoint{-0.000000in}{0.000000in}}%
\pgfpathlineto{\pgfqpoint{-0.020833in}{0.000000in}}%
\pgfusepath{stroke,fill}%
}%
\begin{pgfscope}%
\pgfsys@transformshift{2.746849in}{2.543724in}%
\pgfsys@useobject{currentmarker}{}%
\end{pgfscope}%
\end{pgfscope}%
\begin{pgfscope}%
\pgfsetbuttcap%
\pgfsetroundjoin%
\definecolor{currentfill}{rgb}{0.000000,0.000000,0.000000}%
\pgfsetfillcolor{currentfill}%
\pgfsetlinewidth{0.501875pt}%
\definecolor{currentstroke}{rgb}{0.000000,0.000000,0.000000}%
\pgfsetstrokecolor{currentstroke}%
\pgfsetdash{}{0pt}%
\pgfsys@defobject{currentmarker}{\pgfqpoint{0.000000in}{0.000000in}}{\pgfqpoint{0.020833in}{0.000000in}}{%
\pgfpathmoveto{\pgfqpoint{0.000000in}{0.000000in}}%
\pgfpathlineto{\pgfqpoint{0.020833in}{0.000000in}}%
\pgfusepath{stroke,fill}%
}%
\begin{pgfscope}%
\pgfsys@transformshift{0.539970in}{2.618852in}%
\pgfsys@useobject{currentmarker}{}%
\end{pgfscope}%
\end{pgfscope}%
\begin{pgfscope}%
\pgfsetbuttcap%
\pgfsetroundjoin%
\definecolor{currentfill}{rgb}{0.000000,0.000000,0.000000}%
\pgfsetfillcolor{currentfill}%
\pgfsetlinewidth{0.501875pt}%
\definecolor{currentstroke}{rgb}{0.000000,0.000000,0.000000}%
\pgfsetstrokecolor{currentstroke}%
\pgfsetdash{}{0pt}%
\pgfsys@defobject{currentmarker}{\pgfqpoint{-0.020833in}{0.000000in}}{\pgfqpoint{-0.000000in}{0.000000in}}{%
\pgfpathmoveto{\pgfqpoint{-0.000000in}{0.000000in}}%
\pgfpathlineto{\pgfqpoint{-0.020833in}{0.000000in}}%
\pgfusepath{stroke,fill}%
}%
\begin{pgfscope}%
\pgfsys@transformshift{2.746849in}{2.618852in}%
\pgfsys@useobject{currentmarker}{}%
\end{pgfscope}%
\end{pgfscope}%
\begin{pgfscope}%
\pgfsetbuttcap%
\pgfsetroundjoin%
\definecolor{currentfill}{rgb}{0.000000,0.000000,0.000000}%
\pgfsetfillcolor{currentfill}%
\pgfsetlinewidth{0.501875pt}%
\definecolor{currentstroke}{rgb}{0.000000,0.000000,0.000000}%
\pgfsetstrokecolor{currentstroke}%
\pgfsetdash{}{0pt}%
\pgfsys@defobject{currentmarker}{\pgfqpoint{0.000000in}{0.000000in}}{\pgfqpoint{0.020833in}{0.000000in}}{%
\pgfpathmoveto{\pgfqpoint{0.000000in}{0.000000in}}%
\pgfpathlineto{\pgfqpoint{0.020833in}{0.000000in}}%
\pgfusepath{stroke,fill}%
}%
\begin{pgfscope}%
\pgfsys@transformshift{0.539970in}{2.693979in}%
\pgfsys@useobject{currentmarker}{}%
\end{pgfscope}%
\end{pgfscope}%
\begin{pgfscope}%
\pgfsetbuttcap%
\pgfsetroundjoin%
\definecolor{currentfill}{rgb}{0.000000,0.000000,0.000000}%
\pgfsetfillcolor{currentfill}%
\pgfsetlinewidth{0.501875pt}%
\definecolor{currentstroke}{rgb}{0.000000,0.000000,0.000000}%
\pgfsetstrokecolor{currentstroke}%
\pgfsetdash{}{0pt}%
\pgfsys@defobject{currentmarker}{\pgfqpoint{-0.020833in}{0.000000in}}{\pgfqpoint{-0.000000in}{0.000000in}}{%
\pgfpathmoveto{\pgfqpoint{-0.000000in}{0.000000in}}%
\pgfpathlineto{\pgfqpoint{-0.020833in}{0.000000in}}%
\pgfusepath{stroke,fill}%
}%
\begin{pgfscope}%
\pgfsys@transformshift{2.746849in}{2.693979in}%
\pgfsys@useobject{currentmarker}{}%
\end{pgfscope}%
\end{pgfscope}%
\begin{pgfscope}%
\pgfsetbuttcap%
\pgfsetroundjoin%
\definecolor{currentfill}{rgb}{0.000000,0.000000,0.000000}%
\pgfsetfillcolor{currentfill}%
\pgfsetlinewidth{0.501875pt}%
\definecolor{currentstroke}{rgb}{0.000000,0.000000,0.000000}%
\pgfsetstrokecolor{currentstroke}%
\pgfsetdash{}{0pt}%
\pgfsys@defobject{currentmarker}{\pgfqpoint{0.000000in}{0.000000in}}{\pgfqpoint{0.020833in}{0.000000in}}{%
\pgfpathmoveto{\pgfqpoint{0.000000in}{0.000000in}}%
\pgfpathlineto{\pgfqpoint{0.020833in}{0.000000in}}%
\pgfusepath{stroke,fill}%
}%
\begin{pgfscope}%
\pgfsys@transformshift{0.539970in}{2.844233in}%
\pgfsys@useobject{currentmarker}{}%
\end{pgfscope}%
\end{pgfscope}%
\begin{pgfscope}%
\pgfsetbuttcap%
\pgfsetroundjoin%
\definecolor{currentfill}{rgb}{0.000000,0.000000,0.000000}%
\pgfsetfillcolor{currentfill}%
\pgfsetlinewidth{0.501875pt}%
\definecolor{currentstroke}{rgb}{0.000000,0.000000,0.000000}%
\pgfsetstrokecolor{currentstroke}%
\pgfsetdash{}{0pt}%
\pgfsys@defobject{currentmarker}{\pgfqpoint{-0.020833in}{0.000000in}}{\pgfqpoint{-0.000000in}{0.000000in}}{%
\pgfpathmoveto{\pgfqpoint{-0.000000in}{0.000000in}}%
\pgfpathlineto{\pgfqpoint{-0.020833in}{0.000000in}}%
\pgfusepath{stroke,fill}%
}%
\begin{pgfscope}%
\pgfsys@transformshift{2.746849in}{2.844233in}%
\pgfsys@useobject{currentmarker}{}%
\end{pgfscope}%
\end{pgfscope}%
\begin{pgfscope}%
\pgfsetbuttcap%
\pgfsetroundjoin%
\definecolor{currentfill}{rgb}{0.000000,0.000000,0.000000}%
\pgfsetfillcolor{currentfill}%
\pgfsetlinewidth{0.501875pt}%
\definecolor{currentstroke}{rgb}{0.000000,0.000000,0.000000}%
\pgfsetstrokecolor{currentstroke}%
\pgfsetdash{}{0pt}%
\pgfsys@defobject{currentmarker}{\pgfqpoint{0.000000in}{0.000000in}}{\pgfqpoint{0.020833in}{0.000000in}}{%
\pgfpathmoveto{\pgfqpoint{0.000000in}{0.000000in}}%
\pgfpathlineto{\pgfqpoint{0.020833in}{0.000000in}}%
\pgfusepath{stroke,fill}%
}%
\begin{pgfscope}%
\pgfsys@transformshift{0.539970in}{2.919360in}%
\pgfsys@useobject{currentmarker}{}%
\end{pgfscope}%
\end{pgfscope}%
\begin{pgfscope}%
\pgfsetbuttcap%
\pgfsetroundjoin%
\definecolor{currentfill}{rgb}{0.000000,0.000000,0.000000}%
\pgfsetfillcolor{currentfill}%
\pgfsetlinewidth{0.501875pt}%
\definecolor{currentstroke}{rgb}{0.000000,0.000000,0.000000}%
\pgfsetstrokecolor{currentstroke}%
\pgfsetdash{}{0pt}%
\pgfsys@defobject{currentmarker}{\pgfqpoint{-0.020833in}{0.000000in}}{\pgfqpoint{-0.000000in}{0.000000in}}{%
\pgfpathmoveto{\pgfqpoint{-0.000000in}{0.000000in}}%
\pgfpathlineto{\pgfqpoint{-0.020833in}{0.000000in}}%
\pgfusepath{stroke,fill}%
}%
\begin{pgfscope}%
\pgfsys@transformshift{2.746849in}{2.919360in}%
\pgfsys@useobject{currentmarker}{}%
\end{pgfscope}%
\end{pgfscope}%
\begin{pgfscope}%
\pgfsetbuttcap%
\pgfsetroundjoin%
\definecolor{currentfill}{rgb}{0.000000,0.000000,0.000000}%
\pgfsetfillcolor{currentfill}%
\pgfsetlinewidth{0.501875pt}%
\definecolor{currentstroke}{rgb}{0.000000,0.000000,0.000000}%
\pgfsetstrokecolor{currentstroke}%
\pgfsetdash{}{0pt}%
\pgfsys@defobject{currentmarker}{\pgfqpoint{0.000000in}{0.000000in}}{\pgfqpoint{0.020833in}{0.000000in}}{%
\pgfpathmoveto{\pgfqpoint{0.000000in}{0.000000in}}%
\pgfpathlineto{\pgfqpoint{0.020833in}{0.000000in}}%
\pgfusepath{stroke,fill}%
}%
\begin{pgfscope}%
\pgfsys@transformshift{0.539970in}{2.994487in}%
\pgfsys@useobject{currentmarker}{}%
\end{pgfscope}%
\end{pgfscope}%
\begin{pgfscope}%
\pgfsetbuttcap%
\pgfsetroundjoin%
\definecolor{currentfill}{rgb}{0.000000,0.000000,0.000000}%
\pgfsetfillcolor{currentfill}%
\pgfsetlinewidth{0.501875pt}%
\definecolor{currentstroke}{rgb}{0.000000,0.000000,0.000000}%
\pgfsetstrokecolor{currentstroke}%
\pgfsetdash{}{0pt}%
\pgfsys@defobject{currentmarker}{\pgfqpoint{-0.020833in}{0.000000in}}{\pgfqpoint{-0.000000in}{0.000000in}}{%
\pgfpathmoveto{\pgfqpoint{-0.000000in}{0.000000in}}%
\pgfpathlineto{\pgfqpoint{-0.020833in}{0.000000in}}%
\pgfusepath{stroke,fill}%
}%
\begin{pgfscope}%
\pgfsys@transformshift{2.746849in}{2.994487in}%
\pgfsys@useobject{currentmarker}{}%
\end{pgfscope}%
\end{pgfscope}%
\begin{pgfscope}%
\pgfsetbuttcap%
\pgfsetroundjoin%
\definecolor{currentfill}{rgb}{0.000000,0.000000,0.000000}%
\pgfsetfillcolor{currentfill}%
\pgfsetlinewidth{0.501875pt}%
\definecolor{currentstroke}{rgb}{0.000000,0.000000,0.000000}%
\pgfsetstrokecolor{currentstroke}%
\pgfsetdash{}{0pt}%
\pgfsys@defobject{currentmarker}{\pgfqpoint{0.000000in}{0.000000in}}{\pgfqpoint{0.020833in}{0.000000in}}{%
\pgfpathmoveto{\pgfqpoint{0.000000in}{0.000000in}}%
\pgfpathlineto{\pgfqpoint{0.020833in}{0.000000in}}%
\pgfusepath{stroke,fill}%
}%
\begin{pgfscope}%
\pgfsys@transformshift{0.539970in}{3.069615in}%
\pgfsys@useobject{currentmarker}{}%
\end{pgfscope}%
\end{pgfscope}%
\begin{pgfscope}%
\pgfsetbuttcap%
\pgfsetroundjoin%
\definecolor{currentfill}{rgb}{0.000000,0.000000,0.000000}%
\pgfsetfillcolor{currentfill}%
\pgfsetlinewidth{0.501875pt}%
\definecolor{currentstroke}{rgb}{0.000000,0.000000,0.000000}%
\pgfsetstrokecolor{currentstroke}%
\pgfsetdash{}{0pt}%
\pgfsys@defobject{currentmarker}{\pgfqpoint{-0.020833in}{0.000000in}}{\pgfqpoint{-0.000000in}{0.000000in}}{%
\pgfpathmoveto{\pgfqpoint{-0.000000in}{0.000000in}}%
\pgfpathlineto{\pgfqpoint{-0.020833in}{0.000000in}}%
\pgfusepath{stroke,fill}%
}%
\begin{pgfscope}%
\pgfsys@transformshift{2.746849in}{3.069615in}%
\pgfsys@useobject{currentmarker}{}%
\end{pgfscope}%
\end{pgfscope}%
\begin{pgfscope}%
\pgfsetbuttcap%
\pgfsetroundjoin%
\definecolor{currentfill}{rgb}{0.000000,0.000000,0.000000}%
\pgfsetfillcolor{currentfill}%
\pgfsetlinewidth{0.501875pt}%
\definecolor{currentstroke}{rgb}{0.000000,0.000000,0.000000}%
\pgfsetstrokecolor{currentstroke}%
\pgfsetdash{}{0pt}%
\pgfsys@defobject{currentmarker}{\pgfqpoint{0.000000in}{0.000000in}}{\pgfqpoint{0.020833in}{0.000000in}}{%
\pgfpathmoveto{\pgfqpoint{0.000000in}{0.000000in}}%
\pgfpathlineto{\pgfqpoint{0.020833in}{0.000000in}}%
\pgfusepath{stroke,fill}%
}%
\begin{pgfscope}%
\pgfsys@transformshift{0.539970in}{3.219869in}%
\pgfsys@useobject{currentmarker}{}%
\end{pgfscope}%
\end{pgfscope}%
\begin{pgfscope}%
\pgfsetbuttcap%
\pgfsetroundjoin%
\definecolor{currentfill}{rgb}{0.000000,0.000000,0.000000}%
\pgfsetfillcolor{currentfill}%
\pgfsetlinewidth{0.501875pt}%
\definecolor{currentstroke}{rgb}{0.000000,0.000000,0.000000}%
\pgfsetstrokecolor{currentstroke}%
\pgfsetdash{}{0pt}%
\pgfsys@defobject{currentmarker}{\pgfqpoint{-0.020833in}{0.000000in}}{\pgfqpoint{-0.000000in}{0.000000in}}{%
\pgfpathmoveto{\pgfqpoint{-0.000000in}{0.000000in}}%
\pgfpathlineto{\pgfqpoint{-0.020833in}{0.000000in}}%
\pgfusepath{stroke,fill}%
}%
\begin{pgfscope}%
\pgfsys@transformshift{2.746849in}{3.219869in}%
\pgfsys@useobject{currentmarker}{}%
\end{pgfscope}%
\end{pgfscope}%
\begin{pgfscope}%
\pgfsetbuttcap%
\pgfsetroundjoin%
\definecolor{currentfill}{rgb}{0.000000,0.000000,0.000000}%
\pgfsetfillcolor{currentfill}%
\pgfsetlinewidth{0.501875pt}%
\definecolor{currentstroke}{rgb}{0.000000,0.000000,0.000000}%
\pgfsetstrokecolor{currentstroke}%
\pgfsetdash{}{0pt}%
\pgfsys@defobject{currentmarker}{\pgfqpoint{0.000000in}{0.000000in}}{\pgfqpoint{0.020833in}{0.000000in}}{%
\pgfpathmoveto{\pgfqpoint{0.000000in}{0.000000in}}%
\pgfpathlineto{\pgfqpoint{0.020833in}{0.000000in}}%
\pgfusepath{stroke,fill}%
}%
\begin{pgfscope}%
\pgfsys@transformshift{0.539970in}{3.294996in}%
\pgfsys@useobject{currentmarker}{}%
\end{pgfscope}%
\end{pgfscope}%
\begin{pgfscope}%
\pgfsetbuttcap%
\pgfsetroundjoin%
\definecolor{currentfill}{rgb}{0.000000,0.000000,0.000000}%
\pgfsetfillcolor{currentfill}%
\pgfsetlinewidth{0.501875pt}%
\definecolor{currentstroke}{rgb}{0.000000,0.000000,0.000000}%
\pgfsetstrokecolor{currentstroke}%
\pgfsetdash{}{0pt}%
\pgfsys@defobject{currentmarker}{\pgfqpoint{-0.020833in}{0.000000in}}{\pgfqpoint{-0.000000in}{0.000000in}}{%
\pgfpathmoveto{\pgfqpoint{-0.000000in}{0.000000in}}%
\pgfpathlineto{\pgfqpoint{-0.020833in}{0.000000in}}%
\pgfusepath{stroke,fill}%
}%
\begin{pgfscope}%
\pgfsys@transformshift{2.746849in}{3.294996in}%
\pgfsys@useobject{currentmarker}{}%
\end{pgfscope}%
\end{pgfscope}%
\begin{pgfscope}%
\pgfsetbuttcap%
\pgfsetroundjoin%
\definecolor{currentfill}{rgb}{0.000000,0.000000,0.000000}%
\pgfsetfillcolor{currentfill}%
\pgfsetlinewidth{0.501875pt}%
\definecolor{currentstroke}{rgb}{0.000000,0.000000,0.000000}%
\pgfsetstrokecolor{currentstroke}%
\pgfsetdash{}{0pt}%
\pgfsys@defobject{currentmarker}{\pgfqpoint{0.000000in}{0.000000in}}{\pgfqpoint{0.020833in}{0.000000in}}{%
\pgfpathmoveto{\pgfqpoint{0.000000in}{0.000000in}}%
\pgfpathlineto{\pgfqpoint{0.020833in}{0.000000in}}%
\pgfusepath{stroke,fill}%
}%
\begin{pgfscope}%
\pgfsys@transformshift{0.539970in}{3.370123in}%
\pgfsys@useobject{currentmarker}{}%
\end{pgfscope}%
\end{pgfscope}%
\begin{pgfscope}%
\pgfsetbuttcap%
\pgfsetroundjoin%
\definecolor{currentfill}{rgb}{0.000000,0.000000,0.000000}%
\pgfsetfillcolor{currentfill}%
\pgfsetlinewidth{0.501875pt}%
\definecolor{currentstroke}{rgb}{0.000000,0.000000,0.000000}%
\pgfsetstrokecolor{currentstroke}%
\pgfsetdash{}{0pt}%
\pgfsys@defobject{currentmarker}{\pgfqpoint{-0.020833in}{0.000000in}}{\pgfqpoint{-0.000000in}{0.000000in}}{%
\pgfpathmoveto{\pgfqpoint{-0.000000in}{0.000000in}}%
\pgfpathlineto{\pgfqpoint{-0.020833in}{0.000000in}}%
\pgfusepath{stroke,fill}%
}%
\begin{pgfscope}%
\pgfsys@transformshift{2.746849in}{3.370123in}%
\pgfsys@useobject{currentmarker}{}%
\end{pgfscope}%
\end{pgfscope}%
\begin{pgfscope}%
\pgfsetbuttcap%
\pgfsetroundjoin%
\definecolor{currentfill}{rgb}{0.000000,0.000000,0.000000}%
\pgfsetfillcolor{currentfill}%
\pgfsetlinewidth{0.501875pt}%
\definecolor{currentstroke}{rgb}{0.000000,0.000000,0.000000}%
\pgfsetstrokecolor{currentstroke}%
\pgfsetdash{}{0pt}%
\pgfsys@defobject{currentmarker}{\pgfqpoint{0.000000in}{0.000000in}}{\pgfqpoint{0.020833in}{0.000000in}}{%
\pgfpathmoveto{\pgfqpoint{0.000000in}{0.000000in}}%
\pgfpathlineto{\pgfqpoint{0.020833in}{0.000000in}}%
\pgfusepath{stroke,fill}%
}%
\begin{pgfscope}%
\pgfsys@transformshift{0.539970in}{3.445251in}%
\pgfsys@useobject{currentmarker}{}%
\end{pgfscope}%
\end{pgfscope}%
\begin{pgfscope}%
\pgfsetbuttcap%
\pgfsetroundjoin%
\definecolor{currentfill}{rgb}{0.000000,0.000000,0.000000}%
\pgfsetfillcolor{currentfill}%
\pgfsetlinewidth{0.501875pt}%
\definecolor{currentstroke}{rgb}{0.000000,0.000000,0.000000}%
\pgfsetstrokecolor{currentstroke}%
\pgfsetdash{}{0pt}%
\pgfsys@defobject{currentmarker}{\pgfqpoint{-0.020833in}{0.000000in}}{\pgfqpoint{-0.000000in}{0.000000in}}{%
\pgfpathmoveto{\pgfqpoint{-0.000000in}{0.000000in}}%
\pgfpathlineto{\pgfqpoint{-0.020833in}{0.000000in}}%
\pgfusepath{stroke,fill}%
}%
\begin{pgfscope}%
\pgfsys@transformshift{2.746849in}{3.445251in}%
\pgfsys@useobject{currentmarker}{}%
\end{pgfscope}%
\end{pgfscope}%
\begin{pgfscope}%
\definecolor{textcolor}{rgb}{0.000000,0.000000,0.000000}%
\pgfsetstrokecolor{textcolor}%
\pgfsetfillcolor{textcolor}%
\pgftext[x=0.188889in,y=2.961093in,,bottom,rotate=90.000000]{\color{textcolor}\rmfamily\fontsize{10.000000}{12.000000}\selectfont \(\displaystyle T(K)\)}%
\end{pgfscope}%
\begin{pgfscope}%
\pgfpathrectangle{\pgfqpoint{0.539970in}{2.347992in}}{\pgfqpoint{2.206879in}{1.226201in}}%
\pgfusepath{clip}%
\pgfsetrectcap%
\pgfsetroundjoin%
\pgfsetlinewidth{1.003750pt}%
\definecolor{currentstroke}{rgb}{0.047059,0.364706,0.647059}%
\pgfsetstrokecolor{currentstroke}%
\pgfsetdash{}{0pt}%
\pgfpathmoveto{\pgfqpoint{0.561820in}{2.442215in}}%
\pgfpathlineto{\pgfqpoint{0.583670in}{2.440363in}}%
\pgfpathlineto{\pgfqpoint{0.605521in}{2.440464in}}%
\pgfpathlineto{\pgfqpoint{0.627371in}{2.444849in}}%
\pgfpathlineto{\pgfqpoint{0.649221in}{2.447812in}}%
\pgfpathlineto{\pgfqpoint{0.671072in}{2.452268in}}%
\pgfpathlineto{\pgfqpoint{0.692922in}{2.448464in}}%
\pgfpathlineto{\pgfqpoint{0.714772in}{2.446750in}}%
\pgfpathlineto{\pgfqpoint{0.736622in}{2.444694in}}%
\pgfpathlineto{\pgfqpoint{0.758473in}{2.443699in}}%
\pgfpathlineto{\pgfqpoint{0.780323in}{2.445553in}}%
\pgfpathlineto{\pgfqpoint{0.802173in}{2.443794in}}%
\pgfpathlineto{\pgfqpoint{0.824024in}{2.442211in}}%
\pgfpathlineto{\pgfqpoint{0.845874in}{2.443065in}}%
\pgfpathlineto{\pgfqpoint{0.867724in}{2.444001in}}%
\pgfpathlineto{\pgfqpoint{0.889574in}{2.443349in}}%
\pgfpathlineto{\pgfqpoint{0.911425in}{2.443532in}}%
\pgfpathlineto{\pgfqpoint{0.933275in}{2.446364in}}%
\pgfpathlineto{\pgfqpoint{0.955125in}{2.446812in}}%
\pgfpathlineto{\pgfqpoint{0.976976in}{2.447984in}}%
\pgfpathlineto{\pgfqpoint{0.998826in}{2.450690in}}%
\pgfpathlineto{\pgfqpoint{1.020676in}{2.450440in}}%
\pgfpathlineto{\pgfqpoint{1.042526in}{2.451107in}}%
\pgfpathlineto{\pgfqpoint{1.064377in}{2.451527in}}%
\pgfpathlineto{\pgfqpoint{1.086227in}{2.452695in}}%
\pgfpathlineto{\pgfqpoint{1.108077in}{2.452586in}}%
\pgfpathlineto{\pgfqpoint{1.129928in}{2.453299in}}%
\pgfpathlineto{\pgfqpoint{1.151778in}{2.452503in}}%
\pgfpathlineto{\pgfqpoint{1.173628in}{2.452745in}}%
\pgfpathlineto{\pgfqpoint{1.195478in}{2.452582in}}%
\pgfpathlineto{\pgfqpoint{1.217329in}{2.453288in}}%
\pgfpathlineto{\pgfqpoint{1.239179in}{2.454094in}}%
\pgfpathlineto{\pgfqpoint{1.261029in}{2.455245in}}%
\pgfpathlineto{\pgfqpoint{1.282880in}{2.454593in}}%
\pgfpathlineto{\pgfqpoint{1.304730in}{2.454969in}}%
\pgfpathlineto{\pgfqpoint{1.326580in}{2.454193in}}%
\pgfpathlineto{\pgfqpoint{1.348430in}{2.455605in}}%
\pgfpathlineto{\pgfqpoint{1.370281in}{2.455740in}}%
\pgfpathlineto{\pgfqpoint{1.392131in}{2.455439in}}%
\pgfpathlineto{\pgfqpoint{1.413981in}{2.456343in}}%
\pgfpathlineto{\pgfqpoint{1.435832in}{2.454926in}}%
\pgfpathlineto{\pgfqpoint{1.457682in}{2.455006in}}%
\pgfpathlineto{\pgfqpoint{1.479532in}{2.455523in}}%
\pgfpathlineto{\pgfqpoint{1.501382in}{2.455246in}}%
\pgfpathlineto{\pgfqpoint{1.523233in}{2.456162in}}%
\pgfpathlineto{\pgfqpoint{1.545083in}{2.456018in}}%
\pgfpathlineto{\pgfqpoint{1.566933in}{2.455791in}}%
\pgfpathlineto{\pgfqpoint{1.588784in}{2.455895in}}%
\pgfpathlineto{\pgfqpoint{1.610634in}{2.455253in}}%
\pgfpathlineto{\pgfqpoint{1.632484in}{2.455873in}}%
\pgfpathlineto{\pgfqpoint{1.654334in}{2.456123in}}%
\pgfpathlineto{\pgfqpoint{1.676185in}{2.454885in}}%
\pgfpathlineto{\pgfqpoint{1.698035in}{2.455349in}}%
\pgfpathlineto{\pgfqpoint{1.719885in}{2.455593in}}%
\pgfpathlineto{\pgfqpoint{1.741736in}{2.455660in}}%
\pgfpathlineto{\pgfqpoint{1.763586in}{2.455550in}}%
\pgfpathlineto{\pgfqpoint{1.785436in}{2.455503in}}%
\pgfpathlineto{\pgfqpoint{1.807286in}{2.455388in}}%
\pgfpathlineto{\pgfqpoint{1.829137in}{2.455940in}}%
\pgfpathlineto{\pgfqpoint{1.850987in}{2.455992in}}%
\pgfpathlineto{\pgfqpoint{1.872837in}{2.456142in}}%
\pgfpathlineto{\pgfqpoint{1.894688in}{2.456312in}}%
\pgfpathlineto{\pgfqpoint{1.916538in}{2.456583in}}%
\pgfpathlineto{\pgfqpoint{1.938388in}{2.457166in}}%
\pgfpathlineto{\pgfqpoint{1.960238in}{2.456935in}}%
\pgfpathlineto{\pgfqpoint{1.982089in}{2.457395in}}%
\pgfpathlineto{\pgfqpoint{2.003939in}{2.457128in}}%
\pgfpathlineto{\pgfqpoint{2.025789in}{2.457849in}}%
\pgfpathlineto{\pgfqpoint{2.047640in}{2.458116in}}%
\pgfpathlineto{\pgfqpoint{2.069490in}{2.458189in}}%
\pgfpathlineto{\pgfqpoint{2.091340in}{2.458371in}}%
\pgfpathlineto{\pgfqpoint{2.113190in}{2.458401in}}%
\pgfpathlineto{\pgfqpoint{2.135041in}{2.458089in}}%
\pgfpathlineto{\pgfqpoint{2.156891in}{2.458593in}}%
\pgfpathlineto{\pgfqpoint{2.178741in}{2.458189in}}%
\pgfpathlineto{\pgfqpoint{2.200592in}{2.458204in}}%
\pgfpathlineto{\pgfqpoint{2.222442in}{2.458209in}}%
\pgfpathlineto{\pgfqpoint{2.244292in}{2.457922in}}%
\pgfpathlineto{\pgfqpoint{2.266142in}{2.457722in}}%
\pgfpathlineto{\pgfqpoint{2.287993in}{2.457850in}}%
\pgfpathlineto{\pgfqpoint{2.309843in}{2.458165in}}%
\pgfpathlineto{\pgfqpoint{2.331693in}{2.458384in}}%
\pgfpathlineto{\pgfqpoint{2.353544in}{2.458804in}}%
\pgfpathlineto{\pgfqpoint{2.375394in}{2.459077in}}%
\pgfpathlineto{\pgfqpoint{2.397244in}{2.458910in}}%
\pgfpathlineto{\pgfqpoint{2.419094in}{2.459197in}}%
\pgfpathlineto{\pgfqpoint{2.440945in}{2.459517in}}%
\pgfpathlineto{\pgfqpoint{2.462795in}{2.459836in}}%
\pgfpathlineto{\pgfqpoint{2.484645in}{2.459869in}}%
\pgfpathlineto{\pgfqpoint{2.506496in}{2.460223in}}%
\pgfpathlineto{\pgfqpoint{2.528346in}{2.460557in}}%
\pgfpathlineto{\pgfqpoint{2.550196in}{2.460926in}}%
\pgfpathlineto{\pgfqpoint{2.572046in}{2.461169in}}%
\pgfpathlineto{\pgfqpoint{2.593897in}{2.461725in}}%
\pgfpathlineto{\pgfqpoint{2.615747in}{2.461592in}}%
\pgfpathlineto{\pgfqpoint{2.637597in}{2.461716in}}%
\pgfpathlineto{\pgfqpoint{2.659448in}{2.461894in}}%
\pgfpathlineto{\pgfqpoint{2.681298in}{2.462350in}}%
\pgfpathlineto{\pgfqpoint{2.703148in}{2.462473in}}%
\pgfusepath{stroke}%
\end{pgfscope}%
\begin{pgfscope}%
\pgfpathrectangle{\pgfqpoint{0.539970in}{2.347992in}}{\pgfqpoint{2.206879in}{1.226201in}}%
\pgfusepath{clip}%
\pgfsetrectcap%
\pgfsetroundjoin%
\pgfsetlinewidth{1.003750pt}%
\definecolor{currentstroke}{rgb}{0.000000,0.725490,0.270588}%
\pgfsetstrokecolor{currentstroke}%
\pgfsetdash{}{0pt}%
\pgfpathmoveto{\pgfqpoint{0.561820in}{2.698130in}}%
\pgfpathlineto{\pgfqpoint{0.583670in}{2.679752in}}%
\pgfpathlineto{\pgfqpoint{0.605521in}{2.673149in}}%
\pgfpathlineto{\pgfqpoint{0.627371in}{2.669146in}}%
\pgfpathlineto{\pgfqpoint{0.649221in}{2.670551in}}%
\pgfpathlineto{\pgfqpoint{0.671072in}{2.670778in}}%
\pgfpathlineto{\pgfqpoint{0.692922in}{2.669817in}}%
\pgfpathlineto{\pgfqpoint{0.714772in}{2.669936in}}%
\pgfpathlineto{\pgfqpoint{0.736622in}{2.668827in}}%
\pgfpathlineto{\pgfqpoint{0.758473in}{2.664663in}}%
\pgfpathlineto{\pgfqpoint{0.780323in}{2.661200in}}%
\pgfpathlineto{\pgfqpoint{0.802173in}{2.657800in}}%
\pgfpathlineto{\pgfqpoint{0.824024in}{2.657155in}}%
\pgfpathlineto{\pgfqpoint{0.845874in}{2.655240in}}%
\pgfpathlineto{\pgfqpoint{0.867724in}{2.654918in}}%
\pgfpathlineto{\pgfqpoint{0.889574in}{2.654338in}}%
\pgfpathlineto{\pgfqpoint{0.911425in}{2.651853in}}%
\pgfpathlineto{\pgfqpoint{0.933275in}{2.649286in}}%
\pgfpathlineto{\pgfqpoint{0.955125in}{2.646908in}}%
\pgfpathlineto{\pgfqpoint{0.976976in}{2.644428in}}%
\pgfpathlineto{\pgfqpoint{0.998826in}{2.642738in}}%
\pgfpathlineto{\pgfqpoint{1.020676in}{2.641374in}}%
\pgfpathlineto{\pgfqpoint{1.042526in}{2.641518in}}%
\pgfpathlineto{\pgfqpoint{1.064377in}{2.641116in}}%
\pgfpathlineto{\pgfqpoint{1.086227in}{2.641048in}}%
\pgfpathlineto{\pgfqpoint{1.108077in}{2.639979in}}%
\pgfpathlineto{\pgfqpoint{1.129928in}{2.638528in}}%
\pgfpathlineto{\pgfqpoint{1.151778in}{2.638484in}}%
\pgfpathlineto{\pgfqpoint{1.173628in}{2.637621in}}%
\pgfpathlineto{\pgfqpoint{1.195478in}{2.637791in}}%
\pgfpathlineto{\pgfqpoint{1.217329in}{2.637221in}}%
\pgfpathlineto{\pgfqpoint{1.239179in}{2.637082in}}%
\pgfpathlineto{\pgfqpoint{1.261029in}{2.635637in}}%
\pgfpathlineto{\pgfqpoint{1.282880in}{2.634147in}}%
\pgfpathlineto{\pgfqpoint{1.304730in}{2.633044in}}%
\pgfpathlineto{\pgfqpoint{1.326580in}{2.632983in}}%
\pgfpathlineto{\pgfqpoint{1.348430in}{2.633002in}}%
\pgfpathlineto{\pgfqpoint{1.370281in}{2.631569in}}%
\pgfpathlineto{\pgfqpoint{1.392131in}{2.630668in}}%
\pgfpathlineto{\pgfqpoint{1.413981in}{2.630042in}}%
\pgfpathlineto{\pgfqpoint{1.435832in}{2.629065in}}%
\pgfpathlineto{\pgfqpoint{1.457682in}{2.629095in}}%
\pgfpathlineto{\pgfqpoint{1.479532in}{2.628945in}}%
\pgfpathlineto{\pgfqpoint{1.501382in}{2.627710in}}%
\pgfpathlineto{\pgfqpoint{1.523233in}{2.626844in}}%
\pgfpathlineto{\pgfqpoint{1.545083in}{2.626258in}}%
\pgfpathlineto{\pgfqpoint{1.566933in}{2.625786in}}%
\pgfpathlineto{\pgfqpoint{1.588784in}{2.624391in}}%
\pgfpathlineto{\pgfqpoint{1.610634in}{2.623678in}}%
\pgfpathlineto{\pgfqpoint{1.632484in}{2.623477in}}%
\pgfpathlineto{\pgfqpoint{1.654334in}{2.623606in}}%
\pgfpathlineto{\pgfqpoint{1.676185in}{2.623380in}}%
\pgfpathlineto{\pgfqpoint{1.698035in}{2.622933in}}%
\pgfpathlineto{\pgfqpoint{1.719885in}{2.623133in}}%
\pgfpathlineto{\pgfqpoint{1.741736in}{2.622722in}}%
\pgfpathlineto{\pgfqpoint{1.763586in}{2.622295in}}%
\pgfpathlineto{\pgfqpoint{1.785436in}{2.622353in}}%
\pgfpathlineto{\pgfqpoint{1.807286in}{2.621797in}}%
\pgfpathlineto{\pgfqpoint{1.829137in}{2.621760in}}%
\pgfpathlineto{\pgfqpoint{1.850987in}{2.621728in}}%
\pgfpathlineto{\pgfqpoint{1.872837in}{2.620842in}}%
\pgfpathlineto{\pgfqpoint{1.894688in}{2.619988in}}%
\pgfpathlineto{\pgfqpoint{1.916538in}{2.619978in}}%
\pgfpathlineto{\pgfqpoint{1.938388in}{2.619717in}}%
\pgfpathlineto{\pgfqpoint{1.960238in}{2.619019in}}%
\pgfpathlineto{\pgfqpoint{1.982089in}{2.618601in}}%
\pgfpathlineto{\pgfqpoint{2.003939in}{2.618518in}}%
\pgfpathlineto{\pgfqpoint{2.025789in}{2.617815in}}%
\pgfpathlineto{\pgfqpoint{2.047640in}{2.617076in}}%
\pgfpathlineto{\pgfqpoint{2.069490in}{2.617071in}}%
\pgfpathlineto{\pgfqpoint{2.091340in}{2.616748in}}%
\pgfpathlineto{\pgfqpoint{2.113190in}{2.616513in}}%
\pgfpathlineto{\pgfqpoint{2.135041in}{2.616150in}}%
\pgfpathlineto{\pgfqpoint{2.156891in}{2.615782in}}%
\pgfpathlineto{\pgfqpoint{2.178741in}{2.615504in}}%
\pgfpathlineto{\pgfqpoint{2.200592in}{2.614759in}}%
\pgfpathlineto{\pgfqpoint{2.222442in}{2.614624in}}%
\pgfpathlineto{\pgfqpoint{2.244292in}{2.614296in}}%
\pgfpathlineto{\pgfqpoint{2.266142in}{2.613614in}}%
\pgfpathlineto{\pgfqpoint{2.287993in}{2.613545in}}%
\pgfpathlineto{\pgfqpoint{2.309843in}{2.613187in}}%
\pgfpathlineto{\pgfqpoint{2.331693in}{2.613353in}}%
\pgfpathlineto{\pgfqpoint{2.353544in}{2.613008in}}%
\pgfpathlineto{\pgfqpoint{2.375394in}{2.612343in}}%
\pgfpathlineto{\pgfqpoint{2.397244in}{2.612257in}}%
\pgfpathlineto{\pgfqpoint{2.419094in}{2.612101in}}%
\pgfpathlineto{\pgfqpoint{2.440945in}{2.611930in}}%
\pgfpathlineto{\pgfqpoint{2.462795in}{2.611618in}}%
\pgfpathlineto{\pgfqpoint{2.484645in}{2.611339in}}%
\pgfpathlineto{\pgfqpoint{2.506496in}{2.611139in}}%
\pgfpathlineto{\pgfqpoint{2.528346in}{2.610845in}}%
\pgfpathlineto{\pgfqpoint{2.550196in}{2.609991in}}%
\pgfpathlineto{\pgfqpoint{2.572046in}{2.609513in}}%
\pgfpathlineto{\pgfqpoint{2.593897in}{2.609041in}}%
\pgfpathlineto{\pgfqpoint{2.615747in}{2.608661in}}%
\pgfpathlineto{\pgfqpoint{2.637597in}{2.608441in}}%
\pgfpathlineto{\pgfqpoint{2.659448in}{2.608141in}}%
\pgfpathlineto{\pgfqpoint{2.681298in}{2.608139in}}%
\pgfpathlineto{\pgfqpoint{2.703148in}{2.607985in}}%
\pgfusepath{stroke}%
\end{pgfscope}%
\begin{pgfscope}%
\pgfpathrectangle{\pgfqpoint{0.539970in}{2.347992in}}{\pgfqpoint{2.206879in}{1.226201in}}%
\pgfusepath{clip}%
\pgfsetrectcap%
\pgfsetroundjoin%
\pgfsetlinewidth{1.003750pt}%
\definecolor{currentstroke}{rgb}{1.000000,0.584314,0.000000}%
\pgfsetstrokecolor{currentstroke}%
\pgfsetdash{}{0pt}%
\pgfpathmoveto{\pgfqpoint{0.561820in}{2.403729in}}%
\pgfpathlineto{\pgfqpoint{0.583670in}{2.411921in}}%
\pgfpathlineto{\pgfqpoint{0.605521in}{2.431573in}}%
\pgfpathlineto{\pgfqpoint{0.627371in}{2.435588in}}%
\pgfpathlineto{\pgfqpoint{0.649221in}{2.436035in}}%
\pgfpathlineto{\pgfqpoint{0.671072in}{2.437272in}}%
\pgfpathlineto{\pgfqpoint{0.692922in}{2.436757in}}%
\pgfpathlineto{\pgfqpoint{0.714772in}{2.437015in}}%
\pgfpathlineto{\pgfqpoint{0.736622in}{2.438991in}}%
\pgfpathlineto{\pgfqpoint{0.758473in}{2.438666in}}%
\pgfpathlineto{\pgfqpoint{0.780323in}{2.436112in}}%
\pgfpathlineto{\pgfqpoint{0.802173in}{2.434010in}}%
\pgfpathlineto{\pgfqpoint{0.824024in}{2.434975in}}%
\pgfpathlineto{\pgfqpoint{0.845874in}{2.434064in}}%
\pgfpathlineto{\pgfqpoint{0.867724in}{2.435978in}}%
\pgfpathlineto{\pgfqpoint{0.889574in}{2.435425in}}%
\pgfpathlineto{\pgfqpoint{0.911425in}{2.436028in}}%
\pgfpathlineto{\pgfqpoint{0.933275in}{2.434845in}}%
\pgfpathlineto{\pgfqpoint{0.955125in}{2.436322in}}%
\pgfpathlineto{\pgfqpoint{0.976976in}{2.437457in}}%
\pgfpathlineto{\pgfqpoint{0.998826in}{2.437816in}}%
\pgfpathlineto{\pgfqpoint{1.020676in}{2.437780in}}%
\pgfpathlineto{\pgfqpoint{1.042526in}{2.436459in}}%
\pgfpathlineto{\pgfqpoint{1.064377in}{2.436139in}}%
\pgfpathlineto{\pgfqpoint{1.086227in}{2.435333in}}%
\pgfpathlineto{\pgfqpoint{1.108077in}{2.435706in}}%
\pgfpathlineto{\pgfqpoint{1.129928in}{2.435452in}}%
\pgfpathlineto{\pgfqpoint{1.151778in}{2.435765in}}%
\pgfpathlineto{\pgfqpoint{1.173628in}{2.436867in}}%
\pgfpathlineto{\pgfqpoint{1.195478in}{2.436907in}}%
\pgfpathlineto{\pgfqpoint{1.217329in}{2.435259in}}%
\pgfpathlineto{\pgfqpoint{1.239179in}{2.435761in}}%
\pgfpathlineto{\pgfqpoint{1.261029in}{2.435693in}}%
\pgfpathlineto{\pgfqpoint{1.282880in}{2.436173in}}%
\pgfpathlineto{\pgfqpoint{1.304730in}{2.436667in}}%
\pgfpathlineto{\pgfqpoint{1.326580in}{2.436846in}}%
\pgfpathlineto{\pgfqpoint{1.348430in}{2.437326in}}%
\pgfpathlineto{\pgfqpoint{1.370281in}{2.438063in}}%
\pgfpathlineto{\pgfqpoint{1.392131in}{2.437171in}}%
\pgfpathlineto{\pgfqpoint{1.413981in}{2.436583in}}%
\pgfpathlineto{\pgfqpoint{1.435832in}{2.435758in}}%
\pgfpathlineto{\pgfqpoint{1.457682in}{2.435502in}}%
\pgfpathlineto{\pgfqpoint{1.479532in}{2.435768in}}%
\pgfpathlineto{\pgfqpoint{1.501382in}{2.436428in}}%
\pgfpathlineto{\pgfqpoint{1.523233in}{2.437356in}}%
\pgfpathlineto{\pgfqpoint{1.545083in}{2.437393in}}%
\pgfpathlineto{\pgfqpoint{1.566933in}{2.437477in}}%
\pgfpathlineto{\pgfqpoint{1.588784in}{2.437041in}}%
\pgfpathlineto{\pgfqpoint{1.610634in}{2.437112in}}%
\pgfpathlineto{\pgfqpoint{1.632484in}{2.437681in}}%
\pgfpathlineto{\pgfqpoint{1.654334in}{2.438389in}}%
\pgfpathlineto{\pgfqpoint{1.676185in}{2.438134in}}%
\pgfpathlineto{\pgfqpoint{1.698035in}{2.438516in}}%
\pgfpathlineto{\pgfqpoint{1.719885in}{2.439554in}}%
\pgfpathlineto{\pgfqpoint{1.741736in}{2.439616in}}%
\pgfpathlineto{\pgfqpoint{1.763586in}{2.439855in}}%
\pgfpathlineto{\pgfqpoint{1.785436in}{2.439702in}}%
\pgfpathlineto{\pgfqpoint{1.807286in}{2.439255in}}%
\pgfpathlineto{\pgfqpoint{1.829137in}{2.439244in}}%
\pgfpathlineto{\pgfqpoint{1.850987in}{2.439980in}}%
\pgfpathlineto{\pgfqpoint{1.872837in}{2.440059in}}%
\pgfpathlineto{\pgfqpoint{1.894688in}{2.440393in}}%
\pgfpathlineto{\pgfqpoint{1.916538in}{2.439976in}}%
\pgfpathlineto{\pgfqpoint{1.938388in}{2.440246in}}%
\pgfpathlineto{\pgfqpoint{1.960238in}{2.440814in}}%
\pgfpathlineto{\pgfqpoint{1.982089in}{2.441422in}}%
\pgfpathlineto{\pgfqpoint{2.003939in}{2.441511in}}%
\pgfpathlineto{\pgfqpoint{2.025789in}{2.441049in}}%
\pgfpathlineto{\pgfqpoint{2.047640in}{2.441374in}}%
\pgfpathlineto{\pgfqpoint{2.069490in}{2.441600in}}%
\pgfpathlineto{\pgfqpoint{2.091340in}{2.441516in}}%
\pgfpathlineto{\pgfqpoint{2.113190in}{2.441880in}}%
\pgfpathlineto{\pgfqpoint{2.135041in}{2.442231in}}%
\pgfpathlineto{\pgfqpoint{2.156891in}{2.442457in}}%
\pgfpathlineto{\pgfqpoint{2.178741in}{2.442731in}}%
\pgfpathlineto{\pgfqpoint{2.200592in}{2.443199in}}%
\pgfpathlineto{\pgfqpoint{2.222442in}{2.443704in}}%
\pgfpathlineto{\pgfqpoint{2.244292in}{2.444260in}}%
\pgfpathlineto{\pgfqpoint{2.266142in}{2.444045in}}%
\pgfpathlineto{\pgfqpoint{2.287993in}{2.444572in}}%
\pgfpathlineto{\pgfqpoint{2.309843in}{2.444870in}}%
\pgfpathlineto{\pgfqpoint{2.331693in}{2.445543in}}%
\pgfpathlineto{\pgfqpoint{2.353544in}{2.445851in}}%
\pgfpathlineto{\pgfqpoint{2.375394in}{2.446048in}}%
\pgfpathlineto{\pgfqpoint{2.397244in}{2.446342in}}%
\pgfpathlineto{\pgfqpoint{2.419094in}{2.446375in}}%
\pgfpathlineto{\pgfqpoint{2.440945in}{2.446845in}}%
\pgfpathlineto{\pgfqpoint{2.462795in}{2.447103in}}%
\pgfpathlineto{\pgfqpoint{2.484645in}{2.447057in}}%
\pgfpathlineto{\pgfqpoint{2.506496in}{2.447250in}}%
\pgfpathlineto{\pgfqpoint{2.528346in}{2.447543in}}%
\pgfpathlineto{\pgfqpoint{2.550196in}{2.448402in}}%
\pgfpathlineto{\pgfqpoint{2.572046in}{2.448619in}}%
\pgfpathlineto{\pgfqpoint{2.593897in}{2.448936in}}%
\pgfpathlineto{\pgfqpoint{2.615747in}{2.449359in}}%
\pgfpathlineto{\pgfqpoint{2.637597in}{2.449507in}}%
\pgfpathlineto{\pgfqpoint{2.659448in}{2.449788in}}%
\pgfpathlineto{\pgfqpoint{2.681298in}{2.450055in}}%
\pgfpathlineto{\pgfqpoint{2.703148in}{2.450154in}}%
\pgfusepath{stroke}%
\end{pgfscope}%
\begin{pgfscope}%
\pgfpathrectangle{\pgfqpoint{0.539970in}{2.347992in}}{\pgfqpoint{2.206879in}{1.226201in}}%
\pgfusepath{clip}%
\pgfsetrectcap%
\pgfsetroundjoin%
\pgfsetlinewidth{1.003750pt}%
\definecolor{currentstroke}{rgb}{1.000000,0.172549,0.000000}%
\pgfsetstrokecolor{currentstroke}%
\pgfsetdash{}{0pt}%
\pgfpathmoveto{\pgfqpoint{0.561820in}{3.518457in}}%
\pgfpathlineto{\pgfqpoint{0.583670in}{3.517863in}}%
\pgfpathlineto{\pgfqpoint{0.605521in}{3.517274in}}%
\pgfpathlineto{\pgfqpoint{0.627371in}{3.516621in}}%
\pgfpathlineto{\pgfqpoint{0.649221in}{3.516010in}}%
\pgfpathlineto{\pgfqpoint{0.671072in}{3.515441in}}%
\pgfpathlineto{\pgfqpoint{0.692922in}{3.514877in}}%
\pgfpathlineto{\pgfqpoint{0.714772in}{3.514302in}}%
\pgfpathlineto{\pgfqpoint{0.736622in}{3.513719in}}%
\pgfpathlineto{\pgfqpoint{0.758473in}{3.513140in}}%
\pgfpathlineto{\pgfqpoint{0.780323in}{3.512586in}}%
\pgfpathlineto{\pgfqpoint{0.802173in}{3.512042in}}%
\pgfpathlineto{\pgfqpoint{0.824024in}{3.511514in}}%
\pgfpathlineto{\pgfqpoint{0.845874in}{3.510966in}}%
\pgfpathlineto{\pgfqpoint{0.867724in}{3.510397in}}%
\pgfpathlineto{\pgfqpoint{0.889574in}{3.509889in}}%
\pgfpathlineto{\pgfqpoint{0.911425in}{3.509327in}}%
\pgfpathlineto{\pgfqpoint{0.933275in}{3.508697in}}%
\pgfpathlineto{\pgfqpoint{0.955125in}{3.508166in}}%
\pgfpathlineto{\pgfqpoint{0.976976in}{3.507614in}}%
\pgfpathlineto{\pgfqpoint{0.998826in}{3.507011in}}%
\pgfpathlineto{\pgfqpoint{1.020676in}{3.506455in}}%
\pgfpathlineto{\pgfqpoint{1.042526in}{3.505910in}}%
\pgfpathlineto{\pgfqpoint{1.064377in}{3.505329in}}%
\pgfpathlineto{\pgfqpoint{1.086227in}{3.504779in}}%
\pgfpathlineto{\pgfqpoint{1.108077in}{3.504247in}}%
\pgfpathlineto{\pgfqpoint{1.129928in}{3.503683in}}%
\pgfpathlineto{\pgfqpoint{1.151778in}{3.503152in}}%
\pgfpathlineto{\pgfqpoint{1.173628in}{3.502609in}}%
\pgfpathlineto{\pgfqpoint{1.195478in}{3.502095in}}%
\pgfpathlineto{\pgfqpoint{1.217329in}{3.501531in}}%
\pgfpathlineto{\pgfqpoint{1.239179in}{3.500982in}}%
\pgfpathlineto{\pgfqpoint{1.261029in}{3.500436in}}%
\pgfpathlineto{\pgfqpoint{1.282880in}{3.499911in}}%
\pgfpathlineto{\pgfqpoint{1.304730in}{3.499351in}}%
\pgfpathlineto{\pgfqpoint{1.326580in}{3.498813in}}%
\pgfpathlineto{\pgfqpoint{1.348430in}{3.498324in}}%
\pgfpathlineto{\pgfqpoint{1.370281in}{3.497801in}}%
\pgfpathlineto{\pgfqpoint{1.392131in}{3.497284in}}%
\pgfpathlineto{\pgfqpoint{1.413981in}{3.496761in}}%
\pgfpathlineto{\pgfqpoint{1.435832in}{3.496211in}}%
\pgfpathlineto{\pgfqpoint{1.457682in}{3.495680in}}%
\pgfpathlineto{\pgfqpoint{1.479532in}{3.495150in}}%
\pgfpathlineto{\pgfqpoint{1.501382in}{3.494641in}}%
\pgfpathlineto{\pgfqpoint{1.523233in}{3.494125in}}%
\pgfpathlineto{\pgfqpoint{1.545083in}{3.493617in}}%
\pgfpathlineto{\pgfqpoint{1.566933in}{3.493071in}}%
\pgfpathlineto{\pgfqpoint{1.588784in}{3.492578in}}%
\pgfpathlineto{\pgfqpoint{1.610634in}{3.492036in}}%
\pgfpathlineto{\pgfqpoint{1.632484in}{3.491518in}}%
\pgfpathlineto{\pgfqpoint{1.654334in}{3.490984in}}%
\pgfpathlineto{\pgfqpoint{1.676185in}{3.490439in}}%
\pgfpathlineto{\pgfqpoint{1.698035in}{3.489950in}}%
\pgfpathlineto{\pgfqpoint{1.719885in}{3.489446in}}%
\pgfpathlineto{\pgfqpoint{1.741736in}{3.488892in}}%
\pgfpathlineto{\pgfqpoint{1.763586in}{3.488369in}}%
\pgfpathlineto{\pgfqpoint{1.785436in}{3.487854in}}%
\pgfpathlineto{\pgfqpoint{1.807286in}{3.487306in}}%
\pgfpathlineto{\pgfqpoint{1.829137in}{3.486815in}}%
\pgfpathlineto{\pgfqpoint{1.850987in}{3.486330in}}%
\pgfpathlineto{\pgfqpoint{1.872837in}{3.485808in}}%
\pgfpathlineto{\pgfqpoint{1.894688in}{3.485322in}}%
\pgfpathlineto{\pgfqpoint{1.916538in}{3.484800in}}%
\pgfpathlineto{\pgfqpoint{1.938388in}{3.484263in}}%
\pgfpathlineto{\pgfqpoint{1.960238in}{3.483756in}}%
\pgfpathlineto{\pgfqpoint{1.982089in}{3.483234in}}%
\pgfpathlineto{\pgfqpoint{2.003939in}{3.482721in}}%
\pgfpathlineto{\pgfqpoint{2.025789in}{3.482154in}}%
\pgfpathlineto{\pgfqpoint{2.047640in}{3.481591in}}%
\pgfpathlineto{\pgfqpoint{2.069490in}{3.481053in}}%
\pgfpathlineto{\pgfqpoint{2.091340in}{3.480532in}}%
\pgfpathlineto{\pgfqpoint{2.113190in}{3.480014in}}%
\pgfpathlineto{\pgfqpoint{2.135041in}{3.479513in}}%
\pgfpathlineto{\pgfqpoint{2.156891in}{3.478970in}}%
\pgfpathlineto{\pgfqpoint{2.178741in}{3.478448in}}%
\pgfpathlineto{\pgfqpoint{2.200592in}{3.477922in}}%
\pgfpathlineto{\pgfqpoint{2.222442in}{3.477405in}}%
\pgfpathlineto{\pgfqpoint{2.244292in}{3.476834in}}%
\pgfpathlineto{\pgfqpoint{2.266142in}{3.476329in}}%
\pgfpathlineto{\pgfqpoint{2.287993in}{3.475777in}}%
\pgfpathlineto{\pgfqpoint{2.309843in}{3.475226in}}%
\pgfpathlineto{\pgfqpoint{2.331693in}{3.474678in}}%
\pgfpathlineto{\pgfqpoint{2.353544in}{3.474111in}}%
\pgfpathlineto{\pgfqpoint{2.375394in}{3.473580in}}%
\pgfpathlineto{\pgfqpoint{2.397244in}{3.472982in}}%
\pgfpathlineto{\pgfqpoint{2.419094in}{3.472406in}}%
\pgfpathlineto{\pgfqpoint{2.440945in}{3.471907in}}%
\pgfpathlineto{\pgfqpoint{2.462795in}{3.471334in}}%
\pgfpathlineto{\pgfqpoint{2.484645in}{3.470764in}}%
\pgfpathlineto{\pgfqpoint{2.506496in}{3.470195in}}%
\pgfpathlineto{\pgfqpoint{2.528346in}{3.469675in}}%
\pgfpathlineto{\pgfqpoint{2.550196in}{3.469104in}}%
\pgfpathlineto{\pgfqpoint{2.572046in}{3.468547in}}%
\pgfpathlineto{\pgfqpoint{2.593897in}{3.467977in}}%
\pgfpathlineto{\pgfqpoint{2.615747in}{3.467397in}}%
\pgfpathlineto{\pgfqpoint{2.637597in}{3.466851in}}%
\pgfpathlineto{\pgfqpoint{2.659448in}{3.466304in}}%
\pgfpathlineto{\pgfqpoint{2.681298in}{3.465745in}}%
\pgfpathlineto{\pgfqpoint{2.703148in}{3.465161in}}%
\pgfusepath{stroke}%
\end{pgfscope}%
\begin{pgfscope}%
\pgfpathrectangle{\pgfqpoint{0.539970in}{2.347992in}}{\pgfqpoint{2.206879in}{1.226201in}}%
\pgfusepath{clip}%
\pgfsetrectcap%
\pgfsetroundjoin%
\pgfsetlinewidth{1.003750pt}%
\definecolor{currentstroke}{rgb}{0.517647,0.356863,0.592157}%
\pgfsetstrokecolor{currentstroke}%
\pgfsetdash{}{0pt}%
\pgfpathmoveto{\pgfqpoint{0.561820in}{2.428160in}}%
\pgfpathlineto{\pgfqpoint{0.583670in}{2.429055in}}%
\pgfpathlineto{\pgfqpoint{0.605521in}{2.432769in}}%
\pgfpathlineto{\pgfqpoint{0.627371in}{2.432597in}}%
\pgfpathlineto{\pgfqpoint{0.649221in}{2.431130in}}%
\pgfpathlineto{\pgfqpoint{0.671072in}{2.434122in}}%
\pgfpathlineto{\pgfqpoint{0.692922in}{2.436818in}}%
\pgfpathlineto{\pgfqpoint{0.714772in}{2.433341in}}%
\pgfpathlineto{\pgfqpoint{0.736622in}{2.430216in}}%
\pgfpathlineto{\pgfqpoint{0.758473in}{2.430831in}}%
\pgfpathlineto{\pgfqpoint{0.780323in}{2.431723in}}%
\pgfpathlineto{\pgfqpoint{0.802173in}{2.432128in}}%
\pgfpathlineto{\pgfqpoint{0.824024in}{2.434563in}}%
\pgfpathlineto{\pgfqpoint{0.845874in}{2.434598in}}%
\pgfpathlineto{\pgfqpoint{0.867724in}{2.435575in}}%
\pgfpathlineto{\pgfqpoint{0.889574in}{2.437361in}}%
\pgfpathlineto{\pgfqpoint{0.911425in}{2.436315in}}%
\pgfpathlineto{\pgfqpoint{0.933275in}{2.435580in}}%
\pgfpathlineto{\pgfqpoint{0.955125in}{2.435744in}}%
\pgfpathlineto{\pgfqpoint{0.976976in}{2.436269in}}%
\pgfpathlineto{\pgfqpoint{0.998826in}{2.435983in}}%
\pgfpathlineto{\pgfqpoint{1.020676in}{2.436228in}}%
\pgfpathlineto{\pgfqpoint{1.042526in}{2.434976in}}%
\pgfpathlineto{\pgfqpoint{1.064377in}{2.434793in}}%
\pgfpathlineto{\pgfqpoint{1.086227in}{2.433583in}}%
\pgfpathlineto{\pgfqpoint{1.108077in}{2.433832in}}%
\pgfpathlineto{\pgfqpoint{1.129928in}{2.433628in}}%
\pgfpathlineto{\pgfqpoint{1.151778in}{2.433700in}}%
\pgfpathlineto{\pgfqpoint{1.173628in}{2.434534in}}%
\pgfpathlineto{\pgfqpoint{1.195478in}{2.434091in}}%
\pgfpathlineto{\pgfqpoint{1.217329in}{2.434614in}}%
\pgfpathlineto{\pgfqpoint{1.239179in}{2.434144in}}%
\pgfpathlineto{\pgfqpoint{1.261029in}{2.433349in}}%
\pgfpathlineto{\pgfqpoint{1.282880in}{2.434297in}}%
\pgfpathlineto{\pgfqpoint{1.304730in}{2.434391in}}%
\pgfpathlineto{\pgfqpoint{1.326580in}{2.433303in}}%
\pgfpathlineto{\pgfqpoint{1.348430in}{2.433466in}}%
\pgfpathlineto{\pgfqpoint{1.370281in}{2.433926in}}%
\pgfpathlineto{\pgfqpoint{1.392131in}{2.434367in}}%
\pgfpathlineto{\pgfqpoint{1.413981in}{2.434613in}}%
\pgfpathlineto{\pgfqpoint{1.435832in}{2.435439in}}%
\pgfpathlineto{\pgfqpoint{1.457682in}{2.435451in}}%
\pgfpathlineto{\pgfqpoint{1.479532in}{2.436046in}}%
\pgfpathlineto{\pgfqpoint{1.501382in}{2.435264in}}%
\pgfpathlineto{\pgfqpoint{1.523233in}{2.435179in}}%
\pgfpathlineto{\pgfqpoint{1.545083in}{2.435243in}}%
\pgfpathlineto{\pgfqpoint{1.566933in}{2.435478in}}%
\pgfpathlineto{\pgfqpoint{1.588784in}{2.435645in}}%
\pgfpathlineto{\pgfqpoint{1.610634in}{2.435794in}}%
\pgfpathlineto{\pgfqpoint{1.632484in}{2.435875in}}%
\pgfpathlineto{\pgfqpoint{1.654334in}{2.435961in}}%
\pgfpathlineto{\pgfqpoint{1.676185in}{2.436248in}}%
\pgfpathlineto{\pgfqpoint{1.698035in}{2.436513in}}%
\pgfpathlineto{\pgfqpoint{1.719885in}{2.436874in}}%
\pgfpathlineto{\pgfqpoint{1.741736in}{2.437049in}}%
\pgfpathlineto{\pgfqpoint{1.763586in}{2.437545in}}%
\pgfpathlineto{\pgfqpoint{1.785436in}{2.437902in}}%
\pgfpathlineto{\pgfqpoint{1.807286in}{2.438039in}}%
\pgfpathlineto{\pgfqpoint{1.829137in}{2.438510in}}%
\pgfpathlineto{\pgfqpoint{1.850987in}{2.438611in}}%
\pgfpathlineto{\pgfqpoint{1.872837in}{2.438706in}}%
\pgfpathlineto{\pgfqpoint{1.894688in}{2.439266in}}%
\pgfpathlineto{\pgfqpoint{1.916538in}{2.439399in}}%
\pgfpathlineto{\pgfqpoint{1.938388in}{2.440007in}}%
\pgfpathlineto{\pgfqpoint{1.960238in}{2.440198in}}%
\pgfpathlineto{\pgfqpoint{1.982089in}{2.440491in}}%
\pgfpathlineto{\pgfqpoint{2.003939in}{2.440660in}}%
\pgfpathlineto{\pgfqpoint{2.025789in}{2.440632in}}%
\pgfpathlineto{\pgfqpoint{2.047640in}{2.440875in}}%
\pgfpathlineto{\pgfqpoint{2.069490in}{2.441409in}}%
\pgfpathlineto{\pgfqpoint{2.091340in}{2.441348in}}%
\pgfpathlineto{\pgfqpoint{2.113190in}{2.441444in}}%
\pgfpathlineto{\pgfqpoint{2.135041in}{2.441601in}}%
\pgfpathlineto{\pgfqpoint{2.156891in}{2.441813in}}%
\pgfpathlineto{\pgfqpoint{2.178741in}{2.442346in}}%
\pgfpathlineto{\pgfqpoint{2.200592in}{2.442549in}}%
\pgfpathlineto{\pgfqpoint{2.222442in}{2.442847in}}%
\pgfpathlineto{\pgfqpoint{2.244292in}{2.443340in}}%
\pgfpathlineto{\pgfqpoint{2.266142in}{2.443720in}}%
\pgfpathlineto{\pgfqpoint{2.287993in}{2.443728in}}%
\pgfpathlineto{\pgfqpoint{2.309843in}{2.443965in}}%
\pgfpathlineto{\pgfqpoint{2.331693in}{2.443735in}}%
\pgfpathlineto{\pgfqpoint{2.353544in}{2.444162in}}%
\pgfpathlineto{\pgfqpoint{2.375394in}{2.443965in}}%
\pgfpathlineto{\pgfqpoint{2.397244in}{2.444325in}}%
\pgfpathlineto{\pgfqpoint{2.419094in}{2.444205in}}%
\pgfpathlineto{\pgfqpoint{2.440945in}{2.444364in}}%
\pgfpathlineto{\pgfqpoint{2.462795in}{2.444953in}}%
\pgfpathlineto{\pgfqpoint{2.484645in}{2.445244in}}%
\pgfpathlineto{\pgfqpoint{2.506496in}{2.445780in}}%
\pgfpathlineto{\pgfqpoint{2.528346in}{2.445877in}}%
\pgfpathlineto{\pgfqpoint{2.550196in}{2.446036in}}%
\pgfpathlineto{\pgfqpoint{2.572046in}{2.446190in}}%
\pgfpathlineto{\pgfqpoint{2.593897in}{2.446209in}}%
\pgfpathlineto{\pgfqpoint{2.615747in}{2.446426in}}%
\pgfpathlineto{\pgfqpoint{2.637597in}{2.446526in}}%
\pgfpathlineto{\pgfqpoint{2.659448in}{2.446452in}}%
\pgfpathlineto{\pgfqpoint{2.681298in}{2.446706in}}%
\pgfpathlineto{\pgfqpoint{2.703148in}{2.446804in}}%
\pgfusepath{stroke}%
\end{pgfscope}%
\begin{pgfscope}%
\pgfsetrectcap%
\pgfsetmiterjoin%
\pgfsetlinewidth{0.501875pt}%
\definecolor{currentstroke}{rgb}{0.000000,0.000000,0.000000}%
\pgfsetstrokecolor{currentstroke}%
\pgfsetdash{}{0pt}%
\pgfpathmoveto{\pgfqpoint{0.539970in}{2.347992in}}%
\pgfpathlineto{\pgfqpoint{0.539970in}{3.574193in}}%
\pgfusepath{stroke}%
\end{pgfscope}%
\begin{pgfscope}%
\pgfsetrectcap%
\pgfsetmiterjoin%
\pgfsetlinewidth{0.501875pt}%
\definecolor{currentstroke}{rgb}{0.000000,0.000000,0.000000}%
\pgfsetstrokecolor{currentstroke}%
\pgfsetdash{}{0pt}%
\pgfpathmoveto{\pgfqpoint{2.746849in}{2.347992in}}%
\pgfpathlineto{\pgfqpoint{2.746849in}{3.574193in}}%
\pgfusepath{stroke}%
\end{pgfscope}%
\begin{pgfscope}%
\pgfsetrectcap%
\pgfsetmiterjoin%
\pgfsetlinewidth{0.501875pt}%
\definecolor{currentstroke}{rgb}{0.000000,0.000000,0.000000}%
\pgfsetstrokecolor{currentstroke}%
\pgfsetdash{}{0pt}%
\pgfpathmoveto{\pgfqpoint{0.539970in}{2.347992in}}%
\pgfpathlineto{\pgfqpoint{2.746849in}{2.347992in}}%
\pgfusepath{stroke}%
\end{pgfscope}%
\begin{pgfscope}%
\pgfsetrectcap%
\pgfsetmiterjoin%
\pgfsetlinewidth{0.501875pt}%
\definecolor{currentstroke}{rgb}{0.000000,0.000000,0.000000}%
\pgfsetstrokecolor{currentstroke}%
\pgfsetdash{}{0pt}%
\pgfpathmoveto{\pgfqpoint{0.539970in}{3.574193in}}%
\pgfpathlineto{\pgfqpoint{2.746849in}{3.574193in}}%
\pgfusepath{stroke}%
\end{pgfscope}%
\begin{pgfscope}%
\definecolor{textcolor}{rgb}{0.000000,0.000000,0.000000}%
\pgfsetstrokecolor{textcolor}%
\pgfsetfillcolor{textcolor}%
\pgftext[x=1.643409in,y=3.657526in,,base]{\color{textcolor}\rmfamily\fontsize{12.000000}{14.400000}\selectfont Vertrauenswürdigkeit}%
\end{pgfscope}%
\begin{pgfscope}%
\pgfsetbuttcap%
\pgfsetmiterjoin%
\definecolor{currentfill}{rgb}{1.000000,1.000000,1.000000}%
\pgfsetfillcolor{currentfill}%
\pgfsetlinewidth{0.000000pt}%
\definecolor{currentstroke}{rgb}{0.000000,0.000000,0.000000}%
\pgfsetstrokecolor{currentstroke}%
\pgfsetstrokeopacity{0.000000}%
\pgfsetdash{}{0pt}%
\pgfpathmoveto{\pgfqpoint{0.539970in}{0.422992in}}%
\pgfpathlineto{\pgfqpoint{2.746849in}{0.422992in}}%
\pgfpathlineto{\pgfqpoint{2.746849in}{1.649193in}}%
\pgfpathlineto{\pgfqpoint{0.539970in}{1.649193in}}%
\pgfpathlineto{\pgfqpoint{0.539970in}{0.422992in}}%
\pgfpathclose%
\pgfusepath{fill}%
\end{pgfscope}%
\begin{pgfscope}%
\pgfsetbuttcap%
\pgfsetroundjoin%
\definecolor{currentfill}{rgb}{0.000000,0.000000,0.000000}%
\pgfsetfillcolor{currentfill}%
\pgfsetlinewidth{0.501875pt}%
\definecolor{currentstroke}{rgb}{0.000000,0.000000,0.000000}%
\pgfsetstrokecolor{currentstroke}%
\pgfsetdash{}{0pt}%
\pgfsys@defobject{currentmarker}{\pgfqpoint{0.000000in}{0.000000in}}{\pgfqpoint{0.000000in}{0.041667in}}{%
\pgfpathmoveto{\pgfqpoint{0.000000in}{0.000000in}}%
\pgfpathlineto{\pgfqpoint{0.000000in}{0.041667in}}%
\pgfusepath{stroke,fill}%
}%
\begin{pgfscope}%
\pgfsys@transformshift{0.539970in}{0.422992in}%
\pgfsys@useobject{currentmarker}{}%
\end{pgfscope}%
\end{pgfscope}%
\begin{pgfscope}%
\pgfsetbuttcap%
\pgfsetroundjoin%
\definecolor{currentfill}{rgb}{0.000000,0.000000,0.000000}%
\pgfsetfillcolor{currentfill}%
\pgfsetlinewidth{0.501875pt}%
\definecolor{currentstroke}{rgb}{0.000000,0.000000,0.000000}%
\pgfsetstrokecolor{currentstroke}%
\pgfsetdash{}{0pt}%
\pgfsys@defobject{currentmarker}{\pgfqpoint{0.000000in}{-0.041667in}}{\pgfqpoint{0.000000in}{0.000000in}}{%
\pgfpathmoveto{\pgfqpoint{0.000000in}{0.000000in}}%
\pgfpathlineto{\pgfqpoint{0.000000in}{-0.041667in}}%
\pgfusepath{stroke,fill}%
}%
\begin{pgfscope}%
\pgfsys@transformshift{0.539970in}{1.649193in}%
\pgfsys@useobject{currentmarker}{}%
\end{pgfscope}%
\end{pgfscope}%
\begin{pgfscope}%
\definecolor{textcolor}{rgb}{0.000000,0.000000,0.000000}%
\pgfsetstrokecolor{textcolor}%
\pgfsetfillcolor{textcolor}%
\pgftext[x=0.539970in,y=0.374381in,,top]{\color{textcolor}\rmfamily\fontsize{10.000000}{12.000000}\selectfont \(\displaystyle {0}\)}%
\end{pgfscope}%
\begin{pgfscope}%
\pgfsetbuttcap%
\pgfsetroundjoin%
\definecolor{currentfill}{rgb}{0.000000,0.000000,0.000000}%
\pgfsetfillcolor{currentfill}%
\pgfsetlinewidth{0.501875pt}%
\definecolor{currentstroke}{rgb}{0.000000,0.000000,0.000000}%
\pgfsetstrokecolor{currentstroke}%
\pgfsetdash{}{0pt}%
\pgfsys@defobject{currentmarker}{\pgfqpoint{0.000000in}{0.000000in}}{\pgfqpoint{0.000000in}{0.041667in}}{%
\pgfpathmoveto{\pgfqpoint{0.000000in}{0.000000in}}%
\pgfpathlineto{\pgfqpoint{0.000000in}{0.041667in}}%
\pgfusepath{stroke,fill}%
}%
\begin{pgfscope}%
\pgfsys@transformshift{0.976976in}{0.422992in}%
\pgfsys@useobject{currentmarker}{}%
\end{pgfscope}%
\end{pgfscope}%
\begin{pgfscope}%
\pgfsetbuttcap%
\pgfsetroundjoin%
\definecolor{currentfill}{rgb}{0.000000,0.000000,0.000000}%
\pgfsetfillcolor{currentfill}%
\pgfsetlinewidth{0.501875pt}%
\definecolor{currentstroke}{rgb}{0.000000,0.000000,0.000000}%
\pgfsetstrokecolor{currentstroke}%
\pgfsetdash{}{0pt}%
\pgfsys@defobject{currentmarker}{\pgfqpoint{0.000000in}{-0.041667in}}{\pgfqpoint{0.000000in}{0.000000in}}{%
\pgfpathmoveto{\pgfqpoint{0.000000in}{0.000000in}}%
\pgfpathlineto{\pgfqpoint{0.000000in}{-0.041667in}}%
\pgfusepath{stroke,fill}%
}%
\begin{pgfscope}%
\pgfsys@transformshift{0.976976in}{1.649193in}%
\pgfsys@useobject{currentmarker}{}%
\end{pgfscope}%
\end{pgfscope}%
\begin{pgfscope}%
\definecolor{textcolor}{rgb}{0.000000,0.000000,0.000000}%
\pgfsetstrokecolor{textcolor}%
\pgfsetfillcolor{textcolor}%
\pgftext[x=0.976976in,y=0.374381in,,top]{\color{textcolor}\rmfamily\fontsize{10.000000}{12.000000}\selectfont \(\displaystyle {20}\)}%
\end{pgfscope}%
\begin{pgfscope}%
\pgfsetbuttcap%
\pgfsetroundjoin%
\definecolor{currentfill}{rgb}{0.000000,0.000000,0.000000}%
\pgfsetfillcolor{currentfill}%
\pgfsetlinewidth{0.501875pt}%
\definecolor{currentstroke}{rgb}{0.000000,0.000000,0.000000}%
\pgfsetstrokecolor{currentstroke}%
\pgfsetdash{}{0pt}%
\pgfsys@defobject{currentmarker}{\pgfqpoint{0.000000in}{0.000000in}}{\pgfqpoint{0.000000in}{0.041667in}}{%
\pgfpathmoveto{\pgfqpoint{0.000000in}{0.000000in}}%
\pgfpathlineto{\pgfqpoint{0.000000in}{0.041667in}}%
\pgfusepath{stroke,fill}%
}%
\begin{pgfscope}%
\pgfsys@transformshift{1.413981in}{0.422992in}%
\pgfsys@useobject{currentmarker}{}%
\end{pgfscope}%
\end{pgfscope}%
\begin{pgfscope}%
\pgfsetbuttcap%
\pgfsetroundjoin%
\definecolor{currentfill}{rgb}{0.000000,0.000000,0.000000}%
\pgfsetfillcolor{currentfill}%
\pgfsetlinewidth{0.501875pt}%
\definecolor{currentstroke}{rgb}{0.000000,0.000000,0.000000}%
\pgfsetstrokecolor{currentstroke}%
\pgfsetdash{}{0pt}%
\pgfsys@defobject{currentmarker}{\pgfqpoint{0.000000in}{-0.041667in}}{\pgfqpoint{0.000000in}{0.000000in}}{%
\pgfpathmoveto{\pgfqpoint{0.000000in}{0.000000in}}%
\pgfpathlineto{\pgfqpoint{0.000000in}{-0.041667in}}%
\pgfusepath{stroke,fill}%
}%
\begin{pgfscope}%
\pgfsys@transformshift{1.413981in}{1.649193in}%
\pgfsys@useobject{currentmarker}{}%
\end{pgfscope}%
\end{pgfscope}%
\begin{pgfscope}%
\definecolor{textcolor}{rgb}{0.000000,0.000000,0.000000}%
\pgfsetstrokecolor{textcolor}%
\pgfsetfillcolor{textcolor}%
\pgftext[x=1.413981in,y=0.374381in,,top]{\color{textcolor}\rmfamily\fontsize{10.000000}{12.000000}\selectfont \(\displaystyle {40}\)}%
\end{pgfscope}%
\begin{pgfscope}%
\pgfsetbuttcap%
\pgfsetroundjoin%
\definecolor{currentfill}{rgb}{0.000000,0.000000,0.000000}%
\pgfsetfillcolor{currentfill}%
\pgfsetlinewidth{0.501875pt}%
\definecolor{currentstroke}{rgb}{0.000000,0.000000,0.000000}%
\pgfsetstrokecolor{currentstroke}%
\pgfsetdash{}{0pt}%
\pgfsys@defobject{currentmarker}{\pgfqpoint{0.000000in}{0.000000in}}{\pgfqpoint{0.000000in}{0.041667in}}{%
\pgfpathmoveto{\pgfqpoint{0.000000in}{0.000000in}}%
\pgfpathlineto{\pgfqpoint{0.000000in}{0.041667in}}%
\pgfusepath{stroke,fill}%
}%
\begin{pgfscope}%
\pgfsys@transformshift{1.850987in}{0.422992in}%
\pgfsys@useobject{currentmarker}{}%
\end{pgfscope}%
\end{pgfscope}%
\begin{pgfscope}%
\pgfsetbuttcap%
\pgfsetroundjoin%
\definecolor{currentfill}{rgb}{0.000000,0.000000,0.000000}%
\pgfsetfillcolor{currentfill}%
\pgfsetlinewidth{0.501875pt}%
\definecolor{currentstroke}{rgb}{0.000000,0.000000,0.000000}%
\pgfsetstrokecolor{currentstroke}%
\pgfsetdash{}{0pt}%
\pgfsys@defobject{currentmarker}{\pgfqpoint{0.000000in}{-0.041667in}}{\pgfqpoint{0.000000in}{0.000000in}}{%
\pgfpathmoveto{\pgfqpoint{0.000000in}{0.000000in}}%
\pgfpathlineto{\pgfqpoint{0.000000in}{-0.041667in}}%
\pgfusepath{stroke,fill}%
}%
\begin{pgfscope}%
\pgfsys@transformshift{1.850987in}{1.649193in}%
\pgfsys@useobject{currentmarker}{}%
\end{pgfscope}%
\end{pgfscope}%
\begin{pgfscope}%
\definecolor{textcolor}{rgb}{0.000000,0.000000,0.000000}%
\pgfsetstrokecolor{textcolor}%
\pgfsetfillcolor{textcolor}%
\pgftext[x=1.850987in,y=0.374381in,,top]{\color{textcolor}\rmfamily\fontsize{10.000000}{12.000000}\selectfont \(\displaystyle {60}\)}%
\end{pgfscope}%
\begin{pgfscope}%
\pgfsetbuttcap%
\pgfsetroundjoin%
\definecolor{currentfill}{rgb}{0.000000,0.000000,0.000000}%
\pgfsetfillcolor{currentfill}%
\pgfsetlinewidth{0.501875pt}%
\definecolor{currentstroke}{rgb}{0.000000,0.000000,0.000000}%
\pgfsetstrokecolor{currentstroke}%
\pgfsetdash{}{0pt}%
\pgfsys@defobject{currentmarker}{\pgfqpoint{0.000000in}{0.000000in}}{\pgfqpoint{0.000000in}{0.041667in}}{%
\pgfpathmoveto{\pgfqpoint{0.000000in}{0.000000in}}%
\pgfpathlineto{\pgfqpoint{0.000000in}{0.041667in}}%
\pgfusepath{stroke,fill}%
}%
\begin{pgfscope}%
\pgfsys@transformshift{2.287993in}{0.422992in}%
\pgfsys@useobject{currentmarker}{}%
\end{pgfscope}%
\end{pgfscope}%
\begin{pgfscope}%
\pgfsetbuttcap%
\pgfsetroundjoin%
\definecolor{currentfill}{rgb}{0.000000,0.000000,0.000000}%
\pgfsetfillcolor{currentfill}%
\pgfsetlinewidth{0.501875pt}%
\definecolor{currentstroke}{rgb}{0.000000,0.000000,0.000000}%
\pgfsetstrokecolor{currentstroke}%
\pgfsetdash{}{0pt}%
\pgfsys@defobject{currentmarker}{\pgfqpoint{0.000000in}{-0.041667in}}{\pgfqpoint{0.000000in}{0.000000in}}{%
\pgfpathmoveto{\pgfqpoint{0.000000in}{0.000000in}}%
\pgfpathlineto{\pgfqpoint{0.000000in}{-0.041667in}}%
\pgfusepath{stroke,fill}%
}%
\begin{pgfscope}%
\pgfsys@transformshift{2.287993in}{1.649193in}%
\pgfsys@useobject{currentmarker}{}%
\end{pgfscope}%
\end{pgfscope}%
\begin{pgfscope}%
\definecolor{textcolor}{rgb}{0.000000,0.000000,0.000000}%
\pgfsetstrokecolor{textcolor}%
\pgfsetfillcolor{textcolor}%
\pgftext[x=2.287993in,y=0.374381in,,top]{\color{textcolor}\rmfamily\fontsize{10.000000}{12.000000}\selectfont \(\displaystyle {80}\)}%
\end{pgfscope}%
\begin{pgfscope}%
\pgfsetbuttcap%
\pgfsetroundjoin%
\definecolor{currentfill}{rgb}{0.000000,0.000000,0.000000}%
\pgfsetfillcolor{currentfill}%
\pgfsetlinewidth{0.501875pt}%
\definecolor{currentstroke}{rgb}{0.000000,0.000000,0.000000}%
\pgfsetstrokecolor{currentstroke}%
\pgfsetdash{}{0pt}%
\pgfsys@defobject{currentmarker}{\pgfqpoint{0.000000in}{0.000000in}}{\pgfqpoint{0.000000in}{0.020833in}}{%
\pgfpathmoveto{\pgfqpoint{0.000000in}{0.000000in}}%
\pgfpathlineto{\pgfqpoint{0.000000in}{0.020833in}}%
\pgfusepath{stroke,fill}%
}%
\begin{pgfscope}%
\pgfsys@transformshift{0.649221in}{0.422992in}%
\pgfsys@useobject{currentmarker}{}%
\end{pgfscope}%
\end{pgfscope}%
\begin{pgfscope}%
\pgfsetbuttcap%
\pgfsetroundjoin%
\definecolor{currentfill}{rgb}{0.000000,0.000000,0.000000}%
\pgfsetfillcolor{currentfill}%
\pgfsetlinewidth{0.501875pt}%
\definecolor{currentstroke}{rgb}{0.000000,0.000000,0.000000}%
\pgfsetstrokecolor{currentstroke}%
\pgfsetdash{}{0pt}%
\pgfsys@defobject{currentmarker}{\pgfqpoint{0.000000in}{-0.020833in}}{\pgfqpoint{0.000000in}{0.000000in}}{%
\pgfpathmoveto{\pgfqpoint{0.000000in}{0.000000in}}%
\pgfpathlineto{\pgfqpoint{0.000000in}{-0.020833in}}%
\pgfusepath{stroke,fill}%
}%
\begin{pgfscope}%
\pgfsys@transformshift{0.649221in}{1.649193in}%
\pgfsys@useobject{currentmarker}{}%
\end{pgfscope}%
\end{pgfscope}%
\begin{pgfscope}%
\pgfsetbuttcap%
\pgfsetroundjoin%
\definecolor{currentfill}{rgb}{0.000000,0.000000,0.000000}%
\pgfsetfillcolor{currentfill}%
\pgfsetlinewidth{0.501875pt}%
\definecolor{currentstroke}{rgb}{0.000000,0.000000,0.000000}%
\pgfsetstrokecolor{currentstroke}%
\pgfsetdash{}{0pt}%
\pgfsys@defobject{currentmarker}{\pgfqpoint{0.000000in}{0.000000in}}{\pgfqpoint{0.000000in}{0.020833in}}{%
\pgfpathmoveto{\pgfqpoint{0.000000in}{0.000000in}}%
\pgfpathlineto{\pgfqpoint{0.000000in}{0.020833in}}%
\pgfusepath{stroke,fill}%
}%
\begin{pgfscope}%
\pgfsys@transformshift{0.758473in}{0.422992in}%
\pgfsys@useobject{currentmarker}{}%
\end{pgfscope}%
\end{pgfscope}%
\begin{pgfscope}%
\pgfsetbuttcap%
\pgfsetroundjoin%
\definecolor{currentfill}{rgb}{0.000000,0.000000,0.000000}%
\pgfsetfillcolor{currentfill}%
\pgfsetlinewidth{0.501875pt}%
\definecolor{currentstroke}{rgb}{0.000000,0.000000,0.000000}%
\pgfsetstrokecolor{currentstroke}%
\pgfsetdash{}{0pt}%
\pgfsys@defobject{currentmarker}{\pgfqpoint{0.000000in}{-0.020833in}}{\pgfqpoint{0.000000in}{0.000000in}}{%
\pgfpathmoveto{\pgfqpoint{0.000000in}{0.000000in}}%
\pgfpathlineto{\pgfqpoint{0.000000in}{-0.020833in}}%
\pgfusepath{stroke,fill}%
}%
\begin{pgfscope}%
\pgfsys@transformshift{0.758473in}{1.649193in}%
\pgfsys@useobject{currentmarker}{}%
\end{pgfscope}%
\end{pgfscope}%
\begin{pgfscope}%
\pgfsetbuttcap%
\pgfsetroundjoin%
\definecolor{currentfill}{rgb}{0.000000,0.000000,0.000000}%
\pgfsetfillcolor{currentfill}%
\pgfsetlinewidth{0.501875pt}%
\definecolor{currentstroke}{rgb}{0.000000,0.000000,0.000000}%
\pgfsetstrokecolor{currentstroke}%
\pgfsetdash{}{0pt}%
\pgfsys@defobject{currentmarker}{\pgfqpoint{0.000000in}{0.000000in}}{\pgfqpoint{0.000000in}{0.020833in}}{%
\pgfpathmoveto{\pgfqpoint{0.000000in}{0.000000in}}%
\pgfpathlineto{\pgfqpoint{0.000000in}{0.020833in}}%
\pgfusepath{stroke,fill}%
}%
\begin{pgfscope}%
\pgfsys@transformshift{0.867724in}{0.422992in}%
\pgfsys@useobject{currentmarker}{}%
\end{pgfscope}%
\end{pgfscope}%
\begin{pgfscope}%
\pgfsetbuttcap%
\pgfsetroundjoin%
\definecolor{currentfill}{rgb}{0.000000,0.000000,0.000000}%
\pgfsetfillcolor{currentfill}%
\pgfsetlinewidth{0.501875pt}%
\definecolor{currentstroke}{rgb}{0.000000,0.000000,0.000000}%
\pgfsetstrokecolor{currentstroke}%
\pgfsetdash{}{0pt}%
\pgfsys@defobject{currentmarker}{\pgfqpoint{0.000000in}{-0.020833in}}{\pgfqpoint{0.000000in}{0.000000in}}{%
\pgfpathmoveto{\pgfqpoint{0.000000in}{0.000000in}}%
\pgfpathlineto{\pgfqpoint{0.000000in}{-0.020833in}}%
\pgfusepath{stroke,fill}%
}%
\begin{pgfscope}%
\pgfsys@transformshift{0.867724in}{1.649193in}%
\pgfsys@useobject{currentmarker}{}%
\end{pgfscope}%
\end{pgfscope}%
\begin{pgfscope}%
\pgfsetbuttcap%
\pgfsetroundjoin%
\definecolor{currentfill}{rgb}{0.000000,0.000000,0.000000}%
\pgfsetfillcolor{currentfill}%
\pgfsetlinewidth{0.501875pt}%
\definecolor{currentstroke}{rgb}{0.000000,0.000000,0.000000}%
\pgfsetstrokecolor{currentstroke}%
\pgfsetdash{}{0pt}%
\pgfsys@defobject{currentmarker}{\pgfqpoint{0.000000in}{0.000000in}}{\pgfqpoint{0.000000in}{0.020833in}}{%
\pgfpathmoveto{\pgfqpoint{0.000000in}{0.000000in}}%
\pgfpathlineto{\pgfqpoint{0.000000in}{0.020833in}}%
\pgfusepath{stroke,fill}%
}%
\begin{pgfscope}%
\pgfsys@transformshift{1.086227in}{0.422992in}%
\pgfsys@useobject{currentmarker}{}%
\end{pgfscope}%
\end{pgfscope}%
\begin{pgfscope}%
\pgfsetbuttcap%
\pgfsetroundjoin%
\definecolor{currentfill}{rgb}{0.000000,0.000000,0.000000}%
\pgfsetfillcolor{currentfill}%
\pgfsetlinewidth{0.501875pt}%
\definecolor{currentstroke}{rgb}{0.000000,0.000000,0.000000}%
\pgfsetstrokecolor{currentstroke}%
\pgfsetdash{}{0pt}%
\pgfsys@defobject{currentmarker}{\pgfqpoint{0.000000in}{-0.020833in}}{\pgfqpoint{0.000000in}{0.000000in}}{%
\pgfpathmoveto{\pgfqpoint{0.000000in}{0.000000in}}%
\pgfpathlineto{\pgfqpoint{0.000000in}{-0.020833in}}%
\pgfusepath{stroke,fill}%
}%
\begin{pgfscope}%
\pgfsys@transformshift{1.086227in}{1.649193in}%
\pgfsys@useobject{currentmarker}{}%
\end{pgfscope}%
\end{pgfscope}%
\begin{pgfscope}%
\pgfsetbuttcap%
\pgfsetroundjoin%
\definecolor{currentfill}{rgb}{0.000000,0.000000,0.000000}%
\pgfsetfillcolor{currentfill}%
\pgfsetlinewidth{0.501875pt}%
\definecolor{currentstroke}{rgb}{0.000000,0.000000,0.000000}%
\pgfsetstrokecolor{currentstroke}%
\pgfsetdash{}{0pt}%
\pgfsys@defobject{currentmarker}{\pgfqpoint{0.000000in}{0.000000in}}{\pgfqpoint{0.000000in}{0.020833in}}{%
\pgfpathmoveto{\pgfqpoint{0.000000in}{0.000000in}}%
\pgfpathlineto{\pgfqpoint{0.000000in}{0.020833in}}%
\pgfusepath{stroke,fill}%
}%
\begin{pgfscope}%
\pgfsys@transformshift{1.195478in}{0.422992in}%
\pgfsys@useobject{currentmarker}{}%
\end{pgfscope}%
\end{pgfscope}%
\begin{pgfscope}%
\pgfsetbuttcap%
\pgfsetroundjoin%
\definecolor{currentfill}{rgb}{0.000000,0.000000,0.000000}%
\pgfsetfillcolor{currentfill}%
\pgfsetlinewidth{0.501875pt}%
\definecolor{currentstroke}{rgb}{0.000000,0.000000,0.000000}%
\pgfsetstrokecolor{currentstroke}%
\pgfsetdash{}{0pt}%
\pgfsys@defobject{currentmarker}{\pgfqpoint{0.000000in}{-0.020833in}}{\pgfqpoint{0.000000in}{0.000000in}}{%
\pgfpathmoveto{\pgfqpoint{0.000000in}{0.000000in}}%
\pgfpathlineto{\pgfqpoint{0.000000in}{-0.020833in}}%
\pgfusepath{stroke,fill}%
}%
\begin{pgfscope}%
\pgfsys@transformshift{1.195478in}{1.649193in}%
\pgfsys@useobject{currentmarker}{}%
\end{pgfscope}%
\end{pgfscope}%
\begin{pgfscope}%
\pgfsetbuttcap%
\pgfsetroundjoin%
\definecolor{currentfill}{rgb}{0.000000,0.000000,0.000000}%
\pgfsetfillcolor{currentfill}%
\pgfsetlinewidth{0.501875pt}%
\definecolor{currentstroke}{rgb}{0.000000,0.000000,0.000000}%
\pgfsetstrokecolor{currentstroke}%
\pgfsetdash{}{0pt}%
\pgfsys@defobject{currentmarker}{\pgfqpoint{0.000000in}{0.000000in}}{\pgfqpoint{0.000000in}{0.020833in}}{%
\pgfpathmoveto{\pgfqpoint{0.000000in}{0.000000in}}%
\pgfpathlineto{\pgfqpoint{0.000000in}{0.020833in}}%
\pgfusepath{stroke,fill}%
}%
\begin{pgfscope}%
\pgfsys@transformshift{1.304730in}{0.422992in}%
\pgfsys@useobject{currentmarker}{}%
\end{pgfscope}%
\end{pgfscope}%
\begin{pgfscope}%
\pgfsetbuttcap%
\pgfsetroundjoin%
\definecolor{currentfill}{rgb}{0.000000,0.000000,0.000000}%
\pgfsetfillcolor{currentfill}%
\pgfsetlinewidth{0.501875pt}%
\definecolor{currentstroke}{rgb}{0.000000,0.000000,0.000000}%
\pgfsetstrokecolor{currentstroke}%
\pgfsetdash{}{0pt}%
\pgfsys@defobject{currentmarker}{\pgfqpoint{0.000000in}{-0.020833in}}{\pgfqpoint{0.000000in}{0.000000in}}{%
\pgfpathmoveto{\pgfqpoint{0.000000in}{0.000000in}}%
\pgfpathlineto{\pgfqpoint{0.000000in}{-0.020833in}}%
\pgfusepath{stroke,fill}%
}%
\begin{pgfscope}%
\pgfsys@transformshift{1.304730in}{1.649193in}%
\pgfsys@useobject{currentmarker}{}%
\end{pgfscope}%
\end{pgfscope}%
\begin{pgfscope}%
\pgfsetbuttcap%
\pgfsetroundjoin%
\definecolor{currentfill}{rgb}{0.000000,0.000000,0.000000}%
\pgfsetfillcolor{currentfill}%
\pgfsetlinewidth{0.501875pt}%
\definecolor{currentstroke}{rgb}{0.000000,0.000000,0.000000}%
\pgfsetstrokecolor{currentstroke}%
\pgfsetdash{}{0pt}%
\pgfsys@defobject{currentmarker}{\pgfqpoint{0.000000in}{0.000000in}}{\pgfqpoint{0.000000in}{0.020833in}}{%
\pgfpathmoveto{\pgfqpoint{0.000000in}{0.000000in}}%
\pgfpathlineto{\pgfqpoint{0.000000in}{0.020833in}}%
\pgfusepath{stroke,fill}%
}%
\begin{pgfscope}%
\pgfsys@transformshift{1.523233in}{0.422992in}%
\pgfsys@useobject{currentmarker}{}%
\end{pgfscope}%
\end{pgfscope}%
\begin{pgfscope}%
\pgfsetbuttcap%
\pgfsetroundjoin%
\definecolor{currentfill}{rgb}{0.000000,0.000000,0.000000}%
\pgfsetfillcolor{currentfill}%
\pgfsetlinewidth{0.501875pt}%
\definecolor{currentstroke}{rgb}{0.000000,0.000000,0.000000}%
\pgfsetstrokecolor{currentstroke}%
\pgfsetdash{}{0pt}%
\pgfsys@defobject{currentmarker}{\pgfqpoint{0.000000in}{-0.020833in}}{\pgfqpoint{0.000000in}{0.000000in}}{%
\pgfpathmoveto{\pgfqpoint{0.000000in}{0.000000in}}%
\pgfpathlineto{\pgfqpoint{0.000000in}{-0.020833in}}%
\pgfusepath{stroke,fill}%
}%
\begin{pgfscope}%
\pgfsys@transformshift{1.523233in}{1.649193in}%
\pgfsys@useobject{currentmarker}{}%
\end{pgfscope}%
\end{pgfscope}%
\begin{pgfscope}%
\pgfsetbuttcap%
\pgfsetroundjoin%
\definecolor{currentfill}{rgb}{0.000000,0.000000,0.000000}%
\pgfsetfillcolor{currentfill}%
\pgfsetlinewidth{0.501875pt}%
\definecolor{currentstroke}{rgb}{0.000000,0.000000,0.000000}%
\pgfsetstrokecolor{currentstroke}%
\pgfsetdash{}{0pt}%
\pgfsys@defobject{currentmarker}{\pgfqpoint{0.000000in}{0.000000in}}{\pgfqpoint{0.000000in}{0.020833in}}{%
\pgfpathmoveto{\pgfqpoint{0.000000in}{0.000000in}}%
\pgfpathlineto{\pgfqpoint{0.000000in}{0.020833in}}%
\pgfusepath{stroke,fill}%
}%
\begin{pgfscope}%
\pgfsys@transformshift{1.632484in}{0.422992in}%
\pgfsys@useobject{currentmarker}{}%
\end{pgfscope}%
\end{pgfscope}%
\begin{pgfscope}%
\pgfsetbuttcap%
\pgfsetroundjoin%
\definecolor{currentfill}{rgb}{0.000000,0.000000,0.000000}%
\pgfsetfillcolor{currentfill}%
\pgfsetlinewidth{0.501875pt}%
\definecolor{currentstroke}{rgb}{0.000000,0.000000,0.000000}%
\pgfsetstrokecolor{currentstroke}%
\pgfsetdash{}{0pt}%
\pgfsys@defobject{currentmarker}{\pgfqpoint{0.000000in}{-0.020833in}}{\pgfqpoint{0.000000in}{0.000000in}}{%
\pgfpathmoveto{\pgfqpoint{0.000000in}{0.000000in}}%
\pgfpathlineto{\pgfqpoint{0.000000in}{-0.020833in}}%
\pgfusepath{stroke,fill}%
}%
\begin{pgfscope}%
\pgfsys@transformshift{1.632484in}{1.649193in}%
\pgfsys@useobject{currentmarker}{}%
\end{pgfscope}%
\end{pgfscope}%
\begin{pgfscope}%
\pgfsetbuttcap%
\pgfsetroundjoin%
\definecolor{currentfill}{rgb}{0.000000,0.000000,0.000000}%
\pgfsetfillcolor{currentfill}%
\pgfsetlinewidth{0.501875pt}%
\definecolor{currentstroke}{rgb}{0.000000,0.000000,0.000000}%
\pgfsetstrokecolor{currentstroke}%
\pgfsetdash{}{0pt}%
\pgfsys@defobject{currentmarker}{\pgfqpoint{0.000000in}{0.000000in}}{\pgfqpoint{0.000000in}{0.020833in}}{%
\pgfpathmoveto{\pgfqpoint{0.000000in}{0.000000in}}%
\pgfpathlineto{\pgfqpoint{0.000000in}{0.020833in}}%
\pgfusepath{stroke,fill}%
}%
\begin{pgfscope}%
\pgfsys@transformshift{1.741736in}{0.422992in}%
\pgfsys@useobject{currentmarker}{}%
\end{pgfscope}%
\end{pgfscope}%
\begin{pgfscope}%
\pgfsetbuttcap%
\pgfsetroundjoin%
\definecolor{currentfill}{rgb}{0.000000,0.000000,0.000000}%
\pgfsetfillcolor{currentfill}%
\pgfsetlinewidth{0.501875pt}%
\definecolor{currentstroke}{rgb}{0.000000,0.000000,0.000000}%
\pgfsetstrokecolor{currentstroke}%
\pgfsetdash{}{0pt}%
\pgfsys@defobject{currentmarker}{\pgfqpoint{0.000000in}{-0.020833in}}{\pgfqpoint{0.000000in}{0.000000in}}{%
\pgfpathmoveto{\pgfqpoint{0.000000in}{0.000000in}}%
\pgfpathlineto{\pgfqpoint{0.000000in}{-0.020833in}}%
\pgfusepath{stroke,fill}%
}%
\begin{pgfscope}%
\pgfsys@transformshift{1.741736in}{1.649193in}%
\pgfsys@useobject{currentmarker}{}%
\end{pgfscope}%
\end{pgfscope}%
\begin{pgfscope}%
\pgfsetbuttcap%
\pgfsetroundjoin%
\definecolor{currentfill}{rgb}{0.000000,0.000000,0.000000}%
\pgfsetfillcolor{currentfill}%
\pgfsetlinewidth{0.501875pt}%
\definecolor{currentstroke}{rgb}{0.000000,0.000000,0.000000}%
\pgfsetstrokecolor{currentstroke}%
\pgfsetdash{}{0pt}%
\pgfsys@defobject{currentmarker}{\pgfqpoint{0.000000in}{0.000000in}}{\pgfqpoint{0.000000in}{0.020833in}}{%
\pgfpathmoveto{\pgfqpoint{0.000000in}{0.000000in}}%
\pgfpathlineto{\pgfqpoint{0.000000in}{0.020833in}}%
\pgfusepath{stroke,fill}%
}%
\begin{pgfscope}%
\pgfsys@transformshift{1.960238in}{0.422992in}%
\pgfsys@useobject{currentmarker}{}%
\end{pgfscope}%
\end{pgfscope}%
\begin{pgfscope}%
\pgfsetbuttcap%
\pgfsetroundjoin%
\definecolor{currentfill}{rgb}{0.000000,0.000000,0.000000}%
\pgfsetfillcolor{currentfill}%
\pgfsetlinewidth{0.501875pt}%
\definecolor{currentstroke}{rgb}{0.000000,0.000000,0.000000}%
\pgfsetstrokecolor{currentstroke}%
\pgfsetdash{}{0pt}%
\pgfsys@defobject{currentmarker}{\pgfqpoint{0.000000in}{-0.020833in}}{\pgfqpoint{0.000000in}{0.000000in}}{%
\pgfpathmoveto{\pgfqpoint{0.000000in}{0.000000in}}%
\pgfpathlineto{\pgfqpoint{0.000000in}{-0.020833in}}%
\pgfusepath{stroke,fill}%
}%
\begin{pgfscope}%
\pgfsys@transformshift{1.960238in}{1.649193in}%
\pgfsys@useobject{currentmarker}{}%
\end{pgfscope}%
\end{pgfscope}%
\begin{pgfscope}%
\pgfsetbuttcap%
\pgfsetroundjoin%
\definecolor{currentfill}{rgb}{0.000000,0.000000,0.000000}%
\pgfsetfillcolor{currentfill}%
\pgfsetlinewidth{0.501875pt}%
\definecolor{currentstroke}{rgb}{0.000000,0.000000,0.000000}%
\pgfsetstrokecolor{currentstroke}%
\pgfsetdash{}{0pt}%
\pgfsys@defobject{currentmarker}{\pgfqpoint{0.000000in}{0.000000in}}{\pgfqpoint{0.000000in}{0.020833in}}{%
\pgfpathmoveto{\pgfqpoint{0.000000in}{0.000000in}}%
\pgfpathlineto{\pgfqpoint{0.000000in}{0.020833in}}%
\pgfusepath{stroke,fill}%
}%
\begin{pgfscope}%
\pgfsys@transformshift{2.069490in}{0.422992in}%
\pgfsys@useobject{currentmarker}{}%
\end{pgfscope}%
\end{pgfscope}%
\begin{pgfscope}%
\pgfsetbuttcap%
\pgfsetroundjoin%
\definecolor{currentfill}{rgb}{0.000000,0.000000,0.000000}%
\pgfsetfillcolor{currentfill}%
\pgfsetlinewidth{0.501875pt}%
\definecolor{currentstroke}{rgb}{0.000000,0.000000,0.000000}%
\pgfsetstrokecolor{currentstroke}%
\pgfsetdash{}{0pt}%
\pgfsys@defobject{currentmarker}{\pgfqpoint{0.000000in}{-0.020833in}}{\pgfqpoint{0.000000in}{0.000000in}}{%
\pgfpathmoveto{\pgfqpoint{0.000000in}{0.000000in}}%
\pgfpathlineto{\pgfqpoint{0.000000in}{-0.020833in}}%
\pgfusepath{stroke,fill}%
}%
\begin{pgfscope}%
\pgfsys@transformshift{2.069490in}{1.649193in}%
\pgfsys@useobject{currentmarker}{}%
\end{pgfscope}%
\end{pgfscope}%
\begin{pgfscope}%
\pgfsetbuttcap%
\pgfsetroundjoin%
\definecolor{currentfill}{rgb}{0.000000,0.000000,0.000000}%
\pgfsetfillcolor{currentfill}%
\pgfsetlinewidth{0.501875pt}%
\definecolor{currentstroke}{rgb}{0.000000,0.000000,0.000000}%
\pgfsetstrokecolor{currentstroke}%
\pgfsetdash{}{0pt}%
\pgfsys@defobject{currentmarker}{\pgfqpoint{0.000000in}{0.000000in}}{\pgfqpoint{0.000000in}{0.020833in}}{%
\pgfpathmoveto{\pgfqpoint{0.000000in}{0.000000in}}%
\pgfpathlineto{\pgfqpoint{0.000000in}{0.020833in}}%
\pgfusepath{stroke,fill}%
}%
\begin{pgfscope}%
\pgfsys@transformshift{2.178741in}{0.422992in}%
\pgfsys@useobject{currentmarker}{}%
\end{pgfscope}%
\end{pgfscope}%
\begin{pgfscope}%
\pgfsetbuttcap%
\pgfsetroundjoin%
\definecolor{currentfill}{rgb}{0.000000,0.000000,0.000000}%
\pgfsetfillcolor{currentfill}%
\pgfsetlinewidth{0.501875pt}%
\definecolor{currentstroke}{rgb}{0.000000,0.000000,0.000000}%
\pgfsetstrokecolor{currentstroke}%
\pgfsetdash{}{0pt}%
\pgfsys@defobject{currentmarker}{\pgfqpoint{0.000000in}{-0.020833in}}{\pgfqpoint{0.000000in}{0.000000in}}{%
\pgfpathmoveto{\pgfqpoint{0.000000in}{0.000000in}}%
\pgfpathlineto{\pgfqpoint{0.000000in}{-0.020833in}}%
\pgfusepath{stroke,fill}%
}%
\begin{pgfscope}%
\pgfsys@transformshift{2.178741in}{1.649193in}%
\pgfsys@useobject{currentmarker}{}%
\end{pgfscope}%
\end{pgfscope}%
\begin{pgfscope}%
\pgfsetbuttcap%
\pgfsetroundjoin%
\definecolor{currentfill}{rgb}{0.000000,0.000000,0.000000}%
\pgfsetfillcolor{currentfill}%
\pgfsetlinewidth{0.501875pt}%
\definecolor{currentstroke}{rgb}{0.000000,0.000000,0.000000}%
\pgfsetstrokecolor{currentstroke}%
\pgfsetdash{}{0pt}%
\pgfsys@defobject{currentmarker}{\pgfqpoint{0.000000in}{0.000000in}}{\pgfqpoint{0.000000in}{0.020833in}}{%
\pgfpathmoveto{\pgfqpoint{0.000000in}{0.000000in}}%
\pgfpathlineto{\pgfqpoint{0.000000in}{0.020833in}}%
\pgfusepath{stroke,fill}%
}%
\begin{pgfscope}%
\pgfsys@transformshift{2.397244in}{0.422992in}%
\pgfsys@useobject{currentmarker}{}%
\end{pgfscope}%
\end{pgfscope}%
\begin{pgfscope}%
\pgfsetbuttcap%
\pgfsetroundjoin%
\definecolor{currentfill}{rgb}{0.000000,0.000000,0.000000}%
\pgfsetfillcolor{currentfill}%
\pgfsetlinewidth{0.501875pt}%
\definecolor{currentstroke}{rgb}{0.000000,0.000000,0.000000}%
\pgfsetstrokecolor{currentstroke}%
\pgfsetdash{}{0pt}%
\pgfsys@defobject{currentmarker}{\pgfqpoint{0.000000in}{-0.020833in}}{\pgfqpoint{0.000000in}{0.000000in}}{%
\pgfpathmoveto{\pgfqpoint{0.000000in}{0.000000in}}%
\pgfpathlineto{\pgfqpoint{0.000000in}{-0.020833in}}%
\pgfusepath{stroke,fill}%
}%
\begin{pgfscope}%
\pgfsys@transformshift{2.397244in}{1.649193in}%
\pgfsys@useobject{currentmarker}{}%
\end{pgfscope}%
\end{pgfscope}%
\begin{pgfscope}%
\pgfsetbuttcap%
\pgfsetroundjoin%
\definecolor{currentfill}{rgb}{0.000000,0.000000,0.000000}%
\pgfsetfillcolor{currentfill}%
\pgfsetlinewidth{0.501875pt}%
\definecolor{currentstroke}{rgb}{0.000000,0.000000,0.000000}%
\pgfsetstrokecolor{currentstroke}%
\pgfsetdash{}{0pt}%
\pgfsys@defobject{currentmarker}{\pgfqpoint{0.000000in}{0.000000in}}{\pgfqpoint{0.000000in}{0.020833in}}{%
\pgfpathmoveto{\pgfqpoint{0.000000in}{0.000000in}}%
\pgfpathlineto{\pgfqpoint{0.000000in}{0.020833in}}%
\pgfusepath{stroke,fill}%
}%
\begin{pgfscope}%
\pgfsys@transformshift{2.506496in}{0.422992in}%
\pgfsys@useobject{currentmarker}{}%
\end{pgfscope}%
\end{pgfscope}%
\begin{pgfscope}%
\pgfsetbuttcap%
\pgfsetroundjoin%
\definecolor{currentfill}{rgb}{0.000000,0.000000,0.000000}%
\pgfsetfillcolor{currentfill}%
\pgfsetlinewidth{0.501875pt}%
\definecolor{currentstroke}{rgb}{0.000000,0.000000,0.000000}%
\pgfsetstrokecolor{currentstroke}%
\pgfsetdash{}{0pt}%
\pgfsys@defobject{currentmarker}{\pgfqpoint{0.000000in}{-0.020833in}}{\pgfqpoint{0.000000in}{0.000000in}}{%
\pgfpathmoveto{\pgfqpoint{0.000000in}{0.000000in}}%
\pgfpathlineto{\pgfqpoint{0.000000in}{-0.020833in}}%
\pgfusepath{stroke,fill}%
}%
\begin{pgfscope}%
\pgfsys@transformshift{2.506496in}{1.649193in}%
\pgfsys@useobject{currentmarker}{}%
\end{pgfscope}%
\end{pgfscope}%
\begin{pgfscope}%
\pgfsetbuttcap%
\pgfsetroundjoin%
\definecolor{currentfill}{rgb}{0.000000,0.000000,0.000000}%
\pgfsetfillcolor{currentfill}%
\pgfsetlinewidth{0.501875pt}%
\definecolor{currentstroke}{rgb}{0.000000,0.000000,0.000000}%
\pgfsetstrokecolor{currentstroke}%
\pgfsetdash{}{0pt}%
\pgfsys@defobject{currentmarker}{\pgfqpoint{0.000000in}{0.000000in}}{\pgfqpoint{0.000000in}{0.020833in}}{%
\pgfpathmoveto{\pgfqpoint{0.000000in}{0.000000in}}%
\pgfpathlineto{\pgfqpoint{0.000000in}{0.020833in}}%
\pgfusepath{stroke,fill}%
}%
\begin{pgfscope}%
\pgfsys@transformshift{2.615747in}{0.422992in}%
\pgfsys@useobject{currentmarker}{}%
\end{pgfscope}%
\end{pgfscope}%
\begin{pgfscope}%
\pgfsetbuttcap%
\pgfsetroundjoin%
\definecolor{currentfill}{rgb}{0.000000,0.000000,0.000000}%
\pgfsetfillcolor{currentfill}%
\pgfsetlinewidth{0.501875pt}%
\definecolor{currentstroke}{rgb}{0.000000,0.000000,0.000000}%
\pgfsetstrokecolor{currentstroke}%
\pgfsetdash{}{0pt}%
\pgfsys@defobject{currentmarker}{\pgfqpoint{0.000000in}{-0.020833in}}{\pgfqpoint{0.000000in}{0.000000in}}{%
\pgfpathmoveto{\pgfqpoint{0.000000in}{0.000000in}}%
\pgfpathlineto{\pgfqpoint{0.000000in}{-0.020833in}}%
\pgfusepath{stroke,fill}%
}%
\begin{pgfscope}%
\pgfsys@transformshift{2.615747in}{1.649193in}%
\pgfsys@useobject{currentmarker}{}%
\end{pgfscope}%
\end{pgfscope}%
\begin{pgfscope}%
\pgfsetbuttcap%
\pgfsetroundjoin%
\definecolor{currentfill}{rgb}{0.000000,0.000000,0.000000}%
\pgfsetfillcolor{currentfill}%
\pgfsetlinewidth{0.501875pt}%
\definecolor{currentstroke}{rgb}{0.000000,0.000000,0.000000}%
\pgfsetstrokecolor{currentstroke}%
\pgfsetdash{}{0pt}%
\pgfsys@defobject{currentmarker}{\pgfqpoint{0.000000in}{0.000000in}}{\pgfqpoint{0.000000in}{0.020833in}}{%
\pgfpathmoveto{\pgfqpoint{0.000000in}{0.000000in}}%
\pgfpathlineto{\pgfqpoint{0.000000in}{0.020833in}}%
\pgfusepath{stroke,fill}%
}%
\begin{pgfscope}%
\pgfsys@transformshift{2.724998in}{0.422992in}%
\pgfsys@useobject{currentmarker}{}%
\end{pgfscope}%
\end{pgfscope}%
\begin{pgfscope}%
\pgfsetbuttcap%
\pgfsetroundjoin%
\definecolor{currentfill}{rgb}{0.000000,0.000000,0.000000}%
\pgfsetfillcolor{currentfill}%
\pgfsetlinewidth{0.501875pt}%
\definecolor{currentstroke}{rgb}{0.000000,0.000000,0.000000}%
\pgfsetstrokecolor{currentstroke}%
\pgfsetdash{}{0pt}%
\pgfsys@defobject{currentmarker}{\pgfqpoint{0.000000in}{-0.020833in}}{\pgfqpoint{0.000000in}{0.000000in}}{%
\pgfpathmoveto{\pgfqpoint{0.000000in}{0.000000in}}%
\pgfpathlineto{\pgfqpoint{0.000000in}{-0.020833in}}%
\pgfusepath{stroke,fill}%
}%
\begin{pgfscope}%
\pgfsys@transformshift{2.724998in}{1.649193in}%
\pgfsys@useobject{currentmarker}{}%
\end{pgfscope}%
\end{pgfscope}%
\begin{pgfscope}%
\definecolor{textcolor}{rgb}{0.000000,0.000000,0.000000}%
\pgfsetstrokecolor{textcolor}%
\pgfsetfillcolor{textcolor}%
\pgftext[x=1.643409in,y=0.184413in,,top]{\color{textcolor}\rmfamily\fontsize{10.000000}{12.000000}\selectfont \(\displaystyle K\)}%
\end{pgfscope}%
\begin{pgfscope}%
\pgfsetbuttcap%
\pgfsetroundjoin%
\definecolor{currentfill}{rgb}{0.000000,0.000000,0.000000}%
\pgfsetfillcolor{currentfill}%
\pgfsetlinewidth{0.501875pt}%
\definecolor{currentstroke}{rgb}{0.000000,0.000000,0.000000}%
\pgfsetstrokecolor{currentstroke}%
\pgfsetdash{}{0pt}%
\pgfsys@defobject{currentmarker}{\pgfqpoint{0.000000in}{0.000000in}}{\pgfqpoint{0.041667in}{0.000000in}}{%
\pgfpathmoveto{\pgfqpoint{0.000000in}{0.000000in}}%
\pgfpathlineto{\pgfqpoint{0.041667in}{0.000000in}}%
\pgfusepath{stroke,fill}%
}%
\begin{pgfscope}%
\pgfsys@transformshift{0.539970in}{1.014655in}%
\pgfsys@useobject{currentmarker}{}%
\end{pgfscope}%
\end{pgfscope}%
\begin{pgfscope}%
\pgfsetbuttcap%
\pgfsetroundjoin%
\definecolor{currentfill}{rgb}{0.000000,0.000000,0.000000}%
\pgfsetfillcolor{currentfill}%
\pgfsetlinewidth{0.501875pt}%
\definecolor{currentstroke}{rgb}{0.000000,0.000000,0.000000}%
\pgfsetstrokecolor{currentstroke}%
\pgfsetdash{}{0pt}%
\pgfsys@defobject{currentmarker}{\pgfqpoint{-0.041667in}{0.000000in}}{\pgfqpoint{-0.000000in}{0.000000in}}{%
\pgfpathmoveto{\pgfqpoint{-0.000000in}{0.000000in}}%
\pgfpathlineto{\pgfqpoint{-0.041667in}{0.000000in}}%
\pgfusepath{stroke,fill}%
}%
\begin{pgfscope}%
\pgfsys@transformshift{2.746849in}{1.014655in}%
\pgfsys@useobject{currentmarker}{}%
\end{pgfscope}%
\end{pgfscope}%
\begin{pgfscope}%
\definecolor{textcolor}{rgb}{0.000000,0.000000,0.000000}%
\pgfsetstrokecolor{textcolor}%
\pgfsetfillcolor{textcolor}%
\pgftext[x=0.244444in, y=0.961893in, left, base]{\color{textcolor}\rmfamily\fontsize{10.000000}{12.000000}\selectfont \(\displaystyle {0.99}\)}%
\end{pgfscope}%
\begin{pgfscope}%
\pgfsetbuttcap%
\pgfsetroundjoin%
\definecolor{currentfill}{rgb}{0.000000,0.000000,0.000000}%
\pgfsetfillcolor{currentfill}%
\pgfsetlinewidth{0.501875pt}%
\definecolor{currentstroke}{rgb}{0.000000,0.000000,0.000000}%
\pgfsetstrokecolor{currentstroke}%
\pgfsetdash{}{0pt}%
\pgfsys@defobject{currentmarker}{\pgfqpoint{0.000000in}{0.000000in}}{\pgfqpoint{0.041667in}{0.000000in}}{%
\pgfpathmoveto{\pgfqpoint{0.000000in}{0.000000in}}%
\pgfpathlineto{\pgfqpoint{0.041667in}{0.000000in}}%
\pgfusepath{stroke,fill}%
}%
\begin{pgfscope}%
\pgfsys@transformshift{0.539970in}{1.608494in}%
\pgfsys@useobject{currentmarker}{}%
\end{pgfscope}%
\end{pgfscope}%
\begin{pgfscope}%
\pgfsetbuttcap%
\pgfsetroundjoin%
\definecolor{currentfill}{rgb}{0.000000,0.000000,0.000000}%
\pgfsetfillcolor{currentfill}%
\pgfsetlinewidth{0.501875pt}%
\definecolor{currentstroke}{rgb}{0.000000,0.000000,0.000000}%
\pgfsetstrokecolor{currentstroke}%
\pgfsetdash{}{0pt}%
\pgfsys@defobject{currentmarker}{\pgfqpoint{-0.041667in}{0.000000in}}{\pgfqpoint{-0.000000in}{0.000000in}}{%
\pgfpathmoveto{\pgfqpoint{-0.000000in}{0.000000in}}%
\pgfpathlineto{\pgfqpoint{-0.041667in}{0.000000in}}%
\pgfusepath{stroke,fill}%
}%
\begin{pgfscope}%
\pgfsys@transformshift{2.746849in}{1.608494in}%
\pgfsys@useobject{currentmarker}{}%
\end{pgfscope}%
\end{pgfscope}%
\begin{pgfscope}%
\definecolor{textcolor}{rgb}{0.000000,0.000000,0.000000}%
\pgfsetstrokecolor{textcolor}%
\pgfsetfillcolor{textcolor}%
\pgftext[x=0.244444in, y=1.555732in, left, base]{\color{textcolor}\rmfamily\fontsize{10.000000}{12.000000}\selectfont \(\displaystyle {1.00}\)}%
\end{pgfscope}%
\begin{pgfscope}%
\pgfsetbuttcap%
\pgfsetroundjoin%
\definecolor{currentfill}{rgb}{0.000000,0.000000,0.000000}%
\pgfsetfillcolor{currentfill}%
\pgfsetlinewidth{0.501875pt}%
\definecolor{currentstroke}{rgb}{0.000000,0.000000,0.000000}%
\pgfsetstrokecolor{currentstroke}%
\pgfsetdash{}{0pt}%
\pgfsys@defobject{currentmarker}{\pgfqpoint{0.000000in}{0.000000in}}{\pgfqpoint{0.020833in}{0.000000in}}{%
\pgfpathmoveto{\pgfqpoint{0.000000in}{0.000000in}}%
\pgfpathlineto{\pgfqpoint{0.020833in}{0.000000in}}%
\pgfusepath{stroke,fill}%
}%
\begin{pgfscope}%
\pgfsys@transformshift{0.539970in}{0.539583in}%
\pgfsys@useobject{currentmarker}{}%
\end{pgfscope}%
\end{pgfscope}%
\begin{pgfscope}%
\pgfsetbuttcap%
\pgfsetroundjoin%
\definecolor{currentfill}{rgb}{0.000000,0.000000,0.000000}%
\pgfsetfillcolor{currentfill}%
\pgfsetlinewidth{0.501875pt}%
\definecolor{currentstroke}{rgb}{0.000000,0.000000,0.000000}%
\pgfsetstrokecolor{currentstroke}%
\pgfsetdash{}{0pt}%
\pgfsys@defobject{currentmarker}{\pgfqpoint{-0.020833in}{0.000000in}}{\pgfqpoint{-0.000000in}{0.000000in}}{%
\pgfpathmoveto{\pgfqpoint{-0.000000in}{0.000000in}}%
\pgfpathlineto{\pgfqpoint{-0.020833in}{0.000000in}}%
\pgfusepath{stroke,fill}%
}%
\begin{pgfscope}%
\pgfsys@transformshift{2.746849in}{0.539583in}%
\pgfsys@useobject{currentmarker}{}%
\end{pgfscope}%
\end{pgfscope}%
\begin{pgfscope}%
\pgfsetbuttcap%
\pgfsetroundjoin%
\definecolor{currentfill}{rgb}{0.000000,0.000000,0.000000}%
\pgfsetfillcolor{currentfill}%
\pgfsetlinewidth{0.501875pt}%
\definecolor{currentstroke}{rgb}{0.000000,0.000000,0.000000}%
\pgfsetstrokecolor{currentstroke}%
\pgfsetdash{}{0pt}%
\pgfsys@defobject{currentmarker}{\pgfqpoint{0.000000in}{0.000000in}}{\pgfqpoint{0.020833in}{0.000000in}}{%
\pgfpathmoveto{\pgfqpoint{0.000000in}{0.000000in}}%
\pgfpathlineto{\pgfqpoint{0.020833in}{0.000000in}}%
\pgfusepath{stroke,fill}%
}%
\begin{pgfscope}%
\pgfsys@transformshift{0.539970in}{0.658351in}%
\pgfsys@useobject{currentmarker}{}%
\end{pgfscope}%
\end{pgfscope}%
\begin{pgfscope}%
\pgfsetbuttcap%
\pgfsetroundjoin%
\definecolor{currentfill}{rgb}{0.000000,0.000000,0.000000}%
\pgfsetfillcolor{currentfill}%
\pgfsetlinewidth{0.501875pt}%
\definecolor{currentstroke}{rgb}{0.000000,0.000000,0.000000}%
\pgfsetstrokecolor{currentstroke}%
\pgfsetdash{}{0pt}%
\pgfsys@defobject{currentmarker}{\pgfqpoint{-0.020833in}{0.000000in}}{\pgfqpoint{-0.000000in}{0.000000in}}{%
\pgfpathmoveto{\pgfqpoint{-0.000000in}{0.000000in}}%
\pgfpathlineto{\pgfqpoint{-0.020833in}{0.000000in}}%
\pgfusepath{stroke,fill}%
}%
\begin{pgfscope}%
\pgfsys@transformshift{2.746849in}{0.658351in}%
\pgfsys@useobject{currentmarker}{}%
\end{pgfscope}%
\end{pgfscope}%
\begin{pgfscope}%
\pgfsetbuttcap%
\pgfsetroundjoin%
\definecolor{currentfill}{rgb}{0.000000,0.000000,0.000000}%
\pgfsetfillcolor{currentfill}%
\pgfsetlinewidth{0.501875pt}%
\definecolor{currentstroke}{rgb}{0.000000,0.000000,0.000000}%
\pgfsetstrokecolor{currentstroke}%
\pgfsetdash{}{0pt}%
\pgfsys@defobject{currentmarker}{\pgfqpoint{0.000000in}{0.000000in}}{\pgfqpoint{0.020833in}{0.000000in}}{%
\pgfpathmoveto{\pgfqpoint{0.000000in}{0.000000in}}%
\pgfpathlineto{\pgfqpoint{0.020833in}{0.000000in}}%
\pgfusepath{stroke,fill}%
}%
\begin{pgfscope}%
\pgfsys@transformshift{0.539970in}{0.777119in}%
\pgfsys@useobject{currentmarker}{}%
\end{pgfscope}%
\end{pgfscope}%
\begin{pgfscope}%
\pgfsetbuttcap%
\pgfsetroundjoin%
\definecolor{currentfill}{rgb}{0.000000,0.000000,0.000000}%
\pgfsetfillcolor{currentfill}%
\pgfsetlinewidth{0.501875pt}%
\definecolor{currentstroke}{rgb}{0.000000,0.000000,0.000000}%
\pgfsetstrokecolor{currentstroke}%
\pgfsetdash{}{0pt}%
\pgfsys@defobject{currentmarker}{\pgfqpoint{-0.020833in}{0.000000in}}{\pgfqpoint{-0.000000in}{0.000000in}}{%
\pgfpathmoveto{\pgfqpoint{-0.000000in}{0.000000in}}%
\pgfpathlineto{\pgfqpoint{-0.020833in}{0.000000in}}%
\pgfusepath{stroke,fill}%
}%
\begin{pgfscope}%
\pgfsys@transformshift{2.746849in}{0.777119in}%
\pgfsys@useobject{currentmarker}{}%
\end{pgfscope}%
\end{pgfscope}%
\begin{pgfscope}%
\pgfsetbuttcap%
\pgfsetroundjoin%
\definecolor{currentfill}{rgb}{0.000000,0.000000,0.000000}%
\pgfsetfillcolor{currentfill}%
\pgfsetlinewidth{0.501875pt}%
\definecolor{currentstroke}{rgb}{0.000000,0.000000,0.000000}%
\pgfsetstrokecolor{currentstroke}%
\pgfsetdash{}{0pt}%
\pgfsys@defobject{currentmarker}{\pgfqpoint{0.000000in}{0.000000in}}{\pgfqpoint{0.020833in}{0.000000in}}{%
\pgfpathmoveto{\pgfqpoint{0.000000in}{0.000000in}}%
\pgfpathlineto{\pgfqpoint{0.020833in}{0.000000in}}%
\pgfusepath{stroke,fill}%
}%
\begin{pgfscope}%
\pgfsys@transformshift{0.539970in}{0.895887in}%
\pgfsys@useobject{currentmarker}{}%
\end{pgfscope}%
\end{pgfscope}%
\begin{pgfscope}%
\pgfsetbuttcap%
\pgfsetroundjoin%
\definecolor{currentfill}{rgb}{0.000000,0.000000,0.000000}%
\pgfsetfillcolor{currentfill}%
\pgfsetlinewidth{0.501875pt}%
\definecolor{currentstroke}{rgb}{0.000000,0.000000,0.000000}%
\pgfsetstrokecolor{currentstroke}%
\pgfsetdash{}{0pt}%
\pgfsys@defobject{currentmarker}{\pgfqpoint{-0.020833in}{0.000000in}}{\pgfqpoint{-0.000000in}{0.000000in}}{%
\pgfpathmoveto{\pgfqpoint{-0.000000in}{0.000000in}}%
\pgfpathlineto{\pgfqpoint{-0.020833in}{0.000000in}}%
\pgfusepath{stroke,fill}%
}%
\begin{pgfscope}%
\pgfsys@transformshift{2.746849in}{0.895887in}%
\pgfsys@useobject{currentmarker}{}%
\end{pgfscope}%
\end{pgfscope}%
\begin{pgfscope}%
\pgfsetbuttcap%
\pgfsetroundjoin%
\definecolor{currentfill}{rgb}{0.000000,0.000000,0.000000}%
\pgfsetfillcolor{currentfill}%
\pgfsetlinewidth{0.501875pt}%
\definecolor{currentstroke}{rgb}{0.000000,0.000000,0.000000}%
\pgfsetstrokecolor{currentstroke}%
\pgfsetdash{}{0pt}%
\pgfsys@defobject{currentmarker}{\pgfqpoint{0.000000in}{0.000000in}}{\pgfqpoint{0.020833in}{0.000000in}}{%
\pgfpathmoveto{\pgfqpoint{0.000000in}{0.000000in}}%
\pgfpathlineto{\pgfqpoint{0.020833in}{0.000000in}}%
\pgfusepath{stroke,fill}%
}%
\begin{pgfscope}%
\pgfsys@transformshift{0.539970in}{1.133423in}%
\pgfsys@useobject{currentmarker}{}%
\end{pgfscope}%
\end{pgfscope}%
\begin{pgfscope}%
\pgfsetbuttcap%
\pgfsetroundjoin%
\definecolor{currentfill}{rgb}{0.000000,0.000000,0.000000}%
\pgfsetfillcolor{currentfill}%
\pgfsetlinewidth{0.501875pt}%
\definecolor{currentstroke}{rgb}{0.000000,0.000000,0.000000}%
\pgfsetstrokecolor{currentstroke}%
\pgfsetdash{}{0pt}%
\pgfsys@defobject{currentmarker}{\pgfqpoint{-0.020833in}{0.000000in}}{\pgfqpoint{-0.000000in}{0.000000in}}{%
\pgfpathmoveto{\pgfqpoint{-0.000000in}{0.000000in}}%
\pgfpathlineto{\pgfqpoint{-0.020833in}{0.000000in}}%
\pgfusepath{stroke,fill}%
}%
\begin{pgfscope}%
\pgfsys@transformshift{2.746849in}{1.133423in}%
\pgfsys@useobject{currentmarker}{}%
\end{pgfscope}%
\end{pgfscope}%
\begin{pgfscope}%
\pgfsetbuttcap%
\pgfsetroundjoin%
\definecolor{currentfill}{rgb}{0.000000,0.000000,0.000000}%
\pgfsetfillcolor{currentfill}%
\pgfsetlinewidth{0.501875pt}%
\definecolor{currentstroke}{rgb}{0.000000,0.000000,0.000000}%
\pgfsetstrokecolor{currentstroke}%
\pgfsetdash{}{0pt}%
\pgfsys@defobject{currentmarker}{\pgfqpoint{0.000000in}{0.000000in}}{\pgfqpoint{0.020833in}{0.000000in}}{%
\pgfpathmoveto{\pgfqpoint{0.000000in}{0.000000in}}%
\pgfpathlineto{\pgfqpoint{0.020833in}{0.000000in}}%
\pgfusepath{stroke,fill}%
}%
\begin{pgfscope}%
\pgfsys@transformshift{0.539970in}{1.252190in}%
\pgfsys@useobject{currentmarker}{}%
\end{pgfscope}%
\end{pgfscope}%
\begin{pgfscope}%
\pgfsetbuttcap%
\pgfsetroundjoin%
\definecolor{currentfill}{rgb}{0.000000,0.000000,0.000000}%
\pgfsetfillcolor{currentfill}%
\pgfsetlinewidth{0.501875pt}%
\definecolor{currentstroke}{rgb}{0.000000,0.000000,0.000000}%
\pgfsetstrokecolor{currentstroke}%
\pgfsetdash{}{0pt}%
\pgfsys@defobject{currentmarker}{\pgfqpoint{-0.020833in}{0.000000in}}{\pgfqpoint{-0.000000in}{0.000000in}}{%
\pgfpathmoveto{\pgfqpoint{-0.000000in}{0.000000in}}%
\pgfpathlineto{\pgfqpoint{-0.020833in}{0.000000in}}%
\pgfusepath{stroke,fill}%
}%
\begin{pgfscope}%
\pgfsys@transformshift{2.746849in}{1.252190in}%
\pgfsys@useobject{currentmarker}{}%
\end{pgfscope}%
\end{pgfscope}%
\begin{pgfscope}%
\pgfsetbuttcap%
\pgfsetroundjoin%
\definecolor{currentfill}{rgb}{0.000000,0.000000,0.000000}%
\pgfsetfillcolor{currentfill}%
\pgfsetlinewidth{0.501875pt}%
\definecolor{currentstroke}{rgb}{0.000000,0.000000,0.000000}%
\pgfsetstrokecolor{currentstroke}%
\pgfsetdash{}{0pt}%
\pgfsys@defobject{currentmarker}{\pgfqpoint{0.000000in}{0.000000in}}{\pgfqpoint{0.020833in}{0.000000in}}{%
\pgfpathmoveto{\pgfqpoint{0.000000in}{0.000000in}}%
\pgfpathlineto{\pgfqpoint{0.020833in}{0.000000in}}%
\pgfusepath{stroke,fill}%
}%
\begin{pgfscope}%
\pgfsys@transformshift{0.539970in}{1.370958in}%
\pgfsys@useobject{currentmarker}{}%
\end{pgfscope}%
\end{pgfscope}%
\begin{pgfscope}%
\pgfsetbuttcap%
\pgfsetroundjoin%
\definecolor{currentfill}{rgb}{0.000000,0.000000,0.000000}%
\pgfsetfillcolor{currentfill}%
\pgfsetlinewidth{0.501875pt}%
\definecolor{currentstroke}{rgb}{0.000000,0.000000,0.000000}%
\pgfsetstrokecolor{currentstroke}%
\pgfsetdash{}{0pt}%
\pgfsys@defobject{currentmarker}{\pgfqpoint{-0.020833in}{0.000000in}}{\pgfqpoint{-0.000000in}{0.000000in}}{%
\pgfpathmoveto{\pgfqpoint{-0.000000in}{0.000000in}}%
\pgfpathlineto{\pgfqpoint{-0.020833in}{0.000000in}}%
\pgfusepath{stroke,fill}%
}%
\begin{pgfscope}%
\pgfsys@transformshift{2.746849in}{1.370958in}%
\pgfsys@useobject{currentmarker}{}%
\end{pgfscope}%
\end{pgfscope}%
\begin{pgfscope}%
\pgfsetbuttcap%
\pgfsetroundjoin%
\definecolor{currentfill}{rgb}{0.000000,0.000000,0.000000}%
\pgfsetfillcolor{currentfill}%
\pgfsetlinewidth{0.501875pt}%
\definecolor{currentstroke}{rgb}{0.000000,0.000000,0.000000}%
\pgfsetstrokecolor{currentstroke}%
\pgfsetdash{}{0pt}%
\pgfsys@defobject{currentmarker}{\pgfqpoint{0.000000in}{0.000000in}}{\pgfqpoint{0.020833in}{0.000000in}}{%
\pgfpathmoveto{\pgfqpoint{0.000000in}{0.000000in}}%
\pgfpathlineto{\pgfqpoint{0.020833in}{0.000000in}}%
\pgfusepath{stroke,fill}%
}%
\begin{pgfscope}%
\pgfsys@transformshift{0.539970in}{1.489726in}%
\pgfsys@useobject{currentmarker}{}%
\end{pgfscope}%
\end{pgfscope}%
\begin{pgfscope}%
\pgfsetbuttcap%
\pgfsetroundjoin%
\definecolor{currentfill}{rgb}{0.000000,0.000000,0.000000}%
\pgfsetfillcolor{currentfill}%
\pgfsetlinewidth{0.501875pt}%
\definecolor{currentstroke}{rgb}{0.000000,0.000000,0.000000}%
\pgfsetstrokecolor{currentstroke}%
\pgfsetdash{}{0pt}%
\pgfsys@defobject{currentmarker}{\pgfqpoint{-0.020833in}{0.000000in}}{\pgfqpoint{-0.000000in}{0.000000in}}{%
\pgfpathmoveto{\pgfqpoint{-0.000000in}{0.000000in}}%
\pgfpathlineto{\pgfqpoint{-0.020833in}{0.000000in}}%
\pgfusepath{stroke,fill}%
}%
\begin{pgfscope}%
\pgfsys@transformshift{2.746849in}{1.489726in}%
\pgfsys@useobject{currentmarker}{}%
\end{pgfscope}%
\end{pgfscope}%
\begin{pgfscope}%
\definecolor{textcolor}{rgb}{0.000000,0.000000,0.000000}%
\pgfsetstrokecolor{textcolor}%
\pgfsetfillcolor{textcolor}%
\pgftext[x=0.188889in,y=1.036093in,,bottom,rotate=90.000000]{\color{textcolor}\rmfamily\fontsize{10.000000}{12.000000}\selectfont \(\displaystyle C(K)\)}%
\end{pgfscope}%
\begin{pgfscope}%
\pgfpathrectangle{\pgfqpoint{0.539970in}{0.422992in}}{\pgfqpoint{2.206879in}{1.226201in}}%
\pgfusepath{clip}%
\pgfsetrectcap%
\pgfsetroundjoin%
\pgfsetlinewidth{1.003750pt}%
\definecolor{currentstroke}{rgb}{0.047059,0.364706,0.647059}%
\pgfsetstrokecolor{currentstroke}%
\pgfsetdash{}{0pt}%
\pgfpathmoveto{\pgfqpoint{0.561820in}{1.581269in}}%
\pgfpathlineto{\pgfqpoint{0.583670in}{1.572798in}}%
\pgfpathlineto{\pgfqpoint{0.605521in}{1.564240in}}%
\pgfpathlineto{\pgfqpoint{0.627371in}{1.555805in}}%
\pgfpathlineto{\pgfqpoint{0.649221in}{1.547701in}}%
\pgfpathlineto{\pgfqpoint{0.671072in}{1.540308in}}%
\pgfpathlineto{\pgfqpoint{0.692922in}{1.532329in}}%
\pgfpathlineto{\pgfqpoint{0.714772in}{1.524464in}}%
\pgfpathlineto{\pgfqpoint{0.736622in}{1.517261in}}%
\pgfpathlineto{\pgfqpoint{0.758473in}{1.509583in}}%
\pgfpathlineto{\pgfqpoint{0.780323in}{1.502528in}}%
\pgfpathlineto{\pgfqpoint{0.802173in}{1.494985in}}%
\pgfpathlineto{\pgfqpoint{0.824024in}{1.487913in}}%
\pgfpathlineto{\pgfqpoint{0.845874in}{1.481023in}}%
\pgfpathlineto{\pgfqpoint{0.867724in}{1.473941in}}%
\pgfpathlineto{\pgfqpoint{0.889574in}{1.466655in}}%
\pgfpathlineto{\pgfqpoint{0.911425in}{1.459428in}}%
\pgfpathlineto{\pgfqpoint{0.933275in}{1.452412in}}%
\pgfpathlineto{\pgfqpoint{0.955125in}{1.445389in}}%
\pgfpathlineto{\pgfqpoint{0.976976in}{1.438197in}}%
\pgfpathlineto{\pgfqpoint{0.998826in}{1.431518in}}%
\pgfpathlineto{\pgfqpoint{1.020676in}{1.424697in}}%
\pgfpathlineto{\pgfqpoint{1.042526in}{1.417853in}}%
\pgfpathlineto{\pgfqpoint{1.064377in}{1.410845in}}%
\pgfpathlineto{\pgfqpoint{1.086227in}{1.403881in}}%
\pgfpathlineto{\pgfqpoint{1.108077in}{1.396977in}}%
\pgfpathlineto{\pgfqpoint{1.129928in}{1.389875in}}%
\pgfpathlineto{\pgfqpoint{1.151778in}{1.383095in}}%
\pgfpathlineto{\pgfqpoint{1.173628in}{1.376321in}}%
\pgfpathlineto{\pgfqpoint{1.195478in}{1.369475in}}%
\pgfpathlineto{\pgfqpoint{1.217329in}{1.362554in}}%
\pgfpathlineto{\pgfqpoint{1.239179in}{1.355844in}}%
\pgfpathlineto{\pgfqpoint{1.261029in}{1.349093in}}%
\pgfpathlineto{\pgfqpoint{1.282880in}{1.342480in}}%
\pgfpathlineto{\pgfqpoint{1.304730in}{1.335959in}}%
\pgfpathlineto{\pgfqpoint{1.326580in}{1.329031in}}%
\pgfpathlineto{\pgfqpoint{1.348430in}{1.322587in}}%
\pgfpathlineto{\pgfqpoint{1.370281in}{1.315829in}}%
\pgfpathlineto{\pgfqpoint{1.392131in}{1.309039in}}%
\pgfpathlineto{\pgfqpoint{1.413981in}{1.302309in}}%
\pgfpathlineto{\pgfqpoint{1.435832in}{1.295555in}}%
\pgfpathlineto{\pgfqpoint{1.457682in}{1.288766in}}%
\pgfpathlineto{\pgfqpoint{1.479532in}{1.282118in}}%
\pgfpathlineto{\pgfqpoint{1.501382in}{1.275427in}}%
\pgfpathlineto{\pgfqpoint{1.523233in}{1.268508in}}%
\pgfpathlineto{\pgfqpoint{1.545083in}{1.261821in}}%
\pgfpathlineto{\pgfqpoint{1.566933in}{1.255297in}}%
\pgfpathlineto{\pgfqpoint{1.588784in}{1.248486in}}%
\pgfpathlineto{\pgfqpoint{1.610634in}{1.241845in}}%
\pgfpathlineto{\pgfqpoint{1.632484in}{1.235303in}}%
\pgfpathlineto{\pgfqpoint{1.654334in}{1.228523in}}%
\pgfpathlineto{\pgfqpoint{1.676185in}{1.222052in}}%
\pgfpathlineto{\pgfqpoint{1.698035in}{1.215714in}}%
\pgfpathlineto{\pgfqpoint{1.719885in}{1.208948in}}%
\pgfpathlineto{\pgfqpoint{1.741736in}{1.202286in}}%
\pgfpathlineto{\pgfqpoint{1.763586in}{1.195495in}}%
\pgfpathlineto{\pgfqpoint{1.785436in}{1.188919in}}%
\pgfpathlineto{\pgfqpoint{1.807286in}{1.182332in}}%
\pgfpathlineto{\pgfqpoint{1.829137in}{1.175826in}}%
\pgfpathlineto{\pgfqpoint{1.850987in}{1.169054in}}%
\pgfpathlineto{\pgfqpoint{1.872837in}{1.162508in}}%
\pgfpathlineto{\pgfqpoint{1.894688in}{1.155814in}}%
\pgfpathlineto{\pgfqpoint{1.916538in}{1.149308in}}%
\pgfpathlineto{\pgfqpoint{1.938388in}{1.142719in}}%
\pgfpathlineto{\pgfqpoint{1.960238in}{1.136027in}}%
\pgfpathlineto{\pgfqpoint{1.982089in}{1.129512in}}%
\pgfpathlineto{\pgfqpoint{2.003939in}{1.122950in}}%
\pgfpathlineto{\pgfqpoint{2.025789in}{1.116598in}}%
\pgfpathlineto{\pgfqpoint{2.047640in}{1.109960in}}%
\pgfpathlineto{\pgfqpoint{2.069490in}{1.103526in}}%
\pgfpathlineto{\pgfqpoint{2.091340in}{1.097239in}}%
\pgfpathlineto{\pgfqpoint{2.113190in}{1.090854in}}%
\pgfpathlineto{\pgfqpoint{2.135041in}{1.084386in}}%
\pgfpathlineto{\pgfqpoint{2.156891in}{1.077959in}}%
\pgfpathlineto{\pgfqpoint{2.178741in}{1.071486in}}%
\pgfpathlineto{\pgfqpoint{2.200592in}{1.065236in}}%
\pgfpathlineto{\pgfqpoint{2.222442in}{1.058775in}}%
\pgfpathlineto{\pgfqpoint{2.244292in}{1.052245in}}%
\pgfpathlineto{\pgfqpoint{2.266142in}{1.045832in}}%
\pgfpathlineto{\pgfqpoint{2.287993in}{1.039644in}}%
\pgfpathlineto{\pgfqpoint{2.309843in}{1.033416in}}%
\pgfpathlineto{\pgfqpoint{2.331693in}{1.027001in}}%
\pgfpathlineto{\pgfqpoint{2.353544in}{1.020831in}}%
\pgfpathlineto{\pgfqpoint{2.375394in}{1.014478in}}%
\pgfpathlineto{\pgfqpoint{2.397244in}{1.008118in}}%
\pgfpathlineto{\pgfqpoint{2.419094in}{1.001771in}}%
\pgfpathlineto{\pgfqpoint{2.440945in}{0.995558in}}%
\pgfpathlineto{\pgfqpoint{2.462795in}{0.989203in}}%
\pgfpathlineto{\pgfqpoint{2.484645in}{0.983010in}}%
\pgfpathlineto{\pgfqpoint{2.506496in}{0.976652in}}%
\pgfpathlineto{\pgfqpoint{2.528346in}{0.970303in}}%
\pgfpathlineto{\pgfqpoint{2.550196in}{0.963812in}}%
\pgfpathlineto{\pgfqpoint{2.572046in}{0.957507in}}%
\pgfpathlineto{\pgfqpoint{2.593897in}{0.951285in}}%
\pgfpathlineto{\pgfqpoint{2.615747in}{0.944909in}}%
\pgfpathlineto{\pgfqpoint{2.637597in}{0.938592in}}%
\pgfpathlineto{\pgfqpoint{2.659448in}{0.932436in}}%
\pgfpathlineto{\pgfqpoint{2.681298in}{0.926285in}}%
\pgfpathlineto{\pgfqpoint{2.703148in}{0.920214in}}%
\pgfusepath{stroke}%
\end{pgfscope}%
\begin{pgfscope}%
\pgfpathrectangle{\pgfqpoint{0.539970in}{0.422992in}}{\pgfqpoint{2.206879in}{1.226201in}}%
\pgfusepath{clip}%
\pgfsetrectcap%
\pgfsetroundjoin%
\pgfsetlinewidth{1.003750pt}%
\definecolor{currentstroke}{rgb}{0.000000,0.725490,0.270588}%
\pgfsetstrokecolor{currentstroke}%
\pgfsetdash{}{0pt}%
\pgfpathmoveto{\pgfqpoint{0.561820in}{1.566528in}}%
\pgfpathlineto{\pgfqpoint{0.583670in}{1.551536in}}%
\pgfpathlineto{\pgfqpoint{0.605521in}{1.536170in}}%
\pgfpathlineto{\pgfqpoint{0.627371in}{1.522026in}}%
\pgfpathlineto{\pgfqpoint{0.649221in}{1.508078in}}%
\pgfpathlineto{\pgfqpoint{0.671072in}{1.494024in}}%
\pgfpathlineto{\pgfqpoint{0.692922in}{1.480704in}}%
\pgfpathlineto{\pgfqpoint{0.714772in}{1.467287in}}%
\pgfpathlineto{\pgfqpoint{0.736622in}{1.454195in}}%
\pgfpathlineto{\pgfqpoint{0.758473in}{1.440830in}}%
\pgfpathlineto{\pgfqpoint{0.780323in}{1.427444in}}%
\pgfpathlineto{\pgfqpoint{0.802173in}{1.414055in}}%
\pgfpathlineto{\pgfqpoint{0.824024in}{1.401447in}}%
\pgfpathlineto{\pgfqpoint{0.845874in}{1.388257in}}%
\pgfpathlineto{\pgfqpoint{0.867724in}{1.375204in}}%
\pgfpathlineto{\pgfqpoint{0.889574in}{1.362682in}}%
\pgfpathlineto{\pgfqpoint{0.911425in}{1.350371in}}%
\pgfpathlineto{\pgfqpoint{0.933275in}{1.336986in}}%
\pgfpathlineto{\pgfqpoint{0.955125in}{1.324756in}}%
\pgfpathlineto{\pgfqpoint{0.976976in}{1.312194in}}%
\pgfpathlineto{\pgfqpoint{0.998826in}{1.299888in}}%
\pgfpathlineto{\pgfqpoint{1.020676in}{1.287905in}}%
\pgfpathlineto{\pgfqpoint{1.042526in}{1.276139in}}%
\pgfpathlineto{\pgfqpoint{1.064377in}{1.263885in}}%
\pgfpathlineto{\pgfqpoint{1.086227in}{1.251714in}}%
\pgfpathlineto{\pgfqpoint{1.108077in}{1.240171in}}%
\pgfpathlineto{\pgfqpoint{1.129928in}{1.228450in}}%
\pgfpathlineto{\pgfqpoint{1.151778in}{1.216402in}}%
\pgfpathlineto{\pgfqpoint{1.173628in}{1.204691in}}%
\pgfpathlineto{\pgfqpoint{1.195478in}{1.193164in}}%
\pgfpathlineto{\pgfqpoint{1.217329in}{1.181183in}}%
\pgfpathlineto{\pgfqpoint{1.239179in}{1.169814in}}%
\pgfpathlineto{\pgfqpoint{1.261029in}{1.158809in}}%
\pgfpathlineto{\pgfqpoint{1.282880in}{1.147427in}}%
\pgfpathlineto{\pgfqpoint{1.304730in}{1.135750in}}%
\pgfpathlineto{\pgfqpoint{1.326580in}{1.124004in}}%
\pgfpathlineto{\pgfqpoint{1.348430in}{1.112875in}}%
\pgfpathlineto{\pgfqpoint{1.370281in}{1.102051in}}%
\pgfpathlineto{\pgfqpoint{1.392131in}{1.091252in}}%
\pgfpathlineto{\pgfqpoint{1.413981in}{1.079726in}}%
\pgfpathlineto{\pgfqpoint{1.435832in}{1.068798in}}%
\pgfpathlineto{\pgfqpoint{1.457682in}{1.057638in}}%
\pgfpathlineto{\pgfqpoint{1.479532in}{1.046749in}}%
\pgfpathlineto{\pgfqpoint{1.501382in}{1.035644in}}%
\pgfpathlineto{\pgfqpoint{1.523233in}{1.025567in}}%
\pgfpathlineto{\pgfqpoint{1.545083in}{1.014988in}}%
\pgfpathlineto{\pgfqpoint{1.566933in}{1.004246in}}%
\pgfpathlineto{\pgfqpoint{1.588784in}{0.994028in}}%
\pgfpathlineto{\pgfqpoint{1.610634in}{0.983218in}}%
\pgfpathlineto{\pgfqpoint{1.632484in}{0.972538in}}%
\pgfpathlineto{\pgfqpoint{1.654334in}{0.961944in}}%
\pgfpathlineto{\pgfqpoint{1.676185in}{0.951691in}}%
\pgfpathlineto{\pgfqpoint{1.698035in}{0.941362in}}%
\pgfpathlineto{\pgfqpoint{1.719885in}{0.930869in}}%
\pgfpathlineto{\pgfqpoint{1.741736in}{0.919780in}}%
\pgfpathlineto{\pgfqpoint{1.763586in}{0.909169in}}%
\pgfpathlineto{\pgfqpoint{1.785436in}{0.898503in}}%
\pgfpathlineto{\pgfqpoint{1.807286in}{0.888100in}}%
\pgfpathlineto{\pgfqpoint{1.829137in}{0.877868in}}%
\pgfpathlineto{\pgfqpoint{1.850987in}{0.867527in}}%
\pgfpathlineto{\pgfqpoint{1.872837in}{0.857106in}}%
\pgfpathlineto{\pgfqpoint{1.894688in}{0.846935in}}%
\pgfpathlineto{\pgfqpoint{1.916538in}{0.836441in}}%
\pgfpathlineto{\pgfqpoint{1.938388in}{0.826147in}}%
\pgfpathlineto{\pgfqpoint{1.960238in}{0.815678in}}%
\pgfpathlineto{\pgfqpoint{1.982089in}{0.805766in}}%
\pgfpathlineto{\pgfqpoint{2.003939in}{0.795123in}}%
\pgfpathlineto{\pgfqpoint{2.025789in}{0.785444in}}%
\pgfpathlineto{\pgfqpoint{2.047640in}{0.775879in}}%
\pgfpathlineto{\pgfqpoint{2.069490in}{0.765324in}}%
\pgfpathlineto{\pgfqpoint{2.091340in}{0.755432in}}%
\pgfpathlineto{\pgfqpoint{2.113190in}{0.745467in}}%
\pgfpathlineto{\pgfqpoint{2.135041in}{0.735106in}}%
\pgfpathlineto{\pgfqpoint{2.156891in}{0.724886in}}%
\pgfpathlineto{\pgfqpoint{2.178741in}{0.714928in}}%
\pgfpathlineto{\pgfqpoint{2.200592in}{0.704803in}}%
\pgfpathlineto{\pgfqpoint{2.222442in}{0.694477in}}%
\pgfpathlineto{\pgfqpoint{2.244292in}{0.684614in}}%
\pgfpathlineto{\pgfqpoint{2.266142in}{0.674556in}}%
\pgfpathlineto{\pgfqpoint{2.287993in}{0.664694in}}%
\pgfpathlineto{\pgfqpoint{2.309843in}{0.654307in}}%
\pgfpathlineto{\pgfqpoint{2.331693in}{0.644097in}}%
\pgfpathlineto{\pgfqpoint{2.353544in}{0.634261in}}%
\pgfpathlineto{\pgfqpoint{2.375394in}{0.624322in}}%
\pgfpathlineto{\pgfqpoint{2.397244in}{0.614333in}}%
\pgfpathlineto{\pgfqpoint{2.419094in}{0.604494in}}%
\pgfpathlineto{\pgfqpoint{2.440945in}{0.594718in}}%
\pgfpathlineto{\pgfqpoint{2.462795in}{0.585318in}}%
\pgfpathlineto{\pgfqpoint{2.484645in}{0.575693in}}%
\pgfpathlineto{\pgfqpoint{2.506496in}{0.566049in}}%
\pgfpathlineto{\pgfqpoint{2.528346in}{0.556221in}}%
\pgfpathlineto{\pgfqpoint{2.550196in}{0.546549in}}%
\pgfpathlineto{\pgfqpoint{2.572046in}{0.536746in}}%
\pgfpathlineto{\pgfqpoint{2.593897in}{0.526484in}}%
\pgfpathlineto{\pgfqpoint{2.615747in}{0.516663in}}%
\pgfpathlineto{\pgfqpoint{2.637597in}{0.506861in}}%
\pgfpathlineto{\pgfqpoint{2.659448in}{0.497561in}}%
\pgfpathlineto{\pgfqpoint{2.681298in}{0.487959in}}%
\pgfpathlineto{\pgfqpoint{2.703148in}{0.478729in}}%
\pgfusepath{stroke}%
\end{pgfscope}%
\begin{pgfscope}%
\pgfpathrectangle{\pgfqpoint{0.539970in}{0.422992in}}{\pgfqpoint{2.206879in}{1.226201in}}%
\pgfusepath{clip}%
\pgfsetrectcap%
\pgfsetroundjoin%
\pgfsetlinewidth{1.003750pt}%
\definecolor{currentstroke}{rgb}{1.000000,0.584314,0.000000}%
\pgfsetstrokecolor{currentstroke}%
\pgfsetdash{}{0pt}%
\pgfpathmoveto{\pgfqpoint{0.561820in}{1.577543in}}%
\pgfpathlineto{\pgfqpoint{0.583670in}{1.568284in}}%
\pgfpathlineto{\pgfqpoint{0.605521in}{1.558570in}}%
\pgfpathlineto{\pgfqpoint{0.627371in}{1.548391in}}%
\pgfpathlineto{\pgfqpoint{0.649221in}{1.538578in}}%
\pgfpathlineto{\pgfqpoint{0.671072in}{1.528590in}}%
\pgfpathlineto{\pgfqpoint{0.692922in}{1.518191in}}%
\pgfpathlineto{\pgfqpoint{0.714772in}{1.509330in}}%
\pgfpathlineto{\pgfqpoint{0.736622in}{1.499594in}}%
\pgfpathlineto{\pgfqpoint{0.758473in}{1.490162in}}%
\pgfpathlineto{\pgfqpoint{0.780323in}{1.480913in}}%
\pgfpathlineto{\pgfqpoint{0.802173in}{1.471683in}}%
\pgfpathlineto{\pgfqpoint{0.824024in}{1.462607in}}%
\pgfpathlineto{\pgfqpoint{0.845874in}{1.453322in}}%
\pgfpathlineto{\pgfqpoint{0.867724in}{1.443875in}}%
\pgfpathlineto{\pgfqpoint{0.889574in}{1.434495in}}%
\pgfpathlineto{\pgfqpoint{0.911425in}{1.425326in}}%
\pgfpathlineto{\pgfqpoint{0.933275in}{1.416930in}}%
\pgfpathlineto{\pgfqpoint{0.955125in}{1.407544in}}%
\pgfpathlineto{\pgfqpoint{0.976976in}{1.398443in}}%
\pgfpathlineto{\pgfqpoint{0.998826in}{1.389290in}}%
\pgfpathlineto{\pgfqpoint{1.020676in}{1.379734in}}%
\pgfpathlineto{\pgfqpoint{1.042526in}{1.370196in}}%
\pgfpathlineto{\pgfqpoint{1.064377in}{1.360731in}}%
\pgfpathlineto{\pgfqpoint{1.086227in}{1.351816in}}%
\pgfpathlineto{\pgfqpoint{1.108077in}{1.342650in}}%
\pgfpathlineto{\pgfqpoint{1.129928in}{1.333692in}}%
\pgfpathlineto{\pgfqpoint{1.151778in}{1.324471in}}%
\pgfpathlineto{\pgfqpoint{1.173628in}{1.315627in}}%
\pgfpathlineto{\pgfqpoint{1.195478in}{1.306056in}}%
\pgfpathlineto{\pgfqpoint{1.217329in}{1.296989in}}%
\pgfpathlineto{\pgfqpoint{1.239179in}{1.287843in}}%
\pgfpathlineto{\pgfqpoint{1.261029in}{1.278561in}}%
\pgfpathlineto{\pgfqpoint{1.282880in}{1.270033in}}%
\pgfpathlineto{\pgfqpoint{1.304730in}{1.261421in}}%
\pgfpathlineto{\pgfqpoint{1.326580in}{1.252644in}}%
\pgfpathlineto{\pgfqpoint{1.348430in}{1.243516in}}%
\pgfpathlineto{\pgfqpoint{1.370281in}{1.234903in}}%
\pgfpathlineto{\pgfqpoint{1.392131in}{1.225981in}}%
\pgfpathlineto{\pgfqpoint{1.413981in}{1.216711in}}%
\pgfpathlineto{\pgfqpoint{1.435832in}{1.207937in}}%
\pgfpathlineto{\pgfqpoint{1.457682in}{1.198865in}}%
\pgfpathlineto{\pgfqpoint{1.479532in}{1.189866in}}%
\pgfpathlineto{\pgfqpoint{1.501382in}{1.181068in}}%
\pgfpathlineto{\pgfqpoint{1.523233in}{1.172057in}}%
\pgfpathlineto{\pgfqpoint{1.545083in}{1.163606in}}%
\pgfpathlineto{\pgfqpoint{1.566933in}{1.154486in}}%
\pgfpathlineto{\pgfqpoint{1.588784in}{1.145618in}}%
\pgfpathlineto{\pgfqpoint{1.610634in}{1.137133in}}%
\pgfpathlineto{\pgfqpoint{1.632484in}{1.128625in}}%
\pgfpathlineto{\pgfqpoint{1.654334in}{1.119794in}}%
\pgfpathlineto{\pgfqpoint{1.676185in}{1.111177in}}%
\pgfpathlineto{\pgfqpoint{1.698035in}{1.102932in}}%
\pgfpathlineto{\pgfqpoint{1.719885in}{1.094073in}}%
\pgfpathlineto{\pgfqpoint{1.741736in}{1.085248in}}%
\pgfpathlineto{\pgfqpoint{1.763586in}{1.076325in}}%
\pgfpathlineto{\pgfqpoint{1.785436in}{1.067779in}}%
\pgfpathlineto{\pgfqpoint{1.807286in}{1.058944in}}%
\pgfpathlineto{\pgfqpoint{1.829137in}{1.050514in}}%
\pgfpathlineto{\pgfqpoint{1.850987in}{1.041681in}}%
\pgfpathlineto{\pgfqpoint{1.872837in}{1.033326in}}%
\pgfpathlineto{\pgfqpoint{1.894688in}{1.024482in}}%
\pgfpathlineto{\pgfqpoint{1.916538in}{1.016291in}}%
\pgfpathlineto{\pgfqpoint{1.938388in}{1.007386in}}%
\pgfpathlineto{\pgfqpoint{1.960238in}{0.998999in}}%
\pgfpathlineto{\pgfqpoint{1.982089in}{0.990410in}}%
\pgfpathlineto{\pgfqpoint{2.003939in}{0.981149in}}%
\pgfpathlineto{\pgfqpoint{2.025789in}{0.972675in}}%
\pgfpathlineto{\pgfqpoint{2.047640in}{0.964000in}}%
\pgfpathlineto{\pgfqpoint{2.069490in}{0.955591in}}%
\pgfpathlineto{\pgfqpoint{2.091340in}{0.946920in}}%
\pgfpathlineto{\pgfqpoint{2.113190in}{0.938435in}}%
\pgfpathlineto{\pgfqpoint{2.135041in}{0.929754in}}%
\pgfpathlineto{\pgfqpoint{2.156891in}{0.920982in}}%
\pgfpathlineto{\pgfqpoint{2.178741in}{0.912075in}}%
\pgfpathlineto{\pgfqpoint{2.200592in}{0.903010in}}%
\pgfpathlineto{\pgfqpoint{2.222442in}{0.894206in}}%
\pgfpathlineto{\pgfqpoint{2.244292in}{0.885232in}}%
\pgfpathlineto{\pgfqpoint{2.266142in}{0.876377in}}%
\pgfpathlineto{\pgfqpoint{2.287993in}{0.867501in}}%
\pgfpathlineto{\pgfqpoint{2.309843in}{0.858566in}}%
\pgfpathlineto{\pgfqpoint{2.331693in}{0.849115in}}%
\pgfpathlineto{\pgfqpoint{2.353544in}{0.840249in}}%
\pgfpathlineto{\pgfqpoint{2.375394in}{0.831145in}}%
\pgfpathlineto{\pgfqpoint{2.397244in}{0.821956in}}%
\pgfpathlineto{\pgfqpoint{2.419094in}{0.812833in}}%
\pgfpathlineto{\pgfqpoint{2.440945in}{0.803830in}}%
\pgfpathlineto{\pgfqpoint{2.462795in}{0.794653in}}%
\pgfpathlineto{\pgfqpoint{2.484645in}{0.785338in}}%
\pgfpathlineto{\pgfqpoint{2.506496in}{0.776270in}}%
\pgfpathlineto{\pgfqpoint{2.528346in}{0.767165in}}%
\pgfpathlineto{\pgfqpoint{2.550196in}{0.757872in}}%
\pgfpathlineto{\pgfqpoint{2.572046in}{0.749142in}}%
\pgfpathlineto{\pgfqpoint{2.593897in}{0.740114in}}%
\pgfpathlineto{\pgfqpoint{2.615747in}{0.730965in}}%
\pgfpathlineto{\pgfqpoint{2.637597in}{0.721700in}}%
\pgfpathlineto{\pgfqpoint{2.659448in}{0.712547in}}%
\pgfpathlineto{\pgfqpoint{2.681298in}{0.702841in}}%
\pgfpathlineto{\pgfqpoint{2.703148in}{0.693618in}}%
\pgfusepath{stroke}%
\end{pgfscope}%
\begin{pgfscope}%
\pgfpathrectangle{\pgfqpoint{0.539970in}{0.422992in}}{\pgfqpoint{2.206879in}{1.226201in}}%
\pgfusepath{clip}%
\pgfsetrectcap%
\pgfsetroundjoin%
\pgfsetlinewidth{1.003750pt}%
\definecolor{currentstroke}{rgb}{1.000000,0.172549,0.000000}%
\pgfsetstrokecolor{currentstroke}%
\pgfsetdash{}{0pt}%
\pgfpathmoveto{\pgfqpoint{0.561820in}{1.593457in}}%
\pgfpathlineto{\pgfqpoint{0.583670in}{1.589184in}}%
\pgfpathlineto{\pgfqpoint{0.605521in}{1.584969in}}%
\pgfpathlineto{\pgfqpoint{0.627371in}{1.580664in}}%
\pgfpathlineto{\pgfqpoint{0.649221in}{1.576027in}}%
\pgfpathlineto{\pgfqpoint{0.671072in}{1.571514in}}%
\pgfpathlineto{\pgfqpoint{0.692922in}{1.567181in}}%
\pgfpathlineto{\pgfqpoint{0.714772in}{1.562928in}}%
\pgfpathlineto{\pgfqpoint{0.736622in}{1.558291in}}%
\pgfpathlineto{\pgfqpoint{0.758473in}{1.553928in}}%
\pgfpathlineto{\pgfqpoint{0.780323in}{1.549626in}}%
\pgfpathlineto{\pgfqpoint{0.802173in}{1.545547in}}%
\pgfpathlineto{\pgfqpoint{0.824024in}{1.541378in}}%
\pgfpathlineto{\pgfqpoint{0.845874in}{1.536942in}}%
\pgfpathlineto{\pgfqpoint{0.867724in}{1.533301in}}%
\pgfpathlineto{\pgfqpoint{0.889574in}{1.528792in}}%
\pgfpathlineto{\pgfqpoint{0.911425in}{1.524758in}}%
\pgfpathlineto{\pgfqpoint{0.933275in}{1.520845in}}%
\pgfpathlineto{\pgfqpoint{0.955125in}{1.516578in}}%
\pgfpathlineto{\pgfqpoint{0.976976in}{1.512556in}}%
\pgfpathlineto{\pgfqpoint{0.998826in}{1.508697in}}%
\pgfpathlineto{\pgfqpoint{1.020676in}{1.504549in}}%
\pgfpathlineto{\pgfqpoint{1.042526in}{1.500687in}}%
\pgfpathlineto{\pgfqpoint{1.064377in}{1.496522in}}%
\pgfpathlineto{\pgfqpoint{1.086227in}{1.492580in}}%
\pgfpathlineto{\pgfqpoint{1.108077in}{1.488866in}}%
\pgfpathlineto{\pgfqpoint{1.129928in}{1.484964in}}%
\pgfpathlineto{\pgfqpoint{1.151778in}{1.480716in}}%
\pgfpathlineto{\pgfqpoint{1.173628in}{1.476455in}}%
\pgfpathlineto{\pgfqpoint{1.195478in}{1.472771in}}%
\pgfpathlineto{\pgfqpoint{1.217329in}{1.468798in}}%
\pgfpathlineto{\pgfqpoint{1.239179in}{1.464872in}}%
\pgfpathlineto{\pgfqpoint{1.261029in}{1.461136in}}%
\pgfpathlineto{\pgfqpoint{1.282880in}{1.457339in}}%
\pgfpathlineto{\pgfqpoint{1.304730in}{1.453619in}}%
\pgfpathlineto{\pgfqpoint{1.326580in}{1.449888in}}%
\pgfpathlineto{\pgfqpoint{1.348430in}{1.445647in}}%
\pgfpathlineto{\pgfqpoint{1.370281in}{1.441942in}}%
\pgfpathlineto{\pgfqpoint{1.392131in}{1.437998in}}%
\pgfpathlineto{\pgfqpoint{1.413981in}{1.434029in}}%
\pgfpathlineto{\pgfqpoint{1.435832in}{1.430020in}}%
\pgfpathlineto{\pgfqpoint{1.457682in}{1.426110in}}%
\pgfpathlineto{\pgfqpoint{1.479532in}{1.422162in}}%
\pgfpathlineto{\pgfqpoint{1.501382in}{1.418518in}}%
\pgfpathlineto{\pgfqpoint{1.523233in}{1.414634in}}%
\pgfpathlineto{\pgfqpoint{1.545083in}{1.410371in}}%
\pgfpathlineto{\pgfqpoint{1.566933in}{1.406118in}}%
\pgfpathlineto{\pgfqpoint{1.588784in}{1.402663in}}%
\pgfpathlineto{\pgfqpoint{1.610634in}{1.398481in}}%
\pgfpathlineto{\pgfqpoint{1.632484in}{1.394557in}}%
\pgfpathlineto{\pgfqpoint{1.654334in}{1.390385in}}%
\pgfpathlineto{\pgfqpoint{1.676185in}{1.386567in}}%
\pgfpathlineto{\pgfqpoint{1.698035in}{1.382398in}}%
\pgfpathlineto{\pgfqpoint{1.719885in}{1.378269in}}%
\pgfpathlineto{\pgfqpoint{1.741736in}{1.374455in}}%
\pgfpathlineto{\pgfqpoint{1.763586in}{1.370299in}}%
\pgfpathlineto{\pgfqpoint{1.785436in}{1.366412in}}%
\pgfpathlineto{\pgfqpoint{1.807286in}{1.362453in}}%
\pgfpathlineto{\pgfqpoint{1.829137in}{1.358516in}}%
\pgfpathlineto{\pgfqpoint{1.850987in}{1.354120in}}%
\pgfpathlineto{\pgfqpoint{1.872837in}{1.349940in}}%
\pgfpathlineto{\pgfqpoint{1.894688in}{1.345785in}}%
\pgfpathlineto{\pgfqpoint{1.916538in}{1.341225in}}%
\pgfpathlineto{\pgfqpoint{1.938388in}{1.336951in}}%
\pgfpathlineto{\pgfqpoint{1.960238in}{1.332575in}}%
\pgfpathlineto{\pgfqpoint{1.982089in}{1.328407in}}%
\pgfpathlineto{\pgfqpoint{2.003939in}{1.323973in}}%
\pgfpathlineto{\pgfqpoint{2.025789in}{1.319697in}}%
\pgfpathlineto{\pgfqpoint{2.047640in}{1.315278in}}%
\pgfpathlineto{\pgfqpoint{2.069490in}{1.310839in}}%
\pgfpathlineto{\pgfqpoint{2.091340in}{1.306028in}}%
\pgfpathlineto{\pgfqpoint{2.113190in}{1.301482in}}%
\pgfpathlineto{\pgfqpoint{2.135041in}{1.297107in}}%
\pgfpathlineto{\pgfqpoint{2.156891in}{1.292521in}}%
\pgfpathlineto{\pgfqpoint{2.178741in}{1.287475in}}%
\pgfpathlineto{\pgfqpoint{2.200592in}{1.283146in}}%
\pgfpathlineto{\pgfqpoint{2.222442in}{1.278419in}}%
\pgfpathlineto{\pgfqpoint{2.244292in}{1.273265in}}%
\pgfpathlineto{\pgfqpoint{2.266142in}{1.268698in}}%
\pgfpathlineto{\pgfqpoint{2.287993in}{1.263596in}}%
\pgfpathlineto{\pgfqpoint{2.309843in}{1.258728in}}%
\pgfpathlineto{\pgfqpoint{2.331693in}{1.253666in}}%
\pgfpathlineto{\pgfqpoint{2.353544in}{1.248558in}}%
\pgfpathlineto{\pgfqpoint{2.375394in}{1.243385in}}%
\pgfpathlineto{\pgfqpoint{2.397244in}{1.237987in}}%
\pgfpathlineto{\pgfqpoint{2.419094in}{1.232685in}}%
\pgfpathlineto{\pgfqpoint{2.440945in}{1.226950in}}%
\pgfpathlineto{\pgfqpoint{2.462795in}{1.221763in}}%
\pgfpathlineto{\pgfqpoint{2.484645in}{1.216076in}}%
\pgfpathlineto{\pgfqpoint{2.506496in}{1.210764in}}%
\pgfpathlineto{\pgfqpoint{2.528346in}{1.205457in}}%
\pgfpathlineto{\pgfqpoint{2.550196in}{1.199614in}}%
\pgfpathlineto{\pgfqpoint{2.572046in}{1.194352in}}%
\pgfpathlineto{\pgfqpoint{2.593897in}{1.188935in}}%
\pgfpathlineto{\pgfqpoint{2.615747in}{1.183402in}}%
\pgfpathlineto{\pgfqpoint{2.637597in}{1.178203in}}%
\pgfpathlineto{\pgfqpoint{2.659448in}{1.172647in}}%
\pgfpathlineto{\pgfqpoint{2.681298in}{1.166890in}}%
\pgfpathlineto{\pgfqpoint{2.703148in}{1.161234in}}%
\pgfusepath{stroke}%
\end{pgfscope}%
\begin{pgfscope}%
\pgfpathrectangle{\pgfqpoint{0.539970in}{0.422992in}}{\pgfqpoint{2.206879in}{1.226201in}}%
\pgfusepath{clip}%
\pgfsetrectcap%
\pgfsetroundjoin%
\pgfsetlinewidth{1.003750pt}%
\definecolor{currentstroke}{rgb}{0.517647,0.356863,0.592157}%
\pgfsetstrokecolor{currentstroke}%
\pgfsetdash{}{0pt}%
\pgfpathmoveto{\pgfqpoint{0.561820in}{1.561773in}}%
\pgfpathlineto{\pgfqpoint{0.583670in}{1.547673in}}%
\pgfpathlineto{\pgfqpoint{0.605521in}{1.534425in}}%
\pgfpathlineto{\pgfqpoint{0.627371in}{1.520934in}}%
\pgfpathlineto{\pgfqpoint{0.649221in}{1.508214in}}%
\pgfpathlineto{\pgfqpoint{0.671072in}{1.495021in}}%
\pgfpathlineto{\pgfqpoint{0.692922in}{1.482387in}}%
\pgfpathlineto{\pgfqpoint{0.714772in}{1.470144in}}%
\pgfpathlineto{\pgfqpoint{0.736622in}{1.457398in}}%
\pgfpathlineto{\pgfqpoint{0.758473in}{1.445052in}}%
\pgfpathlineto{\pgfqpoint{0.780323in}{1.432778in}}%
\pgfpathlineto{\pgfqpoint{0.802173in}{1.420610in}}%
\pgfpathlineto{\pgfqpoint{0.824024in}{1.408839in}}%
\pgfpathlineto{\pgfqpoint{0.845874in}{1.397211in}}%
\pgfpathlineto{\pgfqpoint{0.867724in}{1.386327in}}%
\pgfpathlineto{\pgfqpoint{0.889574in}{1.375202in}}%
\pgfpathlineto{\pgfqpoint{0.911425in}{1.364244in}}%
\pgfpathlineto{\pgfqpoint{0.933275in}{1.352774in}}%
\pgfpathlineto{\pgfqpoint{0.955125in}{1.342141in}}%
\pgfpathlineto{\pgfqpoint{0.976976in}{1.331123in}}%
\pgfpathlineto{\pgfqpoint{0.998826in}{1.320517in}}%
\pgfpathlineto{\pgfqpoint{1.020676in}{1.309975in}}%
\pgfpathlineto{\pgfqpoint{1.042526in}{1.299274in}}%
\pgfpathlineto{\pgfqpoint{1.064377in}{1.289205in}}%
\pgfpathlineto{\pgfqpoint{1.086227in}{1.279286in}}%
\pgfpathlineto{\pgfqpoint{1.108077in}{1.268743in}}%
\pgfpathlineto{\pgfqpoint{1.129928in}{1.257952in}}%
\pgfpathlineto{\pgfqpoint{1.151778in}{1.248003in}}%
\pgfpathlineto{\pgfqpoint{1.173628in}{1.237927in}}%
\pgfpathlineto{\pgfqpoint{1.195478in}{1.227663in}}%
\pgfpathlineto{\pgfqpoint{1.217329in}{1.217728in}}%
\pgfpathlineto{\pgfqpoint{1.239179in}{1.207518in}}%
\pgfpathlineto{\pgfqpoint{1.261029in}{1.197534in}}%
\pgfpathlineto{\pgfqpoint{1.282880in}{1.187201in}}%
\pgfpathlineto{\pgfqpoint{1.304730in}{1.176745in}}%
\pgfpathlineto{\pgfqpoint{1.326580in}{1.166674in}}%
\pgfpathlineto{\pgfqpoint{1.348430in}{1.156535in}}%
\pgfpathlineto{\pgfqpoint{1.370281in}{1.146255in}}%
\pgfpathlineto{\pgfqpoint{1.392131in}{1.136290in}}%
\pgfpathlineto{\pgfqpoint{1.413981in}{1.126831in}}%
\pgfpathlineto{\pgfqpoint{1.435832in}{1.116293in}}%
\pgfpathlineto{\pgfqpoint{1.457682in}{1.106056in}}%
\pgfpathlineto{\pgfqpoint{1.479532in}{1.095954in}}%
\pgfpathlineto{\pgfqpoint{1.501382in}{1.085940in}}%
\pgfpathlineto{\pgfqpoint{1.523233in}{1.076130in}}%
\pgfpathlineto{\pgfqpoint{1.545083in}{1.066529in}}%
\pgfpathlineto{\pgfqpoint{1.566933in}{1.056749in}}%
\pgfpathlineto{\pgfqpoint{1.588784in}{1.046580in}}%
\pgfpathlineto{\pgfqpoint{1.610634in}{1.037016in}}%
\pgfpathlineto{\pgfqpoint{1.632484in}{1.026603in}}%
\pgfpathlineto{\pgfqpoint{1.654334in}{1.017022in}}%
\pgfpathlineto{\pgfqpoint{1.676185in}{1.006641in}}%
\pgfpathlineto{\pgfqpoint{1.698035in}{0.997034in}}%
\pgfpathlineto{\pgfqpoint{1.719885in}{0.987011in}}%
\pgfpathlineto{\pgfqpoint{1.741736in}{0.976832in}}%
\pgfpathlineto{\pgfqpoint{1.763586in}{0.966929in}}%
\pgfpathlineto{\pgfqpoint{1.785436in}{0.956565in}}%
\pgfpathlineto{\pgfqpoint{1.807286in}{0.946666in}}%
\pgfpathlineto{\pgfqpoint{1.829137in}{0.936782in}}%
\pgfpathlineto{\pgfqpoint{1.850987in}{0.927016in}}%
\pgfpathlineto{\pgfqpoint{1.872837in}{0.917266in}}%
\pgfpathlineto{\pgfqpoint{1.894688in}{0.907488in}}%
\pgfpathlineto{\pgfqpoint{1.916538in}{0.898011in}}%
\pgfpathlineto{\pgfqpoint{1.938388in}{0.887834in}}%
\pgfpathlineto{\pgfqpoint{1.960238in}{0.878076in}}%
\pgfpathlineto{\pgfqpoint{1.982089in}{0.868692in}}%
\pgfpathlineto{\pgfqpoint{2.003939in}{0.859174in}}%
\pgfpathlineto{\pgfqpoint{2.025789in}{0.849620in}}%
\pgfpathlineto{\pgfqpoint{2.047640in}{0.839752in}}%
\pgfpathlineto{\pgfqpoint{2.069490in}{0.830002in}}%
\pgfpathlineto{\pgfqpoint{2.091340in}{0.820479in}}%
\pgfpathlineto{\pgfqpoint{2.113190in}{0.810532in}}%
\pgfpathlineto{\pgfqpoint{2.135041in}{0.801045in}}%
\pgfpathlineto{\pgfqpoint{2.156891in}{0.791718in}}%
\pgfpathlineto{\pgfqpoint{2.178741in}{0.782339in}}%
\pgfpathlineto{\pgfqpoint{2.200592in}{0.772407in}}%
\pgfpathlineto{\pgfqpoint{2.222442in}{0.762679in}}%
\pgfpathlineto{\pgfqpoint{2.244292in}{0.752996in}}%
\pgfpathlineto{\pgfqpoint{2.266142in}{0.743647in}}%
\pgfpathlineto{\pgfqpoint{2.287993in}{0.733651in}}%
\pgfpathlineto{\pgfqpoint{2.309843in}{0.724318in}}%
\pgfpathlineto{\pgfqpoint{2.331693in}{0.714154in}}%
\pgfpathlineto{\pgfqpoint{2.353544in}{0.704434in}}%
\pgfpathlineto{\pgfqpoint{2.375394in}{0.694404in}}%
\pgfpathlineto{\pgfqpoint{2.397244in}{0.684539in}}%
\pgfpathlineto{\pgfqpoint{2.419094in}{0.674529in}}%
\pgfpathlineto{\pgfqpoint{2.440945in}{0.664194in}}%
\pgfpathlineto{\pgfqpoint{2.462795in}{0.654430in}}%
\pgfpathlineto{\pgfqpoint{2.484645in}{0.644690in}}%
\pgfpathlineto{\pgfqpoint{2.506496in}{0.634604in}}%
\pgfpathlineto{\pgfqpoint{2.528346in}{0.624786in}}%
\pgfpathlineto{\pgfqpoint{2.550196in}{0.615406in}}%
\pgfpathlineto{\pgfqpoint{2.572046in}{0.605733in}}%
\pgfpathlineto{\pgfqpoint{2.593897in}{0.596115in}}%
\pgfpathlineto{\pgfqpoint{2.615747in}{0.586614in}}%
\pgfpathlineto{\pgfqpoint{2.637597in}{0.576743in}}%
\pgfpathlineto{\pgfqpoint{2.659448in}{0.566925in}}%
\pgfpathlineto{\pgfqpoint{2.681298in}{0.556860in}}%
\pgfpathlineto{\pgfqpoint{2.703148in}{0.547425in}}%
\pgfusepath{stroke}%
\end{pgfscope}%
\begin{pgfscope}%
\pgfsetrectcap%
\pgfsetmiterjoin%
\pgfsetlinewidth{0.501875pt}%
\definecolor{currentstroke}{rgb}{0.000000,0.000000,0.000000}%
\pgfsetstrokecolor{currentstroke}%
\pgfsetdash{}{0pt}%
\pgfpathmoveto{\pgfqpoint{0.539970in}{0.422992in}}%
\pgfpathlineto{\pgfqpoint{0.539970in}{1.649193in}}%
\pgfusepath{stroke}%
\end{pgfscope}%
\begin{pgfscope}%
\pgfsetrectcap%
\pgfsetmiterjoin%
\pgfsetlinewidth{0.501875pt}%
\definecolor{currentstroke}{rgb}{0.000000,0.000000,0.000000}%
\pgfsetstrokecolor{currentstroke}%
\pgfsetdash{}{0pt}%
\pgfpathmoveto{\pgfqpoint{2.746849in}{0.422992in}}%
\pgfpathlineto{\pgfqpoint{2.746849in}{1.649193in}}%
\pgfusepath{stroke}%
\end{pgfscope}%
\begin{pgfscope}%
\pgfsetrectcap%
\pgfsetmiterjoin%
\pgfsetlinewidth{0.501875pt}%
\definecolor{currentstroke}{rgb}{0.000000,0.000000,0.000000}%
\pgfsetstrokecolor{currentstroke}%
\pgfsetdash{}{0pt}%
\pgfpathmoveto{\pgfqpoint{0.539970in}{0.422992in}}%
\pgfpathlineto{\pgfqpoint{2.746849in}{0.422992in}}%
\pgfusepath{stroke}%
\end{pgfscope}%
\begin{pgfscope}%
\pgfsetrectcap%
\pgfsetmiterjoin%
\pgfsetlinewidth{0.501875pt}%
\definecolor{currentstroke}{rgb}{0.000000,0.000000,0.000000}%
\pgfsetstrokecolor{currentstroke}%
\pgfsetdash{}{0pt}%
\pgfpathmoveto{\pgfqpoint{0.539970in}{1.649193in}}%
\pgfpathlineto{\pgfqpoint{2.746849in}{1.649193in}}%
\pgfusepath{stroke}%
\end{pgfscope}%
\begin{pgfscope}%
\definecolor{textcolor}{rgb}{0.000000,0.000000,0.000000}%
\pgfsetstrokecolor{textcolor}%
\pgfsetfillcolor{textcolor}%
\pgftext[x=1.643409in,y=1.732526in,,base]{\color{textcolor}\rmfamily\fontsize{12.000000}{14.400000}\selectfont Koninuität}%
\end{pgfscope}%
\begin{pgfscope}%
\pgfsetbuttcap%
\pgfsetmiterjoin%
\definecolor{currentfill}{rgb}{1.000000,1.000000,1.000000}%
\pgfsetfillcolor{currentfill}%
\pgfsetlinewidth{0.000000pt}%
\definecolor{currentstroke}{rgb}{0.000000,0.000000,0.000000}%
\pgfsetstrokecolor{currentstroke}%
\pgfsetstrokeopacity{0.000000}%
\pgfsetdash{}{0pt}%
\pgfpathmoveto{\pgfqpoint{3.385377in}{0.422992in}}%
\pgfpathlineto{\pgfqpoint{5.592256in}{0.422992in}}%
\pgfpathlineto{\pgfqpoint{5.592256in}{3.574193in}}%
\pgfpathlineto{\pgfqpoint{3.385377in}{3.574193in}}%
\pgfpathlineto{\pgfqpoint{3.385377in}{0.422992in}}%
\pgfpathclose%
\pgfusepath{fill}%
\end{pgfscope}%
\begin{pgfscope}%
\pgfsetbuttcap%
\pgfsetroundjoin%
\definecolor{currentfill}{rgb}{0.000000,0.000000,0.000000}%
\pgfsetfillcolor{currentfill}%
\pgfsetlinewidth{0.501875pt}%
\definecolor{currentstroke}{rgb}{0.000000,0.000000,0.000000}%
\pgfsetstrokecolor{currentstroke}%
\pgfsetdash{}{0pt}%
\pgfsys@defobject{currentmarker}{\pgfqpoint{0.000000in}{0.000000in}}{\pgfqpoint{0.000000in}{0.041667in}}{%
\pgfpathmoveto{\pgfqpoint{0.000000in}{0.000000in}}%
\pgfpathlineto{\pgfqpoint{0.000000in}{0.041667in}}%
\pgfusepath{stroke,fill}%
}%
\begin{pgfscope}%
\pgfsys@transformshift{3.385377in}{0.422992in}%
\pgfsys@useobject{currentmarker}{}%
\end{pgfscope}%
\end{pgfscope}%
\begin{pgfscope}%
\pgfsetbuttcap%
\pgfsetroundjoin%
\definecolor{currentfill}{rgb}{0.000000,0.000000,0.000000}%
\pgfsetfillcolor{currentfill}%
\pgfsetlinewidth{0.501875pt}%
\definecolor{currentstroke}{rgb}{0.000000,0.000000,0.000000}%
\pgfsetstrokecolor{currentstroke}%
\pgfsetdash{}{0pt}%
\pgfsys@defobject{currentmarker}{\pgfqpoint{0.000000in}{-0.041667in}}{\pgfqpoint{0.000000in}{0.000000in}}{%
\pgfpathmoveto{\pgfqpoint{0.000000in}{0.000000in}}%
\pgfpathlineto{\pgfqpoint{0.000000in}{-0.041667in}}%
\pgfusepath{stroke,fill}%
}%
\begin{pgfscope}%
\pgfsys@transformshift{3.385377in}{3.574193in}%
\pgfsys@useobject{currentmarker}{}%
\end{pgfscope}%
\end{pgfscope}%
\begin{pgfscope}%
\definecolor{textcolor}{rgb}{0.000000,0.000000,0.000000}%
\pgfsetstrokecolor{textcolor}%
\pgfsetfillcolor{textcolor}%
\pgftext[x=3.385377in,y=0.374381in,,top]{\color{textcolor}\rmfamily\fontsize{10.000000}{12.000000}\selectfont \(\displaystyle {0}\)}%
\end{pgfscope}%
\begin{pgfscope}%
\pgfsetbuttcap%
\pgfsetroundjoin%
\definecolor{currentfill}{rgb}{0.000000,0.000000,0.000000}%
\pgfsetfillcolor{currentfill}%
\pgfsetlinewidth{0.501875pt}%
\definecolor{currentstroke}{rgb}{0.000000,0.000000,0.000000}%
\pgfsetstrokecolor{currentstroke}%
\pgfsetdash{}{0pt}%
\pgfsys@defobject{currentmarker}{\pgfqpoint{0.000000in}{0.000000in}}{\pgfqpoint{0.000000in}{0.041667in}}{%
\pgfpathmoveto{\pgfqpoint{0.000000in}{0.000000in}}%
\pgfpathlineto{\pgfqpoint{0.000000in}{0.041667in}}%
\pgfusepath{stroke,fill}%
}%
\begin{pgfscope}%
\pgfsys@transformshift{3.822383in}{0.422992in}%
\pgfsys@useobject{currentmarker}{}%
\end{pgfscope}%
\end{pgfscope}%
\begin{pgfscope}%
\pgfsetbuttcap%
\pgfsetroundjoin%
\definecolor{currentfill}{rgb}{0.000000,0.000000,0.000000}%
\pgfsetfillcolor{currentfill}%
\pgfsetlinewidth{0.501875pt}%
\definecolor{currentstroke}{rgb}{0.000000,0.000000,0.000000}%
\pgfsetstrokecolor{currentstroke}%
\pgfsetdash{}{0pt}%
\pgfsys@defobject{currentmarker}{\pgfqpoint{0.000000in}{-0.041667in}}{\pgfqpoint{0.000000in}{0.000000in}}{%
\pgfpathmoveto{\pgfqpoint{0.000000in}{0.000000in}}%
\pgfpathlineto{\pgfqpoint{0.000000in}{-0.041667in}}%
\pgfusepath{stroke,fill}%
}%
\begin{pgfscope}%
\pgfsys@transformshift{3.822383in}{3.574193in}%
\pgfsys@useobject{currentmarker}{}%
\end{pgfscope}%
\end{pgfscope}%
\begin{pgfscope}%
\definecolor{textcolor}{rgb}{0.000000,0.000000,0.000000}%
\pgfsetstrokecolor{textcolor}%
\pgfsetfillcolor{textcolor}%
\pgftext[x=3.822383in,y=0.374381in,,top]{\color{textcolor}\rmfamily\fontsize{10.000000}{12.000000}\selectfont \(\displaystyle {20}\)}%
\end{pgfscope}%
\begin{pgfscope}%
\pgfsetbuttcap%
\pgfsetroundjoin%
\definecolor{currentfill}{rgb}{0.000000,0.000000,0.000000}%
\pgfsetfillcolor{currentfill}%
\pgfsetlinewidth{0.501875pt}%
\definecolor{currentstroke}{rgb}{0.000000,0.000000,0.000000}%
\pgfsetstrokecolor{currentstroke}%
\pgfsetdash{}{0pt}%
\pgfsys@defobject{currentmarker}{\pgfqpoint{0.000000in}{0.000000in}}{\pgfqpoint{0.000000in}{0.041667in}}{%
\pgfpathmoveto{\pgfqpoint{0.000000in}{0.000000in}}%
\pgfpathlineto{\pgfqpoint{0.000000in}{0.041667in}}%
\pgfusepath{stroke,fill}%
}%
\begin{pgfscope}%
\pgfsys@transformshift{4.259389in}{0.422992in}%
\pgfsys@useobject{currentmarker}{}%
\end{pgfscope}%
\end{pgfscope}%
\begin{pgfscope}%
\pgfsetbuttcap%
\pgfsetroundjoin%
\definecolor{currentfill}{rgb}{0.000000,0.000000,0.000000}%
\pgfsetfillcolor{currentfill}%
\pgfsetlinewidth{0.501875pt}%
\definecolor{currentstroke}{rgb}{0.000000,0.000000,0.000000}%
\pgfsetstrokecolor{currentstroke}%
\pgfsetdash{}{0pt}%
\pgfsys@defobject{currentmarker}{\pgfqpoint{0.000000in}{-0.041667in}}{\pgfqpoint{0.000000in}{0.000000in}}{%
\pgfpathmoveto{\pgfqpoint{0.000000in}{0.000000in}}%
\pgfpathlineto{\pgfqpoint{0.000000in}{-0.041667in}}%
\pgfusepath{stroke,fill}%
}%
\begin{pgfscope}%
\pgfsys@transformshift{4.259389in}{3.574193in}%
\pgfsys@useobject{currentmarker}{}%
\end{pgfscope}%
\end{pgfscope}%
\begin{pgfscope}%
\definecolor{textcolor}{rgb}{0.000000,0.000000,0.000000}%
\pgfsetstrokecolor{textcolor}%
\pgfsetfillcolor{textcolor}%
\pgftext[x=4.259389in,y=0.374381in,,top]{\color{textcolor}\rmfamily\fontsize{10.000000}{12.000000}\selectfont \(\displaystyle {40}\)}%
\end{pgfscope}%
\begin{pgfscope}%
\pgfsetbuttcap%
\pgfsetroundjoin%
\definecolor{currentfill}{rgb}{0.000000,0.000000,0.000000}%
\pgfsetfillcolor{currentfill}%
\pgfsetlinewidth{0.501875pt}%
\definecolor{currentstroke}{rgb}{0.000000,0.000000,0.000000}%
\pgfsetstrokecolor{currentstroke}%
\pgfsetdash{}{0pt}%
\pgfsys@defobject{currentmarker}{\pgfqpoint{0.000000in}{0.000000in}}{\pgfqpoint{0.000000in}{0.041667in}}{%
\pgfpathmoveto{\pgfqpoint{0.000000in}{0.000000in}}%
\pgfpathlineto{\pgfqpoint{0.000000in}{0.041667in}}%
\pgfusepath{stroke,fill}%
}%
\begin{pgfscope}%
\pgfsys@transformshift{4.696394in}{0.422992in}%
\pgfsys@useobject{currentmarker}{}%
\end{pgfscope}%
\end{pgfscope}%
\begin{pgfscope}%
\pgfsetbuttcap%
\pgfsetroundjoin%
\definecolor{currentfill}{rgb}{0.000000,0.000000,0.000000}%
\pgfsetfillcolor{currentfill}%
\pgfsetlinewidth{0.501875pt}%
\definecolor{currentstroke}{rgb}{0.000000,0.000000,0.000000}%
\pgfsetstrokecolor{currentstroke}%
\pgfsetdash{}{0pt}%
\pgfsys@defobject{currentmarker}{\pgfqpoint{0.000000in}{-0.041667in}}{\pgfqpoint{0.000000in}{0.000000in}}{%
\pgfpathmoveto{\pgfqpoint{0.000000in}{0.000000in}}%
\pgfpathlineto{\pgfqpoint{0.000000in}{-0.041667in}}%
\pgfusepath{stroke,fill}%
}%
\begin{pgfscope}%
\pgfsys@transformshift{4.696394in}{3.574193in}%
\pgfsys@useobject{currentmarker}{}%
\end{pgfscope}%
\end{pgfscope}%
\begin{pgfscope}%
\definecolor{textcolor}{rgb}{0.000000,0.000000,0.000000}%
\pgfsetstrokecolor{textcolor}%
\pgfsetfillcolor{textcolor}%
\pgftext[x=4.696394in,y=0.374381in,,top]{\color{textcolor}\rmfamily\fontsize{10.000000}{12.000000}\selectfont \(\displaystyle {60}\)}%
\end{pgfscope}%
\begin{pgfscope}%
\pgfsetbuttcap%
\pgfsetroundjoin%
\definecolor{currentfill}{rgb}{0.000000,0.000000,0.000000}%
\pgfsetfillcolor{currentfill}%
\pgfsetlinewidth{0.501875pt}%
\definecolor{currentstroke}{rgb}{0.000000,0.000000,0.000000}%
\pgfsetstrokecolor{currentstroke}%
\pgfsetdash{}{0pt}%
\pgfsys@defobject{currentmarker}{\pgfqpoint{0.000000in}{0.000000in}}{\pgfqpoint{0.000000in}{0.041667in}}{%
\pgfpathmoveto{\pgfqpoint{0.000000in}{0.000000in}}%
\pgfpathlineto{\pgfqpoint{0.000000in}{0.041667in}}%
\pgfusepath{stroke,fill}%
}%
\begin{pgfscope}%
\pgfsys@transformshift{5.133400in}{0.422992in}%
\pgfsys@useobject{currentmarker}{}%
\end{pgfscope}%
\end{pgfscope}%
\begin{pgfscope}%
\pgfsetbuttcap%
\pgfsetroundjoin%
\definecolor{currentfill}{rgb}{0.000000,0.000000,0.000000}%
\pgfsetfillcolor{currentfill}%
\pgfsetlinewidth{0.501875pt}%
\definecolor{currentstroke}{rgb}{0.000000,0.000000,0.000000}%
\pgfsetstrokecolor{currentstroke}%
\pgfsetdash{}{0pt}%
\pgfsys@defobject{currentmarker}{\pgfqpoint{0.000000in}{-0.041667in}}{\pgfqpoint{0.000000in}{0.000000in}}{%
\pgfpathmoveto{\pgfqpoint{0.000000in}{0.000000in}}%
\pgfpathlineto{\pgfqpoint{0.000000in}{-0.041667in}}%
\pgfusepath{stroke,fill}%
}%
\begin{pgfscope}%
\pgfsys@transformshift{5.133400in}{3.574193in}%
\pgfsys@useobject{currentmarker}{}%
\end{pgfscope}%
\end{pgfscope}%
\begin{pgfscope}%
\definecolor{textcolor}{rgb}{0.000000,0.000000,0.000000}%
\pgfsetstrokecolor{textcolor}%
\pgfsetfillcolor{textcolor}%
\pgftext[x=5.133400in,y=0.374381in,,top]{\color{textcolor}\rmfamily\fontsize{10.000000}{12.000000}\selectfont \(\displaystyle {80}\)}%
\end{pgfscope}%
\begin{pgfscope}%
\pgfsetbuttcap%
\pgfsetroundjoin%
\definecolor{currentfill}{rgb}{0.000000,0.000000,0.000000}%
\pgfsetfillcolor{currentfill}%
\pgfsetlinewidth{0.501875pt}%
\definecolor{currentstroke}{rgb}{0.000000,0.000000,0.000000}%
\pgfsetstrokecolor{currentstroke}%
\pgfsetdash{}{0pt}%
\pgfsys@defobject{currentmarker}{\pgfqpoint{0.000000in}{0.000000in}}{\pgfqpoint{0.000000in}{0.020833in}}{%
\pgfpathmoveto{\pgfqpoint{0.000000in}{0.000000in}}%
\pgfpathlineto{\pgfqpoint{0.000000in}{0.020833in}}%
\pgfusepath{stroke,fill}%
}%
\begin{pgfscope}%
\pgfsys@transformshift{3.494629in}{0.422992in}%
\pgfsys@useobject{currentmarker}{}%
\end{pgfscope}%
\end{pgfscope}%
\begin{pgfscope}%
\pgfsetbuttcap%
\pgfsetroundjoin%
\definecolor{currentfill}{rgb}{0.000000,0.000000,0.000000}%
\pgfsetfillcolor{currentfill}%
\pgfsetlinewidth{0.501875pt}%
\definecolor{currentstroke}{rgb}{0.000000,0.000000,0.000000}%
\pgfsetstrokecolor{currentstroke}%
\pgfsetdash{}{0pt}%
\pgfsys@defobject{currentmarker}{\pgfqpoint{0.000000in}{-0.020833in}}{\pgfqpoint{0.000000in}{0.000000in}}{%
\pgfpathmoveto{\pgfqpoint{0.000000in}{0.000000in}}%
\pgfpathlineto{\pgfqpoint{0.000000in}{-0.020833in}}%
\pgfusepath{stroke,fill}%
}%
\begin{pgfscope}%
\pgfsys@transformshift{3.494629in}{3.574193in}%
\pgfsys@useobject{currentmarker}{}%
\end{pgfscope}%
\end{pgfscope}%
\begin{pgfscope}%
\pgfsetbuttcap%
\pgfsetroundjoin%
\definecolor{currentfill}{rgb}{0.000000,0.000000,0.000000}%
\pgfsetfillcolor{currentfill}%
\pgfsetlinewidth{0.501875pt}%
\definecolor{currentstroke}{rgb}{0.000000,0.000000,0.000000}%
\pgfsetstrokecolor{currentstroke}%
\pgfsetdash{}{0pt}%
\pgfsys@defobject{currentmarker}{\pgfqpoint{0.000000in}{0.000000in}}{\pgfqpoint{0.000000in}{0.020833in}}{%
\pgfpathmoveto{\pgfqpoint{0.000000in}{0.000000in}}%
\pgfpathlineto{\pgfqpoint{0.000000in}{0.020833in}}%
\pgfusepath{stroke,fill}%
}%
\begin{pgfscope}%
\pgfsys@transformshift{3.603880in}{0.422992in}%
\pgfsys@useobject{currentmarker}{}%
\end{pgfscope}%
\end{pgfscope}%
\begin{pgfscope}%
\pgfsetbuttcap%
\pgfsetroundjoin%
\definecolor{currentfill}{rgb}{0.000000,0.000000,0.000000}%
\pgfsetfillcolor{currentfill}%
\pgfsetlinewidth{0.501875pt}%
\definecolor{currentstroke}{rgb}{0.000000,0.000000,0.000000}%
\pgfsetstrokecolor{currentstroke}%
\pgfsetdash{}{0pt}%
\pgfsys@defobject{currentmarker}{\pgfqpoint{0.000000in}{-0.020833in}}{\pgfqpoint{0.000000in}{0.000000in}}{%
\pgfpathmoveto{\pgfqpoint{0.000000in}{0.000000in}}%
\pgfpathlineto{\pgfqpoint{0.000000in}{-0.020833in}}%
\pgfusepath{stroke,fill}%
}%
\begin{pgfscope}%
\pgfsys@transformshift{3.603880in}{3.574193in}%
\pgfsys@useobject{currentmarker}{}%
\end{pgfscope}%
\end{pgfscope}%
\begin{pgfscope}%
\pgfsetbuttcap%
\pgfsetroundjoin%
\definecolor{currentfill}{rgb}{0.000000,0.000000,0.000000}%
\pgfsetfillcolor{currentfill}%
\pgfsetlinewidth{0.501875pt}%
\definecolor{currentstroke}{rgb}{0.000000,0.000000,0.000000}%
\pgfsetstrokecolor{currentstroke}%
\pgfsetdash{}{0pt}%
\pgfsys@defobject{currentmarker}{\pgfqpoint{0.000000in}{0.000000in}}{\pgfqpoint{0.000000in}{0.020833in}}{%
\pgfpathmoveto{\pgfqpoint{0.000000in}{0.000000in}}%
\pgfpathlineto{\pgfqpoint{0.000000in}{0.020833in}}%
\pgfusepath{stroke,fill}%
}%
\begin{pgfscope}%
\pgfsys@transformshift{3.713132in}{0.422992in}%
\pgfsys@useobject{currentmarker}{}%
\end{pgfscope}%
\end{pgfscope}%
\begin{pgfscope}%
\pgfsetbuttcap%
\pgfsetroundjoin%
\definecolor{currentfill}{rgb}{0.000000,0.000000,0.000000}%
\pgfsetfillcolor{currentfill}%
\pgfsetlinewidth{0.501875pt}%
\definecolor{currentstroke}{rgb}{0.000000,0.000000,0.000000}%
\pgfsetstrokecolor{currentstroke}%
\pgfsetdash{}{0pt}%
\pgfsys@defobject{currentmarker}{\pgfqpoint{0.000000in}{-0.020833in}}{\pgfqpoint{0.000000in}{0.000000in}}{%
\pgfpathmoveto{\pgfqpoint{0.000000in}{0.000000in}}%
\pgfpathlineto{\pgfqpoint{0.000000in}{-0.020833in}}%
\pgfusepath{stroke,fill}%
}%
\begin{pgfscope}%
\pgfsys@transformshift{3.713132in}{3.574193in}%
\pgfsys@useobject{currentmarker}{}%
\end{pgfscope}%
\end{pgfscope}%
\begin{pgfscope}%
\pgfsetbuttcap%
\pgfsetroundjoin%
\definecolor{currentfill}{rgb}{0.000000,0.000000,0.000000}%
\pgfsetfillcolor{currentfill}%
\pgfsetlinewidth{0.501875pt}%
\definecolor{currentstroke}{rgb}{0.000000,0.000000,0.000000}%
\pgfsetstrokecolor{currentstroke}%
\pgfsetdash{}{0pt}%
\pgfsys@defobject{currentmarker}{\pgfqpoint{0.000000in}{0.000000in}}{\pgfqpoint{0.000000in}{0.020833in}}{%
\pgfpathmoveto{\pgfqpoint{0.000000in}{0.000000in}}%
\pgfpathlineto{\pgfqpoint{0.000000in}{0.020833in}}%
\pgfusepath{stroke,fill}%
}%
\begin{pgfscope}%
\pgfsys@transformshift{3.931634in}{0.422992in}%
\pgfsys@useobject{currentmarker}{}%
\end{pgfscope}%
\end{pgfscope}%
\begin{pgfscope}%
\pgfsetbuttcap%
\pgfsetroundjoin%
\definecolor{currentfill}{rgb}{0.000000,0.000000,0.000000}%
\pgfsetfillcolor{currentfill}%
\pgfsetlinewidth{0.501875pt}%
\definecolor{currentstroke}{rgb}{0.000000,0.000000,0.000000}%
\pgfsetstrokecolor{currentstroke}%
\pgfsetdash{}{0pt}%
\pgfsys@defobject{currentmarker}{\pgfqpoint{0.000000in}{-0.020833in}}{\pgfqpoint{0.000000in}{0.000000in}}{%
\pgfpathmoveto{\pgfqpoint{0.000000in}{0.000000in}}%
\pgfpathlineto{\pgfqpoint{0.000000in}{-0.020833in}}%
\pgfusepath{stroke,fill}%
}%
\begin{pgfscope}%
\pgfsys@transformshift{3.931634in}{3.574193in}%
\pgfsys@useobject{currentmarker}{}%
\end{pgfscope}%
\end{pgfscope}%
\begin{pgfscope}%
\pgfsetbuttcap%
\pgfsetroundjoin%
\definecolor{currentfill}{rgb}{0.000000,0.000000,0.000000}%
\pgfsetfillcolor{currentfill}%
\pgfsetlinewidth{0.501875pt}%
\definecolor{currentstroke}{rgb}{0.000000,0.000000,0.000000}%
\pgfsetstrokecolor{currentstroke}%
\pgfsetdash{}{0pt}%
\pgfsys@defobject{currentmarker}{\pgfqpoint{0.000000in}{0.000000in}}{\pgfqpoint{0.000000in}{0.020833in}}{%
\pgfpathmoveto{\pgfqpoint{0.000000in}{0.000000in}}%
\pgfpathlineto{\pgfqpoint{0.000000in}{0.020833in}}%
\pgfusepath{stroke,fill}%
}%
\begin{pgfscope}%
\pgfsys@transformshift{4.040886in}{0.422992in}%
\pgfsys@useobject{currentmarker}{}%
\end{pgfscope}%
\end{pgfscope}%
\begin{pgfscope}%
\pgfsetbuttcap%
\pgfsetroundjoin%
\definecolor{currentfill}{rgb}{0.000000,0.000000,0.000000}%
\pgfsetfillcolor{currentfill}%
\pgfsetlinewidth{0.501875pt}%
\definecolor{currentstroke}{rgb}{0.000000,0.000000,0.000000}%
\pgfsetstrokecolor{currentstroke}%
\pgfsetdash{}{0pt}%
\pgfsys@defobject{currentmarker}{\pgfqpoint{0.000000in}{-0.020833in}}{\pgfqpoint{0.000000in}{0.000000in}}{%
\pgfpathmoveto{\pgfqpoint{0.000000in}{0.000000in}}%
\pgfpathlineto{\pgfqpoint{0.000000in}{-0.020833in}}%
\pgfusepath{stroke,fill}%
}%
\begin{pgfscope}%
\pgfsys@transformshift{4.040886in}{3.574193in}%
\pgfsys@useobject{currentmarker}{}%
\end{pgfscope}%
\end{pgfscope}%
\begin{pgfscope}%
\pgfsetbuttcap%
\pgfsetroundjoin%
\definecolor{currentfill}{rgb}{0.000000,0.000000,0.000000}%
\pgfsetfillcolor{currentfill}%
\pgfsetlinewidth{0.501875pt}%
\definecolor{currentstroke}{rgb}{0.000000,0.000000,0.000000}%
\pgfsetstrokecolor{currentstroke}%
\pgfsetdash{}{0pt}%
\pgfsys@defobject{currentmarker}{\pgfqpoint{0.000000in}{0.000000in}}{\pgfqpoint{0.000000in}{0.020833in}}{%
\pgfpathmoveto{\pgfqpoint{0.000000in}{0.000000in}}%
\pgfpathlineto{\pgfqpoint{0.000000in}{0.020833in}}%
\pgfusepath{stroke,fill}%
}%
\begin{pgfscope}%
\pgfsys@transformshift{4.150137in}{0.422992in}%
\pgfsys@useobject{currentmarker}{}%
\end{pgfscope}%
\end{pgfscope}%
\begin{pgfscope}%
\pgfsetbuttcap%
\pgfsetroundjoin%
\definecolor{currentfill}{rgb}{0.000000,0.000000,0.000000}%
\pgfsetfillcolor{currentfill}%
\pgfsetlinewidth{0.501875pt}%
\definecolor{currentstroke}{rgb}{0.000000,0.000000,0.000000}%
\pgfsetstrokecolor{currentstroke}%
\pgfsetdash{}{0pt}%
\pgfsys@defobject{currentmarker}{\pgfqpoint{0.000000in}{-0.020833in}}{\pgfqpoint{0.000000in}{0.000000in}}{%
\pgfpathmoveto{\pgfqpoint{0.000000in}{0.000000in}}%
\pgfpathlineto{\pgfqpoint{0.000000in}{-0.020833in}}%
\pgfusepath{stroke,fill}%
}%
\begin{pgfscope}%
\pgfsys@transformshift{4.150137in}{3.574193in}%
\pgfsys@useobject{currentmarker}{}%
\end{pgfscope}%
\end{pgfscope}%
\begin{pgfscope}%
\pgfsetbuttcap%
\pgfsetroundjoin%
\definecolor{currentfill}{rgb}{0.000000,0.000000,0.000000}%
\pgfsetfillcolor{currentfill}%
\pgfsetlinewidth{0.501875pt}%
\definecolor{currentstroke}{rgb}{0.000000,0.000000,0.000000}%
\pgfsetstrokecolor{currentstroke}%
\pgfsetdash{}{0pt}%
\pgfsys@defobject{currentmarker}{\pgfqpoint{0.000000in}{0.000000in}}{\pgfqpoint{0.000000in}{0.020833in}}{%
\pgfpathmoveto{\pgfqpoint{0.000000in}{0.000000in}}%
\pgfpathlineto{\pgfqpoint{0.000000in}{0.020833in}}%
\pgfusepath{stroke,fill}%
}%
\begin{pgfscope}%
\pgfsys@transformshift{4.368640in}{0.422992in}%
\pgfsys@useobject{currentmarker}{}%
\end{pgfscope}%
\end{pgfscope}%
\begin{pgfscope}%
\pgfsetbuttcap%
\pgfsetroundjoin%
\definecolor{currentfill}{rgb}{0.000000,0.000000,0.000000}%
\pgfsetfillcolor{currentfill}%
\pgfsetlinewidth{0.501875pt}%
\definecolor{currentstroke}{rgb}{0.000000,0.000000,0.000000}%
\pgfsetstrokecolor{currentstroke}%
\pgfsetdash{}{0pt}%
\pgfsys@defobject{currentmarker}{\pgfqpoint{0.000000in}{-0.020833in}}{\pgfqpoint{0.000000in}{0.000000in}}{%
\pgfpathmoveto{\pgfqpoint{0.000000in}{0.000000in}}%
\pgfpathlineto{\pgfqpoint{0.000000in}{-0.020833in}}%
\pgfusepath{stroke,fill}%
}%
\begin{pgfscope}%
\pgfsys@transformshift{4.368640in}{3.574193in}%
\pgfsys@useobject{currentmarker}{}%
\end{pgfscope}%
\end{pgfscope}%
\begin{pgfscope}%
\pgfsetbuttcap%
\pgfsetroundjoin%
\definecolor{currentfill}{rgb}{0.000000,0.000000,0.000000}%
\pgfsetfillcolor{currentfill}%
\pgfsetlinewidth{0.501875pt}%
\definecolor{currentstroke}{rgb}{0.000000,0.000000,0.000000}%
\pgfsetstrokecolor{currentstroke}%
\pgfsetdash{}{0pt}%
\pgfsys@defobject{currentmarker}{\pgfqpoint{0.000000in}{0.000000in}}{\pgfqpoint{0.000000in}{0.020833in}}{%
\pgfpathmoveto{\pgfqpoint{0.000000in}{0.000000in}}%
\pgfpathlineto{\pgfqpoint{0.000000in}{0.020833in}}%
\pgfusepath{stroke,fill}%
}%
\begin{pgfscope}%
\pgfsys@transformshift{4.477892in}{0.422992in}%
\pgfsys@useobject{currentmarker}{}%
\end{pgfscope}%
\end{pgfscope}%
\begin{pgfscope}%
\pgfsetbuttcap%
\pgfsetroundjoin%
\definecolor{currentfill}{rgb}{0.000000,0.000000,0.000000}%
\pgfsetfillcolor{currentfill}%
\pgfsetlinewidth{0.501875pt}%
\definecolor{currentstroke}{rgb}{0.000000,0.000000,0.000000}%
\pgfsetstrokecolor{currentstroke}%
\pgfsetdash{}{0pt}%
\pgfsys@defobject{currentmarker}{\pgfqpoint{0.000000in}{-0.020833in}}{\pgfqpoint{0.000000in}{0.000000in}}{%
\pgfpathmoveto{\pgfqpoint{0.000000in}{0.000000in}}%
\pgfpathlineto{\pgfqpoint{0.000000in}{-0.020833in}}%
\pgfusepath{stroke,fill}%
}%
\begin{pgfscope}%
\pgfsys@transformshift{4.477892in}{3.574193in}%
\pgfsys@useobject{currentmarker}{}%
\end{pgfscope}%
\end{pgfscope}%
\begin{pgfscope}%
\pgfsetbuttcap%
\pgfsetroundjoin%
\definecolor{currentfill}{rgb}{0.000000,0.000000,0.000000}%
\pgfsetfillcolor{currentfill}%
\pgfsetlinewidth{0.501875pt}%
\definecolor{currentstroke}{rgb}{0.000000,0.000000,0.000000}%
\pgfsetstrokecolor{currentstroke}%
\pgfsetdash{}{0pt}%
\pgfsys@defobject{currentmarker}{\pgfqpoint{0.000000in}{0.000000in}}{\pgfqpoint{0.000000in}{0.020833in}}{%
\pgfpathmoveto{\pgfqpoint{0.000000in}{0.000000in}}%
\pgfpathlineto{\pgfqpoint{0.000000in}{0.020833in}}%
\pgfusepath{stroke,fill}%
}%
\begin{pgfscope}%
\pgfsys@transformshift{4.587143in}{0.422992in}%
\pgfsys@useobject{currentmarker}{}%
\end{pgfscope}%
\end{pgfscope}%
\begin{pgfscope}%
\pgfsetbuttcap%
\pgfsetroundjoin%
\definecolor{currentfill}{rgb}{0.000000,0.000000,0.000000}%
\pgfsetfillcolor{currentfill}%
\pgfsetlinewidth{0.501875pt}%
\definecolor{currentstroke}{rgb}{0.000000,0.000000,0.000000}%
\pgfsetstrokecolor{currentstroke}%
\pgfsetdash{}{0pt}%
\pgfsys@defobject{currentmarker}{\pgfqpoint{0.000000in}{-0.020833in}}{\pgfqpoint{0.000000in}{0.000000in}}{%
\pgfpathmoveto{\pgfqpoint{0.000000in}{0.000000in}}%
\pgfpathlineto{\pgfqpoint{0.000000in}{-0.020833in}}%
\pgfusepath{stroke,fill}%
}%
\begin{pgfscope}%
\pgfsys@transformshift{4.587143in}{3.574193in}%
\pgfsys@useobject{currentmarker}{}%
\end{pgfscope}%
\end{pgfscope}%
\begin{pgfscope}%
\pgfsetbuttcap%
\pgfsetroundjoin%
\definecolor{currentfill}{rgb}{0.000000,0.000000,0.000000}%
\pgfsetfillcolor{currentfill}%
\pgfsetlinewidth{0.501875pt}%
\definecolor{currentstroke}{rgb}{0.000000,0.000000,0.000000}%
\pgfsetstrokecolor{currentstroke}%
\pgfsetdash{}{0pt}%
\pgfsys@defobject{currentmarker}{\pgfqpoint{0.000000in}{0.000000in}}{\pgfqpoint{0.000000in}{0.020833in}}{%
\pgfpathmoveto{\pgfqpoint{0.000000in}{0.000000in}}%
\pgfpathlineto{\pgfqpoint{0.000000in}{0.020833in}}%
\pgfusepath{stroke,fill}%
}%
\begin{pgfscope}%
\pgfsys@transformshift{4.805646in}{0.422992in}%
\pgfsys@useobject{currentmarker}{}%
\end{pgfscope}%
\end{pgfscope}%
\begin{pgfscope}%
\pgfsetbuttcap%
\pgfsetroundjoin%
\definecolor{currentfill}{rgb}{0.000000,0.000000,0.000000}%
\pgfsetfillcolor{currentfill}%
\pgfsetlinewidth{0.501875pt}%
\definecolor{currentstroke}{rgb}{0.000000,0.000000,0.000000}%
\pgfsetstrokecolor{currentstroke}%
\pgfsetdash{}{0pt}%
\pgfsys@defobject{currentmarker}{\pgfqpoint{0.000000in}{-0.020833in}}{\pgfqpoint{0.000000in}{0.000000in}}{%
\pgfpathmoveto{\pgfqpoint{0.000000in}{0.000000in}}%
\pgfpathlineto{\pgfqpoint{0.000000in}{-0.020833in}}%
\pgfusepath{stroke,fill}%
}%
\begin{pgfscope}%
\pgfsys@transformshift{4.805646in}{3.574193in}%
\pgfsys@useobject{currentmarker}{}%
\end{pgfscope}%
\end{pgfscope}%
\begin{pgfscope}%
\pgfsetbuttcap%
\pgfsetroundjoin%
\definecolor{currentfill}{rgb}{0.000000,0.000000,0.000000}%
\pgfsetfillcolor{currentfill}%
\pgfsetlinewidth{0.501875pt}%
\definecolor{currentstroke}{rgb}{0.000000,0.000000,0.000000}%
\pgfsetstrokecolor{currentstroke}%
\pgfsetdash{}{0pt}%
\pgfsys@defobject{currentmarker}{\pgfqpoint{0.000000in}{0.000000in}}{\pgfqpoint{0.000000in}{0.020833in}}{%
\pgfpathmoveto{\pgfqpoint{0.000000in}{0.000000in}}%
\pgfpathlineto{\pgfqpoint{0.000000in}{0.020833in}}%
\pgfusepath{stroke,fill}%
}%
\begin{pgfscope}%
\pgfsys@transformshift{4.914897in}{0.422992in}%
\pgfsys@useobject{currentmarker}{}%
\end{pgfscope}%
\end{pgfscope}%
\begin{pgfscope}%
\pgfsetbuttcap%
\pgfsetroundjoin%
\definecolor{currentfill}{rgb}{0.000000,0.000000,0.000000}%
\pgfsetfillcolor{currentfill}%
\pgfsetlinewidth{0.501875pt}%
\definecolor{currentstroke}{rgb}{0.000000,0.000000,0.000000}%
\pgfsetstrokecolor{currentstroke}%
\pgfsetdash{}{0pt}%
\pgfsys@defobject{currentmarker}{\pgfqpoint{0.000000in}{-0.020833in}}{\pgfqpoint{0.000000in}{0.000000in}}{%
\pgfpathmoveto{\pgfqpoint{0.000000in}{0.000000in}}%
\pgfpathlineto{\pgfqpoint{0.000000in}{-0.020833in}}%
\pgfusepath{stroke,fill}%
}%
\begin{pgfscope}%
\pgfsys@transformshift{4.914897in}{3.574193in}%
\pgfsys@useobject{currentmarker}{}%
\end{pgfscope}%
\end{pgfscope}%
\begin{pgfscope}%
\pgfsetbuttcap%
\pgfsetroundjoin%
\definecolor{currentfill}{rgb}{0.000000,0.000000,0.000000}%
\pgfsetfillcolor{currentfill}%
\pgfsetlinewidth{0.501875pt}%
\definecolor{currentstroke}{rgb}{0.000000,0.000000,0.000000}%
\pgfsetstrokecolor{currentstroke}%
\pgfsetdash{}{0pt}%
\pgfsys@defobject{currentmarker}{\pgfqpoint{0.000000in}{0.000000in}}{\pgfqpoint{0.000000in}{0.020833in}}{%
\pgfpathmoveto{\pgfqpoint{0.000000in}{0.000000in}}%
\pgfpathlineto{\pgfqpoint{0.000000in}{0.020833in}}%
\pgfusepath{stroke,fill}%
}%
\begin{pgfscope}%
\pgfsys@transformshift{5.024149in}{0.422992in}%
\pgfsys@useobject{currentmarker}{}%
\end{pgfscope}%
\end{pgfscope}%
\begin{pgfscope}%
\pgfsetbuttcap%
\pgfsetroundjoin%
\definecolor{currentfill}{rgb}{0.000000,0.000000,0.000000}%
\pgfsetfillcolor{currentfill}%
\pgfsetlinewidth{0.501875pt}%
\definecolor{currentstroke}{rgb}{0.000000,0.000000,0.000000}%
\pgfsetstrokecolor{currentstroke}%
\pgfsetdash{}{0pt}%
\pgfsys@defobject{currentmarker}{\pgfqpoint{0.000000in}{-0.020833in}}{\pgfqpoint{0.000000in}{0.000000in}}{%
\pgfpathmoveto{\pgfqpoint{0.000000in}{0.000000in}}%
\pgfpathlineto{\pgfqpoint{0.000000in}{-0.020833in}}%
\pgfusepath{stroke,fill}%
}%
\begin{pgfscope}%
\pgfsys@transformshift{5.024149in}{3.574193in}%
\pgfsys@useobject{currentmarker}{}%
\end{pgfscope}%
\end{pgfscope}%
\begin{pgfscope}%
\pgfsetbuttcap%
\pgfsetroundjoin%
\definecolor{currentfill}{rgb}{0.000000,0.000000,0.000000}%
\pgfsetfillcolor{currentfill}%
\pgfsetlinewidth{0.501875pt}%
\definecolor{currentstroke}{rgb}{0.000000,0.000000,0.000000}%
\pgfsetstrokecolor{currentstroke}%
\pgfsetdash{}{0pt}%
\pgfsys@defobject{currentmarker}{\pgfqpoint{0.000000in}{0.000000in}}{\pgfqpoint{0.000000in}{0.020833in}}{%
\pgfpathmoveto{\pgfqpoint{0.000000in}{0.000000in}}%
\pgfpathlineto{\pgfqpoint{0.000000in}{0.020833in}}%
\pgfusepath{stroke,fill}%
}%
\begin{pgfscope}%
\pgfsys@transformshift{5.242652in}{0.422992in}%
\pgfsys@useobject{currentmarker}{}%
\end{pgfscope}%
\end{pgfscope}%
\begin{pgfscope}%
\pgfsetbuttcap%
\pgfsetroundjoin%
\definecolor{currentfill}{rgb}{0.000000,0.000000,0.000000}%
\pgfsetfillcolor{currentfill}%
\pgfsetlinewidth{0.501875pt}%
\definecolor{currentstroke}{rgb}{0.000000,0.000000,0.000000}%
\pgfsetstrokecolor{currentstroke}%
\pgfsetdash{}{0pt}%
\pgfsys@defobject{currentmarker}{\pgfqpoint{0.000000in}{-0.020833in}}{\pgfqpoint{0.000000in}{0.000000in}}{%
\pgfpathmoveto{\pgfqpoint{0.000000in}{0.000000in}}%
\pgfpathlineto{\pgfqpoint{0.000000in}{-0.020833in}}%
\pgfusepath{stroke,fill}%
}%
\begin{pgfscope}%
\pgfsys@transformshift{5.242652in}{3.574193in}%
\pgfsys@useobject{currentmarker}{}%
\end{pgfscope}%
\end{pgfscope}%
\begin{pgfscope}%
\pgfsetbuttcap%
\pgfsetroundjoin%
\definecolor{currentfill}{rgb}{0.000000,0.000000,0.000000}%
\pgfsetfillcolor{currentfill}%
\pgfsetlinewidth{0.501875pt}%
\definecolor{currentstroke}{rgb}{0.000000,0.000000,0.000000}%
\pgfsetstrokecolor{currentstroke}%
\pgfsetdash{}{0pt}%
\pgfsys@defobject{currentmarker}{\pgfqpoint{0.000000in}{0.000000in}}{\pgfqpoint{0.000000in}{0.020833in}}{%
\pgfpathmoveto{\pgfqpoint{0.000000in}{0.000000in}}%
\pgfpathlineto{\pgfqpoint{0.000000in}{0.020833in}}%
\pgfusepath{stroke,fill}%
}%
\begin{pgfscope}%
\pgfsys@transformshift{5.351903in}{0.422992in}%
\pgfsys@useobject{currentmarker}{}%
\end{pgfscope}%
\end{pgfscope}%
\begin{pgfscope}%
\pgfsetbuttcap%
\pgfsetroundjoin%
\definecolor{currentfill}{rgb}{0.000000,0.000000,0.000000}%
\pgfsetfillcolor{currentfill}%
\pgfsetlinewidth{0.501875pt}%
\definecolor{currentstroke}{rgb}{0.000000,0.000000,0.000000}%
\pgfsetstrokecolor{currentstroke}%
\pgfsetdash{}{0pt}%
\pgfsys@defobject{currentmarker}{\pgfqpoint{0.000000in}{-0.020833in}}{\pgfqpoint{0.000000in}{0.000000in}}{%
\pgfpathmoveto{\pgfqpoint{0.000000in}{0.000000in}}%
\pgfpathlineto{\pgfqpoint{0.000000in}{-0.020833in}}%
\pgfusepath{stroke,fill}%
}%
\begin{pgfscope}%
\pgfsys@transformshift{5.351903in}{3.574193in}%
\pgfsys@useobject{currentmarker}{}%
\end{pgfscope}%
\end{pgfscope}%
\begin{pgfscope}%
\pgfsetbuttcap%
\pgfsetroundjoin%
\definecolor{currentfill}{rgb}{0.000000,0.000000,0.000000}%
\pgfsetfillcolor{currentfill}%
\pgfsetlinewidth{0.501875pt}%
\definecolor{currentstroke}{rgb}{0.000000,0.000000,0.000000}%
\pgfsetstrokecolor{currentstroke}%
\pgfsetdash{}{0pt}%
\pgfsys@defobject{currentmarker}{\pgfqpoint{0.000000in}{0.000000in}}{\pgfqpoint{0.000000in}{0.020833in}}{%
\pgfpathmoveto{\pgfqpoint{0.000000in}{0.000000in}}%
\pgfpathlineto{\pgfqpoint{0.000000in}{0.020833in}}%
\pgfusepath{stroke,fill}%
}%
\begin{pgfscope}%
\pgfsys@transformshift{5.461154in}{0.422992in}%
\pgfsys@useobject{currentmarker}{}%
\end{pgfscope}%
\end{pgfscope}%
\begin{pgfscope}%
\pgfsetbuttcap%
\pgfsetroundjoin%
\definecolor{currentfill}{rgb}{0.000000,0.000000,0.000000}%
\pgfsetfillcolor{currentfill}%
\pgfsetlinewidth{0.501875pt}%
\definecolor{currentstroke}{rgb}{0.000000,0.000000,0.000000}%
\pgfsetstrokecolor{currentstroke}%
\pgfsetdash{}{0pt}%
\pgfsys@defobject{currentmarker}{\pgfqpoint{0.000000in}{-0.020833in}}{\pgfqpoint{0.000000in}{0.000000in}}{%
\pgfpathmoveto{\pgfqpoint{0.000000in}{0.000000in}}%
\pgfpathlineto{\pgfqpoint{0.000000in}{-0.020833in}}%
\pgfusepath{stroke,fill}%
}%
\begin{pgfscope}%
\pgfsys@transformshift{5.461154in}{3.574193in}%
\pgfsys@useobject{currentmarker}{}%
\end{pgfscope}%
\end{pgfscope}%
\begin{pgfscope}%
\pgfsetbuttcap%
\pgfsetroundjoin%
\definecolor{currentfill}{rgb}{0.000000,0.000000,0.000000}%
\pgfsetfillcolor{currentfill}%
\pgfsetlinewidth{0.501875pt}%
\definecolor{currentstroke}{rgb}{0.000000,0.000000,0.000000}%
\pgfsetstrokecolor{currentstroke}%
\pgfsetdash{}{0pt}%
\pgfsys@defobject{currentmarker}{\pgfqpoint{0.000000in}{0.000000in}}{\pgfqpoint{0.000000in}{0.020833in}}{%
\pgfpathmoveto{\pgfqpoint{0.000000in}{0.000000in}}%
\pgfpathlineto{\pgfqpoint{0.000000in}{0.020833in}}%
\pgfusepath{stroke,fill}%
}%
\begin{pgfscope}%
\pgfsys@transformshift{5.570406in}{0.422992in}%
\pgfsys@useobject{currentmarker}{}%
\end{pgfscope}%
\end{pgfscope}%
\begin{pgfscope}%
\pgfsetbuttcap%
\pgfsetroundjoin%
\definecolor{currentfill}{rgb}{0.000000,0.000000,0.000000}%
\pgfsetfillcolor{currentfill}%
\pgfsetlinewidth{0.501875pt}%
\definecolor{currentstroke}{rgb}{0.000000,0.000000,0.000000}%
\pgfsetstrokecolor{currentstroke}%
\pgfsetdash{}{0pt}%
\pgfsys@defobject{currentmarker}{\pgfqpoint{0.000000in}{-0.020833in}}{\pgfqpoint{0.000000in}{0.000000in}}{%
\pgfpathmoveto{\pgfqpoint{0.000000in}{0.000000in}}%
\pgfpathlineto{\pgfqpoint{0.000000in}{-0.020833in}}%
\pgfusepath{stroke,fill}%
}%
\begin{pgfscope}%
\pgfsys@transformshift{5.570406in}{3.574193in}%
\pgfsys@useobject{currentmarker}{}%
\end{pgfscope}%
\end{pgfscope}%
\begin{pgfscope}%
\definecolor{textcolor}{rgb}{0.000000,0.000000,0.000000}%
\pgfsetstrokecolor{textcolor}%
\pgfsetfillcolor{textcolor}%
\pgftext[x=4.488817in,y=0.184413in,,top]{\color{textcolor}\rmfamily\fontsize{10.000000}{12.000000}\selectfont \(\displaystyle K\)}%
\end{pgfscope}%
\begin{pgfscope}%
\pgfsetbuttcap%
\pgfsetroundjoin%
\definecolor{currentfill}{rgb}{0.000000,0.000000,0.000000}%
\pgfsetfillcolor{currentfill}%
\pgfsetlinewidth{0.501875pt}%
\definecolor{currentstroke}{rgb}{0.000000,0.000000,0.000000}%
\pgfsetstrokecolor{currentstroke}%
\pgfsetdash{}{0pt}%
\pgfsys@defobject{currentmarker}{\pgfqpoint{0.000000in}{0.000000in}}{\pgfqpoint{0.041667in}{0.000000in}}{%
\pgfpathmoveto{\pgfqpoint{0.000000in}{0.000000in}}%
\pgfpathlineto{\pgfqpoint{0.041667in}{0.000000in}}%
\pgfusepath{stroke,fill}%
}%
\begin{pgfscope}%
\pgfsys@transformshift{3.385377in}{0.735224in}%
\pgfsys@useobject{currentmarker}{}%
\end{pgfscope}%
\end{pgfscope}%
\begin{pgfscope}%
\pgfsetbuttcap%
\pgfsetroundjoin%
\definecolor{currentfill}{rgb}{0.000000,0.000000,0.000000}%
\pgfsetfillcolor{currentfill}%
\pgfsetlinewidth{0.501875pt}%
\definecolor{currentstroke}{rgb}{0.000000,0.000000,0.000000}%
\pgfsetstrokecolor{currentstroke}%
\pgfsetdash{}{0pt}%
\pgfsys@defobject{currentmarker}{\pgfqpoint{-0.041667in}{0.000000in}}{\pgfqpoint{-0.000000in}{0.000000in}}{%
\pgfpathmoveto{\pgfqpoint{-0.000000in}{0.000000in}}%
\pgfpathlineto{\pgfqpoint{-0.041667in}{0.000000in}}%
\pgfusepath{stroke,fill}%
}%
\begin{pgfscope}%
\pgfsys@transformshift{5.592256in}{0.735224in}%
\pgfsys@useobject{currentmarker}{}%
\end{pgfscope}%
\end{pgfscope}%
\begin{pgfscope}%
\definecolor{textcolor}{rgb}{0.000000,0.000000,0.000000}%
\pgfsetstrokecolor{textcolor}%
\pgfsetfillcolor{textcolor}%
\pgftext[x=3.089852in, y=0.682462in, left, base]{\color{textcolor}\rmfamily\fontsize{10.000000}{12.000000}\selectfont \(\displaystyle {0.35}\)}%
\end{pgfscope}%
\begin{pgfscope}%
\pgfsetbuttcap%
\pgfsetroundjoin%
\definecolor{currentfill}{rgb}{0.000000,0.000000,0.000000}%
\pgfsetfillcolor{currentfill}%
\pgfsetlinewidth{0.501875pt}%
\definecolor{currentstroke}{rgb}{0.000000,0.000000,0.000000}%
\pgfsetstrokecolor{currentstroke}%
\pgfsetdash{}{0pt}%
\pgfsys@defobject{currentmarker}{\pgfqpoint{0.000000in}{0.000000in}}{\pgfqpoint{0.041667in}{0.000000in}}{%
\pgfpathmoveto{\pgfqpoint{0.000000in}{0.000000in}}%
\pgfpathlineto{\pgfqpoint{0.041667in}{0.000000in}}%
\pgfusepath{stroke,fill}%
}%
\begin{pgfscope}%
\pgfsys@transformshift{3.385377in}{1.186267in}%
\pgfsys@useobject{currentmarker}{}%
\end{pgfscope}%
\end{pgfscope}%
\begin{pgfscope}%
\pgfsetbuttcap%
\pgfsetroundjoin%
\definecolor{currentfill}{rgb}{0.000000,0.000000,0.000000}%
\pgfsetfillcolor{currentfill}%
\pgfsetlinewidth{0.501875pt}%
\definecolor{currentstroke}{rgb}{0.000000,0.000000,0.000000}%
\pgfsetstrokecolor{currentstroke}%
\pgfsetdash{}{0pt}%
\pgfsys@defobject{currentmarker}{\pgfqpoint{-0.041667in}{0.000000in}}{\pgfqpoint{-0.000000in}{0.000000in}}{%
\pgfpathmoveto{\pgfqpoint{-0.000000in}{0.000000in}}%
\pgfpathlineto{\pgfqpoint{-0.041667in}{0.000000in}}%
\pgfusepath{stroke,fill}%
}%
\begin{pgfscope}%
\pgfsys@transformshift{5.592256in}{1.186267in}%
\pgfsys@useobject{currentmarker}{}%
\end{pgfscope}%
\end{pgfscope}%
\begin{pgfscope}%
\definecolor{textcolor}{rgb}{0.000000,0.000000,0.000000}%
\pgfsetstrokecolor{textcolor}%
\pgfsetfillcolor{textcolor}%
\pgftext[x=3.089852in, y=1.133505in, left, base]{\color{textcolor}\rmfamily\fontsize{10.000000}{12.000000}\selectfont \(\displaystyle {0.40}\)}%
\end{pgfscope}%
\begin{pgfscope}%
\pgfsetbuttcap%
\pgfsetroundjoin%
\definecolor{currentfill}{rgb}{0.000000,0.000000,0.000000}%
\pgfsetfillcolor{currentfill}%
\pgfsetlinewidth{0.501875pt}%
\definecolor{currentstroke}{rgb}{0.000000,0.000000,0.000000}%
\pgfsetstrokecolor{currentstroke}%
\pgfsetdash{}{0pt}%
\pgfsys@defobject{currentmarker}{\pgfqpoint{0.000000in}{0.000000in}}{\pgfqpoint{0.041667in}{0.000000in}}{%
\pgfpathmoveto{\pgfqpoint{0.000000in}{0.000000in}}%
\pgfpathlineto{\pgfqpoint{0.041667in}{0.000000in}}%
\pgfusepath{stroke,fill}%
}%
\begin{pgfscope}%
\pgfsys@transformshift{3.385377in}{1.637310in}%
\pgfsys@useobject{currentmarker}{}%
\end{pgfscope}%
\end{pgfscope}%
\begin{pgfscope}%
\pgfsetbuttcap%
\pgfsetroundjoin%
\definecolor{currentfill}{rgb}{0.000000,0.000000,0.000000}%
\pgfsetfillcolor{currentfill}%
\pgfsetlinewidth{0.501875pt}%
\definecolor{currentstroke}{rgb}{0.000000,0.000000,0.000000}%
\pgfsetstrokecolor{currentstroke}%
\pgfsetdash{}{0pt}%
\pgfsys@defobject{currentmarker}{\pgfqpoint{-0.041667in}{0.000000in}}{\pgfqpoint{-0.000000in}{0.000000in}}{%
\pgfpathmoveto{\pgfqpoint{-0.000000in}{0.000000in}}%
\pgfpathlineto{\pgfqpoint{-0.041667in}{0.000000in}}%
\pgfusepath{stroke,fill}%
}%
\begin{pgfscope}%
\pgfsys@transformshift{5.592256in}{1.637310in}%
\pgfsys@useobject{currentmarker}{}%
\end{pgfscope}%
\end{pgfscope}%
\begin{pgfscope}%
\definecolor{textcolor}{rgb}{0.000000,0.000000,0.000000}%
\pgfsetstrokecolor{textcolor}%
\pgfsetfillcolor{textcolor}%
\pgftext[x=3.089852in, y=1.584548in, left, base]{\color{textcolor}\rmfamily\fontsize{10.000000}{12.000000}\selectfont \(\displaystyle {0.45}\)}%
\end{pgfscope}%
\begin{pgfscope}%
\pgfsetbuttcap%
\pgfsetroundjoin%
\definecolor{currentfill}{rgb}{0.000000,0.000000,0.000000}%
\pgfsetfillcolor{currentfill}%
\pgfsetlinewidth{0.501875pt}%
\definecolor{currentstroke}{rgb}{0.000000,0.000000,0.000000}%
\pgfsetstrokecolor{currentstroke}%
\pgfsetdash{}{0pt}%
\pgfsys@defobject{currentmarker}{\pgfqpoint{0.000000in}{0.000000in}}{\pgfqpoint{0.041667in}{0.000000in}}{%
\pgfpathmoveto{\pgfqpoint{0.000000in}{0.000000in}}%
\pgfpathlineto{\pgfqpoint{0.041667in}{0.000000in}}%
\pgfusepath{stroke,fill}%
}%
\begin{pgfscope}%
\pgfsys@transformshift{3.385377in}{2.088353in}%
\pgfsys@useobject{currentmarker}{}%
\end{pgfscope}%
\end{pgfscope}%
\begin{pgfscope}%
\pgfsetbuttcap%
\pgfsetroundjoin%
\definecolor{currentfill}{rgb}{0.000000,0.000000,0.000000}%
\pgfsetfillcolor{currentfill}%
\pgfsetlinewidth{0.501875pt}%
\definecolor{currentstroke}{rgb}{0.000000,0.000000,0.000000}%
\pgfsetstrokecolor{currentstroke}%
\pgfsetdash{}{0pt}%
\pgfsys@defobject{currentmarker}{\pgfqpoint{-0.041667in}{0.000000in}}{\pgfqpoint{-0.000000in}{0.000000in}}{%
\pgfpathmoveto{\pgfqpoint{-0.000000in}{0.000000in}}%
\pgfpathlineto{\pgfqpoint{-0.041667in}{0.000000in}}%
\pgfusepath{stroke,fill}%
}%
\begin{pgfscope}%
\pgfsys@transformshift{5.592256in}{2.088353in}%
\pgfsys@useobject{currentmarker}{}%
\end{pgfscope}%
\end{pgfscope}%
\begin{pgfscope}%
\definecolor{textcolor}{rgb}{0.000000,0.000000,0.000000}%
\pgfsetstrokecolor{textcolor}%
\pgfsetfillcolor{textcolor}%
\pgftext[x=3.089852in, y=2.035592in, left, base]{\color{textcolor}\rmfamily\fontsize{10.000000}{12.000000}\selectfont \(\displaystyle {0.50}\)}%
\end{pgfscope}%
\begin{pgfscope}%
\pgfsetbuttcap%
\pgfsetroundjoin%
\definecolor{currentfill}{rgb}{0.000000,0.000000,0.000000}%
\pgfsetfillcolor{currentfill}%
\pgfsetlinewidth{0.501875pt}%
\definecolor{currentstroke}{rgb}{0.000000,0.000000,0.000000}%
\pgfsetstrokecolor{currentstroke}%
\pgfsetdash{}{0pt}%
\pgfsys@defobject{currentmarker}{\pgfqpoint{0.000000in}{0.000000in}}{\pgfqpoint{0.041667in}{0.000000in}}{%
\pgfpathmoveto{\pgfqpoint{0.000000in}{0.000000in}}%
\pgfpathlineto{\pgfqpoint{0.041667in}{0.000000in}}%
\pgfusepath{stroke,fill}%
}%
\begin{pgfscope}%
\pgfsys@transformshift{3.385377in}{2.539396in}%
\pgfsys@useobject{currentmarker}{}%
\end{pgfscope}%
\end{pgfscope}%
\begin{pgfscope}%
\pgfsetbuttcap%
\pgfsetroundjoin%
\definecolor{currentfill}{rgb}{0.000000,0.000000,0.000000}%
\pgfsetfillcolor{currentfill}%
\pgfsetlinewidth{0.501875pt}%
\definecolor{currentstroke}{rgb}{0.000000,0.000000,0.000000}%
\pgfsetstrokecolor{currentstroke}%
\pgfsetdash{}{0pt}%
\pgfsys@defobject{currentmarker}{\pgfqpoint{-0.041667in}{0.000000in}}{\pgfqpoint{-0.000000in}{0.000000in}}{%
\pgfpathmoveto{\pgfqpoint{-0.000000in}{0.000000in}}%
\pgfpathlineto{\pgfqpoint{-0.041667in}{0.000000in}}%
\pgfusepath{stroke,fill}%
}%
\begin{pgfscope}%
\pgfsys@transformshift{5.592256in}{2.539396in}%
\pgfsys@useobject{currentmarker}{}%
\end{pgfscope}%
\end{pgfscope}%
\begin{pgfscope}%
\definecolor{textcolor}{rgb}{0.000000,0.000000,0.000000}%
\pgfsetstrokecolor{textcolor}%
\pgfsetfillcolor{textcolor}%
\pgftext[x=3.089852in, y=2.486635in, left, base]{\color{textcolor}\rmfamily\fontsize{10.000000}{12.000000}\selectfont \(\displaystyle {0.55}\)}%
\end{pgfscope}%
\begin{pgfscope}%
\pgfsetbuttcap%
\pgfsetroundjoin%
\definecolor{currentfill}{rgb}{0.000000,0.000000,0.000000}%
\pgfsetfillcolor{currentfill}%
\pgfsetlinewidth{0.501875pt}%
\definecolor{currentstroke}{rgb}{0.000000,0.000000,0.000000}%
\pgfsetstrokecolor{currentstroke}%
\pgfsetdash{}{0pt}%
\pgfsys@defobject{currentmarker}{\pgfqpoint{0.000000in}{0.000000in}}{\pgfqpoint{0.041667in}{0.000000in}}{%
\pgfpathmoveto{\pgfqpoint{0.000000in}{0.000000in}}%
\pgfpathlineto{\pgfqpoint{0.041667in}{0.000000in}}%
\pgfusepath{stroke,fill}%
}%
\begin{pgfscope}%
\pgfsys@transformshift{3.385377in}{2.990439in}%
\pgfsys@useobject{currentmarker}{}%
\end{pgfscope}%
\end{pgfscope}%
\begin{pgfscope}%
\pgfsetbuttcap%
\pgfsetroundjoin%
\definecolor{currentfill}{rgb}{0.000000,0.000000,0.000000}%
\pgfsetfillcolor{currentfill}%
\pgfsetlinewidth{0.501875pt}%
\definecolor{currentstroke}{rgb}{0.000000,0.000000,0.000000}%
\pgfsetstrokecolor{currentstroke}%
\pgfsetdash{}{0pt}%
\pgfsys@defobject{currentmarker}{\pgfqpoint{-0.041667in}{0.000000in}}{\pgfqpoint{-0.000000in}{0.000000in}}{%
\pgfpathmoveto{\pgfqpoint{-0.000000in}{0.000000in}}%
\pgfpathlineto{\pgfqpoint{-0.041667in}{0.000000in}}%
\pgfusepath{stroke,fill}%
}%
\begin{pgfscope}%
\pgfsys@transformshift{5.592256in}{2.990439in}%
\pgfsys@useobject{currentmarker}{}%
\end{pgfscope}%
\end{pgfscope}%
\begin{pgfscope}%
\definecolor{textcolor}{rgb}{0.000000,0.000000,0.000000}%
\pgfsetstrokecolor{textcolor}%
\pgfsetfillcolor{textcolor}%
\pgftext[x=3.089852in, y=2.937678in, left, base]{\color{textcolor}\rmfamily\fontsize{10.000000}{12.000000}\selectfont \(\displaystyle {0.60}\)}%
\end{pgfscope}%
\begin{pgfscope}%
\pgfsetbuttcap%
\pgfsetroundjoin%
\definecolor{currentfill}{rgb}{0.000000,0.000000,0.000000}%
\pgfsetfillcolor{currentfill}%
\pgfsetlinewidth{0.501875pt}%
\definecolor{currentstroke}{rgb}{0.000000,0.000000,0.000000}%
\pgfsetstrokecolor{currentstroke}%
\pgfsetdash{}{0pt}%
\pgfsys@defobject{currentmarker}{\pgfqpoint{0.000000in}{0.000000in}}{\pgfqpoint{0.041667in}{0.000000in}}{%
\pgfpathmoveto{\pgfqpoint{0.000000in}{0.000000in}}%
\pgfpathlineto{\pgfqpoint{0.041667in}{0.000000in}}%
\pgfusepath{stroke,fill}%
}%
\begin{pgfscope}%
\pgfsys@transformshift{3.385377in}{3.441483in}%
\pgfsys@useobject{currentmarker}{}%
\end{pgfscope}%
\end{pgfscope}%
\begin{pgfscope}%
\pgfsetbuttcap%
\pgfsetroundjoin%
\definecolor{currentfill}{rgb}{0.000000,0.000000,0.000000}%
\pgfsetfillcolor{currentfill}%
\pgfsetlinewidth{0.501875pt}%
\definecolor{currentstroke}{rgb}{0.000000,0.000000,0.000000}%
\pgfsetstrokecolor{currentstroke}%
\pgfsetdash{}{0pt}%
\pgfsys@defobject{currentmarker}{\pgfqpoint{-0.041667in}{0.000000in}}{\pgfqpoint{-0.000000in}{0.000000in}}{%
\pgfpathmoveto{\pgfqpoint{-0.000000in}{0.000000in}}%
\pgfpathlineto{\pgfqpoint{-0.041667in}{0.000000in}}%
\pgfusepath{stroke,fill}%
}%
\begin{pgfscope}%
\pgfsys@transformshift{5.592256in}{3.441483in}%
\pgfsys@useobject{currentmarker}{}%
\end{pgfscope}%
\end{pgfscope}%
\begin{pgfscope}%
\definecolor{textcolor}{rgb}{0.000000,0.000000,0.000000}%
\pgfsetstrokecolor{textcolor}%
\pgfsetfillcolor{textcolor}%
\pgftext[x=3.089852in, y=3.388721in, left, base]{\color{textcolor}\rmfamily\fontsize{10.000000}{12.000000}\selectfont \(\displaystyle {0.65}\)}%
\end{pgfscope}%
\begin{pgfscope}%
\pgfsetbuttcap%
\pgfsetroundjoin%
\definecolor{currentfill}{rgb}{0.000000,0.000000,0.000000}%
\pgfsetfillcolor{currentfill}%
\pgfsetlinewidth{0.501875pt}%
\definecolor{currentstroke}{rgb}{0.000000,0.000000,0.000000}%
\pgfsetstrokecolor{currentstroke}%
\pgfsetdash{}{0pt}%
\pgfsys@defobject{currentmarker}{\pgfqpoint{0.000000in}{0.000000in}}{\pgfqpoint{0.020833in}{0.000000in}}{%
\pgfpathmoveto{\pgfqpoint{0.000000in}{0.000000in}}%
\pgfpathlineto{\pgfqpoint{0.020833in}{0.000000in}}%
\pgfusepath{stroke,fill}%
}%
\begin{pgfscope}%
\pgfsys@transformshift{3.385377in}{0.464598in}%
\pgfsys@useobject{currentmarker}{}%
\end{pgfscope}%
\end{pgfscope}%
\begin{pgfscope}%
\pgfsetbuttcap%
\pgfsetroundjoin%
\definecolor{currentfill}{rgb}{0.000000,0.000000,0.000000}%
\pgfsetfillcolor{currentfill}%
\pgfsetlinewidth{0.501875pt}%
\definecolor{currentstroke}{rgb}{0.000000,0.000000,0.000000}%
\pgfsetstrokecolor{currentstroke}%
\pgfsetdash{}{0pt}%
\pgfsys@defobject{currentmarker}{\pgfqpoint{-0.020833in}{0.000000in}}{\pgfqpoint{-0.000000in}{0.000000in}}{%
\pgfpathmoveto{\pgfqpoint{-0.000000in}{0.000000in}}%
\pgfpathlineto{\pgfqpoint{-0.020833in}{0.000000in}}%
\pgfusepath{stroke,fill}%
}%
\begin{pgfscope}%
\pgfsys@transformshift{5.592256in}{0.464598in}%
\pgfsys@useobject{currentmarker}{}%
\end{pgfscope}%
\end{pgfscope}%
\begin{pgfscope}%
\pgfsetbuttcap%
\pgfsetroundjoin%
\definecolor{currentfill}{rgb}{0.000000,0.000000,0.000000}%
\pgfsetfillcolor{currentfill}%
\pgfsetlinewidth{0.501875pt}%
\definecolor{currentstroke}{rgb}{0.000000,0.000000,0.000000}%
\pgfsetstrokecolor{currentstroke}%
\pgfsetdash{}{0pt}%
\pgfsys@defobject{currentmarker}{\pgfqpoint{0.000000in}{0.000000in}}{\pgfqpoint{0.020833in}{0.000000in}}{%
\pgfpathmoveto{\pgfqpoint{0.000000in}{0.000000in}}%
\pgfpathlineto{\pgfqpoint{0.020833in}{0.000000in}}%
\pgfusepath{stroke,fill}%
}%
\begin{pgfscope}%
\pgfsys@transformshift{3.385377in}{0.554806in}%
\pgfsys@useobject{currentmarker}{}%
\end{pgfscope}%
\end{pgfscope}%
\begin{pgfscope}%
\pgfsetbuttcap%
\pgfsetroundjoin%
\definecolor{currentfill}{rgb}{0.000000,0.000000,0.000000}%
\pgfsetfillcolor{currentfill}%
\pgfsetlinewidth{0.501875pt}%
\definecolor{currentstroke}{rgb}{0.000000,0.000000,0.000000}%
\pgfsetstrokecolor{currentstroke}%
\pgfsetdash{}{0pt}%
\pgfsys@defobject{currentmarker}{\pgfqpoint{-0.020833in}{0.000000in}}{\pgfqpoint{-0.000000in}{0.000000in}}{%
\pgfpathmoveto{\pgfqpoint{-0.000000in}{0.000000in}}%
\pgfpathlineto{\pgfqpoint{-0.020833in}{0.000000in}}%
\pgfusepath{stroke,fill}%
}%
\begin{pgfscope}%
\pgfsys@transformshift{5.592256in}{0.554806in}%
\pgfsys@useobject{currentmarker}{}%
\end{pgfscope}%
\end{pgfscope}%
\begin{pgfscope}%
\pgfsetbuttcap%
\pgfsetroundjoin%
\definecolor{currentfill}{rgb}{0.000000,0.000000,0.000000}%
\pgfsetfillcolor{currentfill}%
\pgfsetlinewidth{0.501875pt}%
\definecolor{currentstroke}{rgb}{0.000000,0.000000,0.000000}%
\pgfsetstrokecolor{currentstroke}%
\pgfsetdash{}{0pt}%
\pgfsys@defobject{currentmarker}{\pgfqpoint{0.000000in}{0.000000in}}{\pgfqpoint{0.020833in}{0.000000in}}{%
\pgfpathmoveto{\pgfqpoint{0.000000in}{0.000000in}}%
\pgfpathlineto{\pgfqpoint{0.020833in}{0.000000in}}%
\pgfusepath{stroke,fill}%
}%
\begin{pgfscope}%
\pgfsys@transformshift{3.385377in}{0.645015in}%
\pgfsys@useobject{currentmarker}{}%
\end{pgfscope}%
\end{pgfscope}%
\begin{pgfscope}%
\pgfsetbuttcap%
\pgfsetroundjoin%
\definecolor{currentfill}{rgb}{0.000000,0.000000,0.000000}%
\pgfsetfillcolor{currentfill}%
\pgfsetlinewidth{0.501875pt}%
\definecolor{currentstroke}{rgb}{0.000000,0.000000,0.000000}%
\pgfsetstrokecolor{currentstroke}%
\pgfsetdash{}{0pt}%
\pgfsys@defobject{currentmarker}{\pgfqpoint{-0.020833in}{0.000000in}}{\pgfqpoint{-0.000000in}{0.000000in}}{%
\pgfpathmoveto{\pgfqpoint{-0.000000in}{0.000000in}}%
\pgfpathlineto{\pgfqpoint{-0.020833in}{0.000000in}}%
\pgfusepath{stroke,fill}%
}%
\begin{pgfscope}%
\pgfsys@transformshift{5.592256in}{0.645015in}%
\pgfsys@useobject{currentmarker}{}%
\end{pgfscope}%
\end{pgfscope}%
\begin{pgfscope}%
\pgfsetbuttcap%
\pgfsetroundjoin%
\definecolor{currentfill}{rgb}{0.000000,0.000000,0.000000}%
\pgfsetfillcolor{currentfill}%
\pgfsetlinewidth{0.501875pt}%
\definecolor{currentstroke}{rgb}{0.000000,0.000000,0.000000}%
\pgfsetstrokecolor{currentstroke}%
\pgfsetdash{}{0pt}%
\pgfsys@defobject{currentmarker}{\pgfqpoint{0.000000in}{0.000000in}}{\pgfqpoint{0.020833in}{0.000000in}}{%
\pgfpathmoveto{\pgfqpoint{0.000000in}{0.000000in}}%
\pgfpathlineto{\pgfqpoint{0.020833in}{0.000000in}}%
\pgfusepath{stroke,fill}%
}%
\begin{pgfscope}%
\pgfsys@transformshift{3.385377in}{0.825432in}%
\pgfsys@useobject{currentmarker}{}%
\end{pgfscope}%
\end{pgfscope}%
\begin{pgfscope}%
\pgfsetbuttcap%
\pgfsetroundjoin%
\definecolor{currentfill}{rgb}{0.000000,0.000000,0.000000}%
\pgfsetfillcolor{currentfill}%
\pgfsetlinewidth{0.501875pt}%
\definecolor{currentstroke}{rgb}{0.000000,0.000000,0.000000}%
\pgfsetstrokecolor{currentstroke}%
\pgfsetdash{}{0pt}%
\pgfsys@defobject{currentmarker}{\pgfqpoint{-0.020833in}{0.000000in}}{\pgfqpoint{-0.000000in}{0.000000in}}{%
\pgfpathmoveto{\pgfqpoint{-0.000000in}{0.000000in}}%
\pgfpathlineto{\pgfqpoint{-0.020833in}{0.000000in}}%
\pgfusepath{stroke,fill}%
}%
\begin{pgfscope}%
\pgfsys@transformshift{5.592256in}{0.825432in}%
\pgfsys@useobject{currentmarker}{}%
\end{pgfscope}%
\end{pgfscope}%
\begin{pgfscope}%
\pgfsetbuttcap%
\pgfsetroundjoin%
\definecolor{currentfill}{rgb}{0.000000,0.000000,0.000000}%
\pgfsetfillcolor{currentfill}%
\pgfsetlinewidth{0.501875pt}%
\definecolor{currentstroke}{rgb}{0.000000,0.000000,0.000000}%
\pgfsetstrokecolor{currentstroke}%
\pgfsetdash{}{0pt}%
\pgfsys@defobject{currentmarker}{\pgfqpoint{0.000000in}{0.000000in}}{\pgfqpoint{0.020833in}{0.000000in}}{%
\pgfpathmoveto{\pgfqpoint{0.000000in}{0.000000in}}%
\pgfpathlineto{\pgfqpoint{0.020833in}{0.000000in}}%
\pgfusepath{stroke,fill}%
}%
\begin{pgfscope}%
\pgfsys@transformshift{3.385377in}{0.915641in}%
\pgfsys@useobject{currentmarker}{}%
\end{pgfscope}%
\end{pgfscope}%
\begin{pgfscope}%
\pgfsetbuttcap%
\pgfsetroundjoin%
\definecolor{currentfill}{rgb}{0.000000,0.000000,0.000000}%
\pgfsetfillcolor{currentfill}%
\pgfsetlinewidth{0.501875pt}%
\definecolor{currentstroke}{rgb}{0.000000,0.000000,0.000000}%
\pgfsetstrokecolor{currentstroke}%
\pgfsetdash{}{0pt}%
\pgfsys@defobject{currentmarker}{\pgfqpoint{-0.020833in}{0.000000in}}{\pgfqpoint{-0.000000in}{0.000000in}}{%
\pgfpathmoveto{\pgfqpoint{-0.000000in}{0.000000in}}%
\pgfpathlineto{\pgfqpoint{-0.020833in}{0.000000in}}%
\pgfusepath{stroke,fill}%
}%
\begin{pgfscope}%
\pgfsys@transformshift{5.592256in}{0.915641in}%
\pgfsys@useobject{currentmarker}{}%
\end{pgfscope}%
\end{pgfscope}%
\begin{pgfscope}%
\pgfsetbuttcap%
\pgfsetroundjoin%
\definecolor{currentfill}{rgb}{0.000000,0.000000,0.000000}%
\pgfsetfillcolor{currentfill}%
\pgfsetlinewidth{0.501875pt}%
\definecolor{currentstroke}{rgb}{0.000000,0.000000,0.000000}%
\pgfsetstrokecolor{currentstroke}%
\pgfsetdash{}{0pt}%
\pgfsys@defobject{currentmarker}{\pgfqpoint{0.000000in}{0.000000in}}{\pgfqpoint{0.020833in}{0.000000in}}{%
\pgfpathmoveto{\pgfqpoint{0.000000in}{0.000000in}}%
\pgfpathlineto{\pgfqpoint{0.020833in}{0.000000in}}%
\pgfusepath{stroke,fill}%
}%
\begin{pgfscope}%
\pgfsys@transformshift{3.385377in}{1.005849in}%
\pgfsys@useobject{currentmarker}{}%
\end{pgfscope}%
\end{pgfscope}%
\begin{pgfscope}%
\pgfsetbuttcap%
\pgfsetroundjoin%
\definecolor{currentfill}{rgb}{0.000000,0.000000,0.000000}%
\pgfsetfillcolor{currentfill}%
\pgfsetlinewidth{0.501875pt}%
\definecolor{currentstroke}{rgb}{0.000000,0.000000,0.000000}%
\pgfsetstrokecolor{currentstroke}%
\pgfsetdash{}{0pt}%
\pgfsys@defobject{currentmarker}{\pgfqpoint{-0.020833in}{0.000000in}}{\pgfqpoint{-0.000000in}{0.000000in}}{%
\pgfpathmoveto{\pgfqpoint{-0.000000in}{0.000000in}}%
\pgfpathlineto{\pgfqpoint{-0.020833in}{0.000000in}}%
\pgfusepath{stroke,fill}%
}%
\begin{pgfscope}%
\pgfsys@transformshift{5.592256in}{1.005849in}%
\pgfsys@useobject{currentmarker}{}%
\end{pgfscope}%
\end{pgfscope}%
\begin{pgfscope}%
\pgfsetbuttcap%
\pgfsetroundjoin%
\definecolor{currentfill}{rgb}{0.000000,0.000000,0.000000}%
\pgfsetfillcolor{currentfill}%
\pgfsetlinewidth{0.501875pt}%
\definecolor{currentstroke}{rgb}{0.000000,0.000000,0.000000}%
\pgfsetstrokecolor{currentstroke}%
\pgfsetdash{}{0pt}%
\pgfsys@defobject{currentmarker}{\pgfqpoint{0.000000in}{0.000000in}}{\pgfqpoint{0.020833in}{0.000000in}}{%
\pgfpathmoveto{\pgfqpoint{0.000000in}{0.000000in}}%
\pgfpathlineto{\pgfqpoint{0.020833in}{0.000000in}}%
\pgfusepath{stroke,fill}%
}%
\begin{pgfscope}%
\pgfsys@transformshift{3.385377in}{1.096058in}%
\pgfsys@useobject{currentmarker}{}%
\end{pgfscope}%
\end{pgfscope}%
\begin{pgfscope}%
\pgfsetbuttcap%
\pgfsetroundjoin%
\definecolor{currentfill}{rgb}{0.000000,0.000000,0.000000}%
\pgfsetfillcolor{currentfill}%
\pgfsetlinewidth{0.501875pt}%
\definecolor{currentstroke}{rgb}{0.000000,0.000000,0.000000}%
\pgfsetstrokecolor{currentstroke}%
\pgfsetdash{}{0pt}%
\pgfsys@defobject{currentmarker}{\pgfqpoint{-0.020833in}{0.000000in}}{\pgfqpoint{-0.000000in}{0.000000in}}{%
\pgfpathmoveto{\pgfqpoint{-0.000000in}{0.000000in}}%
\pgfpathlineto{\pgfqpoint{-0.020833in}{0.000000in}}%
\pgfusepath{stroke,fill}%
}%
\begin{pgfscope}%
\pgfsys@transformshift{5.592256in}{1.096058in}%
\pgfsys@useobject{currentmarker}{}%
\end{pgfscope}%
\end{pgfscope}%
\begin{pgfscope}%
\pgfsetbuttcap%
\pgfsetroundjoin%
\definecolor{currentfill}{rgb}{0.000000,0.000000,0.000000}%
\pgfsetfillcolor{currentfill}%
\pgfsetlinewidth{0.501875pt}%
\definecolor{currentstroke}{rgb}{0.000000,0.000000,0.000000}%
\pgfsetstrokecolor{currentstroke}%
\pgfsetdash{}{0pt}%
\pgfsys@defobject{currentmarker}{\pgfqpoint{0.000000in}{0.000000in}}{\pgfqpoint{0.020833in}{0.000000in}}{%
\pgfpathmoveto{\pgfqpoint{0.000000in}{0.000000in}}%
\pgfpathlineto{\pgfqpoint{0.020833in}{0.000000in}}%
\pgfusepath{stroke,fill}%
}%
\begin{pgfscope}%
\pgfsys@transformshift{3.385377in}{1.276475in}%
\pgfsys@useobject{currentmarker}{}%
\end{pgfscope}%
\end{pgfscope}%
\begin{pgfscope}%
\pgfsetbuttcap%
\pgfsetroundjoin%
\definecolor{currentfill}{rgb}{0.000000,0.000000,0.000000}%
\pgfsetfillcolor{currentfill}%
\pgfsetlinewidth{0.501875pt}%
\definecolor{currentstroke}{rgb}{0.000000,0.000000,0.000000}%
\pgfsetstrokecolor{currentstroke}%
\pgfsetdash{}{0pt}%
\pgfsys@defobject{currentmarker}{\pgfqpoint{-0.020833in}{0.000000in}}{\pgfqpoint{-0.000000in}{0.000000in}}{%
\pgfpathmoveto{\pgfqpoint{-0.000000in}{0.000000in}}%
\pgfpathlineto{\pgfqpoint{-0.020833in}{0.000000in}}%
\pgfusepath{stroke,fill}%
}%
\begin{pgfscope}%
\pgfsys@transformshift{5.592256in}{1.276475in}%
\pgfsys@useobject{currentmarker}{}%
\end{pgfscope}%
\end{pgfscope}%
\begin{pgfscope}%
\pgfsetbuttcap%
\pgfsetroundjoin%
\definecolor{currentfill}{rgb}{0.000000,0.000000,0.000000}%
\pgfsetfillcolor{currentfill}%
\pgfsetlinewidth{0.501875pt}%
\definecolor{currentstroke}{rgb}{0.000000,0.000000,0.000000}%
\pgfsetstrokecolor{currentstroke}%
\pgfsetdash{}{0pt}%
\pgfsys@defobject{currentmarker}{\pgfqpoint{0.000000in}{0.000000in}}{\pgfqpoint{0.020833in}{0.000000in}}{%
\pgfpathmoveto{\pgfqpoint{0.000000in}{0.000000in}}%
\pgfpathlineto{\pgfqpoint{0.020833in}{0.000000in}}%
\pgfusepath{stroke,fill}%
}%
\begin{pgfscope}%
\pgfsys@transformshift{3.385377in}{1.366684in}%
\pgfsys@useobject{currentmarker}{}%
\end{pgfscope}%
\end{pgfscope}%
\begin{pgfscope}%
\pgfsetbuttcap%
\pgfsetroundjoin%
\definecolor{currentfill}{rgb}{0.000000,0.000000,0.000000}%
\pgfsetfillcolor{currentfill}%
\pgfsetlinewidth{0.501875pt}%
\definecolor{currentstroke}{rgb}{0.000000,0.000000,0.000000}%
\pgfsetstrokecolor{currentstroke}%
\pgfsetdash{}{0pt}%
\pgfsys@defobject{currentmarker}{\pgfqpoint{-0.020833in}{0.000000in}}{\pgfqpoint{-0.000000in}{0.000000in}}{%
\pgfpathmoveto{\pgfqpoint{-0.000000in}{0.000000in}}%
\pgfpathlineto{\pgfqpoint{-0.020833in}{0.000000in}}%
\pgfusepath{stroke,fill}%
}%
\begin{pgfscope}%
\pgfsys@transformshift{5.592256in}{1.366684in}%
\pgfsys@useobject{currentmarker}{}%
\end{pgfscope}%
\end{pgfscope}%
\begin{pgfscope}%
\pgfsetbuttcap%
\pgfsetroundjoin%
\definecolor{currentfill}{rgb}{0.000000,0.000000,0.000000}%
\pgfsetfillcolor{currentfill}%
\pgfsetlinewidth{0.501875pt}%
\definecolor{currentstroke}{rgb}{0.000000,0.000000,0.000000}%
\pgfsetstrokecolor{currentstroke}%
\pgfsetdash{}{0pt}%
\pgfsys@defobject{currentmarker}{\pgfqpoint{0.000000in}{0.000000in}}{\pgfqpoint{0.020833in}{0.000000in}}{%
\pgfpathmoveto{\pgfqpoint{0.000000in}{0.000000in}}%
\pgfpathlineto{\pgfqpoint{0.020833in}{0.000000in}}%
\pgfusepath{stroke,fill}%
}%
\begin{pgfscope}%
\pgfsys@transformshift{3.385377in}{1.456893in}%
\pgfsys@useobject{currentmarker}{}%
\end{pgfscope}%
\end{pgfscope}%
\begin{pgfscope}%
\pgfsetbuttcap%
\pgfsetroundjoin%
\definecolor{currentfill}{rgb}{0.000000,0.000000,0.000000}%
\pgfsetfillcolor{currentfill}%
\pgfsetlinewidth{0.501875pt}%
\definecolor{currentstroke}{rgb}{0.000000,0.000000,0.000000}%
\pgfsetstrokecolor{currentstroke}%
\pgfsetdash{}{0pt}%
\pgfsys@defobject{currentmarker}{\pgfqpoint{-0.020833in}{0.000000in}}{\pgfqpoint{-0.000000in}{0.000000in}}{%
\pgfpathmoveto{\pgfqpoint{-0.000000in}{0.000000in}}%
\pgfpathlineto{\pgfqpoint{-0.020833in}{0.000000in}}%
\pgfusepath{stroke,fill}%
}%
\begin{pgfscope}%
\pgfsys@transformshift{5.592256in}{1.456893in}%
\pgfsys@useobject{currentmarker}{}%
\end{pgfscope}%
\end{pgfscope}%
\begin{pgfscope}%
\pgfsetbuttcap%
\pgfsetroundjoin%
\definecolor{currentfill}{rgb}{0.000000,0.000000,0.000000}%
\pgfsetfillcolor{currentfill}%
\pgfsetlinewidth{0.501875pt}%
\definecolor{currentstroke}{rgb}{0.000000,0.000000,0.000000}%
\pgfsetstrokecolor{currentstroke}%
\pgfsetdash{}{0pt}%
\pgfsys@defobject{currentmarker}{\pgfqpoint{0.000000in}{0.000000in}}{\pgfqpoint{0.020833in}{0.000000in}}{%
\pgfpathmoveto{\pgfqpoint{0.000000in}{0.000000in}}%
\pgfpathlineto{\pgfqpoint{0.020833in}{0.000000in}}%
\pgfusepath{stroke,fill}%
}%
\begin{pgfscope}%
\pgfsys@transformshift{3.385377in}{1.547101in}%
\pgfsys@useobject{currentmarker}{}%
\end{pgfscope}%
\end{pgfscope}%
\begin{pgfscope}%
\pgfsetbuttcap%
\pgfsetroundjoin%
\definecolor{currentfill}{rgb}{0.000000,0.000000,0.000000}%
\pgfsetfillcolor{currentfill}%
\pgfsetlinewidth{0.501875pt}%
\definecolor{currentstroke}{rgb}{0.000000,0.000000,0.000000}%
\pgfsetstrokecolor{currentstroke}%
\pgfsetdash{}{0pt}%
\pgfsys@defobject{currentmarker}{\pgfqpoint{-0.020833in}{0.000000in}}{\pgfqpoint{-0.000000in}{0.000000in}}{%
\pgfpathmoveto{\pgfqpoint{-0.000000in}{0.000000in}}%
\pgfpathlineto{\pgfqpoint{-0.020833in}{0.000000in}}%
\pgfusepath{stroke,fill}%
}%
\begin{pgfscope}%
\pgfsys@transformshift{5.592256in}{1.547101in}%
\pgfsys@useobject{currentmarker}{}%
\end{pgfscope}%
\end{pgfscope}%
\begin{pgfscope}%
\pgfsetbuttcap%
\pgfsetroundjoin%
\definecolor{currentfill}{rgb}{0.000000,0.000000,0.000000}%
\pgfsetfillcolor{currentfill}%
\pgfsetlinewidth{0.501875pt}%
\definecolor{currentstroke}{rgb}{0.000000,0.000000,0.000000}%
\pgfsetstrokecolor{currentstroke}%
\pgfsetdash{}{0pt}%
\pgfsys@defobject{currentmarker}{\pgfqpoint{0.000000in}{0.000000in}}{\pgfqpoint{0.020833in}{0.000000in}}{%
\pgfpathmoveto{\pgfqpoint{0.000000in}{0.000000in}}%
\pgfpathlineto{\pgfqpoint{0.020833in}{0.000000in}}%
\pgfusepath{stroke,fill}%
}%
\begin{pgfscope}%
\pgfsys@transformshift{3.385377in}{1.727519in}%
\pgfsys@useobject{currentmarker}{}%
\end{pgfscope}%
\end{pgfscope}%
\begin{pgfscope}%
\pgfsetbuttcap%
\pgfsetroundjoin%
\definecolor{currentfill}{rgb}{0.000000,0.000000,0.000000}%
\pgfsetfillcolor{currentfill}%
\pgfsetlinewidth{0.501875pt}%
\definecolor{currentstroke}{rgb}{0.000000,0.000000,0.000000}%
\pgfsetstrokecolor{currentstroke}%
\pgfsetdash{}{0pt}%
\pgfsys@defobject{currentmarker}{\pgfqpoint{-0.020833in}{0.000000in}}{\pgfqpoint{-0.000000in}{0.000000in}}{%
\pgfpathmoveto{\pgfqpoint{-0.000000in}{0.000000in}}%
\pgfpathlineto{\pgfqpoint{-0.020833in}{0.000000in}}%
\pgfusepath{stroke,fill}%
}%
\begin{pgfscope}%
\pgfsys@transformshift{5.592256in}{1.727519in}%
\pgfsys@useobject{currentmarker}{}%
\end{pgfscope}%
\end{pgfscope}%
\begin{pgfscope}%
\pgfsetbuttcap%
\pgfsetroundjoin%
\definecolor{currentfill}{rgb}{0.000000,0.000000,0.000000}%
\pgfsetfillcolor{currentfill}%
\pgfsetlinewidth{0.501875pt}%
\definecolor{currentstroke}{rgb}{0.000000,0.000000,0.000000}%
\pgfsetstrokecolor{currentstroke}%
\pgfsetdash{}{0pt}%
\pgfsys@defobject{currentmarker}{\pgfqpoint{0.000000in}{0.000000in}}{\pgfqpoint{0.020833in}{0.000000in}}{%
\pgfpathmoveto{\pgfqpoint{0.000000in}{0.000000in}}%
\pgfpathlineto{\pgfqpoint{0.020833in}{0.000000in}}%
\pgfusepath{stroke,fill}%
}%
\begin{pgfscope}%
\pgfsys@transformshift{3.385377in}{1.817727in}%
\pgfsys@useobject{currentmarker}{}%
\end{pgfscope}%
\end{pgfscope}%
\begin{pgfscope}%
\pgfsetbuttcap%
\pgfsetroundjoin%
\definecolor{currentfill}{rgb}{0.000000,0.000000,0.000000}%
\pgfsetfillcolor{currentfill}%
\pgfsetlinewidth{0.501875pt}%
\definecolor{currentstroke}{rgb}{0.000000,0.000000,0.000000}%
\pgfsetstrokecolor{currentstroke}%
\pgfsetdash{}{0pt}%
\pgfsys@defobject{currentmarker}{\pgfqpoint{-0.020833in}{0.000000in}}{\pgfqpoint{-0.000000in}{0.000000in}}{%
\pgfpathmoveto{\pgfqpoint{-0.000000in}{0.000000in}}%
\pgfpathlineto{\pgfqpoint{-0.020833in}{0.000000in}}%
\pgfusepath{stroke,fill}%
}%
\begin{pgfscope}%
\pgfsys@transformshift{5.592256in}{1.817727in}%
\pgfsys@useobject{currentmarker}{}%
\end{pgfscope}%
\end{pgfscope}%
\begin{pgfscope}%
\pgfsetbuttcap%
\pgfsetroundjoin%
\definecolor{currentfill}{rgb}{0.000000,0.000000,0.000000}%
\pgfsetfillcolor{currentfill}%
\pgfsetlinewidth{0.501875pt}%
\definecolor{currentstroke}{rgb}{0.000000,0.000000,0.000000}%
\pgfsetstrokecolor{currentstroke}%
\pgfsetdash{}{0pt}%
\pgfsys@defobject{currentmarker}{\pgfqpoint{0.000000in}{0.000000in}}{\pgfqpoint{0.020833in}{0.000000in}}{%
\pgfpathmoveto{\pgfqpoint{0.000000in}{0.000000in}}%
\pgfpathlineto{\pgfqpoint{0.020833in}{0.000000in}}%
\pgfusepath{stroke,fill}%
}%
\begin{pgfscope}%
\pgfsys@transformshift{3.385377in}{1.907936in}%
\pgfsys@useobject{currentmarker}{}%
\end{pgfscope}%
\end{pgfscope}%
\begin{pgfscope}%
\pgfsetbuttcap%
\pgfsetroundjoin%
\definecolor{currentfill}{rgb}{0.000000,0.000000,0.000000}%
\pgfsetfillcolor{currentfill}%
\pgfsetlinewidth{0.501875pt}%
\definecolor{currentstroke}{rgb}{0.000000,0.000000,0.000000}%
\pgfsetstrokecolor{currentstroke}%
\pgfsetdash{}{0pt}%
\pgfsys@defobject{currentmarker}{\pgfqpoint{-0.020833in}{0.000000in}}{\pgfqpoint{-0.000000in}{0.000000in}}{%
\pgfpathmoveto{\pgfqpoint{-0.000000in}{0.000000in}}%
\pgfpathlineto{\pgfqpoint{-0.020833in}{0.000000in}}%
\pgfusepath{stroke,fill}%
}%
\begin{pgfscope}%
\pgfsys@transformshift{5.592256in}{1.907936in}%
\pgfsys@useobject{currentmarker}{}%
\end{pgfscope}%
\end{pgfscope}%
\begin{pgfscope}%
\pgfsetbuttcap%
\pgfsetroundjoin%
\definecolor{currentfill}{rgb}{0.000000,0.000000,0.000000}%
\pgfsetfillcolor{currentfill}%
\pgfsetlinewidth{0.501875pt}%
\definecolor{currentstroke}{rgb}{0.000000,0.000000,0.000000}%
\pgfsetstrokecolor{currentstroke}%
\pgfsetdash{}{0pt}%
\pgfsys@defobject{currentmarker}{\pgfqpoint{0.000000in}{0.000000in}}{\pgfqpoint{0.020833in}{0.000000in}}{%
\pgfpathmoveto{\pgfqpoint{0.000000in}{0.000000in}}%
\pgfpathlineto{\pgfqpoint{0.020833in}{0.000000in}}%
\pgfusepath{stroke,fill}%
}%
\begin{pgfscope}%
\pgfsys@transformshift{3.385377in}{1.998144in}%
\pgfsys@useobject{currentmarker}{}%
\end{pgfscope}%
\end{pgfscope}%
\begin{pgfscope}%
\pgfsetbuttcap%
\pgfsetroundjoin%
\definecolor{currentfill}{rgb}{0.000000,0.000000,0.000000}%
\pgfsetfillcolor{currentfill}%
\pgfsetlinewidth{0.501875pt}%
\definecolor{currentstroke}{rgb}{0.000000,0.000000,0.000000}%
\pgfsetstrokecolor{currentstroke}%
\pgfsetdash{}{0pt}%
\pgfsys@defobject{currentmarker}{\pgfqpoint{-0.020833in}{0.000000in}}{\pgfqpoint{-0.000000in}{0.000000in}}{%
\pgfpathmoveto{\pgfqpoint{-0.000000in}{0.000000in}}%
\pgfpathlineto{\pgfqpoint{-0.020833in}{0.000000in}}%
\pgfusepath{stroke,fill}%
}%
\begin{pgfscope}%
\pgfsys@transformshift{5.592256in}{1.998144in}%
\pgfsys@useobject{currentmarker}{}%
\end{pgfscope}%
\end{pgfscope}%
\begin{pgfscope}%
\pgfsetbuttcap%
\pgfsetroundjoin%
\definecolor{currentfill}{rgb}{0.000000,0.000000,0.000000}%
\pgfsetfillcolor{currentfill}%
\pgfsetlinewidth{0.501875pt}%
\definecolor{currentstroke}{rgb}{0.000000,0.000000,0.000000}%
\pgfsetstrokecolor{currentstroke}%
\pgfsetdash{}{0pt}%
\pgfsys@defobject{currentmarker}{\pgfqpoint{0.000000in}{0.000000in}}{\pgfqpoint{0.020833in}{0.000000in}}{%
\pgfpathmoveto{\pgfqpoint{0.000000in}{0.000000in}}%
\pgfpathlineto{\pgfqpoint{0.020833in}{0.000000in}}%
\pgfusepath{stroke,fill}%
}%
\begin{pgfscope}%
\pgfsys@transformshift{3.385377in}{2.178562in}%
\pgfsys@useobject{currentmarker}{}%
\end{pgfscope}%
\end{pgfscope}%
\begin{pgfscope}%
\pgfsetbuttcap%
\pgfsetroundjoin%
\definecolor{currentfill}{rgb}{0.000000,0.000000,0.000000}%
\pgfsetfillcolor{currentfill}%
\pgfsetlinewidth{0.501875pt}%
\definecolor{currentstroke}{rgb}{0.000000,0.000000,0.000000}%
\pgfsetstrokecolor{currentstroke}%
\pgfsetdash{}{0pt}%
\pgfsys@defobject{currentmarker}{\pgfqpoint{-0.020833in}{0.000000in}}{\pgfqpoint{-0.000000in}{0.000000in}}{%
\pgfpathmoveto{\pgfqpoint{-0.000000in}{0.000000in}}%
\pgfpathlineto{\pgfqpoint{-0.020833in}{0.000000in}}%
\pgfusepath{stroke,fill}%
}%
\begin{pgfscope}%
\pgfsys@transformshift{5.592256in}{2.178562in}%
\pgfsys@useobject{currentmarker}{}%
\end{pgfscope}%
\end{pgfscope}%
\begin{pgfscope}%
\pgfsetbuttcap%
\pgfsetroundjoin%
\definecolor{currentfill}{rgb}{0.000000,0.000000,0.000000}%
\pgfsetfillcolor{currentfill}%
\pgfsetlinewidth{0.501875pt}%
\definecolor{currentstroke}{rgb}{0.000000,0.000000,0.000000}%
\pgfsetstrokecolor{currentstroke}%
\pgfsetdash{}{0pt}%
\pgfsys@defobject{currentmarker}{\pgfqpoint{0.000000in}{0.000000in}}{\pgfqpoint{0.020833in}{0.000000in}}{%
\pgfpathmoveto{\pgfqpoint{0.000000in}{0.000000in}}%
\pgfpathlineto{\pgfqpoint{0.020833in}{0.000000in}}%
\pgfusepath{stroke,fill}%
}%
\begin{pgfscope}%
\pgfsys@transformshift{3.385377in}{2.268770in}%
\pgfsys@useobject{currentmarker}{}%
\end{pgfscope}%
\end{pgfscope}%
\begin{pgfscope}%
\pgfsetbuttcap%
\pgfsetroundjoin%
\definecolor{currentfill}{rgb}{0.000000,0.000000,0.000000}%
\pgfsetfillcolor{currentfill}%
\pgfsetlinewidth{0.501875pt}%
\definecolor{currentstroke}{rgb}{0.000000,0.000000,0.000000}%
\pgfsetstrokecolor{currentstroke}%
\pgfsetdash{}{0pt}%
\pgfsys@defobject{currentmarker}{\pgfqpoint{-0.020833in}{0.000000in}}{\pgfqpoint{-0.000000in}{0.000000in}}{%
\pgfpathmoveto{\pgfqpoint{-0.000000in}{0.000000in}}%
\pgfpathlineto{\pgfqpoint{-0.020833in}{0.000000in}}%
\pgfusepath{stroke,fill}%
}%
\begin{pgfscope}%
\pgfsys@transformshift{5.592256in}{2.268770in}%
\pgfsys@useobject{currentmarker}{}%
\end{pgfscope}%
\end{pgfscope}%
\begin{pgfscope}%
\pgfsetbuttcap%
\pgfsetroundjoin%
\definecolor{currentfill}{rgb}{0.000000,0.000000,0.000000}%
\pgfsetfillcolor{currentfill}%
\pgfsetlinewidth{0.501875pt}%
\definecolor{currentstroke}{rgb}{0.000000,0.000000,0.000000}%
\pgfsetstrokecolor{currentstroke}%
\pgfsetdash{}{0pt}%
\pgfsys@defobject{currentmarker}{\pgfqpoint{0.000000in}{0.000000in}}{\pgfqpoint{0.020833in}{0.000000in}}{%
\pgfpathmoveto{\pgfqpoint{0.000000in}{0.000000in}}%
\pgfpathlineto{\pgfqpoint{0.020833in}{0.000000in}}%
\pgfusepath{stroke,fill}%
}%
\begin{pgfscope}%
\pgfsys@transformshift{3.385377in}{2.358979in}%
\pgfsys@useobject{currentmarker}{}%
\end{pgfscope}%
\end{pgfscope}%
\begin{pgfscope}%
\pgfsetbuttcap%
\pgfsetroundjoin%
\definecolor{currentfill}{rgb}{0.000000,0.000000,0.000000}%
\pgfsetfillcolor{currentfill}%
\pgfsetlinewidth{0.501875pt}%
\definecolor{currentstroke}{rgb}{0.000000,0.000000,0.000000}%
\pgfsetstrokecolor{currentstroke}%
\pgfsetdash{}{0pt}%
\pgfsys@defobject{currentmarker}{\pgfqpoint{-0.020833in}{0.000000in}}{\pgfqpoint{-0.000000in}{0.000000in}}{%
\pgfpathmoveto{\pgfqpoint{-0.000000in}{0.000000in}}%
\pgfpathlineto{\pgfqpoint{-0.020833in}{0.000000in}}%
\pgfusepath{stroke,fill}%
}%
\begin{pgfscope}%
\pgfsys@transformshift{5.592256in}{2.358979in}%
\pgfsys@useobject{currentmarker}{}%
\end{pgfscope}%
\end{pgfscope}%
\begin{pgfscope}%
\pgfsetbuttcap%
\pgfsetroundjoin%
\definecolor{currentfill}{rgb}{0.000000,0.000000,0.000000}%
\pgfsetfillcolor{currentfill}%
\pgfsetlinewidth{0.501875pt}%
\definecolor{currentstroke}{rgb}{0.000000,0.000000,0.000000}%
\pgfsetstrokecolor{currentstroke}%
\pgfsetdash{}{0pt}%
\pgfsys@defobject{currentmarker}{\pgfqpoint{0.000000in}{0.000000in}}{\pgfqpoint{0.020833in}{0.000000in}}{%
\pgfpathmoveto{\pgfqpoint{0.000000in}{0.000000in}}%
\pgfpathlineto{\pgfqpoint{0.020833in}{0.000000in}}%
\pgfusepath{stroke,fill}%
}%
\begin{pgfscope}%
\pgfsys@transformshift{3.385377in}{2.449188in}%
\pgfsys@useobject{currentmarker}{}%
\end{pgfscope}%
\end{pgfscope}%
\begin{pgfscope}%
\pgfsetbuttcap%
\pgfsetroundjoin%
\definecolor{currentfill}{rgb}{0.000000,0.000000,0.000000}%
\pgfsetfillcolor{currentfill}%
\pgfsetlinewidth{0.501875pt}%
\definecolor{currentstroke}{rgb}{0.000000,0.000000,0.000000}%
\pgfsetstrokecolor{currentstroke}%
\pgfsetdash{}{0pt}%
\pgfsys@defobject{currentmarker}{\pgfqpoint{-0.020833in}{0.000000in}}{\pgfqpoint{-0.000000in}{0.000000in}}{%
\pgfpathmoveto{\pgfqpoint{-0.000000in}{0.000000in}}%
\pgfpathlineto{\pgfqpoint{-0.020833in}{0.000000in}}%
\pgfusepath{stroke,fill}%
}%
\begin{pgfscope}%
\pgfsys@transformshift{5.592256in}{2.449188in}%
\pgfsys@useobject{currentmarker}{}%
\end{pgfscope}%
\end{pgfscope}%
\begin{pgfscope}%
\pgfsetbuttcap%
\pgfsetroundjoin%
\definecolor{currentfill}{rgb}{0.000000,0.000000,0.000000}%
\pgfsetfillcolor{currentfill}%
\pgfsetlinewidth{0.501875pt}%
\definecolor{currentstroke}{rgb}{0.000000,0.000000,0.000000}%
\pgfsetstrokecolor{currentstroke}%
\pgfsetdash{}{0pt}%
\pgfsys@defobject{currentmarker}{\pgfqpoint{0.000000in}{0.000000in}}{\pgfqpoint{0.020833in}{0.000000in}}{%
\pgfpathmoveto{\pgfqpoint{0.000000in}{0.000000in}}%
\pgfpathlineto{\pgfqpoint{0.020833in}{0.000000in}}%
\pgfusepath{stroke,fill}%
}%
\begin{pgfscope}%
\pgfsys@transformshift{3.385377in}{2.629605in}%
\pgfsys@useobject{currentmarker}{}%
\end{pgfscope}%
\end{pgfscope}%
\begin{pgfscope}%
\pgfsetbuttcap%
\pgfsetroundjoin%
\definecolor{currentfill}{rgb}{0.000000,0.000000,0.000000}%
\pgfsetfillcolor{currentfill}%
\pgfsetlinewidth{0.501875pt}%
\definecolor{currentstroke}{rgb}{0.000000,0.000000,0.000000}%
\pgfsetstrokecolor{currentstroke}%
\pgfsetdash{}{0pt}%
\pgfsys@defobject{currentmarker}{\pgfqpoint{-0.020833in}{0.000000in}}{\pgfqpoint{-0.000000in}{0.000000in}}{%
\pgfpathmoveto{\pgfqpoint{-0.000000in}{0.000000in}}%
\pgfpathlineto{\pgfqpoint{-0.020833in}{0.000000in}}%
\pgfusepath{stroke,fill}%
}%
\begin{pgfscope}%
\pgfsys@transformshift{5.592256in}{2.629605in}%
\pgfsys@useobject{currentmarker}{}%
\end{pgfscope}%
\end{pgfscope}%
\begin{pgfscope}%
\pgfsetbuttcap%
\pgfsetroundjoin%
\definecolor{currentfill}{rgb}{0.000000,0.000000,0.000000}%
\pgfsetfillcolor{currentfill}%
\pgfsetlinewidth{0.501875pt}%
\definecolor{currentstroke}{rgb}{0.000000,0.000000,0.000000}%
\pgfsetstrokecolor{currentstroke}%
\pgfsetdash{}{0pt}%
\pgfsys@defobject{currentmarker}{\pgfqpoint{0.000000in}{0.000000in}}{\pgfqpoint{0.020833in}{0.000000in}}{%
\pgfpathmoveto{\pgfqpoint{0.000000in}{0.000000in}}%
\pgfpathlineto{\pgfqpoint{0.020833in}{0.000000in}}%
\pgfusepath{stroke,fill}%
}%
\begin{pgfscope}%
\pgfsys@transformshift{3.385377in}{2.719814in}%
\pgfsys@useobject{currentmarker}{}%
\end{pgfscope}%
\end{pgfscope}%
\begin{pgfscope}%
\pgfsetbuttcap%
\pgfsetroundjoin%
\definecolor{currentfill}{rgb}{0.000000,0.000000,0.000000}%
\pgfsetfillcolor{currentfill}%
\pgfsetlinewidth{0.501875pt}%
\definecolor{currentstroke}{rgb}{0.000000,0.000000,0.000000}%
\pgfsetstrokecolor{currentstroke}%
\pgfsetdash{}{0pt}%
\pgfsys@defobject{currentmarker}{\pgfqpoint{-0.020833in}{0.000000in}}{\pgfqpoint{-0.000000in}{0.000000in}}{%
\pgfpathmoveto{\pgfqpoint{-0.000000in}{0.000000in}}%
\pgfpathlineto{\pgfqpoint{-0.020833in}{0.000000in}}%
\pgfusepath{stroke,fill}%
}%
\begin{pgfscope}%
\pgfsys@transformshift{5.592256in}{2.719814in}%
\pgfsys@useobject{currentmarker}{}%
\end{pgfscope}%
\end{pgfscope}%
\begin{pgfscope}%
\pgfsetbuttcap%
\pgfsetroundjoin%
\definecolor{currentfill}{rgb}{0.000000,0.000000,0.000000}%
\pgfsetfillcolor{currentfill}%
\pgfsetlinewidth{0.501875pt}%
\definecolor{currentstroke}{rgb}{0.000000,0.000000,0.000000}%
\pgfsetstrokecolor{currentstroke}%
\pgfsetdash{}{0pt}%
\pgfsys@defobject{currentmarker}{\pgfqpoint{0.000000in}{0.000000in}}{\pgfqpoint{0.020833in}{0.000000in}}{%
\pgfpathmoveto{\pgfqpoint{0.000000in}{0.000000in}}%
\pgfpathlineto{\pgfqpoint{0.020833in}{0.000000in}}%
\pgfusepath{stroke,fill}%
}%
\begin{pgfscope}%
\pgfsys@transformshift{3.385377in}{2.810022in}%
\pgfsys@useobject{currentmarker}{}%
\end{pgfscope}%
\end{pgfscope}%
\begin{pgfscope}%
\pgfsetbuttcap%
\pgfsetroundjoin%
\definecolor{currentfill}{rgb}{0.000000,0.000000,0.000000}%
\pgfsetfillcolor{currentfill}%
\pgfsetlinewidth{0.501875pt}%
\definecolor{currentstroke}{rgb}{0.000000,0.000000,0.000000}%
\pgfsetstrokecolor{currentstroke}%
\pgfsetdash{}{0pt}%
\pgfsys@defobject{currentmarker}{\pgfqpoint{-0.020833in}{0.000000in}}{\pgfqpoint{-0.000000in}{0.000000in}}{%
\pgfpathmoveto{\pgfqpoint{-0.000000in}{0.000000in}}%
\pgfpathlineto{\pgfqpoint{-0.020833in}{0.000000in}}%
\pgfusepath{stroke,fill}%
}%
\begin{pgfscope}%
\pgfsys@transformshift{5.592256in}{2.810022in}%
\pgfsys@useobject{currentmarker}{}%
\end{pgfscope}%
\end{pgfscope}%
\begin{pgfscope}%
\pgfsetbuttcap%
\pgfsetroundjoin%
\definecolor{currentfill}{rgb}{0.000000,0.000000,0.000000}%
\pgfsetfillcolor{currentfill}%
\pgfsetlinewidth{0.501875pt}%
\definecolor{currentstroke}{rgb}{0.000000,0.000000,0.000000}%
\pgfsetstrokecolor{currentstroke}%
\pgfsetdash{}{0pt}%
\pgfsys@defobject{currentmarker}{\pgfqpoint{0.000000in}{0.000000in}}{\pgfqpoint{0.020833in}{0.000000in}}{%
\pgfpathmoveto{\pgfqpoint{0.000000in}{0.000000in}}%
\pgfpathlineto{\pgfqpoint{0.020833in}{0.000000in}}%
\pgfusepath{stroke,fill}%
}%
\begin{pgfscope}%
\pgfsys@transformshift{3.385377in}{2.900231in}%
\pgfsys@useobject{currentmarker}{}%
\end{pgfscope}%
\end{pgfscope}%
\begin{pgfscope}%
\pgfsetbuttcap%
\pgfsetroundjoin%
\definecolor{currentfill}{rgb}{0.000000,0.000000,0.000000}%
\pgfsetfillcolor{currentfill}%
\pgfsetlinewidth{0.501875pt}%
\definecolor{currentstroke}{rgb}{0.000000,0.000000,0.000000}%
\pgfsetstrokecolor{currentstroke}%
\pgfsetdash{}{0pt}%
\pgfsys@defobject{currentmarker}{\pgfqpoint{-0.020833in}{0.000000in}}{\pgfqpoint{-0.000000in}{0.000000in}}{%
\pgfpathmoveto{\pgfqpoint{-0.000000in}{0.000000in}}%
\pgfpathlineto{\pgfqpoint{-0.020833in}{0.000000in}}%
\pgfusepath{stroke,fill}%
}%
\begin{pgfscope}%
\pgfsys@transformshift{5.592256in}{2.900231in}%
\pgfsys@useobject{currentmarker}{}%
\end{pgfscope}%
\end{pgfscope}%
\begin{pgfscope}%
\pgfsetbuttcap%
\pgfsetroundjoin%
\definecolor{currentfill}{rgb}{0.000000,0.000000,0.000000}%
\pgfsetfillcolor{currentfill}%
\pgfsetlinewidth{0.501875pt}%
\definecolor{currentstroke}{rgb}{0.000000,0.000000,0.000000}%
\pgfsetstrokecolor{currentstroke}%
\pgfsetdash{}{0pt}%
\pgfsys@defobject{currentmarker}{\pgfqpoint{0.000000in}{0.000000in}}{\pgfqpoint{0.020833in}{0.000000in}}{%
\pgfpathmoveto{\pgfqpoint{0.000000in}{0.000000in}}%
\pgfpathlineto{\pgfqpoint{0.020833in}{0.000000in}}%
\pgfusepath{stroke,fill}%
}%
\begin{pgfscope}%
\pgfsys@transformshift{3.385377in}{3.080648in}%
\pgfsys@useobject{currentmarker}{}%
\end{pgfscope}%
\end{pgfscope}%
\begin{pgfscope}%
\pgfsetbuttcap%
\pgfsetroundjoin%
\definecolor{currentfill}{rgb}{0.000000,0.000000,0.000000}%
\pgfsetfillcolor{currentfill}%
\pgfsetlinewidth{0.501875pt}%
\definecolor{currentstroke}{rgb}{0.000000,0.000000,0.000000}%
\pgfsetstrokecolor{currentstroke}%
\pgfsetdash{}{0pt}%
\pgfsys@defobject{currentmarker}{\pgfqpoint{-0.020833in}{0.000000in}}{\pgfqpoint{-0.000000in}{0.000000in}}{%
\pgfpathmoveto{\pgfqpoint{-0.000000in}{0.000000in}}%
\pgfpathlineto{\pgfqpoint{-0.020833in}{0.000000in}}%
\pgfusepath{stroke,fill}%
}%
\begin{pgfscope}%
\pgfsys@transformshift{5.592256in}{3.080648in}%
\pgfsys@useobject{currentmarker}{}%
\end{pgfscope}%
\end{pgfscope}%
\begin{pgfscope}%
\pgfsetbuttcap%
\pgfsetroundjoin%
\definecolor{currentfill}{rgb}{0.000000,0.000000,0.000000}%
\pgfsetfillcolor{currentfill}%
\pgfsetlinewidth{0.501875pt}%
\definecolor{currentstroke}{rgb}{0.000000,0.000000,0.000000}%
\pgfsetstrokecolor{currentstroke}%
\pgfsetdash{}{0pt}%
\pgfsys@defobject{currentmarker}{\pgfqpoint{0.000000in}{0.000000in}}{\pgfqpoint{0.020833in}{0.000000in}}{%
\pgfpathmoveto{\pgfqpoint{0.000000in}{0.000000in}}%
\pgfpathlineto{\pgfqpoint{0.020833in}{0.000000in}}%
\pgfusepath{stroke,fill}%
}%
\begin{pgfscope}%
\pgfsys@transformshift{3.385377in}{3.170857in}%
\pgfsys@useobject{currentmarker}{}%
\end{pgfscope}%
\end{pgfscope}%
\begin{pgfscope}%
\pgfsetbuttcap%
\pgfsetroundjoin%
\definecolor{currentfill}{rgb}{0.000000,0.000000,0.000000}%
\pgfsetfillcolor{currentfill}%
\pgfsetlinewidth{0.501875pt}%
\definecolor{currentstroke}{rgb}{0.000000,0.000000,0.000000}%
\pgfsetstrokecolor{currentstroke}%
\pgfsetdash{}{0pt}%
\pgfsys@defobject{currentmarker}{\pgfqpoint{-0.020833in}{0.000000in}}{\pgfqpoint{-0.000000in}{0.000000in}}{%
\pgfpathmoveto{\pgfqpoint{-0.000000in}{0.000000in}}%
\pgfpathlineto{\pgfqpoint{-0.020833in}{0.000000in}}%
\pgfusepath{stroke,fill}%
}%
\begin{pgfscope}%
\pgfsys@transformshift{5.592256in}{3.170857in}%
\pgfsys@useobject{currentmarker}{}%
\end{pgfscope}%
\end{pgfscope}%
\begin{pgfscope}%
\pgfsetbuttcap%
\pgfsetroundjoin%
\definecolor{currentfill}{rgb}{0.000000,0.000000,0.000000}%
\pgfsetfillcolor{currentfill}%
\pgfsetlinewidth{0.501875pt}%
\definecolor{currentstroke}{rgb}{0.000000,0.000000,0.000000}%
\pgfsetstrokecolor{currentstroke}%
\pgfsetdash{}{0pt}%
\pgfsys@defobject{currentmarker}{\pgfqpoint{0.000000in}{0.000000in}}{\pgfqpoint{0.020833in}{0.000000in}}{%
\pgfpathmoveto{\pgfqpoint{0.000000in}{0.000000in}}%
\pgfpathlineto{\pgfqpoint{0.020833in}{0.000000in}}%
\pgfusepath{stroke,fill}%
}%
\begin{pgfscope}%
\pgfsys@transformshift{3.385377in}{3.261065in}%
\pgfsys@useobject{currentmarker}{}%
\end{pgfscope}%
\end{pgfscope}%
\begin{pgfscope}%
\pgfsetbuttcap%
\pgfsetroundjoin%
\definecolor{currentfill}{rgb}{0.000000,0.000000,0.000000}%
\pgfsetfillcolor{currentfill}%
\pgfsetlinewidth{0.501875pt}%
\definecolor{currentstroke}{rgb}{0.000000,0.000000,0.000000}%
\pgfsetstrokecolor{currentstroke}%
\pgfsetdash{}{0pt}%
\pgfsys@defobject{currentmarker}{\pgfqpoint{-0.020833in}{0.000000in}}{\pgfqpoint{-0.000000in}{0.000000in}}{%
\pgfpathmoveto{\pgfqpoint{-0.000000in}{0.000000in}}%
\pgfpathlineto{\pgfqpoint{-0.020833in}{0.000000in}}%
\pgfusepath{stroke,fill}%
}%
\begin{pgfscope}%
\pgfsys@transformshift{5.592256in}{3.261065in}%
\pgfsys@useobject{currentmarker}{}%
\end{pgfscope}%
\end{pgfscope}%
\begin{pgfscope}%
\pgfsetbuttcap%
\pgfsetroundjoin%
\definecolor{currentfill}{rgb}{0.000000,0.000000,0.000000}%
\pgfsetfillcolor{currentfill}%
\pgfsetlinewidth{0.501875pt}%
\definecolor{currentstroke}{rgb}{0.000000,0.000000,0.000000}%
\pgfsetstrokecolor{currentstroke}%
\pgfsetdash{}{0pt}%
\pgfsys@defobject{currentmarker}{\pgfqpoint{0.000000in}{0.000000in}}{\pgfqpoint{0.020833in}{0.000000in}}{%
\pgfpathmoveto{\pgfqpoint{0.000000in}{0.000000in}}%
\pgfpathlineto{\pgfqpoint{0.020833in}{0.000000in}}%
\pgfusepath{stroke,fill}%
}%
\begin{pgfscope}%
\pgfsys@transformshift{3.385377in}{3.351274in}%
\pgfsys@useobject{currentmarker}{}%
\end{pgfscope}%
\end{pgfscope}%
\begin{pgfscope}%
\pgfsetbuttcap%
\pgfsetroundjoin%
\definecolor{currentfill}{rgb}{0.000000,0.000000,0.000000}%
\pgfsetfillcolor{currentfill}%
\pgfsetlinewidth{0.501875pt}%
\definecolor{currentstroke}{rgb}{0.000000,0.000000,0.000000}%
\pgfsetstrokecolor{currentstroke}%
\pgfsetdash{}{0pt}%
\pgfsys@defobject{currentmarker}{\pgfqpoint{-0.020833in}{0.000000in}}{\pgfqpoint{-0.000000in}{0.000000in}}{%
\pgfpathmoveto{\pgfqpoint{-0.000000in}{0.000000in}}%
\pgfpathlineto{\pgfqpoint{-0.020833in}{0.000000in}}%
\pgfusepath{stroke,fill}%
}%
\begin{pgfscope}%
\pgfsys@transformshift{5.592256in}{3.351274in}%
\pgfsys@useobject{currentmarker}{}%
\end{pgfscope}%
\end{pgfscope}%
\begin{pgfscope}%
\pgfsetbuttcap%
\pgfsetroundjoin%
\definecolor{currentfill}{rgb}{0.000000,0.000000,0.000000}%
\pgfsetfillcolor{currentfill}%
\pgfsetlinewidth{0.501875pt}%
\definecolor{currentstroke}{rgb}{0.000000,0.000000,0.000000}%
\pgfsetstrokecolor{currentstroke}%
\pgfsetdash{}{0pt}%
\pgfsys@defobject{currentmarker}{\pgfqpoint{0.000000in}{0.000000in}}{\pgfqpoint{0.020833in}{0.000000in}}{%
\pgfpathmoveto{\pgfqpoint{0.000000in}{0.000000in}}%
\pgfpathlineto{\pgfqpoint{0.020833in}{0.000000in}}%
\pgfusepath{stroke,fill}%
}%
\begin{pgfscope}%
\pgfsys@transformshift{3.385377in}{3.531691in}%
\pgfsys@useobject{currentmarker}{}%
\end{pgfscope}%
\end{pgfscope}%
\begin{pgfscope}%
\pgfsetbuttcap%
\pgfsetroundjoin%
\definecolor{currentfill}{rgb}{0.000000,0.000000,0.000000}%
\pgfsetfillcolor{currentfill}%
\pgfsetlinewidth{0.501875pt}%
\definecolor{currentstroke}{rgb}{0.000000,0.000000,0.000000}%
\pgfsetstrokecolor{currentstroke}%
\pgfsetdash{}{0pt}%
\pgfsys@defobject{currentmarker}{\pgfqpoint{-0.020833in}{0.000000in}}{\pgfqpoint{-0.000000in}{0.000000in}}{%
\pgfpathmoveto{\pgfqpoint{-0.000000in}{0.000000in}}%
\pgfpathlineto{\pgfqpoint{-0.020833in}{0.000000in}}%
\pgfusepath{stroke,fill}%
}%
\begin{pgfscope}%
\pgfsys@transformshift{5.592256in}{3.531691in}%
\pgfsys@useobject{currentmarker}{}%
\end{pgfscope}%
\end{pgfscope}%
\begin{pgfscope}%
\definecolor{textcolor}{rgb}{0.000000,0.000000,0.000000}%
\pgfsetstrokecolor{textcolor}%
\pgfsetfillcolor{textcolor}%
\pgftext[x=3.034296in,y=1.998593in,,bottom,rotate=90.000000]{\color{textcolor}\rmfamily\fontsize{10.000000}{12.000000}\selectfont LCMC\(\displaystyle (K)\)}%
\end{pgfscope}%
\begin{pgfscope}%
\pgfpathrectangle{\pgfqpoint{3.385377in}{0.422992in}}{\pgfqpoint{2.206879in}{3.151201in}}%
\pgfusepath{clip}%
\pgfsetrectcap%
\pgfsetroundjoin%
\pgfsetlinewidth{1.003750pt}%
\definecolor{currentstroke}{rgb}{0.047059,0.364706,0.647059}%
\pgfsetstrokecolor{currentstroke}%
\pgfsetdash{}{0pt}%
\pgfpathmoveto{\pgfqpoint{3.407228in}{1.141875in}}%
\pgfpathlineto{\pgfqpoint{3.429078in}{1.217665in}}%
\pgfpathlineto{\pgfqpoint{3.450928in}{1.275410in}}%
\pgfpathlineto{\pgfqpoint{3.472778in}{1.317816in}}%
\pgfpathlineto{\pgfqpoint{3.494629in}{1.361305in}}%
\pgfpathlineto{\pgfqpoint{3.516479in}{1.400824in}}%
\pgfpathlineto{\pgfqpoint{3.538329in}{1.401210in}}%
\pgfpathlineto{\pgfqpoint{3.560180in}{1.405560in}}%
\pgfpathlineto{\pgfqpoint{3.582030in}{1.412151in}}%
\pgfpathlineto{\pgfqpoint{3.603880in}{1.420311in}}%
\pgfpathlineto{\pgfqpoint{3.625730in}{1.431089in}}%
\pgfpathlineto{\pgfqpoint{3.647581in}{1.428641in}}%
\pgfpathlineto{\pgfqpoint{3.669431in}{1.428930in}}%
\pgfpathlineto{\pgfqpoint{3.691281in}{1.432013in}}%
\pgfpathlineto{\pgfqpoint{3.713132in}{1.438415in}}%
\pgfpathlineto{\pgfqpoint{3.734982in}{1.440181in}}%
\pgfpathlineto{\pgfqpoint{3.756832in}{1.441952in}}%
\pgfpathlineto{\pgfqpoint{3.778682in}{1.450544in}}%
\pgfpathlineto{\pgfqpoint{3.800533in}{1.452723in}}%
\pgfpathlineto{\pgfqpoint{3.822383in}{1.452609in}}%
\pgfpathlineto{\pgfqpoint{3.844233in}{1.457833in}}%
\pgfpathlineto{\pgfqpoint{3.866084in}{1.460942in}}%
\pgfpathlineto{\pgfqpoint{3.887934in}{1.463937in}}%
\pgfpathlineto{\pgfqpoint{3.909784in}{1.464126in}}%
\pgfpathlineto{\pgfqpoint{3.931634in}{1.464806in}}%
\pgfpathlineto{\pgfqpoint{3.953485in}{1.465849in}}%
\pgfpathlineto{\pgfqpoint{3.975335in}{1.469088in}}%
\pgfpathlineto{\pgfqpoint{3.997185in}{1.468421in}}%
\pgfpathlineto{\pgfqpoint{4.019036in}{1.466556in}}%
\pgfpathlineto{\pgfqpoint{4.040886in}{1.466741in}}%
\pgfpathlineto{\pgfqpoint{4.062736in}{1.463537in}}%
\pgfpathlineto{\pgfqpoint{4.084586in}{1.465665in}}%
\pgfpathlineto{\pgfqpoint{4.106437in}{1.467117in}}%
\pgfpathlineto{\pgfqpoint{4.128287in}{1.463972in}}%
\pgfpathlineto{\pgfqpoint{4.150137in}{1.465235in}}%
\pgfpathlineto{\pgfqpoint{4.171988in}{1.464072in}}%
\pgfpathlineto{\pgfqpoint{4.193838in}{1.467166in}}%
\pgfpathlineto{\pgfqpoint{4.215688in}{1.468672in}}%
\pgfpathlineto{\pgfqpoint{4.237538in}{1.465983in}}%
\pgfpathlineto{\pgfqpoint{4.259389in}{1.467669in}}%
\pgfpathlineto{\pgfqpoint{4.281239in}{1.464740in}}%
\pgfpathlineto{\pgfqpoint{4.303089in}{1.463711in}}%
\pgfpathlineto{\pgfqpoint{4.324940in}{1.463025in}}%
\pgfpathlineto{\pgfqpoint{4.346790in}{1.461303in}}%
\pgfpathlineto{\pgfqpoint{4.368640in}{1.459377in}}%
\pgfpathlineto{\pgfqpoint{4.390490in}{1.457848in}}%
\pgfpathlineto{\pgfqpoint{4.412341in}{1.456692in}}%
\pgfpathlineto{\pgfqpoint{4.434191in}{1.455997in}}%
\pgfpathlineto{\pgfqpoint{4.456041in}{1.454631in}}%
\pgfpathlineto{\pgfqpoint{4.477892in}{1.457326in}}%
\pgfpathlineto{\pgfqpoint{4.499742in}{1.457190in}}%
\pgfpathlineto{\pgfqpoint{4.521592in}{1.453833in}}%
\pgfpathlineto{\pgfqpoint{4.543442in}{1.453904in}}%
\pgfpathlineto{\pgfqpoint{4.565293in}{1.454174in}}%
\pgfpathlineto{\pgfqpoint{4.587143in}{1.454433in}}%
\pgfpathlineto{\pgfqpoint{4.608993in}{1.452589in}}%
\pgfpathlineto{\pgfqpoint{4.630844in}{1.451538in}}%
\pgfpathlineto{\pgfqpoint{4.652694in}{1.451394in}}%
\pgfpathlineto{\pgfqpoint{4.674544in}{1.451163in}}%
\pgfpathlineto{\pgfqpoint{4.696394in}{1.450760in}}%
\pgfpathlineto{\pgfqpoint{4.718245in}{1.451138in}}%
\pgfpathlineto{\pgfqpoint{4.740095in}{1.451243in}}%
\pgfpathlineto{\pgfqpoint{4.761945in}{1.452175in}}%
\pgfpathlineto{\pgfqpoint{4.783796in}{1.454713in}}%
\pgfpathlineto{\pgfqpoint{4.805646in}{1.454036in}}%
\pgfpathlineto{\pgfqpoint{4.827496in}{1.455594in}}%
\pgfpathlineto{\pgfqpoint{4.849346in}{1.452985in}}%
\pgfpathlineto{\pgfqpoint{4.871197in}{1.453743in}}%
\pgfpathlineto{\pgfqpoint{4.893047in}{1.453459in}}%
\pgfpathlineto{\pgfqpoint{4.914897in}{1.454163in}}%
\pgfpathlineto{\pgfqpoint{4.936748in}{1.453475in}}%
\pgfpathlineto{\pgfqpoint{4.958598in}{1.453106in}}%
\pgfpathlineto{\pgfqpoint{4.980448in}{1.451759in}}%
\pgfpathlineto{\pgfqpoint{5.002298in}{1.452350in}}%
\pgfpathlineto{\pgfqpoint{5.024149in}{1.451723in}}%
\pgfpathlineto{\pgfqpoint{5.045999in}{1.452299in}}%
\pgfpathlineto{\pgfqpoint{5.067849in}{1.452391in}}%
\pgfpathlineto{\pgfqpoint{5.089700in}{1.450792in}}%
\pgfpathlineto{\pgfqpoint{5.111550in}{1.449622in}}%
\pgfpathlineto{\pgfqpoint{5.133400in}{1.449610in}}%
\pgfpathlineto{\pgfqpoint{5.155250in}{1.450488in}}%
\pgfpathlineto{\pgfqpoint{5.177101in}{1.450817in}}%
\pgfpathlineto{\pgfqpoint{5.198951in}{1.451486in}}%
\pgfpathlineto{\pgfqpoint{5.220801in}{1.452074in}}%
\pgfpathlineto{\pgfqpoint{5.242652in}{1.451439in}}%
\pgfpathlineto{\pgfqpoint{5.264502in}{1.451112in}}%
\pgfpathlineto{\pgfqpoint{5.286352in}{1.451788in}}%
\pgfpathlineto{\pgfqpoint{5.308202in}{1.451854in}}%
\pgfpathlineto{\pgfqpoint{5.330053in}{1.451229in}}%
\pgfpathlineto{\pgfqpoint{5.351903in}{1.452984in}}%
\pgfpathlineto{\pgfqpoint{5.373753in}{1.453571in}}%
\pgfpathlineto{\pgfqpoint{5.395604in}{1.453242in}}%
\pgfpathlineto{\pgfqpoint{5.417454in}{1.452862in}}%
\pgfpathlineto{\pgfqpoint{5.439304in}{1.454506in}}%
\pgfpathlineto{\pgfqpoint{5.461154in}{1.455222in}}%
\pgfpathlineto{\pgfqpoint{5.483005in}{1.454908in}}%
\pgfpathlineto{\pgfqpoint{5.504855in}{1.455606in}}%
\pgfpathlineto{\pgfqpoint{5.526705in}{1.456970in}}%
\pgfpathlineto{\pgfqpoint{5.548556in}{1.457122in}}%
\pgfusepath{stroke}%
\end{pgfscope}%
\begin{pgfscope}%
\pgfpathrectangle{\pgfqpoint{3.385377in}{0.422992in}}{\pgfqpoint{2.206879in}{3.151201in}}%
\pgfusepath{clip}%
\pgfsetrectcap%
\pgfsetroundjoin%
\pgfsetlinewidth{1.003750pt}%
\definecolor{currentstroke}{rgb}{0.000000,0.725490,0.270588}%
\pgfsetstrokecolor{currentstroke}%
\pgfsetdash{}{0pt}%
\pgfpathmoveto{\pgfqpoint{3.407228in}{0.728637in}}%
\pgfpathlineto{\pgfqpoint{3.429078in}{0.857660in}}%
\pgfpathlineto{\pgfqpoint{3.450928in}{0.934954in}}%
\pgfpathlineto{\pgfqpoint{3.472778in}{1.011496in}}%
\pgfpathlineto{\pgfqpoint{3.494629in}{1.023135in}}%
\pgfpathlineto{\pgfqpoint{3.516479in}{1.055556in}}%
\pgfpathlineto{\pgfqpoint{3.538329in}{1.091604in}}%
\pgfpathlineto{\pgfqpoint{3.560180in}{1.096533in}}%
\pgfpathlineto{\pgfqpoint{3.582030in}{1.109992in}}%
\pgfpathlineto{\pgfqpoint{3.603880in}{1.118052in}}%
\pgfpathlineto{\pgfqpoint{3.625730in}{1.127927in}}%
\pgfpathlineto{\pgfqpoint{3.647581in}{1.128938in}}%
\pgfpathlineto{\pgfqpoint{3.669431in}{1.135069in}}%
\pgfpathlineto{\pgfqpoint{3.691281in}{1.133492in}}%
\pgfpathlineto{\pgfqpoint{3.713132in}{1.139103in}}%
\pgfpathlineto{\pgfqpoint{3.734982in}{1.138486in}}%
\pgfpathlineto{\pgfqpoint{3.756832in}{1.134863in}}%
\pgfpathlineto{\pgfqpoint{3.778682in}{1.137458in}}%
\pgfpathlineto{\pgfqpoint{3.800533in}{1.139494in}}%
\pgfpathlineto{\pgfqpoint{3.822383in}{1.138620in}}%
\pgfpathlineto{\pgfqpoint{3.844233in}{1.136024in}}%
\pgfpathlineto{\pgfqpoint{3.866084in}{1.138340in}}%
\pgfpathlineto{\pgfqpoint{3.887934in}{1.140141in}}%
\pgfpathlineto{\pgfqpoint{3.909784in}{1.141415in}}%
\pgfpathlineto{\pgfqpoint{3.931634in}{1.145475in}}%
\pgfpathlineto{\pgfqpoint{3.953485in}{1.146794in}}%
\pgfpathlineto{\pgfqpoint{3.975335in}{1.151089in}}%
\pgfpathlineto{\pgfqpoint{3.997185in}{1.150694in}}%
\pgfpathlineto{\pgfqpoint{4.019036in}{1.154496in}}%
\pgfpathlineto{\pgfqpoint{4.040886in}{1.159128in}}%
\pgfpathlineto{\pgfqpoint{4.062736in}{1.161772in}}%
\pgfpathlineto{\pgfqpoint{4.084586in}{1.160078in}}%
\pgfpathlineto{\pgfqpoint{4.106437in}{1.159088in}}%
\pgfpathlineto{\pgfqpoint{4.128287in}{1.157785in}}%
\pgfpathlineto{\pgfqpoint{4.150137in}{1.157124in}}%
\pgfpathlineto{\pgfqpoint{4.171988in}{1.156098in}}%
\pgfpathlineto{\pgfqpoint{4.193838in}{1.156249in}}%
\pgfpathlineto{\pgfqpoint{4.215688in}{1.154683in}}%
\pgfpathlineto{\pgfqpoint{4.237538in}{1.155465in}}%
\pgfpathlineto{\pgfqpoint{4.259389in}{1.154492in}}%
\pgfpathlineto{\pgfqpoint{4.281239in}{1.152600in}}%
\pgfpathlineto{\pgfqpoint{4.303089in}{1.152344in}}%
\pgfpathlineto{\pgfqpoint{4.324940in}{1.152015in}}%
\pgfpathlineto{\pgfqpoint{4.346790in}{1.150349in}}%
\pgfpathlineto{\pgfqpoint{4.368640in}{1.149598in}}%
\pgfpathlineto{\pgfqpoint{4.390490in}{1.148528in}}%
\pgfpathlineto{\pgfqpoint{4.412341in}{1.149614in}}%
\pgfpathlineto{\pgfqpoint{4.434191in}{1.147948in}}%
\pgfpathlineto{\pgfqpoint{4.456041in}{1.147455in}}%
\pgfpathlineto{\pgfqpoint{4.477892in}{1.147307in}}%
\pgfpathlineto{\pgfqpoint{4.499742in}{1.151127in}}%
\pgfpathlineto{\pgfqpoint{4.521592in}{1.150497in}}%
\pgfpathlineto{\pgfqpoint{4.543442in}{1.151594in}}%
\pgfpathlineto{\pgfqpoint{4.565293in}{1.151547in}}%
\pgfpathlineto{\pgfqpoint{4.587143in}{1.148713in}}%
\pgfpathlineto{\pgfqpoint{4.608993in}{1.148010in}}%
\pgfpathlineto{\pgfqpoint{4.630844in}{1.149073in}}%
\pgfpathlineto{\pgfqpoint{4.652694in}{1.151406in}}%
\pgfpathlineto{\pgfqpoint{4.674544in}{1.150785in}}%
\pgfpathlineto{\pgfqpoint{4.696394in}{1.152831in}}%
\pgfpathlineto{\pgfqpoint{4.718245in}{1.152237in}}%
\pgfpathlineto{\pgfqpoint{4.740095in}{1.149682in}}%
\pgfpathlineto{\pgfqpoint{4.761945in}{1.150732in}}%
\pgfpathlineto{\pgfqpoint{4.783796in}{1.150959in}}%
\pgfpathlineto{\pgfqpoint{4.805646in}{1.149320in}}%
\pgfpathlineto{\pgfqpoint{4.827496in}{1.149917in}}%
\pgfpathlineto{\pgfqpoint{4.849346in}{1.149635in}}%
\pgfpathlineto{\pgfqpoint{4.871197in}{1.148618in}}%
\pgfpathlineto{\pgfqpoint{4.893047in}{1.148990in}}%
\pgfpathlineto{\pgfqpoint{4.914897in}{1.150151in}}%
\pgfpathlineto{\pgfqpoint{4.936748in}{1.149347in}}%
\pgfpathlineto{\pgfqpoint{4.958598in}{1.149619in}}%
\pgfpathlineto{\pgfqpoint{4.980448in}{1.149265in}}%
\pgfpathlineto{\pgfqpoint{5.002298in}{1.149993in}}%
\pgfpathlineto{\pgfqpoint{5.024149in}{1.148272in}}%
\pgfpathlineto{\pgfqpoint{5.045999in}{1.147736in}}%
\pgfpathlineto{\pgfqpoint{5.067849in}{1.147308in}}%
\pgfpathlineto{\pgfqpoint{5.089700in}{1.146659in}}%
\pgfpathlineto{\pgfqpoint{5.111550in}{1.146689in}}%
\pgfpathlineto{\pgfqpoint{5.133400in}{1.148162in}}%
\pgfpathlineto{\pgfqpoint{5.155250in}{1.145990in}}%
\pgfpathlineto{\pgfqpoint{5.177101in}{1.146973in}}%
\pgfpathlineto{\pgfqpoint{5.198951in}{1.144519in}}%
\pgfpathlineto{\pgfqpoint{5.220801in}{1.144852in}}%
\pgfpathlineto{\pgfqpoint{5.242652in}{1.145284in}}%
\pgfpathlineto{\pgfqpoint{5.264502in}{1.144362in}}%
\pgfpathlineto{\pgfqpoint{5.286352in}{1.144685in}}%
\pgfpathlineto{\pgfqpoint{5.308202in}{1.144325in}}%
\pgfpathlineto{\pgfqpoint{5.330053in}{1.144154in}}%
\pgfpathlineto{\pgfqpoint{5.351903in}{1.144489in}}%
\pgfpathlineto{\pgfqpoint{5.373753in}{1.144539in}}%
\pgfpathlineto{\pgfqpoint{5.395604in}{1.143215in}}%
\pgfpathlineto{\pgfqpoint{5.417454in}{1.142734in}}%
\pgfpathlineto{\pgfqpoint{5.439304in}{1.141573in}}%
\pgfpathlineto{\pgfqpoint{5.461154in}{1.141974in}}%
\pgfpathlineto{\pgfqpoint{5.483005in}{1.140863in}}%
\pgfpathlineto{\pgfqpoint{5.504855in}{1.139180in}}%
\pgfpathlineto{\pgfqpoint{5.526705in}{1.138930in}}%
\pgfpathlineto{\pgfqpoint{5.548556in}{1.138285in}}%
\pgfusepath{stroke}%
\end{pgfscope}%
\begin{pgfscope}%
\pgfpathrectangle{\pgfqpoint{3.385377in}{0.422992in}}{\pgfqpoint{2.206879in}{3.151201in}}%
\pgfusepath{clip}%
\pgfsetrectcap%
\pgfsetroundjoin%
\pgfsetlinewidth{1.003750pt}%
\definecolor{currentstroke}{rgb}{1.000000,0.584314,0.000000}%
\pgfsetstrokecolor{currentstroke}%
\pgfsetdash{}{0pt}%
\pgfpathmoveto{\pgfqpoint{3.407228in}{0.876608in}}%
\pgfpathlineto{\pgfqpoint{3.429078in}{1.105784in}}%
\pgfpathlineto{\pgfqpoint{3.450928in}{1.214055in}}%
\pgfpathlineto{\pgfqpoint{3.472778in}{1.299319in}}%
\pgfpathlineto{\pgfqpoint{3.494629in}{1.344703in}}%
\pgfpathlineto{\pgfqpoint{3.516479in}{1.364131in}}%
\pgfpathlineto{\pgfqpoint{3.538329in}{1.378782in}}%
\pgfpathlineto{\pgfqpoint{3.560180in}{1.381875in}}%
\pgfpathlineto{\pgfqpoint{3.582030in}{1.398116in}}%
\pgfpathlineto{\pgfqpoint{3.603880in}{1.414176in}}%
\pgfpathlineto{\pgfqpoint{3.625730in}{1.420590in}}%
\pgfpathlineto{\pgfqpoint{3.647581in}{1.427288in}}%
\pgfpathlineto{\pgfqpoint{3.669431in}{1.443228in}}%
\pgfpathlineto{\pgfqpoint{3.691281in}{1.448770in}}%
\pgfpathlineto{\pgfqpoint{3.713132in}{1.457182in}}%
\pgfpathlineto{\pgfqpoint{3.734982in}{1.468828in}}%
\pgfpathlineto{\pgfqpoint{3.756832in}{1.478892in}}%
\pgfpathlineto{\pgfqpoint{3.778682in}{1.484229in}}%
\pgfpathlineto{\pgfqpoint{3.800533in}{1.488054in}}%
\pgfpathlineto{\pgfqpoint{3.822383in}{1.502143in}}%
\pgfpathlineto{\pgfqpoint{3.844233in}{1.508274in}}%
\pgfpathlineto{\pgfqpoint{3.866084in}{1.509172in}}%
\pgfpathlineto{\pgfqpoint{3.887934in}{1.514385in}}%
\pgfpathlineto{\pgfqpoint{3.909784in}{1.519992in}}%
\pgfpathlineto{\pgfqpoint{3.931634in}{1.518436in}}%
\pgfpathlineto{\pgfqpoint{3.953485in}{1.524843in}}%
\pgfpathlineto{\pgfqpoint{3.975335in}{1.528570in}}%
\pgfpathlineto{\pgfqpoint{3.997185in}{1.534544in}}%
\pgfpathlineto{\pgfqpoint{4.019036in}{1.537555in}}%
\pgfpathlineto{\pgfqpoint{4.040886in}{1.543914in}}%
\pgfpathlineto{\pgfqpoint{4.062736in}{1.544915in}}%
\pgfpathlineto{\pgfqpoint{4.084586in}{1.552169in}}%
\pgfpathlineto{\pgfqpoint{4.106437in}{1.551492in}}%
\pgfpathlineto{\pgfqpoint{4.128287in}{1.556216in}}%
\pgfpathlineto{\pgfqpoint{4.150137in}{1.557472in}}%
\pgfpathlineto{\pgfqpoint{4.171988in}{1.561567in}}%
\pgfpathlineto{\pgfqpoint{4.193838in}{1.567439in}}%
\pgfpathlineto{\pgfqpoint{4.215688in}{1.569488in}}%
\pgfpathlineto{\pgfqpoint{4.237538in}{1.570415in}}%
\pgfpathlineto{\pgfqpoint{4.259389in}{1.572242in}}%
\pgfpathlineto{\pgfqpoint{4.281239in}{1.571824in}}%
\pgfpathlineto{\pgfqpoint{4.303089in}{1.574046in}}%
\pgfpathlineto{\pgfqpoint{4.324940in}{1.575955in}}%
\pgfpathlineto{\pgfqpoint{4.346790in}{1.579049in}}%
\pgfpathlineto{\pgfqpoint{4.368640in}{1.579198in}}%
\pgfpathlineto{\pgfqpoint{4.390490in}{1.584479in}}%
\pgfpathlineto{\pgfqpoint{4.412341in}{1.584775in}}%
\pgfpathlineto{\pgfqpoint{4.434191in}{1.584608in}}%
\pgfpathlineto{\pgfqpoint{4.456041in}{1.586767in}}%
\pgfpathlineto{\pgfqpoint{4.477892in}{1.590320in}}%
\pgfpathlineto{\pgfqpoint{4.499742in}{1.594830in}}%
\pgfpathlineto{\pgfqpoint{4.521592in}{1.595488in}}%
\pgfpathlineto{\pgfqpoint{4.543442in}{1.597075in}}%
\pgfpathlineto{\pgfqpoint{4.565293in}{1.600675in}}%
\pgfpathlineto{\pgfqpoint{4.587143in}{1.603127in}}%
\pgfpathlineto{\pgfqpoint{4.608993in}{1.604106in}}%
\pgfpathlineto{\pgfqpoint{4.630844in}{1.604828in}}%
\pgfpathlineto{\pgfqpoint{4.652694in}{1.602695in}}%
\pgfpathlineto{\pgfqpoint{4.674544in}{1.602010in}}%
\pgfpathlineto{\pgfqpoint{4.696394in}{1.605168in}}%
\pgfpathlineto{\pgfqpoint{4.718245in}{1.608044in}}%
\pgfpathlineto{\pgfqpoint{4.740095in}{1.608499in}}%
\pgfpathlineto{\pgfqpoint{4.761945in}{1.609656in}}%
\pgfpathlineto{\pgfqpoint{4.783796in}{1.610777in}}%
\pgfpathlineto{\pgfqpoint{4.805646in}{1.611836in}}%
\pgfpathlineto{\pgfqpoint{4.827496in}{1.614995in}}%
\pgfpathlineto{\pgfqpoint{4.849346in}{1.616551in}}%
\pgfpathlineto{\pgfqpoint{4.871197in}{1.617133in}}%
\pgfpathlineto{\pgfqpoint{4.893047in}{1.618744in}}%
\pgfpathlineto{\pgfqpoint{4.914897in}{1.619948in}}%
\pgfpathlineto{\pgfqpoint{4.936748in}{1.620712in}}%
\pgfpathlineto{\pgfqpoint{4.958598in}{1.621028in}}%
\pgfpathlineto{\pgfqpoint{4.980448in}{1.623437in}}%
\pgfpathlineto{\pgfqpoint{5.002298in}{1.623708in}}%
\pgfpathlineto{\pgfqpoint{5.024149in}{1.625415in}}%
\pgfpathlineto{\pgfqpoint{5.045999in}{1.626531in}}%
\pgfpathlineto{\pgfqpoint{5.067849in}{1.627361in}}%
\pgfpathlineto{\pgfqpoint{5.089700in}{1.628539in}}%
\pgfpathlineto{\pgfqpoint{5.111550in}{1.627632in}}%
\pgfpathlineto{\pgfqpoint{5.133400in}{1.628055in}}%
\pgfpathlineto{\pgfqpoint{5.155250in}{1.628446in}}%
\pgfpathlineto{\pgfqpoint{5.177101in}{1.630544in}}%
\pgfpathlineto{\pgfqpoint{5.198951in}{1.629439in}}%
\pgfpathlineto{\pgfqpoint{5.220801in}{1.630122in}}%
\pgfpathlineto{\pgfqpoint{5.242652in}{1.631298in}}%
\pgfpathlineto{\pgfqpoint{5.264502in}{1.631628in}}%
\pgfpathlineto{\pgfqpoint{5.286352in}{1.633444in}}%
\pgfpathlineto{\pgfqpoint{5.308202in}{1.633558in}}%
\pgfpathlineto{\pgfqpoint{5.330053in}{1.633548in}}%
\pgfpathlineto{\pgfqpoint{5.351903in}{1.633859in}}%
\pgfpathlineto{\pgfqpoint{5.373753in}{1.635253in}}%
\pgfpathlineto{\pgfqpoint{5.395604in}{1.636716in}}%
\pgfpathlineto{\pgfqpoint{5.417454in}{1.637797in}}%
\pgfpathlineto{\pgfqpoint{5.439304in}{1.638338in}}%
\pgfpathlineto{\pgfqpoint{5.461154in}{1.638335in}}%
\pgfpathlineto{\pgfqpoint{5.483005in}{1.639008in}}%
\pgfpathlineto{\pgfqpoint{5.504855in}{1.639371in}}%
\pgfpathlineto{\pgfqpoint{5.526705in}{1.639026in}}%
\pgfpathlineto{\pgfqpoint{5.548556in}{1.639417in}}%
\pgfusepath{stroke}%
\end{pgfscope}%
\begin{pgfscope}%
\pgfpathrectangle{\pgfqpoint{3.385377in}{0.422992in}}{\pgfqpoint{2.206879in}{3.151201in}}%
\pgfusepath{clip}%
\pgfsetrectcap%
\pgfsetroundjoin%
\pgfsetlinewidth{1.003750pt}%
\definecolor{currentstroke}{rgb}{1.000000,0.172549,0.000000}%
\pgfsetstrokecolor{currentstroke}%
\pgfsetdash{}{0pt}%
\pgfpathmoveto{\pgfqpoint{3.407228in}{2.968063in}}%
\pgfpathlineto{\pgfqpoint{3.429078in}{3.145809in}}%
\pgfpathlineto{\pgfqpoint{3.450928in}{3.253178in}}%
\pgfpathlineto{\pgfqpoint{3.472778in}{3.251374in}}%
\pgfpathlineto{\pgfqpoint{3.494629in}{3.273749in}}%
\pgfpathlineto{\pgfqpoint{3.516479in}{3.301599in}}%
\pgfpathlineto{\pgfqpoint{3.538329in}{3.340052in}}%
\pgfpathlineto{\pgfqpoint{3.560180in}{3.349719in}}%
\pgfpathlineto{\pgfqpoint{3.582030in}{3.353428in}}%
\pgfpathlineto{\pgfqpoint{3.603880in}{3.357117in}}%
\pgfpathlineto{\pgfqpoint{3.625730in}{3.366205in}}%
\pgfpathlineto{\pgfqpoint{3.647581in}{3.367913in}}%
\pgfpathlineto{\pgfqpoint{3.669431in}{3.380741in}}%
\pgfpathlineto{\pgfqpoint{3.691281in}{3.377429in}}%
\pgfpathlineto{\pgfqpoint{3.713132in}{3.391642in}}%
\pgfpathlineto{\pgfqpoint{3.734982in}{3.398439in}}%
\pgfpathlineto{\pgfqpoint{3.756832in}{3.406240in}}%
\pgfpathlineto{\pgfqpoint{3.778682in}{3.405456in}}%
\pgfpathlineto{\pgfqpoint{3.800533in}{3.401999in}}%
\pgfpathlineto{\pgfqpoint{3.822383in}{3.405385in}}%
\pgfpathlineto{\pgfqpoint{3.844233in}{3.405011in}}%
\pgfpathlineto{\pgfqpoint{3.866084in}{3.408608in}}%
\pgfpathlineto{\pgfqpoint{3.887934in}{3.416442in}}%
\pgfpathlineto{\pgfqpoint{3.909784in}{3.415729in}}%
\pgfpathlineto{\pgfqpoint{3.931634in}{3.417527in}}%
\pgfpathlineto{\pgfqpoint{3.953485in}{3.421755in}}%
\pgfpathlineto{\pgfqpoint{3.975335in}{3.421058in}}%
\pgfpathlineto{\pgfqpoint{3.997185in}{3.421442in}}%
\pgfpathlineto{\pgfqpoint{4.019036in}{3.418004in}}%
\pgfpathlineto{\pgfqpoint{4.040886in}{3.422013in}}%
\pgfpathlineto{\pgfqpoint{4.062736in}{3.419302in}}%
\pgfpathlineto{\pgfqpoint{4.084586in}{3.419298in}}%
\pgfpathlineto{\pgfqpoint{4.106437in}{3.418638in}}%
\pgfpathlineto{\pgfqpoint{4.128287in}{3.423377in}}%
\pgfpathlineto{\pgfqpoint{4.150137in}{3.422690in}}%
\pgfpathlineto{\pgfqpoint{4.171988in}{3.426803in}}%
\pgfpathlineto{\pgfqpoint{4.193838in}{3.426889in}}%
\pgfpathlineto{\pgfqpoint{4.215688in}{3.429250in}}%
\pgfpathlineto{\pgfqpoint{4.237538in}{3.428853in}}%
\pgfpathlineto{\pgfqpoint{4.259389in}{3.430957in}}%
\pgfpathlineto{\pgfqpoint{4.281239in}{3.429569in}}%
\pgfpathlineto{\pgfqpoint{4.303089in}{3.427431in}}%
\pgfpathlineto{\pgfqpoint{4.324940in}{3.427448in}}%
\pgfpathlineto{\pgfqpoint{4.346790in}{3.427793in}}%
\pgfpathlineto{\pgfqpoint{4.368640in}{3.427120in}}%
\pgfpathlineto{\pgfqpoint{4.390490in}{3.427340in}}%
\pgfpathlineto{\pgfqpoint{4.412341in}{3.424056in}}%
\pgfpathlineto{\pgfqpoint{4.434191in}{3.423803in}}%
\pgfpathlineto{\pgfqpoint{4.456041in}{3.423267in}}%
\pgfpathlineto{\pgfqpoint{4.477892in}{3.421019in}}%
\pgfpathlineto{\pgfqpoint{4.499742in}{3.418187in}}%
\pgfpathlineto{\pgfqpoint{4.521592in}{3.415430in}}%
\pgfpathlineto{\pgfqpoint{4.543442in}{3.414241in}}%
\pgfpathlineto{\pgfqpoint{4.565293in}{3.413362in}}%
\pgfpathlineto{\pgfqpoint{4.587143in}{3.411302in}}%
\pgfpathlineto{\pgfqpoint{4.608993in}{3.409090in}}%
\pgfpathlineto{\pgfqpoint{4.630844in}{3.408634in}}%
\pgfpathlineto{\pgfqpoint{4.652694in}{3.404553in}}%
\pgfpathlineto{\pgfqpoint{4.674544in}{3.406115in}}%
\pgfpathlineto{\pgfqpoint{4.696394in}{3.406152in}}%
\pgfpathlineto{\pgfqpoint{4.718245in}{3.404650in}}%
\pgfpathlineto{\pgfqpoint{4.740095in}{3.405466in}}%
\pgfpathlineto{\pgfqpoint{4.761945in}{3.404050in}}%
\pgfpathlineto{\pgfqpoint{4.783796in}{3.403891in}}%
\pgfpathlineto{\pgfqpoint{4.805646in}{3.403515in}}%
\pgfpathlineto{\pgfqpoint{4.827496in}{3.404026in}}%
\pgfpathlineto{\pgfqpoint{4.849346in}{3.405544in}}%
\pgfpathlineto{\pgfqpoint{4.871197in}{3.402719in}}%
\pgfpathlineto{\pgfqpoint{4.893047in}{3.400682in}}%
\pgfpathlineto{\pgfqpoint{4.914897in}{3.401616in}}%
\pgfpathlineto{\pgfqpoint{4.936748in}{3.400617in}}%
\pgfpathlineto{\pgfqpoint{4.958598in}{3.400023in}}%
\pgfpathlineto{\pgfqpoint{4.980448in}{3.400631in}}%
\pgfpathlineto{\pgfqpoint{5.002298in}{3.400491in}}%
\pgfpathlineto{\pgfqpoint{5.024149in}{3.400787in}}%
\pgfpathlineto{\pgfqpoint{5.045999in}{3.399699in}}%
\pgfpathlineto{\pgfqpoint{5.067849in}{3.397842in}}%
\pgfpathlineto{\pgfqpoint{5.089700in}{3.396126in}}%
\pgfpathlineto{\pgfqpoint{5.111550in}{3.395435in}}%
\pgfpathlineto{\pgfqpoint{5.133400in}{3.394310in}}%
\pgfpathlineto{\pgfqpoint{5.155250in}{3.392567in}}%
\pgfpathlineto{\pgfqpoint{5.177101in}{3.392429in}}%
\pgfpathlineto{\pgfqpoint{5.198951in}{3.391294in}}%
\pgfpathlineto{\pgfqpoint{5.220801in}{3.392055in}}%
\pgfpathlineto{\pgfqpoint{5.242652in}{3.392140in}}%
\pgfpathlineto{\pgfqpoint{5.264502in}{3.391447in}}%
\pgfpathlineto{\pgfqpoint{5.286352in}{3.392533in}}%
\pgfpathlineto{\pgfqpoint{5.308202in}{3.390333in}}%
\pgfpathlineto{\pgfqpoint{5.330053in}{3.388528in}}%
\pgfpathlineto{\pgfqpoint{5.351903in}{3.388607in}}%
\pgfpathlineto{\pgfqpoint{5.373753in}{3.390093in}}%
\pgfpathlineto{\pgfqpoint{5.395604in}{3.388781in}}%
\pgfpathlineto{\pgfqpoint{5.417454in}{3.389359in}}%
\pgfpathlineto{\pgfqpoint{5.439304in}{3.389253in}}%
\pgfpathlineto{\pgfqpoint{5.461154in}{3.389549in}}%
\pgfpathlineto{\pgfqpoint{5.483005in}{3.391060in}}%
\pgfpathlineto{\pgfqpoint{5.504855in}{3.390977in}}%
\pgfpathlineto{\pgfqpoint{5.526705in}{3.391946in}}%
\pgfpathlineto{\pgfqpoint{5.548556in}{3.390817in}}%
\pgfusepath{stroke}%
\end{pgfscope}%
\begin{pgfscope}%
\pgfpathrectangle{\pgfqpoint{3.385377in}{0.422992in}}{\pgfqpoint{2.206879in}{3.151201in}}%
\pgfusepath{clip}%
\pgfsetrectcap%
\pgfsetroundjoin%
\pgfsetlinewidth{1.003750pt}%
\definecolor{currentstroke}{rgb}{0.517647,0.356863,0.592157}%
\pgfsetstrokecolor{currentstroke}%
\pgfsetdash{}{0pt}%
\pgfpathmoveto{\pgfqpoint{3.407228in}{0.566229in}}%
\pgfpathlineto{\pgfqpoint{3.429078in}{0.705177in}}%
\pgfpathlineto{\pgfqpoint{3.450928in}{0.797810in}}%
\pgfpathlineto{\pgfqpoint{3.472778in}{0.859915in}}%
\pgfpathlineto{\pgfqpoint{3.494629in}{0.901870in}}%
\pgfpathlineto{\pgfqpoint{3.516479in}{0.912397in}}%
\pgfpathlineto{\pgfqpoint{3.538329in}{0.924040in}}%
\pgfpathlineto{\pgfqpoint{3.560180in}{0.930516in}}%
\pgfpathlineto{\pgfqpoint{3.582030in}{0.940767in}}%
\pgfpathlineto{\pgfqpoint{3.603880in}{0.954922in}}%
\pgfpathlineto{\pgfqpoint{3.625730in}{0.959613in}}%
\pgfpathlineto{\pgfqpoint{3.647581in}{0.967884in}}%
\pgfpathlineto{\pgfqpoint{3.669431in}{0.976825in}}%
\pgfpathlineto{\pgfqpoint{3.691281in}{0.974951in}}%
\pgfpathlineto{\pgfqpoint{3.713132in}{0.988845in}}%
\pgfpathlineto{\pgfqpoint{3.734982in}{0.990965in}}%
\pgfpathlineto{\pgfqpoint{3.756832in}{0.990713in}}%
\pgfpathlineto{\pgfqpoint{3.778682in}{0.991391in}}%
\pgfpathlineto{\pgfqpoint{3.800533in}{0.994656in}}%
\pgfpathlineto{\pgfqpoint{3.822383in}{0.999220in}}%
\pgfpathlineto{\pgfqpoint{3.844233in}{1.003606in}}%
\pgfpathlineto{\pgfqpoint{3.866084in}{1.005379in}}%
\pgfpathlineto{\pgfqpoint{3.887934in}{1.003938in}}%
\pgfpathlineto{\pgfqpoint{3.909784in}{1.004948in}}%
\pgfpathlineto{\pgfqpoint{3.931634in}{1.005371in}}%
\pgfpathlineto{\pgfqpoint{3.953485in}{1.007012in}}%
\pgfpathlineto{\pgfqpoint{3.975335in}{1.009399in}}%
\pgfpathlineto{\pgfqpoint{3.997185in}{1.013937in}}%
\pgfpathlineto{\pgfqpoint{4.019036in}{1.019219in}}%
\pgfpathlineto{\pgfqpoint{4.040886in}{1.021803in}}%
\pgfpathlineto{\pgfqpoint{4.062736in}{1.026840in}}%
\pgfpathlineto{\pgfqpoint{4.084586in}{1.027389in}}%
\pgfpathlineto{\pgfqpoint{4.106437in}{1.024350in}}%
\pgfpathlineto{\pgfqpoint{4.128287in}{1.025471in}}%
\pgfpathlineto{\pgfqpoint{4.150137in}{1.027661in}}%
\pgfpathlineto{\pgfqpoint{4.171988in}{1.026773in}}%
\pgfpathlineto{\pgfqpoint{4.193838in}{1.031541in}}%
\pgfpathlineto{\pgfqpoint{4.215688in}{1.034064in}}%
\pgfpathlineto{\pgfqpoint{4.237538in}{1.036597in}}%
\pgfpathlineto{\pgfqpoint{4.259389in}{1.041438in}}%
\pgfpathlineto{\pgfqpoint{4.281239in}{1.044284in}}%
\pgfpathlineto{\pgfqpoint{4.303089in}{1.046907in}}%
\pgfpathlineto{\pgfqpoint{4.324940in}{1.046094in}}%
\pgfpathlineto{\pgfqpoint{4.346790in}{1.047860in}}%
\pgfpathlineto{\pgfqpoint{4.368640in}{1.051953in}}%
\pgfpathlineto{\pgfqpoint{4.390490in}{1.055359in}}%
\pgfpathlineto{\pgfqpoint{4.412341in}{1.057736in}}%
\pgfpathlineto{\pgfqpoint{4.434191in}{1.059300in}}%
\pgfpathlineto{\pgfqpoint{4.456041in}{1.061979in}}%
\pgfpathlineto{\pgfqpoint{4.477892in}{1.061339in}}%
\pgfpathlineto{\pgfqpoint{4.499742in}{1.062457in}}%
\pgfpathlineto{\pgfqpoint{4.521592in}{1.061728in}}%
\pgfpathlineto{\pgfqpoint{4.543442in}{1.062457in}}%
\pgfpathlineto{\pgfqpoint{4.565293in}{1.062323in}}%
\pgfpathlineto{\pgfqpoint{4.587143in}{1.062390in}}%
\pgfpathlineto{\pgfqpoint{4.608993in}{1.064776in}}%
\pgfpathlineto{\pgfqpoint{4.630844in}{1.066317in}}%
\pgfpathlineto{\pgfqpoint{4.652694in}{1.067682in}}%
\pgfpathlineto{\pgfqpoint{4.674544in}{1.068388in}}%
\pgfpathlineto{\pgfqpoint{4.696394in}{1.068740in}}%
\pgfpathlineto{\pgfqpoint{4.718245in}{1.069524in}}%
\pgfpathlineto{\pgfqpoint{4.740095in}{1.071243in}}%
\pgfpathlineto{\pgfqpoint{4.761945in}{1.070731in}}%
\pgfpathlineto{\pgfqpoint{4.783796in}{1.070658in}}%
\pgfpathlineto{\pgfqpoint{4.805646in}{1.072113in}}%
\pgfpathlineto{\pgfqpoint{4.827496in}{1.072787in}}%
\pgfpathlineto{\pgfqpoint{4.849346in}{1.073494in}}%
\pgfpathlineto{\pgfqpoint{4.871197in}{1.074446in}}%
\pgfpathlineto{\pgfqpoint{4.893047in}{1.076547in}}%
\pgfpathlineto{\pgfqpoint{4.914897in}{1.079233in}}%
\pgfpathlineto{\pgfqpoint{4.936748in}{1.080877in}}%
\pgfpathlineto{\pgfqpoint{4.958598in}{1.079442in}}%
\pgfpathlineto{\pgfqpoint{4.980448in}{1.079852in}}%
\pgfpathlineto{\pgfqpoint{5.002298in}{1.082616in}}%
\pgfpathlineto{\pgfqpoint{5.024149in}{1.084945in}}%
\pgfpathlineto{\pgfqpoint{5.045999in}{1.087023in}}%
\pgfpathlineto{\pgfqpoint{5.067849in}{1.088438in}}%
\pgfpathlineto{\pgfqpoint{5.089700in}{1.088983in}}%
\pgfpathlineto{\pgfqpoint{5.111550in}{1.091868in}}%
\pgfpathlineto{\pgfqpoint{5.133400in}{1.092583in}}%
\pgfpathlineto{\pgfqpoint{5.155250in}{1.094638in}}%
\pgfpathlineto{\pgfqpoint{5.177101in}{1.093673in}}%
\pgfpathlineto{\pgfqpoint{5.198951in}{1.095688in}}%
\pgfpathlineto{\pgfqpoint{5.220801in}{1.096152in}}%
\pgfpathlineto{\pgfqpoint{5.242652in}{1.097453in}}%
\pgfpathlineto{\pgfqpoint{5.264502in}{1.098703in}}%
\pgfpathlineto{\pgfqpoint{5.286352in}{1.099178in}}%
\pgfpathlineto{\pgfqpoint{5.308202in}{1.100893in}}%
\pgfpathlineto{\pgfqpoint{5.330053in}{1.103319in}}%
\pgfpathlineto{\pgfqpoint{5.351903in}{1.105612in}}%
\pgfpathlineto{\pgfqpoint{5.373753in}{1.107358in}}%
\pgfpathlineto{\pgfqpoint{5.395604in}{1.108537in}}%
\pgfpathlineto{\pgfqpoint{5.417454in}{1.110233in}}%
\pgfpathlineto{\pgfqpoint{5.439304in}{1.109686in}}%
\pgfpathlineto{\pgfqpoint{5.461154in}{1.109701in}}%
\pgfpathlineto{\pgfqpoint{5.483005in}{1.110017in}}%
\pgfpathlineto{\pgfqpoint{5.504855in}{1.110549in}}%
\pgfpathlineto{\pgfqpoint{5.526705in}{1.110997in}}%
\pgfpathlineto{\pgfqpoint{5.548556in}{1.112766in}}%
\pgfusepath{stroke}%
\end{pgfscope}%
\begin{pgfscope}%
\pgfsetrectcap%
\pgfsetmiterjoin%
\pgfsetlinewidth{0.501875pt}%
\definecolor{currentstroke}{rgb}{0.000000,0.000000,0.000000}%
\pgfsetstrokecolor{currentstroke}%
\pgfsetdash{}{0pt}%
\pgfpathmoveto{\pgfqpoint{3.385377in}{0.422992in}}%
\pgfpathlineto{\pgfqpoint{3.385377in}{3.574193in}}%
\pgfusepath{stroke}%
\end{pgfscope}%
\begin{pgfscope}%
\pgfsetrectcap%
\pgfsetmiterjoin%
\pgfsetlinewidth{0.501875pt}%
\definecolor{currentstroke}{rgb}{0.000000,0.000000,0.000000}%
\pgfsetstrokecolor{currentstroke}%
\pgfsetdash{}{0pt}%
\pgfpathmoveto{\pgfqpoint{5.592256in}{0.422992in}}%
\pgfpathlineto{\pgfqpoint{5.592256in}{3.574193in}}%
\pgfusepath{stroke}%
\end{pgfscope}%
\begin{pgfscope}%
\pgfsetrectcap%
\pgfsetmiterjoin%
\pgfsetlinewidth{0.501875pt}%
\definecolor{currentstroke}{rgb}{0.000000,0.000000,0.000000}%
\pgfsetstrokecolor{currentstroke}%
\pgfsetdash{}{0pt}%
\pgfpathmoveto{\pgfqpoint{3.385377in}{0.422992in}}%
\pgfpathlineto{\pgfqpoint{5.592256in}{0.422992in}}%
\pgfusepath{stroke}%
\end{pgfscope}%
\begin{pgfscope}%
\pgfsetrectcap%
\pgfsetmiterjoin%
\pgfsetlinewidth{0.501875pt}%
\definecolor{currentstroke}{rgb}{0.000000,0.000000,0.000000}%
\pgfsetstrokecolor{currentstroke}%
\pgfsetdash{}{0pt}%
\pgfpathmoveto{\pgfqpoint{3.385377in}{3.574193in}}%
\pgfpathlineto{\pgfqpoint{5.592256in}{3.574193in}}%
\pgfusepath{stroke}%
\end{pgfscope}%
\begin{pgfscope}%
\definecolor{textcolor}{rgb}{0.000000,0.000000,0.000000}%
\pgfsetstrokecolor{textcolor}%
\pgfsetfillcolor{textcolor}%
\pgftext[x=4.488817in,y=3.657526in,,base]{\color{textcolor}\rmfamily\fontsize{12.000000}{14.400000}\selectfont LCMC}%
\end{pgfscope}%
\begin{pgfscope}%
\pgfsetrectcap%
\pgfsetroundjoin%
\pgfsetlinewidth{1.003750pt}%
\definecolor{currentstroke}{rgb}{0.047059,0.364706,0.647059}%
\pgfsetstrokecolor{currentstroke}%
\pgfsetdash{}{0pt}%
\pgfpathmoveto{\pgfqpoint{3.510377in}{2.416602in}}%
\pgfpathlineto{\pgfqpoint{3.649266in}{2.416602in}}%
\pgfpathlineto{\pgfqpoint{3.788155in}{2.416602in}}%
\pgfusepath{stroke}%
\end{pgfscope}%
\begin{pgfscope}%
\definecolor{textcolor}{rgb}{0.000000,0.000000,0.000000}%
\pgfsetstrokecolor{textcolor}%
\pgfsetfillcolor{textcolor}%
\pgftext[x=3.899266in,y=2.367990in,left,base]{\color{textcolor}\rmfamily\fontsize{10.000000}{12.000000}\selectfont PCA}%
\end{pgfscope}%
\begin{pgfscope}%
\pgfsetrectcap%
\pgfsetroundjoin%
\pgfsetlinewidth{1.003750pt}%
\definecolor{currentstroke}{rgb}{0.000000,0.725490,0.270588}%
\pgfsetstrokecolor{currentstroke}%
\pgfsetdash{}{0pt}%
\pgfpathmoveto{\pgfqpoint{3.510377in}{2.212744in}}%
\pgfpathlineto{\pgfqpoint{3.649266in}{2.212744in}}%
\pgfpathlineto{\pgfqpoint{3.788155in}{2.212744in}}%
\pgfusepath{stroke}%
\end{pgfscope}%
\begin{pgfscope}%
\definecolor{textcolor}{rgb}{0.000000,0.000000,0.000000}%
\pgfsetstrokecolor{textcolor}%
\pgfsetfillcolor{textcolor}%
\pgftext[x=3.899266in,y=2.164133in,left,base]{\color{textcolor}\rmfamily\fontsize{10.000000}{12.000000}\selectfont KernelPCA}%
\end{pgfscope}%
\begin{pgfscope}%
\pgfsetrectcap%
\pgfsetroundjoin%
\pgfsetlinewidth{1.003750pt}%
\definecolor{currentstroke}{rgb}{1.000000,0.584314,0.000000}%
\pgfsetstrokecolor{currentstroke}%
\pgfsetdash{}{0pt}%
\pgfpathmoveto{\pgfqpoint{3.510377in}{2.008887in}}%
\pgfpathlineto{\pgfqpoint{3.649266in}{2.008887in}}%
\pgfpathlineto{\pgfqpoint{3.788155in}{2.008887in}}%
\pgfusepath{stroke}%
\end{pgfscope}%
\begin{pgfscope}%
\definecolor{textcolor}{rgb}{0.000000,0.000000,0.000000}%
\pgfsetstrokecolor{textcolor}%
\pgfsetfillcolor{textcolor}%
\pgftext[x=3.899266in,y=1.960276in,left,base]{\color{textcolor}\rmfamily\fontsize{10.000000}{12.000000}\selectfont AE}%
\end{pgfscope}%
\begin{pgfscope}%
\pgfsetrectcap%
\pgfsetroundjoin%
\pgfsetlinewidth{1.003750pt}%
\definecolor{currentstroke}{rgb}{1.000000,0.172549,0.000000}%
\pgfsetstrokecolor{currentstroke}%
\pgfsetdash{}{0pt}%
\pgfpathmoveto{\pgfqpoint{3.510377in}{1.805030in}}%
\pgfpathlineto{\pgfqpoint{3.649266in}{1.805030in}}%
\pgfpathlineto{\pgfqpoint{3.788155in}{1.805030in}}%
\pgfusepath{stroke}%
\end{pgfscope}%
\begin{pgfscope}%
\definecolor{textcolor}{rgb}{0.000000,0.000000,0.000000}%
\pgfsetstrokecolor{textcolor}%
\pgfsetfillcolor{textcolor}%
\pgftext[x=3.899266in,y=1.756419in,left,base]{\color{textcolor}\rmfamily\fontsize{10.000000}{12.000000}\selectfont LLE}%
\end{pgfscope}%
\begin{pgfscope}%
\pgfsetrectcap%
\pgfsetroundjoin%
\pgfsetlinewidth{1.003750pt}%
\definecolor{currentstroke}{rgb}{0.517647,0.356863,0.592157}%
\pgfsetstrokecolor{currentstroke}%
\pgfsetdash{}{0pt}%
\pgfpathmoveto{\pgfqpoint{3.510377in}{1.601173in}}%
\pgfpathlineto{\pgfqpoint{3.649266in}{1.601173in}}%
\pgfpathlineto{\pgfqpoint{3.788155in}{1.601173in}}%
\pgfusepath{stroke}%
\end{pgfscope}%
\begin{pgfscope}%
\definecolor{textcolor}{rgb}{0.000000,0.000000,0.000000}%
\pgfsetstrokecolor{textcolor}%
\pgfsetfillcolor{textcolor}%
\pgftext[x=3.899266in,y=1.552562in,left,base]{\color{textcolor}\rmfamily\fontsize{10.000000}{12.000000}\selectfont CAE}%
\end{pgfscope}%
\end{pgfpicture}%
\makeatother%
\endgroup%

	\end{center}
	\caption[Qualitätskriterien für die Swiss Roll]{Die Vertrauenswürdigkeit und Kontinuität der Dimensionsreduktion, sowie das Local Continuity Meta-Criterion (LCMC) für den Swiss Roll Datensatz. Locally Linear Embedding (LLE) schneidet mit Abstand am besten ab. Die restlichen Methoden unterscheiden sich nicht stark, lediglich bei der Vertrauenswürdigkeit können sich die Kernel PCA und der Contractive Autoencoder etwas nach oben absetzen. Die PCA und der Autoencoder sind hinsichtlich der Kontinuität und des LCMC sehr ähnlich. (Eigene Darstellung)}
	\label{fig:SwissRollMetrics}
\end{figure}
und die Qualitätskriterien für den Twin Peaks Datensatz in \figref{fig:TwinPeaksMetrics} abgebildet. Wie dort zu erkennen ist, zeigt Locally Linear Embedding eine sehr starke Performance auf den beiden künstlichen Datensätzen. Der Autoencoder kann sich hinsichtlich des Twin Peaks Datensatzes von den restlichen Methoden abheben. Bei der Swiss Roll ist dies jedoch nicht der Fall. Hier sind die PCA, die Kernel PCA, sowie beide Varianten des Autoencoders relativ ähnlich.

Wie eingangs in \subsecref{ch:Vergleich:sec:VerwendeteDatensaetze:kuenstlich} erwähnt, ist die
Möglichkeit der Visualisierung der latenten Repräsentation ein Vorteil Evaluierung auf künstlichen
Datensätzen. Als Illustration wurden dafür in \figref{fig:SwissRollEmbeddings}
\begin{figure}[ht]
	\centering
	\includegraphics{SwissRollEmbeddings.pdf}
	\caption[Latente zweidimensionale Repräsentation $\mat{Y}$ fünf unterschiedlicher Methodes des Swiss Roll Datensatzes.]{Abgebildet sind die latenten zweidimensionalen Repräsentationen $\mat{Y}$ fünf unterschiedlicher Methoden des Swiss Roll Datensatzes. Nur Locally Linear Embedding ist in der Lage, die Swiss Roll zu \enquote{entfalten}. Dies ist angesichts der hohen Qualitätskriterien von Locally Linear Embedding und der niedrigeren Qualitätskriterien der restlichen Methoden auf diesem Datensatz stimmig (siehe \figref{fig:SwissRollMetrics}). (Eigene Darstellung)}
	\label{fig:SwissRollEmbeddings}
\end{figure}
die zweidimensionalen latenten Repräsentationen $\mat{Y} \in \real^{n \times 2}$ durch Anwendung der fünf Methoden auf den Swiss Roll Datensatz in einem Scatter-Plot dargestellt. Locally Linear Embedding ist als einzige Methode in der Lage, die Swiss Roll zu \enquote{entfalten}. Dies bestätigt die hohe Vertrauenswürdigkeit und Kontinuität der Dimensionsreduktion durch Locally Linear Embedding und die niedrigeren Werte der restlichen Methoden (siehe \figref{fig:SwissRollMetrics}). Die niedrigere Vertrauenswürdigkeit hat insbesondere für die Visualisierung eine große Bedeutung, da dies anzeigt, dass Punkte in der latenten Repräsentation Nachbarn werden, obwohl sie es ursprünglich nicht waren.

\subsection{Resultate auf natürlichen Datensätzen}
\label{ch:Vergleich:sec:Resultate:natuerlich}

Die Qualitätskriterien auf den natürlichen Datensätzen zeigen ein etwas anderes Bild, als es bei
den künstlichen Datensätzen der Fall war. Die starke Performance von Locally Linear Embedding setzt
sich nicht auf den natürlichen hochdimensionalen Datensätzen fort.
\begin{figure}[ht]
	\begin{center}
		%% Creator: Matplotlib, PGF backend
%%
%% To include the figure in your LaTeX document, write
%%   \input{<filename>.pgf}
%%
%% Make sure the required packages are loaded in your preamble
%%   \usepackage{pgf}
%%
%% Also ensure that all the required font packages are loaded; for instance,
%% the lmodern package is sometimes necessary when using math font.
%%   \usepackage{lmodern}
%%
%% Figures using additional raster images can only be included by \input if
%% they are in the same directory as the main LaTeX file. For loading figures
%% from other directories you can use the `import` package
%%   \usepackage{import}
%%
%% and then include the figures with
%%   \import{<path to file>}{<filename>.pgf}
%%
%% Matplotlib used the following preamble
%%   
%%   \usepackage{fontspec}
%%   \setmainfont{DejaVuSerif.ttf}[Path=\detokenize{/Users/moritzmistol/.pyenv/versions/3.9.13/envs/thesis/lib/python3.9/site-packages/matplotlib/mpl-data/fonts/ttf/}]
%%   \setsansfont{DejaVuSans.ttf}[Path=\detokenize{/Users/moritzmistol/.pyenv/versions/3.9.13/envs/thesis/lib/python3.9/site-packages/matplotlib/mpl-data/fonts/ttf/}]
%%   \setmonofont{DejaVuSansMono.ttf}[Path=\detokenize{/Users/moritzmistol/.pyenv/versions/3.9.13/envs/thesis/lib/python3.9/site-packages/matplotlib/mpl-data/fonts/ttf/}]
%%   \makeatletter\@ifpackageloaded{underscore}{}{\usepackage[strings]{underscore}}\makeatother
%%
\begingroup%
\makeatletter%
\begin{pgfpicture}%
\pgfpathrectangle{\pgfpointorigin}{\pgfqpoint{5.785156in}{3.835212in}}%
\pgfusepath{use as bounding box, clip}%
\begin{pgfscope}%
\pgfsetbuttcap%
\pgfsetmiterjoin%
\definecolor{currentfill}{rgb}{1.000000,1.000000,1.000000}%
\pgfsetfillcolor{currentfill}%
\pgfsetlinewidth{0.000000pt}%
\definecolor{currentstroke}{rgb}{1.000000,1.000000,1.000000}%
\pgfsetstrokecolor{currentstroke}%
\pgfsetdash{}{0pt}%
\pgfpathmoveto{\pgfqpoint{0.000000in}{0.000000in}}%
\pgfpathlineto{\pgfqpoint{5.785156in}{0.000000in}}%
\pgfpathlineto{\pgfqpoint{5.785156in}{3.835212in}}%
\pgfpathlineto{\pgfqpoint{0.000000in}{3.835212in}}%
\pgfpathlineto{\pgfqpoint{0.000000in}{0.000000in}}%
\pgfpathclose%
\pgfusepath{fill}%
\end{pgfscope}%
\begin{pgfscope}%
\pgfsetbuttcap%
\pgfsetmiterjoin%
\definecolor{currentfill}{rgb}{1.000000,1.000000,1.000000}%
\pgfsetfillcolor{currentfill}%
\pgfsetlinewidth{0.000000pt}%
\definecolor{currentstroke}{rgb}{0.000000,0.000000,0.000000}%
\pgfsetstrokecolor{currentstroke}%
\pgfsetstrokeopacity{0.000000}%
\pgfsetdash{}{0pt}%
\pgfpathmoveto{\pgfqpoint{0.609415in}{2.347992in}}%
\pgfpathlineto{\pgfqpoint{2.829105in}{2.347992in}}%
\pgfpathlineto{\pgfqpoint{2.829105in}{3.574193in}}%
\pgfpathlineto{\pgfqpoint{0.609415in}{3.574193in}}%
\pgfpathlineto{\pgfqpoint{0.609415in}{2.347992in}}%
\pgfpathclose%
\pgfusepath{fill}%
\end{pgfscope}%
\begin{pgfscope}%
\pgfsetbuttcap%
\pgfsetroundjoin%
\definecolor{currentfill}{rgb}{0.000000,0.000000,0.000000}%
\pgfsetfillcolor{currentfill}%
\pgfsetlinewidth{0.501875pt}%
\definecolor{currentstroke}{rgb}{0.000000,0.000000,0.000000}%
\pgfsetstrokecolor{currentstroke}%
\pgfsetdash{}{0pt}%
\pgfsys@defobject{currentmarker}{\pgfqpoint{0.000000in}{0.000000in}}{\pgfqpoint{0.000000in}{0.041667in}}{%
\pgfpathmoveto{\pgfqpoint{0.000000in}{0.000000in}}%
\pgfpathlineto{\pgfqpoint{0.000000in}{0.041667in}}%
\pgfusepath{stroke,fill}%
}%
\begin{pgfscope}%
\pgfsys@transformshift{0.609415in}{2.347992in}%
\pgfsys@useobject{currentmarker}{}%
\end{pgfscope}%
\end{pgfscope}%
\begin{pgfscope}%
\pgfsetbuttcap%
\pgfsetroundjoin%
\definecolor{currentfill}{rgb}{0.000000,0.000000,0.000000}%
\pgfsetfillcolor{currentfill}%
\pgfsetlinewidth{0.501875pt}%
\definecolor{currentstroke}{rgb}{0.000000,0.000000,0.000000}%
\pgfsetstrokecolor{currentstroke}%
\pgfsetdash{}{0pt}%
\pgfsys@defobject{currentmarker}{\pgfqpoint{0.000000in}{-0.041667in}}{\pgfqpoint{0.000000in}{0.000000in}}{%
\pgfpathmoveto{\pgfqpoint{0.000000in}{0.000000in}}%
\pgfpathlineto{\pgfqpoint{0.000000in}{-0.041667in}}%
\pgfusepath{stroke,fill}%
}%
\begin{pgfscope}%
\pgfsys@transformshift{0.609415in}{3.574193in}%
\pgfsys@useobject{currentmarker}{}%
\end{pgfscope}%
\end{pgfscope}%
\begin{pgfscope}%
\definecolor{textcolor}{rgb}{0.000000,0.000000,0.000000}%
\pgfsetstrokecolor{textcolor}%
\pgfsetfillcolor{textcolor}%
\pgftext[x=0.609415in,y=2.299381in,,top]{\color{textcolor}\rmfamily\fontsize{10.000000}{12.000000}\selectfont \(\displaystyle {0}\)}%
\end{pgfscope}%
\begin{pgfscope}%
\pgfsetbuttcap%
\pgfsetroundjoin%
\definecolor{currentfill}{rgb}{0.000000,0.000000,0.000000}%
\pgfsetfillcolor{currentfill}%
\pgfsetlinewidth{0.501875pt}%
\definecolor{currentstroke}{rgb}{0.000000,0.000000,0.000000}%
\pgfsetstrokecolor{currentstroke}%
\pgfsetdash{}{0pt}%
\pgfsys@defobject{currentmarker}{\pgfqpoint{0.000000in}{0.000000in}}{\pgfqpoint{0.000000in}{0.041667in}}{%
\pgfpathmoveto{\pgfqpoint{0.000000in}{0.000000in}}%
\pgfpathlineto{\pgfqpoint{0.000000in}{0.041667in}}%
\pgfusepath{stroke,fill}%
}%
\begin{pgfscope}%
\pgfsys@transformshift{1.044648in}{2.347992in}%
\pgfsys@useobject{currentmarker}{}%
\end{pgfscope}%
\end{pgfscope}%
\begin{pgfscope}%
\pgfsetbuttcap%
\pgfsetroundjoin%
\definecolor{currentfill}{rgb}{0.000000,0.000000,0.000000}%
\pgfsetfillcolor{currentfill}%
\pgfsetlinewidth{0.501875pt}%
\definecolor{currentstroke}{rgb}{0.000000,0.000000,0.000000}%
\pgfsetstrokecolor{currentstroke}%
\pgfsetdash{}{0pt}%
\pgfsys@defobject{currentmarker}{\pgfqpoint{0.000000in}{-0.041667in}}{\pgfqpoint{0.000000in}{0.000000in}}{%
\pgfpathmoveto{\pgfqpoint{0.000000in}{0.000000in}}%
\pgfpathlineto{\pgfqpoint{0.000000in}{-0.041667in}}%
\pgfusepath{stroke,fill}%
}%
\begin{pgfscope}%
\pgfsys@transformshift{1.044648in}{3.574193in}%
\pgfsys@useobject{currentmarker}{}%
\end{pgfscope}%
\end{pgfscope}%
\begin{pgfscope}%
\definecolor{textcolor}{rgb}{0.000000,0.000000,0.000000}%
\pgfsetstrokecolor{textcolor}%
\pgfsetfillcolor{textcolor}%
\pgftext[x=1.044648in,y=2.299381in,,top]{\color{textcolor}\rmfamily\fontsize{10.000000}{12.000000}\selectfont \(\displaystyle {20}\)}%
\end{pgfscope}%
\begin{pgfscope}%
\pgfsetbuttcap%
\pgfsetroundjoin%
\definecolor{currentfill}{rgb}{0.000000,0.000000,0.000000}%
\pgfsetfillcolor{currentfill}%
\pgfsetlinewidth{0.501875pt}%
\definecolor{currentstroke}{rgb}{0.000000,0.000000,0.000000}%
\pgfsetstrokecolor{currentstroke}%
\pgfsetdash{}{0pt}%
\pgfsys@defobject{currentmarker}{\pgfqpoint{0.000000in}{0.000000in}}{\pgfqpoint{0.000000in}{0.041667in}}{%
\pgfpathmoveto{\pgfqpoint{0.000000in}{0.000000in}}%
\pgfpathlineto{\pgfqpoint{0.000000in}{0.041667in}}%
\pgfusepath{stroke,fill}%
}%
\begin{pgfscope}%
\pgfsys@transformshift{1.479881in}{2.347992in}%
\pgfsys@useobject{currentmarker}{}%
\end{pgfscope}%
\end{pgfscope}%
\begin{pgfscope}%
\pgfsetbuttcap%
\pgfsetroundjoin%
\definecolor{currentfill}{rgb}{0.000000,0.000000,0.000000}%
\pgfsetfillcolor{currentfill}%
\pgfsetlinewidth{0.501875pt}%
\definecolor{currentstroke}{rgb}{0.000000,0.000000,0.000000}%
\pgfsetstrokecolor{currentstroke}%
\pgfsetdash{}{0pt}%
\pgfsys@defobject{currentmarker}{\pgfqpoint{0.000000in}{-0.041667in}}{\pgfqpoint{0.000000in}{0.000000in}}{%
\pgfpathmoveto{\pgfqpoint{0.000000in}{0.000000in}}%
\pgfpathlineto{\pgfqpoint{0.000000in}{-0.041667in}}%
\pgfusepath{stroke,fill}%
}%
\begin{pgfscope}%
\pgfsys@transformshift{1.479881in}{3.574193in}%
\pgfsys@useobject{currentmarker}{}%
\end{pgfscope}%
\end{pgfscope}%
\begin{pgfscope}%
\definecolor{textcolor}{rgb}{0.000000,0.000000,0.000000}%
\pgfsetstrokecolor{textcolor}%
\pgfsetfillcolor{textcolor}%
\pgftext[x=1.479881in,y=2.299381in,,top]{\color{textcolor}\rmfamily\fontsize{10.000000}{12.000000}\selectfont \(\displaystyle {40}\)}%
\end{pgfscope}%
\begin{pgfscope}%
\pgfsetbuttcap%
\pgfsetroundjoin%
\definecolor{currentfill}{rgb}{0.000000,0.000000,0.000000}%
\pgfsetfillcolor{currentfill}%
\pgfsetlinewidth{0.501875pt}%
\definecolor{currentstroke}{rgb}{0.000000,0.000000,0.000000}%
\pgfsetstrokecolor{currentstroke}%
\pgfsetdash{}{0pt}%
\pgfsys@defobject{currentmarker}{\pgfqpoint{0.000000in}{0.000000in}}{\pgfqpoint{0.000000in}{0.041667in}}{%
\pgfpathmoveto{\pgfqpoint{0.000000in}{0.000000in}}%
\pgfpathlineto{\pgfqpoint{0.000000in}{0.041667in}}%
\pgfusepath{stroke,fill}%
}%
\begin{pgfscope}%
\pgfsys@transformshift{1.915115in}{2.347992in}%
\pgfsys@useobject{currentmarker}{}%
\end{pgfscope}%
\end{pgfscope}%
\begin{pgfscope}%
\pgfsetbuttcap%
\pgfsetroundjoin%
\definecolor{currentfill}{rgb}{0.000000,0.000000,0.000000}%
\pgfsetfillcolor{currentfill}%
\pgfsetlinewidth{0.501875pt}%
\definecolor{currentstroke}{rgb}{0.000000,0.000000,0.000000}%
\pgfsetstrokecolor{currentstroke}%
\pgfsetdash{}{0pt}%
\pgfsys@defobject{currentmarker}{\pgfqpoint{0.000000in}{-0.041667in}}{\pgfqpoint{0.000000in}{0.000000in}}{%
\pgfpathmoveto{\pgfqpoint{0.000000in}{0.000000in}}%
\pgfpathlineto{\pgfqpoint{0.000000in}{-0.041667in}}%
\pgfusepath{stroke,fill}%
}%
\begin{pgfscope}%
\pgfsys@transformshift{1.915115in}{3.574193in}%
\pgfsys@useobject{currentmarker}{}%
\end{pgfscope}%
\end{pgfscope}%
\begin{pgfscope}%
\definecolor{textcolor}{rgb}{0.000000,0.000000,0.000000}%
\pgfsetstrokecolor{textcolor}%
\pgfsetfillcolor{textcolor}%
\pgftext[x=1.915115in,y=2.299381in,,top]{\color{textcolor}\rmfamily\fontsize{10.000000}{12.000000}\selectfont \(\displaystyle {60}\)}%
\end{pgfscope}%
\begin{pgfscope}%
\pgfsetbuttcap%
\pgfsetroundjoin%
\definecolor{currentfill}{rgb}{0.000000,0.000000,0.000000}%
\pgfsetfillcolor{currentfill}%
\pgfsetlinewidth{0.501875pt}%
\definecolor{currentstroke}{rgb}{0.000000,0.000000,0.000000}%
\pgfsetstrokecolor{currentstroke}%
\pgfsetdash{}{0pt}%
\pgfsys@defobject{currentmarker}{\pgfqpoint{0.000000in}{0.000000in}}{\pgfqpoint{0.000000in}{0.041667in}}{%
\pgfpathmoveto{\pgfqpoint{0.000000in}{0.000000in}}%
\pgfpathlineto{\pgfqpoint{0.000000in}{0.041667in}}%
\pgfusepath{stroke,fill}%
}%
\begin{pgfscope}%
\pgfsys@transformshift{2.350348in}{2.347992in}%
\pgfsys@useobject{currentmarker}{}%
\end{pgfscope}%
\end{pgfscope}%
\begin{pgfscope}%
\pgfsetbuttcap%
\pgfsetroundjoin%
\definecolor{currentfill}{rgb}{0.000000,0.000000,0.000000}%
\pgfsetfillcolor{currentfill}%
\pgfsetlinewidth{0.501875pt}%
\definecolor{currentstroke}{rgb}{0.000000,0.000000,0.000000}%
\pgfsetstrokecolor{currentstroke}%
\pgfsetdash{}{0pt}%
\pgfsys@defobject{currentmarker}{\pgfqpoint{0.000000in}{-0.041667in}}{\pgfqpoint{0.000000in}{0.000000in}}{%
\pgfpathmoveto{\pgfqpoint{0.000000in}{0.000000in}}%
\pgfpathlineto{\pgfqpoint{0.000000in}{-0.041667in}}%
\pgfusepath{stroke,fill}%
}%
\begin{pgfscope}%
\pgfsys@transformshift{2.350348in}{3.574193in}%
\pgfsys@useobject{currentmarker}{}%
\end{pgfscope}%
\end{pgfscope}%
\begin{pgfscope}%
\definecolor{textcolor}{rgb}{0.000000,0.000000,0.000000}%
\pgfsetstrokecolor{textcolor}%
\pgfsetfillcolor{textcolor}%
\pgftext[x=2.350348in,y=2.299381in,,top]{\color{textcolor}\rmfamily\fontsize{10.000000}{12.000000}\selectfont \(\displaystyle {80}\)}%
\end{pgfscope}%
\begin{pgfscope}%
\pgfsetbuttcap%
\pgfsetroundjoin%
\definecolor{currentfill}{rgb}{0.000000,0.000000,0.000000}%
\pgfsetfillcolor{currentfill}%
\pgfsetlinewidth{0.501875pt}%
\definecolor{currentstroke}{rgb}{0.000000,0.000000,0.000000}%
\pgfsetstrokecolor{currentstroke}%
\pgfsetdash{}{0pt}%
\pgfsys@defobject{currentmarker}{\pgfqpoint{0.000000in}{0.000000in}}{\pgfqpoint{0.000000in}{0.041667in}}{%
\pgfpathmoveto{\pgfqpoint{0.000000in}{0.000000in}}%
\pgfpathlineto{\pgfqpoint{0.000000in}{0.041667in}}%
\pgfusepath{stroke,fill}%
}%
\begin{pgfscope}%
\pgfsys@transformshift{2.785581in}{2.347992in}%
\pgfsys@useobject{currentmarker}{}%
\end{pgfscope}%
\end{pgfscope}%
\begin{pgfscope}%
\pgfsetbuttcap%
\pgfsetroundjoin%
\definecolor{currentfill}{rgb}{0.000000,0.000000,0.000000}%
\pgfsetfillcolor{currentfill}%
\pgfsetlinewidth{0.501875pt}%
\definecolor{currentstroke}{rgb}{0.000000,0.000000,0.000000}%
\pgfsetstrokecolor{currentstroke}%
\pgfsetdash{}{0pt}%
\pgfsys@defobject{currentmarker}{\pgfqpoint{0.000000in}{-0.041667in}}{\pgfqpoint{0.000000in}{0.000000in}}{%
\pgfpathmoveto{\pgfqpoint{0.000000in}{0.000000in}}%
\pgfpathlineto{\pgfqpoint{0.000000in}{-0.041667in}}%
\pgfusepath{stroke,fill}%
}%
\begin{pgfscope}%
\pgfsys@transformshift{2.785581in}{3.574193in}%
\pgfsys@useobject{currentmarker}{}%
\end{pgfscope}%
\end{pgfscope}%
\begin{pgfscope}%
\definecolor{textcolor}{rgb}{0.000000,0.000000,0.000000}%
\pgfsetstrokecolor{textcolor}%
\pgfsetfillcolor{textcolor}%
\pgftext[x=2.785581in,y=2.299381in,,top]{\color{textcolor}\rmfamily\fontsize{10.000000}{12.000000}\selectfont \(\displaystyle {100}\)}%
\end{pgfscope}%
\begin{pgfscope}%
\pgfsetbuttcap%
\pgfsetroundjoin%
\definecolor{currentfill}{rgb}{0.000000,0.000000,0.000000}%
\pgfsetfillcolor{currentfill}%
\pgfsetlinewidth{0.501875pt}%
\definecolor{currentstroke}{rgb}{0.000000,0.000000,0.000000}%
\pgfsetstrokecolor{currentstroke}%
\pgfsetdash{}{0pt}%
\pgfsys@defobject{currentmarker}{\pgfqpoint{0.000000in}{0.000000in}}{\pgfqpoint{0.000000in}{0.020833in}}{%
\pgfpathmoveto{\pgfqpoint{0.000000in}{0.000000in}}%
\pgfpathlineto{\pgfqpoint{0.000000in}{0.020833in}}%
\pgfusepath{stroke,fill}%
}%
\begin{pgfscope}%
\pgfsys@transformshift{0.718223in}{2.347992in}%
\pgfsys@useobject{currentmarker}{}%
\end{pgfscope}%
\end{pgfscope}%
\begin{pgfscope}%
\pgfsetbuttcap%
\pgfsetroundjoin%
\definecolor{currentfill}{rgb}{0.000000,0.000000,0.000000}%
\pgfsetfillcolor{currentfill}%
\pgfsetlinewidth{0.501875pt}%
\definecolor{currentstroke}{rgb}{0.000000,0.000000,0.000000}%
\pgfsetstrokecolor{currentstroke}%
\pgfsetdash{}{0pt}%
\pgfsys@defobject{currentmarker}{\pgfqpoint{0.000000in}{-0.020833in}}{\pgfqpoint{0.000000in}{0.000000in}}{%
\pgfpathmoveto{\pgfqpoint{0.000000in}{0.000000in}}%
\pgfpathlineto{\pgfqpoint{0.000000in}{-0.020833in}}%
\pgfusepath{stroke,fill}%
}%
\begin{pgfscope}%
\pgfsys@transformshift{0.718223in}{3.574193in}%
\pgfsys@useobject{currentmarker}{}%
\end{pgfscope}%
\end{pgfscope}%
\begin{pgfscope}%
\pgfsetbuttcap%
\pgfsetroundjoin%
\definecolor{currentfill}{rgb}{0.000000,0.000000,0.000000}%
\pgfsetfillcolor{currentfill}%
\pgfsetlinewidth{0.501875pt}%
\definecolor{currentstroke}{rgb}{0.000000,0.000000,0.000000}%
\pgfsetstrokecolor{currentstroke}%
\pgfsetdash{}{0pt}%
\pgfsys@defobject{currentmarker}{\pgfqpoint{0.000000in}{0.000000in}}{\pgfqpoint{0.000000in}{0.020833in}}{%
\pgfpathmoveto{\pgfqpoint{0.000000in}{0.000000in}}%
\pgfpathlineto{\pgfqpoint{0.000000in}{0.020833in}}%
\pgfusepath{stroke,fill}%
}%
\begin{pgfscope}%
\pgfsys@transformshift{0.827031in}{2.347992in}%
\pgfsys@useobject{currentmarker}{}%
\end{pgfscope}%
\end{pgfscope}%
\begin{pgfscope}%
\pgfsetbuttcap%
\pgfsetroundjoin%
\definecolor{currentfill}{rgb}{0.000000,0.000000,0.000000}%
\pgfsetfillcolor{currentfill}%
\pgfsetlinewidth{0.501875pt}%
\definecolor{currentstroke}{rgb}{0.000000,0.000000,0.000000}%
\pgfsetstrokecolor{currentstroke}%
\pgfsetdash{}{0pt}%
\pgfsys@defobject{currentmarker}{\pgfqpoint{0.000000in}{-0.020833in}}{\pgfqpoint{0.000000in}{0.000000in}}{%
\pgfpathmoveto{\pgfqpoint{0.000000in}{0.000000in}}%
\pgfpathlineto{\pgfqpoint{0.000000in}{-0.020833in}}%
\pgfusepath{stroke,fill}%
}%
\begin{pgfscope}%
\pgfsys@transformshift{0.827031in}{3.574193in}%
\pgfsys@useobject{currentmarker}{}%
\end{pgfscope}%
\end{pgfscope}%
\begin{pgfscope}%
\pgfsetbuttcap%
\pgfsetroundjoin%
\definecolor{currentfill}{rgb}{0.000000,0.000000,0.000000}%
\pgfsetfillcolor{currentfill}%
\pgfsetlinewidth{0.501875pt}%
\definecolor{currentstroke}{rgb}{0.000000,0.000000,0.000000}%
\pgfsetstrokecolor{currentstroke}%
\pgfsetdash{}{0pt}%
\pgfsys@defobject{currentmarker}{\pgfqpoint{0.000000in}{0.000000in}}{\pgfqpoint{0.000000in}{0.020833in}}{%
\pgfpathmoveto{\pgfqpoint{0.000000in}{0.000000in}}%
\pgfpathlineto{\pgfqpoint{0.000000in}{0.020833in}}%
\pgfusepath{stroke,fill}%
}%
\begin{pgfscope}%
\pgfsys@transformshift{0.935840in}{2.347992in}%
\pgfsys@useobject{currentmarker}{}%
\end{pgfscope}%
\end{pgfscope}%
\begin{pgfscope}%
\pgfsetbuttcap%
\pgfsetroundjoin%
\definecolor{currentfill}{rgb}{0.000000,0.000000,0.000000}%
\pgfsetfillcolor{currentfill}%
\pgfsetlinewidth{0.501875pt}%
\definecolor{currentstroke}{rgb}{0.000000,0.000000,0.000000}%
\pgfsetstrokecolor{currentstroke}%
\pgfsetdash{}{0pt}%
\pgfsys@defobject{currentmarker}{\pgfqpoint{0.000000in}{-0.020833in}}{\pgfqpoint{0.000000in}{0.000000in}}{%
\pgfpathmoveto{\pgfqpoint{0.000000in}{0.000000in}}%
\pgfpathlineto{\pgfqpoint{0.000000in}{-0.020833in}}%
\pgfusepath{stroke,fill}%
}%
\begin{pgfscope}%
\pgfsys@transformshift{0.935840in}{3.574193in}%
\pgfsys@useobject{currentmarker}{}%
\end{pgfscope}%
\end{pgfscope}%
\begin{pgfscope}%
\pgfsetbuttcap%
\pgfsetroundjoin%
\definecolor{currentfill}{rgb}{0.000000,0.000000,0.000000}%
\pgfsetfillcolor{currentfill}%
\pgfsetlinewidth{0.501875pt}%
\definecolor{currentstroke}{rgb}{0.000000,0.000000,0.000000}%
\pgfsetstrokecolor{currentstroke}%
\pgfsetdash{}{0pt}%
\pgfsys@defobject{currentmarker}{\pgfqpoint{0.000000in}{0.000000in}}{\pgfqpoint{0.000000in}{0.020833in}}{%
\pgfpathmoveto{\pgfqpoint{0.000000in}{0.000000in}}%
\pgfpathlineto{\pgfqpoint{0.000000in}{0.020833in}}%
\pgfusepath{stroke,fill}%
}%
\begin{pgfscope}%
\pgfsys@transformshift{1.153456in}{2.347992in}%
\pgfsys@useobject{currentmarker}{}%
\end{pgfscope}%
\end{pgfscope}%
\begin{pgfscope}%
\pgfsetbuttcap%
\pgfsetroundjoin%
\definecolor{currentfill}{rgb}{0.000000,0.000000,0.000000}%
\pgfsetfillcolor{currentfill}%
\pgfsetlinewidth{0.501875pt}%
\definecolor{currentstroke}{rgb}{0.000000,0.000000,0.000000}%
\pgfsetstrokecolor{currentstroke}%
\pgfsetdash{}{0pt}%
\pgfsys@defobject{currentmarker}{\pgfqpoint{0.000000in}{-0.020833in}}{\pgfqpoint{0.000000in}{0.000000in}}{%
\pgfpathmoveto{\pgfqpoint{0.000000in}{0.000000in}}%
\pgfpathlineto{\pgfqpoint{0.000000in}{-0.020833in}}%
\pgfusepath{stroke,fill}%
}%
\begin{pgfscope}%
\pgfsys@transformshift{1.153456in}{3.574193in}%
\pgfsys@useobject{currentmarker}{}%
\end{pgfscope}%
\end{pgfscope}%
\begin{pgfscope}%
\pgfsetbuttcap%
\pgfsetroundjoin%
\definecolor{currentfill}{rgb}{0.000000,0.000000,0.000000}%
\pgfsetfillcolor{currentfill}%
\pgfsetlinewidth{0.501875pt}%
\definecolor{currentstroke}{rgb}{0.000000,0.000000,0.000000}%
\pgfsetstrokecolor{currentstroke}%
\pgfsetdash{}{0pt}%
\pgfsys@defobject{currentmarker}{\pgfqpoint{0.000000in}{0.000000in}}{\pgfqpoint{0.000000in}{0.020833in}}{%
\pgfpathmoveto{\pgfqpoint{0.000000in}{0.000000in}}%
\pgfpathlineto{\pgfqpoint{0.000000in}{0.020833in}}%
\pgfusepath{stroke,fill}%
}%
\begin{pgfscope}%
\pgfsys@transformshift{1.262265in}{2.347992in}%
\pgfsys@useobject{currentmarker}{}%
\end{pgfscope}%
\end{pgfscope}%
\begin{pgfscope}%
\pgfsetbuttcap%
\pgfsetroundjoin%
\definecolor{currentfill}{rgb}{0.000000,0.000000,0.000000}%
\pgfsetfillcolor{currentfill}%
\pgfsetlinewidth{0.501875pt}%
\definecolor{currentstroke}{rgb}{0.000000,0.000000,0.000000}%
\pgfsetstrokecolor{currentstroke}%
\pgfsetdash{}{0pt}%
\pgfsys@defobject{currentmarker}{\pgfqpoint{0.000000in}{-0.020833in}}{\pgfqpoint{0.000000in}{0.000000in}}{%
\pgfpathmoveto{\pgfqpoint{0.000000in}{0.000000in}}%
\pgfpathlineto{\pgfqpoint{0.000000in}{-0.020833in}}%
\pgfusepath{stroke,fill}%
}%
\begin{pgfscope}%
\pgfsys@transformshift{1.262265in}{3.574193in}%
\pgfsys@useobject{currentmarker}{}%
\end{pgfscope}%
\end{pgfscope}%
\begin{pgfscope}%
\pgfsetbuttcap%
\pgfsetroundjoin%
\definecolor{currentfill}{rgb}{0.000000,0.000000,0.000000}%
\pgfsetfillcolor{currentfill}%
\pgfsetlinewidth{0.501875pt}%
\definecolor{currentstroke}{rgb}{0.000000,0.000000,0.000000}%
\pgfsetstrokecolor{currentstroke}%
\pgfsetdash{}{0pt}%
\pgfsys@defobject{currentmarker}{\pgfqpoint{0.000000in}{0.000000in}}{\pgfqpoint{0.000000in}{0.020833in}}{%
\pgfpathmoveto{\pgfqpoint{0.000000in}{0.000000in}}%
\pgfpathlineto{\pgfqpoint{0.000000in}{0.020833in}}%
\pgfusepath{stroke,fill}%
}%
\begin{pgfscope}%
\pgfsys@transformshift{1.371073in}{2.347992in}%
\pgfsys@useobject{currentmarker}{}%
\end{pgfscope}%
\end{pgfscope}%
\begin{pgfscope}%
\pgfsetbuttcap%
\pgfsetroundjoin%
\definecolor{currentfill}{rgb}{0.000000,0.000000,0.000000}%
\pgfsetfillcolor{currentfill}%
\pgfsetlinewidth{0.501875pt}%
\definecolor{currentstroke}{rgb}{0.000000,0.000000,0.000000}%
\pgfsetstrokecolor{currentstroke}%
\pgfsetdash{}{0pt}%
\pgfsys@defobject{currentmarker}{\pgfqpoint{0.000000in}{-0.020833in}}{\pgfqpoint{0.000000in}{0.000000in}}{%
\pgfpathmoveto{\pgfqpoint{0.000000in}{0.000000in}}%
\pgfpathlineto{\pgfqpoint{0.000000in}{-0.020833in}}%
\pgfusepath{stroke,fill}%
}%
\begin{pgfscope}%
\pgfsys@transformshift{1.371073in}{3.574193in}%
\pgfsys@useobject{currentmarker}{}%
\end{pgfscope}%
\end{pgfscope}%
\begin{pgfscope}%
\pgfsetbuttcap%
\pgfsetroundjoin%
\definecolor{currentfill}{rgb}{0.000000,0.000000,0.000000}%
\pgfsetfillcolor{currentfill}%
\pgfsetlinewidth{0.501875pt}%
\definecolor{currentstroke}{rgb}{0.000000,0.000000,0.000000}%
\pgfsetstrokecolor{currentstroke}%
\pgfsetdash{}{0pt}%
\pgfsys@defobject{currentmarker}{\pgfqpoint{0.000000in}{0.000000in}}{\pgfqpoint{0.000000in}{0.020833in}}{%
\pgfpathmoveto{\pgfqpoint{0.000000in}{0.000000in}}%
\pgfpathlineto{\pgfqpoint{0.000000in}{0.020833in}}%
\pgfusepath{stroke,fill}%
}%
\begin{pgfscope}%
\pgfsys@transformshift{1.588690in}{2.347992in}%
\pgfsys@useobject{currentmarker}{}%
\end{pgfscope}%
\end{pgfscope}%
\begin{pgfscope}%
\pgfsetbuttcap%
\pgfsetroundjoin%
\definecolor{currentfill}{rgb}{0.000000,0.000000,0.000000}%
\pgfsetfillcolor{currentfill}%
\pgfsetlinewidth{0.501875pt}%
\definecolor{currentstroke}{rgb}{0.000000,0.000000,0.000000}%
\pgfsetstrokecolor{currentstroke}%
\pgfsetdash{}{0pt}%
\pgfsys@defobject{currentmarker}{\pgfqpoint{0.000000in}{-0.020833in}}{\pgfqpoint{0.000000in}{0.000000in}}{%
\pgfpathmoveto{\pgfqpoint{0.000000in}{0.000000in}}%
\pgfpathlineto{\pgfqpoint{0.000000in}{-0.020833in}}%
\pgfusepath{stroke,fill}%
}%
\begin{pgfscope}%
\pgfsys@transformshift{1.588690in}{3.574193in}%
\pgfsys@useobject{currentmarker}{}%
\end{pgfscope}%
\end{pgfscope}%
\begin{pgfscope}%
\pgfsetbuttcap%
\pgfsetroundjoin%
\definecolor{currentfill}{rgb}{0.000000,0.000000,0.000000}%
\pgfsetfillcolor{currentfill}%
\pgfsetlinewidth{0.501875pt}%
\definecolor{currentstroke}{rgb}{0.000000,0.000000,0.000000}%
\pgfsetstrokecolor{currentstroke}%
\pgfsetdash{}{0pt}%
\pgfsys@defobject{currentmarker}{\pgfqpoint{0.000000in}{0.000000in}}{\pgfqpoint{0.000000in}{0.020833in}}{%
\pgfpathmoveto{\pgfqpoint{0.000000in}{0.000000in}}%
\pgfpathlineto{\pgfqpoint{0.000000in}{0.020833in}}%
\pgfusepath{stroke,fill}%
}%
\begin{pgfscope}%
\pgfsys@transformshift{1.697498in}{2.347992in}%
\pgfsys@useobject{currentmarker}{}%
\end{pgfscope}%
\end{pgfscope}%
\begin{pgfscope}%
\pgfsetbuttcap%
\pgfsetroundjoin%
\definecolor{currentfill}{rgb}{0.000000,0.000000,0.000000}%
\pgfsetfillcolor{currentfill}%
\pgfsetlinewidth{0.501875pt}%
\definecolor{currentstroke}{rgb}{0.000000,0.000000,0.000000}%
\pgfsetstrokecolor{currentstroke}%
\pgfsetdash{}{0pt}%
\pgfsys@defobject{currentmarker}{\pgfqpoint{0.000000in}{-0.020833in}}{\pgfqpoint{0.000000in}{0.000000in}}{%
\pgfpathmoveto{\pgfqpoint{0.000000in}{0.000000in}}%
\pgfpathlineto{\pgfqpoint{0.000000in}{-0.020833in}}%
\pgfusepath{stroke,fill}%
}%
\begin{pgfscope}%
\pgfsys@transformshift{1.697498in}{3.574193in}%
\pgfsys@useobject{currentmarker}{}%
\end{pgfscope}%
\end{pgfscope}%
\begin{pgfscope}%
\pgfsetbuttcap%
\pgfsetroundjoin%
\definecolor{currentfill}{rgb}{0.000000,0.000000,0.000000}%
\pgfsetfillcolor{currentfill}%
\pgfsetlinewidth{0.501875pt}%
\definecolor{currentstroke}{rgb}{0.000000,0.000000,0.000000}%
\pgfsetstrokecolor{currentstroke}%
\pgfsetdash{}{0pt}%
\pgfsys@defobject{currentmarker}{\pgfqpoint{0.000000in}{0.000000in}}{\pgfqpoint{0.000000in}{0.020833in}}{%
\pgfpathmoveto{\pgfqpoint{0.000000in}{0.000000in}}%
\pgfpathlineto{\pgfqpoint{0.000000in}{0.020833in}}%
\pgfusepath{stroke,fill}%
}%
\begin{pgfscope}%
\pgfsys@transformshift{1.806306in}{2.347992in}%
\pgfsys@useobject{currentmarker}{}%
\end{pgfscope}%
\end{pgfscope}%
\begin{pgfscope}%
\pgfsetbuttcap%
\pgfsetroundjoin%
\definecolor{currentfill}{rgb}{0.000000,0.000000,0.000000}%
\pgfsetfillcolor{currentfill}%
\pgfsetlinewidth{0.501875pt}%
\definecolor{currentstroke}{rgb}{0.000000,0.000000,0.000000}%
\pgfsetstrokecolor{currentstroke}%
\pgfsetdash{}{0pt}%
\pgfsys@defobject{currentmarker}{\pgfqpoint{0.000000in}{-0.020833in}}{\pgfqpoint{0.000000in}{0.000000in}}{%
\pgfpathmoveto{\pgfqpoint{0.000000in}{0.000000in}}%
\pgfpathlineto{\pgfqpoint{0.000000in}{-0.020833in}}%
\pgfusepath{stroke,fill}%
}%
\begin{pgfscope}%
\pgfsys@transformshift{1.806306in}{3.574193in}%
\pgfsys@useobject{currentmarker}{}%
\end{pgfscope}%
\end{pgfscope}%
\begin{pgfscope}%
\pgfsetbuttcap%
\pgfsetroundjoin%
\definecolor{currentfill}{rgb}{0.000000,0.000000,0.000000}%
\pgfsetfillcolor{currentfill}%
\pgfsetlinewidth{0.501875pt}%
\definecolor{currentstroke}{rgb}{0.000000,0.000000,0.000000}%
\pgfsetstrokecolor{currentstroke}%
\pgfsetdash{}{0pt}%
\pgfsys@defobject{currentmarker}{\pgfqpoint{0.000000in}{0.000000in}}{\pgfqpoint{0.000000in}{0.020833in}}{%
\pgfpathmoveto{\pgfqpoint{0.000000in}{0.000000in}}%
\pgfpathlineto{\pgfqpoint{0.000000in}{0.020833in}}%
\pgfusepath{stroke,fill}%
}%
\begin{pgfscope}%
\pgfsys@transformshift{2.023923in}{2.347992in}%
\pgfsys@useobject{currentmarker}{}%
\end{pgfscope}%
\end{pgfscope}%
\begin{pgfscope}%
\pgfsetbuttcap%
\pgfsetroundjoin%
\definecolor{currentfill}{rgb}{0.000000,0.000000,0.000000}%
\pgfsetfillcolor{currentfill}%
\pgfsetlinewidth{0.501875pt}%
\definecolor{currentstroke}{rgb}{0.000000,0.000000,0.000000}%
\pgfsetstrokecolor{currentstroke}%
\pgfsetdash{}{0pt}%
\pgfsys@defobject{currentmarker}{\pgfqpoint{0.000000in}{-0.020833in}}{\pgfqpoint{0.000000in}{0.000000in}}{%
\pgfpathmoveto{\pgfqpoint{0.000000in}{0.000000in}}%
\pgfpathlineto{\pgfqpoint{0.000000in}{-0.020833in}}%
\pgfusepath{stroke,fill}%
}%
\begin{pgfscope}%
\pgfsys@transformshift{2.023923in}{3.574193in}%
\pgfsys@useobject{currentmarker}{}%
\end{pgfscope}%
\end{pgfscope}%
\begin{pgfscope}%
\pgfsetbuttcap%
\pgfsetroundjoin%
\definecolor{currentfill}{rgb}{0.000000,0.000000,0.000000}%
\pgfsetfillcolor{currentfill}%
\pgfsetlinewidth{0.501875pt}%
\definecolor{currentstroke}{rgb}{0.000000,0.000000,0.000000}%
\pgfsetstrokecolor{currentstroke}%
\pgfsetdash{}{0pt}%
\pgfsys@defobject{currentmarker}{\pgfqpoint{0.000000in}{0.000000in}}{\pgfqpoint{0.000000in}{0.020833in}}{%
\pgfpathmoveto{\pgfqpoint{0.000000in}{0.000000in}}%
\pgfpathlineto{\pgfqpoint{0.000000in}{0.020833in}}%
\pgfusepath{stroke,fill}%
}%
\begin{pgfscope}%
\pgfsys@transformshift{2.132731in}{2.347992in}%
\pgfsys@useobject{currentmarker}{}%
\end{pgfscope}%
\end{pgfscope}%
\begin{pgfscope}%
\pgfsetbuttcap%
\pgfsetroundjoin%
\definecolor{currentfill}{rgb}{0.000000,0.000000,0.000000}%
\pgfsetfillcolor{currentfill}%
\pgfsetlinewidth{0.501875pt}%
\definecolor{currentstroke}{rgb}{0.000000,0.000000,0.000000}%
\pgfsetstrokecolor{currentstroke}%
\pgfsetdash{}{0pt}%
\pgfsys@defobject{currentmarker}{\pgfqpoint{0.000000in}{-0.020833in}}{\pgfqpoint{0.000000in}{0.000000in}}{%
\pgfpathmoveto{\pgfqpoint{0.000000in}{0.000000in}}%
\pgfpathlineto{\pgfqpoint{0.000000in}{-0.020833in}}%
\pgfusepath{stroke,fill}%
}%
\begin{pgfscope}%
\pgfsys@transformshift{2.132731in}{3.574193in}%
\pgfsys@useobject{currentmarker}{}%
\end{pgfscope}%
\end{pgfscope}%
\begin{pgfscope}%
\pgfsetbuttcap%
\pgfsetroundjoin%
\definecolor{currentfill}{rgb}{0.000000,0.000000,0.000000}%
\pgfsetfillcolor{currentfill}%
\pgfsetlinewidth{0.501875pt}%
\definecolor{currentstroke}{rgb}{0.000000,0.000000,0.000000}%
\pgfsetstrokecolor{currentstroke}%
\pgfsetdash{}{0pt}%
\pgfsys@defobject{currentmarker}{\pgfqpoint{0.000000in}{0.000000in}}{\pgfqpoint{0.000000in}{0.020833in}}{%
\pgfpathmoveto{\pgfqpoint{0.000000in}{0.000000in}}%
\pgfpathlineto{\pgfqpoint{0.000000in}{0.020833in}}%
\pgfusepath{stroke,fill}%
}%
\begin{pgfscope}%
\pgfsys@transformshift{2.241540in}{2.347992in}%
\pgfsys@useobject{currentmarker}{}%
\end{pgfscope}%
\end{pgfscope}%
\begin{pgfscope}%
\pgfsetbuttcap%
\pgfsetroundjoin%
\definecolor{currentfill}{rgb}{0.000000,0.000000,0.000000}%
\pgfsetfillcolor{currentfill}%
\pgfsetlinewidth{0.501875pt}%
\definecolor{currentstroke}{rgb}{0.000000,0.000000,0.000000}%
\pgfsetstrokecolor{currentstroke}%
\pgfsetdash{}{0pt}%
\pgfsys@defobject{currentmarker}{\pgfqpoint{0.000000in}{-0.020833in}}{\pgfqpoint{0.000000in}{0.000000in}}{%
\pgfpathmoveto{\pgfqpoint{0.000000in}{0.000000in}}%
\pgfpathlineto{\pgfqpoint{0.000000in}{-0.020833in}}%
\pgfusepath{stroke,fill}%
}%
\begin{pgfscope}%
\pgfsys@transformshift{2.241540in}{3.574193in}%
\pgfsys@useobject{currentmarker}{}%
\end{pgfscope}%
\end{pgfscope}%
\begin{pgfscope}%
\pgfsetbuttcap%
\pgfsetroundjoin%
\definecolor{currentfill}{rgb}{0.000000,0.000000,0.000000}%
\pgfsetfillcolor{currentfill}%
\pgfsetlinewidth{0.501875pt}%
\definecolor{currentstroke}{rgb}{0.000000,0.000000,0.000000}%
\pgfsetstrokecolor{currentstroke}%
\pgfsetdash{}{0pt}%
\pgfsys@defobject{currentmarker}{\pgfqpoint{0.000000in}{0.000000in}}{\pgfqpoint{0.000000in}{0.020833in}}{%
\pgfpathmoveto{\pgfqpoint{0.000000in}{0.000000in}}%
\pgfpathlineto{\pgfqpoint{0.000000in}{0.020833in}}%
\pgfusepath{stroke,fill}%
}%
\begin{pgfscope}%
\pgfsys@transformshift{2.459156in}{2.347992in}%
\pgfsys@useobject{currentmarker}{}%
\end{pgfscope}%
\end{pgfscope}%
\begin{pgfscope}%
\pgfsetbuttcap%
\pgfsetroundjoin%
\definecolor{currentfill}{rgb}{0.000000,0.000000,0.000000}%
\pgfsetfillcolor{currentfill}%
\pgfsetlinewidth{0.501875pt}%
\definecolor{currentstroke}{rgb}{0.000000,0.000000,0.000000}%
\pgfsetstrokecolor{currentstroke}%
\pgfsetdash{}{0pt}%
\pgfsys@defobject{currentmarker}{\pgfqpoint{0.000000in}{-0.020833in}}{\pgfqpoint{0.000000in}{0.000000in}}{%
\pgfpathmoveto{\pgfqpoint{0.000000in}{0.000000in}}%
\pgfpathlineto{\pgfqpoint{0.000000in}{-0.020833in}}%
\pgfusepath{stroke,fill}%
}%
\begin{pgfscope}%
\pgfsys@transformshift{2.459156in}{3.574193in}%
\pgfsys@useobject{currentmarker}{}%
\end{pgfscope}%
\end{pgfscope}%
\begin{pgfscope}%
\pgfsetbuttcap%
\pgfsetroundjoin%
\definecolor{currentfill}{rgb}{0.000000,0.000000,0.000000}%
\pgfsetfillcolor{currentfill}%
\pgfsetlinewidth{0.501875pt}%
\definecolor{currentstroke}{rgb}{0.000000,0.000000,0.000000}%
\pgfsetstrokecolor{currentstroke}%
\pgfsetdash{}{0pt}%
\pgfsys@defobject{currentmarker}{\pgfqpoint{0.000000in}{0.000000in}}{\pgfqpoint{0.000000in}{0.020833in}}{%
\pgfpathmoveto{\pgfqpoint{0.000000in}{0.000000in}}%
\pgfpathlineto{\pgfqpoint{0.000000in}{0.020833in}}%
\pgfusepath{stroke,fill}%
}%
\begin{pgfscope}%
\pgfsys@transformshift{2.567965in}{2.347992in}%
\pgfsys@useobject{currentmarker}{}%
\end{pgfscope}%
\end{pgfscope}%
\begin{pgfscope}%
\pgfsetbuttcap%
\pgfsetroundjoin%
\definecolor{currentfill}{rgb}{0.000000,0.000000,0.000000}%
\pgfsetfillcolor{currentfill}%
\pgfsetlinewidth{0.501875pt}%
\definecolor{currentstroke}{rgb}{0.000000,0.000000,0.000000}%
\pgfsetstrokecolor{currentstroke}%
\pgfsetdash{}{0pt}%
\pgfsys@defobject{currentmarker}{\pgfqpoint{0.000000in}{-0.020833in}}{\pgfqpoint{0.000000in}{0.000000in}}{%
\pgfpathmoveto{\pgfqpoint{0.000000in}{0.000000in}}%
\pgfpathlineto{\pgfqpoint{0.000000in}{-0.020833in}}%
\pgfusepath{stroke,fill}%
}%
\begin{pgfscope}%
\pgfsys@transformshift{2.567965in}{3.574193in}%
\pgfsys@useobject{currentmarker}{}%
\end{pgfscope}%
\end{pgfscope}%
\begin{pgfscope}%
\pgfsetbuttcap%
\pgfsetroundjoin%
\definecolor{currentfill}{rgb}{0.000000,0.000000,0.000000}%
\pgfsetfillcolor{currentfill}%
\pgfsetlinewidth{0.501875pt}%
\definecolor{currentstroke}{rgb}{0.000000,0.000000,0.000000}%
\pgfsetstrokecolor{currentstroke}%
\pgfsetdash{}{0pt}%
\pgfsys@defobject{currentmarker}{\pgfqpoint{0.000000in}{0.000000in}}{\pgfqpoint{0.000000in}{0.020833in}}{%
\pgfpathmoveto{\pgfqpoint{0.000000in}{0.000000in}}%
\pgfpathlineto{\pgfqpoint{0.000000in}{0.020833in}}%
\pgfusepath{stroke,fill}%
}%
\begin{pgfscope}%
\pgfsys@transformshift{2.676773in}{2.347992in}%
\pgfsys@useobject{currentmarker}{}%
\end{pgfscope}%
\end{pgfscope}%
\begin{pgfscope}%
\pgfsetbuttcap%
\pgfsetroundjoin%
\definecolor{currentfill}{rgb}{0.000000,0.000000,0.000000}%
\pgfsetfillcolor{currentfill}%
\pgfsetlinewidth{0.501875pt}%
\definecolor{currentstroke}{rgb}{0.000000,0.000000,0.000000}%
\pgfsetstrokecolor{currentstroke}%
\pgfsetdash{}{0pt}%
\pgfsys@defobject{currentmarker}{\pgfqpoint{0.000000in}{-0.020833in}}{\pgfqpoint{0.000000in}{0.000000in}}{%
\pgfpathmoveto{\pgfqpoint{0.000000in}{0.000000in}}%
\pgfpathlineto{\pgfqpoint{0.000000in}{-0.020833in}}%
\pgfusepath{stroke,fill}%
}%
\begin{pgfscope}%
\pgfsys@transformshift{2.676773in}{3.574193in}%
\pgfsys@useobject{currentmarker}{}%
\end{pgfscope}%
\end{pgfscope}%
\begin{pgfscope}%
\definecolor{textcolor}{rgb}{0.000000,0.000000,0.000000}%
\pgfsetstrokecolor{textcolor}%
\pgfsetfillcolor{textcolor}%
\pgftext[x=1.719260in,y=2.109413in,,top]{\color{textcolor}\rmfamily\fontsize{10.000000}{12.000000}\selectfont \(\displaystyle K\)}%
\end{pgfscope}%
\begin{pgfscope}%
\pgfsetbuttcap%
\pgfsetroundjoin%
\definecolor{currentfill}{rgb}{0.000000,0.000000,0.000000}%
\pgfsetfillcolor{currentfill}%
\pgfsetlinewidth{0.501875pt}%
\definecolor{currentstroke}{rgb}{0.000000,0.000000,0.000000}%
\pgfsetstrokecolor{currentstroke}%
\pgfsetdash{}{0pt}%
\pgfsys@defobject{currentmarker}{\pgfqpoint{0.000000in}{0.000000in}}{\pgfqpoint{0.041667in}{0.000000in}}{%
\pgfpathmoveto{\pgfqpoint{0.000000in}{0.000000in}}%
\pgfpathlineto{\pgfqpoint{0.041667in}{0.000000in}}%
\pgfusepath{stroke,fill}%
}%
\begin{pgfscope}%
\pgfsys@transformshift{0.609415in}{2.491490in}%
\pgfsys@useobject{currentmarker}{}%
\end{pgfscope}%
\end{pgfscope}%
\begin{pgfscope}%
\pgfsetbuttcap%
\pgfsetroundjoin%
\definecolor{currentfill}{rgb}{0.000000,0.000000,0.000000}%
\pgfsetfillcolor{currentfill}%
\pgfsetlinewidth{0.501875pt}%
\definecolor{currentstroke}{rgb}{0.000000,0.000000,0.000000}%
\pgfsetstrokecolor{currentstroke}%
\pgfsetdash{}{0pt}%
\pgfsys@defobject{currentmarker}{\pgfqpoint{-0.041667in}{0.000000in}}{\pgfqpoint{-0.000000in}{0.000000in}}{%
\pgfpathmoveto{\pgfqpoint{-0.000000in}{0.000000in}}%
\pgfpathlineto{\pgfqpoint{-0.041667in}{0.000000in}}%
\pgfusepath{stroke,fill}%
}%
\begin{pgfscope}%
\pgfsys@transformshift{2.829105in}{2.491490in}%
\pgfsys@useobject{currentmarker}{}%
\end{pgfscope}%
\end{pgfscope}%
\begin{pgfscope}%
\definecolor{textcolor}{rgb}{0.000000,0.000000,0.000000}%
\pgfsetstrokecolor{textcolor}%
\pgfsetfillcolor{textcolor}%
\pgftext[x=0.313889in, y=2.438728in, left, base]{\color{textcolor}\rmfamily\fontsize{10.000000}{12.000000}\selectfont \(\displaystyle {0.85}\)}%
\end{pgfscope}%
\begin{pgfscope}%
\pgfsetbuttcap%
\pgfsetroundjoin%
\definecolor{currentfill}{rgb}{0.000000,0.000000,0.000000}%
\pgfsetfillcolor{currentfill}%
\pgfsetlinewidth{0.501875pt}%
\definecolor{currentstroke}{rgb}{0.000000,0.000000,0.000000}%
\pgfsetstrokecolor{currentstroke}%
\pgfsetdash{}{0pt}%
\pgfsys@defobject{currentmarker}{\pgfqpoint{0.000000in}{0.000000in}}{\pgfqpoint{0.041667in}{0.000000in}}{%
\pgfpathmoveto{\pgfqpoint{0.000000in}{0.000000in}}%
\pgfpathlineto{\pgfqpoint{0.041667in}{0.000000in}}%
\pgfusepath{stroke,fill}%
}%
\begin{pgfscope}%
\pgfsys@transformshift{0.609415in}{2.833812in}%
\pgfsys@useobject{currentmarker}{}%
\end{pgfscope}%
\end{pgfscope}%
\begin{pgfscope}%
\pgfsetbuttcap%
\pgfsetroundjoin%
\definecolor{currentfill}{rgb}{0.000000,0.000000,0.000000}%
\pgfsetfillcolor{currentfill}%
\pgfsetlinewidth{0.501875pt}%
\definecolor{currentstroke}{rgb}{0.000000,0.000000,0.000000}%
\pgfsetstrokecolor{currentstroke}%
\pgfsetdash{}{0pt}%
\pgfsys@defobject{currentmarker}{\pgfqpoint{-0.041667in}{0.000000in}}{\pgfqpoint{-0.000000in}{0.000000in}}{%
\pgfpathmoveto{\pgfqpoint{-0.000000in}{0.000000in}}%
\pgfpathlineto{\pgfqpoint{-0.041667in}{0.000000in}}%
\pgfusepath{stroke,fill}%
}%
\begin{pgfscope}%
\pgfsys@transformshift{2.829105in}{2.833812in}%
\pgfsys@useobject{currentmarker}{}%
\end{pgfscope}%
\end{pgfscope}%
\begin{pgfscope}%
\definecolor{textcolor}{rgb}{0.000000,0.000000,0.000000}%
\pgfsetstrokecolor{textcolor}%
\pgfsetfillcolor{textcolor}%
\pgftext[x=0.313889in, y=2.781051in, left, base]{\color{textcolor}\rmfamily\fontsize{10.000000}{12.000000}\selectfont \(\displaystyle {0.90}\)}%
\end{pgfscope}%
\begin{pgfscope}%
\pgfsetbuttcap%
\pgfsetroundjoin%
\definecolor{currentfill}{rgb}{0.000000,0.000000,0.000000}%
\pgfsetfillcolor{currentfill}%
\pgfsetlinewidth{0.501875pt}%
\definecolor{currentstroke}{rgb}{0.000000,0.000000,0.000000}%
\pgfsetstrokecolor{currentstroke}%
\pgfsetdash{}{0pt}%
\pgfsys@defobject{currentmarker}{\pgfqpoint{0.000000in}{0.000000in}}{\pgfqpoint{0.041667in}{0.000000in}}{%
\pgfpathmoveto{\pgfqpoint{0.000000in}{0.000000in}}%
\pgfpathlineto{\pgfqpoint{0.041667in}{0.000000in}}%
\pgfusepath{stroke,fill}%
}%
\begin{pgfscope}%
\pgfsys@transformshift{0.609415in}{3.176134in}%
\pgfsys@useobject{currentmarker}{}%
\end{pgfscope}%
\end{pgfscope}%
\begin{pgfscope}%
\pgfsetbuttcap%
\pgfsetroundjoin%
\definecolor{currentfill}{rgb}{0.000000,0.000000,0.000000}%
\pgfsetfillcolor{currentfill}%
\pgfsetlinewidth{0.501875pt}%
\definecolor{currentstroke}{rgb}{0.000000,0.000000,0.000000}%
\pgfsetstrokecolor{currentstroke}%
\pgfsetdash{}{0pt}%
\pgfsys@defobject{currentmarker}{\pgfqpoint{-0.041667in}{0.000000in}}{\pgfqpoint{-0.000000in}{0.000000in}}{%
\pgfpathmoveto{\pgfqpoint{-0.000000in}{0.000000in}}%
\pgfpathlineto{\pgfqpoint{-0.041667in}{0.000000in}}%
\pgfusepath{stroke,fill}%
}%
\begin{pgfscope}%
\pgfsys@transformshift{2.829105in}{3.176134in}%
\pgfsys@useobject{currentmarker}{}%
\end{pgfscope}%
\end{pgfscope}%
\begin{pgfscope}%
\definecolor{textcolor}{rgb}{0.000000,0.000000,0.000000}%
\pgfsetstrokecolor{textcolor}%
\pgfsetfillcolor{textcolor}%
\pgftext[x=0.313889in, y=3.123373in, left, base]{\color{textcolor}\rmfamily\fontsize{10.000000}{12.000000}\selectfont \(\displaystyle {0.95}\)}%
\end{pgfscope}%
\begin{pgfscope}%
\pgfsetbuttcap%
\pgfsetroundjoin%
\definecolor{currentfill}{rgb}{0.000000,0.000000,0.000000}%
\pgfsetfillcolor{currentfill}%
\pgfsetlinewidth{0.501875pt}%
\definecolor{currentstroke}{rgb}{0.000000,0.000000,0.000000}%
\pgfsetstrokecolor{currentstroke}%
\pgfsetdash{}{0pt}%
\pgfsys@defobject{currentmarker}{\pgfqpoint{0.000000in}{0.000000in}}{\pgfqpoint{0.041667in}{0.000000in}}{%
\pgfpathmoveto{\pgfqpoint{0.000000in}{0.000000in}}%
\pgfpathlineto{\pgfqpoint{0.041667in}{0.000000in}}%
\pgfusepath{stroke,fill}%
}%
\begin{pgfscope}%
\pgfsys@transformshift{0.609415in}{3.518457in}%
\pgfsys@useobject{currentmarker}{}%
\end{pgfscope}%
\end{pgfscope}%
\begin{pgfscope}%
\pgfsetbuttcap%
\pgfsetroundjoin%
\definecolor{currentfill}{rgb}{0.000000,0.000000,0.000000}%
\pgfsetfillcolor{currentfill}%
\pgfsetlinewidth{0.501875pt}%
\definecolor{currentstroke}{rgb}{0.000000,0.000000,0.000000}%
\pgfsetstrokecolor{currentstroke}%
\pgfsetdash{}{0pt}%
\pgfsys@defobject{currentmarker}{\pgfqpoint{-0.041667in}{0.000000in}}{\pgfqpoint{-0.000000in}{0.000000in}}{%
\pgfpathmoveto{\pgfqpoint{-0.000000in}{0.000000in}}%
\pgfpathlineto{\pgfqpoint{-0.041667in}{0.000000in}}%
\pgfusepath{stroke,fill}%
}%
\begin{pgfscope}%
\pgfsys@transformshift{2.829105in}{3.518457in}%
\pgfsys@useobject{currentmarker}{}%
\end{pgfscope}%
\end{pgfscope}%
\begin{pgfscope}%
\definecolor{textcolor}{rgb}{0.000000,0.000000,0.000000}%
\pgfsetstrokecolor{textcolor}%
\pgfsetfillcolor{textcolor}%
\pgftext[x=0.313889in, y=3.465695in, left, base]{\color{textcolor}\rmfamily\fontsize{10.000000}{12.000000}\selectfont \(\displaystyle {1.00}\)}%
\end{pgfscope}%
\begin{pgfscope}%
\pgfsetbuttcap%
\pgfsetroundjoin%
\definecolor{currentfill}{rgb}{0.000000,0.000000,0.000000}%
\pgfsetfillcolor{currentfill}%
\pgfsetlinewidth{0.501875pt}%
\definecolor{currentstroke}{rgb}{0.000000,0.000000,0.000000}%
\pgfsetstrokecolor{currentstroke}%
\pgfsetdash{}{0pt}%
\pgfsys@defobject{currentmarker}{\pgfqpoint{0.000000in}{0.000000in}}{\pgfqpoint{0.020833in}{0.000000in}}{%
\pgfpathmoveto{\pgfqpoint{0.000000in}{0.000000in}}%
\pgfpathlineto{\pgfqpoint{0.020833in}{0.000000in}}%
\pgfusepath{stroke,fill}%
}%
\begin{pgfscope}%
\pgfsys@transformshift{0.609415in}{2.354561in}%
\pgfsys@useobject{currentmarker}{}%
\end{pgfscope}%
\end{pgfscope}%
\begin{pgfscope}%
\pgfsetbuttcap%
\pgfsetroundjoin%
\definecolor{currentfill}{rgb}{0.000000,0.000000,0.000000}%
\pgfsetfillcolor{currentfill}%
\pgfsetlinewidth{0.501875pt}%
\definecolor{currentstroke}{rgb}{0.000000,0.000000,0.000000}%
\pgfsetstrokecolor{currentstroke}%
\pgfsetdash{}{0pt}%
\pgfsys@defobject{currentmarker}{\pgfqpoint{-0.020833in}{0.000000in}}{\pgfqpoint{-0.000000in}{0.000000in}}{%
\pgfpathmoveto{\pgfqpoint{-0.000000in}{0.000000in}}%
\pgfpathlineto{\pgfqpoint{-0.020833in}{0.000000in}}%
\pgfusepath{stroke,fill}%
}%
\begin{pgfscope}%
\pgfsys@transformshift{2.829105in}{2.354561in}%
\pgfsys@useobject{currentmarker}{}%
\end{pgfscope}%
\end{pgfscope}%
\begin{pgfscope}%
\pgfsetbuttcap%
\pgfsetroundjoin%
\definecolor{currentfill}{rgb}{0.000000,0.000000,0.000000}%
\pgfsetfillcolor{currentfill}%
\pgfsetlinewidth{0.501875pt}%
\definecolor{currentstroke}{rgb}{0.000000,0.000000,0.000000}%
\pgfsetstrokecolor{currentstroke}%
\pgfsetdash{}{0pt}%
\pgfsys@defobject{currentmarker}{\pgfqpoint{0.000000in}{0.000000in}}{\pgfqpoint{0.020833in}{0.000000in}}{%
\pgfpathmoveto{\pgfqpoint{0.000000in}{0.000000in}}%
\pgfpathlineto{\pgfqpoint{0.020833in}{0.000000in}}%
\pgfusepath{stroke,fill}%
}%
\begin{pgfscope}%
\pgfsys@transformshift{0.609415in}{2.423025in}%
\pgfsys@useobject{currentmarker}{}%
\end{pgfscope}%
\end{pgfscope}%
\begin{pgfscope}%
\pgfsetbuttcap%
\pgfsetroundjoin%
\definecolor{currentfill}{rgb}{0.000000,0.000000,0.000000}%
\pgfsetfillcolor{currentfill}%
\pgfsetlinewidth{0.501875pt}%
\definecolor{currentstroke}{rgb}{0.000000,0.000000,0.000000}%
\pgfsetstrokecolor{currentstroke}%
\pgfsetdash{}{0pt}%
\pgfsys@defobject{currentmarker}{\pgfqpoint{-0.020833in}{0.000000in}}{\pgfqpoint{-0.000000in}{0.000000in}}{%
\pgfpathmoveto{\pgfqpoint{-0.000000in}{0.000000in}}%
\pgfpathlineto{\pgfqpoint{-0.020833in}{0.000000in}}%
\pgfusepath{stroke,fill}%
}%
\begin{pgfscope}%
\pgfsys@transformshift{2.829105in}{2.423025in}%
\pgfsys@useobject{currentmarker}{}%
\end{pgfscope}%
\end{pgfscope}%
\begin{pgfscope}%
\pgfsetbuttcap%
\pgfsetroundjoin%
\definecolor{currentfill}{rgb}{0.000000,0.000000,0.000000}%
\pgfsetfillcolor{currentfill}%
\pgfsetlinewidth{0.501875pt}%
\definecolor{currentstroke}{rgb}{0.000000,0.000000,0.000000}%
\pgfsetstrokecolor{currentstroke}%
\pgfsetdash{}{0pt}%
\pgfsys@defobject{currentmarker}{\pgfqpoint{0.000000in}{0.000000in}}{\pgfqpoint{0.020833in}{0.000000in}}{%
\pgfpathmoveto{\pgfqpoint{0.000000in}{0.000000in}}%
\pgfpathlineto{\pgfqpoint{0.020833in}{0.000000in}}%
\pgfusepath{stroke,fill}%
}%
\begin{pgfscope}%
\pgfsys@transformshift{0.609415in}{2.559954in}%
\pgfsys@useobject{currentmarker}{}%
\end{pgfscope}%
\end{pgfscope}%
\begin{pgfscope}%
\pgfsetbuttcap%
\pgfsetroundjoin%
\definecolor{currentfill}{rgb}{0.000000,0.000000,0.000000}%
\pgfsetfillcolor{currentfill}%
\pgfsetlinewidth{0.501875pt}%
\definecolor{currentstroke}{rgb}{0.000000,0.000000,0.000000}%
\pgfsetstrokecolor{currentstroke}%
\pgfsetdash{}{0pt}%
\pgfsys@defobject{currentmarker}{\pgfqpoint{-0.020833in}{0.000000in}}{\pgfqpoint{-0.000000in}{0.000000in}}{%
\pgfpathmoveto{\pgfqpoint{-0.000000in}{0.000000in}}%
\pgfpathlineto{\pgfqpoint{-0.020833in}{0.000000in}}%
\pgfusepath{stroke,fill}%
}%
\begin{pgfscope}%
\pgfsys@transformshift{2.829105in}{2.559954in}%
\pgfsys@useobject{currentmarker}{}%
\end{pgfscope}%
\end{pgfscope}%
\begin{pgfscope}%
\pgfsetbuttcap%
\pgfsetroundjoin%
\definecolor{currentfill}{rgb}{0.000000,0.000000,0.000000}%
\pgfsetfillcolor{currentfill}%
\pgfsetlinewidth{0.501875pt}%
\definecolor{currentstroke}{rgb}{0.000000,0.000000,0.000000}%
\pgfsetstrokecolor{currentstroke}%
\pgfsetdash{}{0pt}%
\pgfsys@defobject{currentmarker}{\pgfqpoint{0.000000in}{0.000000in}}{\pgfqpoint{0.020833in}{0.000000in}}{%
\pgfpathmoveto{\pgfqpoint{0.000000in}{0.000000in}}%
\pgfpathlineto{\pgfqpoint{0.020833in}{0.000000in}}%
\pgfusepath{stroke,fill}%
}%
\begin{pgfscope}%
\pgfsys@transformshift{0.609415in}{2.628419in}%
\pgfsys@useobject{currentmarker}{}%
\end{pgfscope}%
\end{pgfscope}%
\begin{pgfscope}%
\pgfsetbuttcap%
\pgfsetroundjoin%
\definecolor{currentfill}{rgb}{0.000000,0.000000,0.000000}%
\pgfsetfillcolor{currentfill}%
\pgfsetlinewidth{0.501875pt}%
\definecolor{currentstroke}{rgb}{0.000000,0.000000,0.000000}%
\pgfsetstrokecolor{currentstroke}%
\pgfsetdash{}{0pt}%
\pgfsys@defobject{currentmarker}{\pgfqpoint{-0.020833in}{0.000000in}}{\pgfqpoint{-0.000000in}{0.000000in}}{%
\pgfpathmoveto{\pgfqpoint{-0.000000in}{0.000000in}}%
\pgfpathlineto{\pgfqpoint{-0.020833in}{0.000000in}}%
\pgfusepath{stroke,fill}%
}%
\begin{pgfscope}%
\pgfsys@transformshift{2.829105in}{2.628419in}%
\pgfsys@useobject{currentmarker}{}%
\end{pgfscope}%
\end{pgfscope}%
\begin{pgfscope}%
\pgfsetbuttcap%
\pgfsetroundjoin%
\definecolor{currentfill}{rgb}{0.000000,0.000000,0.000000}%
\pgfsetfillcolor{currentfill}%
\pgfsetlinewidth{0.501875pt}%
\definecolor{currentstroke}{rgb}{0.000000,0.000000,0.000000}%
\pgfsetstrokecolor{currentstroke}%
\pgfsetdash{}{0pt}%
\pgfsys@defobject{currentmarker}{\pgfqpoint{0.000000in}{0.000000in}}{\pgfqpoint{0.020833in}{0.000000in}}{%
\pgfpathmoveto{\pgfqpoint{0.000000in}{0.000000in}}%
\pgfpathlineto{\pgfqpoint{0.020833in}{0.000000in}}%
\pgfusepath{stroke,fill}%
}%
\begin{pgfscope}%
\pgfsys@transformshift{0.609415in}{2.696883in}%
\pgfsys@useobject{currentmarker}{}%
\end{pgfscope}%
\end{pgfscope}%
\begin{pgfscope}%
\pgfsetbuttcap%
\pgfsetroundjoin%
\definecolor{currentfill}{rgb}{0.000000,0.000000,0.000000}%
\pgfsetfillcolor{currentfill}%
\pgfsetlinewidth{0.501875pt}%
\definecolor{currentstroke}{rgb}{0.000000,0.000000,0.000000}%
\pgfsetstrokecolor{currentstroke}%
\pgfsetdash{}{0pt}%
\pgfsys@defobject{currentmarker}{\pgfqpoint{-0.020833in}{0.000000in}}{\pgfqpoint{-0.000000in}{0.000000in}}{%
\pgfpathmoveto{\pgfqpoint{-0.000000in}{0.000000in}}%
\pgfpathlineto{\pgfqpoint{-0.020833in}{0.000000in}}%
\pgfusepath{stroke,fill}%
}%
\begin{pgfscope}%
\pgfsys@transformshift{2.829105in}{2.696883in}%
\pgfsys@useobject{currentmarker}{}%
\end{pgfscope}%
\end{pgfscope}%
\begin{pgfscope}%
\pgfsetbuttcap%
\pgfsetroundjoin%
\definecolor{currentfill}{rgb}{0.000000,0.000000,0.000000}%
\pgfsetfillcolor{currentfill}%
\pgfsetlinewidth{0.501875pt}%
\definecolor{currentstroke}{rgb}{0.000000,0.000000,0.000000}%
\pgfsetstrokecolor{currentstroke}%
\pgfsetdash{}{0pt}%
\pgfsys@defobject{currentmarker}{\pgfqpoint{0.000000in}{0.000000in}}{\pgfqpoint{0.020833in}{0.000000in}}{%
\pgfpathmoveto{\pgfqpoint{0.000000in}{0.000000in}}%
\pgfpathlineto{\pgfqpoint{0.020833in}{0.000000in}}%
\pgfusepath{stroke,fill}%
}%
\begin{pgfscope}%
\pgfsys@transformshift{0.609415in}{2.765348in}%
\pgfsys@useobject{currentmarker}{}%
\end{pgfscope}%
\end{pgfscope}%
\begin{pgfscope}%
\pgfsetbuttcap%
\pgfsetroundjoin%
\definecolor{currentfill}{rgb}{0.000000,0.000000,0.000000}%
\pgfsetfillcolor{currentfill}%
\pgfsetlinewidth{0.501875pt}%
\definecolor{currentstroke}{rgb}{0.000000,0.000000,0.000000}%
\pgfsetstrokecolor{currentstroke}%
\pgfsetdash{}{0pt}%
\pgfsys@defobject{currentmarker}{\pgfqpoint{-0.020833in}{0.000000in}}{\pgfqpoint{-0.000000in}{0.000000in}}{%
\pgfpathmoveto{\pgfqpoint{-0.000000in}{0.000000in}}%
\pgfpathlineto{\pgfqpoint{-0.020833in}{0.000000in}}%
\pgfusepath{stroke,fill}%
}%
\begin{pgfscope}%
\pgfsys@transformshift{2.829105in}{2.765348in}%
\pgfsys@useobject{currentmarker}{}%
\end{pgfscope}%
\end{pgfscope}%
\begin{pgfscope}%
\pgfsetbuttcap%
\pgfsetroundjoin%
\definecolor{currentfill}{rgb}{0.000000,0.000000,0.000000}%
\pgfsetfillcolor{currentfill}%
\pgfsetlinewidth{0.501875pt}%
\definecolor{currentstroke}{rgb}{0.000000,0.000000,0.000000}%
\pgfsetstrokecolor{currentstroke}%
\pgfsetdash{}{0pt}%
\pgfsys@defobject{currentmarker}{\pgfqpoint{0.000000in}{0.000000in}}{\pgfqpoint{0.020833in}{0.000000in}}{%
\pgfpathmoveto{\pgfqpoint{0.000000in}{0.000000in}}%
\pgfpathlineto{\pgfqpoint{0.020833in}{0.000000in}}%
\pgfusepath{stroke,fill}%
}%
\begin{pgfscope}%
\pgfsys@transformshift{0.609415in}{2.902277in}%
\pgfsys@useobject{currentmarker}{}%
\end{pgfscope}%
\end{pgfscope}%
\begin{pgfscope}%
\pgfsetbuttcap%
\pgfsetroundjoin%
\definecolor{currentfill}{rgb}{0.000000,0.000000,0.000000}%
\pgfsetfillcolor{currentfill}%
\pgfsetlinewidth{0.501875pt}%
\definecolor{currentstroke}{rgb}{0.000000,0.000000,0.000000}%
\pgfsetstrokecolor{currentstroke}%
\pgfsetdash{}{0pt}%
\pgfsys@defobject{currentmarker}{\pgfqpoint{-0.020833in}{0.000000in}}{\pgfqpoint{-0.000000in}{0.000000in}}{%
\pgfpathmoveto{\pgfqpoint{-0.000000in}{0.000000in}}%
\pgfpathlineto{\pgfqpoint{-0.020833in}{0.000000in}}%
\pgfusepath{stroke,fill}%
}%
\begin{pgfscope}%
\pgfsys@transformshift{2.829105in}{2.902277in}%
\pgfsys@useobject{currentmarker}{}%
\end{pgfscope}%
\end{pgfscope}%
\begin{pgfscope}%
\pgfsetbuttcap%
\pgfsetroundjoin%
\definecolor{currentfill}{rgb}{0.000000,0.000000,0.000000}%
\pgfsetfillcolor{currentfill}%
\pgfsetlinewidth{0.501875pt}%
\definecolor{currentstroke}{rgb}{0.000000,0.000000,0.000000}%
\pgfsetstrokecolor{currentstroke}%
\pgfsetdash{}{0pt}%
\pgfsys@defobject{currentmarker}{\pgfqpoint{0.000000in}{0.000000in}}{\pgfqpoint{0.020833in}{0.000000in}}{%
\pgfpathmoveto{\pgfqpoint{0.000000in}{0.000000in}}%
\pgfpathlineto{\pgfqpoint{0.020833in}{0.000000in}}%
\pgfusepath{stroke,fill}%
}%
\begin{pgfscope}%
\pgfsys@transformshift{0.609415in}{2.970741in}%
\pgfsys@useobject{currentmarker}{}%
\end{pgfscope}%
\end{pgfscope}%
\begin{pgfscope}%
\pgfsetbuttcap%
\pgfsetroundjoin%
\definecolor{currentfill}{rgb}{0.000000,0.000000,0.000000}%
\pgfsetfillcolor{currentfill}%
\pgfsetlinewidth{0.501875pt}%
\definecolor{currentstroke}{rgb}{0.000000,0.000000,0.000000}%
\pgfsetstrokecolor{currentstroke}%
\pgfsetdash{}{0pt}%
\pgfsys@defobject{currentmarker}{\pgfqpoint{-0.020833in}{0.000000in}}{\pgfqpoint{-0.000000in}{0.000000in}}{%
\pgfpathmoveto{\pgfqpoint{-0.000000in}{0.000000in}}%
\pgfpathlineto{\pgfqpoint{-0.020833in}{0.000000in}}%
\pgfusepath{stroke,fill}%
}%
\begin{pgfscope}%
\pgfsys@transformshift{2.829105in}{2.970741in}%
\pgfsys@useobject{currentmarker}{}%
\end{pgfscope}%
\end{pgfscope}%
\begin{pgfscope}%
\pgfsetbuttcap%
\pgfsetroundjoin%
\definecolor{currentfill}{rgb}{0.000000,0.000000,0.000000}%
\pgfsetfillcolor{currentfill}%
\pgfsetlinewidth{0.501875pt}%
\definecolor{currentstroke}{rgb}{0.000000,0.000000,0.000000}%
\pgfsetstrokecolor{currentstroke}%
\pgfsetdash{}{0pt}%
\pgfsys@defobject{currentmarker}{\pgfqpoint{0.000000in}{0.000000in}}{\pgfqpoint{0.020833in}{0.000000in}}{%
\pgfpathmoveto{\pgfqpoint{0.000000in}{0.000000in}}%
\pgfpathlineto{\pgfqpoint{0.020833in}{0.000000in}}%
\pgfusepath{stroke,fill}%
}%
\begin{pgfscope}%
\pgfsys@transformshift{0.609415in}{3.039205in}%
\pgfsys@useobject{currentmarker}{}%
\end{pgfscope}%
\end{pgfscope}%
\begin{pgfscope}%
\pgfsetbuttcap%
\pgfsetroundjoin%
\definecolor{currentfill}{rgb}{0.000000,0.000000,0.000000}%
\pgfsetfillcolor{currentfill}%
\pgfsetlinewidth{0.501875pt}%
\definecolor{currentstroke}{rgb}{0.000000,0.000000,0.000000}%
\pgfsetstrokecolor{currentstroke}%
\pgfsetdash{}{0pt}%
\pgfsys@defobject{currentmarker}{\pgfqpoint{-0.020833in}{0.000000in}}{\pgfqpoint{-0.000000in}{0.000000in}}{%
\pgfpathmoveto{\pgfqpoint{-0.000000in}{0.000000in}}%
\pgfpathlineto{\pgfqpoint{-0.020833in}{0.000000in}}%
\pgfusepath{stroke,fill}%
}%
\begin{pgfscope}%
\pgfsys@transformshift{2.829105in}{3.039205in}%
\pgfsys@useobject{currentmarker}{}%
\end{pgfscope}%
\end{pgfscope}%
\begin{pgfscope}%
\pgfsetbuttcap%
\pgfsetroundjoin%
\definecolor{currentfill}{rgb}{0.000000,0.000000,0.000000}%
\pgfsetfillcolor{currentfill}%
\pgfsetlinewidth{0.501875pt}%
\definecolor{currentstroke}{rgb}{0.000000,0.000000,0.000000}%
\pgfsetstrokecolor{currentstroke}%
\pgfsetdash{}{0pt}%
\pgfsys@defobject{currentmarker}{\pgfqpoint{0.000000in}{0.000000in}}{\pgfqpoint{0.020833in}{0.000000in}}{%
\pgfpathmoveto{\pgfqpoint{0.000000in}{0.000000in}}%
\pgfpathlineto{\pgfqpoint{0.020833in}{0.000000in}}%
\pgfusepath{stroke,fill}%
}%
\begin{pgfscope}%
\pgfsys@transformshift{0.609415in}{3.107670in}%
\pgfsys@useobject{currentmarker}{}%
\end{pgfscope}%
\end{pgfscope}%
\begin{pgfscope}%
\pgfsetbuttcap%
\pgfsetroundjoin%
\definecolor{currentfill}{rgb}{0.000000,0.000000,0.000000}%
\pgfsetfillcolor{currentfill}%
\pgfsetlinewidth{0.501875pt}%
\definecolor{currentstroke}{rgb}{0.000000,0.000000,0.000000}%
\pgfsetstrokecolor{currentstroke}%
\pgfsetdash{}{0pt}%
\pgfsys@defobject{currentmarker}{\pgfqpoint{-0.020833in}{0.000000in}}{\pgfqpoint{-0.000000in}{0.000000in}}{%
\pgfpathmoveto{\pgfqpoint{-0.000000in}{0.000000in}}%
\pgfpathlineto{\pgfqpoint{-0.020833in}{0.000000in}}%
\pgfusepath{stroke,fill}%
}%
\begin{pgfscope}%
\pgfsys@transformshift{2.829105in}{3.107670in}%
\pgfsys@useobject{currentmarker}{}%
\end{pgfscope}%
\end{pgfscope}%
\begin{pgfscope}%
\pgfsetbuttcap%
\pgfsetroundjoin%
\definecolor{currentfill}{rgb}{0.000000,0.000000,0.000000}%
\pgfsetfillcolor{currentfill}%
\pgfsetlinewidth{0.501875pt}%
\definecolor{currentstroke}{rgb}{0.000000,0.000000,0.000000}%
\pgfsetstrokecolor{currentstroke}%
\pgfsetdash{}{0pt}%
\pgfsys@defobject{currentmarker}{\pgfqpoint{0.000000in}{0.000000in}}{\pgfqpoint{0.020833in}{0.000000in}}{%
\pgfpathmoveto{\pgfqpoint{0.000000in}{0.000000in}}%
\pgfpathlineto{\pgfqpoint{0.020833in}{0.000000in}}%
\pgfusepath{stroke,fill}%
}%
\begin{pgfscope}%
\pgfsys@transformshift{0.609415in}{3.244599in}%
\pgfsys@useobject{currentmarker}{}%
\end{pgfscope}%
\end{pgfscope}%
\begin{pgfscope}%
\pgfsetbuttcap%
\pgfsetroundjoin%
\definecolor{currentfill}{rgb}{0.000000,0.000000,0.000000}%
\pgfsetfillcolor{currentfill}%
\pgfsetlinewidth{0.501875pt}%
\definecolor{currentstroke}{rgb}{0.000000,0.000000,0.000000}%
\pgfsetstrokecolor{currentstroke}%
\pgfsetdash{}{0pt}%
\pgfsys@defobject{currentmarker}{\pgfqpoint{-0.020833in}{0.000000in}}{\pgfqpoint{-0.000000in}{0.000000in}}{%
\pgfpathmoveto{\pgfqpoint{-0.000000in}{0.000000in}}%
\pgfpathlineto{\pgfqpoint{-0.020833in}{0.000000in}}%
\pgfusepath{stroke,fill}%
}%
\begin{pgfscope}%
\pgfsys@transformshift{2.829105in}{3.244599in}%
\pgfsys@useobject{currentmarker}{}%
\end{pgfscope}%
\end{pgfscope}%
\begin{pgfscope}%
\pgfsetbuttcap%
\pgfsetroundjoin%
\definecolor{currentfill}{rgb}{0.000000,0.000000,0.000000}%
\pgfsetfillcolor{currentfill}%
\pgfsetlinewidth{0.501875pt}%
\definecolor{currentstroke}{rgb}{0.000000,0.000000,0.000000}%
\pgfsetstrokecolor{currentstroke}%
\pgfsetdash{}{0pt}%
\pgfsys@defobject{currentmarker}{\pgfqpoint{0.000000in}{0.000000in}}{\pgfqpoint{0.020833in}{0.000000in}}{%
\pgfpathmoveto{\pgfqpoint{0.000000in}{0.000000in}}%
\pgfpathlineto{\pgfqpoint{0.020833in}{0.000000in}}%
\pgfusepath{stroke,fill}%
}%
\begin{pgfscope}%
\pgfsys@transformshift{0.609415in}{3.313063in}%
\pgfsys@useobject{currentmarker}{}%
\end{pgfscope}%
\end{pgfscope}%
\begin{pgfscope}%
\pgfsetbuttcap%
\pgfsetroundjoin%
\definecolor{currentfill}{rgb}{0.000000,0.000000,0.000000}%
\pgfsetfillcolor{currentfill}%
\pgfsetlinewidth{0.501875pt}%
\definecolor{currentstroke}{rgb}{0.000000,0.000000,0.000000}%
\pgfsetstrokecolor{currentstroke}%
\pgfsetdash{}{0pt}%
\pgfsys@defobject{currentmarker}{\pgfqpoint{-0.020833in}{0.000000in}}{\pgfqpoint{-0.000000in}{0.000000in}}{%
\pgfpathmoveto{\pgfqpoint{-0.000000in}{0.000000in}}%
\pgfpathlineto{\pgfqpoint{-0.020833in}{0.000000in}}%
\pgfusepath{stroke,fill}%
}%
\begin{pgfscope}%
\pgfsys@transformshift{2.829105in}{3.313063in}%
\pgfsys@useobject{currentmarker}{}%
\end{pgfscope}%
\end{pgfscope}%
\begin{pgfscope}%
\pgfsetbuttcap%
\pgfsetroundjoin%
\definecolor{currentfill}{rgb}{0.000000,0.000000,0.000000}%
\pgfsetfillcolor{currentfill}%
\pgfsetlinewidth{0.501875pt}%
\definecolor{currentstroke}{rgb}{0.000000,0.000000,0.000000}%
\pgfsetstrokecolor{currentstroke}%
\pgfsetdash{}{0pt}%
\pgfsys@defobject{currentmarker}{\pgfqpoint{0.000000in}{0.000000in}}{\pgfqpoint{0.020833in}{0.000000in}}{%
\pgfpathmoveto{\pgfqpoint{0.000000in}{0.000000in}}%
\pgfpathlineto{\pgfqpoint{0.020833in}{0.000000in}}%
\pgfusepath{stroke,fill}%
}%
\begin{pgfscope}%
\pgfsys@transformshift{0.609415in}{3.381528in}%
\pgfsys@useobject{currentmarker}{}%
\end{pgfscope}%
\end{pgfscope}%
\begin{pgfscope}%
\pgfsetbuttcap%
\pgfsetroundjoin%
\definecolor{currentfill}{rgb}{0.000000,0.000000,0.000000}%
\pgfsetfillcolor{currentfill}%
\pgfsetlinewidth{0.501875pt}%
\definecolor{currentstroke}{rgb}{0.000000,0.000000,0.000000}%
\pgfsetstrokecolor{currentstroke}%
\pgfsetdash{}{0pt}%
\pgfsys@defobject{currentmarker}{\pgfqpoint{-0.020833in}{0.000000in}}{\pgfqpoint{-0.000000in}{0.000000in}}{%
\pgfpathmoveto{\pgfqpoint{-0.000000in}{0.000000in}}%
\pgfpathlineto{\pgfqpoint{-0.020833in}{0.000000in}}%
\pgfusepath{stroke,fill}%
}%
\begin{pgfscope}%
\pgfsys@transformshift{2.829105in}{3.381528in}%
\pgfsys@useobject{currentmarker}{}%
\end{pgfscope}%
\end{pgfscope}%
\begin{pgfscope}%
\pgfsetbuttcap%
\pgfsetroundjoin%
\definecolor{currentfill}{rgb}{0.000000,0.000000,0.000000}%
\pgfsetfillcolor{currentfill}%
\pgfsetlinewidth{0.501875pt}%
\definecolor{currentstroke}{rgb}{0.000000,0.000000,0.000000}%
\pgfsetstrokecolor{currentstroke}%
\pgfsetdash{}{0pt}%
\pgfsys@defobject{currentmarker}{\pgfqpoint{0.000000in}{0.000000in}}{\pgfqpoint{0.020833in}{0.000000in}}{%
\pgfpathmoveto{\pgfqpoint{0.000000in}{0.000000in}}%
\pgfpathlineto{\pgfqpoint{0.020833in}{0.000000in}}%
\pgfusepath{stroke,fill}%
}%
\begin{pgfscope}%
\pgfsys@transformshift{0.609415in}{3.449992in}%
\pgfsys@useobject{currentmarker}{}%
\end{pgfscope}%
\end{pgfscope}%
\begin{pgfscope}%
\pgfsetbuttcap%
\pgfsetroundjoin%
\definecolor{currentfill}{rgb}{0.000000,0.000000,0.000000}%
\pgfsetfillcolor{currentfill}%
\pgfsetlinewidth{0.501875pt}%
\definecolor{currentstroke}{rgb}{0.000000,0.000000,0.000000}%
\pgfsetstrokecolor{currentstroke}%
\pgfsetdash{}{0pt}%
\pgfsys@defobject{currentmarker}{\pgfqpoint{-0.020833in}{0.000000in}}{\pgfqpoint{-0.000000in}{0.000000in}}{%
\pgfpathmoveto{\pgfqpoint{-0.000000in}{0.000000in}}%
\pgfpathlineto{\pgfqpoint{-0.020833in}{0.000000in}}%
\pgfusepath{stroke,fill}%
}%
\begin{pgfscope}%
\pgfsys@transformshift{2.829105in}{3.449992in}%
\pgfsys@useobject{currentmarker}{}%
\end{pgfscope}%
\end{pgfscope}%
\begin{pgfscope}%
\definecolor{textcolor}{rgb}{0.000000,0.000000,0.000000}%
\pgfsetstrokecolor{textcolor}%
\pgfsetfillcolor{textcolor}%
\pgftext[x=0.258334in,y=2.961093in,,bottom,rotate=90.000000]{\color{textcolor}\rmfamily\fontsize{10.000000}{12.000000}\selectfont \(\displaystyle T(K)\)}%
\end{pgfscope}%
\begin{pgfscope}%
\pgfpathrectangle{\pgfqpoint{0.609415in}{2.347992in}}{\pgfqpoint{2.219690in}{1.226201in}}%
\pgfusepath{clip}%
\pgfsetrectcap%
\pgfsetroundjoin%
\pgfsetlinewidth{1.003750pt}%
\definecolor{currentstroke}{rgb}{0.047059,0.364706,0.647059}%
\pgfsetstrokecolor{currentstroke}%
\pgfsetdash{}{0pt}%
\pgfpathmoveto{\pgfqpoint{0.631176in}{3.518457in}}%
\pgfpathlineto{\pgfqpoint{0.652938in}{3.480250in}}%
\pgfpathlineto{\pgfqpoint{0.674700in}{3.450527in}}%
\pgfpathlineto{\pgfqpoint{0.696461in}{3.433408in}}%
\pgfpathlineto{\pgfqpoint{0.718223in}{3.419978in}}%
\pgfpathlineto{\pgfqpoint{0.739985in}{3.409069in}}%
\pgfpathlineto{\pgfqpoint{0.761746in}{3.399335in}}%
\pgfpathlineto{\pgfqpoint{0.783508in}{3.392468in}}%
\pgfpathlineto{\pgfqpoint{0.805270in}{3.384788in}}%
\pgfpathlineto{\pgfqpoint{0.827031in}{3.378307in}}%
\pgfpathlineto{\pgfqpoint{0.848793in}{3.372815in}}%
\pgfpathlineto{\pgfqpoint{0.870555in}{3.368210in}}%
\pgfpathlineto{\pgfqpoint{0.892316in}{3.363319in}}%
\pgfpathlineto{\pgfqpoint{0.914078in}{3.358883in}}%
\pgfpathlineto{\pgfqpoint{0.935840in}{3.354484in}}%
\pgfpathlineto{\pgfqpoint{0.957601in}{3.350777in}}%
\pgfpathlineto{\pgfqpoint{0.979363in}{3.347846in}}%
\pgfpathlineto{\pgfqpoint{1.001125in}{3.345126in}}%
\pgfpathlineto{\pgfqpoint{1.022886in}{3.341924in}}%
\pgfpathlineto{\pgfqpoint{1.044648in}{3.338790in}}%
\pgfpathlineto{\pgfqpoint{1.066410in}{3.335588in}}%
\pgfpathlineto{\pgfqpoint{1.088171in}{3.332689in}}%
\pgfpathlineto{\pgfqpoint{1.109933in}{3.330315in}}%
\pgfpathlineto{\pgfqpoint{1.131695in}{3.327770in}}%
\pgfpathlineto{\pgfqpoint{1.153456in}{3.325522in}}%
\pgfpathlineto{\pgfqpoint{1.175218in}{3.323393in}}%
\pgfpathlineto{\pgfqpoint{1.196980in}{3.320993in}}%
\pgfpathlineto{\pgfqpoint{1.218741in}{3.319250in}}%
\pgfpathlineto{\pgfqpoint{1.240503in}{3.317354in}}%
\pgfpathlineto{\pgfqpoint{1.262265in}{3.315206in}}%
\pgfpathlineto{\pgfqpoint{1.284026in}{3.313383in}}%
\pgfpathlineto{\pgfqpoint{1.305788in}{3.311571in}}%
\pgfpathlineto{\pgfqpoint{1.327550in}{3.309630in}}%
\pgfpathlineto{\pgfqpoint{1.349311in}{3.307802in}}%
\pgfpathlineto{\pgfqpoint{1.371073in}{3.306020in}}%
\pgfpathlineto{\pgfqpoint{1.392835in}{3.303736in}}%
\pgfpathlineto{\pgfqpoint{1.414596in}{3.301790in}}%
\pgfpathlineto{\pgfqpoint{1.436358in}{3.300141in}}%
\pgfpathlineto{\pgfqpoint{1.458120in}{3.299073in}}%
\pgfpathlineto{\pgfqpoint{1.479881in}{3.297468in}}%
\pgfpathlineto{\pgfqpoint{1.501643in}{3.295850in}}%
\pgfpathlineto{\pgfqpoint{1.523405in}{3.294188in}}%
\pgfpathlineto{\pgfqpoint{1.545166in}{3.292955in}}%
\pgfpathlineto{\pgfqpoint{1.566928in}{3.291722in}}%
\pgfpathlineto{\pgfqpoint{1.588690in}{3.290324in}}%
\pgfpathlineto{\pgfqpoint{1.610451in}{3.288942in}}%
\pgfpathlineto{\pgfqpoint{1.632213in}{3.287695in}}%
\pgfpathlineto{\pgfqpoint{1.653975in}{3.286469in}}%
\pgfpathlineto{\pgfqpoint{1.675736in}{3.285307in}}%
\pgfpathlineto{\pgfqpoint{1.697498in}{3.284197in}}%
\pgfpathlineto{\pgfqpoint{1.719260in}{3.283066in}}%
\pgfpathlineto{\pgfqpoint{1.741021in}{3.281806in}}%
\pgfpathlineto{\pgfqpoint{1.762783in}{3.280557in}}%
\pgfpathlineto{\pgfqpoint{1.784545in}{3.279558in}}%
\pgfpathlineto{\pgfqpoint{1.806306in}{3.278225in}}%
\pgfpathlineto{\pgfqpoint{1.828068in}{3.277054in}}%
\pgfpathlineto{\pgfqpoint{1.849830in}{3.275898in}}%
\pgfpathlineto{\pgfqpoint{1.871591in}{3.275049in}}%
\pgfpathlineto{\pgfqpoint{1.893353in}{3.273994in}}%
\pgfpathlineto{\pgfqpoint{1.915115in}{3.273029in}}%
\pgfpathlineto{\pgfqpoint{1.936876in}{3.272076in}}%
\pgfpathlineto{\pgfqpoint{1.958638in}{3.270937in}}%
\pgfpathlineto{\pgfqpoint{1.980400in}{3.270038in}}%
\pgfpathlineto{\pgfqpoint{2.002161in}{3.269030in}}%
\pgfpathlineto{\pgfqpoint{2.023923in}{3.268093in}}%
\pgfpathlineto{\pgfqpoint{2.045685in}{3.267010in}}%
\pgfpathlineto{\pgfqpoint{2.067446in}{3.266017in}}%
\pgfpathlineto{\pgfqpoint{2.089208in}{3.265038in}}%
\pgfpathlineto{\pgfqpoint{2.110970in}{3.264067in}}%
\pgfpathlineto{\pgfqpoint{2.132731in}{3.262966in}}%
\pgfpathlineto{\pgfqpoint{2.154493in}{3.261919in}}%
\pgfpathlineto{\pgfqpoint{2.176255in}{3.261065in}}%
\pgfpathlineto{\pgfqpoint{2.198016in}{3.260188in}}%
\pgfpathlineto{\pgfqpoint{2.219778in}{3.259252in}}%
\pgfpathlineto{\pgfqpoint{2.241540in}{3.258297in}}%
\pgfpathlineto{\pgfqpoint{2.263301in}{3.257523in}}%
\pgfpathlineto{\pgfqpoint{2.285063in}{3.256729in}}%
\pgfpathlineto{\pgfqpoint{2.306825in}{3.255800in}}%
\pgfpathlineto{\pgfqpoint{2.328586in}{3.254910in}}%
\pgfpathlineto{\pgfqpoint{2.350348in}{3.253955in}}%
\pgfpathlineto{\pgfqpoint{2.372110in}{3.253145in}}%
\pgfpathlineto{\pgfqpoint{2.393871in}{3.252220in}}%
\pgfpathlineto{\pgfqpoint{2.415633in}{3.251487in}}%
\pgfpathlineto{\pgfqpoint{2.437395in}{3.250557in}}%
\pgfpathlineto{\pgfqpoint{2.459156in}{3.249679in}}%
\pgfpathlineto{\pgfqpoint{2.480918in}{3.248979in}}%
\pgfpathlineto{\pgfqpoint{2.502680in}{3.248155in}}%
\pgfpathlineto{\pgfqpoint{2.524441in}{3.247337in}}%
\pgfpathlineto{\pgfqpoint{2.546203in}{3.246444in}}%
\pgfpathlineto{\pgfqpoint{2.567965in}{3.245567in}}%
\pgfpathlineto{\pgfqpoint{2.589726in}{3.244762in}}%
\pgfpathlineto{\pgfqpoint{2.611488in}{3.244035in}}%
\pgfpathlineto{\pgfqpoint{2.633250in}{3.243280in}}%
\pgfpathlineto{\pgfqpoint{2.655011in}{3.242515in}}%
\pgfpathlineto{\pgfqpoint{2.676773in}{3.241668in}}%
\pgfpathlineto{\pgfqpoint{2.698535in}{3.240891in}}%
\pgfpathlineto{\pgfqpoint{2.720296in}{3.239942in}}%
\pgfpathlineto{\pgfqpoint{2.742058in}{3.239061in}}%
\pgfpathlineto{\pgfqpoint{2.763820in}{3.238344in}}%
\pgfpathlineto{\pgfqpoint{2.785581in}{3.237601in}}%
\pgfusepath{stroke}%
\end{pgfscope}%
\begin{pgfscope}%
\pgfpathrectangle{\pgfqpoint{0.609415in}{2.347992in}}{\pgfqpoint{2.219690in}{1.226201in}}%
\pgfusepath{clip}%
\pgfsetrectcap%
\pgfsetroundjoin%
\pgfsetlinewidth{1.003750pt}%
\definecolor{currentstroke}{rgb}{0.000000,0.725490,0.270588}%
\pgfsetstrokecolor{currentstroke}%
\pgfsetdash{}{0pt}%
\pgfpathmoveto{\pgfqpoint{0.631176in}{3.518457in}}%
\pgfpathlineto{\pgfqpoint{0.652938in}{3.378902in}}%
\pgfpathlineto{\pgfqpoint{0.674700in}{3.303495in}}%
\pgfpathlineto{\pgfqpoint{0.696461in}{3.249639in}}%
\pgfpathlineto{\pgfqpoint{0.718223in}{3.209186in}}%
\pgfpathlineto{\pgfqpoint{0.739985in}{3.179663in}}%
\pgfpathlineto{\pgfqpoint{0.761746in}{3.153711in}}%
\pgfpathlineto{\pgfqpoint{0.783508in}{3.133275in}}%
\pgfpathlineto{\pgfqpoint{0.805270in}{3.114806in}}%
\pgfpathlineto{\pgfqpoint{0.827031in}{3.097136in}}%
\pgfpathlineto{\pgfqpoint{0.848793in}{3.081945in}}%
\pgfpathlineto{\pgfqpoint{0.870555in}{3.067284in}}%
\pgfpathlineto{\pgfqpoint{0.892316in}{3.053629in}}%
\pgfpathlineto{\pgfqpoint{0.914078in}{3.041152in}}%
\pgfpathlineto{\pgfqpoint{0.935840in}{3.029103in}}%
\pgfpathlineto{\pgfqpoint{0.957601in}{3.017789in}}%
\pgfpathlineto{\pgfqpoint{0.979363in}{3.008019in}}%
\pgfpathlineto{\pgfqpoint{1.001125in}{2.998479in}}%
\pgfpathlineto{\pgfqpoint{1.022886in}{2.989257in}}%
\pgfpathlineto{\pgfqpoint{1.044648in}{2.981099in}}%
\pgfpathlineto{\pgfqpoint{1.066410in}{2.972743in}}%
\pgfpathlineto{\pgfqpoint{1.088171in}{2.965004in}}%
\pgfpathlineto{\pgfqpoint{1.109933in}{2.957172in}}%
\pgfpathlineto{\pgfqpoint{1.131695in}{2.949293in}}%
\pgfpathlineto{\pgfqpoint{1.153456in}{2.942008in}}%
\pgfpathlineto{\pgfqpoint{1.175218in}{2.935356in}}%
\pgfpathlineto{\pgfqpoint{1.196980in}{2.928418in}}%
\pgfpathlineto{\pgfqpoint{1.218741in}{2.921582in}}%
\pgfpathlineto{\pgfqpoint{1.240503in}{2.915525in}}%
\pgfpathlineto{\pgfqpoint{1.262265in}{2.909334in}}%
\pgfpathlineto{\pgfqpoint{1.284026in}{2.903396in}}%
\pgfpathlineto{\pgfqpoint{1.305788in}{2.897693in}}%
\pgfpathlineto{\pgfqpoint{1.327550in}{2.892218in}}%
\pgfpathlineto{\pgfqpoint{1.349311in}{2.886976in}}%
\pgfpathlineto{\pgfqpoint{1.371073in}{2.882128in}}%
\pgfpathlineto{\pgfqpoint{1.392835in}{2.876902in}}%
\pgfpathlineto{\pgfqpoint{1.414596in}{2.871719in}}%
\pgfpathlineto{\pgfqpoint{1.436358in}{2.866213in}}%
\pgfpathlineto{\pgfqpoint{1.458120in}{2.861309in}}%
\pgfpathlineto{\pgfqpoint{1.479881in}{2.856127in}}%
\pgfpathlineto{\pgfqpoint{1.501643in}{2.851276in}}%
\pgfpathlineto{\pgfqpoint{1.523405in}{2.846358in}}%
\pgfpathlineto{\pgfqpoint{1.545166in}{2.841609in}}%
\pgfpathlineto{\pgfqpoint{1.566928in}{2.837541in}}%
\pgfpathlineto{\pgfqpoint{1.588690in}{2.833228in}}%
\pgfpathlineto{\pgfqpoint{1.610451in}{2.828707in}}%
\pgfpathlineto{\pgfqpoint{1.632213in}{2.824202in}}%
\pgfpathlineto{\pgfqpoint{1.653975in}{2.820116in}}%
\pgfpathlineto{\pgfqpoint{1.675736in}{2.816323in}}%
\pgfpathlineto{\pgfqpoint{1.697498in}{2.812708in}}%
\pgfpathlineto{\pgfqpoint{1.719260in}{2.808754in}}%
\pgfpathlineto{\pgfqpoint{1.741021in}{2.804884in}}%
\pgfpathlineto{\pgfqpoint{1.762783in}{2.801018in}}%
\pgfpathlineto{\pgfqpoint{1.784545in}{2.797236in}}%
\pgfpathlineto{\pgfqpoint{1.806306in}{2.793597in}}%
\pgfpathlineto{\pgfqpoint{1.828068in}{2.789927in}}%
\pgfpathlineto{\pgfqpoint{1.849830in}{2.786233in}}%
\pgfpathlineto{\pgfqpoint{1.871591in}{2.782365in}}%
\pgfpathlineto{\pgfqpoint{1.893353in}{2.778927in}}%
\pgfpathlineto{\pgfqpoint{1.915115in}{2.775293in}}%
\pgfpathlineto{\pgfqpoint{1.936876in}{2.771794in}}%
\pgfpathlineto{\pgfqpoint{1.958638in}{2.768432in}}%
\pgfpathlineto{\pgfqpoint{1.980400in}{2.765085in}}%
\pgfpathlineto{\pgfqpoint{2.002161in}{2.761910in}}%
\pgfpathlineto{\pgfqpoint{2.023923in}{2.758873in}}%
\pgfpathlineto{\pgfqpoint{2.045685in}{2.755709in}}%
\pgfpathlineto{\pgfqpoint{2.067446in}{2.752396in}}%
\pgfpathlineto{\pgfqpoint{2.089208in}{2.749258in}}%
\pgfpathlineto{\pgfqpoint{2.110970in}{2.746172in}}%
\pgfpathlineto{\pgfqpoint{2.132731in}{2.743126in}}%
\pgfpathlineto{\pgfqpoint{2.154493in}{2.740021in}}%
\pgfpathlineto{\pgfqpoint{2.176255in}{2.736998in}}%
\pgfpathlineto{\pgfqpoint{2.198016in}{2.734153in}}%
\pgfpathlineto{\pgfqpoint{2.219778in}{2.731278in}}%
\pgfpathlineto{\pgfqpoint{2.241540in}{2.728042in}}%
\pgfpathlineto{\pgfqpoint{2.263301in}{2.725095in}}%
\pgfpathlineto{\pgfqpoint{2.285063in}{2.722179in}}%
\pgfpathlineto{\pgfqpoint{2.306825in}{2.719356in}}%
\pgfpathlineto{\pgfqpoint{2.328586in}{2.716467in}}%
\pgfpathlineto{\pgfqpoint{2.350348in}{2.713548in}}%
\pgfpathlineto{\pgfqpoint{2.372110in}{2.710709in}}%
\pgfpathlineto{\pgfqpoint{2.393871in}{2.707900in}}%
\pgfpathlineto{\pgfqpoint{2.415633in}{2.705277in}}%
\pgfpathlineto{\pgfqpoint{2.437395in}{2.702437in}}%
\pgfpathlineto{\pgfqpoint{2.459156in}{2.699850in}}%
\pgfpathlineto{\pgfqpoint{2.480918in}{2.697157in}}%
\pgfpathlineto{\pgfqpoint{2.502680in}{2.694567in}}%
\pgfpathlineto{\pgfqpoint{2.524441in}{2.691828in}}%
\pgfpathlineto{\pgfqpoint{2.546203in}{2.689112in}}%
\pgfpathlineto{\pgfqpoint{2.567965in}{2.686707in}}%
\pgfpathlineto{\pgfqpoint{2.589726in}{2.684099in}}%
\pgfpathlineto{\pgfqpoint{2.611488in}{2.681470in}}%
\pgfpathlineto{\pgfqpoint{2.633250in}{2.678766in}}%
\pgfpathlineto{\pgfqpoint{2.655011in}{2.676242in}}%
\pgfpathlineto{\pgfqpoint{2.676773in}{2.673786in}}%
\pgfpathlineto{\pgfqpoint{2.698535in}{2.671475in}}%
\pgfpathlineto{\pgfqpoint{2.720296in}{2.669057in}}%
\pgfpathlineto{\pgfqpoint{2.742058in}{2.666556in}}%
\pgfpathlineto{\pgfqpoint{2.763820in}{2.664056in}}%
\pgfpathlineto{\pgfqpoint{2.785581in}{2.661466in}}%
\pgfusepath{stroke}%
\end{pgfscope}%
\begin{pgfscope}%
\pgfpathrectangle{\pgfqpoint{0.609415in}{2.347992in}}{\pgfqpoint{2.219690in}{1.226201in}}%
\pgfusepath{clip}%
\pgfsetrectcap%
\pgfsetroundjoin%
\pgfsetlinewidth{1.003750pt}%
\definecolor{currentstroke}{rgb}{1.000000,0.584314,0.000000}%
\pgfsetstrokecolor{currentstroke}%
\pgfsetdash{}{0pt}%
\pgfpathmoveto{\pgfqpoint{0.631176in}{3.518457in}}%
\pgfpathlineto{\pgfqpoint{0.652938in}{3.395763in}}%
\pgfpathlineto{\pgfqpoint{0.674700in}{3.308372in}}%
\pgfpathlineto{\pgfqpoint{0.696461in}{3.250146in}}%
\pgfpathlineto{\pgfqpoint{0.718223in}{3.204408in}}%
\pgfpathlineto{\pgfqpoint{0.739985in}{3.166387in}}%
\pgfpathlineto{\pgfqpoint{0.761746in}{3.135372in}}%
\pgfpathlineto{\pgfqpoint{0.783508in}{3.109022in}}%
\pgfpathlineto{\pgfqpoint{0.805270in}{3.087422in}}%
\pgfpathlineto{\pgfqpoint{0.827031in}{3.066738in}}%
\pgfpathlineto{\pgfqpoint{0.848793in}{3.049294in}}%
\pgfpathlineto{\pgfqpoint{0.870555in}{3.032631in}}%
\pgfpathlineto{\pgfqpoint{0.892316in}{3.017188in}}%
\pgfpathlineto{\pgfqpoint{0.914078in}{3.001961in}}%
\pgfpathlineto{\pgfqpoint{0.935840in}{2.988393in}}%
\pgfpathlineto{\pgfqpoint{0.957601in}{2.974432in}}%
\pgfpathlineto{\pgfqpoint{0.979363in}{2.962368in}}%
\pgfpathlineto{\pgfqpoint{1.001125in}{2.950080in}}%
\pgfpathlineto{\pgfqpoint{1.022886in}{2.938639in}}%
\pgfpathlineto{\pgfqpoint{1.044648in}{2.928708in}}%
\pgfpathlineto{\pgfqpoint{1.066410in}{2.918283in}}%
\pgfpathlineto{\pgfqpoint{1.088171in}{2.908916in}}%
\pgfpathlineto{\pgfqpoint{1.109933in}{2.900126in}}%
\pgfpathlineto{\pgfqpoint{1.131695in}{2.891960in}}%
\pgfpathlineto{\pgfqpoint{1.153456in}{2.883887in}}%
\pgfpathlineto{\pgfqpoint{1.175218in}{2.876034in}}%
\pgfpathlineto{\pgfqpoint{1.196980in}{2.868743in}}%
\pgfpathlineto{\pgfqpoint{1.218741in}{2.861552in}}%
\pgfpathlineto{\pgfqpoint{1.240503in}{2.854574in}}%
\pgfpathlineto{\pgfqpoint{1.262265in}{2.846986in}}%
\pgfpathlineto{\pgfqpoint{1.284026in}{2.839867in}}%
\pgfpathlineto{\pgfqpoint{1.305788in}{2.833471in}}%
\pgfpathlineto{\pgfqpoint{1.327550in}{2.826919in}}%
\pgfpathlineto{\pgfqpoint{1.349311in}{2.820375in}}%
\pgfpathlineto{\pgfqpoint{1.371073in}{2.814395in}}%
\pgfpathlineto{\pgfqpoint{1.392835in}{2.808337in}}%
\pgfpathlineto{\pgfqpoint{1.414596in}{2.802638in}}%
\pgfpathlineto{\pgfqpoint{1.436358in}{2.797642in}}%
\pgfpathlineto{\pgfqpoint{1.458120in}{2.792257in}}%
\pgfpathlineto{\pgfqpoint{1.479881in}{2.786814in}}%
\pgfpathlineto{\pgfqpoint{1.501643in}{2.781913in}}%
\pgfpathlineto{\pgfqpoint{1.523405in}{2.776561in}}%
\pgfpathlineto{\pgfqpoint{1.545166in}{2.771333in}}%
\pgfpathlineto{\pgfqpoint{1.566928in}{2.766538in}}%
\pgfpathlineto{\pgfqpoint{1.588690in}{2.762336in}}%
\pgfpathlineto{\pgfqpoint{1.610451in}{2.757681in}}%
\pgfpathlineto{\pgfqpoint{1.632213in}{2.753168in}}%
\pgfpathlineto{\pgfqpoint{1.653975in}{2.748918in}}%
\pgfpathlineto{\pgfqpoint{1.675736in}{2.744900in}}%
\pgfpathlineto{\pgfqpoint{1.697498in}{2.740609in}}%
\pgfpathlineto{\pgfqpoint{1.719260in}{2.736406in}}%
\pgfpathlineto{\pgfqpoint{1.741021in}{2.732591in}}%
\pgfpathlineto{\pgfqpoint{1.762783in}{2.728476in}}%
\pgfpathlineto{\pgfqpoint{1.784545in}{2.724345in}}%
\pgfpathlineto{\pgfqpoint{1.806306in}{2.720541in}}%
\pgfpathlineto{\pgfqpoint{1.828068in}{2.716540in}}%
\pgfpathlineto{\pgfqpoint{1.849830in}{2.712719in}}%
\pgfpathlineto{\pgfqpoint{1.871591in}{2.708853in}}%
\pgfpathlineto{\pgfqpoint{1.893353in}{2.705212in}}%
\pgfpathlineto{\pgfqpoint{1.915115in}{2.701222in}}%
\pgfpathlineto{\pgfqpoint{1.936876in}{2.697526in}}%
\pgfpathlineto{\pgfqpoint{1.958638in}{2.693679in}}%
\pgfpathlineto{\pgfqpoint{1.980400in}{2.689852in}}%
\pgfpathlineto{\pgfqpoint{2.002161in}{2.686268in}}%
\pgfpathlineto{\pgfqpoint{2.023923in}{2.682565in}}%
\pgfpathlineto{\pgfqpoint{2.045685in}{2.678932in}}%
\pgfpathlineto{\pgfqpoint{2.067446in}{2.675261in}}%
\pgfpathlineto{\pgfqpoint{2.089208in}{2.671903in}}%
\pgfpathlineto{\pgfqpoint{2.110970in}{2.668450in}}%
\pgfpathlineto{\pgfqpoint{2.132731in}{2.665244in}}%
\pgfpathlineto{\pgfqpoint{2.154493in}{2.662164in}}%
\pgfpathlineto{\pgfqpoint{2.176255in}{2.658817in}}%
\pgfpathlineto{\pgfqpoint{2.198016in}{2.655796in}}%
\pgfpathlineto{\pgfqpoint{2.219778in}{2.652607in}}%
\pgfpathlineto{\pgfqpoint{2.241540in}{2.649245in}}%
\pgfpathlineto{\pgfqpoint{2.263301in}{2.646150in}}%
\pgfpathlineto{\pgfqpoint{2.285063in}{2.642911in}}%
\pgfpathlineto{\pgfqpoint{2.306825in}{2.639272in}}%
\pgfpathlineto{\pgfqpoint{2.328586in}{2.636116in}}%
\pgfpathlineto{\pgfqpoint{2.350348in}{2.632991in}}%
\pgfpathlineto{\pgfqpoint{2.372110in}{2.630259in}}%
\pgfpathlineto{\pgfqpoint{2.393871in}{2.627321in}}%
\pgfpathlineto{\pgfqpoint{2.415633in}{2.624555in}}%
\pgfpathlineto{\pgfqpoint{2.437395in}{2.621604in}}%
\pgfpathlineto{\pgfqpoint{2.459156in}{2.618688in}}%
\pgfpathlineto{\pgfqpoint{2.480918in}{2.615641in}}%
\pgfpathlineto{\pgfqpoint{2.502680in}{2.612961in}}%
\pgfpathlineto{\pgfqpoint{2.524441in}{2.610166in}}%
\pgfpathlineto{\pgfqpoint{2.546203in}{2.607432in}}%
\pgfpathlineto{\pgfqpoint{2.567965in}{2.604852in}}%
\pgfpathlineto{\pgfqpoint{2.589726in}{2.602164in}}%
\pgfpathlineto{\pgfqpoint{2.611488in}{2.599355in}}%
\pgfpathlineto{\pgfqpoint{2.633250in}{2.596650in}}%
\pgfpathlineto{\pgfqpoint{2.655011in}{2.594038in}}%
\pgfpathlineto{\pgfqpoint{2.676773in}{2.591293in}}%
\pgfpathlineto{\pgfqpoint{2.698535in}{2.588710in}}%
\pgfpathlineto{\pgfqpoint{2.720296in}{2.586160in}}%
\pgfpathlineto{\pgfqpoint{2.742058in}{2.583518in}}%
\pgfpathlineto{\pgfqpoint{2.763820in}{2.580903in}}%
\pgfpathlineto{\pgfqpoint{2.785581in}{2.578463in}}%
\pgfusepath{stroke}%
\end{pgfscope}%
\begin{pgfscope}%
\pgfpathrectangle{\pgfqpoint{0.609415in}{2.347992in}}{\pgfqpoint{2.219690in}{1.226201in}}%
\pgfusepath{clip}%
\pgfsetrectcap%
\pgfsetroundjoin%
\pgfsetlinewidth{1.003750pt}%
\definecolor{currentstroke}{rgb}{1.000000,0.172549,0.000000}%
\pgfsetstrokecolor{currentstroke}%
\pgfsetdash{}{0pt}%
\pgfpathmoveto{\pgfqpoint{0.631176in}{3.518457in}}%
\pgfpathlineto{\pgfqpoint{0.652938in}{3.472897in}}%
\pgfpathlineto{\pgfqpoint{0.674700in}{3.439179in}}%
\pgfpathlineto{\pgfqpoint{0.696461in}{3.415016in}}%
\pgfpathlineto{\pgfqpoint{0.718223in}{3.397737in}}%
\pgfpathlineto{\pgfqpoint{0.739985in}{3.382963in}}%
\pgfpathlineto{\pgfqpoint{0.761746in}{3.369978in}}%
\pgfpathlineto{\pgfqpoint{0.783508in}{3.359819in}}%
\pgfpathlineto{\pgfqpoint{0.805270in}{3.351094in}}%
\pgfpathlineto{\pgfqpoint{0.827031in}{3.342681in}}%
\pgfpathlineto{\pgfqpoint{0.848793in}{3.334996in}}%
\pgfpathlineto{\pgfqpoint{0.870555in}{3.327673in}}%
\pgfpathlineto{\pgfqpoint{0.892316in}{3.321688in}}%
\pgfpathlineto{\pgfqpoint{0.914078in}{3.316394in}}%
\pgfpathlineto{\pgfqpoint{0.935840in}{3.312000in}}%
\pgfpathlineto{\pgfqpoint{0.957601in}{3.306384in}}%
\pgfpathlineto{\pgfqpoint{0.979363in}{3.300575in}}%
\pgfpathlineto{\pgfqpoint{1.001125in}{3.295712in}}%
\pgfpathlineto{\pgfqpoint{1.022886in}{3.291460in}}%
\pgfpathlineto{\pgfqpoint{1.044648in}{3.286785in}}%
\pgfpathlineto{\pgfqpoint{1.066410in}{3.282608in}}%
\pgfpathlineto{\pgfqpoint{1.088171in}{3.278632in}}%
\pgfpathlineto{\pgfqpoint{1.109933in}{3.275031in}}%
\pgfpathlineto{\pgfqpoint{1.131695in}{3.271332in}}%
\pgfpathlineto{\pgfqpoint{1.153456in}{3.267925in}}%
\pgfpathlineto{\pgfqpoint{1.175218in}{3.264774in}}%
\pgfpathlineto{\pgfqpoint{1.196980in}{3.261475in}}%
\pgfpathlineto{\pgfqpoint{1.218741in}{3.258282in}}%
\pgfpathlineto{\pgfqpoint{1.240503in}{3.255665in}}%
\pgfpathlineto{\pgfqpoint{1.262265in}{3.252243in}}%
\pgfpathlineto{\pgfqpoint{1.284026in}{3.248993in}}%
\pgfpathlineto{\pgfqpoint{1.305788in}{3.246161in}}%
\pgfpathlineto{\pgfqpoint{1.327550in}{3.243010in}}%
\pgfpathlineto{\pgfqpoint{1.349311in}{3.240260in}}%
\pgfpathlineto{\pgfqpoint{1.371073in}{3.237808in}}%
\pgfpathlineto{\pgfqpoint{1.392835in}{3.234814in}}%
\pgfpathlineto{\pgfqpoint{1.414596in}{3.232137in}}%
\pgfpathlineto{\pgfqpoint{1.436358in}{3.229941in}}%
\pgfpathlineto{\pgfqpoint{1.458120in}{3.227650in}}%
\pgfpathlineto{\pgfqpoint{1.479881in}{3.225135in}}%
\pgfpathlineto{\pgfqpoint{1.501643in}{3.222564in}}%
\pgfpathlineto{\pgfqpoint{1.523405in}{3.219984in}}%
\pgfpathlineto{\pgfqpoint{1.545166in}{3.217732in}}%
\pgfpathlineto{\pgfqpoint{1.566928in}{3.215818in}}%
\pgfpathlineto{\pgfqpoint{1.588690in}{3.213925in}}%
\pgfpathlineto{\pgfqpoint{1.610451in}{3.211909in}}%
\pgfpathlineto{\pgfqpoint{1.632213in}{3.209931in}}%
\pgfpathlineto{\pgfqpoint{1.653975in}{3.207663in}}%
\pgfpathlineto{\pgfqpoint{1.675736in}{3.205751in}}%
\pgfpathlineto{\pgfqpoint{1.697498in}{3.203623in}}%
\pgfpathlineto{\pgfqpoint{1.719260in}{3.201540in}}%
\pgfpathlineto{\pgfqpoint{1.741021in}{3.199544in}}%
\pgfpathlineto{\pgfqpoint{1.762783in}{3.197532in}}%
\pgfpathlineto{\pgfqpoint{1.784545in}{3.195869in}}%
\pgfpathlineto{\pgfqpoint{1.806306in}{3.193824in}}%
\pgfpathlineto{\pgfqpoint{1.828068in}{3.192077in}}%
\pgfpathlineto{\pgfqpoint{1.849830in}{3.190368in}}%
\pgfpathlineto{\pgfqpoint{1.871591in}{3.188675in}}%
\pgfpathlineto{\pgfqpoint{1.893353in}{3.186671in}}%
\pgfpathlineto{\pgfqpoint{1.915115in}{3.184904in}}%
\pgfpathlineto{\pgfqpoint{1.936876in}{3.183284in}}%
\pgfpathlineto{\pgfqpoint{1.958638in}{3.181589in}}%
\pgfpathlineto{\pgfqpoint{1.980400in}{3.179879in}}%
\pgfpathlineto{\pgfqpoint{2.002161in}{3.178399in}}%
\pgfpathlineto{\pgfqpoint{2.023923in}{3.177012in}}%
\pgfpathlineto{\pgfqpoint{2.045685in}{3.175456in}}%
\pgfpathlineto{\pgfqpoint{2.067446in}{3.173831in}}%
\pgfpathlineto{\pgfqpoint{2.089208in}{3.172344in}}%
\pgfpathlineto{\pgfqpoint{2.110970in}{3.170952in}}%
\pgfpathlineto{\pgfqpoint{2.132731in}{3.169512in}}%
\pgfpathlineto{\pgfqpoint{2.154493in}{3.168294in}}%
\pgfpathlineto{\pgfqpoint{2.176255in}{3.166761in}}%
\pgfpathlineto{\pgfqpoint{2.198016in}{3.165212in}}%
\pgfpathlineto{\pgfqpoint{2.219778in}{3.163809in}}%
\pgfpathlineto{\pgfqpoint{2.241540in}{3.162467in}}%
\pgfpathlineto{\pgfqpoint{2.263301in}{3.161081in}}%
\pgfpathlineto{\pgfqpoint{2.285063in}{3.159634in}}%
\pgfpathlineto{\pgfqpoint{2.306825in}{3.158242in}}%
\pgfpathlineto{\pgfqpoint{2.328586in}{3.156899in}}%
\pgfpathlineto{\pgfqpoint{2.350348in}{3.155460in}}%
\pgfpathlineto{\pgfqpoint{2.372110in}{3.154136in}}%
\pgfpathlineto{\pgfqpoint{2.393871in}{3.152923in}}%
\pgfpathlineto{\pgfqpoint{2.415633in}{3.151754in}}%
\pgfpathlineto{\pgfqpoint{2.437395in}{3.150478in}}%
\pgfpathlineto{\pgfqpoint{2.459156in}{3.149110in}}%
\pgfpathlineto{\pgfqpoint{2.480918in}{3.147709in}}%
\pgfpathlineto{\pgfqpoint{2.502680in}{3.146336in}}%
\pgfpathlineto{\pgfqpoint{2.524441in}{3.144924in}}%
\pgfpathlineto{\pgfqpoint{2.546203in}{3.143607in}}%
\pgfpathlineto{\pgfqpoint{2.567965in}{3.142347in}}%
\pgfpathlineto{\pgfqpoint{2.589726in}{3.141136in}}%
\pgfpathlineto{\pgfqpoint{2.611488in}{3.139917in}}%
\pgfpathlineto{\pgfqpoint{2.633250in}{3.138608in}}%
\pgfpathlineto{\pgfqpoint{2.655011in}{3.137388in}}%
\pgfpathlineto{\pgfqpoint{2.676773in}{3.136226in}}%
\pgfpathlineto{\pgfqpoint{2.698535in}{3.134987in}}%
\pgfpathlineto{\pgfqpoint{2.720296in}{3.133822in}}%
\pgfpathlineto{\pgfqpoint{2.742058in}{3.132499in}}%
\pgfpathlineto{\pgfqpoint{2.763820in}{3.131232in}}%
\pgfpathlineto{\pgfqpoint{2.785581in}{3.130198in}}%
\pgfusepath{stroke}%
\end{pgfscope}%
\begin{pgfscope}%
\pgfpathrectangle{\pgfqpoint{0.609415in}{2.347992in}}{\pgfqpoint{2.219690in}{1.226201in}}%
\pgfusepath{clip}%
\pgfsetrectcap%
\pgfsetroundjoin%
\pgfsetlinewidth{1.003750pt}%
\definecolor{currentstroke}{rgb}{0.517647,0.356863,0.592157}%
\pgfsetstrokecolor{currentstroke}%
\pgfsetdash{}{0pt}%
\pgfpathmoveto{\pgfqpoint{0.631176in}{3.518457in}}%
\pgfpathlineto{\pgfqpoint{0.652938in}{3.182298in}}%
\pgfpathlineto{\pgfqpoint{0.674700in}{3.024377in}}%
\pgfpathlineto{\pgfqpoint{0.696461in}{2.936861in}}%
\pgfpathlineto{\pgfqpoint{0.718223in}{2.872918in}}%
\pgfpathlineto{\pgfqpoint{0.739985in}{2.826373in}}%
\pgfpathlineto{\pgfqpoint{0.761746in}{2.792067in}}%
\pgfpathlineto{\pgfqpoint{0.783508in}{2.767373in}}%
\pgfpathlineto{\pgfqpoint{0.805270in}{2.741950in}}%
\pgfpathlineto{\pgfqpoint{0.827031in}{2.720932in}}%
\pgfpathlineto{\pgfqpoint{0.848793in}{2.704244in}}%
\pgfpathlineto{\pgfqpoint{0.870555in}{2.693577in}}%
\pgfpathlineto{\pgfqpoint{0.892316in}{2.678653in}}%
\pgfpathlineto{\pgfqpoint{0.914078in}{2.666697in}}%
\pgfpathlineto{\pgfqpoint{0.935840in}{2.656783in}}%
\pgfpathlineto{\pgfqpoint{0.957601in}{2.647594in}}%
\pgfpathlineto{\pgfqpoint{0.979363in}{2.640343in}}%
\pgfpathlineto{\pgfqpoint{1.001125in}{2.633200in}}%
\pgfpathlineto{\pgfqpoint{1.022886in}{2.626176in}}%
\pgfpathlineto{\pgfqpoint{1.044648in}{2.617990in}}%
\pgfpathlineto{\pgfqpoint{1.066410in}{2.611391in}}%
\pgfpathlineto{\pgfqpoint{1.088171in}{2.604202in}}%
\pgfpathlineto{\pgfqpoint{1.109933in}{2.596594in}}%
\pgfpathlineto{\pgfqpoint{1.131695in}{2.591672in}}%
\pgfpathlineto{\pgfqpoint{1.153456in}{2.585621in}}%
\pgfpathlineto{\pgfqpoint{1.175218in}{2.579079in}}%
\pgfpathlineto{\pgfqpoint{1.196980in}{2.574681in}}%
\pgfpathlineto{\pgfqpoint{1.218741in}{2.568567in}}%
\pgfpathlineto{\pgfqpoint{1.240503in}{2.562818in}}%
\pgfpathlineto{\pgfqpoint{1.262265in}{2.558596in}}%
\pgfpathlineto{\pgfqpoint{1.284026in}{2.553645in}}%
\pgfpathlineto{\pgfqpoint{1.305788in}{2.550069in}}%
\pgfpathlineto{\pgfqpoint{1.327550in}{2.544734in}}%
\pgfpathlineto{\pgfqpoint{1.349311in}{2.541410in}}%
\pgfpathlineto{\pgfqpoint{1.371073in}{2.537242in}}%
\pgfpathlineto{\pgfqpoint{1.392835in}{2.532603in}}%
\pgfpathlineto{\pgfqpoint{1.414596in}{2.529543in}}%
\pgfpathlineto{\pgfqpoint{1.436358in}{2.526730in}}%
\pgfpathlineto{\pgfqpoint{1.458120in}{2.523662in}}%
\pgfpathlineto{\pgfqpoint{1.479881in}{2.519242in}}%
\pgfpathlineto{\pgfqpoint{1.501643in}{2.516955in}}%
\pgfpathlineto{\pgfqpoint{1.523405in}{2.514363in}}%
\pgfpathlineto{\pgfqpoint{1.545166in}{2.510648in}}%
\pgfpathlineto{\pgfqpoint{1.566928in}{2.508665in}}%
\pgfpathlineto{\pgfqpoint{1.588690in}{2.505792in}}%
\pgfpathlineto{\pgfqpoint{1.610451in}{2.501923in}}%
\pgfpathlineto{\pgfqpoint{1.632213in}{2.499519in}}%
\pgfpathlineto{\pgfqpoint{1.653975in}{2.496543in}}%
\pgfpathlineto{\pgfqpoint{1.675736in}{2.493643in}}%
\pgfpathlineto{\pgfqpoint{1.697498in}{2.491221in}}%
\pgfpathlineto{\pgfqpoint{1.719260in}{2.488432in}}%
\pgfpathlineto{\pgfqpoint{1.741021in}{2.485496in}}%
\pgfpathlineto{\pgfqpoint{1.762783in}{2.483118in}}%
\pgfpathlineto{\pgfqpoint{1.784545in}{2.481426in}}%
\pgfpathlineto{\pgfqpoint{1.806306in}{2.479828in}}%
\pgfpathlineto{\pgfqpoint{1.828068in}{2.478110in}}%
\pgfpathlineto{\pgfqpoint{1.849830in}{2.476476in}}%
\pgfpathlineto{\pgfqpoint{1.871591in}{2.473805in}}%
\pgfpathlineto{\pgfqpoint{1.893353in}{2.471921in}}%
\pgfpathlineto{\pgfqpoint{1.915115in}{2.469398in}}%
\pgfpathlineto{\pgfqpoint{1.936876in}{2.467593in}}%
\pgfpathlineto{\pgfqpoint{1.958638in}{2.465679in}}%
\pgfpathlineto{\pgfqpoint{1.980400in}{2.463771in}}%
\pgfpathlineto{\pgfqpoint{2.002161in}{2.461713in}}%
\pgfpathlineto{\pgfqpoint{2.023923in}{2.459422in}}%
\pgfpathlineto{\pgfqpoint{2.045685in}{2.457344in}}%
\pgfpathlineto{\pgfqpoint{2.067446in}{2.454687in}}%
\pgfpathlineto{\pgfqpoint{2.089208in}{2.452953in}}%
\pgfpathlineto{\pgfqpoint{2.110970in}{2.451081in}}%
\pgfpathlineto{\pgfqpoint{2.132731in}{2.449139in}}%
\pgfpathlineto{\pgfqpoint{2.154493in}{2.447371in}}%
\pgfpathlineto{\pgfqpoint{2.176255in}{2.446314in}}%
\pgfpathlineto{\pgfqpoint{2.198016in}{2.444903in}}%
\pgfpathlineto{\pgfqpoint{2.219778in}{2.443259in}}%
\pgfpathlineto{\pgfqpoint{2.241540in}{2.441235in}}%
\pgfpathlineto{\pgfqpoint{2.263301in}{2.439645in}}%
\pgfpathlineto{\pgfqpoint{2.285063in}{2.438034in}}%
\pgfpathlineto{\pgfqpoint{2.306825in}{2.436704in}}%
\pgfpathlineto{\pgfqpoint{2.328586in}{2.434803in}}%
\pgfpathlineto{\pgfqpoint{2.350348in}{2.432481in}}%
\pgfpathlineto{\pgfqpoint{2.372110in}{2.430929in}}%
\pgfpathlineto{\pgfqpoint{2.393871in}{2.428664in}}%
\pgfpathlineto{\pgfqpoint{2.415633in}{2.427605in}}%
\pgfpathlineto{\pgfqpoint{2.437395in}{2.425819in}}%
\pgfpathlineto{\pgfqpoint{2.459156in}{2.424153in}}%
\pgfpathlineto{\pgfqpoint{2.480918in}{2.423145in}}%
\pgfpathlineto{\pgfqpoint{2.502680in}{2.421533in}}%
\pgfpathlineto{\pgfqpoint{2.524441in}{2.420127in}}%
\pgfpathlineto{\pgfqpoint{2.546203in}{2.419031in}}%
\pgfpathlineto{\pgfqpoint{2.567965in}{2.417421in}}%
\pgfpathlineto{\pgfqpoint{2.589726in}{2.416194in}}%
\pgfpathlineto{\pgfqpoint{2.611488in}{2.414056in}}%
\pgfpathlineto{\pgfqpoint{2.633250in}{2.412658in}}%
\pgfpathlineto{\pgfqpoint{2.655011in}{2.411277in}}%
\pgfpathlineto{\pgfqpoint{2.676773in}{2.410171in}}%
\pgfpathlineto{\pgfqpoint{2.698535in}{2.408968in}}%
\pgfpathlineto{\pgfqpoint{2.720296in}{2.407758in}}%
\pgfpathlineto{\pgfqpoint{2.742058in}{2.406652in}}%
\pgfpathlineto{\pgfqpoint{2.763820in}{2.405310in}}%
\pgfpathlineto{\pgfqpoint{2.785581in}{2.403729in}}%
\pgfusepath{stroke}%
\end{pgfscope}%
\begin{pgfscope}%
\pgfpathrectangle{\pgfqpoint{0.609415in}{2.347992in}}{\pgfqpoint{2.219690in}{1.226201in}}%
\pgfusepath{clip}%
\pgfsetrectcap%
\pgfsetroundjoin%
\pgfsetlinewidth{1.003750pt}%
\definecolor{currentstroke}{rgb}{0.278431,0.278431,0.278431}%
\pgfsetstrokecolor{currentstroke}%
\pgfsetdash{}{0pt}%
\pgfpathmoveto{\pgfqpoint{0.631176in}{3.518457in}}%
\pgfpathlineto{\pgfqpoint{0.652938in}{3.511678in}}%
\pgfpathlineto{\pgfqpoint{0.674700in}{3.505504in}}%
\pgfpathlineto{\pgfqpoint{0.696461in}{3.500832in}}%
\pgfpathlineto{\pgfqpoint{0.718223in}{3.497772in}}%
\pgfpathlineto{\pgfqpoint{0.739985in}{3.494744in}}%
\pgfpathlineto{\pgfqpoint{0.761746in}{3.491828in}}%
\pgfpathlineto{\pgfqpoint{0.783508in}{3.489442in}}%
\pgfpathlineto{\pgfqpoint{0.805270in}{3.486879in}}%
\pgfpathlineto{\pgfqpoint{0.827031in}{3.484326in}}%
\pgfpathlineto{\pgfqpoint{0.848793in}{3.482371in}}%
\pgfpathlineto{\pgfqpoint{0.870555in}{3.480597in}}%
\pgfpathlineto{\pgfqpoint{0.892316in}{3.478881in}}%
\pgfpathlineto{\pgfqpoint{0.914078in}{3.477042in}}%
\pgfpathlineto{\pgfqpoint{0.935840in}{3.475245in}}%
\pgfpathlineto{\pgfqpoint{0.957601in}{3.473781in}}%
\pgfpathlineto{\pgfqpoint{0.979363in}{3.472003in}}%
\pgfpathlineto{\pgfqpoint{1.001125in}{3.470497in}}%
\pgfpathlineto{\pgfqpoint{1.022886in}{3.468946in}}%
\pgfpathlineto{\pgfqpoint{1.044648in}{3.467723in}}%
\pgfpathlineto{\pgfqpoint{1.066410in}{3.466064in}}%
\pgfpathlineto{\pgfqpoint{1.088171in}{3.464730in}}%
\pgfpathlineto{\pgfqpoint{1.109933in}{3.463494in}}%
\pgfpathlineto{\pgfqpoint{1.131695in}{3.461999in}}%
\pgfpathlineto{\pgfqpoint{1.153456in}{3.460607in}}%
\pgfpathlineto{\pgfqpoint{1.175218in}{3.459299in}}%
\pgfpathlineto{\pgfqpoint{1.196980in}{3.457974in}}%
\pgfpathlineto{\pgfqpoint{1.218741in}{3.456760in}}%
\pgfpathlineto{\pgfqpoint{1.240503in}{3.455670in}}%
\pgfpathlineto{\pgfqpoint{1.262265in}{3.454339in}}%
\pgfpathlineto{\pgfqpoint{1.284026in}{3.452801in}}%
\pgfpathlineto{\pgfqpoint{1.305788in}{3.451442in}}%
\pgfpathlineto{\pgfqpoint{1.327550in}{3.450244in}}%
\pgfpathlineto{\pgfqpoint{1.349311in}{3.448989in}}%
\pgfpathlineto{\pgfqpoint{1.371073in}{3.447680in}}%
\pgfpathlineto{\pgfqpoint{1.392835in}{3.446405in}}%
\pgfpathlineto{\pgfqpoint{1.414596in}{3.445025in}}%
\pgfpathlineto{\pgfqpoint{1.436358in}{3.443928in}}%
\pgfpathlineto{\pgfqpoint{1.458120in}{3.442898in}}%
\pgfpathlineto{\pgfqpoint{1.479881in}{3.441785in}}%
\pgfpathlineto{\pgfqpoint{1.501643in}{3.440861in}}%
\pgfpathlineto{\pgfqpoint{1.523405in}{3.439797in}}%
\pgfpathlineto{\pgfqpoint{1.545166in}{3.438769in}}%
\pgfpathlineto{\pgfqpoint{1.566928in}{3.437695in}}%
\pgfpathlineto{\pgfqpoint{1.588690in}{3.436597in}}%
\pgfpathlineto{\pgfqpoint{1.610451in}{3.435670in}}%
\pgfpathlineto{\pgfqpoint{1.632213in}{3.434684in}}%
\pgfpathlineto{\pgfqpoint{1.653975in}{3.433844in}}%
\pgfpathlineto{\pgfqpoint{1.675736in}{3.432868in}}%
\pgfpathlineto{\pgfqpoint{1.697498in}{3.431831in}}%
\pgfpathlineto{\pgfqpoint{1.719260in}{3.430954in}}%
\pgfpathlineto{\pgfqpoint{1.741021in}{3.430009in}}%
\pgfpathlineto{\pgfqpoint{1.762783in}{3.428912in}}%
\pgfpathlineto{\pgfqpoint{1.784545in}{3.427935in}}%
\pgfpathlineto{\pgfqpoint{1.806306in}{3.427103in}}%
\pgfpathlineto{\pgfqpoint{1.828068in}{3.426323in}}%
\pgfpathlineto{\pgfqpoint{1.849830in}{3.425419in}}%
\pgfpathlineto{\pgfqpoint{1.871591in}{3.424503in}}%
\pgfpathlineto{\pgfqpoint{1.893353in}{3.423752in}}%
\pgfpathlineto{\pgfqpoint{1.915115in}{3.422828in}}%
\pgfpathlineto{\pgfqpoint{1.936876in}{3.421933in}}%
\pgfpathlineto{\pgfqpoint{1.958638in}{3.420925in}}%
\pgfpathlineto{\pgfqpoint{1.980400in}{3.420018in}}%
\pgfpathlineto{\pgfqpoint{2.002161in}{3.419101in}}%
\pgfpathlineto{\pgfqpoint{2.023923in}{3.418195in}}%
\pgfpathlineto{\pgfqpoint{2.045685in}{3.417319in}}%
\pgfpathlineto{\pgfqpoint{2.067446in}{3.416526in}}%
\pgfpathlineto{\pgfqpoint{2.089208in}{3.415774in}}%
\pgfpathlineto{\pgfqpoint{2.110970in}{3.414966in}}%
\pgfpathlineto{\pgfqpoint{2.132731in}{3.414024in}}%
\pgfpathlineto{\pgfqpoint{2.154493in}{3.413239in}}%
\pgfpathlineto{\pgfqpoint{2.176255in}{3.412379in}}%
\pgfpathlineto{\pgfqpoint{2.198016in}{3.411462in}}%
\pgfpathlineto{\pgfqpoint{2.219778in}{3.410712in}}%
\pgfpathlineto{\pgfqpoint{2.241540in}{3.409897in}}%
\pgfpathlineto{\pgfqpoint{2.263301in}{3.409124in}}%
\pgfpathlineto{\pgfqpoint{2.285063in}{3.408374in}}%
\pgfpathlineto{\pgfqpoint{2.306825in}{3.407569in}}%
\pgfpathlineto{\pgfqpoint{2.328586in}{3.406787in}}%
\pgfpathlineto{\pgfqpoint{2.350348in}{3.405936in}}%
\pgfpathlineto{\pgfqpoint{2.372110in}{3.405241in}}%
\pgfpathlineto{\pgfqpoint{2.393871in}{3.404479in}}%
\pgfpathlineto{\pgfqpoint{2.415633in}{3.403721in}}%
\pgfpathlineto{\pgfqpoint{2.437395in}{3.403034in}}%
\pgfpathlineto{\pgfqpoint{2.459156in}{3.402329in}}%
\pgfpathlineto{\pgfqpoint{2.480918in}{3.401583in}}%
\pgfpathlineto{\pgfqpoint{2.502680in}{3.400801in}}%
\pgfpathlineto{\pgfqpoint{2.524441in}{3.400046in}}%
\pgfpathlineto{\pgfqpoint{2.546203in}{3.399335in}}%
\pgfpathlineto{\pgfqpoint{2.567965in}{3.398729in}}%
\pgfpathlineto{\pgfqpoint{2.589726in}{3.397950in}}%
\pgfpathlineto{\pgfqpoint{2.611488in}{3.397142in}}%
\pgfpathlineto{\pgfqpoint{2.633250in}{3.396378in}}%
\pgfpathlineto{\pgfqpoint{2.655011in}{3.395634in}}%
\pgfpathlineto{\pgfqpoint{2.676773in}{3.394959in}}%
\pgfpathlineto{\pgfqpoint{2.698535in}{3.394322in}}%
\pgfpathlineto{\pgfqpoint{2.720296in}{3.393674in}}%
\pgfpathlineto{\pgfqpoint{2.742058in}{3.392986in}}%
\pgfpathlineto{\pgfqpoint{2.763820in}{3.392277in}}%
\pgfpathlineto{\pgfqpoint{2.785581in}{3.391611in}}%
\pgfusepath{stroke}%
\end{pgfscope}%
\begin{pgfscope}%
\pgfsetrectcap%
\pgfsetmiterjoin%
\pgfsetlinewidth{0.501875pt}%
\definecolor{currentstroke}{rgb}{0.000000,0.000000,0.000000}%
\pgfsetstrokecolor{currentstroke}%
\pgfsetdash{}{0pt}%
\pgfpathmoveto{\pgfqpoint{0.609415in}{2.347992in}}%
\pgfpathlineto{\pgfqpoint{0.609415in}{3.574193in}}%
\pgfusepath{stroke}%
\end{pgfscope}%
\begin{pgfscope}%
\pgfsetrectcap%
\pgfsetmiterjoin%
\pgfsetlinewidth{0.501875pt}%
\definecolor{currentstroke}{rgb}{0.000000,0.000000,0.000000}%
\pgfsetstrokecolor{currentstroke}%
\pgfsetdash{}{0pt}%
\pgfpathmoveto{\pgfqpoint{2.829105in}{2.347992in}}%
\pgfpathlineto{\pgfqpoint{2.829105in}{3.574193in}}%
\pgfusepath{stroke}%
\end{pgfscope}%
\begin{pgfscope}%
\pgfsetrectcap%
\pgfsetmiterjoin%
\pgfsetlinewidth{0.501875pt}%
\definecolor{currentstroke}{rgb}{0.000000,0.000000,0.000000}%
\pgfsetstrokecolor{currentstroke}%
\pgfsetdash{}{0pt}%
\pgfpathmoveto{\pgfqpoint{0.609415in}{2.347992in}}%
\pgfpathlineto{\pgfqpoint{2.829105in}{2.347992in}}%
\pgfusepath{stroke}%
\end{pgfscope}%
\begin{pgfscope}%
\pgfsetrectcap%
\pgfsetmiterjoin%
\pgfsetlinewidth{0.501875pt}%
\definecolor{currentstroke}{rgb}{0.000000,0.000000,0.000000}%
\pgfsetstrokecolor{currentstroke}%
\pgfsetdash{}{0pt}%
\pgfpathmoveto{\pgfqpoint{0.609415in}{3.574193in}}%
\pgfpathlineto{\pgfqpoint{2.829105in}{3.574193in}}%
\pgfusepath{stroke}%
\end{pgfscope}%
\begin{pgfscope}%
\definecolor{textcolor}{rgb}{0.000000,0.000000,0.000000}%
\pgfsetstrokecolor{textcolor}%
\pgfsetfillcolor{textcolor}%
\pgftext[x=1.719260in,y=3.657526in,,base]{\color{textcolor}\rmfamily\fontsize{12.000000}{14.400000}\selectfont Vertrauenswürdigkeit}%
\end{pgfscope}%
\begin{pgfscope}%
\pgfsetbuttcap%
\pgfsetmiterjoin%
\definecolor{currentfill}{rgb}{1.000000,1.000000,1.000000}%
\pgfsetfillcolor{currentfill}%
\pgfsetlinewidth{0.000000pt}%
\definecolor{currentstroke}{rgb}{0.000000,0.000000,0.000000}%
\pgfsetstrokecolor{currentstroke}%
\pgfsetstrokeopacity{0.000000}%
\pgfsetdash{}{0pt}%
\pgfpathmoveto{\pgfqpoint{0.609415in}{0.422992in}}%
\pgfpathlineto{\pgfqpoint{2.829105in}{0.422992in}}%
\pgfpathlineto{\pgfqpoint{2.829105in}{1.649193in}}%
\pgfpathlineto{\pgfqpoint{0.609415in}{1.649193in}}%
\pgfpathlineto{\pgfqpoint{0.609415in}{0.422992in}}%
\pgfpathclose%
\pgfusepath{fill}%
\end{pgfscope}%
\begin{pgfscope}%
\pgfsetbuttcap%
\pgfsetroundjoin%
\definecolor{currentfill}{rgb}{0.000000,0.000000,0.000000}%
\pgfsetfillcolor{currentfill}%
\pgfsetlinewidth{0.501875pt}%
\definecolor{currentstroke}{rgb}{0.000000,0.000000,0.000000}%
\pgfsetstrokecolor{currentstroke}%
\pgfsetdash{}{0pt}%
\pgfsys@defobject{currentmarker}{\pgfqpoint{0.000000in}{0.000000in}}{\pgfqpoint{0.000000in}{0.041667in}}{%
\pgfpathmoveto{\pgfqpoint{0.000000in}{0.000000in}}%
\pgfpathlineto{\pgfqpoint{0.000000in}{0.041667in}}%
\pgfusepath{stroke,fill}%
}%
\begin{pgfscope}%
\pgfsys@transformshift{0.609415in}{0.422992in}%
\pgfsys@useobject{currentmarker}{}%
\end{pgfscope}%
\end{pgfscope}%
\begin{pgfscope}%
\pgfsetbuttcap%
\pgfsetroundjoin%
\definecolor{currentfill}{rgb}{0.000000,0.000000,0.000000}%
\pgfsetfillcolor{currentfill}%
\pgfsetlinewidth{0.501875pt}%
\definecolor{currentstroke}{rgb}{0.000000,0.000000,0.000000}%
\pgfsetstrokecolor{currentstroke}%
\pgfsetdash{}{0pt}%
\pgfsys@defobject{currentmarker}{\pgfqpoint{0.000000in}{-0.041667in}}{\pgfqpoint{0.000000in}{0.000000in}}{%
\pgfpathmoveto{\pgfqpoint{0.000000in}{0.000000in}}%
\pgfpathlineto{\pgfqpoint{0.000000in}{-0.041667in}}%
\pgfusepath{stroke,fill}%
}%
\begin{pgfscope}%
\pgfsys@transformshift{0.609415in}{1.649193in}%
\pgfsys@useobject{currentmarker}{}%
\end{pgfscope}%
\end{pgfscope}%
\begin{pgfscope}%
\definecolor{textcolor}{rgb}{0.000000,0.000000,0.000000}%
\pgfsetstrokecolor{textcolor}%
\pgfsetfillcolor{textcolor}%
\pgftext[x=0.609415in,y=0.374381in,,top]{\color{textcolor}\rmfamily\fontsize{10.000000}{12.000000}\selectfont \(\displaystyle {0}\)}%
\end{pgfscope}%
\begin{pgfscope}%
\pgfsetbuttcap%
\pgfsetroundjoin%
\definecolor{currentfill}{rgb}{0.000000,0.000000,0.000000}%
\pgfsetfillcolor{currentfill}%
\pgfsetlinewidth{0.501875pt}%
\definecolor{currentstroke}{rgb}{0.000000,0.000000,0.000000}%
\pgfsetstrokecolor{currentstroke}%
\pgfsetdash{}{0pt}%
\pgfsys@defobject{currentmarker}{\pgfqpoint{0.000000in}{0.000000in}}{\pgfqpoint{0.000000in}{0.041667in}}{%
\pgfpathmoveto{\pgfqpoint{0.000000in}{0.000000in}}%
\pgfpathlineto{\pgfqpoint{0.000000in}{0.041667in}}%
\pgfusepath{stroke,fill}%
}%
\begin{pgfscope}%
\pgfsys@transformshift{1.044648in}{0.422992in}%
\pgfsys@useobject{currentmarker}{}%
\end{pgfscope}%
\end{pgfscope}%
\begin{pgfscope}%
\pgfsetbuttcap%
\pgfsetroundjoin%
\definecolor{currentfill}{rgb}{0.000000,0.000000,0.000000}%
\pgfsetfillcolor{currentfill}%
\pgfsetlinewidth{0.501875pt}%
\definecolor{currentstroke}{rgb}{0.000000,0.000000,0.000000}%
\pgfsetstrokecolor{currentstroke}%
\pgfsetdash{}{0pt}%
\pgfsys@defobject{currentmarker}{\pgfqpoint{0.000000in}{-0.041667in}}{\pgfqpoint{0.000000in}{0.000000in}}{%
\pgfpathmoveto{\pgfqpoint{0.000000in}{0.000000in}}%
\pgfpathlineto{\pgfqpoint{0.000000in}{-0.041667in}}%
\pgfusepath{stroke,fill}%
}%
\begin{pgfscope}%
\pgfsys@transformshift{1.044648in}{1.649193in}%
\pgfsys@useobject{currentmarker}{}%
\end{pgfscope}%
\end{pgfscope}%
\begin{pgfscope}%
\definecolor{textcolor}{rgb}{0.000000,0.000000,0.000000}%
\pgfsetstrokecolor{textcolor}%
\pgfsetfillcolor{textcolor}%
\pgftext[x=1.044648in,y=0.374381in,,top]{\color{textcolor}\rmfamily\fontsize{10.000000}{12.000000}\selectfont \(\displaystyle {20}\)}%
\end{pgfscope}%
\begin{pgfscope}%
\pgfsetbuttcap%
\pgfsetroundjoin%
\definecolor{currentfill}{rgb}{0.000000,0.000000,0.000000}%
\pgfsetfillcolor{currentfill}%
\pgfsetlinewidth{0.501875pt}%
\definecolor{currentstroke}{rgb}{0.000000,0.000000,0.000000}%
\pgfsetstrokecolor{currentstroke}%
\pgfsetdash{}{0pt}%
\pgfsys@defobject{currentmarker}{\pgfqpoint{0.000000in}{0.000000in}}{\pgfqpoint{0.000000in}{0.041667in}}{%
\pgfpathmoveto{\pgfqpoint{0.000000in}{0.000000in}}%
\pgfpathlineto{\pgfqpoint{0.000000in}{0.041667in}}%
\pgfusepath{stroke,fill}%
}%
\begin{pgfscope}%
\pgfsys@transformshift{1.479881in}{0.422992in}%
\pgfsys@useobject{currentmarker}{}%
\end{pgfscope}%
\end{pgfscope}%
\begin{pgfscope}%
\pgfsetbuttcap%
\pgfsetroundjoin%
\definecolor{currentfill}{rgb}{0.000000,0.000000,0.000000}%
\pgfsetfillcolor{currentfill}%
\pgfsetlinewidth{0.501875pt}%
\definecolor{currentstroke}{rgb}{0.000000,0.000000,0.000000}%
\pgfsetstrokecolor{currentstroke}%
\pgfsetdash{}{0pt}%
\pgfsys@defobject{currentmarker}{\pgfqpoint{0.000000in}{-0.041667in}}{\pgfqpoint{0.000000in}{0.000000in}}{%
\pgfpathmoveto{\pgfqpoint{0.000000in}{0.000000in}}%
\pgfpathlineto{\pgfqpoint{0.000000in}{-0.041667in}}%
\pgfusepath{stroke,fill}%
}%
\begin{pgfscope}%
\pgfsys@transformshift{1.479881in}{1.649193in}%
\pgfsys@useobject{currentmarker}{}%
\end{pgfscope}%
\end{pgfscope}%
\begin{pgfscope}%
\definecolor{textcolor}{rgb}{0.000000,0.000000,0.000000}%
\pgfsetstrokecolor{textcolor}%
\pgfsetfillcolor{textcolor}%
\pgftext[x=1.479881in,y=0.374381in,,top]{\color{textcolor}\rmfamily\fontsize{10.000000}{12.000000}\selectfont \(\displaystyle {40}\)}%
\end{pgfscope}%
\begin{pgfscope}%
\pgfsetbuttcap%
\pgfsetroundjoin%
\definecolor{currentfill}{rgb}{0.000000,0.000000,0.000000}%
\pgfsetfillcolor{currentfill}%
\pgfsetlinewidth{0.501875pt}%
\definecolor{currentstroke}{rgb}{0.000000,0.000000,0.000000}%
\pgfsetstrokecolor{currentstroke}%
\pgfsetdash{}{0pt}%
\pgfsys@defobject{currentmarker}{\pgfqpoint{0.000000in}{0.000000in}}{\pgfqpoint{0.000000in}{0.041667in}}{%
\pgfpathmoveto{\pgfqpoint{0.000000in}{0.000000in}}%
\pgfpathlineto{\pgfqpoint{0.000000in}{0.041667in}}%
\pgfusepath{stroke,fill}%
}%
\begin{pgfscope}%
\pgfsys@transformshift{1.915115in}{0.422992in}%
\pgfsys@useobject{currentmarker}{}%
\end{pgfscope}%
\end{pgfscope}%
\begin{pgfscope}%
\pgfsetbuttcap%
\pgfsetroundjoin%
\definecolor{currentfill}{rgb}{0.000000,0.000000,0.000000}%
\pgfsetfillcolor{currentfill}%
\pgfsetlinewidth{0.501875pt}%
\definecolor{currentstroke}{rgb}{0.000000,0.000000,0.000000}%
\pgfsetstrokecolor{currentstroke}%
\pgfsetdash{}{0pt}%
\pgfsys@defobject{currentmarker}{\pgfqpoint{0.000000in}{-0.041667in}}{\pgfqpoint{0.000000in}{0.000000in}}{%
\pgfpathmoveto{\pgfqpoint{0.000000in}{0.000000in}}%
\pgfpathlineto{\pgfqpoint{0.000000in}{-0.041667in}}%
\pgfusepath{stroke,fill}%
}%
\begin{pgfscope}%
\pgfsys@transformshift{1.915115in}{1.649193in}%
\pgfsys@useobject{currentmarker}{}%
\end{pgfscope}%
\end{pgfscope}%
\begin{pgfscope}%
\definecolor{textcolor}{rgb}{0.000000,0.000000,0.000000}%
\pgfsetstrokecolor{textcolor}%
\pgfsetfillcolor{textcolor}%
\pgftext[x=1.915115in,y=0.374381in,,top]{\color{textcolor}\rmfamily\fontsize{10.000000}{12.000000}\selectfont \(\displaystyle {60}\)}%
\end{pgfscope}%
\begin{pgfscope}%
\pgfsetbuttcap%
\pgfsetroundjoin%
\definecolor{currentfill}{rgb}{0.000000,0.000000,0.000000}%
\pgfsetfillcolor{currentfill}%
\pgfsetlinewidth{0.501875pt}%
\definecolor{currentstroke}{rgb}{0.000000,0.000000,0.000000}%
\pgfsetstrokecolor{currentstroke}%
\pgfsetdash{}{0pt}%
\pgfsys@defobject{currentmarker}{\pgfqpoint{0.000000in}{0.000000in}}{\pgfqpoint{0.000000in}{0.041667in}}{%
\pgfpathmoveto{\pgfqpoint{0.000000in}{0.000000in}}%
\pgfpathlineto{\pgfqpoint{0.000000in}{0.041667in}}%
\pgfusepath{stroke,fill}%
}%
\begin{pgfscope}%
\pgfsys@transformshift{2.350348in}{0.422992in}%
\pgfsys@useobject{currentmarker}{}%
\end{pgfscope}%
\end{pgfscope}%
\begin{pgfscope}%
\pgfsetbuttcap%
\pgfsetroundjoin%
\definecolor{currentfill}{rgb}{0.000000,0.000000,0.000000}%
\pgfsetfillcolor{currentfill}%
\pgfsetlinewidth{0.501875pt}%
\definecolor{currentstroke}{rgb}{0.000000,0.000000,0.000000}%
\pgfsetstrokecolor{currentstroke}%
\pgfsetdash{}{0pt}%
\pgfsys@defobject{currentmarker}{\pgfqpoint{0.000000in}{-0.041667in}}{\pgfqpoint{0.000000in}{0.000000in}}{%
\pgfpathmoveto{\pgfqpoint{0.000000in}{0.000000in}}%
\pgfpathlineto{\pgfqpoint{0.000000in}{-0.041667in}}%
\pgfusepath{stroke,fill}%
}%
\begin{pgfscope}%
\pgfsys@transformshift{2.350348in}{1.649193in}%
\pgfsys@useobject{currentmarker}{}%
\end{pgfscope}%
\end{pgfscope}%
\begin{pgfscope}%
\definecolor{textcolor}{rgb}{0.000000,0.000000,0.000000}%
\pgfsetstrokecolor{textcolor}%
\pgfsetfillcolor{textcolor}%
\pgftext[x=2.350348in,y=0.374381in,,top]{\color{textcolor}\rmfamily\fontsize{10.000000}{12.000000}\selectfont \(\displaystyle {80}\)}%
\end{pgfscope}%
\begin{pgfscope}%
\pgfsetbuttcap%
\pgfsetroundjoin%
\definecolor{currentfill}{rgb}{0.000000,0.000000,0.000000}%
\pgfsetfillcolor{currentfill}%
\pgfsetlinewidth{0.501875pt}%
\definecolor{currentstroke}{rgb}{0.000000,0.000000,0.000000}%
\pgfsetstrokecolor{currentstroke}%
\pgfsetdash{}{0pt}%
\pgfsys@defobject{currentmarker}{\pgfqpoint{0.000000in}{0.000000in}}{\pgfqpoint{0.000000in}{0.041667in}}{%
\pgfpathmoveto{\pgfqpoint{0.000000in}{0.000000in}}%
\pgfpathlineto{\pgfqpoint{0.000000in}{0.041667in}}%
\pgfusepath{stroke,fill}%
}%
\begin{pgfscope}%
\pgfsys@transformshift{2.785581in}{0.422992in}%
\pgfsys@useobject{currentmarker}{}%
\end{pgfscope}%
\end{pgfscope}%
\begin{pgfscope}%
\pgfsetbuttcap%
\pgfsetroundjoin%
\definecolor{currentfill}{rgb}{0.000000,0.000000,0.000000}%
\pgfsetfillcolor{currentfill}%
\pgfsetlinewidth{0.501875pt}%
\definecolor{currentstroke}{rgb}{0.000000,0.000000,0.000000}%
\pgfsetstrokecolor{currentstroke}%
\pgfsetdash{}{0pt}%
\pgfsys@defobject{currentmarker}{\pgfqpoint{0.000000in}{-0.041667in}}{\pgfqpoint{0.000000in}{0.000000in}}{%
\pgfpathmoveto{\pgfqpoint{0.000000in}{0.000000in}}%
\pgfpathlineto{\pgfqpoint{0.000000in}{-0.041667in}}%
\pgfusepath{stroke,fill}%
}%
\begin{pgfscope}%
\pgfsys@transformshift{2.785581in}{1.649193in}%
\pgfsys@useobject{currentmarker}{}%
\end{pgfscope}%
\end{pgfscope}%
\begin{pgfscope}%
\definecolor{textcolor}{rgb}{0.000000,0.000000,0.000000}%
\pgfsetstrokecolor{textcolor}%
\pgfsetfillcolor{textcolor}%
\pgftext[x=2.785581in,y=0.374381in,,top]{\color{textcolor}\rmfamily\fontsize{10.000000}{12.000000}\selectfont \(\displaystyle {100}\)}%
\end{pgfscope}%
\begin{pgfscope}%
\pgfsetbuttcap%
\pgfsetroundjoin%
\definecolor{currentfill}{rgb}{0.000000,0.000000,0.000000}%
\pgfsetfillcolor{currentfill}%
\pgfsetlinewidth{0.501875pt}%
\definecolor{currentstroke}{rgb}{0.000000,0.000000,0.000000}%
\pgfsetstrokecolor{currentstroke}%
\pgfsetdash{}{0pt}%
\pgfsys@defobject{currentmarker}{\pgfqpoint{0.000000in}{0.000000in}}{\pgfqpoint{0.000000in}{0.020833in}}{%
\pgfpathmoveto{\pgfqpoint{0.000000in}{0.000000in}}%
\pgfpathlineto{\pgfqpoint{0.000000in}{0.020833in}}%
\pgfusepath{stroke,fill}%
}%
\begin{pgfscope}%
\pgfsys@transformshift{0.718223in}{0.422992in}%
\pgfsys@useobject{currentmarker}{}%
\end{pgfscope}%
\end{pgfscope}%
\begin{pgfscope}%
\pgfsetbuttcap%
\pgfsetroundjoin%
\definecolor{currentfill}{rgb}{0.000000,0.000000,0.000000}%
\pgfsetfillcolor{currentfill}%
\pgfsetlinewidth{0.501875pt}%
\definecolor{currentstroke}{rgb}{0.000000,0.000000,0.000000}%
\pgfsetstrokecolor{currentstroke}%
\pgfsetdash{}{0pt}%
\pgfsys@defobject{currentmarker}{\pgfqpoint{0.000000in}{-0.020833in}}{\pgfqpoint{0.000000in}{0.000000in}}{%
\pgfpathmoveto{\pgfqpoint{0.000000in}{0.000000in}}%
\pgfpathlineto{\pgfqpoint{0.000000in}{-0.020833in}}%
\pgfusepath{stroke,fill}%
}%
\begin{pgfscope}%
\pgfsys@transformshift{0.718223in}{1.649193in}%
\pgfsys@useobject{currentmarker}{}%
\end{pgfscope}%
\end{pgfscope}%
\begin{pgfscope}%
\pgfsetbuttcap%
\pgfsetroundjoin%
\definecolor{currentfill}{rgb}{0.000000,0.000000,0.000000}%
\pgfsetfillcolor{currentfill}%
\pgfsetlinewidth{0.501875pt}%
\definecolor{currentstroke}{rgb}{0.000000,0.000000,0.000000}%
\pgfsetstrokecolor{currentstroke}%
\pgfsetdash{}{0pt}%
\pgfsys@defobject{currentmarker}{\pgfqpoint{0.000000in}{0.000000in}}{\pgfqpoint{0.000000in}{0.020833in}}{%
\pgfpathmoveto{\pgfqpoint{0.000000in}{0.000000in}}%
\pgfpathlineto{\pgfqpoint{0.000000in}{0.020833in}}%
\pgfusepath{stroke,fill}%
}%
\begin{pgfscope}%
\pgfsys@transformshift{0.827031in}{0.422992in}%
\pgfsys@useobject{currentmarker}{}%
\end{pgfscope}%
\end{pgfscope}%
\begin{pgfscope}%
\pgfsetbuttcap%
\pgfsetroundjoin%
\definecolor{currentfill}{rgb}{0.000000,0.000000,0.000000}%
\pgfsetfillcolor{currentfill}%
\pgfsetlinewidth{0.501875pt}%
\definecolor{currentstroke}{rgb}{0.000000,0.000000,0.000000}%
\pgfsetstrokecolor{currentstroke}%
\pgfsetdash{}{0pt}%
\pgfsys@defobject{currentmarker}{\pgfqpoint{0.000000in}{-0.020833in}}{\pgfqpoint{0.000000in}{0.000000in}}{%
\pgfpathmoveto{\pgfqpoint{0.000000in}{0.000000in}}%
\pgfpathlineto{\pgfqpoint{0.000000in}{-0.020833in}}%
\pgfusepath{stroke,fill}%
}%
\begin{pgfscope}%
\pgfsys@transformshift{0.827031in}{1.649193in}%
\pgfsys@useobject{currentmarker}{}%
\end{pgfscope}%
\end{pgfscope}%
\begin{pgfscope}%
\pgfsetbuttcap%
\pgfsetroundjoin%
\definecolor{currentfill}{rgb}{0.000000,0.000000,0.000000}%
\pgfsetfillcolor{currentfill}%
\pgfsetlinewidth{0.501875pt}%
\definecolor{currentstroke}{rgb}{0.000000,0.000000,0.000000}%
\pgfsetstrokecolor{currentstroke}%
\pgfsetdash{}{0pt}%
\pgfsys@defobject{currentmarker}{\pgfqpoint{0.000000in}{0.000000in}}{\pgfqpoint{0.000000in}{0.020833in}}{%
\pgfpathmoveto{\pgfqpoint{0.000000in}{0.000000in}}%
\pgfpathlineto{\pgfqpoint{0.000000in}{0.020833in}}%
\pgfusepath{stroke,fill}%
}%
\begin{pgfscope}%
\pgfsys@transformshift{0.935840in}{0.422992in}%
\pgfsys@useobject{currentmarker}{}%
\end{pgfscope}%
\end{pgfscope}%
\begin{pgfscope}%
\pgfsetbuttcap%
\pgfsetroundjoin%
\definecolor{currentfill}{rgb}{0.000000,0.000000,0.000000}%
\pgfsetfillcolor{currentfill}%
\pgfsetlinewidth{0.501875pt}%
\definecolor{currentstroke}{rgb}{0.000000,0.000000,0.000000}%
\pgfsetstrokecolor{currentstroke}%
\pgfsetdash{}{0pt}%
\pgfsys@defobject{currentmarker}{\pgfqpoint{0.000000in}{-0.020833in}}{\pgfqpoint{0.000000in}{0.000000in}}{%
\pgfpathmoveto{\pgfqpoint{0.000000in}{0.000000in}}%
\pgfpathlineto{\pgfqpoint{0.000000in}{-0.020833in}}%
\pgfusepath{stroke,fill}%
}%
\begin{pgfscope}%
\pgfsys@transformshift{0.935840in}{1.649193in}%
\pgfsys@useobject{currentmarker}{}%
\end{pgfscope}%
\end{pgfscope}%
\begin{pgfscope}%
\pgfsetbuttcap%
\pgfsetroundjoin%
\definecolor{currentfill}{rgb}{0.000000,0.000000,0.000000}%
\pgfsetfillcolor{currentfill}%
\pgfsetlinewidth{0.501875pt}%
\definecolor{currentstroke}{rgb}{0.000000,0.000000,0.000000}%
\pgfsetstrokecolor{currentstroke}%
\pgfsetdash{}{0pt}%
\pgfsys@defobject{currentmarker}{\pgfqpoint{0.000000in}{0.000000in}}{\pgfqpoint{0.000000in}{0.020833in}}{%
\pgfpathmoveto{\pgfqpoint{0.000000in}{0.000000in}}%
\pgfpathlineto{\pgfqpoint{0.000000in}{0.020833in}}%
\pgfusepath{stroke,fill}%
}%
\begin{pgfscope}%
\pgfsys@transformshift{1.153456in}{0.422992in}%
\pgfsys@useobject{currentmarker}{}%
\end{pgfscope}%
\end{pgfscope}%
\begin{pgfscope}%
\pgfsetbuttcap%
\pgfsetroundjoin%
\definecolor{currentfill}{rgb}{0.000000,0.000000,0.000000}%
\pgfsetfillcolor{currentfill}%
\pgfsetlinewidth{0.501875pt}%
\definecolor{currentstroke}{rgb}{0.000000,0.000000,0.000000}%
\pgfsetstrokecolor{currentstroke}%
\pgfsetdash{}{0pt}%
\pgfsys@defobject{currentmarker}{\pgfqpoint{0.000000in}{-0.020833in}}{\pgfqpoint{0.000000in}{0.000000in}}{%
\pgfpathmoveto{\pgfqpoint{0.000000in}{0.000000in}}%
\pgfpathlineto{\pgfqpoint{0.000000in}{-0.020833in}}%
\pgfusepath{stroke,fill}%
}%
\begin{pgfscope}%
\pgfsys@transformshift{1.153456in}{1.649193in}%
\pgfsys@useobject{currentmarker}{}%
\end{pgfscope}%
\end{pgfscope}%
\begin{pgfscope}%
\pgfsetbuttcap%
\pgfsetroundjoin%
\definecolor{currentfill}{rgb}{0.000000,0.000000,0.000000}%
\pgfsetfillcolor{currentfill}%
\pgfsetlinewidth{0.501875pt}%
\definecolor{currentstroke}{rgb}{0.000000,0.000000,0.000000}%
\pgfsetstrokecolor{currentstroke}%
\pgfsetdash{}{0pt}%
\pgfsys@defobject{currentmarker}{\pgfqpoint{0.000000in}{0.000000in}}{\pgfqpoint{0.000000in}{0.020833in}}{%
\pgfpathmoveto{\pgfqpoint{0.000000in}{0.000000in}}%
\pgfpathlineto{\pgfqpoint{0.000000in}{0.020833in}}%
\pgfusepath{stroke,fill}%
}%
\begin{pgfscope}%
\pgfsys@transformshift{1.262265in}{0.422992in}%
\pgfsys@useobject{currentmarker}{}%
\end{pgfscope}%
\end{pgfscope}%
\begin{pgfscope}%
\pgfsetbuttcap%
\pgfsetroundjoin%
\definecolor{currentfill}{rgb}{0.000000,0.000000,0.000000}%
\pgfsetfillcolor{currentfill}%
\pgfsetlinewidth{0.501875pt}%
\definecolor{currentstroke}{rgb}{0.000000,0.000000,0.000000}%
\pgfsetstrokecolor{currentstroke}%
\pgfsetdash{}{0pt}%
\pgfsys@defobject{currentmarker}{\pgfqpoint{0.000000in}{-0.020833in}}{\pgfqpoint{0.000000in}{0.000000in}}{%
\pgfpathmoveto{\pgfqpoint{0.000000in}{0.000000in}}%
\pgfpathlineto{\pgfqpoint{0.000000in}{-0.020833in}}%
\pgfusepath{stroke,fill}%
}%
\begin{pgfscope}%
\pgfsys@transformshift{1.262265in}{1.649193in}%
\pgfsys@useobject{currentmarker}{}%
\end{pgfscope}%
\end{pgfscope}%
\begin{pgfscope}%
\pgfsetbuttcap%
\pgfsetroundjoin%
\definecolor{currentfill}{rgb}{0.000000,0.000000,0.000000}%
\pgfsetfillcolor{currentfill}%
\pgfsetlinewidth{0.501875pt}%
\definecolor{currentstroke}{rgb}{0.000000,0.000000,0.000000}%
\pgfsetstrokecolor{currentstroke}%
\pgfsetdash{}{0pt}%
\pgfsys@defobject{currentmarker}{\pgfqpoint{0.000000in}{0.000000in}}{\pgfqpoint{0.000000in}{0.020833in}}{%
\pgfpathmoveto{\pgfqpoint{0.000000in}{0.000000in}}%
\pgfpathlineto{\pgfqpoint{0.000000in}{0.020833in}}%
\pgfusepath{stroke,fill}%
}%
\begin{pgfscope}%
\pgfsys@transformshift{1.371073in}{0.422992in}%
\pgfsys@useobject{currentmarker}{}%
\end{pgfscope}%
\end{pgfscope}%
\begin{pgfscope}%
\pgfsetbuttcap%
\pgfsetroundjoin%
\definecolor{currentfill}{rgb}{0.000000,0.000000,0.000000}%
\pgfsetfillcolor{currentfill}%
\pgfsetlinewidth{0.501875pt}%
\definecolor{currentstroke}{rgb}{0.000000,0.000000,0.000000}%
\pgfsetstrokecolor{currentstroke}%
\pgfsetdash{}{0pt}%
\pgfsys@defobject{currentmarker}{\pgfqpoint{0.000000in}{-0.020833in}}{\pgfqpoint{0.000000in}{0.000000in}}{%
\pgfpathmoveto{\pgfqpoint{0.000000in}{0.000000in}}%
\pgfpathlineto{\pgfqpoint{0.000000in}{-0.020833in}}%
\pgfusepath{stroke,fill}%
}%
\begin{pgfscope}%
\pgfsys@transformshift{1.371073in}{1.649193in}%
\pgfsys@useobject{currentmarker}{}%
\end{pgfscope}%
\end{pgfscope}%
\begin{pgfscope}%
\pgfsetbuttcap%
\pgfsetroundjoin%
\definecolor{currentfill}{rgb}{0.000000,0.000000,0.000000}%
\pgfsetfillcolor{currentfill}%
\pgfsetlinewidth{0.501875pt}%
\definecolor{currentstroke}{rgb}{0.000000,0.000000,0.000000}%
\pgfsetstrokecolor{currentstroke}%
\pgfsetdash{}{0pt}%
\pgfsys@defobject{currentmarker}{\pgfqpoint{0.000000in}{0.000000in}}{\pgfqpoint{0.000000in}{0.020833in}}{%
\pgfpathmoveto{\pgfqpoint{0.000000in}{0.000000in}}%
\pgfpathlineto{\pgfqpoint{0.000000in}{0.020833in}}%
\pgfusepath{stroke,fill}%
}%
\begin{pgfscope}%
\pgfsys@transformshift{1.588690in}{0.422992in}%
\pgfsys@useobject{currentmarker}{}%
\end{pgfscope}%
\end{pgfscope}%
\begin{pgfscope}%
\pgfsetbuttcap%
\pgfsetroundjoin%
\definecolor{currentfill}{rgb}{0.000000,0.000000,0.000000}%
\pgfsetfillcolor{currentfill}%
\pgfsetlinewidth{0.501875pt}%
\definecolor{currentstroke}{rgb}{0.000000,0.000000,0.000000}%
\pgfsetstrokecolor{currentstroke}%
\pgfsetdash{}{0pt}%
\pgfsys@defobject{currentmarker}{\pgfqpoint{0.000000in}{-0.020833in}}{\pgfqpoint{0.000000in}{0.000000in}}{%
\pgfpathmoveto{\pgfqpoint{0.000000in}{0.000000in}}%
\pgfpathlineto{\pgfqpoint{0.000000in}{-0.020833in}}%
\pgfusepath{stroke,fill}%
}%
\begin{pgfscope}%
\pgfsys@transformshift{1.588690in}{1.649193in}%
\pgfsys@useobject{currentmarker}{}%
\end{pgfscope}%
\end{pgfscope}%
\begin{pgfscope}%
\pgfsetbuttcap%
\pgfsetroundjoin%
\definecolor{currentfill}{rgb}{0.000000,0.000000,0.000000}%
\pgfsetfillcolor{currentfill}%
\pgfsetlinewidth{0.501875pt}%
\definecolor{currentstroke}{rgb}{0.000000,0.000000,0.000000}%
\pgfsetstrokecolor{currentstroke}%
\pgfsetdash{}{0pt}%
\pgfsys@defobject{currentmarker}{\pgfqpoint{0.000000in}{0.000000in}}{\pgfqpoint{0.000000in}{0.020833in}}{%
\pgfpathmoveto{\pgfqpoint{0.000000in}{0.000000in}}%
\pgfpathlineto{\pgfqpoint{0.000000in}{0.020833in}}%
\pgfusepath{stroke,fill}%
}%
\begin{pgfscope}%
\pgfsys@transformshift{1.697498in}{0.422992in}%
\pgfsys@useobject{currentmarker}{}%
\end{pgfscope}%
\end{pgfscope}%
\begin{pgfscope}%
\pgfsetbuttcap%
\pgfsetroundjoin%
\definecolor{currentfill}{rgb}{0.000000,0.000000,0.000000}%
\pgfsetfillcolor{currentfill}%
\pgfsetlinewidth{0.501875pt}%
\definecolor{currentstroke}{rgb}{0.000000,0.000000,0.000000}%
\pgfsetstrokecolor{currentstroke}%
\pgfsetdash{}{0pt}%
\pgfsys@defobject{currentmarker}{\pgfqpoint{0.000000in}{-0.020833in}}{\pgfqpoint{0.000000in}{0.000000in}}{%
\pgfpathmoveto{\pgfqpoint{0.000000in}{0.000000in}}%
\pgfpathlineto{\pgfqpoint{0.000000in}{-0.020833in}}%
\pgfusepath{stroke,fill}%
}%
\begin{pgfscope}%
\pgfsys@transformshift{1.697498in}{1.649193in}%
\pgfsys@useobject{currentmarker}{}%
\end{pgfscope}%
\end{pgfscope}%
\begin{pgfscope}%
\pgfsetbuttcap%
\pgfsetroundjoin%
\definecolor{currentfill}{rgb}{0.000000,0.000000,0.000000}%
\pgfsetfillcolor{currentfill}%
\pgfsetlinewidth{0.501875pt}%
\definecolor{currentstroke}{rgb}{0.000000,0.000000,0.000000}%
\pgfsetstrokecolor{currentstroke}%
\pgfsetdash{}{0pt}%
\pgfsys@defobject{currentmarker}{\pgfqpoint{0.000000in}{0.000000in}}{\pgfqpoint{0.000000in}{0.020833in}}{%
\pgfpathmoveto{\pgfqpoint{0.000000in}{0.000000in}}%
\pgfpathlineto{\pgfqpoint{0.000000in}{0.020833in}}%
\pgfusepath{stroke,fill}%
}%
\begin{pgfscope}%
\pgfsys@transformshift{1.806306in}{0.422992in}%
\pgfsys@useobject{currentmarker}{}%
\end{pgfscope}%
\end{pgfscope}%
\begin{pgfscope}%
\pgfsetbuttcap%
\pgfsetroundjoin%
\definecolor{currentfill}{rgb}{0.000000,0.000000,0.000000}%
\pgfsetfillcolor{currentfill}%
\pgfsetlinewidth{0.501875pt}%
\definecolor{currentstroke}{rgb}{0.000000,0.000000,0.000000}%
\pgfsetstrokecolor{currentstroke}%
\pgfsetdash{}{0pt}%
\pgfsys@defobject{currentmarker}{\pgfqpoint{0.000000in}{-0.020833in}}{\pgfqpoint{0.000000in}{0.000000in}}{%
\pgfpathmoveto{\pgfqpoint{0.000000in}{0.000000in}}%
\pgfpathlineto{\pgfqpoint{0.000000in}{-0.020833in}}%
\pgfusepath{stroke,fill}%
}%
\begin{pgfscope}%
\pgfsys@transformshift{1.806306in}{1.649193in}%
\pgfsys@useobject{currentmarker}{}%
\end{pgfscope}%
\end{pgfscope}%
\begin{pgfscope}%
\pgfsetbuttcap%
\pgfsetroundjoin%
\definecolor{currentfill}{rgb}{0.000000,0.000000,0.000000}%
\pgfsetfillcolor{currentfill}%
\pgfsetlinewidth{0.501875pt}%
\definecolor{currentstroke}{rgb}{0.000000,0.000000,0.000000}%
\pgfsetstrokecolor{currentstroke}%
\pgfsetdash{}{0pt}%
\pgfsys@defobject{currentmarker}{\pgfqpoint{0.000000in}{0.000000in}}{\pgfqpoint{0.000000in}{0.020833in}}{%
\pgfpathmoveto{\pgfqpoint{0.000000in}{0.000000in}}%
\pgfpathlineto{\pgfqpoint{0.000000in}{0.020833in}}%
\pgfusepath{stroke,fill}%
}%
\begin{pgfscope}%
\pgfsys@transformshift{2.023923in}{0.422992in}%
\pgfsys@useobject{currentmarker}{}%
\end{pgfscope}%
\end{pgfscope}%
\begin{pgfscope}%
\pgfsetbuttcap%
\pgfsetroundjoin%
\definecolor{currentfill}{rgb}{0.000000,0.000000,0.000000}%
\pgfsetfillcolor{currentfill}%
\pgfsetlinewidth{0.501875pt}%
\definecolor{currentstroke}{rgb}{0.000000,0.000000,0.000000}%
\pgfsetstrokecolor{currentstroke}%
\pgfsetdash{}{0pt}%
\pgfsys@defobject{currentmarker}{\pgfqpoint{0.000000in}{-0.020833in}}{\pgfqpoint{0.000000in}{0.000000in}}{%
\pgfpathmoveto{\pgfqpoint{0.000000in}{0.000000in}}%
\pgfpathlineto{\pgfqpoint{0.000000in}{-0.020833in}}%
\pgfusepath{stroke,fill}%
}%
\begin{pgfscope}%
\pgfsys@transformshift{2.023923in}{1.649193in}%
\pgfsys@useobject{currentmarker}{}%
\end{pgfscope}%
\end{pgfscope}%
\begin{pgfscope}%
\pgfsetbuttcap%
\pgfsetroundjoin%
\definecolor{currentfill}{rgb}{0.000000,0.000000,0.000000}%
\pgfsetfillcolor{currentfill}%
\pgfsetlinewidth{0.501875pt}%
\definecolor{currentstroke}{rgb}{0.000000,0.000000,0.000000}%
\pgfsetstrokecolor{currentstroke}%
\pgfsetdash{}{0pt}%
\pgfsys@defobject{currentmarker}{\pgfqpoint{0.000000in}{0.000000in}}{\pgfqpoint{0.000000in}{0.020833in}}{%
\pgfpathmoveto{\pgfqpoint{0.000000in}{0.000000in}}%
\pgfpathlineto{\pgfqpoint{0.000000in}{0.020833in}}%
\pgfusepath{stroke,fill}%
}%
\begin{pgfscope}%
\pgfsys@transformshift{2.132731in}{0.422992in}%
\pgfsys@useobject{currentmarker}{}%
\end{pgfscope}%
\end{pgfscope}%
\begin{pgfscope}%
\pgfsetbuttcap%
\pgfsetroundjoin%
\definecolor{currentfill}{rgb}{0.000000,0.000000,0.000000}%
\pgfsetfillcolor{currentfill}%
\pgfsetlinewidth{0.501875pt}%
\definecolor{currentstroke}{rgb}{0.000000,0.000000,0.000000}%
\pgfsetstrokecolor{currentstroke}%
\pgfsetdash{}{0pt}%
\pgfsys@defobject{currentmarker}{\pgfqpoint{0.000000in}{-0.020833in}}{\pgfqpoint{0.000000in}{0.000000in}}{%
\pgfpathmoveto{\pgfqpoint{0.000000in}{0.000000in}}%
\pgfpathlineto{\pgfqpoint{0.000000in}{-0.020833in}}%
\pgfusepath{stroke,fill}%
}%
\begin{pgfscope}%
\pgfsys@transformshift{2.132731in}{1.649193in}%
\pgfsys@useobject{currentmarker}{}%
\end{pgfscope}%
\end{pgfscope}%
\begin{pgfscope}%
\pgfsetbuttcap%
\pgfsetroundjoin%
\definecolor{currentfill}{rgb}{0.000000,0.000000,0.000000}%
\pgfsetfillcolor{currentfill}%
\pgfsetlinewidth{0.501875pt}%
\definecolor{currentstroke}{rgb}{0.000000,0.000000,0.000000}%
\pgfsetstrokecolor{currentstroke}%
\pgfsetdash{}{0pt}%
\pgfsys@defobject{currentmarker}{\pgfqpoint{0.000000in}{0.000000in}}{\pgfqpoint{0.000000in}{0.020833in}}{%
\pgfpathmoveto{\pgfqpoint{0.000000in}{0.000000in}}%
\pgfpathlineto{\pgfqpoint{0.000000in}{0.020833in}}%
\pgfusepath{stroke,fill}%
}%
\begin{pgfscope}%
\pgfsys@transformshift{2.241540in}{0.422992in}%
\pgfsys@useobject{currentmarker}{}%
\end{pgfscope}%
\end{pgfscope}%
\begin{pgfscope}%
\pgfsetbuttcap%
\pgfsetroundjoin%
\definecolor{currentfill}{rgb}{0.000000,0.000000,0.000000}%
\pgfsetfillcolor{currentfill}%
\pgfsetlinewidth{0.501875pt}%
\definecolor{currentstroke}{rgb}{0.000000,0.000000,0.000000}%
\pgfsetstrokecolor{currentstroke}%
\pgfsetdash{}{0pt}%
\pgfsys@defobject{currentmarker}{\pgfqpoint{0.000000in}{-0.020833in}}{\pgfqpoint{0.000000in}{0.000000in}}{%
\pgfpathmoveto{\pgfqpoint{0.000000in}{0.000000in}}%
\pgfpathlineto{\pgfqpoint{0.000000in}{-0.020833in}}%
\pgfusepath{stroke,fill}%
}%
\begin{pgfscope}%
\pgfsys@transformshift{2.241540in}{1.649193in}%
\pgfsys@useobject{currentmarker}{}%
\end{pgfscope}%
\end{pgfscope}%
\begin{pgfscope}%
\pgfsetbuttcap%
\pgfsetroundjoin%
\definecolor{currentfill}{rgb}{0.000000,0.000000,0.000000}%
\pgfsetfillcolor{currentfill}%
\pgfsetlinewidth{0.501875pt}%
\definecolor{currentstroke}{rgb}{0.000000,0.000000,0.000000}%
\pgfsetstrokecolor{currentstroke}%
\pgfsetdash{}{0pt}%
\pgfsys@defobject{currentmarker}{\pgfqpoint{0.000000in}{0.000000in}}{\pgfqpoint{0.000000in}{0.020833in}}{%
\pgfpathmoveto{\pgfqpoint{0.000000in}{0.000000in}}%
\pgfpathlineto{\pgfqpoint{0.000000in}{0.020833in}}%
\pgfusepath{stroke,fill}%
}%
\begin{pgfscope}%
\pgfsys@transformshift{2.459156in}{0.422992in}%
\pgfsys@useobject{currentmarker}{}%
\end{pgfscope}%
\end{pgfscope}%
\begin{pgfscope}%
\pgfsetbuttcap%
\pgfsetroundjoin%
\definecolor{currentfill}{rgb}{0.000000,0.000000,0.000000}%
\pgfsetfillcolor{currentfill}%
\pgfsetlinewidth{0.501875pt}%
\definecolor{currentstroke}{rgb}{0.000000,0.000000,0.000000}%
\pgfsetstrokecolor{currentstroke}%
\pgfsetdash{}{0pt}%
\pgfsys@defobject{currentmarker}{\pgfqpoint{0.000000in}{-0.020833in}}{\pgfqpoint{0.000000in}{0.000000in}}{%
\pgfpathmoveto{\pgfqpoint{0.000000in}{0.000000in}}%
\pgfpathlineto{\pgfqpoint{0.000000in}{-0.020833in}}%
\pgfusepath{stroke,fill}%
}%
\begin{pgfscope}%
\pgfsys@transformshift{2.459156in}{1.649193in}%
\pgfsys@useobject{currentmarker}{}%
\end{pgfscope}%
\end{pgfscope}%
\begin{pgfscope}%
\pgfsetbuttcap%
\pgfsetroundjoin%
\definecolor{currentfill}{rgb}{0.000000,0.000000,0.000000}%
\pgfsetfillcolor{currentfill}%
\pgfsetlinewidth{0.501875pt}%
\definecolor{currentstroke}{rgb}{0.000000,0.000000,0.000000}%
\pgfsetstrokecolor{currentstroke}%
\pgfsetdash{}{0pt}%
\pgfsys@defobject{currentmarker}{\pgfqpoint{0.000000in}{0.000000in}}{\pgfqpoint{0.000000in}{0.020833in}}{%
\pgfpathmoveto{\pgfqpoint{0.000000in}{0.000000in}}%
\pgfpathlineto{\pgfqpoint{0.000000in}{0.020833in}}%
\pgfusepath{stroke,fill}%
}%
\begin{pgfscope}%
\pgfsys@transformshift{2.567965in}{0.422992in}%
\pgfsys@useobject{currentmarker}{}%
\end{pgfscope}%
\end{pgfscope}%
\begin{pgfscope}%
\pgfsetbuttcap%
\pgfsetroundjoin%
\definecolor{currentfill}{rgb}{0.000000,0.000000,0.000000}%
\pgfsetfillcolor{currentfill}%
\pgfsetlinewidth{0.501875pt}%
\definecolor{currentstroke}{rgb}{0.000000,0.000000,0.000000}%
\pgfsetstrokecolor{currentstroke}%
\pgfsetdash{}{0pt}%
\pgfsys@defobject{currentmarker}{\pgfqpoint{0.000000in}{-0.020833in}}{\pgfqpoint{0.000000in}{0.000000in}}{%
\pgfpathmoveto{\pgfqpoint{0.000000in}{0.000000in}}%
\pgfpathlineto{\pgfqpoint{0.000000in}{-0.020833in}}%
\pgfusepath{stroke,fill}%
}%
\begin{pgfscope}%
\pgfsys@transformshift{2.567965in}{1.649193in}%
\pgfsys@useobject{currentmarker}{}%
\end{pgfscope}%
\end{pgfscope}%
\begin{pgfscope}%
\pgfsetbuttcap%
\pgfsetroundjoin%
\definecolor{currentfill}{rgb}{0.000000,0.000000,0.000000}%
\pgfsetfillcolor{currentfill}%
\pgfsetlinewidth{0.501875pt}%
\definecolor{currentstroke}{rgb}{0.000000,0.000000,0.000000}%
\pgfsetstrokecolor{currentstroke}%
\pgfsetdash{}{0pt}%
\pgfsys@defobject{currentmarker}{\pgfqpoint{0.000000in}{0.000000in}}{\pgfqpoint{0.000000in}{0.020833in}}{%
\pgfpathmoveto{\pgfqpoint{0.000000in}{0.000000in}}%
\pgfpathlineto{\pgfqpoint{0.000000in}{0.020833in}}%
\pgfusepath{stroke,fill}%
}%
\begin{pgfscope}%
\pgfsys@transformshift{2.676773in}{0.422992in}%
\pgfsys@useobject{currentmarker}{}%
\end{pgfscope}%
\end{pgfscope}%
\begin{pgfscope}%
\pgfsetbuttcap%
\pgfsetroundjoin%
\definecolor{currentfill}{rgb}{0.000000,0.000000,0.000000}%
\pgfsetfillcolor{currentfill}%
\pgfsetlinewidth{0.501875pt}%
\definecolor{currentstroke}{rgb}{0.000000,0.000000,0.000000}%
\pgfsetstrokecolor{currentstroke}%
\pgfsetdash{}{0pt}%
\pgfsys@defobject{currentmarker}{\pgfqpoint{0.000000in}{-0.020833in}}{\pgfqpoint{0.000000in}{0.000000in}}{%
\pgfpathmoveto{\pgfqpoint{0.000000in}{0.000000in}}%
\pgfpathlineto{\pgfqpoint{0.000000in}{-0.020833in}}%
\pgfusepath{stroke,fill}%
}%
\begin{pgfscope}%
\pgfsys@transformshift{2.676773in}{1.649193in}%
\pgfsys@useobject{currentmarker}{}%
\end{pgfscope}%
\end{pgfscope}%
\begin{pgfscope}%
\definecolor{textcolor}{rgb}{0.000000,0.000000,0.000000}%
\pgfsetstrokecolor{textcolor}%
\pgfsetfillcolor{textcolor}%
\pgftext[x=1.719260in,y=0.184413in,,top]{\color{textcolor}\rmfamily\fontsize{10.000000}{12.000000}\selectfont \(\displaystyle K\)}%
\end{pgfscope}%
\begin{pgfscope}%
\pgfsetbuttcap%
\pgfsetroundjoin%
\definecolor{currentfill}{rgb}{0.000000,0.000000,0.000000}%
\pgfsetfillcolor{currentfill}%
\pgfsetlinewidth{0.501875pt}%
\definecolor{currentstroke}{rgb}{0.000000,0.000000,0.000000}%
\pgfsetstrokecolor{currentstroke}%
\pgfsetdash{}{0pt}%
\pgfsys@defobject{currentmarker}{\pgfqpoint{0.000000in}{0.000000in}}{\pgfqpoint{0.041667in}{0.000000in}}{%
\pgfpathmoveto{\pgfqpoint{0.000000in}{0.000000in}}%
\pgfpathlineto{\pgfqpoint{0.041667in}{0.000000in}}%
\pgfusepath{stroke,fill}%
}%
\begin{pgfscope}%
\pgfsys@transformshift{0.609415in}{0.469229in}%
\pgfsys@useobject{currentmarker}{}%
\end{pgfscope}%
\end{pgfscope}%
\begin{pgfscope}%
\pgfsetbuttcap%
\pgfsetroundjoin%
\definecolor{currentfill}{rgb}{0.000000,0.000000,0.000000}%
\pgfsetfillcolor{currentfill}%
\pgfsetlinewidth{0.501875pt}%
\definecolor{currentstroke}{rgb}{0.000000,0.000000,0.000000}%
\pgfsetstrokecolor{currentstroke}%
\pgfsetdash{}{0pt}%
\pgfsys@defobject{currentmarker}{\pgfqpoint{-0.041667in}{0.000000in}}{\pgfqpoint{-0.000000in}{0.000000in}}{%
\pgfpathmoveto{\pgfqpoint{-0.000000in}{0.000000in}}%
\pgfpathlineto{\pgfqpoint{-0.041667in}{0.000000in}}%
\pgfusepath{stroke,fill}%
}%
\begin{pgfscope}%
\pgfsys@transformshift{2.829105in}{0.469229in}%
\pgfsys@useobject{currentmarker}{}%
\end{pgfscope}%
\end{pgfscope}%
\begin{pgfscope}%
\definecolor{textcolor}{rgb}{0.000000,0.000000,0.000000}%
\pgfsetstrokecolor{textcolor}%
\pgfsetfillcolor{textcolor}%
\pgftext[x=0.244444in, y=0.416467in, left, base]{\color{textcolor}\rmfamily\fontsize{10.000000}{12.000000}\selectfont \(\displaystyle {0.925}\)}%
\end{pgfscope}%
\begin{pgfscope}%
\pgfsetbuttcap%
\pgfsetroundjoin%
\definecolor{currentfill}{rgb}{0.000000,0.000000,0.000000}%
\pgfsetfillcolor{currentfill}%
\pgfsetlinewidth{0.501875pt}%
\definecolor{currentstroke}{rgb}{0.000000,0.000000,0.000000}%
\pgfsetstrokecolor{currentstroke}%
\pgfsetdash{}{0pt}%
\pgfsys@defobject{currentmarker}{\pgfqpoint{0.000000in}{0.000000in}}{\pgfqpoint{0.041667in}{0.000000in}}{%
\pgfpathmoveto{\pgfqpoint{0.000000in}{0.000000in}}%
\pgfpathlineto{\pgfqpoint{0.041667in}{0.000000in}}%
\pgfusepath{stroke,fill}%
}%
\begin{pgfscope}%
\pgfsys@transformshift{0.609415in}{0.843971in}%
\pgfsys@useobject{currentmarker}{}%
\end{pgfscope}%
\end{pgfscope}%
\begin{pgfscope}%
\pgfsetbuttcap%
\pgfsetroundjoin%
\definecolor{currentfill}{rgb}{0.000000,0.000000,0.000000}%
\pgfsetfillcolor{currentfill}%
\pgfsetlinewidth{0.501875pt}%
\definecolor{currentstroke}{rgb}{0.000000,0.000000,0.000000}%
\pgfsetstrokecolor{currentstroke}%
\pgfsetdash{}{0pt}%
\pgfsys@defobject{currentmarker}{\pgfqpoint{-0.041667in}{0.000000in}}{\pgfqpoint{-0.000000in}{0.000000in}}{%
\pgfpathmoveto{\pgfqpoint{-0.000000in}{0.000000in}}%
\pgfpathlineto{\pgfqpoint{-0.041667in}{0.000000in}}%
\pgfusepath{stroke,fill}%
}%
\begin{pgfscope}%
\pgfsys@transformshift{2.829105in}{0.843971in}%
\pgfsys@useobject{currentmarker}{}%
\end{pgfscope}%
\end{pgfscope}%
\begin{pgfscope}%
\definecolor{textcolor}{rgb}{0.000000,0.000000,0.000000}%
\pgfsetstrokecolor{textcolor}%
\pgfsetfillcolor{textcolor}%
\pgftext[x=0.244444in, y=0.791210in, left, base]{\color{textcolor}\rmfamily\fontsize{10.000000}{12.000000}\selectfont \(\displaystyle {0.950}\)}%
\end{pgfscope}%
\begin{pgfscope}%
\pgfsetbuttcap%
\pgfsetroundjoin%
\definecolor{currentfill}{rgb}{0.000000,0.000000,0.000000}%
\pgfsetfillcolor{currentfill}%
\pgfsetlinewidth{0.501875pt}%
\definecolor{currentstroke}{rgb}{0.000000,0.000000,0.000000}%
\pgfsetstrokecolor{currentstroke}%
\pgfsetdash{}{0pt}%
\pgfsys@defobject{currentmarker}{\pgfqpoint{0.000000in}{0.000000in}}{\pgfqpoint{0.041667in}{0.000000in}}{%
\pgfpathmoveto{\pgfqpoint{0.000000in}{0.000000in}}%
\pgfpathlineto{\pgfqpoint{0.041667in}{0.000000in}}%
\pgfusepath{stroke,fill}%
}%
\begin{pgfscope}%
\pgfsys@transformshift{0.609415in}{1.218714in}%
\pgfsys@useobject{currentmarker}{}%
\end{pgfscope}%
\end{pgfscope}%
\begin{pgfscope}%
\pgfsetbuttcap%
\pgfsetroundjoin%
\definecolor{currentfill}{rgb}{0.000000,0.000000,0.000000}%
\pgfsetfillcolor{currentfill}%
\pgfsetlinewidth{0.501875pt}%
\definecolor{currentstroke}{rgb}{0.000000,0.000000,0.000000}%
\pgfsetstrokecolor{currentstroke}%
\pgfsetdash{}{0pt}%
\pgfsys@defobject{currentmarker}{\pgfqpoint{-0.041667in}{0.000000in}}{\pgfqpoint{-0.000000in}{0.000000in}}{%
\pgfpathmoveto{\pgfqpoint{-0.000000in}{0.000000in}}%
\pgfpathlineto{\pgfqpoint{-0.041667in}{0.000000in}}%
\pgfusepath{stroke,fill}%
}%
\begin{pgfscope}%
\pgfsys@transformshift{2.829105in}{1.218714in}%
\pgfsys@useobject{currentmarker}{}%
\end{pgfscope}%
\end{pgfscope}%
\begin{pgfscope}%
\definecolor{textcolor}{rgb}{0.000000,0.000000,0.000000}%
\pgfsetstrokecolor{textcolor}%
\pgfsetfillcolor{textcolor}%
\pgftext[x=0.244444in, y=1.165953in, left, base]{\color{textcolor}\rmfamily\fontsize{10.000000}{12.000000}\selectfont \(\displaystyle {0.975}\)}%
\end{pgfscope}%
\begin{pgfscope}%
\pgfsetbuttcap%
\pgfsetroundjoin%
\definecolor{currentfill}{rgb}{0.000000,0.000000,0.000000}%
\pgfsetfillcolor{currentfill}%
\pgfsetlinewidth{0.501875pt}%
\definecolor{currentstroke}{rgb}{0.000000,0.000000,0.000000}%
\pgfsetstrokecolor{currentstroke}%
\pgfsetdash{}{0pt}%
\pgfsys@defobject{currentmarker}{\pgfqpoint{0.000000in}{0.000000in}}{\pgfqpoint{0.041667in}{0.000000in}}{%
\pgfpathmoveto{\pgfqpoint{0.000000in}{0.000000in}}%
\pgfpathlineto{\pgfqpoint{0.041667in}{0.000000in}}%
\pgfusepath{stroke,fill}%
}%
\begin{pgfscope}%
\pgfsys@transformshift{0.609415in}{1.593457in}%
\pgfsys@useobject{currentmarker}{}%
\end{pgfscope}%
\end{pgfscope}%
\begin{pgfscope}%
\pgfsetbuttcap%
\pgfsetroundjoin%
\definecolor{currentfill}{rgb}{0.000000,0.000000,0.000000}%
\pgfsetfillcolor{currentfill}%
\pgfsetlinewidth{0.501875pt}%
\definecolor{currentstroke}{rgb}{0.000000,0.000000,0.000000}%
\pgfsetstrokecolor{currentstroke}%
\pgfsetdash{}{0pt}%
\pgfsys@defobject{currentmarker}{\pgfqpoint{-0.041667in}{0.000000in}}{\pgfqpoint{-0.000000in}{0.000000in}}{%
\pgfpathmoveto{\pgfqpoint{-0.000000in}{0.000000in}}%
\pgfpathlineto{\pgfqpoint{-0.041667in}{0.000000in}}%
\pgfusepath{stroke,fill}%
}%
\begin{pgfscope}%
\pgfsys@transformshift{2.829105in}{1.593457in}%
\pgfsys@useobject{currentmarker}{}%
\end{pgfscope}%
\end{pgfscope}%
\begin{pgfscope}%
\definecolor{textcolor}{rgb}{0.000000,0.000000,0.000000}%
\pgfsetstrokecolor{textcolor}%
\pgfsetfillcolor{textcolor}%
\pgftext[x=0.244444in, y=1.540695in, left, base]{\color{textcolor}\rmfamily\fontsize{10.000000}{12.000000}\selectfont \(\displaystyle {1.000}\)}%
\end{pgfscope}%
\begin{pgfscope}%
\pgfsetbuttcap%
\pgfsetroundjoin%
\definecolor{currentfill}{rgb}{0.000000,0.000000,0.000000}%
\pgfsetfillcolor{currentfill}%
\pgfsetlinewidth{0.501875pt}%
\definecolor{currentstroke}{rgb}{0.000000,0.000000,0.000000}%
\pgfsetstrokecolor{currentstroke}%
\pgfsetdash{}{0pt}%
\pgfsys@defobject{currentmarker}{\pgfqpoint{0.000000in}{0.000000in}}{\pgfqpoint{0.020833in}{0.000000in}}{%
\pgfpathmoveto{\pgfqpoint{0.000000in}{0.000000in}}%
\pgfpathlineto{\pgfqpoint{0.020833in}{0.000000in}}%
\pgfusepath{stroke,fill}%
}%
\begin{pgfscope}%
\pgfsys@transformshift{0.609415in}{0.544177in}%
\pgfsys@useobject{currentmarker}{}%
\end{pgfscope}%
\end{pgfscope}%
\begin{pgfscope}%
\pgfsetbuttcap%
\pgfsetroundjoin%
\definecolor{currentfill}{rgb}{0.000000,0.000000,0.000000}%
\pgfsetfillcolor{currentfill}%
\pgfsetlinewidth{0.501875pt}%
\definecolor{currentstroke}{rgb}{0.000000,0.000000,0.000000}%
\pgfsetstrokecolor{currentstroke}%
\pgfsetdash{}{0pt}%
\pgfsys@defobject{currentmarker}{\pgfqpoint{-0.020833in}{0.000000in}}{\pgfqpoint{-0.000000in}{0.000000in}}{%
\pgfpathmoveto{\pgfqpoint{-0.000000in}{0.000000in}}%
\pgfpathlineto{\pgfqpoint{-0.020833in}{0.000000in}}%
\pgfusepath{stroke,fill}%
}%
\begin{pgfscope}%
\pgfsys@transformshift{2.829105in}{0.544177in}%
\pgfsys@useobject{currentmarker}{}%
\end{pgfscope}%
\end{pgfscope}%
\begin{pgfscope}%
\pgfsetbuttcap%
\pgfsetroundjoin%
\definecolor{currentfill}{rgb}{0.000000,0.000000,0.000000}%
\pgfsetfillcolor{currentfill}%
\pgfsetlinewidth{0.501875pt}%
\definecolor{currentstroke}{rgb}{0.000000,0.000000,0.000000}%
\pgfsetstrokecolor{currentstroke}%
\pgfsetdash{}{0pt}%
\pgfsys@defobject{currentmarker}{\pgfqpoint{0.000000in}{0.000000in}}{\pgfqpoint{0.020833in}{0.000000in}}{%
\pgfpathmoveto{\pgfqpoint{0.000000in}{0.000000in}}%
\pgfpathlineto{\pgfqpoint{0.020833in}{0.000000in}}%
\pgfusepath{stroke,fill}%
}%
\begin{pgfscope}%
\pgfsys@transformshift{0.609415in}{0.619126in}%
\pgfsys@useobject{currentmarker}{}%
\end{pgfscope}%
\end{pgfscope}%
\begin{pgfscope}%
\pgfsetbuttcap%
\pgfsetroundjoin%
\definecolor{currentfill}{rgb}{0.000000,0.000000,0.000000}%
\pgfsetfillcolor{currentfill}%
\pgfsetlinewidth{0.501875pt}%
\definecolor{currentstroke}{rgb}{0.000000,0.000000,0.000000}%
\pgfsetstrokecolor{currentstroke}%
\pgfsetdash{}{0pt}%
\pgfsys@defobject{currentmarker}{\pgfqpoint{-0.020833in}{0.000000in}}{\pgfqpoint{-0.000000in}{0.000000in}}{%
\pgfpathmoveto{\pgfqpoint{-0.000000in}{0.000000in}}%
\pgfpathlineto{\pgfqpoint{-0.020833in}{0.000000in}}%
\pgfusepath{stroke,fill}%
}%
\begin{pgfscope}%
\pgfsys@transformshift{2.829105in}{0.619126in}%
\pgfsys@useobject{currentmarker}{}%
\end{pgfscope}%
\end{pgfscope}%
\begin{pgfscope}%
\pgfsetbuttcap%
\pgfsetroundjoin%
\definecolor{currentfill}{rgb}{0.000000,0.000000,0.000000}%
\pgfsetfillcolor{currentfill}%
\pgfsetlinewidth{0.501875pt}%
\definecolor{currentstroke}{rgb}{0.000000,0.000000,0.000000}%
\pgfsetstrokecolor{currentstroke}%
\pgfsetdash{}{0pt}%
\pgfsys@defobject{currentmarker}{\pgfqpoint{0.000000in}{0.000000in}}{\pgfqpoint{0.020833in}{0.000000in}}{%
\pgfpathmoveto{\pgfqpoint{0.000000in}{0.000000in}}%
\pgfpathlineto{\pgfqpoint{0.020833in}{0.000000in}}%
\pgfusepath{stroke,fill}%
}%
\begin{pgfscope}%
\pgfsys@transformshift{0.609415in}{0.694074in}%
\pgfsys@useobject{currentmarker}{}%
\end{pgfscope}%
\end{pgfscope}%
\begin{pgfscope}%
\pgfsetbuttcap%
\pgfsetroundjoin%
\definecolor{currentfill}{rgb}{0.000000,0.000000,0.000000}%
\pgfsetfillcolor{currentfill}%
\pgfsetlinewidth{0.501875pt}%
\definecolor{currentstroke}{rgb}{0.000000,0.000000,0.000000}%
\pgfsetstrokecolor{currentstroke}%
\pgfsetdash{}{0pt}%
\pgfsys@defobject{currentmarker}{\pgfqpoint{-0.020833in}{0.000000in}}{\pgfqpoint{-0.000000in}{0.000000in}}{%
\pgfpathmoveto{\pgfqpoint{-0.000000in}{0.000000in}}%
\pgfpathlineto{\pgfqpoint{-0.020833in}{0.000000in}}%
\pgfusepath{stroke,fill}%
}%
\begin{pgfscope}%
\pgfsys@transformshift{2.829105in}{0.694074in}%
\pgfsys@useobject{currentmarker}{}%
\end{pgfscope}%
\end{pgfscope}%
\begin{pgfscope}%
\pgfsetbuttcap%
\pgfsetroundjoin%
\definecolor{currentfill}{rgb}{0.000000,0.000000,0.000000}%
\pgfsetfillcolor{currentfill}%
\pgfsetlinewidth{0.501875pt}%
\definecolor{currentstroke}{rgb}{0.000000,0.000000,0.000000}%
\pgfsetstrokecolor{currentstroke}%
\pgfsetdash{}{0pt}%
\pgfsys@defobject{currentmarker}{\pgfqpoint{0.000000in}{0.000000in}}{\pgfqpoint{0.020833in}{0.000000in}}{%
\pgfpathmoveto{\pgfqpoint{0.000000in}{0.000000in}}%
\pgfpathlineto{\pgfqpoint{0.020833in}{0.000000in}}%
\pgfusepath{stroke,fill}%
}%
\begin{pgfscope}%
\pgfsys@transformshift{0.609415in}{0.769023in}%
\pgfsys@useobject{currentmarker}{}%
\end{pgfscope}%
\end{pgfscope}%
\begin{pgfscope}%
\pgfsetbuttcap%
\pgfsetroundjoin%
\definecolor{currentfill}{rgb}{0.000000,0.000000,0.000000}%
\pgfsetfillcolor{currentfill}%
\pgfsetlinewidth{0.501875pt}%
\definecolor{currentstroke}{rgb}{0.000000,0.000000,0.000000}%
\pgfsetstrokecolor{currentstroke}%
\pgfsetdash{}{0pt}%
\pgfsys@defobject{currentmarker}{\pgfqpoint{-0.020833in}{0.000000in}}{\pgfqpoint{-0.000000in}{0.000000in}}{%
\pgfpathmoveto{\pgfqpoint{-0.000000in}{0.000000in}}%
\pgfpathlineto{\pgfqpoint{-0.020833in}{0.000000in}}%
\pgfusepath{stroke,fill}%
}%
\begin{pgfscope}%
\pgfsys@transformshift{2.829105in}{0.769023in}%
\pgfsys@useobject{currentmarker}{}%
\end{pgfscope}%
\end{pgfscope}%
\begin{pgfscope}%
\pgfsetbuttcap%
\pgfsetroundjoin%
\definecolor{currentfill}{rgb}{0.000000,0.000000,0.000000}%
\pgfsetfillcolor{currentfill}%
\pgfsetlinewidth{0.501875pt}%
\definecolor{currentstroke}{rgb}{0.000000,0.000000,0.000000}%
\pgfsetstrokecolor{currentstroke}%
\pgfsetdash{}{0pt}%
\pgfsys@defobject{currentmarker}{\pgfqpoint{0.000000in}{0.000000in}}{\pgfqpoint{0.020833in}{0.000000in}}{%
\pgfpathmoveto{\pgfqpoint{0.000000in}{0.000000in}}%
\pgfpathlineto{\pgfqpoint{0.020833in}{0.000000in}}%
\pgfusepath{stroke,fill}%
}%
\begin{pgfscope}%
\pgfsys@transformshift{0.609415in}{0.918920in}%
\pgfsys@useobject{currentmarker}{}%
\end{pgfscope}%
\end{pgfscope}%
\begin{pgfscope}%
\pgfsetbuttcap%
\pgfsetroundjoin%
\definecolor{currentfill}{rgb}{0.000000,0.000000,0.000000}%
\pgfsetfillcolor{currentfill}%
\pgfsetlinewidth{0.501875pt}%
\definecolor{currentstroke}{rgb}{0.000000,0.000000,0.000000}%
\pgfsetstrokecolor{currentstroke}%
\pgfsetdash{}{0pt}%
\pgfsys@defobject{currentmarker}{\pgfqpoint{-0.020833in}{0.000000in}}{\pgfqpoint{-0.000000in}{0.000000in}}{%
\pgfpathmoveto{\pgfqpoint{-0.000000in}{0.000000in}}%
\pgfpathlineto{\pgfqpoint{-0.020833in}{0.000000in}}%
\pgfusepath{stroke,fill}%
}%
\begin{pgfscope}%
\pgfsys@transformshift{2.829105in}{0.918920in}%
\pgfsys@useobject{currentmarker}{}%
\end{pgfscope}%
\end{pgfscope}%
\begin{pgfscope}%
\pgfsetbuttcap%
\pgfsetroundjoin%
\definecolor{currentfill}{rgb}{0.000000,0.000000,0.000000}%
\pgfsetfillcolor{currentfill}%
\pgfsetlinewidth{0.501875pt}%
\definecolor{currentstroke}{rgb}{0.000000,0.000000,0.000000}%
\pgfsetstrokecolor{currentstroke}%
\pgfsetdash{}{0pt}%
\pgfsys@defobject{currentmarker}{\pgfqpoint{0.000000in}{0.000000in}}{\pgfqpoint{0.020833in}{0.000000in}}{%
\pgfpathmoveto{\pgfqpoint{0.000000in}{0.000000in}}%
\pgfpathlineto{\pgfqpoint{0.020833in}{0.000000in}}%
\pgfusepath{stroke,fill}%
}%
\begin{pgfscope}%
\pgfsys@transformshift{0.609415in}{0.993868in}%
\pgfsys@useobject{currentmarker}{}%
\end{pgfscope}%
\end{pgfscope}%
\begin{pgfscope}%
\pgfsetbuttcap%
\pgfsetroundjoin%
\definecolor{currentfill}{rgb}{0.000000,0.000000,0.000000}%
\pgfsetfillcolor{currentfill}%
\pgfsetlinewidth{0.501875pt}%
\definecolor{currentstroke}{rgb}{0.000000,0.000000,0.000000}%
\pgfsetstrokecolor{currentstroke}%
\pgfsetdash{}{0pt}%
\pgfsys@defobject{currentmarker}{\pgfqpoint{-0.020833in}{0.000000in}}{\pgfqpoint{-0.000000in}{0.000000in}}{%
\pgfpathmoveto{\pgfqpoint{-0.000000in}{0.000000in}}%
\pgfpathlineto{\pgfqpoint{-0.020833in}{0.000000in}}%
\pgfusepath{stroke,fill}%
}%
\begin{pgfscope}%
\pgfsys@transformshift{2.829105in}{0.993868in}%
\pgfsys@useobject{currentmarker}{}%
\end{pgfscope}%
\end{pgfscope}%
\begin{pgfscope}%
\pgfsetbuttcap%
\pgfsetroundjoin%
\definecolor{currentfill}{rgb}{0.000000,0.000000,0.000000}%
\pgfsetfillcolor{currentfill}%
\pgfsetlinewidth{0.501875pt}%
\definecolor{currentstroke}{rgb}{0.000000,0.000000,0.000000}%
\pgfsetstrokecolor{currentstroke}%
\pgfsetdash{}{0pt}%
\pgfsys@defobject{currentmarker}{\pgfqpoint{0.000000in}{0.000000in}}{\pgfqpoint{0.020833in}{0.000000in}}{%
\pgfpathmoveto{\pgfqpoint{0.000000in}{0.000000in}}%
\pgfpathlineto{\pgfqpoint{0.020833in}{0.000000in}}%
\pgfusepath{stroke,fill}%
}%
\begin{pgfscope}%
\pgfsys@transformshift{0.609415in}{1.068817in}%
\pgfsys@useobject{currentmarker}{}%
\end{pgfscope}%
\end{pgfscope}%
\begin{pgfscope}%
\pgfsetbuttcap%
\pgfsetroundjoin%
\definecolor{currentfill}{rgb}{0.000000,0.000000,0.000000}%
\pgfsetfillcolor{currentfill}%
\pgfsetlinewidth{0.501875pt}%
\definecolor{currentstroke}{rgb}{0.000000,0.000000,0.000000}%
\pgfsetstrokecolor{currentstroke}%
\pgfsetdash{}{0pt}%
\pgfsys@defobject{currentmarker}{\pgfqpoint{-0.020833in}{0.000000in}}{\pgfqpoint{-0.000000in}{0.000000in}}{%
\pgfpathmoveto{\pgfqpoint{-0.000000in}{0.000000in}}%
\pgfpathlineto{\pgfqpoint{-0.020833in}{0.000000in}}%
\pgfusepath{stroke,fill}%
}%
\begin{pgfscope}%
\pgfsys@transformshift{2.829105in}{1.068817in}%
\pgfsys@useobject{currentmarker}{}%
\end{pgfscope}%
\end{pgfscope}%
\begin{pgfscope}%
\pgfsetbuttcap%
\pgfsetroundjoin%
\definecolor{currentfill}{rgb}{0.000000,0.000000,0.000000}%
\pgfsetfillcolor{currentfill}%
\pgfsetlinewidth{0.501875pt}%
\definecolor{currentstroke}{rgb}{0.000000,0.000000,0.000000}%
\pgfsetstrokecolor{currentstroke}%
\pgfsetdash{}{0pt}%
\pgfsys@defobject{currentmarker}{\pgfqpoint{0.000000in}{0.000000in}}{\pgfqpoint{0.020833in}{0.000000in}}{%
\pgfpathmoveto{\pgfqpoint{0.000000in}{0.000000in}}%
\pgfpathlineto{\pgfqpoint{0.020833in}{0.000000in}}%
\pgfusepath{stroke,fill}%
}%
\begin{pgfscope}%
\pgfsys@transformshift{0.609415in}{1.143766in}%
\pgfsys@useobject{currentmarker}{}%
\end{pgfscope}%
\end{pgfscope}%
\begin{pgfscope}%
\pgfsetbuttcap%
\pgfsetroundjoin%
\definecolor{currentfill}{rgb}{0.000000,0.000000,0.000000}%
\pgfsetfillcolor{currentfill}%
\pgfsetlinewidth{0.501875pt}%
\definecolor{currentstroke}{rgb}{0.000000,0.000000,0.000000}%
\pgfsetstrokecolor{currentstroke}%
\pgfsetdash{}{0pt}%
\pgfsys@defobject{currentmarker}{\pgfqpoint{-0.020833in}{0.000000in}}{\pgfqpoint{-0.000000in}{0.000000in}}{%
\pgfpathmoveto{\pgfqpoint{-0.000000in}{0.000000in}}%
\pgfpathlineto{\pgfqpoint{-0.020833in}{0.000000in}}%
\pgfusepath{stroke,fill}%
}%
\begin{pgfscope}%
\pgfsys@transformshift{2.829105in}{1.143766in}%
\pgfsys@useobject{currentmarker}{}%
\end{pgfscope}%
\end{pgfscope}%
\begin{pgfscope}%
\pgfsetbuttcap%
\pgfsetroundjoin%
\definecolor{currentfill}{rgb}{0.000000,0.000000,0.000000}%
\pgfsetfillcolor{currentfill}%
\pgfsetlinewidth{0.501875pt}%
\definecolor{currentstroke}{rgb}{0.000000,0.000000,0.000000}%
\pgfsetstrokecolor{currentstroke}%
\pgfsetdash{}{0pt}%
\pgfsys@defobject{currentmarker}{\pgfqpoint{0.000000in}{0.000000in}}{\pgfqpoint{0.020833in}{0.000000in}}{%
\pgfpathmoveto{\pgfqpoint{0.000000in}{0.000000in}}%
\pgfpathlineto{\pgfqpoint{0.020833in}{0.000000in}}%
\pgfusepath{stroke,fill}%
}%
\begin{pgfscope}%
\pgfsys@transformshift{0.609415in}{1.293663in}%
\pgfsys@useobject{currentmarker}{}%
\end{pgfscope}%
\end{pgfscope}%
\begin{pgfscope}%
\pgfsetbuttcap%
\pgfsetroundjoin%
\definecolor{currentfill}{rgb}{0.000000,0.000000,0.000000}%
\pgfsetfillcolor{currentfill}%
\pgfsetlinewidth{0.501875pt}%
\definecolor{currentstroke}{rgb}{0.000000,0.000000,0.000000}%
\pgfsetstrokecolor{currentstroke}%
\pgfsetdash{}{0pt}%
\pgfsys@defobject{currentmarker}{\pgfqpoint{-0.020833in}{0.000000in}}{\pgfqpoint{-0.000000in}{0.000000in}}{%
\pgfpathmoveto{\pgfqpoint{-0.000000in}{0.000000in}}%
\pgfpathlineto{\pgfqpoint{-0.020833in}{0.000000in}}%
\pgfusepath{stroke,fill}%
}%
\begin{pgfscope}%
\pgfsys@transformshift{2.829105in}{1.293663in}%
\pgfsys@useobject{currentmarker}{}%
\end{pgfscope}%
\end{pgfscope}%
\begin{pgfscope}%
\pgfsetbuttcap%
\pgfsetroundjoin%
\definecolor{currentfill}{rgb}{0.000000,0.000000,0.000000}%
\pgfsetfillcolor{currentfill}%
\pgfsetlinewidth{0.501875pt}%
\definecolor{currentstroke}{rgb}{0.000000,0.000000,0.000000}%
\pgfsetstrokecolor{currentstroke}%
\pgfsetdash{}{0pt}%
\pgfsys@defobject{currentmarker}{\pgfqpoint{0.000000in}{0.000000in}}{\pgfqpoint{0.020833in}{0.000000in}}{%
\pgfpathmoveto{\pgfqpoint{0.000000in}{0.000000in}}%
\pgfpathlineto{\pgfqpoint{0.020833in}{0.000000in}}%
\pgfusepath{stroke,fill}%
}%
\begin{pgfscope}%
\pgfsys@transformshift{0.609415in}{1.368611in}%
\pgfsys@useobject{currentmarker}{}%
\end{pgfscope}%
\end{pgfscope}%
\begin{pgfscope}%
\pgfsetbuttcap%
\pgfsetroundjoin%
\definecolor{currentfill}{rgb}{0.000000,0.000000,0.000000}%
\pgfsetfillcolor{currentfill}%
\pgfsetlinewidth{0.501875pt}%
\definecolor{currentstroke}{rgb}{0.000000,0.000000,0.000000}%
\pgfsetstrokecolor{currentstroke}%
\pgfsetdash{}{0pt}%
\pgfsys@defobject{currentmarker}{\pgfqpoint{-0.020833in}{0.000000in}}{\pgfqpoint{-0.000000in}{0.000000in}}{%
\pgfpathmoveto{\pgfqpoint{-0.000000in}{0.000000in}}%
\pgfpathlineto{\pgfqpoint{-0.020833in}{0.000000in}}%
\pgfusepath{stroke,fill}%
}%
\begin{pgfscope}%
\pgfsys@transformshift{2.829105in}{1.368611in}%
\pgfsys@useobject{currentmarker}{}%
\end{pgfscope}%
\end{pgfscope}%
\begin{pgfscope}%
\pgfsetbuttcap%
\pgfsetroundjoin%
\definecolor{currentfill}{rgb}{0.000000,0.000000,0.000000}%
\pgfsetfillcolor{currentfill}%
\pgfsetlinewidth{0.501875pt}%
\definecolor{currentstroke}{rgb}{0.000000,0.000000,0.000000}%
\pgfsetstrokecolor{currentstroke}%
\pgfsetdash{}{0pt}%
\pgfsys@defobject{currentmarker}{\pgfqpoint{0.000000in}{0.000000in}}{\pgfqpoint{0.020833in}{0.000000in}}{%
\pgfpathmoveto{\pgfqpoint{0.000000in}{0.000000in}}%
\pgfpathlineto{\pgfqpoint{0.020833in}{0.000000in}}%
\pgfusepath{stroke,fill}%
}%
\begin{pgfscope}%
\pgfsys@transformshift{0.609415in}{1.443560in}%
\pgfsys@useobject{currentmarker}{}%
\end{pgfscope}%
\end{pgfscope}%
\begin{pgfscope}%
\pgfsetbuttcap%
\pgfsetroundjoin%
\definecolor{currentfill}{rgb}{0.000000,0.000000,0.000000}%
\pgfsetfillcolor{currentfill}%
\pgfsetlinewidth{0.501875pt}%
\definecolor{currentstroke}{rgb}{0.000000,0.000000,0.000000}%
\pgfsetstrokecolor{currentstroke}%
\pgfsetdash{}{0pt}%
\pgfsys@defobject{currentmarker}{\pgfqpoint{-0.020833in}{0.000000in}}{\pgfqpoint{-0.000000in}{0.000000in}}{%
\pgfpathmoveto{\pgfqpoint{-0.000000in}{0.000000in}}%
\pgfpathlineto{\pgfqpoint{-0.020833in}{0.000000in}}%
\pgfusepath{stroke,fill}%
}%
\begin{pgfscope}%
\pgfsys@transformshift{2.829105in}{1.443560in}%
\pgfsys@useobject{currentmarker}{}%
\end{pgfscope}%
\end{pgfscope}%
\begin{pgfscope}%
\pgfsetbuttcap%
\pgfsetroundjoin%
\definecolor{currentfill}{rgb}{0.000000,0.000000,0.000000}%
\pgfsetfillcolor{currentfill}%
\pgfsetlinewidth{0.501875pt}%
\definecolor{currentstroke}{rgb}{0.000000,0.000000,0.000000}%
\pgfsetstrokecolor{currentstroke}%
\pgfsetdash{}{0pt}%
\pgfsys@defobject{currentmarker}{\pgfqpoint{0.000000in}{0.000000in}}{\pgfqpoint{0.020833in}{0.000000in}}{%
\pgfpathmoveto{\pgfqpoint{0.000000in}{0.000000in}}%
\pgfpathlineto{\pgfqpoint{0.020833in}{0.000000in}}%
\pgfusepath{stroke,fill}%
}%
\begin{pgfscope}%
\pgfsys@transformshift{0.609415in}{1.518508in}%
\pgfsys@useobject{currentmarker}{}%
\end{pgfscope}%
\end{pgfscope}%
\begin{pgfscope}%
\pgfsetbuttcap%
\pgfsetroundjoin%
\definecolor{currentfill}{rgb}{0.000000,0.000000,0.000000}%
\pgfsetfillcolor{currentfill}%
\pgfsetlinewidth{0.501875pt}%
\definecolor{currentstroke}{rgb}{0.000000,0.000000,0.000000}%
\pgfsetstrokecolor{currentstroke}%
\pgfsetdash{}{0pt}%
\pgfsys@defobject{currentmarker}{\pgfqpoint{-0.020833in}{0.000000in}}{\pgfqpoint{-0.000000in}{0.000000in}}{%
\pgfpathmoveto{\pgfqpoint{-0.000000in}{0.000000in}}%
\pgfpathlineto{\pgfqpoint{-0.020833in}{0.000000in}}%
\pgfusepath{stroke,fill}%
}%
\begin{pgfscope}%
\pgfsys@transformshift{2.829105in}{1.518508in}%
\pgfsys@useobject{currentmarker}{}%
\end{pgfscope}%
\end{pgfscope}%
\begin{pgfscope}%
\definecolor{textcolor}{rgb}{0.000000,0.000000,0.000000}%
\pgfsetstrokecolor{textcolor}%
\pgfsetfillcolor{textcolor}%
\pgftext[x=0.188889in,y=1.036093in,,bottom,rotate=90.000000]{\color{textcolor}\rmfamily\fontsize{10.000000}{12.000000}\selectfont \(\displaystyle C(K)\)}%
\end{pgfscope}%
\begin{pgfscope}%
\pgfpathrectangle{\pgfqpoint{0.609415in}{0.422992in}}{\pgfqpoint{2.219690in}{1.226201in}}%
\pgfusepath{clip}%
\pgfsetrectcap%
\pgfsetroundjoin%
\pgfsetlinewidth{1.003750pt}%
\definecolor{currentstroke}{rgb}{0.047059,0.364706,0.647059}%
\pgfsetstrokecolor{currentstroke}%
\pgfsetdash{}{0pt}%
\pgfpathmoveto{\pgfqpoint{0.631176in}{1.593457in}}%
\pgfpathlineto{\pgfqpoint{0.652938in}{1.586694in}}%
\pgfpathlineto{\pgfqpoint{0.674700in}{1.581643in}}%
\pgfpathlineto{\pgfqpoint{0.696461in}{1.576501in}}%
\pgfpathlineto{\pgfqpoint{0.718223in}{1.573171in}}%
\pgfpathlineto{\pgfqpoint{0.739985in}{1.570258in}}%
\pgfpathlineto{\pgfqpoint{0.761746in}{1.567799in}}%
\pgfpathlineto{\pgfqpoint{0.783508in}{1.565522in}}%
\pgfpathlineto{\pgfqpoint{0.805270in}{1.563025in}}%
\pgfpathlineto{\pgfqpoint{0.827031in}{1.561211in}}%
\pgfpathlineto{\pgfqpoint{0.848793in}{1.559469in}}%
\pgfpathlineto{\pgfqpoint{0.870555in}{1.557617in}}%
\pgfpathlineto{\pgfqpoint{0.892316in}{1.556306in}}%
\pgfpathlineto{\pgfqpoint{0.914078in}{1.555126in}}%
\pgfpathlineto{\pgfqpoint{0.935840in}{1.553562in}}%
\pgfpathlineto{\pgfqpoint{0.957601in}{1.552351in}}%
\pgfpathlineto{\pgfqpoint{0.979363in}{1.551300in}}%
\pgfpathlineto{\pgfqpoint{1.001125in}{1.550400in}}%
\pgfpathlineto{\pgfqpoint{1.022886in}{1.549364in}}%
\pgfpathlineto{\pgfqpoint{1.044648in}{1.548364in}}%
\pgfpathlineto{\pgfqpoint{1.066410in}{1.547315in}}%
\pgfpathlineto{\pgfqpoint{1.088171in}{1.546503in}}%
\pgfpathlineto{\pgfqpoint{1.109933in}{1.545653in}}%
\pgfpathlineto{\pgfqpoint{1.131695in}{1.544812in}}%
\pgfpathlineto{\pgfqpoint{1.153456in}{1.543955in}}%
\pgfpathlineto{\pgfqpoint{1.175218in}{1.543276in}}%
\pgfpathlineto{\pgfqpoint{1.196980in}{1.542627in}}%
\pgfpathlineto{\pgfqpoint{1.218741in}{1.541874in}}%
\pgfpathlineto{\pgfqpoint{1.240503in}{1.541168in}}%
\pgfpathlineto{\pgfqpoint{1.262265in}{1.540493in}}%
\pgfpathlineto{\pgfqpoint{1.284026in}{1.539851in}}%
\pgfpathlineto{\pgfqpoint{1.305788in}{1.539174in}}%
\pgfpathlineto{\pgfqpoint{1.327550in}{1.538534in}}%
\pgfpathlineto{\pgfqpoint{1.349311in}{1.537845in}}%
\pgfpathlineto{\pgfqpoint{1.371073in}{1.537119in}}%
\pgfpathlineto{\pgfqpoint{1.392835in}{1.536553in}}%
\pgfpathlineto{\pgfqpoint{1.414596in}{1.536010in}}%
\pgfpathlineto{\pgfqpoint{1.436358in}{1.535381in}}%
\pgfpathlineto{\pgfqpoint{1.458120in}{1.534844in}}%
\pgfpathlineto{\pgfqpoint{1.479881in}{1.534270in}}%
\pgfpathlineto{\pgfqpoint{1.501643in}{1.533660in}}%
\pgfpathlineto{\pgfqpoint{1.523405in}{1.533096in}}%
\pgfpathlineto{\pgfqpoint{1.545166in}{1.532525in}}%
\pgfpathlineto{\pgfqpoint{1.566928in}{1.531978in}}%
\pgfpathlineto{\pgfqpoint{1.588690in}{1.531502in}}%
\pgfpathlineto{\pgfqpoint{1.610451in}{1.531053in}}%
\pgfpathlineto{\pgfqpoint{1.632213in}{1.530633in}}%
\pgfpathlineto{\pgfqpoint{1.653975in}{1.530174in}}%
\pgfpathlineto{\pgfqpoint{1.675736in}{1.529649in}}%
\pgfpathlineto{\pgfqpoint{1.697498in}{1.529154in}}%
\pgfpathlineto{\pgfqpoint{1.719260in}{1.528726in}}%
\pgfpathlineto{\pgfqpoint{1.741021in}{1.528333in}}%
\pgfpathlineto{\pgfqpoint{1.762783in}{1.527881in}}%
\pgfpathlineto{\pgfqpoint{1.784545in}{1.527435in}}%
\pgfpathlineto{\pgfqpoint{1.806306in}{1.527003in}}%
\pgfpathlineto{\pgfqpoint{1.828068in}{1.526522in}}%
\pgfpathlineto{\pgfqpoint{1.849830in}{1.526113in}}%
\pgfpathlineto{\pgfqpoint{1.871591in}{1.525689in}}%
\pgfpathlineto{\pgfqpoint{1.893353in}{1.525221in}}%
\pgfpathlineto{\pgfqpoint{1.915115in}{1.524843in}}%
\pgfpathlineto{\pgfqpoint{1.936876in}{1.524403in}}%
\pgfpathlineto{\pgfqpoint{1.958638in}{1.523989in}}%
\pgfpathlineto{\pgfqpoint{1.980400in}{1.523585in}}%
\pgfpathlineto{\pgfqpoint{2.002161in}{1.523137in}}%
\pgfpathlineto{\pgfqpoint{2.023923in}{1.522718in}}%
\pgfpathlineto{\pgfqpoint{2.045685in}{1.522316in}}%
\pgfpathlineto{\pgfqpoint{2.067446in}{1.521937in}}%
\pgfpathlineto{\pgfqpoint{2.089208in}{1.521567in}}%
\pgfpathlineto{\pgfqpoint{2.110970in}{1.521125in}}%
\pgfpathlineto{\pgfqpoint{2.132731in}{1.520759in}}%
\pgfpathlineto{\pgfqpoint{2.154493in}{1.520382in}}%
\pgfpathlineto{\pgfqpoint{2.176255in}{1.520022in}}%
\pgfpathlineto{\pgfqpoint{2.198016in}{1.519647in}}%
\pgfpathlineto{\pgfqpoint{2.219778in}{1.519254in}}%
\pgfpathlineto{\pgfqpoint{2.241540in}{1.518869in}}%
\pgfpathlineto{\pgfqpoint{2.263301in}{1.518502in}}%
\pgfpathlineto{\pgfqpoint{2.285063in}{1.518148in}}%
\pgfpathlineto{\pgfqpoint{2.306825in}{1.517748in}}%
\pgfpathlineto{\pgfqpoint{2.328586in}{1.517373in}}%
\pgfpathlineto{\pgfqpoint{2.350348in}{1.516970in}}%
\pgfpathlineto{\pgfqpoint{2.372110in}{1.516623in}}%
\pgfpathlineto{\pgfqpoint{2.393871in}{1.516278in}}%
\pgfpathlineto{\pgfqpoint{2.415633in}{1.515948in}}%
\pgfpathlineto{\pgfqpoint{2.437395in}{1.515626in}}%
\pgfpathlineto{\pgfqpoint{2.459156in}{1.515294in}}%
\pgfpathlineto{\pgfqpoint{2.480918in}{1.514921in}}%
\pgfpathlineto{\pgfqpoint{2.502680in}{1.514564in}}%
\pgfpathlineto{\pgfqpoint{2.524441in}{1.514195in}}%
\pgfpathlineto{\pgfqpoint{2.546203in}{1.513848in}}%
\pgfpathlineto{\pgfqpoint{2.567965in}{1.513508in}}%
\pgfpathlineto{\pgfqpoint{2.589726in}{1.513158in}}%
\pgfpathlineto{\pgfqpoint{2.611488in}{1.512830in}}%
\pgfpathlineto{\pgfqpoint{2.633250in}{1.512508in}}%
\pgfpathlineto{\pgfqpoint{2.655011in}{1.512157in}}%
\pgfpathlineto{\pgfqpoint{2.676773in}{1.511803in}}%
\pgfpathlineto{\pgfqpoint{2.698535in}{1.511473in}}%
\pgfpathlineto{\pgfqpoint{2.720296in}{1.511134in}}%
\pgfpathlineto{\pgfqpoint{2.742058in}{1.510821in}}%
\pgfpathlineto{\pgfqpoint{2.763820in}{1.510482in}}%
\pgfpathlineto{\pgfqpoint{2.785581in}{1.510137in}}%
\pgfusepath{stroke}%
\end{pgfscope}%
\begin{pgfscope}%
\pgfpathrectangle{\pgfqpoint{0.609415in}{0.422992in}}{\pgfqpoint{2.219690in}{1.226201in}}%
\pgfusepath{clip}%
\pgfsetrectcap%
\pgfsetroundjoin%
\pgfsetlinewidth{1.003750pt}%
\definecolor{currentstroke}{rgb}{0.000000,0.725490,0.270588}%
\pgfsetstrokecolor{currentstroke}%
\pgfsetdash{}{0pt}%
\pgfpathmoveto{\pgfqpoint{0.631176in}{1.593457in}}%
\pgfpathlineto{\pgfqpoint{0.652938in}{1.571733in}}%
\pgfpathlineto{\pgfqpoint{0.674700in}{1.556103in}}%
\pgfpathlineto{\pgfqpoint{0.696461in}{1.542603in}}%
\pgfpathlineto{\pgfqpoint{0.718223in}{1.531259in}}%
\pgfpathlineto{\pgfqpoint{0.739985in}{1.520925in}}%
\pgfpathlineto{\pgfqpoint{0.761746in}{1.511767in}}%
\pgfpathlineto{\pgfqpoint{0.783508in}{1.503103in}}%
\pgfpathlineto{\pgfqpoint{0.805270in}{1.494297in}}%
\pgfpathlineto{\pgfqpoint{0.827031in}{1.486112in}}%
\pgfpathlineto{\pgfqpoint{0.848793in}{1.478031in}}%
\pgfpathlineto{\pgfqpoint{0.870555in}{1.470338in}}%
\pgfpathlineto{\pgfqpoint{0.892316in}{1.462821in}}%
\pgfpathlineto{\pgfqpoint{0.914078in}{1.455492in}}%
\pgfpathlineto{\pgfqpoint{0.935840in}{1.448196in}}%
\pgfpathlineto{\pgfqpoint{0.957601in}{1.441486in}}%
\pgfpathlineto{\pgfqpoint{0.979363in}{1.434449in}}%
\pgfpathlineto{\pgfqpoint{1.001125in}{1.427979in}}%
\pgfpathlineto{\pgfqpoint{1.022886in}{1.421325in}}%
\pgfpathlineto{\pgfqpoint{1.044648in}{1.414401in}}%
\pgfpathlineto{\pgfqpoint{1.066410in}{1.407994in}}%
\pgfpathlineto{\pgfqpoint{1.088171in}{1.401492in}}%
\pgfpathlineto{\pgfqpoint{1.109933in}{1.394992in}}%
\pgfpathlineto{\pgfqpoint{1.131695in}{1.388804in}}%
\pgfpathlineto{\pgfqpoint{1.153456in}{1.382889in}}%
\pgfpathlineto{\pgfqpoint{1.175218in}{1.376815in}}%
\pgfpathlineto{\pgfqpoint{1.196980in}{1.370591in}}%
\pgfpathlineto{\pgfqpoint{1.218741in}{1.364839in}}%
\pgfpathlineto{\pgfqpoint{1.240503in}{1.359020in}}%
\pgfpathlineto{\pgfqpoint{1.262265in}{1.353243in}}%
\pgfpathlineto{\pgfqpoint{1.284026in}{1.347762in}}%
\pgfpathlineto{\pgfqpoint{1.305788in}{1.342094in}}%
\pgfpathlineto{\pgfqpoint{1.327550in}{1.336503in}}%
\pgfpathlineto{\pgfqpoint{1.349311in}{1.331264in}}%
\pgfpathlineto{\pgfqpoint{1.371073in}{1.325985in}}%
\pgfpathlineto{\pgfqpoint{1.392835in}{1.320748in}}%
\pgfpathlineto{\pgfqpoint{1.414596in}{1.315410in}}%
\pgfpathlineto{\pgfqpoint{1.436358in}{1.310038in}}%
\pgfpathlineto{\pgfqpoint{1.458120in}{1.304828in}}%
\pgfpathlineto{\pgfqpoint{1.479881in}{1.299790in}}%
\pgfpathlineto{\pgfqpoint{1.501643in}{1.294754in}}%
\pgfpathlineto{\pgfqpoint{1.523405in}{1.289709in}}%
\pgfpathlineto{\pgfqpoint{1.545166in}{1.284839in}}%
\pgfpathlineto{\pgfqpoint{1.566928in}{1.279717in}}%
\pgfpathlineto{\pgfqpoint{1.588690in}{1.274790in}}%
\pgfpathlineto{\pgfqpoint{1.610451in}{1.269875in}}%
\pgfpathlineto{\pgfqpoint{1.632213in}{1.265169in}}%
\pgfpathlineto{\pgfqpoint{1.653975in}{1.260227in}}%
\pgfpathlineto{\pgfqpoint{1.675736in}{1.255202in}}%
\pgfpathlineto{\pgfqpoint{1.697498in}{1.250580in}}%
\pgfpathlineto{\pgfqpoint{1.719260in}{1.245865in}}%
\pgfpathlineto{\pgfqpoint{1.741021in}{1.241147in}}%
\pgfpathlineto{\pgfqpoint{1.762783in}{1.236413in}}%
\pgfpathlineto{\pgfqpoint{1.784545in}{1.231538in}}%
\pgfpathlineto{\pgfqpoint{1.806306in}{1.226874in}}%
\pgfpathlineto{\pgfqpoint{1.828068in}{1.222462in}}%
\pgfpathlineto{\pgfqpoint{1.849830in}{1.218219in}}%
\pgfpathlineto{\pgfqpoint{1.871591in}{1.213778in}}%
\pgfpathlineto{\pgfqpoint{1.893353in}{1.209249in}}%
\pgfpathlineto{\pgfqpoint{1.915115in}{1.204531in}}%
\pgfpathlineto{\pgfqpoint{1.936876in}{1.200063in}}%
\pgfpathlineto{\pgfqpoint{1.958638in}{1.195526in}}%
\pgfpathlineto{\pgfqpoint{1.980400in}{1.190982in}}%
\pgfpathlineto{\pgfqpoint{2.002161in}{1.186438in}}%
\pgfpathlineto{\pgfqpoint{2.023923in}{1.182135in}}%
\pgfpathlineto{\pgfqpoint{2.045685in}{1.177758in}}%
\pgfpathlineto{\pgfqpoint{2.067446in}{1.173429in}}%
\pgfpathlineto{\pgfqpoint{2.089208in}{1.168993in}}%
\pgfpathlineto{\pgfqpoint{2.110970in}{1.164714in}}%
\pgfpathlineto{\pgfqpoint{2.132731in}{1.160554in}}%
\pgfpathlineto{\pgfqpoint{2.154493in}{1.156393in}}%
\pgfpathlineto{\pgfqpoint{2.176255in}{1.151956in}}%
\pgfpathlineto{\pgfqpoint{2.198016in}{1.147729in}}%
\pgfpathlineto{\pgfqpoint{2.219778in}{1.143408in}}%
\pgfpathlineto{\pgfqpoint{2.241540in}{1.139401in}}%
\pgfpathlineto{\pgfqpoint{2.263301in}{1.135166in}}%
\pgfpathlineto{\pgfqpoint{2.285063in}{1.131130in}}%
\pgfpathlineto{\pgfqpoint{2.306825in}{1.127018in}}%
\pgfpathlineto{\pgfqpoint{2.328586in}{1.122912in}}%
\pgfpathlineto{\pgfqpoint{2.350348in}{1.118977in}}%
\pgfpathlineto{\pgfqpoint{2.372110in}{1.114866in}}%
\pgfpathlineto{\pgfqpoint{2.393871in}{1.111038in}}%
\pgfpathlineto{\pgfqpoint{2.415633in}{1.107289in}}%
\pgfpathlineto{\pgfqpoint{2.437395in}{1.103332in}}%
\pgfpathlineto{\pgfqpoint{2.459156in}{1.099293in}}%
\pgfpathlineto{\pgfqpoint{2.480918in}{1.095508in}}%
\pgfpathlineto{\pgfqpoint{2.502680in}{1.091770in}}%
\pgfpathlineto{\pgfqpoint{2.524441in}{1.087835in}}%
\pgfpathlineto{\pgfqpoint{2.546203in}{1.083901in}}%
\pgfpathlineto{\pgfqpoint{2.567965in}{1.080036in}}%
\pgfpathlineto{\pgfqpoint{2.589726in}{1.076242in}}%
\pgfpathlineto{\pgfqpoint{2.611488in}{1.072265in}}%
\pgfpathlineto{\pgfqpoint{2.633250in}{1.068331in}}%
\pgfpathlineto{\pgfqpoint{2.655011in}{1.064685in}}%
\pgfpathlineto{\pgfqpoint{2.676773in}{1.060885in}}%
\pgfpathlineto{\pgfqpoint{2.698535in}{1.057019in}}%
\pgfpathlineto{\pgfqpoint{2.720296in}{1.053463in}}%
\pgfpathlineto{\pgfqpoint{2.742058in}{1.049829in}}%
\pgfpathlineto{\pgfqpoint{2.763820in}{1.046085in}}%
\pgfpathlineto{\pgfqpoint{2.785581in}{1.042299in}}%
\pgfusepath{stroke}%
\end{pgfscope}%
\begin{pgfscope}%
\pgfpathrectangle{\pgfqpoint{0.609415in}{0.422992in}}{\pgfqpoint{2.219690in}{1.226201in}}%
\pgfusepath{clip}%
\pgfsetrectcap%
\pgfsetroundjoin%
\pgfsetlinewidth{1.003750pt}%
\definecolor{currentstroke}{rgb}{1.000000,0.584314,0.000000}%
\pgfsetstrokecolor{currentstroke}%
\pgfsetdash{}{0pt}%
\pgfpathmoveto{\pgfqpoint{0.631176in}{1.593457in}}%
\pgfpathlineto{\pgfqpoint{0.652938in}{1.519031in}}%
\pgfpathlineto{\pgfqpoint{0.674700in}{1.465979in}}%
\pgfpathlineto{\pgfqpoint{0.696461in}{1.427144in}}%
\pgfpathlineto{\pgfqpoint{0.718223in}{1.393009in}}%
\pgfpathlineto{\pgfqpoint{0.739985in}{1.363404in}}%
\pgfpathlineto{\pgfqpoint{0.761746in}{1.339745in}}%
\pgfpathlineto{\pgfqpoint{0.783508in}{1.317183in}}%
\pgfpathlineto{\pgfqpoint{0.805270in}{1.295876in}}%
\pgfpathlineto{\pgfqpoint{0.827031in}{1.276873in}}%
\pgfpathlineto{\pgfqpoint{0.848793in}{1.259978in}}%
\pgfpathlineto{\pgfqpoint{0.870555in}{1.242803in}}%
\pgfpathlineto{\pgfqpoint{0.892316in}{1.227292in}}%
\pgfpathlineto{\pgfqpoint{0.914078in}{1.210912in}}%
\pgfpathlineto{\pgfqpoint{0.935840in}{1.195745in}}%
\pgfpathlineto{\pgfqpoint{0.957601in}{1.180005in}}%
\pgfpathlineto{\pgfqpoint{0.979363in}{1.165950in}}%
\pgfpathlineto{\pgfqpoint{1.001125in}{1.152236in}}%
\pgfpathlineto{\pgfqpoint{1.022886in}{1.138652in}}%
\pgfpathlineto{\pgfqpoint{1.044648in}{1.126492in}}%
\pgfpathlineto{\pgfqpoint{1.066410in}{1.113641in}}%
\pgfpathlineto{\pgfqpoint{1.088171in}{1.100863in}}%
\pgfpathlineto{\pgfqpoint{1.109933in}{1.089441in}}%
\pgfpathlineto{\pgfqpoint{1.131695in}{1.077320in}}%
\pgfpathlineto{\pgfqpoint{1.153456in}{1.065532in}}%
\pgfpathlineto{\pgfqpoint{1.175218in}{1.053924in}}%
\pgfpathlineto{\pgfqpoint{1.196980in}{1.042725in}}%
\pgfpathlineto{\pgfqpoint{1.218741in}{1.031510in}}%
\pgfpathlineto{\pgfqpoint{1.240503in}{1.020553in}}%
\pgfpathlineto{\pgfqpoint{1.262265in}{1.009665in}}%
\pgfpathlineto{\pgfqpoint{1.284026in}{0.999336in}}%
\pgfpathlineto{\pgfqpoint{1.305788in}{0.989334in}}%
\pgfpathlineto{\pgfqpoint{1.327550in}{0.978971in}}%
\pgfpathlineto{\pgfqpoint{1.349311in}{0.968415in}}%
\pgfpathlineto{\pgfqpoint{1.371073in}{0.958782in}}%
\pgfpathlineto{\pgfqpoint{1.392835in}{0.949269in}}%
\pgfpathlineto{\pgfqpoint{1.414596in}{0.939285in}}%
\pgfpathlineto{\pgfqpoint{1.436358in}{0.930178in}}%
\pgfpathlineto{\pgfqpoint{1.458120in}{0.920828in}}%
\pgfpathlineto{\pgfqpoint{1.479881in}{0.911284in}}%
\pgfpathlineto{\pgfqpoint{1.501643in}{0.902209in}}%
\pgfpathlineto{\pgfqpoint{1.523405in}{0.893909in}}%
\pgfpathlineto{\pgfqpoint{1.545166in}{0.885077in}}%
\pgfpathlineto{\pgfqpoint{1.566928in}{0.877094in}}%
\pgfpathlineto{\pgfqpoint{1.588690in}{0.867614in}}%
\pgfpathlineto{\pgfqpoint{1.610451in}{0.858740in}}%
\pgfpathlineto{\pgfqpoint{1.632213in}{0.850781in}}%
\pgfpathlineto{\pgfqpoint{1.653975in}{0.842336in}}%
\pgfpathlineto{\pgfqpoint{1.675736in}{0.834107in}}%
\pgfpathlineto{\pgfqpoint{1.697498in}{0.826103in}}%
\pgfpathlineto{\pgfqpoint{1.719260in}{0.818164in}}%
\pgfpathlineto{\pgfqpoint{1.741021in}{0.810072in}}%
\pgfpathlineto{\pgfqpoint{1.762783in}{0.802046in}}%
\pgfpathlineto{\pgfqpoint{1.784545in}{0.794420in}}%
\pgfpathlineto{\pgfqpoint{1.806306in}{0.786546in}}%
\pgfpathlineto{\pgfqpoint{1.828068in}{0.778958in}}%
\pgfpathlineto{\pgfqpoint{1.849830in}{0.771331in}}%
\pgfpathlineto{\pgfqpoint{1.871591in}{0.763428in}}%
\pgfpathlineto{\pgfqpoint{1.893353in}{0.755878in}}%
\pgfpathlineto{\pgfqpoint{1.915115in}{0.747828in}}%
\pgfpathlineto{\pgfqpoint{1.936876in}{0.740100in}}%
\pgfpathlineto{\pgfqpoint{1.958638in}{0.732316in}}%
\pgfpathlineto{\pgfqpoint{1.980400in}{0.724618in}}%
\pgfpathlineto{\pgfqpoint{2.002161in}{0.717344in}}%
\pgfpathlineto{\pgfqpoint{2.023923in}{0.710281in}}%
\pgfpathlineto{\pgfqpoint{2.045685in}{0.702585in}}%
\pgfpathlineto{\pgfqpoint{2.067446in}{0.695527in}}%
\pgfpathlineto{\pgfqpoint{2.089208in}{0.687980in}}%
\pgfpathlineto{\pgfqpoint{2.110970in}{0.680507in}}%
\pgfpathlineto{\pgfqpoint{2.132731in}{0.673764in}}%
\pgfpathlineto{\pgfqpoint{2.154493in}{0.666579in}}%
\pgfpathlineto{\pgfqpoint{2.176255in}{0.660082in}}%
\pgfpathlineto{\pgfqpoint{2.198016in}{0.653287in}}%
\pgfpathlineto{\pgfqpoint{2.219778in}{0.646352in}}%
\pgfpathlineto{\pgfqpoint{2.241540in}{0.639379in}}%
\pgfpathlineto{\pgfqpoint{2.263301in}{0.632910in}}%
\pgfpathlineto{\pgfqpoint{2.285063in}{0.626418in}}%
\pgfpathlineto{\pgfqpoint{2.306825in}{0.619562in}}%
\pgfpathlineto{\pgfqpoint{2.328586in}{0.613137in}}%
\pgfpathlineto{\pgfqpoint{2.350348in}{0.606347in}}%
\pgfpathlineto{\pgfqpoint{2.372110in}{0.599953in}}%
\pgfpathlineto{\pgfqpoint{2.393871in}{0.593544in}}%
\pgfpathlineto{\pgfqpoint{2.415633in}{0.586692in}}%
\pgfpathlineto{\pgfqpoint{2.437395in}{0.579804in}}%
\pgfpathlineto{\pgfqpoint{2.459156in}{0.573516in}}%
\pgfpathlineto{\pgfqpoint{2.480918in}{0.566887in}}%
\pgfpathlineto{\pgfqpoint{2.502680in}{0.560258in}}%
\pgfpathlineto{\pgfqpoint{2.524441in}{0.553820in}}%
\pgfpathlineto{\pgfqpoint{2.546203in}{0.547305in}}%
\pgfpathlineto{\pgfqpoint{2.567965in}{0.540794in}}%
\pgfpathlineto{\pgfqpoint{2.589726in}{0.534669in}}%
\pgfpathlineto{\pgfqpoint{2.611488in}{0.528263in}}%
\pgfpathlineto{\pgfqpoint{2.633250in}{0.522032in}}%
\pgfpathlineto{\pgfqpoint{2.655011in}{0.515865in}}%
\pgfpathlineto{\pgfqpoint{2.676773in}{0.509756in}}%
\pgfpathlineto{\pgfqpoint{2.698535in}{0.503744in}}%
\pgfpathlineto{\pgfqpoint{2.720296in}{0.497444in}}%
\pgfpathlineto{\pgfqpoint{2.742058in}{0.491254in}}%
\pgfpathlineto{\pgfqpoint{2.763820in}{0.485067in}}%
\pgfpathlineto{\pgfqpoint{2.785581in}{0.478729in}}%
\pgfusepath{stroke}%
\end{pgfscope}%
\begin{pgfscope}%
\pgfpathrectangle{\pgfqpoint{0.609415in}{0.422992in}}{\pgfqpoint{2.219690in}{1.226201in}}%
\pgfusepath{clip}%
\pgfsetrectcap%
\pgfsetroundjoin%
\pgfsetlinewidth{1.003750pt}%
\definecolor{currentstroke}{rgb}{1.000000,0.172549,0.000000}%
\pgfsetstrokecolor{currentstroke}%
\pgfsetdash{}{0pt}%
\pgfpathmoveto{\pgfqpoint{0.631176in}{1.593457in}}%
\pgfpathlineto{\pgfqpoint{0.652938in}{1.586368in}}%
\pgfpathlineto{\pgfqpoint{0.674700in}{1.579736in}}%
\pgfpathlineto{\pgfqpoint{0.696461in}{1.574123in}}%
\pgfpathlineto{\pgfqpoint{0.718223in}{1.569664in}}%
\pgfpathlineto{\pgfqpoint{0.739985in}{1.565659in}}%
\pgfpathlineto{\pgfqpoint{0.761746in}{1.562612in}}%
\pgfpathlineto{\pgfqpoint{0.783508in}{1.559972in}}%
\pgfpathlineto{\pgfqpoint{0.805270in}{1.557824in}}%
\pgfpathlineto{\pgfqpoint{0.827031in}{1.555057in}}%
\pgfpathlineto{\pgfqpoint{0.848793in}{1.552347in}}%
\pgfpathlineto{\pgfqpoint{0.870555in}{1.550269in}}%
\pgfpathlineto{\pgfqpoint{0.892316in}{1.548209in}}%
\pgfpathlineto{\pgfqpoint{0.914078in}{1.546175in}}%
\pgfpathlineto{\pgfqpoint{0.935840in}{1.544031in}}%
\pgfpathlineto{\pgfqpoint{0.957601in}{1.542192in}}%
\pgfpathlineto{\pgfqpoint{0.979363in}{1.540095in}}%
\pgfpathlineto{\pgfqpoint{1.001125in}{1.538052in}}%
\pgfpathlineto{\pgfqpoint{1.022886in}{1.536164in}}%
\pgfpathlineto{\pgfqpoint{1.044648in}{1.534333in}}%
\pgfpathlineto{\pgfqpoint{1.066410in}{1.532662in}}%
\pgfpathlineto{\pgfqpoint{1.088171in}{1.531114in}}%
\pgfpathlineto{\pgfqpoint{1.109933in}{1.529517in}}%
\pgfpathlineto{\pgfqpoint{1.131695in}{1.528128in}}%
\pgfpathlineto{\pgfqpoint{1.153456in}{1.526652in}}%
\pgfpathlineto{\pgfqpoint{1.175218in}{1.525205in}}%
\pgfpathlineto{\pgfqpoint{1.196980in}{1.523678in}}%
\pgfpathlineto{\pgfqpoint{1.218741in}{1.522242in}}%
\pgfpathlineto{\pgfqpoint{1.240503in}{1.520852in}}%
\pgfpathlineto{\pgfqpoint{1.262265in}{1.519291in}}%
\pgfpathlineto{\pgfqpoint{1.284026in}{1.518101in}}%
\pgfpathlineto{\pgfqpoint{1.305788in}{1.516752in}}%
\pgfpathlineto{\pgfqpoint{1.327550in}{1.515345in}}%
\pgfpathlineto{\pgfqpoint{1.349311in}{1.514078in}}%
\pgfpathlineto{\pgfqpoint{1.371073in}{1.512686in}}%
\pgfpathlineto{\pgfqpoint{1.392835in}{1.511357in}}%
\pgfpathlineto{\pgfqpoint{1.414596in}{1.510137in}}%
\pgfpathlineto{\pgfqpoint{1.436358in}{1.508963in}}%
\pgfpathlineto{\pgfqpoint{1.458120in}{1.507776in}}%
\pgfpathlineto{\pgfqpoint{1.479881in}{1.506501in}}%
\pgfpathlineto{\pgfqpoint{1.501643in}{1.505346in}}%
\pgfpathlineto{\pgfqpoint{1.523405in}{1.504226in}}%
\pgfpathlineto{\pgfqpoint{1.545166in}{1.503106in}}%
\pgfpathlineto{\pgfqpoint{1.566928in}{1.502014in}}%
\pgfpathlineto{\pgfqpoint{1.588690in}{1.500938in}}%
\pgfpathlineto{\pgfqpoint{1.610451in}{1.499881in}}%
\pgfpathlineto{\pgfqpoint{1.632213in}{1.498828in}}%
\pgfpathlineto{\pgfqpoint{1.653975in}{1.497801in}}%
\pgfpathlineto{\pgfqpoint{1.675736in}{1.496598in}}%
\pgfpathlineto{\pgfqpoint{1.697498in}{1.495615in}}%
\pgfpathlineto{\pgfqpoint{1.719260in}{1.494618in}}%
\pgfpathlineto{\pgfqpoint{1.741021in}{1.493517in}}%
\pgfpathlineto{\pgfqpoint{1.762783in}{1.492608in}}%
\pgfpathlineto{\pgfqpoint{1.784545in}{1.491629in}}%
\pgfpathlineto{\pgfqpoint{1.806306in}{1.490588in}}%
\pgfpathlineto{\pgfqpoint{1.828068in}{1.489581in}}%
\pgfpathlineto{\pgfqpoint{1.849830in}{1.488667in}}%
\pgfpathlineto{\pgfqpoint{1.871591in}{1.487693in}}%
\pgfpathlineto{\pgfqpoint{1.893353in}{1.486647in}}%
\pgfpathlineto{\pgfqpoint{1.915115in}{1.485652in}}%
\pgfpathlineto{\pgfqpoint{1.936876in}{1.484763in}}%
\pgfpathlineto{\pgfqpoint{1.958638in}{1.483822in}}%
\pgfpathlineto{\pgfqpoint{1.980400in}{1.482831in}}%
\pgfpathlineto{\pgfqpoint{2.002161in}{1.481923in}}%
\pgfpathlineto{\pgfqpoint{2.023923in}{1.481005in}}%
\pgfpathlineto{\pgfqpoint{2.045685in}{1.480063in}}%
\pgfpathlineto{\pgfqpoint{2.067446in}{1.479180in}}%
\pgfpathlineto{\pgfqpoint{2.089208in}{1.478281in}}%
\pgfpathlineto{\pgfqpoint{2.110970in}{1.477358in}}%
\pgfpathlineto{\pgfqpoint{2.132731in}{1.476428in}}%
\pgfpathlineto{\pgfqpoint{2.154493in}{1.475510in}}%
\pgfpathlineto{\pgfqpoint{2.176255in}{1.474725in}}%
\pgfpathlineto{\pgfqpoint{2.198016in}{1.473835in}}%
\pgfpathlineto{\pgfqpoint{2.219778in}{1.472991in}}%
\pgfpathlineto{\pgfqpoint{2.241540in}{1.472217in}}%
\pgfpathlineto{\pgfqpoint{2.263301in}{1.471371in}}%
\pgfpathlineto{\pgfqpoint{2.285063in}{1.470558in}}%
\pgfpathlineto{\pgfqpoint{2.306825in}{1.469704in}}%
\pgfpathlineto{\pgfqpoint{2.328586in}{1.468881in}}%
\pgfpathlineto{\pgfqpoint{2.350348in}{1.468087in}}%
\pgfpathlineto{\pgfqpoint{2.372110in}{1.467279in}}%
\pgfpathlineto{\pgfqpoint{2.393871in}{1.466468in}}%
\pgfpathlineto{\pgfqpoint{2.415633in}{1.465629in}}%
\pgfpathlineto{\pgfqpoint{2.437395in}{1.464735in}}%
\pgfpathlineto{\pgfqpoint{2.459156in}{1.463937in}}%
\pgfpathlineto{\pgfqpoint{2.480918in}{1.463107in}}%
\pgfpathlineto{\pgfqpoint{2.502680in}{1.462280in}}%
\pgfpathlineto{\pgfqpoint{2.524441in}{1.461518in}}%
\pgfpathlineto{\pgfqpoint{2.546203in}{1.460686in}}%
\pgfpathlineto{\pgfqpoint{2.567965in}{1.459806in}}%
\pgfpathlineto{\pgfqpoint{2.589726in}{1.459049in}}%
\pgfpathlineto{\pgfqpoint{2.611488in}{1.458241in}}%
\pgfpathlineto{\pgfqpoint{2.633250in}{1.457372in}}%
\pgfpathlineto{\pgfqpoint{2.655011in}{1.456556in}}%
\pgfpathlineto{\pgfqpoint{2.676773in}{1.455764in}}%
\pgfpathlineto{\pgfqpoint{2.698535in}{1.454994in}}%
\pgfpathlineto{\pgfqpoint{2.720296in}{1.454247in}}%
\pgfpathlineto{\pgfqpoint{2.742058in}{1.453442in}}%
\pgfpathlineto{\pgfqpoint{2.763820in}{1.452577in}}%
\pgfpathlineto{\pgfqpoint{2.785581in}{1.451792in}}%
\pgfusepath{stroke}%
\end{pgfscope}%
\begin{pgfscope}%
\pgfpathrectangle{\pgfqpoint{0.609415in}{0.422992in}}{\pgfqpoint{2.219690in}{1.226201in}}%
\pgfusepath{clip}%
\pgfsetrectcap%
\pgfsetroundjoin%
\pgfsetlinewidth{1.003750pt}%
\definecolor{currentstroke}{rgb}{0.517647,0.356863,0.592157}%
\pgfsetstrokecolor{currentstroke}%
\pgfsetdash{}{0pt}%
\pgfpathmoveto{\pgfqpoint{0.631176in}{1.593457in}}%
\pgfpathlineto{\pgfqpoint{0.652938in}{1.484236in}}%
\pgfpathlineto{\pgfqpoint{0.674700in}{1.408090in}}%
\pgfpathlineto{\pgfqpoint{0.696461in}{1.356640in}}%
\pgfpathlineto{\pgfqpoint{0.718223in}{1.318068in}}%
\pgfpathlineto{\pgfqpoint{0.739985in}{1.289156in}}%
\pgfpathlineto{\pgfqpoint{0.761746in}{1.264658in}}%
\pgfpathlineto{\pgfqpoint{0.783508in}{1.243369in}}%
\pgfpathlineto{\pgfqpoint{0.805270in}{1.225776in}}%
\pgfpathlineto{\pgfqpoint{0.827031in}{1.209137in}}%
\pgfpathlineto{\pgfqpoint{0.848793in}{1.195094in}}%
\pgfpathlineto{\pgfqpoint{0.870555in}{1.181391in}}%
\pgfpathlineto{\pgfqpoint{0.892316in}{1.168781in}}%
\pgfpathlineto{\pgfqpoint{0.914078in}{1.157114in}}%
\pgfpathlineto{\pgfqpoint{0.935840in}{1.146345in}}%
\pgfpathlineto{\pgfqpoint{0.957601in}{1.137118in}}%
\pgfpathlineto{\pgfqpoint{0.979363in}{1.127450in}}%
\pgfpathlineto{\pgfqpoint{1.001125in}{1.118816in}}%
\pgfpathlineto{\pgfqpoint{1.022886in}{1.110566in}}%
\pgfpathlineto{\pgfqpoint{1.044648in}{1.103419in}}%
\pgfpathlineto{\pgfqpoint{1.066410in}{1.096556in}}%
\pgfpathlineto{\pgfqpoint{1.088171in}{1.089512in}}%
\pgfpathlineto{\pgfqpoint{1.109933in}{1.082705in}}%
\pgfpathlineto{\pgfqpoint{1.131695in}{1.075967in}}%
\pgfpathlineto{\pgfqpoint{1.153456in}{1.069912in}}%
\pgfpathlineto{\pgfqpoint{1.175218in}{1.063866in}}%
\pgfpathlineto{\pgfqpoint{1.196980in}{1.058276in}}%
\pgfpathlineto{\pgfqpoint{1.218741in}{1.053258in}}%
\pgfpathlineto{\pgfqpoint{1.240503in}{1.048098in}}%
\pgfpathlineto{\pgfqpoint{1.262265in}{1.043073in}}%
\pgfpathlineto{\pgfqpoint{1.284026in}{1.038082in}}%
\pgfpathlineto{\pgfqpoint{1.305788in}{1.033396in}}%
\pgfpathlineto{\pgfqpoint{1.327550in}{1.028833in}}%
\pgfpathlineto{\pgfqpoint{1.349311in}{1.023990in}}%
\pgfpathlineto{\pgfqpoint{1.371073in}{1.019394in}}%
\pgfpathlineto{\pgfqpoint{1.392835in}{1.014782in}}%
\pgfpathlineto{\pgfqpoint{1.414596in}{1.010626in}}%
\pgfpathlineto{\pgfqpoint{1.436358in}{1.005981in}}%
\pgfpathlineto{\pgfqpoint{1.458120in}{1.001849in}}%
\pgfpathlineto{\pgfqpoint{1.479881in}{0.997449in}}%
\pgfpathlineto{\pgfqpoint{1.501643in}{0.993407in}}%
\pgfpathlineto{\pgfqpoint{1.523405in}{0.989719in}}%
\pgfpathlineto{\pgfqpoint{1.545166in}{0.985646in}}%
\pgfpathlineto{\pgfqpoint{1.566928in}{0.981800in}}%
\pgfpathlineto{\pgfqpoint{1.588690in}{0.978120in}}%
\pgfpathlineto{\pgfqpoint{1.610451in}{0.974393in}}%
\pgfpathlineto{\pgfqpoint{1.632213in}{0.970639in}}%
\pgfpathlineto{\pgfqpoint{1.653975in}{0.967228in}}%
\pgfpathlineto{\pgfqpoint{1.675736in}{0.963325in}}%
\pgfpathlineto{\pgfqpoint{1.697498in}{0.959957in}}%
\pgfpathlineto{\pgfqpoint{1.719260in}{0.956534in}}%
\pgfpathlineto{\pgfqpoint{1.741021in}{0.953136in}}%
\pgfpathlineto{\pgfqpoint{1.762783in}{0.949692in}}%
\pgfpathlineto{\pgfqpoint{1.784545in}{0.946404in}}%
\pgfpathlineto{\pgfqpoint{1.806306in}{0.943237in}}%
\pgfpathlineto{\pgfqpoint{1.828068in}{0.940002in}}%
\pgfpathlineto{\pgfqpoint{1.849830in}{0.936868in}}%
\pgfpathlineto{\pgfqpoint{1.871591in}{0.933510in}}%
\pgfpathlineto{\pgfqpoint{1.893353in}{0.930550in}}%
\pgfpathlineto{\pgfqpoint{1.915115in}{0.927413in}}%
\pgfpathlineto{\pgfqpoint{1.936876in}{0.924132in}}%
\pgfpathlineto{\pgfqpoint{1.958638in}{0.921180in}}%
\pgfpathlineto{\pgfqpoint{1.980400in}{0.918297in}}%
\pgfpathlineto{\pgfqpoint{2.002161in}{0.915472in}}%
\pgfpathlineto{\pgfqpoint{2.023923in}{0.912547in}}%
\pgfpathlineto{\pgfqpoint{2.045685in}{0.909907in}}%
\pgfpathlineto{\pgfqpoint{2.067446in}{0.907418in}}%
\pgfpathlineto{\pgfqpoint{2.089208in}{0.904535in}}%
\pgfpathlineto{\pgfqpoint{2.110970in}{0.901827in}}%
\pgfpathlineto{\pgfqpoint{2.132731in}{0.899222in}}%
\pgfpathlineto{\pgfqpoint{2.154493in}{0.896469in}}%
\pgfpathlineto{\pgfqpoint{2.176255in}{0.893928in}}%
\pgfpathlineto{\pgfqpoint{2.198016in}{0.891241in}}%
\pgfpathlineto{\pgfqpoint{2.219778in}{0.888782in}}%
\pgfpathlineto{\pgfqpoint{2.241540in}{0.886135in}}%
\pgfpathlineto{\pgfqpoint{2.263301in}{0.883390in}}%
\pgfpathlineto{\pgfqpoint{2.285063in}{0.880849in}}%
\pgfpathlineto{\pgfqpoint{2.306825in}{0.878294in}}%
\pgfpathlineto{\pgfqpoint{2.328586in}{0.875757in}}%
\pgfpathlineto{\pgfqpoint{2.350348in}{0.873629in}}%
\pgfpathlineto{\pgfqpoint{2.372110in}{0.871268in}}%
\pgfpathlineto{\pgfqpoint{2.393871in}{0.868805in}}%
\pgfpathlineto{\pgfqpoint{2.415633in}{0.866359in}}%
\pgfpathlineto{\pgfqpoint{2.437395in}{0.863842in}}%
\pgfpathlineto{\pgfqpoint{2.459156in}{0.861524in}}%
\pgfpathlineto{\pgfqpoint{2.480918in}{0.859061in}}%
\pgfpathlineto{\pgfqpoint{2.502680in}{0.856882in}}%
\pgfpathlineto{\pgfqpoint{2.524441in}{0.854708in}}%
\pgfpathlineto{\pgfqpoint{2.546203in}{0.852323in}}%
\pgfpathlineto{\pgfqpoint{2.567965in}{0.849965in}}%
\pgfpathlineto{\pgfqpoint{2.589726in}{0.847763in}}%
\pgfpathlineto{\pgfqpoint{2.611488in}{0.845390in}}%
\pgfpathlineto{\pgfqpoint{2.633250in}{0.843171in}}%
\pgfpathlineto{\pgfqpoint{2.655011in}{0.840980in}}%
\pgfpathlineto{\pgfqpoint{2.676773in}{0.838812in}}%
\pgfpathlineto{\pgfqpoint{2.698535in}{0.836572in}}%
\pgfpathlineto{\pgfqpoint{2.720296in}{0.834517in}}%
\pgfpathlineto{\pgfqpoint{2.742058in}{0.832221in}}%
\pgfpathlineto{\pgfqpoint{2.763820in}{0.829922in}}%
\pgfpathlineto{\pgfqpoint{2.785581in}{0.827835in}}%
\pgfusepath{stroke}%
\end{pgfscope}%
\begin{pgfscope}%
\pgfpathrectangle{\pgfqpoint{0.609415in}{0.422992in}}{\pgfqpoint{2.219690in}{1.226201in}}%
\pgfusepath{clip}%
\pgfsetrectcap%
\pgfsetroundjoin%
\pgfsetlinewidth{1.003750pt}%
\definecolor{currentstroke}{rgb}{0.278431,0.278431,0.278431}%
\pgfsetstrokecolor{currentstroke}%
\pgfsetdash{}{0pt}%
\pgfpathmoveto{\pgfqpoint{0.631176in}{1.593457in}}%
\pgfpathlineto{\pgfqpoint{0.652938in}{1.591127in}}%
\pgfpathlineto{\pgfqpoint{0.674700in}{1.588778in}}%
\pgfpathlineto{\pgfqpoint{0.696461in}{1.586833in}}%
\pgfpathlineto{\pgfqpoint{0.718223in}{1.585018in}}%
\pgfpathlineto{\pgfqpoint{0.739985in}{1.583492in}}%
\pgfpathlineto{\pgfqpoint{0.761746in}{1.581864in}}%
\pgfpathlineto{\pgfqpoint{0.783508in}{1.580372in}}%
\pgfpathlineto{\pgfqpoint{0.805270in}{1.578960in}}%
\pgfpathlineto{\pgfqpoint{0.827031in}{1.577599in}}%
\pgfpathlineto{\pgfqpoint{0.848793in}{1.576106in}}%
\pgfpathlineto{\pgfqpoint{0.870555in}{1.574632in}}%
\pgfpathlineto{\pgfqpoint{0.892316in}{1.573396in}}%
\pgfpathlineto{\pgfqpoint{0.914078in}{1.572256in}}%
\pgfpathlineto{\pgfqpoint{0.935840in}{1.570967in}}%
\pgfpathlineto{\pgfqpoint{0.957601in}{1.569548in}}%
\pgfpathlineto{\pgfqpoint{0.979363in}{1.568278in}}%
\pgfpathlineto{\pgfqpoint{1.001125in}{1.566874in}}%
\pgfpathlineto{\pgfqpoint{1.022886in}{1.565597in}}%
\pgfpathlineto{\pgfqpoint{1.044648in}{1.564396in}}%
\pgfpathlineto{\pgfqpoint{1.066410in}{1.563136in}}%
\pgfpathlineto{\pgfqpoint{1.088171in}{1.562002in}}%
\pgfpathlineto{\pgfqpoint{1.109933in}{1.560786in}}%
\pgfpathlineto{\pgfqpoint{1.131695in}{1.559567in}}%
\pgfpathlineto{\pgfqpoint{1.153456in}{1.558396in}}%
\pgfpathlineto{\pgfqpoint{1.175218in}{1.557210in}}%
\pgfpathlineto{\pgfqpoint{1.196980in}{1.556013in}}%
\pgfpathlineto{\pgfqpoint{1.218741in}{1.554869in}}%
\pgfpathlineto{\pgfqpoint{1.240503in}{1.553693in}}%
\pgfpathlineto{\pgfqpoint{1.262265in}{1.552441in}}%
\pgfpathlineto{\pgfqpoint{1.284026in}{1.551260in}}%
\pgfpathlineto{\pgfqpoint{1.305788in}{1.550135in}}%
\pgfpathlineto{\pgfqpoint{1.327550in}{1.548996in}}%
\pgfpathlineto{\pgfqpoint{1.349311in}{1.547858in}}%
\pgfpathlineto{\pgfqpoint{1.371073in}{1.546739in}}%
\pgfpathlineto{\pgfqpoint{1.392835in}{1.545673in}}%
\pgfpathlineto{\pgfqpoint{1.414596in}{1.544598in}}%
\pgfpathlineto{\pgfqpoint{1.436358in}{1.543542in}}%
\pgfpathlineto{\pgfqpoint{1.458120in}{1.542402in}}%
\pgfpathlineto{\pgfqpoint{1.479881in}{1.541322in}}%
\pgfpathlineto{\pgfqpoint{1.501643in}{1.540212in}}%
\pgfpathlineto{\pgfqpoint{1.523405in}{1.539108in}}%
\pgfpathlineto{\pgfqpoint{1.545166in}{1.537974in}}%
\pgfpathlineto{\pgfqpoint{1.566928in}{1.536839in}}%
\pgfpathlineto{\pgfqpoint{1.588690in}{1.535768in}}%
\pgfpathlineto{\pgfqpoint{1.610451in}{1.534729in}}%
\pgfpathlineto{\pgfqpoint{1.632213in}{1.533744in}}%
\pgfpathlineto{\pgfqpoint{1.653975in}{1.532637in}}%
\pgfpathlineto{\pgfqpoint{1.675736in}{1.531527in}}%
\pgfpathlineto{\pgfqpoint{1.697498in}{1.530437in}}%
\pgfpathlineto{\pgfqpoint{1.719260in}{1.529280in}}%
\pgfpathlineto{\pgfqpoint{1.741021in}{1.528279in}}%
\pgfpathlineto{\pgfqpoint{1.762783in}{1.527285in}}%
\pgfpathlineto{\pgfqpoint{1.784545in}{1.526255in}}%
\pgfpathlineto{\pgfqpoint{1.806306in}{1.525181in}}%
\pgfpathlineto{\pgfqpoint{1.828068in}{1.524172in}}%
\pgfpathlineto{\pgfqpoint{1.849830in}{1.523179in}}%
\pgfpathlineto{\pgfqpoint{1.871591in}{1.522197in}}%
\pgfpathlineto{\pgfqpoint{1.893353in}{1.521133in}}%
\pgfpathlineto{\pgfqpoint{1.915115in}{1.520098in}}%
\pgfpathlineto{\pgfqpoint{1.936876in}{1.519032in}}%
\pgfpathlineto{\pgfqpoint{1.958638in}{1.517968in}}%
\pgfpathlineto{\pgfqpoint{1.980400in}{1.516884in}}%
\pgfpathlineto{\pgfqpoint{2.002161in}{1.515825in}}%
\pgfpathlineto{\pgfqpoint{2.023923in}{1.514794in}}%
\pgfpathlineto{\pgfqpoint{2.045685in}{1.513776in}}%
\pgfpathlineto{\pgfqpoint{2.067446in}{1.512783in}}%
\pgfpathlineto{\pgfqpoint{2.089208in}{1.511814in}}%
\pgfpathlineto{\pgfqpoint{2.110970in}{1.510791in}}%
\pgfpathlineto{\pgfqpoint{2.132731in}{1.509857in}}%
\pgfpathlineto{\pgfqpoint{2.154493in}{1.508879in}}%
\pgfpathlineto{\pgfqpoint{2.176255in}{1.507823in}}%
\pgfpathlineto{\pgfqpoint{2.198016in}{1.506891in}}%
\pgfpathlineto{\pgfqpoint{2.219778in}{1.505836in}}%
\pgfpathlineto{\pgfqpoint{2.241540in}{1.504872in}}%
\pgfpathlineto{\pgfqpoint{2.263301in}{1.503835in}}%
\pgfpathlineto{\pgfqpoint{2.285063in}{1.502830in}}%
\pgfpathlineto{\pgfqpoint{2.306825in}{1.501885in}}%
\pgfpathlineto{\pgfqpoint{2.328586in}{1.500898in}}%
\pgfpathlineto{\pgfqpoint{2.350348in}{1.499905in}}%
\pgfpathlineto{\pgfqpoint{2.372110in}{1.498858in}}%
\pgfpathlineto{\pgfqpoint{2.393871in}{1.497936in}}%
\pgfpathlineto{\pgfqpoint{2.415633in}{1.496931in}}%
\pgfpathlineto{\pgfqpoint{2.437395in}{1.495982in}}%
\pgfpathlineto{\pgfqpoint{2.459156in}{1.495015in}}%
\pgfpathlineto{\pgfqpoint{2.480918in}{1.494021in}}%
\pgfpathlineto{\pgfqpoint{2.502680in}{1.493060in}}%
\pgfpathlineto{\pgfqpoint{2.524441in}{1.492033in}}%
\pgfpathlineto{\pgfqpoint{2.546203in}{1.491032in}}%
\pgfpathlineto{\pgfqpoint{2.567965in}{1.490005in}}%
\pgfpathlineto{\pgfqpoint{2.589726in}{1.488984in}}%
\pgfpathlineto{\pgfqpoint{2.611488in}{1.488024in}}%
\pgfpathlineto{\pgfqpoint{2.633250in}{1.487004in}}%
\pgfpathlineto{\pgfqpoint{2.655011in}{1.485959in}}%
\pgfpathlineto{\pgfqpoint{2.676773in}{1.484923in}}%
\pgfpathlineto{\pgfqpoint{2.698535in}{1.483938in}}%
\pgfpathlineto{\pgfqpoint{2.720296in}{1.482938in}}%
\pgfpathlineto{\pgfqpoint{2.742058in}{1.481938in}}%
\pgfpathlineto{\pgfqpoint{2.763820in}{1.480940in}}%
\pgfpathlineto{\pgfqpoint{2.785581in}{1.479901in}}%
\pgfusepath{stroke}%
\end{pgfscope}%
\begin{pgfscope}%
\pgfsetrectcap%
\pgfsetmiterjoin%
\pgfsetlinewidth{0.501875pt}%
\definecolor{currentstroke}{rgb}{0.000000,0.000000,0.000000}%
\pgfsetstrokecolor{currentstroke}%
\pgfsetdash{}{0pt}%
\pgfpathmoveto{\pgfqpoint{0.609415in}{0.422992in}}%
\pgfpathlineto{\pgfqpoint{0.609415in}{1.649193in}}%
\pgfusepath{stroke}%
\end{pgfscope}%
\begin{pgfscope}%
\pgfsetrectcap%
\pgfsetmiterjoin%
\pgfsetlinewidth{0.501875pt}%
\definecolor{currentstroke}{rgb}{0.000000,0.000000,0.000000}%
\pgfsetstrokecolor{currentstroke}%
\pgfsetdash{}{0pt}%
\pgfpathmoveto{\pgfqpoint{2.829105in}{0.422992in}}%
\pgfpathlineto{\pgfqpoint{2.829105in}{1.649193in}}%
\pgfusepath{stroke}%
\end{pgfscope}%
\begin{pgfscope}%
\pgfsetrectcap%
\pgfsetmiterjoin%
\pgfsetlinewidth{0.501875pt}%
\definecolor{currentstroke}{rgb}{0.000000,0.000000,0.000000}%
\pgfsetstrokecolor{currentstroke}%
\pgfsetdash{}{0pt}%
\pgfpathmoveto{\pgfqpoint{0.609415in}{0.422992in}}%
\pgfpathlineto{\pgfqpoint{2.829105in}{0.422992in}}%
\pgfusepath{stroke}%
\end{pgfscope}%
\begin{pgfscope}%
\pgfsetrectcap%
\pgfsetmiterjoin%
\pgfsetlinewidth{0.501875pt}%
\definecolor{currentstroke}{rgb}{0.000000,0.000000,0.000000}%
\pgfsetstrokecolor{currentstroke}%
\pgfsetdash{}{0pt}%
\pgfpathmoveto{\pgfqpoint{0.609415in}{1.649193in}}%
\pgfpathlineto{\pgfqpoint{2.829105in}{1.649193in}}%
\pgfusepath{stroke}%
\end{pgfscope}%
\begin{pgfscope}%
\definecolor{textcolor}{rgb}{0.000000,0.000000,0.000000}%
\pgfsetstrokecolor{textcolor}%
\pgfsetfillcolor{textcolor}%
\pgftext[x=1.719260in,y=1.732526in,,base]{\color{textcolor}\rmfamily\fontsize{12.000000}{14.400000}\selectfont Kontinuität}%
\end{pgfscope}%
\begin{pgfscope}%
\pgfsetbuttcap%
\pgfsetmiterjoin%
\definecolor{currentfill}{rgb}{1.000000,1.000000,1.000000}%
\pgfsetfillcolor{currentfill}%
\pgfsetlinewidth{0.000000pt}%
\definecolor{currentstroke}{rgb}{0.000000,0.000000,0.000000}%
\pgfsetstrokecolor{currentstroke}%
\pgfsetstrokeopacity{0.000000}%
\pgfsetdash{}{0pt}%
\pgfpathmoveto{\pgfqpoint{3.454822in}{0.422992in}}%
\pgfpathlineto{\pgfqpoint{5.674512in}{0.422992in}}%
\pgfpathlineto{\pgfqpoint{5.674512in}{3.574193in}}%
\pgfpathlineto{\pgfqpoint{3.454822in}{3.574193in}}%
\pgfpathlineto{\pgfqpoint{3.454822in}{0.422992in}}%
\pgfpathclose%
\pgfusepath{fill}%
\end{pgfscope}%
\begin{pgfscope}%
\pgfsetbuttcap%
\pgfsetroundjoin%
\definecolor{currentfill}{rgb}{0.000000,0.000000,0.000000}%
\pgfsetfillcolor{currentfill}%
\pgfsetlinewidth{0.501875pt}%
\definecolor{currentstroke}{rgb}{0.000000,0.000000,0.000000}%
\pgfsetstrokecolor{currentstroke}%
\pgfsetdash{}{0pt}%
\pgfsys@defobject{currentmarker}{\pgfqpoint{0.000000in}{0.000000in}}{\pgfqpoint{0.000000in}{0.041667in}}{%
\pgfpathmoveto{\pgfqpoint{0.000000in}{0.000000in}}%
\pgfpathlineto{\pgfqpoint{0.000000in}{0.041667in}}%
\pgfusepath{stroke,fill}%
}%
\begin{pgfscope}%
\pgfsys@transformshift{3.454822in}{0.422992in}%
\pgfsys@useobject{currentmarker}{}%
\end{pgfscope}%
\end{pgfscope}%
\begin{pgfscope}%
\pgfsetbuttcap%
\pgfsetroundjoin%
\definecolor{currentfill}{rgb}{0.000000,0.000000,0.000000}%
\pgfsetfillcolor{currentfill}%
\pgfsetlinewidth{0.501875pt}%
\definecolor{currentstroke}{rgb}{0.000000,0.000000,0.000000}%
\pgfsetstrokecolor{currentstroke}%
\pgfsetdash{}{0pt}%
\pgfsys@defobject{currentmarker}{\pgfqpoint{0.000000in}{-0.041667in}}{\pgfqpoint{0.000000in}{0.000000in}}{%
\pgfpathmoveto{\pgfqpoint{0.000000in}{0.000000in}}%
\pgfpathlineto{\pgfqpoint{0.000000in}{-0.041667in}}%
\pgfusepath{stroke,fill}%
}%
\begin{pgfscope}%
\pgfsys@transformshift{3.454822in}{3.574193in}%
\pgfsys@useobject{currentmarker}{}%
\end{pgfscope}%
\end{pgfscope}%
\begin{pgfscope}%
\definecolor{textcolor}{rgb}{0.000000,0.000000,0.000000}%
\pgfsetstrokecolor{textcolor}%
\pgfsetfillcolor{textcolor}%
\pgftext[x=3.454822in,y=0.374381in,,top]{\color{textcolor}\rmfamily\fontsize{10.000000}{12.000000}\selectfont \(\displaystyle {0}\)}%
\end{pgfscope}%
\begin{pgfscope}%
\pgfsetbuttcap%
\pgfsetroundjoin%
\definecolor{currentfill}{rgb}{0.000000,0.000000,0.000000}%
\pgfsetfillcolor{currentfill}%
\pgfsetlinewidth{0.501875pt}%
\definecolor{currentstroke}{rgb}{0.000000,0.000000,0.000000}%
\pgfsetstrokecolor{currentstroke}%
\pgfsetdash{}{0pt}%
\pgfsys@defobject{currentmarker}{\pgfqpoint{0.000000in}{0.000000in}}{\pgfqpoint{0.000000in}{0.041667in}}{%
\pgfpathmoveto{\pgfqpoint{0.000000in}{0.000000in}}%
\pgfpathlineto{\pgfqpoint{0.000000in}{0.041667in}}%
\pgfusepath{stroke,fill}%
}%
\begin{pgfscope}%
\pgfsys@transformshift{3.890055in}{0.422992in}%
\pgfsys@useobject{currentmarker}{}%
\end{pgfscope}%
\end{pgfscope}%
\begin{pgfscope}%
\pgfsetbuttcap%
\pgfsetroundjoin%
\definecolor{currentfill}{rgb}{0.000000,0.000000,0.000000}%
\pgfsetfillcolor{currentfill}%
\pgfsetlinewidth{0.501875pt}%
\definecolor{currentstroke}{rgb}{0.000000,0.000000,0.000000}%
\pgfsetstrokecolor{currentstroke}%
\pgfsetdash{}{0pt}%
\pgfsys@defobject{currentmarker}{\pgfqpoint{0.000000in}{-0.041667in}}{\pgfqpoint{0.000000in}{0.000000in}}{%
\pgfpathmoveto{\pgfqpoint{0.000000in}{0.000000in}}%
\pgfpathlineto{\pgfqpoint{0.000000in}{-0.041667in}}%
\pgfusepath{stroke,fill}%
}%
\begin{pgfscope}%
\pgfsys@transformshift{3.890055in}{3.574193in}%
\pgfsys@useobject{currentmarker}{}%
\end{pgfscope}%
\end{pgfscope}%
\begin{pgfscope}%
\definecolor{textcolor}{rgb}{0.000000,0.000000,0.000000}%
\pgfsetstrokecolor{textcolor}%
\pgfsetfillcolor{textcolor}%
\pgftext[x=3.890055in,y=0.374381in,,top]{\color{textcolor}\rmfamily\fontsize{10.000000}{12.000000}\selectfont \(\displaystyle {20}\)}%
\end{pgfscope}%
\begin{pgfscope}%
\pgfsetbuttcap%
\pgfsetroundjoin%
\definecolor{currentfill}{rgb}{0.000000,0.000000,0.000000}%
\pgfsetfillcolor{currentfill}%
\pgfsetlinewidth{0.501875pt}%
\definecolor{currentstroke}{rgb}{0.000000,0.000000,0.000000}%
\pgfsetstrokecolor{currentstroke}%
\pgfsetdash{}{0pt}%
\pgfsys@defobject{currentmarker}{\pgfqpoint{0.000000in}{0.000000in}}{\pgfqpoint{0.000000in}{0.041667in}}{%
\pgfpathmoveto{\pgfqpoint{0.000000in}{0.000000in}}%
\pgfpathlineto{\pgfqpoint{0.000000in}{0.041667in}}%
\pgfusepath{stroke,fill}%
}%
\begin{pgfscope}%
\pgfsys@transformshift{4.325289in}{0.422992in}%
\pgfsys@useobject{currentmarker}{}%
\end{pgfscope}%
\end{pgfscope}%
\begin{pgfscope}%
\pgfsetbuttcap%
\pgfsetroundjoin%
\definecolor{currentfill}{rgb}{0.000000,0.000000,0.000000}%
\pgfsetfillcolor{currentfill}%
\pgfsetlinewidth{0.501875pt}%
\definecolor{currentstroke}{rgb}{0.000000,0.000000,0.000000}%
\pgfsetstrokecolor{currentstroke}%
\pgfsetdash{}{0pt}%
\pgfsys@defobject{currentmarker}{\pgfqpoint{0.000000in}{-0.041667in}}{\pgfqpoint{0.000000in}{0.000000in}}{%
\pgfpathmoveto{\pgfqpoint{0.000000in}{0.000000in}}%
\pgfpathlineto{\pgfqpoint{0.000000in}{-0.041667in}}%
\pgfusepath{stroke,fill}%
}%
\begin{pgfscope}%
\pgfsys@transformshift{4.325289in}{3.574193in}%
\pgfsys@useobject{currentmarker}{}%
\end{pgfscope}%
\end{pgfscope}%
\begin{pgfscope}%
\definecolor{textcolor}{rgb}{0.000000,0.000000,0.000000}%
\pgfsetstrokecolor{textcolor}%
\pgfsetfillcolor{textcolor}%
\pgftext[x=4.325289in,y=0.374381in,,top]{\color{textcolor}\rmfamily\fontsize{10.000000}{12.000000}\selectfont \(\displaystyle {40}\)}%
\end{pgfscope}%
\begin{pgfscope}%
\pgfsetbuttcap%
\pgfsetroundjoin%
\definecolor{currentfill}{rgb}{0.000000,0.000000,0.000000}%
\pgfsetfillcolor{currentfill}%
\pgfsetlinewidth{0.501875pt}%
\definecolor{currentstroke}{rgb}{0.000000,0.000000,0.000000}%
\pgfsetstrokecolor{currentstroke}%
\pgfsetdash{}{0pt}%
\pgfsys@defobject{currentmarker}{\pgfqpoint{0.000000in}{0.000000in}}{\pgfqpoint{0.000000in}{0.041667in}}{%
\pgfpathmoveto{\pgfqpoint{0.000000in}{0.000000in}}%
\pgfpathlineto{\pgfqpoint{0.000000in}{0.041667in}}%
\pgfusepath{stroke,fill}%
}%
\begin{pgfscope}%
\pgfsys@transformshift{4.760522in}{0.422992in}%
\pgfsys@useobject{currentmarker}{}%
\end{pgfscope}%
\end{pgfscope}%
\begin{pgfscope}%
\pgfsetbuttcap%
\pgfsetroundjoin%
\definecolor{currentfill}{rgb}{0.000000,0.000000,0.000000}%
\pgfsetfillcolor{currentfill}%
\pgfsetlinewidth{0.501875pt}%
\definecolor{currentstroke}{rgb}{0.000000,0.000000,0.000000}%
\pgfsetstrokecolor{currentstroke}%
\pgfsetdash{}{0pt}%
\pgfsys@defobject{currentmarker}{\pgfqpoint{0.000000in}{-0.041667in}}{\pgfqpoint{0.000000in}{0.000000in}}{%
\pgfpathmoveto{\pgfqpoint{0.000000in}{0.000000in}}%
\pgfpathlineto{\pgfqpoint{0.000000in}{-0.041667in}}%
\pgfusepath{stroke,fill}%
}%
\begin{pgfscope}%
\pgfsys@transformshift{4.760522in}{3.574193in}%
\pgfsys@useobject{currentmarker}{}%
\end{pgfscope}%
\end{pgfscope}%
\begin{pgfscope}%
\definecolor{textcolor}{rgb}{0.000000,0.000000,0.000000}%
\pgfsetstrokecolor{textcolor}%
\pgfsetfillcolor{textcolor}%
\pgftext[x=4.760522in,y=0.374381in,,top]{\color{textcolor}\rmfamily\fontsize{10.000000}{12.000000}\selectfont \(\displaystyle {60}\)}%
\end{pgfscope}%
\begin{pgfscope}%
\pgfsetbuttcap%
\pgfsetroundjoin%
\definecolor{currentfill}{rgb}{0.000000,0.000000,0.000000}%
\pgfsetfillcolor{currentfill}%
\pgfsetlinewidth{0.501875pt}%
\definecolor{currentstroke}{rgb}{0.000000,0.000000,0.000000}%
\pgfsetstrokecolor{currentstroke}%
\pgfsetdash{}{0pt}%
\pgfsys@defobject{currentmarker}{\pgfqpoint{0.000000in}{0.000000in}}{\pgfqpoint{0.000000in}{0.041667in}}{%
\pgfpathmoveto{\pgfqpoint{0.000000in}{0.000000in}}%
\pgfpathlineto{\pgfqpoint{0.000000in}{0.041667in}}%
\pgfusepath{stroke,fill}%
}%
\begin{pgfscope}%
\pgfsys@transformshift{5.195755in}{0.422992in}%
\pgfsys@useobject{currentmarker}{}%
\end{pgfscope}%
\end{pgfscope}%
\begin{pgfscope}%
\pgfsetbuttcap%
\pgfsetroundjoin%
\definecolor{currentfill}{rgb}{0.000000,0.000000,0.000000}%
\pgfsetfillcolor{currentfill}%
\pgfsetlinewidth{0.501875pt}%
\definecolor{currentstroke}{rgb}{0.000000,0.000000,0.000000}%
\pgfsetstrokecolor{currentstroke}%
\pgfsetdash{}{0pt}%
\pgfsys@defobject{currentmarker}{\pgfqpoint{0.000000in}{-0.041667in}}{\pgfqpoint{0.000000in}{0.000000in}}{%
\pgfpathmoveto{\pgfqpoint{0.000000in}{0.000000in}}%
\pgfpathlineto{\pgfqpoint{0.000000in}{-0.041667in}}%
\pgfusepath{stroke,fill}%
}%
\begin{pgfscope}%
\pgfsys@transformshift{5.195755in}{3.574193in}%
\pgfsys@useobject{currentmarker}{}%
\end{pgfscope}%
\end{pgfscope}%
\begin{pgfscope}%
\definecolor{textcolor}{rgb}{0.000000,0.000000,0.000000}%
\pgfsetstrokecolor{textcolor}%
\pgfsetfillcolor{textcolor}%
\pgftext[x=5.195755in,y=0.374381in,,top]{\color{textcolor}\rmfamily\fontsize{10.000000}{12.000000}\selectfont \(\displaystyle {80}\)}%
\end{pgfscope}%
\begin{pgfscope}%
\pgfsetbuttcap%
\pgfsetroundjoin%
\definecolor{currentfill}{rgb}{0.000000,0.000000,0.000000}%
\pgfsetfillcolor{currentfill}%
\pgfsetlinewidth{0.501875pt}%
\definecolor{currentstroke}{rgb}{0.000000,0.000000,0.000000}%
\pgfsetstrokecolor{currentstroke}%
\pgfsetdash{}{0pt}%
\pgfsys@defobject{currentmarker}{\pgfqpoint{0.000000in}{0.000000in}}{\pgfqpoint{0.000000in}{0.041667in}}{%
\pgfpathmoveto{\pgfqpoint{0.000000in}{0.000000in}}%
\pgfpathlineto{\pgfqpoint{0.000000in}{0.041667in}}%
\pgfusepath{stroke,fill}%
}%
\begin{pgfscope}%
\pgfsys@transformshift{5.630989in}{0.422992in}%
\pgfsys@useobject{currentmarker}{}%
\end{pgfscope}%
\end{pgfscope}%
\begin{pgfscope}%
\pgfsetbuttcap%
\pgfsetroundjoin%
\definecolor{currentfill}{rgb}{0.000000,0.000000,0.000000}%
\pgfsetfillcolor{currentfill}%
\pgfsetlinewidth{0.501875pt}%
\definecolor{currentstroke}{rgb}{0.000000,0.000000,0.000000}%
\pgfsetstrokecolor{currentstroke}%
\pgfsetdash{}{0pt}%
\pgfsys@defobject{currentmarker}{\pgfqpoint{0.000000in}{-0.041667in}}{\pgfqpoint{0.000000in}{0.000000in}}{%
\pgfpathmoveto{\pgfqpoint{0.000000in}{0.000000in}}%
\pgfpathlineto{\pgfqpoint{0.000000in}{-0.041667in}}%
\pgfusepath{stroke,fill}%
}%
\begin{pgfscope}%
\pgfsys@transformshift{5.630989in}{3.574193in}%
\pgfsys@useobject{currentmarker}{}%
\end{pgfscope}%
\end{pgfscope}%
\begin{pgfscope}%
\definecolor{textcolor}{rgb}{0.000000,0.000000,0.000000}%
\pgfsetstrokecolor{textcolor}%
\pgfsetfillcolor{textcolor}%
\pgftext[x=5.630989in,y=0.374381in,,top]{\color{textcolor}\rmfamily\fontsize{10.000000}{12.000000}\selectfont \(\displaystyle {100}\)}%
\end{pgfscope}%
\begin{pgfscope}%
\pgfsetbuttcap%
\pgfsetroundjoin%
\definecolor{currentfill}{rgb}{0.000000,0.000000,0.000000}%
\pgfsetfillcolor{currentfill}%
\pgfsetlinewidth{0.501875pt}%
\definecolor{currentstroke}{rgb}{0.000000,0.000000,0.000000}%
\pgfsetstrokecolor{currentstroke}%
\pgfsetdash{}{0pt}%
\pgfsys@defobject{currentmarker}{\pgfqpoint{0.000000in}{0.000000in}}{\pgfqpoint{0.000000in}{0.020833in}}{%
\pgfpathmoveto{\pgfqpoint{0.000000in}{0.000000in}}%
\pgfpathlineto{\pgfqpoint{0.000000in}{0.020833in}}%
\pgfusepath{stroke,fill}%
}%
\begin{pgfscope}%
\pgfsys@transformshift{3.563630in}{0.422992in}%
\pgfsys@useobject{currentmarker}{}%
\end{pgfscope}%
\end{pgfscope}%
\begin{pgfscope}%
\pgfsetbuttcap%
\pgfsetroundjoin%
\definecolor{currentfill}{rgb}{0.000000,0.000000,0.000000}%
\pgfsetfillcolor{currentfill}%
\pgfsetlinewidth{0.501875pt}%
\definecolor{currentstroke}{rgb}{0.000000,0.000000,0.000000}%
\pgfsetstrokecolor{currentstroke}%
\pgfsetdash{}{0pt}%
\pgfsys@defobject{currentmarker}{\pgfqpoint{0.000000in}{-0.020833in}}{\pgfqpoint{0.000000in}{0.000000in}}{%
\pgfpathmoveto{\pgfqpoint{0.000000in}{0.000000in}}%
\pgfpathlineto{\pgfqpoint{0.000000in}{-0.020833in}}%
\pgfusepath{stroke,fill}%
}%
\begin{pgfscope}%
\pgfsys@transformshift{3.563630in}{3.574193in}%
\pgfsys@useobject{currentmarker}{}%
\end{pgfscope}%
\end{pgfscope}%
\begin{pgfscope}%
\pgfsetbuttcap%
\pgfsetroundjoin%
\definecolor{currentfill}{rgb}{0.000000,0.000000,0.000000}%
\pgfsetfillcolor{currentfill}%
\pgfsetlinewidth{0.501875pt}%
\definecolor{currentstroke}{rgb}{0.000000,0.000000,0.000000}%
\pgfsetstrokecolor{currentstroke}%
\pgfsetdash{}{0pt}%
\pgfsys@defobject{currentmarker}{\pgfqpoint{0.000000in}{0.000000in}}{\pgfqpoint{0.000000in}{0.020833in}}{%
\pgfpathmoveto{\pgfqpoint{0.000000in}{0.000000in}}%
\pgfpathlineto{\pgfqpoint{0.000000in}{0.020833in}}%
\pgfusepath{stroke,fill}%
}%
\begin{pgfscope}%
\pgfsys@transformshift{3.672439in}{0.422992in}%
\pgfsys@useobject{currentmarker}{}%
\end{pgfscope}%
\end{pgfscope}%
\begin{pgfscope}%
\pgfsetbuttcap%
\pgfsetroundjoin%
\definecolor{currentfill}{rgb}{0.000000,0.000000,0.000000}%
\pgfsetfillcolor{currentfill}%
\pgfsetlinewidth{0.501875pt}%
\definecolor{currentstroke}{rgb}{0.000000,0.000000,0.000000}%
\pgfsetstrokecolor{currentstroke}%
\pgfsetdash{}{0pt}%
\pgfsys@defobject{currentmarker}{\pgfqpoint{0.000000in}{-0.020833in}}{\pgfqpoint{0.000000in}{0.000000in}}{%
\pgfpathmoveto{\pgfqpoint{0.000000in}{0.000000in}}%
\pgfpathlineto{\pgfqpoint{0.000000in}{-0.020833in}}%
\pgfusepath{stroke,fill}%
}%
\begin{pgfscope}%
\pgfsys@transformshift{3.672439in}{3.574193in}%
\pgfsys@useobject{currentmarker}{}%
\end{pgfscope}%
\end{pgfscope}%
\begin{pgfscope}%
\pgfsetbuttcap%
\pgfsetroundjoin%
\definecolor{currentfill}{rgb}{0.000000,0.000000,0.000000}%
\pgfsetfillcolor{currentfill}%
\pgfsetlinewidth{0.501875pt}%
\definecolor{currentstroke}{rgb}{0.000000,0.000000,0.000000}%
\pgfsetstrokecolor{currentstroke}%
\pgfsetdash{}{0pt}%
\pgfsys@defobject{currentmarker}{\pgfqpoint{0.000000in}{0.000000in}}{\pgfqpoint{0.000000in}{0.020833in}}{%
\pgfpathmoveto{\pgfqpoint{0.000000in}{0.000000in}}%
\pgfpathlineto{\pgfqpoint{0.000000in}{0.020833in}}%
\pgfusepath{stroke,fill}%
}%
\begin{pgfscope}%
\pgfsys@transformshift{3.781247in}{0.422992in}%
\pgfsys@useobject{currentmarker}{}%
\end{pgfscope}%
\end{pgfscope}%
\begin{pgfscope}%
\pgfsetbuttcap%
\pgfsetroundjoin%
\definecolor{currentfill}{rgb}{0.000000,0.000000,0.000000}%
\pgfsetfillcolor{currentfill}%
\pgfsetlinewidth{0.501875pt}%
\definecolor{currentstroke}{rgb}{0.000000,0.000000,0.000000}%
\pgfsetstrokecolor{currentstroke}%
\pgfsetdash{}{0pt}%
\pgfsys@defobject{currentmarker}{\pgfqpoint{0.000000in}{-0.020833in}}{\pgfqpoint{0.000000in}{0.000000in}}{%
\pgfpathmoveto{\pgfqpoint{0.000000in}{0.000000in}}%
\pgfpathlineto{\pgfqpoint{0.000000in}{-0.020833in}}%
\pgfusepath{stroke,fill}%
}%
\begin{pgfscope}%
\pgfsys@transformshift{3.781247in}{3.574193in}%
\pgfsys@useobject{currentmarker}{}%
\end{pgfscope}%
\end{pgfscope}%
\begin{pgfscope}%
\pgfsetbuttcap%
\pgfsetroundjoin%
\definecolor{currentfill}{rgb}{0.000000,0.000000,0.000000}%
\pgfsetfillcolor{currentfill}%
\pgfsetlinewidth{0.501875pt}%
\definecolor{currentstroke}{rgb}{0.000000,0.000000,0.000000}%
\pgfsetstrokecolor{currentstroke}%
\pgfsetdash{}{0pt}%
\pgfsys@defobject{currentmarker}{\pgfqpoint{0.000000in}{0.000000in}}{\pgfqpoint{0.000000in}{0.020833in}}{%
\pgfpathmoveto{\pgfqpoint{0.000000in}{0.000000in}}%
\pgfpathlineto{\pgfqpoint{0.000000in}{0.020833in}}%
\pgfusepath{stroke,fill}%
}%
\begin{pgfscope}%
\pgfsys@transformshift{3.998864in}{0.422992in}%
\pgfsys@useobject{currentmarker}{}%
\end{pgfscope}%
\end{pgfscope}%
\begin{pgfscope}%
\pgfsetbuttcap%
\pgfsetroundjoin%
\definecolor{currentfill}{rgb}{0.000000,0.000000,0.000000}%
\pgfsetfillcolor{currentfill}%
\pgfsetlinewidth{0.501875pt}%
\definecolor{currentstroke}{rgb}{0.000000,0.000000,0.000000}%
\pgfsetstrokecolor{currentstroke}%
\pgfsetdash{}{0pt}%
\pgfsys@defobject{currentmarker}{\pgfqpoint{0.000000in}{-0.020833in}}{\pgfqpoint{0.000000in}{0.000000in}}{%
\pgfpathmoveto{\pgfqpoint{0.000000in}{0.000000in}}%
\pgfpathlineto{\pgfqpoint{0.000000in}{-0.020833in}}%
\pgfusepath{stroke,fill}%
}%
\begin{pgfscope}%
\pgfsys@transformshift{3.998864in}{3.574193in}%
\pgfsys@useobject{currentmarker}{}%
\end{pgfscope}%
\end{pgfscope}%
\begin{pgfscope}%
\pgfsetbuttcap%
\pgfsetroundjoin%
\definecolor{currentfill}{rgb}{0.000000,0.000000,0.000000}%
\pgfsetfillcolor{currentfill}%
\pgfsetlinewidth{0.501875pt}%
\definecolor{currentstroke}{rgb}{0.000000,0.000000,0.000000}%
\pgfsetstrokecolor{currentstroke}%
\pgfsetdash{}{0pt}%
\pgfsys@defobject{currentmarker}{\pgfqpoint{0.000000in}{0.000000in}}{\pgfqpoint{0.000000in}{0.020833in}}{%
\pgfpathmoveto{\pgfqpoint{0.000000in}{0.000000in}}%
\pgfpathlineto{\pgfqpoint{0.000000in}{0.020833in}}%
\pgfusepath{stroke,fill}%
}%
\begin{pgfscope}%
\pgfsys@transformshift{4.107672in}{0.422992in}%
\pgfsys@useobject{currentmarker}{}%
\end{pgfscope}%
\end{pgfscope}%
\begin{pgfscope}%
\pgfsetbuttcap%
\pgfsetroundjoin%
\definecolor{currentfill}{rgb}{0.000000,0.000000,0.000000}%
\pgfsetfillcolor{currentfill}%
\pgfsetlinewidth{0.501875pt}%
\definecolor{currentstroke}{rgb}{0.000000,0.000000,0.000000}%
\pgfsetstrokecolor{currentstroke}%
\pgfsetdash{}{0pt}%
\pgfsys@defobject{currentmarker}{\pgfqpoint{0.000000in}{-0.020833in}}{\pgfqpoint{0.000000in}{0.000000in}}{%
\pgfpathmoveto{\pgfqpoint{0.000000in}{0.000000in}}%
\pgfpathlineto{\pgfqpoint{0.000000in}{-0.020833in}}%
\pgfusepath{stroke,fill}%
}%
\begin{pgfscope}%
\pgfsys@transformshift{4.107672in}{3.574193in}%
\pgfsys@useobject{currentmarker}{}%
\end{pgfscope}%
\end{pgfscope}%
\begin{pgfscope}%
\pgfsetbuttcap%
\pgfsetroundjoin%
\definecolor{currentfill}{rgb}{0.000000,0.000000,0.000000}%
\pgfsetfillcolor{currentfill}%
\pgfsetlinewidth{0.501875pt}%
\definecolor{currentstroke}{rgb}{0.000000,0.000000,0.000000}%
\pgfsetstrokecolor{currentstroke}%
\pgfsetdash{}{0pt}%
\pgfsys@defobject{currentmarker}{\pgfqpoint{0.000000in}{0.000000in}}{\pgfqpoint{0.000000in}{0.020833in}}{%
\pgfpathmoveto{\pgfqpoint{0.000000in}{0.000000in}}%
\pgfpathlineto{\pgfqpoint{0.000000in}{0.020833in}}%
\pgfusepath{stroke,fill}%
}%
\begin{pgfscope}%
\pgfsys@transformshift{4.216480in}{0.422992in}%
\pgfsys@useobject{currentmarker}{}%
\end{pgfscope}%
\end{pgfscope}%
\begin{pgfscope}%
\pgfsetbuttcap%
\pgfsetroundjoin%
\definecolor{currentfill}{rgb}{0.000000,0.000000,0.000000}%
\pgfsetfillcolor{currentfill}%
\pgfsetlinewidth{0.501875pt}%
\definecolor{currentstroke}{rgb}{0.000000,0.000000,0.000000}%
\pgfsetstrokecolor{currentstroke}%
\pgfsetdash{}{0pt}%
\pgfsys@defobject{currentmarker}{\pgfqpoint{0.000000in}{-0.020833in}}{\pgfqpoint{0.000000in}{0.000000in}}{%
\pgfpathmoveto{\pgfqpoint{0.000000in}{0.000000in}}%
\pgfpathlineto{\pgfqpoint{0.000000in}{-0.020833in}}%
\pgfusepath{stroke,fill}%
}%
\begin{pgfscope}%
\pgfsys@transformshift{4.216480in}{3.574193in}%
\pgfsys@useobject{currentmarker}{}%
\end{pgfscope}%
\end{pgfscope}%
\begin{pgfscope}%
\pgfsetbuttcap%
\pgfsetroundjoin%
\definecolor{currentfill}{rgb}{0.000000,0.000000,0.000000}%
\pgfsetfillcolor{currentfill}%
\pgfsetlinewidth{0.501875pt}%
\definecolor{currentstroke}{rgb}{0.000000,0.000000,0.000000}%
\pgfsetstrokecolor{currentstroke}%
\pgfsetdash{}{0pt}%
\pgfsys@defobject{currentmarker}{\pgfqpoint{0.000000in}{0.000000in}}{\pgfqpoint{0.000000in}{0.020833in}}{%
\pgfpathmoveto{\pgfqpoint{0.000000in}{0.000000in}}%
\pgfpathlineto{\pgfqpoint{0.000000in}{0.020833in}}%
\pgfusepath{stroke,fill}%
}%
\begin{pgfscope}%
\pgfsys@transformshift{4.434097in}{0.422992in}%
\pgfsys@useobject{currentmarker}{}%
\end{pgfscope}%
\end{pgfscope}%
\begin{pgfscope}%
\pgfsetbuttcap%
\pgfsetroundjoin%
\definecolor{currentfill}{rgb}{0.000000,0.000000,0.000000}%
\pgfsetfillcolor{currentfill}%
\pgfsetlinewidth{0.501875pt}%
\definecolor{currentstroke}{rgb}{0.000000,0.000000,0.000000}%
\pgfsetstrokecolor{currentstroke}%
\pgfsetdash{}{0pt}%
\pgfsys@defobject{currentmarker}{\pgfqpoint{0.000000in}{-0.020833in}}{\pgfqpoint{0.000000in}{0.000000in}}{%
\pgfpathmoveto{\pgfqpoint{0.000000in}{0.000000in}}%
\pgfpathlineto{\pgfqpoint{0.000000in}{-0.020833in}}%
\pgfusepath{stroke,fill}%
}%
\begin{pgfscope}%
\pgfsys@transformshift{4.434097in}{3.574193in}%
\pgfsys@useobject{currentmarker}{}%
\end{pgfscope}%
\end{pgfscope}%
\begin{pgfscope}%
\pgfsetbuttcap%
\pgfsetroundjoin%
\definecolor{currentfill}{rgb}{0.000000,0.000000,0.000000}%
\pgfsetfillcolor{currentfill}%
\pgfsetlinewidth{0.501875pt}%
\definecolor{currentstroke}{rgb}{0.000000,0.000000,0.000000}%
\pgfsetstrokecolor{currentstroke}%
\pgfsetdash{}{0pt}%
\pgfsys@defobject{currentmarker}{\pgfqpoint{0.000000in}{0.000000in}}{\pgfqpoint{0.000000in}{0.020833in}}{%
\pgfpathmoveto{\pgfqpoint{0.000000in}{0.000000in}}%
\pgfpathlineto{\pgfqpoint{0.000000in}{0.020833in}}%
\pgfusepath{stroke,fill}%
}%
\begin{pgfscope}%
\pgfsys@transformshift{4.542905in}{0.422992in}%
\pgfsys@useobject{currentmarker}{}%
\end{pgfscope}%
\end{pgfscope}%
\begin{pgfscope}%
\pgfsetbuttcap%
\pgfsetroundjoin%
\definecolor{currentfill}{rgb}{0.000000,0.000000,0.000000}%
\pgfsetfillcolor{currentfill}%
\pgfsetlinewidth{0.501875pt}%
\definecolor{currentstroke}{rgb}{0.000000,0.000000,0.000000}%
\pgfsetstrokecolor{currentstroke}%
\pgfsetdash{}{0pt}%
\pgfsys@defobject{currentmarker}{\pgfqpoint{0.000000in}{-0.020833in}}{\pgfqpoint{0.000000in}{0.000000in}}{%
\pgfpathmoveto{\pgfqpoint{0.000000in}{0.000000in}}%
\pgfpathlineto{\pgfqpoint{0.000000in}{-0.020833in}}%
\pgfusepath{stroke,fill}%
}%
\begin{pgfscope}%
\pgfsys@transformshift{4.542905in}{3.574193in}%
\pgfsys@useobject{currentmarker}{}%
\end{pgfscope}%
\end{pgfscope}%
\begin{pgfscope}%
\pgfsetbuttcap%
\pgfsetroundjoin%
\definecolor{currentfill}{rgb}{0.000000,0.000000,0.000000}%
\pgfsetfillcolor{currentfill}%
\pgfsetlinewidth{0.501875pt}%
\definecolor{currentstroke}{rgb}{0.000000,0.000000,0.000000}%
\pgfsetstrokecolor{currentstroke}%
\pgfsetdash{}{0pt}%
\pgfsys@defobject{currentmarker}{\pgfqpoint{0.000000in}{0.000000in}}{\pgfqpoint{0.000000in}{0.020833in}}{%
\pgfpathmoveto{\pgfqpoint{0.000000in}{0.000000in}}%
\pgfpathlineto{\pgfqpoint{0.000000in}{0.020833in}}%
\pgfusepath{stroke,fill}%
}%
\begin{pgfscope}%
\pgfsys@transformshift{4.651714in}{0.422992in}%
\pgfsys@useobject{currentmarker}{}%
\end{pgfscope}%
\end{pgfscope}%
\begin{pgfscope}%
\pgfsetbuttcap%
\pgfsetroundjoin%
\definecolor{currentfill}{rgb}{0.000000,0.000000,0.000000}%
\pgfsetfillcolor{currentfill}%
\pgfsetlinewidth{0.501875pt}%
\definecolor{currentstroke}{rgb}{0.000000,0.000000,0.000000}%
\pgfsetstrokecolor{currentstroke}%
\pgfsetdash{}{0pt}%
\pgfsys@defobject{currentmarker}{\pgfqpoint{0.000000in}{-0.020833in}}{\pgfqpoint{0.000000in}{0.000000in}}{%
\pgfpathmoveto{\pgfqpoint{0.000000in}{0.000000in}}%
\pgfpathlineto{\pgfqpoint{0.000000in}{-0.020833in}}%
\pgfusepath{stroke,fill}%
}%
\begin{pgfscope}%
\pgfsys@transformshift{4.651714in}{3.574193in}%
\pgfsys@useobject{currentmarker}{}%
\end{pgfscope}%
\end{pgfscope}%
\begin{pgfscope}%
\pgfsetbuttcap%
\pgfsetroundjoin%
\definecolor{currentfill}{rgb}{0.000000,0.000000,0.000000}%
\pgfsetfillcolor{currentfill}%
\pgfsetlinewidth{0.501875pt}%
\definecolor{currentstroke}{rgb}{0.000000,0.000000,0.000000}%
\pgfsetstrokecolor{currentstroke}%
\pgfsetdash{}{0pt}%
\pgfsys@defobject{currentmarker}{\pgfqpoint{0.000000in}{0.000000in}}{\pgfqpoint{0.000000in}{0.020833in}}{%
\pgfpathmoveto{\pgfqpoint{0.000000in}{0.000000in}}%
\pgfpathlineto{\pgfqpoint{0.000000in}{0.020833in}}%
\pgfusepath{stroke,fill}%
}%
\begin{pgfscope}%
\pgfsys@transformshift{4.869330in}{0.422992in}%
\pgfsys@useobject{currentmarker}{}%
\end{pgfscope}%
\end{pgfscope}%
\begin{pgfscope}%
\pgfsetbuttcap%
\pgfsetroundjoin%
\definecolor{currentfill}{rgb}{0.000000,0.000000,0.000000}%
\pgfsetfillcolor{currentfill}%
\pgfsetlinewidth{0.501875pt}%
\definecolor{currentstroke}{rgb}{0.000000,0.000000,0.000000}%
\pgfsetstrokecolor{currentstroke}%
\pgfsetdash{}{0pt}%
\pgfsys@defobject{currentmarker}{\pgfqpoint{0.000000in}{-0.020833in}}{\pgfqpoint{0.000000in}{0.000000in}}{%
\pgfpathmoveto{\pgfqpoint{0.000000in}{0.000000in}}%
\pgfpathlineto{\pgfqpoint{0.000000in}{-0.020833in}}%
\pgfusepath{stroke,fill}%
}%
\begin{pgfscope}%
\pgfsys@transformshift{4.869330in}{3.574193in}%
\pgfsys@useobject{currentmarker}{}%
\end{pgfscope}%
\end{pgfscope}%
\begin{pgfscope}%
\pgfsetbuttcap%
\pgfsetroundjoin%
\definecolor{currentfill}{rgb}{0.000000,0.000000,0.000000}%
\pgfsetfillcolor{currentfill}%
\pgfsetlinewidth{0.501875pt}%
\definecolor{currentstroke}{rgb}{0.000000,0.000000,0.000000}%
\pgfsetstrokecolor{currentstroke}%
\pgfsetdash{}{0pt}%
\pgfsys@defobject{currentmarker}{\pgfqpoint{0.000000in}{0.000000in}}{\pgfqpoint{0.000000in}{0.020833in}}{%
\pgfpathmoveto{\pgfqpoint{0.000000in}{0.000000in}}%
\pgfpathlineto{\pgfqpoint{0.000000in}{0.020833in}}%
\pgfusepath{stroke,fill}%
}%
\begin{pgfscope}%
\pgfsys@transformshift{4.978139in}{0.422992in}%
\pgfsys@useobject{currentmarker}{}%
\end{pgfscope}%
\end{pgfscope}%
\begin{pgfscope}%
\pgfsetbuttcap%
\pgfsetroundjoin%
\definecolor{currentfill}{rgb}{0.000000,0.000000,0.000000}%
\pgfsetfillcolor{currentfill}%
\pgfsetlinewidth{0.501875pt}%
\definecolor{currentstroke}{rgb}{0.000000,0.000000,0.000000}%
\pgfsetstrokecolor{currentstroke}%
\pgfsetdash{}{0pt}%
\pgfsys@defobject{currentmarker}{\pgfqpoint{0.000000in}{-0.020833in}}{\pgfqpoint{0.000000in}{0.000000in}}{%
\pgfpathmoveto{\pgfqpoint{0.000000in}{0.000000in}}%
\pgfpathlineto{\pgfqpoint{0.000000in}{-0.020833in}}%
\pgfusepath{stroke,fill}%
}%
\begin{pgfscope}%
\pgfsys@transformshift{4.978139in}{3.574193in}%
\pgfsys@useobject{currentmarker}{}%
\end{pgfscope}%
\end{pgfscope}%
\begin{pgfscope}%
\pgfsetbuttcap%
\pgfsetroundjoin%
\definecolor{currentfill}{rgb}{0.000000,0.000000,0.000000}%
\pgfsetfillcolor{currentfill}%
\pgfsetlinewidth{0.501875pt}%
\definecolor{currentstroke}{rgb}{0.000000,0.000000,0.000000}%
\pgfsetstrokecolor{currentstroke}%
\pgfsetdash{}{0pt}%
\pgfsys@defobject{currentmarker}{\pgfqpoint{0.000000in}{0.000000in}}{\pgfqpoint{0.000000in}{0.020833in}}{%
\pgfpathmoveto{\pgfqpoint{0.000000in}{0.000000in}}%
\pgfpathlineto{\pgfqpoint{0.000000in}{0.020833in}}%
\pgfusepath{stroke,fill}%
}%
\begin{pgfscope}%
\pgfsys@transformshift{5.086947in}{0.422992in}%
\pgfsys@useobject{currentmarker}{}%
\end{pgfscope}%
\end{pgfscope}%
\begin{pgfscope}%
\pgfsetbuttcap%
\pgfsetroundjoin%
\definecolor{currentfill}{rgb}{0.000000,0.000000,0.000000}%
\pgfsetfillcolor{currentfill}%
\pgfsetlinewidth{0.501875pt}%
\definecolor{currentstroke}{rgb}{0.000000,0.000000,0.000000}%
\pgfsetstrokecolor{currentstroke}%
\pgfsetdash{}{0pt}%
\pgfsys@defobject{currentmarker}{\pgfqpoint{0.000000in}{-0.020833in}}{\pgfqpoint{0.000000in}{0.000000in}}{%
\pgfpathmoveto{\pgfqpoint{0.000000in}{0.000000in}}%
\pgfpathlineto{\pgfqpoint{0.000000in}{-0.020833in}}%
\pgfusepath{stroke,fill}%
}%
\begin{pgfscope}%
\pgfsys@transformshift{5.086947in}{3.574193in}%
\pgfsys@useobject{currentmarker}{}%
\end{pgfscope}%
\end{pgfscope}%
\begin{pgfscope}%
\pgfsetbuttcap%
\pgfsetroundjoin%
\definecolor{currentfill}{rgb}{0.000000,0.000000,0.000000}%
\pgfsetfillcolor{currentfill}%
\pgfsetlinewidth{0.501875pt}%
\definecolor{currentstroke}{rgb}{0.000000,0.000000,0.000000}%
\pgfsetstrokecolor{currentstroke}%
\pgfsetdash{}{0pt}%
\pgfsys@defobject{currentmarker}{\pgfqpoint{0.000000in}{0.000000in}}{\pgfqpoint{0.000000in}{0.020833in}}{%
\pgfpathmoveto{\pgfqpoint{0.000000in}{0.000000in}}%
\pgfpathlineto{\pgfqpoint{0.000000in}{0.020833in}}%
\pgfusepath{stroke,fill}%
}%
\begin{pgfscope}%
\pgfsys@transformshift{5.304564in}{0.422992in}%
\pgfsys@useobject{currentmarker}{}%
\end{pgfscope}%
\end{pgfscope}%
\begin{pgfscope}%
\pgfsetbuttcap%
\pgfsetroundjoin%
\definecolor{currentfill}{rgb}{0.000000,0.000000,0.000000}%
\pgfsetfillcolor{currentfill}%
\pgfsetlinewidth{0.501875pt}%
\definecolor{currentstroke}{rgb}{0.000000,0.000000,0.000000}%
\pgfsetstrokecolor{currentstroke}%
\pgfsetdash{}{0pt}%
\pgfsys@defobject{currentmarker}{\pgfqpoint{0.000000in}{-0.020833in}}{\pgfqpoint{0.000000in}{0.000000in}}{%
\pgfpathmoveto{\pgfqpoint{0.000000in}{0.000000in}}%
\pgfpathlineto{\pgfqpoint{0.000000in}{-0.020833in}}%
\pgfusepath{stroke,fill}%
}%
\begin{pgfscope}%
\pgfsys@transformshift{5.304564in}{3.574193in}%
\pgfsys@useobject{currentmarker}{}%
\end{pgfscope}%
\end{pgfscope}%
\begin{pgfscope}%
\pgfsetbuttcap%
\pgfsetroundjoin%
\definecolor{currentfill}{rgb}{0.000000,0.000000,0.000000}%
\pgfsetfillcolor{currentfill}%
\pgfsetlinewidth{0.501875pt}%
\definecolor{currentstroke}{rgb}{0.000000,0.000000,0.000000}%
\pgfsetstrokecolor{currentstroke}%
\pgfsetdash{}{0pt}%
\pgfsys@defobject{currentmarker}{\pgfqpoint{0.000000in}{0.000000in}}{\pgfqpoint{0.000000in}{0.020833in}}{%
\pgfpathmoveto{\pgfqpoint{0.000000in}{0.000000in}}%
\pgfpathlineto{\pgfqpoint{0.000000in}{0.020833in}}%
\pgfusepath{stroke,fill}%
}%
\begin{pgfscope}%
\pgfsys@transformshift{5.413372in}{0.422992in}%
\pgfsys@useobject{currentmarker}{}%
\end{pgfscope}%
\end{pgfscope}%
\begin{pgfscope}%
\pgfsetbuttcap%
\pgfsetroundjoin%
\definecolor{currentfill}{rgb}{0.000000,0.000000,0.000000}%
\pgfsetfillcolor{currentfill}%
\pgfsetlinewidth{0.501875pt}%
\definecolor{currentstroke}{rgb}{0.000000,0.000000,0.000000}%
\pgfsetstrokecolor{currentstroke}%
\pgfsetdash{}{0pt}%
\pgfsys@defobject{currentmarker}{\pgfqpoint{0.000000in}{-0.020833in}}{\pgfqpoint{0.000000in}{0.000000in}}{%
\pgfpathmoveto{\pgfqpoint{0.000000in}{0.000000in}}%
\pgfpathlineto{\pgfqpoint{0.000000in}{-0.020833in}}%
\pgfusepath{stroke,fill}%
}%
\begin{pgfscope}%
\pgfsys@transformshift{5.413372in}{3.574193in}%
\pgfsys@useobject{currentmarker}{}%
\end{pgfscope}%
\end{pgfscope}%
\begin{pgfscope}%
\pgfsetbuttcap%
\pgfsetroundjoin%
\definecolor{currentfill}{rgb}{0.000000,0.000000,0.000000}%
\pgfsetfillcolor{currentfill}%
\pgfsetlinewidth{0.501875pt}%
\definecolor{currentstroke}{rgb}{0.000000,0.000000,0.000000}%
\pgfsetstrokecolor{currentstroke}%
\pgfsetdash{}{0pt}%
\pgfsys@defobject{currentmarker}{\pgfqpoint{0.000000in}{0.000000in}}{\pgfqpoint{0.000000in}{0.020833in}}{%
\pgfpathmoveto{\pgfqpoint{0.000000in}{0.000000in}}%
\pgfpathlineto{\pgfqpoint{0.000000in}{0.020833in}}%
\pgfusepath{stroke,fill}%
}%
\begin{pgfscope}%
\pgfsys@transformshift{5.522180in}{0.422992in}%
\pgfsys@useobject{currentmarker}{}%
\end{pgfscope}%
\end{pgfscope}%
\begin{pgfscope}%
\pgfsetbuttcap%
\pgfsetroundjoin%
\definecolor{currentfill}{rgb}{0.000000,0.000000,0.000000}%
\pgfsetfillcolor{currentfill}%
\pgfsetlinewidth{0.501875pt}%
\definecolor{currentstroke}{rgb}{0.000000,0.000000,0.000000}%
\pgfsetstrokecolor{currentstroke}%
\pgfsetdash{}{0pt}%
\pgfsys@defobject{currentmarker}{\pgfqpoint{0.000000in}{-0.020833in}}{\pgfqpoint{0.000000in}{0.000000in}}{%
\pgfpathmoveto{\pgfqpoint{0.000000in}{0.000000in}}%
\pgfpathlineto{\pgfqpoint{0.000000in}{-0.020833in}}%
\pgfusepath{stroke,fill}%
}%
\begin{pgfscope}%
\pgfsys@transformshift{5.522180in}{3.574193in}%
\pgfsys@useobject{currentmarker}{}%
\end{pgfscope}%
\end{pgfscope}%
\begin{pgfscope}%
\definecolor{textcolor}{rgb}{0.000000,0.000000,0.000000}%
\pgfsetstrokecolor{textcolor}%
\pgfsetfillcolor{textcolor}%
\pgftext[x=4.564667in,y=0.184413in,,top]{\color{textcolor}\rmfamily\fontsize{10.000000}{12.000000}\selectfont \(\displaystyle K\)}%
\end{pgfscope}%
\begin{pgfscope}%
\pgfsetbuttcap%
\pgfsetroundjoin%
\definecolor{currentfill}{rgb}{0.000000,0.000000,0.000000}%
\pgfsetfillcolor{currentfill}%
\pgfsetlinewidth{0.501875pt}%
\definecolor{currentstroke}{rgb}{0.000000,0.000000,0.000000}%
\pgfsetstrokecolor{currentstroke}%
\pgfsetdash{}{0pt}%
\pgfsys@defobject{currentmarker}{\pgfqpoint{0.000000in}{0.000000in}}{\pgfqpoint{0.041667in}{0.000000in}}{%
\pgfpathmoveto{\pgfqpoint{0.000000in}{0.000000in}}%
\pgfpathlineto{\pgfqpoint{0.041667in}{0.000000in}}%
\pgfusepath{stroke,fill}%
}%
\begin{pgfscope}%
\pgfsys@transformshift{3.454822in}{0.715529in}%
\pgfsys@useobject{currentmarker}{}%
\end{pgfscope}%
\end{pgfscope}%
\begin{pgfscope}%
\pgfsetbuttcap%
\pgfsetroundjoin%
\definecolor{currentfill}{rgb}{0.000000,0.000000,0.000000}%
\pgfsetfillcolor{currentfill}%
\pgfsetlinewidth{0.501875pt}%
\definecolor{currentstroke}{rgb}{0.000000,0.000000,0.000000}%
\pgfsetstrokecolor{currentstroke}%
\pgfsetdash{}{0pt}%
\pgfsys@defobject{currentmarker}{\pgfqpoint{-0.041667in}{0.000000in}}{\pgfqpoint{-0.000000in}{0.000000in}}{%
\pgfpathmoveto{\pgfqpoint{-0.000000in}{0.000000in}}%
\pgfpathlineto{\pgfqpoint{-0.041667in}{0.000000in}}%
\pgfusepath{stroke,fill}%
}%
\begin{pgfscope}%
\pgfsys@transformshift{5.674512in}{0.715529in}%
\pgfsys@useobject{currentmarker}{}%
\end{pgfscope}%
\end{pgfscope}%
\begin{pgfscope}%
\definecolor{textcolor}{rgb}{0.000000,0.000000,0.000000}%
\pgfsetstrokecolor{textcolor}%
\pgfsetfillcolor{textcolor}%
\pgftext[x=3.228741in, y=0.662767in, left, base]{\color{textcolor}\rmfamily\fontsize{10.000000}{12.000000}\selectfont \(\displaystyle {0.1}\)}%
\end{pgfscope}%
\begin{pgfscope}%
\pgfsetbuttcap%
\pgfsetroundjoin%
\definecolor{currentfill}{rgb}{0.000000,0.000000,0.000000}%
\pgfsetfillcolor{currentfill}%
\pgfsetlinewidth{0.501875pt}%
\definecolor{currentstroke}{rgb}{0.000000,0.000000,0.000000}%
\pgfsetstrokecolor{currentstroke}%
\pgfsetdash{}{0pt}%
\pgfsys@defobject{currentmarker}{\pgfqpoint{0.000000in}{0.000000in}}{\pgfqpoint{0.041667in}{0.000000in}}{%
\pgfpathmoveto{\pgfqpoint{0.000000in}{0.000000in}}%
\pgfpathlineto{\pgfqpoint{0.041667in}{0.000000in}}%
\pgfusepath{stroke,fill}%
}%
\begin{pgfscope}%
\pgfsys@transformshift{3.454822in}{1.253255in}%
\pgfsys@useobject{currentmarker}{}%
\end{pgfscope}%
\end{pgfscope}%
\begin{pgfscope}%
\pgfsetbuttcap%
\pgfsetroundjoin%
\definecolor{currentfill}{rgb}{0.000000,0.000000,0.000000}%
\pgfsetfillcolor{currentfill}%
\pgfsetlinewidth{0.501875pt}%
\definecolor{currentstroke}{rgb}{0.000000,0.000000,0.000000}%
\pgfsetstrokecolor{currentstroke}%
\pgfsetdash{}{0pt}%
\pgfsys@defobject{currentmarker}{\pgfqpoint{-0.041667in}{0.000000in}}{\pgfqpoint{-0.000000in}{0.000000in}}{%
\pgfpathmoveto{\pgfqpoint{-0.000000in}{0.000000in}}%
\pgfpathlineto{\pgfqpoint{-0.041667in}{0.000000in}}%
\pgfusepath{stroke,fill}%
}%
\begin{pgfscope}%
\pgfsys@transformshift{5.674512in}{1.253255in}%
\pgfsys@useobject{currentmarker}{}%
\end{pgfscope}%
\end{pgfscope}%
\begin{pgfscope}%
\definecolor{textcolor}{rgb}{0.000000,0.000000,0.000000}%
\pgfsetstrokecolor{textcolor}%
\pgfsetfillcolor{textcolor}%
\pgftext[x=3.228741in, y=1.200494in, left, base]{\color{textcolor}\rmfamily\fontsize{10.000000}{12.000000}\selectfont \(\displaystyle {0.2}\)}%
\end{pgfscope}%
\begin{pgfscope}%
\pgfsetbuttcap%
\pgfsetroundjoin%
\definecolor{currentfill}{rgb}{0.000000,0.000000,0.000000}%
\pgfsetfillcolor{currentfill}%
\pgfsetlinewidth{0.501875pt}%
\definecolor{currentstroke}{rgb}{0.000000,0.000000,0.000000}%
\pgfsetstrokecolor{currentstroke}%
\pgfsetdash{}{0pt}%
\pgfsys@defobject{currentmarker}{\pgfqpoint{0.000000in}{0.000000in}}{\pgfqpoint{0.041667in}{0.000000in}}{%
\pgfpathmoveto{\pgfqpoint{0.000000in}{0.000000in}}%
\pgfpathlineto{\pgfqpoint{0.041667in}{0.000000in}}%
\pgfusepath{stroke,fill}%
}%
\begin{pgfscope}%
\pgfsys@transformshift{3.454822in}{1.790982in}%
\pgfsys@useobject{currentmarker}{}%
\end{pgfscope}%
\end{pgfscope}%
\begin{pgfscope}%
\pgfsetbuttcap%
\pgfsetroundjoin%
\definecolor{currentfill}{rgb}{0.000000,0.000000,0.000000}%
\pgfsetfillcolor{currentfill}%
\pgfsetlinewidth{0.501875pt}%
\definecolor{currentstroke}{rgb}{0.000000,0.000000,0.000000}%
\pgfsetstrokecolor{currentstroke}%
\pgfsetdash{}{0pt}%
\pgfsys@defobject{currentmarker}{\pgfqpoint{-0.041667in}{0.000000in}}{\pgfqpoint{-0.000000in}{0.000000in}}{%
\pgfpathmoveto{\pgfqpoint{-0.000000in}{0.000000in}}%
\pgfpathlineto{\pgfqpoint{-0.041667in}{0.000000in}}%
\pgfusepath{stroke,fill}%
}%
\begin{pgfscope}%
\pgfsys@transformshift{5.674512in}{1.790982in}%
\pgfsys@useobject{currentmarker}{}%
\end{pgfscope}%
\end{pgfscope}%
\begin{pgfscope}%
\definecolor{textcolor}{rgb}{0.000000,0.000000,0.000000}%
\pgfsetstrokecolor{textcolor}%
\pgfsetfillcolor{textcolor}%
\pgftext[x=3.228741in, y=1.738220in, left, base]{\color{textcolor}\rmfamily\fontsize{10.000000}{12.000000}\selectfont \(\displaystyle {0.3}\)}%
\end{pgfscope}%
\begin{pgfscope}%
\pgfsetbuttcap%
\pgfsetroundjoin%
\definecolor{currentfill}{rgb}{0.000000,0.000000,0.000000}%
\pgfsetfillcolor{currentfill}%
\pgfsetlinewidth{0.501875pt}%
\definecolor{currentstroke}{rgb}{0.000000,0.000000,0.000000}%
\pgfsetstrokecolor{currentstroke}%
\pgfsetdash{}{0pt}%
\pgfsys@defobject{currentmarker}{\pgfqpoint{0.000000in}{0.000000in}}{\pgfqpoint{0.041667in}{0.000000in}}{%
\pgfpathmoveto{\pgfqpoint{0.000000in}{0.000000in}}%
\pgfpathlineto{\pgfqpoint{0.041667in}{0.000000in}}%
\pgfusepath{stroke,fill}%
}%
\begin{pgfscope}%
\pgfsys@transformshift{3.454822in}{2.328708in}%
\pgfsys@useobject{currentmarker}{}%
\end{pgfscope}%
\end{pgfscope}%
\begin{pgfscope}%
\pgfsetbuttcap%
\pgfsetroundjoin%
\definecolor{currentfill}{rgb}{0.000000,0.000000,0.000000}%
\pgfsetfillcolor{currentfill}%
\pgfsetlinewidth{0.501875pt}%
\definecolor{currentstroke}{rgb}{0.000000,0.000000,0.000000}%
\pgfsetstrokecolor{currentstroke}%
\pgfsetdash{}{0pt}%
\pgfsys@defobject{currentmarker}{\pgfqpoint{-0.041667in}{0.000000in}}{\pgfqpoint{-0.000000in}{0.000000in}}{%
\pgfpathmoveto{\pgfqpoint{-0.000000in}{0.000000in}}%
\pgfpathlineto{\pgfqpoint{-0.041667in}{0.000000in}}%
\pgfusepath{stroke,fill}%
}%
\begin{pgfscope}%
\pgfsys@transformshift{5.674512in}{2.328708in}%
\pgfsys@useobject{currentmarker}{}%
\end{pgfscope}%
\end{pgfscope}%
\begin{pgfscope}%
\definecolor{textcolor}{rgb}{0.000000,0.000000,0.000000}%
\pgfsetstrokecolor{textcolor}%
\pgfsetfillcolor{textcolor}%
\pgftext[x=3.228741in, y=2.275946in, left, base]{\color{textcolor}\rmfamily\fontsize{10.000000}{12.000000}\selectfont \(\displaystyle {0.4}\)}%
\end{pgfscope}%
\begin{pgfscope}%
\pgfsetbuttcap%
\pgfsetroundjoin%
\definecolor{currentfill}{rgb}{0.000000,0.000000,0.000000}%
\pgfsetfillcolor{currentfill}%
\pgfsetlinewidth{0.501875pt}%
\definecolor{currentstroke}{rgb}{0.000000,0.000000,0.000000}%
\pgfsetstrokecolor{currentstroke}%
\pgfsetdash{}{0pt}%
\pgfsys@defobject{currentmarker}{\pgfqpoint{0.000000in}{0.000000in}}{\pgfqpoint{0.041667in}{0.000000in}}{%
\pgfpathmoveto{\pgfqpoint{0.000000in}{0.000000in}}%
\pgfpathlineto{\pgfqpoint{0.041667in}{0.000000in}}%
\pgfusepath{stroke,fill}%
}%
\begin{pgfscope}%
\pgfsys@transformshift{3.454822in}{2.866434in}%
\pgfsys@useobject{currentmarker}{}%
\end{pgfscope}%
\end{pgfscope}%
\begin{pgfscope}%
\pgfsetbuttcap%
\pgfsetroundjoin%
\definecolor{currentfill}{rgb}{0.000000,0.000000,0.000000}%
\pgfsetfillcolor{currentfill}%
\pgfsetlinewidth{0.501875pt}%
\definecolor{currentstroke}{rgb}{0.000000,0.000000,0.000000}%
\pgfsetstrokecolor{currentstroke}%
\pgfsetdash{}{0pt}%
\pgfsys@defobject{currentmarker}{\pgfqpoint{-0.041667in}{0.000000in}}{\pgfqpoint{-0.000000in}{0.000000in}}{%
\pgfpathmoveto{\pgfqpoint{-0.000000in}{0.000000in}}%
\pgfpathlineto{\pgfqpoint{-0.041667in}{0.000000in}}%
\pgfusepath{stroke,fill}%
}%
\begin{pgfscope}%
\pgfsys@transformshift{5.674512in}{2.866434in}%
\pgfsys@useobject{currentmarker}{}%
\end{pgfscope}%
\end{pgfscope}%
\begin{pgfscope}%
\definecolor{textcolor}{rgb}{0.000000,0.000000,0.000000}%
\pgfsetstrokecolor{textcolor}%
\pgfsetfillcolor{textcolor}%
\pgftext[x=3.228741in, y=2.813673in, left, base]{\color{textcolor}\rmfamily\fontsize{10.000000}{12.000000}\selectfont \(\displaystyle {0.5}\)}%
\end{pgfscope}%
\begin{pgfscope}%
\pgfsetbuttcap%
\pgfsetroundjoin%
\definecolor{currentfill}{rgb}{0.000000,0.000000,0.000000}%
\pgfsetfillcolor{currentfill}%
\pgfsetlinewidth{0.501875pt}%
\definecolor{currentstroke}{rgb}{0.000000,0.000000,0.000000}%
\pgfsetstrokecolor{currentstroke}%
\pgfsetdash{}{0pt}%
\pgfsys@defobject{currentmarker}{\pgfqpoint{0.000000in}{0.000000in}}{\pgfqpoint{0.041667in}{0.000000in}}{%
\pgfpathmoveto{\pgfqpoint{0.000000in}{0.000000in}}%
\pgfpathlineto{\pgfqpoint{0.041667in}{0.000000in}}%
\pgfusepath{stroke,fill}%
}%
\begin{pgfscope}%
\pgfsys@transformshift{3.454822in}{3.404161in}%
\pgfsys@useobject{currentmarker}{}%
\end{pgfscope}%
\end{pgfscope}%
\begin{pgfscope}%
\pgfsetbuttcap%
\pgfsetroundjoin%
\definecolor{currentfill}{rgb}{0.000000,0.000000,0.000000}%
\pgfsetfillcolor{currentfill}%
\pgfsetlinewidth{0.501875pt}%
\definecolor{currentstroke}{rgb}{0.000000,0.000000,0.000000}%
\pgfsetstrokecolor{currentstroke}%
\pgfsetdash{}{0pt}%
\pgfsys@defobject{currentmarker}{\pgfqpoint{-0.041667in}{0.000000in}}{\pgfqpoint{-0.000000in}{0.000000in}}{%
\pgfpathmoveto{\pgfqpoint{-0.000000in}{0.000000in}}%
\pgfpathlineto{\pgfqpoint{-0.041667in}{0.000000in}}%
\pgfusepath{stroke,fill}%
}%
\begin{pgfscope}%
\pgfsys@transformshift{5.674512in}{3.404161in}%
\pgfsys@useobject{currentmarker}{}%
\end{pgfscope}%
\end{pgfscope}%
\begin{pgfscope}%
\definecolor{textcolor}{rgb}{0.000000,0.000000,0.000000}%
\pgfsetstrokecolor{textcolor}%
\pgfsetfillcolor{textcolor}%
\pgftext[x=3.228741in, y=3.351399in, left, base]{\color{textcolor}\rmfamily\fontsize{10.000000}{12.000000}\selectfont \(\displaystyle {0.6}\)}%
\end{pgfscope}%
\begin{pgfscope}%
\pgfsetbuttcap%
\pgfsetroundjoin%
\definecolor{currentfill}{rgb}{0.000000,0.000000,0.000000}%
\pgfsetfillcolor{currentfill}%
\pgfsetlinewidth{0.501875pt}%
\definecolor{currentstroke}{rgb}{0.000000,0.000000,0.000000}%
\pgfsetstrokecolor{currentstroke}%
\pgfsetdash{}{0pt}%
\pgfsys@defobject{currentmarker}{\pgfqpoint{0.000000in}{0.000000in}}{\pgfqpoint{0.020833in}{0.000000in}}{%
\pgfpathmoveto{\pgfqpoint{0.000000in}{0.000000in}}%
\pgfpathlineto{\pgfqpoint{0.020833in}{0.000000in}}%
\pgfusepath{stroke,fill}%
}%
\begin{pgfscope}%
\pgfsys@transformshift{3.454822in}{0.500438in}%
\pgfsys@useobject{currentmarker}{}%
\end{pgfscope}%
\end{pgfscope}%
\begin{pgfscope}%
\pgfsetbuttcap%
\pgfsetroundjoin%
\definecolor{currentfill}{rgb}{0.000000,0.000000,0.000000}%
\pgfsetfillcolor{currentfill}%
\pgfsetlinewidth{0.501875pt}%
\definecolor{currentstroke}{rgb}{0.000000,0.000000,0.000000}%
\pgfsetstrokecolor{currentstroke}%
\pgfsetdash{}{0pt}%
\pgfsys@defobject{currentmarker}{\pgfqpoint{-0.020833in}{0.000000in}}{\pgfqpoint{-0.000000in}{0.000000in}}{%
\pgfpathmoveto{\pgfqpoint{-0.000000in}{0.000000in}}%
\pgfpathlineto{\pgfqpoint{-0.020833in}{0.000000in}}%
\pgfusepath{stroke,fill}%
}%
\begin{pgfscope}%
\pgfsys@transformshift{5.674512in}{0.500438in}%
\pgfsys@useobject{currentmarker}{}%
\end{pgfscope}%
\end{pgfscope}%
\begin{pgfscope}%
\pgfsetbuttcap%
\pgfsetroundjoin%
\definecolor{currentfill}{rgb}{0.000000,0.000000,0.000000}%
\pgfsetfillcolor{currentfill}%
\pgfsetlinewidth{0.501875pt}%
\definecolor{currentstroke}{rgb}{0.000000,0.000000,0.000000}%
\pgfsetstrokecolor{currentstroke}%
\pgfsetdash{}{0pt}%
\pgfsys@defobject{currentmarker}{\pgfqpoint{0.000000in}{0.000000in}}{\pgfqpoint{0.020833in}{0.000000in}}{%
\pgfpathmoveto{\pgfqpoint{0.000000in}{0.000000in}}%
\pgfpathlineto{\pgfqpoint{0.020833in}{0.000000in}}%
\pgfusepath{stroke,fill}%
}%
\begin{pgfscope}%
\pgfsys@transformshift{3.454822in}{0.607984in}%
\pgfsys@useobject{currentmarker}{}%
\end{pgfscope}%
\end{pgfscope}%
\begin{pgfscope}%
\pgfsetbuttcap%
\pgfsetroundjoin%
\definecolor{currentfill}{rgb}{0.000000,0.000000,0.000000}%
\pgfsetfillcolor{currentfill}%
\pgfsetlinewidth{0.501875pt}%
\definecolor{currentstroke}{rgb}{0.000000,0.000000,0.000000}%
\pgfsetstrokecolor{currentstroke}%
\pgfsetdash{}{0pt}%
\pgfsys@defobject{currentmarker}{\pgfqpoint{-0.020833in}{0.000000in}}{\pgfqpoint{-0.000000in}{0.000000in}}{%
\pgfpathmoveto{\pgfqpoint{-0.000000in}{0.000000in}}%
\pgfpathlineto{\pgfqpoint{-0.020833in}{0.000000in}}%
\pgfusepath{stroke,fill}%
}%
\begin{pgfscope}%
\pgfsys@transformshift{5.674512in}{0.607984in}%
\pgfsys@useobject{currentmarker}{}%
\end{pgfscope}%
\end{pgfscope}%
\begin{pgfscope}%
\pgfsetbuttcap%
\pgfsetroundjoin%
\definecolor{currentfill}{rgb}{0.000000,0.000000,0.000000}%
\pgfsetfillcolor{currentfill}%
\pgfsetlinewidth{0.501875pt}%
\definecolor{currentstroke}{rgb}{0.000000,0.000000,0.000000}%
\pgfsetstrokecolor{currentstroke}%
\pgfsetdash{}{0pt}%
\pgfsys@defobject{currentmarker}{\pgfqpoint{0.000000in}{0.000000in}}{\pgfqpoint{0.020833in}{0.000000in}}{%
\pgfpathmoveto{\pgfqpoint{0.000000in}{0.000000in}}%
\pgfpathlineto{\pgfqpoint{0.020833in}{0.000000in}}%
\pgfusepath{stroke,fill}%
}%
\begin{pgfscope}%
\pgfsys@transformshift{3.454822in}{0.823074in}%
\pgfsys@useobject{currentmarker}{}%
\end{pgfscope}%
\end{pgfscope}%
\begin{pgfscope}%
\pgfsetbuttcap%
\pgfsetroundjoin%
\definecolor{currentfill}{rgb}{0.000000,0.000000,0.000000}%
\pgfsetfillcolor{currentfill}%
\pgfsetlinewidth{0.501875pt}%
\definecolor{currentstroke}{rgb}{0.000000,0.000000,0.000000}%
\pgfsetstrokecolor{currentstroke}%
\pgfsetdash{}{0pt}%
\pgfsys@defobject{currentmarker}{\pgfqpoint{-0.020833in}{0.000000in}}{\pgfqpoint{-0.000000in}{0.000000in}}{%
\pgfpathmoveto{\pgfqpoint{-0.000000in}{0.000000in}}%
\pgfpathlineto{\pgfqpoint{-0.020833in}{0.000000in}}%
\pgfusepath{stroke,fill}%
}%
\begin{pgfscope}%
\pgfsys@transformshift{5.674512in}{0.823074in}%
\pgfsys@useobject{currentmarker}{}%
\end{pgfscope}%
\end{pgfscope}%
\begin{pgfscope}%
\pgfsetbuttcap%
\pgfsetroundjoin%
\definecolor{currentfill}{rgb}{0.000000,0.000000,0.000000}%
\pgfsetfillcolor{currentfill}%
\pgfsetlinewidth{0.501875pt}%
\definecolor{currentstroke}{rgb}{0.000000,0.000000,0.000000}%
\pgfsetstrokecolor{currentstroke}%
\pgfsetdash{}{0pt}%
\pgfsys@defobject{currentmarker}{\pgfqpoint{0.000000in}{0.000000in}}{\pgfqpoint{0.020833in}{0.000000in}}{%
\pgfpathmoveto{\pgfqpoint{0.000000in}{0.000000in}}%
\pgfpathlineto{\pgfqpoint{0.020833in}{0.000000in}}%
\pgfusepath{stroke,fill}%
}%
\begin{pgfscope}%
\pgfsys@transformshift{3.454822in}{0.930620in}%
\pgfsys@useobject{currentmarker}{}%
\end{pgfscope}%
\end{pgfscope}%
\begin{pgfscope}%
\pgfsetbuttcap%
\pgfsetroundjoin%
\definecolor{currentfill}{rgb}{0.000000,0.000000,0.000000}%
\pgfsetfillcolor{currentfill}%
\pgfsetlinewidth{0.501875pt}%
\definecolor{currentstroke}{rgb}{0.000000,0.000000,0.000000}%
\pgfsetstrokecolor{currentstroke}%
\pgfsetdash{}{0pt}%
\pgfsys@defobject{currentmarker}{\pgfqpoint{-0.020833in}{0.000000in}}{\pgfqpoint{-0.000000in}{0.000000in}}{%
\pgfpathmoveto{\pgfqpoint{-0.000000in}{0.000000in}}%
\pgfpathlineto{\pgfqpoint{-0.020833in}{0.000000in}}%
\pgfusepath{stroke,fill}%
}%
\begin{pgfscope}%
\pgfsys@transformshift{5.674512in}{0.930620in}%
\pgfsys@useobject{currentmarker}{}%
\end{pgfscope}%
\end{pgfscope}%
\begin{pgfscope}%
\pgfsetbuttcap%
\pgfsetroundjoin%
\definecolor{currentfill}{rgb}{0.000000,0.000000,0.000000}%
\pgfsetfillcolor{currentfill}%
\pgfsetlinewidth{0.501875pt}%
\definecolor{currentstroke}{rgb}{0.000000,0.000000,0.000000}%
\pgfsetstrokecolor{currentstroke}%
\pgfsetdash{}{0pt}%
\pgfsys@defobject{currentmarker}{\pgfqpoint{0.000000in}{0.000000in}}{\pgfqpoint{0.020833in}{0.000000in}}{%
\pgfpathmoveto{\pgfqpoint{0.000000in}{0.000000in}}%
\pgfpathlineto{\pgfqpoint{0.020833in}{0.000000in}}%
\pgfusepath{stroke,fill}%
}%
\begin{pgfscope}%
\pgfsys@transformshift{3.454822in}{1.038165in}%
\pgfsys@useobject{currentmarker}{}%
\end{pgfscope}%
\end{pgfscope}%
\begin{pgfscope}%
\pgfsetbuttcap%
\pgfsetroundjoin%
\definecolor{currentfill}{rgb}{0.000000,0.000000,0.000000}%
\pgfsetfillcolor{currentfill}%
\pgfsetlinewidth{0.501875pt}%
\definecolor{currentstroke}{rgb}{0.000000,0.000000,0.000000}%
\pgfsetstrokecolor{currentstroke}%
\pgfsetdash{}{0pt}%
\pgfsys@defobject{currentmarker}{\pgfqpoint{-0.020833in}{0.000000in}}{\pgfqpoint{-0.000000in}{0.000000in}}{%
\pgfpathmoveto{\pgfqpoint{-0.000000in}{0.000000in}}%
\pgfpathlineto{\pgfqpoint{-0.020833in}{0.000000in}}%
\pgfusepath{stroke,fill}%
}%
\begin{pgfscope}%
\pgfsys@transformshift{5.674512in}{1.038165in}%
\pgfsys@useobject{currentmarker}{}%
\end{pgfscope}%
\end{pgfscope}%
\begin{pgfscope}%
\pgfsetbuttcap%
\pgfsetroundjoin%
\definecolor{currentfill}{rgb}{0.000000,0.000000,0.000000}%
\pgfsetfillcolor{currentfill}%
\pgfsetlinewidth{0.501875pt}%
\definecolor{currentstroke}{rgb}{0.000000,0.000000,0.000000}%
\pgfsetstrokecolor{currentstroke}%
\pgfsetdash{}{0pt}%
\pgfsys@defobject{currentmarker}{\pgfqpoint{0.000000in}{0.000000in}}{\pgfqpoint{0.020833in}{0.000000in}}{%
\pgfpathmoveto{\pgfqpoint{0.000000in}{0.000000in}}%
\pgfpathlineto{\pgfqpoint{0.020833in}{0.000000in}}%
\pgfusepath{stroke,fill}%
}%
\begin{pgfscope}%
\pgfsys@transformshift{3.454822in}{1.145710in}%
\pgfsys@useobject{currentmarker}{}%
\end{pgfscope}%
\end{pgfscope}%
\begin{pgfscope}%
\pgfsetbuttcap%
\pgfsetroundjoin%
\definecolor{currentfill}{rgb}{0.000000,0.000000,0.000000}%
\pgfsetfillcolor{currentfill}%
\pgfsetlinewidth{0.501875pt}%
\definecolor{currentstroke}{rgb}{0.000000,0.000000,0.000000}%
\pgfsetstrokecolor{currentstroke}%
\pgfsetdash{}{0pt}%
\pgfsys@defobject{currentmarker}{\pgfqpoint{-0.020833in}{0.000000in}}{\pgfqpoint{-0.000000in}{0.000000in}}{%
\pgfpathmoveto{\pgfqpoint{-0.000000in}{0.000000in}}%
\pgfpathlineto{\pgfqpoint{-0.020833in}{0.000000in}}%
\pgfusepath{stroke,fill}%
}%
\begin{pgfscope}%
\pgfsys@transformshift{5.674512in}{1.145710in}%
\pgfsys@useobject{currentmarker}{}%
\end{pgfscope}%
\end{pgfscope}%
\begin{pgfscope}%
\pgfsetbuttcap%
\pgfsetroundjoin%
\definecolor{currentfill}{rgb}{0.000000,0.000000,0.000000}%
\pgfsetfillcolor{currentfill}%
\pgfsetlinewidth{0.501875pt}%
\definecolor{currentstroke}{rgb}{0.000000,0.000000,0.000000}%
\pgfsetstrokecolor{currentstroke}%
\pgfsetdash{}{0pt}%
\pgfsys@defobject{currentmarker}{\pgfqpoint{0.000000in}{0.000000in}}{\pgfqpoint{0.020833in}{0.000000in}}{%
\pgfpathmoveto{\pgfqpoint{0.000000in}{0.000000in}}%
\pgfpathlineto{\pgfqpoint{0.020833in}{0.000000in}}%
\pgfusepath{stroke,fill}%
}%
\begin{pgfscope}%
\pgfsys@transformshift{3.454822in}{1.360801in}%
\pgfsys@useobject{currentmarker}{}%
\end{pgfscope}%
\end{pgfscope}%
\begin{pgfscope}%
\pgfsetbuttcap%
\pgfsetroundjoin%
\definecolor{currentfill}{rgb}{0.000000,0.000000,0.000000}%
\pgfsetfillcolor{currentfill}%
\pgfsetlinewidth{0.501875pt}%
\definecolor{currentstroke}{rgb}{0.000000,0.000000,0.000000}%
\pgfsetstrokecolor{currentstroke}%
\pgfsetdash{}{0pt}%
\pgfsys@defobject{currentmarker}{\pgfqpoint{-0.020833in}{0.000000in}}{\pgfqpoint{-0.000000in}{0.000000in}}{%
\pgfpathmoveto{\pgfqpoint{-0.000000in}{0.000000in}}%
\pgfpathlineto{\pgfqpoint{-0.020833in}{0.000000in}}%
\pgfusepath{stroke,fill}%
}%
\begin{pgfscope}%
\pgfsys@transformshift{5.674512in}{1.360801in}%
\pgfsys@useobject{currentmarker}{}%
\end{pgfscope}%
\end{pgfscope}%
\begin{pgfscope}%
\pgfsetbuttcap%
\pgfsetroundjoin%
\definecolor{currentfill}{rgb}{0.000000,0.000000,0.000000}%
\pgfsetfillcolor{currentfill}%
\pgfsetlinewidth{0.501875pt}%
\definecolor{currentstroke}{rgb}{0.000000,0.000000,0.000000}%
\pgfsetstrokecolor{currentstroke}%
\pgfsetdash{}{0pt}%
\pgfsys@defobject{currentmarker}{\pgfqpoint{0.000000in}{0.000000in}}{\pgfqpoint{0.020833in}{0.000000in}}{%
\pgfpathmoveto{\pgfqpoint{0.000000in}{0.000000in}}%
\pgfpathlineto{\pgfqpoint{0.020833in}{0.000000in}}%
\pgfusepath{stroke,fill}%
}%
\begin{pgfscope}%
\pgfsys@transformshift{3.454822in}{1.468346in}%
\pgfsys@useobject{currentmarker}{}%
\end{pgfscope}%
\end{pgfscope}%
\begin{pgfscope}%
\pgfsetbuttcap%
\pgfsetroundjoin%
\definecolor{currentfill}{rgb}{0.000000,0.000000,0.000000}%
\pgfsetfillcolor{currentfill}%
\pgfsetlinewidth{0.501875pt}%
\definecolor{currentstroke}{rgb}{0.000000,0.000000,0.000000}%
\pgfsetstrokecolor{currentstroke}%
\pgfsetdash{}{0pt}%
\pgfsys@defobject{currentmarker}{\pgfqpoint{-0.020833in}{0.000000in}}{\pgfqpoint{-0.000000in}{0.000000in}}{%
\pgfpathmoveto{\pgfqpoint{-0.000000in}{0.000000in}}%
\pgfpathlineto{\pgfqpoint{-0.020833in}{0.000000in}}%
\pgfusepath{stroke,fill}%
}%
\begin{pgfscope}%
\pgfsys@transformshift{5.674512in}{1.468346in}%
\pgfsys@useobject{currentmarker}{}%
\end{pgfscope}%
\end{pgfscope}%
\begin{pgfscope}%
\pgfsetbuttcap%
\pgfsetroundjoin%
\definecolor{currentfill}{rgb}{0.000000,0.000000,0.000000}%
\pgfsetfillcolor{currentfill}%
\pgfsetlinewidth{0.501875pt}%
\definecolor{currentstroke}{rgb}{0.000000,0.000000,0.000000}%
\pgfsetstrokecolor{currentstroke}%
\pgfsetdash{}{0pt}%
\pgfsys@defobject{currentmarker}{\pgfqpoint{0.000000in}{0.000000in}}{\pgfqpoint{0.020833in}{0.000000in}}{%
\pgfpathmoveto{\pgfqpoint{0.000000in}{0.000000in}}%
\pgfpathlineto{\pgfqpoint{0.020833in}{0.000000in}}%
\pgfusepath{stroke,fill}%
}%
\begin{pgfscope}%
\pgfsys@transformshift{3.454822in}{1.575891in}%
\pgfsys@useobject{currentmarker}{}%
\end{pgfscope}%
\end{pgfscope}%
\begin{pgfscope}%
\pgfsetbuttcap%
\pgfsetroundjoin%
\definecolor{currentfill}{rgb}{0.000000,0.000000,0.000000}%
\pgfsetfillcolor{currentfill}%
\pgfsetlinewidth{0.501875pt}%
\definecolor{currentstroke}{rgb}{0.000000,0.000000,0.000000}%
\pgfsetstrokecolor{currentstroke}%
\pgfsetdash{}{0pt}%
\pgfsys@defobject{currentmarker}{\pgfqpoint{-0.020833in}{0.000000in}}{\pgfqpoint{-0.000000in}{0.000000in}}{%
\pgfpathmoveto{\pgfqpoint{-0.000000in}{0.000000in}}%
\pgfpathlineto{\pgfqpoint{-0.020833in}{0.000000in}}%
\pgfusepath{stroke,fill}%
}%
\begin{pgfscope}%
\pgfsys@transformshift{5.674512in}{1.575891in}%
\pgfsys@useobject{currentmarker}{}%
\end{pgfscope}%
\end{pgfscope}%
\begin{pgfscope}%
\pgfsetbuttcap%
\pgfsetroundjoin%
\definecolor{currentfill}{rgb}{0.000000,0.000000,0.000000}%
\pgfsetfillcolor{currentfill}%
\pgfsetlinewidth{0.501875pt}%
\definecolor{currentstroke}{rgb}{0.000000,0.000000,0.000000}%
\pgfsetstrokecolor{currentstroke}%
\pgfsetdash{}{0pt}%
\pgfsys@defobject{currentmarker}{\pgfqpoint{0.000000in}{0.000000in}}{\pgfqpoint{0.020833in}{0.000000in}}{%
\pgfpathmoveto{\pgfqpoint{0.000000in}{0.000000in}}%
\pgfpathlineto{\pgfqpoint{0.020833in}{0.000000in}}%
\pgfusepath{stroke,fill}%
}%
\begin{pgfscope}%
\pgfsys@transformshift{3.454822in}{1.683436in}%
\pgfsys@useobject{currentmarker}{}%
\end{pgfscope}%
\end{pgfscope}%
\begin{pgfscope}%
\pgfsetbuttcap%
\pgfsetroundjoin%
\definecolor{currentfill}{rgb}{0.000000,0.000000,0.000000}%
\pgfsetfillcolor{currentfill}%
\pgfsetlinewidth{0.501875pt}%
\definecolor{currentstroke}{rgb}{0.000000,0.000000,0.000000}%
\pgfsetstrokecolor{currentstroke}%
\pgfsetdash{}{0pt}%
\pgfsys@defobject{currentmarker}{\pgfqpoint{-0.020833in}{0.000000in}}{\pgfqpoint{-0.000000in}{0.000000in}}{%
\pgfpathmoveto{\pgfqpoint{-0.000000in}{0.000000in}}%
\pgfpathlineto{\pgfqpoint{-0.020833in}{0.000000in}}%
\pgfusepath{stroke,fill}%
}%
\begin{pgfscope}%
\pgfsys@transformshift{5.674512in}{1.683436in}%
\pgfsys@useobject{currentmarker}{}%
\end{pgfscope}%
\end{pgfscope}%
\begin{pgfscope}%
\pgfsetbuttcap%
\pgfsetroundjoin%
\definecolor{currentfill}{rgb}{0.000000,0.000000,0.000000}%
\pgfsetfillcolor{currentfill}%
\pgfsetlinewidth{0.501875pt}%
\definecolor{currentstroke}{rgb}{0.000000,0.000000,0.000000}%
\pgfsetstrokecolor{currentstroke}%
\pgfsetdash{}{0pt}%
\pgfsys@defobject{currentmarker}{\pgfqpoint{0.000000in}{0.000000in}}{\pgfqpoint{0.020833in}{0.000000in}}{%
\pgfpathmoveto{\pgfqpoint{0.000000in}{0.000000in}}%
\pgfpathlineto{\pgfqpoint{0.020833in}{0.000000in}}%
\pgfusepath{stroke,fill}%
}%
\begin{pgfscope}%
\pgfsys@transformshift{3.454822in}{1.898527in}%
\pgfsys@useobject{currentmarker}{}%
\end{pgfscope}%
\end{pgfscope}%
\begin{pgfscope}%
\pgfsetbuttcap%
\pgfsetroundjoin%
\definecolor{currentfill}{rgb}{0.000000,0.000000,0.000000}%
\pgfsetfillcolor{currentfill}%
\pgfsetlinewidth{0.501875pt}%
\definecolor{currentstroke}{rgb}{0.000000,0.000000,0.000000}%
\pgfsetstrokecolor{currentstroke}%
\pgfsetdash{}{0pt}%
\pgfsys@defobject{currentmarker}{\pgfqpoint{-0.020833in}{0.000000in}}{\pgfqpoint{-0.000000in}{0.000000in}}{%
\pgfpathmoveto{\pgfqpoint{-0.000000in}{0.000000in}}%
\pgfpathlineto{\pgfqpoint{-0.020833in}{0.000000in}}%
\pgfusepath{stroke,fill}%
}%
\begin{pgfscope}%
\pgfsys@transformshift{5.674512in}{1.898527in}%
\pgfsys@useobject{currentmarker}{}%
\end{pgfscope}%
\end{pgfscope}%
\begin{pgfscope}%
\pgfsetbuttcap%
\pgfsetroundjoin%
\definecolor{currentfill}{rgb}{0.000000,0.000000,0.000000}%
\pgfsetfillcolor{currentfill}%
\pgfsetlinewidth{0.501875pt}%
\definecolor{currentstroke}{rgb}{0.000000,0.000000,0.000000}%
\pgfsetstrokecolor{currentstroke}%
\pgfsetdash{}{0pt}%
\pgfsys@defobject{currentmarker}{\pgfqpoint{0.000000in}{0.000000in}}{\pgfqpoint{0.020833in}{0.000000in}}{%
\pgfpathmoveto{\pgfqpoint{0.000000in}{0.000000in}}%
\pgfpathlineto{\pgfqpoint{0.020833in}{0.000000in}}%
\pgfusepath{stroke,fill}%
}%
\begin{pgfscope}%
\pgfsys@transformshift{3.454822in}{2.006072in}%
\pgfsys@useobject{currentmarker}{}%
\end{pgfscope}%
\end{pgfscope}%
\begin{pgfscope}%
\pgfsetbuttcap%
\pgfsetroundjoin%
\definecolor{currentfill}{rgb}{0.000000,0.000000,0.000000}%
\pgfsetfillcolor{currentfill}%
\pgfsetlinewidth{0.501875pt}%
\definecolor{currentstroke}{rgb}{0.000000,0.000000,0.000000}%
\pgfsetstrokecolor{currentstroke}%
\pgfsetdash{}{0pt}%
\pgfsys@defobject{currentmarker}{\pgfqpoint{-0.020833in}{0.000000in}}{\pgfqpoint{-0.000000in}{0.000000in}}{%
\pgfpathmoveto{\pgfqpoint{-0.000000in}{0.000000in}}%
\pgfpathlineto{\pgfqpoint{-0.020833in}{0.000000in}}%
\pgfusepath{stroke,fill}%
}%
\begin{pgfscope}%
\pgfsys@transformshift{5.674512in}{2.006072in}%
\pgfsys@useobject{currentmarker}{}%
\end{pgfscope}%
\end{pgfscope}%
\begin{pgfscope}%
\pgfsetbuttcap%
\pgfsetroundjoin%
\definecolor{currentfill}{rgb}{0.000000,0.000000,0.000000}%
\pgfsetfillcolor{currentfill}%
\pgfsetlinewidth{0.501875pt}%
\definecolor{currentstroke}{rgb}{0.000000,0.000000,0.000000}%
\pgfsetstrokecolor{currentstroke}%
\pgfsetdash{}{0pt}%
\pgfsys@defobject{currentmarker}{\pgfqpoint{0.000000in}{0.000000in}}{\pgfqpoint{0.020833in}{0.000000in}}{%
\pgfpathmoveto{\pgfqpoint{0.000000in}{0.000000in}}%
\pgfpathlineto{\pgfqpoint{0.020833in}{0.000000in}}%
\pgfusepath{stroke,fill}%
}%
\begin{pgfscope}%
\pgfsys@transformshift{3.454822in}{2.113617in}%
\pgfsys@useobject{currentmarker}{}%
\end{pgfscope}%
\end{pgfscope}%
\begin{pgfscope}%
\pgfsetbuttcap%
\pgfsetroundjoin%
\definecolor{currentfill}{rgb}{0.000000,0.000000,0.000000}%
\pgfsetfillcolor{currentfill}%
\pgfsetlinewidth{0.501875pt}%
\definecolor{currentstroke}{rgb}{0.000000,0.000000,0.000000}%
\pgfsetstrokecolor{currentstroke}%
\pgfsetdash{}{0pt}%
\pgfsys@defobject{currentmarker}{\pgfqpoint{-0.020833in}{0.000000in}}{\pgfqpoint{-0.000000in}{0.000000in}}{%
\pgfpathmoveto{\pgfqpoint{-0.000000in}{0.000000in}}%
\pgfpathlineto{\pgfqpoint{-0.020833in}{0.000000in}}%
\pgfusepath{stroke,fill}%
}%
\begin{pgfscope}%
\pgfsys@transformshift{5.674512in}{2.113617in}%
\pgfsys@useobject{currentmarker}{}%
\end{pgfscope}%
\end{pgfscope}%
\begin{pgfscope}%
\pgfsetbuttcap%
\pgfsetroundjoin%
\definecolor{currentfill}{rgb}{0.000000,0.000000,0.000000}%
\pgfsetfillcolor{currentfill}%
\pgfsetlinewidth{0.501875pt}%
\definecolor{currentstroke}{rgb}{0.000000,0.000000,0.000000}%
\pgfsetstrokecolor{currentstroke}%
\pgfsetdash{}{0pt}%
\pgfsys@defobject{currentmarker}{\pgfqpoint{0.000000in}{0.000000in}}{\pgfqpoint{0.020833in}{0.000000in}}{%
\pgfpathmoveto{\pgfqpoint{0.000000in}{0.000000in}}%
\pgfpathlineto{\pgfqpoint{0.020833in}{0.000000in}}%
\pgfusepath{stroke,fill}%
}%
\begin{pgfscope}%
\pgfsys@transformshift{3.454822in}{2.221163in}%
\pgfsys@useobject{currentmarker}{}%
\end{pgfscope}%
\end{pgfscope}%
\begin{pgfscope}%
\pgfsetbuttcap%
\pgfsetroundjoin%
\definecolor{currentfill}{rgb}{0.000000,0.000000,0.000000}%
\pgfsetfillcolor{currentfill}%
\pgfsetlinewidth{0.501875pt}%
\definecolor{currentstroke}{rgb}{0.000000,0.000000,0.000000}%
\pgfsetstrokecolor{currentstroke}%
\pgfsetdash{}{0pt}%
\pgfsys@defobject{currentmarker}{\pgfqpoint{-0.020833in}{0.000000in}}{\pgfqpoint{-0.000000in}{0.000000in}}{%
\pgfpathmoveto{\pgfqpoint{-0.000000in}{0.000000in}}%
\pgfpathlineto{\pgfqpoint{-0.020833in}{0.000000in}}%
\pgfusepath{stroke,fill}%
}%
\begin{pgfscope}%
\pgfsys@transformshift{5.674512in}{2.221163in}%
\pgfsys@useobject{currentmarker}{}%
\end{pgfscope}%
\end{pgfscope}%
\begin{pgfscope}%
\pgfsetbuttcap%
\pgfsetroundjoin%
\definecolor{currentfill}{rgb}{0.000000,0.000000,0.000000}%
\pgfsetfillcolor{currentfill}%
\pgfsetlinewidth{0.501875pt}%
\definecolor{currentstroke}{rgb}{0.000000,0.000000,0.000000}%
\pgfsetstrokecolor{currentstroke}%
\pgfsetdash{}{0pt}%
\pgfsys@defobject{currentmarker}{\pgfqpoint{0.000000in}{0.000000in}}{\pgfqpoint{0.020833in}{0.000000in}}{%
\pgfpathmoveto{\pgfqpoint{0.000000in}{0.000000in}}%
\pgfpathlineto{\pgfqpoint{0.020833in}{0.000000in}}%
\pgfusepath{stroke,fill}%
}%
\begin{pgfscope}%
\pgfsys@transformshift{3.454822in}{2.436253in}%
\pgfsys@useobject{currentmarker}{}%
\end{pgfscope}%
\end{pgfscope}%
\begin{pgfscope}%
\pgfsetbuttcap%
\pgfsetroundjoin%
\definecolor{currentfill}{rgb}{0.000000,0.000000,0.000000}%
\pgfsetfillcolor{currentfill}%
\pgfsetlinewidth{0.501875pt}%
\definecolor{currentstroke}{rgb}{0.000000,0.000000,0.000000}%
\pgfsetstrokecolor{currentstroke}%
\pgfsetdash{}{0pt}%
\pgfsys@defobject{currentmarker}{\pgfqpoint{-0.020833in}{0.000000in}}{\pgfqpoint{-0.000000in}{0.000000in}}{%
\pgfpathmoveto{\pgfqpoint{-0.000000in}{0.000000in}}%
\pgfpathlineto{\pgfqpoint{-0.020833in}{0.000000in}}%
\pgfusepath{stroke,fill}%
}%
\begin{pgfscope}%
\pgfsys@transformshift{5.674512in}{2.436253in}%
\pgfsys@useobject{currentmarker}{}%
\end{pgfscope}%
\end{pgfscope}%
\begin{pgfscope}%
\pgfsetbuttcap%
\pgfsetroundjoin%
\definecolor{currentfill}{rgb}{0.000000,0.000000,0.000000}%
\pgfsetfillcolor{currentfill}%
\pgfsetlinewidth{0.501875pt}%
\definecolor{currentstroke}{rgb}{0.000000,0.000000,0.000000}%
\pgfsetstrokecolor{currentstroke}%
\pgfsetdash{}{0pt}%
\pgfsys@defobject{currentmarker}{\pgfqpoint{0.000000in}{0.000000in}}{\pgfqpoint{0.020833in}{0.000000in}}{%
\pgfpathmoveto{\pgfqpoint{0.000000in}{0.000000in}}%
\pgfpathlineto{\pgfqpoint{0.020833in}{0.000000in}}%
\pgfusepath{stroke,fill}%
}%
\begin{pgfscope}%
\pgfsys@transformshift{3.454822in}{2.543798in}%
\pgfsys@useobject{currentmarker}{}%
\end{pgfscope}%
\end{pgfscope}%
\begin{pgfscope}%
\pgfsetbuttcap%
\pgfsetroundjoin%
\definecolor{currentfill}{rgb}{0.000000,0.000000,0.000000}%
\pgfsetfillcolor{currentfill}%
\pgfsetlinewidth{0.501875pt}%
\definecolor{currentstroke}{rgb}{0.000000,0.000000,0.000000}%
\pgfsetstrokecolor{currentstroke}%
\pgfsetdash{}{0pt}%
\pgfsys@defobject{currentmarker}{\pgfqpoint{-0.020833in}{0.000000in}}{\pgfqpoint{-0.000000in}{0.000000in}}{%
\pgfpathmoveto{\pgfqpoint{-0.000000in}{0.000000in}}%
\pgfpathlineto{\pgfqpoint{-0.020833in}{0.000000in}}%
\pgfusepath{stroke,fill}%
}%
\begin{pgfscope}%
\pgfsys@transformshift{5.674512in}{2.543798in}%
\pgfsys@useobject{currentmarker}{}%
\end{pgfscope}%
\end{pgfscope}%
\begin{pgfscope}%
\pgfsetbuttcap%
\pgfsetroundjoin%
\definecolor{currentfill}{rgb}{0.000000,0.000000,0.000000}%
\pgfsetfillcolor{currentfill}%
\pgfsetlinewidth{0.501875pt}%
\definecolor{currentstroke}{rgb}{0.000000,0.000000,0.000000}%
\pgfsetstrokecolor{currentstroke}%
\pgfsetdash{}{0pt}%
\pgfsys@defobject{currentmarker}{\pgfqpoint{0.000000in}{0.000000in}}{\pgfqpoint{0.020833in}{0.000000in}}{%
\pgfpathmoveto{\pgfqpoint{0.000000in}{0.000000in}}%
\pgfpathlineto{\pgfqpoint{0.020833in}{0.000000in}}%
\pgfusepath{stroke,fill}%
}%
\begin{pgfscope}%
\pgfsys@transformshift{3.454822in}{2.651344in}%
\pgfsys@useobject{currentmarker}{}%
\end{pgfscope}%
\end{pgfscope}%
\begin{pgfscope}%
\pgfsetbuttcap%
\pgfsetroundjoin%
\definecolor{currentfill}{rgb}{0.000000,0.000000,0.000000}%
\pgfsetfillcolor{currentfill}%
\pgfsetlinewidth{0.501875pt}%
\definecolor{currentstroke}{rgb}{0.000000,0.000000,0.000000}%
\pgfsetstrokecolor{currentstroke}%
\pgfsetdash{}{0pt}%
\pgfsys@defobject{currentmarker}{\pgfqpoint{-0.020833in}{0.000000in}}{\pgfqpoint{-0.000000in}{0.000000in}}{%
\pgfpathmoveto{\pgfqpoint{-0.000000in}{0.000000in}}%
\pgfpathlineto{\pgfqpoint{-0.020833in}{0.000000in}}%
\pgfusepath{stroke,fill}%
}%
\begin{pgfscope}%
\pgfsys@transformshift{5.674512in}{2.651344in}%
\pgfsys@useobject{currentmarker}{}%
\end{pgfscope}%
\end{pgfscope}%
\begin{pgfscope}%
\pgfsetbuttcap%
\pgfsetroundjoin%
\definecolor{currentfill}{rgb}{0.000000,0.000000,0.000000}%
\pgfsetfillcolor{currentfill}%
\pgfsetlinewidth{0.501875pt}%
\definecolor{currentstroke}{rgb}{0.000000,0.000000,0.000000}%
\pgfsetstrokecolor{currentstroke}%
\pgfsetdash{}{0pt}%
\pgfsys@defobject{currentmarker}{\pgfqpoint{0.000000in}{0.000000in}}{\pgfqpoint{0.020833in}{0.000000in}}{%
\pgfpathmoveto{\pgfqpoint{0.000000in}{0.000000in}}%
\pgfpathlineto{\pgfqpoint{0.020833in}{0.000000in}}%
\pgfusepath{stroke,fill}%
}%
\begin{pgfscope}%
\pgfsys@transformshift{3.454822in}{2.758889in}%
\pgfsys@useobject{currentmarker}{}%
\end{pgfscope}%
\end{pgfscope}%
\begin{pgfscope}%
\pgfsetbuttcap%
\pgfsetroundjoin%
\definecolor{currentfill}{rgb}{0.000000,0.000000,0.000000}%
\pgfsetfillcolor{currentfill}%
\pgfsetlinewidth{0.501875pt}%
\definecolor{currentstroke}{rgb}{0.000000,0.000000,0.000000}%
\pgfsetstrokecolor{currentstroke}%
\pgfsetdash{}{0pt}%
\pgfsys@defobject{currentmarker}{\pgfqpoint{-0.020833in}{0.000000in}}{\pgfqpoint{-0.000000in}{0.000000in}}{%
\pgfpathmoveto{\pgfqpoint{-0.000000in}{0.000000in}}%
\pgfpathlineto{\pgfqpoint{-0.020833in}{0.000000in}}%
\pgfusepath{stroke,fill}%
}%
\begin{pgfscope}%
\pgfsys@transformshift{5.674512in}{2.758889in}%
\pgfsys@useobject{currentmarker}{}%
\end{pgfscope}%
\end{pgfscope}%
\begin{pgfscope}%
\pgfsetbuttcap%
\pgfsetroundjoin%
\definecolor{currentfill}{rgb}{0.000000,0.000000,0.000000}%
\pgfsetfillcolor{currentfill}%
\pgfsetlinewidth{0.501875pt}%
\definecolor{currentstroke}{rgb}{0.000000,0.000000,0.000000}%
\pgfsetstrokecolor{currentstroke}%
\pgfsetdash{}{0pt}%
\pgfsys@defobject{currentmarker}{\pgfqpoint{0.000000in}{0.000000in}}{\pgfqpoint{0.020833in}{0.000000in}}{%
\pgfpathmoveto{\pgfqpoint{0.000000in}{0.000000in}}%
\pgfpathlineto{\pgfqpoint{0.020833in}{0.000000in}}%
\pgfusepath{stroke,fill}%
}%
\begin{pgfscope}%
\pgfsys@transformshift{3.454822in}{2.973980in}%
\pgfsys@useobject{currentmarker}{}%
\end{pgfscope}%
\end{pgfscope}%
\begin{pgfscope}%
\pgfsetbuttcap%
\pgfsetroundjoin%
\definecolor{currentfill}{rgb}{0.000000,0.000000,0.000000}%
\pgfsetfillcolor{currentfill}%
\pgfsetlinewidth{0.501875pt}%
\definecolor{currentstroke}{rgb}{0.000000,0.000000,0.000000}%
\pgfsetstrokecolor{currentstroke}%
\pgfsetdash{}{0pt}%
\pgfsys@defobject{currentmarker}{\pgfqpoint{-0.020833in}{0.000000in}}{\pgfqpoint{-0.000000in}{0.000000in}}{%
\pgfpathmoveto{\pgfqpoint{-0.000000in}{0.000000in}}%
\pgfpathlineto{\pgfqpoint{-0.020833in}{0.000000in}}%
\pgfusepath{stroke,fill}%
}%
\begin{pgfscope}%
\pgfsys@transformshift{5.674512in}{2.973980in}%
\pgfsys@useobject{currentmarker}{}%
\end{pgfscope}%
\end{pgfscope}%
\begin{pgfscope}%
\pgfsetbuttcap%
\pgfsetroundjoin%
\definecolor{currentfill}{rgb}{0.000000,0.000000,0.000000}%
\pgfsetfillcolor{currentfill}%
\pgfsetlinewidth{0.501875pt}%
\definecolor{currentstroke}{rgb}{0.000000,0.000000,0.000000}%
\pgfsetstrokecolor{currentstroke}%
\pgfsetdash{}{0pt}%
\pgfsys@defobject{currentmarker}{\pgfqpoint{0.000000in}{0.000000in}}{\pgfqpoint{0.020833in}{0.000000in}}{%
\pgfpathmoveto{\pgfqpoint{0.000000in}{0.000000in}}%
\pgfpathlineto{\pgfqpoint{0.020833in}{0.000000in}}%
\pgfusepath{stroke,fill}%
}%
\begin{pgfscope}%
\pgfsys@transformshift{3.454822in}{3.081525in}%
\pgfsys@useobject{currentmarker}{}%
\end{pgfscope}%
\end{pgfscope}%
\begin{pgfscope}%
\pgfsetbuttcap%
\pgfsetroundjoin%
\definecolor{currentfill}{rgb}{0.000000,0.000000,0.000000}%
\pgfsetfillcolor{currentfill}%
\pgfsetlinewidth{0.501875pt}%
\definecolor{currentstroke}{rgb}{0.000000,0.000000,0.000000}%
\pgfsetstrokecolor{currentstroke}%
\pgfsetdash{}{0pt}%
\pgfsys@defobject{currentmarker}{\pgfqpoint{-0.020833in}{0.000000in}}{\pgfqpoint{-0.000000in}{0.000000in}}{%
\pgfpathmoveto{\pgfqpoint{-0.000000in}{0.000000in}}%
\pgfpathlineto{\pgfqpoint{-0.020833in}{0.000000in}}%
\pgfusepath{stroke,fill}%
}%
\begin{pgfscope}%
\pgfsys@transformshift{5.674512in}{3.081525in}%
\pgfsys@useobject{currentmarker}{}%
\end{pgfscope}%
\end{pgfscope}%
\begin{pgfscope}%
\pgfsetbuttcap%
\pgfsetroundjoin%
\definecolor{currentfill}{rgb}{0.000000,0.000000,0.000000}%
\pgfsetfillcolor{currentfill}%
\pgfsetlinewidth{0.501875pt}%
\definecolor{currentstroke}{rgb}{0.000000,0.000000,0.000000}%
\pgfsetstrokecolor{currentstroke}%
\pgfsetdash{}{0pt}%
\pgfsys@defobject{currentmarker}{\pgfqpoint{0.000000in}{0.000000in}}{\pgfqpoint{0.020833in}{0.000000in}}{%
\pgfpathmoveto{\pgfqpoint{0.000000in}{0.000000in}}%
\pgfpathlineto{\pgfqpoint{0.020833in}{0.000000in}}%
\pgfusepath{stroke,fill}%
}%
\begin{pgfscope}%
\pgfsys@transformshift{3.454822in}{3.189070in}%
\pgfsys@useobject{currentmarker}{}%
\end{pgfscope}%
\end{pgfscope}%
\begin{pgfscope}%
\pgfsetbuttcap%
\pgfsetroundjoin%
\definecolor{currentfill}{rgb}{0.000000,0.000000,0.000000}%
\pgfsetfillcolor{currentfill}%
\pgfsetlinewidth{0.501875pt}%
\definecolor{currentstroke}{rgb}{0.000000,0.000000,0.000000}%
\pgfsetstrokecolor{currentstroke}%
\pgfsetdash{}{0pt}%
\pgfsys@defobject{currentmarker}{\pgfqpoint{-0.020833in}{0.000000in}}{\pgfqpoint{-0.000000in}{0.000000in}}{%
\pgfpathmoveto{\pgfqpoint{-0.000000in}{0.000000in}}%
\pgfpathlineto{\pgfqpoint{-0.020833in}{0.000000in}}%
\pgfusepath{stroke,fill}%
}%
\begin{pgfscope}%
\pgfsys@transformshift{5.674512in}{3.189070in}%
\pgfsys@useobject{currentmarker}{}%
\end{pgfscope}%
\end{pgfscope}%
\begin{pgfscope}%
\pgfsetbuttcap%
\pgfsetroundjoin%
\definecolor{currentfill}{rgb}{0.000000,0.000000,0.000000}%
\pgfsetfillcolor{currentfill}%
\pgfsetlinewidth{0.501875pt}%
\definecolor{currentstroke}{rgb}{0.000000,0.000000,0.000000}%
\pgfsetstrokecolor{currentstroke}%
\pgfsetdash{}{0pt}%
\pgfsys@defobject{currentmarker}{\pgfqpoint{0.000000in}{0.000000in}}{\pgfqpoint{0.020833in}{0.000000in}}{%
\pgfpathmoveto{\pgfqpoint{0.000000in}{0.000000in}}%
\pgfpathlineto{\pgfqpoint{0.020833in}{0.000000in}}%
\pgfusepath{stroke,fill}%
}%
\begin{pgfscope}%
\pgfsys@transformshift{3.454822in}{3.296615in}%
\pgfsys@useobject{currentmarker}{}%
\end{pgfscope}%
\end{pgfscope}%
\begin{pgfscope}%
\pgfsetbuttcap%
\pgfsetroundjoin%
\definecolor{currentfill}{rgb}{0.000000,0.000000,0.000000}%
\pgfsetfillcolor{currentfill}%
\pgfsetlinewidth{0.501875pt}%
\definecolor{currentstroke}{rgb}{0.000000,0.000000,0.000000}%
\pgfsetstrokecolor{currentstroke}%
\pgfsetdash{}{0pt}%
\pgfsys@defobject{currentmarker}{\pgfqpoint{-0.020833in}{0.000000in}}{\pgfqpoint{-0.000000in}{0.000000in}}{%
\pgfpathmoveto{\pgfqpoint{-0.000000in}{0.000000in}}%
\pgfpathlineto{\pgfqpoint{-0.020833in}{0.000000in}}%
\pgfusepath{stroke,fill}%
}%
\begin{pgfscope}%
\pgfsys@transformshift{5.674512in}{3.296615in}%
\pgfsys@useobject{currentmarker}{}%
\end{pgfscope}%
\end{pgfscope}%
\begin{pgfscope}%
\pgfsetbuttcap%
\pgfsetroundjoin%
\definecolor{currentfill}{rgb}{0.000000,0.000000,0.000000}%
\pgfsetfillcolor{currentfill}%
\pgfsetlinewidth{0.501875pt}%
\definecolor{currentstroke}{rgb}{0.000000,0.000000,0.000000}%
\pgfsetstrokecolor{currentstroke}%
\pgfsetdash{}{0pt}%
\pgfsys@defobject{currentmarker}{\pgfqpoint{0.000000in}{0.000000in}}{\pgfqpoint{0.020833in}{0.000000in}}{%
\pgfpathmoveto{\pgfqpoint{0.000000in}{0.000000in}}%
\pgfpathlineto{\pgfqpoint{0.020833in}{0.000000in}}%
\pgfusepath{stroke,fill}%
}%
\begin{pgfscope}%
\pgfsys@transformshift{3.454822in}{3.511706in}%
\pgfsys@useobject{currentmarker}{}%
\end{pgfscope}%
\end{pgfscope}%
\begin{pgfscope}%
\pgfsetbuttcap%
\pgfsetroundjoin%
\definecolor{currentfill}{rgb}{0.000000,0.000000,0.000000}%
\pgfsetfillcolor{currentfill}%
\pgfsetlinewidth{0.501875pt}%
\definecolor{currentstroke}{rgb}{0.000000,0.000000,0.000000}%
\pgfsetstrokecolor{currentstroke}%
\pgfsetdash{}{0pt}%
\pgfsys@defobject{currentmarker}{\pgfqpoint{-0.020833in}{0.000000in}}{\pgfqpoint{-0.000000in}{0.000000in}}{%
\pgfpathmoveto{\pgfqpoint{-0.000000in}{0.000000in}}%
\pgfpathlineto{\pgfqpoint{-0.020833in}{0.000000in}}%
\pgfusepath{stroke,fill}%
}%
\begin{pgfscope}%
\pgfsys@transformshift{5.674512in}{3.511706in}%
\pgfsys@useobject{currentmarker}{}%
\end{pgfscope}%
\end{pgfscope}%
\begin{pgfscope}%
\definecolor{textcolor}{rgb}{0.000000,0.000000,0.000000}%
\pgfsetstrokecolor{textcolor}%
\pgfsetfillcolor{textcolor}%
\pgftext[x=3.173186in,y=1.998593in,,bottom,rotate=90.000000]{\color{textcolor}\rmfamily\fontsize{10.000000}{12.000000}\selectfont LCMC\(\displaystyle (K)\)}%
\end{pgfscope}%
\begin{pgfscope}%
\pgfpathrectangle{\pgfqpoint{3.454822in}{0.422992in}}{\pgfqpoint{2.219690in}{3.151201in}}%
\pgfusepath{clip}%
\pgfsetrectcap%
\pgfsetroundjoin%
\pgfsetlinewidth{1.003750pt}%
\definecolor{currentstroke}{rgb}{0.047059,0.364706,0.647059}%
\pgfsetstrokecolor{currentstroke}%
\pgfsetdash{}{0pt}%
\pgfpathmoveto{\pgfqpoint{3.476584in}{2.010662in}}%
\pgfpathlineto{\pgfqpoint{3.498345in}{2.086700in}}%
\pgfpathlineto{\pgfqpoint{3.520107in}{2.117638in}}%
\pgfpathlineto{\pgfqpoint{3.541869in}{2.153918in}}%
\pgfpathlineto{\pgfqpoint{3.563630in}{2.199347in}}%
\pgfpathlineto{\pgfqpoint{3.585392in}{2.233362in}}%
\pgfpathlineto{\pgfqpoint{3.607154in}{2.265278in}}%
\pgfpathlineto{\pgfqpoint{3.628915in}{2.288597in}}%
\pgfpathlineto{\pgfqpoint{3.650677in}{2.312064in}}%
\pgfpathlineto{\pgfqpoint{3.672439in}{2.333999in}}%
\pgfpathlineto{\pgfqpoint{3.694200in}{2.358966in}}%
\pgfpathlineto{\pgfqpoint{3.715962in}{2.378930in}}%
\pgfpathlineto{\pgfqpoint{3.737724in}{2.396898in}}%
\pgfpathlineto{\pgfqpoint{3.759485in}{2.412621in}}%
\pgfpathlineto{\pgfqpoint{3.781247in}{2.429116in}}%
\pgfpathlineto{\pgfqpoint{3.803009in}{2.443079in}}%
\pgfpathlineto{\pgfqpoint{3.824770in}{2.457372in}}%
\pgfpathlineto{\pgfqpoint{3.846532in}{2.470592in}}%
\pgfpathlineto{\pgfqpoint{3.868294in}{2.484243in}}%
\pgfpathlineto{\pgfqpoint{3.890055in}{2.494173in}}%
\pgfpathlineto{\pgfqpoint{3.911817in}{2.506087in}}%
\pgfpathlineto{\pgfqpoint{3.933579in}{2.518169in}}%
\pgfpathlineto{\pgfqpoint{3.955340in}{2.528144in}}%
\pgfpathlineto{\pgfqpoint{3.977102in}{2.539062in}}%
\pgfpathlineto{\pgfqpoint{3.998864in}{2.548806in}}%
\pgfpathlineto{\pgfqpoint{4.020625in}{2.557709in}}%
\pgfpathlineto{\pgfqpoint{4.042387in}{2.566239in}}%
\pgfpathlineto{\pgfqpoint{4.064149in}{2.574836in}}%
\pgfpathlineto{\pgfqpoint{4.085910in}{2.583412in}}%
\pgfpathlineto{\pgfqpoint{4.107672in}{2.592498in}}%
\pgfpathlineto{\pgfqpoint{4.129434in}{2.600880in}}%
\pgfpathlineto{\pgfqpoint{4.151195in}{2.608456in}}%
\pgfpathlineto{\pgfqpoint{4.172957in}{2.615306in}}%
\pgfpathlineto{\pgfqpoint{4.194719in}{2.623138in}}%
\pgfpathlineto{\pgfqpoint{4.216480in}{2.630830in}}%
\pgfpathlineto{\pgfqpoint{4.238242in}{2.638232in}}%
\pgfpathlineto{\pgfqpoint{4.260004in}{2.645768in}}%
\pgfpathlineto{\pgfqpoint{4.281765in}{2.654550in}}%
\pgfpathlineto{\pgfqpoint{4.303527in}{2.662407in}}%
\pgfpathlineto{\pgfqpoint{4.325289in}{2.669107in}}%
\pgfpathlineto{\pgfqpoint{4.347050in}{2.675229in}}%
\pgfpathlineto{\pgfqpoint{4.368812in}{2.682565in}}%
\pgfpathlineto{\pgfqpoint{4.390574in}{2.689105in}}%
\pgfpathlineto{\pgfqpoint{4.412335in}{2.695982in}}%
\pgfpathlineto{\pgfqpoint{4.434097in}{2.702469in}}%
\pgfpathlineto{\pgfqpoint{4.455859in}{2.708331in}}%
\pgfpathlineto{\pgfqpoint{4.477620in}{2.714160in}}%
\pgfpathlineto{\pgfqpoint{4.499382in}{2.720261in}}%
\pgfpathlineto{\pgfqpoint{4.521144in}{2.725897in}}%
\pgfpathlineto{\pgfqpoint{4.542905in}{2.731098in}}%
\pgfpathlineto{\pgfqpoint{4.564667in}{2.735972in}}%
\pgfpathlineto{\pgfqpoint{4.586429in}{2.740291in}}%
\pgfpathlineto{\pgfqpoint{4.608190in}{2.744913in}}%
\pgfpathlineto{\pgfqpoint{4.629952in}{2.749424in}}%
\pgfpathlineto{\pgfqpoint{4.651714in}{2.754400in}}%
\pgfpathlineto{\pgfqpoint{4.673475in}{2.758742in}}%
\pgfpathlineto{\pgfqpoint{4.695237in}{2.763569in}}%
\pgfpathlineto{\pgfqpoint{4.716999in}{2.768983in}}%
\pgfpathlineto{\pgfqpoint{4.738760in}{2.773655in}}%
\pgfpathlineto{\pgfqpoint{4.760522in}{2.778562in}}%
\pgfpathlineto{\pgfqpoint{4.782284in}{2.783030in}}%
\pgfpathlineto{\pgfqpoint{4.804045in}{2.788599in}}%
\pgfpathlineto{\pgfqpoint{4.825807in}{2.793042in}}%
\pgfpathlineto{\pgfqpoint{4.847569in}{2.797373in}}%
\pgfpathlineto{\pgfqpoint{4.869330in}{2.801674in}}%
\pgfpathlineto{\pgfqpoint{4.891092in}{2.805955in}}%
\pgfpathlineto{\pgfqpoint{4.912854in}{2.809880in}}%
\pgfpathlineto{\pgfqpoint{4.934615in}{2.814142in}}%
\pgfpathlineto{\pgfqpoint{4.956377in}{2.818375in}}%
\pgfpathlineto{\pgfqpoint{4.978139in}{2.822209in}}%
\pgfpathlineto{\pgfqpoint{4.999900in}{2.826384in}}%
\pgfpathlineto{\pgfqpoint{5.021662in}{2.830829in}}%
\pgfpathlineto{\pgfqpoint{5.043424in}{2.834992in}}%
\pgfpathlineto{\pgfqpoint{5.065185in}{2.839043in}}%
\pgfpathlineto{\pgfqpoint{5.086947in}{2.843437in}}%
\pgfpathlineto{\pgfqpoint{5.108709in}{2.847089in}}%
\pgfpathlineto{\pgfqpoint{5.130470in}{2.850445in}}%
\pgfpathlineto{\pgfqpoint{5.152232in}{2.854063in}}%
\pgfpathlineto{\pgfqpoint{5.173994in}{2.858139in}}%
\pgfpathlineto{\pgfqpoint{5.195755in}{2.861806in}}%
\pgfpathlineto{\pgfqpoint{5.217517in}{2.865678in}}%
\pgfpathlineto{\pgfqpoint{5.239279in}{2.868902in}}%
\pgfpathlineto{\pgfqpoint{5.261040in}{2.872473in}}%
\pgfpathlineto{\pgfqpoint{5.282802in}{2.875813in}}%
\pgfpathlineto{\pgfqpoint{5.304564in}{2.879763in}}%
\pgfpathlineto{\pgfqpoint{5.326325in}{2.883189in}}%
\pgfpathlineto{\pgfqpoint{5.348087in}{2.886152in}}%
\pgfpathlineto{\pgfqpoint{5.369849in}{2.889134in}}%
\pgfpathlineto{\pgfqpoint{5.391610in}{2.892355in}}%
\pgfpathlineto{\pgfqpoint{5.413372in}{2.895981in}}%
\pgfpathlineto{\pgfqpoint{5.435134in}{2.899132in}}%
\pgfpathlineto{\pgfqpoint{5.456895in}{2.902472in}}%
\pgfpathlineto{\pgfqpoint{5.478657in}{2.905159in}}%
\pgfpathlineto{\pgfqpoint{5.500419in}{2.908096in}}%
\pgfpathlineto{\pgfqpoint{5.522180in}{2.910980in}}%
\pgfpathlineto{\pgfqpoint{5.543942in}{2.914230in}}%
\pgfpathlineto{\pgfqpoint{5.565704in}{2.917072in}}%
\pgfpathlineto{\pgfqpoint{5.587465in}{2.919602in}}%
\pgfpathlineto{\pgfqpoint{5.609227in}{2.922449in}}%
\pgfpathlineto{\pgfqpoint{5.630989in}{2.925278in}}%
\pgfusepath{stroke}%
\end{pgfscope}%
\begin{pgfscope}%
\pgfpathrectangle{\pgfqpoint{3.454822in}{0.422992in}}{\pgfqpoint{2.219690in}{3.151201in}}%
\pgfusepath{clip}%
\pgfsetrectcap%
\pgfsetroundjoin%
\pgfsetlinewidth{1.003750pt}%
\definecolor{currentstroke}{rgb}{0.000000,0.725490,0.270588}%
\pgfsetstrokecolor{currentstroke}%
\pgfsetdash{}{0pt}%
\pgfpathmoveto{\pgfqpoint{3.476584in}{1.885735in}}%
\pgfpathlineto{\pgfqpoint{3.498345in}{1.950625in}}%
\pgfpathlineto{\pgfqpoint{3.520107in}{2.028182in}}%
\pgfpathlineto{\pgfqpoint{3.541869in}{2.075834in}}%
\pgfpathlineto{\pgfqpoint{3.563630in}{2.111810in}}%
\pgfpathlineto{\pgfqpoint{3.585392in}{2.144698in}}%
\pgfpathlineto{\pgfqpoint{3.607154in}{2.166703in}}%
\pgfpathlineto{\pgfqpoint{3.628915in}{2.189212in}}%
\pgfpathlineto{\pgfqpoint{3.650677in}{2.215762in}}%
\pgfpathlineto{\pgfqpoint{3.672439in}{2.234026in}}%
\pgfpathlineto{\pgfqpoint{3.694200in}{2.249621in}}%
\pgfpathlineto{\pgfqpoint{3.715962in}{2.262258in}}%
\pgfpathlineto{\pgfqpoint{3.737724in}{2.276729in}}%
\pgfpathlineto{\pgfqpoint{3.759485in}{2.292411in}}%
\pgfpathlineto{\pgfqpoint{3.781247in}{2.308057in}}%
\pgfpathlineto{\pgfqpoint{3.803009in}{2.323339in}}%
\pgfpathlineto{\pgfqpoint{3.824770in}{2.335156in}}%
\pgfpathlineto{\pgfqpoint{3.846532in}{2.345800in}}%
\pgfpathlineto{\pgfqpoint{3.868294in}{2.357192in}}%
\pgfpathlineto{\pgfqpoint{3.890055in}{2.364487in}}%
\pgfpathlineto{\pgfqpoint{3.911817in}{2.372094in}}%
\pgfpathlineto{\pgfqpoint{3.933579in}{2.378276in}}%
\pgfpathlineto{\pgfqpoint{3.955340in}{2.385760in}}%
\pgfpathlineto{\pgfqpoint{3.977102in}{2.394981in}}%
\pgfpathlineto{\pgfqpoint{3.998864in}{2.402847in}}%
\pgfpathlineto{\pgfqpoint{4.020625in}{2.409901in}}%
\pgfpathlineto{\pgfqpoint{4.042387in}{2.417601in}}%
\pgfpathlineto{\pgfqpoint{4.064149in}{2.424239in}}%
\pgfpathlineto{\pgfqpoint{4.085910in}{2.428813in}}%
\pgfpathlineto{\pgfqpoint{4.107672in}{2.435698in}}%
\pgfpathlineto{\pgfqpoint{4.129434in}{2.438253in}}%
\pgfpathlineto{\pgfqpoint{4.151195in}{2.441131in}}%
\pgfpathlineto{\pgfqpoint{4.172957in}{2.446409in}}%
\pgfpathlineto{\pgfqpoint{4.194719in}{2.451566in}}%
\pgfpathlineto{\pgfqpoint{4.216480in}{2.455957in}}%
\pgfpathlineto{\pgfqpoint{4.238242in}{2.458710in}}%
\pgfpathlineto{\pgfqpoint{4.260004in}{2.462332in}}%
\pgfpathlineto{\pgfqpoint{4.281765in}{2.466536in}}%
\pgfpathlineto{\pgfqpoint{4.303527in}{2.471527in}}%
\pgfpathlineto{\pgfqpoint{4.325289in}{2.476000in}}%
\pgfpathlineto{\pgfqpoint{4.347050in}{2.479607in}}%
\pgfpathlineto{\pgfqpoint{4.368812in}{2.483897in}}%
\pgfpathlineto{\pgfqpoint{4.390574in}{2.488245in}}%
\pgfpathlineto{\pgfqpoint{4.412335in}{2.492615in}}%
\pgfpathlineto{\pgfqpoint{4.434097in}{2.495541in}}%
\pgfpathlineto{\pgfqpoint{4.455859in}{2.498986in}}%
\pgfpathlineto{\pgfqpoint{4.477620in}{2.501865in}}%
\pgfpathlineto{\pgfqpoint{4.499382in}{2.504639in}}%
\pgfpathlineto{\pgfqpoint{4.521144in}{2.507212in}}%
\pgfpathlineto{\pgfqpoint{4.542905in}{2.509904in}}%
\pgfpathlineto{\pgfqpoint{4.564667in}{2.512413in}}%
\pgfpathlineto{\pgfqpoint{4.586429in}{2.515185in}}%
\pgfpathlineto{\pgfqpoint{4.608190in}{2.517730in}}%
\pgfpathlineto{\pgfqpoint{4.629952in}{2.520672in}}%
\pgfpathlineto{\pgfqpoint{4.651714in}{2.522236in}}%
\pgfpathlineto{\pgfqpoint{4.673475in}{2.523738in}}%
\pgfpathlineto{\pgfqpoint{4.695237in}{2.526356in}}%
\pgfpathlineto{\pgfqpoint{4.716999in}{2.528990in}}%
\pgfpathlineto{\pgfqpoint{4.738760in}{2.531067in}}%
\pgfpathlineto{\pgfqpoint{4.760522in}{2.533540in}}%
\pgfpathlineto{\pgfqpoint{4.782284in}{2.535615in}}%
\pgfpathlineto{\pgfqpoint{4.804045in}{2.538219in}}%
\pgfpathlineto{\pgfqpoint{4.825807in}{2.541087in}}%
\pgfpathlineto{\pgfqpoint{4.847569in}{2.543053in}}%
\pgfpathlineto{\pgfqpoint{4.869330in}{2.545240in}}%
\pgfpathlineto{\pgfqpoint{4.891092in}{2.547203in}}%
\pgfpathlineto{\pgfqpoint{4.912854in}{2.549038in}}%
\pgfpathlineto{\pgfqpoint{4.934615in}{2.552121in}}%
\pgfpathlineto{\pgfqpoint{4.956377in}{2.554258in}}%
\pgfpathlineto{\pgfqpoint{4.978139in}{2.555498in}}%
\pgfpathlineto{\pgfqpoint{4.999900in}{2.557123in}}%
\pgfpathlineto{\pgfqpoint{5.021662in}{2.559918in}}%
\pgfpathlineto{\pgfqpoint{5.043424in}{2.561934in}}%
\pgfpathlineto{\pgfqpoint{5.065185in}{2.563265in}}%
\pgfpathlineto{\pgfqpoint{5.086947in}{2.564394in}}%
\pgfpathlineto{\pgfqpoint{5.108709in}{2.566300in}}%
\pgfpathlineto{\pgfqpoint{5.130470in}{2.567504in}}%
\pgfpathlineto{\pgfqpoint{5.152232in}{2.569252in}}%
\pgfpathlineto{\pgfqpoint{5.173994in}{2.570806in}}%
\pgfpathlineto{\pgfqpoint{5.195755in}{2.572554in}}%
\pgfpathlineto{\pgfqpoint{5.217517in}{2.573936in}}%
\pgfpathlineto{\pgfqpoint{5.239279in}{2.574956in}}%
\pgfpathlineto{\pgfqpoint{5.261040in}{2.576492in}}%
\pgfpathlineto{\pgfqpoint{5.282802in}{2.578388in}}%
\pgfpathlineto{\pgfqpoint{5.304564in}{2.580518in}}%
\pgfpathlineto{\pgfqpoint{5.326325in}{2.582936in}}%
\pgfpathlineto{\pgfqpoint{5.348087in}{2.584626in}}%
\pgfpathlineto{\pgfqpoint{5.369849in}{2.585843in}}%
\pgfpathlineto{\pgfqpoint{5.391610in}{2.587442in}}%
\pgfpathlineto{\pgfqpoint{5.413372in}{2.589174in}}%
\pgfpathlineto{\pgfqpoint{5.435134in}{2.590048in}}%
\pgfpathlineto{\pgfqpoint{5.456895in}{2.591156in}}%
\pgfpathlineto{\pgfqpoint{5.478657in}{2.592688in}}%
\pgfpathlineto{\pgfqpoint{5.500419in}{2.593802in}}%
\pgfpathlineto{\pgfqpoint{5.522180in}{2.594515in}}%
\pgfpathlineto{\pgfqpoint{5.543942in}{2.595878in}}%
\pgfpathlineto{\pgfqpoint{5.565704in}{2.597741in}}%
\pgfpathlineto{\pgfqpoint{5.587465in}{2.598846in}}%
\pgfpathlineto{\pgfqpoint{5.609227in}{2.599555in}}%
\pgfpathlineto{\pgfqpoint{5.630989in}{2.601337in}}%
\pgfusepath{stroke}%
\end{pgfscope}%
\begin{pgfscope}%
\pgfpathrectangle{\pgfqpoint{3.454822in}{0.422992in}}{\pgfqpoint{2.219690in}{3.151201in}}%
\pgfusepath{clip}%
\pgfsetrectcap%
\pgfsetroundjoin%
\pgfsetlinewidth{1.003750pt}%
\definecolor{currentstroke}{rgb}{1.000000,0.584314,0.000000}%
\pgfsetstrokecolor{currentstroke}%
\pgfsetdash{}{0pt}%
\pgfpathmoveto{\pgfqpoint{3.476584in}{1.246587in}}%
\pgfpathlineto{\pgfqpoint{3.498345in}{1.127071in}}%
\pgfpathlineto{\pgfqpoint{3.520107in}{1.076657in}}%
\pgfpathlineto{\pgfqpoint{3.541869in}{1.063415in}}%
\pgfpathlineto{\pgfqpoint{3.563630in}{1.068645in}}%
\pgfpathlineto{\pgfqpoint{3.585392in}{1.086874in}}%
\pgfpathlineto{\pgfqpoint{3.607154in}{1.101163in}}%
\pgfpathlineto{\pgfqpoint{3.628915in}{1.120484in}}%
\pgfpathlineto{\pgfqpoint{3.650677in}{1.137005in}}%
\pgfpathlineto{\pgfqpoint{3.672439in}{1.154631in}}%
\pgfpathlineto{\pgfqpoint{3.694200in}{1.170275in}}%
\pgfpathlineto{\pgfqpoint{3.715962in}{1.184051in}}%
\pgfpathlineto{\pgfqpoint{3.737724in}{1.198375in}}%
\pgfpathlineto{\pgfqpoint{3.759485in}{1.214475in}}%
\pgfpathlineto{\pgfqpoint{3.781247in}{1.226242in}}%
\pgfpathlineto{\pgfqpoint{3.803009in}{1.240857in}}%
\pgfpathlineto{\pgfqpoint{3.824770in}{1.252249in}}%
\pgfpathlineto{\pgfqpoint{3.846532in}{1.263929in}}%
\pgfpathlineto{\pgfqpoint{3.868294in}{1.274975in}}%
\pgfpathlineto{\pgfqpoint{3.890055in}{1.284861in}}%
\pgfpathlineto{\pgfqpoint{3.911817in}{1.295215in}}%
\pgfpathlineto{\pgfqpoint{3.933579in}{1.307072in}}%
\pgfpathlineto{\pgfqpoint{3.955340in}{1.317301in}}%
\pgfpathlineto{\pgfqpoint{3.977102in}{1.326454in}}%
\pgfpathlineto{\pgfqpoint{3.998864in}{1.335617in}}%
\pgfpathlineto{\pgfqpoint{4.020625in}{1.344768in}}%
\pgfpathlineto{\pgfqpoint{4.042387in}{1.354307in}}%
\pgfpathlineto{\pgfqpoint{4.064149in}{1.361935in}}%
\pgfpathlineto{\pgfqpoint{4.085910in}{1.369936in}}%
\pgfpathlineto{\pgfqpoint{4.107672in}{1.376893in}}%
\pgfpathlineto{\pgfqpoint{4.129434in}{1.385527in}}%
\pgfpathlineto{\pgfqpoint{4.151195in}{1.393133in}}%
\pgfpathlineto{\pgfqpoint{4.172957in}{1.400115in}}%
\pgfpathlineto{\pgfqpoint{4.194719in}{1.407209in}}%
\pgfpathlineto{\pgfqpoint{4.216480in}{1.414005in}}%
\pgfpathlineto{\pgfqpoint{4.238242in}{1.420415in}}%
\pgfpathlineto{\pgfqpoint{4.260004in}{1.428020in}}%
\pgfpathlineto{\pgfqpoint{4.281765in}{1.435189in}}%
\pgfpathlineto{\pgfqpoint{4.303527in}{1.441536in}}%
\pgfpathlineto{\pgfqpoint{4.325289in}{1.447726in}}%
\pgfpathlineto{\pgfqpoint{4.347050in}{1.452946in}}%
\pgfpathlineto{\pgfqpoint{4.368812in}{1.458692in}}%
\pgfpathlineto{\pgfqpoint{4.390574in}{1.464333in}}%
\pgfpathlineto{\pgfqpoint{4.412335in}{1.469821in}}%
\pgfpathlineto{\pgfqpoint{4.434097in}{1.475460in}}%
\pgfpathlineto{\pgfqpoint{4.455859in}{1.481608in}}%
\pgfpathlineto{\pgfqpoint{4.477620in}{1.486510in}}%
\pgfpathlineto{\pgfqpoint{4.499382in}{1.492239in}}%
\pgfpathlineto{\pgfqpoint{4.521144in}{1.497454in}}%
\pgfpathlineto{\pgfqpoint{4.542905in}{1.501707in}}%
\pgfpathlineto{\pgfqpoint{4.564667in}{1.506348in}}%
\pgfpathlineto{\pgfqpoint{4.586429in}{1.511404in}}%
\pgfpathlineto{\pgfqpoint{4.608190in}{1.517097in}}%
\pgfpathlineto{\pgfqpoint{4.629952in}{1.522011in}}%
\pgfpathlineto{\pgfqpoint{4.651714in}{1.526257in}}%
\pgfpathlineto{\pgfqpoint{4.673475in}{1.530616in}}%
\pgfpathlineto{\pgfqpoint{4.695237in}{1.534714in}}%
\pgfpathlineto{\pgfqpoint{4.716999in}{1.539602in}}%
\pgfpathlineto{\pgfqpoint{4.738760in}{1.543927in}}%
\pgfpathlineto{\pgfqpoint{4.760522in}{1.547504in}}%
\pgfpathlineto{\pgfqpoint{4.782284in}{1.551523in}}%
\pgfpathlineto{\pgfqpoint{4.804045in}{1.554827in}}%
\pgfpathlineto{\pgfqpoint{4.825807in}{1.559900in}}%
\pgfpathlineto{\pgfqpoint{4.847569in}{1.564016in}}%
\pgfpathlineto{\pgfqpoint{4.869330in}{1.567890in}}%
\pgfpathlineto{\pgfqpoint{4.891092in}{1.572135in}}%
\pgfpathlineto{\pgfqpoint{4.912854in}{1.576542in}}%
\pgfpathlineto{\pgfqpoint{4.934615in}{1.580531in}}%
\pgfpathlineto{\pgfqpoint{4.956377in}{1.584759in}}%
\pgfpathlineto{\pgfqpoint{4.978139in}{1.588847in}}%
\pgfpathlineto{\pgfqpoint{4.999900in}{1.593115in}}%
\pgfpathlineto{\pgfqpoint{5.021662in}{1.596734in}}%
\pgfpathlineto{\pgfqpoint{5.043424in}{1.600641in}}%
\pgfpathlineto{\pgfqpoint{5.065185in}{1.604239in}}%
\pgfpathlineto{\pgfqpoint{5.086947in}{1.607838in}}%
\pgfpathlineto{\pgfqpoint{5.108709in}{1.611395in}}%
\pgfpathlineto{\pgfqpoint{5.130470in}{1.613980in}}%
\pgfpathlineto{\pgfqpoint{5.152232in}{1.617463in}}%
\pgfpathlineto{\pgfqpoint{5.173994in}{1.620716in}}%
\pgfpathlineto{\pgfqpoint{5.195755in}{1.623423in}}%
\pgfpathlineto{\pgfqpoint{5.217517in}{1.626160in}}%
\pgfpathlineto{\pgfqpoint{5.239279in}{1.629466in}}%
\pgfpathlineto{\pgfqpoint{5.261040in}{1.631789in}}%
\pgfpathlineto{\pgfqpoint{5.282802in}{1.634946in}}%
\pgfpathlineto{\pgfqpoint{5.304564in}{1.638330in}}%
\pgfpathlineto{\pgfqpoint{5.326325in}{1.641500in}}%
\pgfpathlineto{\pgfqpoint{5.348087in}{1.644211in}}%
\pgfpathlineto{\pgfqpoint{5.369849in}{1.647102in}}%
\pgfpathlineto{\pgfqpoint{5.391610in}{1.649865in}}%
\pgfpathlineto{\pgfqpoint{5.413372in}{1.652820in}}%
\pgfpathlineto{\pgfqpoint{5.435134in}{1.655157in}}%
\pgfpathlineto{\pgfqpoint{5.456895in}{1.657608in}}%
\pgfpathlineto{\pgfqpoint{5.478657in}{1.660245in}}%
\pgfpathlineto{\pgfqpoint{5.500419in}{1.663076in}}%
\pgfpathlineto{\pgfqpoint{5.522180in}{1.665552in}}%
\pgfpathlineto{\pgfqpoint{5.543942in}{1.667823in}}%
\pgfpathlineto{\pgfqpoint{5.565704in}{1.670615in}}%
\pgfpathlineto{\pgfqpoint{5.587465in}{1.672821in}}%
\pgfpathlineto{\pgfqpoint{5.609227in}{1.675162in}}%
\pgfpathlineto{\pgfqpoint{5.630989in}{1.677329in}}%
\pgfusepath{stroke}%
\end{pgfscope}%
\begin{pgfscope}%
\pgfpathrectangle{\pgfqpoint{3.454822in}{0.422992in}}{\pgfqpoint{2.219690in}{3.151201in}}%
\pgfusepath{clip}%
\pgfsetrectcap%
\pgfsetroundjoin%
\pgfsetlinewidth{1.003750pt}%
\definecolor{currentstroke}{rgb}{1.000000,0.172549,0.000000}%
\pgfsetstrokecolor{currentstroke}%
\pgfsetdash{}{0pt}%
\pgfpathmoveto{\pgfqpoint{3.476584in}{2.093906in}}%
\pgfpathlineto{\pgfqpoint{3.498345in}{2.256413in}}%
\pgfpathlineto{\pgfqpoint{3.520107in}{2.314741in}}%
\pgfpathlineto{\pgfqpoint{3.541869in}{2.360414in}}%
\pgfpathlineto{\pgfqpoint{3.563630in}{2.405584in}}%
\pgfpathlineto{\pgfqpoint{3.585392in}{2.439534in}}%
\pgfpathlineto{\pgfqpoint{3.607154in}{2.464306in}}%
\pgfpathlineto{\pgfqpoint{3.628915in}{2.489123in}}%
\pgfpathlineto{\pgfqpoint{3.650677in}{2.507470in}}%
\pgfpathlineto{\pgfqpoint{3.672439in}{2.524491in}}%
\pgfpathlineto{\pgfqpoint{3.694200in}{2.540471in}}%
\pgfpathlineto{\pgfqpoint{3.715962in}{2.555383in}}%
\pgfpathlineto{\pgfqpoint{3.737724in}{2.567438in}}%
\pgfpathlineto{\pgfqpoint{3.759485in}{2.580998in}}%
\pgfpathlineto{\pgfqpoint{3.781247in}{2.591430in}}%
\pgfpathlineto{\pgfqpoint{3.803009in}{2.599321in}}%
\pgfpathlineto{\pgfqpoint{3.824770in}{2.607512in}}%
\pgfpathlineto{\pgfqpoint{3.846532in}{2.615115in}}%
\pgfpathlineto{\pgfqpoint{3.868294in}{2.623932in}}%
\pgfpathlineto{\pgfqpoint{3.890055in}{2.632912in}}%
\pgfpathlineto{\pgfqpoint{3.911817in}{2.643351in}}%
\pgfpathlineto{\pgfqpoint{3.933579in}{2.650914in}}%
\pgfpathlineto{\pgfqpoint{3.955340in}{2.657157in}}%
\pgfpathlineto{\pgfqpoint{3.977102in}{2.664223in}}%
\pgfpathlineto{\pgfqpoint{3.998864in}{2.670397in}}%
\pgfpathlineto{\pgfqpoint{4.020625in}{2.675203in}}%
\pgfpathlineto{\pgfqpoint{4.042387in}{2.681971in}}%
\pgfpathlineto{\pgfqpoint{4.064149in}{2.688970in}}%
\pgfpathlineto{\pgfqpoint{4.085910in}{2.694514in}}%
\pgfpathlineto{\pgfqpoint{4.107672in}{2.700277in}}%
\pgfpathlineto{\pgfqpoint{4.129434in}{2.705869in}}%
\pgfpathlineto{\pgfqpoint{4.151195in}{2.710749in}}%
\pgfpathlineto{\pgfqpoint{4.172957in}{2.716911in}}%
\pgfpathlineto{\pgfqpoint{4.194719in}{2.721805in}}%
\pgfpathlineto{\pgfqpoint{4.216480in}{2.726475in}}%
\pgfpathlineto{\pgfqpoint{4.238242in}{2.732033in}}%
\pgfpathlineto{\pgfqpoint{4.260004in}{2.737116in}}%
\pgfpathlineto{\pgfqpoint{4.281765in}{2.741127in}}%
\pgfpathlineto{\pgfqpoint{4.303527in}{2.744889in}}%
\pgfpathlineto{\pgfqpoint{4.325289in}{2.748296in}}%
\pgfpathlineto{\pgfqpoint{4.347050in}{2.752817in}}%
\pgfpathlineto{\pgfqpoint{4.368812in}{2.757087in}}%
\pgfpathlineto{\pgfqpoint{4.390574in}{2.760798in}}%
\pgfpathlineto{\pgfqpoint{4.412335in}{2.764281in}}%
\pgfpathlineto{\pgfqpoint{4.434097in}{2.767825in}}%
\pgfpathlineto{\pgfqpoint{4.455859in}{2.771482in}}%
\pgfpathlineto{\pgfqpoint{4.477620in}{2.775294in}}%
\pgfpathlineto{\pgfqpoint{4.499382in}{2.779431in}}%
\pgfpathlineto{\pgfqpoint{4.521144in}{2.783566in}}%
\pgfpathlineto{\pgfqpoint{4.542905in}{2.786245in}}%
\pgfpathlineto{\pgfqpoint{4.564667in}{2.789430in}}%
\pgfpathlineto{\pgfqpoint{4.586429in}{2.792613in}}%
\pgfpathlineto{\pgfqpoint{4.608190in}{2.795059in}}%
\pgfpathlineto{\pgfqpoint{4.629952in}{2.797996in}}%
\pgfpathlineto{\pgfqpoint{4.651714in}{2.800975in}}%
\pgfpathlineto{\pgfqpoint{4.673475in}{2.804366in}}%
\pgfpathlineto{\pgfqpoint{4.695237in}{2.807646in}}%
\pgfpathlineto{\pgfqpoint{4.716999in}{2.810282in}}%
\pgfpathlineto{\pgfqpoint{4.738760in}{2.813474in}}%
\pgfpathlineto{\pgfqpoint{4.760522in}{2.815968in}}%
\pgfpathlineto{\pgfqpoint{4.782284in}{2.819050in}}%
\pgfpathlineto{\pgfqpoint{4.804045in}{2.821377in}}%
\pgfpathlineto{\pgfqpoint{4.825807in}{2.824723in}}%
\pgfpathlineto{\pgfqpoint{4.847569in}{2.827460in}}%
\pgfpathlineto{\pgfqpoint{4.869330in}{2.830153in}}%
\pgfpathlineto{\pgfqpoint{4.891092in}{2.833299in}}%
\pgfpathlineto{\pgfqpoint{4.912854in}{2.836443in}}%
\pgfpathlineto{\pgfqpoint{4.934615in}{2.838290in}}%
\pgfpathlineto{\pgfqpoint{4.956377in}{2.840963in}}%
\pgfpathlineto{\pgfqpoint{4.978139in}{2.843673in}}%
\pgfpathlineto{\pgfqpoint{4.999900in}{2.845931in}}%
\pgfpathlineto{\pgfqpoint{5.021662in}{2.848500in}}%
\pgfpathlineto{\pgfqpoint{5.043424in}{2.851234in}}%
\pgfpathlineto{\pgfqpoint{5.065185in}{2.853525in}}%
\pgfpathlineto{\pgfqpoint{5.086947in}{2.855565in}}%
\pgfpathlineto{\pgfqpoint{5.108709in}{2.857722in}}%
\pgfpathlineto{\pgfqpoint{5.130470in}{2.859798in}}%
\pgfpathlineto{\pgfqpoint{5.152232in}{2.862443in}}%
\pgfpathlineto{\pgfqpoint{5.173994in}{2.865120in}}%
\pgfpathlineto{\pgfqpoint{5.195755in}{2.867450in}}%
\pgfpathlineto{\pgfqpoint{5.217517in}{2.869614in}}%
\pgfpathlineto{\pgfqpoint{5.239279in}{2.871546in}}%
\pgfpathlineto{\pgfqpoint{5.261040in}{2.874093in}}%
\pgfpathlineto{\pgfqpoint{5.282802in}{2.876187in}}%
\pgfpathlineto{\pgfqpoint{5.304564in}{2.877898in}}%
\pgfpathlineto{\pgfqpoint{5.326325in}{2.880225in}}%
\pgfpathlineto{\pgfqpoint{5.348087in}{2.881937in}}%
\pgfpathlineto{\pgfqpoint{5.369849in}{2.883974in}}%
\pgfpathlineto{\pgfqpoint{5.391610in}{2.885279in}}%
\pgfpathlineto{\pgfqpoint{5.413372in}{2.886999in}}%
\pgfpathlineto{\pgfqpoint{5.435134in}{2.888795in}}%
\pgfpathlineto{\pgfqpoint{5.456895in}{2.890284in}}%
\pgfpathlineto{\pgfqpoint{5.478657in}{2.892096in}}%
\pgfpathlineto{\pgfqpoint{5.500419in}{2.893879in}}%
\pgfpathlineto{\pgfqpoint{5.522180in}{2.895389in}}%
\pgfpathlineto{\pgfqpoint{5.543942in}{2.896930in}}%
\pgfpathlineto{\pgfqpoint{5.565704in}{2.899110in}}%
\pgfpathlineto{\pgfqpoint{5.587465in}{2.901143in}}%
\pgfpathlineto{\pgfqpoint{5.609227in}{2.903067in}}%
\pgfpathlineto{\pgfqpoint{5.630989in}{2.905121in}}%
\pgfusepath{stroke}%
\end{pgfscope}%
\begin{pgfscope}%
\pgfpathrectangle{\pgfqpoint{3.454822in}{0.422992in}}{\pgfqpoint{2.219690in}{3.151201in}}%
\pgfusepath{clip}%
\pgfsetrectcap%
\pgfsetroundjoin%
\pgfsetlinewidth{1.003750pt}%
\definecolor{currentstroke}{rgb}{0.517647,0.356863,0.592157}%
\pgfsetstrokecolor{currentstroke}%
\pgfsetdash{}{0pt}%
\pgfpathmoveto{\pgfqpoint{3.476584in}{0.698773in}}%
\pgfpathlineto{\pgfqpoint{3.498345in}{0.643062in}}%
\pgfpathlineto{\pgfqpoint{3.520107in}{0.591582in}}%
\pgfpathlineto{\pgfqpoint{3.541869in}{0.568799in}}%
\pgfpathlineto{\pgfqpoint{3.563630in}{0.566229in}}%
\pgfpathlineto{\pgfqpoint{3.585392in}{0.569175in}}%
\pgfpathlineto{\pgfqpoint{3.607154in}{0.577150in}}%
\pgfpathlineto{\pgfqpoint{3.628915in}{0.583964in}}%
\pgfpathlineto{\pgfqpoint{3.650677in}{0.591271in}}%
\pgfpathlineto{\pgfqpoint{3.672439in}{0.597848in}}%
\pgfpathlineto{\pgfqpoint{3.694200in}{0.607180in}}%
\pgfpathlineto{\pgfqpoint{3.715962in}{0.615726in}}%
\pgfpathlineto{\pgfqpoint{3.737724in}{0.623670in}}%
\pgfpathlineto{\pgfqpoint{3.759485in}{0.631447in}}%
\pgfpathlineto{\pgfqpoint{3.781247in}{0.640094in}}%
\pgfpathlineto{\pgfqpoint{3.803009in}{0.648453in}}%
\pgfpathlineto{\pgfqpoint{3.824770in}{0.659106in}}%
\pgfpathlineto{\pgfqpoint{3.846532in}{0.668695in}}%
\pgfpathlineto{\pgfqpoint{3.868294in}{0.676651in}}%
\pgfpathlineto{\pgfqpoint{3.890055in}{0.684286in}}%
\pgfpathlineto{\pgfqpoint{3.911817in}{0.692729in}}%
\pgfpathlineto{\pgfqpoint{3.933579in}{0.700229in}}%
\pgfpathlineto{\pgfqpoint{3.955340in}{0.708377in}}%
\pgfpathlineto{\pgfqpoint{3.977102in}{0.715846in}}%
\pgfpathlineto{\pgfqpoint{3.998864in}{0.722700in}}%
\pgfpathlineto{\pgfqpoint{4.020625in}{0.730400in}}%
\pgfpathlineto{\pgfqpoint{4.042387in}{0.737315in}}%
\pgfpathlineto{\pgfqpoint{4.064149in}{0.744143in}}%
\pgfpathlineto{\pgfqpoint{4.085910in}{0.750782in}}%
\pgfpathlineto{\pgfqpoint{4.107672in}{0.757824in}}%
\pgfpathlineto{\pgfqpoint{4.129434in}{0.764142in}}%
\pgfpathlineto{\pgfqpoint{4.151195in}{0.770824in}}%
\pgfpathlineto{\pgfqpoint{4.172957in}{0.776404in}}%
\pgfpathlineto{\pgfqpoint{4.194719in}{0.783123in}}%
\pgfpathlineto{\pgfqpoint{4.216480in}{0.788684in}}%
\pgfpathlineto{\pgfqpoint{4.238242in}{0.794790in}}%
\pgfpathlineto{\pgfqpoint{4.260004in}{0.800282in}}%
\pgfpathlineto{\pgfqpoint{4.281765in}{0.806101in}}%
\pgfpathlineto{\pgfqpoint{4.303527in}{0.811617in}}%
\pgfpathlineto{\pgfqpoint{4.325289in}{0.817927in}}%
\pgfpathlineto{\pgfqpoint{4.347050in}{0.823740in}}%
\pgfpathlineto{\pgfqpoint{4.368812in}{0.828846in}}%
\pgfpathlineto{\pgfqpoint{4.390574in}{0.834960in}}%
\pgfpathlineto{\pgfqpoint{4.412335in}{0.839188in}}%
\pgfpathlineto{\pgfqpoint{4.434097in}{0.844246in}}%
\pgfpathlineto{\pgfqpoint{4.455859in}{0.850352in}}%
\pgfpathlineto{\pgfqpoint{4.477620in}{0.855543in}}%
\pgfpathlineto{\pgfqpoint{4.499382in}{0.861159in}}%
\pgfpathlineto{\pgfqpoint{4.521144in}{0.866558in}}%
\pgfpathlineto{\pgfqpoint{4.542905in}{0.872280in}}%
\pgfpathlineto{\pgfqpoint{4.564667in}{0.877233in}}%
\pgfpathlineto{\pgfqpoint{4.586429in}{0.881880in}}%
\pgfpathlineto{\pgfqpoint{4.608190in}{0.886424in}}%
\pgfpathlineto{\pgfqpoint{4.629952in}{0.891656in}}%
\pgfpathlineto{\pgfqpoint{4.651714in}{0.897833in}}%
\pgfpathlineto{\pgfqpoint{4.673475in}{0.903020in}}%
\pgfpathlineto{\pgfqpoint{4.695237in}{0.908075in}}%
\pgfpathlineto{\pgfqpoint{4.716999in}{0.912885in}}%
\pgfpathlineto{\pgfqpoint{4.738760in}{0.916937in}}%
\pgfpathlineto{\pgfqpoint{4.760522in}{0.921855in}}%
\pgfpathlineto{\pgfqpoint{4.782284in}{0.926069in}}%
\pgfpathlineto{\pgfqpoint{4.804045in}{0.931426in}}%
\pgfpathlineto{\pgfqpoint{4.825807in}{0.936057in}}%
\pgfpathlineto{\pgfqpoint{4.847569in}{0.940419in}}%
\pgfpathlineto{\pgfqpoint{4.869330in}{0.944492in}}%
\pgfpathlineto{\pgfqpoint{4.891092in}{0.948538in}}%
\pgfpathlineto{\pgfqpoint{4.912854in}{0.952840in}}%
\pgfpathlineto{\pgfqpoint{4.934615in}{0.957846in}}%
\pgfpathlineto{\pgfqpoint{4.956377in}{0.961605in}}%
\pgfpathlineto{\pgfqpoint{4.978139in}{0.966030in}}%
\pgfpathlineto{\pgfqpoint{4.999900in}{0.970461in}}%
\pgfpathlineto{\pgfqpoint{5.021662in}{0.974912in}}%
\pgfpathlineto{\pgfqpoint{5.043424in}{0.979070in}}%
\pgfpathlineto{\pgfqpoint{5.065185in}{0.982630in}}%
\pgfpathlineto{\pgfqpoint{5.086947in}{0.986543in}}%
\pgfpathlineto{\pgfqpoint{5.108709in}{0.989869in}}%
\pgfpathlineto{\pgfqpoint{5.130470in}{0.994271in}}%
\pgfpathlineto{\pgfqpoint{5.152232in}{0.998259in}}%
\pgfpathlineto{\pgfqpoint{5.173994in}{1.001615in}}%
\pgfpathlineto{\pgfqpoint{5.195755in}{1.005651in}}%
\pgfpathlineto{\pgfqpoint{5.217517in}{1.008828in}}%
\pgfpathlineto{\pgfqpoint{5.239279in}{1.012712in}}%
\pgfpathlineto{\pgfqpoint{5.261040in}{1.016264in}}%
\pgfpathlineto{\pgfqpoint{5.282802in}{1.019971in}}%
\pgfpathlineto{\pgfqpoint{5.304564in}{1.023754in}}%
\pgfpathlineto{\pgfqpoint{5.326325in}{1.027491in}}%
\pgfpathlineto{\pgfqpoint{5.348087in}{1.030672in}}%
\pgfpathlineto{\pgfqpoint{5.369849in}{1.034285in}}%
\pgfpathlineto{\pgfqpoint{5.391610in}{1.038027in}}%
\pgfpathlineto{\pgfqpoint{5.413372in}{1.041293in}}%
\pgfpathlineto{\pgfqpoint{5.435134in}{1.044949in}}%
\pgfpathlineto{\pgfqpoint{5.456895in}{1.048184in}}%
\pgfpathlineto{\pgfqpoint{5.478657in}{1.051578in}}%
\pgfpathlineto{\pgfqpoint{5.500419in}{1.054619in}}%
\pgfpathlineto{\pgfqpoint{5.522180in}{1.058064in}}%
\pgfpathlineto{\pgfqpoint{5.543942in}{1.061783in}}%
\pgfpathlineto{\pgfqpoint{5.565704in}{1.065248in}}%
\pgfpathlineto{\pgfqpoint{5.587465in}{1.068312in}}%
\pgfpathlineto{\pgfqpoint{5.609227in}{1.071352in}}%
\pgfpathlineto{\pgfqpoint{5.630989in}{1.074847in}}%
\pgfusepath{stroke}%
\end{pgfscope}%
\begin{pgfscope}%
\pgfpathrectangle{\pgfqpoint{3.454822in}{0.422992in}}{\pgfqpoint{2.219690in}{3.151201in}}%
\pgfusepath{clip}%
\pgfsetrectcap%
\pgfsetroundjoin%
\pgfsetlinewidth{1.003750pt}%
\definecolor{currentstroke}{rgb}{0.278431,0.278431,0.278431}%
\pgfsetstrokecolor{currentstroke}%
\pgfsetdash{}{0pt}%
\pgfpathmoveto{\pgfqpoint{3.476584in}{2.492054in}}%
\pgfpathlineto{\pgfqpoint{3.498345in}{2.734256in}}%
\pgfpathlineto{\pgfqpoint{3.520107in}{2.851341in}}%
\pgfpathlineto{\pgfqpoint{3.541869in}{2.928598in}}%
\pgfpathlineto{\pgfqpoint{3.563630in}{2.984158in}}%
\pgfpathlineto{\pgfqpoint{3.585392in}{3.033244in}}%
\pgfpathlineto{\pgfqpoint{3.607154in}{3.076540in}}%
\pgfpathlineto{\pgfqpoint{3.628915in}{3.103957in}}%
\pgfpathlineto{\pgfqpoint{3.650677in}{3.125473in}}%
\pgfpathlineto{\pgfqpoint{3.672439in}{3.146321in}}%
\pgfpathlineto{\pgfqpoint{3.694200in}{3.162107in}}%
\pgfpathlineto{\pgfqpoint{3.715962in}{3.177539in}}%
\pgfpathlineto{\pgfqpoint{3.737724in}{3.189124in}}%
\pgfpathlineto{\pgfqpoint{3.759485in}{3.200959in}}%
\pgfpathlineto{\pgfqpoint{3.781247in}{3.213181in}}%
\pgfpathlineto{\pgfqpoint{3.803009in}{3.222437in}}%
\pgfpathlineto{\pgfqpoint{3.824770in}{3.229427in}}%
\pgfpathlineto{\pgfqpoint{3.846532in}{3.236464in}}%
\pgfpathlineto{\pgfqpoint{3.868294in}{3.244493in}}%
\pgfpathlineto{\pgfqpoint{3.890055in}{3.252515in}}%
\pgfpathlineto{\pgfqpoint{3.911817in}{3.258411in}}%
\pgfpathlineto{\pgfqpoint{3.933579in}{3.264465in}}%
\pgfpathlineto{\pgfqpoint{3.955340in}{3.271844in}}%
\pgfpathlineto{\pgfqpoint{3.977102in}{3.280168in}}%
\pgfpathlineto{\pgfqpoint{3.998864in}{3.286277in}}%
\pgfpathlineto{\pgfqpoint{4.020625in}{3.292024in}}%
\pgfpathlineto{\pgfqpoint{4.042387in}{3.298516in}}%
\pgfpathlineto{\pgfqpoint{4.064149in}{3.302194in}}%
\pgfpathlineto{\pgfqpoint{4.085910in}{3.307168in}}%
\pgfpathlineto{\pgfqpoint{4.107672in}{3.311632in}}%
\pgfpathlineto{\pgfqpoint{4.129434in}{3.315759in}}%
\pgfpathlineto{\pgfqpoint{4.151195in}{3.320616in}}%
\pgfpathlineto{\pgfqpoint{4.172957in}{3.324996in}}%
\pgfpathlineto{\pgfqpoint{4.194719in}{3.328119in}}%
\pgfpathlineto{\pgfqpoint{4.216480in}{3.332631in}}%
\pgfpathlineto{\pgfqpoint{4.238242in}{3.336491in}}%
\pgfpathlineto{\pgfqpoint{4.260004in}{3.339754in}}%
\pgfpathlineto{\pgfqpoint{4.281765in}{3.341922in}}%
\pgfpathlineto{\pgfqpoint{4.303527in}{3.346058in}}%
\pgfpathlineto{\pgfqpoint{4.325289in}{3.349762in}}%
\pgfpathlineto{\pgfqpoint{4.347050in}{3.352172in}}%
\pgfpathlineto{\pgfqpoint{4.368812in}{3.354248in}}%
\pgfpathlineto{\pgfqpoint{4.390574in}{3.356407in}}%
\pgfpathlineto{\pgfqpoint{4.412335in}{3.359236in}}%
\pgfpathlineto{\pgfqpoint{4.434097in}{3.361408in}}%
\pgfpathlineto{\pgfqpoint{4.455859in}{3.363374in}}%
\pgfpathlineto{\pgfqpoint{4.477620in}{3.365008in}}%
\pgfpathlineto{\pgfqpoint{4.499382in}{3.366826in}}%
\pgfpathlineto{\pgfqpoint{4.521144in}{3.369592in}}%
\pgfpathlineto{\pgfqpoint{4.542905in}{3.372807in}}%
\pgfpathlineto{\pgfqpoint{4.564667in}{3.374968in}}%
\pgfpathlineto{\pgfqpoint{4.586429in}{3.377476in}}%
\pgfpathlineto{\pgfqpoint{4.608190in}{3.379058in}}%
\pgfpathlineto{\pgfqpoint{4.629952in}{3.381043in}}%
\pgfpathlineto{\pgfqpoint{4.651714in}{3.382474in}}%
\pgfpathlineto{\pgfqpoint{4.673475in}{3.383467in}}%
\pgfpathlineto{\pgfqpoint{4.695237in}{3.385432in}}%
\pgfpathlineto{\pgfqpoint{4.716999in}{3.386736in}}%
\pgfpathlineto{\pgfqpoint{4.738760in}{3.388284in}}%
\pgfpathlineto{\pgfqpoint{4.760522in}{3.388573in}}%
\pgfpathlineto{\pgfqpoint{4.782284in}{3.389881in}}%
\pgfpathlineto{\pgfqpoint{4.804045in}{3.391230in}}%
\pgfpathlineto{\pgfqpoint{4.825807in}{3.392281in}}%
\pgfpathlineto{\pgfqpoint{4.847569in}{3.393524in}}%
\pgfpathlineto{\pgfqpoint{4.869330in}{3.395225in}}%
\pgfpathlineto{\pgfqpoint{4.891092in}{3.396774in}}%
\pgfpathlineto{\pgfqpoint{4.912854in}{3.398090in}}%
\pgfpathlineto{\pgfqpoint{4.934615in}{3.400069in}}%
\pgfpathlineto{\pgfqpoint{4.956377in}{3.400879in}}%
\pgfpathlineto{\pgfqpoint{4.978139in}{3.402350in}}%
\pgfpathlineto{\pgfqpoint{4.999900in}{3.404477in}}%
\pgfpathlineto{\pgfqpoint{5.021662in}{3.405475in}}%
\pgfpathlineto{\pgfqpoint{5.043424in}{3.406858in}}%
\pgfpathlineto{\pgfqpoint{5.065185in}{3.407478in}}%
\pgfpathlineto{\pgfqpoint{5.086947in}{3.408688in}}%
\pgfpathlineto{\pgfqpoint{5.108709in}{3.410351in}}%
\pgfpathlineto{\pgfqpoint{5.130470in}{3.411485in}}%
\pgfpathlineto{\pgfqpoint{5.152232in}{3.412421in}}%
\pgfpathlineto{\pgfqpoint{5.173994in}{3.412849in}}%
\pgfpathlineto{\pgfqpoint{5.195755in}{3.413420in}}%
\pgfpathlineto{\pgfqpoint{5.217517in}{3.414290in}}%
\pgfpathlineto{\pgfqpoint{5.239279in}{3.415681in}}%
\pgfpathlineto{\pgfqpoint{5.261040in}{3.416645in}}%
\pgfpathlineto{\pgfqpoint{5.282802in}{3.417420in}}%
\pgfpathlineto{\pgfqpoint{5.304564in}{3.418172in}}%
\pgfpathlineto{\pgfqpoint{5.326325in}{3.419253in}}%
\pgfpathlineto{\pgfqpoint{5.348087in}{3.419830in}}%
\pgfpathlineto{\pgfqpoint{5.369849in}{3.420666in}}%
\pgfpathlineto{\pgfqpoint{5.391610in}{3.421161in}}%
\pgfpathlineto{\pgfqpoint{5.413372in}{3.422403in}}%
\pgfpathlineto{\pgfqpoint{5.435134in}{3.423220in}}%
\pgfpathlineto{\pgfqpoint{5.456895in}{3.423741in}}%
\pgfpathlineto{\pgfqpoint{5.478657in}{3.424411in}}%
\pgfpathlineto{\pgfqpoint{5.500419in}{3.425469in}}%
\pgfpathlineto{\pgfqpoint{5.522180in}{3.426394in}}%
\pgfpathlineto{\pgfqpoint{5.543942in}{3.427053in}}%
\pgfpathlineto{\pgfqpoint{5.565704in}{3.427983in}}%
\pgfpathlineto{\pgfqpoint{5.587465in}{3.428887in}}%
\pgfpathlineto{\pgfqpoint{5.609227in}{3.429833in}}%
\pgfpathlineto{\pgfqpoint{5.630989in}{3.430957in}}%
\pgfusepath{stroke}%
\end{pgfscope}%
\begin{pgfscope}%
\pgfsetrectcap%
\pgfsetmiterjoin%
\pgfsetlinewidth{0.501875pt}%
\definecolor{currentstroke}{rgb}{0.000000,0.000000,0.000000}%
\pgfsetstrokecolor{currentstroke}%
\pgfsetdash{}{0pt}%
\pgfpathmoveto{\pgfqpoint{3.454822in}{0.422992in}}%
\pgfpathlineto{\pgfqpoint{3.454822in}{3.574193in}}%
\pgfusepath{stroke}%
\end{pgfscope}%
\begin{pgfscope}%
\pgfsetrectcap%
\pgfsetmiterjoin%
\pgfsetlinewidth{0.501875pt}%
\definecolor{currentstroke}{rgb}{0.000000,0.000000,0.000000}%
\pgfsetstrokecolor{currentstroke}%
\pgfsetdash{}{0pt}%
\pgfpathmoveto{\pgfqpoint{5.674512in}{0.422992in}}%
\pgfpathlineto{\pgfqpoint{5.674512in}{3.574193in}}%
\pgfusepath{stroke}%
\end{pgfscope}%
\begin{pgfscope}%
\pgfsetrectcap%
\pgfsetmiterjoin%
\pgfsetlinewidth{0.501875pt}%
\definecolor{currentstroke}{rgb}{0.000000,0.000000,0.000000}%
\pgfsetstrokecolor{currentstroke}%
\pgfsetdash{}{0pt}%
\pgfpathmoveto{\pgfqpoint{3.454822in}{0.422992in}}%
\pgfpathlineto{\pgfqpoint{5.674512in}{0.422992in}}%
\pgfusepath{stroke}%
\end{pgfscope}%
\begin{pgfscope}%
\pgfsetrectcap%
\pgfsetmiterjoin%
\pgfsetlinewidth{0.501875pt}%
\definecolor{currentstroke}{rgb}{0.000000,0.000000,0.000000}%
\pgfsetstrokecolor{currentstroke}%
\pgfsetdash{}{0pt}%
\pgfpathmoveto{\pgfqpoint{3.454822in}{3.574193in}}%
\pgfpathlineto{\pgfqpoint{5.674512in}{3.574193in}}%
\pgfusepath{stroke}%
\end{pgfscope}%
\begin{pgfscope}%
\definecolor{textcolor}{rgb}{0.000000,0.000000,0.000000}%
\pgfsetstrokecolor{textcolor}%
\pgfsetfillcolor{textcolor}%
\pgftext[x=4.564667in,y=3.657526in,,base]{\color{textcolor}\rmfamily\fontsize{12.000000}{14.400000}\selectfont LCMC}%
\end{pgfscope}%
\begin{pgfscope}%
\pgfsetrectcap%
\pgfsetroundjoin%
\pgfsetlinewidth{1.003750pt}%
\definecolor{currentstroke}{rgb}{0.047059,0.364706,0.647059}%
\pgfsetstrokecolor{currentstroke}%
\pgfsetdash{}{0pt}%
\pgfpathmoveto{\pgfqpoint{3.579822in}{1.644779in}}%
\pgfpathlineto{\pgfqpoint{3.718711in}{1.644779in}}%
\pgfpathlineto{\pgfqpoint{3.857600in}{1.644779in}}%
\pgfusepath{stroke}%
\end{pgfscope}%
\begin{pgfscope}%
\definecolor{textcolor}{rgb}{0.000000,0.000000,0.000000}%
\pgfsetstrokecolor{textcolor}%
\pgfsetfillcolor{textcolor}%
\pgftext[x=3.968711in,y=1.596168in,left,base]{\color{textcolor}\rmfamily\fontsize{10.000000}{12.000000}\selectfont PCA}%
\end{pgfscope}%
\begin{pgfscope}%
\pgfsetrectcap%
\pgfsetroundjoin%
\pgfsetlinewidth{1.003750pt}%
\definecolor{currentstroke}{rgb}{0.000000,0.725490,0.270588}%
\pgfsetstrokecolor{currentstroke}%
\pgfsetdash{}{0pt}%
\pgfpathmoveto{\pgfqpoint{3.579822in}{1.440922in}}%
\pgfpathlineto{\pgfqpoint{3.718711in}{1.440922in}}%
\pgfpathlineto{\pgfqpoint{3.857600in}{1.440922in}}%
\pgfusepath{stroke}%
\end{pgfscope}%
\begin{pgfscope}%
\definecolor{textcolor}{rgb}{0.000000,0.000000,0.000000}%
\pgfsetstrokecolor{textcolor}%
\pgfsetfillcolor{textcolor}%
\pgftext[x=3.968711in,y=1.392311in,left,base]{\color{textcolor}\rmfamily\fontsize{10.000000}{12.000000}\selectfont KernelPCA}%
\end{pgfscope}%
\begin{pgfscope}%
\pgfsetrectcap%
\pgfsetroundjoin%
\pgfsetlinewidth{1.003750pt}%
\definecolor{currentstroke}{rgb}{1.000000,0.584314,0.000000}%
\pgfsetstrokecolor{currentstroke}%
\pgfsetdash{}{0pt}%
\pgfpathmoveto{\pgfqpoint{3.579822in}{1.237065in}}%
\pgfpathlineto{\pgfqpoint{3.718711in}{1.237065in}}%
\pgfpathlineto{\pgfqpoint{3.857600in}{1.237065in}}%
\pgfusepath{stroke}%
\end{pgfscope}%
\begin{pgfscope}%
\definecolor{textcolor}{rgb}{0.000000,0.000000,0.000000}%
\pgfsetstrokecolor{textcolor}%
\pgfsetfillcolor{textcolor}%
\pgftext[x=3.968711in,y=1.188454in,left,base]{\color{textcolor}\rmfamily\fontsize{10.000000}{12.000000}\selectfont LLE}%
\end{pgfscope}%
\begin{pgfscope}%
\pgfsetrectcap%
\pgfsetroundjoin%
\pgfsetlinewidth{1.003750pt}%
\definecolor{currentstroke}{rgb}{1.000000,0.172549,0.000000}%
\pgfsetstrokecolor{currentstroke}%
\pgfsetdash{}{0pt}%
\pgfpathmoveto{\pgfqpoint{3.579822in}{1.033208in}}%
\pgfpathlineto{\pgfqpoint{3.718711in}{1.033208in}}%
\pgfpathlineto{\pgfqpoint{3.857600in}{1.033208in}}%
\pgfusepath{stroke}%
\end{pgfscope}%
\begin{pgfscope}%
\definecolor{textcolor}{rgb}{0.000000,0.000000,0.000000}%
\pgfsetstrokecolor{textcolor}%
\pgfsetfillcolor{textcolor}%
\pgftext[x=3.968711in,y=0.984596in,left,base]{\color{textcolor}\rmfamily\fontsize{10.000000}{12.000000}\selectfont AE}%
\end{pgfscope}%
\begin{pgfscope}%
\pgfsetrectcap%
\pgfsetroundjoin%
\pgfsetlinewidth{1.003750pt}%
\definecolor{currentstroke}{rgb}{0.517647,0.356863,0.592157}%
\pgfsetstrokecolor{currentstroke}%
\pgfsetdash{}{0pt}%
\pgfpathmoveto{\pgfqpoint{3.579822in}{0.829350in}}%
\pgfpathlineto{\pgfqpoint{3.718711in}{0.829350in}}%
\pgfpathlineto{\pgfqpoint{3.857600in}{0.829350in}}%
\pgfusepath{stroke}%
\end{pgfscope}%
\begin{pgfscope}%
\definecolor{textcolor}{rgb}{0.000000,0.000000,0.000000}%
\pgfsetstrokecolor{textcolor}%
\pgfsetfillcolor{textcolor}%
\pgftext[x=3.968711in,y=0.780739in,left,base]{\color{textcolor}\rmfamily\fontsize{10.000000}{12.000000}\selectfont CAE}%
\end{pgfscope}%
\begin{pgfscope}%
\pgfsetrectcap%
\pgfsetroundjoin%
\pgfsetlinewidth{1.003750pt}%
\definecolor{currentstroke}{rgb}{0.278431,0.278431,0.278431}%
\pgfsetstrokecolor{currentstroke}%
\pgfsetdash{}{0pt}%
\pgfpathmoveto{\pgfqpoint{3.579822in}{0.625493in}}%
\pgfpathlineto{\pgfqpoint{3.718711in}{0.625493in}}%
\pgfpathlineto{\pgfqpoint{3.857600in}{0.625493in}}%
\pgfusepath{stroke}%
\end{pgfscope}%
\begin{pgfscope}%
\definecolor{textcolor}{rgb}{0.000000,0.000000,0.000000}%
\pgfsetstrokecolor{textcolor}%
\pgfsetfillcolor{textcolor}%
\pgftext[x=3.968711in,y=0.576882in,left,base]{\color{textcolor}\rmfamily\fontsize{10.000000}{12.000000}\selectfont ConvAE}%
\end{pgfscope}%
\end{pgfpicture}%
\makeatother%
\endgroup%

	\end{center}
	\caption[Qualitätskriterien für MNIST]{Die Vertrauenswürdigkeit und Kontinuität der Dimensionsreduktion, sowie das Local Continuity Meta-Criterion (LCMC) für den MNIST-Datensatz. Auf diesem Datensatz kann der der domänenspezifische Convolutional Autoencoder (ConvAE) seine starke Performance zeigen, da der Datensatz von hoher Qualität ist und eine hohe Stichprobengröße aufweisen kann. Nichtsdestotrotz kann die Hauptkomponentenanalyse nur knapp dahinter sehr gut mithalten und hinsichtlich der Kontinuität sogar übertreffen. Die Kernel PCA und der klassische vollvernetzte Autoencoder schneiden ebenfalls relativ gut ab, jedoch ist die Reduktion der Kernel PCA weniger vertrauenswürdig. Locally Linear Embedding und der Contractive Autoencoder können vor allem auf dem LCMC nicht mithalten. (Eigene Darstellung)}
	\label{fig:MNISTMetrics}
\end{figure}
Für die Olivetti Faces (\figref{fig:OlivettiFacesMetrics}), den FER- (\figref{fig:FER2013Metrics}) und den LFW-Datensatz (\figref{fig:LfwPeopleMetrics})
schneidet LLE auf allen drei Kriterien relativ gesehen sehr schlecht ab. insbesondere nehmen die
Qualitätskriterien von LLE mit größer werdender Nachbarschaftsgröße $K$ deutlicher ab, als bei den
restlichen Methoden. Aufgrund der Wahl der Nachbarschaftsgröße von $K=10$ für Locally Linear
Embedding und der lokalen Eigenschaften macht dies jedoch Sinn. Aus diesem Grund ist die
Betrachtung einer größeren Nachbarschaft der Qualitätskriterien für LLE benachteiligend, wurde aber
der Vollständigkeit halber und zu Vergleichszwecken trotzdem mit aufgenommen. Überraschenderweise
kann die Hauptkomponentenanalyse auf fast allen natürlichen Datensätzen eine sehr gute oder sogar die
beste relative Performance aufweisen. Auf den vier Bilddatensätzen kann jedoch der Convolutional
Autoencoder seine Stärken zeigen, insbesondere wenn genügend Trainingsdaten zur Verfügung stehen, was für den MNIST-Datensatz in \figref{fig:MNISTMetrics} zutrifft.
Ist dies jedoch nicht der Fall, so kann die Performance stark darunter leiden. Dies ist
beispielsweise auf dem Olivetti Faces Datensatz der Fall, wie in \figref{fig:OlivettiFacesMetrics}
zu erkennen ist. Das Problem einer geringen Menge an Trainingsdaten gilt gleichermaßen für alle
Varianten von neuronalen Netzen, insbesondere jedoch für vollvernetzte Autoencoder aufgrund der
hohen Anzahl an Freiheitsgraden im Netz.

Interessant ist außerdem, wie gut die Dimensionsreduktionsmethoden ein originales Bild wieder
rekonstruieren können. In diesem Vergleich können allerdings nur die Hauptkomponentenanalyse und
alle Varianten der Autoencoder für die Rekonstruktion verwendet werden. Kernel PCA und Locally
Linear Embedding erlauben ohne Approximation keine inverse Transformation aus der latenten
Repräsentation $\vect{y}$ zurück in eine approximierte ursprüngliche Repräsentation
$\estNormal{\vect{x}}$. Für die Hauptkomponentenanalyse, den klassischen vollvernetzten
Autoencoder, den Convolutional Autoencoder und für den Contractive Autoencoder wurde auf dem
MNIST-Datensatz je ein Beispielbild für eine der Zahlen rekonstruiert und in
\figref{fig:MNIST-reconstructions} abgebildet. Die erste Zeile besteht aus den Originalbildern als
Referenz.
\begin{figure}[ht]
	\centering
	\includegraphics{reconstructions_mnist_pca_ae_convae_cae.pdf}
	\caption[Rekonstruktierte MNIST-Zahlen]{Gezeigt sind einige rekonstruierten Zahlen aus dem MNIST-Datensatz von mehreren Methoden. Jede Zeile gehört zu einer Methode. Von oben nach unten: (1) Originale Bilder als Referenz, (2) Rekonstruktion durch PCA, (3) Rekonstruktion durch einen vollvernetzten Autoencoder, (4) Rekonstruktion durch den Convolutional Autoencoder und (5) Rekonstruktion durch den Contractive Autoencoder.}
	\label{fig:MNIST-reconstructions}
\end{figure}
Visuell ist die beste Rekonstruktion die des Convolutional Autoencoders, gefolgt vom vollvernetzten Autoencoder. Die Rekonstrukionen der Hauptkomponentenanalyse und des Contractive Autoencoders sind visuell nicht überzeugend, da sie deutlich unscharfer sind. Der Convolutional Autoencoder hat auf dem MNIST-Datensatz hinsichtlich der Qualitätskriterien ebenfalls am besten abgeschnitten (siehe \figref{fig:MNISTMetrics}). Der hier eingesetzte tiefe Convolutional Autoencoder mit mehr als 10 Schichten hat über zwölf Millionen trainierbare Gewichte. Allerdings ist MNIST ein relativ simpler Datensatz von hoher Qualität, so dass dieser tiefe Autoencoder scheinbar eine sinnvolle Repräsentation lernen kann. Beispielsweise ist der FER- oder LFW-Datensatz deutlich komplexer als MNIST mit einer geringeren Anzahl an Trainingsdaten. Auf diesen Datensätzen ist die Rekonstruktion des Convolutional Autoencoders nicht mehr so gut wie auf dem MNIST-Datensatz.

\subsection{PCA und Autoencoder}
\label{ch:Vergleich:sec:Resultate:PCA_AE}

In diesem Abschnitt wird die Hauptkomponentenanalyse mit Autoencodern verglichen. Dazu wird zuerst
der theoretische Zusammenhang zwischen den beiden Methoden erklärt. Im Anschluss wird die
Hauptkomponentenanalyse und unterschiedliche Modelle des Autoencoders für einen empirischen
Vergleich auf dem FER-Datensatz trainiert, wobei aufgrund der sieben unterschiedlichen Emotionen im
Datensatz eine Reduktion auf sieben Dimensionen durchgeführt wurde. Ein Beispielbild für jede
Emotion des FER-Datensatzes ist in \figref{fig:FER-Datensatz-Beispiele} dargestellt.\rewrite{FER
	hat 7 Emotionen. wütend fehlt hier}

\begin{figure}[ht]
	\centering
	%% Creator: Matplotlib, PGF backend
%%
%% To include the figure in your LaTeX document, write
%%   \input{<filename>.pgf}
%%
%% Make sure the required packages are loaded in your preamble
%%   \usepackage{pgf}
%%
%% Also ensure that all the required font packages are loaded; for instance,
%% the lmodern package is sometimes necessary when using math font.
%%   \usepackage{lmodern}
%%
%% Figures using additional raster images can only be included by \input if
%% they are in the same directory as the main LaTeX file. For loading figures
%% from other directories you can use the `import` package
%%   \usepackage{import}
%%
%% and then include the figures with
%%   \import{<path to file>}{<filename>.pgf}
%%
%% Matplotlib used the following preamble
%%   
%%   \usepackage{fontspec}
%%   \setmainfont{DejaVuSerif.ttf}[Path=\detokenize{/Users/moritzmistol/.pyenv/versions/3.9.13/envs/thesis/lib/python3.9/site-packages/matplotlib/mpl-data/fonts/ttf/}]
%%   \setsansfont{DejaVuSans.ttf}[Path=\detokenize{/Users/moritzmistol/.pyenv/versions/3.9.13/envs/thesis/lib/python3.9/site-packages/matplotlib/mpl-data/fonts/ttf/}]
%%   \setmonofont{DejaVuSansMono.ttf}[Path=\detokenize{/Users/moritzmistol/.pyenv/versions/3.9.13/envs/thesis/lib/python3.9/site-packages/matplotlib/mpl-data/fonts/ttf/}]
%%   \makeatletter\@ifpackageloaded{underscore}{}{\usepackage[strings]{underscore}}\makeatother
%%
\begingroup%
\makeatletter%
\begin{pgfpicture}%
\pgfpathrectangle{\pgfpointorigin}{\pgfqpoint{4.680250in}{3.327511in}}%
\pgfusepath{use as bounding box, clip}%
\begin{pgfscope}%
\pgfsetbuttcap%
\pgfsetmiterjoin%
\definecolor{currentfill}{rgb}{1.000000,1.000000,1.000000}%
\pgfsetfillcolor{currentfill}%
\pgfsetlinewidth{0.000000pt}%
\definecolor{currentstroke}{rgb}{1.000000,1.000000,1.000000}%
\pgfsetstrokecolor{currentstroke}%
\pgfsetdash{}{0pt}%
\pgfpathmoveto{\pgfqpoint{0.000000in}{0.000000in}}%
\pgfpathlineto{\pgfqpoint{4.680250in}{0.000000in}}%
\pgfpathlineto{\pgfqpoint{4.680250in}{3.327511in}}%
\pgfpathlineto{\pgfqpoint{0.000000in}{3.327511in}}%
\pgfpathlineto{\pgfqpoint{0.000000in}{0.000000in}}%
\pgfpathclose%
\pgfusepath{fill}%
\end{pgfscope}%
\begin{pgfscope}%
\pgfpathrectangle{\pgfqpoint{0.050000in}{1.730000in}}{\pgfqpoint{1.347132in}{1.347132in}}%
\pgfusepath{clip}%
\pgfsys@transformshift{0.050000in}{1.730000in}%
\pgftext[left,bottom]{\includegraphics[interpolate=true,width=1.350000in,height=1.350000in]{fer2013-images-img0.png}}%
\end{pgfscope}%
\begin{pgfscope}%
\definecolor{textcolor}{rgb}{0.000000,0.000000,0.000000}%
\pgfsetstrokecolor{textcolor}%
\pgfsetfillcolor{textcolor}%
\pgftext[x=0.723566in,y=3.160466in,,base]{\color{textcolor}\rmfamily\fontsize{11.000000}{13.200000}\selectfont empört}%
\end{pgfscope}%
\begin{pgfscope}%
\pgfpathrectangle{\pgfqpoint{1.666559in}{1.730000in}}{\pgfqpoint{1.347132in}{1.347132in}}%
\pgfusepath{clip}%
\pgfsys@transformshift{1.666559in}{1.730000in}%
\pgftext[left,bottom]{\includegraphics[interpolate=true,width=1.350000in,height=1.350000in]{fer2013-images-img1.png}}%
\end{pgfscope}%
\begin{pgfscope}%
\definecolor{textcolor}{rgb}{0.000000,0.000000,0.000000}%
\pgfsetstrokecolor{textcolor}%
\pgfsetfillcolor{textcolor}%
\pgftext[x=2.340125in,y=3.160466in,,base]{\color{textcolor}\rmfamily\fontsize{11.000000}{13.200000}\selectfont ängstlich}%
\end{pgfscope}%
\begin{pgfscope}%
\pgfpathrectangle{\pgfqpoint{3.283118in}{1.730000in}}{\pgfqpoint{1.347132in}{1.347132in}}%
\pgfusepath{clip}%
\pgfsys@transformshift{3.283118in}{1.730000in}%
\pgftext[left,bottom]{\includegraphics[interpolate=true,width=1.350000in,height=1.350000in]{fer2013-images-img2.png}}%
\end{pgfscope}%
\begin{pgfscope}%
\definecolor{textcolor}{rgb}{0.000000,0.000000,0.000000}%
\pgfsetstrokecolor{textcolor}%
\pgfsetfillcolor{textcolor}%
\pgftext[x=3.956684in,y=3.160466in,,base]{\color{textcolor}\rmfamily\fontsize{11.000000}{13.200000}\selectfont glücklich}%
\end{pgfscope}%
\begin{pgfscope}%
\pgfpathrectangle{\pgfqpoint{0.050000in}{0.050000in}}{\pgfqpoint{1.347132in}{1.347132in}}%
\pgfusepath{clip}%
\pgfsys@transformshift{0.050000in}{0.050000in}%
\pgftext[left,bottom]{\includegraphics[interpolate=true,width=1.350000in,height=1.350000in]{fer2013-images-img3.png}}%
\end{pgfscope}%
\begin{pgfscope}%
\definecolor{textcolor}{rgb}{0.000000,0.000000,0.000000}%
\pgfsetstrokecolor{textcolor}%
\pgfsetfillcolor{textcolor}%
\pgftext[x=0.723566in,y=1.480466in,,base]{\color{textcolor}\rmfamily\fontsize{11.000000}{13.200000}\selectfont traurig}%
\end{pgfscope}%
\begin{pgfscope}%
\pgfpathrectangle{\pgfqpoint{1.666559in}{0.050000in}}{\pgfqpoint{1.347132in}{1.347132in}}%
\pgfusepath{clip}%
\pgfsys@transformshift{1.666559in}{0.050000in}%
\pgftext[left,bottom]{\includegraphics[interpolate=true,width=1.350000in,height=1.350000in]{fer2013-images-img4.png}}%
\end{pgfscope}%
\begin{pgfscope}%
\definecolor{textcolor}{rgb}{0.000000,0.000000,0.000000}%
\pgfsetstrokecolor{textcolor}%
\pgfsetfillcolor{textcolor}%
\pgftext[x=2.340125in,y=1.480466in,,base]{\color{textcolor}\rmfamily\fontsize{11.000000}{13.200000}\selectfont überrascht}%
\end{pgfscope}%
\begin{pgfscope}%
\pgfpathrectangle{\pgfqpoint{3.283118in}{0.050000in}}{\pgfqpoint{1.347132in}{1.347132in}}%
\pgfusepath{clip}%
\pgfsys@transformshift{3.280000in}{0.050000in}%
\pgftext[left,bottom]{\includegraphics[interpolate=true,width=1.350000in,height=1.350000in]{fer2013-images-img5.png}}%
\end{pgfscope}%
\begin{pgfscope}%
\definecolor{textcolor}{rgb}{0.000000,0.000000,0.000000}%
\pgfsetstrokecolor{textcolor}%
\pgfsetfillcolor{textcolor}%
\pgftext[x=3.956684in,y=1.480466in,,base]{\color{textcolor}\rmfamily\fontsize{11.000000}{13.200000}\selectfont neutral}%
\end{pgfscope}%
\end{pgfpicture}%
\makeatother%
\endgroup%

	\caption[Beispielbilder des FER-Datensatzes]{Abgebildet ist je ein Bild für eine der sieben Emotionen im FER-Datensatz.}
	\label{fig:FER-Datensatz-Beispiele}
\end{figure}

\textcites{Baldi.1989}{Bourlard.1988} haben gezeigt, dass die Kodierung $\vect{y} = f(\vect{x})$ (siehe \secref{ch:MethodenDerDimRed:ML:AE:MathematischeFormulierung}) eines linearen Autoencoders äquivalent -- aber nicht identisch -- zu den Hauptkomponenten $\tr{\mat{A}}\vect{x}$ der PCA (siehe \secref{ch:MethodenDerDimRed:statistisch:PCA:HerleitungPC}) ist: Die Gewichte eines linearen Autoencoders, der den quadrierten Fehler (\eqref{eq:AE_objectiveFunction}) minimiert, spannen den gleichen Unterraum auf, wie die Ladungsmatrix $\mat{A}$. Dies gilt sogar dann, wenn im Encoder eines dreischichtigen Autoencoders Sigmoid-Aktivierungsfunktionen verwendet werden \parencite[291, 293]{Bourlard.1988}. Jedoch unterscheidet sich die latente Repräsentation $\mat{Y}$
eines linearen Autoencoders mit der von PCA wie folgt \parencite[3]{Plaut.2018}: (1) Die Kovarianzmatrix von $\mat{Y}$ ist nicht diagonal, das heißt die
latente Repräsentation ist nicht unkorreliert.\footnote{Äquivalent dazu ist die Korrelationsmatrix
	von $\mat{Y}$, welche lediglich eine skalierte Kovarianzmatrix darstellt, keine Einheitsmatrix.}
(2) Die transformierten Daten $\mat{Y}$ sind nicht nach absteigender Varianz sortiert. (3) Hat man
einen Encoder $f: \real^D \rightarrow \real^{k_1}$ und möchte man einen Vektor $\vect{x} \in
	\real^D$ auf nur $k_2$ Dimensionen mit $k_2 < k_1$ reduzieren, so können nicht einfach die ersten
$k_2$ Koordinaten von $f(\vect{x})$ verwendet werden. Das bedeutet, dass die Lösungen von
Autoencodern auf unterschiedliche Dimensionen nicht geschachtelt sind.

\begin{figure}[ht]
	\centering
	%% Creator: Matplotlib, PGF backend
%%
%% To include the figure in your LaTeX document, write
%%   \input{<filename>.pgf}
%%
%% Make sure the required packages are loaded in your preamble
%%   \usepackage{pgf}
%%
%% Also ensure that all the required font packages are loaded; for instance,
%% the lmodern package is sometimes necessary when using math font.
%%   \usepackage{lmodern}
%%
%% Figures using additional raster images can only be included by \input if
%% they are in the same directory as the main LaTeX file. For loading figures
%% from other directories you can use the `import` package
%%   \usepackage{import}
%%
%% and then include the figures with
%%   \import{<path to file>}{<filename>.pgf}
%%
%% Matplotlib used the following preamble
%%   
%%   \usepackage{fontspec}
%%   \setmainfont{DejaVuSerif.ttf}[Path=\detokenize{/Users/moritzmistol/.pyenv/versions/3.9.13/envs/thesis/lib/python3.9/site-packages/matplotlib/mpl-data/fonts/ttf/}]
%%   \setsansfont{DejaVuSans.ttf}[Path=\detokenize{/Users/moritzmistol/.pyenv/versions/3.9.13/envs/thesis/lib/python3.9/site-packages/matplotlib/mpl-data/fonts/ttf/}]
%%   \setmonofont{DejaVuSansMono.ttf}[Path=\detokenize{/Users/moritzmistol/.pyenv/versions/3.9.13/envs/thesis/lib/python3.9/site-packages/matplotlib/mpl-data/fonts/ttf/}]
%%   \makeatletter\@ifpackageloaded{underscore}{}{\usepackage[strings]{underscore}}\makeatother
%%
\begingroup%
\makeatletter%
\begin{pgfpicture}%
\pgfpathrectangle{\pgfpointorigin}{\pgfqpoint{4.390031in}{1.627769in}}%
\pgfusepath{use as bounding box, clip}%
\begin{pgfscope}%
\pgfsetbuttcap%
\pgfsetmiterjoin%
\definecolor{currentfill}{rgb}{1.000000,1.000000,1.000000}%
\pgfsetfillcolor{currentfill}%
\pgfsetlinewidth{0.000000pt}%
\definecolor{currentstroke}{rgb}{1.000000,1.000000,1.000000}%
\pgfsetstrokecolor{currentstroke}%
\pgfsetdash{}{0pt}%
\pgfpathmoveto{\pgfqpoint{0.000000in}{0.000000in}}%
\pgfpathlineto{\pgfqpoint{4.390031in}{0.000000in}}%
\pgfpathlineto{\pgfqpoint{4.390031in}{1.627769in}}%
\pgfpathlineto{\pgfqpoint{0.000000in}{1.627769in}}%
\pgfpathlineto{\pgfqpoint{0.000000in}{0.000000in}}%
\pgfpathclose%
\pgfusepath{fill}%
\end{pgfscope}%
\begin{pgfscope}%
\pgfsetbuttcap%
\pgfsetmiterjoin%
\definecolor{currentfill}{rgb}{1.000000,1.000000,1.000000}%
\pgfsetfillcolor{currentfill}%
\pgfsetlinewidth{0.000000pt}%
\definecolor{currentstroke}{rgb}{0.000000,0.000000,0.000000}%
\pgfsetstrokecolor{currentstroke}%
\pgfsetstrokeopacity{0.000000}%
\pgfsetdash{}{0pt}%
\pgfpathmoveto{\pgfqpoint{0.050000in}{0.294066in}}%
\pgfpathlineto{\pgfqpoint{1.125882in}{0.294066in}}%
\pgfpathlineto{\pgfqpoint{1.125882in}{1.369949in}}%
\pgfpathlineto{\pgfqpoint{0.050000in}{1.369949in}}%
\pgfpathlineto{\pgfqpoint{0.050000in}{0.294066in}}%
\pgfpathclose%
\pgfusepath{fill}%
\end{pgfscope}%
\begin{pgfscope}%
\pgfpathrectangle{\pgfqpoint{0.050000in}{0.294066in}}{\pgfqpoint{1.075882in}{1.075882in}}%
\pgfusepath{clip}%
\pgfsys@transformshift{0.050000in}{0.294066in}%
\pgftext[left,bottom]{\includegraphics[interpolate=true,width=1.080000in,height=1.080000in]{correlation-matrices-img0.png}}%
\end{pgfscope}%
\begin{pgfscope}%
\pgfsetrectcap%
\pgfsetmiterjoin%
\pgfsetlinewidth{0.501875pt}%
\definecolor{currentstroke}{rgb}{0.000000,0.000000,0.000000}%
\pgfsetstrokecolor{currentstroke}%
\pgfsetdash{}{0pt}%
\pgfpathmoveto{\pgfqpoint{0.050000in}{0.294066in}}%
\pgfpathlineto{\pgfqpoint{0.050000in}{1.369949in}}%
\pgfusepath{stroke}%
\end{pgfscope}%
\begin{pgfscope}%
\pgfsetrectcap%
\pgfsetmiterjoin%
\pgfsetlinewidth{0.501875pt}%
\definecolor{currentstroke}{rgb}{0.000000,0.000000,0.000000}%
\pgfsetstrokecolor{currentstroke}%
\pgfsetdash{}{0pt}%
\pgfpathmoveto{\pgfqpoint{1.125882in}{0.294066in}}%
\pgfpathlineto{\pgfqpoint{1.125882in}{1.369949in}}%
\pgfusepath{stroke}%
\end{pgfscope}%
\begin{pgfscope}%
\pgfsetrectcap%
\pgfsetmiterjoin%
\pgfsetlinewidth{0.501875pt}%
\definecolor{currentstroke}{rgb}{0.000000,0.000000,0.000000}%
\pgfsetstrokecolor{currentstroke}%
\pgfsetdash{}{0pt}%
\pgfpathmoveto{\pgfqpoint{0.050000in}{0.294066in}}%
\pgfpathlineto{\pgfqpoint{1.125882in}{0.294066in}}%
\pgfusepath{stroke}%
\end{pgfscope}%
\begin{pgfscope}%
\pgfsetrectcap%
\pgfsetmiterjoin%
\pgfsetlinewidth{0.501875pt}%
\definecolor{currentstroke}{rgb}{0.000000,0.000000,0.000000}%
\pgfsetstrokecolor{currentstroke}%
\pgfsetdash{}{0pt}%
\pgfpathmoveto{\pgfqpoint{0.050000in}{1.369949in}}%
\pgfpathlineto{\pgfqpoint{1.125882in}{1.369949in}}%
\pgfusepath{stroke}%
\end{pgfscope}%
\begin{pgfscope}%
\pgfsetbuttcap%
\pgfsetmiterjoin%
\definecolor{currentfill}{rgb}{1.000000,1.000000,1.000000}%
\pgfsetfillcolor{currentfill}%
\pgfsetlinewidth{0.000000pt}%
\definecolor{currentstroke}{rgb}{0.000000,0.000000,0.000000}%
\pgfsetstrokecolor{currentstroke}%
\pgfsetstrokeopacity{0.000000}%
\pgfsetdash{}{0pt}%
\pgfpathmoveto{\pgfqpoint{1.341059in}{0.294066in}}%
\pgfpathlineto{\pgfqpoint{2.416941in}{0.294066in}}%
\pgfpathlineto{\pgfqpoint{2.416941in}{1.369949in}}%
\pgfpathlineto{\pgfqpoint{1.341059in}{1.369949in}}%
\pgfpathlineto{\pgfqpoint{1.341059in}{0.294066in}}%
\pgfpathclose%
\pgfusepath{fill}%
\end{pgfscope}%
\begin{pgfscope}%
\pgfpathrectangle{\pgfqpoint{1.341059in}{0.294066in}}{\pgfqpoint{1.075882in}{1.075882in}}%
\pgfusepath{clip}%
\pgfsys@transformshift{1.341059in}{0.294066in}%
\pgftext[left,bottom]{\includegraphics[interpolate=true,width=1.080000in,height=1.080000in]{correlation-matrices-img1.png}}%
\end{pgfscope}%
\begin{pgfscope}%
\pgfsetrectcap%
\pgfsetmiterjoin%
\pgfsetlinewidth{0.501875pt}%
\definecolor{currentstroke}{rgb}{0.000000,0.000000,0.000000}%
\pgfsetstrokecolor{currentstroke}%
\pgfsetdash{}{0pt}%
\pgfpathmoveto{\pgfqpoint{1.341059in}{0.294066in}}%
\pgfpathlineto{\pgfqpoint{1.341059in}{1.369949in}}%
\pgfusepath{stroke}%
\end{pgfscope}%
\begin{pgfscope}%
\pgfsetrectcap%
\pgfsetmiterjoin%
\pgfsetlinewidth{0.501875pt}%
\definecolor{currentstroke}{rgb}{0.000000,0.000000,0.000000}%
\pgfsetstrokecolor{currentstroke}%
\pgfsetdash{}{0pt}%
\pgfpathmoveto{\pgfqpoint{2.416941in}{0.294066in}}%
\pgfpathlineto{\pgfqpoint{2.416941in}{1.369949in}}%
\pgfusepath{stroke}%
\end{pgfscope}%
\begin{pgfscope}%
\pgfsetrectcap%
\pgfsetmiterjoin%
\pgfsetlinewidth{0.501875pt}%
\definecolor{currentstroke}{rgb}{0.000000,0.000000,0.000000}%
\pgfsetstrokecolor{currentstroke}%
\pgfsetdash{}{0pt}%
\pgfpathmoveto{\pgfqpoint{1.341059in}{0.294066in}}%
\pgfpathlineto{\pgfqpoint{2.416941in}{0.294066in}}%
\pgfusepath{stroke}%
\end{pgfscope}%
\begin{pgfscope}%
\pgfsetrectcap%
\pgfsetmiterjoin%
\pgfsetlinewidth{0.501875pt}%
\definecolor{currentstroke}{rgb}{0.000000,0.000000,0.000000}%
\pgfsetstrokecolor{currentstroke}%
\pgfsetdash{}{0pt}%
\pgfpathmoveto{\pgfqpoint{1.341059in}{1.369949in}}%
\pgfpathlineto{\pgfqpoint{2.416941in}{1.369949in}}%
\pgfusepath{stroke}%
\end{pgfscope}%
\begin{pgfscope}%
\pgfsetbuttcap%
\pgfsetmiterjoin%
\definecolor{currentfill}{rgb}{1.000000,1.000000,1.000000}%
\pgfsetfillcolor{currentfill}%
\pgfsetlinewidth{0.000000pt}%
\definecolor{currentstroke}{rgb}{0.000000,0.000000,0.000000}%
\pgfsetstrokecolor{currentstroke}%
\pgfsetstrokeopacity{0.000000}%
\pgfsetdash{}{0pt}%
\pgfpathmoveto{\pgfqpoint{2.632118in}{0.294066in}}%
\pgfpathlineto{\pgfqpoint{3.708000in}{0.294066in}}%
\pgfpathlineto{\pgfqpoint{3.708000in}{1.369949in}}%
\pgfpathlineto{\pgfqpoint{2.632118in}{1.369949in}}%
\pgfpathlineto{\pgfqpoint{2.632118in}{0.294066in}}%
\pgfpathclose%
\pgfusepath{fill}%
\end{pgfscope}%
\begin{pgfscope}%
\pgfpathrectangle{\pgfqpoint{2.632118in}{0.294066in}}{\pgfqpoint{1.075882in}{1.075882in}}%
\pgfusepath{clip}%
\pgfsys@transformshift{2.632118in}{0.294066in}%
\pgftext[left,bottom]{\includegraphics[interpolate=true,width=1.080000in,height=1.080000in]{correlation-matrices-img2.png}}%
\end{pgfscope}%
\begin{pgfscope}%
\pgfsetrectcap%
\pgfsetmiterjoin%
\pgfsetlinewidth{0.501875pt}%
\definecolor{currentstroke}{rgb}{0.000000,0.000000,0.000000}%
\pgfsetstrokecolor{currentstroke}%
\pgfsetdash{}{0pt}%
\pgfpathmoveto{\pgfqpoint{2.632118in}{0.294066in}}%
\pgfpathlineto{\pgfqpoint{2.632118in}{1.369949in}}%
\pgfusepath{stroke}%
\end{pgfscope}%
\begin{pgfscope}%
\pgfsetrectcap%
\pgfsetmiterjoin%
\pgfsetlinewidth{0.501875pt}%
\definecolor{currentstroke}{rgb}{0.000000,0.000000,0.000000}%
\pgfsetstrokecolor{currentstroke}%
\pgfsetdash{}{0pt}%
\pgfpathmoveto{\pgfqpoint{3.708000in}{0.294066in}}%
\pgfpathlineto{\pgfqpoint{3.708000in}{1.369949in}}%
\pgfusepath{stroke}%
\end{pgfscope}%
\begin{pgfscope}%
\pgfsetrectcap%
\pgfsetmiterjoin%
\pgfsetlinewidth{0.501875pt}%
\definecolor{currentstroke}{rgb}{0.000000,0.000000,0.000000}%
\pgfsetstrokecolor{currentstroke}%
\pgfsetdash{}{0pt}%
\pgfpathmoveto{\pgfqpoint{2.632118in}{0.294066in}}%
\pgfpathlineto{\pgfqpoint{3.708000in}{0.294066in}}%
\pgfusepath{stroke}%
\end{pgfscope}%
\begin{pgfscope}%
\pgfsetrectcap%
\pgfsetmiterjoin%
\pgfsetlinewidth{0.501875pt}%
\definecolor{currentstroke}{rgb}{0.000000,0.000000,0.000000}%
\pgfsetstrokecolor{currentstroke}%
\pgfsetdash{}{0pt}%
\pgfpathmoveto{\pgfqpoint{2.632118in}{1.369949in}}%
\pgfpathlineto{\pgfqpoint{3.708000in}{1.369949in}}%
\pgfusepath{stroke}%
\end{pgfscope}%
\begin{pgfscope}%
\definecolor{textcolor}{rgb}{0.000000,0.000000,0.000000}%
\pgfsetstrokecolor{textcolor}%
\pgfsetfillcolor{textcolor}%
\pgftext[x=0.587941in,y=0.078890in,,base]{\color{textcolor}\rmfamily\fontsize{10.000000}{12.000000}\selectfont (a)}%
\end{pgfscope}%
\begin{pgfscope}%
\definecolor{textcolor}{rgb}{0.000000,0.000000,0.000000}%
\pgfsetstrokecolor{textcolor}%
\pgfsetfillcolor{textcolor}%
\pgftext[x=1.879000in,y=0.078890in,,base]{\color{textcolor}\rmfamily\fontsize{10.000000}{12.000000}\selectfont (b)}%
\end{pgfscope}%
\begin{pgfscope}%
\definecolor{textcolor}{rgb}{0.000000,0.000000,0.000000}%
\pgfsetstrokecolor{textcolor}%
\pgfsetfillcolor{textcolor}%
\pgftext[x=3.170059in,y=0.078890in,,base]{\color{textcolor}\rmfamily\fontsize{10.000000}{12.000000}\selectfont (c)}%
\end{pgfscope}%
\begin{pgfscope}%
\pgfsetbuttcap%
\pgfsetmiterjoin%
\definecolor{currentfill}{rgb}{1.000000,1.000000,1.000000}%
\pgfsetfillcolor{currentfill}%
\pgfsetlinewidth{0.000000pt}%
\definecolor{currentstroke}{rgb}{0.000000,0.000000,0.000000}%
\pgfsetstrokecolor{currentstroke}%
\pgfsetstrokeopacity{0.000000}%
\pgfsetdash{}{0pt}%
\pgfpathmoveto{\pgfqpoint{3.936625in}{0.139007in}}%
\pgfpathlineto{\pgfqpoint{4.005925in}{0.139007in}}%
\pgfpathlineto{\pgfqpoint{4.005925in}{1.525007in}}%
\pgfpathlineto{\pgfqpoint{3.936625in}{1.525007in}}%
\pgfpathlineto{\pgfqpoint{3.936625in}{0.139007in}}%
\pgfpathclose%
\pgfusepath{fill}%
\end{pgfscope}%
\begin{pgfscope}%
\pgfpathrectangle{\pgfqpoint{3.936625in}{0.139007in}}{\pgfqpoint{0.069300in}{1.386000in}}%
\pgfusepath{clip}%
\pgfsetbuttcap%
\pgfsetmiterjoin%
\definecolor{currentfill}{rgb}{1.000000,1.000000,1.000000}%
\pgfsetfillcolor{currentfill}%
\pgfsetlinewidth{0.010037pt}%
\definecolor{currentstroke}{rgb}{1.000000,1.000000,1.000000}%
\pgfsetstrokecolor{currentstroke}%
\pgfsetdash{}{0pt}%
\pgfusepath{stroke,fill}%
\end{pgfscope}%
\begin{pgfscope}%
\pgfsys@transformshift{3.940000in}{0.147769in}%
\pgftext[left,bottom]{\includegraphics[interpolate=true,width=0.070000in,height=1.390000in]{correlation-matrices-img3.png}}%
\end{pgfscope}%
\begin{pgfscope}%
\pgfsetbuttcap%
\pgfsetroundjoin%
\definecolor{currentfill}{rgb}{0.000000,0.000000,0.000000}%
\pgfsetfillcolor{currentfill}%
\pgfsetlinewidth{0.501875pt}%
\definecolor{currentstroke}{rgb}{0.000000,0.000000,0.000000}%
\pgfsetstrokecolor{currentstroke}%
\pgfsetdash{}{0pt}%
\pgfsys@defobject{currentmarker}{\pgfqpoint{-0.041667in}{0.000000in}}{\pgfqpoint{-0.000000in}{0.000000in}}{%
\pgfpathmoveto{\pgfqpoint{-0.000000in}{0.000000in}}%
\pgfpathlineto{\pgfqpoint{-0.041667in}{0.000000in}}%
\pgfusepath{stroke,fill}%
}%
\begin{pgfscope}%
\pgfsys@transformshift{4.005925in}{0.139007in}%
\pgfsys@useobject{currentmarker}{}%
\end{pgfscope}%
\end{pgfscope}%
\begin{pgfscope}%
\definecolor{textcolor}{rgb}{0.000000,0.000000,0.000000}%
\pgfsetstrokecolor{textcolor}%
\pgfsetfillcolor{textcolor}%
\pgftext[x=4.054536in, y=0.086246in, left, base]{\color{textcolor}\rmfamily\fontsize{10.000000}{12.000000}\selectfont \(\displaystyle {\ensuremath{-}1.0}\)}%
\end{pgfscope}%
\begin{pgfscope}%
\pgfsetbuttcap%
\pgfsetroundjoin%
\definecolor{currentfill}{rgb}{0.000000,0.000000,0.000000}%
\pgfsetfillcolor{currentfill}%
\pgfsetlinewidth{0.501875pt}%
\definecolor{currentstroke}{rgb}{0.000000,0.000000,0.000000}%
\pgfsetstrokecolor{currentstroke}%
\pgfsetdash{}{0pt}%
\pgfsys@defobject{currentmarker}{\pgfqpoint{-0.041667in}{0.000000in}}{\pgfqpoint{-0.000000in}{0.000000in}}{%
\pgfpathmoveto{\pgfqpoint{-0.000000in}{0.000000in}}%
\pgfpathlineto{\pgfqpoint{-0.041667in}{0.000000in}}%
\pgfusepath{stroke,fill}%
}%
\begin{pgfscope}%
\pgfsys@transformshift{4.005925in}{0.485507in}%
\pgfsys@useobject{currentmarker}{}%
\end{pgfscope}%
\end{pgfscope}%
\begin{pgfscope}%
\definecolor{textcolor}{rgb}{0.000000,0.000000,0.000000}%
\pgfsetstrokecolor{textcolor}%
\pgfsetfillcolor{textcolor}%
\pgftext[x=4.054536in, y=0.432746in, left, base]{\color{textcolor}\rmfamily\fontsize{10.000000}{12.000000}\selectfont \(\displaystyle {\ensuremath{-}0.5}\)}%
\end{pgfscope}%
\begin{pgfscope}%
\pgfsetbuttcap%
\pgfsetroundjoin%
\definecolor{currentfill}{rgb}{0.000000,0.000000,0.000000}%
\pgfsetfillcolor{currentfill}%
\pgfsetlinewidth{0.501875pt}%
\definecolor{currentstroke}{rgb}{0.000000,0.000000,0.000000}%
\pgfsetstrokecolor{currentstroke}%
\pgfsetdash{}{0pt}%
\pgfsys@defobject{currentmarker}{\pgfqpoint{-0.041667in}{0.000000in}}{\pgfqpoint{-0.000000in}{0.000000in}}{%
\pgfpathmoveto{\pgfqpoint{-0.000000in}{0.000000in}}%
\pgfpathlineto{\pgfqpoint{-0.041667in}{0.000000in}}%
\pgfusepath{stroke,fill}%
}%
\begin{pgfscope}%
\pgfsys@transformshift{4.005925in}{0.832007in}%
\pgfsys@useobject{currentmarker}{}%
\end{pgfscope}%
\end{pgfscope}%
\begin{pgfscope}%
\definecolor{textcolor}{rgb}{0.000000,0.000000,0.000000}%
\pgfsetstrokecolor{textcolor}%
\pgfsetfillcolor{textcolor}%
\pgftext[x=4.054536in, y=0.779246in, left, base]{\color{textcolor}\rmfamily\fontsize{10.000000}{12.000000}\selectfont \(\displaystyle {0.0}\)}%
\end{pgfscope}%
\begin{pgfscope}%
\pgfsetbuttcap%
\pgfsetroundjoin%
\definecolor{currentfill}{rgb}{0.000000,0.000000,0.000000}%
\pgfsetfillcolor{currentfill}%
\pgfsetlinewidth{0.501875pt}%
\definecolor{currentstroke}{rgb}{0.000000,0.000000,0.000000}%
\pgfsetstrokecolor{currentstroke}%
\pgfsetdash{}{0pt}%
\pgfsys@defobject{currentmarker}{\pgfqpoint{-0.041667in}{0.000000in}}{\pgfqpoint{-0.000000in}{0.000000in}}{%
\pgfpathmoveto{\pgfqpoint{-0.000000in}{0.000000in}}%
\pgfpathlineto{\pgfqpoint{-0.041667in}{0.000000in}}%
\pgfusepath{stroke,fill}%
}%
\begin{pgfscope}%
\pgfsys@transformshift{4.005925in}{1.178507in}%
\pgfsys@useobject{currentmarker}{}%
\end{pgfscope}%
\end{pgfscope}%
\begin{pgfscope}%
\definecolor{textcolor}{rgb}{0.000000,0.000000,0.000000}%
\pgfsetstrokecolor{textcolor}%
\pgfsetfillcolor{textcolor}%
\pgftext[x=4.054536in, y=1.125746in, left, base]{\color{textcolor}\rmfamily\fontsize{10.000000}{12.000000}\selectfont \(\displaystyle {0.5}\)}%
\end{pgfscope}%
\begin{pgfscope}%
\pgfsetbuttcap%
\pgfsetroundjoin%
\definecolor{currentfill}{rgb}{0.000000,0.000000,0.000000}%
\pgfsetfillcolor{currentfill}%
\pgfsetlinewidth{0.501875pt}%
\definecolor{currentstroke}{rgb}{0.000000,0.000000,0.000000}%
\pgfsetstrokecolor{currentstroke}%
\pgfsetdash{}{0pt}%
\pgfsys@defobject{currentmarker}{\pgfqpoint{-0.041667in}{0.000000in}}{\pgfqpoint{-0.000000in}{0.000000in}}{%
\pgfpathmoveto{\pgfqpoint{-0.000000in}{0.000000in}}%
\pgfpathlineto{\pgfqpoint{-0.041667in}{0.000000in}}%
\pgfusepath{stroke,fill}%
}%
\begin{pgfscope}%
\pgfsys@transformshift{4.005925in}{1.525007in}%
\pgfsys@useobject{currentmarker}{}%
\end{pgfscope}%
\end{pgfscope}%
\begin{pgfscope}%
\definecolor{textcolor}{rgb}{0.000000,0.000000,0.000000}%
\pgfsetstrokecolor{textcolor}%
\pgfsetfillcolor{textcolor}%
\pgftext[x=4.054536in, y=1.472246in, left, base]{\color{textcolor}\rmfamily\fontsize{10.000000}{12.000000}\selectfont \(\displaystyle {1.0}\)}%
\end{pgfscope}%
\begin{pgfscope}%
\pgfsetbuttcap%
\pgfsetroundjoin%
\definecolor{currentfill}{rgb}{0.000000,0.000000,0.000000}%
\pgfsetfillcolor{currentfill}%
\pgfsetlinewidth{0.501875pt}%
\definecolor{currentstroke}{rgb}{0.000000,0.000000,0.000000}%
\pgfsetstrokecolor{currentstroke}%
\pgfsetdash{}{0pt}%
\pgfsys@defobject{currentmarker}{\pgfqpoint{-0.020833in}{0.000000in}}{\pgfqpoint{-0.000000in}{0.000000in}}{%
\pgfpathmoveto{\pgfqpoint{-0.000000in}{0.000000in}}%
\pgfpathlineto{\pgfqpoint{-0.020833in}{0.000000in}}%
\pgfusepath{stroke,fill}%
}%
\begin{pgfscope}%
\pgfsys@transformshift{4.005925in}{0.208307in}%
\pgfsys@useobject{currentmarker}{}%
\end{pgfscope}%
\end{pgfscope}%
\begin{pgfscope}%
\pgfsetbuttcap%
\pgfsetroundjoin%
\definecolor{currentfill}{rgb}{0.000000,0.000000,0.000000}%
\pgfsetfillcolor{currentfill}%
\pgfsetlinewidth{0.501875pt}%
\definecolor{currentstroke}{rgb}{0.000000,0.000000,0.000000}%
\pgfsetstrokecolor{currentstroke}%
\pgfsetdash{}{0pt}%
\pgfsys@defobject{currentmarker}{\pgfqpoint{-0.020833in}{0.000000in}}{\pgfqpoint{-0.000000in}{0.000000in}}{%
\pgfpathmoveto{\pgfqpoint{-0.000000in}{0.000000in}}%
\pgfpathlineto{\pgfqpoint{-0.020833in}{0.000000in}}%
\pgfusepath{stroke,fill}%
}%
\begin{pgfscope}%
\pgfsys@transformshift{4.005925in}{0.277607in}%
\pgfsys@useobject{currentmarker}{}%
\end{pgfscope}%
\end{pgfscope}%
\begin{pgfscope}%
\pgfsetbuttcap%
\pgfsetroundjoin%
\definecolor{currentfill}{rgb}{0.000000,0.000000,0.000000}%
\pgfsetfillcolor{currentfill}%
\pgfsetlinewidth{0.501875pt}%
\definecolor{currentstroke}{rgb}{0.000000,0.000000,0.000000}%
\pgfsetstrokecolor{currentstroke}%
\pgfsetdash{}{0pt}%
\pgfsys@defobject{currentmarker}{\pgfqpoint{-0.020833in}{0.000000in}}{\pgfqpoint{-0.000000in}{0.000000in}}{%
\pgfpathmoveto{\pgfqpoint{-0.000000in}{0.000000in}}%
\pgfpathlineto{\pgfqpoint{-0.020833in}{0.000000in}}%
\pgfusepath{stroke,fill}%
}%
\begin{pgfscope}%
\pgfsys@transformshift{4.005925in}{0.346907in}%
\pgfsys@useobject{currentmarker}{}%
\end{pgfscope}%
\end{pgfscope}%
\begin{pgfscope}%
\pgfsetbuttcap%
\pgfsetroundjoin%
\definecolor{currentfill}{rgb}{0.000000,0.000000,0.000000}%
\pgfsetfillcolor{currentfill}%
\pgfsetlinewidth{0.501875pt}%
\definecolor{currentstroke}{rgb}{0.000000,0.000000,0.000000}%
\pgfsetstrokecolor{currentstroke}%
\pgfsetdash{}{0pt}%
\pgfsys@defobject{currentmarker}{\pgfqpoint{-0.020833in}{0.000000in}}{\pgfqpoint{-0.000000in}{0.000000in}}{%
\pgfpathmoveto{\pgfqpoint{-0.000000in}{0.000000in}}%
\pgfpathlineto{\pgfqpoint{-0.020833in}{0.000000in}}%
\pgfusepath{stroke,fill}%
}%
\begin{pgfscope}%
\pgfsys@transformshift{4.005925in}{0.416207in}%
\pgfsys@useobject{currentmarker}{}%
\end{pgfscope}%
\end{pgfscope}%
\begin{pgfscope}%
\pgfsetbuttcap%
\pgfsetroundjoin%
\definecolor{currentfill}{rgb}{0.000000,0.000000,0.000000}%
\pgfsetfillcolor{currentfill}%
\pgfsetlinewidth{0.501875pt}%
\definecolor{currentstroke}{rgb}{0.000000,0.000000,0.000000}%
\pgfsetstrokecolor{currentstroke}%
\pgfsetdash{}{0pt}%
\pgfsys@defobject{currentmarker}{\pgfqpoint{-0.020833in}{0.000000in}}{\pgfqpoint{-0.000000in}{0.000000in}}{%
\pgfpathmoveto{\pgfqpoint{-0.000000in}{0.000000in}}%
\pgfpathlineto{\pgfqpoint{-0.020833in}{0.000000in}}%
\pgfusepath{stroke,fill}%
}%
\begin{pgfscope}%
\pgfsys@transformshift{4.005925in}{0.554807in}%
\pgfsys@useobject{currentmarker}{}%
\end{pgfscope}%
\end{pgfscope}%
\begin{pgfscope}%
\pgfsetbuttcap%
\pgfsetroundjoin%
\definecolor{currentfill}{rgb}{0.000000,0.000000,0.000000}%
\pgfsetfillcolor{currentfill}%
\pgfsetlinewidth{0.501875pt}%
\definecolor{currentstroke}{rgb}{0.000000,0.000000,0.000000}%
\pgfsetstrokecolor{currentstroke}%
\pgfsetdash{}{0pt}%
\pgfsys@defobject{currentmarker}{\pgfqpoint{-0.020833in}{0.000000in}}{\pgfqpoint{-0.000000in}{0.000000in}}{%
\pgfpathmoveto{\pgfqpoint{-0.000000in}{0.000000in}}%
\pgfpathlineto{\pgfqpoint{-0.020833in}{0.000000in}}%
\pgfusepath{stroke,fill}%
}%
\begin{pgfscope}%
\pgfsys@transformshift{4.005925in}{0.624107in}%
\pgfsys@useobject{currentmarker}{}%
\end{pgfscope}%
\end{pgfscope}%
\begin{pgfscope}%
\pgfsetbuttcap%
\pgfsetroundjoin%
\definecolor{currentfill}{rgb}{0.000000,0.000000,0.000000}%
\pgfsetfillcolor{currentfill}%
\pgfsetlinewidth{0.501875pt}%
\definecolor{currentstroke}{rgb}{0.000000,0.000000,0.000000}%
\pgfsetstrokecolor{currentstroke}%
\pgfsetdash{}{0pt}%
\pgfsys@defobject{currentmarker}{\pgfqpoint{-0.020833in}{0.000000in}}{\pgfqpoint{-0.000000in}{0.000000in}}{%
\pgfpathmoveto{\pgfqpoint{-0.000000in}{0.000000in}}%
\pgfpathlineto{\pgfqpoint{-0.020833in}{0.000000in}}%
\pgfusepath{stroke,fill}%
}%
\begin{pgfscope}%
\pgfsys@transformshift{4.005925in}{0.693407in}%
\pgfsys@useobject{currentmarker}{}%
\end{pgfscope}%
\end{pgfscope}%
\begin{pgfscope}%
\pgfsetbuttcap%
\pgfsetroundjoin%
\definecolor{currentfill}{rgb}{0.000000,0.000000,0.000000}%
\pgfsetfillcolor{currentfill}%
\pgfsetlinewidth{0.501875pt}%
\definecolor{currentstroke}{rgb}{0.000000,0.000000,0.000000}%
\pgfsetstrokecolor{currentstroke}%
\pgfsetdash{}{0pt}%
\pgfsys@defobject{currentmarker}{\pgfqpoint{-0.020833in}{0.000000in}}{\pgfqpoint{-0.000000in}{0.000000in}}{%
\pgfpathmoveto{\pgfqpoint{-0.000000in}{0.000000in}}%
\pgfpathlineto{\pgfqpoint{-0.020833in}{0.000000in}}%
\pgfusepath{stroke,fill}%
}%
\begin{pgfscope}%
\pgfsys@transformshift{4.005925in}{0.762707in}%
\pgfsys@useobject{currentmarker}{}%
\end{pgfscope}%
\end{pgfscope}%
\begin{pgfscope}%
\pgfsetbuttcap%
\pgfsetroundjoin%
\definecolor{currentfill}{rgb}{0.000000,0.000000,0.000000}%
\pgfsetfillcolor{currentfill}%
\pgfsetlinewidth{0.501875pt}%
\definecolor{currentstroke}{rgb}{0.000000,0.000000,0.000000}%
\pgfsetstrokecolor{currentstroke}%
\pgfsetdash{}{0pt}%
\pgfsys@defobject{currentmarker}{\pgfqpoint{-0.020833in}{0.000000in}}{\pgfqpoint{-0.000000in}{0.000000in}}{%
\pgfpathmoveto{\pgfqpoint{-0.000000in}{0.000000in}}%
\pgfpathlineto{\pgfqpoint{-0.020833in}{0.000000in}}%
\pgfusepath{stroke,fill}%
}%
\begin{pgfscope}%
\pgfsys@transformshift{4.005925in}{0.901307in}%
\pgfsys@useobject{currentmarker}{}%
\end{pgfscope}%
\end{pgfscope}%
\begin{pgfscope}%
\pgfsetbuttcap%
\pgfsetroundjoin%
\definecolor{currentfill}{rgb}{0.000000,0.000000,0.000000}%
\pgfsetfillcolor{currentfill}%
\pgfsetlinewidth{0.501875pt}%
\definecolor{currentstroke}{rgb}{0.000000,0.000000,0.000000}%
\pgfsetstrokecolor{currentstroke}%
\pgfsetdash{}{0pt}%
\pgfsys@defobject{currentmarker}{\pgfqpoint{-0.020833in}{0.000000in}}{\pgfqpoint{-0.000000in}{0.000000in}}{%
\pgfpathmoveto{\pgfqpoint{-0.000000in}{0.000000in}}%
\pgfpathlineto{\pgfqpoint{-0.020833in}{0.000000in}}%
\pgfusepath{stroke,fill}%
}%
\begin{pgfscope}%
\pgfsys@transformshift{4.005925in}{0.970607in}%
\pgfsys@useobject{currentmarker}{}%
\end{pgfscope}%
\end{pgfscope}%
\begin{pgfscope}%
\pgfsetbuttcap%
\pgfsetroundjoin%
\definecolor{currentfill}{rgb}{0.000000,0.000000,0.000000}%
\pgfsetfillcolor{currentfill}%
\pgfsetlinewidth{0.501875pt}%
\definecolor{currentstroke}{rgb}{0.000000,0.000000,0.000000}%
\pgfsetstrokecolor{currentstroke}%
\pgfsetdash{}{0pt}%
\pgfsys@defobject{currentmarker}{\pgfqpoint{-0.020833in}{0.000000in}}{\pgfqpoint{-0.000000in}{0.000000in}}{%
\pgfpathmoveto{\pgfqpoint{-0.000000in}{0.000000in}}%
\pgfpathlineto{\pgfqpoint{-0.020833in}{0.000000in}}%
\pgfusepath{stroke,fill}%
}%
\begin{pgfscope}%
\pgfsys@transformshift{4.005925in}{1.039907in}%
\pgfsys@useobject{currentmarker}{}%
\end{pgfscope}%
\end{pgfscope}%
\begin{pgfscope}%
\pgfsetbuttcap%
\pgfsetroundjoin%
\definecolor{currentfill}{rgb}{0.000000,0.000000,0.000000}%
\pgfsetfillcolor{currentfill}%
\pgfsetlinewidth{0.501875pt}%
\definecolor{currentstroke}{rgb}{0.000000,0.000000,0.000000}%
\pgfsetstrokecolor{currentstroke}%
\pgfsetdash{}{0pt}%
\pgfsys@defobject{currentmarker}{\pgfqpoint{-0.020833in}{0.000000in}}{\pgfqpoint{-0.000000in}{0.000000in}}{%
\pgfpathmoveto{\pgfqpoint{-0.000000in}{0.000000in}}%
\pgfpathlineto{\pgfqpoint{-0.020833in}{0.000000in}}%
\pgfusepath{stroke,fill}%
}%
\begin{pgfscope}%
\pgfsys@transformshift{4.005925in}{1.109207in}%
\pgfsys@useobject{currentmarker}{}%
\end{pgfscope}%
\end{pgfscope}%
\begin{pgfscope}%
\pgfsetbuttcap%
\pgfsetroundjoin%
\definecolor{currentfill}{rgb}{0.000000,0.000000,0.000000}%
\pgfsetfillcolor{currentfill}%
\pgfsetlinewidth{0.501875pt}%
\definecolor{currentstroke}{rgb}{0.000000,0.000000,0.000000}%
\pgfsetstrokecolor{currentstroke}%
\pgfsetdash{}{0pt}%
\pgfsys@defobject{currentmarker}{\pgfqpoint{-0.020833in}{0.000000in}}{\pgfqpoint{-0.000000in}{0.000000in}}{%
\pgfpathmoveto{\pgfqpoint{-0.000000in}{0.000000in}}%
\pgfpathlineto{\pgfqpoint{-0.020833in}{0.000000in}}%
\pgfusepath{stroke,fill}%
}%
\begin{pgfscope}%
\pgfsys@transformshift{4.005925in}{1.247807in}%
\pgfsys@useobject{currentmarker}{}%
\end{pgfscope}%
\end{pgfscope}%
\begin{pgfscope}%
\pgfsetbuttcap%
\pgfsetroundjoin%
\definecolor{currentfill}{rgb}{0.000000,0.000000,0.000000}%
\pgfsetfillcolor{currentfill}%
\pgfsetlinewidth{0.501875pt}%
\definecolor{currentstroke}{rgb}{0.000000,0.000000,0.000000}%
\pgfsetstrokecolor{currentstroke}%
\pgfsetdash{}{0pt}%
\pgfsys@defobject{currentmarker}{\pgfqpoint{-0.020833in}{0.000000in}}{\pgfqpoint{-0.000000in}{0.000000in}}{%
\pgfpathmoveto{\pgfqpoint{-0.000000in}{0.000000in}}%
\pgfpathlineto{\pgfqpoint{-0.020833in}{0.000000in}}%
\pgfusepath{stroke,fill}%
}%
\begin{pgfscope}%
\pgfsys@transformshift{4.005925in}{1.317107in}%
\pgfsys@useobject{currentmarker}{}%
\end{pgfscope}%
\end{pgfscope}%
\begin{pgfscope}%
\pgfsetbuttcap%
\pgfsetroundjoin%
\definecolor{currentfill}{rgb}{0.000000,0.000000,0.000000}%
\pgfsetfillcolor{currentfill}%
\pgfsetlinewidth{0.501875pt}%
\definecolor{currentstroke}{rgb}{0.000000,0.000000,0.000000}%
\pgfsetstrokecolor{currentstroke}%
\pgfsetdash{}{0pt}%
\pgfsys@defobject{currentmarker}{\pgfqpoint{-0.020833in}{0.000000in}}{\pgfqpoint{-0.000000in}{0.000000in}}{%
\pgfpathmoveto{\pgfqpoint{-0.000000in}{0.000000in}}%
\pgfpathlineto{\pgfqpoint{-0.020833in}{0.000000in}}%
\pgfusepath{stroke,fill}%
}%
\begin{pgfscope}%
\pgfsys@transformshift{4.005925in}{1.386407in}%
\pgfsys@useobject{currentmarker}{}%
\end{pgfscope}%
\end{pgfscope}%
\begin{pgfscope}%
\pgfsetbuttcap%
\pgfsetroundjoin%
\definecolor{currentfill}{rgb}{0.000000,0.000000,0.000000}%
\pgfsetfillcolor{currentfill}%
\pgfsetlinewidth{0.501875pt}%
\definecolor{currentstroke}{rgb}{0.000000,0.000000,0.000000}%
\pgfsetstrokecolor{currentstroke}%
\pgfsetdash{}{0pt}%
\pgfsys@defobject{currentmarker}{\pgfqpoint{-0.020833in}{0.000000in}}{\pgfqpoint{-0.000000in}{0.000000in}}{%
\pgfpathmoveto{\pgfqpoint{-0.000000in}{0.000000in}}%
\pgfpathlineto{\pgfqpoint{-0.020833in}{0.000000in}}%
\pgfusepath{stroke,fill}%
}%
\begin{pgfscope}%
\pgfsys@transformshift{4.005925in}{1.455707in}%
\pgfsys@useobject{currentmarker}{}%
\end{pgfscope}%
\end{pgfscope}%
\begin{pgfscope}%
\pgfsetrectcap%
\pgfsetmiterjoin%
\pgfsetlinewidth{0.501875pt}%
\definecolor{currentstroke}{rgb}{0.000000,0.000000,0.000000}%
\pgfsetstrokecolor{currentstroke}%
\pgfsetdash{}{0pt}%
\pgfpathmoveto{\pgfqpoint{3.936625in}{0.139007in}}%
\pgfpathlineto{\pgfqpoint{3.971275in}{0.139007in}}%
\pgfpathlineto{\pgfqpoint{4.005925in}{0.139007in}}%
\pgfpathlineto{\pgfqpoint{4.005925in}{1.525007in}}%
\pgfpathlineto{\pgfqpoint{3.971275in}{1.525007in}}%
\pgfpathlineto{\pgfqpoint{3.936625in}{1.525007in}}%
\pgfpathlineto{\pgfqpoint{3.936625in}{0.139007in}}%
\pgfpathclose%
\pgfusepath{stroke}%
\end{pgfscope}%
\end{pgfpicture}%
\makeatother%
\endgroup%

	\caption[Korrelationsmatrizen der transformierten Daten $\mat{Y}$ für den FER-Datensatz von vier Methoden]{Hier sind die Korrelationsmatrizen der transformierten Daten $\mat{Y}$ für den FER-Datensatz abgebildet, wobei in \captiona die Hauptkomponentenanalyse, in \captionb ein linearer dreischichtiger Autoencoder und in \captionc ein nichtlinearer dreischichtiger Autoencoder mit Sigmoid-Encoder und linearem Decoder verwendet wurde, um die Daten auf sieben Dimensionen zu reduzieren. Die Korrelationsmatrix bei Transformation mittels der Hauptkomponentenanalyse ist per Konstruktion eine Einheitsmatrix, das heißt es besteht keine Korrelation in $\mat{Y}$. Bei Transformation durch die Autoencoder besteht Korrelation, wobei der lineare Autoencoder zufälligerweise betragsmäßig größere Korrelationen erzeugt als der nichtlineare Autoencoder.  (Eigene Darstellung, angelehnt an \textcite[5]{Plaut.2018})}
	\label{fig:Korrelationsmatrizen}
\end{figure}
Punkt 1 ist beispielsweise in \figref{fig:Korrelationsmatrizen} zu erkennen. Dargestellt sind die Korrelationsmatrizen der transformierten Daten, wobei in \captiona die Hauptkomponentenanalyse, in \captionb ein linearer dreischichtiger Autoencoder und in \captionc ein nichtlinearer Autoencoder verwendet wurde. Die Korrelationsmatrix der Hauptkomponentenanalyse ist eine Einheitsmatrix, während die Korrelationsmatrizen der durch die Autoencoder gefundenen latenten Repräsentationen von Null verschiedene Werte auf der Nicht-diagonalen aufweisen.

Um die gefundenen Lösungen einer Hauptkomponentenanalyse und von Autoencodern genauer betrachten zu
können, kann die Ladungsmatrix $\mat{A}$ mit den Gewichtsmatrizen eines dreischichtigen
Autoencoders verglichen werden. Im Falle eines dreischichtigen Autoencoders sind die Gewichte des
Encoders eine Matrix $\mat{W}_1 \in \real^{d \times D}$ und die Gewichte des Decoders eine Matrix
$\mat{W}_2 \in \real^{D \times d}$. Auf Bilddatensätzen können diese Matrizen so umgeformt werden,
dass sich für jede Zeile von $\mat{W}_1$ oder Spalte von $\mat{W}_2$ ein Bild ergibt. Auf diese
Weise kann die \enquote{Arbeitsweise} des Autoencoders anschaulich untersucht werden.
\figref{fig:Gewichtsvergleich} setzt dieses Vorgehen auf dem FER-Datensatz um.
\begin{figure}[ht]
	\centering
	%% Creator: Matplotlib, PGF backend
%%
%% To include the figure in your LaTeX document, write
%%   \input{<filename>.pgf}
%%
%% Make sure the required packages are loaded in your preamble
%%   \usepackage{pgf}
%%
%% Also ensure that all the required font packages are loaded; for instance,
%% the lmodern package is sometimes necessary when using math font.
%%   \usepackage{lmodern}
%%
%% Figures using additional raster images can only be included by \input if
%% they are in the same directory as the main LaTeX file. For loading figures
%% from other directories you can use the `import` package
%%   \usepackage{import}
%%
%% and then include the figures with
%%   \import{<path to file>}{<filename>.pgf}
%%
%% Matplotlib used the following preamble
%%   
%%   \usepackage{fontspec}
%%   \setmainfont{DejaVuSerif.ttf}[Path=\detokenize{/Users/moritzmistol/.pyenv/versions/3.9.13/envs/thesis/lib/python3.9/site-packages/matplotlib/mpl-data/fonts/ttf/}]
%%   \setsansfont{DejaVuSans.ttf}[Path=\detokenize{/Users/moritzmistol/.pyenv/versions/3.9.13/envs/thesis/lib/python3.9/site-packages/matplotlib/mpl-data/fonts/ttf/}]
%%   \setmonofont{DejaVuSansMono.ttf}[Path=\detokenize{/Users/moritzmistol/.pyenv/versions/3.9.13/envs/thesis/lib/python3.9/site-packages/matplotlib/mpl-data/fonts/ttf/}]
%%   \makeatletter\@ifpackageloaded{underscore}{}{\usepackage[strings]{underscore}}\makeatother
%%
\begingroup%
\makeatletter%
\begin{pgfpicture}%
\pgfpathrectangle{\pgfpointorigin}{\pgfqpoint{5.544109in}{2.710000in}}%
\pgfusepath{use as bounding box, clip}%
\begin{pgfscope}%
\pgfsetbuttcap%
\pgfsetmiterjoin%
\definecolor{currentfill}{rgb}{1.000000,1.000000,1.000000}%
\pgfsetfillcolor{currentfill}%
\pgfsetlinewidth{0.000000pt}%
\definecolor{currentstroke}{rgb}{1.000000,1.000000,1.000000}%
\pgfsetstrokecolor{currentstroke}%
\pgfsetdash{}{0pt}%
\pgfpathmoveto{\pgfqpoint{-0.000000in}{0.000000in}}%
\pgfpathlineto{\pgfqpoint{5.544109in}{0.000000in}}%
\pgfpathlineto{\pgfqpoint{5.544109in}{2.710000in}}%
\pgfpathlineto{\pgfqpoint{-0.000000in}{2.710000in}}%
\pgfpathlineto{\pgfqpoint{-0.000000in}{0.000000in}}%
\pgfpathclose%
\pgfusepath{fill}%
\end{pgfscope}%
\begin{pgfscope}%
\pgfsetbuttcap%
\pgfsetmiterjoin%
\definecolor{currentfill}{rgb}{1.000000,1.000000,1.000000}%
\pgfsetfillcolor{currentfill}%
\pgfsetlinewidth{0.000000pt}%
\definecolor{currentstroke}{rgb}{1.000000,1.000000,1.000000}%
\pgfsetstrokecolor{currentstroke}%
\pgfsetdash{}{0pt}%
\pgfpathmoveto{\pgfqpoint{-0.202529in}{-0.280000in}}%
\pgfpathlineto{\pgfqpoint{1.764138in}{-0.280000in}}%
\pgfpathlineto{\pgfqpoint{1.764138in}{2.720000in}}%
\pgfpathlineto{\pgfqpoint{-0.202529in}{2.720000in}}%
\pgfpathlineto{\pgfqpoint{-0.202529in}{-0.280000in}}%
\pgfpathclose%
\pgfusepath{fill}%
\end{pgfscope}%
\begin{pgfscope}%
\pgfpathrectangle{\pgfqpoint{0.050000in}{1.680588in}}{\pgfqpoint{0.679412in}{0.679412in}}%
\pgfusepath{clip}%
\pgfsys@transformshift{0.050000in}{1.680588in}%
\pgftext[left,bottom]{\includegraphics[interpolate=true,width=0.680000in,height=0.680000in]{weights-comparison-img0.png}}%
\end{pgfscope}%
\begin{pgfscope}%
\pgfpathrectangle{\pgfqpoint{0.881364in}{1.680588in}}{\pgfqpoint{0.679412in}{0.679412in}}%
\pgfusepath{clip}%
\pgfsys@transformshift{0.881364in}{1.680588in}%
\pgftext[left,bottom]{\includegraphics[interpolate=true,width=0.680000in,height=0.680000in]{weights-comparison-img1.png}}%
\end{pgfscope}%
\begin{pgfscope}%
\pgfpathrectangle{\pgfqpoint{0.050000in}{0.865294in}}{\pgfqpoint{0.679412in}{0.679412in}}%
\pgfusepath{clip}%
\pgfsys@transformshift{0.050000in}{0.865294in}%
\pgftext[left,bottom]{\includegraphics[interpolate=true,width=0.680000in,height=0.680000in]{weights-comparison-img2.png}}%
\end{pgfscope}%
\begin{pgfscope}%
\pgfpathrectangle{\pgfqpoint{0.881364in}{0.865294in}}{\pgfqpoint{0.679412in}{0.679412in}}%
\pgfusepath{clip}%
\pgfsys@transformshift{0.881364in}{0.865294in}%
\pgftext[left,bottom]{\includegraphics[interpolate=true,width=0.680000in,height=0.680000in]{weights-comparison-img3.png}}%
\end{pgfscope}%
\begin{pgfscope}%
\pgfpathrectangle{\pgfqpoint{0.050000in}{0.050000in}}{\pgfqpoint{0.679412in}{0.679412in}}%
\pgfusepath{clip}%
\pgfsys@transformshift{0.050000in}{0.050000in}%
\pgftext[left,bottom]{\includegraphics[interpolate=true,width=0.680000in,height=0.680000in]{weights-comparison-img4.png}}%
\end{pgfscope}%
\begin{pgfscope}%
\pgfpathrectangle{\pgfqpoint{0.881364in}{0.050000in}}{\pgfqpoint{0.679412in}{0.679412in}}%
\pgfusepath{clip}%
\pgfsys@transformshift{0.881364in}{0.050000in}%
\pgftext[left,bottom]{\includegraphics[interpolate=true,width=0.680000in,height=0.680000in]{weights-comparison-img5.png}}%
\end{pgfscope}%
\begin{pgfscope}%
\definecolor{textcolor}{rgb}{0.000000,0.000000,0.000000}%
\pgfsetstrokecolor{textcolor}%
\pgfsetfillcolor{textcolor}%
\pgftext[x=0.780804in,y=2.660000in,,top]{\color{textcolor}\rmfamily\fontsize{12.000000}{14.400000}\selectfont (a)}%
\end{pgfscope}%
\begin{pgfscope}%
\pgfsetbuttcap%
\pgfsetmiterjoin%
\definecolor{currentfill}{rgb}{1.000000,1.000000,1.000000}%
\pgfsetfillcolor{currentfill}%
\pgfsetlinewidth{0.000000pt}%
\definecolor{currentstroke}{rgb}{1.000000,1.000000,1.000000}%
\pgfsetstrokecolor{currentstroke}%
\pgfsetdash{}{0pt}%
\pgfpathmoveto{\pgfqpoint{1.764138in}{-0.280000in}}%
\pgfpathlineto{\pgfqpoint{3.730804in}{-0.280000in}}%
\pgfpathlineto{\pgfqpoint{3.730804in}{2.720000in}}%
\pgfpathlineto{\pgfqpoint{1.764138in}{2.720000in}}%
\pgfpathlineto{\pgfqpoint{1.764138in}{-0.280000in}}%
\pgfpathclose%
\pgfusepath{fill}%
\end{pgfscope}%
\begin{pgfscope}%
\pgfpathrectangle{\pgfqpoint{2.016667in}{1.680588in}}{\pgfqpoint{0.679412in}{0.679412in}}%
\pgfusepath{clip}%
\pgfsys@transformshift{2.016667in}{1.680588in}%
\pgftext[left,bottom]{\includegraphics[interpolate=true,width=0.680000in,height=0.680000in]{weights-comparison-img6.png}}%
\end{pgfscope}%
\begin{pgfscope}%
\pgfpathrectangle{\pgfqpoint{2.848030in}{1.680588in}}{\pgfqpoint{0.679412in}{0.679412in}}%
\pgfusepath{clip}%
\pgfsys@transformshift{2.848030in}{1.680588in}%
\pgftext[left,bottom]{\includegraphics[interpolate=true,width=0.680000in,height=0.680000in]{weights-comparison-img7.png}}%
\end{pgfscope}%
\begin{pgfscope}%
\pgfpathrectangle{\pgfqpoint{2.016667in}{0.865294in}}{\pgfqpoint{0.679412in}{0.679412in}}%
\pgfusepath{clip}%
\pgfsys@transformshift{2.016667in}{0.865294in}%
\pgftext[left,bottom]{\includegraphics[interpolate=true,width=0.680000in,height=0.680000in]{weights-comparison-img8.png}}%
\end{pgfscope}%
\begin{pgfscope}%
\pgfpathrectangle{\pgfqpoint{2.848030in}{0.865294in}}{\pgfqpoint{0.679412in}{0.679412in}}%
\pgfusepath{clip}%
\pgfsys@transformshift{2.848030in}{0.865294in}%
\pgftext[left,bottom]{\includegraphics[interpolate=true,width=0.680000in,height=0.680000in]{weights-comparison-img9.png}}%
\end{pgfscope}%
\begin{pgfscope}%
\pgfpathrectangle{\pgfqpoint{2.016667in}{0.050000in}}{\pgfqpoint{0.679412in}{0.679412in}}%
\pgfusepath{clip}%
\pgfsys@transformshift{2.016667in}{0.050000in}%
\pgftext[left,bottom]{\includegraphics[interpolate=true,width=0.680000in,height=0.680000in]{weights-comparison-img10.png}}%
\end{pgfscope}%
\begin{pgfscope}%
\pgfpathrectangle{\pgfqpoint{2.848030in}{0.050000in}}{\pgfqpoint{0.679412in}{0.679412in}}%
\pgfusepath{clip}%
\pgfsys@transformshift{2.848030in}{0.050000in}%
\pgftext[left,bottom]{\includegraphics[interpolate=true,width=0.680000in,height=0.680000in]{weights-comparison-img11.png}}%
\end{pgfscope}%
\begin{pgfscope}%
\definecolor{textcolor}{rgb}{0.000000,0.000000,0.000000}%
\pgfsetstrokecolor{textcolor}%
\pgfsetfillcolor{textcolor}%
\pgftext[x=2.747471in,y=2.660000in,,top]{\color{textcolor}\rmfamily\fontsize{12.000000}{14.400000}\selectfont (b)}%
\end{pgfscope}%
\begin{pgfscope}%
\pgfsetbuttcap%
\pgfsetmiterjoin%
\definecolor{currentfill}{rgb}{1.000000,1.000000,1.000000}%
\pgfsetfillcolor{currentfill}%
\pgfsetlinewidth{0.000000pt}%
\definecolor{currentstroke}{rgb}{1.000000,1.000000,1.000000}%
\pgfsetstrokecolor{currentstroke}%
\pgfsetdash{}{0pt}%
\pgfpathmoveto{\pgfqpoint{3.730804in}{-0.280000in}}%
\pgfpathlineto{\pgfqpoint{5.697471in}{-0.280000in}}%
\pgfpathlineto{\pgfqpoint{5.697471in}{2.720000in}}%
\pgfpathlineto{\pgfqpoint{3.730804in}{2.720000in}}%
\pgfpathlineto{\pgfqpoint{3.730804in}{-0.280000in}}%
\pgfpathclose%
\pgfusepath{fill}%
\end{pgfscope}%
\begin{pgfscope}%
\pgfpathrectangle{\pgfqpoint{3.983333in}{1.680588in}}{\pgfqpoint{0.679412in}{0.679412in}}%
\pgfusepath{clip}%
\pgfsys@transformshift{3.983333in}{1.680588in}%
\pgftext[left,bottom]{\includegraphics[interpolate=true,width=0.680000in,height=0.680000in]{weights-comparison-img12.png}}%
\end{pgfscope}%
\begin{pgfscope}%
\pgfpathrectangle{\pgfqpoint{4.814697in}{1.680588in}}{\pgfqpoint{0.679412in}{0.679412in}}%
\pgfusepath{clip}%
\pgfsys@transformshift{4.814697in}{1.680588in}%
\pgftext[left,bottom]{\includegraphics[interpolate=true,width=0.680000in,height=0.680000in]{weights-comparison-img13.png}}%
\end{pgfscope}%
\begin{pgfscope}%
\pgfpathrectangle{\pgfqpoint{3.983333in}{0.865294in}}{\pgfqpoint{0.679412in}{0.679412in}}%
\pgfusepath{clip}%
\pgfsys@transformshift{3.983333in}{0.865294in}%
\pgftext[left,bottom]{\includegraphics[interpolate=true,width=0.680000in,height=0.680000in]{weights-comparison-img14.png}}%
\end{pgfscope}%
\begin{pgfscope}%
\pgfpathrectangle{\pgfqpoint{4.814697in}{0.865294in}}{\pgfqpoint{0.679412in}{0.679412in}}%
\pgfusepath{clip}%
\pgfsys@transformshift{4.814697in}{0.865294in}%
\pgftext[left,bottom]{\includegraphics[interpolate=true,width=0.680000in,height=0.680000in]{weights-comparison-img15.png}}%
\end{pgfscope}%
\begin{pgfscope}%
\pgfpathrectangle{\pgfqpoint{3.983333in}{0.050000in}}{\pgfqpoint{0.679412in}{0.679412in}}%
\pgfusepath{clip}%
\pgfsys@transformshift{3.983333in}{0.050000in}%
\pgftext[left,bottom]{\includegraphics[interpolate=true,width=0.680000in,height=0.680000in]{weights-comparison-img16.png}}%
\end{pgfscope}%
\begin{pgfscope}%
\pgfpathrectangle{\pgfqpoint{4.814697in}{0.050000in}}{\pgfqpoint{0.679412in}{0.679412in}}%
\pgfusepath{clip}%
\pgfsys@transformshift{4.814697in}{0.050000in}%
\pgftext[left,bottom]{\includegraphics[interpolate=true,width=0.680000in,height=0.680000in]{weights-comparison-img17.png}}%
\end{pgfscope}%
\begin{pgfscope}%
\definecolor{textcolor}{rgb}{0.000000,0.000000,0.000000}%
\pgfsetstrokecolor{textcolor}%
\pgfsetfillcolor{textcolor}%
\pgftext[x=4.714138in,y=2.660000in,,top]{\color{textcolor}\rmfamily\fontsize{12.000000}{14.400000}\selectfont (c)}%
\end{pgfscope}%
\end{pgfpicture}%
\makeatother%
\endgroup%

	\caption[Die Gewichtsmatrizen von ausgewählten Methoden auf dem FER-Datensatz]{Gezeigt sind die Gewichtsmatrizen von drei Methoden bei einer Reduktion des FER-Datensatzes auf sieben Dimensionen. Ein einzelnes Bild entspricht einer Spalte der Ladungs- beziehungsweise Gewichtsmatrix, welche in die Größe des Bildes umgeformt wurde. \captiona Die Ladungsmatrix $\mat{A}$ der Hauptkomponentenanalyse, \captionb die Decoder-Gewichtsmatrix eines linearen dreischichtigen Autoencoders und \captionc die Decoder-Gewichtsmatrix eines nichtlinearen dreischichtigen Autoencoders, wobei der Encoder eine Sigmoid-Aktivierungsfunktion einsetzt und der Decoder linear ist. (Eigene Darstellung, angelehnt an \textcite[5]{Plaut.2018})}
	\label{fig:Gewichtsvergleich}
\end{figure}
Hierbei ist in \captiona die Ladungsmatrix von PCA, in \captionb die Gewichtsmatrix $\mat{W}_2$ eines linearen dreischichtigen Autoencoders und in \captionc die Gewichtsmatrix $\mat{W}_2$ eines nichtlinearen Autoencoders dargestellt. Wie zu erkennen ist, sind
die Gewichte des linearen Autoencoders bis auf das Vorzeichen sehr ähnlich zur Ladungsmatrix von
PCA.