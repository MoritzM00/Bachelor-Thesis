\begin{table}[ht]
	\tymax=400pt
	\tymin=200pt
	\centering
	\ra{1.3}
	\begin{tabulary}{\textwidth}{lLLLLLLL}
		\toprule
		{} & \multicolumn{3}{c}{Encoder-Architektur} \\ \cmidrule{2-4} & Swiss Roll & Twin Peaks & ICMR                                                             \\ \midrule
		AE      & $\vect{x} \in \real^3
	\newline \rightarrow \text{FC}_{5} \rightarrow \text{BN} \rightarrow \text{Sigmoid}
	\newline \rightarrow \text{FC}_{10} \rightarrow \text{BN} \rightarrow \text{Sigmoid}
	\newline \rightarrow \text{FC}_{5} \rightarrow \text{BN} \rightarrow \text{Sigmoid}
	\newline \rightarrow \text{FC}_{2} \rightarrow \text{BN} \rightarrow \text{Sigmoid}$ & $\vect{x} \in \real^3
	\newline \rightarrow \text{FC}_{3} \rightarrow \text{BN} \rightarrow \text{Sigmoid}
	\newline \rightarrow \text{FC}_{2} \rightarrow \text{BN} \rightarrow \text{Sigmoid}$ & $\vect{x} \in \real^{20531}
	\newline \rightarrow \text{FC}_{128} \rightarrow \text{BN} \rightarrow \text{Sigmoid}
	\newline \rightarrow \text{FC}_{22} \rightarrow \text{BN} \rightarrow \text{Sigmoid}$          \\ \midrule
		CAE & $\vect{x} \in \real^3
	\newline \rightarrow \text{FC}_{2} \rightarrow \text{Sigmoid}$      & $\vect{x} \in \real^3
	\newline \rightarrow \text{FC}_{10} \newline \rightarrow \text{Sigmoid}
	\newline \rightarrow \text{FC}_{2} \newline \rightarrow  \text{Sigmoid}$ & $\vect{x} \in \real^{\num{20531}}
	\newline \rightarrow \text{FC}_{22} \rightarrow \text{Sigmoid}$
		\\ \bottomrule
	\end{tabulary}
	\caption[Die Architektur des Encoders des vollvernetzten und Contractive Autoencoders für den Swiss Roll-, Twin Peaks- und ICMR-Datensatz]{Die Architektur des Encoders des vollvernetzten und Contractive Autoencoders für den Swiss Roll-, Twin Peaks- und ICMR-Datensatz. Die Architektur des Decoders ist symmetrisch zum Encoder. $\text{FC}_{x}$ steht für eine vollvernetzte Schicht mit $x$ Neuronen. BN steht für \textit{Batch Normalization}.}
	\label{tab:uebersicht-autoencoder}
\end{table}