\begin{table}[ht]
	\tymax=400pt
	\tymin=300pt
	\centering
	\ra{1.3}
	\begin{tabulary}{\textwidth}{lLL}
		\toprule
		{} & \multicolumn{2}{c}{Encoder-Architektur} \\ \cmidrule{2-3} & MNIST & Fashion MNIST                                                             \\ \midrule
		AE      & A & A \\ \midrule
		CAE & A       & A  \\ \midrule
		ConvAE & \input{tables/architectures/MNIST_ConvAE.tex} & $\vect{x} \in \real^{28 \times 28} \newline
	\rightarrow \text{Conv}_{128} \rightarrow \text{BN} \rightarrow \text{ReLU}  \newline
	\rightarrow \text{Conv}_{256} \rightarrow \text{BN} \rightarrow \text{ReLU} \newline
	\rightarrow \text{Conv}_{512} \rightarrow \text{BN} \rightarrow \text{ReLU} \newline
	\rightarrow \text{Conv}_{1024} \rightarrow \text{BN} \rightarrow \text{ReLU} \newline
	\rightarrow \text{Flatten} \rightarrow \text{FC}_{18}$
		\\ \bottomrule
	\end{tabulary}
	\caption[Die Architektur des Encoders der Autoencoder für den MNIST und Fashion MNIST-Datensatz]{Die Architektur des Encoders der Autoencoder für den MNIST und Fashion MNIST-Datensatz. Die Architektur des Decoders ist symmetrisch zum Encoder. In einem Convolutional Autoencoder werden im Decoder Convolutional Schichten gegen Transpose Convolutional Schichten ausgetauscht. $\text{FC}_{x}$ steht für eine vollvernetzte Schicht mit $x$ Neuronen. BN steht für \textit{Batch Normalization}. $\text{Conv}_n$ steht für eine Convolutional Schicht mit $n$ Filter. Die Kernel-Größe der Convolutional Schichten beträgt $3 \times 3$ und die Schrittgröße ist zwei.}
	\label{tab:uebersicht-autoencoder}
\end{table}