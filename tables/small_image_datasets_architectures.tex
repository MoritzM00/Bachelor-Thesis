\begin{table}[ht]
	\tymax=400pt
	\tymin=300pt
	\centering
	\ra{1.3}
	\begin{tabulary}{\textwidth}{lLL}
		\toprule
		\multirow{2}[2]{*}{Methode} & \multicolumn{2}{c}{Encoder-Architektur} \\ \cmidrule{2-3} & Olivetti Faces & FER                                                             \\ \midrule
		AE      & $\vect{x} \in \real^{4096}
	\newline \rightarrow \text{FC}_{64} \rightarrow \text{BN} \rightarrow \text{Sigmoid}
	\newline \rightarrow \text{FC}_{9} \rightarrow \text{BN} \rightarrow \text{Sigmoid}$ & $\vect{x} \in \real^{2304}
	\newline \rightarrow \text{FC}_{256} \rightarrow \text{BN} \rightarrow \text{Sigmoid}
	\newline \rightarrow \text{FC}_{26} \rightarrow \text{BN} \rightarrow \text{Sigmoid}$ \\ \midrule
		CAE & $\vect{x} \in \real^{4096}
	\newline \rightarrow \text{FC}_{9} \rightarrow \text{Sigmoid}$       & $\vect{x} \in \real^{2304}
	\newline \rightarrow \text{FC}_{26} \rightarrow \text{Sigmoid}$ \\ \midrule
		ConvAE & $\vect{x} \in \real^{4096} \newline
	\rightarrow \text{Conv}_{32} \rightarrow \text{BN} \rightarrow \text{ReLU}  \newline
	\rightarrow \text{Conv}_{64} \rightarrow \text{BN} \rightarrow \text{ReLU}  \newline
	\rightarrow \text{Conv}_{64} \rightarrow \text{BN} \rightarrow \text{ReLU}  \newline
	\rightarrow \text{Flatten} \rightarrow \text{FC}_{9}
$ & $\vect{x} \in \real^{48 \times 48} \newline
	\rightarrow \text{Conv}_{64} \rightarrow \text{BN} \rightarrow \text{ReLU}  \newline
	\rightarrow \text{Conv}_{128} \rightarrow \text{BN} \rightarrow \text{ReLU}  \newline
	\rightarrow \text{Conv}_{256} \rightarrow \text{BN} \rightarrow \text{ReLU}  \newline
	\rightarrow \text{Flatten} \rightarrow \text{FC}_{26}
$
		\\ \bottomrule
	\end{tabulary}
	\caption[Die Architektur des Encoders der Autoencoder für den Olivetti Faces und FER-Datensatz]{Die Architektur des Encoders der Autoencoder für den Olivetti Faces und FER-Datensatz. Die Architektur des Decoders ist symmetrisch zum Encoder. Die Kernel-Größe der Convolutional Schichten beträgt $3 \times 3$ für Olivetti Faces und $4 \times 4$ für FER.}
	\label{tab:uebersicht-autoencoder-small}
\end{table}