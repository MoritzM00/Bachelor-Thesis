\def\sym#1{\ifmmode^{#1}\else\(^{#1}\)\fi}
\begin{table}[ht]
	\centering
	\ra{1.3}
	\resizebox{\textwidth}{!}{
		\begin{tabular}{l*7{S[table-align-text-post=false]}}
			\toprule
			\multirow{2}[2]{*}{Methode} & \multicolumn{7}{c}{Laufzeit in Sekunden}                                                                                                                        \\
			\cmidrule{2-8}              & {MNIST}                                  & {Fashion MNIST}  & {Twin Peaks}    & {Swiss Roll}     & {ORL}           & {ICMR}          & {FER}                    \\
			\midrule
			PCA                         & 2,00                                     & 2,54             & 0,01            & 0,00             & 0,35            & 0,78            & 2,51                     \\
			Kernel PCA                  & 101,00\sym{*}                            & 101,72\sym{*}    & 0,62            & 0,59             & 0,03            & 1,01            & 103,17\sym{*}            \\
			LLE                         & 417,41\sym{**}                           & 447,95\sym{**}   & 0,45            & 0,42             & 0,24            & 1,14            & \bfseries 426,50\sym{**} \\
			AE                          & 161,35                                   & 121,14           & \bfseries 45,32 & \bfseries 102,34 & 43,02           & 54,51           & 102,26                   \\
			CAE                         & 403,74                                   & 641,00           & 30,04           & 22,68            & 5,76            & \bfseries 93,06 & 361,72                   \\
			ConvAE                      & \bfseries 467,61                         & \bfseries 754,83 & {--}            & {--}             & \bfseries 50,20 & {--}            & 249,64                   \\
			\bottomrule
			\multicolumn{8}{l}{\footnotesize Hinweis: Die Berechnung erfolgte auf einer zufällig gezogenen Teilmenge der Größe (\sym{*}) \num{15000} bzw. (\sym{**}) \num{20000}}                         \\
		\end{tabular}
	}
	\caption[Laufzeiten der Dimensionsreduktionsmethoden in Sekunden.]{Laufzeiten der einzelnen Dimensionsreduktionsmethoden in Sekunden (niedriger ist besser). Alle statistischen Methoden wurden auf einer Intel Xeon Gold 5315Y CPU mit \num{3,20} GHz Taktfrequenz und alle Machine Learning Methoden auf einer NVIDIA RTX A4000 Grafikkarte trainiert. Die höchsten Werte pro Datensatz wurden fett markiert. PCA weist auf allen Datensätzen, inklusive der etwas größeren Datensätze MNIST und Fashion MNIST, eine kurze Laufzeit von maximal \num{2,5} Sekunden auf. Die Laufzeiten der anderen beiden statistischen Methoden Kernel PCA und LLE steigen bei Datensätzen mit einer höheren Stichprobengröße stark an. Die längsten Laufzeiten weisen der Convolutional und Contractive Autoencoder auf.}
	\label{tab:training_times}
\end{table}
