%----------------------------------------------------------------------------------------
%	Math commands
%----------------------------------------------------------------------------------------

\usepackage{amsmath}
\usepackage{amsfonts}
\usepackage{amssymb}

% Highlight a newly defined term
\newcommand{\newterm}[1]{{\bf #1}}

% Mark sections of captions for referring to divisions of figures
\newcommand{\figleft}{{\em (Left)}}
\newcommand{\figcenter}{{\em (Center)}}
\newcommand{\figright}{{\em (Right)}}
\newcommand{\figtop}{{\em (Top)}}
\newcommand{\figbottom}{{\em (Bottom)}}
\newcommand{\captiona}{{\em (a)}}
\newcommand{\captionb}{{\em (b)}}
\newcommand{\captionc}{{\em (c)}}
\newcommand{\captiond}{{\em (d)}}

% references ----------

% Figure reference
\def\figref#1{Abbildung~\ref{#1}}

% Section reference
\def\secref#1{Abschnitt~\ref{#1}}

\def\subsecref#1{Unterabschnitt~\ref{#1}}

\def\chapref#1{Kapitel~\ref{#1}}

% Reference to two sections.
\def\twosecrefs#1#2{Abschnitte \ref{#1} und \ref{#2}}
% Reference to three sections.
\def\secrefs#1#2#3{Abschnitte \ref{#1}, \ref{#2} und \ref{#3}}
% Reference to an equation
\def\eqref#1{Gleichung~\ref{#1}}
% A raw reference to an equation---avoid using if possible
\def\plaineqref#1{\ref{#1}}

% Reference to a range of chapters
\def\rangechapref#1#2{Kapitel \ref{#1}--\ref{#2}}

% Reference to an algorithm, upper case.
\def\Algref#1{Algorithmus~\ref{#1}}
\def\twoalgref#1#2{Algorithmen \ref{#1} und \ref{#2}}

% ------------



% Common matrices



% Math operators ----------

\newcommand{\ceil}[1]{\lceil #1 \rceil}
\newcommand{\floor}[1]{\lfloor #1 \rfloor}

\newcommand{\vect}[1]{\mathbf{#1}}
\newcommand{\mat}[1]{\mathbf{#1}}

\newcommand{\estWide}[1]{\widehat{#1}}
\newcommand{\estNormal}[1]{\hat{#1}}

\newcommand{\empCov}{\estWide{\mat{\Sigma}}}

\DeclareMathOperator{\diag}{diag}
\DeclareMathOperator{\Cov}{Cov}
\DeclareMathOperator{\Var}{Var}
\DeclareMathOperator{\Exp}{\mathbb{E}}

\DeclareMathOperator*{\argmax}{arg\,max}
\DeclareMathOperator*{\argmin}{arg\,min}