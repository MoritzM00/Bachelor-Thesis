%----------------------------------------------------------------------------------------
%	Math commands
%----------------------------------------------------------------------------------------

\usepackage{amsmath}
\usepackage{amsfonts}
\usepackage{amssymb}
\usepackage{bm}
\usepackage{mathtools}

% Highlight a newly defined term
\newcommand{\newterm}[1]{{\textbf{#1}}}

% Mark sections of captions for referring to divisions of figures
\newcommand{\figleft}{{\em (Links)}\;}
\newcommand{\figcenter}{{\em (Mitte)}\;}
\newcommand{\figright}{{\em (Rechts)}\;}
\newcommand{\figtop}{{\em (Oben)}\;}
\newcommand{\figbottom}{{\em (Unten)}\;}
\newcommand{\captiona}{{\em (a)}\;}
\newcommand{\captionb}{{\em (b)}\;}
\newcommand{\captionc}{{\em (c)}\;}
\newcommand{\captiond}{{\em (d)}\;}

% references ----------

% Figure reference
\def\figref#1{Abbildung~\ref{#1}}

% Section reference
\def\secref#1{Abschnitt~\ref{#1}}

\def\subsecref#1{Unterabschnitt~\ref{#1}}

\def\chapref#1{Kapitel~\ref{#1}}

% Reference to two sections.
\def\twosecrefs#1#2{Abschnitte \ref{#1} und \ref{#2}}
% Reference to three sections.
\def\secrefs#1#2#3{Abschnitte \ref{#1}, \ref{#2} und \ref{#3}}
% Reference to an equation
\def\eqref#1{Gleichung~\ref{#1}}
% A raw reference to an equation---avoid using if possible
\def\plaineqref#1{\ref{#1}}

\def\tabref#1{Tabelle~\ref{#1}}

% Reference to a range of chapters
\def\rangechapref#1#2{Kapitel \ref{#1}--\ref{#2}}

% Reference to an algorithm, upper case.
\def\Algref#1{Algorithmus~\ref{#1}}
\def\twoalgref#1#2{Algorithmen \ref{#1} und \ref{#2}}

\def\expnum#1#2{#1 \mathrm{e}{#2}}

% ------------

% Common matrices
\newcommand{\datamatrix}{\mat{X}}

\newcommand{\identity}{\mat{I}}

% common vectors
\newcommand{\ones}{\vect{1}}

% Math operators ----------

\newcommand{\ceil}[1]{\lceil #1 \rceil}
\newcommand{\floor}[1]{\lfloor #1 \rfloor}

\newcommand{\tr}[1]{#1^\top}
\DeclarePairedDelimiter{\norm}{\lVert}{\rVert}

\DeclareMathOperator{\diag}{diag}
\DeclareMathOperator{\Cov}{Cov}
\DeclareMathOperator{\Var}{Var}
\DeclareMathOperator{\Exp}{\mathbb{E}}

\DeclareMathOperator{\lcmc}{LCMC}

\DeclareMathOperator{\Spur}{Spur}

\DeclareMathOperator*{\argmax}{arg\,max}
\DeclareMathOperator*{\argmin}{arg\,min}

% real-valued
%\newcommand{\sc}[1]{\textnormal{#1}}
\newcommand{\vect}[1]{\bm{#1}}
\newcommand{\mat}[1]{\mathbf{#1}}

% random variables
\newcommand{\rv}[1]{#1}
\newcommand{\rvect}[1]{\bm{#1}}
\newcommand{\rmat}[1]{\bm{#1}}

% others
\newcommand{\estWide}[1]{\widehat{#1}}
\newcommand{\estNormal}[1]{\hat{#1}}

\newcommand{\empCov}{\estWide{\mat{\Sigma}}}

% neighborhood shortcuts (for id, lle and quality criteria)
\newcommand{\Klle}{K_{\text{LLE}}}
\newcommand{\Kid}{K_{\text{ID}}}
\newcommand{\Kqk}{K_{\text{QK}}}
% for KL-Divergence
\DeclarePairedDelimiterX{\infdivx}[2]{(}{)}{%
	#1\;\delimsize\|\;#2%
}
\newcommand{\KLDiv}{D_{\textnormal{KL}}\infdivx}
% usage of KL Div: \KLDiv[\bigg] for fixed-size delimiter, \KLDiv* for extensible size

% sets
\newcommand{\set}[1]{\mathbb{#1}}

\newcommand{\real}{\set{R}}
\newcommand{\N}{\set{N}}

% common distributions
\newcommand{\gaussian}{\mathcal{N}}