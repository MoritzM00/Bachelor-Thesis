% code adapted from https://tex.stackexchange.com/questions/255159/help-on-drawing-a-tree-in-latex?rq=1
\tikzset{
	my node/.style={
			draw=gray,
			inner color=gray!5,
			outer color=gray!10,
			thick,
			minimum width=1cm,
			rounded corners=3,
			text height=1.5ex,
			text depth=0ex,
		}
}
\begin{figure}[h]
	\centering
	\begin{forest}
		for tree={%
		my node,
		l sep+=5pt,
		grow'=south,
		edge={gray, thick},
		parent anchor=south,
		child anchor=north,
		fit=band,
		if n children=0{tier=last}{},
		if={isodd(n_children())}{
				for children={
						if={equal(n,(n_children("!u")+1)/2)}{calign with current}{}
					}
			}{}
		}
		[Dimensionsreduktion, s sep=15mm,
		[Konvex
			[nicht-vollwertige EWZ [LLE]] [vollwertige EWZ [Kernel PCA] [PCA]]]
		[Nicht-konvex
		[CAE] [AE]]
		]
	\end{forest}
	\caption[Alternative Kategorisierung der Dimensionsreduktionsmethoden]{Ein Auszug aus der Kategorisierung nach \textcite{vanderMaaten.2009}. Eine vollwertige Eigenwertzerlegung (EWZ) zerlegt eine (dichte) Matrix in ihre Eigenwerte und -vektoren. Eine nicht-vollwertige Eigenwertzerlegung meint die Zerlegung einer dünn besetzten Matrix \parencite[3,7]{vanderMaaten.2009}.} \label{fig:KategorisierungMaaten}
\end{figure}