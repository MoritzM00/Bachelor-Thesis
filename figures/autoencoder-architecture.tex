\def\a{3}  % width of trapezium
\def\b{.9} % small height of trapezium
\def\c{2}  % tall height of trapezium
\tikzset
{
myTrapezium/.pic =
	{
		\draw [fill=gray!30] (0,0) -- (0,\b) -- (\a,\c) -- (\a,-\c) -- (0,-\b) -- cycle ;
		\coordinate (-center) at (\a/2,0);
		\coordinate (-out) at (\a,0);
	},
myArrows/.style=
{
line width=0.4mm,
black,
-{Triangle[length=2mm,width=3mm]},
shorten >=2pt,
shorten <=2pt,
}
}
\begin{figure}[h]
	\centering
	\begin{tikzpicture}
		[
			node distance=1mm, % space between drawn parts
			every node/.style={align=center},
		]

		\node (middleThing)
		[
			draw,
			fill=gray!10,
			%minimum width=1cm,
			minimum height=2*\b cm, font=\tiny, ]
		%{\begin{tabular}{r}-0.2 \\ -0.1 \\ 0.1 \\ 0.4 \\ -0.3 \\ 1.1\end{tabular}};
		{$\vect{y} \in \real^d$};
		\pic (right)[right=of middleThing.east] {myTrapezium} ;
		\pic (left)[left=of middleThing.west, rotate=180] {myTrapezium} ;
		\node at (right-center) {Decoder} ;
		\node at (left-center) {Encoder};
		\node at (right-center) {Decoder} ;

		\def\d{.9}
		\coordinate (u) at (\d,0);
		\draw [myArrows] (right-out) -- ++(u) node [anchor=west] {Output $\estNormal{\vect{x}} \in \real^D$};
		\draw [myArrows] ($(left-out)-(u)$) node [anchor=east] {Input $\vect{x} \in \real^D$} -- ++(u) ;

	\end{tikzpicture}
	\caption[Schematische Architektur eines Autoencoders]{Schematische Architektur eines Autoencoders. Der Inputvektor $\vect{x} \in \real^D$ wird durch den Encoder in eine niedrigdimensionale Repräsentation $\vect{y} \in \real^d$ kodiert. Der Decoder nimmt diese niedrigdimensionale Repräsentation und liefert ein Outputvektor $\estNormal{\vect{x}}$, der möglichst dem ursprünglichen Vektor $\vect{x}$ entspricht, d.h. es soll $\estNormal{\vect{x}} \approx \vect{x}$ gelten. Eigene Darstellung, angelehnt an \textcite[15]{Luong.2016}}
	\label{fig:SchematischerAutoencoder}
\end{figure}