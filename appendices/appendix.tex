%% appendix.tex
%%

\chapter{Appendix}
\label{ch:Appendix}

\section{Autoencoder Architektur und Training Details}
\label{ch:Appendix:Architektur-Details}

\dots

\section{Qualitätskriterien für die Datensätze}
\label{ch:Appendix:Qualitaetskriterien}
\begin{figure}[ht]
	\begin{center}
		%% Creator: Matplotlib, PGF backend
%%
%% To include the figure in your LaTeX document, write
%%   \input{<filename>.pgf}
%%
%% Make sure the required packages are loaded in your preamble
%%   \usepackage{pgf}
%%
%% Also ensure that all the required font packages are loaded; for instance,
%% the lmodern package is sometimes necessary when using math font.
%%   \usepackage{lmodern}
%%
%% Figures using additional raster images can only be included by \input if
%% they are in the same directory as the main LaTeX file. For loading figures
%% from other directories you can use the `import` package
%%   \usepackage{import}
%%
%% and then include the figures with
%%   \import{<path to file>}{<filename>.pgf}
%%
%% Matplotlib used the following preamble
%%   
%%   \usepackage{fontspec}
%%   \setmainfont{DejaVuSerif.ttf}[Path=\detokenize{/Users/moritzmistol/.pyenv/versions/3.9.13/envs/thesis/lib/python3.9/site-packages/matplotlib/mpl-data/fonts/ttf/}]
%%   \setsansfont{DejaVuSans.ttf}[Path=\detokenize{/Users/moritzmistol/.pyenv/versions/3.9.13/envs/thesis/lib/python3.9/site-packages/matplotlib/mpl-data/fonts/ttf/}]
%%   \setmonofont{DejaVuSansMono.ttf}[Path=\detokenize{/Users/moritzmistol/.pyenv/versions/3.9.13/envs/thesis/lib/python3.9/site-packages/matplotlib/mpl-data/fonts/ttf/}]
%%   \makeatletter\@ifpackageloaded{underscore}{}{\usepackage[strings]{underscore}}\makeatother
%%
\begingroup%
\makeatletter%
\begin{pgfpicture}%
\pgfpathrectangle{\pgfpointorigin}{\pgfqpoint{5.711701in}{4.634154in}}%
\pgfusepath{use as bounding box, clip}%
\begin{pgfscope}%
\pgfsetbuttcap%
\pgfsetmiterjoin%
\definecolor{currentfill}{rgb}{1.000000,1.000000,1.000000}%
\pgfsetfillcolor{currentfill}%
\pgfsetlinewidth{0.000000pt}%
\definecolor{currentstroke}{rgb}{1.000000,1.000000,1.000000}%
\pgfsetstrokecolor{currentstroke}%
\pgfsetdash{}{0pt}%
\pgfpathmoveto{\pgfqpoint{-0.000000in}{-0.000000in}}%
\pgfpathlineto{\pgfqpoint{5.711701in}{-0.000000in}}%
\pgfpathlineto{\pgfqpoint{5.711701in}{4.634154in}}%
\pgfpathlineto{\pgfqpoint{-0.000000in}{4.634154in}}%
\pgfpathlineto{\pgfqpoint{-0.000000in}{-0.000000in}}%
\pgfpathclose%
\pgfusepath{fill}%
\end{pgfscope}%
\begin{pgfscope}%
\pgfsetbuttcap%
\pgfsetmiterjoin%
\definecolor{currentfill}{rgb}{1.000000,1.000000,1.000000}%
\pgfsetfillcolor{currentfill}%
\pgfsetlinewidth{0.000000pt}%
\definecolor{currentstroke}{rgb}{0.000000,0.000000,0.000000}%
\pgfsetstrokecolor{currentstroke}%
\pgfsetstrokeopacity{0.000000}%
\pgfsetdash{}{0pt}%
\pgfpathmoveto{\pgfqpoint{0.609415in}{2.747992in}}%
\pgfpathlineto{\pgfqpoint{2.850886in}{2.747992in}}%
\pgfpathlineto{\pgfqpoint{2.850886in}{4.374193in}}%
\pgfpathlineto{\pgfqpoint{0.609415in}{4.374193in}}%
\pgfpathlineto{\pgfqpoint{0.609415in}{2.747992in}}%
\pgfpathclose%
\pgfusepath{fill}%
\end{pgfscope}%
\begin{pgfscope}%
\pgfsetbuttcap%
\pgfsetroundjoin%
\definecolor{currentfill}{rgb}{0.000000,0.000000,0.000000}%
\pgfsetfillcolor{currentfill}%
\pgfsetlinewidth{0.501875pt}%
\definecolor{currentstroke}{rgb}{0.000000,0.000000,0.000000}%
\pgfsetstrokecolor{currentstroke}%
\pgfsetdash{}{0pt}%
\pgfsys@defobject{currentmarker}{\pgfqpoint{0.000000in}{0.000000in}}{\pgfqpoint{0.000000in}{0.041667in}}{%
\pgfpathmoveto{\pgfqpoint{0.000000in}{0.000000in}}%
\pgfpathlineto{\pgfqpoint{0.000000in}{0.041667in}}%
\pgfusepath{stroke,fill}%
}%
\begin{pgfscope}%
\pgfsys@transformshift{0.609415in}{2.747992in}%
\pgfsys@useobject{currentmarker}{}%
\end{pgfscope}%
\end{pgfscope}%
\begin{pgfscope}%
\pgfsetbuttcap%
\pgfsetroundjoin%
\definecolor{currentfill}{rgb}{0.000000,0.000000,0.000000}%
\pgfsetfillcolor{currentfill}%
\pgfsetlinewidth{0.501875pt}%
\definecolor{currentstroke}{rgb}{0.000000,0.000000,0.000000}%
\pgfsetstrokecolor{currentstroke}%
\pgfsetdash{}{0pt}%
\pgfsys@defobject{currentmarker}{\pgfqpoint{0.000000in}{-0.041667in}}{\pgfqpoint{0.000000in}{0.000000in}}{%
\pgfpathmoveto{\pgfqpoint{0.000000in}{0.000000in}}%
\pgfpathlineto{\pgfqpoint{0.000000in}{-0.041667in}}%
\pgfusepath{stroke,fill}%
}%
\begin{pgfscope}%
\pgfsys@transformshift{0.609415in}{4.374193in}%
\pgfsys@useobject{currentmarker}{}%
\end{pgfscope}%
\end{pgfscope}%
\begin{pgfscope}%
\definecolor{textcolor}{rgb}{0.000000,0.000000,0.000000}%
\pgfsetstrokecolor{textcolor}%
\pgfsetfillcolor{textcolor}%
\pgftext[x=0.609415in,y=2.699381in,,top]{\color{textcolor}\rmfamily\fontsize{10.000000}{12.000000}\selectfont \(\displaystyle {0}\)}%
\end{pgfscope}%
\begin{pgfscope}%
\pgfsetbuttcap%
\pgfsetroundjoin%
\definecolor{currentfill}{rgb}{0.000000,0.000000,0.000000}%
\pgfsetfillcolor{currentfill}%
\pgfsetlinewidth{0.501875pt}%
\definecolor{currentstroke}{rgb}{0.000000,0.000000,0.000000}%
\pgfsetstrokecolor{currentstroke}%
\pgfsetdash{}{0pt}%
\pgfsys@defobject{currentmarker}{\pgfqpoint{0.000000in}{0.000000in}}{\pgfqpoint{0.000000in}{0.041667in}}{%
\pgfpathmoveto{\pgfqpoint{0.000000in}{0.000000in}}%
\pgfpathlineto{\pgfqpoint{0.000000in}{0.041667in}}%
\pgfusepath{stroke,fill}%
}%
\begin{pgfscope}%
\pgfsys@transformshift{1.053270in}{2.747992in}%
\pgfsys@useobject{currentmarker}{}%
\end{pgfscope}%
\end{pgfscope}%
\begin{pgfscope}%
\pgfsetbuttcap%
\pgfsetroundjoin%
\definecolor{currentfill}{rgb}{0.000000,0.000000,0.000000}%
\pgfsetfillcolor{currentfill}%
\pgfsetlinewidth{0.501875pt}%
\definecolor{currentstroke}{rgb}{0.000000,0.000000,0.000000}%
\pgfsetstrokecolor{currentstroke}%
\pgfsetdash{}{0pt}%
\pgfsys@defobject{currentmarker}{\pgfqpoint{0.000000in}{-0.041667in}}{\pgfqpoint{0.000000in}{0.000000in}}{%
\pgfpathmoveto{\pgfqpoint{0.000000in}{0.000000in}}%
\pgfpathlineto{\pgfqpoint{0.000000in}{-0.041667in}}%
\pgfusepath{stroke,fill}%
}%
\begin{pgfscope}%
\pgfsys@transformshift{1.053270in}{4.374193in}%
\pgfsys@useobject{currentmarker}{}%
\end{pgfscope}%
\end{pgfscope}%
\begin{pgfscope}%
\definecolor{textcolor}{rgb}{0.000000,0.000000,0.000000}%
\pgfsetstrokecolor{textcolor}%
\pgfsetfillcolor{textcolor}%
\pgftext[x=1.053270in,y=2.699381in,,top]{\color{textcolor}\rmfamily\fontsize{10.000000}{12.000000}\selectfont \(\displaystyle {20}\)}%
\end{pgfscope}%
\begin{pgfscope}%
\pgfsetbuttcap%
\pgfsetroundjoin%
\definecolor{currentfill}{rgb}{0.000000,0.000000,0.000000}%
\pgfsetfillcolor{currentfill}%
\pgfsetlinewidth{0.501875pt}%
\definecolor{currentstroke}{rgb}{0.000000,0.000000,0.000000}%
\pgfsetstrokecolor{currentstroke}%
\pgfsetdash{}{0pt}%
\pgfsys@defobject{currentmarker}{\pgfqpoint{0.000000in}{0.000000in}}{\pgfqpoint{0.000000in}{0.041667in}}{%
\pgfpathmoveto{\pgfqpoint{0.000000in}{0.000000in}}%
\pgfpathlineto{\pgfqpoint{0.000000in}{0.041667in}}%
\pgfusepath{stroke,fill}%
}%
\begin{pgfscope}%
\pgfsys@transformshift{1.497126in}{2.747992in}%
\pgfsys@useobject{currentmarker}{}%
\end{pgfscope}%
\end{pgfscope}%
\begin{pgfscope}%
\pgfsetbuttcap%
\pgfsetroundjoin%
\definecolor{currentfill}{rgb}{0.000000,0.000000,0.000000}%
\pgfsetfillcolor{currentfill}%
\pgfsetlinewidth{0.501875pt}%
\definecolor{currentstroke}{rgb}{0.000000,0.000000,0.000000}%
\pgfsetstrokecolor{currentstroke}%
\pgfsetdash{}{0pt}%
\pgfsys@defobject{currentmarker}{\pgfqpoint{0.000000in}{-0.041667in}}{\pgfqpoint{0.000000in}{0.000000in}}{%
\pgfpathmoveto{\pgfqpoint{0.000000in}{0.000000in}}%
\pgfpathlineto{\pgfqpoint{0.000000in}{-0.041667in}}%
\pgfusepath{stroke,fill}%
}%
\begin{pgfscope}%
\pgfsys@transformshift{1.497126in}{4.374193in}%
\pgfsys@useobject{currentmarker}{}%
\end{pgfscope}%
\end{pgfscope}%
\begin{pgfscope}%
\definecolor{textcolor}{rgb}{0.000000,0.000000,0.000000}%
\pgfsetstrokecolor{textcolor}%
\pgfsetfillcolor{textcolor}%
\pgftext[x=1.497126in,y=2.699381in,,top]{\color{textcolor}\rmfamily\fontsize{10.000000}{12.000000}\selectfont \(\displaystyle {40}\)}%
\end{pgfscope}%
\begin{pgfscope}%
\pgfsetbuttcap%
\pgfsetroundjoin%
\definecolor{currentfill}{rgb}{0.000000,0.000000,0.000000}%
\pgfsetfillcolor{currentfill}%
\pgfsetlinewidth{0.501875pt}%
\definecolor{currentstroke}{rgb}{0.000000,0.000000,0.000000}%
\pgfsetstrokecolor{currentstroke}%
\pgfsetdash{}{0pt}%
\pgfsys@defobject{currentmarker}{\pgfqpoint{0.000000in}{0.000000in}}{\pgfqpoint{0.000000in}{0.041667in}}{%
\pgfpathmoveto{\pgfqpoint{0.000000in}{0.000000in}}%
\pgfpathlineto{\pgfqpoint{0.000000in}{0.041667in}}%
\pgfusepath{stroke,fill}%
}%
\begin{pgfscope}%
\pgfsys@transformshift{1.940982in}{2.747992in}%
\pgfsys@useobject{currentmarker}{}%
\end{pgfscope}%
\end{pgfscope}%
\begin{pgfscope}%
\pgfsetbuttcap%
\pgfsetroundjoin%
\definecolor{currentfill}{rgb}{0.000000,0.000000,0.000000}%
\pgfsetfillcolor{currentfill}%
\pgfsetlinewidth{0.501875pt}%
\definecolor{currentstroke}{rgb}{0.000000,0.000000,0.000000}%
\pgfsetstrokecolor{currentstroke}%
\pgfsetdash{}{0pt}%
\pgfsys@defobject{currentmarker}{\pgfqpoint{0.000000in}{-0.041667in}}{\pgfqpoint{0.000000in}{0.000000in}}{%
\pgfpathmoveto{\pgfqpoint{0.000000in}{0.000000in}}%
\pgfpathlineto{\pgfqpoint{0.000000in}{-0.041667in}}%
\pgfusepath{stroke,fill}%
}%
\begin{pgfscope}%
\pgfsys@transformshift{1.940982in}{4.374193in}%
\pgfsys@useobject{currentmarker}{}%
\end{pgfscope}%
\end{pgfscope}%
\begin{pgfscope}%
\definecolor{textcolor}{rgb}{0.000000,0.000000,0.000000}%
\pgfsetstrokecolor{textcolor}%
\pgfsetfillcolor{textcolor}%
\pgftext[x=1.940982in,y=2.699381in,,top]{\color{textcolor}\rmfamily\fontsize{10.000000}{12.000000}\selectfont \(\displaystyle {60}\)}%
\end{pgfscope}%
\begin{pgfscope}%
\pgfsetbuttcap%
\pgfsetroundjoin%
\definecolor{currentfill}{rgb}{0.000000,0.000000,0.000000}%
\pgfsetfillcolor{currentfill}%
\pgfsetlinewidth{0.501875pt}%
\definecolor{currentstroke}{rgb}{0.000000,0.000000,0.000000}%
\pgfsetstrokecolor{currentstroke}%
\pgfsetdash{}{0pt}%
\pgfsys@defobject{currentmarker}{\pgfqpoint{0.000000in}{0.000000in}}{\pgfqpoint{0.000000in}{0.041667in}}{%
\pgfpathmoveto{\pgfqpoint{0.000000in}{0.000000in}}%
\pgfpathlineto{\pgfqpoint{0.000000in}{0.041667in}}%
\pgfusepath{stroke,fill}%
}%
\begin{pgfscope}%
\pgfsys@transformshift{2.384838in}{2.747992in}%
\pgfsys@useobject{currentmarker}{}%
\end{pgfscope}%
\end{pgfscope}%
\begin{pgfscope}%
\pgfsetbuttcap%
\pgfsetroundjoin%
\definecolor{currentfill}{rgb}{0.000000,0.000000,0.000000}%
\pgfsetfillcolor{currentfill}%
\pgfsetlinewidth{0.501875pt}%
\definecolor{currentstroke}{rgb}{0.000000,0.000000,0.000000}%
\pgfsetstrokecolor{currentstroke}%
\pgfsetdash{}{0pt}%
\pgfsys@defobject{currentmarker}{\pgfqpoint{0.000000in}{-0.041667in}}{\pgfqpoint{0.000000in}{0.000000in}}{%
\pgfpathmoveto{\pgfqpoint{0.000000in}{0.000000in}}%
\pgfpathlineto{\pgfqpoint{0.000000in}{-0.041667in}}%
\pgfusepath{stroke,fill}%
}%
\begin{pgfscope}%
\pgfsys@transformshift{2.384838in}{4.374193in}%
\pgfsys@useobject{currentmarker}{}%
\end{pgfscope}%
\end{pgfscope}%
\begin{pgfscope}%
\definecolor{textcolor}{rgb}{0.000000,0.000000,0.000000}%
\pgfsetstrokecolor{textcolor}%
\pgfsetfillcolor{textcolor}%
\pgftext[x=2.384838in,y=2.699381in,,top]{\color{textcolor}\rmfamily\fontsize{10.000000}{12.000000}\selectfont \(\displaystyle {80}\)}%
\end{pgfscope}%
\begin{pgfscope}%
\pgfsetbuttcap%
\pgfsetroundjoin%
\definecolor{currentfill}{rgb}{0.000000,0.000000,0.000000}%
\pgfsetfillcolor{currentfill}%
\pgfsetlinewidth{0.501875pt}%
\definecolor{currentstroke}{rgb}{0.000000,0.000000,0.000000}%
\pgfsetstrokecolor{currentstroke}%
\pgfsetdash{}{0pt}%
\pgfsys@defobject{currentmarker}{\pgfqpoint{0.000000in}{0.000000in}}{\pgfqpoint{0.000000in}{0.020833in}}{%
\pgfpathmoveto{\pgfqpoint{0.000000in}{0.000000in}}%
\pgfpathlineto{\pgfqpoint{0.000000in}{0.020833in}}%
\pgfusepath{stroke,fill}%
}%
\begin{pgfscope}%
\pgfsys@transformshift{0.720379in}{2.747992in}%
\pgfsys@useobject{currentmarker}{}%
\end{pgfscope}%
\end{pgfscope}%
\begin{pgfscope}%
\pgfsetbuttcap%
\pgfsetroundjoin%
\definecolor{currentfill}{rgb}{0.000000,0.000000,0.000000}%
\pgfsetfillcolor{currentfill}%
\pgfsetlinewidth{0.501875pt}%
\definecolor{currentstroke}{rgb}{0.000000,0.000000,0.000000}%
\pgfsetstrokecolor{currentstroke}%
\pgfsetdash{}{0pt}%
\pgfsys@defobject{currentmarker}{\pgfqpoint{0.000000in}{-0.020833in}}{\pgfqpoint{0.000000in}{0.000000in}}{%
\pgfpathmoveto{\pgfqpoint{0.000000in}{0.000000in}}%
\pgfpathlineto{\pgfqpoint{0.000000in}{-0.020833in}}%
\pgfusepath{stroke,fill}%
}%
\begin{pgfscope}%
\pgfsys@transformshift{0.720379in}{4.374193in}%
\pgfsys@useobject{currentmarker}{}%
\end{pgfscope}%
\end{pgfscope}%
\begin{pgfscope}%
\pgfsetbuttcap%
\pgfsetroundjoin%
\definecolor{currentfill}{rgb}{0.000000,0.000000,0.000000}%
\pgfsetfillcolor{currentfill}%
\pgfsetlinewidth{0.501875pt}%
\definecolor{currentstroke}{rgb}{0.000000,0.000000,0.000000}%
\pgfsetstrokecolor{currentstroke}%
\pgfsetdash{}{0pt}%
\pgfsys@defobject{currentmarker}{\pgfqpoint{0.000000in}{0.000000in}}{\pgfqpoint{0.000000in}{0.020833in}}{%
\pgfpathmoveto{\pgfqpoint{0.000000in}{0.000000in}}%
\pgfpathlineto{\pgfqpoint{0.000000in}{0.020833in}}%
\pgfusepath{stroke,fill}%
}%
\begin{pgfscope}%
\pgfsys@transformshift{0.831342in}{2.747992in}%
\pgfsys@useobject{currentmarker}{}%
\end{pgfscope}%
\end{pgfscope}%
\begin{pgfscope}%
\pgfsetbuttcap%
\pgfsetroundjoin%
\definecolor{currentfill}{rgb}{0.000000,0.000000,0.000000}%
\pgfsetfillcolor{currentfill}%
\pgfsetlinewidth{0.501875pt}%
\definecolor{currentstroke}{rgb}{0.000000,0.000000,0.000000}%
\pgfsetstrokecolor{currentstroke}%
\pgfsetdash{}{0pt}%
\pgfsys@defobject{currentmarker}{\pgfqpoint{0.000000in}{-0.020833in}}{\pgfqpoint{0.000000in}{0.000000in}}{%
\pgfpathmoveto{\pgfqpoint{0.000000in}{0.000000in}}%
\pgfpathlineto{\pgfqpoint{0.000000in}{-0.020833in}}%
\pgfusepath{stroke,fill}%
}%
\begin{pgfscope}%
\pgfsys@transformshift{0.831342in}{4.374193in}%
\pgfsys@useobject{currentmarker}{}%
\end{pgfscope}%
\end{pgfscope}%
\begin{pgfscope}%
\pgfsetbuttcap%
\pgfsetroundjoin%
\definecolor{currentfill}{rgb}{0.000000,0.000000,0.000000}%
\pgfsetfillcolor{currentfill}%
\pgfsetlinewidth{0.501875pt}%
\definecolor{currentstroke}{rgb}{0.000000,0.000000,0.000000}%
\pgfsetstrokecolor{currentstroke}%
\pgfsetdash{}{0pt}%
\pgfsys@defobject{currentmarker}{\pgfqpoint{0.000000in}{0.000000in}}{\pgfqpoint{0.000000in}{0.020833in}}{%
\pgfpathmoveto{\pgfqpoint{0.000000in}{0.000000in}}%
\pgfpathlineto{\pgfqpoint{0.000000in}{0.020833in}}%
\pgfusepath{stroke,fill}%
}%
\begin{pgfscope}%
\pgfsys@transformshift{0.942306in}{2.747992in}%
\pgfsys@useobject{currentmarker}{}%
\end{pgfscope}%
\end{pgfscope}%
\begin{pgfscope}%
\pgfsetbuttcap%
\pgfsetroundjoin%
\definecolor{currentfill}{rgb}{0.000000,0.000000,0.000000}%
\pgfsetfillcolor{currentfill}%
\pgfsetlinewidth{0.501875pt}%
\definecolor{currentstroke}{rgb}{0.000000,0.000000,0.000000}%
\pgfsetstrokecolor{currentstroke}%
\pgfsetdash{}{0pt}%
\pgfsys@defobject{currentmarker}{\pgfqpoint{0.000000in}{-0.020833in}}{\pgfqpoint{0.000000in}{0.000000in}}{%
\pgfpathmoveto{\pgfqpoint{0.000000in}{0.000000in}}%
\pgfpathlineto{\pgfqpoint{0.000000in}{-0.020833in}}%
\pgfusepath{stroke,fill}%
}%
\begin{pgfscope}%
\pgfsys@transformshift{0.942306in}{4.374193in}%
\pgfsys@useobject{currentmarker}{}%
\end{pgfscope}%
\end{pgfscope}%
\begin{pgfscope}%
\pgfsetbuttcap%
\pgfsetroundjoin%
\definecolor{currentfill}{rgb}{0.000000,0.000000,0.000000}%
\pgfsetfillcolor{currentfill}%
\pgfsetlinewidth{0.501875pt}%
\definecolor{currentstroke}{rgb}{0.000000,0.000000,0.000000}%
\pgfsetstrokecolor{currentstroke}%
\pgfsetdash{}{0pt}%
\pgfsys@defobject{currentmarker}{\pgfqpoint{0.000000in}{0.000000in}}{\pgfqpoint{0.000000in}{0.020833in}}{%
\pgfpathmoveto{\pgfqpoint{0.000000in}{0.000000in}}%
\pgfpathlineto{\pgfqpoint{0.000000in}{0.020833in}}%
\pgfusepath{stroke,fill}%
}%
\begin{pgfscope}%
\pgfsys@transformshift{1.164234in}{2.747992in}%
\pgfsys@useobject{currentmarker}{}%
\end{pgfscope}%
\end{pgfscope}%
\begin{pgfscope}%
\pgfsetbuttcap%
\pgfsetroundjoin%
\definecolor{currentfill}{rgb}{0.000000,0.000000,0.000000}%
\pgfsetfillcolor{currentfill}%
\pgfsetlinewidth{0.501875pt}%
\definecolor{currentstroke}{rgb}{0.000000,0.000000,0.000000}%
\pgfsetstrokecolor{currentstroke}%
\pgfsetdash{}{0pt}%
\pgfsys@defobject{currentmarker}{\pgfqpoint{0.000000in}{-0.020833in}}{\pgfqpoint{0.000000in}{0.000000in}}{%
\pgfpathmoveto{\pgfqpoint{0.000000in}{0.000000in}}%
\pgfpathlineto{\pgfqpoint{0.000000in}{-0.020833in}}%
\pgfusepath{stroke,fill}%
}%
\begin{pgfscope}%
\pgfsys@transformshift{1.164234in}{4.374193in}%
\pgfsys@useobject{currentmarker}{}%
\end{pgfscope}%
\end{pgfscope}%
\begin{pgfscope}%
\pgfsetbuttcap%
\pgfsetroundjoin%
\definecolor{currentfill}{rgb}{0.000000,0.000000,0.000000}%
\pgfsetfillcolor{currentfill}%
\pgfsetlinewidth{0.501875pt}%
\definecolor{currentstroke}{rgb}{0.000000,0.000000,0.000000}%
\pgfsetstrokecolor{currentstroke}%
\pgfsetdash{}{0pt}%
\pgfsys@defobject{currentmarker}{\pgfqpoint{0.000000in}{0.000000in}}{\pgfqpoint{0.000000in}{0.020833in}}{%
\pgfpathmoveto{\pgfqpoint{0.000000in}{0.000000in}}%
\pgfpathlineto{\pgfqpoint{0.000000in}{0.020833in}}%
\pgfusepath{stroke,fill}%
}%
\begin{pgfscope}%
\pgfsys@transformshift{1.275198in}{2.747992in}%
\pgfsys@useobject{currentmarker}{}%
\end{pgfscope}%
\end{pgfscope}%
\begin{pgfscope}%
\pgfsetbuttcap%
\pgfsetroundjoin%
\definecolor{currentfill}{rgb}{0.000000,0.000000,0.000000}%
\pgfsetfillcolor{currentfill}%
\pgfsetlinewidth{0.501875pt}%
\definecolor{currentstroke}{rgb}{0.000000,0.000000,0.000000}%
\pgfsetstrokecolor{currentstroke}%
\pgfsetdash{}{0pt}%
\pgfsys@defobject{currentmarker}{\pgfqpoint{0.000000in}{-0.020833in}}{\pgfqpoint{0.000000in}{0.000000in}}{%
\pgfpathmoveto{\pgfqpoint{0.000000in}{0.000000in}}%
\pgfpathlineto{\pgfqpoint{0.000000in}{-0.020833in}}%
\pgfusepath{stroke,fill}%
}%
\begin{pgfscope}%
\pgfsys@transformshift{1.275198in}{4.374193in}%
\pgfsys@useobject{currentmarker}{}%
\end{pgfscope}%
\end{pgfscope}%
\begin{pgfscope}%
\pgfsetbuttcap%
\pgfsetroundjoin%
\definecolor{currentfill}{rgb}{0.000000,0.000000,0.000000}%
\pgfsetfillcolor{currentfill}%
\pgfsetlinewidth{0.501875pt}%
\definecolor{currentstroke}{rgb}{0.000000,0.000000,0.000000}%
\pgfsetstrokecolor{currentstroke}%
\pgfsetdash{}{0pt}%
\pgfsys@defobject{currentmarker}{\pgfqpoint{0.000000in}{0.000000in}}{\pgfqpoint{0.000000in}{0.020833in}}{%
\pgfpathmoveto{\pgfqpoint{0.000000in}{0.000000in}}%
\pgfpathlineto{\pgfqpoint{0.000000in}{0.020833in}}%
\pgfusepath{stroke,fill}%
}%
\begin{pgfscope}%
\pgfsys@transformshift{1.386162in}{2.747992in}%
\pgfsys@useobject{currentmarker}{}%
\end{pgfscope}%
\end{pgfscope}%
\begin{pgfscope}%
\pgfsetbuttcap%
\pgfsetroundjoin%
\definecolor{currentfill}{rgb}{0.000000,0.000000,0.000000}%
\pgfsetfillcolor{currentfill}%
\pgfsetlinewidth{0.501875pt}%
\definecolor{currentstroke}{rgb}{0.000000,0.000000,0.000000}%
\pgfsetstrokecolor{currentstroke}%
\pgfsetdash{}{0pt}%
\pgfsys@defobject{currentmarker}{\pgfqpoint{0.000000in}{-0.020833in}}{\pgfqpoint{0.000000in}{0.000000in}}{%
\pgfpathmoveto{\pgfqpoint{0.000000in}{0.000000in}}%
\pgfpathlineto{\pgfqpoint{0.000000in}{-0.020833in}}%
\pgfusepath{stroke,fill}%
}%
\begin{pgfscope}%
\pgfsys@transformshift{1.386162in}{4.374193in}%
\pgfsys@useobject{currentmarker}{}%
\end{pgfscope}%
\end{pgfscope}%
\begin{pgfscope}%
\pgfsetbuttcap%
\pgfsetroundjoin%
\definecolor{currentfill}{rgb}{0.000000,0.000000,0.000000}%
\pgfsetfillcolor{currentfill}%
\pgfsetlinewidth{0.501875pt}%
\definecolor{currentstroke}{rgb}{0.000000,0.000000,0.000000}%
\pgfsetstrokecolor{currentstroke}%
\pgfsetdash{}{0pt}%
\pgfsys@defobject{currentmarker}{\pgfqpoint{0.000000in}{0.000000in}}{\pgfqpoint{0.000000in}{0.020833in}}{%
\pgfpathmoveto{\pgfqpoint{0.000000in}{0.000000in}}%
\pgfpathlineto{\pgfqpoint{0.000000in}{0.020833in}}%
\pgfusepath{stroke,fill}%
}%
\begin{pgfscope}%
\pgfsys@transformshift{1.608090in}{2.747992in}%
\pgfsys@useobject{currentmarker}{}%
\end{pgfscope}%
\end{pgfscope}%
\begin{pgfscope}%
\pgfsetbuttcap%
\pgfsetroundjoin%
\definecolor{currentfill}{rgb}{0.000000,0.000000,0.000000}%
\pgfsetfillcolor{currentfill}%
\pgfsetlinewidth{0.501875pt}%
\definecolor{currentstroke}{rgb}{0.000000,0.000000,0.000000}%
\pgfsetstrokecolor{currentstroke}%
\pgfsetdash{}{0pt}%
\pgfsys@defobject{currentmarker}{\pgfqpoint{0.000000in}{-0.020833in}}{\pgfqpoint{0.000000in}{0.000000in}}{%
\pgfpathmoveto{\pgfqpoint{0.000000in}{0.000000in}}%
\pgfpathlineto{\pgfqpoint{0.000000in}{-0.020833in}}%
\pgfusepath{stroke,fill}%
}%
\begin{pgfscope}%
\pgfsys@transformshift{1.608090in}{4.374193in}%
\pgfsys@useobject{currentmarker}{}%
\end{pgfscope}%
\end{pgfscope}%
\begin{pgfscope}%
\pgfsetbuttcap%
\pgfsetroundjoin%
\definecolor{currentfill}{rgb}{0.000000,0.000000,0.000000}%
\pgfsetfillcolor{currentfill}%
\pgfsetlinewidth{0.501875pt}%
\definecolor{currentstroke}{rgb}{0.000000,0.000000,0.000000}%
\pgfsetstrokecolor{currentstroke}%
\pgfsetdash{}{0pt}%
\pgfsys@defobject{currentmarker}{\pgfqpoint{0.000000in}{0.000000in}}{\pgfqpoint{0.000000in}{0.020833in}}{%
\pgfpathmoveto{\pgfqpoint{0.000000in}{0.000000in}}%
\pgfpathlineto{\pgfqpoint{0.000000in}{0.020833in}}%
\pgfusepath{stroke,fill}%
}%
\begin{pgfscope}%
\pgfsys@transformshift{1.719054in}{2.747992in}%
\pgfsys@useobject{currentmarker}{}%
\end{pgfscope}%
\end{pgfscope}%
\begin{pgfscope}%
\pgfsetbuttcap%
\pgfsetroundjoin%
\definecolor{currentfill}{rgb}{0.000000,0.000000,0.000000}%
\pgfsetfillcolor{currentfill}%
\pgfsetlinewidth{0.501875pt}%
\definecolor{currentstroke}{rgb}{0.000000,0.000000,0.000000}%
\pgfsetstrokecolor{currentstroke}%
\pgfsetdash{}{0pt}%
\pgfsys@defobject{currentmarker}{\pgfqpoint{0.000000in}{-0.020833in}}{\pgfqpoint{0.000000in}{0.000000in}}{%
\pgfpathmoveto{\pgfqpoint{0.000000in}{0.000000in}}%
\pgfpathlineto{\pgfqpoint{0.000000in}{-0.020833in}}%
\pgfusepath{stroke,fill}%
}%
\begin{pgfscope}%
\pgfsys@transformshift{1.719054in}{4.374193in}%
\pgfsys@useobject{currentmarker}{}%
\end{pgfscope}%
\end{pgfscope}%
\begin{pgfscope}%
\pgfsetbuttcap%
\pgfsetroundjoin%
\definecolor{currentfill}{rgb}{0.000000,0.000000,0.000000}%
\pgfsetfillcolor{currentfill}%
\pgfsetlinewidth{0.501875pt}%
\definecolor{currentstroke}{rgb}{0.000000,0.000000,0.000000}%
\pgfsetstrokecolor{currentstroke}%
\pgfsetdash{}{0pt}%
\pgfsys@defobject{currentmarker}{\pgfqpoint{0.000000in}{0.000000in}}{\pgfqpoint{0.000000in}{0.020833in}}{%
\pgfpathmoveto{\pgfqpoint{0.000000in}{0.000000in}}%
\pgfpathlineto{\pgfqpoint{0.000000in}{0.020833in}}%
\pgfusepath{stroke,fill}%
}%
\begin{pgfscope}%
\pgfsys@transformshift{1.830018in}{2.747992in}%
\pgfsys@useobject{currentmarker}{}%
\end{pgfscope}%
\end{pgfscope}%
\begin{pgfscope}%
\pgfsetbuttcap%
\pgfsetroundjoin%
\definecolor{currentfill}{rgb}{0.000000,0.000000,0.000000}%
\pgfsetfillcolor{currentfill}%
\pgfsetlinewidth{0.501875pt}%
\definecolor{currentstroke}{rgb}{0.000000,0.000000,0.000000}%
\pgfsetstrokecolor{currentstroke}%
\pgfsetdash{}{0pt}%
\pgfsys@defobject{currentmarker}{\pgfqpoint{0.000000in}{-0.020833in}}{\pgfqpoint{0.000000in}{0.000000in}}{%
\pgfpathmoveto{\pgfqpoint{0.000000in}{0.000000in}}%
\pgfpathlineto{\pgfqpoint{0.000000in}{-0.020833in}}%
\pgfusepath{stroke,fill}%
}%
\begin{pgfscope}%
\pgfsys@transformshift{1.830018in}{4.374193in}%
\pgfsys@useobject{currentmarker}{}%
\end{pgfscope}%
\end{pgfscope}%
\begin{pgfscope}%
\pgfsetbuttcap%
\pgfsetroundjoin%
\definecolor{currentfill}{rgb}{0.000000,0.000000,0.000000}%
\pgfsetfillcolor{currentfill}%
\pgfsetlinewidth{0.501875pt}%
\definecolor{currentstroke}{rgb}{0.000000,0.000000,0.000000}%
\pgfsetstrokecolor{currentstroke}%
\pgfsetdash{}{0pt}%
\pgfsys@defobject{currentmarker}{\pgfqpoint{0.000000in}{0.000000in}}{\pgfqpoint{0.000000in}{0.020833in}}{%
\pgfpathmoveto{\pgfqpoint{0.000000in}{0.000000in}}%
\pgfpathlineto{\pgfqpoint{0.000000in}{0.020833in}}%
\pgfusepath{stroke,fill}%
}%
\begin{pgfscope}%
\pgfsys@transformshift{2.051946in}{2.747992in}%
\pgfsys@useobject{currentmarker}{}%
\end{pgfscope}%
\end{pgfscope}%
\begin{pgfscope}%
\pgfsetbuttcap%
\pgfsetroundjoin%
\definecolor{currentfill}{rgb}{0.000000,0.000000,0.000000}%
\pgfsetfillcolor{currentfill}%
\pgfsetlinewidth{0.501875pt}%
\definecolor{currentstroke}{rgb}{0.000000,0.000000,0.000000}%
\pgfsetstrokecolor{currentstroke}%
\pgfsetdash{}{0pt}%
\pgfsys@defobject{currentmarker}{\pgfqpoint{0.000000in}{-0.020833in}}{\pgfqpoint{0.000000in}{0.000000in}}{%
\pgfpathmoveto{\pgfqpoint{0.000000in}{0.000000in}}%
\pgfpathlineto{\pgfqpoint{0.000000in}{-0.020833in}}%
\pgfusepath{stroke,fill}%
}%
\begin{pgfscope}%
\pgfsys@transformshift{2.051946in}{4.374193in}%
\pgfsys@useobject{currentmarker}{}%
\end{pgfscope}%
\end{pgfscope}%
\begin{pgfscope}%
\pgfsetbuttcap%
\pgfsetroundjoin%
\definecolor{currentfill}{rgb}{0.000000,0.000000,0.000000}%
\pgfsetfillcolor{currentfill}%
\pgfsetlinewidth{0.501875pt}%
\definecolor{currentstroke}{rgb}{0.000000,0.000000,0.000000}%
\pgfsetstrokecolor{currentstroke}%
\pgfsetdash{}{0pt}%
\pgfsys@defobject{currentmarker}{\pgfqpoint{0.000000in}{0.000000in}}{\pgfqpoint{0.000000in}{0.020833in}}{%
\pgfpathmoveto{\pgfqpoint{0.000000in}{0.000000in}}%
\pgfpathlineto{\pgfqpoint{0.000000in}{0.020833in}}%
\pgfusepath{stroke,fill}%
}%
\begin{pgfscope}%
\pgfsys@transformshift{2.162910in}{2.747992in}%
\pgfsys@useobject{currentmarker}{}%
\end{pgfscope}%
\end{pgfscope}%
\begin{pgfscope}%
\pgfsetbuttcap%
\pgfsetroundjoin%
\definecolor{currentfill}{rgb}{0.000000,0.000000,0.000000}%
\pgfsetfillcolor{currentfill}%
\pgfsetlinewidth{0.501875pt}%
\definecolor{currentstroke}{rgb}{0.000000,0.000000,0.000000}%
\pgfsetstrokecolor{currentstroke}%
\pgfsetdash{}{0pt}%
\pgfsys@defobject{currentmarker}{\pgfqpoint{0.000000in}{-0.020833in}}{\pgfqpoint{0.000000in}{0.000000in}}{%
\pgfpathmoveto{\pgfqpoint{0.000000in}{0.000000in}}%
\pgfpathlineto{\pgfqpoint{0.000000in}{-0.020833in}}%
\pgfusepath{stroke,fill}%
}%
\begin{pgfscope}%
\pgfsys@transformshift{2.162910in}{4.374193in}%
\pgfsys@useobject{currentmarker}{}%
\end{pgfscope}%
\end{pgfscope}%
\begin{pgfscope}%
\pgfsetbuttcap%
\pgfsetroundjoin%
\definecolor{currentfill}{rgb}{0.000000,0.000000,0.000000}%
\pgfsetfillcolor{currentfill}%
\pgfsetlinewidth{0.501875pt}%
\definecolor{currentstroke}{rgb}{0.000000,0.000000,0.000000}%
\pgfsetstrokecolor{currentstroke}%
\pgfsetdash{}{0pt}%
\pgfsys@defobject{currentmarker}{\pgfqpoint{0.000000in}{0.000000in}}{\pgfqpoint{0.000000in}{0.020833in}}{%
\pgfpathmoveto{\pgfqpoint{0.000000in}{0.000000in}}%
\pgfpathlineto{\pgfqpoint{0.000000in}{0.020833in}}%
\pgfusepath{stroke,fill}%
}%
\begin{pgfscope}%
\pgfsys@transformshift{2.273874in}{2.747992in}%
\pgfsys@useobject{currentmarker}{}%
\end{pgfscope}%
\end{pgfscope}%
\begin{pgfscope}%
\pgfsetbuttcap%
\pgfsetroundjoin%
\definecolor{currentfill}{rgb}{0.000000,0.000000,0.000000}%
\pgfsetfillcolor{currentfill}%
\pgfsetlinewidth{0.501875pt}%
\definecolor{currentstroke}{rgb}{0.000000,0.000000,0.000000}%
\pgfsetstrokecolor{currentstroke}%
\pgfsetdash{}{0pt}%
\pgfsys@defobject{currentmarker}{\pgfqpoint{0.000000in}{-0.020833in}}{\pgfqpoint{0.000000in}{0.000000in}}{%
\pgfpathmoveto{\pgfqpoint{0.000000in}{0.000000in}}%
\pgfpathlineto{\pgfqpoint{0.000000in}{-0.020833in}}%
\pgfusepath{stroke,fill}%
}%
\begin{pgfscope}%
\pgfsys@transformshift{2.273874in}{4.374193in}%
\pgfsys@useobject{currentmarker}{}%
\end{pgfscope}%
\end{pgfscope}%
\begin{pgfscope}%
\pgfsetbuttcap%
\pgfsetroundjoin%
\definecolor{currentfill}{rgb}{0.000000,0.000000,0.000000}%
\pgfsetfillcolor{currentfill}%
\pgfsetlinewidth{0.501875pt}%
\definecolor{currentstroke}{rgb}{0.000000,0.000000,0.000000}%
\pgfsetstrokecolor{currentstroke}%
\pgfsetdash{}{0pt}%
\pgfsys@defobject{currentmarker}{\pgfqpoint{0.000000in}{0.000000in}}{\pgfqpoint{0.000000in}{0.020833in}}{%
\pgfpathmoveto{\pgfqpoint{0.000000in}{0.000000in}}%
\pgfpathlineto{\pgfqpoint{0.000000in}{0.020833in}}%
\pgfusepath{stroke,fill}%
}%
\begin{pgfscope}%
\pgfsys@transformshift{2.495801in}{2.747992in}%
\pgfsys@useobject{currentmarker}{}%
\end{pgfscope}%
\end{pgfscope}%
\begin{pgfscope}%
\pgfsetbuttcap%
\pgfsetroundjoin%
\definecolor{currentfill}{rgb}{0.000000,0.000000,0.000000}%
\pgfsetfillcolor{currentfill}%
\pgfsetlinewidth{0.501875pt}%
\definecolor{currentstroke}{rgb}{0.000000,0.000000,0.000000}%
\pgfsetstrokecolor{currentstroke}%
\pgfsetdash{}{0pt}%
\pgfsys@defobject{currentmarker}{\pgfqpoint{0.000000in}{-0.020833in}}{\pgfqpoint{0.000000in}{0.000000in}}{%
\pgfpathmoveto{\pgfqpoint{0.000000in}{0.000000in}}%
\pgfpathlineto{\pgfqpoint{0.000000in}{-0.020833in}}%
\pgfusepath{stroke,fill}%
}%
\begin{pgfscope}%
\pgfsys@transformshift{2.495801in}{4.374193in}%
\pgfsys@useobject{currentmarker}{}%
\end{pgfscope}%
\end{pgfscope}%
\begin{pgfscope}%
\pgfsetbuttcap%
\pgfsetroundjoin%
\definecolor{currentfill}{rgb}{0.000000,0.000000,0.000000}%
\pgfsetfillcolor{currentfill}%
\pgfsetlinewidth{0.501875pt}%
\definecolor{currentstroke}{rgb}{0.000000,0.000000,0.000000}%
\pgfsetstrokecolor{currentstroke}%
\pgfsetdash{}{0pt}%
\pgfsys@defobject{currentmarker}{\pgfqpoint{0.000000in}{0.000000in}}{\pgfqpoint{0.000000in}{0.020833in}}{%
\pgfpathmoveto{\pgfqpoint{0.000000in}{0.000000in}}%
\pgfpathlineto{\pgfqpoint{0.000000in}{0.020833in}}%
\pgfusepath{stroke,fill}%
}%
\begin{pgfscope}%
\pgfsys@transformshift{2.606765in}{2.747992in}%
\pgfsys@useobject{currentmarker}{}%
\end{pgfscope}%
\end{pgfscope}%
\begin{pgfscope}%
\pgfsetbuttcap%
\pgfsetroundjoin%
\definecolor{currentfill}{rgb}{0.000000,0.000000,0.000000}%
\pgfsetfillcolor{currentfill}%
\pgfsetlinewidth{0.501875pt}%
\definecolor{currentstroke}{rgb}{0.000000,0.000000,0.000000}%
\pgfsetstrokecolor{currentstroke}%
\pgfsetdash{}{0pt}%
\pgfsys@defobject{currentmarker}{\pgfqpoint{0.000000in}{-0.020833in}}{\pgfqpoint{0.000000in}{0.000000in}}{%
\pgfpathmoveto{\pgfqpoint{0.000000in}{0.000000in}}%
\pgfpathlineto{\pgfqpoint{0.000000in}{-0.020833in}}%
\pgfusepath{stroke,fill}%
}%
\begin{pgfscope}%
\pgfsys@transformshift{2.606765in}{4.374193in}%
\pgfsys@useobject{currentmarker}{}%
\end{pgfscope}%
\end{pgfscope}%
\begin{pgfscope}%
\pgfsetbuttcap%
\pgfsetroundjoin%
\definecolor{currentfill}{rgb}{0.000000,0.000000,0.000000}%
\pgfsetfillcolor{currentfill}%
\pgfsetlinewidth{0.501875pt}%
\definecolor{currentstroke}{rgb}{0.000000,0.000000,0.000000}%
\pgfsetstrokecolor{currentstroke}%
\pgfsetdash{}{0pt}%
\pgfsys@defobject{currentmarker}{\pgfqpoint{0.000000in}{0.000000in}}{\pgfqpoint{0.000000in}{0.020833in}}{%
\pgfpathmoveto{\pgfqpoint{0.000000in}{0.000000in}}%
\pgfpathlineto{\pgfqpoint{0.000000in}{0.020833in}}%
\pgfusepath{stroke,fill}%
}%
\begin{pgfscope}%
\pgfsys@transformshift{2.717729in}{2.747992in}%
\pgfsys@useobject{currentmarker}{}%
\end{pgfscope}%
\end{pgfscope}%
\begin{pgfscope}%
\pgfsetbuttcap%
\pgfsetroundjoin%
\definecolor{currentfill}{rgb}{0.000000,0.000000,0.000000}%
\pgfsetfillcolor{currentfill}%
\pgfsetlinewidth{0.501875pt}%
\definecolor{currentstroke}{rgb}{0.000000,0.000000,0.000000}%
\pgfsetstrokecolor{currentstroke}%
\pgfsetdash{}{0pt}%
\pgfsys@defobject{currentmarker}{\pgfqpoint{0.000000in}{-0.020833in}}{\pgfqpoint{0.000000in}{0.000000in}}{%
\pgfpathmoveto{\pgfqpoint{0.000000in}{0.000000in}}%
\pgfpathlineto{\pgfqpoint{0.000000in}{-0.020833in}}%
\pgfusepath{stroke,fill}%
}%
\begin{pgfscope}%
\pgfsys@transformshift{2.717729in}{4.374193in}%
\pgfsys@useobject{currentmarker}{}%
\end{pgfscope}%
\end{pgfscope}%
\begin{pgfscope}%
\pgfsetbuttcap%
\pgfsetroundjoin%
\definecolor{currentfill}{rgb}{0.000000,0.000000,0.000000}%
\pgfsetfillcolor{currentfill}%
\pgfsetlinewidth{0.501875pt}%
\definecolor{currentstroke}{rgb}{0.000000,0.000000,0.000000}%
\pgfsetstrokecolor{currentstroke}%
\pgfsetdash{}{0pt}%
\pgfsys@defobject{currentmarker}{\pgfqpoint{0.000000in}{0.000000in}}{\pgfqpoint{0.000000in}{0.020833in}}{%
\pgfpathmoveto{\pgfqpoint{0.000000in}{0.000000in}}%
\pgfpathlineto{\pgfqpoint{0.000000in}{0.020833in}}%
\pgfusepath{stroke,fill}%
}%
\begin{pgfscope}%
\pgfsys@transformshift{2.828693in}{2.747992in}%
\pgfsys@useobject{currentmarker}{}%
\end{pgfscope}%
\end{pgfscope}%
\begin{pgfscope}%
\pgfsetbuttcap%
\pgfsetroundjoin%
\definecolor{currentfill}{rgb}{0.000000,0.000000,0.000000}%
\pgfsetfillcolor{currentfill}%
\pgfsetlinewidth{0.501875pt}%
\definecolor{currentstroke}{rgb}{0.000000,0.000000,0.000000}%
\pgfsetstrokecolor{currentstroke}%
\pgfsetdash{}{0pt}%
\pgfsys@defobject{currentmarker}{\pgfqpoint{0.000000in}{-0.020833in}}{\pgfqpoint{0.000000in}{0.000000in}}{%
\pgfpathmoveto{\pgfqpoint{0.000000in}{0.000000in}}%
\pgfpathlineto{\pgfqpoint{0.000000in}{-0.020833in}}%
\pgfusepath{stroke,fill}%
}%
\begin{pgfscope}%
\pgfsys@transformshift{2.828693in}{4.374193in}%
\pgfsys@useobject{currentmarker}{}%
\end{pgfscope}%
\end{pgfscope}%
\begin{pgfscope}%
\definecolor{textcolor}{rgb}{0.000000,0.000000,0.000000}%
\pgfsetstrokecolor{textcolor}%
\pgfsetfillcolor{textcolor}%
\pgftext[x=1.730150in,y=2.509413in,,top]{\color{textcolor}\rmfamily\fontsize{10.000000}{12.000000}\selectfont \(\displaystyle K\)}%
\end{pgfscope}%
\begin{pgfscope}%
\pgfsetbuttcap%
\pgfsetroundjoin%
\definecolor{currentfill}{rgb}{0.000000,0.000000,0.000000}%
\pgfsetfillcolor{currentfill}%
\pgfsetlinewidth{0.501875pt}%
\definecolor{currentstroke}{rgb}{0.000000,0.000000,0.000000}%
\pgfsetstrokecolor{currentstroke}%
\pgfsetdash{}{0pt}%
\pgfsys@defobject{currentmarker}{\pgfqpoint{0.000000in}{0.000000in}}{\pgfqpoint{0.041667in}{0.000000in}}{%
\pgfpathmoveto{\pgfqpoint{0.000000in}{0.000000in}}%
\pgfpathlineto{\pgfqpoint{0.041667in}{0.000000in}}%
\pgfusepath{stroke,fill}%
}%
\begin{pgfscope}%
\pgfsys@transformshift{0.609415in}{3.046024in}%
\pgfsys@useobject{currentmarker}{}%
\end{pgfscope}%
\end{pgfscope}%
\begin{pgfscope}%
\pgfsetbuttcap%
\pgfsetroundjoin%
\definecolor{currentfill}{rgb}{0.000000,0.000000,0.000000}%
\pgfsetfillcolor{currentfill}%
\pgfsetlinewidth{0.501875pt}%
\definecolor{currentstroke}{rgb}{0.000000,0.000000,0.000000}%
\pgfsetstrokecolor{currentstroke}%
\pgfsetdash{}{0pt}%
\pgfsys@defobject{currentmarker}{\pgfqpoint{-0.041667in}{0.000000in}}{\pgfqpoint{-0.000000in}{0.000000in}}{%
\pgfpathmoveto{\pgfqpoint{-0.000000in}{0.000000in}}%
\pgfpathlineto{\pgfqpoint{-0.041667in}{0.000000in}}%
\pgfusepath{stroke,fill}%
}%
\begin{pgfscope}%
\pgfsys@transformshift{2.850886in}{3.046024in}%
\pgfsys@useobject{currentmarker}{}%
\end{pgfscope}%
\end{pgfscope}%
\begin{pgfscope}%
\definecolor{textcolor}{rgb}{0.000000,0.000000,0.000000}%
\pgfsetstrokecolor{textcolor}%
\pgfsetfillcolor{textcolor}%
\pgftext[x=0.313889in, y=2.993262in, left, base]{\color{textcolor}\rmfamily\fontsize{10.000000}{12.000000}\selectfont \(\displaystyle {0.85}\)}%
\end{pgfscope}%
\begin{pgfscope}%
\pgfsetbuttcap%
\pgfsetroundjoin%
\definecolor{currentfill}{rgb}{0.000000,0.000000,0.000000}%
\pgfsetfillcolor{currentfill}%
\pgfsetlinewidth{0.501875pt}%
\definecolor{currentstroke}{rgb}{0.000000,0.000000,0.000000}%
\pgfsetstrokecolor{currentstroke}%
\pgfsetdash{}{0pt}%
\pgfsys@defobject{currentmarker}{\pgfqpoint{0.000000in}{0.000000in}}{\pgfqpoint{0.041667in}{0.000000in}}{%
\pgfpathmoveto{\pgfqpoint{0.000000in}{0.000000in}}%
\pgfpathlineto{\pgfqpoint{0.041667in}{0.000000in}}%
\pgfusepath{stroke,fill}%
}%
\begin{pgfscope}%
\pgfsys@transformshift{0.609415in}{3.465034in}%
\pgfsys@useobject{currentmarker}{}%
\end{pgfscope}%
\end{pgfscope}%
\begin{pgfscope}%
\pgfsetbuttcap%
\pgfsetroundjoin%
\definecolor{currentfill}{rgb}{0.000000,0.000000,0.000000}%
\pgfsetfillcolor{currentfill}%
\pgfsetlinewidth{0.501875pt}%
\definecolor{currentstroke}{rgb}{0.000000,0.000000,0.000000}%
\pgfsetstrokecolor{currentstroke}%
\pgfsetdash{}{0pt}%
\pgfsys@defobject{currentmarker}{\pgfqpoint{-0.041667in}{0.000000in}}{\pgfqpoint{-0.000000in}{0.000000in}}{%
\pgfpathmoveto{\pgfqpoint{-0.000000in}{0.000000in}}%
\pgfpathlineto{\pgfqpoint{-0.041667in}{0.000000in}}%
\pgfusepath{stroke,fill}%
}%
\begin{pgfscope}%
\pgfsys@transformshift{2.850886in}{3.465034in}%
\pgfsys@useobject{currentmarker}{}%
\end{pgfscope}%
\end{pgfscope}%
\begin{pgfscope}%
\definecolor{textcolor}{rgb}{0.000000,0.000000,0.000000}%
\pgfsetstrokecolor{textcolor}%
\pgfsetfillcolor{textcolor}%
\pgftext[x=0.313889in, y=3.412272in, left, base]{\color{textcolor}\rmfamily\fontsize{10.000000}{12.000000}\selectfont \(\displaystyle {0.90}\)}%
\end{pgfscope}%
\begin{pgfscope}%
\pgfsetbuttcap%
\pgfsetroundjoin%
\definecolor{currentfill}{rgb}{0.000000,0.000000,0.000000}%
\pgfsetfillcolor{currentfill}%
\pgfsetlinewidth{0.501875pt}%
\definecolor{currentstroke}{rgb}{0.000000,0.000000,0.000000}%
\pgfsetstrokecolor{currentstroke}%
\pgfsetdash{}{0pt}%
\pgfsys@defobject{currentmarker}{\pgfqpoint{0.000000in}{0.000000in}}{\pgfqpoint{0.041667in}{0.000000in}}{%
\pgfpathmoveto{\pgfqpoint{0.000000in}{0.000000in}}%
\pgfpathlineto{\pgfqpoint{0.041667in}{0.000000in}}%
\pgfusepath{stroke,fill}%
}%
\begin{pgfscope}%
\pgfsys@transformshift{0.609415in}{3.884043in}%
\pgfsys@useobject{currentmarker}{}%
\end{pgfscope}%
\end{pgfscope}%
\begin{pgfscope}%
\pgfsetbuttcap%
\pgfsetroundjoin%
\definecolor{currentfill}{rgb}{0.000000,0.000000,0.000000}%
\pgfsetfillcolor{currentfill}%
\pgfsetlinewidth{0.501875pt}%
\definecolor{currentstroke}{rgb}{0.000000,0.000000,0.000000}%
\pgfsetstrokecolor{currentstroke}%
\pgfsetdash{}{0pt}%
\pgfsys@defobject{currentmarker}{\pgfqpoint{-0.041667in}{0.000000in}}{\pgfqpoint{-0.000000in}{0.000000in}}{%
\pgfpathmoveto{\pgfqpoint{-0.000000in}{0.000000in}}%
\pgfpathlineto{\pgfqpoint{-0.041667in}{0.000000in}}%
\pgfusepath{stroke,fill}%
}%
\begin{pgfscope}%
\pgfsys@transformshift{2.850886in}{3.884043in}%
\pgfsys@useobject{currentmarker}{}%
\end{pgfscope}%
\end{pgfscope}%
\begin{pgfscope}%
\definecolor{textcolor}{rgb}{0.000000,0.000000,0.000000}%
\pgfsetstrokecolor{textcolor}%
\pgfsetfillcolor{textcolor}%
\pgftext[x=0.313889in, y=3.831282in, left, base]{\color{textcolor}\rmfamily\fontsize{10.000000}{12.000000}\selectfont \(\displaystyle {0.95}\)}%
\end{pgfscope}%
\begin{pgfscope}%
\pgfsetbuttcap%
\pgfsetroundjoin%
\definecolor{currentfill}{rgb}{0.000000,0.000000,0.000000}%
\pgfsetfillcolor{currentfill}%
\pgfsetlinewidth{0.501875pt}%
\definecolor{currentstroke}{rgb}{0.000000,0.000000,0.000000}%
\pgfsetstrokecolor{currentstroke}%
\pgfsetdash{}{0pt}%
\pgfsys@defobject{currentmarker}{\pgfqpoint{0.000000in}{0.000000in}}{\pgfqpoint{0.041667in}{0.000000in}}{%
\pgfpathmoveto{\pgfqpoint{0.000000in}{0.000000in}}%
\pgfpathlineto{\pgfqpoint{0.041667in}{0.000000in}}%
\pgfusepath{stroke,fill}%
}%
\begin{pgfscope}%
\pgfsys@transformshift{0.609415in}{4.303053in}%
\pgfsys@useobject{currentmarker}{}%
\end{pgfscope}%
\end{pgfscope}%
\begin{pgfscope}%
\pgfsetbuttcap%
\pgfsetroundjoin%
\definecolor{currentfill}{rgb}{0.000000,0.000000,0.000000}%
\pgfsetfillcolor{currentfill}%
\pgfsetlinewidth{0.501875pt}%
\definecolor{currentstroke}{rgb}{0.000000,0.000000,0.000000}%
\pgfsetstrokecolor{currentstroke}%
\pgfsetdash{}{0pt}%
\pgfsys@defobject{currentmarker}{\pgfqpoint{-0.041667in}{0.000000in}}{\pgfqpoint{-0.000000in}{0.000000in}}{%
\pgfpathmoveto{\pgfqpoint{-0.000000in}{0.000000in}}%
\pgfpathlineto{\pgfqpoint{-0.041667in}{0.000000in}}%
\pgfusepath{stroke,fill}%
}%
\begin{pgfscope}%
\pgfsys@transformshift{2.850886in}{4.303053in}%
\pgfsys@useobject{currentmarker}{}%
\end{pgfscope}%
\end{pgfscope}%
\begin{pgfscope}%
\definecolor{textcolor}{rgb}{0.000000,0.000000,0.000000}%
\pgfsetstrokecolor{textcolor}%
\pgfsetfillcolor{textcolor}%
\pgftext[x=0.313889in, y=4.250292in, left, base]{\color{textcolor}\rmfamily\fontsize{10.000000}{12.000000}\selectfont \(\displaystyle {1.00}\)}%
\end{pgfscope}%
\begin{pgfscope}%
\pgfsetbuttcap%
\pgfsetroundjoin%
\definecolor{currentfill}{rgb}{0.000000,0.000000,0.000000}%
\pgfsetfillcolor{currentfill}%
\pgfsetlinewidth{0.501875pt}%
\definecolor{currentstroke}{rgb}{0.000000,0.000000,0.000000}%
\pgfsetstrokecolor{currentstroke}%
\pgfsetdash{}{0pt}%
\pgfsys@defobject{currentmarker}{\pgfqpoint{0.000000in}{0.000000in}}{\pgfqpoint{0.020833in}{0.000000in}}{%
\pgfpathmoveto{\pgfqpoint{0.000000in}{0.000000in}}%
\pgfpathlineto{\pgfqpoint{0.020833in}{0.000000in}}%
\pgfusepath{stroke,fill}%
}%
\begin{pgfscope}%
\pgfsys@transformshift{0.609415in}{2.794618in}%
\pgfsys@useobject{currentmarker}{}%
\end{pgfscope}%
\end{pgfscope}%
\begin{pgfscope}%
\pgfsetbuttcap%
\pgfsetroundjoin%
\definecolor{currentfill}{rgb}{0.000000,0.000000,0.000000}%
\pgfsetfillcolor{currentfill}%
\pgfsetlinewidth{0.501875pt}%
\definecolor{currentstroke}{rgb}{0.000000,0.000000,0.000000}%
\pgfsetstrokecolor{currentstroke}%
\pgfsetdash{}{0pt}%
\pgfsys@defobject{currentmarker}{\pgfqpoint{-0.020833in}{0.000000in}}{\pgfqpoint{-0.000000in}{0.000000in}}{%
\pgfpathmoveto{\pgfqpoint{-0.000000in}{0.000000in}}%
\pgfpathlineto{\pgfqpoint{-0.020833in}{0.000000in}}%
\pgfusepath{stroke,fill}%
}%
\begin{pgfscope}%
\pgfsys@transformshift{2.850886in}{2.794618in}%
\pgfsys@useobject{currentmarker}{}%
\end{pgfscope}%
\end{pgfscope}%
\begin{pgfscope}%
\pgfsetbuttcap%
\pgfsetroundjoin%
\definecolor{currentfill}{rgb}{0.000000,0.000000,0.000000}%
\pgfsetfillcolor{currentfill}%
\pgfsetlinewidth{0.501875pt}%
\definecolor{currentstroke}{rgb}{0.000000,0.000000,0.000000}%
\pgfsetstrokecolor{currentstroke}%
\pgfsetdash{}{0pt}%
\pgfsys@defobject{currentmarker}{\pgfqpoint{0.000000in}{0.000000in}}{\pgfqpoint{0.020833in}{0.000000in}}{%
\pgfpathmoveto{\pgfqpoint{0.000000in}{0.000000in}}%
\pgfpathlineto{\pgfqpoint{0.020833in}{0.000000in}}%
\pgfusepath{stroke,fill}%
}%
\begin{pgfscope}%
\pgfsys@transformshift{0.609415in}{2.878420in}%
\pgfsys@useobject{currentmarker}{}%
\end{pgfscope}%
\end{pgfscope}%
\begin{pgfscope}%
\pgfsetbuttcap%
\pgfsetroundjoin%
\definecolor{currentfill}{rgb}{0.000000,0.000000,0.000000}%
\pgfsetfillcolor{currentfill}%
\pgfsetlinewidth{0.501875pt}%
\definecolor{currentstroke}{rgb}{0.000000,0.000000,0.000000}%
\pgfsetstrokecolor{currentstroke}%
\pgfsetdash{}{0pt}%
\pgfsys@defobject{currentmarker}{\pgfqpoint{-0.020833in}{0.000000in}}{\pgfqpoint{-0.000000in}{0.000000in}}{%
\pgfpathmoveto{\pgfqpoint{-0.000000in}{0.000000in}}%
\pgfpathlineto{\pgfqpoint{-0.020833in}{0.000000in}}%
\pgfusepath{stroke,fill}%
}%
\begin{pgfscope}%
\pgfsys@transformshift{2.850886in}{2.878420in}%
\pgfsys@useobject{currentmarker}{}%
\end{pgfscope}%
\end{pgfscope}%
\begin{pgfscope}%
\pgfsetbuttcap%
\pgfsetroundjoin%
\definecolor{currentfill}{rgb}{0.000000,0.000000,0.000000}%
\pgfsetfillcolor{currentfill}%
\pgfsetlinewidth{0.501875pt}%
\definecolor{currentstroke}{rgb}{0.000000,0.000000,0.000000}%
\pgfsetstrokecolor{currentstroke}%
\pgfsetdash{}{0pt}%
\pgfsys@defobject{currentmarker}{\pgfqpoint{0.000000in}{0.000000in}}{\pgfqpoint{0.020833in}{0.000000in}}{%
\pgfpathmoveto{\pgfqpoint{0.000000in}{0.000000in}}%
\pgfpathlineto{\pgfqpoint{0.020833in}{0.000000in}}%
\pgfusepath{stroke,fill}%
}%
\begin{pgfscope}%
\pgfsys@transformshift{0.609415in}{2.962222in}%
\pgfsys@useobject{currentmarker}{}%
\end{pgfscope}%
\end{pgfscope}%
\begin{pgfscope}%
\pgfsetbuttcap%
\pgfsetroundjoin%
\definecolor{currentfill}{rgb}{0.000000,0.000000,0.000000}%
\pgfsetfillcolor{currentfill}%
\pgfsetlinewidth{0.501875pt}%
\definecolor{currentstroke}{rgb}{0.000000,0.000000,0.000000}%
\pgfsetstrokecolor{currentstroke}%
\pgfsetdash{}{0pt}%
\pgfsys@defobject{currentmarker}{\pgfqpoint{-0.020833in}{0.000000in}}{\pgfqpoint{-0.000000in}{0.000000in}}{%
\pgfpathmoveto{\pgfqpoint{-0.000000in}{0.000000in}}%
\pgfpathlineto{\pgfqpoint{-0.020833in}{0.000000in}}%
\pgfusepath{stroke,fill}%
}%
\begin{pgfscope}%
\pgfsys@transformshift{2.850886in}{2.962222in}%
\pgfsys@useobject{currentmarker}{}%
\end{pgfscope}%
\end{pgfscope}%
\begin{pgfscope}%
\pgfsetbuttcap%
\pgfsetroundjoin%
\definecolor{currentfill}{rgb}{0.000000,0.000000,0.000000}%
\pgfsetfillcolor{currentfill}%
\pgfsetlinewidth{0.501875pt}%
\definecolor{currentstroke}{rgb}{0.000000,0.000000,0.000000}%
\pgfsetstrokecolor{currentstroke}%
\pgfsetdash{}{0pt}%
\pgfsys@defobject{currentmarker}{\pgfqpoint{0.000000in}{0.000000in}}{\pgfqpoint{0.020833in}{0.000000in}}{%
\pgfpathmoveto{\pgfqpoint{0.000000in}{0.000000in}}%
\pgfpathlineto{\pgfqpoint{0.020833in}{0.000000in}}%
\pgfusepath{stroke,fill}%
}%
\begin{pgfscope}%
\pgfsys@transformshift{0.609415in}{3.129826in}%
\pgfsys@useobject{currentmarker}{}%
\end{pgfscope}%
\end{pgfscope}%
\begin{pgfscope}%
\pgfsetbuttcap%
\pgfsetroundjoin%
\definecolor{currentfill}{rgb}{0.000000,0.000000,0.000000}%
\pgfsetfillcolor{currentfill}%
\pgfsetlinewidth{0.501875pt}%
\definecolor{currentstroke}{rgb}{0.000000,0.000000,0.000000}%
\pgfsetstrokecolor{currentstroke}%
\pgfsetdash{}{0pt}%
\pgfsys@defobject{currentmarker}{\pgfqpoint{-0.020833in}{0.000000in}}{\pgfqpoint{-0.000000in}{0.000000in}}{%
\pgfpathmoveto{\pgfqpoint{-0.000000in}{0.000000in}}%
\pgfpathlineto{\pgfqpoint{-0.020833in}{0.000000in}}%
\pgfusepath{stroke,fill}%
}%
\begin{pgfscope}%
\pgfsys@transformshift{2.850886in}{3.129826in}%
\pgfsys@useobject{currentmarker}{}%
\end{pgfscope}%
\end{pgfscope}%
\begin{pgfscope}%
\pgfsetbuttcap%
\pgfsetroundjoin%
\definecolor{currentfill}{rgb}{0.000000,0.000000,0.000000}%
\pgfsetfillcolor{currentfill}%
\pgfsetlinewidth{0.501875pt}%
\definecolor{currentstroke}{rgb}{0.000000,0.000000,0.000000}%
\pgfsetstrokecolor{currentstroke}%
\pgfsetdash{}{0pt}%
\pgfsys@defobject{currentmarker}{\pgfqpoint{0.000000in}{0.000000in}}{\pgfqpoint{0.020833in}{0.000000in}}{%
\pgfpathmoveto{\pgfqpoint{0.000000in}{0.000000in}}%
\pgfpathlineto{\pgfqpoint{0.020833in}{0.000000in}}%
\pgfusepath{stroke,fill}%
}%
\begin{pgfscope}%
\pgfsys@transformshift{0.609415in}{3.213628in}%
\pgfsys@useobject{currentmarker}{}%
\end{pgfscope}%
\end{pgfscope}%
\begin{pgfscope}%
\pgfsetbuttcap%
\pgfsetroundjoin%
\definecolor{currentfill}{rgb}{0.000000,0.000000,0.000000}%
\pgfsetfillcolor{currentfill}%
\pgfsetlinewidth{0.501875pt}%
\definecolor{currentstroke}{rgb}{0.000000,0.000000,0.000000}%
\pgfsetstrokecolor{currentstroke}%
\pgfsetdash{}{0pt}%
\pgfsys@defobject{currentmarker}{\pgfqpoint{-0.020833in}{0.000000in}}{\pgfqpoint{-0.000000in}{0.000000in}}{%
\pgfpathmoveto{\pgfqpoint{-0.000000in}{0.000000in}}%
\pgfpathlineto{\pgfqpoint{-0.020833in}{0.000000in}}%
\pgfusepath{stroke,fill}%
}%
\begin{pgfscope}%
\pgfsys@transformshift{2.850886in}{3.213628in}%
\pgfsys@useobject{currentmarker}{}%
\end{pgfscope}%
\end{pgfscope}%
\begin{pgfscope}%
\pgfsetbuttcap%
\pgfsetroundjoin%
\definecolor{currentfill}{rgb}{0.000000,0.000000,0.000000}%
\pgfsetfillcolor{currentfill}%
\pgfsetlinewidth{0.501875pt}%
\definecolor{currentstroke}{rgb}{0.000000,0.000000,0.000000}%
\pgfsetstrokecolor{currentstroke}%
\pgfsetdash{}{0pt}%
\pgfsys@defobject{currentmarker}{\pgfqpoint{0.000000in}{0.000000in}}{\pgfqpoint{0.020833in}{0.000000in}}{%
\pgfpathmoveto{\pgfqpoint{0.000000in}{0.000000in}}%
\pgfpathlineto{\pgfqpoint{0.020833in}{0.000000in}}%
\pgfusepath{stroke,fill}%
}%
\begin{pgfscope}%
\pgfsys@transformshift{0.609415in}{3.297430in}%
\pgfsys@useobject{currentmarker}{}%
\end{pgfscope}%
\end{pgfscope}%
\begin{pgfscope}%
\pgfsetbuttcap%
\pgfsetroundjoin%
\definecolor{currentfill}{rgb}{0.000000,0.000000,0.000000}%
\pgfsetfillcolor{currentfill}%
\pgfsetlinewidth{0.501875pt}%
\definecolor{currentstroke}{rgb}{0.000000,0.000000,0.000000}%
\pgfsetstrokecolor{currentstroke}%
\pgfsetdash{}{0pt}%
\pgfsys@defobject{currentmarker}{\pgfqpoint{-0.020833in}{0.000000in}}{\pgfqpoint{-0.000000in}{0.000000in}}{%
\pgfpathmoveto{\pgfqpoint{-0.000000in}{0.000000in}}%
\pgfpathlineto{\pgfqpoint{-0.020833in}{0.000000in}}%
\pgfusepath{stroke,fill}%
}%
\begin{pgfscope}%
\pgfsys@transformshift{2.850886in}{3.297430in}%
\pgfsys@useobject{currentmarker}{}%
\end{pgfscope}%
\end{pgfscope}%
\begin{pgfscope}%
\pgfsetbuttcap%
\pgfsetroundjoin%
\definecolor{currentfill}{rgb}{0.000000,0.000000,0.000000}%
\pgfsetfillcolor{currentfill}%
\pgfsetlinewidth{0.501875pt}%
\definecolor{currentstroke}{rgb}{0.000000,0.000000,0.000000}%
\pgfsetstrokecolor{currentstroke}%
\pgfsetdash{}{0pt}%
\pgfsys@defobject{currentmarker}{\pgfqpoint{0.000000in}{0.000000in}}{\pgfqpoint{0.020833in}{0.000000in}}{%
\pgfpathmoveto{\pgfqpoint{0.000000in}{0.000000in}}%
\pgfpathlineto{\pgfqpoint{0.020833in}{0.000000in}}%
\pgfusepath{stroke,fill}%
}%
\begin{pgfscope}%
\pgfsys@transformshift{0.609415in}{3.381232in}%
\pgfsys@useobject{currentmarker}{}%
\end{pgfscope}%
\end{pgfscope}%
\begin{pgfscope}%
\pgfsetbuttcap%
\pgfsetroundjoin%
\definecolor{currentfill}{rgb}{0.000000,0.000000,0.000000}%
\pgfsetfillcolor{currentfill}%
\pgfsetlinewidth{0.501875pt}%
\definecolor{currentstroke}{rgb}{0.000000,0.000000,0.000000}%
\pgfsetstrokecolor{currentstroke}%
\pgfsetdash{}{0pt}%
\pgfsys@defobject{currentmarker}{\pgfqpoint{-0.020833in}{0.000000in}}{\pgfqpoint{-0.000000in}{0.000000in}}{%
\pgfpathmoveto{\pgfqpoint{-0.000000in}{0.000000in}}%
\pgfpathlineto{\pgfqpoint{-0.020833in}{0.000000in}}%
\pgfusepath{stroke,fill}%
}%
\begin{pgfscope}%
\pgfsys@transformshift{2.850886in}{3.381232in}%
\pgfsys@useobject{currentmarker}{}%
\end{pgfscope}%
\end{pgfscope}%
\begin{pgfscope}%
\pgfsetbuttcap%
\pgfsetroundjoin%
\definecolor{currentfill}{rgb}{0.000000,0.000000,0.000000}%
\pgfsetfillcolor{currentfill}%
\pgfsetlinewidth{0.501875pt}%
\definecolor{currentstroke}{rgb}{0.000000,0.000000,0.000000}%
\pgfsetstrokecolor{currentstroke}%
\pgfsetdash{}{0pt}%
\pgfsys@defobject{currentmarker}{\pgfqpoint{0.000000in}{0.000000in}}{\pgfqpoint{0.020833in}{0.000000in}}{%
\pgfpathmoveto{\pgfqpoint{0.000000in}{0.000000in}}%
\pgfpathlineto{\pgfqpoint{0.020833in}{0.000000in}}%
\pgfusepath{stroke,fill}%
}%
\begin{pgfscope}%
\pgfsys@transformshift{0.609415in}{3.548836in}%
\pgfsys@useobject{currentmarker}{}%
\end{pgfscope}%
\end{pgfscope}%
\begin{pgfscope}%
\pgfsetbuttcap%
\pgfsetroundjoin%
\definecolor{currentfill}{rgb}{0.000000,0.000000,0.000000}%
\pgfsetfillcolor{currentfill}%
\pgfsetlinewidth{0.501875pt}%
\definecolor{currentstroke}{rgb}{0.000000,0.000000,0.000000}%
\pgfsetstrokecolor{currentstroke}%
\pgfsetdash{}{0pt}%
\pgfsys@defobject{currentmarker}{\pgfqpoint{-0.020833in}{0.000000in}}{\pgfqpoint{-0.000000in}{0.000000in}}{%
\pgfpathmoveto{\pgfqpoint{-0.000000in}{0.000000in}}%
\pgfpathlineto{\pgfqpoint{-0.020833in}{0.000000in}}%
\pgfusepath{stroke,fill}%
}%
\begin{pgfscope}%
\pgfsys@transformshift{2.850886in}{3.548836in}%
\pgfsys@useobject{currentmarker}{}%
\end{pgfscope}%
\end{pgfscope}%
\begin{pgfscope}%
\pgfsetbuttcap%
\pgfsetroundjoin%
\definecolor{currentfill}{rgb}{0.000000,0.000000,0.000000}%
\pgfsetfillcolor{currentfill}%
\pgfsetlinewidth{0.501875pt}%
\definecolor{currentstroke}{rgb}{0.000000,0.000000,0.000000}%
\pgfsetstrokecolor{currentstroke}%
\pgfsetdash{}{0pt}%
\pgfsys@defobject{currentmarker}{\pgfqpoint{0.000000in}{0.000000in}}{\pgfqpoint{0.020833in}{0.000000in}}{%
\pgfpathmoveto{\pgfqpoint{0.000000in}{0.000000in}}%
\pgfpathlineto{\pgfqpoint{0.020833in}{0.000000in}}%
\pgfusepath{stroke,fill}%
}%
\begin{pgfscope}%
\pgfsys@transformshift{0.609415in}{3.632638in}%
\pgfsys@useobject{currentmarker}{}%
\end{pgfscope}%
\end{pgfscope}%
\begin{pgfscope}%
\pgfsetbuttcap%
\pgfsetroundjoin%
\definecolor{currentfill}{rgb}{0.000000,0.000000,0.000000}%
\pgfsetfillcolor{currentfill}%
\pgfsetlinewidth{0.501875pt}%
\definecolor{currentstroke}{rgb}{0.000000,0.000000,0.000000}%
\pgfsetstrokecolor{currentstroke}%
\pgfsetdash{}{0pt}%
\pgfsys@defobject{currentmarker}{\pgfqpoint{-0.020833in}{0.000000in}}{\pgfqpoint{-0.000000in}{0.000000in}}{%
\pgfpathmoveto{\pgfqpoint{-0.000000in}{0.000000in}}%
\pgfpathlineto{\pgfqpoint{-0.020833in}{0.000000in}}%
\pgfusepath{stroke,fill}%
}%
\begin{pgfscope}%
\pgfsys@transformshift{2.850886in}{3.632638in}%
\pgfsys@useobject{currentmarker}{}%
\end{pgfscope}%
\end{pgfscope}%
\begin{pgfscope}%
\pgfsetbuttcap%
\pgfsetroundjoin%
\definecolor{currentfill}{rgb}{0.000000,0.000000,0.000000}%
\pgfsetfillcolor{currentfill}%
\pgfsetlinewidth{0.501875pt}%
\definecolor{currentstroke}{rgb}{0.000000,0.000000,0.000000}%
\pgfsetstrokecolor{currentstroke}%
\pgfsetdash{}{0pt}%
\pgfsys@defobject{currentmarker}{\pgfqpoint{0.000000in}{0.000000in}}{\pgfqpoint{0.020833in}{0.000000in}}{%
\pgfpathmoveto{\pgfqpoint{0.000000in}{0.000000in}}%
\pgfpathlineto{\pgfqpoint{0.020833in}{0.000000in}}%
\pgfusepath{stroke,fill}%
}%
\begin{pgfscope}%
\pgfsys@transformshift{0.609415in}{3.716440in}%
\pgfsys@useobject{currentmarker}{}%
\end{pgfscope}%
\end{pgfscope}%
\begin{pgfscope}%
\pgfsetbuttcap%
\pgfsetroundjoin%
\definecolor{currentfill}{rgb}{0.000000,0.000000,0.000000}%
\pgfsetfillcolor{currentfill}%
\pgfsetlinewidth{0.501875pt}%
\definecolor{currentstroke}{rgb}{0.000000,0.000000,0.000000}%
\pgfsetstrokecolor{currentstroke}%
\pgfsetdash{}{0pt}%
\pgfsys@defobject{currentmarker}{\pgfqpoint{-0.020833in}{0.000000in}}{\pgfqpoint{-0.000000in}{0.000000in}}{%
\pgfpathmoveto{\pgfqpoint{-0.000000in}{0.000000in}}%
\pgfpathlineto{\pgfqpoint{-0.020833in}{0.000000in}}%
\pgfusepath{stroke,fill}%
}%
\begin{pgfscope}%
\pgfsys@transformshift{2.850886in}{3.716440in}%
\pgfsys@useobject{currentmarker}{}%
\end{pgfscope}%
\end{pgfscope}%
\begin{pgfscope}%
\pgfsetbuttcap%
\pgfsetroundjoin%
\definecolor{currentfill}{rgb}{0.000000,0.000000,0.000000}%
\pgfsetfillcolor{currentfill}%
\pgfsetlinewidth{0.501875pt}%
\definecolor{currentstroke}{rgb}{0.000000,0.000000,0.000000}%
\pgfsetstrokecolor{currentstroke}%
\pgfsetdash{}{0pt}%
\pgfsys@defobject{currentmarker}{\pgfqpoint{0.000000in}{0.000000in}}{\pgfqpoint{0.020833in}{0.000000in}}{%
\pgfpathmoveto{\pgfqpoint{0.000000in}{0.000000in}}%
\pgfpathlineto{\pgfqpoint{0.020833in}{0.000000in}}%
\pgfusepath{stroke,fill}%
}%
\begin{pgfscope}%
\pgfsys@transformshift{0.609415in}{3.800241in}%
\pgfsys@useobject{currentmarker}{}%
\end{pgfscope}%
\end{pgfscope}%
\begin{pgfscope}%
\pgfsetbuttcap%
\pgfsetroundjoin%
\definecolor{currentfill}{rgb}{0.000000,0.000000,0.000000}%
\pgfsetfillcolor{currentfill}%
\pgfsetlinewidth{0.501875pt}%
\definecolor{currentstroke}{rgb}{0.000000,0.000000,0.000000}%
\pgfsetstrokecolor{currentstroke}%
\pgfsetdash{}{0pt}%
\pgfsys@defobject{currentmarker}{\pgfqpoint{-0.020833in}{0.000000in}}{\pgfqpoint{-0.000000in}{0.000000in}}{%
\pgfpathmoveto{\pgfqpoint{-0.000000in}{0.000000in}}%
\pgfpathlineto{\pgfqpoint{-0.020833in}{0.000000in}}%
\pgfusepath{stroke,fill}%
}%
\begin{pgfscope}%
\pgfsys@transformshift{2.850886in}{3.800241in}%
\pgfsys@useobject{currentmarker}{}%
\end{pgfscope}%
\end{pgfscope}%
\begin{pgfscope}%
\pgfsetbuttcap%
\pgfsetroundjoin%
\definecolor{currentfill}{rgb}{0.000000,0.000000,0.000000}%
\pgfsetfillcolor{currentfill}%
\pgfsetlinewidth{0.501875pt}%
\definecolor{currentstroke}{rgb}{0.000000,0.000000,0.000000}%
\pgfsetstrokecolor{currentstroke}%
\pgfsetdash{}{0pt}%
\pgfsys@defobject{currentmarker}{\pgfqpoint{0.000000in}{0.000000in}}{\pgfqpoint{0.020833in}{0.000000in}}{%
\pgfpathmoveto{\pgfqpoint{0.000000in}{0.000000in}}%
\pgfpathlineto{\pgfqpoint{0.020833in}{0.000000in}}%
\pgfusepath{stroke,fill}%
}%
\begin{pgfscope}%
\pgfsys@transformshift{0.609415in}{3.967845in}%
\pgfsys@useobject{currentmarker}{}%
\end{pgfscope}%
\end{pgfscope}%
\begin{pgfscope}%
\pgfsetbuttcap%
\pgfsetroundjoin%
\definecolor{currentfill}{rgb}{0.000000,0.000000,0.000000}%
\pgfsetfillcolor{currentfill}%
\pgfsetlinewidth{0.501875pt}%
\definecolor{currentstroke}{rgb}{0.000000,0.000000,0.000000}%
\pgfsetstrokecolor{currentstroke}%
\pgfsetdash{}{0pt}%
\pgfsys@defobject{currentmarker}{\pgfqpoint{-0.020833in}{0.000000in}}{\pgfqpoint{-0.000000in}{0.000000in}}{%
\pgfpathmoveto{\pgfqpoint{-0.000000in}{0.000000in}}%
\pgfpathlineto{\pgfqpoint{-0.020833in}{0.000000in}}%
\pgfusepath{stroke,fill}%
}%
\begin{pgfscope}%
\pgfsys@transformshift{2.850886in}{3.967845in}%
\pgfsys@useobject{currentmarker}{}%
\end{pgfscope}%
\end{pgfscope}%
\begin{pgfscope}%
\pgfsetbuttcap%
\pgfsetroundjoin%
\definecolor{currentfill}{rgb}{0.000000,0.000000,0.000000}%
\pgfsetfillcolor{currentfill}%
\pgfsetlinewidth{0.501875pt}%
\definecolor{currentstroke}{rgb}{0.000000,0.000000,0.000000}%
\pgfsetstrokecolor{currentstroke}%
\pgfsetdash{}{0pt}%
\pgfsys@defobject{currentmarker}{\pgfqpoint{0.000000in}{0.000000in}}{\pgfqpoint{0.020833in}{0.000000in}}{%
\pgfpathmoveto{\pgfqpoint{0.000000in}{0.000000in}}%
\pgfpathlineto{\pgfqpoint{0.020833in}{0.000000in}}%
\pgfusepath{stroke,fill}%
}%
\begin{pgfscope}%
\pgfsys@transformshift{0.609415in}{4.051647in}%
\pgfsys@useobject{currentmarker}{}%
\end{pgfscope}%
\end{pgfscope}%
\begin{pgfscope}%
\pgfsetbuttcap%
\pgfsetroundjoin%
\definecolor{currentfill}{rgb}{0.000000,0.000000,0.000000}%
\pgfsetfillcolor{currentfill}%
\pgfsetlinewidth{0.501875pt}%
\definecolor{currentstroke}{rgb}{0.000000,0.000000,0.000000}%
\pgfsetstrokecolor{currentstroke}%
\pgfsetdash{}{0pt}%
\pgfsys@defobject{currentmarker}{\pgfqpoint{-0.020833in}{0.000000in}}{\pgfqpoint{-0.000000in}{0.000000in}}{%
\pgfpathmoveto{\pgfqpoint{-0.000000in}{0.000000in}}%
\pgfpathlineto{\pgfqpoint{-0.020833in}{0.000000in}}%
\pgfusepath{stroke,fill}%
}%
\begin{pgfscope}%
\pgfsys@transformshift{2.850886in}{4.051647in}%
\pgfsys@useobject{currentmarker}{}%
\end{pgfscope}%
\end{pgfscope}%
\begin{pgfscope}%
\pgfsetbuttcap%
\pgfsetroundjoin%
\definecolor{currentfill}{rgb}{0.000000,0.000000,0.000000}%
\pgfsetfillcolor{currentfill}%
\pgfsetlinewidth{0.501875pt}%
\definecolor{currentstroke}{rgb}{0.000000,0.000000,0.000000}%
\pgfsetstrokecolor{currentstroke}%
\pgfsetdash{}{0pt}%
\pgfsys@defobject{currentmarker}{\pgfqpoint{0.000000in}{0.000000in}}{\pgfqpoint{0.020833in}{0.000000in}}{%
\pgfpathmoveto{\pgfqpoint{0.000000in}{0.000000in}}%
\pgfpathlineto{\pgfqpoint{0.020833in}{0.000000in}}%
\pgfusepath{stroke,fill}%
}%
\begin{pgfscope}%
\pgfsys@transformshift{0.609415in}{4.135449in}%
\pgfsys@useobject{currentmarker}{}%
\end{pgfscope}%
\end{pgfscope}%
\begin{pgfscope}%
\pgfsetbuttcap%
\pgfsetroundjoin%
\definecolor{currentfill}{rgb}{0.000000,0.000000,0.000000}%
\pgfsetfillcolor{currentfill}%
\pgfsetlinewidth{0.501875pt}%
\definecolor{currentstroke}{rgb}{0.000000,0.000000,0.000000}%
\pgfsetstrokecolor{currentstroke}%
\pgfsetdash{}{0pt}%
\pgfsys@defobject{currentmarker}{\pgfqpoint{-0.020833in}{0.000000in}}{\pgfqpoint{-0.000000in}{0.000000in}}{%
\pgfpathmoveto{\pgfqpoint{-0.000000in}{0.000000in}}%
\pgfpathlineto{\pgfqpoint{-0.020833in}{0.000000in}}%
\pgfusepath{stroke,fill}%
}%
\begin{pgfscope}%
\pgfsys@transformshift{2.850886in}{4.135449in}%
\pgfsys@useobject{currentmarker}{}%
\end{pgfscope}%
\end{pgfscope}%
\begin{pgfscope}%
\pgfsetbuttcap%
\pgfsetroundjoin%
\definecolor{currentfill}{rgb}{0.000000,0.000000,0.000000}%
\pgfsetfillcolor{currentfill}%
\pgfsetlinewidth{0.501875pt}%
\definecolor{currentstroke}{rgb}{0.000000,0.000000,0.000000}%
\pgfsetstrokecolor{currentstroke}%
\pgfsetdash{}{0pt}%
\pgfsys@defobject{currentmarker}{\pgfqpoint{0.000000in}{0.000000in}}{\pgfqpoint{0.020833in}{0.000000in}}{%
\pgfpathmoveto{\pgfqpoint{0.000000in}{0.000000in}}%
\pgfpathlineto{\pgfqpoint{0.020833in}{0.000000in}}%
\pgfusepath{stroke,fill}%
}%
\begin{pgfscope}%
\pgfsys@transformshift{0.609415in}{4.219251in}%
\pgfsys@useobject{currentmarker}{}%
\end{pgfscope}%
\end{pgfscope}%
\begin{pgfscope}%
\pgfsetbuttcap%
\pgfsetroundjoin%
\definecolor{currentfill}{rgb}{0.000000,0.000000,0.000000}%
\pgfsetfillcolor{currentfill}%
\pgfsetlinewidth{0.501875pt}%
\definecolor{currentstroke}{rgb}{0.000000,0.000000,0.000000}%
\pgfsetstrokecolor{currentstroke}%
\pgfsetdash{}{0pt}%
\pgfsys@defobject{currentmarker}{\pgfqpoint{-0.020833in}{0.000000in}}{\pgfqpoint{-0.000000in}{0.000000in}}{%
\pgfpathmoveto{\pgfqpoint{-0.000000in}{0.000000in}}%
\pgfpathlineto{\pgfqpoint{-0.020833in}{0.000000in}}%
\pgfusepath{stroke,fill}%
}%
\begin{pgfscope}%
\pgfsys@transformshift{2.850886in}{4.219251in}%
\pgfsys@useobject{currentmarker}{}%
\end{pgfscope}%
\end{pgfscope}%
\begin{pgfscope}%
\definecolor{textcolor}{rgb}{0.000000,0.000000,0.000000}%
\pgfsetstrokecolor{textcolor}%
\pgfsetfillcolor{textcolor}%
\pgftext[x=0.258334in,y=3.561093in,,bottom,rotate=90.000000]{\color{textcolor}\rmfamily\fontsize{10.000000}{12.000000}\selectfont \(\displaystyle T(K)\)}%
\end{pgfscope}%
\begin{pgfscope}%
\pgfpathrectangle{\pgfqpoint{0.609415in}{2.747992in}}{\pgfqpoint{2.241471in}{1.626201in}}%
\pgfusepath{clip}%
\pgfsetrectcap%
\pgfsetroundjoin%
\pgfsetlinewidth{1.003750pt}%
\definecolor{currentstroke}{rgb}{0.047059,0.364706,0.647059}%
\pgfsetstrokecolor{currentstroke}%
\pgfsetdash{}{0pt}%
\pgfpathmoveto{\pgfqpoint{0.631607in}{2.821910in}}%
\pgfpathlineto{\pgfqpoint{0.653800in}{2.835345in}}%
\pgfpathlineto{\pgfqpoint{0.675993in}{2.842744in}}%
\pgfpathlineto{\pgfqpoint{0.698186in}{2.844403in}}%
\pgfpathlineto{\pgfqpoint{0.720379in}{2.837817in}}%
\pgfpathlineto{\pgfqpoint{0.742571in}{2.838933in}}%
\pgfpathlineto{\pgfqpoint{0.764764in}{2.842233in}}%
\pgfpathlineto{\pgfqpoint{0.786957in}{2.844556in}}%
\pgfpathlineto{\pgfqpoint{0.809150in}{2.841808in}}%
\pgfpathlineto{\pgfqpoint{0.831342in}{2.839101in}}%
\pgfpathlineto{\pgfqpoint{0.853535in}{2.840119in}}%
\pgfpathlineto{\pgfqpoint{0.875728in}{2.839700in}}%
\pgfpathlineto{\pgfqpoint{0.897921in}{2.837730in}}%
\pgfpathlineto{\pgfqpoint{0.920114in}{2.838000in}}%
\pgfpathlineto{\pgfqpoint{0.942306in}{2.838435in}}%
\pgfpathlineto{\pgfqpoint{0.964499in}{2.837324in}}%
\pgfpathlineto{\pgfqpoint{0.986692in}{2.835634in}}%
\pgfpathlineto{\pgfqpoint{1.008885in}{2.832961in}}%
\pgfpathlineto{\pgfqpoint{1.031078in}{2.831965in}}%
\pgfpathlineto{\pgfqpoint{1.053270in}{2.832016in}}%
\pgfpathlineto{\pgfqpoint{1.075463in}{2.832448in}}%
\pgfpathlineto{\pgfqpoint{1.097656in}{2.831708in}}%
\pgfpathlineto{\pgfqpoint{1.119849in}{2.832887in}}%
\pgfpathlineto{\pgfqpoint{1.142041in}{2.833665in}}%
\pgfpathlineto{\pgfqpoint{1.164234in}{2.835003in}}%
\pgfpathlineto{\pgfqpoint{1.186427in}{2.833643in}}%
\pgfpathlineto{\pgfqpoint{1.208620in}{2.834745in}}%
\pgfpathlineto{\pgfqpoint{1.230813in}{2.835446in}}%
\pgfpathlineto{\pgfqpoint{1.253005in}{2.834736in}}%
\pgfpathlineto{\pgfqpoint{1.275198in}{2.834775in}}%
\pgfpathlineto{\pgfqpoint{1.297391in}{2.836715in}}%
\pgfpathlineto{\pgfqpoint{1.319584in}{2.836411in}}%
\pgfpathlineto{\pgfqpoint{1.341777in}{2.836184in}}%
\pgfpathlineto{\pgfqpoint{1.363969in}{2.838068in}}%
\pgfpathlineto{\pgfqpoint{1.386162in}{2.838965in}}%
\pgfpathlineto{\pgfqpoint{1.408355in}{2.838953in}}%
\pgfpathlineto{\pgfqpoint{1.430548in}{2.839136in}}%
\pgfpathlineto{\pgfqpoint{1.452740in}{2.838752in}}%
\pgfpathlineto{\pgfqpoint{1.474933in}{2.839471in}}%
\pgfpathlineto{\pgfqpoint{1.497126in}{2.840048in}}%
\pgfpathlineto{\pgfqpoint{1.519319in}{2.841673in}}%
\pgfpathlineto{\pgfqpoint{1.541512in}{2.843258in}}%
\pgfpathlineto{\pgfqpoint{1.563704in}{2.843463in}}%
\pgfpathlineto{\pgfqpoint{1.585897in}{2.843974in}}%
\pgfpathlineto{\pgfqpoint{1.608090in}{2.845259in}}%
\pgfpathlineto{\pgfqpoint{1.630283in}{2.846707in}}%
\pgfpathlineto{\pgfqpoint{1.652476in}{2.846980in}}%
\pgfpathlineto{\pgfqpoint{1.674668in}{2.847485in}}%
\pgfpathlineto{\pgfqpoint{1.696861in}{2.847722in}}%
\pgfpathlineto{\pgfqpoint{1.719054in}{2.847939in}}%
\pgfpathlineto{\pgfqpoint{1.741247in}{2.848959in}}%
\pgfpathlineto{\pgfqpoint{1.763439in}{2.849190in}}%
\pgfpathlineto{\pgfqpoint{1.785632in}{2.848997in}}%
\pgfpathlineto{\pgfqpoint{1.807825in}{2.848764in}}%
\pgfpathlineto{\pgfqpoint{1.830018in}{2.849179in}}%
\pgfpathlineto{\pgfqpoint{1.852211in}{2.849838in}}%
\pgfpathlineto{\pgfqpoint{1.874403in}{2.849864in}}%
\pgfpathlineto{\pgfqpoint{1.896596in}{2.850233in}}%
\pgfpathlineto{\pgfqpoint{1.918789in}{2.850869in}}%
\pgfpathlineto{\pgfqpoint{1.940982in}{2.850620in}}%
\pgfpathlineto{\pgfqpoint{1.963175in}{2.850448in}}%
\pgfpathlineto{\pgfqpoint{1.985367in}{2.851113in}}%
\pgfpathlineto{\pgfqpoint{2.007560in}{2.850447in}}%
\pgfpathlineto{\pgfqpoint{2.029753in}{2.850562in}}%
\pgfpathlineto{\pgfqpoint{2.051946in}{2.851233in}}%
\pgfpathlineto{\pgfqpoint{2.074139in}{2.850955in}}%
\pgfpathlineto{\pgfqpoint{2.096331in}{2.850558in}}%
\pgfpathlineto{\pgfqpoint{2.118524in}{2.850388in}}%
\pgfpathlineto{\pgfqpoint{2.140717in}{2.850809in}}%
\pgfpathlineto{\pgfqpoint{2.162910in}{2.850540in}}%
\pgfpathlineto{\pgfqpoint{2.185102in}{2.850899in}}%
\pgfpathlineto{\pgfqpoint{2.207295in}{2.850224in}}%
\pgfpathlineto{\pgfqpoint{2.229488in}{2.850628in}}%
\pgfpathlineto{\pgfqpoint{2.251681in}{2.851248in}}%
\pgfpathlineto{\pgfqpoint{2.273874in}{2.851122in}}%
\pgfpathlineto{\pgfqpoint{2.296066in}{2.851336in}}%
\pgfpathlineto{\pgfqpoint{2.318259in}{2.850861in}}%
\pgfpathlineto{\pgfqpoint{2.340452in}{2.851313in}}%
\pgfpathlineto{\pgfqpoint{2.362645in}{2.851745in}}%
\pgfpathlineto{\pgfqpoint{2.384838in}{2.851589in}}%
\pgfpathlineto{\pgfqpoint{2.407030in}{2.852143in}}%
\pgfpathlineto{\pgfqpoint{2.429223in}{2.852196in}}%
\pgfpathlineto{\pgfqpoint{2.451416in}{2.852453in}}%
\pgfpathlineto{\pgfqpoint{2.473609in}{2.852716in}}%
\pgfpathlineto{\pgfqpoint{2.495801in}{2.852596in}}%
\pgfpathlineto{\pgfqpoint{2.517994in}{2.852424in}}%
\pgfpathlineto{\pgfqpoint{2.540187in}{2.852377in}}%
\pgfpathlineto{\pgfqpoint{2.562380in}{2.852645in}}%
\pgfpathlineto{\pgfqpoint{2.584573in}{2.852863in}}%
\pgfpathlineto{\pgfqpoint{2.606765in}{2.853403in}}%
\pgfpathlineto{\pgfqpoint{2.628958in}{2.853638in}}%
\pgfpathlineto{\pgfqpoint{2.651151in}{2.854017in}}%
\pgfpathlineto{\pgfqpoint{2.673344in}{2.854285in}}%
\pgfpathlineto{\pgfqpoint{2.695537in}{2.854687in}}%
\pgfpathlineto{\pgfqpoint{2.717729in}{2.855400in}}%
\pgfpathlineto{\pgfqpoint{2.739922in}{2.855181in}}%
\pgfpathlineto{\pgfqpoint{2.762115in}{2.855378in}}%
\pgfpathlineto{\pgfqpoint{2.784308in}{2.855757in}}%
\pgfpathlineto{\pgfqpoint{2.806500in}{2.855823in}}%
\pgfusepath{stroke}%
\end{pgfscope}%
\begin{pgfscope}%
\pgfpathrectangle{\pgfqpoint{0.609415in}{2.747992in}}{\pgfqpoint{2.241471in}{1.626201in}}%
\pgfusepath{clip}%
\pgfsetrectcap%
\pgfsetroundjoin%
\pgfsetlinewidth{1.003750pt}%
\definecolor{currentstroke}{rgb}{0.000000,0.725490,0.270588}%
\pgfsetstrokecolor{currentstroke}%
\pgfsetdash{}{0pt}%
\pgfpathmoveto{\pgfqpoint{0.631607in}{4.089842in}}%
\pgfpathlineto{\pgfqpoint{0.653800in}{4.089537in}}%
\pgfpathlineto{\pgfqpoint{0.675993in}{4.086284in}}%
\pgfpathlineto{\pgfqpoint{0.698186in}{4.085113in}}%
\pgfpathlineto{\pgfqpoint{0.720379in}{4.083341in}}%
\pgfpathlineto{\pgfqpoint{0.742571in}{4.082845in}}%
\pgfpathlineto{\pgfqpoint{0.764764in}{4.080031in}}%
\pgfpathlineto{\pgfqpoint{0.786957in}{4.078458in}}%
\pgfpathlineto{\pgfqpoint{0.809150in}{4.078144in}}%
\pgfpathlineto{\pgfqpoint{0.831342in}{4.077494in}}%
\pgfpathlineto{\pgfqpoint{0.853535in}{4.076442in}}%
\pgfpathlineto{\pgfqpoint{0.875728in}{4.076181in}}%
\pgfpathlineto{\pgfqpoint{0.897921in}{4.075716in}}%
\pgfpathlineto{\pgfqpoint{0.920114in}{4.075248in}}%
\pgfpathlineto{\pgfqpoint{0.942306in}{4.075243in}}%
\pgfpathlineto{\pgfqpoint{0.964499in}{4.075276in}}%
\pgfpathlineto{\pgfqpoint{0.986692in}{4.074134in}}%
\pgfpathlineto{\pgfqpoint{1.008885in}{4.073761in}}%
\pgfpathlineto{\pgfqpoint{1.031078in}{4.073443in}}%
\pgfpathlineto{\pgfqpoint{1.053270in}{4.072344in}}%
\pgfpathlineto{\pgfqpoint{1.075463in}{4.071545in}}%
\pgfpathlineto{\pgfqpoint{1.097656in}{4.071214in}}%
\pgfpathlineto{\pgfqpoint{1.119849in}{4.070417in}}%
\pgfpathlineto{\pgfqpoint{1.142041in}{4.070433in}}%
\pgfpathlineto{\pgfqpoint{1.164234in}{4.070499in}}%
\pgfpathlineto{\pgfqpoint{1.186427in}{4.070137in}}%
\pgfpathlineto{\pgfqpoint{1.208620in}{4.069799in}}%
\pgfpathlineto{\pgfqpoint{1.230813in}{4.069720in}}%
\pgfpathlineto{\pgfqpoint{1.253005in}{4.069518in}}%
\pgfpathlineto{\pgfqpoint{1.275198in}{4.069375in}}%
\pgfpathlineto{\pgfqpoint{1.297391in}{4.069228in}}%
\pgfpathlineto{\pgfqpoint{1.319584in}{4.069105in}}%
\pgfpathlineto{\pgfqpoint{1.341777in}{4.068961in}}%
\pgfpathlineto{\pgfqpoint{1.363969in}{4.068610in}}%
\pgfpathlineto{\pgfqpoint{1.386162in}{4.068470in}}%
\pgfpathlineto{\pgfqpoint{1.408355in}{4.068124in}}%
\pgfpathlineto{\pgfqpoint{1.430548in}{4.067838in}}%
\pgfpathlineto{\pgfqpoint{1.452740in}{4.067513in}}%
\pgfpathlineto{\pgfqpoint{1.474933in}{4.067458in}}%
\pgfpathlineto{\pgfqpoint{1.497126in}{4.067229in}}%
\pgfpathlineto{\pgfqpoint{1.519319in}{4.066840in}}%
\pgfpathlineto{\pgfqpoint{1.541512in}{4.066639in}}%
\pgfpathlineto{\pgfqpoint{1.563704in}{4.065869in}}%
\pgfpathlineto{\pgfqpoint{1.585897in}{4.065968in}}%
\pgfpathlineto{\pgfqpoint{1.608090in}{4.065791in}}%
\pgfpathlineto{\pgfqpoint{1.630283in}{4.065640in}}%
\pgfpathlineto{\pgfqpoint{1.652476in}{4.065252in}}%
\pgfpathlineto{\pgfqpoint{1.674668in}{4.065066in}}%
\pgfpathlineto{\pgfqpoint{1.696861in}{4.064876in}}%
\pgfpathlineto{\pgfqpoint{1.719054in}{4.064623in}}%
\pgfpathlineto{\pgfqpoint{1.741247in}{4.064259in}}%
\pgfpathlineto{\pgfqpoint{1.763439in}{4.063913in}}%
\pgfpathlineto{\pgfqpoint{1.785632in}{4.063854in}}%
\pgfpathlineto{\pgfqpoint{1.807825in}{4.063579in}}%
\pgfpathlineto{\pgfqpoint{1.830018in}{4.063536in}}%
\pgfpathlineto{\pgfqpoint{1.852211in}{4.063446in}}%
\pgfpathlineto{\pgfqpoint{1.874403in}{4.063083in}}%
\pgfpathlineto{\pgfqpoint{1.896596in}{4.062861in}}%
\pgfpathlineto{\pgfqpoint{1.918789in}{4.062787in}}%
\pgfpathlineto{\pgfqpoint{1.940982in}{4.062702in}}%
\pgfpathlineto{\pgfqpoint{1.963175in}{4.062662in}}%
\pgfpathlineto{\pgfqpoint{1.985367in}{4.062612in}}%
\pgfpathlineto{\pgfqpoint{2.007560in}{4.062462in}}%
\pgfpathlineto{\pgfqpoint{2.029753in}{4.062348in}}%
\pgfpathlineto{\pgfqpoint{2.051946in}{4.062292in}}%
\pgfpathlineto{\pgfqpoint{2.074139in}{4.062340in}}%
\pgfpathlineto{\pgfqpoint{2.096331in}{4.062254in}}%
\pgfpathlineto{\pgfqpoint{2.118524in}{4.062259in}}%
\pgfpathlineto{\pgfqpoint{2.140717in}{4.062057in}}%
\pgfpathlineto{\pgfqpoint{2.162910in}{4.061865in}}%
\pgfpathlineto{\pgfqpoint{2.185102in}{4.061628in}}%
\pgfpathlineto{\pgfqpoint{2.207295in}{4.061350in}}%
\pgfpathlineto{\pgfqpoint{2.229488in}{4.061041in}}%
\pgfpathlineto{\pgfqpoint{2.251681in}{4.060873in}}%
\pgfpathlineto{\pgfqpoint{2.273874in}{4.060727in}}%
\pgfpathlineto{\pgfqpoint{2.296066in}{4.060563in}}%
\pgfpathlineto{\pgfqpoint{2.318259in}{4.060440in}}%
\pgfpathlineto{\pgfqpoint{2.340452in}{4.060312in}}%
\pgfpathlineto{\pgfqpoint{2.362645in}{4.060276in}}%
\pgfpathlineto{\pgfqpoint{2.384838in}{4.060163in}}%
\pgfpathlineto{\pgfqpoint{2.407030in}{4.059941in}}%
\pgfpathlineto{\pgfqpoint{2.429223in}{4.059940in}}%
\pgfpathlineto{\pgfqpoint{2.451416in}{4.059782in}}%
\pgfpathlineto{\pgfqpoint{2.473609in}{4.059494in}}%
\pgfpathlineto{\pgfqpoint{2.495801in}{4.059399in}}%
\pgfpathlineto{\pgfqpoint{2.517994in}{4.059378in}}%
\pgfpathlineto{\pgfqpoint{2.540187in}{4.059155in}}%
\pgfpathlineto{\pgfqpoint{2.562380in}{4.058982in}}%
\pgfpathlineto{\pgfqpoint{2.584573in}{4.058832in}}%
\pgfpathlineto{\pgfqpoint{2.606765in}{4.058816in}}%
\pgfpathlineto{\pgfqpoint{2.628958in}{4.058552in}}%
\pgfpathlineto{\pgfqpoint{2.651151in}{4.058299in}}%
\pgfpathlineto{\pgfqpoint{2.673344in}{4.058075in}}%
\pgfpathlineto{\pgfqpoint{2.695537in}{4.057908in}}%
\pgfpathlineto{\pgfqpoint{2.717729in}{4.057718in}}%
\pgfpathlineto{\pgfqpoint{2.739922in}{4.057439in}}%
\pgfpathlineto{\pgfqpoint{2.762115in}{4.057160in}}%
\pgfpathlineto{\pgfqpoint{2.784308in}{4.057085in}}%
\pgfpathlineto{\pgfqpoint{2.806500in}{4.056922in}}%
\pgfusepath{stroke}%
\end{pgfscope}%
\begin{pgfscope}%
\pgfpathrectangle{\pgfqpoint{0.609415in}{2.747992in}}{\pgfqpoint{2.241471in}{1.626201in}}%
\pgfusepath{clip}%
\pgfsetrectcap%
\pgfsetroundjoin%
\pgfsetlinewidth{1.003750pt}%
\definecolor{currentstroke}{rgb}{1.000000,0.584314,0.000000}%
\pgfsetstrokecolor{currentstroke}%
\pgfsetdash{}{0pt}%
\pgfpathmoveto{\pgfqpoint{0.631607in}{4.300275in}}%
\pgfpathlineto{\pgfqpoint{0.653800in}{4.299204in}}%
\pgfpathlineto{\pgfqpoint{0.675993in}{4.298009in}}%
\pgfpathlineto{\pgfqpoint{0.698186in}{4.296810in}}%
\pgfpathlineto{\pgfqpoint{0.720379in}{4.295646in}}%
\pgfpathlineto{\pgfqpoint{0.742571in}{4.294443in}}%
\pgfpathlineto{\pgfqpoint{0.764764in}{4.293261in}}%
\pgfpathlineto{\pgfqpoint{0.786957in}{4.292152in}}%
\pgfpathlineto{\pgfqpoint{0.809150in}{4.290895in}}%
\pgfpathlineto{\pgfqpoint{0.831342in}{4.289868in}}%
\pgfpathlineto{\pgfqpoint{0.853535in}{4.288673in}}%
\pgfpathlineto{\pgfqpoint{0.875728in}{4.287479in}}%
\pgfpathlineto{\pgfqpoint{0.897921in}{4.286248in}}%
\pgfpathlineto{\pgfqpoint{0.920114in}{4.284966in}}%
\pgfpathlineto{\pgfqpoint{0.942306in}{4.283799in}}%
\pgfpathlineto{\pgfqpoint{0.964499in}{4.282542in}}%
\pgfpathlineto{\pgfqpoint{0.986692in}{4.281343in}}%
\pgfpathlineto{\pgfqpoint{1.008885in}{4.280054in}}%
\pgfpathlineto{\pgfqpoint{1.031078in}{4.278762in}}%
\pgfpathlineto{\pgfqpoint{1.053270in}{4.277583in}}%
\pgfpathlineto{\pgfqpoint{1.075463in}{4.276358in}}%
\pgfpathlineto{\pgfqpoint{1.097656in}{4.275167in}}%
\pgfpathlineto{\pgfqpoint{1.119849in}{4.273913in}}%
\pgfpathlineto{\pgfqpoint{1.142041in}{4.272522in}}%
\pgfpathlineto{\pgfqpoint{1.164234in}{4.271190in}}%
\pgfpathlineto{\pgfqpoint{1.186427in}{4.269879in}}%
\pgfpathlineto{\pgfqpoint{1.208620in}{4.268556in}}%
\pgfpathlineto{\pgfqpoint{1.230813in}{4.267285in}}%
\pgfpathlineto{\pgfqpoint{1.253005in}{4.265878in}}%
\pgfpathlineto{\pgfqpoint{1.275198in}{4.264571in}}%
\pgfpathlineto{\pgfqpoint{1.297391in}{4.263321in}}%
\pgfpathlineto{\pgfqpoint{1.319584in}{4.262054in}}%
\pgfpathlineto{\pgfqpoint{1.341777in}{4.260681in}}%
\pgfpathlineto{\pgfqpoint{1.363969in}{4.259371in}}%
\pgfpathlineto{\pgfqpoint{1.386162in}{4.257984in}}%
\pgfpathlineto{\pgfqpoint{1.408355in}{4.256851in}}%
\pgfpathlineto{\pgfqpoint{1.430548in}{4.255706in}}%
\pgfpathlineto{\pgfqpoint{1.452740in}{4.254360in}}%
\pgfpathlineto{\pgfqpoint{1.474933in}{4.253049in}}%
\pgfpathlineto{\pgfqpoint{1.497126in}{4.251705in}}%
\pgfpathlineto{\pgfqpoint{1.519319in}{4.250364in}}%
\pgfpathlineto{\pgfqpoint{1.541512in}{4.248980in}}%
\pgfpathlineto{\pgfqpoint{1.563704in}{4.247649in}}%
\pgfpathlineto{\pgfqpoint{1.585897in}{4.246429in}}%
\pgfpathlineto{\pgfqpoint{1.608090in}{4.245201in}}%
\pgfpathlineto{\pgfqpoint{1.630283in}{4.243984in}}%
\pgfpathlineto{\pgfqpoint{1.652476in}{4.242796in}}%
\pgfpathlineto{\pgfqpoint{1.674668in}{4.241573in}}%
\pgfpathlineto{\pgfqpoint{1.696861in}{4.240406in}}%
\pgfpathlineto{\pgfqpoint{1.719054in}{4.239246in}}%
\pgfpathlineto{\pgfqpoint{1.741247in}{4.238082in}}%
\pgfpathlineto{\pgfqpoint{1.763439in}{4.236913in}}%
\pgfpathlineto{\pgfqpoint{1.785632in}{4.235774in}}%
\pgfpathlineto{\pgfqpoint{1.807825in}{4.234704in}}%
\pgfpathlineto{\pgfqpoint{1.830018in}{4.233616in}}%
\pgfpathlineto{\pgfqpoint{1.852211in}{4.232520in}}%
\pgfpathlineto{\pgfqpoint{1.874403in}{4.231429in}}%
\pgfpathlineto{\pgfqpoint{1.896596in}{4.230394in}}%
\pgfpathlineto{\pgfqpoint{1.918789in}{4.229308in}}%
\pgfpathlineto{\pgfqpoint{1.940982in}{4.228392in}}%
\pgfpathlineto{\pgfqpoint{1.963175in}{4.227201in}}%
\pgfpathlineto{\pgfqpoint{1.985367in}{4.226114in}}%
\pgfpathlineto{\pgfqpoint{2.007560in}{4.224912in}}%
\pgfpathlineto{\pgfqpoint{2.029753in}{4.223885in}}%
\pgfpathlineto{\pgfqpoint{2.051946in}{4.222814in}}%
\pgfpathlineto{\pgfqpoint{2.074139in}{4.221849in}}%
\pgfpathlineto{\pgfqpoint{2.096331in}{4.220826in}}%
\pgfpathlineto{\pgfqpoint{2.118524in}{4.219755in}}%
\pgfpathlineto{\pgfqpoint{2.140717in}{4.218690in}}%
\pgfpathlineto{\pgfqpoint{2.162910in}{4.217783in}}%
\pgfpathlineto{\pgfqpoint{2.185102in}{4.216786in}}%
\pgfpathlineto{\pgfqpoint{2.207295in}{4.215807in}}%
\pgfpathlineto{\pgfqpoint{2.229488in}{4.214812in}}%
\pgfpathlineto{\pgfqpoint{2.251681in}{4.213753in}}%
\pgfpathlineto{\pgfqpoint{2.273874in}{4.212795in}}%
\pgfpathlineto{\pgfqpoint{2.296066in}{4.211687in}}%
\pgfpathlineto{\pgfqpoint{2.318259in}{4.210730in}}%
\pgfpathlineto{\pgfqpoint{2.340452in}{4.209712in}}%
\pgfpathlineto{\pgfqpoint{2.362645in}{4.208786in}}%
\pgfpathlineto{\pgfqpoint{2.384838in}{4.207813in}}%
\pgfpathlineto{\pgfqpoint{2.407030in}{4.206869in}}%
\pgfpathlineto{\pgfqpoint{2.429223in}{4.205899in}}%
\pgfpathlineto{\pgfqpoint{2.451416in}{4.204956in}}%
\pgfpathlineto{\pgfqpoint{2.473609in}{4.204046in}}%
\pgfpathlineto{\pgfqpoint{2.495801in}{4.203131in}}%
\pgfpathlineto{\pgfqpoint{2.517994in}{4.202299in}}%
\pgfpathlineto{\pgfqpoint{2.540187in}{4.201367in}}%
\pgfpathlineto{\pgfqpoint{2.562380in}{4.200388in}}%
\pgfpathlineto{\pgfqpoint{2.584573in}{4.199392in}}%
\pgfpathlineto{\pgfqpoint{2.606765in}{4.198422in}}%
\pgfpathlineto{\pgfqpoint{2.628958in}{4.197507in}}%
\pgfpathlineto{\pgfqpoint{2.651151in}{4.196607in}}%
\pgfpathlineto{\pgfqpoint{2.673344in}{4.195746in}}%
\pgfpathlineto{\pgfqpoint{2.695537in}{4.194887in}}%
\pgfpathlineto{\pgfqpoint{2.717729in}{4.193971in}}%
\pgfpathlineto{\pgfqpoint{2.739922in}{4.193049in}}%
\pgfpathlineto{\pgfqpoint{2.762115in}{4.192234in}}%
\pgfpathlineto{\pgfqpoint{2.784308in}{4.191415in}}%
\pgfpathlineto{\pgfqpoint{2.806500in}{4.190586in}}%
\pgfusepath{stroke}%
\end{pgfscope}%
\begin{pgfscope}%
\pgfpathrectangle{\pgfqpoint{0.609415in}{2.747992in}}{\pgfqpoint{2.241471in}{1.626201in}}%
\pgfusepath{clip}%
\pgfsetrectcap%
\pgfsetroundjoin%
\pgfsetlinewidth{1.003750pt}%
\definecolor{currentstroke}{rgb}{1.000000,0.172549,0.000000}%
\pgfsetstrokecolor{currentstroke}%
\pgfsetdash{}{0pt}%
\pgfpathmoveto{\pgfqpoint{0.631607in}{4.227889in}}%
\pgfpathlineto{\pgfqpoint{0.653800in}{4.228070in}}%
\pgfpathlineto{\pgfqpoint{0.675993in}{4.225670in}}%
\pgfpathlineto{\pgfqpoint{0.698186in}{4.226572in}}%
\pgfpathlineto{\pgfqpoint{0.720379in}{4.227968in}}%
\pgfpathlineto{\pgfqpoint{0.742571in}{4.228261in}}%
\pgfpathlineto{\pgfqpoint{0.764764in}{4.228149in}}%
\pgfpathlineto{\pgfqpoint{0.786957in}{4.228561in}}%
\pgfpathlineto{\pgfqpoint{0.809150in}{4.228367in}}%
\pgfpathlineto{\pgfqpoint{0.831342in}{4.228422in}}%
\pgfpathlineto{\pgfqpoint{0.853535in}{4.228075in}}%
\pgfpathlineto{\pgfqpoint{0.875728in}{4.228483in}}%
\pgfpathlineto{\pgfqpoint{0.897921in}{4.228421in}}%
\pgfpathlineto{\pgfqpoint{0.920114in}{4.228193in}}%
\pgfpathlineto{\pgfqpoint{0.942306in}{4.228484in}}%
\pgfpathlineto{\pgfqpoint{0.964499in}{4.228408in}}%
\pgfpathlineto{\pgfqpoint{0.986692in}{4.228119in}}%
\pgfpathlineto{\pgfqpoint{1.008885in}{4.228286in}}%
\pgfpathlineto{\pgfqpoint{1.031078in}{4.228162in}}%
\pgfpathlineto{\pgfqpoint{1.053270in}{4.228240in}}%
\pgfpathlineto{\pgfqpoint{1.075463in}{4.227794in}}%
\pgfpathlineto{\pgfqpoint{1.097656in}{4.227568in}}%
\pgfpathlineto{\pgfqpoint{1.119849in}{4.227651in}}%
\pgfpathlineto{\pgfqpoint{1.142041in}{4.227445in}}%
\pgfpathlineto{\pgfqpoint{1.164234in}{4.227192in}}%
\pgfpathlineto{\pgfqpoint{1.186427in}{4.226913in}}%
\pgfpathlineto{\pgfqpoint{1.208620in}{4.226975in}}%
\pgfpathlineto{\pgfqpoint{1.230813in}{4.226670in}}%
\pgfpathlineto{\pgfqpoint{1.253005in}{4.226477in}}%
\pgfpathlineto{\pgfqpoint{1.275198in}{4.226353in}}%
\pgfpathlineto{\pgfqpoint{1.297391in}{4.226280in}}%
\pgfpathlineto{\pgfqpoint{1.319584in}{4.225823in}}%
\pgfpathlineto{\pgfqpoint{1.341777in}{4.225358in}}%
\pgfpathlineto{\pgfqpoint{1.363969in}{4.225224in}}%
\pgfpathlineto{\pgfqpoint{1.386162in}{4.224991in}}%
\pgfpathlineto{\pgfqpoint{1.408355in}{4.224916in}}%
\pgfpathlineto{\pgfqpoint{1.430548in}{4.224788in}}%
\pgfpathlineto{\pgfqpoint{1.452740in}{4.224720in}}%
\pgfpathlineto{\pgfqpoint{1.474933in}{4.224616in}}%
\pgfpathlineto{\pgfqpoint{1.497126in}{4.224535in}}%
\pgfpathlineto{\pgfqpoint{1.519319in}{4.224433in}}%
\pgfpathlineto{\pgfqpoint{1.541512in}{4.224284in}}%
\pgfpathlineto{\pgfqpoint{1.563704in}{4.224196in}}%
\pgfpathlineto{\pgfqpoint{1.585897in}{4.224105in}}%
\pgfpathlineto{\pgfqpoint{1.608090in}{4.223841in}}%
\pgfpathlineto{\pgfqpoint{1.630283in}{4.223677in}}%
\pgfpathlineto{\pgfqpoint{1.652476in}{4.223480in}}%
\pgfpathlineto{\pgfqpoint{1.674668in}{4.223281in}}%
\pgfpathlineto{\pgfqpoint{1.696861in}{4.223084in}}%
\pgfpathlineto{\pgfqpoint{1.719054in}{4.222906in}}%
\pgfpathlineto{\pgfqpoint{1.741247in}{4.222722in}}%
\pgfpathlineto{\pgfqpoint{1.763439in}{4.222664in}}%
\pgfpathlineto{\pgfqpoint{1.785632in}{4.222688in}}%
\pgfpathlineto{\pgfqpoint{1.807825in}{4.222427in}}%
\pgfpathlineto{\pgfqpoint{1.830018in}{4.222225in}}%
\pgfpathlineto{\pgfqpoint{1.852211in}{4.222097in}}%
\pgfpathlineto{\pgfqpoint{1.874403in}{4.221976in}}%
\pgfpathlineto{\pgfqpoint{1.896596in}{4.221742in}}%
\pgfpathlineto{\pgfqpoint{1.918789in}{4.221598in}}%
\pgfpathlineto{\pgfqpoint{1.940982in}{4.221480in}}%
\pgfpathlineto{\pgfqpoint{1.963175in}{4.221411in}}%
\pgfpathlineto{\pgfqpoint{1.985367in}{4.221384in}}%
\pgfpathlineto{\pgfqpoint{2.007560in}{4.221238in}}%
\pgfpathlineto{\pgfqpoint{2.029753in}{4.221195in}}%
\pgfpathlineto{\pgfqpoint{2.051946in}{4.221083in}}%
\pgfpathlineto{\pgfqpoint{2.074139in}{4.220976in}}%
\pgfpathlineto{\pgfqpoint{2.096331in}{4.220821in}}%
\pgfpathlineto{\pgfqpoint{2.118524in}{4.220657in}}%
\pgfpathlineto{\pgfqpoint{2.140717in}{4.220562in}}%
\pgfpathlineto{\pgfqpoint{2.162910in}{4.220519in}}%
\pgfpathlineto{\pgfqpoint{2.185102in}{4.220379in}}%
\pgfpathlineto{\pgfqpoint{2.207295in}{4.220271in}}%
\pgfpathlineto{\pgfqpoint{2.229488in}{4.220179in}}%
\pgfpathlineto{\pgfqpoint{2.251681in}{4.220027in}}%
\pgfpathlineto{\pgfqpoint{2.273874in}{4.219888in}}%
\pgfpathlineto{\pgfqpoint{2.296066in}{4.219794in}}%
\pgfpathlineto{\pgfqpoint{2.318259in}{4.219637in}}%
\pgfpathlineto{\pgfqpoint{2.340452in}{4.219469in}}%
\pgfpathlineto{\pgfqpoint{2.362645in}{4.219373in}}%
\pgfpathlineto{\pgfqpoint{2.384838in}{4.219290in}}%
\pgfpathlineto{\pgfqpoint{2.407030in}{4.219126in}}%
\pgfpathlineto{\pgfqpoint{2.429223in}{4.218980in}}%
\pgfpathlineto{\pgfqpoint{2.451416in}{4.218819in}}%
\pgfpathlineto{\pgfqpoint{2.473609in}{4.218584in}}%
\pgfpathlineto{\pgfqpoint{2.495801in}{4.218519in}}%
\pgfpathlineto{\pgfqpoint{2.517994in}{4.218358in}}%
\pgfpathlineto{\pgfqpoint{2.540187in}{4.218240in}}%
\pgfpathlineto{\pgfqpoint{2.562380in}{4.218117in}}%
\pgfpathlineto{\pgfqpoint{2.584573in}{4.217946in}}%
\pgfpathlineto{\pgfqpoint{2.606765in}{4.217809in}}%
\pgfpathlineto{\pgfqpoint{2.628958in}{4.217684in}}%
\pgfpathlineto{\pgfqpoint{2.651151in}{4.217485in}}%
\pgfpathlineto{\pgfqpoint{2.673344in}{4.217334in}}%
\pgfpathlineto{\pgfqpoint{2.695537in}{4.217178in}}%
\pgfpathlineto{\pgfqpoint{2.717729in}{4.217033in}}%
\pgfpathlineto{\pgfqpoint{2.739922in}{4.216909in}}%
\pgfpathlineto{\pgfqpoint{2.762115in}{4.216734in}}%
\pgfpathlineto{\pgfqpoint{2.784308in}{4.216583in}}%
\pgfpathlineto{\pgfqpoint{2.806500in}{4.216430in}}%
\pgfusepath{stroke}%
\end{pgfscope}%
\begin{pgfscope}%
\pgfpathrectangle{\pgfqpoint{0.609415in}{2.747992in}}{\pgfqpoint{2.241471in}{1.626201in}}%
\pgfusepath{clip}%
\pgfsetrectcap%
\pgfsetroundjoin%
\pgfsetlinewidth{1.003750pt}%
\definecolor{currentstroke}{rgb}{0.517647,0.356863,0.592157}%
\pgfsetstrokecolor{currentstroke}%
\pgfsetdash{}{0pt}%
\pgfpathmoveto{\pgfqpoint{0.631607in}{3.693598in}}%
\pgfpathlineto{\pgfqpoint{0.653800in}{3.684034in}}%
\pgfpathlineto{\pgfqpoint{0.675993in}{3.678806in}}%
\pgfpathlineto{\pgfqpoint{0.698186in}{3.670485in}}%
\pgfpathlineto{\pgfqpoint{0.720379in}{3.666551in}}%
\pgfpathlineto{\pgfqpoint{0.742571in}{3.658255in}}%
\pgfpathlineto{\pgfqpoint{0.764764in}{3.656492in}}%
\pgfpathlineto{\pgfqpoint{0.786957in}{3.653854in}}%
\pgfpathlineto{\pgfqpoint{0.809150in}{3.651412in}}%
\pgfpathlineto{\pgfqpoint{0.831342in}{3.648027in}}%
\pgfpathlineto{\pgfqpoint{0.853535in}{3.643355in}}%
\pgfpathlineto{\pgfqpoint{0.875728in}{3.641740in}}%
\pgfpathlineto{\pgfqpoint{0.897921in}{3.637290in}}%
\pgfpathlineto{\pgfqpoint{0.920114in}{3.635957in}}%
\pgfpathlineto{\pgfqpoint{0.942306in}{3.633384in}}%
\pgfpathlineto{\pgfqpoint{0.964499in}{3.631896in}}%
\pgfpathlineto{\pgfqpoint{0.986692in}{3.631859in}}%
\pgfpathlineto{\pgfqpoint{1.008885in}{3.632114in}}%
\pgfpathlineto{\pgfqpoint{1.031078in}{3.633174in}}%
\pgfpathlineto{\pgfqpoint{1.053270in}{3.632019in}}%
\pgfpathlineto{\pgfqpoint{1.075463in}{3.632144in}}%
\pgfpathlineto{\pgfqpoint{1.097656in}{3.630675in}}%
\pgfpathlineto{\pgfqpoint{1.119849in}{3.630505in}}%
\pgfpathlineto{\pgfqpoint{1.142041in}{3.628347in}}%
\pgfpathlineto{\pgfqpoint{1.164234in}{3.626475in}}%
\pgfpathlineto{\pgfqpoint{1.186427in}{3.625444in}}%
\pgfpathlineto{\pgfqpoint{1.208620in}{3.624260in}}%
\pgfpathlineto{\pgfqpoint{1.230813in}{3.623294in}}%
\pgfpathlineto{\pgfqpoint{1.253005in}{3.622937in}}%
\pgfpathlineto{\pgfqpoint{1.275198in}{3.621767in}}%
\pgfpathlineto{\pgfqpoint{1.297391in}{3.621139in}}%
\pgfpathlineto{\pgfqpoint{1.319584in}{3.619938in}}%
\pgfpathlineto{\pgfqpoint{1.341777in}{3.619508in}}%
\pgfpathlineto{\pgfqpoint{1.363969in}{3.618292in}}%
\pgfpathlineto{\pgfqpoint{1.386162in}{3.617993in}}%
\pgfpathlineto{\pgfqpoint{1.408355in}{3.617032in}}%
\pgfpathlineto{\pgfqpoint{1.430548in}{3.616535in}}%
\pgfpathlineto{\pgfqpoint{1.452740in}{3.615856in}}%
\pgfpathlineto{\pgfqpoint{1.474933in}{3.614807in}}%
\pgfpathlineto{\pgfqpoint{1.497126in}{3.613843in}}%
\pgfpathlineto{\pgfqpoint{1.519319in}{3.613250in}}%
\pgfpathlineto{\pgfqpoint{1.541512in}{3.612598in}}%
\pgfpathlineto{\pgfqpoint{1.563704in}{3.612373in}}%
\pgfpathlineto{\pgfqpoint{1.585897in}{3.612213in}}%
\pgfpathlineto{\pgfqpoint{1.608090in}{3.611907in}}%
\pgfpathlineto{\pgfqpoint{1.630283in}{3.610790in}}%
\pgfpathlineto{\pgfqpoint{1.652476in}{3.609662in}}%
\pgfpathlineto{\pgfqpoint{1.674668in}{3.609262in}}%
\pgfpathlineto{\pgfqpoint{1.696861in}{3.608980in}}%
\pgfpathlineto{\pgfqpoint{1.719054in}{3.608375in}}%
\pgfpathlineto{\pgfqpoint{1.741247in}{3.607995in}}%
\pgfpathlineto{\pgfqpoint{1.763439in}{3.607213in}}%
\pgfpathlineto{\pgfqpoint{1.785632in}{3.606465in}}%
\pgfpathlineto{\pgfqpoint{1.807825in}{3.605549in}}%
\pgfpathlineto{\pgfqpoint{1.830018in}{3.605342in}}%
\pgfpathlineto{\pgfqpoint{1.852211in}{3.604751in}}%
\pgfpathlineto{\pgfqpoint{1.874403in}{3.605047in}}%
\pgfpathlineto{\pgfqpoint{1.896596in}{3.604895in}}%
\pgfpathlineto{\pgfqpoint{1.918789in}{3.604550in}}%
\pgfpathlineto{\pgfqpoint{1.940982in}{3.604038in}}%
\pgfpathlineto{\pgfqpoint{1.963175in}{3.603538in}}%
\pgfpathlineto{\pgfqpoint{1.985367in}{3.603453in}}%
\pgfpathlineto{\pgfqpoint{2.007560in}{3.603142in}}%
\pgfpathlineto{\pgfqpoint{2.029753in}{3.602919in}}%
\pgfpathlineto{\pgfqpoint{2.051946in}{3.602768in}}%
\pgfpathlineto{\pgfqpoint{2.074139in}{3.602520in}}%
\pgfpathlineto{\pgfqpoint{2.096331in}{3.602483in}}%
\pgfpathlineto{\pgfqpoint{2.118524in}{3.601994in}}%
\pgfpathlineto{\pgfqpoint{2.140717in}{3.601659in}}%
\pgfpathlineto{\pgfqpoint{2.162910in}{3.601782in}}%
\pgfpathlineto{\pgfqpoint{2.185102in}{3.601707in}}%
\pgfpathlineto{\pgfqpoint{2.207295in}{3.601548in}}%
\pgfpathlineto{\pgfqpoint{2.229488in}{3.601737in}}%
\pgfpathlineto{\pgfqpoint{2.251681in}{3.601285in}}%
\pgfpathlineto{\pgfqpoint{2.273874in}{3.600811in}}%
\pgfpathlineto{\pgfqpoint{2.296066in}{3.600561in}}%
\pgfpathlineto{\pgfqpoint{2.318259in}{3.600500in}}%
\pgfpathlineto{\pgfqpoint{2.340452in}{3.599999in}}%
\pgfpathlineto{\pgfqpoint{2.362645in}{3.600096in}}%
\pgfpathlineto{\pgfqpoint{2.384838in}{3.599622in}}%
\pgfpathlineto{\pgfqpoint{2.407030in}{3.599013in}}%
\pgfpathlineto{\pgfqpoint{2.429223in}{3.598889in}}%
\pgfpathlineto{\pgfqpoint{2.451416in}{3.598383in}}%
\pgfpathlineto{\pgfqpoint{2.473609in}{3.597906in}}%
\pgfpathlineto{\pgfqpoint{2.495801in}{3.597715in}}%
\pgfpathlineto{\pgfqpoint{2.517994in}{3.597383in}}%
\pgfpathlineto{\pgfqpoint{2.540187in}{3.597047in}}%
\pgfpathlineto{\pgfqpoint{2.562380in}{3.597047in}}%
\pgfpathlineto{\pgfqpoint{2.584573in}{3.596883in}}%
\pgfpathlineto{\pgfqpoint{2.606765in}{3.596844in}}%
\pgfpathlineto{\pgfqpoint{2.628958in}{3.596707in}}%
\pgfpathlineto{\pgfqpoint{2.651151in}{3.596470in}}%
\pgfpathlineto{\pgfqpoint{2.673344in}{3.596618in}}%
\pgfpathlineto{\pgfqpoint{2.695537in}{3.596480in}}%
\pgfpathlineto{\pgfqpoint{2.717729in}{3.596251in}}%
\pgfpathlineto{\pgfqpoint{2.739922in}{3.595903in}}%
\pgfpathlineto{\pgfqpoint{2.762115in}{3.595900in}}%
\pgfpathlineto{\pgfqpoint{2.784308in}{3.595578in}}%
\pgfpathlineto{\pgfqpoint{2.806500in}{3.595357in}}%
\pgfusepath{stroke}%
\end{pgfscope}%
\begin{pgfscope}%
\pgfsetrectcap%
\pgfsetmiterjoin%
\pgfsetlinewidth{0.501875pt}%
\definecolor{currentstroke}{rgb}{0.000000,0.000000,0.000000}%
\pgfsetstrokecolor{currentstroke}%
\pgfsetdash{}{0pt}%
\pgfpathmoveto{\pgfqpoint{0.609415in}{2.747992in}}%
\pgfpathlineto{\pgfqpoint{0.609415in}{4.374193in}}%
\pgfusepath{stroke}%
\end{pgfscope}%
\begin{pgfscope}%
\pgfsetrectcap%
\pgfsetmiterjoin%
\pgfsetlinewidth{0.501875pt}%
\definecolor{currentstroke}{rgb}{0.000000,0.000000,0.000000}%
\pgfsetstrokecolor{currentstroke}%
\pgfsetdash{}{0pt}%
\pgfpathmoveto{\pgfqpoint{2.850886in}{2.747992in}}%
\pgfpathlineto{\pgfqpoint{2.850886in}{4.374193in}}%
\pgfusepath{stroke}%
\end{pgfscope}%
\begin{pgfscope}%
\pgfsetrectcap%
\pgfsetmiterjoin%
\pgfsetlinewidth{0.501875pt}%
\definecolor{currentstroke}{rgb}{0.000000,0.000000,0.000000}%
\pgfsetstrokecolor{currentstroke}%
\pgfsetdash{}{0pt}%
\pgfpathmoveto{\pgfqpoint{0.609415in}{2.747992in}}%
\pgfpathlineto{\pgfqpoint{2.850886in}{2.747992in}}%
\pgfusepath{stroke}%
\end{pgfscope}%
\begin{pgfscope}%
\pgfsetrectcap%
\pgfsetmiterjoin%
\pgfsetlinewidth{0.501875pt}%
\definecolor{currentstroke}{rgb}{0.000000,0.000000,0.000000}%
\pgfsetstrokecolor{currentstroke}%
\pgfsetdash{}{0pt}%
\pgfpathmoveto{\pgfqpoint{0.609415in}{4.374193in}}%
\pgfpathlineto{\pgfqpoint{2.850886in}{4.374193in}}%
\pgfusepath{stroke}%
\end{pgfscope}%
\begin{pgfscope}%
\definecolor{textcolor}{rgb}{0.000000,0.000000,0.000000}%
\pgfsetstrokecolor{textcolor}%
\pgfsetfillcolor{textcolor}%
\pgftext[x=1.730150in,y=4.457526in,,base]{\color{textcolor}\rmfamily\fontsize{12.000000}{14.400000}\selectfont Trustworthiness}%
\end{pgfscope}%
\begin{pgfscope}%
\pgfsetbuttcap%
\pgfsetmiterjoin%
\definecolor{currentfill}{rgb}{1.000000,1.000000,1.000000}%
\pgfsetfillcolor{currentfill}%
\pgfsetlinewidth{0.000000pt}%
\definecolor{currentstroke}{rgb}{0.000000,0.000000,0.000000}%
\pgfsetstrokecolor{currentstroke}%
\pgfsetstrokeopacity{0.000000}%
\pgfsetdash{}{0pt}%
\pgfpathmoveto{\pgfqpoint{0.609415in}{0.422992in}}%
\pgfpathlineto{\pgfqpoint{2.850886in}{0.422992in}}%
\pgfpathlineto{\pgfqpoint{2.850886in}{2.049193in}}%
\pgfpathlineto{\pgfqpoint{0.609415in}{2.049193in}}%
\pgfpathlineto{\pgfqpoint{0.609415in}{0.422992in}}%
\pgfpathclose%
\pgfusepath{fill}%
\end{pgfscope}%
\begin{pgfscope}%
\pgfsetbuttcap%
\pgfsetroundjoin%
\definecolor{currentfill}{rgb}{0.000000,0.000000,0.000000}%
\pgfsetfillcolor{currentfill}%
\pgfsetlinewidth{0.501875pt}%
\definecolor{currentstroke}{rgb}{0.000000,0.000000,0.000000}%
\pgfsetstrokecolor{currentstroke}%
\pgfsetdash{}{0pt}%
\pgfsys@defobject{currentmarker}{\pgfqpoint{0.000000in}{0.000000in}}{\pgfqpoint{0.000000in}{0.041667in}}{%
\pgfpathmoveto{\pgfqpoint{0.000000in}{0.000000in}}%
\pgfpathlineto{\pgfqpoint{0.000000in}{0.041667in}}%
\pgfusepath{stroke,fill}%
}%
\begin{pgfscope}%
\pgfsys@transformshift{0.609415in}{0.422992in}%
\pgfsys@useobject{currentmarker}{}%
\end{pgfscope}%
\end{pgfscope}%
\begin{pgfscope}%
\pgfsetbuttcap%
\pgfsetroundjoin%
\definecolor{currentfill}{rgb}{0.000000,0.000000,0.000000}%
\pgfsetfillcolor{currentfill}%
\pgfsetlinewidth{0.501875pt}%
\definecolor{currentstroke}{rgb}{0.000000,0.000000,0.000000}%
\pgfsetstrokecolor{currentstroke}%
\pgfsetdash{}{0pt}%
\pgfsys@defobject{currentmarker}{\pgfqpoint{0.000000in}{-0.041667in}}{\pgfqpoint{0.000000in}{0.000000in}}{%
\pgfpathmoveto{\pgfqpoint{0.000000in}{0.000000in}}%
\pgfpathlineto{\pgfqpoint{0.000000in}{-0.041667in}}%
\pgfusepath{stroke,fill}%
}%
\begin{pgfscope}%
\pgfsys@transformshift{0.609415in}{2.049193in}%
\pgfsys@useobject{currentmarker}{}%
\end{pgfscope}%
\end{pgfscope}%
\begin{pgfscope}%
\definecolor{textcolor}{rgb}{0.000000,0.000000,0.000000}%
\pgfsetstrokecolor{textcolor}%
\pgfsetfillcolor{textcolor}%
\pgftext[x=0.609415in,y=0.374381in,,top]{\color{textcolor}\rmfamily\fontsize{10.000000}{12.000000}\selectfont \(\displaystyle {0}\)}%
\end{pgfscope}%
\begin{pgfscope}%
\pgfsetbuttcap%
\pgfsetroundjoin%
\definecolor{currentfill}{rgb}{0.000000,0.000000,0.000000}%
\pgfsetfillcolor{currentfill}%
\pgfsetlinewidth{0.501875pt}%
\definecolor{currentstroke}{rgb}{0.000000,0.000000,0.000000}%
\pgfsetstrokecolor{currentstroke}%
\pgfsetdash{}{0pt}%
\pgfsys@defobject{currentmarker}{\pgfqpoint{0.000000in}{0.000000in}}{\pgfqpoint{0.000000in}{0.041667in}}{%
\pgfpathmoveto{\pgfqpoint{0.000000in}{0.000000in}}%
\pgfpathlineto{\pgfqpoint{0.000000in}{0.041667in}}%
\pgfusepath{stroke,fill}%
}%
\begin{pgfscope}%
\pgfsys@transformshift{1.053270in}{0.422992in}%
\pgfsys@useobject{currentmarker}{}%
\end{pgfscope}%
\end{pgfscope}%
\begin{pgfscope}%
\pgfsetbuttcap%
\pgfsetroundjoin%
\definecolor{currentfill}{rgb}{0.000000,0.000000,0.000000}%
\pgfsetfillcolor{currentfill}%
\pgfsetlinewidth{0.501875pt}%
\definecolor{currentstroke}{rgb}{0.000000,0.000000,0.000000}%
\pgfsetstrokecolor{currentstroke}%
\pgfsetdash{}{0pt}%
\pgfsys@defobject{currentmarker}{\pgfqpoint{0.000000in}{-0.041667in}}{\pgfqpoint{0.000000in}{0.000000in}}{%
\pgfpathmoveto{\pgfqpoint{0.000000in}{0.000000in}}%
\pgfpathlineto{\pgfqpoint{0.000000in}{-0.041667in}}%
\pgfusepath{stroke,fill}%
}%
\begin{pgfscope}%
\pgfsys@transformshift{1.053270in}{2.049193in}%
\pgfsys@useobject{currentmarker}{}%
\end{pgfscope}%
\end{pgfscope}%
\begin{pgfscope}%
\definecolor{textcolor}{rgb}{0.000000,0.000000,0.000000}%
\pgfsetstrokecolor{textcolor}%
\pgfsetfillcolor{textcolor}%
\pgftext[x=1.053270in,y=0.374381in,,top]{\color{textcolor}\rmfamily\fontsize{10.000000}{12.000000}\selectfont \(\displaystyle {20}\)}%
\end{pgfscope}%
\begin{pgfscope}%
\pgfsetbuttcap%
\pgfsetroundjoin%
\definecolor{currentfill}{rgb}{0.000000,0.000000,0.000000}%
\pgfsetfillcolor{currentfill}%
\pgfsetlinewidth{0.501875pt}%
\definecolor{currentstroke}{rgb}{0.000000,0.000000,0.000000}%
\pgfsetstrokecolor{currentstroke}%
\pgfsetdash{}{0pt}%
\pgfsys@defobject{currentmarker}{\pgfqpoint{0.000000in}{0.000000in}}{\pgfqpoint{0.000000in}{0.041667in}}{%
\pgfpathmoveto{\pgfqpoint{0.000000in}{0.000000in}}%
\pgfpathlineto{\pgfqpoint{0.000000in}{0.041667in}}%
\pgfusepath{stroke,fill}%
}%
\begin{pgfscope}%
\pgfsys@transformshift{1.497126in}{0.422992in}%
\pgfsys@useobject{currentmarker}{}%
\end{pgfscope}%
\end{pgfscope}%
\begin{pgfscope}%
\pgfsetbuttcap%
\pgfsetroundjoin%
\definecolor{currentfill}{rgb}{0.000000,0.000000,0.000000}%
\pgfsetfillcolor{currentfill}%
\pgfsetlinewidth{0.501875pt}%
\definecolor{currentstroke}{rgb}{0.000000,0.000000,0.000000}%
\pgfsetstrokecolor{currentstroke}%
\pgfsetdash{}{0pt}%
\pgfsys@defobject{currentmarker}{\pgfqpoint{0.000000in}{-0.041667in}}{\pgfqpoint{0.000000in}{0.000000in}}{%
\pgfpathmoveto{\pgfqpoint{0.000000in}{0.000000in}}%
\pgfpathlineto{\pgfqpoint{0.000000in}{-0.041667in}}%
\pgfusepath{stroke,fill}%
}%
\begin{pgfscope}%
\pgfsys@transformshift{1.497126in}{2.049193in}%
\pgfsys@useobject{currentmarker}{}%
\end{pgfscope}%
\end{pgfscope}%
\begin{pgfscope}%
\definecolor{textcolor}{rgb}{0.000000,0.000000,0.000000}%
\pgfsetstrokecolor{textcolor}%
\pgfsetfillcolor{textcolor}%
\pgftext[x=1.497126in,y=0.374381in,,top]{\color{textcolor}\rmfamily\fontsize{10.000000}{12.000000}\selectfont \(\displaystyle {40}\)}%
\end{pgfscope}%
\begin{pgfscope}%
\pgfsetbuttcap%
\pgfsetroundjoin%
\definecolor{currentfill}{rgb}{0.000000,0.000000,0.000000}%
\pgfsetfillcolor{currentfill}%
\pgfsetlinewidth{0.501875pt}%
\definecolor{currentstroke}{rgb}{0.000000,0.000000,0.000000}%
\pgfsetstrokecolor{currentstroke}%
\pgfsetdash{}{0pt}%
\pgfsys@defobject{currentmarker}{\pgfqpoint{0.000000in}{0.000000in}}{\pgfqpoint{0.000000in}{0.041667in}}{%
\pgfpathmoveto{\pgfqpoint{0.000000in}{0.000000in}}%
\pgfpathlineto{\pgfqpoint{0.000000in}{0.041667in}}%
\pgfusepath{stroke,fill}%
}%
\begin{pgfscope}%
\pgfsys@transformshift{1.940982in}{0.422992in}%
\pgfsys@useobject{currentmarker}{}%
\end{pgfscope}%
\end{pgfscope}%
\begin{pgfscope}%
\pgfsetbuttcap%
\pgfsetroundjoin%
\definecolor{currentfill}{rgb}{0.000000,0.000000,0.000000}%
\pgfsetfillcolor{currentfill}%
\pgfsetlinewidth{0.501875pt}%
\definecolor{currentstroke}{rgb}{0.000000,0.000000,0.000000}%
\pgfsetstrokecolor{currentstroke}%
\pgfsetdash{}{0pt}%
\pgfsys@defobject{currentmarker}{\pgfqpoint{0.000000in}{-0.041667in}}{\pgfqpoint{0.000000in}{0.000000in}}{%
\pgfpathmoveto{\pgfqpoint{0.000000in}{0.000000in}}%
\pgfpathlineto{\pgfqpoint{0.000000in}{-0.041667in}}%
\pgfusepath{stroke,fill}%
}%
\begin{pgfscope}%
\pgfsys@transformshift{1.940982in}{2.049193in}%
\pgfsys@useobject{currentmarker}{}%
\end{pgfscope}%
\end{pgfscope}%
\begin{pgfscope}%
\definecolor{textcolor}{rgb}{0.000000,0.000000,0.000000}%
\pgfsetstrokecolor{textcolor}%
\pgfsetfillcolor{textcolor}%
\pgftext[x=1.940982in,y=0.374381in,,top]{\color{textcolor}\rmfamily\fontsize{10.000000}{12.000000}\selectfont \(\displaystyle {60}\)}%
\end{pgfscope}%
\begin{pgfscope}%
\pgfsetbuttcap%
\pgfsetroundjoin%
\definecolor{currentfill}{rgb}{0.000000,0.000000,0.000000}%
\pgfsetfillcolor{currentfill}%
\pgfsetlinewidth{0.501875pt}%
\definecolor{currentstroke}{rgb}{0.000000,0.000000,0.000000}%
\pgfsetstrokecolor{currentstroke}%
\pgfsetdash{}{0pt}%
\pgfsys@defobject{currentmarker}{\pgfqpoint{0.000000in}{0.000000in}}{\pgfqpoint{0.000000in}{0.041667in}}{%
\pgfpathmoveto{\pgfqpoint{0.000000in}{0.000000in}}%
\pgfpathlineto{\pgfqpoint{0.000000in}{0.041667in}}%
\pgfusepath{stroke,fill}%
}%
\begin{pgfscope}%
\pgfsys@transformshift{2.384838in}{0.422992in}%
\pgfsys@useobject{currentmarker}{}%
\end{pgfscope}%
\end{pgfscope}%
\begin{pgfscope}%
\pgfsetbuttcap%
\pgfsetroundjoin%
\definecolor{currentfill}{rgb}{0.000000,0.000000,0.000000}%
\pgfsetfillcolor{currentfill}%
\pgfsetlinewidth{0.501875pt}%
\definecolor{currentstroke}{rgb}{0.000000,0.000000,0.000000}%
\pgfsetstrokecolor{currentstroke}%
\pgfsetdash{}{0pt}%
\pgfsys@defobject{currentmarker}{\pgfqpoint{0.000000in}{-0.041667in}}{\pgfqpoint{0.000000in}{0.000000in}}{%
\pgfpathmoveto{\pgfqpoint{0.000000in}{0.000000in}}%
\pgfpathlineto{\pgfqpoint{0.000000in}{-0.041667in}}%
\pgfusepath{stroke,fill}%
}%
\begin{pgfscope}%
\pgfsys@transformshift{2.384838in}{2.049193in}%
\pgfsys@useobject{currentmarker}{}%
\end{pgfscope}%
\end{pgfscope}%
\begin{pgfscope}%
\definecolor{textcolor}{rgb}{0.000000,0.000000,0.000000}%
\pgfsetstrokecolor{textcolor}%
\pgfsetfillcolor{textcolor}%
\pgftext[x=2.384838in,y=0.374381in,,top]{\color{textcolor}\rmfamily\fontsize{10.000000}{12.000000}\selectfont \(\displaystyle {80}\)}%
\end{pgfscope}%
\begin{pgfscope}%
\pgfsetbuttcap%
\pgfsetroundjoin%
\definecolor{currentfill}{rgb}{0.000000,0.000000,0.000000}%
\pgfsetfillcolor{currentfill}%
\pgfsetlinewidth{0.501875pt}%
\definecolor{currentstroke}{rgb}{0.000000,0.000000,0.000000}%
\pgfsetstrokecolor{currentstroke}%
\pgfsetdash{}{0pt}%
\pgfsys@defobject{currentmarker}{\pgfqpoint{0.000000in}{0.000000in}}{\pgfqpoint{0.000000in}{0.020833in}}{%
\pgfpathmoveto{\pgfqpoint{0.000000in}{0.000000in}}%
\pgfpathlineto{\pgfqpoint{0.000000in}{0.020833in}}%
\pgfusepath{stroke,fill}%
}%
\begin{pgfscope}%
\pgfsys@transformshift{0.720379in}{0.422992in}%
\pgfsys@useobject{currentmarker}{}%
\end{pgfscope}%
\end{pgfscope}%
\begin{pgfscope}%
\pgfsetbuttcap%
\pgfsetroundjoin%
\definecolor{currentfill}{rgb}{0.000000,0.000000,0.000000}%
\pgfsetfillcolor{currentfill}%
\pgfsetlinewidth{0.501875pt}%
\definecolor{currentstroke}{rgb}{0.000000,0.000000,0.000000}%
\pgfsetstrokecolor{currentstroke}%
\pgfsetdash{}{0pt}%
\pgfsys@defobject{currentmarker}{\pgfqpoint{0.000000in}{-0.020833in}}{\pgfqpoint{0.000000in}{0.000000in}}{%
\pgfpathmoveto{\pgfqpoint{0.000000in}{0.000000in}}%
\pgfpathlineto{\pgfqpoint{0.000000in}{-0.020833in}}%
\pgfusepath{stroke,fill}%
}%
\begin{pgfscope}%
\pgfsys@transformshift{0.720379in}{2.049193in}%
\pgfsys@useobject{currentmarker}{}%
\end{pgfscope}%
\end{pgfscope}%
\begin{pgfscope}%
\pgfsetbuttcap%
\pgfsetroundjoin%
\definecolor{currentfill}{rgb}{0.000000,0.000000,0.000000}%
\pgfsetfillcolor{currentfill}%
\pgfsetlinewidth{0.501875pt}%
\definecolor{currentstroke}{rgb}{0.000000,0.000000,0.000000}%
\pgfsetstrokecolor{currentstroke}%
\pgfsetdash{}{0pt}%
\pgfsys@defobject{currentmarker}{\pgfqpoint{0.000000in}{0.000000in}}{\pgfqpoint{0.000000in}{0.020833in}}{%
\pgfpathmoveto{\pgfqpoint{0.000000in}{0.000000in}}%
\pgfpathlineto{\pgfqpoint{0.000000in}{0.020833in}}%
\pgfusepath{stroke,fill}%
}%
\begin{pgfscope}%
\pgfsys@transformshift{0.831342in}{0.422992in}%
\pgfsys@useobject{currentmarker}{}%
\end{pgfscope}%
\end{pgfscope}%
\begin{pgfscope}%
\pgfsetbuttcap%
\pgfsetroundjoin%
\definecolor{currentfill}{rgb}{0.000000,0.000000,0.000000}%
\pgfsetfillcolor{currentfill}%
\pgfsetlinewidth{0.501875pt}%
\definecolor{currentstroke}{rgb}{0.000000,0.000000,0.000000}%
\pgfsetstrokecolor{currentstroke}%
\pgfsetdash{}{0pt}%
\pgfsys@defobject{currentmarker}{\pgfqpoint{0.000000in}{-0.020833in}}{\pgfqpoint{0.000000in}{0.000000in}}{%
\pgfpathmoveto{\pgfqpoint{0.000000in}{0.000000in}}%
\pgfpathlineto{\pgfqpoint{0.000000in}{-0.020833in}}%
\pgfusepath{stroke,fill}%
}%
\begin{pgfscope}%
\pgfsys@transformshift{0.831342in}{2.049193in}%
\pgfsys@useobject{currentmarker}{}%
\end{pgfscope}%
\end{pgfscope}%
\begin{pgfscope}%
\pgfsetbuttcap%
\pgfsetroundjoin%
\definecolor{currentfill}{rgb}{0.000000,0.000000,0.000000}%
\pgfsetfillcolor{currentfill}%
\pgfsetlinewidth{0.501875pt}%
\definecolor{currentstroke}{rgb}{0.000000,0.000000,0.000000}%
\pgfsetstrokecolor{currentstroke}%
\pgfsetdash{}{0pt}%
\pgfsys@defobject{currentmarker}{\pgfqpoint{0.000000in}{0.000000in}}{\pgfqpoint{0.000000in}{0.020833in}}{%
\pgfpathmoveto{\pgfqpoint{0.000000in}{0.000000in}}%
\pgfpathlineto{\pgfqpoint{0.000000in}{0.020833in}}%
\pgfusepath{stroke,fill}%
}%
\begin{pgfscope}%
\pgfsys@transformshift{0.942306in}{0.422992in}%
\pgfsys@useobject{currentmarker}{}%
\end{pgfscope}%
\end{pgfscope}%
\begin{pgfscope}%
\pgfsetbuttcap%
\pgfsetroundjoin%
\definecolor{currentfill}{rgb}{0.000000,0.000000,0.000000}%
\pgfsetfillcolor{currentfill}%
\pgfsetlinewidth{0.501875pt}%
\definecolor{currentstroke}{rgb}{0.000000,0.000000,0.000000}%
\pgfsetstrokecolor{currentstroke}%
\pgfsetdash{}{0pt}%
\pgfsys@defobject{currentmarker}{\pgfqpoint{0.000000in}{-0.020833in}}{\pgfqpoint{0.000000in}{0.000000in}}{%
\pgfpathmoveto{\pgfqpoint{0.000000in}{0.000000in}}%
\pgfpathlineto{\pgfqpoint{0.000000in}{-0.020833in}}%
\pgfusepath{stroke,fill}%
}%
\begin{pgfscope}%
\pgfsys@transformshift{0.942306in}{2.049193in}%
\pgfsys@useobject{currentmarker}{}%
\end{pgfscope}%
\end{pgfscope}%
\begin{pgfscope}%
\pgfsetbuttcap%
\pgfsetroundjoin%
\definecolor{currentfill}{rgb}{0.000000,0.000000,0.000000}%
\pgfsetfillcolor{currentfill}%
\pgfsetlinewidth{0.501875pt}%
\definecolor{currentstroke}{rgb}{0.000000,0.000000,0.000000}%
\pgfsetstrokecolor{currentstroke}%
\pgfsetdash{}{0pt}%
\pgfsys@defobject{currentmarker}{\pgfqpoint{0.000000in}{0.000000in}}{\pgfqpoint{0.000000in}{0.020833in}}{%
\pgfpathmoveto{\pgfqpoint{0.000000in}{0.000000in}}%
\pgfpathlineto{\pgfqpoint{0.000000in}{0.020833in}}%
\pgfusepath{stroke,fill}%
}%
\begin{pgfscope}%
\pgfsys@transformshift{1.164234in}{0.422992in}%
\pgfsys@useobject{currentmarker}{}%
\end{pgfscope}%
\end{pgfscope}%
\begin{pgfscope}%
\pgfsetbuttcap%
\pgfsetroundjoin%
\definecolor{currentfill}{rgb}{0.000000,0.000000,0.000000}%
\pgfsetfillcolor{currentfill}%
\pgfsetlinewidth{0.501875pt}%
\definecolor{currentstroke}{rgb}{0.000000,0.000000,0.000000}%
\pgfsetstrokecolor{currentstroke}%
\pgfsetdash{}{0pt}%
\pgfsys@defobject{currentmarker}{\pgfqpoint{0.000000in}{-0.020833in}}{\pgfqpoint{0.000000in}{0.000000in}}{%
\pgfpathmoveto{\pgfqpoint{0.000000in}{0.000000in}}%
\pgfpathlineto{\pgfqpoint{0.000000in}{-0.020833in}}%
\pgfusepath{stroke,fill}%
}%
\begin{pgfscope}%
\pgfsys@transformshift{1.164234in}{2.049193in}%
\pgfsys@useobject{currentmarker}{}%
\end{pgfscope}%
\end{pgfscope}%
\begin{pgfscope}%
\pgfsetbuttcap%
\pgfsetroundjoin%
\definecolor{currentfill}{rgb}{0.000000,0.000000,0.000000}%
\pgfsetfillcolor{currentfill}%
\pgfsetlinewidth{0.501875pt}%
\definecolor{currentstroke}{rgb}{0.000000,0.000000,0.000000}%
\pgfsetstrokecolor{currentstroke}%
\pgfsetdash{}{0pt}%
\pgfsys@defobject{currentmarker}{\pgfqpoint{0.000000in}{0.000000in}}{\pgfqpoint{0.000000in}{0.020833in}}{%
\pgfpathmoveto{\pgfqpoint{0.000000in}{0.000000in}}%
\pgfpathlineto{\pgfqpoint{0.000000in}{0.020833in}}%
\pgfusepath{stroke,fill}%
}%
\begin{pgfscope}%
\pgfsys@transformshift{1.275198in}{0.422992in}%
\pgfsys@useobject{currentmarker}{}%
\end{pgfscope}%
\end{pgfscope}%
\begin{pgfscope}%
\pgfsetbuttcap%
\pgfsetroundjoin%
\definecolor{currentfill}{rgb}{0.000000,0.000000,0.000000}%
\pgfsetfillcolor{currentfill}%
\pgfsetlinewidth{0.501875pt}%
\definecolor{currentstroke}{rgb}{0.000000,0.000000,0.000000}%
\pgfsetstrokecolor{currentstroke}%
\pgfsetdash{}{0pt}%
\pgfsys@defobject{currentmarker}{\pgfqpoint{0.000000in}{-0.020833in}}{\pgfqpoint{0.000000in}{0.000000in}}{%
\pgfpathmoveto{\pgfqpoint{0.000000in}{0.000000in}}%
\pgfpathlineto{\pgfqpoint{0.000000in}{-0.020833in}}%
\pgfusepath{stroke,fill}%
}%
\begin{pgfscope}%
\pgfsys@transformshift{1.275198in}{2.049193in}%
\pgfsys@useobject{currentmarker}{}%
\end{pgfscope}%
\end{pgfscope}%
\begin{pgfscope}%
\pgfsetbuttcap%
\pgfsetroundjoin%
\definecolor{currentfill}{rgb}{0.000000,0.000000,0.000000}%
\pgfsetfillcolor{currentfill}%
\pgfsetlinewidth{0.501875pt}%
\definecolor{currentstroke}{rgb}{0.000000,0.000000,0.000000}%
\pgfsetstrokecolor{currentstroke}%
\pgfsetdash{}{0pt}%
\pgfsys@defobject{currentmarker}{\pgfqpoint{0.000000in}{0.000000in}}{\pgfqpoint{0.000000in}{0.020833in}}{%
\pgfpathmoveto{\pgfqpoint{0.000000in}{0.000000in}}%
\pgfpathlineto{\pgfqpoint{0.000000in}{0.020833in}}%
\pgfusepath{stroke,fill}%
}%
\begin{pgfscope}%
\pgfsys@transformshift{1.386162in}{0.422992in}%
\pgfsys@useobject{currentmarker}{}%
\end{pgfscope}%
\end{pgfscope}%
\begin{pgfscope}%
\pgfsetbuttcap%
\pgfsetroundjoin%
\definecolor{currentfill}{rgb}{0.000000,0.000000,0.000000}%
\pgfsetfillcolor{currentfill}%
\pgfsetlinewidth{0.501875pt}%
\definecolor{currentstroke}{rgb}{0.000000,0.000000,0.000000}%
\pgfsetstrokecolor{currentstroke}%
\pgfsetdash{}{0pt}%
\pgfsys@defobject{currentmarker}{\pgfqpoint{0.000000in}{-0.020833in}}{\pgfqpoint{0.000000in}{0.000000in}}{%
\pgfpathmoveto{\pgfqpoint{0.000000in}{0.000000in}}%
\pgfpathlineto{\pgfqpoint{0.000000in}{-0.020833in}}%
\pgfusepath{stroke,fill}%
}%
\begin{pgfscope}%
\pgfsys@transformshift{1.386162in}{2.049193in}%
\pgfsys@useobject{currentmarker}{}%
\end{pgfscope}%
\end{pgfscope}%
\begin{pgfscope}%
\pgfsetbuttcap%
\pgfsetroundjoin%
\definecolor{currentfill}{rgb}{0.000000,0.000000,0.000000}%
\pgfsetfillcolor{currentfill}%
\pgfsetlinewidth{0.501875pt}%
\definecolor{currentstroke}{rgb}{0.000000,0.000000,0.000000}%
\pgfsetstrokecolor{currentstroke}%
\pgfsetdash{}{0pt}%
\pgfsys@defobject{currentmarker}{\pgfqpoint{0.000000in}{0.000000in}}{\pgfqpoint{0.000000in}{0.020833in}}{%
\pgfpathmoveto{\pgfqpoint{0.000000in}{0.000000in}}%
\pgfpathlineto{\pgfqpoint{0.000000in}{0.020833in}}%
\pgfusepath{stroke,fill}%
}%
\begin{pgfscope}%
\pgfsys@transformshift{1.608090in}{0.422992in}%
\pgfsys@useobject{currentmarker}{}%
\end{pgfscope}%
\end{pgfscope}%
\begin{pgfscope}%
\pgfsetbuttcap%
\pgfsetroundjoin%
\definecolor{currentfill}{rgb}{0.000000,0.000000,0.000000}%
\pgfsetfillcolor{currentfill}%
\pgfsetlinewidth{0.501875pt}%
\definecolor{currentstroke}{rgb}{0.000000,0.000000,0.000000}%
\pgfsetstrokecolor{currentstroke}%
\pgfsetdash{}{0pt}%
\pgfsys@defobject{currentmarker}{\pgfqpoint{0.000000in}{-0.020833in}}{\pgfqpoint{0.000000in}{0.000000in}}{%
\pgfpathmoveto{\pgfqpoint{0.000000in}{0.000000in}}%
\pgfpathlineto{\pgfqpoint{0.000000in}{-0.020833in}}%
\pgfusepath{stroke,fill}%
}%
\begin{pgfscope}%
\pgfsys@transformshift{1.608090in}{2.049193in}%
\pgfsys@useobject{currentmarker}{}%
\end{pgfscope}%
\end{pgfscope}%
\begin{pgfscope}%
\pgfsetbuttcap%
\pgfsetroundjoin%
\definecolor{currentfill}{rgb}{0.000000,0.000000,0.000000}%
\pgfsetfillcolor{currentfill}%
\pgfsetlinewidth{0.501875pt}%
\definecolor{currentstroke}{rgb}{0.000000,0.000000,0.000000}%
\pgfsetstrokecolor{currentstroke}%
\pgfsetdash{}{0pt}%
\pgfsys@defobject{currentmarker}{\pgfqpoint{0.000000in}{0.000000in}}{\pgfqpoint{0.000000in}{0.020833in}}{%
\pgfpathmoveto{\pgfqpoint{0.000000in}{0.000000in}}%
\pgfpathlineto{\pgfqpoint{0.000000in}{0.020833in}}%
\pgfusepath{stroke,fill}%
}%
\begin{pgfscope}%
\pgfsys@transformshift{1.719054in}{0.422992in}%
\pgfsys@useobject{currentmarker}{}%
\end{pgfscope}%
\end{pgfscope}%
\begin{pgfscope}%
\pgfsetbuttcap%
\pgfsetroundjoin%
\definecolor{currentfill}{rgb}{0.000000,0.000000,0.000000}%
\pgfsetfillcolor{currentfill}%
\pgfsetlinewidth{0.501875pt}%
\definecolor{currentstroke}{rgb}{0.000000,0.000000,0.000000}%
\pgfsetstrokecolor{currentstroke}%
\pgfsetdash{}{0pt}%
\pgfsys@defobject{currentmarker}{\pgfqpoint{0.000000in}{-0.020833in}}{\pgfqpoint{0.000000in}{0.000000in}}{%
\pgfpathmoveto{\pgfqpoint{0.000000in}{0.000000in}}%
\pgfpathlineto{\pgfqpoint{0.000000in}{-0.020833in}}%
\pgfusepath{stroke,fill}%
}%
\begin{pgfscope}%
\pgfsys@transformshift{1.719054in}{2.049193in}%
\pgfsys@useobject{currentmarker}{}%
\end{pgfscope}%
\end{pgfscope}%
\begin{pgfscope}%
\pgfsetbuttcap%
\pgfsetroundjoin%
\definecolor{currentfill}{rgb}{0.000000,0.000000,0.000000}%
\pgfsetfillcolor{currentfill}%
\pgfsetlinewidth{0.501875pt}%
\definecolor{currentstroke}{rgb}{0.000000,0.000000,0.000000}%
\pgfsetstrokecolor{currentstroke}%
\pgfsetdash{}{0pt}%
\pgfsys@defobject{currentmarker}{\pgfqpoint{0.000000in}{0.000000in}}{\pgfqpoint{0.000000in}{0.020833in}}{%
\pgfpathmoveto{\pgfqpoint{0.000000in}{0.000000in}}%
\pgfpathlineto{\pgfqpoint{0.000000in}{0.020833in}}%
\pgfusepath{stroke,fill}%
}%
\begin{pgfscope}%
\pgfsys@transformshift{1.830018in}{0.422992in}%
\pgfsys@useobject{currentmarker}{}%
\end{pgfscope}%
\end{pgfscope}%
\begin{pgfscope}%
\pgfsetbuttcap%
\pgfsetroundjoin%
\definecolor{currentfill}{rgb}{0.000000,0.000000,0.000000}%
\pgfsetfillcolor{currentfill}%
\pgfsetlinewidth{0.501875pt}%
\definecolor{currentstroke}{rgb}{0.000000,0.000000,0.000000}%
\pgfsetstrokecolor{currentstroke}%
\pgfsetdash{}{0pt}%
\pgfsys@defobject{currentmarker}{\pgfqpoint{0.000000in}{-0.020833in}}{\pgfqpoint{0.000000in}{0.000000in}}{%
\pgfpathmoveto{\pgfqpoint{0.000000in}{0.000000in}}%
\pgfpathlineto{\pgfqpoint{0.000000in}{-0.020833in}}%
\pgfusepath{stroke,fill}%
}%
\begin{pgfscope}%
\pgfsys@transformshift{1.830018in}{2.049193in}%
\pgfsys@useobject{currentmarker}{}%
\end{pgfscope}%
\end{pgfscope}%
\begin{pgfscope}%
\pgfsetbuttcap%
\pgfsetroundjoin%
\definecolor{currentfill}{rgb}{0.000000,0.000000,0.000000}%
\pgfsetfillcolor{currentfill}%
\pgfsetlinewidth{0.501875pt}%
\definecolor{currentstroke}{rgb}{0.000000,0.000000,0.000000}%
\pgfsetstrokecolor{currentstroke}%
\pgfsetdash{}{0pt}%
\pgfsys@defobject{currentmarker}{\pgfqpoint{0.000000in}{0.000000in}}{\pgfqpoint{0.000000in}{0.020833in}}{%
\pgfpathmoveto{\pgfqpoint{0.000000in}{0.000000in}}%
\pgfpathlineto{\pgfqpoint{0.000000in}{0.020833in}}%
\pgfusepath{stroke,fill}%
}%
\begin{pgfscope}%
\pgfsys@transformshift{2.051946in}{0.422992in}%
\pgfsys@useobject{currentmarker}{}%
\end{pgfscope}%
\end{pgfscope}%
\begin{pgfscope}%
\pgfsetbuttcap%
\pgfsetroundjoin%
\definecolor{currentfill}{rgb}{0.000000,0.000000,0.000000}%
\pgfsetfillcolor{currentfill}%
\pgfsetlinewidth{0.501875pt}%
\definecolor{currentstroke}{rgb}{0.000000,0.000000,0.000000}%
\pgfsetstrokecolor{currentstroke}%
\pgfsetdash{}{0pt}%
\pgfsys@defobject{currentmarker}{\pgfqpoint{0.000000in}{-0.020833in}}{\pgfqpoint{0.000000in}{0.000000in}}{%
\pgfpathmoveto{\pgfqpoint{0.000000in}{0.000000in}}%
\pgfpathlineto{\pgfqpoint{0.000000in}{-0.020833in}}%
\pgfusepath{stroke,fill}%
}%
\begin{pgfscope}%
\pgfsys@transformshift{2.051946in}{2.049193in}%
\pgfsys@useobject{currentmarker}{}%
\end{pgfscope}%
\end{pgfscope}%
\begin{pgfscope}%
\pgfsetbuttcap%
\pgfsetroundjoin%
\definecolor{currentfill}{rgb}{0.000000,0.000000,0.000000}%
\pgfsetfillcolor{currentfill}%
\pgfsetlinewidth{0.501875pt}%
\definecolor{currentstroke}{rgb}{0.000000,0.000000,0.000000}%
\pgfsetstrokecolor{currentstroke}%
\pgfsetdash{}{0pt}%
\pgfsys@defobject{currentmarker}{\pgfqpoint{0.000000in}{0.000000in}}{\pgfqpoint{0.000000in}{0.020833in}}{%
\pgfpathmoveto{\pgfqpoint{0.000000in}{0.000000in}}%
\pgfpathlineto{\pgfqpoint{0.000000in}{0.020833in}}%
\pgfusepath{stroke,fill}%
}%
\begin{pgfscope}%
\pgfsys@transformshift{2.162910in}{0.422992in}%
\pgfsys@useobject{currentmarker}{}%
\end{pgfscope}%
\end{pgfscope}%
\begin{pgfscope}%
\pgfsetbuttcap%
\pgfsetroundjoin%
\definecolor{currentfill}{rgb}{0.000000,0.000000,0.000000}%
\pgfsetfillcolor{currentfill}%
\pgfsetlinewidth{0.501875pt}%
\definecolor{currentstroke}{rgb}{0.000000,0.000000,0.000000}%
\pgfsetstrokecolor{currentstroke}%
\pgfsetdash{}{0pt}%
\pgfsys@defobject{currentmarker}{\pgfqpoint{0.000000in}{-0.020833in}}{\pgfqpoint{0.000000in}{0.000000in}}{%
\pgfpathmoveto{\pgfqpoint{0.000000in}{0.000000in}}%
\pgfpathlineto{\pgfqpoint{0.000000in}{-0.020833in}}%
\pgfusepath{stroke,fill}%
}%
\begin{pgfscope}%
\pgfsys@transformshift{2.162910in}{2.049193in}%
\pgfsys@useobject{currentmarker}{}%
\end{pgfscope}%
\end{pgfscope}%
\begin{pgfscope}%
\pgfsetbuttcap%
\pgfsetroundjoin%
\definecolor{currentfill}{rgb}{0.000000,0.000000,0.000000}%
\pgfsetfillcolor{currentfill}%
\pgfsetlinewidth{0.501875pt}%
\definecolor{currentstroke}{rgb}{0.000000,0.000000,0.000000}%
\pgfsetstrokecolor{currentstroke}%
\pgfsetdash{}{0pt}%
\pgfsys@defobject{currentmarker}{\pgfqpoint{0.000000in}{0.000000in}}{\pgfqpoint{0.000000in}{0.020833in}}{%
\pgfpathmoveto{\pgfqpoint{0.000000in}{0.000000in}}%
\pgfpathlineto{\pgfqpoint{0.000000in}{0.020833in}}%
\pgfusepath{stroke,fill}%
}%
\begin{pgfscope}%
\pgfsys@transformshift{2.273874in}{0.422992in}%
\pgfsys@useobject{currentmarker}{}%
\end{pgfscope}%
\end{pgfscope}%
\begin{pgfscope}%
\pgfsetbuttcap%
\pgfsetroundjoin%
\definecolor{currentfill}{rgb}{0.000000,0.000000,0.000000}%
\pgfsetfillcolor{currentfill}%
\pgfsetlinewidth{0.501875pt}%
\definecolor{currentstroke}{rgb}{0.000000,0.000000,0.000000}%
\pgfsetstrokecolor{currentstroke}%
\pgfsetdash{}{0pt}%
\pgfsys@defobject{currentmarker}{\pgfqpoint{0.000000in}{-0.020833in}}{\pgfqpoint{0.000000in}{0.000000in}}{%
\pgfpathmoveto{\pgfqpoint{0.000000in}{0.000000in}}%
\pgfpathlineto{\pgfqpoint{0.000000in}{-0.020833in}}%
\pgfusepath{stroke,fill}%
}%
\begin{pgfscope}%
\pgfsys@transformshift{2.273874in}{2.049193in}%
\pgfsys@useobject{currentmarker}{}%
\end{pgfscope}%
\end{pgfscope}%
\begin{pgfscope}%
\pgfsetbuttcap%
\pgfsetroundjoin%
\definecolor{currentfill}{rgb}{0.000000,0.000000,0.000000}%
\pgfsetfillcolor{currentfill}%
\pgfsetlinewidth{0.501875pt}%
\definecolor{currentstroke}{rgb}{0.000000,0.000000,0.000000}%
\pgfsetstrokecolor{currentstroke}%
\pgfsetdash{}{0pt}%
\pgfsys@defobject{currentmarker}{\pgfqpoint{0.000000in}{0.000000in}}{\pgfqpoint{0.000000in}{0.020833in}}{%
\pgfpathmoveto{\pgfqpoint{0.000000in}{0.000000in}}%
\pgfpathlineto{\pgfqpoint{0.000000in}{0.020833in}}%
\pgfusepath{stroke,fill}%
}%
\begin{pgfscope}%
\pgfsys@transformshift{2.495801in}{0.422992in}%
\pgfsys@useobject{currentmarker}{}%
\end{pgfscope}%
\end{pgfscope}%
\begin{pgfscope}%
\pgfsetbuttcap%
\pgfsetroundjoin%
\definecolor{currentfill}{rgb}{0.000000,0.000000,0.000000}%
\pgfsetfillcolor{currentfill}%
\pgfsetlinewidth{0.501875pt}%
\definecolor{currentstroke}{rgb}{0.000000,0.000000,0.000000}%
\pgfsetstrokecolor{currentstroke}%
\pgfsetdash{}{0pt}%
\pgfsys@defobject{currentmarker}{\pgfqpoint{0.000000in}{-0.020833in}}{\pgfqpoint{0.000000in}{0.000000in}}{%
\pgfpathmoveto{\pgfqpoint{0.000000in}{0.000000in}}%
\pgfpathlineto{\pgfqpoint{0.000000in}{-0.020833in}}%
\pgfusepath{stroke,fill}%
}%
\begin{pgfscope}%
\pgfsys@transformshift{2.495801in}{2.049193in}%
\pgfsys@useobject{currentmarker}{}%
\end{pgfscope}%
\end{pgfscope}%
\begin{pgfscope}%
\pgfsetbuttcap%
\pgfsetroundjoin%
\definecolor{currentfill}{rgb}{0.000000,0.000000,0.000000}%
\pgfsetfillcolor{currentfill}%
\pgfsetlinewidth{0.501875pt}%
\definecolor{currentstroke}{rgb}{0.000000,0.000000,0.000000}%
\pgfsetstrokecolor{currentstroke}%
\pgfsetdash{}{0pt}%
\pgfsys@defobject{currentmarker}{\pgfqpoint{0.000000in}{0.000000in}}{\pgfqpoint{0.000000in}{0.020833in}}{%
\pgfpathmoveto{\pgfqpoint{0.000000in}{0.000000in}}%
\pgfpathlineto{\pgfqpoint{0.000000in}{0.020833in}}%
\pgfusepath{stroke,fill}%
}%
\begin{pgfscope}%
\pgfsys@transformshift{2.606765in}{0.422992in}%
\pgfsys@useobject{currentmarker}{}%
\end{pgfscope}%
\end{pgfscope}%
\begin{pgfscope}%
\pgfsetbuttcap%
\pgfsetroundjoin%
\definecolor{currentfill}{rgb}{0.000000,0.000000,0.000000}%
\pgfsetfillcolor{currentfill}%
\pgfsetlinewidth{0.501875pt}%
\definecolor{currentstroke}{rgb}{0.000000,0.000000,0.000000}%
\pgfsetstrokecolor{currentstroke}%
\pgfsetdash{}{0pt}%
\pgfsys@defobject{currentmarker}{\pgfqpoint{0.000000in}{-0.020833in}}{\pgfqpoint{0.000000in}{0.000000in}}{%
\pgfpathmoveto{\pgfqpoint{0.000000in}{0.000000in}}%
\pgfpathlineto{\pgfqpoint{0.000000in}{-0.020833in}}%
\pgfusepath{stroke,fill}%
}%
\begin{pgfscope}%
\pgfsys@transformshift{2.606765in}{2.049193in}%
\pgfsys@useobject{currentmarker}{}%
\end{pgfscope}%
\end{pgfscope}%
\begin{pgfscope}%
\pgfsetbuttcap%
\pgfsetroundjoin%
\definecolor{currentfill}{rgb}{0.000000,0.000000,0.000000}%
\pgfsetfillcolor{currentfill}%
\pgfsetlinewidth{0.501875pt}%
\definecolor{currentstroke}{rgb}{0.000000,0.000000,0.000000}%
\pgfsetstrokecolor{currentstroke}%
\pgfsetdash{}{0pt}%
\pgfsys@defobject{currentmarker}{\pgfqpoint{0.000000in}{0.000000in}}{\pgfqpoint{0.000000in}{0.020833in}}{%
\pgfpathmoveto{\pgfqpoint{0.000000in}{0.000000in}}%
\pgfpathlineto{\pgfqpoint{0.000000in}{0.020833in}}%
\pgfusepath{stroke,fill}%
}%
\begin{pgfscope}%
\pgfsys@transformshift{2.717729in}{0.422992in}%
\pgfsys@useobject{currentmarker}{}%
\end{pgfscope}%
\end{pgfscope}%
\begin{pgfscope}%
\pgfsetbuttcap%
\pgfsetroundjoin%
\definecolor{currentfill}{rgb}{0.000000,0.000000,0.000000}%
\pgfsetfillcolor{currentfill}%
\pgfsetlinewidth{0.501875pt}%
\definecolor{currentstroke}{rgb}{0.000000,0.000000,0.000000}%
\pgfsetstrokecolor{currentstroke}%
\pgfsetdash{}{0pt}%
\pgfsys@defobject{currentmarker}{\pgfqpoint{0.000000in}{-0.020833in}}{\pgfqpoint{0.000000in}{0.000000in}}{%
\pgfpathmoveto{\pgfqpoint{0.000000in}{0.000000in}}%
\pgfpathlineto{\pgfqpoint{0.000000in}{-0.020833in}}%
\pgfusepath{stroke,fill}%
}%
\begin{pgfscope}%
\pgfsys@transformshift{2.717729in}{2.049193in}%
\pgfsys@useobject{currentmarker}{}%
\end{pgfscope}%
\end{pgfscope}%
\begin{pgfscope}%
\pgfsetbuttcap%
\pgfsetroundjoin%
\definecolor{currentfill}{rgb}{0.000000,0.000000,0.000000}%
\pgfsetfillcolor{currentfill}%
\pgfsetlinewidth{0.501875pt}%
\definecolor{currentstroke}{rgb}{0.000000,0.000000,0.000000}%
\pgfsetstrokecolor{currentstroke}%
\pgfsetdash{}{0pt}%
\pgfsys@defobject{currentmarker}{\pgfqpoint{0.000000in}{0.000000in}}{\pgfqpoint{0.000000in}{0.020833in}}{%
\pgfpathmoveto{\pgfqpoint{0.000000in}{0.000000in}}%
\pgfpathlineto{\pgfqpoint{0.000000in}{0.020833in}}%
\pgfusepath{stroke,fill}%
}%
\begin{pgfscope}%
\pgfsys@transformshift{2.828693in}{0.422992in}%
\pgfsys@useobject{currentmarker}{}%
\end{pgfscope}%
\end{pgfscope}%
\begin{pgfscope}%
\pgfsetbuttcap%
\pgfsetroundjoin%
\definecolor{currentfill}{rgb}{0.000000,0.000000,0.000000}%
\pgfsetfillcolor{currentfill}%
\pgfsetlinewidth{0.501875pt}%
\definecolor{currentstroke}{rgb}{0.000000,0.000000,0.000000}%
\pgfsetstrokecolor{currentstroke}%
\pgfsetdash{}{0pt}%
\pgfsys@defobject{currentmarker}{\pgfqpoint{0.000000in}{-0.020833in}}{\pgfqpoint{0.000000in}{0.000000in}}{%
\pgfpathmoveto{\pgfqpoint{0.000000in}{0.000000in}}%
\pgfpathlineto{\pgfqpoint{0.000000in}{-0.020833in}}%
\pgfusepath{stroke,fill}%
}%
\begin{pgfscope}%
\pgfsys@transformshift{2.828693in}{2.049193in}%
\pgfsys@useobject{currentmarker}{}%
\end{pgfscope}%
\end{pgfscope}%
\begin{pgfscope}%
\definecolor{textcolor}{rgb}{0.000000,0.000000,0.000000}%
\pgfsetstrokecolor{textcolor}%
\pgfsetfillcolor{textcolor}%
\pgftext[x=1.730150in,y=0.184413in,,top]{\color{textcolor}\rmfamily\fontsize{10.000000}{12.000000}\selectfont \(\displaystyle K\)}%
\end{pgfscope}%
\begin{pgfscope}%
\pgfsetbuttcap%
\pgfsetroundjoin%
\definecolor{currentfill}{rgb}{0.000000,0.000000,0.000000}%
\pgfsetfillcolor{currentfill}%
\pgfsetlinewidth{0.501875pt}%
\definecolor{currentstroke}{rgb}{0.000000,0.000000,0.000000}%
\pgfsetstrokecolor{currentstroke}%
\pgfsetdash{}{0pt}%
\pgfsys@defobject{currentmarker}{\pgfqpoint{0.000000in}{0.000000in}}{\pgfqpoint{0.041667in}{0.000000in}}{%
\pgfpathmoveto{\pgfqpoint{0.000000in}{0.000000in}}%
\pgfpathlineto{\pgfqpoint{0.041667in}{0.000000in}}%
\pgfusepath{stroke,fill}%
}%
\begin{pgfscope}%
\pgfsys@transformshift{0.609415in}{0.606098in}%
\pgfsys@useobject{currentmarker}{}%
\end{pgfscope}%
\end{pgfscope}%
\begin{pgfscope}%
\pgfsetbuttcap%
\pgfsetroundjoin%
\definecolor{currentfill}{rgb}{0.000000,0.000000,0.000000}%
\pgfsetfillcolor{currentfill}%
\pgfsetlinewidth{0.501875pt}%
\definecolor{currentstroke}{rgb}{0.000000,0.000000,0.000000}%
\pgfsetstrokecolor{currentstroke}%
\pgfsetdash{}{0pt}%
\pgfsys@defobject{currentmarker}{\pgfqpoint{-0.041667in}{0.000000in}}{\pgfqpoint{-0.000000in}{0.000000in}}{%
\pgfpathmoveto{\pgfqpoint{-0.000000in}{0.000000in}}%
\pgfpathlineto{\pgfqpoint{-0.041667in}{0.000000in}}%
\pgfusepath{stroke,fill}%
}%
\begin{pgfscope}%
\pgfsys@transformshift{2.850886in}{0.606098in}%
\pgfsys@useobject{currentmarker}{}%
\end{pgfscope}%
\end{pgfscope}%
\begin{pgfscope}%
\definecolor{textcolor}{rgb}{0.000000,0.000000,0.000000}%
\pgfsetstrokecolor{textcolor}%
\pgfsetfillcolor{textcolor}%
\pgftext[x=0.244444in, y=0.553336in, left, base]{\color{textcolor}\rmfamily\fontsize{10.000000}{12.000000}\selectfont \(\displaystyle {0.985}\)}%
\end{pgfscope}%
\begin{pgfscope}%
\pgfsetbuttcap%
\pgfsetroundjoin%
\definecolor{currentfill}{rgb}{0.000000,0.000000,0.000000}%
\pgfsetfillcolor{currentfill}%
\pgfsetlinewidth{0.501875pt}%
\definecolor{currentstroke}{rgb}{0.000000,0.000000,0.000000}%
\pgfsetstrokecolor{currentstroke}%
\pgfsetdash{}{0pt}%
\pgfsys@defobject{currentmarker}{\pgfqpoint{0.000000in}{0.000000in}}{\pgfqpoint{0.041667in}{0.000000in}}{%
\pgfpathmoveto{\pgfqpoint{0.000000in}{0.000000in}}%
\pgfpathlineto{\pgfqpoint{0.041667in}{0.000000in}}%
\pgfusepath{stroke,fill}%
}%
\begin{pgfscope}%
\pgfsys@transformshift{0.609415in}{1.069372in}%
\pgfsys@useobject{currentmarker}{}%
\end{pgfscope}%
\end{pgfscope}%
\begin{pgfscope}%
\pgfsetbuttcap%
\pgfsetroundjoin%
\definecolor{currentfill}{rgb}{0.000000,0.000000,0.000000}%
\pgfsetfillcolor{currentfill}%
\pgfsetlinewidth{0.501875pt}%
\definecolor{currentstroke}{rgb}{0.000000,0.000000,0.000000}%
\pgfsetstrokecolor{currentstroke}%
\pgfsetdash{}{0pt}%
\pgfsys@defobject{currentmarker}{\pgfqpoint{-0.041667in}{0.000000in}}{\pgfqpoint{-0.000000in}{0.000000in}}{%
\pgfpathmoveto{\pgfqpoint{-0.000000in}{0.000000in}}%
\pgfpathlineto{\pgfqpoint{-0.041667in}{0.000000in}}%
\pgfusepath{stroke,fill}%
}%
\begin{pgfscope}%
\pgfsys@transformshift{2.850886in}{1.069372in}%
\pgfsys@useobject{currentmarker}{}%
\end{pgfscope}%
\end{pgfscope}%
\begin{pgfscope}%
\definecolor{textcolor}{rgb}{0.000000,0.000000,0.000000}%
\pgfsetstrokecolor{textcolor}%
\pgfsetfillcolor{textcolor}%
\pgftext[x=0.244444in, y=1.016610in, left, base]{\color{textcolor}\rmfamily\fontsize{10.000000}{12.000000}\selectfont \(\displaystyle {0.990}\)}%
\end{pgfscope}%
\begin{pgfscope}%
\pgfsetbuttcap%
\pgfsetroundjoin%
\definecolor{currentfill}{rgb}{0.000000,0.000000,0.000000}%
\pgfsetfillcolor{currentfill}%
\pgfsetlinewidth{0.501875pt}%
\definecolor{currentstroke}{rgb}{0.000000,0.000000,0.000000}%
\pgfsetstrokecolor{currentstroke}%
\pgfsetdash{}{0pt}%
\pgfsys@defobject{currentmarker}{\pgfqpoint{0.000000in}{0.000000in}}{\pgfqpoint{0.041667in}{0.000000in}}{%
\pgfpathmoveto{\pgfqpoint{0.000000in}{0.000000in}}%
\pgfpathlineto{\pgfqpoint{0.041667in}{0.000000in}}%
\pgfusepath{stroke,fill}%
}%
\begin{pgfscope}%
\pgfsys@transformshift{0.609415in}{1.532646in}%
\pgfsys@useobject{currentmarker}{}%
\end{pgfscope}%
\end{pgfscope}%
\begin{pgfscope}%
\pgfsetbuttcap%
\pgfsetroundjoin%
\definecolor{currentfill}{rgb}{0.000000,0.000000,0.000000}%
\pgfsetfillcolor{currentfill}%
\pgfsetlinewidth{0.501875pt}%
\definecolor{currentstroke}{rgb}{0.000000,0.000000,0.000000}%
\pgfsetstrokecolor{currentstroke}%
\pgfsetdash{}{0pt}%
\pgfsys@defobject{currentmarker}{\pgfqpoint{-0.041667in}{0.000000in}}{\pgfqpoint{-0.000000in}{0.000000in}}{%
\pgfpathmoveto{\pgfqpoint{-0.000000in}{0.000000in}}%
\pgfpathlineto{\pgfqpoint{-0.041667in}{0.000000in}}%
\pgfusepath{stroke,fill}%
}%
\begin{pgfscope}%
\pgfsys@transformshift{2.850886in}{1.532646in}%
\pgfsys@useobject{currentmarker}{}%
\end{pgfscope}%
\end{pgfscope}%
\begin{pgfscope}%
\definecolor{textcolor}{rgb}{0.000000,0.000000,0.000000}%
\pgfsetstrokecolor{textcolor}%
\pgfsetfillcolor{textcolor}%
\pgftext[x=0.244444in, y=1.479884in, left, base]{\color{textcolor}\rmfamily\fontsize{10.000000}{12.000000}\selectfont \(\displaystyle {0.995}\)}%
\end{pgfscope}%
\begin{pgfscope}%
\pgfsetbuttcap%
\pgfsetroundjoin%
\definecolor{currentfill}{rgb}{0.000000,0.000000,0.000000}%
\pgfsetfillcolor{currentfill}%
\pgfsetlinewidth{0.501875pt}%
\definecolor{currentstroke}{rgb}{0.000000,0.000000,0.000000}%
\pgfsetstrokecolor{currentstroke}%
\pgfsetdash{}{0pt}%
\pgfsys@defobject{currentmarker}{\pgfqpoint{0.000000in}{0.000000in}}{\pgfqpoint{0.041667in}{0.000000in}}{%
\pgfpathmoveto{\pgfqpoint{0.000000in}{0.000000in}}%
\pgfpathlineto{\pgfqpoint{0.041667in}{0.000000in}}%
\pgfusepath{stroke,fill}%
}%
\begin{pgfscope}%
\pgfsys@transformshift{0.609415in}{1.995919in}%
\pgfsys@useobject{currentmarker}{}%
\end{pgfscope}%
\end{pgfscope}%
\begin{pgfscope}%
\pgfsetbuttcap%
\pgfsetroundjoin%
\definecolor{currentfill}{rgb}{0.000000,0.000000,0.000000}%
\pgfsetfillcolor{currentfill}%
\pgfsetlinewidth{0.501875pt}%
\definecolor{currentstroke}{rgb}{0.000000,0.000000,0.000000}%
\pgfsetstrokecolor{currentstroke}%
\pgfsetdash{}{0pt}%
\pgfsys@defobject{currentmarker}{\pgfqpoint{-0.041667in}{0.000000in}}{\pgfqpoint{-0.000000in}{0.000000in}}{%
\pgfpathmoveto{\pgfqpoint{-0.000000in}{0.000000in}}%
\pgfpathlineto{\pgfqpoint{-0.041667in}{0.000000in}}%
\pgfusepath{stroke,fill}%
}%
\begin{pgfscope}%
\pgfsys@transformshift{2.850886in}{1.995919in}%
\pgfsys@useobject{currentmarker}{}%
\end{pgfscope}%
\end{pgfscope}%
\begin{pgfscope}%
\definecolor{textcolor}{rgb}{0.000000,0.000000,0.000000}%
\pgfsetstrokecolor{textcolor}%
\pgfsetfillcolor{textcolor}%
\pgftext[x=0.244444in, y=1.943158in, left, base]{\color{textcolor}\rmfamily\fontsize{10.000000}{12.000000}\selectfont \(\displaystyle {1.000}\)}%
\end{pgfscope}%
\begin{pgfscope}%
\pgfsetbuttcap%
\pgfsetroundjoin%
\definecolor{currentfill}{rgb}{0.000000,0.000000,0.000000}%
\pgfsetfillcolor{currentfill}%
\pgfsetlinewidth{0.501875pt}%
\definecolor{currentstroke}{rgb}{0.000000,0.000000,0.000000}%
\pgfsetstrokecolor{currentstroke}%
\pgfsetdash{}{0pt}%
\pgfsys@defobject{currentmarker}{\pgfqpoint{0.000000in}{0.000000in}}{\pgfqpoint{0.020833in}{0.000000in}}{%
\pgfpathmoveto{\pgfqpoint{0.000000in}{0.000000in}}%
\pgfpathlineto{\pgfqpoint{0.020833in}{0.000000in}}%
\pgfusepath{stroke,fill}%
}%
\begin{pgfscope}%
\pgfsys@transformshift{0.609415in}{0.513443in}%
\pgfsys@useobject{currentmarker}{}%
\end{pgfscope}%
\end{pgfscope}%
\begin{pgfscope}%
\pgfsetbuttcap%
\pgfsetroundjoin%
\definecolor{currentfill}{rgb}{0.000000,0.000000,0.000000}%
\pgfsetfillcolor{currentfill}%
\pgfsetlinewidth{0.501875pt}%
\definecolor{currentstroke}{rgb}{0.000000,0.000000,0.000000}%
\pgfsetstrokecolor{currentstroke}%
\pgfsetdash{}{0pt}%
\pgfsys@defobject{currentmarker}{\pgfqpoint{-0.020833in}{0.000000in}}{\pgfqpoint{-0.000000in}{0.000000in}}{%
\pgfpathmoveto{\pgfqpoint{-0.000000in}{0.000000in}}%
\pgfpathlineto{\pgfqpoint{-0.020833in}{0.000000in}}%
\pgfusepath{stroke,fill}%
}%
\begin{pgfscope}%
\pgfsys@transformshift{2.850886in}{0.513443in}%
\pgfsys@useobject{currentmarker}{}%
\end{pgfscope}%
\end{pgfscope}%
\begin{pgfscope}%
\pgfsetbuttcap%
\pgfsetroundjoin%
\definecolor{currentfill}{rgb}{0.000000,0.000000,0.000000}%
\pgfsetfillcolor{currentfill}%
\pgfsetlinewidth{0.501875pt}%
\definecolor{currentstroke}{rgb}{0.000000,0.000000,0.000000}%
\pgfsetstrokecolor{currentstroke}%
\pgfsetdash{}{0pt}%
\pgfsys@defobject{currentmarker}{\pgfqpoint{0.000000in}{0.000000in}}{\pgfqpoint{0.020833in}{0.000000in}}{%
\pgfpathmoveto{\pgfqpoint{0.000000in}{0.000000in}}%
\pgfpathlineto{\pgfqpoint{0.020833in}{0.000000in}}%
\pgfusepath{stroke,fill}%
}%
\begin{pgfscope}%
\pgfsys@transformshift{0.609415in}{0.698753in}%
\pgfsys@useobject{currentmarker}{}%
\end{pgfscope}%
\end{pgfscope}%
\begin{pgfscope}%
\pgfsetbuttcap%
\pgfsetroundjoin%
\definecolor{currentfill}{rgb}{0.000000,0.000000,0.000000}%
\pgfsetfillcolor{currentfill}%
\pgfsetlinewidth{0.501875pt}%
\definecolor{currentstroke}{rgb}{0.000000,0.000000,0.000000}%
\pgfsetstrokecolor{currentstroke}%
\pgfsetdash{}{0pt}%
\pgfsys@defobject{currentmarker}{\pgfqpoint{-0.020833in}{0.000000in}}{\pgfqpoint{-0.000000in}{0.000000in}}{%
\pgfpathmoveto{\pgfqpoint{-0.000000in}{0.000000in}}%
\pgfpathlineto{\pgfqpoint{-0.020833in}{0.000000in}}%
\pgfusepath{stroke,fill}%
}%
\begin{pgfscope}%
\pgfsys@transformshift{2.850886in}{0.698753in}%
\pgfsys@useobject{currentmarker}{}%
\end{pgfscope}%
\end{pgfscope}%
\begin{pgfscope}%
\pgfsetbuttcap%
\pgfsetroundjoin%
\definecolor{currentfill}{rgb}{0.000000,0.000000,0.000000}%
\pgfsetfillcolor{currentfill}%
\pgfsetlinewidth{0.501875pt}%
\definecolor{currentstroke}{rgb}{0.000000,0.000000,0.000000}%
\pgfsetstrokecolor{currentstroke}%
\pgfsetdash{}{0pt}%
\pgfsys@defobject{currentmarker}{\pgfqpoint{0.000000in}{0.000000in}}{\pgfqpoint{0.020833in}{0.000000in}}{%
\pgfpathmoveto{\pgfqpoint{0.000000in}{0.000000in}}%
\pgfpathlineto{\pgfqpoint{0.020833in}{0.000000in}}%
\pgfusepath{stroke,fill}%
}%
\begin{pgfscope}%
\pgfsys@transformshift{0.609415in}{0.791408in}%
\pgfsys@useobject{currentmarker}{}%
\end{pgfscope}%
\end{pgfscope}%
\begin{pgfscope}%
\pgfsetbuttcap%
\pgfsetroundjoin%
\definecolor{currentfill}{rgb}{0.000000,0.000000,0.000000}%
\pgfsetfillcolor{currentfill}%
\pgfsetlinewidth{0.501875pt}%
\definecolor{currentstroke}{rgb}{0.000000,0.000000,0.000000}%
\pgfsetstrokecolor{currentstroke}%
\pgfsetdash{}{0pt}%
\pgfsys@defobject{currentmarker}{\pgfqpoint{-0.020833in}{0.000000in}}{\pgfqpoint{-0.000000in}{0.000000in}}{%
\pgfpathmoveto{\pgfqpoint{-0.000000in}{0.000000in}}%
\pgfpathlineto{\pgfqpoint{-0.020833in}{0.000000in}}%
\pgfusepath{stroke,fill}%
}%
\begin{pgfscope}%
\pgfsys@transformshift{2.850886in}{0.791408in}%
\pgfsys@useobject{currentmarker}{}%
\end{pgfscope}%
\end{pgfscope}%
\begin{pgfscope}%
\pgfsetbuttcap%
\pgfsetroundjoin%
\definecolor{currentfill}{rgb}{0.000000,0.000000,0.000000}%
\pgfsetfillcolor{currentfill}%
\pgfsetlinewidth{0.501875pt}%
\definecolor{currentstroke}{rgb}{0.000000,0.000000,0.000000}%
\pgfsetstrokecolor{currentstroke}%
\pgfsetdash{}{0pt}%
\pgfsys@defobject{currentmarker}{\pgfqpoint{0.000000in}{0.000000in}}{\pgfqpoint{0.020833in}{0.000000in}}{%
\pgfpathmoveto{\pgfqpoint{0.000000in}{0.000000in}}%
\pgfpathlineto{\pgfqpoint{0.020833in}{0.000000in}}%
\pgfusepath{stroke,fill}%
}%
\begin{pgfscope}%
\pgfsys@transformshift{0.609415in}{0.884062in}%
\pgfsys@useobject{currentmarker}{}%
\end{pgfscope}%
\end{pgfscope}%
\begin{pgfscope}%
\pgfsetbuttcap%
\pgfsetroundjoin%
\definecolor{currentfill}{rgb}{0.000000,0.000000,0.000000}%
\pgfsetfillcolor{currentfill}%
\pgfsetlinewidth{0.501875pt}%
\definecolor{currentstroke}{rgb}{0.000000,0.000000,0.000000}%
\pgfsetstrokecolor{currentstroke}%
\pgfsetdash{}{0pt}%
\pgfsys@defobject{currentmarker}{\pgfqpoint{-0.020833in}{0.000000in}}{\pgfqpoint{-0.000000in}{0.000000in}}{%
\pgfpathmoveto{\pgfqpoint{-0.000000in}{0.000000in}}%
\pgfpathlineto{\pgfqpoint{-0.020833in}{0.000000in}}%
\pgfusepath{stroke,fill}%
}%
\begin{pgfscope}%
\pgfsys@transformshift{2.850886in}{0.884062in}%
\pgfsys@useobject{currentmarker}{}%
\end{pgfscope}%
\end{pgfscope}%
\begin{pgfscope}%
\pgfsetbuttcap%
\pgfsetroundjoin%
\definecolor{currentfill}{rgb}{0.000000,0.000000,0.000000}%
\pgfsetfillcolor{currentfill}%
\pgfsetlinewidth{0.501875pt}%
\definecolor{currentstroke}{rgb}{0.000000,0.000000,0.000000}%
\pgfsetstrokecolor{currentstroke}%
\pgfsetdash{}{0pt}%
\pgfsys@defobject{currentmarker}{\pgfqpoint{0.000000in}{0.000000in}}{\pgfqpoint{0.020833in}{0.000000in}}{%
\pgfpathmoveto{\pgfqpoint{0.000000in}{0.000000in}}%
\pgfpathlineto{\pgfqpoint{0.020833in}{0.000000in}}%
\pgfusepath{stroke,fill}%
}%
\begin{pgfscope}%
\pgfsys@transformshift{0.609415in}{0.976717in}%
\pgfsys@useobject{currentmarker}{}%
\end{pgfscope}%
\end{pgfscope}%
\begin{pgfscope}%
\pgfsetbuttcap%
\pgfsetroundjoin%
\definecolor{currentfill}{rgb}{0.000000,0.000000,0.000000}%
\pgfsetfillcolor{currentfill}%
\pgfsetlinewidth{0.501875pt}%
\definecolor{currentstroke}{rgb}{0.000000,0.000000,0.000000}%
\pgfsetstrokecolor{currentstroke}%
\pgfsetdash{}{0pt}%
\pgfsys@defobject{currentmarker}{\pgfqpoint{-0.020833in}{0.000000in}}{\pgfqpoint{-0.000000in}{0.000000in}}{%
\pgfpathmoveto{\pgfqpoint{-0.000000in}{0.000000in}}%
\pgfpathlineto{\pgfqpoint{-0.020833in}{0.000000in}}%
\pgfusepath{stroke,fill}%
}%
\begin{pgfscope}%
\pgfsys@transformshift{2.850886in}{0.976717in}%
\pgfsys@useobject{currentmarker}{}%
\end{pgfscope}%
\end{pgfscope}%
\begin{pgfscope}%
\pgfsetbuttcap%
\pgfsetroundjoin%
\definecolor{currentfill}{rgb}{0.000000,0.000000,0.000000}%
\pgfsetfillcolor{currentfill}%
\pgfsetlinewidth{0.501875pt}%
\definecolor{currentstroke}{rgb}{0.000000,0.000000,0.000000}%
\pgfsetstrokecolor{currentstroke}%
\pgfsetdash{}{0pt}%
\pgfsys@defobject{currentmarker}{\pgfqpoint{0.000000in}{0.000000in}}{\pgfqpoint{0.020833in}{0.000000in}}{%
\pgfpathmoveto{\pgfqpoint{0.000000in}{0.000000in}}%
\pgfpathlineto{\pgfqpoint{0.020833in}{0.000000in}}%
\pgfusepath{stroke,fill}%
}%
\begin{pgfscope}%
\pgfsys@transformshift{0.609415in}{1.162027in}%
\pgfsys@useobject{currentmarker}{}%
\end{pgfscope}%
\end{pgfscope}%
\begin{pgfscope}%
\pgfsetbuttcap%
\pgfsetroundjoin%
\definecolor{currentfill}{rgb}{0.000000,0.000000,0.000000}%
\pgfsetfillcolor{currentfill}%
\pgfsetlinewidth{0.501875pt}%
\definecolor{currentstroke}{rgb}{0.000000,0.000000,0.000000}%
\pgfsetstrokecolor{currentstroke}%
\pgfsetdash{}{0pt}%
\pgfsys@defobject{currentmarker}{\pgfqpoint{-0.020833in}{0.000000in}}{\pgfqpoint{-0.000000in}{0.000000in}}{%
\pgfpathmoveto{\pgfqpoint{-0.000000in}{0.000000in}}%
\pgfpathlineto{\pgfqpoint{-0.020833in}{0.000000in}}%
\pgfusepath{stroke,fill}%
}%
\begin{pgfscope}%
\pgfsys@transformshift{2.850886in}{1.162027in}%
\pgfsys@useobject{currentmarker}{}%
\end{pgfscope}%
\end{pgfscope}%
\begin{pgfscope}%
\pgfsetbuttcap%
\pgfsetroundjoin%
\definecolor{currentfill}{rgb}{0.000000,0.000000,0.000000}%
\pgfsetfillcolor{currentfill}%
\pgfsetlinewidth{0.501875pt}%
\definecolor{currentstroke}{rgb}{0.000000,0.000000,0.000000}%
\pgfsetstrokecolor{currentstroke}%
\pgfsetdash{}{0pt}%
\pgfsys@defobject{currentmarker}{\pgfqpoint{0.000000in}{0.000000in}}{\pgfqpoint{0.020833in}{0.000000in}}{%
\pgfpathmoveto{\pgfqpoint{0.000000in}{0.000000in}}%
\pgfpathlineto{\pgfqpoint{0.020833in}{0.000000in}}%
\pgfusepath{stroke,fill}%
}%
\begin{pgfscope}%
\pgfsys@transformshift{0.609415in}{1.254681in}%
\pgfsys@useobject{currentmarker}{}%
\end{pgfscope}%
\end{pgfscope}%
\begin{pgfscope}%
\pgfsetbuttcap%
\pgfsetroundjoin%
\definecolor{currentfill}{rgb}{0.000000,0.000000,0.000000}%
\pgfsetfillcolor{currentfill}%
\pgfsetlinewidth{0.501875pt}%
\definecolor{currentstroke}{rgb}{0.000000,0.000000,0.000000}%
\pgfsetstrokecolor{currentstroke}%
\pgfsetdash{}{0pt}%
\pgfsys@defobject{currentmarker}{\pgfqpoint{-0.020833in}{0.000000in}}{\pgfqpoint{-0.000000in}{0.000000in}}{%
\pgfpathmoveto{\pgfqpoint{-0.000000in}{0.000000in}}%
\pgfpathlineto{\pgfqpoint{-0.020833in}{0.000000in}}%
\pgfusepath{stroke,fill}%
}%
\begin{pgfscope}%
\pgfsys@transformshift{2.850886in}{1.254681in}%
\pgfsys@useobject{currentmarker}{}%
\end{pgfscope}%
\end{pgfscope}%
\begin{pgfscope}%
\pgfsetbuttcap%
\pgfsetroundjoin%
\definecolor{currentfill}{rgb}{0.000000,0.000000,0.000000}%
\pgfsetfillcolor{currentfill}%
\pgfsetlinewidth{0.501875pt}%
\definecolor{currentstroke}{rgb}{0.000000,0.000000,0.000000}%
\pgfsetstrokecolor{currentstroke}%
\pgfsetdash{}{0pt}%
\pgfsys@defobject{currentmarker}{\pgfqpoint{0.000000in}{0.000000in}}{\pgfqpoint{0.020833in}{0.000000in}}{%
\pgfpathmoveto{\pgfqpoint{0.000000in}{0.000000in}}%
\pgfpathlineto{\pgfqpoint{0.020833in}{0.000000in}}%
\pgfusepath{stroke,fill}%
}%
\begin{pgfscope}%
\pgfsys@transformshift{0.609415in}{1.347336in}%
\pgfsys@useobject{currentmarker}{}%
\end{pgfscope}%
\end{pgfscope}%
\begin{pgfscope}%
\pgfsetbuttcap%
\pgfsetroundjoin%
\definecolor{currentfill}{rgb}{0.000000,0.000000,0.000000}%
\pgfsetfillcolor{currentfill}%
\pgfsetlinewidth{0.501875pt}%
\definecolor{currentstroke}{rgb}{0.000000,0.000000,0.000000}%
\pgfsetstrokecolor{currentstroke}%
\pgfsetdash{}{0pt}%
\pgfsys@defobject{currentmarker}{\pgfqpoint{-0.020833in}{0.000000in}}{\pgfqpoint{-0.000000in}{0.000000in}}{%
\pgfpathmoveto{\pgfqpoint{-0.000000in}{0.000000in}}%
\pgfpathlineto{\pgfqpoint{-0.020833in}{0.000000in}}%
\pgfusepath{stroke,fill}%
}%
\begin{pgfscope}%
\pgfsys@transformshift{2.850886in}{1.347336in}%
\pgfsys@useobject{currentmarker}{}%
\end{pgfscope}%
\end{pgfscope}%
\begin{pgfscope}%
\pgfsetbuttcap%
\pgfsetroundjoin%
\definecolor{currentfill}{rgb}{0.000000,0.000000,0.000000}%
\pgfsetfillcolor{currentfill}%
\pgfsetlinewidth{0.501875pt}%
\definecolor{currentstroke}{rgb}{0.000000,0.000000,0.000000}%
\pgfsetstrokecolor{currentstroke}%
\pgfsetdash{}{0pt}%
\pgfsys@defobject{currentmarker}{\pgfqpoint{0.000000in}{0.000000in}}{\pgfqpoint{0.020833in}{0.000000in}}{%
\pgfpathmoveto{\pgfqpoint{0.000000in}{0.000000in}}%
\pgfpathlineto{\pgfqpoint{0.020833in}{0.000000in}}%
\pgfusepath{stroke,fill}%
}%
\begin{pgfscope}%
\pgfsys@transformshift{0.609415in}{1.439991in}%
\pgfsys@useobject{currentmarker}{}%
\end{pgfscope}%
\end{pgfscope}%
\begin{pgfscope}%
\pgfsetbuttcap%
\pgfsetroundjoin%
\definecolor{currentfill}{rgb}{0.000000,0.000000,0.000000}%
\pgfsetfillcolor{currentfill}%
\pgfsetlinewidth{0.501875pt}%
\definecolor{currentstroke}{rgb}{0.000000,0.000000,0.000000}%
\pgfsetstrokecolor{currentstroke}%
\pgfsetdash{}{0pt}%
\pgfsys@defobject{currentmarker}{\pgfqpoint{-0.020833in}{0.000000in}}{\pgfqpoint{-0.000000in}{0.000000in}}{%
\pgfpathmoveto{\pgfqpoint{-0.000000in}{0.000000in}}%
\pgfpathlineto{\pgfqpoint{-0.020833in}{0.000000in}}%
\pgfusepath{stroke,fill}%
}%
\begin{pgfscope}%
\pgfsys@transformshift{2.850886in}{1.439991in}%
\pgfsys@useobject{currentmarker}{}%
\end{pgfscope}%
\end{pgfscope}%
\begin{pgfscope}%
\pgfsetbuttcap%
\pgfsetroundjoin%
\definecolor{currentfill}{rgb}{0.000000,0.000000,0.000000}%
\pgfsetfillcolor{currentfill}%
\pgfsetlinewidth{0.501875pt}%
\definecolor{currentstroke}{rgb}{0.000000,0.000000,0.000000}%
\pgfsetstrokecolor{currentstroke}%
\pgfsetdash{}{0pt}%
\pgfsys@defobject{currentmarker}{\pgfqpoint{0.000000in}{0.000000in}}{\pgfqpoint{0.020833in}{0.000000in}}{%
\pgfpathmoveto{\pgfqpoint{0.000000in}{0.000000in}}%
\pgfpathlineto{\pgfqpoint{0.020833in}{0.000000in}}%
\pgfusepath{stroke,fill}%
}%
\begin{pgfscope}%
\pgfsys@transformshift{0.609415in}{1.625300in}%
\pgfsys@useobject{currentmarker}{}%
\end{pgfscope}%
\end{pgfscope}%
\begin{pgfscope}%
\pgfsetbuttcap%
\pgfsetroundjoin%
\definecolor{currentfill}{rgb}{0.000000,0.000000,0.000000}%
\pgfsetfillcolor{currentfill}%
\pgfsetlinewidth{0.501875pt}%
\definecolor{currentstroke}{rgb}{0.000000,0.000000,0.000000}%
\pgfsetstrokecolor{currentstroke}%
\pgfsetdash{}{0pt}%
\pgfsys@defobject{currentmarker}{\pgfqpoint{-0.020833in}{0.000000in}}{\pgfqpoint{-0.000000in}{0.000000in}}{%
\pgfpathmoveto{\pgfqpoint{-0.000000in}{0.000000in}}%
\pgfpathlineto{\pgfqpoint{-0.020833in}{0.000000in}}%
\pgfusepath{stroke,fill}%
}%
\begin{pgfscope}%
\pgfsys@transformshift{2.850886in}{1.625300in}%
\pgfsys@useobject{currentmarker}{}%
\end{pgfscope}%
\end{pgfscope}%
\begin{pgfscope}%
\pgfsetbuttcap%
\pgfsetroundjoin%
\definecolor{currentfill}{rgb}{0.000000,0.000000,0.000000}%
\pgfsetfillcolor{currentfill}%
\pgfsetlinewidth{0.501875pt}%
\definecolor{currentstroke}{rgb}{0.000000,0.000000,0.000000}%
\pgfsetstrokecolor{currentstroke}%
\pgfsetdash{}{0pt}%
\pgfsys@defobject{currentmarker}{\pgfqpoint{0.000000in}{0.000000in}}{\pgfqpoint{0.020833in}{0.000000in}}{%
\pgfpathmoveto{\pgfqpoint{0.000000in}{0.000000in}}%
\pgfpathlineto{\pgfqpoint{0.020833in}{0.000000in}}%
\pgfusepath{stroke,fill}%
}%
\begin{pgfscope}%
\pgfsys@transformshift{0.609415in}{1.717955in}%
\pgfsys@useobject{currentmarker}{}%
\end{pgfscope}%
\end{pgfscope}%
\begin{pgfscope}%
\pgfsetbuttcap%
\pgfsetroundjoin%
\definecolor{currentfill}{rgb}{0.000000,0.000000,0.000000}%
\pgfsetfillcolor{currentfill}%
\pgfsetlinewidth{0.501875pt}%
\definecolor{currentstroke}{rgb}{0.000000,0.000000,0.000000}%
\pgfsetstrokecolor{currentstroke}%
\pgfsetdash{}{0pt}%
\pgfsys@defobject{currentmarker}{\pgfqpoint{-0.020833in}{0.000000in}}{\pgfqpoint{-0.000000in}{0.000000in}}{%
\pgfpathmoveto{\pgfqpoint{-0.000000in}{0.000000in}}%
\pgfpathlineto{\pgfqpoint{-0.020833in}{0.000000in}}%
\pgfusepath{stroke,fill}%
}%
\begin{pgfscope}%
\pgfsys@transformshift{2.850886in}{1.717955in}%
\pgfsys@useobject{currentmarker}{}%
\end{pgfscope}%
\end{pgfscope}%
\begin{pgfscope}%
\pgfsetbuttcap%
\pgfsetroundjoin%
\definecolor{currentfill}{rgb}{0.000000,0.000000,0.000000}%
\pgfsetfillcolor{currentfill}%
\pgfsetlinewidth{0.501875pt}%
\definecolor{currentstroke}{rgb}{0.000000,0.000000,0.000000}%
\pgfsetstrokecolor{currentstroke}%
\pgfsetdash{}{0pt}%
\pgfsys@defobject{currentmarker}{\pgfqpoint{0.000000in}{0.000000in}}{\pgfqpoint{0.020833in}{0.000000in}}{%
\pgfpathmoveto{\pgfqpoint{0.000000in}{0.000000in}}%
\pgfpathlineto{\pgfqpoint{0.020833in}{0.000000in}}%
\pgfusepath{stroke,fill}%
}%
\begin{pgfscope}%
\pgfsys@transformshift{0.609415in}{1.810610in}%
\pgfsys@useobject{currentmarker}{}%
\end{pgfscope}%
\end{pgfscope}%
\begin{pgfscope}%
\pgfsetbuttcap%
\pgfsetroundjoin%
\definecolor{currentfill}{rgb}{0.000000,0.000000,0.000000}%
\pgfsetfillcolor{currentfill}%
\pgfsetlinewidth{0.501875pt}%
\definecolor{currentstroke}{rgb}{0.000000,0.000000,0.000000}%
\pgfsetstrokecolor{currentstroke}%
\pgfsetdash{}{0pt}%
\pgfsys@defobject{currentmarker}{\pgfqpoint{-0.020833in}{0.000000in}}{\pgfqpoint{-0.000000in}{0.000000in}}{%
\pgfpathmoveto{\pgfqpoint{-0.000000in}{0.000000in}}%
\pgfpathlineto{\pgfqpoint{-0.020833in}{0.000000in}}%
\pgfusepath{stroke,fill}%
}%
\begin{pgfscope}%
\pgfsys@transformshift{2.850886in}{1.810610in}%
\pgfsys@useobject{currentmarker}{}%
\end{pgfscope}%
\end{pgfscope}%
\begin{pgfscope}%
\pgfsetbuttcap%
\pgfsetroundjoin%
\definecolor{currentfill}{rgb}{0.000000,0.000000,0.000000}%
\pgfsetfillcolor{currentfill}%
\pgfsetlinewidth{0.501875pt}%
\definecolor{currentstroke}{rgb}{0.000000,0.000000,0.000000}%
\pgfsetstrokecolor{currentstroke}%
\pgfsetdash{}{0pt}%
\pgfsys@defobject{currentmarker}{\pgfqpoint{0.000000in}{0.000000in}}{\pgfqpoint{0.020833in}{0.000000in}}{%
\pgfpathmoveto{\pgfqpoint{0.000000in}{0.000000in}}%
\pgfpathlineto{\pgfqpoint{0.020833in}{0.000000in}}%
\pgfusepath{stroke,fill}%
}%
\begin{pgfscope}%
\pgfsys@transformshift{0.609415in}{1.903264in}%
\pgfsys@useobject{currentmarker}{}%
\end{pgfscope}%
\end{pgfscope}%
\begin{pgfscope}%
\pgfsetbuttcap%
\pgfsetroundjoin%
\definecolor{currentfill}{rgb}{0.000000,0.000000,0.000000}%
\pgfsetfillcolor{currentfill}%
\pgfsetlinewidth{0.501875pt}%
\definecolor{currentstroke}{rgb}{0.000000,0.000000,0.000000}%
\pgfsetstrokecolor{currentstroke}%
\pgfsetdash{}{0pt}%
\pgfsys@defobject{currentmarker}{\pgfqpoint{-0.020833in}{0.000000in}}{\pgfqpoint{-0.000000in}{0.000000in}}{%
\pgfpathmoveto{\pgfqpoint{-0.000000in}{0.000000in}}%
\pgfpathlineto{\pgfqpoint{-0.020833in}{0.000000in}}%
\pgfusepath{stroke,fill}%
}%
\begin{pgfscope}%
\pgfsys@transformshift{2.850886in}{1.903264in}%
\pgfsys@useobject{currentmarker}{}%
\end{pgfscope}%
\end{pgfscope}%
\begin{pgfscope}%
\definecolor{textcolor}{rgb}{0.000000,0.000000,0.000000}%
\pgfsetstrokecolor{textcolor}%
\pgfsetfillcolor{textcolor}%
\pgftext[x=0.188889in,y=1.236093in,,bottom,rotate=90.000000]{\color{textcolor}\rmfamily\fontsize{10.000000}{12.000000}\selectfont \(\displaystyle C(K)\)}%
\end{pgfscope}%
\begin{pgfscope}%
\pgfpathrectangle{\pgfqpoint{0.609415in}{0.422992in}}{\pgfqpoint{2.241471in}{1.626201in}}%
\pgfusepath{clip}%
\pgfsetrectcap%
\pgfsetroundjoin%
\pgfsetlinewidth{1.003750pt}%
\definecolor{currentstroke}{rgb}{0.047059,0.364706,0.647059}%
\pgfsetstrokecolor{currentstroke}%
\pgfsetdash{}{0pt}%
\pgfpathmoveto{\pgfqpoint{0.631607in}{1.944034in}}%
\pgfpathlineto{\pgfqpoint{0.653800in}{1.928267in}}%
\pgfpathlineto{\pgfqpoint{0.675993in}{1.912727in}}%
\pgfpathlineto{\pgfqpoint{0.698186in}{1.897002in}}%
\pgfpathlineto{\pgfqpoint{0.720379in}{1.882119in}}%
\pgfpathlineto{\pgfqpoint{0.742571in}{1.867403in}}%
\pgfpathlineto{\pgfqpoint{0.764764in}{1.852853in}}%
\pgfpathlineto{\pgfqpoint{0.786957in}{1.838474in}}%
\pgfpathlineto{\pgfqpoint{0.809150in}{1.823835in}}%
\pgfpathlineto{\pgfqpoint{0.831342in}{1.809828in}}%
\pgfpathlineto{\pgfqpoint{0.853535in}{1.796723in}}%
\pgfpathlineto{\pgfqpoint{0.875728in}{1.783365in}}%
\pgfpathlineto{\pgfqpoint{0.897921in}{1.770230in}}%
\pgfpathlineto{\pgfqpoint{0.920114in}{1.757181in}}%
\pgfpathlineto{\pgfqpoint{0.942306in}{1.744370in}}%
\pgfpathlineto{\pgfqpoint{0.964499in}{1.732158in}}%
\pgfpathlineto{\pgfqpoint{0.986692in}{1.720003in}}%
\pgfpathlineto{\pgfqpoint{1.008885in}{1.707758in}}%
\pgfpathlineto{\pgfqpoint{1.031078in}{1.695176in}}%
\pgfpathlineto{\pgfqpoint{1.053270in}{1.683134in}}%
\pgfpathlineto{\pgfqpoint{1.075463in}{1.671236in}}%
\pgfpathlineto{\pgfqpoint{1.097656in}{1.659513in}}%
\pgfpathlineto{\pgfqpoint{1.119849in}{1.647718in}}%
\pgfpathlineto{\pgfqpoint{1.142041in}{1.635847in}}%
\pgfpathlineto{\pgfqpoint{1.164234in}{1.624347in}}%
\pgfpathlineto{\pgfqpoint{1.186427in}{1.613075in}}%
\pgfpathlineto{\pgfqpoint{1.208620in}{1.601571in}}%
\pgfpathlineto{\pgfqpoint{1.230813in}{1.590202in}}%
\pgfpathlineto{\pgfqpoint{1.253005in}{1.578806in}}%
\pgfpathlineto{\pgfqpoint{1.275198in}{1.567546in}}%
\pgfpathlineto{\pgfqpoint{1.297391in}{1.556374in}}%
\pgfpathlineto{\pgfqpoint{1.319584in}{1.545001in}}%
\pgfpathlineto{\pgfqpoint{1.341777in}{1.534219in}}%
\pgfpathlineto{\pgfqpoint{1.363969in}{1.523147in}}%
\pgfpathlineto{\pgfqpoint{1.386162in}{1.512089in}}%
\pgfpathlineto{\pgfqpoint{1.408355in}{1.501354in}}%
\pgfpathlineto{\pgfqpoint{1.430548in}{1.490304in}}%
\pgfpathlineto{\pgfqpoint{1.452740in}{1.479612in}}%
\pgfpathlineto{\pgfqpoint{1.474933in}{1.468711in}}%
\pgfpathlineto{\pgfqpoint{1.497126in}{1.457965in}}%
\pgfpathlineto{\pgfqpoint{1.519319in}{1.446807in}}%
\pgfpathlineto{\pgfqpoint{1.541512in}{1.436808in}}%
\pgfpathlineto{\pgfqpoint{1.563704in}{1.425979in}}%
\pgfpathlineto{\pgfqpoint{1.585897in}{1.415188in}}%
\pgfpathlineto{\pgfqpoint{1.608090in}{1.404336in}}%
\pgfpathlineto{\pgfqpoint{1.630283in}{1.393858in}}%
\pgfpathlineto{\pgfqpoint{1.652476in}{1.383399in}}%
\pgfpathlineto{\pgfqpoint{1.674668in}{1.372677in}}%
\pgfpathlineto{\pgfqpoint{1.696861in}{1.362201in}}%
\pgfpathlineto{\pgfqpoint{1.719054in}{1.351453in}}%
\pgfpathlineto{\pgfqpoint{1.741247in}{1.341032in}}%
\pgfpathlineto{\pgfqpoint{1.763439in}{1.331079in}}%
\pgfpathlineto{\pgfqpoint{1.785632in}{1.320971in}}%
\pgfpathlineto{\pgfqpoint{1.807825in}{1.311052in}}%
\pgfpathlineto{\pgfqpoint{1.830018in}{1.300869in}}%
\pgfpathlineto{\pgfqpoint{1.852211in}{1.291133in}}%
\pgfpathlineto{\pgfqpoint{1.874403in}{1.281309in}}%
\pgfpathlineto{\pgfqpoint{1.896596in}{1.271480in}}%
\pgfpathlineto{\pgfqpoint{1.918789in}{1.261440in}}%
\pgfpathlineto{\pgfqpoint{1.940982in}{1.251543in}}%
\pgfpathlineto{\pgfqpoint{1.963175in}{1.241561in}}%
\pgfpathlineto{\pgfqpoint{1.985367in}{1.231705in}}%
\pgfpathlineto{\pgfqpoint{2.007560in}{1.221818in}}%
\pgfpathlineto{\pgfqpoint{2.029753in}{1.212300in}}%
\pgfpathlineto{\pgfqpoint{2.051946in}{1.202461in}}%
\pgfpathlineto{\pgfqpoint{2.074139in}{1.192693in}}%
\pgfpathlineto{\pgfqpoint{2.096331in}{1.182938in}}%
\pgfpathlineto{\pgfqpoint{2.118524in}{1.173159in}}%
\pgfpathlineto{\pgfqpoint{2.140717in}{1.163628in}}%
\pgfpathlineto{\pgfqpoint{2.162910in}{1.153487in}}%
\pgfpathlineto{\pgfqpoint{2.185102in}{1.143689in}}%
\pgfpathlineto{\pgfqpoint{2.207295in}{1.134100in}}%
\pgfpathlineto{\pgfqpoint{2.229488in}{1.124505in}}%
\pgfpathlineto{\pgfqpoint{2.251681in}{1.114686in}}%
\pgfpathlineto{\pgfqpoint{2.273874in}{1.105695in}}%
\pgfpathlineto{\pgfqpoint{2.296066in}{1.096101in}}%
\pgfpathlineto{\pgfqpoint{2.318259in}{1.086315in}}%
\pgfpathlineto{\pgfqpoint{2.340452in}{1.076530in}}%
\pgfpathlineto{\pgfqpoint{2.362645in}{1.067728in}}%
\pgfpathlineto{\pgfqpoint{2.384838in}{1.058529in}}%
\pgfpathlineto{\pgfqpoint{2.407030in}{1.048994in}}%
\pgfpathlineto{\pgfqpoint{2.429223in}{1.039846in}}%
\pgfpathlineto{\pgfqpoint{2.451416in}{1.030625in}}%
\pgfpathlineto{\pgfqpoint{2.473609in}{1.021518in}}%
\pgfpathlineto{\pgfqpoint{2.495801in}{1.012304in}}%
\pgfpathlineto{\pgfqpoint{2.517994in}{1.003611in}}%
\pgfpathlineto{\pgfqpoint{2.540187in}{0.994510in}}%
\pgfpathlineto{\pgfqpoint{2.562380in}{0.985457in}}%
\pgfpathlineto{\pgfqpoint{2.584573in}{0.976434in}}%
\pgfpathlineto{\pgfqpoint{2.606765in}{0.967379in}}%
\pgfpathlineto{\pgfqpoint{2.628958in}{0.958270in}}%
\pgfpathlineto{\pgfqpoint{2.651151in}{0.949445in}}%
\pgfpathlineto{\pgfqpoint{2.673344in}{0.940754in}}%
\pgfpathlineto{\pgfqpoint{2.695537in}{0.932095in}}%
\pgfpathlineto{\pgfqpoint{2.717729in}{0.923198in}}%
\pgfpathlineto{\pgfqpoint{2.739922in}{0.914055in}}%
\pgfpathlineto{\pgfqpoint{2.762115in}{0.905183in}}%
\pgfpathlineto{\pgfqpoint{2.784308in}{0.896249in}}%
\pgfpathlineto{\pgfqpoint{2.806500in}{0.887585in}}%
\pgfusepath{stroke}%
\end{pgfscope}%
\begin{pgfscope}%
\pgfpathrectangle{\pgfqpoint{0.609415in}{0.422992in}}{\pgfqpoint{2.241471in}{1.626201in}}%
\pgfusepath{clip}%
\pgfsetrectcap%
\pgfsetroundjoin%
\pgfsetlinewidth{1.003750pt}%
\definecolor{currentstroke}{rgb}{0.000000,0.725490,0.270588}%
\pgfsetstrokecolor{currentstroke}%
\pgfsetdash{}{0pt}%
\pgfpathmoveto{\pgfqpoint{0.631607in}{1.961842in}}%
\pgfpathlineto{\pgfqpoint{0.653800in}{1.951288in}}%
\pgfpathlineto{\pgfqpoint{0.675993in}{1.942307in}}%
\pgfpathlineto{\pgfqpoint{0.698186in}{1.932730in}}%
\pgfpathlineto{\pgfqpoint{0.720379in}{1.923845in}}%
\pgfpathlineto{\pgfqpoint{0.742571in}{1.915754in}}%
\pgfpathlineto{\pgfqpoint{0.764764in}{1.908072in}}%
\pgfpathlineto{\pgfqpoint{0.786957in}{1.900454in}}%
\pgfpathlineto{\pgfqpoint{0.809150in}{1.892849in}}%
\pgfpathlineto{\pgfqpoint{0.831342in}{1.885667in}}%
\pgfpathlineto{\pgfqpoint{0.853535in}{1.878274in}}%
\pgfpathlineto{\pgfqpoint{0.875728in}{1.871141in}}%
\pgfpathlineto{\pgfqpoint{0.897921in}{1.863759in}}%
\pgfpathlineto{\pgfqpoint{0.920114in}{1.856628in}}%
\pgfpathlineto{\pgfqpoint{0.942306in}{1.849255in}}%
\pgfpathlineto{\pgfqpoint{0.964499in}{1.842098in}}%
\pgfpathlineto{\pgfqpoint{0.986692in}{1.835302in}}%
\pgfpathlineto{\pgfqpoint{1.008885in}{1.828622in}}%
\pgfpathlineto{\pgfqpoint{1.031078in}{1.821834in}}%
\pgfpathlineto{\pgfqpoint{1.053270in}{1.815203in}}%
\pgfpathlineto{\pgfqpoint{1.075463in}{1.808447in}}%
\pgfpathlineto{\pgfqpoint{1.097656in}{1.802089in}}%
\pgfpathlineto{\pgfqpoint{1.119849in}{1.795986in}}%
\pgfpathlineto{\pgfqpoint{1.142041in}{1.789533in}}%
\pgfpathlineto{\pgfqpoint{1.164234in}{1.782965in}}%
\pgfpathlineto{\pgfqpoint{1.186427in}{1.776839in}}%
\pgfpathlineto{\pgfqpoint{1.208620in}{1.770551in}}%
\pgfpathlineto{\pgfqpoint{1.230813in}{1.764628in}}%
\pgfpathlineto{\pgfqpoint{1.253005in}{1.758245in}}%
\pgfpathlineto{\pgfqpoint{1.275198in}{1.752363in}}%
\pgfpathlineto{\pgfqpoint{1.297391in}{1.746386in}}%
\pgfpathlineto{\pgfqpoint{1.319584in}{1.740200in}}%
\pgfpathlineto{\pgfqpoint{1.341777in}{1.734400in}}%
\pgfpathlineto{\pgfqpoint{1.363969in}{1.728196in}}%
\pgfpathlineto{\pgfqpoint{1.386162in}{1.722436in}}%
\pgfpathlineto{\pgfqpoint{1.408355in}{1.716538in}}%
\pgfpathlineto{\pgfqpoint{1.430548in}{1.710435in}}%
\pgfpathlineto{\pgfqpoint{1.452740in}{1.704451in}}%
\pgfpathlineto{\pgfqpoint{1.474933in}{1.698445in}}%
\pgfpathlineto{\pgfqpoint{1.497126in}{1.692439in}}%
\pgfpathlineto{\pgfqpoint{1.519319in}{1.686726in}}%
\pgfpathlineto{\pgfqpoint{1.541512in}{1.680503in}}%
\pgfpathlineto{\pgfqpoint{1.563704in}{1.674723in}}%
\pgfpathlineto{\pgfqpoint{1.585897in}{1.669199in}}%
\pgfpathlineto{\pgfqpoint{1.608090in}{1.663129in}}%
\pgfpathlineto{\pgfqpoint{1.630283in}{1.657263in}}%
\pgfpathlineto{\pgfqpoint{1.652476in}{1.651381in}}%
\pgfpathlineto{\pgfqpoint{1.674668in}{1.645566in}}%
\pgfpathlineto{\pgfqpoint{1.696861in}{1.640005in}}%
\pgfpathlineto{\pgfqpoint{1.719054in}{1.634185in}}%
\pgfpathlineto{\pgfqpoint{1.741247in}{1.628474in}}%
\pgfpathlineto{\pgfqpoint{1.763439in}{1.623013in}}%
\pgfpathlineto{\pgfqpoint{1.785632in}{1.617377in}}%
\pgfpathlineto{\pgfqpoint{1.807825in}{1.611825in}}%
\pgfpathlineto{\pgfqpoint{1.830018in}{1.606303in}}%
\pgfpathlineto{\pgfqpoint{1.852211in}{1.600469in}}%
\pgfpathlineto{\pgfqpoint{1.874403in}{1.594425in}}%
\pgfpathlineto{\pgfqpoint{1.896596in}{1.589204in}}%
\pgfpathlineto{\pgfqpoint{1.918789in}{1.583647in}}%
\pgfpathlineto{\pgfqpoint{1.940982in}{1.577841in}}%
\pgfpathlineto{\pgfqpoint{1.963175in}{1.572175in}}%
\pgfpathlineto{\pgfqpoint{1.985367in}{1.566765in}}%
\pgfpathlineto{\pgfqpoint{2.007560in}{1.560733in}}%
\pgfpathlineto{\pgfqpoint{2.029753in}{1.555534in}}%
\pgfpathlineto{\pgfqpoint{2.051946in}{1.549929in}}%
\pgfpathlineto{\pgfqpoint{2.074139in}{1.544869in}}%
\pgfpathlineto{\pgfqpoint{2.096331in}{1.539611in}}%
\pgfpathlineto{\pgfqpoint{2.118524in}{1.534193in}}%
\pgfpathlineto{\pgfqpoint{2.140717in}{1.528943in}}%
\pgfpathlineto{\pgfqpoint{2.162910in}{1.523397in}}%
\pgfpathlineto{\pgfqpoint{2.185102in}{1.518368in}}%
\pgfpathlineto{\pgfqpoint{2.207295in}{1.512662in}}%
\pgfpathlineto{\pgfqpoint{2.229488in}{1.507318in}}%
\pgfpathlineto{\pgfqpoint{2.251681in}{1.501814in}}%
\pgfpathlineto{\pgfqpoint{2.273874in}{1.495604in}}%
\pgfpathlineto{\pgfqpoint{2.296066in}{1.490250in}}%
\pgfpathlineto{\pgfqpoint{2.318259in}{1.484772in}}%
\pgfpathlineto{\pgfqpoint{2.340452in}{1.479209in}}%
\pgfpathlineto{\pgfqpoint{2.362645in}{1.473461in}}%
\pgfpathlineto{\pgfqpoint{2.384838in}{1.468604in}}%
\pgfpathlineto{\pgfqpoint{2.407030in}{1.463366in}}%
\pgfpathlineto{\pgfqpoint{2.429223in}{1.457775in}}%
\pgfpathlineto{\pgfqpoint{2.451416in}{1.452370in}}%
\pgfpathlineto{\pgfqpoint{2.473609in}{1.446838in}}%
\pgfpathlineto{\pgfqpoint{2.495801in}{1.441286in}}%
\pgfpathlineto{\pgfqpoint{2.517994in}{1.436027in}}%
\pgfpathlineto{\pgfqpoint{2.540187in}{1.430802in}}%
\pgfpathlineto{\pgfqpoint{2.562380in}{1.425280in}}%
\pgfpathlineto{\pgfqpoint{2.584573in}{1.419831in}}%
\pgfpathlineto{\pgfqpoint{2.606765in}{1.414709in}}%
\pgfpathlineto{\pgfqpoint{2.628958in}{1.409455in}}%
\pgfpathlineto{\pgfqpoint{2.651151in}{1.404208in}}%
\pgfpathlineto{\pgfqpoint{2.673344in}{1.398808in}}%
\pgfpathlineto{\pgfqpoint{2.695537in}{1.393626in}}%
\pgfpathlineto{\pgfqpoint{2.717729in}{1.388259in}}%
\pgfpathlineto{\pgfqpoint{2.739922in}{1.382779in}}%
\pgfpathlineto{\pgfqpoint{2.762115in}{1.377528in}}%
\pgfpathlineto{\pgfqpoint{2.784308in}{1.372158in}}%
\pgfpathlineto{\pgfqpoint{2.806500in}{1.366638in}}%
\pgfusepath{stroke}%
\end{pgfscope}%
\begin{pgfscope}%
\pgfpathrectangle{\pgfqpoint{0.609415in}{0.422992in}}{\pgfqpoint{2.241471in}{1.626201in}}%
\pgfusepath{clip}%
\pgfsetrectcap%
\pgfsetroundjoin%
\pgfsetlinewidth{1.003750pt}%
\definecolor{currentstroke}{rgb}{1.000000,0.584314,0.000000}%
\pgfsetstrokecolor{currentstroke}%
\pgfsetdash{}{0pt}%
\pgfpathmoveto{\pgfqpoint{0.631607in}{1.968501in}}%
\pgfpathlineto{\pgfqpoint{0.653800in}{1.959176in}}%
\pgfpathlineto{\pgfqpoint{0.675993in}{1.950514in}}%
\pgfpathlineto{\pgfqpoint{0.698186in}{1.942096in}}%
\pgfpathlineto{\pgfqpoint{0.720379in}{1.934180in}}%
\pgfpathlineto{\pgfqpoint{0.742571in}{1.925616in}}%
\pgfpathlineto{\pgfqpoint{0.764764in}{1.917786in}}%
\pgfpathlineto{\pgfqpoint{0.786957in}{1.910685in}}%
\pgfpathlineto{\pgfqpoint{0.809150in}{1.902532in}}%
\pgfpathlineto{\pgfqpoint{0.831342in}{1.895339in}}%
\pgfpathlineto{\pgfqpoint{0.853535in}{1.888207in}}%
\pgfpathlineto{\pgfqpoint{0.875728in}{1.881360in}}%
\pgfpathlineto{\pgfqpoint{0.897921in}{1.874471in}}%
\pgfpathlineto{\pgfqpoint{0.920114in}{1.867517in}}%
\pgfpathlineto{\pgfqpoint{0.942306in}{1.860611in}}%
\pgfpathlineto{\pgfqpoint{0.964499in}{1.854266in}}%
\pgfpathlineto{\pgfqpoint{0.986692in}{1.846736in}}%
\pgfpathlineto{\pgfqpoint{1.008885in}{1.840256in}}%
\pgfpathlineto{\pgfqpoint{1.031078in}{1.833692in}}%
\pgfpathlineto{\pgfqpoint{1.053270in}{1.826634in}}%
\pgfpathlineto{\pgfqpoint{1.075463in}{1.820524in}}%
\pgfpathlineto{\pgfqpoint{1.097656in}{1.813993in}}%
\pgfpathlineto{\pgfqpoint{1.119849in}{1.807898in}}%
\pgfpathlineto{\pgfqpoint{1.142041in}{1.801236in}}%
\pgfpathlineto{\pgfqpoint{1.164234in}{1.795250in}}%
\pgfpathlineto{\pgfqpoint{1.186427in}{1.788927in}}%
\pgfpathlineto{\pgfqpoint{1.208620in}{1.783048in}}%
\pgfpathlineto{\pgfqpoint{1.230813in}{1.777551in}}%
\pgfpathlineto{\pgfqpoint{1.253005in}{1.771704in}}%
\pgfpathlineto{\pgfqpoint{1.275198in}{1.765453in}}%
\pgfpathlineto{\pgfqpoint{1.297391in}{1.759650in}}%
\pgfpathlineto{\pgfqpoint{1.319584in}{1.753880in}}%
\pgfpathlineto{\pgfqpoint{1.341777in}{1.748227in}}%
\pgfpathlineto{\pgfqpoint{1.363969in}{1.742093in}}%
\pgfpathlineto{\pgfqpoint{1.386162in}{1.736263in}}%
\pgfpathlineto{\pgfqpoint{1.408355in}{1.730324in}}%
\pgfpathlineto{\pgfqpoint{1.430548in}{1.724527in}}%
\pgfpathlineto{\pgfqpoint{1.452740in}{1.718818in}}%
\pgfpathlineto{\pgfqpoint{1.474933in}{1.713505in}}%
\pgfpathlineto{\pgfqpoint{1.497126in}{1.708229in}}%
\pgfpathlineto{\pgfqpoint{1.519319in}{1.703501in}}%
\pgfpathlineto{\pgfqpoint{1.541512in}{1.697953in}}%
\pgfpathlineto{\pgfqpoint{1.563704in}{1.693097in}}%
\pgfpathlineto{\pgfqpoint{1.585897in}{1.687726in}}%
\pgfpathlineto{\pgfqpoint{1.608090in}{1.682349in}}%
\pgfpathlineto{\pgfqpoint{1.630283in}{1.676919in}}%
\pgfpathlineto{\pgfqpoint{1.652476in}{1.671509in}}%
\pgfpathlineto{\pgfqpoint{1.674668in}{1.666071in}}%
\pgfpathlineto{\pgfqpoint{1.696861in}{1.660768in}}%
\pgfpathlineto{\pgfqpoint{1.719054in}{1.655762in}}%
\pgfpathlineto{\pgfqpoint{1.741247in}{1.650291in}}%
\pgfpathlineto{\pgfqpoint{1.763439in}{1.645537in}}%
\pgfpathlineto{\pgfqpoint{1.785632in}{1.640515in}}%
\pgfpathlineto{\pgfqpoint{1.807825in}{1.635431in}}%
\pgfpathlineto{\pgfqpoint{1.830018in}{1.630433in}}%
\pgfpathlineto{\pgfqpoint{1.852211in}{1.625037in}}%
\pgfpathlineto{\pgfqpoint{1.874403in}{1.620259in}}%
\pgfpathlineto{\pgfqpoint{1.896596in}{1.615710in}}%
\pgfpathlineto{\pgfqpoint{1.918789in}{1.610988in}}%
\pgfpathlineto{\pgfqpoint{1.940982in}{1.605757in}}%
\pgfpathlineto{\pgfqpoint{1.963175in}{1.601057in}}%
\pgfpathlineto{\pgfqpoint{1.985367in}{1.596342in}}%
\pgfpathlineto{\pgfqpoint{2.007560in}{1.591513in}}%
\pgfpathlineto{\pgfqpoint{2.029753in}{1.586776in}}%
\pgfpathlineto{\pgfqpoint{2.051946in}{1.582032in}}%
\pgfpathlineto{\pgfqpoint{2.074139in}{1.576879in}}%
\pgfpathlineto{\pgfqpoint{2.096331in}{1.571905in}}%
\pgfpathlineto{\pgfqpoint{2.118524in}{1.566846in}}%
\pgfpathlineto{\pgfqpoint{2.140717in}{1.561890in}}%
\pgfpathlineto{\pgfqpoint{2.162910in}{1.557443in}}%
\pgfpathlineto{\pgfqpoint{2.185102in}{1.552652in}}%
\pgfpathlineto{\pgfqpoint{2.207295in}{1.547765in}}%
\pgfpathlineto{\pgfqpoint{2.229488in}{1.543346in}}%
\pgfpathlineto{\pgfqpoint{2.251681in}{1.538929in}}%
\pgfpathlineto{\pgfqpoint{2.273874in}{1.533950in}}%
\pgfpathlineto{\pgfqpoint{2.296066in}{1.528879in}}%
\pgfpathlineto{\pgfqpoint{2.318259in}{1.524401in}}%
\pgfpathlineto{\pgfqpoint{2.340452in}{1.519910in}}%
\pgfpathlineto{\pgfqpoint{2.362645in}{1.514942in}}%
\pgfpathlineto{\pgfqpoint{2.384838in}{1.510153in}}%
\pgfpathlineto{\pgfqpoint{2.407030in}{1.505407in}}%
\pgfpathlineto{\pgfqpoint{2.429223in}{1.500611in}}%
\pgfpathlineto{\pgfqpoint{2.451416in}{1.495906in}}%
\pgfpathlineto{\pgfqpoint{2.473609in}{1.491325in}}%
\pgfpathlineto{\pgfqpoint{2.495801in}{1.486662in}}%
\pgfpathlineto{\pgfqpoint{2.517994in}{1.482387in}}%
\pgfpathlineto{\pgfqpoint{2.540187in}{1.477592in}}%
\pgfpathlineto{\pgfqpoint{2.562380in}{1.473220in}}%
\pgfpathlineto{\pgfqpoint{2.584573in}{1.468545in}}%
\pgfpathlineto{\pgfqpoint{2.606765in}{1.463585in}}%
\pgfpathlineto{\pgfqpoint{2.628958in}{1.458771in}}%
\pgfpathlineto{\pgfqpoint{2.651151in}{1.454593in}}%
\pgfpathlineto{\pgfqpoint{2.673344in}{1.450434in}}%
\pgfpathlineto{\pgfqpoint{2.695537in}{1.445778in}}%
\pgfpathlineto{\pgfqpoint{2.717729in}{1.441322in}}%
\pgfpathlineto{\pgfqpoint{2.739922in}{1.436699in}}%
\pgfpathlineto{\pgfqpoint{2.762115in}{1.432219in}}%
\pgfpathlineto{\pgfqpoint{2.784308in}{1.428042in}}%
\pgfpathlineto{\pgfqpoint{2.806500in}{1.423586in}}%
\pgfusepath{stroke}%
\end{pgfscope}%
\begin{pgfscope}%
\pgfpathrectangle{\pgfqpoint{0.609415in}{0.422992in}}{\pgfqpoint{2.241471in}{1.626201in}}%
\pgfusepath{clip}%
\pgfsetrectcap%
\pgfsetroundjoin%
\pgfsetlinewidth{1.003750pt}%
\definecolor{currentstroke}{rgb}{1.000000,0.172549,0.000000}%
\pgfsetstrokecolor{currentstroke}%
\pgfsetdash{}{0pt}%
\pgfpathmoveto{\pgfqpoint{0.631607in}{1.975275in}}%
\pgfpathlineto{\pgfqpoint{0.653800in}{1.969553in}}%
\pgfpathlineto{\pgfqpoint{0.675993in}{1.965175in}}%
\pgfpathlineto{\pgfqpoint{0.698186in}{1.960292in}}%
\pgfpathlineto{\pgfqpoint{0.720379in}{1.955135in}}%
\pgfpathlineto{\pgfqpoint{0.742571in}{1.950548in}}%
\pgfpathlineto{\pgfqpoint{0.764764in}{1.945847in}}%
\pgfpathlineto{\pgfqpoint{0.786957in}{1.941245in}}%
\pgfpathlineto{\pgfqpoint{0.809150in}{1.936656in}}%
\pgfpathlineto{\pgfqpoint{0.831342in}{1.931824in}}%
\pgfpathlineto{\pgfqpoint{0.853535in}{1.927099in}}%
\pgfpathlineto{\pgfqpoint{0.875728in}{1.922163in}}%
\pgfpathlineto{\pgfqpoint{0.897921in}{1.917930in}}%
\pgfpathlineto{\pgfqpoint{0.920114in}{1.912894in}}%
\pgfpathlineto{\pgfqpoint{0.942306in}{1.908138in}}%
\pgfpathlineto{\pgfqpoint{0.964499in}{1.903601in}}%
\pgfpathlineto{\pgfqpoint{0.986692in}{1.899269in}}%
\pgfpathlineto{\pgfqpoint{1.008885in}{1.894658in}}%
\pgfpathlineto{\pgfqpoint{1.031078in}{1.890732in}}%
\pgfpathlineto{\pgfqpoint{1.053270in}{1.886294in}}%
\pgfpathlineto{\pgfqpoint{1.075463in}{1.882184in}}%
\pgfpathlineto{\pgfqpoint{1.097656in}{1.877902in}}%
\pgfpathlineto{\pgfqpoint{1.119849in}{1.873477in}}%
\pgfpathlineto{\pgfqpoint{1.142041in}{1.869146in}}%
\pgfpathlineto{\pgfqpoint{1.164234in}{1.864604in}}%
\pgfpathlineto{\pgfqpoint{1.186427in}{1.859853in}}%
\pgfpathlineto{\pgfqpoint{1.208620in}{1.855668in}}%
\pgfpathlineto{\pgfqpoint{1.230813in}{1.851367in}}%
\pgfpathlineto{\pgfqpoint{1.253005in}{1.847255in}}%
\pgfpathlineto{\pgfqpoint{1.275198in}{1.842901in}}%
\pgfpathlineto{\pgfqpoint{1.297391in}{1.839120in}}%
\pgfpathlineto{\pgfqpoint{1.319584in}{1.835140in}}%
\pgfpathlineto{\pgfqpoint{1.341777in}{1.830894in}}%
\pgfpathlineto{\pgfqpoint{1.363969in}{1.826905in}}%
\pgfpathlineto{\pgfqpoint{1.386162in}{1.823132in}}%
\pgfpathlineto{\pgfqpoint{1.408355in}{1.819083in}}%
\pgfpathlineto{\pgfqpoint{1.430548in}{1.815175in}}%
\pgfpathlineto{\pgfqpoint{1.452740in}{1.810924in}}%
\pgfpathlineto{\pgfqpoint{1.474933in}{1.807123in}}%
\pgfpathlineto{\pgfqpoint{1.497126in}{1.803144in}}%
\pgfpathlineto{\pgfqpoint{1.519319in}{1.799330in}}%
\pgfpathlineto{\pgfqpoint{1.541512in}{1.795521in}}%
\pgfpathlineto{\pgfqpoint{1.563704in}{1.791745in}}%
\pgfpathlineto{\pgfqpoint{1.585897in}{1.787881in}}%
\pgfpathlineto{\pgfqpoint{1.608090in}{1.784305in}}%
\pgfpathlineto{\pgfqpoint{1.630283in}{1.780760in}}%
\pgfpathlineto{\pgfqpoint{1.652476in}{1.776699in}}%
\pgfpathlineto{\pgfqpoint{1.674668in}{1.772525in}}%
\pgfpathlineto{\pgfqpoint{1.696861in}{1.768655in}}%
\pgfpathlineto{\pgfqpoint{1.719054in}{1.764825in}}%
\pgfpathlineto{\pgfqpoint{1.741247in}{1.760905in}}%
\pgfpathlineto{\pgfqpoint{1.763439in}{1.757103in}}%
\pgfpathlineto{\pgfqpoint{1.785632in}{1.753151in}}%
\pgfpathlineto{\pgfqpoint{1.807825in}{1.748787in}}%
\pgfpathlineto{\pgfqpoint{1.830018in}{1.745084in}}%
\pgfpathlineto{\pgfqpoint{1.852211in}{1.741639in}}%
\pgfpathlineto{\pgfqpoint{1.874403in}{1.737895in}}%
\pgfpathlineto{\pgfqpoint{1.896596in}{1.734300in}}%
\pgfpathlineto{\pgfqpoint{1.918789in}{1.731088in}}%
\pgfpathlineto{\pgfqpoint{1.940982in}{1.727333in}}%
\pgfpathlineto{\pgfqpoint{1.963175in}{1.723813in}}%
\pgfpathlineto{\pgfqpoint{1.985367in}{1.720396in}}%
\pgfpathlineto{\pgfqpoint{2.007560in}{1.717012in}}%
\pgfpathlineto{\pgfqpoint{2.029753in}{1.713374in}}%
\pgfpathlineto{\pgfqpoint{2.051946in}{1.710023in}}%
\pgfpathlineto{\pgfqpoint{2.074139in}{1.706521in}}%
\pgfpathlineto{\pgfqpoint{2.096331in}{1.702877in}}%
\pgfpathlineto{\pgfqpoint{2.118524in}{1.699380in}}%
\pgfpathlineto{\pgfqpoint{2.140717in}{1.695691in}}%
\pgfpathlineto{\pgfqpoint{2.162910in}{1.692004in}}%
\pgfpathlineto{\pgfqpoint{2.185102in}{1.688101in}}%
\pgfpathlineto{\pgfqpoint{2.207295in}{1.684326in}}%
\pgfpathlineto{\pgfqpoint{2.229488in}{1.680695in}}%
\pgfpathlineto{\pgfqpoint{2.251681in}{1.677341in}}%
\pgfpathlineto{\pgfqpoint{2.273874in}{1.673407in}}%
\pgfpathlineto{\pgfqpoint{2.296066in}{1.669395in}}%
\pgfpathlineto{\pgfqpoint{2.318259in}{1.665765in}}%
\pgfpathlineto{\pgfqpoint{2.340452in}{1.662436in}}%
\pgfpathlineto{\pgfqpoint{2.362645in}{1.659149in}}%
\pgfpathlineto{\pgfqpoint{2.384838in}{1.655541in}}%
\pgfpathlineto{\pgfqpoint{2.407030in}{1.652208in}}%
\pgfpathlineto{\pgfqpoint{2.429223in}{1.648788in}}%
\pgfpathlineto{\pgfqpoint{2.451416in}{1.645361in}}%
\pgfpathlineto{\pgfqpoint{2.473609in}{1.642113in}}%
\pgfpathlineto{\pgfqpoint{2.495801in}{1.638706in}}%
\pgfpathlineto{\pgfqpoint{2.517994in}{1.635299in}}%
\pgfpathlineto{\pgfqpoint{2.540187in}{1.631907in}}%
\pgfpathlineto{\pgfqpoint{2.562380in}{1.628201in}}%
\pgfpathlineto{\pgfqpoint{2.584573in}{1.624914in}}%
\pgfpathlineto{\pgfqpoint{2.606765in}{1.621061in}}%
\pgfpathlineto{\pgfqpoint{2.628958in}{1.617338in}}%
\pgfpathlineto{\pgfqpoint{2.651151in}{1.613729in}}%
\pgfpathlineto{\pgfqpoint{2.673344in}{1.610425in}}%
\pgfpathlineto{\pgfqpoint{2.695537in}{1.606919in}}%
\pgfpathlineto{\pgfqpoint{2.717729in}{1.602949in}}%
\pgfpathlineto{\pgfqpoint{2.739922in}{1.599464in}}%
\pgfpathlineto{\pgfqpoint{2.762115in}{1.596226in}}%
\pgfpathlineto{\pgfqpoint{2.784308in}{1.592618in}}%
\pgfpathlineto{\pgfqpoint{2.806500in}{1.588717in}}%
\pgfusepath{stroke}%
\end{pgfscope}%
\begin{pgfscope}%
\pgfpathrectangle{\pgfqpoint{0.609415in}{0.422992in}}{\pgfqpoint{2.241471in}{1.626201in}}%
\pgfusepath{clip}%
\pgfsetrectcap%
\pgfsetroundjoin%
\pgfsetlinewidth{1.003750pt}%
\definecolor{currentstroke}{rgb}{0.517647,0.356863,0.592157}%
\pgfsetstrokecolor{currentstroke}%
\pgfsetdash{}{0pt}%
\pgfpathmoveto{\pgfqpoint{0.631607in}{1.912626in}}%
\pgfpathlineto{\pgfqpoint{0.653800in}{1.888220in}}%
\pgfpathlineto{\pgfqpoint{0.675993in}{1.864487in}}%
\pgfpathlineto{\pgfqpoint{0.698186in}{1.841979in}}%
\pgfpathlineto{\pgfqpoint{0.720379in}{1.821086in}}%
\pgfpathlineto{\pgfqpoint{0.742571in}{1.801222in}}%
\pgfpathlineto{\pgfqpoint{0.764764in}{1.781516in}}%
\pgfpathlineto{\pgfqpoint{0.786957in}{1.762475in}}%
\pgfpathlineto{\pgfqpoint{0.809150in}{1.742945in}}%
\pgfpathlineto{\pgfqpoint{0.831342in}{1.724954in}}%
\pgfpathlineto{\pgfqpoint{0.853535in}{1.706468in}}%
\pgfpathlineto{\pgfqpoint{0.875728in}{1.687506in}}%
\pgfpathlineto{\pgfqpoint{0.897921in}{1.669221in}}%
\pgfpathlineto{\pgfqpoint{0.920114in}{1.652394in}}%
\pgfpathlineto{\pgfqpoint{0.942306in}{1.635687in}}%
\pgfpathlineto{\pgfqpoint{0.964499in}{1.619229in}}%
\pgfpathlineto{\pgfqpoint{0.986692in}{1.602243in}}%
\pgfpathlineto{\pgfqpoint{1.008885in}{1.585480in}}%
\pgfpathlineto{\pgfqpoint{1.031078in}{1.568971in}}%
\pgfpathlineto{\pgfqpoint{1.053270in}{1.552279in}}%
\pgfpathlineto{\pgfqpoint{1.075463in}{1.536345in}}%
\pgfpathlineto{\pgfqpoint{1.097656in}{1.521688in}}%
\pgfpathlineto{\pgfqpoint{1.119849in}{1.506364in}}%
\pgfpathlineto{\pgfqpoint{1.142041in}{1.491720in}}%
\pgfpathlineto{\pgfqpoint{1.164234in}{1.475104in}}%
\pgfpathlineto{\pgfqpoint{1.186427in}{1.460044in}}%
\pgfpathlineto{\pgfqpoint{1.208620in}{1.444275in}}%
\pgfpathlineto{\pgfqpoint{1.230813in}{1.429243in}}%
\pgfpathlineto{\pgfqpoint{1.253005in}{1.415357in}}%
\pgfpathlineto{\pgfqpoint{1.275198in}{1.400996in}}%
\pgfpathlineto{\pgfqpoint{1.297391in}{1.385414in}}%
\pgfpathlineto{\pgfqpoint{1.319584in}{1.369849in}}%
\pgfpathlineto{\pgfqpoint{1.341777in}{1.355482in}}%
\pgfpathlineto{\pgfqpoint{1.363969in}{1.341086in}}%
\pgfpathlineto{\pgfqpoint{1.386162in}{1.326664in}}%
\pgfpathlineto{\pgfqpoint{1.408355in}{1.312367in}}%
\pgfpathlineto{\pgfqpoint{1.430548in}{1.298497in}}%
\pgfpathlineto{\pgfqpoint{1.452740in}{1.285425in}}%
\pgfpathlineto{\pgfqpoint{1.474933in}{1.270961in}}%
\pgfpathlineto{\pgfqpoint{1.497126in}{1.257580in}}%
\pgfpathlineto{\pgfqpoint{1.519319in}{1.243324in}}%
\pgfpathlineto{\pgfqpoint{1.541512in}{1.229248in}}%
\pgfpathlineto{\pgfqpoint{1.563704in}{1.215399in}}%
\pgfpathlineto{\pgfqpoint{1.585897in}{1.202315in}}%
\pgfpathlineto{\pgfqpoint{1.608090in}{1.188365in}}%
\pgfpathlineto{\pgfqpoint{1.630283in}{1.174701in}}%
\pgfpathlineto{\pgfqpoint{1.652476in}{1.161673in}}%
\pgfpathlineto{\pgfqpoint{1.674668in}{1.147944in}}%
\pgfpathlineto{\pgfqpoint{1.696861in}{1.134606in}}%
\pgfpathlineto{\pgfqpoint{1.719054in}{1.122018in}}%
\pgfpathlineto{\pgfqpoint{1.741247in}{1.109119in}}%
\pgfpathlineto{\pgfqpoint{1.763439in}{1.096563in}}%
\pgfpathlineto{\pgfqpoint{1.785632in}{1.082944in}}%
\pgfpathlineto{\pgfqpoint{1.807825in}{1.069014in}}%
\pgfpathlineto{\pgfqpoint{1.830018in}{1.055305in}}%
\pgfpathlineto{\pgfqpoint{1.852211in}{1.042008in}}%
\pgfpathlineto{\pgfqpoint{1.874403in}{1.028282in}}%
\pgfpathlineto{\pgfqpoint{1.896596in}{1.014643in}}%
\pgfpathlineto{\pgfqpoint{1.918789in}{1.001134in}}%
\pgfpathlineto{\pgfqpoint{1.940982in}{0.988100in}}%
\pgfpathlineto{\pgfqpoint{1.963175in}{0.975007in}}%
\pgfpathlineto{\pgfqpoint{1.985367in}{0.961282in}}%
\pgfpathlineto{\pgfqpoint{2.007560in}{0.947743in}}%
\pgfpathlineto{\pgfqpoint{2.029753in}{0.934413in}}%
\pgfpathlineto{\pgfqpoint{2.051946in}{0.919984in}}%
\pgfpathlineto{\pgfqpoint{2.074139in}{0.906860in}}%
\pgfpathlineto{\pgfqpoint{2.096331in}{0.893445in}}%
\pgfpathlineto{\pgfqpoint{2.118524in}{0.880514in}}%
\pgfpathlineto{\pgfqpoint{2.140717in}{0.868195in}}%
\pgfpathlineto{\pgfqpoint{2.162910in}{0.855068in}}%
\pgfpathlineto{\pgfqpoint{2.185102in}{0.843002in}}%
\pgfpathlineto{\pgfqpoint{2.207295in}{0.831226in}}%
\pgfpathlineto{\pgfqpoint{2.229488in}{0.818774in}}%
\pgfpathlineto{\pgfqpoint{2.251681in}{0.805906in}}%
\pgfpathlineto{\pgfqpoint{2.273874in}{0.793549in}}%
\pgfpathlineto{\pgfqpoint{2.296066in}{0.780034in}}%
\pgfpathlineto{\pgfqpoint{2.318259in}{0.767575in}}%
\pgfpathlineto{\pgfqpoint{2.340452in}{0.755138in}}%
\pgfpathlineto{\pgfqpoint{2.362645in}{0.743119in}}%
\pgfpathlineto{\pgfqpoint{2.384838in}{0.730158in}}%
\pgfpathlineto{\pgfqpoint{2.407030in}{0.717139in}}%
\pgfpathlineto{\pgfqpoint{2.429223in}{0.705129in}}%
\pgfpathlineto{\pgfqpoint{2.451416in}{0.692092in}}%
\pgfpathlineto{\pgfqpoint{2.473609in}{0.679856in}}%
\pgfpathlineto{\pgfqpoint{2.495801in}{0.667507in}}%
\pgfpathlineto{\pgfqpoint{2.517994in}{0.654215in}}%
\pgfpathlineto{\pgfqpoint{2.540187in}{0.641826in}}%
\pgfpathlineto{\pgfqpoint{2.562380in}{0.629735in}}%
\pgfpathlineto{\pgfqpoint{2.584573in}{0.617386in}}%
\pgfpathlineto{\pgfqpoint{2.606765in}{0.604864in}}%
\pgfpathlineto{\pgfqpoint{2.628958in}{0.591967in}}%
\pgfpathlineto{\pgfqpoint{2.651151in}{0.579587in}}%
\pgfpathlineto{\pgfqpoint{2.673344in}{0.567814in}}%
\pgfpathlineto{\pgfqpoint{2.695537in}{0.555458in}}%
\pgfpathlineto{\pgfqpoint{2.717729in}{0.543833in}}%
\pgfpathlineto{\pgfqpoint{2.739922in}{0.531779in}}%
\pgfpathlineto{\pgfqpoint{2.762115in}{0.520271in}}%
\pgfpathlineto{\pgfqpoint{2.784308in}{0.508339in}}%
\pgfpathlineto{\pgfqpoint{2.806500in}{0.496910in}}%
\pgfusepath{stroke}%
\end{pgfscope}%
\begin{pgfscope}%
\pgfsetrectcap%
\pgfsetmiterjoin%
\pgfsetlinewidth{0.501875pt}%
\definecolor{currentstroke}{rgb}{0.000000,0.000000,0.000000}%
\pgfsetstrokecolor{currentstroke}%
\pgfsetdash{}{0pt}%
\pgfpathmoveto{\pgfqpoint{0.609415in}{0.422992in}}%
\pgfpathlineto{\pgfqpoint{0.609415in}{2.049193in}}%
\pgfusepath{stroke}%
\end{pgfscope}%
\begin{pgfscope}%
\pgfsetrectcap%
\pgfsetmiterjoin%
\pgfsetlinewidth{0.501875pt}%
\definecolor{currentstroke}{rgb}{0.000000,0.000000,0.000000}%
\pgfsetstrokecolor{currentstroke}%
\pgfsetdash{}{0pt}%
\pgfpathmoveto{\pgfqpoint{2.850886in}{0.422992in}}%
\pgfpathlineto{\pgfqpoint{2.850886in}{2.049193in}}%
\pgfusepath{stroke}%
\end{pgfscope}%
\begin{pgfscope}%
\pgfsetrectcap%
\pgfsetmiterjoin%
\pgfsetlinewidth{0.501875pt}%
\definecolor{currentstroke}{rgb}{0.000000,0.000000,0.000000}%
\pgfsetstrokecolor{currentstroke}%
\pgfsetdash{}{0pt}%
\pgfpathmoveto{\pgfqpoint{0.609415in}{0.422992in}}%
\pgfpathlineto{\pgfqpoint{2.850886in}{0.422992in}}%
\pgfusepath{stroke}%
\end{pgfscope}%
\begin{pgfscope}%
\pgfsetrectcap%
\pgfsetmiterjoin%
\pgfsetlinewidth{0.501875pt}%
\definecolor{currentstroke}{rgb}{0.000000,0.000000,0.000000}%
\pgfsetstrokecolor{currentstroke}%
\pgfsetdash{}{0pt}%
\pgfpathmoveto{\pgfqpoint{0.609415in}{2.049193in}}%
\pgfpathlineto{\pgfqpoint{2.850886in}{2.049193in}}%
\pgfusepath{stroke}%
\end{pgfscope}%
\begin{pgfscope}%
\definecolor{textcolor}{rgb}{0.000000,0.000000,0.000000}%
\pgfsetstrokecolor{textcolor}%
\pgfsetfillcolor{textcolor}%
\pgftext[x=1.730150in,y=2.132526in,,base]{\color{textcolor}\rmfamily\fontsize{12.000000}{14.400000}\selectfont Continuity}%
\end{pgfscope}%
\begin{pgfscope}%
\pgfsetbuttcap%
\pgfsetmiterjoin%
\definecolor{currentfill}{rgb}{1.000000,1.000000,1.000000}%
\pgfsetfillcolor{currentfill}%
\pgfsetlinewidth{0.000000pt}%
\definecolor{currentstroke}{rgb}{0.000000,0.000000,0.000000}%
\pgfsetstrokecolor{currentstroke}%
\pgfsetstrokeopacity{0.000000}%
\pgfsetdash{}{0pt}%
\pgfpathmoveto{\pgfqpoint{3.420229in}{0.422992in}}%
\pgfpathlineto{\pgfqpoint{5.661701in}{0.422992in}}%
\pgfpathlineto{\pgfqpoint{5.661701in}{4.374193in}}%
\pgfpathlineto{\pgfqpoint{3.420229in}{4.374193in}}%
\pgfpathlineto{\pgfqpoint{3.420229in}{0.422992in}}%
\pgfpathclose%
\pgfusepath{fill}%
\end{pgfscope}%
\begin{pgfscope}%
\pgfsetbuttcap%
\pgfsetroundjoin%
\definecolor{currentfill}{rgb}{0.000000,0.000000,0.000000}%
\pgfsetfillcolor{currentfill}%
\pgfsetlinewidth{0.501875pt}%
\definecolor{currentstroke}{rgb}{0.000000,0.000000,0.000000}%
\pgfsetstrokecolor{currentstroke}%
\pgfsetdash{}{0pt}%
\pgfsys@defobject{currentmarker}{\pgfqpoint{0.000000in}{0.000000in}}{\pgfqpoint{0.000000in}{0.041667in}}{%
\pgfpathmoveto{\pgfqpoint{0.000000in}{0.000000in}}%
\pgfpathlineto{\pgfqpoint{0.000000in}{0.041667in}}%
\pgfusepath{stroke,fill}%
}%
\begin{pgfscope}%
\pgfsys@transformshift{3.420229in}{0.422992in}%
\pgfsys@useobject{currentmarker}{}%
\end{pgfscope}%
\end{pgfscope}%
\begin{pgfscope}%
\pgfsetbuttcap%
\pgfsetroundjoin%
\definecolor{currentfill}{rgb}{0.000000,0.000000,0.000000}%
\pgfsetfillcolor{currentfill}%
\pgfsetlinewidth{0.501875pt}%
\definecolor{currentstroke}{rgb}{0.000000,0.000000,0.000000}%
\pgfsetstrokecolor{currentstroke}%
\pgfsetdash{}{0pt}%
\pgfsys@defobject{currentmarker}{\pgfqpoint{0.000000in}{-0.041667in}}{\pgfqpoint{0.000000in}{0.000000in}}{%
\pgfpathmoveto{\pgfqpoint{0.000000in}{0.000000in}}%
\pgfpathlineto{\pgfqpoint{0.000000in}{-0.041667in}}%
\pgfusepath{stroke,fill}%
}%
\begin{pgfscope}%
\pgfsys@transformshift{3.420229in}{4.374193in}%
\pgfsys@useobject{currentmarker}{}%
\end{pgfscope}%
\end{pgfscope}%
\begin{pgfscope}%
\definecolor{textcolor}{rgb}{0.000000,0.000000,0.000000}%
\pgfsetstrokecolor{textcolor}%
\pgfsetfillcolor{textcolor}%
\pgftext[x=3.420229in,y=0.374381in,,top]{\color{textcolor}\rmfamily\fontsize{10.000000}{12.000000}\selectfont \(\displaystyle {0}\)}%
\end{pgfscope}%
\begin{pgfscope}%
\pgfsetbuttcap%
\pgfsetroundjoin%
\definecolor{currentfill}{rgb}{0.000000,0.000000,0.000000}%
\pgfsetfillcolor{currentfill}%
\pgfsetlinewidth{0.501875pt}%
\definecolor{currentstroke}{rgb}{0.000000,0.000000,0.000000}%
\pgfsetstrokecolor{currentstroke}%
\pgfsetdash{}{0pt}%
\pgfsys@defobject{currentmarker}{\pgfqpoint{0.000000in}{0.000000in}}{\pgfqpoint{0.000000in}{0.041667in}}{%
\pgfpathmoveto{\pgfqpoint{0.000000in}{0.000000in}}%
\pgfpathlineto{\pgfqpoint{0.000000in}{0.041667in}}%
\pgfusepath{stroke,fill}%
}%
\begin{pgfscope}%
\pgfsys@transformshift{3.864085in}{0.422992in}%
\pgfsys@useobject{currentmarker}{}%
\end{pgfscope}%
\end{pgfscope}%
\begin{pgfscope}%
\pgfsetbuttcap%
\pgfsetroundjoin%
\definecolor{currentfill}{rgb}{0.000000,0.000000,0.000000}%
\pgfsetfillcolor{currentfill}%
\pgfsetlinewidth{0.501875pt}%
\definecolor{currentstroke}{rgb}{0.000000,0.000000,0.000000}%
\pgfsetstrokecolor{currentstroke}%
\pgfsetdash{}{0pt}%
\pgfsys@defobject{currentmarker}{\pgfqpoint{0.000000in}{-0.041667in}}{\pgfqpoint{0.000000in}{0.000000in}}{%
\pgfpathmoveto{\pgfqpoint{0.000000in}{0.000000in}}%
\pgfpathlineto{\pgfqpoint{0.000000in}{-0.041667in}}%
\pgfusepath{stroke,fill}%
}%
\begin{pgfscope}%
\pgfsys@transformshift{3.864085in}{4.374193in}%
\pgfsys@useobject{currentmarker}{}%
\end{pgfscope}%
\end{pgfscope}%
\begin{pgfscope}%
\definecolor{textcolor}{rgb}{0.000000,0.000000,0.000000}%
\pgfsetstrokecolor{textcolor}%
\pgfsetfillcolor{textcolor}%
\pgftext[x=3.864085in,y=0.374381in,,top]{\color{textcolor}\rmfamily\fontsize{10.000000}{12.000000}\selectfont \(\displaystyle {20}\)}%
\end{pgfscope}%
\begin{pgfscope}%
\pgfsetbuttcap%
\pgfsetroundjoin%
\definecolor{currentfill}{rgb}{0.000000,0.000000,0.000000}%
\pgfsetfillcolor{currentfill}%
\pgfsetlinewidth{0.501875pt}%
\definecolor{currentstroke}{rgb}{0.000000,0.000000,0.000000}%
\pgfsetstrokecolor{currentstroke}%
\pgfsetdash{}{0pt}%
\pgfsys@defobject{currentmarker}{\pgfqpoint{0.000000in}{0.000000in}}{\pgfqpoint{0.000000in}{0.041667in}}{%
\pgfpathmoveto{\pgfqpoint{0.000000in}{0.000000in}}%
\pgfpathlineto{\pgfqpoint{0.000000in}{0.041667in}}%
\pgfusepath{stroke,fill}%
}%
\begin{pgfscope}%
\pgfsys@transformshift{4.307941in}{0.422992in}%
\pgfsys@useobject{currentmarker}{}%
\end{pgfscope}%
\end{pgfscope}%
\begin{pgfscope}%
\pgfsetbuttcap%
\pgfsetroundjoin%
\definecolor{currentfill}{rgb}{0.000000,0.000000,0.000000}%
\pgfsetfillcolor{currentfill}%
\pgfsetlinewidth{0.501875pt}%
\definecolor{currentstroke}{rgb}{0.000000,0.000000,0.000000}%
\pgfsetstrokecolor{currentstroke}%
\pgfsetdash{}{0pt}%
\pgfsys@defobject{currentmarker}{\pgfqpoint{0.000000in}{-0.041667in}}{\pgfqpoint{0.000000in}{0.000000in}}{%
\pgfpathmoveto{\pgfqpoint{0.000000in}{0.000000in}}%
\pgfpathlineto{\pgfqpoint{0.000000in}{-0.041667in}}%
\pgfusepath{stroke,fill}%
}%
\begin{pgfscope}%
\pgfsys@transformshift{4.307941in}{4.374193in}%
\pgfsys@useobject{currentmarker}{}%
\end{pgfscope}%
\end{pgfscope}%
\begin{pgfscope}%
\definecolor{textcolor}{rgb}{0.000000,0.000000,0.000000}%
\pgfsetstrokecolor{textcolor}%
\pgfsetfillcolor{textcolor}%
\pgftext[x=4.307941in,y=0.374381in,,top]{\color{textcolor}\rmfamily\fontsize{10.000000}{12.000000}\selectfont \(\displaystyle {40}\)}%
\end{pgfscope}%
\begin{pgfscope}%
\pgfsetbuttcap%
\pgfsetroundjoin%
\definecolor{currentfill}{rgb}{0.000000,0.000000,0.000000}%
\pgfsetfillcolor{currentfill}%
\pgfsetlinewidth{0.501875pt}%
\definecolor{currentstroke}{rgb}{0.000000,0.000000,0.000000}%
\pgfsetstrokecolor{currentstroke}%
\pgfsetdash{}{0pt}%
\pgfsys@defobject{currentmarker}{\pgfqpoint{0.000000in}{0.000000in}}{\pgfqpoint{0.000000in}{0.041667in}}{%
\pgfpathmoveto{\pgfqpoint{0.000000in}{0.000000in}}%
\pgfpathlineto{\pgfqpoint{0.000000in}{0.041667in}}%
\pgfusepath{stroke,fill}%
}%
\begin{pgfscope}%
\pgfsys@transformshift{4.751797in}{0.422992in}%
\pgfsys@useobject{currentmarker}{}%
\end{pgfscope}%
\end{pgfscope}%
\begin{pgfscope}%
\pgfsetbuttcap%
\pgfsetroundjoin%
\definecolor{currentfill}{rgb}{0.000000,0.000000,0.000000}%
\pgfsetfillcolor{currentfill}%
\pgfsetlinewidth{0.501875pt}%
\definecolor{currentstroke}{rgb}{0.000000,0.000000,0.000000}%
\pgfsetstrokecolor{currentstroke}%
\pgfsetdash{}{0pt}%
\pgfsys@defobject{currentmarker}{\pgfqpoint{0.000000in}{-0.041667in}}{\pgfqpoint{0.000000in}{0.000000in}}{%
\pgfpathmoveto{\pgfqpoint{0.000000in}{0.000000in}}%
\pgfpathlineto{\pgfqpoint{0.000000in}{-0.041667in}}%
\pgfusepath{stroke,fill}%
}%
\begin{pgfscope}%
\pgfsys@transformshift{4.751797in}{4.374193in}%
\pgfsys@useobject{currentmarker}{}%
\end{pgfscope}%
\end{pgfscope}%
\begin{pgfscope}%
\definecolor{textcolor}{rgb}{0.000000,0.000000,0.000000}%
\pgfsetstrokecolor{textcolor}%
\pgfsetfillcolor{textcolor}%
\pgftext[x=4.751797in,y=0.374381in,,top]{\color{textcolor}\rmfamily\fontsize{10.000000}{12.000000}\selectfont \(\displaystyle {60}\)}%
\end{pgfscope}%
\begin{pgfscope}%
\pgfsetbuttcap%
\pgfsetroundjoin%
\definecolor{currentfill}{rgb}{0.000000,0.000000,0.000000}%
\pgfsetfillcolor{currentfill}%
\pgfsetlinewidth{0.501875pt}%
\definecolor{currentstroke}{rgb}{0.000000,0.000000,0.000000}%
\pgfsetstrokecolor{currentstroke}%
\pgfsetdash{}{0pt}%
\pgfsys@defobject{currentmarker}{\pgfqpoint{0.000000in}{0.000000in}}{\pgfqpoint{0.000000in}{0.041667in}}{%
\pgfpathmoveto{\pgfqpoint{0.000000in}{0.000000in}}%
\pgfpathlineto{\pgfqpoint{0.000000in}{0.041667in}}%
\pgfusepath{stroke,fill}%
}%
\begin{pgfscope}%
\pgfsys@transformshift{5.195652in}{0.422992in}%
\pgfsys@useobject{currentmarker}{}%
\end{pgfscope}%
\end{pgfscope}%
\begin{pgfscope}%
\pgfsetbuttcap%
\pgfsetroundjoin%
\definecolor{currentfill}{rgb}{0.000000,0.000000,0.000000}%
\pgfsetfillcolor{currentfill}%
\pgfsetlinewidth{0.501875pt}%
\definecolor{currentstroke}{rgb}{0.000000,0.000000,0.000000}%
\pgfsetstrokecolor{currentstroke}%
\pgfsetdash{}{0pt}%
\pgfsys@defobject{currentmarker}{\pgfqpoint{0.000000in}{-0.041667in}}{\pgfqpoint{0.000000in}{0.000000in}}{%
\pgfpathmoveto{\pgfqpoint{0.000000in}{0.000000in}}%
\pgfpathlineto{\pgfqpoint{0.000000in}{-0.041667in}}%
\pgfusepath{stroke,fill}%
}%
\begin{pgfscope}%
\pgfsys@transformshift{5.195652in}{4.374193in}%
\pgfsys@useobject{currentmarker}{}%
\end{pgfscope}%
\end{pgfscope}%
\begin{pgfscope}%
\definecolor{textcolor}{rgb}{0.000000,0.000000,0.000000}%
\pgfsetstrokecolor{textcolor}%
\pgfsetfillcolor{textcolor}%
\pgftext[x=5.195652in,y=0.374381in,,top]{\color{textcolor}\rmfamily\fontsize{10.000000}{12.000000}\selectfont \(\displaystyle {80}\)}%
\end{pgfscope}%
\begin{pgfscope}%
\pgfsetbuttcap%
\pgfsetroundjoin%
\definecolor{currentfill}{rgb}{0.000000,0.000000,0.000000}%
\pgfsetfillcolor{currentfill}%
\pgfsetlinewidth{0.501875pt}%
\definecolor{currentstroke}{rgb}{0.000000,0.000000,0.000000}%
\pgfsetstrokecolor{currentstroke}%
\pgfsetdash{}{0pt}%
\pgfsys@defobject{currentmarker}{\pgfqpoint{0.000000in}{0.000000in}}{\pgfqpoint{0.000000in}{0.020833in}}{%
\pgfpathmoveto{\pgfqpoint{0.000000in}{0.000000in}}%
\pgfpathlineto{\pgfqpoint{0.000000in}{0.020833in}}%
\pgfusepath{stroke,fill}%
}%
\begin{pgfscope}%
\pgfsys@transformshift{3.531193in}{0.422992in}%
\pgfsys@useobject{currentmarker}{}%
\end{pgfscope}%
\end{pgfscope}%
\begin{pgfscope}%
\pgfsetbuttcap%
\pgfsetroundjoin%
\definecolor{currentfill}{rgb}{0.000000,0.000000,0.000000}%
\pgfsetfillcolor{currentfill}%
\pgfsetlinewidth{0.501875pt}%
\definecolor{currentstroke}{rgb}{0.000000,0.000000,0.000000}%
\pgfsetstrokecolor{currentstroke}%
\pgfsetdash{}{0pt}%
\pgfsys@defobject{currentmarker}{\pgfqpoint{0.000000in}{-0.020833in}}{\pgfqpoint{0.000000in}{0.000000in}}{%
\pgfpathmoveto{\pgfqpoint{0.000000in}{0.000000in}}%
\pgfpathlineto{\pgfqpoint{0.000000in}{-0.020833in}}%
\pgfusepath{stroke,fill}%
}%
\begin{pgfscope}%
\pgfsys@transformshift{3.531193in}{4.374193in}%
\pgfsys@useobject{currentmarker}{}%
\end{pgfscope}%
\end{pgfscope}%
\begin{pgfscope}%
\pgfsetbuttcap%
\pgfsetroundjoin%
\definecolor{currentfill}{rgb}{0.000000,0.000000,0.000000}%
\pgfsetfillcolor{currentfill}%
\pgfsetlinewidth{0.501875pt}%
\definecolor{currentstroke}{rgb}{0.000000,0.000000,0.000000}%
\pgfsetstrokecolor{currentstroke}%
\pgfsetdash{}{0pt}%
\pgfsys@defobject{currentmarker}{\pgfqpoint{0.000000in}{0.000000in}}{\pgfqpoint{0.000000in}{0.020833in}}{%
\pgfpathmoveto{\pgfqpoint{0.000000in}{0.000000in}}%
\pgfpathlineto{\pgfqpoint{0.000000in}{0.020833in}}%
\pgfusepath{stroke,fill}%
}%
\begin{pgfscope}%
\pgfsys@transformshift{3.642157in}{0.422992in}%
\pgfsys@useobject{currentmarker}{}%
\end{pgfscope}%
\end{pgfscope}%
\begin{pgfscope}%
\pgfsetbuttcap%
\pgfsetroundjoin%
\definecolor{currentfill}{rgb}{0.000000,0.000000,0.000000}%
\pgfsetfillcolor{currentfill}%
\pgfsetlinewidth{0.501875pt}%
\definecolor{currentstroke}{rgb}{0.000000,0.000000,0.000000}%
\pgfsetstrokecolor{currentstroke}%
\pgfsetdash{}{0pt}%
\pgfsys@defobject{currentmarker}{\pgfqpoint{0.000000in}{-0.020833in}}{\pgfqpoint{0.000000in}{0.000000in}}{%
\pgfpathmoveto{\pgfqpoint{0.000000in}{0.000000in}}%
\pgfpathlineto{\pgfqpoint{0.000000in}{-0.020833in}}%
\pgfusepath{stroke,fill}%
}%
\begin{pgfscope}%
\pgfsys@transformshift{3.642157in}{4.374193in}%
\pgfsys@useobject{currentmarker}{}%
\end{pgfscope}%
\end{pgfscope}%
\begin{pgfscope}%
\pgfsetbuttcap%
\pgfsetroundjoin%
\definecolor{currentfill}{rgb}{0.000000,0.000000,0.000000}%
\pgfsetfillcolor{currentfill}%
\pgfsetlinewidth{0.501875pt}%
\definecolor{currentstroke}{rgb}{0.000000,0.000000,0.000000}%
\pgfsetstrokecolor{currentstroke}%
\pgfsetdash{}{0pt}%
\pgfsys@defobject{currentmarker}{\pgfqpoint{0.000000in}{0.000000in}}{\pgfqpoint{0.000000in}{0.020833in}}{%
\pgfpathmoveto{\pgfqpoint{0.000000in}{0.000000in}}%
\pgfpathlineto{\pgfqpoint{0.000000in}{0.020833in}}%
\pgfusepath{stroke,fill}%
}%
\begin{pgfscope}%
\pgfsys@transformshift{3.753121in}{0.422992in}%
\pgfsys@useobject{currentmarker}{}%
\end{pgfscope}%
\end{pgfscope}%
\begin{pgfscope}%
\pgfsetbuttcap%
\pgfsetroundjoin%
\definecolor{currentfill}{rgb}{0.000000,0.000000,0.000000}%
\pgfsetfillcolor{currentfill}%
\pgfsetlinewidth{0.501875pt}%
\definecolor{currentstroke}{rgb}{0.000000,0.000000,0.000000}%
\pgfsetstrokecolor{currentstroke}%
\pgfsetdash{}{0pt}%
\pgfsys@defobject{currentmarker}{\pgfqpoint{0.000000in}{-0.020833in}}{\pgfqpoint{0.000000in}{0.000000in}}{%
\pgfpathmoveto{\pgfqpoint{0.000000in}{0.000000in}}%
\pgfpathlineto{\pgfqpoint{0.000000in}{-0.020833in}}%
\pgfusepath{stroke,fill}%
}%
\begin{pgfscope}%
\pgfsys@transformshift{3.753121in}{4.374193in}%
\pgfsys@useobject{currentmarker}{}%
\end{pgfscope}%
\end{pgfscope}%
\begin{pgfscope}%
\pgfsetbuttcap%
\pgfsetroundjoin%
\definecolor{currentfill}{rgb}{0.000000,0.000000,0.000000}%
\pgfsetfillcolor{currentfill}%
\pgfsetlinewidth{0.501875pt}%
\definecolor{currentstroke}{rgb}{0.000000,0.000000,0.000000}%
\pgfsetstrokecolor{currentstroke}%
\pgfsetdash{}{0pt}%
\pgfsys@defobject{currentmarker}{\pgfqpoint{0.000000in}{0.000000in}}{\pgfqpoint{0.000000in}{0.020833in}}{%
\pgfpathmoveto{\pgfqpoint{0.000000in}{0.000000in}}%
\pgfpathlineto{\pgfqpoint{0.000000in}{0.020833in}}%
\pgfusepath{stroke,fill}%
}%
\begin{pgfscope}%
\pgfsys@transformshift{3.975049in}{0.422992in}%
\pgfsys@useobject{currentmarker}{}%
\end{pgfscope}%
\end{pgfscope}%
\begin{pgfscope}%
\pgfsetbuttcap%
\pgfsetroundjoin%
\definecolor{currentfill}{rgb}{0.000000,0.000000,0.000000}%
\pgfsetfillcolor{currentfill}%
\pgfsetlinewidth{0.501875pt}%
\definecolor{currentstroke}{rgb}{0.000000,0.000000,0.000000}%
\pgfsetstrokecolor{currentstroke}%
\pgfsetdash{}{0pt}%
\pgfsys@defobject{currentmarker}{\pgfqpoint{0.000000in}{-0.020833in}}{\pgfqpoint{0.000000in}{0.000000in}}{%
\pgfpathmoveto{\pgfqpoint{0.000000in}{0.000000in}}%
\pgfpathlineto{\pgfqpoint{0.000000in}{-0.020833in}}%
\pgfusepath{stroke,fill}%
}%
\begin{pgfscope}%
\pgfsys@transformshift{3.975049in}{4.374193in}%
\pgfsys@useobject{currentmarker}{}%
\end{pgfscope}%
\end{pgfscope}%
\begin{pgfscope}%
\pgfsetbuttcap%
\pgfsetroundjoin%
\definecolor{currentfill}{rgb}{0.000000,0.000000,0.000000}%
\pgfsetfillcolor{currentfill}%
\pgfsetlinewidth{0.501875pt}%
\definecolor{currentstroke}{rgb}{0.000000,0.000000,0.000000}%
\pgfsetstrokecolor{currentstroke}%
\pgfsetdash{}{0pt}%
\pgfsys@defobject{currentmarker}{\pgfqpoint{0.000000in}{0.000000in}}{\pgfqpoint{0.000000in}{0.020833in}}{%
\pgfpathmoveto{\pgfqpoint{0.000000in}{0.000000in}}%
\pgfpathlineto{\pgfqpoint{0.000000in}{0.020833in}}%
\pgfusepath{stroke,fill}%
}%
\begin{pgfscope}%
\pgfsys@transformshift{4.086013in}{0.422992in}%
\pgfsys@useobject{currentmarker}{}%
\end{pgfscope}%
\end{pgfscope}%
\begin{pgfscope}%
\pgfsetbuttcap%
\pgfsetroundjoin%
\definecolor{currentfill}{rgb}{0.000000,0.000000,0.000000}%
\pgfsetfillcolor{currentfill}%
\pgfsetlinewidth{0.501875pt}%
\definecolor{currentstroke}{rgb}{0.000000,0.000000,0.000000}%
\pgfsetstrokecolor{currentstroke}%
\pgfsetdash{}{0pt}%
\pgfsys@defobject{currentmarker}{\pgfqpoint{0.000000in}{-0.020833in}}{\pgfqpoint{0.000000in}{0.000000in}}{%
\pgfpathmoveto{\pgfqpoint{0.000000in}{0.000000in}}%
\pgfpathlineto{\pgfqpoint{0.000000in}{-0.020833in}}%
\pgfusepath{stroke,fill}%
}%
\begin{pgfscope}%
\pgfsys@transformshift{4.086013in}{4.374193in}%
\pgfsys@useobject{currentmarker}{}%
\end{pgfscope}%
\end{pgfscope}%
\begin{pgfscope}%
\pgfsetbuttcap%
\pgfsetroundjoin%
\definecolor{currentfill}{rgb}{0.000000,0.000000,0.000000}%
\pgfsetfillcolor{currentfill}%
\pgfsetlinewidth{0.501875pt}%
\definecolor{currentstroke}{rgb}{0.000000,0.000000,0.000000}%
\pgfsetstrokecolor{currentstroke}%
\pgfsetdash{}{0pt}%
\pgfsys@defobject{currentmarker}{\pgfqpoint{0.000000in}{0.000000in}}{\pgfqpoint{0.000000in}{0.020833in}}{%
\pgfpathmoveto{\pgfqpoint{0.000000in}{0.000000in}}%
\pgfpathlineto{\pgfqpoint{0.000000in}{0.020833in}}%
\pgfusepath{stroke,fill}%
}%
\begin{pgfscope}%
\pgfsys@transformshift{4.196977in}{0.422992in}%
\pgfsys@useobject{currentmarker}{}%
\end{pgfscope}%
\end{pgfscope}%
\begin{pgfscope}%
\pgfsetbuttcap%
\pgfsetroundjoin%
\definecolor{currentfill}{rgb}{0.000000,0.000000,0.000000}%
\pgfsetfillcolor{currentfill}%
\pgfsetlinewidth{0.501875pt}%
\definecolor{currentstroke}{rgb}{0.000000,0.000000,0.000000}%
\pgfsetstrokecolor{currentstroke}%
\pgfsetdash{}{0pt}%
\pgfsys@defobject{currentmarker}{\pgfqpoint{0.000000in}{-0.020833in}}{\pgfqpoint{0.000000in}{0.000000in}}{%
\pgfpathmoveto{\pgfqpoint{0.000000in}{0.000000in}}%
\pgfpathlineto{\pgfqpoint{0.000000in}{-0.020833in}}%
\pgfusepath{stroke,fill}%
}%
\begin{pgfscope}%
\pgfsys@transformshift{4.196977in}{4.374193in}%
\pgfsys@useobject{currentmarker}{}%
\end{pgfscope}%
\end{pgfscope}%
\begin{pgfscope}%
\pgfsetbuttcap%
\pgfsetroundjoin%
\definecolor{currentfill}{rgb}{0.000000,0.000000,0.000000}%
\pgfsetfillcolor{currentfill}%
\pgfsetlinewidth{0.501875pt}%
\definecolor{currentstroke}{rgb}{0.000000,0.000000,0.000000}%
\pgfsetstrokecolor{currentstroke}%
\pgfsetdash{}{0pt}%
\pgfsys@defobject{currentmarker}{\pgfqpoint{0.000000in}{0.000000in}}{\pgfqpoint{0.000000in}{0.020833in}}{%
\pgfpathmoveto{\pgfqpoint{0.000000in}{0.000000in}}%
\pgfpathlineto{\pgfqpoint{0.000000in}{0.020833in}}%
\pgfusepath{stroke,fill}%
}%
\begin{pgfscope}%
\pgfsys@transformshift{4.418905in}{0.422992in}%
\pgfsys@useobject{currentmarker}{}%
\end{pgfscope}%
\end{pgfscope}%
\begin{pgfscope}%
\pgfsetbuttcap%
\pgfsetroundjoin%
\definecolor{currentfill}{rgb}{0.000000,0.000000,0.000000}%
\pgfsetfillcolor{currentfill}%
\pgfsetlinewidth{0.501875pt}%
\definecolor{currentstroke}{rgb}{0.000000,0.000000,0.000000}%
\pgfsetstrokecolor{currentstroke}%
\pgfsetdash{}{0pt}%
\pgfsys@defobject{currentmarker}{\pgfqpoint{0.000000in}{-0.020833in}}{\pgfqpoint{0.000000in}{0.000000in}}{%
\pgfpathmoveto{\pgfqpoint{0.000000in}{0.000000in}}%
\pgfpathlineto{\pgfqpoint{0.000000in}{-0.020833in}}%
\pgfusepath{stroke,fill}%
}%
\begin{pgfscope}%
\pgfsys@transformshift{4.418905in}{4.374193in}%
\pgfsys@useobject{currentmarker}{}%
\end{pgfscope}%
\end{pgfscope}%
\begin{pgfscope}%
\pgfsetbuttcap%
\pgfsetroundjoin%
\definecolor{currentfill}{rgb}{0.000000,0.000000,0.000000}%
\pgfsetfillcolor{currentfill}%
\pgfsetlinewidth{0.501875pt}%
\definecolor{currentstroke}{rgb}{0.000000,0.000000,0.000000}%
\pgfsetstrokecolor{currentstroke}%
\pgfsetdash{}{0pt}%
\pgfsys@defobject{currentmarker}{\pgfqpoint{0.000000in}{0.000000in}}{\pgfqpoint{0.000000in}{0.020833in}}{%
\pgfpathmoveto{\pgfqpoint{0.000000in}{0.000000in}}%
\pgfpathlineto{\pgfqpoint{0.000000in}{0.020833in}}%
\pgfusepath{stroke,fill}%
}%
\begin{pgfscope}%
\pgfsys@transformshift{4.529869in}{0.422992in}%
\pgfsys@useobject{currentmarker}{}%
\end{pgfscope}%
\end{pgfscope}%
\begin{pgfscope}%
\pgfsetbuttcap%
\pgfsetroundjoin%
\definecolor{currentfill}{rgb}{0.000000,0.000000,0.000000}%
\pgfsetfillcolor{currentfill}%
\pgfsetlinewidth{0.501875pt}%
\definecolor{currentstroke}{rgb}{0.000000,0.000000,0.000000}%
\pgfsetstrokecolor{currentstroke}%
\pgfsetdash{}{0pt}%
\pgfsys@defobject{currentmarker}{\pgfqpoint{0.000000in}{-0.020833in}}{\pgfqpoint{0.000000in}{0.000000in}}{%
\pgfpathmoveto{\pgfqpoint{0.000000in}{0.000000in}}%
\pgfpathlineto{\pgfqpoint{0.000000in}{-0.020833in}}%
\pgfusepath{stroke,fill}%
}%
\begin{pgfscope}%
\pgfsys@transformshift{4.529869in}{4.374193in}%
\pgfsys@useobject{currentmarker}{}%
\end{pgfscope}%
\end{pgfscope}%
\begin{pgfscope}%
\pgfsetbuttcap%
\pgfsetroundjoin%
\definecolor{currentfill}{rgb}{0.000000,0.000000,0.000000}%
\pgfsetfillcolor{currentfill}%
\pgfsetlinewidth{0.501875pt}%
\definecolor{currentstroke}{rgb}{0.000000,0.000000,0.000000}%
\pgfsetstrokecolor{currentstroke}%
\pgfsetdash{}{0pt}%
\pgfsys@defobject{currentmarker}{\pgfqpoint{0.000000in}{0.000000in}}{\pgfqpoint{0.000000in}{0.020833in}}{%
\pgfpathmoveto{\pgfqpoint{0.000000in}{0.000000in}}%
\pgfpathlineto{\pgfqpoint{0.000000in}{0.020833in}}%
\pgfusepath{stroke,fill}%
}%
\begin{pgfscope}%
\pgfsys@transformshift{4.640833in}{0.422992in}%
\pgfsys@useobject{currentmarker}{}%
\end{pgfscope}%
\end{pgfscope}%
\begin{pgfscope}%
\pgfsetbuttcap%
\pgfsetroundjoin%
\definecolor{currentfill}{rgb}{0.000000,0.000000,0.000000}%
\pgfsetfillcolor{currentfill}%
\pgfsetlinewidth{0.501875pt}%
\definecolor{currentstroke}{rgb}{0.000000,0.000000,0.000000}%
\pgfsetstrokecolor{currentstroke}%
\pgfsetdash{}{0pt}%
\pgfsys@defobject{currentmarker}{\pgfqpoint{0.000000in}{-0.020833in}}{\pgfqpoint{0.000000in}{0.000000in}}{%
\pgfpathmoveto{\pgfqpoint{0.000000in}{0.000000in}}%
\pgfpathlineto{\pgfqpoint{0.000000in}{-0.020833in}}%
\pgfusepath{stroke,fill}%
}%
\begin{pgfscope}%
\pgfsys@transformshift{4.640833in}{4.374193in}%
\pgfsys@useobject{currentmarker}{}%
\end{pgfscope}%
\end{pgfscope}%
\begin{pgfscope}%
\pgfsetbuttcap%
\pgfsetroundjoin%
\definecolor{currentfill}{rgb}{0.000000,0.000000,0.000000}%
\pgfsetfillcolor{currentfill}%
\pgfsetlinewidth{0.501875pt}%
\definecolor{currentstroke}{rgb}{0.000000,0.000000,0.000000}%
\pgfsetstrokecolor{currentstroke}%
\pgfsetdash{}{0pt}%
\pgfsys@defobject{currentmarker}{\pgfqpoint{0.000000in}{0.000000in}}{\pgfqpoint{0.000000in}{0.020833in}}{%
\pgfpathmoveto{\pgfqpoint{0.000000in}{0.000000in}}%
\pgfpathlineto{\pgfqpoint{0.000000in}{0.020833in}}%
\pgfusepath{stroke,fill}%
}%
\begin{pgfscope}%
\pgfsys@transformshift{4.862761in}{0.422992in}%
\pgfsys@useobject{currentmarker}{}%
\end{pgfscope}%
\end{pgfscope}%
\begin{pgfscope}%
\pgfsetbuttcap%
\pgfsetroundjoin%
\definecolor{currentfill}{rgb}{0.000000,0.000000,0.000000}%
\pgfsetfillcolor{currentfill}%
\pgfsetlinewidth{0.501875pt}%
\definecolor{currentstroke}{rgb}{0.000000,0.000000,0.000000}%
\pgfsetstrokecolor{currentstroke}%
\pgfsetdash{}{0pt}%
\pgfsys@defobject{currentmarker}{\pgfqpoint{0.000000in}{-0.020833in}}{\pgfqpoint{0.000000in}{0.000000in}}{%
\pgfpathmoveto{\pgfqpoint{0.000000in}{0.000000in}}%
\pgfpathlineto{\pgfqpoint{0.000000in}{-0.020833in}}%
\pgfusepath{stroke,fill}%
}%
\begin{pgfscope}%
\pgfsys@transformshift{4.862761in}{4.374193in}%
\pgfsys@useobject{currentmarker}{}%
\end{pgfscope}%
\end{pgfscope}%
\begin{pgfscope}%
\pgfsetbuttcap%
\pgfsetroundjoin%
\definecolor{currentfill}{rgb}{0.000000,0.000000,0.000000}%
\pgfsetfillcolor{currentfill}%
\pgfsetlinewidth{0.501875pt}%
\definecolor{currentstroke}{rgb}{0.000000,0.000000,0.000000}%
\pgfsetstrokecolor{currentstroke}%
\pgfsetdash{}{0pt}%
\pgfsys@defobject{currentmarker}{\pgfqpoint{0.000000in}{0.000000in}}{\pgfqpoint{0.000000in}{0.020833in}}{%
\pgfpathmoveto{\pgfqpoint{0.000000in}{0.000000in}}%
\pgfpathlineto{\pgfqpoint{0.000000in}{0.020833in}}%
\pgfusepath{stroke,fill}%
}%
\begin{pgfscope}%
\pgfsys@transformshift{4.973724in}{0.422992in}%
\pgfsys@useobject{currentmarker}{}%
\end{pgfscope}%
\end{pgfscope}%
\begin{pgfscope}%
\pgfsetbuttcap%
\pgfsetroundjoin%
\definecolor{currentfill}{rgb}{0.000000,0.000000,0.000000}%
\pgfsetfillcolor{currentfill}%
\pgfsetlinewidth{0.501875pt}%
\definecolor{currentstroke}{rgb}{0.000000,0.000000,0.000000}%
\pgfsetstrokecolor{currentstroke}%
\pgfsetdash{}{0pt}%
\pgfsys@defobject{currentmarker}{\pgfqpoint{0.000000in}{-0.020833in}}{\pgfqpoint{0.000000in}{0.000000in}}{%
\pgfpathmoveto{\pgfqpoint{0.000000in}{0.000000in}}%
\pgfpathlineto{\pgfqpoint{0.000000in}{-0.020833in}}%
\pgfusepath{stroke,fill}%
}%
\begin{pgfscope}%
\pgfsys@transformshift{4.973724in}{4.374193in}%
\pgfsys@useobject{currentmarker}{}%
\end{pgfscope}%
\end{pgfscope}%
\begin{pgfscope}%
\pgfsetbuttcap%
\pgfsetroundjoin%
\definecolor{currentfill}{rgb}{0.000000,0.000000,0.000000}%
\pgfsetfillcolor{currentfill}%
\pgfsetlinewidth{0.501875pt}%
\definecolor{currentstroke}{rgb}{0.000000,0.000000,0.000000}%
\pgfsetstrokecolor{currentstroke}%
\pgfsetdash{}{0pt}%
\pgfsys@defobject{currentmarker}{\pgfqpoint{0.000000in}{0.000000in}}{\pgfqpoint{0.000000in}{0.020833in}}{%
\pgfpathmoveto{\pgfqpoint{0.000000in}{0.000000in}}%
\pgfpathlineto{\pgfqpoint{0.000000in}{0.020833in}}%
\pgfusepath{stroke,fill}%
}%
\begin{pgfscope}%
\pgfsys@transformshift{5.084688in}{0.422992in}%
\pgfsys@useobject{currentmarker}{}%
\end{pgfscope}%
\end{pgfscope}%
\begin{pgfscope}%
\pgfsetbuttcap%
\pgfsetroundjoin%
\definecolor{currentfill}{rgb}{0.000000,0.000000,0.000000}%
\pgfsetfillcolor{currentfill}%
\pgfsetlinewidth{0.501875pt}%
\definecolor{currentstroke}{rgb}{0.000000,0.000000,0.000000}%
\pgfsetstrokecolor{currentstroke}%
\pgfsetdash{}{0pt}%
\pgfsys@defobject{currentmarker}{\pgfqpoint{0.000000in}{-0.020833in}}{\pgfqpoint{0.000000in}{0.000000in}}{%
\pgfpathmoveto{\pgfqpoint{0.000000in}{0.000000in}}%
\pgfpathlineto{\pgfqpoint{0.000000in}{-0.020833in}}%
\pgfusepath{stroke,fill}%
}%
\begin{pgfscope}%
\pgfsys@transformshift{5.084688in}{4.374193in}%
\pgfsys@useobject{currentmarker}{}%
\end{pgfscope}%
\end{pgfscope}%
\begin{pgfscope}%
\pgfsetbuttcap%
\pgfsetroundjoin%
\definecolor{currentfill}{rgb}{0.000000,0.000000,0.000000}%
\pgfsetfillcolor{currentfill}%
\pgfsetlinewidth{0.501875pt}%
\definecolor{currentstroke}{rgb}{0.000000,0.000000,0.000000}%
\pgfsetstrokecolor{currentstroke}%
\pgfsetdash{}{0pt}%
\pgfsys@defobject{currentmarker}{\pgfqpoint{0.000000in}{0.000000in}}{\pgfqpoint{0.000000in}{0.020833in}}{%
\pgfpathmoveto{\pgfqpoint{0.000000in}{0.000000in}}%
\pgfpathlineto{\pgfqpoint{0.000000in}{0.020833in}}%
\pgfusepath{stroke,fill}%
}%
\begin{pgfscope}%
\pgfsys@transformshift{5.306616in}{0.422992in}%
\pgfsys@useobject{currentmarker}{}%
\end{pgfscope}%
\end{pgfscope}%
\begin{pgfscope}%
\pgfsetbuttcap%
\pgfsetroundjoin%
\definecolor{currentfill}{rgb}{0.000000,0.000000,0.000000}%
\pgfsetfillcolor{currentfill}%
\pgfsetlinewidth{0.501875pt}%
\definecolor{currentstroke}{rgb}{0.000000,0.000000,0.000000}%
\pgfsetstrokecolor{currentstroke}%
\pgfsetdash{}{0pt}%
\pgfsys@defobject{currentmarker}{\pgfqpoint{0.000000in}{-0.020833in}}{\pgfqpoint{0.000000in}{0.000000in}}{%
\pgfpathmoveto{\pgfqpoint{0.000000in}{0.000000in}}%
\pgfpathlineto{\pgfqpoint{0.000000in}{-0.020833in}}%
\pgfusepath{stroke,fill}%
}%
\begin{pgfscope}%
\pgfsys@transformshift{5.306616in}{4.374193in}%
\pgfsys@useobject{currentmarker}{}%
\end{pgfscope}%
\end{pgfscope}%
\begin{pgfscope}%
\pgfsetbuttcap%
\pgfsetroundjoin%
\definecolor{currentfill}{rgb}{0.000000,0.000000,0.000000}%
\pgfsetfillcolor{currentfill}%
\pgfsetlinewidth{0.501875pt}%
\definecolor{currentstroke}{rgb}{0.000000,0.000000,0.000000}%
\pgfsetstrokecolor{currentstroke}%
\pgfsetdash{}{0pt}%
\pgfsys@defobject{currentmarker}{\pgfqpoint{0.000000in}{0.000000in}}{\pgfqpoint{0.000000in}{0.020833in}}{%
\pgfpathmoveto{\pgfqpoint{0.000000in}{0.000000in}}%
\pgfpathlineto{\pgfqpoint{0.000000in}{0.020833in}}%
\pgfusepath{stroke,fill}%
}%
\begin{pgfscope}%
\pgfsys@transformshift{5.417580in}{0.422992in}%
\pgfsys@useobject{currentmarker}{}%
\end{pgfscope}%
\end{pgfscope}%
\begin{pgfscope}%
\pgfsetbuttcap%
\pgfsetroundjoin%
\definecolor{currentfill}{rgb}{0.000000,0.000000,0.000000}%
\pgfsetfillcolor{currentfill}%
\pgfsetlinewidth{0.501875pt}%
\definecolor{currentstroke}{rgb}{0.000000,0.000000,0.000000}%
\pgfsetstrokecolor{currentstroke}%
\pgfsetdash{}{0pt}%
\pgfsys@defobject{currentmarker}{\pgfqpoint{0.000000in}{-0.020833in}}{\pgfqpoint{0.000000in}{0.000000in}}{%
\pgfpathmoveto{\pgfqpoint{0.000000in}{0.000000in}}%
\pgfpathlineto{\pgfqpoint{0.000000in}{-0.020833in}}%
\pgfusepath{stroke,fill}%
}%
\begin{pgfscope}%
\pgfsys@transformshift{5.417580in}{4.374193in}%
\pgfsys@useobject{currentmarker}{}%
\end{pgfscope}%
\end{pgfscope}%
\begin{pgfscope}%
\pgfsetbuttcap%
\pgfsetroundjoin%
\definecolor{currentfill}{rgb}{0.000000,0.000000,0.000000}%
\pgfsetfillcolor{currentfill}%
\pgfsetlinewidth{0.501875pt}%
\definecolor{currentstroke}{rgb}{0.000000,0.000000,0.000000}%
\pgfsetstrokecolor{currentstroke}%
\pgfsetdash{}{0pt}%
\pgfsys@defobject{currentmarker}{\pgfqpoint{0.000000in}{0.000000in}}{\pgfqpoint{0.000000in}{0.020833in}}{%
\pgfpathmoveto{\pgfqpoint{0.000000in}{0.000000in}}%
\pgfpathlineto{\pgfqpoint{0.000000in}{0.020833in}}%
\pgfusepath{stroke,fill}%
}%
\begin{pgfscope}%
\pgfsys@transformshift{5.528544in}{0.422992in}%
\pgfsys@useobject{currentmarker}{}%
\end{pgfscope}%
\end{pgfscope}%
\begin{pgfscope}%
\pgfsetbuttcap%
\pgfsetroundjoin%
\definecolor{currentfill}{rgb}{0.000000,0.000000,0.000000}%
\pgfsetfillcolor{currentfill}%
\pgfsetlinewidth{0.501875pt}%
\definecolor{currentstroke}{rgb}{0.000000,0.000000,0.000000}%
\pgfsetstrokecolor{currentstroke}%
\pgfsetdash{}{0pt}%
\pgfsys@defobject{currentmarker}{\pgfqpoint{0.000000in}{-0.020833in}}{\pgfqpoint{0.000000in}{0.000000in}}{%
\pgfpathmoveto{\pgfqpoint{0.000000in}{0.000000in}}%
\pgfpathlineto{\pgfqpoint{0.000000in}{-0.020833in}}%
\pgfusepath{stroke,fill}%
}%
\begin{pgfscope}%
\pgfsys@transformshift{5.528544in}{4.374193in}%
\pgfsys@useobject{currentmarker}{}%
\end{pgfscope}%
\end{pgfscope}%
\begin{pgfscope}%
\pgfsetbuttcap%
\pgfsetroundjoin%
\definecolor{currentfill}{rgb}{0.000000,0.000000,0.000000}%
\pgfsetfillcolor{currentfill}%
\pgfsetlinewidth{0.501875pt}%
\definecolor{currentstroke}{rgb}{0.000000,0.000000,0.000000}%
\pgfsetstrokecolor{currentstroke}%
\pgfsetdash{}{0pt}%
\pgfsys@defobject{currentmarker}{\pgfqpoint{0.000000in}{0.000000in}}{\pgfqpoint{0.000000in}{0.020833in}}{%
\pgfpathmoveto{\pgfqpoint{0.000000in}{0.000000in}}%
\pgfpathlineto{\pgfqpoint{0.000000in}{0.020833in}}%
\pgfusepath{stroke,fill}%
}%
\begin{pgfscope}%
\pgfsys@transformshift{5.639508in}{0.422992in}%
\pgfsys@useobject{currentmarker}{}%
\end{pgfscope}%
\end{pgfscope}%
\begin{pgfscope}%
\pgfsetbuttcap%
\pgfsetroundjoin%
\definecolor{currentfill}{rgb}{0.000000,0.000000,0.000000}%
\pgfsetfillcolor{currentfill}%
\pgfsetlinewidth{0.501875pt}%
\definecolor{currentstroke}{rgb}{0.000000,0.000000,0.000000}%
\pgfsetstrokecolor{currentstroke}%
\pgfsetdash{}{0pt}%
\pgfsys@defobject{currentmarker}{\pgfqpoint{0.000000in}{-0.020833in}}{\pgfqpoint{0.000000in}{0.000000in}}{%
\pgfpathmoveto{\pgfqpoint{0.000000in}{0.000000in}}%
\pgfpathlineto{\pgfqpoint{0.000000in}{-0.020833in}}%
\pgfusepath{stroke,fill}%
}%
\begin{pgfscope}%
\pgfsys@transformshift{5.639508in}{4.374193in}%
\pgfsys@useobject{currentmarker}{}%
\end{pgfscope}%
\end{pgfscope}%
\begin{pgfscope}%
\definecolor{textcolor}{rgb}{0.000000,0.000000,0.000000}%
\pgfsetstrokecolor{textcolor}%
\pgfsetfillcolor{textcolor}%
\pgftext[x=4.540965in,y=0.184413in,,top]{\color{textcolor}\rmfamily\fontsize{10.000000}{12.000000}\selectfont \(\displaystyle K\)}%
\end{pgfscope}%
\begin{pgfscope}%
\pgfsetbuttcap%
\pgfsetroundjoin%
\definecolor{currentfill}{rgb}{0.000000,0.000000,0.000000}%
\pgfsetfillcolor{currentfill}%
\pgfsetlinewidth{0.501875pt}%
\definecolor{currentstroke}{rgb}{0.000000,0.000000,0.000000}%
\pgfsetstrokecolor{currentstroke}%
\pgfsetdash{}{0pt}%
\pgfsys@defobject{currentmarker}{\pgfqpoint{0.000000in}{0.000000in}}{\pgfqpoint{0.041667in}{0.000000in}}{%
\pgfpathmoveto{\pgfqpoint{0.000000in}{0.000000in}}%
\pgfpathlineto{\pgfqpoint{0.041667in}{0.000000in}}%
\pgfusepath{stroke,fill}%
}%
\begin{pgfscope}%
\pgfsys@transformshift{3.420229in}{0.530141in}%
\pgfsys@useobject{currentmarker}{}%
\end{pgfscope}%
\end{pgfscope}%
\begin{pgfscope}%
\pgfsetbuttcap%
\pgfsetroundjoin%
\definecolor{currentfill}{rgb}{0.000000,0.000000,0.000000}%
\pgfsetfillcolor{currentfill}%
\pgfsetlinewidth{0.501875pt}%
\definecolor{currentstroke}{rgb}{0.000000,0.000000,0.000000}%
\pgfsetstrokecolor{currentstroke}%
\pgfsetdash{}{0pt}%
\pgfsys@defobject{currentmarker}{\pgfqpoint{-0.041667in}{0.000000in}}{\pgfqpoint{-0.000000in}{0.000000in}}{%
\pgfpathmoveto{\pgfqpoint{-0.000000in}{0.000000in}}%
\pgfpathlineto{\pgfqpoint{-0.041667in}{0.000000in}}%
\pgfusepath{stroke,fill}%
}%
\begin{pgfscope}%
\pgfsys@transformshift{5.661701in}{0.530141in}%
\pgfsys@useobject{currentmarker}{}%
\end{pgfscope}%
\end{pgfscope}%
\begin{pgfscope}%
\definecolor{textcolor}{rgb}{0.000000,0.000000,0.000000}%
\pgfsetstrokecolor{textcolor}%
\pgfsetfillcolor{textcolor}%
\pgftext[x=3.194149in, y=0.477379in, left, base]{\color{textcolor}\rmfamily\fontsize{10.000000}{12.000000}\selectfont \(\displaystyle {0.3}\)}%
\end{pgfscope}%
\begin{pgfscope}%
\pgfsetbuttcap%
\pgfsetroundjoin%
\definecolor{currentfill}{rgb}{0.000000,0.000000,0.000000}%
\pgfsetfillcolor{currentfill}%
\pgfsetlinewidth{0.501875pt}%
\definecolor{currentstroke}{rgb}{0.000000,0.000000,0.000000}%
\pgfsetstrokecolor{currentstroke}%
\pgfsetdash{}{0pt}%
\pgfsys@defobject{currentmarker}{\pgfqpoint{0.000000in}{0.000000in}}{\pgfqpoint{0.041667in}{0.000000in}}{%
\pgfpathmoveto{\pgfqpoint{0.000000in}{0.000000in}}%
\pgfpathlineto{\pgfqpoint{0.041667in}{0.000000in}}%
\pgfusepath{stroke,fill}%
}%
\begin{pgfscope}%
\pgfsys@transformshift{3.420229in}{1.407109in}%
\pgfsys@useobject{currentmarker}{}%
\end{pgfscope}%
\end{pgfscope}%
\begin{pgfscope}%
\pgfsetbuttcap%
\pgfsetroundjoin%
\definecolor{currentfill}{rgb}{0.000000,0.000000,0.000000}%
\pgfsetfillcolor{currentfill}%
\pgfsetlinewidth{0.501875pt}%
\definecolor{currentstroke}{rgb}{0.000000,0.000000,0.000000}%
\pgfsetstrokecolor{currentstroke}%
\pgfsetdash{}{0pt}%
\pgfsys@defobject{currentmarker}{\pgfqpoint{-0.041667in}{0.000000in}}{\pgfqpoint{-0.000000in}{0.000000in}}{%
\pgfpathmoveto{\pgfqpoint{-0.000000in}{0.000000in}}%
\pgfpathlineto{\pgfqpoint{-0.041667in}{0.000000in}}%
\pgfusepath{stroke,fill}%
}%
\begin{pgfscope}%
\pgfsys@transformshift{5.661701in}{1.407109in}%
\pgfsys@useobject{currentmarker}{}%
\end{pgfscope}%
\end{pgfscope}%
\begin{pgfscope}%
\definecolor{textcolor}{rgb}{0.000000,0.000000,0.000000}%
\pgfsetstrokecolor{textcolor}%
\pgfsetfillcolor{textcolor}%
\pgftext[x=3.194149in, y=1.354347in, left, base]{\color{textcolor}\rmfamily\fontsize{10.000000}{12.000000}\selectfont \(\displaystyle {0.4}\)}%
\end{pgfscope}%
\begin{pgfscope}%
\pgfsetbuttcap%
\pgfsetroundjoin%
\definecolor{currentfill}{rgb}{0.000000,0.000000,0.000000}%
\pgfsetfillcolor{currentfill}%
\pgfsetlinewidth{0.501875pt}%
\definecolor{currentstroke}{rgb}{0.000000,0.000000,0.000000}%
\pgfsetstrokecolor{currentstroke}%
\pgfsetdash{}{0pt}%
\pgfsys@defobject{currentmarker}{\pgfqpoint{0.000000in}{0.000000in}}{\pgfqpoint{0.041667in}{0.000000in}}{%
\pgfpathmoveto{\pgfqpoint{0.000000in}{0.000000in}}%
\pgfpathlineto{\pgfqpoint{0.041667in}{0.000000in}}%
\pgfusepath{stroke,fill}%
}%
\begin{pgfscope}%
\pgfsys@transformshift{3.420229in}{2.284077in}%
\pgfsys@useobject{currentmarker}{}%
\end{pgfscope}%
\end{pgfscope}%
\begin{pgfscope}%
\pgfsetbuttcap%
\pgfsetroundjoin%
\definecolor{currentfill}{rgb}{0.000000,0.000000,0.000000}%
\pgfsetfillcolor{currentfill}%
\pgfsetlinewidth{0.501875pt}%
\definecolor{currentstroke}{rgb}{0.000000,0.000000,0.000000}%
\pgfsetstrokecolor{currentstroke}%
\pgfsetdash{}{0pt}%
\pgfsys@defobject{currentmarker}{\pgfqpoint{-0.041667in}{0.000000in}}{\pgfqpoint{-0.000000in}{0.000000in}}{%
\pgfpathmoveto{\pgfqpoint{-0.000000in}{0.000000in}}%
\pgfpathlineto{\pgfqpoint{-0.041667in}{0.000000in}}%
\pgfusepath{stroke,fill}%
}%
\begin{pgfscope}%
\pgfsys@transformshift{5.661701in}{2.284077in}%
\pgfsys@useobject{currentmarker}{}%
\end{pgfscope}%
\end{pgfscope}%
\begin{pgfscope}%
\definecolor{textcolor}{rgb}{0.000000,0.000000,0.000000}%
\pgfsetstrokecolor{textcolor}%
\pgfsetfillcolor{textcolor}%
\pgftext[x=3.194149in, y=2.231316in, left, base]{\color{textcolor}\rmfamily\fontsize{10.000000}{12.000000}\selectfont \(\displaystyle {0.5}\)}%
\end{pgfscope}%
\begin{pgfscope}%
\pgfsetbuttcap%
\pgfsetroundjoin%
\definecolor{currentfill}{rgb}{0.000000,0.000000,0.000000}%
\pgfsetfillcolor{currentfill}%
\pgfsetlinewidth{0.501875pt}%
\definecolor{currentstroke}{rgb}{0.000000,0.000000,0.000000}%
\pgfsetstrokecolor{currentstroke}%
\pgfsetdash{}{0pt}%
\pgfsys@defobject{currentmarker}{\pgfqpoint{0.000000in}{0.000000in}}{\pgfqpoint{0.041667in}{0.000000in}}{%
\pgfpathmoveto{\pgfqpoint{0.000000in}{0.000000in}}%
\pgfpathlineto{\pgfqpoint{0.041667in}{0.000000in}}%
\pgfusepath{stroke,fill}%
}%
\begin{pgfscope}%
\pgfsys@transformshift{3.420229in}{3.161046in}%
\pgfsys@useobject{currentmarker}{}%
\end{pgfscope}%
\end{pgfscope}%
\begin{pgfscope}%
\pgfsetbuttcap%
\pgfsetroundjoin%
\definecolor{currentfill}{rgb}{0.000000,0.000000,0.000000}%
\pgfsetfillcolor{currentfill}%
\pgfsetlinewidth{0.501875pt}%
\definecolor{currentstroke}{rgb}{0.000000,0.000000,0.000000}%
\pgfsetstrokecolor{currentstroke}%
\pgfsetdash{}{0pt}%
\pgfsys@defobject{currentmarker}{\pgfqpoint{-0.041667in}{0.000000in}}{\pgfqpoint{-0.000000in}{0.000000in}}{%
\pgfpathmoveto{\pgfqpoint{-0.000000in}{0.000000in}}%
\pgfpathlineto{\pgfqpoint{-0.041667in}{0.000000in}}%
\pgfusepath{stroke,fill}%
}%
\begin{pgfscope}%
\pgfsys@transformshift{5.661701in}{3.161046in}%
\pgfsys@useobject{currentmarker}{}%
\end{pgfscope}%
\end{pgfscope}%
\begin{pgfscope}%
\definecolor{textcolor}{rgb}{0.000000,0.000000,0.000000}%
\pgfsetstrokecolor{textcolor}%
\pgfsetfillcolor{textcolor}%
\pgftext[x=3.194149in, y=3.108284in, left, base]{\color{textcolor}\rmfamily\fontsize{10.000000}{12.000000}\selectfont \(\displaystyle {0.6}\)}%
\end{pgfscope}%
\begin{pgfscope}%
\pgfsetbuttcap%
\pgfsetroundjoin%
\definecolor{currentfill}{rgb}{0.000000,0.000000,0.000000}%
\pgfsetfillcolor{currentfill}%
\pgfsetlinewidth{0.501875pt}%
\definecolor{currentstroke}{rgb}{0.000000,0.000000,0.000000}%
\pgfsetstrokecolor{currentstroke}%
\pgfsetdash{}{0pt}%
\pgfsys@defobject{currentmarker}{\pgfqpoint{0.000000in}{0.000000in}}{\pgfqpoint{0.041667in}{0.000000in}}{%
\pgfpathmoveto{\pgfqpoint{0.000000in}{0.000000in}}%
\pgfpathlineto{\pgfqpoint{0.041667in}{0.000000in}}%
\pgfusepath{stroke,fill}%
}%
\begin{pgfscope}%
\pgfsys@transformshift{3.420229in}{4.038014in}%
\pgfsys@useobject{currentmarker}{}%
\end{pgfscope}%
\end{pgfscope}%
\begin{pgfscope}%
\pgfsetbuttcap%
\pgfsetroundjoin%
\definecolor{currentfill}{rgb}{0.000000,0.000000,0.000000}%
\pgfsetfillcolor{currentfill}%
\pgfsetlinewidth{0.501875pt}%
\definecolor{currentstroke}{rgb}{0.000000,0.000000,0.000000}%
\pgfsetstrokecolor{currentstroke}%
\pgfsetdash{}{0pt}%
\pgfsys@defobject{currentmarker}{\pgfqpoint{-0.041667in}{0.000000in}}{\pgfqpoint{-0.000000in}{0.000000in}}{%
\pgfpathmoveto{\pgfqpoint{-0.000000in}{0.000000in}}%
\pgfpathlineto{\pgfqpoint{-0.041667in}{0.000000in}}%
\pgfusepath{stroke,fill}%
}%
\begin{pgfscope}%
\pgfsys@transformshift{5.661701in}{4.038014in}%
\pgfsys@useobject{currentmarker}{}%
\end{pgfscope}%
\end{pgfscope}%
\begin{pgfscope}%
\definecolor{textcolor}{rgb}{0.000000,0.000000,0.000000}%
\pgfsetstrokecolor{textcolor}%
\pgfsetfillcolor{textcolor}%
\pgftext[x=3.194149in, y=3.985253in, left, base]{\color{textcolor}\rmfamily\fontsize{10.000000}{12.000000}\selectfont \(\displaystyle {0.7}\)}%
\end{pgfscope}%
\begin{pgfscope}%
\pgfsetbuttcap%
\pgfsetroundjoin%
\definecolor{currentfill}{rgb}{0.000000,0.000000,0.000000}%
\pgfsetfillcolor{currentfill}%
\pgfsetlinewidth{0.501875pt}%
\definecolor{currentstroke}{rgb}{0.000000,0.000000,0.000000}%
\pgfsetstrokecolor{currentstroke}%
\pgfsetdash{}{0pt}%
\pgfsys@defobject{currentmarker}{\pgfqpoint{0.000000in}{0.000000in}}{\pgfqpoint{0.020833in}{0.000000in}}{%
\pgfpathmoveto{\pgfqpoint{0.000000in}{0.000000in}}%
\pgfpathlineto{\pgfqpoint{0.020833in}{0.000000in}}%
\pgfusepath{stroke,fill}%
}%
\begin{pgfscope}%
\pgfsys@transformshift{3.420229in}{0.705534in}%
\pgfsys@useobject{currentmarker}{}%
\end{pgfscope}%
\end{pgfscope}%
\begin{pgfscope}%
\pgfsetbuttcap%
\pgfsetroundjoin%
\definecolor{currentfill}{rgb}{0.000000,0.000000,0.000000}%
\pgfsetfillcolor{currentfill}%
\pgfsetlinewidth{0.501875pt}%
\definecolor{currentstroke}{rgb}{0.000000,0.000000,0.000000}%
\pgfsetstrokecolor{currentstroke}%
\pgfsetdash{}{0pt}%
\pgfsys@defobject{currentmarker}{\pgfqpoint{-0.020833in}{0.000000in}}{\pgfqpoint{-0.000000in}{0.000000in}}{%
\pgfpathmoveto{\pgfqpoint{-0.000000in}{0.000000in}}%
\pgfpathlineto{\pgfqpoint{-0.020833in}{0.000000in}}%
\pgfusepath{stroke,fill}%
}%
\begin{pgfscope}%
\pgfsys@transformshift{5.661701in}{0.705534in}%
\pgfsys@useobject{currentmarker}{}%
\end{pgfscope}%
\end{pgfscope}%
\begin{pgfscope}%
\pgfsetbuttcap%
\pgfsetroundjoin%
\definecolor{currentfill}{rgb}{0.000000,0.000000,0.000000}%
\pgfsetfillcolor{currentfill}%
\pgfsetlinewidth{0.501875pt}%
\definecolor{currentstroke}{rgb}{0.000000,0.000000,0.000000}%
\pgfsetstrokecolor{currentstroke}%
\pgfsetdash{}{0pt}%
\pgfsys@defobject{currentmarker}{\pgfqpoint{0.000000in}{0.000000in}}{\pgfqpoint{0.020833in}{0.000000in}}{%
\pgfpathmoveto{\pgfqpoint{0.000000in}{0.000000in}}%
\pgfpathlineto{\pgfqpoint{0.020833in}{0.000000in}}%
\pgfusepath{stroke,fill}%
}%
\begin{pgfscope}%
\pgfsys@transformshift{3.420229in}{0.880928in}%
\pgfsys@useobject{currentmarker}{}%
\end{pgfscope}%
\end{pgfscope}%
\begin{pgfscope}%
\pgfsetbuttcap%
\pgfsetroundjoin%
\definecolor{currentfill}{rgb}{0.000000,0.000000,0.000000}%
\pgfsetfillcolor{currentfill}%
\pgfsetlinewidth{0.501875pt}%
\definecolor{currentstroke}{rgb}{0.000000,0.000000,0.000000}%
\pgfsetstrokecolor{currentstroke}%
\pgfsetdash{}{0pt}%
\pgfsys@defobject{currentmarker}{\pgfqpoint{-0.020833in}{0.000000in}}{\pgfqpoint{-0.000000in}{0.000000in}}{%
\pgfpathmoveto{\pgfqpoint{-0.000000in}{0.000000in}}%
\pgfpathlineto{\pgfqpoint{-0.020833in}{0.000000in}}%
\pgfusepath{stroke,fill}%
}%
\begin{pgfscope}%
\pgfsys@transformshift{5.661701in}{0.880928in}%
\pgfsys@useobject{currentmarker}{}%
\end{pgfscope}%
\end{pgfscope}%
\begin{pgfscope}%
\pgfsetbuttcap%
\pgfsetroundjoin%
\definecolor{currentfill}{rgb}{0.000000,0.000000,0.000000}%
\pgfsetfillcolor{currentfill}%
\pgfsetlinewidth{0.501875pt}%
\definecolor{currentstroke}{rgb}{0.000000,0.000000,0.000000}%
\pgfsetstrokecolor{currentstroke}%
\pgfsetdash{}{0pt}%
\pgfsys@defobject{currentmarker}{\pgfqpoint{0.000000in}{0.000000in}}{\pgfqpoint{0.020833in}{0.000000in}}{%
\pgfpathmoveto{\pgfqpoint{0.000000in}{0.000000in}}%
\pgfpathlineto{\pgfqpoint{0.020833in}{0.000000in}}%
\pgfusepath{stroke,fill}%
}%
\begin{pgfscope}%
\pgfsys@transformshift{3.420229in}{1.056322in}%
\pgfsys@useobject{currentmarker}{}%
\end{pgfscope}%
\end{pgfscope}%
\begin{pgfscope}%
\pgfsetbuttcap%
\pgfsetroundjoin%
\definecolor{currentfill}{rgb}{0.000000,0.000000,0.000000}%
\pgfsetfillcolor{currentfill}%
\pgfsetlinewidth{0.501875pt}%
\definecolor{currentstroke}{rgb}{0.000000,0.000000,0.000000}%
\pgfsetstrokecolor{currentstroke}%
\pgfsetdash{}{0pt}%
\pgfsys@defobject{currentmarker}{\pgfqpoint{-0.020833in}{0.000000in}}{\pgfqpoint{-0.000000in}{0.000000in}}{%
\pgfpathmoveto{\pgfqpoint{-0.000000in}{0.000000in}}%
\pgfpathlineto{\pgfqpoint{-0.020833in}{0.000000in}}%
\pgfusepath{stroke,fill}%
}%
\begin{pgfscope}%
\pgfsys@transformshift{5.661701in}{1.056322in}%
\pgfsys@useobject{currentmarker}{}%
\end{pgfscope}%
\end{pgfscope}%
\begin{pgfscope}%
\pgfsetbuttcap%
\pgfsetroundjoin%
\definecolor{currentfill}{rgb}{0.000000,0.000000,0.000000}%
\pgfsetfillcolor{currentfill}%
\pgfsetlinewidth{0.501875pt}%
\definecolor{currentstroke}{rgb}{0.000000,0.000000,0.000000}%
\pgfsetstrokecolor{currentstroke}%
\pgfsetdash{}{0pt}%
\pgfsys@defobject{currentmarker}{\pgfqpoint{0.000000in}{0.000000in}}{\pgfqpoint{0.020833in}{0.000000in}}{%
\pgfpathmoveto{\pgfqpoint{0.000000in}{0.000000in}}%
\pgfpathlineto{\pgfqpoint{0.020833in}{0.000000in}}%
\pgfusepath{stroke,fill}%
}%
\begin{pgfscope}%
\pgfsys@transformshift{3.420229in}{1.231715in}%
\pgfsys@useobject{currentmarker}{}%
\end{pgfscope}%
\end{pgfscope}%
\begin{pgfscope}%
\pgfsetbuttcap%
\pgfsetroundjoin%
\definecolor{currentfill}{rgb}{0.000000,0.000000,0.000000}%
\pgfsetfillcolor{currentfill}%
\pgfsetlinewidth{0.501875pt}%
\definecolor{currentstroke}{rgb}{0.000000,0.000000,0.000000}%
\pgfsetstrokecolor{currentstroke}%
\pgfsetdash{}{0pt}%
\pgfsys@defobject{currentmarker}{\pgfqpoint{-0.020833in}{0.000000in}}{\pgfqpoint{-0.000000in}{0.000000in}}{%
\pgfpathmoveto{\pgfqpoint{-0.000000in}{0.000000in}}%
\pgfpathlineto{\pgfqpoint{-0.020833in}{0.000000in}}%
\pgfusepath{stroke,fill}%
}%
\begin{pgfscope}%
\pgfsys@transformshift{5.661701in}{1.231715in}%
\pgfsys@useobject{currentmarker}{}%
\end{pgfscope}%
\end{pgfscope}%
\begin{pgfscope}%
\pgfsetbuttcap%
\pgfsetroundjoin%
\definecolor{currentfill}{rgb}{0.000000,0.000000,0.000000}%
\pgfsetfillcolor{currentfill}%
\pgfsetlinewidth{0.501875pt}%
\definecolor{currentstroke}{rgb}{0.000000,0.000000,0.000000}%
\pgfsetstrokecolor{currentstroke}%
\pgfsetdash{}{0pt}%
\pgfsys@defobject{currentmarker}{\pgfqpoint{0.000000in}{0.000000in}}{\pgfqpoint{0.020833in}{0.000000in}}{%
\pgfpathmoveto{\pgfqpoint{0.000000in}{0.000000in}}%
\pgfpathlineto{\pgfqpoint{0.020833in}{0.000000in}}%
\pgfusepath{stroke,fill}%
}%
\begin{pgfscope}%
\pgfsys@transformshift{3.420229in}{1.582503in}%
\pgfsys@useobject{currentmarker}{}%
\end{pgfscope}%
\end{pgfscope}%
\begin{pgfscope}%
\pgfsetbuttcap%
\pgfsetroundjoin%
\definecolor{currentfill}{rgb}{0.000000,0.000000,0.000000}%
\pgfsetfillcolor{currentfill}%
\pgfsetlinewidth{0.501875pt}%
\definecolor{currentstroke}{rgb}{0.000000,0.000000,0.000000}%
\pgfsetstrokecolor{currentstroke}%
\pgfsetdash{}{0pt}%
\pgfsys@defobject{currentmarker}{\pgfqpoint{-0.020833in}{0.000000in}}{\pgfqpoint{-0.000000in}{0.000000in}}{%
\pgfpathmoveto{\pgfqpoint{-0.000000in}{0.000000in}}%
\pgfpathlineto{\pgfqpoint{-0.020833in}{0.000000in}}%
\pgfusepath{stroke,fill}%
}%
\begin{pgfscope}%
\pgfsys@transformshift{5.661701in}{1.582503in}%
\pgfsys@useobject{currentmarker}{}%
\end{pgfscope}%
\end{pgfscope}%
\begin{pgfscope}%
\pgfsetbuttcap%
\pgfsetroundjoin%
\definecolor{currentfill}{rgb}{0.000000,0.000000,0.000000}%
\pgfsetfillcolor{currentfill}%
\pgfsetlinewidth{0.501875pt}%
\definecolor{currentstroke}{rgb}{0.000000,0.000000,0.000000}%
\pgfsetstrokecolor{currentstroke}%
\pgfsetdash{}{0pt}%
\pgfsys@defobject{currentmarker}{\pgfqpoint{0.000000in}{0.000000in}}{\pgfqpoint{0.020833in}{0.000000in}}{%
\pgfpathmoveto{\pgfqpoint{0.000000in}{0.000000in}}%
\pgfpathlineto{\pgfqpoint{0.020833in}{0.000000in}}%
\pgfusepath{stroke,fill}%
}%
\begin{pgfscope}%
\pgfsys@transformshift{3.420229in}{1.757896in}%
\pgfsys@useobject{currentmarker}{}%
\end{pgfscope}%
\end{pgfscope}%
\begin{pgfscope}%
\pgfsetbuttcap%
\pgfsetroundjoin%
\definecolor{currentfill}{rgb}{0.000000,0.000000,0.000000}%
\pgfsetfillcolor{currentfill}%
\pgfsetlinewidth{0.501875pt}%
\definecolor{currentstroke}{rgb}{0.000000,0.000000,0.000000}%
\pgfsetstrokecolor{currentstroke}%
\pgfsetdash{}{0pt}%
\pgfsys@defobject{currentmarker}{\pgfqpoint{-0.020833in}{0.000000in}}{\pgfqpoint{-0.000000in}{0.000000in}}{%
\pgfpathmoveto{\pgfqpoint{-0.000000in}{0.000000in}}%
\pgfpathlineto{\pgfqpoint{-0.020833in}{0.000000in}}%
\pgfusepath{stroke,fill}%
}%
\begin{pgfscope}%
\pgfsys@transformshift{5.661701in}{1.757896in}%
\pgfsys@useobject{currentmarker}{}%
\end{pgfscope}%
\end{pgfscope}%
\begin{pgfscope}%
\pgfsetbuttcap%
\pgfsetroundjoin%
\definecolor{currentfill}{rgb}{0.000000,0.000000,0.000000}%
\pgfsetfillcolor{currentfill}%
\pgfsetlinewidth{0.501875pt}%
\definecolor{currentstroke}{rgb}{0.000000,0.000000,0.000000}%
\pgfsetstrokecolor{currentstroke}%
\pgfsetdash{}{0pt}%
\pgfsys@defobject{currentmarker}{\pgfqpoint{0.000000in}{0.000000in}}{\pgfqpoint{0.020833in}{0.000000in}}{%
\pgfpathmoveto{\pgfqpoint{0.000000in}{0.000000in}}%
\pgfpathlineto{\pgfqpoint{0.020833in}{0.000000in}}%
\pgfusepath{stroke,fill}%
}%
\begin{pgfscope}%
\pgfsys@transformshift{3.420229in}{1.933290in}%
\pgfsys@useobject{currentmarker}{}%
\end{pgfscope}%
\end{pgfscope}%
\begin{pgfscope}%
\pgfsetbuttcap%
\pgfsetroundjoin%
\definecolor{currentfill}{rgb}{0.000000,0.000000,0.000000}%
\pgfsetfillcolor{currentfill}%
\pgfsetlinewidth{0.501875pt}%
\definecolor{currentstroke}{rgb}{0.000000,0.000000,0.000000}%
\pgfsetstrokecolor{currentstroke}%
\pgfsetdash{}{0pt}%
\pgfsys@defobject{currentmarker}{\pgfqpoint{-0.020833in}{0.000000in}}{\pgfqpoint{-0.000000in}{0.000000in}}{%
\pgfpathmoveto{\pgfqpoint{-0.000000in}{0.000000in}}%
\pgfpathlineto{\pgfqpoint{-0.020833in}{0.000000in}}%
\pgfusepath{stroke,fill}%
}%
\begin{pgfscope}%
\pgfsys@transformshift{5.661701in}{1.933290in}%
\pgfsys@useobject{currentmarker}{}%
\end{pgfscope}%
\end{pgfscope}%
\begin{pgfscope}%
\pgfsetbuttcap%
\pgfsetroundjoin%
\definecolor{currentfill}{rgb}{0.000000,0.000000,0.000000}%
\pgfsetfillcolor{currentfill}%
\pgfsetlinewidth{0.501875pt}%
\definecolor{currentstroke}{rgb}{0.000000,0.000000,0.000000}%
\pgfsetstrokecolor{currentstroke}%
\pgfsetdash{}{0pt}%
\pgfsys@defobject{currentmarker}{\pgfqpoint{0.000000in}{0.000000in}}{\pgfqpoint{0.020833in}{0.000000in}}{%
\pgfpathmoveto{\pgfqpoint{0.000000in}{0.000000in}}%
\pgfpathlineto{\pgfqpoint{0.020833in}{0.000000in}}%
\pgfusepath{stroke,fill}%
}%
\begin{pgfscope}%
\pgfsys@transformshift{3.420229in}{2.108684in}%
\pgfsys@useobject{currentmarker}{}%
\end{pgfscope}%
\end{pgfscope}%
\begin{pgfscope}%
\pgfsetbuttcap%
\pgfsetroundjoin%
\definecolor{currentfill}{rgb}{0.000000,0.000000,0.000000}%
\pgfsetfillcolor{currentfill}%
\pgfsetlinewidth{0.501875pt}%
\definecolor{currentstroke}{rgb}{0.000000,0.000000,0.000000}%
\pgfsetstrokecolor{currentstroke}%
\pgfsetdash{}{0pt}%
\pgfsys@defobject{currentmarker}{\pgfqpoint{-0.020833in}{0.000000in}}{\pgfqpoint{-0.000000in}{0.000000in}}{%
\pgfpathmoveto{\pgfqpoint{-0.000000in}{0.000000in}}%
\pgfpathlineto{\pgfqpoint{-0.020833in}{0.000000in}}%
\pgfusepath{stroke,fill}%
}%
\begin{pgfscope}%
\pgfsys@transformshift{5.661701in}{2.108684in}%
\pgfsys@useobject{currentmarker}{}%
\end{pgfscope}%
\end{pgfscope}%
\begin{pgfscope}%
\pgfsetbuttcap%
\pgfsetroundjoin%
\definecolor{currentfill}{rgb}{0.000000,0.000000,0.000000}%
\pgfsetfillcolor{currentfill}%
\pgfsetlinewidth{0.501875pt}%
\definecolor{currentstroke}{rgb}{0.000000,0.000000,0.000000}%
\pgfsetstrokecolor{currentstroke}%
\pgfsetdash{}{0pt}%
\pgfsys@defobject{currentmarker}{\pgfqpoint{0.000000in}{0.000000in}}{\pgfqpoint{0.020833in}{0.000000in}}{%
\pgfpathmoveto{\pgfqpoint{0.000000in}{0.000000in}}%
\pgfpathlineto{\pgfqpoint{0.020833in}{0.000000in}}%
\pgfusepath{stroke,fill}%
}%
\begin{pgfscope}%
\pgfsys@transformshift{3.420229in}{2.459471in}%
\pgfsys@useobject{currentmarker}{}%
\end{pgfscope}%
\end{pgfscope}%
\begin{pgfscope}%
\pgfsetbuttcap%
\pgfsetroundjoin%
\definecolor{currentfill}{rgb}{0.000000,0.000000,0.000000}%
\pgfsetfillcolor{currentfill}%
\pgfsetlinewidth{0.501875pt}%
\definecolor{currentstroke}{rgb}{0.000000,0.000000,0.000000}%
\pgfsetstrokecolor{currentstroke}%
\pgfsetdash{}{0pt}%
\pgfsys@defobject{currentmarker}{\pgfqpoint{-0.020833in}{0.000000in}}{\pgfqpoint{-0.000000in}{0.000000in}}{%
\pgfpathmoveto{\pgfqpoint{-0.000000in}{0.000000in}}%
\pgfpathlineto{\pgfqpoint{-0.020833in}{0.000000in}}%
\pgfusepath{stroke,fill}%
}%
\begin{pgfscope}%
\pgfsys@transformshift{5.661701in}{2.459471in}%
\pgfsys@useobject{currentmarker}{}%
\end{pgfscope}%
\end{pgfscope}%
\begin{pgfscope}%
\pgfsetbuttcap%
\pgfsetroundjoin%
\definecolor{currentfill}{rgb}{0.000000,0.000000,0.000000}%
\pgfsetfillcolor{currentfill}%
\pgfsetlinewidth{0.501875pt}%
\definecolor{currentstroke}{rgb}{0.000000,0.000000,0.000000}%
\pgfsetstrokecolor{currentstroke}%
\pgfsetdash{}{0pt}%
\pgfsys@defobject{currentmarker}{\pgfqpoint{0.000000in}{0.000000in}}{\pgfqpoint{0.020833in}{0.000000in}}{%
\pgfpathmoveto{\pgfqpoint{0.000000in}{0.000000in}}%
\pgfpathlineto{\pgfqpoint{0.020833in}{0.000000in}}%
\pgfusepath{stroke,fill}%
}%
\begin{pgfscope}%
\pgfsys@transformshift{3.420229in}{2.634865in}%
\pgfsys@useobject{currentmarker}{}%
\end{pgfscope}%
\end{pgfscope}%
\begin{pgfscope}%
\pgfsetbuttcap%
\pgfsetroundjoin%
\definecolor{currentfill}{rgb}{0.000000,0.000000,0.000000}%
\pgfsetfillcolor{currentfill}%
\pgfsetlinewidth{0.501875pt}%
\definecolor{currentstroke}{rgb}{0.000000,0.000000,0.000000}%
\pgfsetstrokecolor{currentstroke}%
\pgfsetdash{}{0pt}%
\pgfsys@defobject{currentmarker}{\pgfqpoint{-0.020833in}{0.000000in}}{\pgfqpoint{-0.000000in}{0.000000in}}{%
\pgfpathmoveto{\pgfqpoint{-0.000000in}{0.000000in}}%
\pgfpathlineto{\pgfqpoint{-0.020833in}{0.000000in}}%
\pgfusepath{stroke,fill}%
}%
\begin{pgfscope}%
\pgfsys@transformshift{5.661701in}{2.634865in}%
\pgfsys@useobject{currentmarker}{}%
\end{pgfscope}%
\end{pgfscope}%
\begin{pgfscope}%
\pgfsetbuttcap%
\pgfsetroundjoin%
\definecolor{currentfill}{rgb}{0.000000,0.000000,0.000000}%
\pgfsetfillcolor{currentfill}%
\pgfsetlinewidth{0.501875pt}%
\definecolor{currentstroke}{rgb}{0.000000,0.000000,0.000000}%
\pgfsetstrokecolor{currentstroke}%
\pgfsetdash{}{0pt}%
\pgfsys@defobject{currentmarker}{\pgfqpoint{0.000000in}{0.000000in}}{\pgfqpoint{0.020833in}{0.000000in}}{%
\pgfpathmoveto{\pgfqpoint{0.000000in}{0.000000in}}%
\pgfpathlineto{\pgfqpoint{0.020833in}{0.000000in}}%
\pgfusepath{stroke,fill}%
}%
\begin{pgfscope}%
\pgfsys@transformshift{3.420229in}{2.810258in}%
\pgfsys@useobject{currentmarker}{}%
\end{pgfscope}%
\end{pgfscope}%
\begin{pgfscope}%
\pgfsetbuttcap%
\pgfsetroundjoin%
\definecolor{currentfill}{rgb}{0.000000,0.000000,0.000000}%
\pgfsetfillcolor{currentfill}%
\pgfsetlinewidth{0.501875pt}%
\definecolor{currentstroke}{rgb}{0.000000,0.000000,0.000000}%
\pgfsetstrokecolor{currentstroke}%
\pgfsetdash{}{0pt}%
\pgfsys@defobject{currentmarker}{\pgfqpoint{-0.020833in}{0.000000in}}{\pgfqpoint{-0.000000in}{0.000000in}}{%
\pgfpathmoveto{\pgfqpoint{-0.000000in}{0.000000in}}%
\pgfpathlineto{\pgfqpoint{-0.020833in}{0.000000in}}%
\pgfusepath{stroke,fill}%
}%
\begin{pgfscope}%
\pgfsys@transformshift{5.661701in}{2.810258in}%
\pgfsys@useobject{currentmarker}{}%
\end{pgfscope}%
\end{pgfscope}%
\begin{pgfscope}%
\pgfsetbuttcap%
\pgfsetroundjoin%
\definecolor{currentfill}{rgb}{0.000000,0.000000,0.000000}%
\pgfsetfillcolor{currentfill}%
\pgfsetlinewidth{0.501875pt}%
\definecolor{currentstroke}{rgb}{0.000000,0.000000,0.000000}%
\pgfsetstrokecolor{currentstroke}%
\pgfsetdash{}{0pt}%
\pgfsys@defobject{currentmarker}{\pgfqpoint{0.000000in}{0.000000in}}{\pgfqpoint{0.020833in}{0.000000in}}{%
\pgfpathmoveto{\pgfqpoint{0.000000in}{0.000000in}}%
\pgfpathlineto{\pgfqpoint{0.020833in}{0.000000in}}%
\pgfusepath{stroke,fill}%
}%
\begin{pgfscope}%
\pgfsys@transformshift{3.420229in}{2.985652in}%
\pgfsys@useobject{currentmarker}{}%
\end{pgfscope}%
\end{pgfscope}%
\begin{pgfscope}%
\pgfsetbuttcap%
\pgfsetroundjoin%
\definecolor{currentfill}{rgb}{0.000000,0.000000,0.000000}%
\pgfsetfillcolor{currentfill}%
\pgfsetlinewidth{0.501875pt}%
\definecolor{currentstroke}{rgb}{0.000000,0.000000,0.000000}%
\pgfsetstrokecolor{currentstroke}%
\pgfsetdash{}{0pt}%
\pgfsys@defobject{currentmarker}{\pgfqpoint{-0.020833in}{0.000000in}}{\pgfqpoint{-0.000000in}{0.000000in}}{%
\pgfpathmoveto{\pgfqpoint{-0.000000in}{0.000000in}}%
\pgfpathlineto{\pgfqpoint{-0.020833in}{0.000000in}}%
\pgfusepath{stroke,fill}%
}%
\begin{pgfscope}%
\pgfsys@transformshift{5.661701in}{2.985652in}%
\pgfsys@useobject{currentmarker}{}%
\end{pgfscope}%
\end{pgfscope}%
\begin{pgfscope}%
\pgfsetbuttcap%
\pgfsetroundjoin%
\definecolor{currentfill}{rgb}{0.000000,0.000000,0.000000}%
\pgfsetfillcolor{currentfill}%
\pgfsetlinewidth{0.501875pt}%
\definecolor{currentstroke}{rgb}{0.000000,0.000000,0.000000}%
\pgfsetstrokecolor{currentstroke}%
\pgfsetdash{}{0pt}%
\pgfsys@defobject{currentmarker}{\pgfqpoint{0.000000in}{0.000000in}}{\pgfqpoint{0.020833in}{0.000000in}}{%
\pgfpathmoveto{\pgfqpoint{0.000000in}{0.000000in}}%
\pgfpathlineto{\pgfqpoint{0.020833in}{0.000000in}}%
\pgfusepath{stroke,fill}%
}%
\begin{pgfscope}%
\pgfsys@transformshift{3.420229in}{3.336439in}%
\pgfsys@useobject{currentmarker}{}%
\end{pgfscope}%
\end{pgfscope}%
\begin{pgfscope}%
\pgfsetbuttcap%
\pgfsetroundjoin%
\definecolor{currentfill}{rgb}{0.000000,0.000000,0.000000}%
\pgfsetfillcolor{currentfill}%
\pgfsetlinewidth{0.501875pt}%
\definecolor{currentstroke}{rgb}{0.000000,0.000000,0.000000}%
\pgfsetstrokecolor{currentstroke}%
\pgfsetdash{}{0pt}%
\pgfsys@defobject{currentmarker}{\pgfqpoint{-0.020833in}{0.000000in}}{\pgfqpoint{-0.000000in}{0.000000in}}{%
\pgfpathmoveto{\pgfqpoint{-0.000000in}{0.000000in}}%
\pgfpathlineto{\pgfqpoint{-0.020833in}{0.000000in}}%
\pgfusepath{stroke,fill}%
}%
\begin{pgfscope}%
\pgfsys@transformshift{5.661701in}{3.336439in}%
\pgfsys@useobject{currentmarker}{}%
\end{pgfscope}%
\end{pgfscope}%
\begin{pgfscope}%
\pgfsetbuttcap%
\pgfsetroundjoin%
\definecolor{currentfill}{rgb}{0.000000,0.000000,0.000000}%
\pgfsetfillcolor{currentfill}%
\pgfsetlinewidth{0.501875pt}%
\definecolor{currentstroke}{rgb}{0.000000,0.000000,0.000000}%
\pgfsetstrokecolor{currentstroke}%
\pgfsetdash{}{0pt}%
\pgfsys@defobject{currentmarker}{\pgfqpoint{0.000000in}{0.000000in}}{\pgfqpoint{0.020833in}{0.000000in}}{%
\pgfpathmoveto{\pgfqpoint{0.000000in}{0.000000in}}%
\pgfpathlineto{\pgfqpoint{0.020833in}{0.000000in}}%
\pgfusepath{stroke,fill}%
}%
\begin{pgfscope}%
\pgfsys@transformshift{3.420229in}{3.511833in}%
\pgfsys@useobject{currentmarker}{}%
\end{pgfscope}%
\end{pgfscope}%
\begin{pgfscope}%
\pgfsetbuttcap%
\pgfsetroundjoin%
\definecolor{currentfill}{rgb}{0.000000,0.000000,0.000000}%
\pgfsetfillcolor{currentfill}%
\pgfsetlinewidth{0.501875pt}%
\definecolor{currentstroke}{rgb}{0.000000,0.000000,0.000000}%
\pgfsetstrokecolor{currentstroke}%
\pgfsetdash{}{0pt}%
\pgfsys@defobject{currentmarker}{\pgfqpoint{-0.020833in}{0.000000in}}{\pgfqpoint{-0.000000in}{0.000000in}}{%
\pgfpathmoveto{\pgfqpoint{-0.000000in}{0.000000in}}%
\pgfpathlineto{\pgfqpoint{-0.020833in}{0.000000in}}%
\pgfusepath{stroke,fill}%
}%
\begin{pgfscope}%
\pgfsys@transformshift{5.661701in}{3.511833in}%
\pgfsys@useobject{currentmarker}{}%
\end{pgfscope}%
\end{pgfscope}%
\begin{pgfscope}%
\pgfsetbuttcap%
\pgfsetroundjoin%
\definecolor{currentfill}{rgb}{0.000000,0.000000,0.000000}%
\pgfsetfillcolor{currentfill}%
\pgfsetlinewidth{0.501875pt}%
\definecolor{currentstroke}{rgb}{0.000000,0.000000,0.000000}%
\pgfsetstrokecolor{currentstroke}%
\pgfsetdash{}{0pt}%
\pgfsys@defobject{currentmarker}{\pgfqpoint{0.000000in}{0.000000in}}{\pgfqpoint{0.020833in}{0.000000in}}{%
\pgfpathmoveto{\pgfqpoint{0.000000in}{0.000000in}}%
\pgfpathlineto{\pgfqpoint{0.020833in}{0.000000in}}%
\pgfusepath{stroke,fill}%
}%
\begin{pgfscope}%
\pgfsys@transformshift{3.420229in}{3.687227in}%
\pgfsys@useobject{currentmarker}{}%
\end{pgfscope}%
\end{pgfscope}%
\begin{pgfscope}%
\pgfsetbuttcap%
\pgfsetroundjoin%
\definecolor{currentfill}{rgb}{0.000000,0.000000,0.000000}%
\pgfsetfillcolor{currentfill}%
\pgfsetlinewidth{0.501875pt}%
\definecolor{currentstroke}{rgb}{0.000000,0.000000,0.000000}%
\pgfsetstrokecolor{currentstroke}%
\pgfsetdash{}{0pt}%
\pgfsys@defobject{currentmarker}{\pgfqpoint{-0.020833in}{0.000000in}}{\pgfqpoint{-0.000000in}{0.000000in}}{%
\pgfpathmoveto{\pgfqpoint{-0.000000in}{0.000000in}}%
\pgfpathlineto{\pgfqpoint{-0.020833in}{0.000000in}}%
\pgfusepath{stroke,fill}%
}%
\begin{pgfscope}%
\pgfsys@transformshift{5.661701in}{3.687227in}%
\pgfsys@useobject{currentmarker}{}%
\end{pgfscope}%
\end{pgfscope}%
\begin{pgfscope}%
\pgfsetbuttcap%
\pgfsetroundjoin%
\definecolor{currentfill}{rgb}{0.000000,0.000000,0.000000}%
\pgfsetfillcolor{currentfill}%
\pgfsetlinewidth{0.501875pt}%
\definecolor{currentstroke}{rgb}{0.000000,0.000000,0.000000}%
\pgfsetstrokecolor{currentstroke}%
\pgfsetdash{}{0pt}%
\pgfsys@defobject{currentmarker}{\pgfqpoint{0.000000in}{0.000000in}}{\pgfqpoint{0.020833in}{0.000000in}}{%
\pgfpathmoveto{\pgfqpoint{0.000000in}{0.000000in}}%
\pgfpathlineto{\pgfqpoint{0.020833in}{0.000000in}}%
\pgfusepath{stroke,fill}%
}%
\begin{pgfscope}%
\pgfsys@transformshift{3.420229in}{3.862620in}%
\pgfsys@useobject{currentmarker}{}%
\end{pgfscope}%
\end{pgfscope}%
\begin{pgfscope}%
\pgfsetbuttcap%
\pgfsetroundjoin%
\definecolor{currentfill}{rgb}{0.000000,0.000000,0.000000}%
\pgfsetfillcolor{currentfill}%
\pgfsetlinewidth{0.501875pt}%
\definecolor{currentstroke}{rgb}{0.000000,0.000000,0.000000}%
\pgfsetstrokecolor{currentstroke}%
\pgfsetdash{}{0pt}%
\pgfsys@defobject{currentmarker}{\pgfqpoint{-0.020833in}{0.000000in}}{\pgfqpoint{-0.000000in}{0.000000in}}{%
\pgfpathmoveto{\pgfqpoint{-0.000000in}{0.000000in}}%
\pgfpathlineto{\pgfqpoint{-0.020833in}{0.000000in}}%
\pgfusepath{stroke,fill}%
}%
\begin{pgfscope}%
\pgfsys@transformshift{5.661701in}{3.862620in}%
\pgfsys@useobject{currentmarker}{}%
\end{pgfscope}%
\end{pgfscope}%
\begin{pgfscope}%
\pgfsetbuttcap%
\pgfsetroundjoin%
\definecolor{currentfill}{rgb}{0.000000,0.000000,0.000000}%
\pgfsetfillcolor{currentfill}%
\pgfsetlinewidth{0.501875pt}%
\definecolor{currentstroke}{rgb}{0.000000,0.000000,0.000000}%
\pgfsetstrokecolor{currentstroke}%
\pgfsetdash{}{0pt}%
\pgfsys@defobject{currentmarker}{\pgfqpoint{0.000000in}{0.000000in}}{\pgfqpoint{0.020833in}{0.000000in}}{%
\pgfpathmoveto{\pgfqpoint{0.000000in}{0.000000in}}%
\pgfpathlineto{\pgfqpoint{0.020833in}{0.000000in}}%
\pgfusepath{stroke,fill}%
}%
\begin{pgfscope}%
\pgfsys@transformshift{3.420229in}{4.213408in}%
\pgfsys@useobject{currentmarker}{}%
\end{pgfscope}%
\end{pgfscope}%
\begin{pgfscope}%
\pgfsetbuttcap%
\pgfsetroundjoin%
\definecolor{currentfill}{rgb}{0.000000,0.000000,0.000000}%
\pgfsetfillcolor{currentfill}%
\pgfsetlinewidth{0.501875pt}%
\definecolor{currentstroke}{rgb}{0.000000,0.000000,0.000000}%
\pgfsetstrokecolor{currentstroke}%
\pgfsetdash{}{0pt}%
\pgfsys@defobject{currentmarker}{\pgfqpoint{-0.020833in}{0.000000in}}{\pgfqpoint{-0.000000in}{0.000000in}}{%
\pgfpathmoveto{\pgfqpoint{-0.000000in}{0.000000in}}%
\pgfpathlineto{\pgfqpoint{-0.020833in}{0.000000in}}%
\pgfusepath{stroke,fill}%
}%
\begin{pgfscope}%
\pgfsys@transformshift{5.661701in}{4.213408in}%
\pgfsys@useobject{currentmarker}{}%
\end{pgfscope}%
\end{pgfscope}%
\begin{pgfscope}%
\definecolor{textcolor}{rgb}{0.000000,0.000000,0.000000}%
\pgfsetstrokecolor{textcolor}%
\pgfsetfillcolor{textcolor}%
\pgftext[x=3.138593in,y=2.398593in,,bottom,rotate=90.000000]{\color{textcolor}\rmfamily\fontsize{10.000000}{12.000000}\selectfont \(\displaystyle LCMC(K)\)}%
\end{pgfscope}%
\begin{pgfscope}%
\pgfpathrectangle{\pgfqpoint{3.420229in}{0.422992in}}{\pgfqpoint{2.241471in}{3.951201in}}%
\pgfusepath{clip}%
\pgfsetrectcap%
\pgfsetroundjoin%
\pgfsetlinewidth{1.003750pt}%
\definecolor{currentstroke}{rgb}{0.047059,0.364706,0.647059}%
\pgfsetstrokecolor{currentstroke}%
\pgfsetdash{}{0pt}%
\pgfpathmoveto{\pgfqpoint{3.442422in}{0.718375in}}%
\pgfpathlineto{\pgfqpoint{3.464615in}{0.899067in}}%
\pgfpathlineto{\pgfqpoint{3.486808in}{1.047011in}}%
\pgfpathlineto{\pgfqpoint{3.509001in}{1.095985in}}%
\pgfpathlineto{\pgfqpoint{3.531193in}{1.124667in}}%
\pgfpathlineto{\pgfqpoint{3.553386in}{1.171272in}}%
\pgfpathlineto{\pgfqpoint{3.575579in}{1.207569in}}%
\pgfpathlineto{\pgfqpoint{3.597772in}{1.223169in}}%
\pgfpathlineto{\pgfqpoint{3.619964in}{1.243099in}}%
\pgfpathlineto{\pgfqpoint{3.642157in}{1.256061in}}%
\pgfpathlineto{\pgfqpoint{3.664350in}{1.271132in}}%
\pgfpathlineto{\pgfqpoint{3.686543in}{1.294947in}}%
\pgfpathlineto{\pgfqpoint{3.708736in}{1.304438in}}%
\pgfpathlineto{\pgfqpoint{3.730928in}{1.309942in}}%
\pgfpathlineto{\pgfqpoint{3.753121in}{1.314243in}}%
\pgfpathlineto{\pgfqpoint{3.775314in}{1.327217in}}%
\pgfpathlineto{\pgfqpoint{3.797507in}{1.334847in}}%
\pgfpathlineto{\pgfqpoint{3.819700in}{1.351960in}}%
\pgfpathlineto{\pgfqpoint{3.841892in}{1.364224in}}%
\pgfpathlineto{\pgfqpoint{3.864085in}{1.371314in}}%
\pgfpathlineto{\pgfqpoint{3.886278in}{1.381238in}}%
\pgfpathlineto{\pgfqpoint{3.908471in}{1.392094in}}%
\pgfpathlineto{\pgfqpoint{3.930663in}{1.408565in}}%
\pgfpathlineto{\pgfqpoint{3.952856in}{1.421544in}}%
\pgfpathlineto{\pgfqpoint{3.975049in}{1.436151in}}%
\pgfpathlineto{\pgfqpoint{3.997242in}{1.447610in}}%
\pgfpathlineto{\pgfqpoint{4.019435in}{1.458481in}}%
\pgfpathlineto{\pgfqpoint{4.041627in}{1.472146in}}%
\pgfpathlineto{\pgfqpoint{4.063820in}{1.479182in}}%
\pgfpathlineto{\pgfqpoint{4.086013in}{1.485164in}}%
\pgfpathlineto{\pgfqpoint{4.108206in}{1.496703in}}%
\pgfpathlineto{\pgfqpoint{4.130399in}{1.505656in}}%
\pgfpathlineto{\pgfqpoint{4.152591in}{1.514598in}}%
\pgfpathlineto{\pgfqpoint{4.174784in}{1.524356in}}%
\pgfpathlineto{\pgfqpoint{4.196977in}{1.532654in}}%
\pgfpathlineto{\pgfqpoint{4.219170in}{1.537713in}}%
\pgfpathlineto{\pgfqpoint{4.241362in}{1.543162in}}%
\pgfpathlineto{\pgfqpoint{4.263555in}{1.549941in}}%
\pgfpathlineto{\pgfqpoint{4.285748in}{1.556237in}}%
\pgfpathlineto{\pgfqpoint{4.307941in}{1.566165in}}%
\pgfpathlineto{\pgfqpoint{4.330134in}{1.576679in}}%
\pgfpathlineto{\pgfqpoint{4.352326in}{1.583977in}}%
\pgfpathlineto{\pgfqpoint{4.374519in}{1.590772in}}%
\pgfpathlineto{\pgfqpoint{4.396712in}{1.596740in}}%
\pgfpathlineto{\pgfqpoint{4.418905in}{1.605367in}}%
\pgfpathlineto{\pgfqpoint{4.441098in}{1.614877in}}%
\pgfpathlineto{\pgfqpoint{4.463290in}{1.620847in}}%
\pgfpathlineto{\pgfqpoint{4.485483in}{1.627409in}}%
\pgfpathlineto{\pgfqpoint{4.507676in}{1.633237in}}%
\pgfpathlineto{\pgfqpoint{4.529869in}{1.638903in}}%
\pgfpathlineto{\pgfqpoint{4.552061in}{1.645757in}}%
\pgfpathlineto{\pgfqpoint{4.574254in}{1.653359in}}%
\pgfpathlineto{\pgfqpoint{4.596447in}{1.658159in}}%
\pgfpathlineto{\pgfqpoint{4.618640in}{1.660831in}}%
\pgfpathlineto{\pgfqpoint{4.640833in}{1.667649in}}%
\pgfpathlineto{\pgfqpoint{4.663025in}{1.674067in}}%
\pgfpathlineto{\pgfqpoint{4.685218in}{1.680075in}}%
\pgfpathlineto{\pgfqpoint{4.707411in}{1.684363in}}%
\pgfpathlineto{\pgfqpoint{4.729604in}{1.690201in}}%
\pgfpathlineto{\pgfqpoint{4.751797in}{1.694089in}}%
\pgfpathlineto{\pgfqpoint{4.773989in}{1.698656in}}%
\pgfpathlineto{\pgfqpoint{4.796182in}{1.704914in}}%
\pgfpathlineto{\pgfqpoint{4.818375in}{1.707911in}}%
\pgfpathlineto{\pgfqpoint{4.840568in}{1.712513in}}%
\pgfpathlineto{\pgfqpoint{4.862761in}{1.719403in}}%
\pgfpathlineto{\pgfqpoint{4.884953in}{1.724011in}}%
\pgfpathlineto{\pgfqpoint{4.907146in}{1.727905in}}%
\pgfpathlineto{\pgfqpoint{4.929339in}{1.733723in}}%
\pgfpathlineto{\pgfqpoint{4.951532in}{1.737974in}}%
\pgfpathlineto{\pgfqpoint{4.973724in}{1.741777in}}%
\pgfpathlineto{\pgfqpoint{4.995917in}{1.747154in}}%
\pgfpathlineto{\pgfqpoint{5.018110in}{1.749896in}}%
\pgfpathlineto{\pgfqpoint{5.040303in}{1.754101in}}%
\pgfpathlineto{\pgfqpoint{5.062496in}{1.758215in}}%
\pgfpathlineto{\pgfqpoint{5.084688in}{1.760443in}}%
\pgfpathlineto{\pgfqpoint{5.106881in}{1.764181in}}%
\pgfpathlineto{\pgfqpoint{5.129074in}{1.765704in}}%
\pgfpathlineto{\pgfqpoint{5.151267in}{1.769526in}}%
\pgfpathlineto{\pgfqpoint{5.173460in}{1.773252in}}%
\pgfpathlineto{\pgfqpoint{5.195652in}{1.777608in}}%
\pgfpathlineto{\pgfqpoint{5.217845in}{1.780276in}}%
\pgfpathlineto{\pgfqpoint{5.240038in}{1.783627in}}%
\pgfpathlineto{\pgfqpoint{5.262231in}{1.787659in}}%
\pgfpathlineto{\pgfqpoint{5.284423in}{1.791218in}}%
\pgfpathlineto{\pgfqpoint{5.306616in}{1.794240in}}%
\pgfpathlineto{\pgfqpoint{5.328809in}{1.797702in}}%
\pgfpathlineto{\pgfqpoint{5.351002in}{1.801668in}}%
\pgfpathlineto{\pgfqpoint{5.373195in}{1.805804in}}%
\pgfpathlineto{\pgfqpoint{5.395387in}{1.810083in}}%
\pgfpathlineto{\pgfqpoint{5.417580in}{1.813721in}}%
\pgfpathlineto{\pgfqpoint{5.439773in}{1.816258in}}%
\pgfpathlineto{\pgfqpoint{5.461966in}{1.819693in}}%
\pgfpathlineto{\pgfqpoint{5.484159in}{1.822639in}}%
\pgfpathlineto{\pgfqpoint{5.506351in}{1.826698in}}%
\pgfpathlineto{\pgfqpoint{5.528544in}{1.831115in}}%
\pgfpathlineto{\pgfqpoint{5.550737in}{1.833484in}}%
\pgfpathlineto{\pgfqpoint{5.572930in}{1.836528in}}%
\pgfpathlineto{\pgfqpoint{5.595122in}{1.840190in}}%
\pgfpathlineto{\pgfqpoint{5.617315in}{1.843265in}}%
\pgfusepath{stroke}%
\end{pgfscope}%
\begin{pgfscope}%
\pgfpathrectangle{\pgfqpoint{3.420229in}{0.422992in}}{\pgfqpoint{2.241471in}{3.951201in}}%
\pgfusepath{clip}%
\pgfsetrectcap%
\pgfsetroundjoin%
\pgfsetlinewidth{1.003750pt}%
\definecolor{currentstroke}{rgb}{0.000000,0.725490,0.270588}%
\pgfsetstrokecolor{currentstroke}%
\pgfsetdash{}{0pt}%
\pgfpathmoveto{\pgfqpoint{3.442422in}{2.302497in}}%
\pgfpathlineto{\pgfqpoint{3.464615in}{2.491083in}}%
\pgfpathlineto{\pgfqpoint{3.486808in}{2.570318in}}%
\pgfpathlineto{\pgfqpoint{3.509001in}{2.589761in}}%
\pgfpathlineto{\pgfqpoint{3.531193in}{2.635109in}}%
\pgfpathlineto{\pgfqpoint{3.553386in}{2.666218in}}%
\pgfpathlineto{\pgfqpoint{3.575579in}{2.682173in}}%
\pgfpathlineto{\pgfqpoint{3.597772in}{2.693701in}}%
\pgfpathlineto{\pgfqpoint{3.619964in}{2.717676in}}%
\pgfpathlineto{\pgfqpoint{3.642157in}{2.729137in}}%
\pgfpathlineto{\pgfqpoint{3.664350in}{2.745850in}}%
\pgfpathlineto{\pgfqpoint{3.686543in}{2.749398in}}%
\pgfpathlineto{\pgfqpoint{3.708736in}{2.761037in}}%
\pgfpathlineto{\pgfqpoint{3.730928in}{2.771263in}}%
\pgfpathlineto{\pgfqpoint{3.753121in}{2.785740in}}%
\pgfpathlineto{\pgfqpoint{3.775314in}{2.795227in}}%
\pgfpathlineto{\pgfqpoint{3.797507in}{2.796788in}}%
\pgfpathlineto{\pgfqpoint{3.819700in}{2.801099in}}%
\pgfpathlineto{\pgfqpoint{3.841892in}{2.813265in}}%
\pgfpathlineto{\pgfqpoint{3.864085in}{2.813339in}}%
\pgfpathlineto{\pgfqpoint{3.886278in}{2.815995in}}%
\pgfpathlineto{\pgfqpoint{3.908471in}{2.828776in}}%
\pgfpathlineto{\pgfqpoint{3.930663in}{2.836327in}}%
\pgfpathlineto{\pgfqpoint{3.952856in}{2.843467in}}%
\pgfpathlineto{\pgfqpoint{3.975049in}{2.854037in}}%
\pgfpathlineto{\pgfqpoint{3.997242in}{2.857720in}}%
\pgfpathlineto{\pgfqpoint{4.019435in}{2.860741in}}%
\pgfpathlineto{\pgfqpoint{4.041627in}{2.866679in}}%
\pgfpathlineto{\pgfqpoint{4.063820in}{2.872026in}}%
\pgfpathlineto{\pgfqpoint{4.086013in}{2.874735in}}%
\pgfpathlineto{\pgfqpoint{4.108206in}{2.881175in}}%
\pgfpathlineto{\pgfqpoint{4.130399in}{2.886171in}}%
\pgfpathlineto{\pgfqpoint{4.152591in}{2.891182in}}%
\pgfpathlineto{\pgfqpoint{4.174784in}{2.895073in}}%
\pgfpathlineto{\pgfqpoint{4.196977in}{2.903554in}}%
\pgfpathlineto{\pgfqpoint{4.219170in}{2.906155in}}%
\pgfpathlineto{\pgfqpoint{4.241362in}{2.911033in}}%
\pgfpathlineto{\pgfqpoint{4.263555in}{2.918701in}}%
\pgfpathlineto{\pgfqpoint{4.285748in}{2.926920in}}%
\pgfpathlineto{\pgfqpoint{4.307941in}{2.929159in}}%
\pgfpathlineto{\pgfqpoint{4.330134in}{2.936166in}}%
\pgfpathlineto{\pgfqpoint{4.352326in}{2.939707in}}%
\pgfpathlineto{\pgfqpoint{4.374519in}{2.943164in}}%
\pgfpathlineto{\pgfqpoint{4.396712in}{2.949734in}}%
\pgfpathlineto{\pgfqpoint{4.418905in}{2.953127in}}%
\pgfpathlineto{\pgfqpoint{4.441098in}{2.956335in}}%
\pgfpathlineto{\pgfqpoint{4.463290in}{2.960451in}}%
\pgfpathlineto{\pgfqpoint{4.485483in}{2.964322in}}%
\pgfpathlineto{\pgfqpoint{4.507676in}{2.968895in}}%
\pgfpathlineto{\pgfqpoint{4.529869in}{2.972723in}}%
\pgfpathlineto{\pgfqpoint{4.552061in}{2.976883in}}%
\pgfpathlineto{\pgfqpoint{4.574254in}{2.979668in}}%
\pgfpathlineto{\pgfqpoint{4.596447in}{2.984400in}}%
\pgfpathlineto{\pgfqpoint{4.618640in}{2.989347in}}%
\pgfpathlineto{\pgfqpoint{4.640833in}{2.995645in}}%
\pgfpathlineto{\pgfqpoint{4.663025in}{3.002125in}}%
\pgfpathlineto{\pgfqpoint{4.685218in}{3.006747in}}%
\pgfpathlineto{\pgfqpoint{4.707411in}{3.014083in}}%
\pgfpathlineto{\pgfqpoint{4.729604in}{3.019118in}}%
\pgfpathlineto{\pgfqpoint{4.751797in}{3.024103in}}%
\pgfpathlineto{\pgfqpoint{4.773989in}{3.030534in}}%
\pgfpathlineto{\pgfqpoint{4.796182in}{3.035428in}}%
\pgfpathlineto{\pgfqpoint{4.818375in}{3.036993in}}%
\pgfpathlineto{\pgfqpoint{4.840568in}{3.044127in}}%
\pgfpathlineto{\pgfqpoint{4.862761in}{3.050287in}}%
\pgfpathlineto{\pgfqpoint{4.884953in}{3.056100in}}%
\pgfpathlineto{\pgfqpoint{4.907146in}{3.060457in}}%
\pgfpathlineto{\pgfqpoint{4.929339in}{3.063035in}}%
\pgfpathlineto{\pgfqpoint{4.951532in}{3.067037in}}%
\pgfpathlineto{\pgfqpoint{4.973724in}{3.069949in}}%
\pgfpathlineto{\pgfqpoint{4.995917in}{3.074458in}}%
\pgfpathlineto{\pgfqpoint{5.018110in}{3.076722in}}%
\pgfpathlineto{\pgfqpoint{5.040303in}{3.079621in}}%
\pgfpathlineto{\pgfqpoint{5.062496in}{3.083959in}}%
\pgfpathlineto{\pgfqpoint{5.084688in}{3.086638in}}%
\pgfpathlineto{\pgfqpoint{5.106881in}{3.090885in}}%
\pgfpathlineto{\pgfqpoint{5.129074in}{3.094451in}}%
\pgfpathlineto{\pgfqpoint{5.151267in}{3.098782in}}%
\pgfpathlineto{\pgfqpoint{5.173460in}{3.102425in}}%
\pgfpathlineto{\pgfqpoint{5.195652in}{3.106021in}}%
\pgfpathlineto{\pgfqpoint{5.217845in}{3.109376in}}%
\pgfpathlineto{\pgfqpoint{5.240038in}{3.112671in}}%
\pgfpathlineto{\pgfqpoint{5.262231in}{3.114238in}}%
\pgfpathlineto{\pgfqpoint{5.284423in}{3.116457in}}%
\pgfpathlineto{\pgfqpoint{5.306616in}{3.118686in}}%
\pgfpathlineto{\pgfqpoint{5.328809in}{3.122597in}}%
\pgfpathlineto{\pgfqpoint{5.351002in}{3.124401in}}%
\pgfpathlineto{\pgfqpoint{5.373195in}{3.126324in}}%
\pgfpathlineto{\pgfqpoint{5.395387in}{3.128203in}}%
\pgfpathlineto{\pgfqpoint{5.417580in}{3.131133in}}%
\pgfpathlineto{\pgfqpoint{5.439773in}{3.133092in}}%
\pgfpathlineto{\pgfqpoint{5.461966in}{3.136190in}}%
\pgfpathlineto{\pgfqpoint{5.484159in}{3.138562in}}%
\pgfpathlineto{\pgfqpoint{5.506351in}{3.141424in}}%
\pgfpathlineto{\pgfqpoint{5.528544in}{3.144762in}}%
\pgfpathlineto{\pgfqpoint{5.550737in}{3.146185in}}%
\pgfpathlineto{\pgfqpoint{5.572930in}{3.148066in}}%
\pgfpathlineto{\pgfqpoint{5.595122in}{3.150804in}}%
\pgfpathlineto{\pgfqpoint{5.617315in}{3.152548in}}%
\pgfusepath{stroke}%
\end{pgfscope}%
\begin{pgfscope}%
\pgfpathrectangle{\pgfqpoint{3.420229in}{0.422992in}}{\pgfqpoint{2.241471in}{3.951201in}}%
\pgfusepath{clip}%
\pgfsetrectcap%
\pgfsetroundjoin%
\pgfsetlinewidth{1.003750pt}%
\definecolor{currentstroke}{rgb}{1.000000,0.584314,0.000000}%
\pgfsetstrokecolor{currentstroke}%
\pgfsetdash{}{0pt}%
\pgfpathmoveto{\pgfqpoint{3.442422in}{3.253321in}}%
\pgfpathlineto{\pgfqpoint{3.464615in}{3.330509in}}%
\pgfpathlineto{\pgfqpoint{3.486808in}{3.415299in}}%
\pgfpathlineto{\pgfqpoint{3.509001in}{3.462519in}}%
\pgfpathlineto{\pgfqpoint{3.531193in}{3.516463in}}%
\pgfpathlineto{\pgfqpoint{3.553386in}{3.522895in}}%
\pgfpathlineto{\pgfqpoint{3.575579in}{3.528742in}}%
\pgfpathlineto{\pgfqpoint{3.597772in}{3.547600in}}%
\pgfpathlineto{\pgfqpoint{3.619964in}{3.552717in}}%
\pgfpathlineto{\pgfqpoint{3.642157in}{3.564178in}}%
\pgfpathlineto{\pgfqpoint{3.664350in}{3.574671in}}%
\pgfpathlineto{\pgfqpoint{3.686543in}{3.582538in}}%
\pgfpathlineto{\pgfqpoint{3.708736in}{3.587306in}}%
\pgfpathlineto{\pgfqpoint{3.730928in}{3.593022in}}%
\pgfpathlineto{\pgfqpoint{3.753121in}{3.597975in}}%
\pgfpathlineto{\pgfqpoint{3.775314in}{3.603296in}}%
\pgfpathlineto{\pgfqpoint{3.797507in}{3.603760in}}%
\pgfpathlineto{\pgfqpoint{3.819700in}{3.597935in}}%
\pgfpathlineto{\pgfqpoint{3.841892in}{3.598448in}}%
\pgfpathlineto{\pgfqpoint{3.864085in}{3.598909in}}%
\pgfpathlineto{\pgfqpoint{3.886278in}{3.599828in}}%
\pgfpathlineto{\pgfqpoint{3.908471in}{3.598111in}}%
\pgfpathlineto{\pgfqpoint{3.930663in}{3.595781in}}%
\pgfpathlineto{\pgfqpoint{3.952856in}{3.598030in}}%
\pgfpathlineto{\pgfqpoint{3.975049in}{3.599328in}}%
\pgfpathlineto{\pgfqpoint{3.997242in}{3.598704in}}%
\pgfpathlineto{\pgfqpoint{4.019435in}{3.601635in}}%
\pgfpathlineto{\pgfqpoint{4.041627in}{3.598468in}}%
\pgfpathlineto{\pgfqpoint{4.063820in}{3.599390in}}%
\pgfpathlineto{\pgfqpoint{4.086013in}{3.596449in}}%
\pgfpathlineto{\pgfqpoint{4.108206in}{3.591379in}}%
\pgfpathlineto{\pgfqpoint{4.130399in}{3.593806in}}%
\pgfpathlineto{\pgfqpoint{4.152591in}{3.594917in}}%
\pgfpathlineto{\pgfqpoint{4.174784in}{3.591938in}}%
\pgfpathlineto{\pgfqpoint{4.196977in}{3.585220in}}%
\pgfpathlineto{\pgfqpoint{4.219170in}{3.584430in}}%
\pgfpathlineto{\pgfqpoint{4.241362in}{3.585295in}}%
\pgfpathlineto{\pgfqpoint{4.263555in}{3.583022in}}%
\pgfpathlineto{\pgfqpoint{4.285748in}{3.582439in}}%
\pgfpathlineto{\pgfqpoint{4.307941in}{3.578114in}}%
\pgfpathlineto{\pgfqpoint{4.330134in}{3.573700in}}%
\pgfpathlineto{\pgfqpoint{4.352326in}{3.572044in}}%
\pgfpathlineto{\pgfqpoint{4.374519in}{3.570914in}}%
\pgfpathlineto{\pgfqpoint{4.396712in}{3.567961in}}%
\pgfpathlineto{\pgfqpoint{4.418905in}{3.566582in}}%
\pgfpathlineto{\pgfqpoint{4.441098in}{3.566140in}}%
\pgfpathlineto{\pgfqpoint{4.463290in}{3.565979in}}%
\pgfpathlineto{\pgfqpoint{4.485483in}{3.562023in}}%
\pgfpathlineto{\pgfqpoint{4.507676in}{3.560483in}}%
\pgfpathlineto{\pgfqpoint{4.529869in}{3.561006in}}%
\pgfpathlineto{\pgfqpoint{4.552061in}{3.557930in}}%
\pgfpathlineto{\pgfqpoint{4.574254in}{3.558448in}}%
\pgfpathlineto{\pgfqpoint{4.596447in}{3.557258in}}%
\pgfpathlineto{\pgfqpoint{4.618640in}{3.558353in}}%
\pgfpathlineto{\pgfqpoint{4.640833in}{3.558261in}}%
\pgfpathlineto{\pgfqpoint{4.663025in}{3.558234in}}%
\pgfpathlineto{\pgfqpoint{4.685218in}{3.559009in}}%
\pgfpathlineto{\pgfqpoint{4.707411in}{3.559545in}}%
\pgfpathlineto{\pgfqpoint{4.729604in}{3.559290in}}%
\pgfpathlineto{\pgfqpoint{4.751797in}{3.559804in}}%
\pgfpathlineto{\pgfqpoint{4.773989in}{3.558345in}}%
\pgfpathlineto{\pgfqpoint{4.796182in}{3.556961in}}%
\pgfpathlineto{\pgfqpoint{4.818375in}{3.556958in}}%
\pgfpathlineto{\pgfqpoint{4.840568in}{3.556325in}}%
\pgfpathlineto{\pgfqpoint{4.862761in}{3.554415in}}%
\pgfpathlineto{\pgfqpoint{4.884953in}{3.553839in}}%
\pgfpathlineto{\pgfqpoint{4.907146in}{3.551370in}}%
\pgfpathlineto{\pgfqpoint{4.929339in}{3.550469in}}%
\pgfpathlineto{\pgfqpoint{4.951532in}{3.548551in}}%
\pgfpathlineto{\pgfqpoint{4.973724in}{3.548794in}}%
\pgfpathlineto{\pgfqpoint{4.995917in}{3.547597in}}%
\pgfpathlineto{\pgfqpoint{5.018110in}{3.545702in}}%
\pgfpathlineto{\pgfqpoint{5.040303in}{3.544988in}}%
\pgfpathlineto{\pgfqpoint{5.062496in}{3.545052in}}%
\pgfpathlineto{\pgfqpoint{5.084688in}{3.544203in}}%
\pgfpathlineto{\pgfqpoint{5.106881in}{3.542198in}}%
\pgfpathlineto{\pgfqpoint{5.129074in}{3.541453in}}%
\pgfpathlineto{\pgfqpoint{5.151267in}{3.541762in}}%
\pgfpathlineto{\pgfqpoint{5.173460in}{3.539886in}}%
\pgfpathlineto{\pgfqpoint{5.195652in}{3.538080in}}%
\pgfpathlineto{\pgfqpoint{5.217845in}{3.536578in}}%
\pgfpathlineto{\pgfqpoint{5.240038in}{3.536011in}}%
\pgfpathlineto{\pgfqpoint{5.262231in}{3.535775in}}%
\pgfpathlineto{\pgfqpoint{5.284423in}{3.534604in}}%
\pgfpathlineto{\pgfqpoint{5.306616in}{3.533812in}}%
\pgfpathlineto{\pgfqpoint{5.328809in}{3.534263in}}%
\pgfpathlineto{\pgfqpoint{5.351002in}{3.534239in}}%
\pgfpathlineto{\pgfqpoint{5.373195in}{3.533219in}}%
\pgfpathlineto{\pgfqpoint{5.395387in}{3.532044in}}%
\pgfpathlineto{\pgfqpoint{5.417580in}{3.533060in}}%
\pgfpathlineto{\pgfqpoint{5.439773in}{3.531373in}}%
\pgfpathlineto{\pgfqpoint{5.461966in}{3.531420in}}%
\pgfpathlineto{\pgfqpoint{5.484159in}{3.530994in}}%
\pgfpathlineto{\pgfqpoint{5.506351in}{3.529085in}}%
\pgfpathlineto{\pgfqpoint{5.528544in}{3.526791in}}%
\pgfpathlineto{\pgfqpoint{5.550737in}{3.526262in}}%
\pgfpathlineto{\pgfqpoint{5.572930in}{3.526992in}}%
\pgfpathlineto{\pgfqpoint{5.595122in}{3.526920in}}%
\pgfpathlineto{\pgfqpoint{5.617315in}{3.525680in}}%
\pgfusepath{stroke}%
\end{pgfscope}%
\begin{pgfscope}%
\pgfpathrectangle{\pgfqpoint{3.420229in}{0.422992in}}{\pgfqpoint{2.241471in}{3.951201in}}%
\pgfusepath{clip}%
\pgfsetrectcap%
\pgfsetroundjoin%
\pgfsetlinewidth{1.003750pt}%
\definecolor{currentstroke}{rgb}{1.000000,0.172549,0.000000}%
\pgfsetstrokecolor{currentstroke}%
\pgfsetdash{}{0pt}%
\pgfpathmoveto{\pgfqpoint{3.442422in}{3.684876in}}%
\pgfpathlineto{\pgfqpoint{3.464615in}{3.805921in}}%
\pgfpathlineto{\pgfqpoint{3.486808in}{3.901822in}}%
\pgfpathlineto{\pgfqpoint{3.509001in}{3.985735in}}%
\pgfpathlineto{\pgfqpoint{3.531193in}{3.988717in}}%
\pgfpathlineto{\pgfqpoint{3.553386in}{4.045672in}}%
\pgfpathlineto{\pgfqpoint{3.575579in}{4.051770in}}%
\pgfpathlineto{\pgfqpoint{3.597772in}{4.075641in}}%
\pgfpathlineto{\pgfqpoint{3.619964in}{4.088944in}}%
\pgfpathlineto{\pgfqpoint{3.642157in}{4.120988in}}%
\pgfpathlineto{\pgfqpoint{3.664350in}{4.124720in}}%
\pgfpathlineto{\pgfqpoint{3.686543in}{4.146396in}}%
\pgfpathlineto{\pgfqpoint{3.708736in}{4.152861in}}%
\pgfpathlineto{\pgfqpoint{3.730928in}{4.172438in}}%
\pgfpathlineto{\pgfqpoint{3.753121in}{4.172212in}}%
\pgfpathlineto{\pgfqpoint{3.775314in}{4.172124in}}%
\pgfpathlineto{\pgfqpoint{3.797507in}{4.172872in}}%
\pgfpathlineto{\pgfqpoint{3.819700in}{4.171587in}}%
\pgfpathlineto{\pgfqpoint{3.841892in}{4.175054in}}%
\pgfpathlineto{\pgfqpoint{3.864085in}{4.180982in}}%
\pgfpathlineto{\pgfqpoint{3.886278in}{4.183003in}}%
\pgfpathlineto{\pgfqpoint{3.908471in}{4.186993in}}%
\pgfpathlineto{\pgfqpoint{3.930663in}{4.192467in}}%
\pgfpathlineto{\pgfqpoint{3.952856in}{4.192076in}}%
\pgfpathlineto{\pgfqpoint{3.975049in}{4.194593in}}%
\pgfpathlineto{\pgfqpoint{3.997242in}{4.193880in}}%
\pgfpathlineto{\pgfqpoint{4.019435in}{4.188152in}}%
\pgfpathlineto{\pgfqpoint{4.041627in}{4.188660in}}%
\pgfpathlineto{\pgfqpoint{4.063820in}{4.187439in}}%
\pgfpathlineto{\pgfqpoint{4.086013in}{4.183317in}}%
\pgfpathlineto{\pgfqpoint{4.108206in}{4.185007in}}%
\pgfpathlineto{\pgfqpoint{4.130399in}{4.180122in}}%
\pgfpathlineto{\pgfqpoint{4.152591in}{4.181966in}}%
\pgfpathlineto{\pgfqpoint{4.174784in}{4.178593in}}%
\pgfpathlineto{\pgfqpoint{4.196977in}{4.177668in}}%
\pgfpathlineto{\pgfqpoint{4.219170in}{4.180206in}}%
\pgfpathlineto{\pgfqpoint{4.241362in}{4.179762in}}%
\pgfpathlineto{\pgfqpoint{4.263555in}{4.182111in}}%
\pgfpathlineto{\pgfqpoint{4.285748in}{4.182900in}}%
\pgfpathlineto{\pgfqpoint{4.307941in}{4.183343in}}%
\pgfpathlineto{\pgfqpoint{4.330134in}{4.185903in}}%
\pgfpathlineto{\pgfqpoint{4.352326in}{4.184750in}}%
\pgfpathlineto{\pgfqpoint{4.374519in}{4.183772in}}%
\pgfpathlineto{\pgfqpoint{4.396712in}{4.185431in}}%
\pgfpathlineto{\pgfqpoint{4.418905in}{4.186041in}}%
\pgfpathlineto{\pgfqpoint{4.441098in}{4.187540in}}%
\pgfpathlineto{\pgfqpoint{4.463290in}{4.189833in}}%
\pgfpathlineto{\pgfqpoint{4.485483in}{4.187682in}}%
\pgfpathlineto{\pgfqpoint{4.507676in}{4.186907in}}%
\pgfpathlineto{\pgfqpoint{4.529869in}{4.184480in}}%
\pgfpathlineto{\pgfqpoint{4.552061in}{4.184108in}}%
\pgfpathlineto{\pgfqpoint{4.574254in}{4.184223in}}%
\pgfpathlineto{\pgfqpoint{4.596447in}{4.184068in}}%
\pgfpathlineto{\pgfqpoint{4.618640in}{4.182393in}}%
\pgfpathlineto{\pgfqpoint{4.640833in}{4.181320in}}%
\pgfpathlineto{\pgfqpoint{4.663025in}{4.181852in}}%
\pgfpathlineto{\pgfqpoint{4.685218in}{4.180735in}}%
\pgfpathlineto{\pgfqpoint{4.707411in}{4.180351in}}%
\pgfpathlineto{\pgfqpoint{4.729604in}{4.182122in}}%
\pgfpathlineto{\pgfqpoint{4.751797in}{4.182780in}}%
\pgfpathlineto{\pgfqpoint{4.773989in}{4.183791in}}%
\pgfpathlineto{\pgfqpoint{4.796182in}{4.185562in}}%
\pgfpathlineto{\pgfqpoint{4.818375in}{4.186719in}}%
\pgfpathlineto{\pgfqpoint{4.840568in}{4.185785in}}%
\pgfpathlineto{\pgfqpoint{4.862761in}{4.185176in}}%
\pgfpathlineto{\pgfqpoint{4.884953in}{4.186234in}}%
\pgfpathlineto{\pgfqpoint{4.907146in}{4.187050in}}%
\pgfpathlineto{\pgfqpoint{4.929339in}{4.187069in}}%
\pgfpathlineto{\pgfqpoint{4.951532in}{4.187697in}}%
\pgfpathlineto{\pgfqpoint{4.973724in}{4.188583in}}%
\pgfpathlineto{\pgfqpoint{4.995917in}{4.187146in}}%
\pgfpathlineto{\pgfqpoint{5.018110in}{4.184238in}}%
\pgfpathlineto{\pgfqpoint{5.040303in}{4.182515in}}%
\pgfpathlineto{\pgfqpoint{5.062496in}{4.183755in}}%
\pgfpathlineto{\pgfqpoint{5.084688in}{4.182857in}}%
\pgfpathlineto{\pgfqpoint{5.106881in}{4.182836in}}%
\pgfpathlineto{\pgfqpoint{5.129074in}{4.181654in}}%
\pgfpathlineto{\pgfqpoint{5.151267in}{4.180278in}}%
\pgfpathlineto{\pgfqpoint{5.173460in}{4.182000in}}%
\pgfpathlineto{\pgfqpoint{5.195652in}{4.179491in}}%
\pgfpathlineto{\pgfqpoint{5.217845in}{4.179167in}}%
\pgfpathlineto{\pgfqpoint{5.240038in}{4.179193in}}%
\pgfpathlineto{\pgfqpoint{5.262231in}{4.177802in}}%
\pgfpathlineto{\pgfqpoint{5.284423in}{4.178156in}}%
\pgfpathlineto{\pgfqpoint{5.306616in}{4.177677in}}%
\pgfpathlineto{\pgfqpoint{5.328809in}{4.176985in}}%
\pgfpathlineto{\pgfqpoint{5.351002in}{4.175986in}}%
\pgfpathlineto{\pgfqpoint{5.373195in}{4.174770in}}%
\pgfpathlineto{\pgfqpoint{5.395387in}{4.173542in}}%
\pgfpathlineto{\pgfqpoint{5.417580in}{4.172069in}}%
\pgfpathlineto{\pgfqpoint{5.439773in}{4.170165in}}%
\pgfpathlineto{\pgfqpoint{5.461966in}{4.168493in}}%
\pgfpathlineto{\pgfqpoint{5.484159in}{4.168385in}}%
\pgfpathlineto{\pgfqpoint{5.506351in}{4.168709in}}%
\pgfpathlineto{\pgfqpoint{5.528544in}{4.168915in}}%
\pgfpathlineto{\pgfqpoint{5.550737in}{4.169172in}}%
\pgfpathlineto{\pgfqpoint{5.572930in}{4.166602in}}%
\pgfpathlineto{\pgfqpoint{5.595122in}{4.166018in}}%
\pgfpathlineto{\pgfqpoint{5.617315in}{4.163673in}}%
\pgfusepath{stroke}%
\end{pgfscope}%
\begin{pgfscope}%
\pgfpathrectangle{\pgfqpoint{3.420229in}{0.422992in}}{\pgfqpoint{2.241471in}{3.951201in}}%
\pgfusepath{clip}%
\pgfsetrectcap%
\pgfsetroundjoin%
\pgfsetlinewidth{1.003750pt}%
\definecolor{currentstroke}{rgb}{0.517647,0.356863,0.592157}%
\pgfsetstrokecolor{currentstroke}%
\pgfsetdash{}{0pt}%
\pgfpathmoveto{\pgfqpoint{3.442422in}{0.602592in}}%
\pgfpathlineto{\pgfqpoint{3.464615in}{0.864858in}}%
\pgfpathlineto{\pgfqpoint{3.486808in}{1.000230in}}%
\pgfpathlineto{\pgfqpoint{3.509001in}{1.075810in}}%
\pgfpathlineto{\pgfqpoint{3.531193in}{1.119755in}}%
\pgfpathlineto{\pgfqpoint{3.553386in}{1.170395in}}%
\pgfpathlineto{\pgfqpoint{3.575579in}{1.208321in}}%
\pgfpathlineto{\pgfqpoint{3.597772in}{1.237642in}}%
\pgfpathlineto{\pgfqpoint{3.619964in}{1.259278in}}%
\pgfpathlineto{\pgfqpoint{3.642157in}{1.277464in}}%
\pgfpathlineto{\pgfqpoint{3.664350in}{1.294097in}}%
\pgfpathlineto{\pgfqpoint{3.686543in}{1.314683in}}%
\pgfpathlineto{\pgfqpoint{3.708736in}{1.324275in}}%
\pgfpathlineto{\pgfqpoint{3.730928in}{1.342020in}}%
\pgfpathlineto{\pgfqpoint{3.753121in}{1.355995in}}%
\pgfpathlineto{\pgfqpoint{3.775314in}{1.371184in}}%
\pgfpathlineto{\pgfqpoint{3.797507in}{1.386547in}}%
\pgfpathlineto{\pgfqpoint{3.819700in}{1.398643in}}%
\pgfpathlineto{\pgfqpoint{3.841892in}{1.416668in}}%
\pgfpathlineto{\pgfqpoint{3.864085in}{1.429031in}}%
\pgfpathlineto{\pgfqpoint{3.886278in}{1.441302in}}%
\pgfpathlineto{\pgfqpoint{3.908471in}{1.451740in}}%
\pgfpathlineto{\pgfqpoint{3.930663in}{1.460431in}}%
\pgfpathlineto{\pgfqpoint{3.952856in}{1.471395in}}%
\pgfpathlineto{\pgfqpoint{3.975049in}{1.482114in}}%
\pgfpathlineto{\pgfqpoint{3.997242in}{1.491468in}}%
\pgfpathlineto{\pgfqpoint{4.019435in}{1.500194in}}%
\pgfpathlineto{\pgfqpoint{4.041627in}{1.511680in}}%
\pgfpathlineto{\pgfqpoint{4.063820in}{1.522374in}}%
\pgfpathlineto{\pgfqpoint{4.086013in}{1.533466in}}%
\pgfpathlineto{\pgfqpoint{4.108206in}{1.545030in}}%
\pgfpathlineto{\pgfqpoint{4.130399in}{1.550061in}}%
\pgfpathlineto{\pgfqpoint{4.152591in}{1.557445in}}%
\pgfpathlineto{\pgfqpoint{4.174784in}{1.564343in}}%
\pgfpathlineto{\pgfqpoint{4.196977in}{1.573804in}}%
\pgfpathlineto{\pgfqpoint{4.219170in}{1.582691in}}%
\pgfpathlineto{\pgfqpoint{4.241362in}{1.590433in}}%
\pgfpathlineto{\pgfqpoint{4.263555in}{1.597214in}}%
\pgfpathlineto{\pgfqpoint{4.285748in}{1.600904in}}%
\pgfpathlineto{\pgfqpoint{4.307941in}{1.607566in}}%
\pgfpathlineto{\pgfqpoint{4.330134in}{1.614931in}}%
\pgfpathlineto{\pgfqpoint{4.352326in}{1.622654in}}%
\pgfpathlineto{\pgfqpoint{4.374519in}{1.628917in}}%
\pgfpathlineto{\pgfqpoint{4.396712in}{1.637567in}}%
\pgfpathlineto{\pgfqpoint{4.418905in}{1.646378in}}%
\pgfpathlineto{\pgfqpoint{4.441098in}{1.650229in}}%
\pgfpathlineto{\pgfqpoint{4.463290in}{1.656082in}}%
\pgfpathlineto{\pgfqpoint{4.485483in}{1.660959in}}%
\pgfpathlineto{\pgfqpoint{4.507676in}{1.668466in}}%
\pgfpathlineto{\pgfqpoint{4.529869in}{1.676375in}}%
\pgfpathlineto{\pgfqpoint{4.552061in}{1.683147in}}%
\pgfpathlineto{\pgfqpoint{4.574254in}{1.688546in}}%
\pgfpathlineto{\pgfqpoint{4.596447in}{1.694105in}}%
\pgfpathlineto{\pgfqpoint{4.618640in}{1.701115in}}%
\pgfpathlineto{\pgfqpoint{4.640833in}{1.706786in}}%
\pgfpathlineto{\pgfqpoint{4.663025in}{1.711565in}}%
\pgfpathlineto{\pgfqpoint{4.685218in}{1.717807in}}%
\pgfpathlineto{\pgfqpoint{4.707411in}{1.724439in}}%
\pgfpathlineto{\pgfqpoint{4.729604in}{1.727338in}}%
\pgfpathlineto{\pgfqpoint{4.751797in}{1.732274in}}%
\pgfpathlineto{\pgfqpoint{4.773989in}{1.738142in}}%
\pgfpathlineto{\pgfqpoint{4.796182in}{1.742094in}}%
\pgfpathlineto{\pgfqpoint{4.818375in}{1.747341in}}%
\pgfpathlineto{\pgfqpoint{4.840568in}{1.753575in}}%
\pgfpathlineto{\pgfqpoint{4.862761in}{1.759509in}}%
\pgfpathlineto{\pgfqpoint{4.884953in}{1.765343in}}%
\pgfpathlineto{\pgfqpoint{4.907146in}{1.769746in}}%
\pgfpathlineto{\pgfqpoint{4.929339in}{1.773220in}}%
\pgfpathlineto{\pgfqpoint{4.951532in}{1.778577in}}%
\pgfpathlineto{\pgfqpoint{4.973724in}{1.785058in}}%
\pgfpathlineto{\pgfqpoint{4.995917in}{1.789158in}}%
\pgfpathlineto{\pgfqpoint{5.018110in}{1.795386in}}%
\pgfpathlineto{\pgfqpoint{5.040303in}{1.800577in}}%
\pgfpathlineto{\pgfqpoint{5.062496in}{1.804491in}}%
\pgfpathlineto{\pgfqpoint{5.084688in}{1.808885in}}%
\pgfpathlineto{\pgfqpoint{5.106881in}{1.812055in}}%
\pgfpathlineto{\pgfqpoint{5.129074in}{1.817148in}}%
\pgfpathlineto{\pgfqpoint{5.151267in}{1.821593in}}%
\pgfpathlineto{\pgfqpoint{5.173460in}{1.826658in}}%
\pgfpathlineto{\pgfqpoint{5.195652in}{1.831421in}}%
\pgfpathlineto{\pgfqpoint{5.217845in}{1.832753in}}%
\pgfpathlineto{\pgfqpoint{5.240038in}{1.837497in}}%
\pgfpathlineto{\pgfqpoint{5.262231in}{1.841598in}}%
\pgfpathlineto{\pgfqpoint{5.284423in}{1.845539in}}%
\pgfpathlineto{\pgfqpoint{5.306616in}{1.849407in}}%
\pgfpathlineto{\pgfqpoint{5.328809in}{1.852656in}}%
\pgfpathlineto{\pgfqpoint{5.351002in}{1.857079in}}%
\pgfpathlineto{\pgfqpoint{5.373195in}{1.859888in}}%
\pgfpathlineto{\pgfqpoint{5.395387in}{1.862436in}}%
\pgfpathlineto{\pgfqpoint{5.417580in}{1.866311in}}%
\pgfpathlineto{\pgfqpoint{5.439773in}{1.868964in}}%
\pgfpathlineto{\pgfqpoint{5.461966in}{1.874266in}}%
\pgfpathlineto{\pgfqpoint{5.484159in}{1.878436in}}%
\pgfpathlineto{\pgfqpoint{5.506351in}{1.881062in}}%
\pgfpathlineto{\pgfqpoint{5.528544in}{1.884205in}}%
\pgfpathlineto{\pgfqpoint{5.550737in}{1.886734in}}%
\pgfpathlineto{\pgfqpoint{5.572930in}{1.890025in}}%
\pgfpathlineto{\pgfqpoint{5.595122in}{1.893445in}}%
\pgfpathlineto{\pgfqpoint{5.617315in}{1.896708in}}%
\pgfusepath{stroke}%
\end{pgfscope}%
\begin{pgfscope}%
\pgfsetrectcap%
\pgfsetmiterjoin%
\pgfsetlinewidth{0.501875pt}%
\definecolor{currentstroke}{rgb}{0.000000,0.000000,0.000000}%
\pgfsetstrokecolor{currentstroke}%
\pgfsetdash{}{0pt}%
\pgfpathmoveto{\pgfqpoint{3.420229in}{0.422992in}}%
\pgfpathlineto{\pgfqpoint{3.420229in}{4.374193in}}%
\pgfusepath{stroke}%
\end{pgfscope}%
\begin{pgfscope}%
\pgfsetrectcap%
\pgfsetmiterjoin%
\pgfsetlinewidth{0.501875pt}%
\definecolor{currentstroke}{rgb}{0.000000,0.000000,0.000000}%
\pgfsetstrokecolor{currentstroke}%
\pgfsetdash{}{0pt}%
\pgfpathmoveto{\pgfqpoint{5.661701in}{0.422992in}}%
\pgfpathlineto{\pgfqpoint{5.661701in}{4.374193in}}%
\pgfusepath{stroke}%
\end{pgfscope}%
\begin{pgfscope}%
\pgfsetrectcap%
\pgfsetmiterjoin%
\pgfsetlinewidth{0.501875pt}%
\definecolor{currentstroke}{rgb}{0.000000,0.000000,0.000000}%
\pgfsetstrokecolor{currentstroke}%
\pgfsetdash{}{0pt}%
\pgfpathmoveto{\pgfqpoint{3.420229in}{0.422992in}}%
\pgfpathlineto{\pgfqpoint{5.661701in}{0.422992in}}%
\pgfusepath{stroke}%
\end{pgfscope}%
\begin{pgfscope}%
\pgfsetrectcap%
\pgfsetmiterjoin%
\pgfsetlinewidth{0.501875pt}%
\definecolor{currentstroke}{rgb}{0.000000,0.000000,0.000000}%
\pgfsetstrokecolor{currentstroke}%
\pgfsetdash{}{0pt}%
\pgfpathmoveto{\pgfqpoint{3.420229in}{4.374193in}}%
\pgfpathlineto{\pgfqpoint{5.661701in}{4.374193in}}%
\pgfusepath{stroke}%
\end{pgfscope}%
\begin{pgfscope}%
\definecolor{textcolor}{rgb}{0.000000,0.000000,0.000000}%
\pgfsetstrokecolor{textcolor}%
\pgfsetfillcolor{textcolor}%
\pgftext[x=4.540965in,y=4.457526in,,base]{\color{textcolor}\rmfamily\fontsize{12.000000}{14.400000}\selectfont LCMC}%
\end{pgfscope}%
\begin{pgfscope}%
\pgfsetrectcap%
\pgfsetroundjoin%
\pgfsetlinewidth{1.003750pt}%
\definecolor{currentstroke}{rgb}{0.047059,0.364706,0.647059}%
\pgfsetstrokecolor{currentstroke}%
\pgfsetdash{}{0pt}%
\pgfpathmoveto{\pgfqpoint{4.383109in}{1.440922in}}%
\pgfpathlineto{\pgfqpoint{4.521998in}{1.440922in}}%
\pgfpathlineto{\pgfqpoint{4.660887in}{1.440922in}}%
\pgfusepath{stroke}%
\end{pgfscope}%
\begin{pgfscope}%
\definecolor{textcolor}{rgb}{0.000000,0.000000,0.000000}%
\pgfsetstrokecolor{textcolor}%
\pgfsetfillcolor{textcolor}%
\pgftext[x=4.771998in,y=1.392311in,left,base]{\color{textcolor}\rmfamily\fontsize{10.000000}{12.000000}\selectfont PCA}%
\end{pgfscope}%
\begin{pgfscope}%
\pgfsetrectcap%
\pgfsetroundjoin%
\pgfsetlinewidth{1.003750pt}%
\definecolor{currentstroke}{rgb}{0.000000,0.725490,0.270588}%
\pgfsetstrokecolor{currentstroke}%
\pgfsetdash{}{0pt}%
\pgfpathmoveto{\pgfqpoint{4.383109in}{1.237065in}}%
\pgfpathlineto{\pgfqpoint{4.521998in}{1.237065in}}%
\pgfpathlineto{\pgfqpoint{4.660887in}{1.237065in}}%
\pgfusepath{stroke}%
\end{pgfscope}%
\begin{pgfscope}%
\definecolor{textcolor}{rgb}{0.000000,0.000000,0.000000}%
\pgfsetstrokecolor{textcolor}%
\pgfsetfillcolor{textcolor}%
\pgftext[x=4.771998in,y=1.188454in,left,base]{\color{textcolor}\rmfamily\fontsize{10.000000}{12.000000}\selectfont KernelPCA}%
\end{pgfscope}%
\begin{pgfscope}%
\pgfsetrectcap%
\pgfsetroundjoin%
\pgfsetlinewidth{1.003750pt}%
\definecolor{currentstroke}{rgb}{1.000000,0.584314,0.000000}%
\pgfsetstrokecolor{currentstroke}%
\pgfsetdash{}{0pt}%
\pgfpathmoveto{\pgfqpoint{4.383109in}{1.033208in}}%
\pgfpathlineto{\pgfqpoint{4.521998in}{1.033208in}}%
\pgfpathlineto{\pgfqpoint{4.660887in}{1.033208in}}%
\pgfusepath{stroke}%
\end{pgfscope}%
\begin{pgfscope}%
\definecolor{textcolor}{rgb}{0.000000,0.000000,0.000000}%
\pgfsetstrokecolor{textcolor}%
\pgfsetfillcolor{textcolor}%
\pgftext[x=4.771998in,y=0.984596in,left,base]{\color{textcolor}\rmfamily\fontsize{10.000000}{12.000000}\selectfont AE}%
\end{pgfscope}%
\begin{pgfscope}%
\pgfsetrectcap%
\pgfsetroundjoin%
\pgfsetlinewidth{1.003750pt}%
\definecolor{currentstroke}{rgb}{1.000000,0.172549,0.000000}%
\pgfsetstrokecolor{currentstroke}%
\pgfsetdash{}{0pt}%
\pgfpathmoveto{\pgfqpoint{4.383109in}{0.829350in}}%
\pgfpathlineto{\pgfqpoint{4.521998in}{0.829350in}}%
\pgfpathlineto{\pgfqpoint{4.660887in}{0.829350in}}%
\pgfusepath{stroke}%
\end{pgfscope}%
\begin{pgfscope}%
\definecolor{textcolor}{rgb}{0.000000,0.000000,0.000000}%
\pgfsetstrokecolor{textcolor}%
\pgfsetfillcolor{textcolor}%
\pgftext[x=4.771998in,y=0.780739in,left,base]{\color{textcolor}\rmfamily\fontsize{10.000000}{12.000000}\selectfont LLE}%
\end{pgfscope}%
\begin{pgfscope}%
\pgfsetrectcap%
\pgfsetroundjoin%
\pgfsetlinewidth{1.003750pt}%
\definecolor{currentstroke}{rgb}{0.517647,0.356863,0.592157}%
\pgfsetstrokecolor{currentstroke}%
\pgfsetdash{}{0pt}%
\pgfpathmoveto{\pgfqpoint{4.383109in}{0.625493in}}%
\pgfpathlineto{\pgfqpoint{4.521998in}{0.625493in}}%
\pgfpathlineto{\pgfqpoint{4.660887in}{0.625493in}}%
\pgfusepath{stroke}%
\end{pgfscope}%
\begin{pgfscope}%
\definecolor{textcolor}{rgb}{0.000000,0.000000,0.000000}%
\pgfsetstrokecolor{textcolor}%
\pgfsetfillcolor{textcolor}%
\pgftext[x=4.771998in,y=0.576882in,left,base]{\color{textcolor}\rmfamily\fontsize{10.000000}{12.000000}\selectfont CAE}%
\end{pgfscope}%
\end{pgfpicture}%
\makeatother%
\endgroup%

	\end{center}
	\caption[Twin Peaks Qualitätskriterien]{Die Vertrauenswürdigkeit und Kontinuität der Dimensionsreduktion, sowie das Local Continuity Meta-Criterion (LCMC) für den Twin Peaks Datensatz. Locally Linear Embedding (LLE) schneidet wie bei der Swiss Roll insgesamt am besten ab, dicht gefolgt vom Autoencoder und der Kernel PCA. Lediglich der Contractive Autoencoder (CAE) und PCA fallen auf diesem künstlichen Datensatz etwas zurück, wobei die Vertrauenswürdigkeit der Dimensionsreduktion von PCA deutlich schlechter als bei den restlichen Methoden ist. Dies ist höchstwahrscheinlich der nichtlinearen Mannigfaltigkeit geschuldet, was der linearen Hauptkomponentenanalyse Schwierigkeiten bereitet. (Eigene Darstellung)}
	\label{fig:TwinPeaksMetrics}
\end{figure}
\begin{figure}[ht]
	\begin{center}
		%% Creator: Matplotlib, PGF backend
%%
%% To include the figure in your LaTeX document, write
%%   \input{<filename>.pgf}
%%
%% Make sure the required packages are loaded in your preamble
%%   \usepackage{pgf}
%%
%% Also ensure that all the required font packages are loaded; for instance,
%% the lmodern package is sometimes necessary when using math font.
%%   \usepackage{lmodern}
%%
%% Figures using additional raster images can only be included by \input if
%% they are in the same directory as the main LaTeX file. For loading figures
%% from other directories you can use the `import` package
%%   \usepackage{import}
%%
%% and then include the figures with
%%   \import{<path to file>}{<filename>.pgf}
%%
%% Matplotlib used the following preamble
%%   
%%   \usepackage{fontspec}
%%   \setmainfont{DejaVuSerif.ttf}[Path=\detokenize{/Users/moritzmistol/.pyenv/versions/3.9.13/envs/thesis/lib/python3.9/site-packages/matplotlib/mpl-data/fonts/ttf/}]
%%   \setsansfont{DejaVuSans.ttf}[Path=\detokenize{/Users/moritzmistol/.pyenv/versions/3.9.13/envs/thesis/lib/python3.9/site-packages/matplotlib/mpl-data/fonts/ttf/}]
%%   \setmonofont{DejaVuSansMono.ttf}[Path=\detokenize{/Users/moritzmistol/.pyenv/versions/3.9.13/envs/thesis/lib/python3.9/site-packages/matplotlib/mpl-data/fonts/ttf/}]
%%   \makeatletter\@ifpackageloaded{underscore}{}{\usepackage[strings]{underscore}}\makeatother
%%
\begingroup%
\makeatletter%
\begin{pgfpicture}%
\pgfpathrectangle{\pgfpointorigin}{\pgfqpoint{5.785156in}{3.835212in}}%
\pgfusepath{use as bounding box, clip}%
\begin{pgfscope}%
\pgfsetbuttcap%
\pgfsetmiterjoin%
\definecolor{currentfill}{rgb}{1.000000,1.000000,1.000000}%
\pgfsetfillcolor{currentfill}%
\pgfsetlinewidth{0.000000pt}%
\definecolor{currentstroke}{rgb}{1.000000,1.000000,1.000000}%
\pgfsetstrokecolor{currentstroke}%
\pgfsetdash{}{0pt}%
\pgfpathmoveto{\pgfqpoint{0.000000in}{0.000000in}}%
\pgfpathlineto{\pgfqpoint{5.785156in}{0.000000in}}%
\pgfpathlineto{\pgfqpoint{5.785156in}{3.835212in}}%
\pgfpathlineto{\pgfqpoint{0.000000in}{3.835212in}}%
\pgfpathlineto{\pgfqpoint{0.000000in}{0.000000in}}%
\pgfpathclose%
\pgfusepath{fill}%
\end{pgfscope}%
\begin{pgfscope}%
\pgfsetbuttcap%
\pgfsetmiterjoin%
\definecolor{currentfill}{rgb}{1.000000,1.000000,1.000000}%
\pgfsetfillcolor{currentfill}%
\pgfsetlinewidth{0.000000pt}%
\definecolor{currentstroke}{rgb}{0.000000,0.000000,0.000000}%
\pgfsetstrokecolor{currentstroke}%
\pgfsetstrokeopacity{0.000000}%
\pgfsetdash{}{0pt}%
\pgfpathmoveto{\pgfqpoint{0.609415in}{2.347992in}}%
\pgfpathlineto{\pgfqpoint{2.829105in}{2.347992in}}%
\pgfpathlineto{\pgfqpoint{2.829105in}{3.574193in}}%
\pgfpathlineto{\pgfqpoint{0.609415in}{3.574193in}}%
\pgfpathlineto{\pgfqpoint{0.609415in}{2.347992in}}%
\pgfpathclose%
\pgfusepath{fill}%
\end{pgfscope}%
\begin{pgfscope}%
\pgfsetbuttcap%
\pgfsetroundjoin%
\definecolor{currentfill}{rgb}{0.000000,0.000000,0.000000}%
\pgfsetfillcolor{currentfill}%
\pgfsetlinewidth{0.501875pt}%
\definecolor{currentstroke}{rgb}{0.000000,0.000000,0.000000}%
\pgfsetstrokecolor{currentstroke}%
\pgfsetdash{}{0pt}%
\pgfsys@defobject{currentmarker}{\pgfqpoint{0.000000in}{0.000000in}}{\pgfqpoint{0.000000in}{0.041667in}}{%
\pgfpathmoveto{\pgfqpoint{0.000000in}{0.000000in}}%
\pgfpathlineto{\pgfqpoint{0.000000in}{0.041667in}}%
\pgfusepath{stroke,fill}%
}%
\begin{pgfscope}%
\pgfsys@transformshift{0.609415in}{2.347992in}%
\pgfsys@useobject{currentmarker}{}%
\end{pgfscope}%
\end{pgfscope}%
\begin{pgfscope}%
\pgfsetbuttcap%
\pgfsetroundjoin%
\definecolor{currentfill}{rgb}{0.000000,0.000000,0.000000}%
\pgfsetfillcolor{currentfill}%
\pgfsetlinewidth{0.501875pt}%
\definecolor{currentstroke}{rgb}{0.000000,0.000000,0.000000}%
\pgfsetstrokecolor{currentstroke}%
\pgfsetdash{}{0pt}%
\pgfsys@defobject{currentmarker}{\pgfqpoint{0.000000in}{-0.041667in}}{\pgfqpoint{0.000000in}{0.000000in}}{%
\pgfpathmoveto{\pgfqpoint{0.000000in}{0.000000in}}%
\pgfpathlineto{\pgfqpoint{0.000000in}{-0.041667in}}%
\pgfusepath{stroke,fill}%
}%
\begin{pgfscope}%
\pgfsys@transformshift{0.609415in}{3.574193in}%
\pgfsys@useobject{currentmarker}{}%
\end{pgfscope}%
\end{pgfscope}%
\begin{pgfscope}%
\definecolor{textcolor}{rgb}{0.000000,0.000000,0.000000}%
\pgfsetstrokecolor{textcolor}%
\pgfsetfillcolor{textcolor}%
\pgftext[x=0.609415in,y=2.299381in,,top]{\color{textcolor}\rmfamily\fontsize{10.000000}{12.000000}\selectfont \(\displaystyle {0}\)}%
\end{pgfscope}%
\begin{pgfscope}%
\pgfsetbuttcap%
\pgfsetroundjoin%
\definecolor{currentfill}{rgb}{0.000000,0.000000,0.000000}%
\pgfsetfillcolor{currentfill}%
\pgfsetlinewidth{0.501875pt}%
\definecolor{currentstroke}{rgb}{0.000000,0.000000,0.000000}%
\pgfsetstrokecolor{currentstroke}%
\pgfsetdash{}{0pt}%
\pgfsys@defobject{currentmarker}{\pgfqpoint{0.000000in}{0.000000in}}{\pgfqpoint{0.000000in}{0.041667in}}{%
\pgfpathmoveto{\pgfqpoint{0.000000in}{0.000000in}}%
\pgfpathlineto{\pgfqpoint{0.000000in}{0.041667in}}%
\pgfusepath{stroke,fill}%
}%
\begin{pgfscope}%
\pgfsys@transformshift{1.044648in}{2.347992in}%
\pgfsys@useobject{currentmarker}{}%
\end{pgfscope}%
\end{pgfscope}%
\begin{pgfscope}%
\pgfsetbuttcap%
\pgfsetroundjoin%
\definecolor{currentfill}{rgb}{0.000000,0.000000,0.000000}%
\pgfsetfillcolor{currentfill}%
\pgfsetlinewidth{0.501875pt}%
\definecolor{currentstroke}{rgb}{0.000000,0.000000,0.000000}%
\pgfsetstrokecolor{currentstroke}%
\pgfsetdash{}{0pt}%
\pgfsys@defobject{currentmarker}{\pgfqpoint{0.000000in}{-0.041667in}}{\pgfqpoint{0.000000in}{0.000000in}}{%
\pgfpathmoveto{\pgfqpoint{0.000000in}{0.000000in}}%
\pgfpathlineto{\pgfqpoint{0.000000in}{-0.041667in}}%
\pgfusepath{stroke,fill}%
}%
\begin{pgfscope}%
\pgfsys@transformshift{1.044648in}{3.574193in}%
\pgfsys@useobject{currentmarker}{}%
\end{pgfscope}%
\end{pgfscope}%
\begin{pgfscope}%
\definecolor{textcolor}{rgb}{0.000000,0.000000,0.000000}%
\pgfsetstrokecolor{textcolor}%
\pgfsetfillcolor{textcolor}%
\pgftext[x=1.044648in,y=2.299381in,,top]{\color{textcolor}\rmfamily\fontsize{10.000000}{12.000000}\selectfont \(\displaystyle {20}\)}%
\end{pgfscope}%
\begin{pgfscope}%
\pgfsetbuttcap%
\pgfsetroundjoin%
\definecolor{currentfill}{rgb}{0.000000,0.000000,0.000000}%
\pgfsetfillcolor{currentfill}%
\pgfsetlinewidth{0.501875pt}%
\definecolor{currentstroke}{rgb}{0.000000,0.000000,0.000000}%
\pgfsetstrokecolor{currentstroke}%
\pgfsetdash{}{0pt}%
\pgfsys@defobject{currentmarker}{\pgfqpoint{0.000000in}{0.000000in}}{\pgfqpoint{0.000000in}{0.041667in}}{%
\pgfpathmoveto{\pgfqpoint{0.000000in}{0.000000in}}%
\pgfpathlineto{\pgfqpoint{0.000000in}{0.041667in}}%
\pgfusepath{stroke,fill}%
}%
\begin{pgfscope}%
\pgfsys@transformshift{1.479881in}{2.347992in}%
\pgfsys@useobject{currentmarker}{}%
\end{pgfscope}%
\end{pgfscope}%
\begin{pgfscope}%
\pgfsetbuttcap%
\pgfsetroundjoin%
\definecolor{currentfill}{rgb}{0.000000,0.000000,0.000000}%
\pgfsetfillcolor{currentfill}%
\pgfsetlinewidth{0.501875pt}%
\definecolor{currentstroke}{rgb}{0.000000,0.000000,0.000000}%
\pgfsetstrokecolor{currentstroke}%
\pgfsetdash{}{0pt}%
\pgfsys@defobject{currentmarker}{\pgfqpoint{0.000000in}{-0.041667in}}{\pgfqpoint{0.000000in}{0.000000in}}{%
\pgfpathmoveto{\pgfqpoint{0.000000in}{0.000000in}}%
\pgfpathlineto{\pgfqpoint{0.000000in}{-0.041667in}}%
\pgfusepath{stroke,fill}%
}%
\begin{pgfscope}%
\pgfsys@transformshift{1.479881in}{3.574193in}%
\pgfsys@useobject{currentmarker}{}%
\end{pgfscope}%
\end{pgfscope}%
\begin{pgfscope}%
\definecolor{textcolor}{rgb}{0.000000,0.000000,0.000000}%
\pgfsetstrokecolor{textcolor}%
\pgfsetfillcolor{textcolor}%
\pgftext[x=1.479881in,y=2.299381in,,top]{\color{textcolor}\rmfamily\fontsize{10.000000}{12.000000}\selectfont \(\displaystyle {40}\)}%
\end{pgfscope}%
\begin{pgfscope}%
\pgfsetbuttcap%
\pgfsetroundjoin%
\definecolor{currentfill}{rgb}{0.000000,0.000000,0.000000}%
\pgfsetfillcolor{currentfill}%
\pgfsetlinewidth{0.501875pt}%
\definecolor{currentstroke}{rgb}{0.000000,0.000000,0.000000}%
\pgfsetstrokecolor{currentstroke}%
\pgfsetdash{}{0pt}%
\pgfsys@defobject{currentmarker}{\pgfqpoint{0.000000in}{0.000000in}}{\pgfqpoint{0.000000in}{0.041667in}}{%
\pgfpathmoveto{\pgfqpoint{0.000000in}{0.000000in}}%
\pgfpathlineto{\pgfqpoint{0.000000in}{0.041667in}}%
\pgfusepath{stroke,fill}%
}%
\begin{pgfscope}%
\pgfsys@transformshift{1.915115in}{2.347992in}%
\pgfsys@useobject{currentmarker}{}%
\end{pgfscope}%
\end{pgfscope}%
\begin{pgfscope}%
\pgfsetbuttcap%
\pgfsetroundjoin%
\definecolor{currentfill}{rgb}{0.000000,0.000000,0.000000}%
\pgfsetfillcolor{currentfill}%
\pgfsetlinewidth{0.501875pt}%
\definecolor{currentstroke}{rgb}{0.000000,0.000000,0.000000}%
\pgfsetstrokecolor{currentstroke}%
\pgfsetdash{}{0pt}%
\pgfsys@defobject{currentmarker}{\pgfqpoint{0.000000in}{-0.041667in}}{\pgfqpoint{0.000000in}{0.000000in}}{%
\pgfpathmoveto{\pgfqpoint{0.000000in}{0.000000in}}%
\pgfpathlineto{\pgfqpoint{0.000000in}{-0.041667in}}%
\pgfusepath{stroke,fill}%
}%
\begin{pgfscope}%
\pgfsys@transformshift{1.915115in}{3.574193in}%
\pgfsys@useobject{currentmarker}{}%
\end{pgfscope}%
\end{pgfscope}%
\begin{pgfscope}%
\definecolor{textcolor}{rgb}{0.000000,0.000000,0.000000}%
\pgfsetstrokecolor{textcolor}%
\pgfsetfillcolor{textcolor}%
\pgftext[x=1.915115in,y=2.299381in,,top]{\color{textcolor}\rmfamily\fontsize{10.000000}{12.000000}\selectfont \(\displaystyle {60}\)}%
\end{pgfscope}%
\begin{pgfscope}%
\pgfsetbuttcap%
\pgfsetroundjoin%
\definecolor{currentfill}{rgb}{0.000000,0.000000,0.000000}%
\pgfsetfillcolor{currentfill}%
\pgfsetlinewidth{0.501875pt}%
\definecolor{currentstroke}{rgb}{0.000000,0.000000,0.000000}%
\pgfsetstrokecolor{currentstroke}%
\pgfsetdash{}{0pt}%
\pgfsys@defobject{currentmarker}{\pgfqpoint{0.000000in}{0.000000in}}{\pgfqpoint{0.000000in}{0.041667in}}{%
\pgfpathmoveto{\pgfqpoint{0.000000in}{0.000000in}}%
\pgfpathlineto{\pgfqpoint{0.000000in}{0.041667in}}%
\pgfusepath{stroke,fill}%
}%
\begin{pgfscope}%
\pgfsys@transformshift{2.350348in}{2.347992in}%
\pgfsys@useobject{currentmarker}{}%
\end{pgfscope}%
\end{pgfscope}%
\begin{pgfscope}%
\pgfsetbuttcap%
\pgfsetroundjoin%
\definecolor{currentfill}{rgb}{0.000000,0.000000,0.000000}%
\pgfsetfillcolor{currentfill}%
\pgfsetlinewidth{0.501875pt}%
\definecolor{currentstroke}{rgb}{0.000000,0.000000,0.000000}%
\pgfsetstrokecolor{currentstroke}%
\pgfsetdash{}{0pt}%
\pgfsys@defobject{currentmarker}{\pgfqpoint{0.000000in}{-0.041667in}}{\pgfqpoint{0.000000in}{0.000000in}}{%
\pgfpathmoveto{\pgfqpoint{0.000000in}{0.000000in}}%
\pgfpathlineto{\pgfqpoint{0.000000in}{-0.041667in}}%
\pgfusepath{stroke,fill}%
}%
\begin{pgfscope}%
\pgfsys@transformshift{2.350348in}{3.574193in}%
\pgfsys@useobject{currentmarker}{}%
\end{pgfscope}%
\end{pgfscope}%
\begin{pgfscope}%
\definecolor{textcolor}{rgb}{0.000000,0.000000,0.000000}%
\pgfsetstrokecolor{textcolor}%
\pgfsetfillcolor{textcolor}%
\pgftext[x=2.350348in,y=2.299381in,,top]{\color{textcolor}\rmfamily\fontsize{10.000000}{12.000000}\selectfont \(\displaystyle {80}\)}%
\end{pgfscope}%
\begin{pgfscope}%
\pgfsetbuttcap%
\pgfsetroundjoin%
\definecolor{currentfill}{rgb}{0.000000,0.000000,0.000000}%
\pgfsetfillcolor{currentfill}%
\pgfsetlinewidth{0.501875pt}%
\definecolor{currentstroke}{rgb}{0.000000,0.000000,0.000000}%
\pgfsetstrokecolor{currentstroke}%
\pgfsetdash{}{0pt}%
\pgfsys@defobject{currentmarker}{\pgfqpoint{0.000000in}{0.000000in}}{\pgfqpoint{0.000000in}{0.041667in}}{%
\pgfpathmoveto{\pgfqpoint{0.000000in}{0.000000in}}%
\pgfpathlineto{\pgfqpoint{0.000000in}{0.041667in}}%
\pgfusepath{stroke,fill}%
}%
\begin{pgfscope}%
\pgfsys@transformshift{2.785581in}{2.347992in}%
\pgfsys@useobject{currentmarker}{}%
\end{pgfscope}%
\end{pgfscope}%
\begin{pgfscope}%
\pgfsetbuttcap%
\pgfsetroundjoin%
\definecolor{currentfill}{rgb}{0.000000,0.000000,0.000000}%
\pgfsetfillcolor{currentfill}%
\pgfsetlinewidth{0.501875pt}%
\definecolor{currentstroke}{rgb}{0.000000,0.000000,0.000000}%
\pgfsetstrokecolor{currentstroke}%
\pgfsetdash{}{0pt}%
\pgfsys@defobject{currentmarker}{\pgfqpoint{0.000000in}{-0.041667in}}{\pgfqpoint{0.000000in}{0.000000in}}{%
\pgfpathmoveto{\pgfqpoint{0.000000in}{0.000000in}}%
\pgfpathlineto{\pgfqpoint{0.000000in}{-0.041667in}}%
\pgfusepath{stroke,fill}%
}%
\begin{pgfscope}%
\pgfsys@transformshift{2.785581in}{3.574193in}%
\pgfsys@useobject{currentmarker}{}%
\end{pgfscope}%
\end{pgfscope}%
\begin{pgfscope}%
\definecolor{textcolor}{rgb}{0.000000,0.000000,0.000000}%
\pgfsetstrokecolor{textcolor}%
\pgfsetfillcolor{textcolor}%
\pgftext[x=2.785581in,y=2.299381in,,top]{\color{textcolor}\rmfamily\fontsize{10.000000}{12.000000}\selectfont \(\displaystyle {100}\)}%
\end{pgfscope}%
\begin{pgfscope}%
\pgfsetbuttcap%
\pgfsetroundjoin%
\definecolor{currentfill}{rgb}{0.000000,0.000000,0.000000}%
\pgfsetfillcolor{currentfill}%
\pgfsetlinewidth{0.501875pt}%
\definecolor{currentstroke}{rgb}{0.000000,0.000000,0.000000}%
\pgfsetstrokecolor{currentstroke}%
\pgfsetdash{}{0pt}%
\pgfsys@defobject{currentmarker}{\pgfqpoint{0.000000in}{0.000000in}}{\pgfqpoint{0.000000in}{0.020833in}}{%
\pgfpathmoveto{\pgfqpoint{0.000000in}{0.000000in}}%
\pgfpathlineto{\pgfqpoint{0.000000in}{0.020833in}}%
\pgfusepath{stroke,fill}%
}%
\begin{pgfscope}%
\pgfsys@transformshift{0.718223in}{2.347992in}%
\pgfsys@useobject{currentmarker}{}%
\end{pgfscope}%
\end{pgfscope}%
\begin{pgfscope}%
\pgfsetbuttcap%
\pgfsetroundjoin%
\definecolor{currentfill}{rgb}{0.000000,0.000000,0.000000}%
\pgfsetfillcolor{currentfill}%
\pgfsetlinewidth{0.501875pt}%
\definecolor{currentstroke}{rgb}{0.000000,0.000000,0.000000}%
\pgfsetstrokecolor{currentstroke}%
\pgfsetdash{}{0pt}%
\pgfsys@defobject{currentmarker}{\pgfqpoint{0.000000in}{-0.020833in}}{\pgfqpoint{0.000000in}{0.000000in}}{%
\pgfpathmoveto{\pgfqpoint{0.000000in}{0.000000in}}%
\pgfpathlineto{\pgfqpoint{0.000000in}{-0.020833in}}%
\pgfusepath{stroke,fill}%
}%
\begin{pgfscope}%
\pgfsys@transformshift{0.718223in}{3.574193in}%
\pgfsys@useobject{currentmarker}{}%
\end{pgfscope}%
\end{pgfscope}%
\begin{pgfscope}%
\pgfsetbuttcap%
\pgfsetroundjoin%
\definecolor{currentfill}{rgb}{0.000000,0.000000,0.000000}%
\pgfsetfillcolor{currentfill}%
\pgfsetlinewidth{0.501875pt}%
\definecolor{currentstroke}{rgb}{0.000000,0.000000,0.000000}%
\pgfsetstrokecolor{currentstroke}%
\pgfsetdash{}{0pt}%
\pgfsys@defobject{currentmarker}{\pgfqpoint{0.000000in}{0.000000in}}{\pgfqpoint{0.000000in}{0.020833in}}{%
\pgfpathmoveto{\pgfqpoint{0.000000in}{0.000000in}}%
\pgfpathlineto{\pgfqpoint{0.000000in}{0.020833in}}%
\pgfusepath{stroke,fill}%
}%
\begin{pgfscope}%
\pgfsys@transformshift{0.827031in}{2.347992in}%
\pgfsys@useobject{currentmarker}{}%
\end{pgfscope}%
\end{pgfscope}%
\begin{pgfscope}%
\pgfsetbuttcap%
\pgfsetroundjoin%
\definecolor{currentfill}{rgb}{0.000000,0.000000,0.000000}%
\pgfsetfillcolor{currentfill}%
\pgfsetlinewidth{0.501875pt}%
\definecolor{currentstroke}{rgb}{0.000000,0.000000,0.000000}%
\pgfsetstrokecolor{currentstroke}%
\pgfsetdash{}{0pt}%
\pgfsys@defobject{currentmarker}{\pgfqpoint{0.000000in}{-0.020833in}}{\pgfqpoint{0.000000in}{0.000000in}}{%
\pgfpathmoveto{\pgfqpoint{0.000000in}{0.000000in}}%
\pgfpathlineto{\pgfqpoint{0.000000in}{-0.020833in}}%
\pgfusepath{stroke,fill}%
}%
\begin{pgfscope}%
\pgfsys@transformshift{0.827031in}{3.574193in}%
\pgfsys@useobject{currentmarker}{}%
\end{pgfscope}%
\end{pgfscope}%
\begin{pgfscope}%
\pgfsetbuttcap%
\pgfsetroundjoin%
\definecolor{currentfill}{rgb}{0.000000,0.000000,0.000000}%
\pgfsetfillcolor{currentfill}%
\pgfsetlinewidth{0.501875pt}%
\definecolor{currentstroke}{rgb}{0.000000,0.000000,0.000000}%
\pgfsetstrokecolor{currentstroke}%
\pgfsetdash{}{0pt}%
\pgfsys@defobject{currentmarker}{\pgfqpoint{0.000000in}{0.000000in}}{\pgfqpoint{0.000000in}{0.020833in}}{%
\pgfpathmoveto{\pgfqpoint{0.000000in}{0.000000in}}%
\pgfpathlineto{\pgfqpoint{0.000000in}{0.020833in}}%
\pgfusepath{stroke,fill}%
}%
\begin{pgfscope}%
\pgfsys@transformshift{0.935840in}{2.347992in}%
\pgfsys@useobject{currentmarker}{}%
\end{pgfscope}%
\end{pgfscope}%
\begin{pgfscope}%
\pgfsetbuttcap%
\pgfsetroundjoin%
\definecolor{currentfill}{rgb}{0.000000,0.000000,0.000000}%
\pgfsetfillcolor{currentfill}%
\pgfsetlinewidth{0.501875pt}%
\definecolor{currentstroke}{rgb}{0.000000,0.000000,0.000000}%
\pgfsetstrokecolor{currentstroke}%
\pgfsetdash{}{0pt}%
\pgfsys@defobject{currentmarker}{\pgfqpoint{0.000000in}{-0.020833in}}{\pgfqpoint{0.000000in}{0.000000in}}{%
\pgfpathmoveto{\pgfqpoint{0.000000in}{0.000000in}}%
\pgfpathlineto{\pgfqpoint{0.000000in}{-0.020833in}}%
\pgfusepath{stroke,fill}%
}%
\begin{pgfscope}%
\pgfsys@transformshift{0.935840in}{3.574193in}%
\pgfsys@useobject{currentmarker}{}%
\end{pgfscope}%
\end{pgfscope}%
\begin{pgfscope}%
\pgfsetbuttcap%
\pgfsetroundjoin%
\definecolor{currentfill}{rgb}{0.000000,0.000000,0.000000}%
\pgfsetfillcolor{currentfill}%
\pgfsetlinewidth{0.501875pt}%
\definecolor{currentstroke}{rgb}{0.000000,0.000000,0.000000}%
\pgfsetstrokecolor{currentstroke}%
\pgfsetdash{}{0pt}%
\pgfsys@defobject{currentmarker}{\pgfqpoint{0.000000in}{0.000000in}}{\pgfqpoint{0.000000in}{0.020833in}}{%
\pgfpathmoveto{\pgfqpoint{0.000000in}{0.000000in}}%
\pgfpathlineto{\pgfqpoint{0.000000in}{0.020833in}}%
\pgfusepath{stroke,fill}%
}%
\begin{pgfscope}%
\pgfsys@transformshift{1.153456in}{2.347992in}%
\pgfsys@useobject{currentmarker}{}%
\end{pgfscope}%
\end{pgfscope}%
\begin{pgfscope}%
\pgfsetbuttcap%
\pgfsetroundjoin%
\definecolor{currentfill}{rgb}{0.000000,0.000000,0.000000}%
\pgfsetfillcolor{currentfill}%
\pgfsetlinewidth{0.501875pt}%
\definecolor{currentstroke}{rgb}{0.000000,0.000000,0.000000}%
\pgfsetstrokecolor{currentstroke}%
\pgfsetdash{}{0pt}%
\pgfsys@defobject{currentmarker}{\pgfqpoint{0.000000in}{-0.020833in}}{\pgfqpoint{0.000000in}{0.000000in}}{%
\pgfpathmoveto{\pgfqpoint{0.000000in}{0.000000in}}%
\pgfpathlineto{\pgfqpoint{0.000000in}{-0.020833in}}%
\pgfusepath{stroke,fill}%
}%
\begin{pgfscope}%
\pgfsys@transformshift{1.153456in}{3.574193in}%
\pgfsys@useobject{currentmarker}{}%
\end{pgfscope}%
\end{pgfscope}%
\begin{pgfscope}%
\pgfsetbuttcap%
\pgfsetroundjoin%
\definecolor{currentfill}{rgb}{0.000000,0.000000,0.000000}%
\pgfsetfillcolor{currentfill}%
\pgfsetlinewidth{0.501875pt}%
\definecolor{currentstroke}{rgb}{0.000000,0.000000,0.000000}%
\pgfsetstrokecolor{currentstroke}%
\pgfsetdash{}{0pt}%
\pgfsys@defobject{currentmarker}{\pgfqpoint{0.000000in}{0.000000in}}{\pgfqpoint{0.000000in}{0.020833in}}{%
\pgfpathmoveto{\pgfqpoint{0.000000in}{0.000000in}}%
\pgfpathlineto{\pgfqpoint{0.000000in}{0.020833in}}%
\pgfusepath{stroke,fill}%
}%
\begin{pgfscope}%
\pgfsys@transformshift{1.262265in}{2.347992in}%
\pgfsys@useobject{currentmarker}{}%
\end{pgfscope}%
\end{pgfscope}%
\begin{pgfscope}%
\pgfsetbuttcap%
\pgfsetroundjoin%
\definecolor{currentfill}{rgb}{0.000000,0.000000,0.000000}%
\pgfsetfillcolor{currentfill}%
\pgfsetlinewidth{0.501875pt}%
\definecolor{currentstroke}{rgb}{0.000000,0.000000,0.000000}%
\pgfsetstrokecolor{currentstroke}%
\pgfsetdash{}{0pt}%
\pgfsys@defobject{currentmarker}{\pgfqpoint{0.000000in}{-0.020833in}}{\pgfqpoint{0.000000in}{0.000000in}}{%
\pgfpathmoveto{\pgfqpoint{0.000000in}{0.000000in}}%
\pgfpathlineto{\pgfqpoint{0.000000in}{-0.020833in}}%
\pgfusepath{stroke,fill}%
}%
\begin{pgfscope}%
\pgfsys@transformshift{1.262265in}{3.574193in}%
\pgfsys@useobject{currentmarker}{}%
\end{pgfscope}%
\end{pgfscope}%
\begin{pgfscope}%
\pgfsetbuttcap%
\pgfsetroundjoin%
\definecolor{currentfill}{rgb}{0.000000,0.000000,0.000000}%
\pgfsetfillcolor{currentfill}%
\pgfsetlinewidth{0.501875pt}%
\definecolor{currentstroke}{rgb}{0.000000,0.000000,0.000000}%
\pgfsetstrokecolor{currentstroke}%
\pgfsetdash{}{0pt}%
\pgfsys@defobject{currentmarker}{\pgfqpoint{0.000000in}{0.000000in}}{\pgfqpoint{0.000000in}{0.020833in}}{%
\pgfpathmoveto{\pgfqpoint{0.000000in}{0.000000in}}%
\pgfpathlineto{\pgfqpoint{0.000000in}{0.020833in}}%
\pgfusepath{stroke,fill}%
}%
\begin{pgfscope}%
\pgfsys@transformshift{1.371073in}{2.347992in}%
\pgfsys@useobject{currentmarker}{}%
\end{pgfscope}%
\end{pgfscope}%
\begin{pgfscope}%
\pgfsetbuttcap%
\pgfsetroundjoin%
\definecolor{currentfill}{rgb}{0.000000,0.000000,0.000000}%
\pgfsetfillcolor{currentfill}%
\pgfsetlinewidth{0.501875pt}%
\definecolor{currentstroke}{rgb}{0.000000,0.000000,0.000000}%
\pgfsetstrokecolor{currentstroke}%
\pgfsetdash{}{0pt}%
\pgfsys@defobject{currentmarker}{\pgfqpoint{0.000000in}{-0.020833in}}{\pgfqpoint{0.000000in}{0.000000in}}{%
\pgfpathmoveto{\pgfqpoint{0.000000in}{0.000000in}}%
\pgfpathlineto{\pgfqpoint{0.000000in}{-0.020833in}}%
\pgfusepath{stroke,fill}%
}%
\begin{pgfscope}%
\pgfsys@transformshift{1.371073in}{3.574193in}%
\pgfsys@useobject{currentmarker}{}%
\end{pgfscope}%
\end{pgfscope}%
\begin{pgfscope}%
\pgfsetbuttcap%
\pgfsetroundjoin%
\definecolor{currentfill}{rgb}{0.000000,0.000000,0.000000}%
\pgfsetfillcolor{currentfill}%
\pgfsetlinewidth{0.501875pt}%
\definecolor{currentstroke}{rgb}{0.000000,0.000000,0.000000}%
\pgfsetstrokecolor{currentstroke}%
\pgfsetdash{}{0pt}%
\pgfsys@defobject{currentmarker}{\pgfqpoint{0.000000in}{0.000000in}}{\pgfqpoint{0.000000in}{0.020833in}}{%
\pgfpathmoveto{\pgfqpoint{0.000000in}{0.000000in}}%
\pgfpathlineto{\pgfqpoint{0.000000in}{0.020833in}}%
\pgfusepath{stroke,fill}%
}%
\begin{pgfscope}%
\pgfsys@transformshift{1.588690in}{2.347992in}%
\pgfsys@useobject{currentmarker}{}%
\end{pgfscope}%
\end{pgfscope}%
\begin{pgfscope}%
\pgfsetbuttcap%
\pgfsetroundjoin%
\definecolor{currentfill}{rgb}{0.000000,0.000000,0.000000}%
\pgfsetfillcolor{currentfill}%
\pgfsetlinewidth{0.501875pt}%
\definecolor{currentstroke}{rgb}{0.000000,0.000000,0.000000}%
\pgfsetstrokecolor{currentstroke}%
\pgfsetdash{}{0pt}%
\pgfsys@defobject{currentmarker}{\pgfqpoint{0.000000in}{-0.020833in}}{\pgfqpoint{0.000000in}{0.000000in}}{%
\pgfpathmoveto{\pgfqpoint{0.000000in}{0.000000in}}%
\pgfpathlineto{\pgfqpoint{0.000000in}{-0.020833in}}%
\pgfusepath{stroke,fill}%
}%
\begin{pgfscope}%
\pgfsys@transformshift{1.588690in}{3.574193in}%
\pgfsys@useobject{currentmarker}{}%
\end{pgfscope}%
\end{pgfscope}%
\begin{pgfscope}%
\pgfsetbuttcap%
\pgfsetroundjoin%
\definecolor{currentfill}{rgb}{0.000000,0.000000,0.000000}%
\pgfsetfillcolor{currentfill}%
\pgfsetlinewidth{0.501875pt}%
\definecolor{currentstroke}{rgb}{0.000000,0.000000,0.000000}%
\pgfsetstrokecolor{currentstroke}%
\pgfsetdash{}{0pt}%
\pgfsys@defobject{currentmarker}{\pgfqpoint{0.000000in}{0.000000in}}{\pgfqpoint{0.000000in}{0.020833in}}{%
\pgfpathmoveto{\pgfqpoint{0.000000in}{0.000000in}}%
\pgfpathlineto{\pgfqpoint{0.000000in}{0.020833in}}%
\pgfusepath{stroke,fill}%
}%
\begin{pgfscope}%
\pgfsys@transformshift{1.697498in}{2.347992in}%
\pgfsys@useobject{currentmarker}{}%
\end{pgfscope}%
\end{pgfscope}%
\begin{pgfscope}%
\pgfsetbuttcap%
\pgfsetroundjoin%
\definecolor{currentfill}{rgb}{0.000000,0.000000,0.000000}%
\pgfsetfillcolor{currentfill}%
\pgfsetlinewidth{0.501875pt}%
\definecolor{currentstroke}{rgb}{0.000000,0.000000,0.000000}%
\pgfsetstrokecolor{currentstroke}%
\pgfsetdash{}{0pt}%
\pgfsys@defobject{currentmarker}{\pgfqpoint{0.000000in}{-0.020833in}}{\pgfqpoint{0.000000in}{0.000000in}}{%
\pgfpathmoveto{\pgfqpoint{0.000000in}{0.000000in}}%
\pgfpathlineto{\pgfqpoint{0.000000in}{-0.020833in}}%
\pgfusepath{stroke,fill}%
}%
\begin{pgfscope}%
\pgfsys@transformshift{1.697498in}{3.574193in}%
\pgfsys@useobject{currentmarker}{}%
\end{pgfscope}%
\end{pgfscope}%
\begin{pgfscope}%
\pgfsetbuttcap%
\pgfsetroundjoin%
\definecolor{currentfill}{rgb}{0.000000,0.000000,0.000000}%
\pgfsetfillcolor{currentfill}%
\pgfsetlinewidth{0.501875pt}%
\definecolor{currentstroke}{rgb}{0.000000,0.000000,0.000000}%
\pgfsetstrokecolor{currentstroke}%
\pgfsetdash{}{0pt}%
\pgfsys@defobject{currentmarker}{\pgfqpoint{0.000000in}{0.000000in}}{\pgfqpoint{0.000000in}{0.020833in}}{%
\pgfpathmoveto{\pgfqpoint{0.000000in}{0.000000in}}%
\pgfpathlineto{\pgfqpoint{0.000000in}{0.020833in}}%
\pgfusepath{stroke,fill}%
}%
\begin{pgfscope}%
\pgfsys@transformshift{1.806306in}{2.347992in}%
\pgfsys@useobject{currentmarker}{}%
\end{pgfscope}%
\end{pgfscope}%
\begin{pgfscope}%
\pgfsetbuttcap%
\pgfsetroundjoin%
\definecolor{currentfill}{rgb}{0.000000,0.000000,0.000000}%
\pgfsetfillcolor{currentfill}%
\pgfsetlinewidth{0.501875pt}%
\definecolor{currentstroke}{rgb}{0.000000,0.000000,0.000000}%
\pgfsetstrokecolor{currentstroke}%
\pgfsetdash{}{0pt}%
\pgfsys@defobject{currentmarker}{\pgfqpoint{0.000000in}{-0.020833in}}{\pgfqpoint{0.000000in}{0.000000in}}{%
\pgfpathmoveto{\pgfqpoint{0.000000in}{0.000000in}}%
\pgfpathlineto{\pgfqpoint{0.000000in}{-0.020833in}}%
\pgfusepath{stroke,fill}%
}%
\begin{pgfscope}%
\pgfsys@transformshift{1.806306in}{3.574193in}%
\pgfsys@useobject{currentmarker}{}%
\end{pgfscope}%
\end{pgfscope}%
\begin{pgfscope}%
\pgfsetbuttcap%
\pgfsetroundjoin%
\definecolor{currentfill}{rgb}{0.000000,0.000000,0.000000}%
\pgfsetfillcolor{currentfill}%
\pgfsetlinewidth{0.501875pt}%
\definecolor{currentstroke}{rgb}{0.000000,0.000000,0.000000}%
\pgfsetstrokecolor{currentstroke}%
\pgfsetdash{}{0pt}%
\pgfsys@defobject{currentmarker}{\pgfqpoint{0.000000in}{0.000000in}}{\pgfqpoint{0.000000in}{0.020833in}}{%
\pgfpathmoveto{\pgfqpoint{0.000000in}{0.000000in}}%
\pgfpathlineto{\pgfqpoint{0.000000in}{0.020833in}}%
\pgfusepath{stroke,fill}%
}%
\begin{pgfscope}%
\pgfsys@transformshift{2.023923in}{2.347992in}%
\pgfsys@useobject{currentmarker}{}%
\end{pgfscope}%
\end{pgfscope}%
\begin{pgfscope}%
\pgfsetbuttcap%
\pgfsetroundjoin%
\definecolor{currentfill}{rgb}{0.000000,0.000000,0.000000}%
\pgfsetfillcolor{currentfill}%
\pgfsetlinewidth{0.501875pt}%
\definecolor{currentstroke}{rgb}{0.000000,0.000000,0.000000}%
\pgfsetstrokecolor{currentstroke}%
\pgfsetdash{}{0pt}%
\pgfsys@defobject{currentmarker}{\pgfqpoint{0.000000in}{-0.020833in}}{\pgfqpoint{0.000000in}{0.000000in}}{%
\pgfpathmoveto{\pgfqpoint{0.000000in}{0.000000in}}%
\pgfpathlineto{\pgfqpoint{0.000000in}{-0.020833in}}%
\pgfusepath{stroke,fill}%
}%
\begin{pgfscope}%
\pgfsys@transformshift{2.023923in}{3.574193in}%
\pgfsys@useobject{currentmarker}{}%
\end{pgfscope}%
\end{pgfscope}%
\begin{pgfscope}%
\pgfsetbuttcap%
\pgfsetroundjoin%
\definecolor{currentfill}{rgb}{0.000000,0.000000,0.000000}%
\pgfsetfillcolor{currentfill}%
\pgfsetlinewidth{0.501875pt}%
\definecolor{currentstroke}{rgb}{0.000000,0.000000,0.000000}%
\pgfsetstrokecolor{currentstroke}%
\pgfsetdash{}{0pt}%
\pgfsys@defobject{currentmarker}{\pgfqpoint{0.000000in}{0.000000in}}{\pgfqpoint{0.000000in}{0.020833in}}{%
\pgfpathmoveto{\pgfqpoint{0.000000in}{0.000000in}}%
\pgfpathlineto{\pgfqpoint{0.000000in}{0.020833in}}%
\pgfusepath{stroke,fill}%
}%
\begin{pgfscope}%
\pgfsys@transformshift{2.132731in}{2.347992in}%
\pgfsys@useobject{currentmarker}{}%
\end{pgfscope}%
\end{pgfscope}%
\begin{pgfscope}%
\pgfsetbuttcap%
\pgfsetroundjoin%
\definecolor{currentfill}{rgb}{0.000000,0.000000,0.000000}%
\pgfsetfillcolor{currentfill}%
\pgfsetlinewidth{0.501875pt}%
\definecolor{currentstroke}{rgb}{0.000000,0.000000,0.000000}%
\pgfsetstrokecolor{currentstroke}%
\pgfsetdash{}{0pt}%
\pgfsys@defobject{currentmarker}{\pgfqpoint{0.000000in}{-0.020833in}}{\pgfqpoint{0.000000in}{0.000000in}}{%
\pgfpathmoveto{\pgfqpoint{0.000000in}{0.000000in}}%
\pgfpathlineto{\pgfqpoint{0.000000in}{-0.020833in}}%
\pgfusepath{stroke,fill}%
}%
\begin{pgfscope}%
\pgfsys@transformshift{2.132731in}{3.574193in}%
\pgfsys@useobject{currentmarker}{}%
\end{pgfscope}%
\end{pgfscope}%
\begin{pgfscope}%
\pgfsetbuttcap%
\pgfsetroundjoin%
\definecolor{currentfill}{rgb}{0.000000,0.000000,0.000000}%
\pgfsetfillcolor{currentfill}%
\pgfsetlinewidth{0.501875pt}%
\definecolor{currentstroke}{rgb}{0.000000,0.000000,0.000000}%
\pgfsetstrokecolor{currentstroke}%
\pgfsetdash{}{0pt}%
\pgfsys@defobject{currentmarker}{\pgfqpoint{0.000000in}{0.000000in}}{\pgfqpoint{0.000000in}{0.020833in}}{%
\pgfpathmoveto{\pgfqpoint{0.000000in}{0.000000in}}%
\pgfpathlineto{\pgfqpoint{0.000000in}{0.020833in}}%
\pgfusepath{stroke,fill}%
}%
\begin{pgfscope}%
\pgfsys@transformshift{2.241540in}{2.347992in}%
\pgfsys@useobject{currentmarker}{}%
\end{pgfscope}%
\end{pgfscope}%
\begin{pgfscope}%
\pgfsetbuttcap%
\pgfsetroundjoin%
\definecolor{currentfill}{rgb}{0.000000,0.000000,0.000000}%
\pgfsetfillcolor{currentfill}%
\pgfsetlinewidth{0.501875pt}%
\definecolor{currentstroke}{rgb}{0.000000,0.000000,0.000000}%
\pgfsetstrokecolor{currentstroke}%
\pgfsetdash{}{0pt}%
\pgfsys@defobject{currentmarker}{\pgfqpoint{0.000000in}{-0.020833in}}{\pgfqpoint{0.000000in}{0.000000in}}{%
\pgfpathmoveto{\pgfqpoint{0.000000in}{0.000000in}}%
\pgfpathlineto{\pgfqpoint{0.000000in}{-0.020833in}}%
\pgfusepath{stroke,fill}%
}%
\begin{pgfscope}%
\pgfsys@transformshift{2.241540in}{3.574193in}%
\pgfsys@useobject{currentmarker}{}%
\end{pgfscope}%
\end{pgfscope}%
\begin{pgfscope}%
\pgfsetbuttcap%
\pgfsetroundjoin%
\definecolor{currentfill}{rgb}{0.000000,0.000000,0.000000}%
\pgfsetfillcolor{currentfill}%
\pgfsetlinewidth{0.501875pt}%
\definecolor{currentstroke}{rgb}{0.000000,0.000000,0.000000}%
\pgfsetstrokecolor{currentstroke}%
\pgfsetdash{}{0pt}%
\pgfsys@defobject{currentmarker}{\pgfqpoint{0.000000in}{0.000000in}}{\pgfqpoint{0.000000in}{0.020833in}}{%
\pgfpathmoveto{\pgfqpoint{0.000000in}{0.000000in}}%
\pgfpathlineto{\pgfqpoint{0.000000in}{0.020833in}}%
\pgfusepath{stroke,fill}%
}%
\begin{pgfscope}%
\pgfsys@transformshift{2.459156in}{2.347992in}%
\pgfsys@useobject{currentmarker}{}%
\end{pgfscope}%
\end{pgfscope}%
\begin{pgfscope}%
\pgfsetbuttcap%
\pgfsetroundjoin%
\definecolor{currentfill}{rgb}{0.000000,0.000000,0.000000}%
\pgfsetfillcolor{currentfill}%
\pgfsetlinewidth{0.501875pt}%
\definecolor{currentstroke}{rgb}{0.000000,0.000000,0.000000}%
\pgfsetstrokecolor{currentstroke}%
\pgfsetdash{}{0pt}%
\pgfsys@defobject{currentmarker}{\pgfqpoint{0.000000in}{-0.020833in}}{\pgfqpoint{0.000000in}{0.000000in}}{%
\pgfpathmoveto{\pgfqpoint{0.000000in}{0.000000in}}%
\pgfpathlineto{\pgfqpoint{0.000000in}{-0.020833in}}%
\pgfusepath{stroke,fill}%
}%
\begin{pgfscope}%
\pgfsys@transformshift{2.459156in}{3.574193in}%
\pgfsys@useobject{currentmarker}{}%
\end{pgfscope}%
\end{pgfscope}%
\begin{pgfscope}%
\pgfsetbuttcap%
\pgfsetroundjoin%
\definecolor{currentfill}{rgb}{0.000000,0.000000,0.000000}%
\pgfsetfillcolor{currentfill}%
\pgfsetlinewidth{0.501875pt}%
\definecolor{currentstroke}{rgb}{0.000000,0.000000,0.000000}%
\pgfsetstrokecolor{currentstroke}%
\pgfsetdash{}{0pt}%
\pgfsys@defobject{currentmarker}{\pgfqpoint{0.000000in}{0.000000in}}{\pgfqpoint{0.000000in}{0.020833in}}{%
\pgfpathmoveto{\pgfqpoint{0.000000in}{0.000000in}}%
\pgfpathlineto{\pgfqpoint{0.000000in}{0.020833in}}%
\pgfusepath{stroke,fill}%
}%
\begin{pgfscope}%
\pgfsys@transformshift{2.567965in}{2.347992in}%
\pgfsys@useobject{currentmarker}{}%
\end{pgfscope}%
\end{pgfscope}%
\begin{pgfscope}%
\pgfsetbuttcap%
\pgfsetroundjoin%
\definecolor{currentfill}{rgb}{0.000000,0.000000,0.000000}%
\pgfsetfillcolor{currentfill}%
\pgfsetlinewidth{0.501875pt}%
\definecolor{currentstroke}{rgb}{0.000000,0.000000,0.000000}%
\pgfsetstrokecolor{currentstroke}%
\pgfsetdash{}{0pt}%
\pgfsys@defobject{currentmarker}{\pgfqpoint{0.000000in}{-0.020833in}}{\pgfqpoint{0.000000in}{0.000000in}}{%
\pgfpathmoveto{\pgfqpoint{0.000000in}{0.000000in}}%
\pgfpathlineto{\pgfqpoint{0.000000in}{-0.020833in}}%
\pgfusepath{stroke,fill}%
}%
\begin{pgfscope}%
\pgfsys@transformshift{2.567965in}{3.574193in}%
\pgfsys@useobject{currentmarker}{}%
\end{pgfscope}%
\end{pgfscope}%
\begin{pgfscope}%
\pgfsetbuttcap%
\pgfsetroundjoin%
\definecolor{currentfill}{rgb}{0.000000,0.000000,0.000000}%
\pgfsetfillcolor{currentfill}%
\pgfsetlinewidth{0.501875pt}%
\definecolor{currentstroke}{rgb}{0.000000,0.000000,0.000000}%
\pgfsetstrokecolor{currentstroke}%
\pgfsetdash{}{0pt}%
\pgfsys@defobject{currentmarker}{\pgfqpoint{0.000000in}{0.000000in}}{\pgfqpoint{0.000000in}{0.020833in}}{%
\pgfpathmoveto{\pgfqpoint{0.000000in}{0.000000in}}%
\pgfpathlineto{\pgfqpoint{0.000000in}{0.020833in}}%
\pgfusepath{stroke,fill}%
}%
\begin{pgfscope}%
\pgfsys@transformshift{2.676773in}{2.347992in}%
\pgfsys@useobject{currentmarker}{}%
\end{pgfscope}%
\end{pgfscope}%
\begin{pgfscope}%
\pgfsetbuttcap%
\pgfsetroundjoin%
\definecolor{currentfill}{rgb}{0.000000,0.000000,0.000000}%
\pgfsetfillcolor{currentfill}%
\pgfsetlinewidth{0.501875pt}%
\definecolor{currentstroke}{rgb}{0.000000,0.000000,0.000000}%
\pgfsetstrokecolor{currentstroke}%
\pgfsetdash{}{0pt}%
\pgfsys@defobject{currentmarker}{\pgfqpoint{0.000000in}{-0.020833in}}{\pgfqpoint{0.000000in}{0.000000in}}{%
\pgfpathmoveto{\pgfqpoint{0.000000in}{0.000000in}}%
\pgfpathlineto{\pgfqpoint{0.000000in}{-0.020833in}}%
\pgfusepath{stroke,fill}%
}%
\begin{pgfscope}%
\pgfsys@transformshift{2.676773in}{3.574193in}%
\pgfsys@useobject{currentmarker}{}%
\end{pgfscope}%
\end{pgfscope}%
\begin{pgfscope}%
\definecolor{textcolor}{rgb}{0.000000,0.000000,0.000000}%
\pgfsetstrokecolor{textcolor}%
\pgfsetfillcolor{textcolor}%
\pgftext[x=1.719260in,y=2.109413in,,top]{\color{textcolor}\rmfamily\fontsize{10.000000}{12.000000}\selectfont \(\displaystyle K\)}%
\end{pgfscope}%
\begin{pgfscope}%
\pgfsetbuttcap%
\pgfsetroundjoin%
\definecolor{currentfill}{rgb}{0.000000,0.000000,0.000000}%
\pgfsetfillcolor{currentfill}%
\pgfsetlinewidth{0.501875pt}%
\definecolor{currentstroke}{rgb}{0.000000,0.000000,0.000000}%
\pgfsetstrokecolor{currentstroke}%
\pgfsetdash{}{0pt}%
\pgfsys@defobject{currentmarker}{\pgfqpoint{0.000000in}{0.000000in}}{\pgfqpoint{0.041667in}{0.000000in}}{%
\pgfpathmoveto{\pgfqpoint{0.000000in}{0.000000in}}%
\pgfpathlineto{\pgfqpoint{0.041667in}{0.000000in}}%
\pgfusepath{stroke,fill}%
}%
\begin{pgfscope}%
\pgfsys@transformshift{0.609415in}{2.491490in}%
\pgfsys@useobject{currentmarker}{}%
\end{pgfscope}%
\end{pgfscope}%
\begin{pgfscope}%
\pgfsetbuttcap%
\pgfsetroundjoin%
\definecolor{currentfill}{rgb}{0.000000,0.000000,0.000000}%
\pgfsetfillcolor{currentfill}%
\pgfsetlinewidth{0.501875pt}%
\definecolor{currentstroke}{rgb}{0.000000,0.000000,0.000000}%
\pgfsetstrokecolor{currentstroke}%
\pgfsetdash{}{0pt}%
\pgfsys@defobject{currentmarker}{\pgfqpoint{-0.041667in}{0.000000in}}{\pgfqpoint{-0.000000in}{0.000000in}}{%
\pgfpathmoveto{\pgfqpoint{-0.000000in}{0.000000in}}%
\pgfpathlineto{\pgfqpoint{-0.041667in}{0.000000in}}%
\pgfusepath{stroke,fill}%
}%
\begin{pgfscope}%
\pgfsys@transformshift{2.829105in}{2.491490in}%
\pgfsys@useobject{currentmarker}{}%
\end{pgfscope}%
\end{pgfscope}%
\begin{pgfscope}%
\definecolor{textcolor}{rgb}{0.000000,0.000000,0.000000}%
\pgfsetstrokecolor{textcolor}%
\pgfsetfillcolor{textcolor}%
\pgftext[x=0.313889in, y=2.438728in, left, base]{\color{textcolor}\rmfamily\fontsize{10.000000}{12.000000}\selectfont \(\displaystyle {0.85}\)}%
\end{pgfscope}%
\begin{pgfscope}%
\pgfsetbuttcap%
\pgfsetroundjoin%
\definecolor{currentfill}{rgb}{0.000000,0.000000,0.000000}%
\pgfsetfillcolor{currentfill}%
\pgfsetlinewidth{0.501875pt}%
\definecolor{currentstroke}{rgb}{0.000000,0.000000,0.000000}%
\pgfsetstrokecolor{currentstroke}%
\pgfsetdash{}{0pt}%
\pgfsys@defobject{currentmarker}{\pgfqpoint{0.000000in}{0.000000in}}{\pgfqpoint{0.041667in}{0.000000in}}{%
\pgfpathmoveto{\pgfqpoint{0.000000in}{0.000000in}}%
\pgfpathlineto{\pgfqpoint{0.041667in}{0.000000in}}%
\pgfusepath{stroke,fill}%
}%
\begin{pgfscope}%
\pgfsys@transformshift{0.609415in}{2.833812in}%
\pgfsys@useobject{currentmarker}{}%
\end{pgfscope}%
\end{pgfscope}%
\begin{pgfscope}%
\pgfsetbuttcap%
\pgfsetroundjoin%
\definecolor{currentfill}{rgb}{0.000000,0.000000,0.000000}%
\pgfsetfillcolor{currentfill}%
\pgfsetlinewidth{0.501875pt}%
\definecolor{currentstroke}{rgb}{0.000000,0.000000,0.000000}%
\pgfsetstrokecolor{currentstroke}%
\pgfsetdash{}{0pt}%
\pgfsys@defobject{currentmarker}{\pgfqpoint{-0.041667in}{0.000000in}}{\pgfqpoint{-0.000000in}{0.000000in}}{%
\pgfpathmoveto{\pgfqpoint{-0.000000in}{0.000000in}}%
\pgfpathlineto{\pgfqpoint{-0.041667in}{0.000000in}}%
\pgfusepath{stroke,fill}%
}%
\begin{pgfscope}%
\pgfsys@transformshift{2.829105in}{2.833812in}%
\pgfsys@useobject{currentmarker}{}%
\end{pgfscope}%
\end{pgfscope}%
\begin{pgfscope}%
\definecolor{textcolor}{rgb}{0.000000,0.000000,0.000000}%
\pgfsetstrokecolor{textcolor}%
\pgfsetfillcolor{textcolor}%
\pgftext[x=0.313889in, y=2.781051in, left, base]{\color{textcolor}\rmfamily\fontsize{10.000000}{12.000000}\selectfont \(\displaystyle {0.90}\)}%
\end{pgfscope}%
\begin{pgfscope}%
\pgfsetbuttcap%
\pgfsetroundjoin%
\definecolor{currentfill}{rgb}{0.000000,0.000000,0.000000}%
\pgfsetfillcolor{currentfill}%
\pgfsetlinewidth{0.501875pt}%
\definecolor{currentstroke}{rgb}{0.000000,0.000000,0.000000}%
\pgfsetstrokecolor{currentstroke}%
\pgfsetdash{}{0pt}%
\pgfsys@defobject{currentmarker}{\pgfqpoint{0.000000in}{0.000000in}}{\pgfqpoint{0.041667in}{0.000000in}}{%
\pgfpathmoveto{\pgfqpoint{0.000000in}{0.000000in}}%
\pgfpathlineto{\pgfqpoint{0.041667in}{0.000000in}}%
\pgfusepath{stroke,fill}%
}%
\begin{pgfscope}%
\pgfsys@transformshift{0.609415in}{3.176134in}%
\pgfsys@useobject{currentmarker}{}%
\end{pgfscope}%
\end{pgfscope}%
\begin{pgfscope}%
\pgfsetbuttcap%
\pgfsetroundjoin%
\definecolor{currentfill}{rgb}{0.000000,0.000000,0.000000}%
\pgfsetfillcolor{currentfill}%
\pgfsetlinewidth{0.501875pt}%
\definecolor{currentstroke}{rgb}{0.000000,0.000000,0.000000}%
\pgfsetstrokecolor{currentstroke}%
\pgfsetdash{}{0pt}%
\pgfsys@defobject{currentmarker}{\pgfqpoint{-0.041667in}{0.000000in}}{\pgfqpoint{-0.000000in}{0.000000in}}{%
\pgfpathmoveto{\pgfqpoint{-0.000000in}{0.000000in}}%
\pgfpathlineto{\pgfqpoint{-0.041667in}{0.000000in}}%
\pgfusepath{stroke,fill}%
}%
\begin{pgfscope}%
\pgfsys@transformshift{2.829105in}{3.176134in}%
\pgfsys@useobject{currentmarker}{}%
\end{pgfscope}%
\end{pgfscope}%
\begin{pgfscope}%
\definecolor{textcolor}{rgb}{0.000000,0.000000,0.000000}%
\pgfsetstrokecolor{textcolor}%
\pgfsetfillcolor{textcolor}%
\pgftext[x=0.313889in, y=3.123373in, left, base]{\color{textcolor}\rmfamily\fontsize{10.000000}{12.000000}\selectfont \(\displaystyle {0.95}\)}%
\end{pgfscope}%
\begin{pgfscope}%
\pgfsetbuttcap%
\pgfsetroundjoin%
\definecolor{currentfill}{rgb}{0.000000,0.000000,0.000000}%
\pgfsetfillcolor{currentfill}%
\pgfsetlinewidth{0.501875pt}%
\definecolor{currentstroke}{rgb}{0.000000,0.000000,0.000000}%
\pgfsetstrokecolor{currentstroke}%
\pgfsetdash{}{0pt}%
\pgfsys@defobject{currentmarker}{\pgfqpoint{0.000000in}{0.000000in}}{\pgfqpoint{0.041667in}{0.000000in}}{%
\pgfpathmoveto{\pgfqpoint{0.000000in}{0.000000in}}%
\pgfpathlineto{\pgfqpoint{0.041667in}{0.000000in}}%
\pgfusepath{stroke,fill}%
}%
\begin{pgfscope}%
\pgfsys@transformshift{0.609415in}{3.518457in}%
\pgfsys@useobject{currentmarker}{}%
\end{pgfscope}%
\end{pgfscope}%
\begin{pgfscope}%
\pgfsetbuttcap%
\pgfsetroundjoin%
\definecolor{currentfill}{rgb}{0.000000,0.000000,0.000000}%
\pgfsetfillcolor{currentfill}%
\pgfsetlinewidth{0.501875pt}%
\definecolor{currentstroke}{rgb}{0.000000,0.000000,0.000000}%
\pgfsetstrokecolor{currentstroke}%
\pgfsetdash{}{0pt}%
\pgfsys@defobject{currentmarker}{\pgfqpoint{-0.041667in}{0.000000in}}{\pgfqpoint{-0.000000in}{0.000000in}}{%
\pgfpathmoveto{\pgfqpoint{-0.000000in}{0.000000in}}%
\pgfpathlineto{\pgfqpoint{-0.041667in}{0.000000in}}%
\pgfusepath{stroke,fill}%
}%
\begin{pgfscope}%
\pgfsys@transformshift{2.829105in}{3.518457in}%
\pgfsys@useobject{currentmarker}{}%
\end{pgfscope}%
\end{pgfscope}%
\begin{pgfscope}%
\definecolor{textcolor}{rgb}{0.000000,0.000000,0.000000}%
\pgfsetstrokecolor{textcolor}%
\pgfsetfillcolor{textcolor}%
\pgftext[x=0.313889in, y=3.465695in, left, base]{\color{textcolor}\rmfamily\fontsize{10.000000}{12.000000}\selectfont \(\displaystyle {1.00}\)}%
\end{pgfscope}%
\begin{pgfscope}%
\pgfsetbuttcap%
\pgfsetroundjoin%
\definecolor{currentfill}{rgb}{0.000000,0.000000,0.000000}%
\pgfsetfillcolor{currentfill}%
\pgfsetlinewidth{0.501875pt}%
\definecolor{currentstroke}{rgb}{0.000000,0.000000,0.000000}%
\pgfsetstrokecolor{currentstroke}%
\pgfsetdash{}{0pt}%
\pgfsys@defobject{currentmarker}{\pgfqpoint{0.000000in}{0.000000in}}{\pgfqpoint{0.020833in}{0.000000in}}{%
\pgfpathmoveto{\pgfqpoint{0.000000in}{0.000000in}}%
\pgfpathlineto{\pgfqpoint{0.020833in}{0.000000in}}%
\pgfusepath{stroke,fill}%
}%
\begin{pgfscope}%
\pgfsys@transformshift{0.609415in}{2.354561in}%
\pgfsys@useobject{currentmarker}{}%
\end{pgfscope}%
\end{pgfscope}%
\begin{pgfscope}%
\pgfsetbuttcap%
\pgfsetroundjoin%
\definecolor{currentfill}{rgb}{0.000000,0.000000,0.000000}%
\pgfsetfillcolor{currentfill}%
\pgfsetlinewidth{0.501875pt}%
\definecolor{currentstroke}{rgb}{0.000000,0.000000,0.000000}%
\pgfsetstrokecolor{currentstroke}%
\pgfsetdash{}{0pt}%
\pgfsys@defobject{currentmarker}{\pgfqpoint{-0.020833in}{0.000000in}}{\pgfqpoint{-0.000000in}{0.000000in}}{%
\pgfpathmoveto{\pgfqpoint{-0.000000in}{0.000000in}}%
\pgfpathlineto{\pgfqpoint{-0.020833in}{0.000000in}}%
\pgfusepath{stroke,fill}%
}%
\begin{pgfscope}%
\pgfsys@transformshift{2.829105in}{2.354561in}%
\pgfsys@useobject{currentmarker}{}%
\end{pgfscope}%
\end{pgfscope}%
\begin{pgfscope}%
\pgfsetbuttcap%
\pgfsetroundjoin%
\definecolor{currentfill}{rgb}{0.000000,0.000000,0.000000}%
\pgfsetfillcolor{currentfill}%
\pgfsetlinewidth{0.501875pt}%
\definecolor{currentstroke}{rgb}{0.000000,0.000000,0.000000}%
\pgfsetstrokecolor{currentstroke}%
\pgfsetdash{}{0pt}%
\pgfsys@defobject{currentmarker}{\pgfqpoint{0.000000in}{0.000000in}}{\pgfqpoint{0.020833in}{0.000000in}}{%
\pgfpathmoveto{\pgfqpoint{0.000000in}{0.000000in}}%
\pgfpathlineto{\pgfqpoint{0.020833in}{0.000000in}}%
\pgfusepath{stroke,fill}%
}%
\begin{pgfscope}%
\pgfsys@transformshift{0.609415in}{2.423025in}%
\pgfsys@useobject{currentmarker}{}%
\end{pgfscope}%
\end{pgfscope}%
\begin{pgfscope}%
\pgfsetbuttcap%
\pgfsetroundjoin%
\definecolor{currentfill}{rgb}{0.000000,0.000000,0.000000}%
\pgfsetfillcolor{currentfill}%
\pgfsetlinewidth{0.501875pt}%
\definecolor{currentstroke}{rgb}{0.000000,0.000000,0.000000}%
\pgfsetstrokecolor{currentstroke}%
\pgfsetdash{}{0pt}%
\pgfsys@defobject{currentmarker}{\pgfqpoint{-0.020833in}{0.000000in}}{\pgfqpoint{-0.000000in}{0.000000in}}{%
\pgfpathmoveto{\pgfqpoint{-0.000000in}{0.000000in}}%
\pgfpathlineto{\pgfqpoint{-0.020833in}{0.000000in}}%
\pgfusepath{stroke,fill}%
}%
\begin{pgfscope}%
\pgfsys@transformshift{2.829105in}{2.423025in}%
\pgfsys@useobject{currentmarker}{}%
\end{pgfscope}%
\end{pgfscope}%
\begin{pgfscope}%
\pgfsetbuttcap%
\pgfsetroundjoin%
\definecolor{currentfill}{rgb}{0.000000,0.000000,0.000000}%
\pgfsetfillcolor{currentfill}%
\pgfsetlinewidth{0.501875pt}%
\definecolor{currentstroke}{rgb}{0.000000,0.000000,0.000000}%
\pgfsetstrokecolor{currentstroke}%
\pgfsetdash{}{0pt}%
\pgfsys@defobject{currentmarker}{\pgfqpoint{0.000000in}{0.000000in}}{\pgfqpoint{0.020833in}{0.000000in}}{%
\pgfpathmoveto{\pgfqpoint{0.000000in}{0.000000in}}%
\pgfpathlineto{\pgfqpoint{0.020833in}{0.000000in}}%
\pgfusepath{stroke,fill}%
}%
\begin{pgfscope}%
\pgfsys@transformshift{0.609415in}{2.559954in}%
\pgfsys@useobject{currentmarker}{}%
\end{pgfscope}%
\end{pgfscope}%
\begin{pgfscope}%
\pgfsetbuttcap%
\pgfsetroundjoin%
\definecolor{currentfill}{rgb}{0.000000,0.000000,0.000000}%
\pgfsetfillcolor{currentfill}%
\pgfsetlinewidth{0.501875pt}%
\definecolor{currentstroke}{rgb}{0.000000,0.000000,0.000000}%
\pgfsetstrokecolor{currentstroke}%
\pgfsetdash{}{0pt}%
\pgfsys@defobject{currentmarker}{\pgfqpoint{-0.020833in}{0.000000in}}{\pgfqpoint{-0.000000in}{0.000000in}}{%
\pgfpathmoveto{\pgfqpoint{-0.000000in}{0.000000in}}%
\pgfpathlineto{\pgfqpoint{-0.020833in}{0.000000in}}%
\pgfusepath{stroke,fill}%
}%
\begin{pgfscope}%
\pgfsys@transformshift{2.829105in}{2.559954in}%
\pgfsys@useobject{currentmarker}{}%
\end{pgfscope}%
\end{pgfscope}%
\begin{pgfscope}%
\pgfsetbuttcap%
\pgfsetroundjoin%
\definecolor{currentfill}{rgb}{0.000000,0.000000,0.000000}%
\pgfsetfillcolor{currentfill}%
\pgfsetlinewidth{0.501875pt}%
\definecolor{currentstroke}{rgb}{0.000000,0.000000,0.000000}%
\pgfsetstrokecolor{currentstroke}%
\pgfsetdash{}{0pt}%
\pgfsys@defobject{currentmarker}{\pgfqpoint{0.000000in}{0.000000in}}{\pgfqpoint{0.020833in}{0.000000in}}{%
\pgfpathmoveto{\pgfqpoint{0.000000in}{0.000000in}}%
\pgfpathlineto{\pgfqpoint{0.020833in}{0.000000in}}%
\pgfusepath{stroke,fill}%
}%
\begin{pgfscope}%
\pgfsys@transformshift{0.609415in}{2.628419in}%
\pgfsys@useobject{currentmarker}{}%
\end{pgfscope}%
\end{pgfscope}%
\begin{pgfscope}%
\pgfsetbuttcap%
\pgfsetroundjoin%
\definecolor{currentfill}{rgb}{0.000000,0.000000,0.000000}%
\pgfsetfillcolor{currentfill}%
\pgfsetlinewidth{0.501875pt}%
\definecolor{currentstroke}{rgb}{0.000000,0.000000,0.000000}%
\pgfsetstrokecolor{currentstroke}%
\pgfsetdash{}{0pt}%
\pgfsys@defobject{currentmarker}{\pgfqpoint{-0.020833in}{0.000000in}}{\pgfqpoint{-0.000000in}{0.000000in}}{%
\pgfpathmoveto{\pgfqpoint{-0.000000in}{0.000000in}}%
\pgfpathlineto{\pgfqpoint{-0.020833in}{0.000000in}}%
\pgfusepath{stroke,fill}%
}%
\begin{pgfscope}%
\pgfsys@transformshift{2.829105in}{2.628419in}%
\pgfsys@useobject{currentmarker}{}%
\end{pgfscope}%
\end{pgfscope}%
\begin{pgfscope}%
\pgfsetbuttcap%
\pgfsetroundjoin%
\definecolor{currentfill}{rgb}{0.000000,0.000000,0.000000}%
\pgfsetfillcolor{currentfill}%
\pgfsetlinewidth{0.501875pt}%
\definecolor{currentstroke}{rgb}{0.000000,0.000000,0.000000}%
\pgfsetstrokecolor{currentstroke}%
\pgfsetdash{}{0pt}%
\pgfsys@defobject{currentmarker}{\pgfqpoint{0.000000in}{0.000000in}}{\pgfqpoint{0.020833in}{0.000000in}}{%
\pgfpathmoveto{\pgfqpoint{0.000000in}{0.000000in}}%
\pgfpathlineto{\pgfqpoint{0.020833in}{0.000000in}}%
\pgfusepath{stroke,fill}%
}%
\begin{pgfscope}%
\pgfsys@transformshift{0.609415in}{2.696883in}%
\pgfsys@useobject{currentmarker}{}%
\end{pgfscope}%
\end{pgfscope}%
\begin{pgfscope}%
\pgfsetbuttcap%
\pgfsetroundjoin%
\definecolor{currentfill}{rgb}{0.000000,0.000000,0.000000}%
\pgfsetfillcolor{currentfill}%
\pgfsetlinewidth{0.501875pt}%
\definecolor{currentstroke}{rgb}{0.000000,0.000000,0.000000}%
\pgfsetstrokecolor{currentstroke}%
\pgfsetdash{}{0pt}%
\pgfsys@defobject{currentmarker}{\pgfqpoint{-0.020833in}{0.000000in}}{\pgfqpoint{-0.000000in}{0.000000in}}{%
\pgfpathmoveto{\pgfqpoint{-0.000000in}{0.000000in}}%
\pgfpathlineto{\pgfqpoint{-0.020833in}{0.000000in}}%
\pgfusepath{stroke,fill}%
}%
\begin{pgfscope}%
\pgfsys@transformshift{2.829105in}{2.696883in}%
\pgfsys@useobject{currentmarker}{}%
\end{pgfscope}%
\end{pgfscope}%
\begin{pgfscope}%
\pgfsetbuttcap%
\pgfsetroundjoin%
\definecolor{currentfill}{rgb}{0.000000,0.000000,0.000000}%
\pgfsetfillcolor{currentfill}%
\pgfsetlinewidth{0.501875pt}%
\definecolor{currentstroke}{rgb}{0.000000,0.000000,0.000000}%
\pgfsetstrokecolor{currentstroke}%
\pgfsetdash{}{0pt}%
\pgfsys@defobject{currentmarker}{\pgfqpoint{0.000000in}{0.000000in}}{\pgfqpoint{0.020833in}{0.000000in}}{%
\pgfpathmoveto{\pgfqpoint{0.000000in}{0.000000in}}%
\pgfpathlineto{\pgfqpoint{0.020833in}{0.000000in}}%
\pgfusepath{stroke,fill}%
}%
\begin{pgfscope}%
\pgfsys@transformshift{0.609415in}{2.765348in}%
\pgfsys@useobject{currentmarker}{}%
\end{pgfscope}%
\end{pgfscope}%
\begin{pgfscope}%
\pgfsetbuttcap%
\pgfsetroundjoin%
\definecolor{currentfill}{rgb}{0.000000,0.000000,0.000000}%
\pgfsetfillcolor{currentfill}%
\pgfsetlinewidth{0.501875pt}%
\definecolor{currentstroke}{rgb}{0.000000,0.000000,0.000000}%
\pgfsetstrokecolor{currentstroke}%
\pgfsetdash{}{0pt}%
\pgfsys@defobject{currentmarker}{\pgfqpoint{-0.020833in}{0.000000in}}{\pgfqpoint{-0.000000in}{0.000000in}}{%
\pgfpathmoveto{\pgfqpoint{-0.000000in}{0.000000in}}%
\pgfpathlineto{\pgfqpoint{-0.020833in}{0.000000in}}%
\pgfusepath{stroke,fill}%
}%
\begin{pgfscope}%
\pgfsys@transformshift{2.829105in}{2.765348in}%
\pgfsys@useobject{currentmarker}{}%
\end{pgfscope}%
\end{pgfscope}%
\begin{pgfscope}%
\pgfsetbuttcap%
\pgfsetroundjoin%
\definecolor{currentfill}{rgb}{0.000000,0.000000,0.000000}%
\pgfsetfillcolor{currentfill}%
\pgfsetlinewidth{0.501875pt}%
\definecolor{currentstroke}{rgb}{0.000000,0.000000,0.000000}%
\pgfsetstrokecolor{currentstroke}%
\pgfsetdash{}{0pt}%
\pgfsys@defobject{currentmarker}{\pgfqpoint{0.000000in}{0.000000in}}{\pgfqpoint{0.020833in}{0.000000in}}{%
\pgfpathmoveto{\pgfqpoint{0.000000in}{0.000000in}}%
\pgfpathlineto{\pgfqpoint{0.020833in}{0.000000in}}%
\pgfusepath{stroke,fill}%
}%
\begin{pgfscope}%
\pgfsys@transformshift{0.609415in}{2.902277in}%
\pgfsys@useobject{currentmarker}{}%
\end{pgfscope}%
\end{pgfscope}%
\begin{pgfscope}%
\pgfsetbuttcap%
\pgfsetroundjoin%
\definecolor{currentfill}{rgb}{0.000000,0.000000,0.000000}%
\pgfsetfillcolor{currentfill}%
\pgfsetlinewidth{0.501875pt}%
\definecolor{currentstroke}{rgb}{0.000000,0.000000,0.000000}%
\pgfsetstrokecolor{currentstroke}%
\pgfsetdash{}{0pt}%
\pgfsys@defobject{currentmarker}{\pgfqpoint{-0.020833in}{0.000000in}}{\pgfqpoint{-0.000000in}{0.000000in}}{%
\pgfpathmoveto{\pgfqpoint{-0.000000in}{0.000000in}}%
\pgfpathlineto{\pgfqpoint{-0.020833in}{0.000000in}}%
\pgfusepath{stroke,fill}%
}%
\begin{pgfscope}%
\pgfsys@transformshift{2.829105in}{2.902277in}%
\pgfsys@useobject{currentmarker}{}%
\end{pgfscope}%
\end{pgfscope}%
\begin{pgfscope}%
\pgfsetbuttcap%
\pgfsetroundjoin%
\definecolor{currentfill}{rgb}{0.000000,0.000000,0.000000}%
\pgfsetfillcolor{currentfill}%
\pgfsetlinewidth{0.501875pt}%
\definecolor{currentstroke}{rgb}{0.000000,0.000000,0.000000}%
\pgfsetstrokecolor{currentstroke}%
\pgfsetdash{}{0pt}%
\pgfsys@defobject{currentmarker}{\pgfqpoint{0.000000in}{0.000000in}}{\pgfqpoint{0.020833in}{0.000000in}}{%
\pgfpathmoveto{\pgfqpoint{0.000000in}{0.000000in}}%
\pgfpathlineto{\pgfqpoint{0.020833in}{0.000000in}}%
\pgfusepath{stroke,fill}%
}%
\begin{pgfscope}%
\pgfsys@transformshift{0.609415in}{2.970741in}%
\pgfsys@useobject{currentmarker}{}%
\end{pgfscope}%
\end{pgfscope}%
\begin{pgfscope}%
\pgfsetbuttcap%
\pgfsetroundjoin%
\definecolor{currentfill}{rgb}{0.000000,0.000000,0.000000}%
\pgfsetfillcolor{currentfill}%
\pgfsetlinewidth{0.501875pt}%
\definecolor{currentstroke}{rgb}{0.000000,0.000000,0.000000}%
\pgfsetstrokecolor{currentstroke}%
\pgfsetdash{}{0pt}%
\pgfsys@defobject{currentmarker}{\pgfqpoint{-0.020833in}{0.000000in}}{\pgfqpoint{-0.000000in}{0.000000in}}{%
\pgfpathmoveto{\pgfqpoint{-0.000000in}{0.000000in}}%
\pgfpathlineto{\pgfqpoint{-0.020833in}{0.000000in}}%
\pgfusepath{stroke,fill}%
}%
\begin{pgfscope}%
\pgfsys@transformshift{2.829105in}{2.970741in}%
\pgfsys@useobject{currentmarker}{}%
\end{pgfscope}%
\end{pgfscope}%
\begin{pgfscope}%
\pgfsetbuttcap%
\pgfsetroundjoin%
\definecolor{currentfill}{rgb}{0.000000,0.000000,0.000000}%
\pgfsetfillcolor{currentfill}%
\pgfsetlinewidth{0.501875pt}%
\definecolor{currentstroke}{rgb}{0.000000,0.000000,0.000000}%
\pgfsetstrokecolor{currentstroke}%
\pgfsetdash{}{0pt}%
\pgfsys@defobject{currentmarker}{\pgfqpoint{0.000000in}{0.000000in}}{\pgfqpoint{0.020833in}{0.000000in}}{%
\pgfpathmoveto{\pgfqpoint{0.000000in}{0.000000in}}%
\pgfpathlineto{\pgfqpoint{0.020833in}{0.000000in}}%
\pgfusepath{stroke,fill}%
}%
\begin{pgfscope}%
\pgfsys@transformshift{0.609415in}{3.039205in}%
\pgfsys@useobject{currentmarker}{}%
\end{pgfscope}%
\end{pgfscope}%
\begin{pgfscope}%
\pgfsetbuttcap%
\pgfsetroundjoin%
\definecolor{currentfill}{rgb}{0.000000,0.000000,0.000000}%
\pgfsetfillcolor{currentfill}%
\pgfsetlinewidth{0.501875pt}%
\definecolor{currentstroke}{rgb}{0.000000,0.000000,0.000000}%
\pgfsetstrokecolor{currentstroke}%
\pgfsetdash{}{0pt}%
\pgfsys@defobject{currentmarker}{\pgfqpoint{-0.020833in}{0.000000in}}{\pgfqpoint{-0.000000in}{0.000000in}}{%
\pgfpathmoveto{\pgfqpoint{-0.000000in}{0.000000in}}%
\pgfpathlineto{\pgfqpoint{-0.020833in}{0.000000in}}%
\pgfusepath{stroke,fill}%
}%
\begin{pgfscope}%
\pgfsys@transformshift{2.829105in}{3.039205in}%
\pgfsys@useobject{currentmarker}{}%
\end{pgfscope}%
\end{pgfscope}%
\begin{pgfscope}%
\pgfsetbuttcap%
\pgfsetroundjoin%
\definecolor{currentfill}{rgb}{0.000000,0.000000,0.000000}%
\pgfsetfillcolor{currentfill}%
\pgfsetlinewidth{0.501875pt}%
\definecolor{currentstroke}{rgb}{0.000000,0.000000,0.000000}%
\pgfsetstrokecolor{currentstroke}%
\pgfsetdash{}{0pt}%
\pgfsys@defobject{currentmarker}{\pgfqpoint{0.000000in}{0.000000in}}{\pgfqpoint{0.020833in}{0.000000in}}{%
\pgfpathmoveto{\pgfqpoint{0.000000in}{0.000000in}}%
\pgfpathlineto{\pgfqpoint{0.020833in}{0.000000in}}%
\pgfusepath{stroke,fill}%
}%
\begin{pgfscope}%
\pgfsys@transformshift{0.609415in}{3.107670in}%
\pgfsys@useobject{currentmarker}{}%
\end{pgfscope}%
\end{pgfscope}%
\begin{pgfscope}%
\pgfsetbuttcap%
\pgfsetroundjoin%
\definecolor{currentfill}{rgb}{0.000000,0.000000,0.000000}%
\pgfsetfillcolor{currentfill}%
\pgfsetlinewidth{0.501875pt}%
\definecolor{currentstroke}{rgb}{0.000000,0.000000,0.000000}%
\pgfsetstrokecolor{currentstroke}%
\pgfsetdash{}{0pt}%
\pgfsys@defobject{currentmarker}{\pgfqpoint{-0.020833in}{0.000000in}}{\pgfqpoint{-0.000000in}{0.000000in}}{%
\pgfpathmoveto{\pgfqpoint{-0.000000in}{0.000000in}}%
\pgfpathlineto{\pgfqpoint{-0.020833in}{0.000000in}}%
\pgfusepath{stroke,fill}%
}%
\begin{pgfscope}%
\pgfsys@transformshift{2.829105in}{3.107670in}%
\pgfsys@useobject{currentmarker}{}%
\end{pgfscope}%
\end{pgfscope}%
\begin{pgfscope}%
\pgfsetbuttcap%
\pgfsetroundjoin%
\definecolor{currentfill}{rgb}{0.000000,0.000000,0.000000}%
\pgfsetfillcolor{currentfill}%
\pgfsetlinewidth{0.501875pt}%
\definecolor{currentstroke}{rgb}{0.000000,0.000000,0.000000}%
\pgfsetstrokecolor{currentstroke}%
\pgfsetdash{}{0pt}%
\pgfsys@defobject{currentmarker}{\pgfqpoint{0.000000in}{0.000000in}}{\pgfqpoint{0.020833in}{0.000000in}}{%
\pgfpathmoveto{\pgfqpoint{0.000000in}{0.000000in}}%
\pgfpathlineto{\pgfqpoint{0.020833in}{0.000000in}}%
\pgfusepath{stroke,fill}%
}%
\begin{pgfscope}%
\pgfsys@transformshift{0.609415in}{3.244599in}%
\pgfsys@useobject{currentmarker}{}%
\end{pgfscope}%
\end{pgfscope}%
\begin{pgfscope}%
\pgfsetbuttcap%
\pgfsetroundjoin%
\definecolor{currentfill}{rgb}{0.000000,0.000000,0.000000}%
\pgfsetfillcolor{currentfill}%
\pgfsetlinewidth{0.501875pt}%
\definecolor{currentstroke}{rgb}{0.000000,0.000000,0.000000}%
\pgfsetstrokecolor{currentstroke}%
\pgfsetdash{}{0pt}%
\pgfsys@defobject{currentmarker}{\pgfqpoint{-0.020833in}{0.000000in}}{\pgfqpoint{-0.000000in}{0.000000in}}{%
\pgfpathmoveto{\pgfqpoint{-0.000000in}{0.000000in}}%
\pgfpathlineto{\pgfqpoint{-0.020833in}{0.000000in}}%
\pgfusepath{stroke,fill}%
}%
\begin{pgfscope}%
\pgfsys@transformshift{2.829105in}{3.244599in}%
\pgfsys@useobject{currentmarker}{}%
\end{pgfscope}%
\end{pgfscope}%
\begin{pgfscope}%
\pgfsetbuttcap%
\pgfsetroundjoin%
\definecolor{currentfill}{rgb}{0.000000,0.000000,0.000000}%
\pgfsetfillcolor{currentfill}%
\pgfsetlinewidth{0.501875pt}%
\definecolor{currentstroke}{rgb}{0.000000,0.000000,0.000000}%
\pgfsetstrokecolor{currentstroke}%
\pgfsetdash{}{0pt}%
\pgfsys@defobject{currentmarker}{\pgfqpoint{0.000000in}{0.000000in}}{\pgfqpoint{0.020833in}{0.000000in}}{%
\pgfpathmoveto{\pgfqpoint{0.000000in}{0.000000in}}%
\pgfpathlineto{\pgfqpoint{0.020833in}{0.000000in}}%
\pgfusepath{stroke,fill}%
}%
\begin{pgfscope}%
\pgfsys@transformshift{0.609415in}{3.313063in}%
\pgfsys@useobject{currentmarker}{}%
\end{pgfscope}%
\end{pgfscope}%
\begin{pgfscope}%
\pgfsetbuttcap%
\pgfsetroundjoin%
\definecolor{currentfill}{rgb}{0.000000,0.000000,0.000000}%
\pgfsetfillcolor{currentfill}%
\pgfsetlinewidth{0.501875pt}%
\definecolor{currentstroke}{rgb}{0.000000,0.000000,0.000000}%
\pgfsetstrokecolor{currentstroke}%
\pgfsetdash{}{0pt}%
\pgfsys@defobject{currentmarker}{\pgfqpoint{-0.020833in}{0.000000in}}{\pgfqpoint{-0.000000in}{0.000000in}}{%
\pgfpathmoveto{\pgfqpoint{-0.000000in}{0.000000in}}%
\pgfpathlineto{\pgfqpoint{-0.020833in}{0.000000in}}%
\pgfusepath{stroke,fill}%
}%
\begin{pgfscope}%
\pgfsys@transformshift{2.829105in}{3.313063in}%
\pgfsys@useobject{currentmarker}{}%
\end{pgfscope}%
\end{pgfscope}%
\begin{pgfscope}%
\pgfsetbuttcap%
\pgfsetroundjoin%
\definecolor{currentfill}{rgb}{0.000000,0.000000,0.000000}%
\pgfsetfillcolor{currentfill}%
\pgfsetlinewidth{0.501875pt}%
\definecolor{currentstroke}{rgb}{0.000000,0.000000,0.000000}%
\pgfsetstrokecolor{currentstroke}%
\pgfsetdash{}{0pt}%
\pgfsys@defobject{currentmarker}{\pgfqpoint{0.000000in}{0.000000in}}{\pgfqpoint{0.020833in}{0.000000in}}{%
\pgfpathmoveto{\pgfqpoint{0.000000in}{0.000000in}}%
\pgfpathlineto{\pgfqpoint{0.020833in}{0.000000in}}%
\pgfusepath{stroke,fill}%
}%
\begin{pgfscope}%
\pgfsys@transformshift{0.609415in}{3.381528in}%
\pgfsys@useobject{currentmarker}{}%
\end{pgfscope}%
\end{pgfscope}%
\begin{pgfscope}%
\pgfsetbuttcap%
\pgfsetroundjoin%
\definecolor{currentfill}{rgb}{0.000000,0.000000,0.000000}%
\pgfsetfillcolor{currentfill}%
\pgfsetlinewidth{0.501875pt}%
\definecolor{currentstroke}{rgb}{0.000000,0.000000,0.000000}%
\pgfsetstrokecolor{currentstroke}%
\pgfsetdash{}{0pt}%
\pgfsys@defobject{currentmarker}{\pgfqpoint{-0.020833in}{0.000000in}}{\pgfqpoint{-0.000000in}{0.000000in}}{%
\pgfpathmoveto{\pgfqpoint{-0.000000in}{0.000000in}}%
\pgfpathlineto{\pgfqpoint{-0.020833in}{0.000000in}}%
\pgfusepath{stroke,fill}%
}%
\begin{pgfscope}%
\pgfsys@transformshift{2.829105in}{3.381528in}%
\pgfsys@useobject{currentmarker}{}%
\end{pgfscope}%
\end{pgfscope}%
\begin{pgfscope}%
\pgfsetbuttcap%
\pgfsetroundjoin%
\definecolor{currentfill}{rgb}{0.000000,0.000000,0.000000}%
\pgfsetfillcolor{currentfill}%
\pgfsetlinewidth{0.501875pt}%
\definecolor{currentstroke}{rgb}{0.000000,0.000000,0.000000}%
\pgfsetstrokecolor{currentstroke}%
\pgfsetdash{}{0pt}%
\pgfsys@defobject{currentmarker}{\pgfqpoint{0.000000in}{0.000000in}}{\pgfqpoint{0.020833in}{0.000000in}}{%
\pgfpathmoveto{\pgfqpoint{0.000000in}{0.000000in}}%
\pgfpathlineto{\pgfqpoint{0.020833in}{0.000000in}}%
\pgfusepath{stroke,fill}%
}%
\begin{pgfscope}%
\pgfsys@transformshift{0.609415in}{3.449992in}%
\pgfsys@useobject{currentmarker}{}%
\end{pgfscope}%
\end{pgfscope}%
\begin{pgfscope}%
\pgfsetbuttcap%
\pgfsetroundjoin%
\definecolor{currentfill}{rgb}{0.000000,0.000000,0.000000}%
\pgfsetfillcolor{currentfill}%
\pgfsetlinewidth{0.501875pt}%
\definecolor{currentstroke}{rgb}{0.000000,0.000000,0.000000}%
\pgfsetstrokecolor{currentstroke}%
\pgfsetdash{}{0pt}%
\pgfsys@defobject{currentmarker}{\pgfqpoint{-0.020833in}{0.000000in}}{\pgfqpoint{-0.000000in}{0.000000in}}{%
\pgfpathmoveto{\pgfqpoint{-0.000000in}{0.000000in}}%
\pgfpathlineto{\pgfqpoint{-0.020833in}{0.000000in}}%
\pgfusepath{stroke,fill}%
}%
\begin{pgfscope}%
\pgfsys@transformshift{2.829105in}{3.449992in}%
\pgfsys@useobject{currentmarker}{}%
\end{pgfscope}%
\end{pgfscope}%
\begin{pgfscope}%
\definecolor{textcolor}{rgb}{0.000000,0.000000,0.000000}%
\pgfsetstrokecolor{textcolor}%
\pgfsetfillcolor{textcolor}%
\pgftext[x=0.258334in,y=2.961093in,,bottom,rotate=90.000000]{\color{textcolor}\rmfamily\fontsize{10.000000}{12.000000}\selectfont \(\displaystyle T(K)\)}%
\end{pgfscope}%
\begin{pgfscope}%
\pgfpathrectangle{\pgfqpoint{0.609415in}{2.347992in}}{\pgfqpoint{2.219690in}{1.226201in}}%
\pgfusepath{clip}%
\pgfsetrectcap%
\pgfsetroundjoin%
\pgfsetlinewidth{1.003750pt}%
\definecolor{currentstroke}{rgb}{0.047059,0.364706,0.647059}%
\pgfsetstrokecolor{currentstroke}%
\pgfsetdash{}{0pt}%
\pgfpathmoveto{\pgfqpoint{0.631176in}{3.518457in}}%
\pgfpathlineto{\pgfqpoint{0.652938in}{3.480250in}}%
\pgfpathlineto{\pgfqpoint{0.674700in}{3.450527in}}%
\pgfpathlineto{\pgfqpoint{0.696461in}{3.433408in}}%
\pgfpathlineto{\pgfqpoint{0.718223in}{3.419978in}}%
\pgfpathlineto{\pgfqpoint{0.739985in}{3.409069in}}%
\pgfpathlineto{\pgfqpoint{0.761746in}{3.399335in}}%
\pgfpathlineto{\pgfqpoint{0.783508in}{3.392468in}}%
\pgfpathlineto{\pgfqpoint{0.805270in}{3.384788in}}%
\pgfpathlineto{\pgfqpoint{0.827031in}{3.378307in}}%
\pgfpathlineto{\pgfqpoint{0.848793in}{3.372815in}}%
\pgfpathlineto{\pgfqpoint{0.870555in}{3.368210in}}%
\pgfpathlineto{\pgfqpoint{0.892316in}{3.363319in}}%
\pgfpathlineto{\pgfqpoint{0.914078in}{3.358883in}}%
\pgfpathlineto{\pgfqpoint{0.935840in}{3.354484in}}%
\pgfpathlineto{\pgfqpoint{0.957601in}{3.350777in}}%
\pgfpathlineto{\pgfqpoint{0.979363in}{3.347846in}}%
\pgfpathlineto{\pgfqpoint{1.001125in}{3.345126in}}%
\pgfpathlineto{\pgfqpoint{1.022886in}{3.341924in}}%
\pgfpathlineto{\pgfqpoint{1.044648in}{3.338790in}}%
\pgfpathlineto{\pgfqpoint{1.066410in}{3.335588in}}%
\pgfpathlineto{\pgfqpoint{1.088171in}{3.332689in}}%
\pgfpathlineto{\pgfqpoint{1.109933in}{3.330315in}}%
\pgfpathlineto{\pgfqpoint{1.131695in}{3.327770in}}%
\pgfpathlineto{\pgfqpoint{1.153456in}{3.325522in}}%
\pgfpathlineto{\pgfqpoint{1.175218in}{3.323393in}}%
\pgfpathlineto{\pgfqpoint{1.196980in}{3.320993in}}%
\pgfpathlineto{\pgfqpoint{1.218741in}{3.319250in}}%
\pgfpathlineto{\pgfqpoint{1.240503in}{3.317354in}}%
\pgfpathlineto{\pgfqpoint{1.262265in}{3.315206in}}%
\pgfpathlineto{\pgfqpoint{1.284026in}{3.313383in}}%
\pgfpathlineto{\pgfqpoint{1.305788in}{3.311571in}}%
\pgfpathlineto{\pgfqpoint{1.327550in}{3.309630in}}%
\pgfpathlineto{\pgfqpoint{1.349311in}{3.307802in}}%
\pgfpathlineto{\pgfqpoint{1.371073in}{3.306020in}}%
\pgfpathlineto{\pgfqpoint{1.392835in}{3.303736in}}%
\pgfpathlineto{\pgfqpoint{1.414596in}{3.301790in}}%
\pgfpathlineto{\pgfqpoint{1.436358in}{3.300141in}}%
\pgfpathlineto{\pgfqpoint{1.458120in}{3.299073in}}%
\pgfpathlineto{\pgfqpoint{1.479881in}{3.297468in}}%
\pgfpathlineto{\pgfqpoint{1.501643in}{3.295850in}}%
\pgfpathlineto{\pgfqpoint{1.523405in}{3.294188in}}%
\pgfpathlineto{\pgfqpoint{1.545166in}{3.292955in}}%
\pgfpathlineto{\pgfqpoint{1.566928in}{3.291722in}}%
\pgfpathlineto{\pgfqpoint{1.588690in}{3.290324in}}%
\pgfpathlineto{\pgfqpoint{1.610451in}{3.288942in}}%
\pgfpathlineto{\pgfqpoint{1.632213in}{3.287695in}}%
\pgfpathlineto{\pgfqpoint{1.653975in}{3.286469in}}%
\pgfpathlineto{\pgfqpoint{1.675736in}{3.285307in}}%
\pgfpathlineto{\pgfqpoint{1.697498in}{3.284197in}}%
\pgfpathlineto{\pgfqpoint{1.719260in}{3.283066in}}%
\pgfpathlineto{\pgfqpoint{1.741021in}{3.281806in}}%
\pgfpathlineto{\pgfqpoint{1.762783in}{3.280557in}}%
\pgfpathlineto{\pgfqpoint{1.784545in}{3.279558in}}%
\pgfpathlineto{\pgfqpoint{1.806306in}{3.278225in}}%
\pgfpathlineto{\pgfqpoint{1.828068in}{3.277054in}}%
\pgfpathlineto{\pgfqpoint{1.849830in}{3.275898in}}%
\pgfpathlineto{\pgfqpoint{1.871591in}{3.275049in}}%
\pgfpathlineto{\pgfqpoint{1.893353in}{3.273994in}}%
\pgfpathlineto{\pgfqpoint{1.915115in}{3.273029in}}%
\pgfpathlineto{\pgfqpoint{1.936876in}{3.272076in}}%
\pgfpathlineto{\pgfqpoint{1.958638in}{3.270937in}}%
\pgfpathlineto{\pgfqpoint{1.980400in}{3.270038in}}%
\pgfpathlineto{\pgfqpoint{2.002161in}{3.269030in}}%
\pgfpathlineto{\pgfqpoint{2.023923in}{3.268093in}}%
\pgfpathlineto{\pgfqpoint{2.045685in}{3.267010in}}%
\pgfpathlineto{\pgfqpoint{2.067446in}{3.266017in}}%
\pgfpathlineto{\pgfqpoint{2.089208in}{3.265038in}}%
\pgfpathlineto{\pgfqpoint{2.110970in}{3.264067in}}%
\pgfpathlineto{\pgfqpoint{2.132731in}{3.262966in}}%
\pgfpathlineto{\pgfqpoint{2.154493in}{3.261919in}}%
\pgfpathlineto{\pgfqpoint{2.176255in}{3.261065in}}%
\pgfpathlineto{\pgfqpoint{2.198016in}{3.260188in}}%
\pgfpathlineto{\pgfqpoint{2.219778in}{3.259252in}}%
\pgfpathlineto{\pgfqpoint{2.241540in}{3.258297in}}%
\pgfpathlineto{\pgfqpoint{2.263301in}{3.257523in}}%
\pgfpathlineto{\pgfqpoint{2.285063in}{3.256729in}}%
\pgfpathlineto{\pgfqpoint{2.306825in}{3.255800in}}%
\pgfpathlineto{\pgfqpoint{2.328586in}{3.254910in}}%
\pgfpathlineto{\pgfqpoint{2.350348in}{3.253955in}}%
\pgfpathlineto{\pgfqpoint{2.372110in}{3.253145in}}%
\pgfpathlineto{\pgfqpoint{2.393871in}{3.252220in}}%
\pgfpathlineto{\pgfqpoint{2.415633in}{3.251487in}}%
\pgfpathlineto{\pgfqpoint{2.437395in}{3.250557in}}%
\pgfpathlineto{\pgfqpoint{2.459156in}{3.249679in}}%
\pgfpathlineto{\pgfqpoint{2.480918in}{3.248979in}}%
\pgfpathlineto{\pgfqpoint{2.502680in}{3.248155in}}%
\pgfpathlineto{\pgfqpoint{2.524441in}{3.247337in}}%
\pgfpathlineto{\pgfqpoint{2.546203in}{3.246444in}}%
\pgfpathlineto{\pgfqpoint{2.567965in}{3.245567in}}%
\pgfpathlineto{\pgfqpoint{2.589726in}{3.244762in}}%
\pgfpathlineto{\pgfqpoint{2.611488in}{3.244035in}}%
\pgfpathlineto{\pgfqpoint{2.633250in}{3.243280in}}%
\pgfpathlineto{\pgfqpoint{2.655011in}{3.242515in}}%
\pgfpathlineto{\pgfqpoint{2.676773in}{3.241668in}}%
\pgfpathlineto{\pgfqpoint{2.698535in}{3.240891in}}%
\pgfpathlineto{\pgfqpoint{2.720296in}{3.239942in}}%
\pgfpathlineto{\pgfqpoint{2.742058in}{3.239061in}}%
\pgfpathlineto{\pgfqpoint{2.763820in}{3.238344in}}%
\pgfpathlineto{\pgfqpoint{2.785581in}{3.237601in}}%
\pgfusepath{stroke}%
\end{pgfscope}%
\begin{pgfscope}%
\pgfpathrectangle{\pgfqpoint{0.609415in}{2.347992in}}{\pgfqpoint{2.219690in}{1.226201in}}%
\pgfusepath{clip}%
\pgfsetrectcap%
\pgfsetroundjoin%
\pgfsetlinewidth{1.003750pt}%
\definecolor{currentstroke}{rgb}{0.000000,0.725490,0.270588}%
\pgfsetstrokecolor{currentstroke}%
\pgfsetdash{}{0pt}%
\pgfpathmoveto{\pgfqpoint{0.631176in}{3.518457in}}%
\pgfpathlineto{\pgfqpoint{0.652938in}{3.378902in}}%
\pgfpathlineto{\pgfqpoint{0.674700in}{3.303495in}}%
\pgfpathlineto{\pgfqpoint{0.696461in}{3.249639in}}%
\pgfpathlineto{\pgfqpoint{0.718223in}{3.209186in}}%
\pgfpathlineto{\pgfqpoint{0.739985in}{3.179663in}}%
\pgfpathlineto{\pgfqpoint{0.761746in}{3.153711in}}%
\pgfpathlineto{\pgfqpoint{0.783508in}{3.133275in}}%
\pgfpathlineto{\pgfqpoint{0.805270in}{3.114806in}}%
\pgfpathlineto{\pgfqpoint{0.827031in}{3.097136in}}%
\pgfpathlineto{\pgfqpoint{0.848793in}{3.081945in}}%
\pgfpathlineto{\pgfqpoint{0.870555in}{3.067284in}}%
\pgfpathlineto{\pgfqpoint{0.892316in}{3.053629in}}%
\pgfpathlineto{\pgfqpoint{0.914078in}{3.041152in}}%
\pgfpathlineto{\pgfqpoint{0.935840in}{3.029103in}}%
\pgfpathlineto{\pgfqpoint{0.957601in}{3.017789in}}%
\pgfpathlineto{\pgfqpoint{0.979363in}{3.008019in}}%
\pgfpathlineto{\pgfqpoint{1.001125in}{2.998479in}}%
\pgfpathlineto{\pgfqpoint{1.022886in}{2.989257in}}%
\pgfpathlineto{\pgfqpoint{1.044648in}{2.981099in}}%
\pgfpathlineto{\pgfqpoint{1.066410in}{2.972743in}}%
\pgfpathlineto{\pgfqpoint{1.088171in}{2.965004in}}%
\pgfpathlineto{\pgfqpoint{1.109933in}{2.957172in}}%
\pgfpathlineto{\pgfqpoint{1.131695in}{2.949293in}}%
\pgfpathlineto{\pgfqpoint{1.153456in}{2.942008in}}%
\pgfpathlineto{\pgfqpoint{1.175218in}{2.935356in}}%
\pgfpathlineto{\pgfqpoint{1.196980in}{2.928418in}}%
\pgfpathlineto{\pgfqpoint{1.218741in}{2.921582in}}%
\pgfpathlineto{\pgfqpoint{1.240503in}{2.915525in}}%
\pgfpathlineto{\pgfqpoint{1.262265in}{2.909334in}}%
\pgfpathlineto{\pgfqpoint{1.284026in}{2.903396in}}%
\pgfpathlineto{\pgfqpoint{1.305788in}{2.897693in}}%
\pgfpathlineto{\pgfqpoint{1.327550in}{2.892218in}}%
\pgfpathlineto{\pgfqpoint{1.349311in}{2.886976in}}%
\pgfpathlineto{\pgfqpoint{1.371073in}{2.882128in}}%
\pgfpathlineto{\pgfqpoint{1.392835in}{2.876902in}}%
\pgfpathlineto{\pgfqpoint{1.414596in}{2.871719in}}%
\pgfpathlineto{\pgfqpoint{1.436358in}{2.866213in}}%
\pgfpathlineto{\pgfqpoint{1.458120in}{2.861309in}}%
\pgfpathlineto{\pgfqpoint{1.479881in}{2.856127in}}%
\pgfpathlineto{\pgfqpoint{1.501643in}{2.851276in}}%
\pgfpathlineto{\pgfqpoint{1.523405in}{2.846358in}}%
\pgfpathlineto{\pgfqpoint{1.545166in}{2.841609in}}%
\pgfpathlineto{\pgfqpoint{1.566928in}{2.837541in}}%
\pgfpathlineto{\pgfqpoint{1.588690in}{2.833228in}}%
\pgfpathlineto{\pgfqpoint{1.610451in}{2.828707in}}%
\pgfpathlineto{\pgfqpoint{1.632213in}{2.824202in}}%
\pgfpathlineto{\pgfqpoint{1.653975in}{2.820116in}}%
\pgfpathlineto{\pgfqpoint{1.675736in}{2.816323in}}%
\pgfpathlineto{\pgfqpoint{1.697498in}{2.812708in}}%
\pgfpathlineto{\pgfqpoint{1.719260in}{2.808754in}}%
\pgfpathlineto{\pgfqpoint{1.741021in}{2.804884in}}%
\pgfpathlineto{\pgfqpoint{1.762783in}{2.801018in}}%
\pgfpathlineto{\pgfqpoint{1.784545in}{2.797236in}}%
\pgfpathlineto{\pgfqpoint{1.806306in}{2.793597in}}%
\pgfpathlineto{\pgfqpoint{1.828068in}{2.789927in}}%
\pgfpathlineto{\pgfqpoint{1.849830in}{2.786233in}}%
\pgfpathlineto{\pgfqpoint{1.871591in}{2.782365in}}%
\pgfpathlineto{\pgfqpoint{1.893353in}{2.778927in}}%
\pgfpathlineto{\pgfqpoint{1.915115in}{2.775293in}}%
\pgfpathlineto{\pgfqpoint{1.936876in}{2.771794in}}%
\pgfpathlineto{\pgfqpoint{1.958638in}{2.768432in}}%
\pgfpathlineto{\pgfqpoint{1.980400in}{2.765085in}}%
\pgfpathlineto{\pgfqpoint{2.002161in}{2.761910in}}%
\pgfpathlineto{\pgfqpoint{2.023923in}{2.758873in}}%
\pgfpathlineto{\pgfqpoint{2.045685in}{2.755709in}}%
\pgfpathlineto{\pgfqpoint{2.067446in}{2.752396in}}%
\pgfpathlineto{\pgfqpoint{2.089208in}{2.749258in}}%
\pgfpathlineto{\pgfqpoint{2.110970in}{2.746172in}}%
\pgfpathlineto{\pgfqpoint{2.132731in}{2.743126in}}%
\pgfpathlineto{\pgfqpoint{2.154493in}{2.740021in}}%
\pgfpathlineto{\pgfqpoint{2.176255in}{2.736998in}}%
\pgfpathlineto{\pgfqpoint{2.198016in}{2.734153in}}%
\pgfpathlineto{\pgfqpoint{2.219778in}{2.731278in}}%
\pgfpathlineto{\pgfqpoint{2.241540in}{2.728042in}}%
\pgfpathlineto{\pgfqpoint{2.263301in}{2.725095in}}%
\pgfpathlineto{\pgfqpoint{2.285063in}{2.722179in}}%
\pgfpathlineto{\pgfqpoint{2.306825in}{2.719356in}}%
\pgfpathlineto{\pgfqpoint{2.328586in}{2.716467in}}%
\pgfpathlineto{\pgfqpoint{2.350348in}{2.713548in}}%
\pgfpathlineto{\pgfqpoint{2.372110in}{2.710709in}}%
\pgfpathlineto{\pgfqpoint{2.393871in}{2.707900in}}%
\pgfpathlineto{\pgfqpoint{2.415633in}{2.705277in}}%
\pgfpathlineto{\pgfqpoint{2.437395in}{2.702437in}}%
\pgfpathlineto{\pgfqpoint{2.459156in}{2.699850in}}%
\pgfpathlineto{\pgfqpoint{2.480918in}{2.697157in}}%
\pgfpathlineto{\pgfqpoint{2.502680in}{2.694567in}}%
\pgfpathlineto{\pgfqpoint{2.524441in}{2.691828in}}%
\pgfpathlineto{\pgfqpoint{2.546203in}{2.689112in}}%
\pgfpathlineto{\pgfqpoint{2.567965in}{2.686707in}}%
\pgfpathlineto{\pgfqpoint{2.589726in}{2.684099in}}%
\pgfpathlineto{\pgfqpoint{2.611488in}{2.681470in}}%
\pgfpathlineto{\pgfqpoint{2.633250in}{2.678766in}}%
\pgfpathlineto{\pgfqpoint{2.655011in}{2.676242in}}%
\pgfpathlineto{\pgfqpoint{2.676773in}{2.673786in}}%
\pgfpathlineto{\pgfqpoint{2.698535in}{2.671475in}}%
\pgfpathlineto{\pgfqpoint{2.720296in}{2.669057in}}%
\pgfpathlineto{\pgfqpoint{2.742058in}{2.666556in}}%
\pgfpathlineto{\pgfqpoint{2.763820in}{2.664056in}}%
\pgfpathlineto{\pgfqpoint{2.785581in}{2.661466in}}%
\pgfusepath{stroke}%
\end{pgfscope}%
\begin{pgfscope}%
\pgfpathrectangle{\pgfqpoint{0.609415in}{2.347992in}}{\pgfqpoint{2.219690in}{1.226201in}}%
\pgfusepath{clip}%
\pgfsetrectcap%
\pgfsetroundjoin%
\pgfsetlinewidth{1.003750pt}%
\definecolor{currentstroke}{rgb}{1.000000,0.584314,0.000000}%
\pgfsetstrokecolor{currentstroke}%
\pgfsetdash{}{0pt}%
\pgfpathmoveto{\pgfqpoint{0.631176in}{3.518457in}}%
\pgfpathlineto{\pgfqpoint{0.652938in}{3.395763in}}%
\pgfpathlineto{\pgfqpoint{0.674700in}{3.308372in}}%
\pgfpathlineto{\pgfqpoint{0.696461in}{3.250146in}}%
\pgfpathlineto{\pgfqpoint{0.718223in}{3.204408in}}%
\pgfpathlineto{\pgfqpoint{0.739985in}{3.166387in}}%
\pgfpathlineto{\pgfqpoint{0.761746in}{3.135372in}}%
\pgfpathlineto{\pgfqpoint{0.783508in}{3.109022in}}%
\pgfpathlineto{\pgfqpoint{0.805270in}{3.087422in}}%
\pgfpathlineto{\pgfqpoint{0.827031in}{3.066738in}}%
\pgfpathlineto{\pgfqpoint{0.848793in}{3.049294in}}%
\pgfpathlineto{\pgfqpoint{0.870555in}{3.032631in}}%
\pgfpathlineto{\pgfqpoint{0.892316in}{3.017188in}}%
\pgfpathlineto{\pgfqpoint{0.914078in}{3.001961in}}%
\pgfpathlineto{\pgfqpoint{0.935840in}{2.988393in}}%
\pgfpathlineto{\pgfqpoint{0.957601in}{2.974432in}}%
\pgfpathlineto{\pgfqpoint{0.979363in}{2.962368in}}%
\pgfpathlineto{\pgfqpoint{1.001125in}{2.950080in}}%
\pgfpathlineto{\pgfqpoint{1.022886in}{2.938639in}}%
\pgfpathlineto{\pgfqpoint{1.044648in}{2.928708in}}%
\pgfpathlineto{\pgfqpoint{1.066410in}{2.918283in}}%
\pgfpathlineto{\pgfqpoint{1.088171in}{2.908916in}}%
\pgfpathlineto{\pgfqpoint{1.109933in}{2.900126in}}%
\pgfpathlineto{\pgfqpoint{1.131695in}{2.891960in}}%
\pgfpathlineto{\pgfqpoint{1.153456in}{2.883887in}}%
\pgfpathlineto{\pgfqpoint{1.175218in}{2.876034in}}%
\pgfpathlineto{\pgfqpoint{1.196980in}{2.868743in}}%
\pgfpathlineto{\pgfqpoint{1.218741in}{2.861552in}}%
\pgfpathlineto{\pgfqpoint{1.240503in}{2.854574in}}%
\pgfpathlineto{\pgfqpoint{1.262265in}{2.846986in}}%
\pgfpathlineto{\pgfqpoint{1.284026in}{2.839867in}}%
\pgfpathlineto{\pgfqpoint{1.305788in}{2.833471in}}%
\pgfpathlineto{\pgfqpoint{1.327550in}{2.826919in}}%
\pgfpathlineto{\pgfqpoint{1.349311in}{2.820375in}}%
\pgfpathlineto{\pgfqpoint{1.371073in}{2.814395in}}%
\pgfpathlineto{\pgfqpoint{1.392835in}{2.808337in}}%
\pgfpathlineto{\pgfqpoint{1.414596in}{2.802638in}}%
\pgfpathlineto{\pgfqpoint{1.436358in}{2.797642in}}%
\pgfpathlineto{\pgfqpoint{1.458120in}{2.792257in}}%
\pgfpathlineto{\pgfqpoint{1.479881in}{2.786814in}}%
\pgfpathlineto{\pgfqpoint{1.501643in}{2.781913in}}%
\pgfpathlineto{\pgfqpoint{1.523405in}{2.776561in}}%
\pgfpathlineto{\pgfqpoint{1.545166in}{2.771333in}}%
\pgfpathlineto{\pgfqpoint{1.566928in}{2.766538in}}%
\pgfpathlineto{\pgfqpoint{1.588690in}{2.762336in}}%
\pgfpathlineto{\pgfqpoint{1.610451in}{2.757681in}}%
\pgfpathlineto{\pgfqpoint{1.632213in}{2.753168in}}%
\pgfpathlineto{\pgfqpoint{1.653975in}{2.748918in}}%
\pgfpathlineto{\pgfqpoint{1.675736in}{2.744900in}}%
\pgfpathlineto{\pgfqpoint{1.697498in}{2.740609in}}%
\pgfpathlineto{\pgfqpoint{1.719260in}{2.736406in}}%
\pgfpathlineto{\pgfqpoint{1.741021in}{2.732591in}}%
\pgfpathlineto{\pgfqpoint{1.762783in}{2.728476in}}%
\pgfpathlineto{\pgfqpoint{1.784545in}{2.724345in}}%
\pgfpathlineto{\pgfqpoint{1.806306in}{2.720541in}}%
\pgfpathlineto{\pgfqpoint{1.828068in}{2.716540in}}%
\pgfpathlineto{\pgfqpoint{1.849830in}{2.712719in}}%
\pgfpathlineto{\pgfqpoint{1.871591in}{2.708853in}}%
\pgfpathlineto{\pgfqpoint{1.893353in}{2.705212in}}%
\pgfpathlineto{\pgfqpoint{1.915115in}{2.701222in}}%
\pgfpathlineto{\pgfqpoint{1.936876in}{2.697526in}}%
\pgfpathlineto{\pgfqpoint{1.958638in}{2.693679in}}%
\pgfpathlineto{\pgfqpoint{1.980400in}{2.689852in}}%
\pgfpathlineto{\pgfqpoint{2.002161in}{2.686268in}}%
\pgfpathlineto{\pgfqpoint{2.023923in}{2.682565in}}%
\pgfpathlineto{\pgfqpoint{2.045685in}{2.678932in}}%
\pgfpathlineto{\pgfqpoint{2.067446in}{2.675261in}}%
\pgfpathlineto{\pgfqpoint{2.089208in}{2.671903in}}%
\pgfpathlineto{\pgfqpoint{2.110970in}{2.668450in}}%
\pgfpathlineto{\pgfqpoint{2.132731in}{2.665244in}}%
\pgfpathlineto{\pgfqpoint{2.154493in}{2.662164in}}%
\pgfpathlineto{\pgfqpoint{2.176255in}{2.658817in}}%
\pgfpathlineto{\pgfqpoint{2.198016in}{2.655796in}}%
\pgfpathlineto{\pgfqpoint{2.219778in}{2.652607in}}%
\pgfpathlineto{\pgfqpoint{2.241540in}{2.649245in}}%
\pgfpathlineto{\pgfqpoint{2.263301in}{2.646150in}}%
\pgfpathlineto{\pgfqpoint{2.285063in}{2.642911in}}%
\pgfpathlineto{\pgfqpoint{2.306825in}{2.639272in}}%
\pgfpathlineto{\pgfqpoint{2.328586in}{2.636116in}}%
\pgfpathlineto{\pgfqpoint{2.350348in}{2.632991in}}%
\pgfpathlineto{\pgfqpoint{2.372110in}{2.630259in}}%
\pgfpathlineto{\pgfqpoint{2.393871in}{2.627321in}}%
\pgfpathlineto{\pgfqpoint{2.415633in}{2.624555in}}%
\pgfpathlineto{\pgfqpoint{2.437395in}{2.621604in}}%
\pgfpathlineto{\pgfqpoint{2.459156in}{2.618688in}}%
\pgfpathlineto{\pgfqpoint{2.480918in}{2.615641in}}%
\pgfpathlineto{\pgfqpoint{2.502680in}{2.612961in}}%
\pgfpathlineto{\pgfqpoint{2.524441in}{2.610166in}}%
\pgfpathlineto{\pgfqpoint{2.546203in}{2.607432in}}%
\pgfpathlineto{\pgfqpoint{2.567965in}{2.604852in}}%
\pgfpathlineto{\pgfqpoint{2.589726in}{2.602164in}}%
\pgfpathlineto{\pgfqpoint{2.611488in}{2.599355in}}%
\pgfpathlineto{\pgfqpoint{2.633250in}{2.596650in}}%
\pgfpathlineto{\pgfqpoint{2.655011in}{2.594038in}}%
\pgfpathlineto{\pgfqpoint{2.676773in}{2.591293in}}%
\pgfpathlineto{\pgfqpoint{2.698535in}{2.588710in}}%
\pgfpathlineto{\pgfqpoint{2.720296in}{2.586160in}}%
\pgfpathlineto{\pgfqpoint{2.742058in}{2.583518in}}%
\pgfpathlineto{\pgfqpoint{2.763820in}{2.580903in}}%
\pgfpathlineto{\pgfqpoint{2.785581in}{2.578463in}}%
\pgfusepath{stroke}%
\end{pgfscope}%
\begin{pgfscope}%
\pgfpathrectangle{\pgfqpoint{0.609415in}{2.347992in}}{\pgfqpoint{2.219690in}{1.226201in}}%
\pgfusepath{clip}%
\pgfsetrectcap%
\pgfsetroundjoin%
\pgfsetlinewidth{1.003750pt}%
\definecolor{currentstroke}{rgb}{1.000000,0.172549,0.000000}%
\pgfsetstrokecolor{currentstroke}%
\pgfsetdash{}{0pt}%
\pgfpathmoveto{\pgfqpoint{0.631176in}{3.518457in}}%
\pgfpathlineto{\pgfqpoint{0.652938in}{3.472897in}}%
\pgfpathlineto{\pgfqpoint{0.674700in}{3.439179in}}%
\pgfpathlineto{\pgfqpoint{0.696461in}{3.415016in}}%
\pgfpathlineto{\pgfqpoint{0.718223in}{3.397737in}}%
\pgfpathlineto{\pgfqpoint{0.739985in}{3.382963in}}%
\pgfpathlineto{\pgfqpoint{0.761746in}{3.369978in}}%
\pgfpathlineto{\pgfqpoint{0.783508in}{3.359819in}}%
\pgfpathlineto{\pgfqpoint{0.805270in}{3.351094in}}%
\pgfpathlineto{\pgfqpoint{0.827031in}{3.342681in}}%
\pgfpathlineto{\pgfqpoint{0.848793in}{3.334996in}}%
\pgfpathlineto{\pgfqpoint{0.870555in}{3.327673in}}%
\pgfpathlineto{\pgfqpoint{0.892316in}{3.321688in}}%
\pgfpathlineto{\pgfqpoint{0.914078in}{3.316394in}}%
\pgfpathlineto{\pgfqpoint{0.935840in}{3.312000in}}%
\pgfpathlineto{\pgfqpoint{0.957601in}{3.306384in}}%
\pgfpathlineto{\pgfqpoint{0.979363in}{3.300575in}}%
\pgfpathlineto{\pgfqpoint{1.001125in}{3.295712in}}%
\pgfpathlineto{\pgfqpoint{1.022886in}{3.291460in}}%
\pgfpathlineto{\pgfqpoint{1.044648in}{3.286785in}}%
\pgfpathlineto{\pgfqpoint{1.066410in}{3.282608in}}%
\pgfpathlineto{\pgfqpoint{1.088171in}{3.278632in}}%
\pgfpathlineto{\pgfqpoint{1.109933in}{3.275031in}}%
\pgfpathlineto{\pgfqpoint{1.131695in}{3.271332in}}%
\pgfpathlineto{\pgfqpoint{1.153456in}{3.267925in}}%
\pgfpathlineto{\pgfqpoint{1.175218in}{3.264774in}}%
\pgfpathlineto{\pgfqpoint{1.196980in}{3.261475in}}%
\pgfpathlineto{\pgfqpoint{1.218741in}{3.258282in}}%
\pgfpathlineto{\pgfqpoint{1.240503in}{3.255665in}}%
\pgfpathlineto{\pgfqpoint{1.262265in}{3.252243in}}%
\pgfpathlineto{\pgfqpoint{1.284026in}{3.248993in}}%
\pgfpathlineto{\pgfqpoint{1.305788in}{3.246161in}}%
\pgfpathlineto{\pgfqpoint{1.327550in}{3.243010in}}%
\pgfpathlineto{\pgfqpoint{1.349311in}{3.240260in}}%
\pgfpathlineto{\pgfqpoint{1.371073in}{3.237808in}}%
\pgfpathlineto{\pgfqpoint{1.392835in}{3.234814in}}%
\pgfpathlineto{\pgfqpoint{1.414596in}{3.232137in}}%
\pgfpathlineto{\pgfqpoint{1.436358in}{3.229941in}}%
\pgfpathlineto{\pgfqpoint{1.458120in}{3.227650in}}%
\pgfpathlineto{\pgfqpoint{1.479881in}{3.225135in}}%
\pgfpathlineto{\pgfqpoint{1.501643in}{3.222564in}}%
\pgfpathlineto{\pgfqpoint{1.523405in}{3.219984in}}%
\pgfpathlineto{\pgfqpoint{1.545166in}{3.217732in}}%
\pgfpathlineto{\pgfqpoint{1.566928in}{3.215818in}}%
\pgfpathlineto{\pgfqpoint{1.588690in}{3.213925in}}%
\pgfpathlineto{\pgfqpoint{1.610451in}{3.211909in}}%
\pgfpathlineto{\pgfqpoint{1.632213in}{3.209931in}}%
\pgfpathlineto{\pgfqpoint{1.653975in}{3.207663in}}%
\pgfpathlineto{\pgfqpoint{1.675736in}{3.205751in}}%
\pgfpathlineto{\pgfqpoint{1.697498in}{3.203623in}}%
\pgfpathlineto{\pgfqpoint{1.719260in}{3.201540in}}%
\pgfpathlineto{\pgfqpoint{1.741021in}{3.199544in}}%
\pgfpathlineto{\pgfqpoint{1.762783in}{3.197532in}}%
\pgfpathlineto{\pgfqpoint{1.784545in}{3.195869in}}%
\pgfpathlineto{\pgfqpoint{1.806306in}{3.193824in}}%
\pgfpathlineto{\pgfqpoint{1.828068in}{3.192077in}}%
\pgfpathlineto{\pgfqpoint{1.849830in}{3.190368in}}%
\pgfpathlineto{\pgfqpoint{1.871591in}{3.188675in}}%
\pgfpathlineto{\pgfqpoint{1.893353in}{3.186671in}}%
\pgfpathlineto{\pgfqpoint{1.915115in}{3.184904in}}%
\pgfpathlineto{\pgfqpoint{1.936876in}{3.183284in}}%
\pgfpathlineto{\pgfqpoint{1.958638in}{3.181589in}}%
\pgfpathlineto{\pgfqpoint{1.980400in}{3.179879in}}%
\pgfpathlineto{\pgfqpoint{2.002161in}{3.178399in}}%
\pgfpathlineto{\pgfqpoint{2.023923in}{3.177012in}}%
\pgfpathlineto{\pgfqpoint{2.045685in}{3.175456in}}%
\pgfpathlineto{\pgfqpoint{2.067446in}{3.173831in}}%
\pgfpathlineto{\pgfqpoint{2.089208in}{3.172344in}}%
\pgfpathlineto{\pgfqpoint{2.110970in}{3.170952in}}%
\pgfpathlineto{\pgfqpoint{2.132731in}{3.169512in}}%
\pgfpathlineto{\pgfqpoint{2.154493in}{3.168294in}}%
\pgfpathlineto{\pgfqpoint{2.176255in}{3.166761in}}%
\pgfpathlineto{\pgfqpoint{2.198016in}{3.165212in}}%
\pgfpathlineto{\pgfqpoint{2.219778in}{3.163809in}}%
\pgfpathlineto{\pgfqpoint{2.241540in}{3.162467in}}%
\pgfpathlineto{\pgfqpoint{2.263301in}{3.161081in}}%
\pgfpathlineto{\pgfqpoint{2.285063in}{3.159634in}}%
\pgfpathlineto{\pgfqpoint{2.306825in}{3.158242in}}%
\pgfpathlineto{\pgfqpoint{2.328586in}{3.156899in}}%
\pgfpathlineto{\pgfqpoint{2.350348in}{3.155460in}}%
\pgfpathlineto{\pgfqpoint{2.372110in}{3.154136in}}%
\pgfpathlineto{\pgfqpoint{2.393871in}{3.152923in}}%
\pgfpathlineto{\pgfqpoint{2.415633in}{3.151754in}}%
\pgfpathlineto{\pgfqpoint{2.437395in}{3.150478in}}%
\pgfpathlineto{\pgfqpoint{2.459156in}{3.149110in}}%
\pgfpathlineto{\pgfqpoint{2.480918in}{3.147709in}}%
\pgfpathlineto{\pgfqpoint{2.502680in}{3.146336in}}%
\pgfpathlineto{\pgfqpoint{2.524441in}{3.144924in}}%
\pgfpathlineto{\pgfqpoint{2.546203in}{3.143607in}}%
\pgfpathlineto{\pgfqpoint{2.567965in}{3.142347in}}%
\pgfpathlineto{\pgfqpoint{2.589726in}{3.141136in}}%
\pgfpathlineto{\pgfqpoint{2.611488in}{3.139917in}}%
\pgfpathlineto{\pgfqpoint{2.633250in}{3.138608in}}%
\pgfpathlineto{\pgfqpoint{2.655011in}{3.137388in}}%
\pgfpathlineto{\pgfqpoint{2.676773in}{3.136226in}}%
\pgfpathlineto{\pgfqpoint{2.698535in}{3.134987in}}%
\pgfpathlineto{\pgfqpoint{2.720296in}{3.133822in}}%
\pgfpathlineto{\pgfqpoint{2.742058in}{3.132499in}}%
\pgfpathlineto{\pgfqpoint{2.763820in}{3.131232in}}%
\pgfpathlineto{\pgfqpoint{2.785581in}{3.130198in}}%
\pgfusepath{stroke}%
\end{pgfscope}%
\begin{pgfscope}%
\pgfpathrectangle{\pgfqpoint{0.609415in}{2.347992in}}{\pgfqpoint{2.219690in}{1.226201in}}%
\pgfusepath{clip}%
\pgfsetrectcap%
\pgfsetroundjoin%
\pgfsetlinewidth{1.003750pt}%
\definecolor{currentstroke}{rgb}{0.517647,0.356863,0.592157}%
\pgfsetstrokecolor{currentstroke}%
\pgfsetdash{}{0pt}%
\pgfpathmoveto{\pgfqpoint{0.631176in}{3.518457in}}%
\pgfpathlineto{\pgfqpoint{0.652938in}{3.182298in}}%
\pgfpathlineto{\pgfqpoint{0.674700in}{3.024377in}}%
\pgfpathlineto{\pgfqpoint{0.696461in}{2.936861in}}%
\pgfpathlineto{\pgfqpoint{0.718223in}{2.872918in}}%
\pgfpathlineto{\pgfqpoint{0.739985in}{2.826373in}}%
\pgfpathlineto{\pgfqpoint{0.761746in}{2.792067in}}%
\pgfpathlineto{\pgfqpoint{0.783508in}{2.767373in}}%
\pgfpathlineto{\pgfqpoint{0.805270in}{2.741950in}}%
\pgfpathlineto{\pgfqpoint{0.827031in}{2.720932in}}%
\pgfpathlineto{\pgfqpoint{0.848793in}{2.704244in}}%
\pgfpathlineto{\pgfqpoint{0.870555in}{2.693577in}}%
\pgfpathlineto{\pgfqpoint{0.892316in}{2.678653in}}%
\pgfpathlineto{\pgfqpoint{0.914078in}{2.666697in}}%
\pgfpathlineto{\pgfqpoint{0.935840in}{2.656783in}}%
\pgfpathlineto{\pgfqpoint{0.957601in}{2.647594in}}%
\pgfpathlineto{\pgfqpoint{0.979363in}{2.640343in}}%
\pgfpathlineto{\pgfqpoint{1.001125in}{2.633200in}}%
\pgfpathlineto{\pgfqpoint{1.022886in}{2.626176in}}%
\pgfpathlineto{\pgfqpoint{1.044648in}{2.617990in}}%
\pgfpathlineto{\pgfqpoint{1.066410in}{2.611391in}}%
\pgfpathlineto{\pgfqpoint{1.088171in}{2.604202in}}%
\pgfpathlineto{\pgfqpoint{1.109933in}{2.596594in}}%
\pgfpathlineto{\pgfqpoint{1.131695in}{2.591672in}}%
\pgfpathlineto{\pgfqpoint{1.153456in}{2.585621in}}%
\pgfpathlineto{\pgfqpoint{1.175218in}{2.579079in}}%
\pgfpathlineto{\pgfqpoint{1.196980in}{2.574681in}}%
\pgfpathlineto{\pgfqpoint{1.218741in}{2.568567in}}%
\pgfpathlineto{\pgfqpoint{1.240503in}{2.562818in}}%
\pgfpathlineto{\pgfqpoint{1.262265in}{2.558596in}}%
\pgfpathlineto{\pgfqpoint{1.284026in}{2.553645in}}%
\pgfpathlineto{\pgfqpoint{1.305788in}{2.550069in}}%
\pgfpathlineto{\pgfqpoint{1.327550in}{2.544734in}}%
\pgfpathlineto{\pgfqpoint{1.349311in}{2.541410in}}%
\pgfpathlineto{\pgfqpoint{1.371073in}{2.537242in}}%
\pgfpathlineto{\pgfqpoint{1.392835in}{2.532603in}}%
\pgfpathlineto{\pgfqpoint{1.414596in}{2.529543in}}%
\pgfpathlineto{\pgfqpoint{1.436358in}{2.526730in}}%
\pgfpathlineto{\pgfqpoint{1.458120in}{2.523662in}}%
\pgfpathlineto{\pgfqpoint{1.479881in}{2.519242in}}%
\pgfpathlineto{\pgfqpoint{1.501643in}{2.516955in}}%
\pgfpathlineto{\pgfqpoint{1.523405in}{2.514363in}}%
\pgfpathlineto{\pgfqpoint{1.545166in}{2.510648in}}%
\pgfpathlineto{\pgfqpoint{1.566928in}{2.508665in}}%
\pgfpathlineto{\pgfqpoint{1.588690in}{2.505792in}}%
\pgfpathlineto{\pgfqpoint{1.610451in}{2.501923in}}%
\pgfpathlineto{\pgfqpoint{1.632213in}{2.499519in}}%
\pgfpathlineto{\pgfqpoint{1.653975in}{2.496543in}}%
\pgfpathlineto{\pgfqpoint{1.675736in}{2.493643in}}%
\pgfpathlineto{\pgfqpoint{1.697498in}{2.491221in}}%
\pgfpathlineto{\pgfqpoint{1.719260in}{2.488432in}}%
\pgfpathlineto{\pgfqpoint{1.741021in}{2.485496in}}%
\pgfpathlineto{\pgfqpoint{1.762783in}{2.483118in}}%
\pgfpathlineto{\pgfqpoint{1.784545in}{2.481426in}}%
\pgfpathlineto{\pgfqpoint{1.806306in}{2.479828in}}%
\pgfpathlineto{\pgfqpoint{1.828068in}{2.478110in}}%
\pgfpathlineto{\pgfqpoint{1.849830in}{2.476476in}}%
\pgfpathlineto{\pgfqpoint{1.871591in}{2.473805in}}%
\pgfpathlineto{\pgfqpoint{1.893353in}{2.471921in}}%
\pgfpathlineto{\pgfqpoint{1.915115in}{2.469398in}}%
\pgfpathlineto{\pgfqpoint{1.936876in}{2.467593in}}%
\pgfpathlineto{\pgfqpoint{1.958638in}{2.465679in}}%
\pgfpathlineto{\pgfqpoint{1.980400in}{2.463771in}}%
\pgfpathlineto{\pgfqpoint{2.002161in}{2.461713in}}%
\pgfpathlineto{\pgfqpoint{2.023923in}{2.459422in}}%
\pgfpathlineto{\pgfqpoint{2.045685in}{2.457344in}}%
\pgfpathlineto{\pgfqpoint{2.067446in}{2.454687in}}%
\pgfpathlineto{\pgfqpoint{2.089208in}{2.452953in}}%
\pgfpathlineto{\pgfqpoint{2.110970in}{2.451081in}}%
\pgfpathlineto{\pgfqpoint{2.132731in}{2.449139in}}%
\pgfpathlineto{\pgfqpoint{2.154493in}{2.447371in}}%
\pgfpathlineto{\pgfqpoint{2.176255in}{2.446314in}}%
\pgfpathlineto{\pgfqpoint{2.198016in}{2.444903in}}%
\pgfpathlineto{\pgfqpoint{2.219778in}{2.443259in}}%
\pgfpathlineto{\pgfqpoint{2.241540in}{2.441235in}}%
\pgfpathlineto{\pgfqpoint{2.263301in}{2.439645in}}%
\pgfpathlineto{\pgfqpoint{2.285063in}{2.438034in}}%
\pgfpathlineto{\pgfqpoint{2.306825in}{2.436704in}}%
\pgfpathlineto{\pgfqpoint{2.328586in}{2.434803in}}%
\pgfpathlineto{\pgfqpoint{2.350348in}{2.432481in}}%
\pgfpathlineto{\pgfqpoint{2.372110in}{2.430929in}}%
\pgfpathlineto{\pgfqpoint{2.393871in}{2.428664in}}%
\pgfpathlineto{\pgfqpoint{2.415633in}{2.427605in}}%
\pgfpathlineto{\pgfqpoint{2.437395in}{2.425819in}}%
\pgfpathlineto{\pgfqpoint{2.459156in}{2.424153in}}%
\pgfpathlineto{\pgfqpoint{2.480918in}{2.423145in}}%
\pgfpathlineto{\pgfqpoint{2.502680in}{2.421533in}}%
\pgfpathlineto{\pgfqpoint{2.524441in}{2.420127in}}%
\pgfpathlineto{\pgfqpoint{2.546203in}{2.419031in}}%
\pgfpathlineto{\pgfqpoint{2.567965in}{2.417421in}}%
\pgfpathlineto{\pgfqpoint{2.589726in}{2.416194in}}%
\pgfpathlineto{\pgfqpoint{2.611488in}{2.414056in}}%
\pgfpathlineto{\pgfqpoint{2.633250in}{2.412658in}}%
\pgfpathlineto{\pgfqpoint{2.655011in}{2.411277in}}%
\pgfpathlineto{\pgfqpoint{2.676773in}{2.410171in}}%
\pgfpathlineto{\pgfqpoint{2.698535in}{2.408968in}}%
\pgfpathlineto{\pgfqpoint{2.720296in}{2.407758in}}%
\pgfpathlineto{\pgfqpoint{2.742058in}{2.406652in}}%
\pgfpathlineto{\pgfqpoint{2.763820in}{2.405310in}}%
\pgfpathlineto{\pgfqpoint{2.785581in}{2.403729in}}%
\pgfusepath{stroke}%
\end{pgfscope}%
\begin{pgfscope}%
\pgfpathrectangle{\pgfqpoint{0.609415in}{2.347992in}}{\pgfqpoint{2.219690in}{1.226201in}}%
\pgfusepath{clip}%
\pgfsetrectcap%
\pgfsetroundjoin%
\pgfsetlinewidth{1.003750pt}%
\definecolor{currentstroke}{rgb}{0.278431,0.278431,0.278431}%
\pgfsetstrokecolor{currentstroke}%
\pgfsetdash{}{0pt}%
\pgfpathmoveto{\pgfqpoint{0.631176in}{3.518457in}}%
\pgfpathlineto{\pgfqpoint{0.652938in}{3.511678in}}%
\pgfpathlineto{\pgfqpoint{0.674700in}{3.505504in}}%
\pgfpathlineto{\pgfqpoint{0.696461in}{3.500832in}}%
\pgfpathlineto{\pgfqpoint{0.718223in}{3.497772in}}%
\pgfpathlineto{\pgfqpoint{0.739985in}{3.494744in}}%
\pgfpathlineto{\pgfqpoint{0.761746in}{3.491828in}}%
\pgfpathlineto{\pgfqpoint{0.783508in}{3.489442in}}%
\pgfpathlineto{\pgfqpoint{0.805270in}{3.486879in}}%
\pgfpathlineto{\pgfqpoint{0.827031in}{3.484326in}}%
\pgfpathlineto{\pgfqpoint{0.848793in}{3.482371in}}%
\pgfpathlineto{\pgfqpoint{0.870555in}{3.480597in}}%
\pgfpathlineto{\pgfqpoint{0.892316in}{3.478881in}}%
\pgfpathlineto{\pgfqpoint{0.914078in}{3.477042in}}%
\pgfpathlineto{\pgfqpoint{0.935840in}{3.475245in}}%
\pgfpathlineto{\pgfqpoint{0.957601in}{3.473781in}}%
\pgfpathlineto{\pgfqpoint{0.979363in}{3.472003in}}%
\pgfpathlineto{\pgfqpoint{1.001125in}{3.470497in}}%
\pgfpathlineto{\pgfqpoint{1.022886in}{3.468946in}}%
\pgfpathlineto{\pgfqpoint{1.044648in}{3.467723in}}%
\pgfpathlineto{\pgfqpoint{1.066410in}{3.466064in}}%
\pgfpathlineto{\pgfqpoint{1.088171in}{3.464730in}}%
\pgfpathlineto{\pgfqpoint{1.109933in}{3.463494in}}%
\pgfpathlineto{\pgfqpoint{1.131695in}{3.461999in}}%
\pgfpathlineto{\pgfqpoint{1.153456in}{3.460607in}}%
\pgfpathlineto{\pgfqpoint{1.175218in}{3.459299in}}%
\pgfpathlineto{\pgfqpoint{1.196980in}{3.457974in}}%
\pgfpathlineto{\pgfqpoint{1.218741in}{3.456760in}}%
\pgfpathlineto{\pgfqpoint{1.240503in}{3.455670in}}%
\pgfpathlineto{\pgfqpoint{1.262265in}{3.454339in}}%
\pgfpathlineto{\pgfqpoint{1.284026in}{3.452801in}}%
\pgfpathlineto{\pgfqpoint{1.305788in}{3.451442in}}%
\pgfpathlineto{\pgfqpoint{1.327550in}{3.450244in}}%
\pgfpathlineto{\pgfqpoint{1.349311in}{3.448989in}}%
\pgfpathlineto{\pgfqpoint{1.371073in}{3.447680in}}%
\pgfpathlineto{\pgfqpoint{1.392835in}{3.446405in}}%
\pgfpathlineto{\pgfqpoint{1.414596in}{3.445025in}}%
\pgfpathlineto{\pgfqpoint{1.436358in}{3.443928in}}%
\pgfpathlineto{\pgfqpoint{1.458120in}{3.442898in}}%
\pgfpathlineto{\pgfqpoint{1.479881in}{3.441785in}}%
\pgfpathlineto{\pgfqpoint{1.501643in}{3.440861in}}%
\pgfpathlineto{\pgfqpoint{1.523405in}{3.439797in}}%
\pgfpathlineto{\pgfqpoint{1.545166in}{3.438769in}}%
\pgfpathlineto{\pgfqpoint{1.566928in}{3.437695in}}%
\pgfpathlineto{\pgfqpoint{1.588690in}{3.436597in}}%
\pgfpathlineto{\pgfqpoint{1.610451in}{3.435670in}}%
\pgfpathlineto{\pgfqpoint{1.632213in}{3.434684in}}%
\pgfpathlineto{\pgfqpoint{1.653975in}{3.433844in}}%
\pgfpathlineto{\pgfqpoint{1.675736in}{3.432868in}}%
\pgfpathlineto{\pgfqpoint{1.697498in}{3.431831in}}%
\pgfpathlineto{\pgfqpoint{1.719260in}{3.430954in}}%
\pgfpathlineto{\pgfqpoint{1.741021in}{3.430009in}}%
\pgfpathlineto{\pgfqpoint{1.762783in}{3.428912in}}%
\pgfpathlineto{\pgfqpoint{1.784545in}{3.427935in}}%
\pgfpathlineto{\pgfqpoint{1.806306in}{3.427103in}}%
\pgfpathlineto{\pgfqpoint{1.828068in}{3.426323in}}%
\pgfpathlineto{\pgfqpoint{1.849830in}{3.425419in}}%
\pgfpathlineto{\pgfqpoint{1.871591in}{3.424503in}}%
\pgfpathlineto{\pgfqpoint{1.893353in}{3.423752in}}%
\pgfpathlineto{\pgfqpoint{1.915115in}{3.422828in}}%
\pgfpathlineto{\pgfqpoint{1.936876in}{3.421933in}}%
\pgfpathlineto{\pgfqpoint{1.958638in}{3.420925in}}%
\pgfpathlineto{\pgfqpoint{1.980400in}{3.420018in}}%
\pgfpathlineto{\pgfqpoint{2.002161in}{3.419101in}}%
\pgfpathlineto{\pgfqpoint{2.023923in}{3.418195in}}%
\pgfpathlineto{\pgfqpoint{2.045685in}{3.417319in}}%
\pgfpathlineto{\pgfqpoint{2.067446in}{3.416526in}}%
\pgfpathlineto{\pgfqpoint{2.089208in}{3.415774in}}%
\pgfpathlineto{\pgfqpoint{2.110970in}{3.414966in}}%
\pgfpathlineto{\pgfqpoint{2.132731in}{3.414024in}}%
\pgfpathlineto{\pgfqpoint{2.154493in}{3.413239in}}%
\pgfpathlineto{\pgfqpoint{2.176255in}{3.412379in}}%
\pgfpathlineto{\pgfqpoint{2.198016in}{3.411462in}}%
\pgfpathlineto{\pgfqpoint{2.219778in}{3.410712in}}%
\pgfpathlineto{\pgfqpoint{2.241540in}{3.409897in}}%
\pgfpathlineto{\pgfqpoint{2.263301in}{3.409124in}}%
\pgfpathlineto{\pgfqpoint{2.285063in}{3.408374in}}%
\pgfpathlineto{\pgfqpoint{2.306825in}{3.407569in}}%
\pgfpathlineto{\pgfqpoint{2.328586in}{3.406787in}}%
\pgfpathlineto{\pgfqpoint{2.350348in}{3.405936in}}%
\pgfpathlineto{\pgfqpoint{2.372110in}{3.405241in}}%
\pgfpathlineto{\pgfqpoint{2.393871in}{3.404479in}}%
\pgfpathlineto{\pgfqpoint{2.415633in}{3.403721in}}%
\pgfpathlineto{\pgfqpoint{2.437395in}{3.403034in}}%
\pgfpathlineto{\pgfqpoint{2.459156in}{3.402329in}}%
\pgfpathlineto{\pgfqpoint{2.480918in}{3.401583in}}%
\pgfpathlineto{\pgfqpoint{2.502680in}{3.400801in}}%
\pgfpathlineto{\pgfqpoint{2.524441in}{3.400046in}}%
\pgfpathlineto{\pgfqpoint{2.546203in}{3.399335in}}%
\pgfpathlineto{\pgfqpoint{2.567965in}{3.398729in}}%
\pgfpathlineto{\pgfqpoint{2.589726in}{3.397950in}}%
\pgfpathlineto{\pgfqpoint{2.611488in}{3.397142in}}%
\pgfpathlineto{\pgfqpoint{2.633250in}{3.396378in}}%
\pgfpathlineto{\pgfqpoint{2.655011in}{3.395634in}}%
\pgfpathlineto{\pgfqpoint{2.676773in}{3.394959in}}%
\pgfpathlineto{\pgfqpoint{2.698535in}{3.394322in}}%
\pgfpathlineto{\pgfqpoint{2.720296in}{3.393674in}}%
\pgfpathlineto{\pgfqpoint{2.742058in}{3.392986in}}%
\pgfpathlineto{\pgfqpoint{2.763820in}{3.392277in}}%
\pgfpathlineto{\pgfqpoint{2.785581in}{3.391611in}}%
\pgfusepath{stroke}%
\end{pgfscope}%
\begin{pgfscope}%
\pgfsetrectcap%
\pgfsetmiterjoin%
\pgfsetlinewidth{0.501875pt}%
\definecolor{currentstroke}{rgb}{0.000000,0.000000,0.000000}%
\pgfsetstrokecolor{currentstroke}%
\pgfsetdash{}{0pt}%
\pgfpathmoveto{\pgfqpoint{0.609415in}{2.347992in}}%
\pgfpathlineto{\pgfqpoint{0.609415in}{3.574193in}}%
\pgfusepath{stroke}%
\end{pgfscope}%
\begin{pgfscope}%
\pgfsetrectcap%
\pgfsetmiterjoin%
\pgfsetlinewidth{0.501875pt}%
\definecolor{currentstroke}{rgb}{0.000000,0.000000,0.000000}%
\pgfsetstrokecolor{currentstroke}%
\pgfsetdash{}{0pt}%
\pgfpathmoveto{\pgfqpoint{2.829105in}{2.347992in}}%
\pgfpathlineto{\pgfqpoint{2.829105in}{3.574193in}}%
\pgfusepath{stroke}%
\end{pgfscope}%
\begin{pgfscope}%
\pgfsetrectcap%
\pgfsetmiterjoin%
\pgfsetlinewidth{0.501875pt}%
\definecolor{currentstroke}{rgb}{0.000000,0.000000,0.000000}%
\pgfsetstrokecolor{currentstroke}%
\pgfsetdash{}{0pt}%
\pgfpathmoveto{\pgfqpoint{0.609415in}{2.347992in}}%
\pgfpathlineto{\pgfqpoint{2.829105in}{2.347992in}}%
\pgfusepath{stroke}%
\end{pgfscope}%
\begin{pgfscope}%
\pgfsetrectcap%
\pgfsetmiterjoin%
\pgfsetlinewidth{0.501875pt}%
\definecolor{currentstroke}{rgb}{0.000000,0.000000,0.000000}%
\pgfsetstrokecolor{currentstroke}%
\pgfsetdash{}{0pt}%
\pgfpathmoveto{\pgfqpoint{0.609415in}{3.574193in}}%
\pgfpathlineto{\pgfqpoint{2.829105in}{3.574193in}}%
\pgfusepath{stroke}%
\end{pgfscope}%
\begin{pgfscope}%
\definecolor{textcolor}{rgb}{0.000000,0.000000,0.000000}%
\pgfsetstrokecolor{textcolor}%
\pgfsetfillcolor{textcolor}%
\pgftext[x=1.719260in,y=3.657526in,,base]{\color{textcolor}\rmfamily\fontsize{12.000000}{14.400000}\selectfont Vertrauenswürdigkeit}%
\end{pgfscope}%
\begin{pgfscope}%
\pgfsetbuttcap%
\pgfsetmiterjoin%
\definecolor{currentfill}{rgb}{1.000000,1.000000,1.000000}%
\pgfsetfillcolor{currentfill}%
\pgfsetlinewidth{0.000000pt}%
\definecolor{currentstroke}{rgb}{0.000000,0.000000,0.000000}%
\pgfsetstrokecolor{currentstroke}%
\pgfsetstrokeopacity{0.000000}%
\pgfsetdash{}{0pt}%
\pgfpathmoveto{\pgfqpoint{0.609415in}{0.422992in}}%
\pgfpathlineto{\pgfqpoint{2.829105in}{0.422992in}}%
\pgfpathlineto{\pgfqpoint{2.829105in}{1.649193in}}%
\pgfpathlineto{\pgfqpoint{0.609415in}{1.649193in}}%
\pgfpathlineto{\pgfqpoint{0.609415in}{0.422992in}}%
\pgfpathclose%
\pgfusepath{fill}%
\end{pgfscope}%
\begin{pgfscope}%
\pgfsetbuttcap%
\pgfsetroundjoin%
\definecolor{currentfill}{rgb}{0.000000,0.000000,0.000000}%
\pgfsetfillcolor{currentfill}%
\pgfsetlinewidth{0.501875pt}%
\definecolor{currentstroke}{rgb}{0.000000,0.000000,0.000000}%
\pgfsetstrokecolor{currentstroke}%
\pgfsetdash{}{0pt}%
\pgfsys@defobject{currentmarker}{\pgfqpoint{0.000000in}{0.000000in}}{\pgfqpoint{0.000000in}{0.041667in}}{%
\pgfpathmoveto{\pgfqpoint{0.000000in}{0.000000in}}%
\pgfpathlineto{\pgfqpoint{0.000000in}{0.041667in}}%
\pgfusepath{stroke,fill}%
}%
\begin{pgfscope}%
\pgfsys@transformshift{0.609415in}{0.422992in}%
\pgfsys@useobject{currentmarker}{}%
\end{pgfscope}%
\end{pgfscope}%
\begin{pgfscope}%
\pgfsetbuttcap%
\pgfsetroundjoin%
\definecolor{currentfill}{rgb}{0.000000,0.000000,0.000000}%
\pgfsetfillcolor{currentfill}%
\pgfsetlinewidth{0.501875pt}%
\definecolor{currentstroke}{rgb}{0.000000,0.000000,0.000000}%
\pgfsetstrokecolor{currentstroke}%
\pgfsetdash{}{0pt}%
\pgfsys@defobject{currentmarker}{\pgfqpoint{0.000000in}{-0.041667in}}{\pgfqpoint{0.000000in}{0.000000in}}{%
\pgfpathmoveto{\pgfqpoint{0.000000in}{0.000000in}}%
\pgfpathlineto{\pgfqpoint{0.000000in}{-0.041667in}}%
\pgfusepath{stroke,fill}%
}%
\begin{pgfscope}%
\pgfsys@transformshift{0.609415in}{1.649193in}%
\pgfsys@useobject{currentmarker}{}%
\end{pgfscope}%
\end{pgfscope}%
\begin{pgfscope}%
\definecolor{textcolor}{rgb}{0.000000,0.000000,0.000000}%
\pgfsetstrokecolor{textcolor}%
\pgfsetfillcolor{textcolor}%
\pgftext[x=0.609415in,y=0.374381in,,top]{\color{textcolor}\rmfamily\fontsize{10.000000}{12.000000}\selectfont \(\displaystyle {0}\)}%
\end{pgfscope}%
\begin{pgfscope}%
\pgfsetbuttcap%
\pgfsetroundjoin%
\definecolor{currentfill}{rgb}{0.000000,0.000000,0.000000}%
\pgfsetfillcolor{currentfill}%
\pgfsetlinewidth{0.501875pt}%
\definecolor{currentstroke}{rgb}{0.000000,0.000000,0.000000}%
\pgfsetstrokecolor{currentstroke}%
\pgfsetdash{}{0pt}%
\pgfsys@defobject{currentmarker}{\pgfqpoint{0.000000in}{0.000000in}}{\pgfqpoint{0.000000in}{0.041667in}}{%
\pgfpathmoveto{\pgfqpoint{0.000000in}{0.000000in}}%
\pgfpathlineto{\pgfqpoint{0.000000in}{0.041667in}}%
\pgfusepath{stroke,fill}%
}%
\begin{pgfscope}%
\pgfsys@transformshift{1.044648in}{0.422992in}%
\pgfsys@useobject{currentmarker}{}%
\end{pgfscope}%
\end{pgfscope}%
\begin{pgfscope}%
\pgfsetbuttcap%
\pgfsetroundjoin%
\definecolor{currentfill}{rgb}{0.000000,0.000000,0.000000}%
\pgfsetfillcolor{currentfill}%
\pgfsetlinewidth{0.501875pt}%
\definecolor{currentstroke}{rgb}{0.000000,0.000000,0.000000}%
\pgfsetstrokecolor{currentstroke}%
\pgfsetdash{}{0pt}%
\pgfsys@defobject{currentmarker}{\pgfqpoint{0.000000in}{-0.041667in}}{\pgfqpoint{0.000000in}{0.000000in}}{%
\pgfpathmoveto{\pgfqpoint{0.000000in}{0.000000in}}%
\pgfpathlineto{\pgfqpoint{0.000000in}{-0.041667in}}%
\pgfusepath{stroke,fill}%
}%
\begin{pgfscope}%
\pgfsys@transformshift{1.044648in}{1.649193in}%
\pgfsys@useobject{currentmarker}{}%
\end{pgfscope}%
\end{pgfscope}%
\begin{pgfscope}%
\definecolor{textcolor}{rgb}{0.000000,0.000000,0.000000}%
\pgfsetstrokecolor{textcolor}%
\pgfsetfillcolor{textcolor}%
\pgftext[x=1.044648in,y=0.374381in,,top]{\color{textcolor}\rmfamily\fontsize{10.000000}{12.000000}\selectfont \(\displaystyle {20}\)}%
\end{pgfscope}%
\begin{pgfscope}%
\pgfsetbuttcap%
\pgfsetroundjoin%
\definecolor{currentfill}{rgb}{0.000000,0.000000,0.000000}%
\pgfsetfillcolor{currentfill}%
\pgfsetlinewidth{0.501875pt}%
\definecolor{currentstroke}{rgb}{0.000000,0.000000,0.000000}%
\pgfsetstrokecolor{currentstroke}%
\pgfsetdash{}{0pt}%
\pgfsys@defobject{currentmarker}{\pgfqpoint{0.000000in}{0.000000in}}{\pgfqpoint{0.000000in}{0.041667in}}{%
\pgfpathmoveto{\pgfqpoint{0.000000in}{0.000000in}}%
\pgfpathlineto{\pgfqpoint{0.000000in}{0.041667in}}%
\pgfusepath{stroke,fill}%
}%
\begin{pgfscope}%
\pgfsys@transformshift{1.479881in}{0.422992in}%
\pgfsys@useobject{currentmarker}{}%
\end{pgfscope}%
\end{pgfscope}%
\begin{pgfscope}%
\pgfsetbuttcap%
\pgfsetroundjoin%
\definecolor{currentfill}{rgb}{0.000000,0.000000,0.000000}%
\pgfsetfillcolor{currentfill}%
\pgfsetlinewidth{0.501875pt}%
\definecolor{currentstroke}{rgb}{0.000000,0.000000,0.000000}%
\pgfsetstrokecolor{currentstroke}%
\pgfsetdash{}{0pt}%
\pgfsys@defobject{currentmarker}{\pgfqpoint{0.000000in}{-0.041667in}}{\pgfqpoint{0.000000in}{0.000000in}}{%
\pgfpathmoveto{\pgfqpoint{0.000000in}{0.000000in}}%
\pgfpathlineto{\pgfqpoint{0.000000in}{-0.041667in}}%
\pgfusepath{stroke,fill}%
}%
\begin{pgfscope}%
\pgfsys@transformshift{1.479881in}{1.649193in}%
\pgfsys@useobject{currentmarker}{}%
\end{pgfscope}%
\end{pgfscope}%
\begin{pgfscope}%
\definecolor{textcolor}{rgb}{0.000000,0.000000,0.000000}%
\pgfsetstrokecolor{textcolor}%
\pgfsetfillcolor{textcolor}%
\pgftext[x=1.479881in,y=0.374381in,,top]{\color{textcolor}\rmfamily\fontsize{10.000000}{12.000000}\selectfont \(\displaystyle {40}\)}%
\end{pgfscope}%
\begin{pgfscope}%
\pgfsetbuttcap%
\pgfsetroundjoin%
\definecolor{currentfill}{rgb}{0.000000,0.000000,0.000000}%
\pgfsetfillcolor{currentfill}%
\pgfsetlinewidth{0.501875pt}%
\definecolor{currentstroke}{rgb}{0.000000,0.000000,0.000000}%
\pgfsetstrokecolor{currentstroke}%
\pgfsetdash{}{0pt}%
\pgfsys@defobject{currentmarker}{\pgfqpoint{0.000000in}{0.000000in}}{\pgfqpoint{0.000000in}{0.041667in}}{%
\pgfpathmoveto{\pgfqpoint{0.000000in}{0.000000in}}%
\pgfpathlineto{\pgfqpoint{0.000000in}{0.041667in}}%
\pgfusepath{stroke,fill}%
}%
\begin{pgfscope}%
\pgfsys@transformshift{1.915115in}{0.422992in}%
\pgfsys@useobject{currentmarker}{}%
\end{pgfscope}%
\end{pgfscope}%
\begin{pgfscope}%
\pgfsetbuttcap%
\pgfsetroundjoin%
\definecolor{currentfill}{rgb}{0.000000,0.000000,0.000000}%
\pgfsetfillcolor{currentfill}%
\pgfsetlinewidth{0.501875pt}%
\definecolor{currentstroke}{rgb}{0.000000,0.000000,0.000000}%
\pgfsetstrokecolor{currentstroke}%
\pgfsetdash{}{0pt}%
\pgfsys@defobject{currentmarker}{\pgfqpoint{0.000000in}{-0.041667in}}{\pgfqpoint{0.000000in}{0.000000in}}{%
\pgfpathmoveto{\pgfqpoint{0.000000in}{0.000000in}}%
\pgfpathlineto{\pgfqpoint{0.000000in}{-0.041667in}}%
\pgfusepath{stroke,fill}%
}%
\begin{pgfscope}%
\pgfsys@transformshift{1.915115in}{1.649193in}%
\pgfsys@useobject{currentmarker}{}%
\end{pgfscope}%
\end{pgfscope}%
\begin{pgfscope}%
\definecolor{textcolor}{rgb}{0.000000,0.000000,0.000000}%
\pgfsetstrokecolor{textcolor}%
\pgfsetfillcolor{textcolor}%
\pgftext[x=1.915115in,y=0.374381in,,top]{\color{textcolor}\rmfamily\fontsize{10.000000}{12.000000}\selectfont \(\displaystyle {60}\)}%
\end{pgfscope}%
\begin{pgfscope}%
\pgfsetbuttcap%
\pgfsetroundjoin%
\definecolor{currentfill}{rgb}{0.000000,0.000000,0.000000}%
\pgfsetfillcolor{currentfill}%
\pgfsetlinewidth{0.501875pt}%
\definecolor{currentstroke}{rgb}{0.000000,0.000000,0.000000}%
\pgfsetstrokecolor{currentstroke}%
\pgfsetdash{}{0pt}%
\pgfsys@defobject{currentmarker}{\pgfqpoint{0.000000in}{0.000000in}}{\pgfqpoint{0.000000in}{0.041667in}}{%
\pgfpathmoveto{\pgfqpoint{0.000000in}{0.000000in}}%
\pgfpathlineto{\pgfqpoint{0.000000in}{0.041667in}}%
\pgfusepath{stroke,fill}%
}%
\begin{pgfscope}%
\pgfsys@transformshift{2.350348in}{0.422992in}%
\pgfsys@useobject{currentmarker}{}%
\end{pgfscope}%
\end{pgfscope}%
\begin{pgfscope}%
\pgfsetbuttcap%
\pgfsetroundjoin%
\definecolor{currentfill}{rgb}{0.000000,0.000000,0.000000}%
\pgfsetfillcolor{currentfill}%
\pgfsetlinewidth{0.501875pt}%
\definecolor{currentstroke}{rgb}{0.000000,0.000000,0.000000}%
\pgfsetstrokecolor{currentstroke}%
\pgfsetdash{}{0pt}%
\pgfsys@defobject{currentmarker}{\pgfqpoint{0.000000in}{-0.041667in}}{\pgfqpoint{0.000000in}{0.000000in}}{%
\pgfpathmoveto{\pgfqpoint{0.000000in}{0.000000in}}%
\pgfpathlineto{\pgfqpoint{0.000000in}{-0.041667in}}%
\pgfusepath{stroke,fill}%
}%
\begin{pgfscope}%
\pgfsys@transformshift{2.350348in}{1.649193in}%
\pgfsys@useobject{currentmarker}{}%
\end{pgfscope}%
\end{pgfscope}%
\begin{pgfscope}%
\definecolor{textcolor}{rgb}{0.000000,0.000000,0.000000}%
\pgfsetstrokecolor{textcolor}%
\pgfsetfillcolor{textcolor}%
\pgftext[x=2.350348in,y=0.374381in,,top]{\color{textcolor}\rmfamily\fontsize{10.000000}{12.000000}\selectfont \(\displaystyle {80}\)}%
\end{pgfscope}%
\begin{pgfscope}%
\pgfsetbuttcap%
\pgfsetroundjoin%
\definecolor{currentfill}{rgb}{0.000000,0.000000,0.000000}%
\pgfsetfillcolor{currentfill}%
\pgfsetlinewidth{0.501875pt}%
\definecolor{currentstroke}{rgb}{0.000000,0.000000,0.000000}%
\pgfsetstrokecolor{currentstroke}%
\pgfsetdash{}{0pt}%
\pgfsys@defobject{currentmarker}{\pgfqpoint{0.000000in}{0.000000in}}{\pgfqpoint{0.000000in}{0.041667in}}{%
\pgfpathmoveto{\pgfqpoint{0.000000in}{0.000000in}}%
\pgfpathlineto{\pgfqpoint{0.000000in}{0.041667in}}%
\pgfusepath{stroke,fill}%
}%
\begin{pgfscope}%
\pgfsys@transformshift{2.785581in}{0.422992in}%
\pgfsys@useobject{currentmarker}{}%
\end{pgfscope}%
\end{pgfscope}%
\begin{pgfscope}%
\pgfsetbuttcap%
\pgfsetroundjoin%
\definecolor{currentfill}{rgb}{0.000000,0.000000,0.000000}%
\pgfsetfillcolor{currentfill}%
\pgfsetlinewidth{0.501875pt}%
\definecolor{currentstroke}{rgb}{0.000000,0.000000,0.000000}%
\pgfsetstrokecolor{currentstroke}%
\pgfsetdash{}{0pt}%
\pgfsys@defobject{currentmarker}{\pgfqpoint{0.000000in}{-0.041667in}}{\pgfqpoint{0.000000in}{0.000000in}}{%
\pgfpathmoveto{\pgfqpoint{0.000000in}{0.000000in}}%
\pgfpathlineto{\pgfqpoint{0.000000in}{-0.041667in}}%
\pgfusepath{stroke,fill}%
}%
\begin{pgfscope}%
\pgfsys@transformshift{2.785581in}{1.649193in}%
\pgfsys@useobject{currentmarker}{}%
\end{pgfscope}%
\end{pgfscope}%
\begin{pgfscope}%
\definecolor{textcolor}{rgb}{0.000000,0.000000,0.000000}%
\pgfsetstrokecolor{textcolor}%
\pgfsetfillcolor{textcolor}%
\pgftext[x=2.785581in,y=0.374381in,,top]{\color{textcolor}\rmfamily\fontsize{10.000000}{12.000000}\selectfont \(\displaystyle {100}\)}%
\end{pgfscope}%
\begin{pgfscope}%
\pgfsetbuttcap%
\pgfsetroundjoin%
\definecolor{currentfill}{rgb}{0.000000,0.000000,0.000000}%
\pgfsetfillcolor{currentfill}%
\pgfsetlinewidth{0.501875pt}%
\definecolor{currentstroke}{rgb}{0.000000,0.000000,0.000000}%
\pgfsetstrokecolor{currentstroke}%
\pgfsetdash{}{0pt}%
\pgfsys@defobject{currentmarker}{\pgfqpoint{0.000000in}{0.000000in}}{\pgfqpoint{0.000000in}{0.020833in}}{%
\pgfpathmoveto{\pgfqpoint{0.000000in}{0.000000in}}%
\pgfpathlineto{\pgfqpoint{0.000000in}{0.020833in}}%
\pgfusepath{stroke,fill}%
}%
\begin{pgfscope}%
\pgfsys@transformshift{0.718223in}{0.422992in}%
\pgfsys@useobject{currentmarker}{}%
\end{pgfscope}%
\end{pgfscope}%
\begin{pgfscope}%
\pgfsetbuttcap%
\pgfsetroundjoin%
\definecolor{currentfill}{rgb}{0.000000,0.000000,0.000000}%
\pgfsetfillcolor{currentfill}%
\pgfsetlinewidth{0.501875pt}%
\definecolor{currentstroke}{rgb}{0.000000,0.000000,0.000000}%
\pgfsetstrokecolor{currentstroke}%
\pgfsetdash{}{0pt}%
\pgfsys@defobject{currentmarker}{\pgfqpoint{0.000000in}{-0.020833in}}{\pgfqpoint{0.000000in}{0.000000in}}{%
\pgfpathmoveto{\pgfqpoint{0.000000in}{0.000000in}}%
\pgfpathlineto{\pgfqpoint{0.000000in}{-0.020833in}}%
\pgfusepath{stroke,fill}%
}%
\begin{pgfscope}%
\pgfsys@transformshift{0.718223in}{1.649193in}%
\pgfsys@useobject{currentmarker}{}%
\end{pgfscope}%
\end{pgfscope}%
\begin{pgfscope}%
\pgfsetbuttcap%
\pgfsetroundjoin%
\definecolor{currentfill}{rgb}{0.000000,0.000000,0.000000}%
\pgfsetfillcolor{currentfill}%
\pgfsetlinewidth{0.501875pt}%
\definecolor{currentstroke}{rgb}{0.000000,0.000000,0.000000}%
\pgfsetstrokecolor{currentstroke}%
\pgfsetdash{}{0pt}%
\pgfsys@defobject{currentmarker}{\pgfqpoint{0.000000in}{0.000000in}}{\pgfqpoint{0.000000in}{0.020833in}}{%
\pgfpathmoveto{\pgfqpoint{0.000000in}{0.000000in}}%
\pgfpathlineto{\pgfqpoint{0.000000in}{0.020833in}}%
\pgfusepath{stroke,fill}%
}%
\begin{pgfscope}%
\pgfsys@transformshift{0.827031in}{0.422992in}%
\pgfsys@useobject{currentmarker}{}%
\end{pgfscope}%
\end{pgfscope}%
\begin{pgfscope}%
\pgfsetbuttcap%
\pgfsetroundjoin%
\definecolor{currentfill}{rgb}{0.000000,0.000000,0.000000}%
\pgfsetfillcolor{currentfill}%
\pgfsetlinewidth{0.501875pt}%
\definecolor{currentstroke}{rgb}{0.000000,0.000000,0.000000}%
\pgfsetstrokecolor{currentstroke}%
\pgfsetdash{}{0pt}%
\pgfsys@defobject{currentmarker}{\pgfqpoint{0.000000in}{-0.020833in}}{\pgfqpoint{0.000000in}{0.000000in}}{%
\pgfpathmoveto{\pgfqpoint{0.000000in}{0.000000in}}%
\pgfpathlineto{\pgfqpoint{0.000000in}{-0.020833in}}%
\pgfusepath{stroke,fill}%
}%
\begin{pgfscope}%
\pgfsys@transformshift{0.827031in}{1.649193in}%
\pgfsys@useobject{currentmarker}{}%
\end{pgfscope}%
\end{pgfscope}%
\begin{pgfscope}%
\pgfsetbuttcap%
\pgfsetroundjoin%
\definecolor{currentfill}{rgb}{0.000000,0.000000,0.000000}%
\pgfsetfillcolor{currentfill}%
\pgfsetlinewidth{0.501875pt}%
\definecolor{currentstroke}{rgb}{0.000000,0.000000,0.000000}%
\pgfsetstrokecolor{currentstroke}%
\pgfsetdash{}{0pt}%
\pgfsys@defobject{currentmarker}{\pgfqpoint{0.000000in}{0.000000in}}{\pgfqpoint{0.000000in}{0.020833in}}{%
\pgfpathmoveto{\pgfqpoint{0.000000in}{0.000000in}}%
\pgfpathlineto{\pgfqpoint{0.000000in}{0.020833in}}%
\pgfusepath{stroke,fill}%
}%
\begin{pgfscope}%
\pgfsys@transformshift{0.935840in}{0.422992in}%
\pgfsys@useobject{currentmarker}{}%
\end{pgfscope}%
\end{pgfscope}%
\begin{pgfscope}%
\pgfsetbuttcap%
\pgfsetroundjoin%
\definecolor{currentfill}{rgb}{0.000000,0.000000,0.000000}%
\pgfsetfillcolor{currentfill}%
\pgfsetlinewidth{0.501875pt}%
\definecolor{currentstroke}{rgb}{0.000000,0.000000,0.000000}%
\pgfsetstrokecolor{currentstroke}%
\pgfsetdash{}{0pt}%
\pgfsys@defobject{currentmarker}{\pgfqpoint{0.000000in}{-0.020833in}}{\pgfqpoint{0.000000in}{0.000000in}}{%
\pgfpathmoveto{\pgfqpoint{0.000000in}{0.000000in}}%
\pgfpathlineto{\pgfqpoint{0.000000in}{-0.020833in}}%
\pgfusepath{stroke,fill}%
}%
\begin{pgfscope}%
\pgfsys@transformshift{0.935840in}{1.649193in}%
\pgfsys@useobject{currentmarker}{}%
\end{pgfscope}%
\end{pgfscope}%
\begin{pgfscope}%
\pgfsetbuttcap%
\pgfsetroundjoin%
\definecolor{currentfill}{rgb}{0.000000,0.000000,0.000000}%
\pgfsetfillcolor{currentfill}%
\pgfsetlinewidth{0.501875pt}%
\definecolor{currentstroke}{rgb}{0.000000,0.000000,0.000000}%
\pgfsetstrokecolor{currentstroke}%
\pgfsetdash{}{0pt}%
\pgfsys@defobject{currentmarker}{\pgfqpoint{0.000000in}{0.000000in}}{\pgfqpoint{0.000000in}{0.020833in}}{%
\pgfpathmoveto{\pgfqpoint{0.000000in}{0.000000in}}%
\pgfpathlineto{\pgfqpoint{0.000000in}{0.020833in}}%
\pgfusepath{stroke,fill}%
}%
\begin{pgfscope}%
\pgfsys@transformshift{1.153456in}{0.422992in}%
\pgfsys@useobject{currentmarker}{}%
\end{pgfscope}%
\end{pgfscope}%
\begin{pgfscope}%
\pgfsetbuttcap%
\pgfsetroundjoin%
\definecolor{currentfill}{rgb}{0.000000,0.000000,0.000000}%
\pgfsetfillcolor{currentfill}%
\pgfsetlinewidth{0.501875pt}%
\definecolor{currentstroke}{rgb}{0.000000,0.000000,0.000000}%
\pgfsetstrokecolor{currentstroke}%
\pgfsetdash{}{0pt}%
\pgfsys@defobject{currentmarker}{\pgfqpoint{0.000000in}{-0.020833in}}{\pgfqpoint{0.000000in}{0.000000in}}{%
\pgfpathmoveto{\pgfqpoint{0.000000in}{0.000000in}}%
\pgfpathlineto{\pgfqpoint{0.000000in}{-0.020833in}}%
\pgfusepath{stroke,fill}%
}%
\begin{pgfscope}%
\pgfsys@transformshift{1.153456in}{1.649193in}%
\pgfsys@useobject{currentmarker}{}%
\end{pgfscope}%
\end{pgfscope}%
\begin{pgfscope}%
\pgfsetbuttcap%
\pgfsetroundjoin%
\definecolor{currentfill}{rgb}{0.000000,0.000000,0.000000}%
\pgfsetfillcolor{currentfill}%
\pgfsetlinewidth{0.501875pt}%
\definecolor{currentstroke}{rgb}{0.000000,0.000000,0.000000}%
\pgfsetstrokecolor{currentstroke}%
\pgfsetdash{}{0pt}%
\pgfsys@defobject{currentmarker}{\pgfqpoint{0.000000in}{0.000000in}}{\pgfqpoint{0.000000in}{0.020833in}}{%
\pgfpathmoveto{\pgfqpoint{0.000000in}{0.000000in}}%
\pgfpathlineto{\pgfqpoint{0.000000in}{0.020833in}}%
\pgfusepath{stroke,fill}%
}%
\begin{pgfscope}%
\pgfsys@transformshift{1.262265in}{0.422992in}%
\pgfsys@useobject{currentmarker}{}%
\end{pgfscope}%
\end{pgfscope}%
\begin{pgfscope}%
\pgfsetbuttcap%
\pgfsetroundjoin%
\definecolor{currentfill}{rgb}{0.000000,0.000000,0.000000}%
\pgfsetfillcolor{currentfill}%
\pgfsetlinewidth{0.501875pt}%
\definecolor{currentstroke}{rgb}{0.000000,0.000000,0.000000}%
\pgfsetstrokecolor{currentstroke}%
\pgfsetdash{}{0pt}%
\pgfsys@defobject{currentmarker}{\pgfqpoint{0.000000in}{-0.020833in}}{\pgfqpoint{0.000000in}{0.000000in}}{%
\pgfpathmoveto{\pgfqpoint{0.000000in}{0.000000in}}%
\pgfpathlineto{\pgfqpoint{0.000000in}{-0.020833in}}%
\pgfusepath{stroke,fill}%
}%
\begin{pgfscope}%
\pgfsys@transformshift{1.262265in}{1.649193in}%
\pgfsys@useobject{currentmarker}{}%
\end{pgfscope}%
\end{pgfscope}%
\begin{pgfscope}%
\pgfsetbuttcap%
\pgfsetroundjoin%
\definecolor{currentfill}{rgb}{0.000000,0.000000,0.000000}%
\pgfsetfillcolor{currentfill}%
\pgfsetlinewidth{0.501875pt}%
\definecolor{currentstroke}{rgb}{0.000000,0.000000,0.000000}%
\pgfsetstrokecolor{currentstroke}%
\pgfsetdash{}{0pt}%
\pgfsys@defobject{currentmarker}{\pgfqpoint{0.000000in}{0.000000in}}{\pgfqpoint{0.000000in}{0.020833in}}{%
\pgfpathmoveto{\pgfqpoint{0.000000in}{0.000000in}}%
\pgfpathlineto{\pgfqpoint{0.000000in}{0.020833in}}%
\pgfusepath{stroke,fill}%
}%
\begin{pgfscope}%
\pgfsys@transformshift{1.371073in}{0.422992in}%
\pgfsys@useobject{currentmarker}{}%
\end{pgfscope}%
\end{pgfscope}%
\begin{pgfscope}%
\pgfsetbuttcap%
\pgfsetroundjoin%
\definecolor{currentfill}{rgb}{0.000000,0.000000,0.000000}%
\pgfsetfillcolor{currentfill}%
\pgfsetlinewidth{0.501875pt}%
\definecolor{currentstroke}{rgb}{0.000000,0.000000,0.000000}%
\pgfsetstrokecolor{currentstroke}%
\pgfsetdash{}{0pt}%
\pgfsys@defobject{currentmarker}{\pgfqpoint{0.000000in}{-0.020833in}}{\pgfqpoint{0.000000in}{0.000000in}}{%
\pgfpathmoveto{\pgfqpoint{0.000000in}{0.000000in}}%
\pgfpathlineto{\pgfqpoint{0.000000in}{-0.020833in}}%
\pgfusepath{stroke,fill}%
}%
\begin{pgfscope}%
\pgfsys@transformshift{1.371073in}{1.649193in}%
\pgfsys@useobject{currentmarker}{}%
\end{pgfscope}%
\end{pgfscope}%
\begin{pgfscope}%
\pgfsetbuttcap%
\pgfsetroundjoin%
\definecolor{currentfill}{rgb}{0.000000,0.000000,0.000000}%
\pgfsetfillcolor{currentfill}%
\pgfsetlinewidth{0.501875pt}%
\definecolor{currentstroke}{rgb}{0.000000,0.000000,0.000000}%
\pgfsetstrokecolor{currentstroke}%
\pgfsetdash{}{0pt}%
\pgfsys@defobject{currentmarker}{\pgfqpoint{0.000000in}{0.000000in}}{\pgfqpoint{0.000000in}{0.020833in}}{%
\pgfpathmoveto{\pgfqpoint{0.000000in}{0.000000in}}%
\pgfpathlineto{\pgfqpoint{0.000000in}{0.020833in}}%
\pgfusepath{stroke,fill}%
}%
\begin{pgfscope}%
\pgfsys@transformshift{1.588690in}{0.422992in}%
\pgfsys@useobject{currentmarker}{}%
\end{pgfscope}%
\end{pgfscope}%
\begin{pgfscope}%
\pgfsetbuttcap%
\pgfsetroundjoin%
\definecolor{currentfill}{rgb}{0.000000,0.000000,0.000000}%
\pgfsetfillcolor{currentfill}%
\pgfsetlinewidth{0.501875pt}%
\definecolor{currentstroke}{rgb}{0.000000,0.000000,0.000000}%
\pgfsetstrokecolor{currentstroke}%
\pgfsetdash{}{0pt}%
\pgfsys@defobject{currentmarker}{\pgfqpoint{0.000000in}{-0.020833in}}{\pgfqpoint{0.000000in}{0.000000in}}{%
\pgfpathmoveto{\pgfqpoint{0.000000in}{0.000000in}}%
\pgfpathlineto{\pgfqpoint{0.000000in}{-0.020833in}}%
\pgfusepath{stroke,fill}%
}%
\begin{pgfscope}%
\pgfsys@transformshift{1.588690in}{1.649193in}%
\pgfsys@useobject{currentmarker}{}%
\end{pgfscope}%
\end{pgfscope}%
\begin{pgfscope}%
\pgfsetbuttcap%
\pgfsetroundjoin%
\definecolor{currentfill}{rgb}{0.000000,0.000000,0.000000}%
\pgfsetfillcolor{currentfill}%
\pgfsetlinewidth{0.501875pt}%
\definecolor{currentstroke}{rgb}{0.000000,0.000000,0.000000}%
\pgfsetstrokecolor{currentstroke}%
\pgfsetdash{}{0pt}%
\pgfsys@defobject{currentmarker}{\pgfqpoint{0.000000in}{0.000000in}}{\pgfqpoint{0.000000in}{0.020833in}}{%
\pgfpathmoveto{\pgfqpoint{0.000000in}{0.000000in}}%
\pgfpathlineto{\pgfqpoint{0.000000in}{0.020833in}}%
\pgfusepath{stroke,fill}%
}%
\begin{pgfscope}%
\pgfsys@transformshift{1.697498in}{0.422992in}%
\pgfsys@useobject{currentmarker}{}%
\end{pgfscope}%
\end{pgfscope}%
\begin{pgfscope}%
\pgfsetbuttcap%
\pgfsetroundjoin%
\definecolor{currentfill}{rgb}{0.000000,0.000000,0.000000}%
\pgfsetfillcolor{currentfill}%
\pgfsetlinewidth{0.501875pt}%
\definecolor{currentstroke}{rgb}{0.000000,0.000000,0.000000}%
\pgfsetstrokecolor{currentstroke}%
\pgfsetdash{}{0pt}%
\pgfsys@defobject{currentmarker}{\pgfqpoint{0.000000in}{-0.020833in}}{\pgfqpoint{0.000000in}{0.000000in}}{%
\pgfpathmoveto{\pgfqpoint{0.000000in}{0.000000in}}%
\pgfpathlineto{\pgfqpoint{0.000000in}{-0.020833in}}%
\pgfusepath{stroke,fill}%
}%
\begin{pgfscope}%
\pgfsys@transformshift{1.697498in}{1.649193in}%
\pgfsys@useobject{currentmarker}{}%
\end{pgfscope}%
\end{pgfscope}%
\begin{pgfscope}%
\pgfsetbuttcap%
\pgfsetroundjoin%
\definecolor{currentfill}{rgb}{0.000000,0.000000,0.000000}%
\pgfsetfillcolor{currentfill}%
\pgfsetlinewidth{0.501875pt}%
\definecolor{currentstroke}{rgb}{0.000000,0.000000,0.000000}%
\pgfsetstrokecolor{currentstroke}%
\pgfsetdash{}{0pt}%
\pgfsys@defobject{currentmarker}{\pgfqpoint{0.000000in}{0.000000in}}{\pgfqpoint{0.000000in}{0.020833in}}{%
\pgfpathmoveto{\pgfqpoint{0.000000in}{0.000000in}}%
\pgfpathlineto{\pgfqpoint{0.000000in}{0.020833in}}%
\pgfusepath{stroke,fill}%
}%
\begin{pgfscope}%
\pgfsys@transformshift{1.806306in}{0.422992in}%
\pgfsys@useobject{currentmarker}{}%
\end{pgfscope}%
\end{pgfscope}%
\begin{pgfscope}%
\pgfsetbuttcap%
\pgfsetroundjoin%
\definecolor{currentfill}{rgb}{0.000000,0.000000,0.000000}%
\pgfsetfillcolor{currentfill}%
\pgfsetlinewidth{0.501875pt}%
\definecolor{currentstroke}{rgb}{0.000000,0.000000,0.000000}%
\pgfsetstrokecolor{currentstroke}%
\pgfsetdash{}{0pt}%
\pgfsys@defobject{currentmarker}{\pgfqpoint{0.000000in}{-0.020833in}}{\pgfqpoint{0.000000in}{0.000000in}}{%
\pgfpathmoveto{\pgfqpoint{0.000000in}{0.000000in}}%
\pgfpathlineto{\pgfqpoint{0.000000in}{-0.020833in}}%
\pgfusepath{stroke,fill}%
}%
\begin{pgfscope}%
\pgfsys@transformshift{1.806306in}{1.649193in}%
\pgfsys@useobject{currentmarker}{}%
\end{pgfscope}%
\end{pgfscope}%
\begin{pgfscope}%
\pgfsetbuttcap%
\pgfsetroundjoin%
\definecolor{currentfill}{rgb}{0.000000,0.000000,0.000000}%
\pgfsetfillcolor{currentfill}%
\pgfsetlinewidth{0.501875pt}%
\definecolor{currentstroke}{rgb}{0.000000,0.000000,0.000000}%
\pgfsetstrokecolor{currentstroke}%
\pgfsetdash{}{0pt}%
\pgfsys@defobject{currentmarker}{\pgfqpoint{0.000000in}{0.000000in}}{\pgfqpoint{0.000000in}{0.020833in}}{%
\pgfpathmoveto{\pgfqpoint{0.000000in}{0.000000in}}%
\pgfpathlineto{\pgfqpoint{0.000000in}{0.020833in}}%
\pgfusepath{stroke,fill}%
}%
\begin{pgfscope}%
\pgfsys@transformshift{2.023923in}{0.422992in}%
\pgfsys@useobject{currentmarker}{}%
\end{pgfscope}%
\end{pgfscope}%
\begin{pgfscope}%
\pgfsetbuttcap%
\pgfsetroundjoin%
\definecolor{currentfill}{rgb}{0.000000,0.000000,0.000000}%
\pgfsetfillcolor{currentfill}%
\pgfsetlinewidth{0.501875pt}%
\definecolor{currentstroke}{rgb}{0.000000,0.000000,0.000000}%
\pgfsetstrokecolor{currentstroke}%
\pgfsetdash{}{0pt}%
\pgfsys@defobject{currentmarker}{\pgfqpoint{0.000000in}{-0.020833in}}{\pgfqpoint{0.000000in}{0.000000in}}{%
\pgfpathmoveto{\pgfqpoint{0.000000in}{0.000000in}}%
\pgfpathlineto{\pgfqpoint{0.000000in}{-0.020833in}}%
\pgfusepath{stroke,fill}%
}%
\begin{pgfscope}%
\pgfsys@transformshift{2.023923in}{1.649193in}%
\pgfsys@useobject{currentmarker}{}%
\end{pgfscope}%
\end{pgfscope}%
\begin{pgfscope}%
\pgfsetbuttcap%
\pgfsetroundjoin%
\definecolor{currentfill}{rgb}{0.000000,0.000000,0.000000}%
\pgfsetfillcolor{currentfill}%
\pgfsetlinewidth{0.501875pt}%
\definecolor{currentstroke}{rgb}{0.000000,0.000000,0.000000}%
\pgfsetstrokecolor{currentstroke}%
\pgfsetdash{}{0pt}%
\pgfsys@defobject{currentmarker}{\pgfqpoint{0.000000in}{0.000000in}}{\pgfqpoint{0.000000in}{0.020833in}}{%
\pgfpathmoveto{\pgfqpoint{0.000000in}{0.000000in}}%
\pgfpathlineto{\pgfqpoint{0.000000in}{0.020833in}}%
\pgfusepath{stroke,fill}%
}%
\begin{pgfscope}%
\pgfsys@transformshift{2.132731in}{0.422992in}%
\pgfsys@useobject{currentmarker}{}%
\end{pgfscope}%
\end{pgfscope}%
\begin{pgfscope}%
\pgfsetbuttcap%
\pgfsetroundjoin%
\definecolor{currentfill}{rgb}{0.000000,0.000000,0.000000}%
\pgfsetfillcolor{currentfill}%
\pgfsetlinewidth{0.501875pt}%
\definecolor{currentstroke}{rgb}{0.000000,0.000000,0.000000}%
\pgfsetstrokecolor{currentstroke}%
\pgfsetdash{}{0pt}%
\pgfsys@defobject{currentmarker}{\pgfqpoint{0.000000in}{-0.020833in}}{\pgfqpoint{0.000000in}{0.000000in}}{%
\pgfpathmoveto{\pgfqpoint{0.000000in}{0.000000in}}%
\pgfpathlineto{\pgfqpoint{0.000000in}{-0.020833in}}%
\pgfusepath{stroke,fill}%
}%
\begin{pgfscope}%
\pgfsys@transformshift{2.132731in}{1.649193in}%
\pgfsys@useobject{currentmarker}{}%
\end{pgfscope}%
\end{pgfscope}%
\begin{pgfscope}%
\pgfsetbuttcap%
\pgfsetroundjoin%
\definecolor{currentfill}{rgb}{0.000000,0.000000,0.000000}%
\pgfsetfillcolor{currentfill}%
\pgfsetlinewidth{0.501875pt}%
\definecolor{currentstroke}{rgb}{0.000000,0.000000,0.000000}%
\pgfsetstrokecolor{currentstroke}%
\pgfsetdash{}{0pt}%
\pgfsys@defobject{currentmarker}{\pgfqpoint{0.000000in}{0.000000in}}{\pgfqpoint{0.000000in}{0.020833in}}{%
\pgfpathmoveto{\pgfqpoint{0.000000in}{0.000000in}}%
\pgfpathlineto{\pgfqpoint{0.000000in}{0.020833in}}%
\pgfusepath{stroke,fill}%
}%
\begin{pgfscope}%
\pgfsys@transformshift{2.241540in}{0.422992in}%
\pgfsys@useobject{currentmarker}{}%
\end{pgfscope}%
\end{pgfscope}%
\begin{pgfscope}%
\pgfsetbuttcap%
\pgfsetroundjoin%
\definecolor{currentfill}{rgb}{0.000000,0.000000,0.000000}%
\pgfsetfillcolor{currentfill}%
\pgfsetlinewidth{0.501875pt}%
\definecolor{currentstroke}{rgb}{0.000000,0.000000,0.000000}%
\pgfsetstrokecolor{currentstroke}%
\pgfsetdash{}{0pt}%
\pgfsys@defobject{currentmarker}{\pgfqpoint{0.000000in}{-0.020833in}}{\pgfqpoint{0.000000in}{0.000000in}}{%
\pgfpathmoveto{\pgfqpoint{0.000000in}{0.000000in}}%
\pgfpathlineto{\pgfqpoint{0.000000in}{-0.020833in}}%
\pgfusepath{stroke,fill}%
}%
\begin{pgfscope}%
\pgfsys@transformshift{2.241540in}{1.649193in}%
\pgfsys@useobject{currentmarker}{}%
\end{pgfscope}%
\end{pgfscope}%
\begin{pgfscope}%
\pgfsetbuttcap%
\pgfsetroundjoin%
\definecolor{currentfill}{rgb}{0.000000,0.000000,0.000000}%
\pgfsetfillcolor{currentfill}%
\pgfsetlinewidth{0.501875pt}%
\definecolor{currentstroke}{rgb}{0.000000,0.000000,0.000000}%
\pgfsetstrokecolor{currentstroke}%
\pgfsetdash{}{0pt}%
\pgfsys@defobject{currentmarker}{\pgfqpoint{0.000000in}{0.000000in}}{\pgfqpoint{0.000000in}{0.020833in}}{%
\pgfpathmoveto{\pgfqpoint{0.000000in}{0.000000in}}%
\pgfpathlineto{\pgfqpoint{0.000000in}{0.020833in}}%
\pgfusepath{stroke,fill}%
}%
\begin{pgfscope}%
\pgfsys@transformshift{2.459156in}{0.422992in}%
\pgfsys@useobject{currentmarker}{}%
\end{pgfscope}%
\end{pgfscope}%
\begin{pgfscope}%
\pgfsetbuttcap%
\pgfsetroundjoin%
\definecolor{currentfill}{rgb}{0.000000,0.000000,0.000000}%
\pgfsetfillcolor{currentfill}%
\pgfsetlinewidth{0.501875pt}%
\definecolor{currentstroke}{rgb}{0.000000,0.000000,0.000000}%
\pgfsetstrokecolor{currentstroke}%
\pgfsetdash{}{0pt}%
\pgfsys@defobject{currentmarker}{\pgfqpoint{0.000000in}{-0.020833in}}{\pgfqpoint{0.000000in}{0.000000in}}{%
\pgfpathmoveto{\pgfqpoint{0.000000in}{0.000000in}}%
\pgfpathlineto{\pgfqpoint{0.000000in}{-0.020833in}}%
\pgfusepath{stroke,fill}%
}%
\begin{pgfscope}%
\pgfsys@transformshift{2.459156in}{1.649193in}%
\pgfsys@useobject{currentmarker}{}%
\end{pgfscope}%
\end{pgfscope}%
\begin{pgfscope}%
\pgfsetbuttcap%
\pgfsetroundjoin%
\definecolor{currentfill}{rgb}{0.000000,0.000000,0.000000}%
\pgfsetfillcolor{currentfill}%
\pgfsetlinewidth{0.501875pt}%
\definecolor{currentstroke}{rgb}{0.000000,0.000000,0.000000}%
\pgfsetstrokecolor{currentstroke}%
\pgfsetdash{}{0pt}%
\pgfsys@defobject{currentmarker}{\pgfqpoint{0.000000in}{0.000000in}}{\pgfqpoint{0.000000in}{0.020833in}}{%
\pgfpathmoveto{\pgfqpoint{0.000000in}{0.000000in}}%
\pgfpathlineto{\pgfqpoint{0.000000in}{0.020833in}}%
\pgfusepath{stroke,fill}%
}%
\begin{pgfscope}%
\pgfsys@transformshift{2.567965in}{0.422992in}%
\pgfsys@useobject{currentmarker}{}%
\end{pgfscope}%
\end{pgfscope}%
\begin{pgfscope}%
\pgfsetbuttcap%
\pgfsetroundjoin%
\definecolor{currentfill}{rgb}{0.000000,0.000000,0.000000}%
\pgfsetfillcolor{currentfill}%
\pgfsetlinewidth{0.501875pt}%
\definecolor{currentstroke}{rgb}{0.000000,0.000000,0.000000}%
\pgfsetstrokecolor{currentstroke}%
\pgfsetdash{}{0pt}%
\pgfsys@defobject{currentmarker}{\pgfqpoint{0.000000in}{-0.020833in}}{\pgfqpoint{0.000000in}{0.000000in}}{%
\pgfpathmoveto{\pgfqpoint{0.000000in}{0.000000in}}%
\pgfpathlineto{\pgfqpoint{0.000000in}{-0.020833in}}%
\pgfusepath{stroke,fill}%
}%
\begin{pgfscope}%
\pgfsys@transformshift{2.567965in}{1.649193in}%
\pgfsys@useobject{currentmarker}{}%
\end{pgfscope}%
\end{pgfscope}%
\begin{pgfscope}%
\pgfsetbuttcap%
\pgfsetroundjoin%
\definecolor{currentfill}{rgb}{0.000000,0.000000,0.000000}%
\pgfsetfillcolor{currentfill}%
\pgfsetlinewidth{0.501875pt}%
\definecolor{currentstroke}{rgb}{0.000000,0.000000,0.000000}%
\pgfsetstrokecolor{currentstroke}%
\pgfsetdash{}{0pt}%
\pgfsys@defobject{currentmarker}{\pgfqpoint{0.000000in}{0.000000in}}{\pgfqpoint{0.000000in}{0.020833in}}{%
\pgfpathmoveto{\pgfqpoint{0.000000in}{0.000000in}}%
\pgfpathlineto{\pgfqpoint{0.000000in}{0.020833in}}%
\pgfusepath{stroke,fill}%
}%
\begin{pgfscope}%
\pgfsys@transformshift{2.676773in}{0.422992in}%
\pgfsys@useobject{currentmarker}{}%
\end{pgfscope}%
\end{pgfscope}%
\begin{pgfscope}%
\pgfsetbuttcap%
\pgfsetroundjoin%
\definecolor{currentfill}{rgb}{0.000000,0.000000,0.000000}%
\pgfsetfillcolor{currentfill}%
\pgfsetlinewidth{0.501875pt}%
\definecolor{currentstroke}{rgb}{0.000000,0.000000,0.000000}%
\pgfsetstrokecolor{currentstroke}%
\pgfsetdash{}{0pt}%
\pgfsys@defobject{currentmarker}{\pgfqpoint{0.000000in}{-0.020833in}}{\pgfqpoint{0.000000in}{0.000000in}}{%
\pgfpathmoveto{\pgfqpoint{0.000000in}{0.000000in}}%
\pgfpathlineto{\pgfqpoint{0.000000in}{-0.020833in}}%
\pgfusepath{stroke,fill}%
}%
\begin{pgfscope}%
\pgfsys@transformshift{2.676773in}{1.649193in}%
\pgfsys@useobject{currentmarker}{}%
\end{pgfscope}%
\end{pgfscope}%
\begin{pgfscope}%
\definecolor{textcolor}{rgb}{0.000000,0.000000,0.000000}%
\pgfsetstrokecolor{textcolor}%
\pgfsetfillcolor{textcolor}%
\pgftext[x=1.719260in,y=0.184413in,,top]{\color{textcolor}\rmfamily\fontsize{10.000000}{12.000000}\selectfont \(\displaystyle K\)}%
\end{pgfscope}%
\begin{pgfscope}%
\pgfsetbuttcap%
\pgfsetroundjoin%
\definecolor{currentfill}{rgb}{0.000000,0.000000,0.000000}%
\pgfsetfillcolor{currentfill}%
\pgfsetlinewidth{0.501875pt}%
\definecolor{currentstroke}{rgb}{0.000000,0.000000,0.000000}%
\pgfsetstrokecolor{currentstroke}%
\pgfsetdash{}{0pt}%
\pgfsys@defobject{currentmarker}{\pgfqpoint{0.000000in}{0.000000in}}{\pgfqpoint{0.041667in}{0.000000in}}{%
\pgfpathmoveto{\pgfqpoint{0.000000in}{0.000000in}}%
\pgfpathlineto{\pgfqpoint{0.041667in}{0.000000in}}%
\pgfusepath{stroke,fill}%
}%
\begin{pgfscope}%
\pgfsys@transformshift{0.609415in}{0.469229in}%
\pgfsys@useobject{currentmarker}{}%
\end{pgfscope}%
\end{pgfscope}%
\begin{pgfscope}%
\pgfsetbuttcap%
\pgfsetroundjoin%
\definecolor{currentfill}{rgb}{0.000000,0.000000,0.000000}%
\pgfsetfillcolor{currentfill}%
\pgfsetlinewidth{0.501875pt}%
\definecolor{currentstroke}{rgb}{0.000000,0.000000,0.000000}%
\pgfsetstrokecolor{currentstroke}%
\pgfsetdash{}{0pt}%
\pgfsys@defobject{currentmarker}{\pgfqpoint{-0.041667in}{0.000000in}}{\pgfqpoint{-0.000000in}{0.000000in}}{%
\pgfpathmoveto{\pgfqpoint{-0.000000in}{0.000000in}}%
\pgfpathlineto{\pgfqpoint{-0.041667in}{0.000000in}}%
\pgfusepath{stroke,fill}%
}%
\begin{pgfscope}%
\pgfsys@transformshift{2.829105in}{0.469229in}%
\pgfsys@useobject{currentmarker}{}%
\end{pgfscope}%
\end{pgfscope}%
\begin{pgfscope}%
\definecolor{textcolor}{rgb}{0.000000,0.000000,0.000000}%
\pgfsetstrokecolor{textcolor}%
\pgfsetfillcolor{textcolor}%
\pgftext[x=0.244444in, y=0.416467in, left, base]{\color{textcolor}\rmfamily\fontsize{10.000000}{12.000000}\selectfont \(\displaystyle {0.925}\)}%
\end{pgfscope}%
\begin{pgfscope}%
\pgfsetbuttcap%
\pgfsetroundjoin%
\definecolor{currentfill}{rgb}{0.000000,0.000000,0.000000}%
\pgfsetfillcolor{currentfill}%
\pgfsetlinewidth{0.501875pt}%
\definecolor{currentstroke}{rgb}{0.000000,0.000000,0.000000}%
\pgfsetstrokecolor{currentstroke}%
\pgfsetdash{}{0pt}%
\pgfsys@defobject{currentmarker}{\pgfqpoint{0.000000in}{0.000000in}}{\pgfqpoint{0.041667in}{0.000000in}}{%
\pgfpathmoveto{\pgfqpoint{0.000000in}{0.000000in}}%
\pgfpathlineto{\pgfqpoint{0.041667in}{0.000000in}}%
\pgfusepath{stroke,fill}%
}%
\begin{pgfscope}%
\pgfsys@transformshift{0.609415in}{0.843971in}%
\pgfsys@useobject{currentmarker}{}%
\end{pgfscope}%
\end{pgfscope}%
\begin{pgfscope}%
\pgfsetbuttcap%
\pgfsetroundjoin%
\definecolor{currentfill}{rgb}{0.000000,0.000000,0.000000}%
\pgfsetfillcolor{currentfill}%
\pgfsetlinewidth{0.501875pt}%
\definecolor{currentstroke}{rgb}{0.000000,0.000000,0.000000}%
\pgfsetstrokecolor{currentstroke}%
\pgfsetdash{}{0pt}%
\pgfsys@defobject{currentmarker}{\pgfqpoint{-0.041667in}{0.000000in}}{\pgfqpoint{-0.000000in}{0.000000in}}{%
\pgfpathmoveto{\pgfqpoint{-0.000000in}{0.000000in}}%
\pgfpathlineto{\pgfqpoint{-0.041667in}{0.000000in}}%
\pgfusepath{stroke,fill}%
}%
\begin{pgfscope}%
\pgfsys@transformshift{2.829105in}{0.843971in}%
\pgfsys@useobject{currentmarker}{}%
\end{pgfscope}%
\end{pgfscope}%
\begin{pgfscope}%
\definecolor{textcolor}{rgb}{0.000000,0.000000,0.000000}%
\pgfsetstrokecolor{textcolor}%
\pgfsetfillcolor{textcolor}%
\pgftext[x=0.244444in, y=0.791210in, left, base]{\color{textcolor}\rmfamily\fontsize{10.000000}{12.000000}\selectfont \(\displaystyle {0.950}\)}%
\end{pgfscope}%
\begin{pgfscope}%
\pgfsetbuttcap%
\pgfsetroundjoin%
\definecolor{currentfill}{rgb}{0.000000,0.000000,0.000000}%
\pgfsetfillcolor{currentfill}%
\pgfsetlinewidth{0.501875pt}%
\definecolor{currentstroke}{rgb}{0.000000,0.000000,0.000000}%
\pgfsetstrokecolor{currentstroke}%
\pgfsetdash{}{0pt}%
\pgfsys@defobject{currentmarker}{\pgfqpoint{0.000000in}{0.000000in}}{\pgfqpoint{0.041667in}{0.000000in}}{%
\pgfpathmoveto{\pgfqpoint{0.000000in}{0.000000in}}%
\pgfpathlineto{\pgfqpoint{0.041667in}{0.000000in}}%
\pgfusepath{stroke,fill}%
}%
\begin{pgfscope}%
\pgfsys@transformshift{0.609415in}{1.218714in}%
\pgfsys@useobject{currentmarker}{}%
\end{pgfscope}%
\end{pgfscope}%
\begin{pgfscope}%
\pgfsetbuttcap%
\pgfsetroundjoin%
\definecolor{currentfill}{rgb}{0.000000,0.000000,0.000000}%
\pgfsetfillcolor{currentfill}%
\pgfsetlinewidth{0.501875pt}%
\definecolor{currentstroke}{rgb}{0.000000,0.000000,0.000000}%
\pgfsetstrokecolor{currentstroke}%
\pgfsetdash{}{0pt}%
\pgfsys@defobject{currentmarker}{\pgfqpoint{-0.041667in}{0.000000in}}{\pgfqpoint{-0.000000in}{0.000000in}}{%
\pgfpathmoveto{\pgfqpoint{-0.000000in}{0.000000in}}%
\pgfpathlineto{\pgfqpoint{-0.041667in}{0.000000in}}%
\pgfusepath{stroke,fill}%
}%
\begin{pgfscope}%
\pgfsys@transformshift{2.829105in}{1.218714in}%
\pgfsys@useobject{currentmarker}{}%
\end{pgfscope}%
\end{pgfscope}%
\begin{pgfscope}%
\definecolor{textcolor}{rgb}{0.000000,0.000000,0.000000}%
\pgfsetstrokecolor{textcolor}%
\pgfsetfillcolor{textcolor}%
\pgftext[x=0.244444in, y=1.165953in, left, base]{\color{textcolor}\rmfamily\fontsize{10.000000}{12.000000}\selectfont \(\displaystyle {0.975}\)}%
\end{pgfscope}%
\begin{pgfscope}%
\pgfsetbuttcap%
\pgfsetroundjoin%
\definecolor{currentfill}{rgb}{0.000000,0.000000,0.000000}%
\pgfsetfillcolor{currentfill}%
\pgfsetlinewidth{0.501875pt}%
\definecolor{currentstroke}{rgb}{0.000000,0.000000,0.000000}%
\pgfsetstrokecolor{currentstroke}%
\pgfsetdash{}{0pt}%
\pgfsys@defobject{currentmarker}{\pgfqpoint{0.000000in}{0.000000in}}{\pgfqpoint{0.041667in}{0.000000in}}{%
\pgfpathmoveto{\pgfqpoint{0.000000in}{0.000000in}}%
\pgfpathlineto{\pgfqpoint{0.041667in}{0.000000in}}%
\pgfusepath{stroke,fill}%
}%
\begin{pgfscope}%
\pgfsys@transformshift{0.609415in}{1.593457in}%
\pgfsys@useobject{currentmarker}{}%
\end{pgfscope}%
\end{pgfscope}%
\begin{pgfscope}%
\pgfsetbuttcap%
\pgfsetroundjoin%
\definecolor{currentfill}{rgb}{0.000000,0.000000,0.000000}%
\pgfsetfillcolor{currentfill}%
\pgfsetlinewidth{0.501875pt}%
\definecolor{currentstroke}{rgb}{0.000000,0.000000,0.000000}%
\pgfsetstrokecolor{currentstroke}%
\pgfsetdash{}{0pt}%
\pgfsys@defobject{currentmarker}{\pgfqpoint{-0.041667in}{0.000000in}}{\pgfqpoint{-0.000000in}{0.000000in}}{%
\pgfpathmoveto{\pgfqpoint{-0.000000in}{0.000000in}}%
\pgfpathlineto{\pgfqpoint{-0.041667in}{0.000000in}}%
\pgfusepath{stroke,fill}%
}%
\begin{pgfscope}%
\pgfsys@transformshift{2.829105in}{1.593457in}%
\pgfsys@useobject{currentmarker}{}%
\end{pgfscope}%
\end{pgfscope}%
\begin{pgfscope}%
\definecolor{textcolor}{rgb}{0.000000,0.000000,0.000000}%
\pgfsetstrokecolor{textcolor}%
\pgfsetfillcolor{textcolor}%
\pgftext[x=0.244444in, y=1.540695in, left, base]{\color{textcolor}\rmfamily\fontsize{10.000000}{12.000000}\selectfont \(\displaystyle {1.000}\)}%
\end{pgfscope}%
\begin{pgfscope}%
\pgfsetbuttcap%
\pgfsetroundjoin%
\definecolor{currentfill}{rgb}{0.000000,0.000000,0.000000}%
\pgfsetfillcolor{currentfill}%
\pgfsetlinewidth{0.501875pt}%
\definecolor{currentstroke}{rgb}{0.000000,0.000000,0.000000}%
\pgfsetstrokecolor{currentstroke}%
\pgfsetdash{}{0pt}%
\pgfsys@defobject{currentmarker}{\pgfqpoint{0.000000in}{0.000000in}}{\pgfqpoint{0.020833in}{0.000000in}}{%
\pgfpathmoveto{\pgfqpoint{0.000000in}{0.000000in}}%
\pgfpathlineto{\pgfqpoint{0.020833in}{0.000000in}}%
\pgfusepath{stroke,fill}%
}%
\begin{pgfscope}%
\pgfsys@transformshift{0.609415in}{0.544177in}%
\pgfsys@useobject{currentmarker}{}%
\end{pgfscope}%
\end{pgfscope}%
\begin{pgfscope}%
\pgfsetbuttcap%
\pgfsetroundjoin%
\definecolor{currentfill}{rgb}{0.000000,0.000000,0.000000}%
\pgfsetfillcolor{currentfill}%
\pgfsetlinewidth{0.501875pt}%
\definecolor{currentstroke}{rgb}{0.000000,0.000000,0.000000}%
\pgfsetstrokecolor{currentstroke}%
\pgfsetdash{}{0pt}%
\pgfsys@defobject{currentmarker}{\pgfqpoint{-0.020833in}{0.000000in}}{\pgfqpoint{-0.000000in}{0.000000in}}{%
\pgfpathmoveto{\pgfqpoint{-0.000000in}{0.000000in}}%
\pgfpathlineto{\pgfqpoint{-0.020833in}{0.000000in}}%
\pgfusepath{stroke,fill}%
}%
\begin{pgfscope}%
\pgfsys@transformshift{2.829105in}{0.544177in}%
\pgfsys@useobject{currentmarker}{}%
\end{pgfscope}%
\end{pgfscope}%
\begin{pgfscope}%
\pgfsetbuttcap%
\pgfsetroundjoin%
\definecolor{currentfill}{rgb}{0.000000,0.000000,0.000000}%
\pgfsetfillcolor{currentfill}%
\pgfsetlinewidth{0.501875pt}%
\definecolor{currentstroke}{rgb}{0.000000,0.000000,0.000000}%
\pgfsetstrokecolor{currentstroke}%
\pgfsetdash{}{0pt}%
\pgfsys@defobject{currentmarker}{\pgfqpoint{0.000000in}{0.000000in}}{\pgfqpoint{0.020833in}{0.000000in}}{%
\pgfpathmoveto{\pgfqpoint{0.000000in}{0.000000in}}%
\pgfpathlineto{\pgfqpoint{0.020833in}{0.000000in}}%
\pgfusepath{stroke,fill}%
}%
\begin{pgfscope}%
\pgfsys@transformshift{0.609415in}{0.619126in}%
\pgfsys@useobject{currentmarker}{}%
\end{pgfscope}%
\end{pgfscope}%
\begin{pgfscope}%
\pgfsetbuttcap%
\pgfsetroundjoin%
\definecolor{currentfill}{rgb}{0.000000,0.000000,0.000000}%
\pgfsetfillcolor{currentfill}%
\pgfsetlinewidth{0.501875pt}%
\definecolor{currentstroke}{rgb}{0.000000,0.000000,0.000000}%
\pgfsetstrokecolor{currentstroke}%
\pgfsetdash{}{0pt}%
\pgfsys@defobject{currentmarker}{\pgfqpoint{-0.020833in}{0.000000in}}{\pgfqpoint{-0.000000in}{0.000000in}}{%
\pgfpathmoveto{\pgfqpoint{-0.000000in}{0.000000in}}%
\pgfpathlineto{\pgfqpoint{-0.020833in}{0.000000in}}%
\pgfusepath{stroke,fill}%
}%
\begin{pgfscope}%
\pgfsys@transformshift{2.829105in}{0.619126in}%
\pgfsys@useobject{currentmarker}{}%
\end{pgfscope}%
\end{pgfscope}%
\begin{pgfscope}%
\pgfsetbuttcap%
\pgfsetroundjoin%
\definecolor{currentfill}{rgb}{0.000000,0.000000,0.000000}%
\pgfsetfillcolor{currentfill}%
\pgfsetlinewidth{0.501875pt}%
\definecolor{currentstroke}{rgb}{0.000000,0.000000,0.000000}%
\pgfsetstrokecolor{currentstroke}%
\pgfsetdash{}{0pt}%
\pgfsys@defobject{currentmarker}{\pgfqpoint{0.000000in}{0.000000in}}{\pgfqpoint{0.020833in}{0.000000in}}{%
\pgfpathmoveto{\pgfqpoint{0.000000in}{0.000000in}}%
\pgfpathlineto{\pgfqpoint{0.020833in}{0.000000in}}%
\pgfusepath{stroke,fill}%
}%
\begin{pgfscope}%
\pgfsys@transformshift{0.609415in}{0.694074in}%
\pgfsys@useobject{currentmarker}{}%
\end{pgfscope}%
\end{pgfscope}%
\begin{pgfscope}%
\pgfsetbuttcap%
\pgfsetroundjoin%
\definecolor{currentfill}{rgb}{0.000000,0.000000,0.000000}%
\pgfsetfillcolor{currentfill}%
\pgfsetlinewidth{0.501875pt}%
\definecolor{currentstroke}{rgb}{0.000000,0.000000,0.000000}%
\pgfsetstrokecolor{currentstroke}%
\pgfsetdash{}{0pt}%
\pgfsys@defobject{currentmarker}{\pgfqpoint{-0.020833in}{0.000000in}}{\pgfqpoint{-0.000000in}{0.000000in}}{%
\pgfpathmoveto{\pgfqpoint{-0.000000in}{0.000000in}}%
\pgfpathlineto{\pgfqpoint{-0.020833in}{0.000000in}}%
\pgfusepath{stroke,fill}%
}%
\begin{pgfscope}%
\pgfsys@transformshift{2.829105in}{0.694074in}%
\pgfsys@useobject{currentmarker}{}%
\end{pgfscope}%
\end{pgfscope}%
\begin{pgfscope}%
\pgfsetbuttcap%
\pgfsetroundjoin%
\definecolor{currentfill}{rgb}{0.000000,0.000000,0.000000}%
\pgfsetfillcolor{currentfill}%
\pgfsetlinewidth{0.501875pt}%
\definecolor{currentstroke}{rgb}{0.000000,0.000000,0.000000}%
\pgfsetstrokecolor{currentstroke}%
\pgfsetdash{}{0pt}%
\pgfsys@defobject{currentmarker}{\pgfqpoint{0.000000in}{0.000000in}}{\pgfqpoint{0.020833in}{0.000000in}}{%
\pgfpathmoveto{\pgfqpoint{0.000000in}{0.000000in}}%
\pgfpathlineto{\pgfqpoint{0.020833in}{0.000000in}}%
\pgfusepath{stroke,fill}%
}%
\begin{pgfscope}%
\pgfsys@transformshift{0.609415in}{0.769023in}%
\pgfsys@useobject{currentmarker}{}%
\end{pgfscope}%
\end{pgfscope}%
\begin{pgfscope}%
\pgfsetbuttcap%
\pgfsetroundjoin%
\definecolor{currentfill}{rgb}{0.000000,0.000000,0.000000}%
\pgfsetfillcolor{currentfill}%
\pgfsetlinewidth{0.501875pt}%
\definecolor{currentstroke}{rgb}{0.000000,0.000000,0.000000}%
\pgfsetstrokecolor{currentstroke}%
\pgfsetdash{}{0pt}%
\pgfsys@defobject{currentmarker}{\pgfqpoint{-0.020833in}{0.000000in}}{\pgfqpoint{-0.000000in}{0.000000in}}{%
\pgfpathmoveto{\pgfqpoint{-0.000000in}{0.000000in}}%
\pgfpathlineto{\pgfqpoint{-0.020833in}{0.000000in}}%
\pgfusepath{stroke,fill}%
}%
\begin{pgfscope}%
\pgfsys@transformshift{2.829105in}{0.769023in}%
\pgfsys@useobject{currentmarker}{}%
\end{pgfscope}%
\end{pgfscope}%
\begin{pgfscope}%
\pgfsetbuttcap%
\pgfsetroundjoin%
\definecolor{currentfill}{rgb}{0.000000,0.000000,0.000000}%
\pgfsetfillcolor{currentfill}%
\pgfsetlinewidth{0.501875pt}%
\definecolor{currentstroke}{rgb}{0.000000,0.000000,0.000000}%
\pgfsetstrokecolor{currentstroke}%
\pgfsetdash{}{0pt}%
\pgfsys@defobject{currentmarker}{\pgfqpoint{0.000000in}{0.000000in}}{\pgfqpoint{0.020833in}{0.000000in}}{%
\pgfpathmoveto{\pgfqpoint{0.000000in}{0.000000in}}%
\pgfpathlineto{\pgfqpoint{0.020833in}{0.000000in}}%
\pgfusepath{stroke,fill}%
}%
\begin{pgfscope}%
\pgfsys@transformshift{0.609415in}{0.918920in}%
\pgfsys@useobject{currentmarker}{}%
\end{pgfscope}%
\end{pgfscope}%
\begin{pgfscope}%
\pgfsetbuttcap%
\pgfsetroundjoin%
\definecolor{currentfill}{rgb}{0.000000,0.000000,0.000000}%
\pgfsetfillcolor{currentfill}%
\pgfsetlinewidth{0.501875pt}%
\definecolor{currentstroke}{rgb}{0.000000,0.000000,0.000000}%
\pgfsetstrokecolor{currentstroke}%
\pgfsetdash{}{0pt}%
\pgfsys@defobject{currentmarker}{\pgfqpoint{-0.020833in}{0.000000in}}{\pgfqpoint{-0.000000in}{0.000000in}}{%
\pgfpathmoveto{\pgfqpoint{-0.000000in}{0.000000in}}%
\pgfpathlineto{\pgfqpoint{-0.020833in}{0.000000in}}%
\pgfusepath{stroke,fill}%
}%
\begin{pgfscope}%
\pgfsys@transformshift{2.829105in}{0.918920in}%
\pgfsys@useobject{currentmarker}{}%
\end{pgfscope}%
\end{pgfscope}%
\begin{pgfscope}%
\pgfsetbuttcap%
\pgfsetroundjoin%
\definecolor{currentfill}{rgb}{0.000000,0.000000,0.000000}%
\pgfsetfillcolor{currentfill}%
\pgfsetlinewidth{0.501875pt}%
\definecolor{currentstroke}{rgb}{0.000000,0.000000,0.000000}%
\pgfsetstrokecolor{currentstroke}%
\pgfsetdash{}{0pt}%
\pgfsys@defobject{currentmarker}{\pgfqpoint{0.000000in}{0.000000in}}{\pgfqpoint{0.020833in}{0.000000in}}{%
\pgfpathmoveto{\pgfqpoint{0.000000in}{0.000000in}}%
\pgfpathlineto{\pgfqpoint{0.020833in}{0.000000in}}%
\pgfusepath{stroke,fill}%
}%
\begin{pgfscope}%
\pgfsys@transformshift{0.609415in}{0.993868in}%
\pgfsys@useobject{currentmarker}{}%
\end{pgfscope}%
\end{pgfscope}%
\begin{pgfscope}%
\pgfsetbuttcap%
\pgfsetroundjoin%
\definecolor{currentfill}{rgb}{0.000000,0.000000,0.000000}%
\pgfsetfillcolor{currentfill}%
\pgfsetlinewidth{0.501875pt}%
\definecolor{currentstroke}{rgb}{0.000000,0.000000,0.000000}%
\pgfsetstrokecolor{currentstroke}%
\pgfsetdash{}{0pt}%
\pgfsys@defobject{currentmarker}{\pgfqpoint{-0.020833in}{0.000000in}}{\pgfqpoint{-0.000000in}{0.000000in}}{%
\pgfpathmoveto{\pgfqpoint{-0.000000in}{0.000000in}}%
\pgfpathlineto{\pgfqpoint{-0.020833in}{0.000000in}}%
\pgfusepath{stroke,fill}%
}%
\begin{pgfscope}%
\pgfsys@transformshift{2.829105in}{0.993868in}%
\pgfsys@useobject{currentmarker}{}%
\end{pgfscope}%
\end{pgfscope}%
\begin{pgfscope}%
\pgfsetbuttcap%
\pgfsetroundjoin%
\definecolor{currentfill}{rgb}{0.000000,0.000000,0.000000}%
\pgfsetfillcolor{currentfill}%
\pgfsetlinewidth{0.501875pt}%
\definecolor{currentstroke}{rgb}{0.000000,0.000000,0.000000}%
\pgfsetstrokecolor{currentstroke}%
\pgfsetdash{}{0pt}%
\pgfsys@defobject{currentmarker}{\pgfqpoint{0.000000in}{0.000000in}}{\pgfqpoint{0.020833in}{0.000000in}}{%
\pgfpathmoveto{\pgfqpoint{0.000000in}{0.000000in}}%
\pgfpathlineto{\pgfqpoint{0.020833in}{0.000000in}}%
\pgfusepath{stroke,fill}%
}%
\begin{pgfscope}%
\pgfsys@transformshift{0.609415in}{1.068817in}%
\pgfsys@useobject{currentmarker}{}%
\end{pgfscope}%
\end{pgfscope}%
\begin{pgfscope}%
\pgfsetbuttcap%
\pgfsetroundjoin%
\definecolor{currentfill}{rgb}{0.000000,0.000000,0.000000}%
\pgfsetfillcolor{currentfill}%
\pgfsetlinewidth{0.501875pt}%
\definecolor{currentstroke}{rgb}{0.000000,0.000000,0.000000}%
\pgfsetstrokecolor{currentstroke}%
\pgfsetdash{}{0pt}%
\pgfsys@defobject{currentmarker}{\pgfqpoint{-0.020833in}{0.000000in}}{\pgfqpoint{-0.000000in}{0.000000in}}{%
\pgfpathmoveto{\pgfqpoint{-0.000000in}{0.000000in}}%
\pgfpathlineto{\pgfqpoint{-0.020833in}{0.000000in}}%
\pgfusepath{stroke,fill}%
}%
\begin{pgfscope}%
\pgfsys@transformshift{2.829105in}{1.068817in}%
\pgfsys@useobject{currentmarker}{}%
\end{pgfscope}%
\end{pgfscope}%
\begin{pgfscope}%
\pgfsetbuttcap%
\pgfsetroundjoin%
\definecolor{currentfill}{rgb}{0.000000,0.000000,0.000000}%
\pgfsetfillcolor{currentfill}%
\pgfsetlinewidth{0.501875pt}%
\definecolor{currentstroke}{rgb}{0.000000,0.000000,0.000000}%
\pgfsetstrokecolor{currentstroke}%
\pgfsetdash{}{0pt}%
\pgfsys@defobject{currentmarker}{\pgfqpoint{0.000000in}{0.000000in}}{\pgfqpoint{0.020833in}{0.000000in}}{%
\pgfpathmoveto{\pgfqpoint{0.000000in}{0.000000in}}%
\pgfpathlineto{\pgfqpoint{0.020833in}{0.000000in}}%
\pgfusepath{stroke,fill}%
}%
\begin{pgfscope}%
\pgfsys@transformshift{0.609415in}{1.143766in}%
\pgfsys@useobject{currentmarker}{}%
\end{pgfscope}%
\end{pgfscope}%
\begin{pgfscope}%
\pgfsetbuttcap%
\pgfsetroundjoin%
\definecolor{currentfill}{rgb}{0.000000,0.000000,0.000000}%
\pgfsetfillcolor{currentfill}%
\pgfsetlinewidth{0.501875pt}%
\definecolor{currentstroke}{rgb}{0.000000,0.000000,0.000000}%
\pgfsetstrokecolor{currentstroke}%
\pgfsetdash{}{0pt}%
\pgfsys@defobject{currentmarker}{\pgfqpoint{-0.020833in}{0.000000in}}{\pgfqpoint{-0.000000in}{0.000000in}}{%
\pgfpathmoveto{\pgfqpoint{-0.000000in}{0.000000in}}%
\pgfpathlineto{\pgfqpoint{-0.020833in}{0.000000in}}%
\pgfusepath{stroke,fill}%
}%
\begin{pgfscope}%
\pgfsys@transformshift{2.829105in}{1.143766in}%
\pgfsys@useobject{currentmarker}{}%
\end{pgfscope}%
\end{pgfscope}%
\begin{pgfscope}%
\pgfsetbuttcap%
\pgfsetroundjoin%
\definecolor{currentfill}{rgb}{0.000000,0.000000,0.000000}%
\pgfsetfillcolor{currentfill}%
\pgfsetlinewidth{0.501875pt}%
\definecolor{currentstroke}{rgb}{0.000000,0.000000,0.000000}%
\pgfsetstrokecolor{currentstroke}%
\pgfsetdash{}{0pt}%
\pgfsys@defobject{currentmarker}{\pgfqpoint{0.000000in}{0.000000in}}{\pgfqpoint{0.020833in}{0.000000in}}{%
\pgfpathmoveto{\pgfqpoint{0.000000in}{0.000000in}}%
\pgfpathlineto{\pgfqpoint{0.020833in}{0.000000in}}%
\pgfusepath{stroke,fill}%
}%
\begin{pgfscope}%
\pgfsys@transformshift{0.609415in}{1.293663in}%
\pgfsys@useobject{currentmarker}{}%
\end{pgfscope}%
\end{pgfscope}%
\begin{pgfscope}%
\pgfsetbuttcap%
\pgfsetroundjoin%
\definecolor{currentfill}{rgb}{0.000000,0.000000,0.000000}%
\pgfsetfillcolor{currentfill}%
\pgfsetlinewidth{0.501875pt}%
\definecolor{currentstroke}{rgb}{0.000000,0.000000,0.000000}%
\pgfsetstrokecolor{currentstroke}%
\pgfsetdash{}{0pt}%
\pgfsys@defobject{currentmarker}{\pgfqpoint{-0.020833in}{0.000000in}}{\pgfqpoint{-0.000000in}{0.000000in}}{%
\pgfpathmoveto{\pgfqpoint{-0.000000in}{0.000000in}}%
\pgfpathlineto{\pgfqpoint{-0.020833in}{0.000000in}}%
\pgfusepath{stroke,fill}%
}%
\begin{pgfscope}%
\pgfsys@transformshift{2.829105in}{1.293663in}%
\pgfsys@useobject{currentmarker}{}%
\end{pgfscope}%
\end{pgfscope}%
\begin{pgfscope}%
\pgfsetbuttcap%
\pgfsetroundjoin%
\definecolor{currentfill}{rgb}{0.000000,0.000000,0.000000}%
\pgfsetfillcolor{currentfill}%
\pgfsetlinewidth{0.501875pt}%
\definecolor{currentstroke}{rgb}{0.000000,0.000000,0.000000}%
\pgfsetstrokecolor{currentstroke}%
\pgfsetdash{}{0pt}%
\pgfsys@defobject{currentmarker}{\pgfqpoint{0.000000in}{0.000000in}}{\pgfqpoint{0.020833in}{0.000000in}}{%
\pgfpathmoveto{\pgfqpoint{0.000000in}{0.000000in}}%
\pgfpathlineto{\pgfqpoint{0.020833in}{0.000000in}}%
\pgfusepath{stroke,fill}%
}%
\begin{pgfscope}%
\pgfsys@transformshift{0.609415in}{1.368611in}%
\pgfsys@useobject{currentmarker}{}%
\end{pgfscope}%
\end{pgfscope}%
\begin{pgfscope}%
\pgfsetbuttcap%
\pgfsetroundjoin%
\definecolor{currentfill}{rgb}{0.000000,0.000000,0.000000}%
\pgfsetfillcolor{currentfill}%
\pgfsetlinewidth{0.501875pt}%
\definecolor{currentstroke}{rgb}{0.000000,0.000000,0.000000}%
\pgfsetstrokecolor{currentstroke}%
\pgfsetdash{}{0pt}%
\pgfsys@defobject{currentmarker}{\pgfqpoint{-0.020833in}{0.000000in}}{\pgfqpoint{-0.000000in}{0.000000in}}{%
\pgfpathmoveto{\pgfqpoint{-0.000000in}{0.000000in}}%
\pgfpathlineto{\pgfqpoint{-0.020833in}{0.000000in}}%
\pgfusepath{stroke,fill}%
}%
\begin{pgfscope}%
\pgfsys@transformshift{2.829105in}{1.368611in}%
\pgfsys@useobject{currentmarker}{}%
\end{pgfscope}%
\end{pgfscope}%
\begin{pgfscope}%
\pgfsetbuttcap%
\pgfsetroundjoin%
\definecolor{currentfill}{rgb}{0.000000,0.000000,0.000000}%
\pgfsetfillcolor{currentfill}%
\pgfsetlinewidth{0.501875pt}%
\definecolor{currentstroke}{rgb}{0.000000,0.000000,0.000000}%
\pgfsetstrokecolor{currentstroke}%
\pgfsetdash{}{0pt}%
\pgfsys@defobject{currentmarker}{\pgfqpoint{0.000000in}{0.000000in}}{\pgfqpoint{0.020833in}{0.000000in}}{%
\pgfpathmoveto{\pgfqpoint{0.000000in}{0.000000in}}%
\pgfpathlineto{\pgfqpoint{0.020833in}{0.000000in}}%
\pgfusepath{stroke,fill}%
}%
\begin{pgfscope}%
\pgfsys@transformshift{0.609415in}{1.443560in}%
\pgfsys@useobject{currentmarker}{}%
\end{pgfscope}%
\end{pgfscope}%
\begin{pgfscope}%
\pgfsetbuttcap%
\pgfsetroundjoin%
\definecolor{currentfill}{rgb}{0.000000,0.000000,0.000000}%
\pgfsetfillcolor{currentfill}%
\pgfsetlinewidth{0.501875pt}%
\definecolor{currentstroke}{rgb}{0.000000,0.000000,0.000000}%
\pgfsetstrokecolor{currentstroke}%
\pgfsetdash{}{0pt}%
\pgfsys@defobject{currentmarker}{\pgfqpoint{-0.020833in}{0.000000in}}{\pgfqpoint{-0.000000in}{0.000000in}}{%
\pgfpathmoveto{\pgfqpoint{-0.000000in}{0.000000in}}%
\pgfpathlineto{\pgfqpoint{-0.020833in}{0.000000in}}%
\pgfusepath{stroke,fill}%
}%
\begin{pgfscope}%
\pgfsys@transformshift{2.829105in}{1.443560in}%
\pgfsys@useobject{currentmarker}{}%
\end{pgfscope}%
\end{pgfscope}%
\begin{pgfscope}%
\pgfsetbuttcap%
\pgfsetroundjoin%
\definecolor{currentfill}{rgb}{0.000000,0.000000,0.000000}%
\pgfsetfillcolor{currentfill}%
\pgfsetlinewidth{0.501875pt}%
\definecolor{currentstroke}{rgb}{0.000000,0.000000,0.000000}%
\pgfsetstrokecolor{currentstroke}%
\pgfsetdash{}{0pt}%
\pgfsys@defobject{currentmarker}{\pgfqpoint{0.000000in}{0.000000in}}{\pgfqpoint{0.020833in}{0.000000in}}{%
\pgfpathmoveto{\pgfqpoint{0.000000in}{0.000000in}}%
\pgfpathlineto{\pgfqpoint{0.020833in}{0.000000in}}%
\pgfusepath{stroke,fill}%
}%
\begin{pgfscope}%
\pgfsys@transformshift{0.609415in}{1.518508in}%
\pgfsys@useobject{currentmarker}{}%
\end{pgfscope}%
\end{pgfscope}%
\begin{pgfscope}%
\pgfsetbuttcap%
\pgfsetroundjoin%
\definecolor{currentfill}{rgb}{0.000000,0.000000,0.000000}%
\pgfsetfillcolor{currentfill}%
\pgfsetlinewidth{0.501875pt}%
\definecolor{currentstroke}{rgb}{0.000000,0.000000,0.000000}%
\pgfsetstrokecolor{currentstroke}%
\pgfsetdash{}{0pt}%
\pgfsys@defobject{currentmarker}{\pgfqpoint{-0.020833in}{0.000000in}}{\pgfqpoint{-0.000000in}{0.000000in}}{%
\pgfpathmoveto{\pgfqpoint{-0.000000in}{0.000000in}}%
\pgfpathlineto{\pgfqpoint{-0.020833in}{0.000000in}}%
\pgfusepath{stroke,fill}%
}%
\begin{pgfscope}%
\pgfsys@transformshift{2.829105in}{1.518508in}%
\pgfsys@useobject{currentmarker}{}%
\end{pgfscope}%
\end{pgfscope}%
\begin{pgfscope}%
\definecolor{textcolor}{rgb}{0.000000,0.000000,0.000000}%
\pgfsetstrokecolor{textcolor}%
\pgfsetfillcolor{textcolor}%
\pgftext[x=0.188889in,y=1.036093in,,bottom,rotate=90.000000]{\color{textcolor}\rmfamily\fontsize{10.000000}{12.000000}\selectfont \(\displaystyle C(K)\)}%
\end{pgfscope}%
\begin{pgfscope}%
\pgfpathrectangle{\pgfqpoint{0.609415in}{0.422992in}}{\pgfqpoint{2.219690in}{1.226201in}}%
\pgfusepath{clip}%
\pgfsetrectcap%
\pgfsetroundjoin%
\pgfsetlinewidth{1.003750pt}%
\definecolor{currentstroke}{rgb}{0.047059,0.364706,0.647059}%
\pgfsetstrokecolor{currentstroke}%
\pgfsetdash{}{0pt}%
\pgfpathmoveto{\pgfqpoint{0.631176in}{1.593457in}}%
\pgfpathlineto{\pgfqpoint{0.652938in}{1.586694in}}%
\pgfpathlineto{\pgfqpoint{0.674700in}{1.581643in}}%
\pgfpathlineto{\pgfqpoint{0.696461in}{1.576501in}}%
\pgfpathlineto{\pgfqpoint{0.718223in}{1.573171in}}%
\pgfpathlineto{\pgfqpoint{0.739985in}{1.570258in}}%
\pgfpathlineto{\pgfqpoint{0.761746in}{1.567799in}}%
\pgfpathlineto{\pgfqpoint{0.783508in}{1.565522in}}%
\pgfpathlineto{\pgfqpoint{0.805270in}{1.563025in}}%
\pgfpathlineto{\pgfqpoint{0.827031in}{1.561211in}}%
\pgfpathlineto{\pgfqpoint{0.848793in}{1.559469in}}%
\pgfpathlineto{\pgfqpoint{0.870555in}{1.557617in}}%
\pgfpathlineto{\pgfqpoint{0.892316in}{1.556306in}}%
\pgfpathlineto{\pgfqpoint{0.914078in}{1.555126in}}%
\pgfpathlineto{\pgfqpoint{0.935840in}{1.553562in}}%
\pgfpathlineto{\pgfqpoint{0.957601in}{1.552351in}}%
\pgfpathlineto{\pgfqpoint{0.979363in}{1.551300in}}%
\pgfpathlineto{\pgfqpoint{1.001125in}{1.550400in}}%
\pgfpathlineto{\pgfqpoint{1.022886in}{1.549364in}}%
\pgfpathlineto{\pgfqpoint{1.044648in}{1.548364in}}%
\pgfpathlineto{\pgfqpoint{1.066410in}{1.547315in}}%
\pgfpathlineto{\pgfqpoint{1.088171in}{1.546503in}}%
\pgfpathlineto{\pgfqpoint{1.109933in}{1.545653in}}%
\pgfpathlineto{\pgfqpoint{1.131695in}{1.544812in}}%
\pgfpathlineto{\pgfqpoint{1.153456in}{1.543955in}}%
\pgfpathlineto{\pgfqpoint{1.175218in}{1.543276in}}%
\pgfpathlineto{\pgfqpoint{1.196980in}{1.542627in}}%
\pgfpathlineto{\pgfqpoint{1.218741in}{1.541874in}}%
\pgfpathlineto{\pgfqpoint{1.240503in}{1.541168in}}%
\pgfpathlineto{\pgfqpoint{1.262265in}{1.540493in}}%
\pgfpathlineto{\pgfqpoint{1.284026in}{1.539851in}}%
\pgfpathlineto{\pgfqpoint{1.305788in}{1.539174in}}%
\pgfpathlineto{\pgfqpoint{1.327550in}{1.538534in}}%
\pgfpathlineto{\pgfqpoint{1.349311in}{1.537845in}}%
\pgfpathlineto{\pgfqpoint{1.371073in}{1.537119in}}%
\pgfpathlineto{\pgfqpoint{1.392835in}{1.536553in}}%
\pgfpathlineto{\pgfqpoint{1.414596in}{1.536010in}}%
\pgfpathlineto{\pgfqpoint{1.436358in}{1.535381in}}%
\pgfpathlineto{\pgfqpoint{1.458120in}{1.534844in}}%
\pgfpathlineto{\pgfqpoint{1.479881in}{1.534270in}}%
\pgfpathlineto{\pgfqpoint{1.501643in}{1.533660in}}%
\pgfpathlineto{\pgfqpoint{1.523405in}{1.533096in}}%
\pgfpathlineto{\pgfqpoint{1.545166in}{1.532525in}}%
\pgfpathlineto{\pgfqpoint{1.566928in}{1.531978in}}%
\pgfpathlineto{\pgfqpoint{1.588690in}{1.531502in}}%
\pgfpathlineto{\pgfqpoint{1.610451in}{1.531053in}}%
\pgfpathlineto{\pgfqpoint{1.632213in}{1.530633in}}%
\pgfpathlineto{\pgfqpoint{1.653975in}{1.530174in}}%
\pgfpathlineto{\pgfqpoint{1.675736in}{1.529649in}}%
\pgfpathlineto{\pgfqpoint{1.697498in}{1.529154in}}%
\pgfpathlineto{\pgfqpoint{1.719260in}{1.528726in}}%
\pgfpathlineto{\pgfqpoint{1.741021in}{1.528333in}}%
\pgfpathlineto{\pgfqpoint{1.762783in}{1.527881in}}%
\pgfpathlineto{\pgfqpoint{1.784545in}{1.527435in}}%
\pgfpathlineto{\pgfqpoint{1.806306in}{1.527003in}}%
\pgfpathlineto{\pgfqpoint{1.828068in}{1.526522in}}%
\pgfpathlineto{\pgfqpoint{1.849830in}{1.526113in}}%
\pgfpathlineto{\pgfqpoint{1.871591in}{1.525689in}}%
\pgfpathlineto{\pgfqpoint{1.893353in}{1.525221in}}%
\pgfpathlineto{\pgfqpoint{1.915115in}{1.524843in}}%
\pgfpathlineto{\pgfqpoint{1.936876in}{1.524403in}}%
\pgfpathlineto{\pgfqpoint{1.958638in}{1.523989in}}%
\pgfpathlineto{\pgfqpoint{1.980400in}{1.523585in}}%
\pgfpathlineto{\pgfqpoint{2.002161in}{1.523137in}}%
\pgfpathlineto{\pgfqpoint{2.023923in}{1.522718in}}%
\pgfpathlineto{\pgfqpoint{2.045685in}{1.522316in}}%
\pgfpathlineto{\pgfqpoint{2.067446in}{1.521937in}}%
\pgfpathlineto{\pgfqpoint{2.089208in}{1.521567in}}%
\pgfpathlineto{\pgfqpoint{2.110970in}{1.521125in}}%
\pgfpathlineto{\pgfqpoint{2.132731in}{1.520759in}}%
\pgfpathlineto{\pgfqpoint{2.154493in}{1.520382in}}%
\pgfpathlineto{\pgfqpoint{2.176255in}{1.520022in}}%
\pgfpathlineto{\pgfqpoint{2.198016in}{1.519647in}}%
\pgfpathlineto{\pgfqpoint{2.219778in}{1.519254in}}%
\pgfpathlineto{\pgfqpoint{2.241540in}{1.518869in}}%
\pgfpathlineto{\pgfqpoint{2.263301in}{1.518502in}}%
\pgfpathlineto{\pgfqpoint{2.285063in}{1.518148in}}%
\pgfpathlineto{\pgfqpoint{2.306825in}{1.517748in}}%
\pgfpathlineto{\pgfqpoint{2.328586in}{1.517373in}}%
\pgfpathlineto{\pgfqpoint{2.350348in}{1.516970in}}%
\pgfpathlineto{\pgfqpoint{2.372110in}{1.516623in}}%
\pgfpathlineto{\pgfqpoint{2.393871in}{1.516278in}}%
\pgfpathlineto{\pgfqpoint{2.415633in}{1.515948in}}%
\pgfpathlineto{\pgfqpoint{2.437395in}{1.515626in}}%
\pgfpathlineto{\pgfqpoint{2.459156in}{1.515294in}}%
\pgfpathlineto{\pgfqpoint{2.480918in}{1.514921in}}%
\pgfpathlineto{\pgfqpoint{2.502680in}{1.514564in}}%
\pgfpathlineto{\pgfqpoint{2.524441in}{1.514195in}}%
\pgfpathlineto{\pgfqpoint{2.546203in}{1.513848in}}%
\pgfpathlineto{\pgfqpoint{2.567965in}{1.513508in}}%
\pgfpathlineto{\pgfqpoint{2.589726in}{1.513158in}}%
\pgfpathlineto{\pgfqpoint{2.611488in}{1.512830in}}%
\pgfpathlineto{\pgfqpoint{2.633250in}{1.512508in}}%
\pgfpathlineto{\pgfqpoint{2.655011in}{1.512157in}}%
\pgfpathlineto{\pgfqpoint{2.676773in}{1.511803in}}%
\pgfpathlineto{\pgfqpoint{2.698535in}{1.511473in}}%
\pgfpathlineto{\pgfqpoint{2.720296in}{1.511134in}}%
\pgfpathlineto{\pgfqpoint{2.742058in}{1.510821in}}%
\pgfpathlineto{\pgfqpoint{2.763820in}{1.510482in}}%
\pgfpathlineto{\pgfqpoint{2.785581in}{1.510137in}}%
\pgfusepath{stroke}%
\end{pgfscope}%
\begin{pgfscope}%
\pgfpathrectangle{\pgfqpoint{0.609415in}{0.422992in}}{\pgfqpoint{2.219690in}{1.226201in}}%
\pgfusepath{clip}%
\pgfsetrectcap%
\pgfsetroundjoin%
\pgfsetlinewidth{1.003750pt}%
\definecolor{currentstroke}{rgb}{0.000000,0.725490,0.270588}%
\pgfsetstrokecolor{currentstroke}%
\pgfsetdash{}{0pt}%
\pgfpathmoveto{\pgfqpoint{0.631176in}{1.593457in}}%
\pgfpathlineto{\pgfqpoint{0.652938in}{1.571733in}}%
\pgfpathlineto{\pgfqpoint{0.674700in}{1.556103in}}%
\pgfpathlineto{\pgfqpoint{0.696461in}{1.542603in}}%
\pgfpathlineto{\pgfqpoint{0.718223in}{1.531259in}}%
\pgfpathlineto{\pgfqpoint{0.739985in}{1.520925in}}%
\pgfpathlineto{\pgfqpoint{0.761746in}{1.511767in}}%
\pgfpathlineto{\pgfqpoint{0.783508in}{1.503103in}}%
\pgfpathlineto{\pgfqpoint{0.805270in}{1.494297in}}%
\pgfpathlineto{\pgfqpoint{0.827031in}{1.486112in}}%
\pgfpathlineto{\pgfqpoint{0.848793in}{1.478031in}}%
\pgfpathlineto{\pgfqpoint{0.870555in}{1.470338in}}%
\pgfpathlineto{\pgfqpoint{0.892316in}{1.462821in}}%
\pgfpathlineto{\pgfqpoint{0.914078in}{1.455492in}}%
\pgfpathlineto{\pgfqpoint{0.935840in}{1.448196in}}%
\pgfpathlineto{\pgfqpoint{0.957601in}{1.441486in}}%
\pgfpathlineto{\pgfqpoint{0.979363in}{1.434449in}}%
\pgfpathlineto{\pgfqpoint{1.001125in}{1.427979in}}%
\pgfpathlineto{\pgfqpoint{1.022886in}{1.421325in}}%
\pgfpathlineto{\pgfqpoint{1.044648in}{1.414401in}}%
\pgfpathlineto{\pgfqpoint{1.066410in}{1.407994in}}%
\pgfpathlineto{\pgfqpoint{1.088171in}{1.401492in}}%
\pgfpathlineto{\pgfqpoint{1.109933in}{1.394992in}}%
\pgfpathlineto{\pgfqpoint{1.131695in}{1.388804in}}%
\pgfpathlineto{\pgfqpoint{1.153456in}{1.382889in}}%
\pgfpathlineto{\pgfqpoint{1.175218in}{1.376815in}}%
\pgfpathlineto{\pgfqpoint{1.196980in}{1.370591in}}%
\pgfpathlineto{\pgfqpoint{1.218741in}{1.364839in}}%
\pgfpathlineto{\pgfqpoint{1.240503in}{1.359020in}}%
\pgfpathlineto{\pgfqpoint{1.262265in}{1.353243in}}%
\pgfpathlineto{\pgfqpoint{1.284026in}{1.347762in}}%
\pgfpathlineto{\pgfqpoint{1.305788in}{1.342094in}}%
\pgfpathlineto{\pgfqpoint{1.327550in}{1.336503in}}%
\pgfpathlineto{\pgfqpoint{1.349311in}{1.331264in}}%
\pgfpathlineto{\pgfqpoint{1.371073in}{1.325985in}}%
\pgfpathlineto{\pgfqpoint{1.392835in}{1.320748in}}%
\pgfpathlineto{\pgfqpoint{1.414596in}{1.315410in}}%
\pgfpathlineto{\pgfqpoint{1.436358in}{1.310038in}}%
\pgfpathlineto{\pgfqpoint{1.458120in}{1.304828in}}%
\pgfpathlineto{\pgfqpoint{1.479881in}{1.299790in}}%
\pgfpathlineto{\pgfqpoint{1.501643in}{1.294754in}}%
\pgfpathlineto{\pgfqpoint{1.523405in}{1.289709in}}%
\pgfpathlineto{\pgfqpoint{1.545166in}{1.284839in}}%
\pgfpathlineto{\pgfqpoint{1.566928in}{1.279717in}}%
\pgfpathlineto{\pgfqpoint{1.588690in}{1.274790in}}%
\pgfpathlineto{\pgfqpoint{1.610451in}{1.269875in}}%
\pgfpathlineto{\pgfqpoint{1.632213in}{1.265169in}}%
\pgfpathlineto{\pgfqpoint{1.653975in}{1.260227in}}%
\pgfpathlineto{\pgfqpoint{1.675736in}{1.255202in}}%
\pgfpathlineto{\pgfqpoint{1.697498in}{1.250580in}}%
\pgfpathlineto{\pgfqpoint{1.719260in}{1.245865in}}%
\pgfpathlineto{\pgfqpoint{1.741021in}{1.241147in}}%
\pgfpathlineto{\pgfqpoint{1.762783in}{1.236413in}}%
\pgfpathlineto{\pgfqpoint{1.784545in}{1.231538in}}%
\pgfpathlineto{\pgfqpoint{1.806306in}{1.226874in}}%
\pgfpathlineto{\pgfqpoint{1.828068in}{1.222462in}}%
\pgfpathlineto{\pgfqpoint{1.849830in}{1.218219in}}%
\pgfpathlineto{\pgfqpoint{1.871591in}{1.213778in}}%
\pgfpathlineto{\pgfqpoint{1.893353in}{1.209249in}}%
\pgfpathlineto{\pgfqpoint{1.915115in}{1.204531in}}%
\pgfpathlineto{\pgfqpoint{1.936876in}{1.200063in}}%
\pgfpathlineto{\pgfqpoint{1.958638in}{1.195526in}}%
\pgfpathlineto{\pgfqpoint{1.980400in}{1.190982in}}%
\pgfpathlineto{\pgfqpoint{2.002161in}{1.186438in}}%
\pgfpathlineto{\pgfqpoint{2.023923in}{1.182135in}}%
\pgfpathlineto{\pgfqpoint{2.045685in}{1.177758in}}%
\pgfpathlineto{\pgfqpoint{2.067446in}{1.173429in}}%
\pgfpathlineto{\pgfqpoint{2.089208in}{1.168993in}}%
\pgfpathlineto{\pgfqpoint{2.110970in}{1.164714in}}%
\pgfpathlineto{\pgfqpoint{2.132731in}{1.160554in}}%
\pgfpathlineto{\pgfqpoint{2.154493in}{1.156393in}}%
\pgfpathlineto{\pgfqpoint{2.176255in}{1.151956in}}%
\pgfpathlineto{\pgfqpoint{2.198016in}{1.147729in}}%
\pgfpathlineto{\pgfqpoint{2.219778in}{1.143408in}}%
\pgfpathlineto{\pgfqpoint{2.241540in}{1.139401in}}%
\pgfpathlineto{\pgfqpoint{2.263301in}{1.135166in}}%
\pgfpathlineto{\pgfqpoint{2.285063in}{1.131130in}}%
\pgfpathlineto{\pgfqpoint{2.306825in}{1.127018in}}%
\pgfpathlineto{\pgfqpoint{2.328586in}{1.122912in}}%
\pgfpathlineto{\pgfqpoint{2.350348in}{1.118977in}}%
\pgfpathlineto{\pgfqpoint{2.372110in}{1.114866in}}%
\pgfpathlineto{\pgfqpoint{2.393871in}{1.111038in}}%
\pgfpathlineto{\pgfqpoint{2.415633in}{1.107289in}}%
\pgfpathlineto{\pgfqpoint{2.437395in}{1.103332in}}%
\pgfpathlineto{\pgfqpoint{2.459156in}{1.099293in}}%
\pgfpathlineto{\pgfqpoint{2.480918in}{1.095508in}}%
\pgfpathlineto{\pgfqpoint{2.502680in}{1.091770in}}%
\pgfpathlineto{\pgfqpoint{2.524441in}{1.087835in}}%
\pgfpathlineto{\pgfqpoint{2.546203in}{1.083901in}}%
\pgfpathlineto{\pgfqpoint{2.567965in}{1.080036in}}%
\pgfpathlineto{\pgfqpoint{2.589726in}{1.076242in}}%
\pgfpathlineto{\pgfqpoint{2.611488in}{1.072265in}}%
\pgfpathlineto{\pgfqpoint{2.633250in}{1.068331in}}%
\pgfpathlineto{\pgfqpoint{2.655011in}{1.064685in}}%
\pgfpathlineto{\pgfqpoint{2.676773in}{1.060885in}}%
\pgfpathlineto{\pgfqpoint{2.698535in}{1.057019in}}%
\pgfpathlineto{\pgfqpoint{2.720296in}{1.053463in}}%
\pgfpathlineto{\pgfqpoint{2.742058in}{1.049829in}}%
\pgfpathlineto{\pgfqpoint{2.763820in}{1.046085in}}%
\pgfpathlineto{\pgfqpoint{2.785581in}{1.042299in}}%
\pgfusepath{stroke}%
\end{pgfscope}%
\begin{pgfscope}%
\pgfpathrectangle{\pgfqpoint{0.609415in}{0.422992in}}{\pgfqpoint{2.219690in}{1.226201in}}%
\pgfusepath{clip}%
\pgfsetrectcap%
\pgfsetroundjoin%
\pgfsetlinewidth{1.003750pt}%
\definecolor{currentstroke}{rgb}{1.000000,0.584314,0.000000}%
\pgfsetstrokecolor{currentstroke}%
\pgfsetdash{}{0pt}%
\pgfpathmoveto{\pgfqpoint{0.631176in}{1.593457in}}%
\pgfpathlineto{\pgfqpoint{0.652938in}{1.519031in}}%
\pgfpathlineto{\pgfqpoint{0.674700in}{1.465979in}}%
\pgfpathlineto{\pgfqpoint{0.696461in}{1.427144in}}%
\pgfpathlineto{\pgfqpoint{0.718223in}{1.393009in}}%
\pgfpathlineto{\pgfqpoint{0.739985in}{1.363404in}}%
\pgfpathlineto{\pgfqpoint{0.761746in}{1.339745in}}%
\pgfpathlineto{\pgfqpoint{0.783508in}{1.317183in}}%
\pgfpathlineto{\pgfqpoint{0.805270in}{1.295876in}}%
\pgfpathlineto{\pgfqpoint{0.827031in}{1.276873in}}%
\pgfpathlineto{\pgfqpoint{0.848793in}{1.259978in}}%
\pgfpathlineto{\pgfqpoint{0.870555in}{1.242803in}}%
\pgfpathlineto{\pgfqpoint{0.892316in}{1.227292in}}%
\pgfpathlineto{\pgfqpoint{0.914078in}{1.210912in}}%
\pgfpathlineto{\pgfqpoint{0.935840in}{1.195745in}}%
\pgfpathlineto{\pgfqpoint{0.957601in}{1.180005in}}%
\pgfpathlineto{\pgfqpoint{0.979363in}{1.165950in}}%
\pgfpathlineto{\pgfqpoint{1.001125in}{1.152236in}}%
\pgfpathlineto{\pgfqpoint{1.022886in}{1.138652in}}%
\pgfpathlineto{\pgfqpoint{1.044648in}{1.126492in}}%
\pgfpathlineto{\pgfqpoint{1.066410in}{1.113641in}}%
\pgfpathlineto{\pgfqpoint{1.088171in}{1.100863in}}%
\pgfpathlineto{\pgfqpoint{1.109933in}{1.089441in}}%
\pgfpathlineto{\pgfqpoint{1.131695in}{1.077320in}}%
\pgfpathlineto{\pgfqpoint{1.153456in}{1.065532in}}%
\pgfpathlineto{\pgfqpoint{1.175218in}{1.053924in}}%
\pgfpathlineto{\pgfqpoint{1.196980in}{1.042725in}}%
\pgfpathlineto{\pgfqpoint{1.218741in}{1.031510in}}%
\pgfpathlineto{\pgfqpoint{1.240503in}{1.020553in}}%
\pgfpathlineto{\pgfqpoint{1.262265in}{1.009665in}}%
\pgfpathlineto{\pgfqpoint{1.284026in}{0.999336in}}%
\pgfpathlineto{\pgfqpoint{1.305788in}{0.989334in}}%
\pgfpathlineto{\pgfqpoint{1.327550in}{0.978971in}}%
\pgfpathlineto{\pgfqpoint{1.349311in}{0.968415in}}%
\pgfpathlineto{\pgfqpoint{1.371073in}{0.958782in}}%
\pgfpathlineto{\pgfqpoint{1.392835in}{0.949269in}}%
\pgfpathlineto{\pgfqpoint{1.414596in}{0.939285in}}%
\pgfpathlineto{\pgfqpoint{1.436358in}{0.930178in}}%
\pgfpathlineto{\pgfqpoint{1.458120in}{0.920828in}}%
\pgfpathlineto{\pgfqpoint{1.479881in}{0.911284in}}%
\pgfpathlineto{\pgfqpoint{1.501643in}{0.902209in}}%
\pgfpathlineto{\pgfqpoint{1.523405in}{0.893909in}}%
\pgfpathlineto{\pgfqpoint{1.545166in}{0.885077in}}%
\pgfpathlineto{\pgfqpoint{1.566928in}{0.877094in}}%
\pgfpathlineto{\pgfqpoint{1.588690in}{0.867614in}}%
\pgfpathlineto{\pgfqpoint{1.610451in}{0.858740in}}%
\pgfpathlineto{\pgfqpoint{1.632213in}{0.850781in}}%
\pgfpathlineto{\pgfqpoint{1.653975in}{0.842336in}}%
\pgfpathlineto{\pgfqpoint{1.675736in}{0.834107in}}%
\pgfpathlineto{\pgfqpoint{1.697498in}{0.826103in}}%
\pgfpathlineto{\pgfqpoint{1.719260in}{0.818164in}}%
\pgfpathlineto{\pgfqpoint{1.741021in}{0.810072in}}%
\pgfpathlineto{\pgfqpoint{1.762783in}{0.802046in}}%
\pgfpathlineto{\pgfqpoint{1.784545in}{0.794420in}}%
\pgfpathlineto{\pgfqpoint{1.806306in}{0.786546in}}%
\pgfpathlineto{\pgfqpoint{1.828068in}{0.778958in}}%
\pgfpathlineto{\pgfqpoint{1.849830in}{0.771331in}}%
\pgfpathlineto{\pgfqpoint{1.871591in}{0.763428in}}%
\pgfpathlineto{\pgfqpoint{1.893353in}{0.755878in}}%
\pgfpathlineto{\pgfqpoint{1.915115in}{0.747828in}}%
\pgfpathlineto{\pgfqpoint{1.936876in}{0.740100in}}%
\pgfpathlineto{\pgfqpoint{1.958638in}{0.732316in}}%
\pgfpathlineto{\pgfqpoint{1.980400in}{0.724618in}}%
\pgfpathlineto{\pgfqpoint{2.002161in}{0.717344in}}%
\pgfpathlineto{\pgfqpoint{2.023923in}{0.710281in}}%
\pgfpathlineto{\pgfqpoint{2.045685in}{0.702585in}}%
\pgfpathlineto{\pgfqpoint{2.067446in}{0.695527in}}%
\pgfpathlineto{\pgfqpoint{2.089208in}{0.687980in}}%
\pgfpathlineto{\pgfqpoint{2.110970in}{0.680507in}}%
\pgfpathlineto{\pgfqpoint{2.132731in}{0.673764in}}%
\pgfpathlineto{\pgfqpoint{2.154493in}{0.666579in}}%
\pgfpathlineto{\pgfqpoint{2.176255in}{0.660082in}}%
\pgfpathlineto{\pgfqpoint{2.198016in}{0.653287in}}%
\pgfpathlineto{\pgfqpoint{2.219778in}{0.646352in}}%
\pgfpathlineto{\pgfqpoint{2.241540in}{0.639379in}}%
\pgfpathlineto{\pgfqpoint{2.263301in}{0.632910in}}%
\pgfpathlineto{\pgfqpoint{2.285063in}{0.626418in}}%
\pgfpathlineto{\pgfqpoint{2.306825in}{0.619562in}}%
\pgfpathlineto{\pgfqpoint{2.328586in}{0.613137in}}%
\pgfpathlineto{\pgfqpoint{2.350348in}{0.606347in}}%
\pgfpathlineto{\pgfqpoint{2.372110in}{0.599953in}}%
\pgfpathlineto{\pgfqpoint{2.393871in}{0.593544in}}%
\pgfpathlineto{\pgfqpoint{2.415633in}{0.586692in}}%
\pgfpathlineto{\pgfqpoint{2.437395in}{0.579804in}}%
\pgfpathlineto{\pgfqpoint{2.459156in}{0.573516in}}%
\pgfpathlineto{\pgfqpoint{2.480918in}{0.566887in}}%
\pgfpathlineto{\pgfqpoint{2.502680in}{0.560258in}}%
\pgfpathlineto{\pgfqpoint{2.524441in}{0.553820in}}%
\pgfpathlineto{\pgfqpoint{2.546203in}{0.547305in}}%
\pgfpathlineto{\pgfqpoint{2.567965in}{0.540794in}}%
\pgfpathlineto{\pgfqpoint{2.589726in}{0.534669in}}%
\pgfpathlineto{\pgfqpoint{2.611488in}{0.528263in}}%
\pgfpathlineto{\pgfqpoint{2.633250in}{0.522032in}}%
\pgfpathlineto{\pgfqpoint{2.655011in}{0.515865in}}%
\pgfpathlineto{\pgfqpoint{2.676773in}{0.509756in}}%
\pgfpathlineto{\pgfqpoint{2.698535in}{0.503744in}}%
\pgfpathlineto{\pgfqpoint{2.720296in}{0.497444in}}%
\pgfpathlineto{\pgfqpoint{2.742058in}{0.491254in}}%
\pgfpathlineto{\pgfqpoint{2.763820in}{0.485067in}}%
\pgfpathlineto{\pgfqpoint{2.785581in}{0.478729in}}%
\pgfusepath{stroke}%
\end{pgfscope}%
\begin{pgfscope}%
\pgfpathrectangle{\pgfqpoint{0.609415in}{0.422992in}}{\pgfqpoint{2.219690in}{1.226201in}}%
\pgfusepath{clip}%
\pgfsetrectcap%
\pgfsetroundjoin%
\pgfsetlinewidth{1.003750pt}%
\definecolor{currentstroke}{rgb}{1.000000,0.172549,0.000000}%
\pgfsetstrokecolor{currentstroke}%
\pgfsetdash{}{0pt}%
\pgfpathmoveto{\pgfqpoint{0.631176in}{1.593457in}}%
\pgfpathlineto{\pgfqpoint{0.652938in}{1.586368in}}%
\pgfpathlineto{\pgfqpoint{0.674700in}{1.579736in}}%
\pgfpathlineto{\pgfqpoint{0.696461in}{1.574123in}}%
\pgfpathlineto{\pgfqpoint{0.718223in}{1.569664in}}%
\pgfpathlineto{\pgfqpoint{0.739985in}{1.565659in}}%
\pgfpathlineto{\pgfqpoint{0.761746in}{1.562612in}}%
\pgfpathlineto{\pgfqpoint{0.783508in}{1.559972in}}%
\pgfpathlineto{\pgfqpoint{0.805270in}{1.557824in}}%
\pgfpathlineto{\pgfqpoint{0.827031in}{1.555057in}}%
\pgfpathlineto{\pgfqpoint{0.848793in}{1.552347in}}%
\pgfpathlineto{\pgfqpoint{0.870555in}{1.550269in}}%
\pgfpathlineto{\pgfqpoint{0.892316in}{1.548209in}}%
\pgfpathlineto{\pgfqpoint{0.914078in}{1.546175in}}%
\pgfpathlineto{\pgfqpoint{0.935840in}{1.544031in}}%
\pgfpathlineto{\pgfqpoint{0.957601in}{1.542192in}}%
\pgfpathlineto{\pgfqpoint{0.979363in}{1.540095in}}%
\pgfpathlineto{\pgfqpoint{1.001125in}{1.538052in}}%
\pgfpathlineto{\pgfqpoint{1.022886in}{1.536164in}}%
\pgfpathlineto{\pgfqpoint{1.044648in}{1.534333in}}%
\pgfpathlineto{\pgfqpoint{1.066410in}{1.532662in}}%
\pgfpathlineto{\pgfqpoint{1.088171in}{1.531114in}}%
\pgfpathlineto{\pgfqpoint{1.109933in}{1.529517in}}%
\pgfpathlineto{\pgfqpoint{1.131695in}{1.528128in}}%
\pgfpathlineto{\pgfqpoint{1.153456in}{1.526652in}}%
\pgfpathlineto{\pgfqpoint{1.175218in}{1.525205in}}%
\pgfpathlineto{\pgfqpoint{1.196980in}{1.523678in}}%
\pgfpathlineto{\pgfqpoint{1.218741in}{1.522242in}}%
\pgfpathlineto{\pgfqpoint{1.240503in}{1.520852in}}%
\pgfpathlineto{\pgfqpoint{1.262265in}{1.519291in}}%
\pgfpathlineto{\pgfqpoint{1.284026in}{1.518101in}}%
\pgfpathlineto{\pgfqpoint{1.305788in}{1.516752in}}%
\pgfpathlineto{\pgfqpoint{1.327550in}{1.515345in}}%
\pgfpathlineto{\pgfqpoint{1.349311in}{1.514078in}}%
\pgfpathlineto{\pgfqpoint{1.371073in}{1.512686in}}%
\pgfpathlineto{\pgfqpoint{1.392835in}{1.511357in}}%
\pgfpathlineto{\pgfqpoint{1.414596in}{1.510137in}}%
\pgfpathlineto{\pgfqpoint{1.436358in}{1.508963in}}%
\pgfpathlineto{\pgfqpoint{1.458120in}{1.507776in}}%
\pgfpathlineto{\pgfqpoint{1.479881in}{1.506501in}}%
\pgfpathlineto{\pgfqpoint{1.501643in}{1.505346in}}%
\pgfpathlineto{\pgfqpoint{1.523405in}{1.504226in}}%
\pgfpathlineto{\pgfqpoint{1.545166in}{1.503106in}}%
\pgfpathlineto{\pgfqpoint{1.566928in}{1.502014in}}%
\pgfpathlineto{\pgfqpoint{1.588690in}{1.500938in}}%
\pgfpathlineto{\pgfqpoint{1.610451in}{1.499881in}}%
\pgfpathlineto{\pgfqpoint{1.632213in}{1.498828in}}%
\pgfpathlineto{\pgfqpoint{1.653975in}{1.497801in}}%
\pgfpathlineto{\pgfqpoint{1.675736in}{1.496598in}}%
\pgfpathlineto{\pgfqpoint{1.697498in}{1.495615in}}%
\pgfpathlineto{\pgfqpoint{1.719260in}{1.494618in}}%
\pgfpathlineto{\pgfqpoint{1.741021in}{1.493517in}}%
\pgfpathlineto{\pgfqpoint{1.762783in}{1.492608in}}%
\pgfpathlineto{\pgfqpoint{1.784545in}{1.491629in}}%
\pgfpathlineto{\pgfqpoint{1.806306in}{1.490588in}}%
\pgfpathlineto{\pgfqpoint{1.828068in}{1.489581in}}%
\pgfpathlineto{\pgfqpoint{1.849830in}{1.488667in}}%
\pgfpathlineto{\pgfqpoint{1.871591in}{1.487693in}}%
\pgfpathlineto{\pgfqpoint{1.893353in}{1.486647in}}%
\pgfpathlineto{\pgfqpoint{1.915115in}{1.485652in}}%
\pgfpathlineto{\pgfqpoint{1.936876in}{1.484763in}}%
\pgfpathlineto{\pgfqpoint{1.958638in}{1.483822in}}%
\pgfpathlineto{\pgfqpoint{1.980400in}{1.482831in}}%
\pgfpathlineto{\pgfqpoint{2.002161in}{1.481923in}}%
\pgfpathlineto{\pgfqpoint{2.023923in}{1.481005in}}%
\pgfpathlineto{\pgfqpoint{2.045685in}{1.480063in}}%
\pgfpathlineto{\pgfqpoint{2.067446in}{1.479180in}}%
\pgfpathlineto{\pgfqpoint{2.089208in}{1.478281in}}%
\pgfpathlineto{\pgfqpoint{2.110970in}{1.477358in}}%
\pgfpathlineto{\pgfqpoint{2.132731in}{1.476428in}}%
\pgfpathlineto{\pgfqpoint{2.154493in}{1.475510in}}%
\pgfpathlineto{\pgfqpoint{2.176255in}{1.474725in}}%
\pgfpathlineto{\pgfqpoint{2.198016in}{1.473835in}}%
\pgfpathlineto{\pgfqpoint{2.219778in}{1.472991in}}%
\pgfpathlineto{\pgfqpoint{2.241540in}{1.472217in}}%
\pgfpathlineto{\pgfqpoint{2.263301in}{1.471371in}}%
\pgfpathlineto{\pgfqpoint{2.285063in}{1.470558in}}%
\pgfpathlineto{\pgfqpoint{2.306825in}{1.469704in}}%
\pgfpathlineto{\pgfqpoint{2.328586in}{1.468881in}}%
\pgfpathlineto{\pgfqpoint{2.350348in}{1.468087in}}%
\pgfpathlineto{\pgfqpoint{2.372110in}{1.467279in}}%
\pgfpathlineto{\pgfqpoint{2.393871in}{1.466468in}}%
\pgfpathlineto{\pgfqpoint{2.415633in}{1.465629in}}%
\pgfpathlineto{\pgfqpoint{2.437395in}{1.464735in}}%
\pgfpathlineto{\pgfqpoint{2.459156in}{1.463937in}}%
\pgfpathlineto{\pgfqpoint{2.480918in}{1.463107in}}%
\pgfpathlineto{\pgfqpoint{2.502680in}{1.462280in}}%
\pgfpathlineto{\pgfqpoint{2.524441in}{1.461518in}}%
\pgfpathlineto{\pgfqpoint{2.546203in}{1.460686in}}%
\pgfpathlineto{\pgfqpoint{2.567965in}{1.459806in}}%
\pgfpathlineto{\pgfqpoint{2.589726in}{1.459049in}}%
\pgfpathlineto{\pgfqpoint{2.611488in}{1.458241in}}%
\pgfpathlineto{\pgfqpoint{2.633250in}{1.457372in}}%
\pgfpathlineto{\pgfqpoint{2.655011in}{1.456556in}}%
\pgfpathlineto{\pgfqpoint{2.676773in}{1.455764in}}%
\pgfpathlineto{\pgfqpoint{2.698535in}{1.454994in}}%
\pgfpathlineto{\pgfqpoint{2.720296in}{1.454247in}}%
\pgfpathlineto{\pgfqpoint{2.742058in}{1.453442in}}%
\pgfpathlineto{\pgfqpoint{2.763820in}{1.452577in}}%
\pgfpathlineto{\pgfqpoint{2.785581in}{1.451792in}}%
\pgfusepath{stroke}%
\end{pgfscope}%
\begin{pgfscope}%
\pgfpathrectangle{\pgfqpoint{0.609415in}{0.422992in}}{\pgfqpoint{2.219690in}{1.226201in}}%
\pgfusepath{clip}%
\pgfsetrectcap%
\pgfsetroundjoin%
\pgfsetlinewidth{1.003750pt}%
\definecolor{currentstroke}{rgb}{0.517647,0.356863,0.592157}%
\pgfsetstrokecolor{currentstroke}%
\pgfsetdash{}{0pt}%
\pgfpathmoveto{\pgfqpoint{0.631176in}{1.593457in}}%
\pgfpathlineto{\pgfqpoint{0.652938in}{1.484236in}}%
\pgfpathlineto{\pgfqpoint{0.674700in}{1.408090in}}%
\pgfpathlineto{\pgfqpoint{0.696461in}{1.356640in}}%
\pgfpathlineto{\pgfqpoint{0.718223in}{1.318068in}}%
\pgfpathlineto{\pgfqpoint{0.739985in}{1.289156in}}%
\pgfpathlineto{\pgfqpoint{0.761746in}{1.264658in}}%
\pgfpathlineto{\pgfqpoint{0.783508in}{1.243369in}}%
\pgfpathlineto{\pgfqpoint{0.805270in}{1.225776in}}%
\pgfpathlineto{\pgfqpoint{0.827031in}{1.209137in}}%
\pgfpathlineto{\pgfqpoint{0.848793in}{1.195094in}}%
\pgfpathlineto{\pgfqpoint{0.870555in}{1.181391in}}%
\pgfpathlineto{\pgfqpoint{0.892316in}{1.168781in}}%
\pgfpathlineto{\pgfqpoint{0.914078in}{1.157114in}}%
\pgfpathlineto{\pgfqpoint{0.935840in}{1.146345in}}%
\pgfpathlineto{\pgfqpoint{0.957601in}{1.137118in}}%
\pgfpathlineto{\pgfqpoint{0.979363in}{1.127450in}}%
\pgfpathlineto{\pgfqpoint{1.001125in}{1.118816in}}%
\pgfpathlineto{\pgfqpoint{1.022886in}{1.110566in}}%
\pgfpathlineto{\pgfqpoint{1.044648in}{1.103419in}}%
\pgfpathlineto{\pgfqpoint{1.066410in}{1.096556in}}%
\pgfpathlineto{\pgfqpoint{1.088171in}{1.089512in}}%
\pgfpathlineto{\pgfqpoint{1.109933in}{1.082705in}}%
\pgfpathlineto{\pgfqpoint{1.131695in}{1.075967in}}%
\pgfpathlineto{\pgfqpoint{1.153456in}{1.069912in}}%
\pgfpathlineto{\pgfqpoint{1.175218in}{1.063866in}}%
\pgfpathlineto{\pgfqpoint{1.196980in}{1.058276in}}%
\pgfpathlineto{\pgfqpoint{1.218741in}{1.053258in}}%
\pgfpathlineto{\pgfqpoint{1.240503in}{1.048098in}}%
\pgfpathlineto{\pgfqpoint{1.262265in}{1.043073in}}%
\pgfpathlineto{\pgfqpoint{1.284026in}{1.038082in}}%
\pgfpathlineto{\pgfqpoint{1.305788in}{1.033396in}}%
\pgfpathlineto{\pgfqpoint{1.327550in}{1.028833in}}%
\pgfpathlineto{\pgfqpoint{1.349311in}{1.023990in}}%
\pgfpathlineto{\pgfqpoint{1.371073in}{1.019394in}}%
\pgfpathlineto{\pgfqpoint{1.392835in}{1.014782in}}%
\pgfpathlineto{\pgfqpoint{1.414596in}{1.010626in}}%
\pgfpathlineto{\pgfqpoint{1.436358in}{1.005981in}}%
\pgfpathlineto{\pgfqpoint{1.458120in}{1.001849in}}%
\pgfpathlineto{\pgfqpoint{1.479881in}{0.997449in}}%
\pgfpathlineto{\pgfqpoint{1.501643in}{0.993407in}}%
\pgfpathlineto{\pgfqpoint{1.523405in}{0.989719in}}%
\pgfpathlineto{\pgfqpoint{1.545166in}{0.985646in}}%
\pgfpathlineto{\pgfqpoint{1.566928in}{0.981800in}}%
\pgfpathlineto{\pgfqpoint{1.588690in}{0.978120in}}%
\pgfpathlineto{\pgfqpoint{1.610451in}{0.974393in}}%
\pgfpathlineto{\pgfqpoint{1.632213in}{0.970639in}}%
\pgfpathlineto{\pgfqpoint{1.653975in}{0.967228in}}%
\pgfpathlineto{\pgfqpoint{1.675736in}{0.963325in}}%
\pgfpathlineto{\pgfqpoint{1.697498in}{0.959957in}}%
\pgfpathlineto{\pgfqpoint{1.719260in}{0.956534in}}%
\pgfpathlineto{\pgfqpoint{1.741021in}{0.953136in}}%
\pgfpathlineto{\pgfqpoint{1.762783in}{0.949692in}}%
\pgfpathlineto{\pgfqpoint{1.784545in}{0.946404in}}%
\pgfpathlineto{\pgfqpoint{1.806306in}{0.943237in}}%
\pgfpathlineto{\pgfqpoint{1.828068in}{0.940002in}}%
\pgfpathlineto{\pgfqpoint{1.849830in}{0.936868in}}%
\pgfpathlineto{\pgfqpoint{1.871591in}{0.933510in}}%
\pgfpathlineto{\pgfqpoint{1.893353in}{0.930550in}}%
\pgfpathlineto{\pgfqpoint{1.915115in}{0.927413in}}%
\pgfpathlineto{\pgfqpoint{1.936876in}{0.924132in}}%
\pgfpathlineto{\pgfqpoint{1.958638in}{0.921180in}}%
\pgfpathlineto{\pgfqpoint{1.980400in}{0.918297in}}%
\pgfpathlineto{\pgfqpoint{2.002161in}{0.915472in}}%
\pgfpathlineto{\pgfqpoint{2.023923in}{0.912547in}}%
\pgfpathlineto{\pgfqpoint{2.045685in}{0.909907in}}%
\pgfpathlineto{\pgfqpoint{2.067446in}{0.907418in}}%
\pgfpathlineto{\pgfqpoint{2.089208in}{0.904535in}}%
\pgfpathlineto{\pgfqpoint{2.110970in}{0.901827in}}%
\pgfpathlineto{\pgfqpoint{2.132731in}{0.899222in}}%
\pgfpathlineto{\pgfqpoint{2.154493in}{0.896469in}}%
\pgfpathlineto{\pgfqpoint{2.176255in}{0.893928in}}%
\pgfpathlineto{\pgfqpoint{2.198016in}{0.891241in}}%
\pgfpathlineto{\pgfqpoint{2.219778in}{0.888782in}}%
\pgfpathlineto{\pgfqpoint{2.241540in}{0.886135in}}%
\pgfpathlineto{\pgfqpoint{2.263301in}{0.883390in}}%
\pgfpathlineto{\pgfqpoint{2.285063in}{0.880849in}}%
\pgfpathlineto{\pgfqpoint{2.306825in}{0.878294in}}%
\pgfpathlineto{\pgfqpoint{2.328586in}{0.875757in}}%
\pgfpathlineto{\pgfqpoint{2.350348in}{0.873629in}}%
\pgfpathlineto{\pgfqpoint{2.372110in}{0.871268in}}%
\pgfpathlineto{\pgfqpoint{2.393871in}{0.868805in}}%
\pgfpathlineto{\pgfqpoint{2.415633in}{0.866359in}}%
\pgfpathlineto{\pgfqpoint{2.437395in}{0.863842in}}%
\pgfpathlineto{\pgfqpoint{2.459156in}{0.861524in}}%
\pgfpathlineto{\pgfqpoint{2.480918in}{0.859061in}}%
\pgfpathlineto{\pgfqpoint{2.502680in}{0.856882in}}%
\pgfpathlineto{\pgfqpoint{2.524441in}{0.854708in}}%
\pgfpathlineto{\pgfqpoint{2.546203in}{0.852323in}}%
\pgfpathlineto{\pgfqpoint{2.567965in}{0.849965in}}%
\pgfpathlineto{\pgfqpoint{2.589726in}{0.847763in}}%
\pgfpathlineto{\pgfqpoint{2.611488in}{0.845390in}}%
\pgfpathlineto{\pgfqpoint{2.633250in}{0.843171in}}%
\pgfpathlineto{\pgfqpoint{2.655011in}{0.840980in}}%
\pgfpathlineto{\pgfqpoint{2.676773in}{0.838812in}}%
\pgfpathlineto{\pgfqpoint{2.698535in}{0.836572in}}%
\pgfpathlineto{\pgfqpoint{2.720296in}{0.834517in}}%
\pgfpathlineto{\pgfqpoint{2.742058in}{0.832221in}}%
\pgfpathlineto{\pgfqpoint{2.763820in}{0.829922in}}%
\pgfpathlineto{\pgfqpoint{2.785581in}{0.827835in}}%
\pgfusepath{stroke}%
\end{pgfscope}%
\begin{pgfscope}%
\pgfpathrectangle{\pgfqpoint{0.609415in}{0.422992in}}{\pgfqpoint{2.219690in}{1.226201in}}%
\pgfusepath{clip}%
\pgfsetrectcap%
\pgfsetroundjoin%
\pgfsetlinewidth{1.003750pt}%
\definecolor{currentstroke}{rgb}{0.278431,0.278431,0.278431}%
\pgfsetstrokecolor{currentstroke}%
\pgfsetdash{}{0pt}%
\pgfpathmoveto{\pgfqpoint{0.631176in}{1.593457in}}%
\pgfpathlineto{\pgfqpoint{0.652938in}{1.591127in}}%
\pgfpathlineto{\pgfqpoint{0.674700in}{1.588778in}}%
\pgfpathlineto{\pgfqpoint{0.696461in}{1.586833in}}%
\pgfpathlineto{\pgfqpoint{0.718223in}{1.585018in}}%
\pgfpathlineto{\pgfqpoint{0.739985in}{1.583492in}}%
\pgfpathlineto{\pgfqpoint{0.761746in}{1.581864in}}%
\pgfpathlineto{\pgfqpoint{0.783508in}{1.580372in}}%
\pgfpathlineto{\pgfqpoint{0.805270in}{1.578960in}}%
\pgfpathlineto{\pgfqpoint{0.827031in}{1.577599in}}%
\pgfpathlineto{\pgfqpoint{0.848793in}{1.576106in}}%
\pgfpathlineto{\pgfqpoint{0.870555in}{1.574632in}}%
\pgfpathlineto{\pgfqpoint{0.892316in}{1.573396in}}%
\pgfpathlineto{\pgfqpoint{0.914078in}{1.572256in}}%
\pgfpathlineto{\pgfqpoint{0.935840in}{1.570967in}}%
\pgfpathlineto{\pgfqpoint{0.957601in}{1.569548in}}%
\pgfpathlineto{\pgfqpoint{0.979363in}{1.568278in}}%
\pgfpathlineto{\pgfqpoint{1.001125in}{1.566874in}}%
\pgfpathlineto{\pgfqpoint{1.022886in}{1.565597in}}%
\pgfpathlineto{\pgfqpoint{1.044648in}{1.564396in}}%
\pgfpathlineto{\pgfqpoint{1.066410in}{1.563136in}}%
\pgfpathlineto{\pgfqpoint{1.088171in}{1.562002in}}%
\pgfpathlineto{\pgfqpoint{1.109933in}{1.560786in}}%
\pgfpathlineto{\pgfqpoint{1.131695in}{1.559567in}}%
\pgfpathlineto{\pgfqpoint{1.153456in}{1.558396in}}%
\pgfpathlineto{\pgfqpoint{1.175218in}{1.557210in}}%
\pgfpathlineto{\pgfqpoint{1.196980in}{1.556013in}}%
\pgfpathlineto{\pgfqpoint{1.218741in}{1.554869in}}%
\pgfpathlineto{\pgfqpoint{1.240503in}{1.553693in}}%
\pgfpathlineto{\pgfqpoint{1.262265in}{1.552441in}}%
\pgfpathlineto{\pgfqpoint{1.284026in}{1.551260in}}%
\pgfpathlineto{\pgfqpoint{1.305788in}{1.550135in}}%
\pgfpathlineto{\pgfqpoint{1.327550in}{1.548996in}}%
\pgfpathlineto{\pgfqpoint{1.349311in}{1.547858in}}%
\pgfpathlineto{\pgfqpoint{1.371073in}{1.546739in}}%
\pgfpathlineto{\pgfqpoint{1.392835in}{1.545673in}}%
\pgfpathlineto{\pgfqpoint{1.414596in}{1.544598in}}%
\pgfpathlineto{\pgfqpoint{1.436358in}{1.543542in}}%
\pgfpathlineto{\pgfqpoint{1.458120in}{1.542402in}}%
\pgfpathlineto{\pgfqpoint{1.479881in}{1.541322in}}%
\pgfpathlineto{\pgfqpoint{1.501643in}{1.540212in}}%
\pgfpathlineto{\pgfqpoint{1.523405in}{1.539108in}}%
\pgfpathlineto{\pgfqpoint{1.545166in}{1.537974in}}%
\pgfpathlineto{\pgfqpoint{1.566928in}{1.536839in}}%
\pgfpathlineto{\pgfqpoint{1.588690in}{1.535768in}}%
\pgfpathlineto{\pgfqpoint{1.610451in}{1.534729in}}%
\pgfpathlineto{\pgfqpoint{1.632213in}{1.533744in}}%
\pgfpathlineto{\pgfqpoint{1.653975in}{1.532637in}}%
\pgfpathlineto{\pgfqpoint{1.675736in}{1.531527in}}%
\pgfpathlineto{\pgfqpoint{1.697498in}{1.530437in}}%
\pgfpathlineto{\pgfqpoint{1.719260in}{1.529280in}}%
\pgfpathlineto{\pgfqpoint{1.741021in}{1.528279in}}%
\pgfpathlineto{\pgfqpoint{1.762783in}{1.527285in}}%
\pgfpathlineto{\pgfqpoint{1.784545in}{1.526255in}}%
\pgfpathlineto{\pgfqpoint{1.806306in}{1.525181in}}%
\pgfpathlineto{\pgfqpoint{1.828068in}{1.524172in}}%
\pgfpathlineto{\pgfqpoint{1.849830in}{1.523179in}}%
\pgfpathlineto{\pgfqpoint{1.871591in}{1.522197in}}%
\pgfpathlineto{\pgfqpoint{1.893353in}{1.521133in}}%
\pgfpathlineto{\pgfqpoint{1.915115in}{1.520098in}}%
\pgfpathlineto{\pgfqpoint{1.936876in}{1.519032in}}%
\pgfpathlineto{\pgfqpoint{1.958638in}{1.517968in}}%
\pgfpathlineto{\pgfqpoint{1.980400in}{1.516884in}}%
\pgfpathlineto{\pgfqpoint{2.002161in}{1.515825in}}%
\pgfpathlineto{\pgfqpoint{2.023923in}{1.514794in}}%
\pgfpathlineto{\pgfqpoint{2.045685in}{1.513776in}}%
\pgfpathlineto{\pgfqpoint{2.067446in}{1.512783in}}%
\pgfpathlineto{\pgfqpoint{2.089208in}{1.511814in}}%
\pgfpathlineto{\pgfqpoint{2.110970in}{1.510791in}}%
\pgfpathlineto{\pgfqpoint{2.132731in}{1.509857in}}%
\pgfpathlineto{\pgfqpoint{2.154493in}{1.508879in}}%
\pgfpathlineto{\pgfqpoint{2.176255in}{1.507823in}}%
\pgfpathlineto{\pgfqpoint{2.198016in}{1.506891in}}%
\pgfpathlineto{\pgfqpoint{2.219778in}{1.505836in}}%
\pgfpathlineto{\pgfqpoint{2.241540in}{1.504872in}}%
\pgfpathlineto{\pgfqpoint{2.263301in}{1.503835in}}%
\pgfpathlineto{\pgfqpoint{2.285063in}{1.502830in}}%
\pgfpathlineto{\pgfqpoint{2.306825in}{1.501885in}}%
\pgfpathlineto{\pgfqpoint{2.328586in}{1.500898in}}%
\pgfpathlineto{\pgfqpoint{2.350348in}{1.499905in}}%
\pgfpathlineto{\pgfqpoint{2.372110in}{1.498858in}}%
\pgfpathlineto{\pgfqpoint{2.393871in}{1.497936in}}%
\pgfpathlineto{\pgfqpoint{2.415633in}{1.496931in}}%
\pgfpathlineto{\pgfqpoint{2.437395in}{1.495982in}}%
\pgfpathlineto{\pgfqpoint{2.459156in}{1.495015in}}%
\pgfpathlineto{\pgfqpoint{2.480918in}{1.494021in}}%
\pgfpathlineto{\pgfqpoint{2.502680in}{1.493060in}}%
\pgfpathlineto{\pgfqpoint{2.524441in}{1.492033in}}%
\pgfpathlineto{\pgfqpoint{2.546203in}{1.491032in}}%
\pgfpathlineto{\pgfqpoint{2.567965in}{1.490005in}}%
\pgfpathlineto{\pgfqpoint{2.589726in}{1.488984in}}%
\pgfpathlineto{\pgfqpoint{2.611488in}{1.488024in}}%
\pgfpathlineto{\pgfqpoint{2.633250in}{1.487004in}}%
\pgfpathlineto{\pgfqpoint{2.655011in}{1.485959in}}%
\pgfpathlineto{\pgfqpoint{2.676773in}{1.484923in}}%
\pgfpathlineto{\pgfqpoint{2.698535in}{1.483938in}}%
\pgfpathlineto{\pgfqpoint{2.720296in}{1.482938in}}%
\pgfpathlineto{\pgfqpoint{2.742058in}{1.481938in}}%
\pgfpathlineto{\pgfqpoint{2.763820in}{1.480940in}}%
\pgfpathlineto{\pgfqpoint{2.785581in}{1.479901in}}%
\pgfusepath{stroke}%
\end{pgfscope}%
\begin{pgfscope}%
\pgfsetrectcap%
\pgfsetmiterjoin%
\pgfsetlinewidth{0.501875pt}%
\definecolor{currentstroke}{rgb}{0.000000,0.000000,0.000000}%
\pgfsetstrokecolor{currentstroke}%
\pgfsetdash{}{0pt}%
\pgfpathmoveto{\pgfqpoint{0.609415in}{0.422992in}}%
\pgfpathlineto{\pgfqpoint{0.609415in}{1.649193in}}%
\pgfusepath{stroke}%
\end{pgfscope}%
\begin{pgfscope}%
\pgfsetrectcap%
\pgfsetmiterjoin%
\pgfsetlinewidth{0.501875pt}%
\definecolor{currentstroke}{rgb}{0.000000,0.000000,0.000000}%
\pgfsetstrokecolor{currentstroke}%
\pgfsetdash{}{0pt}%
\pgfpathmoveto{\pgfqpoint{2.829105in}{0.422992in}}%
\pgfpathlineto{\pgfqpoint{2.829105in}{1.649193in}}%
\pgfusepath{stroke}%
\end{pgfscope}%
\begin{pgfscope}%
\pgfsetrectcap%
\pgfsetmiterjoin%
\pgfsetlinewidth{0.501875pt}%
\definecolor{currentstroke}{rgb}{0.000000,0.000000,0.000000}%
\pgfsetstrokecolor{currentstroke}%
\pgfsetdash{}{0pt}%
\pgfpathmoveto{\pgfqpoint{0.609415in}{0.422992in}}%
\pgfpathlineto{\pgfqpoint{2.829105in}{0.422992in}}%
\pgfusepath{stroke}%
\end{pgfscope}%
\begin{pgfscope}%
\pgfsetrectcap%
\pgfsetmiterjoin%
\pgfsetlinewidth{0.501875pt}%
\definecolor{currentstroke}{rgb}{0.000000,0.000000,0.000000}%
\pgfsetstrokecolor{currentstroke}%
\pgfsetdash{}{0pt}%
\pgfpathmoveto{\pgfqpoint{0.609415in}{1.649193in}}%
\pgfpathlineto{\pgfqpoint{2.829105in}{1.649193in}}%
\pgfusepath{stroke}%
\end{pgfscope}%
\begin{pgfscope}%
\definecolor{textcolor}{rgb}{0.000000,0.000000,0.000000}%
\pgfsetstrokecolor{textcolor}%
\pgfsetfillcolor{textcolor}%
\pgftext[x=1.719260in,y=1.732526in,,base]{\color{textcolor}\rmfamily\fontsize{12.000000}{14.400000}\selectfont Kontinuität}%
\end{pgfscope}%
\begin{pgfscope}%
\pgfsetbuttcap%
\pgfsetmiterjoin%
\definecolor{currentfill}{rgb}{1.000000,1.000000,1.000000}%
\pgfsetfillcolor{currentfill}%
\pgfsetlinewidth{0.000000pt}%
\definecolor{currentstroke}{rgb}{0.000000,0.000000,0.000000}%
\pgfsetstrokecolor{currentstroke}%
\pgfsetstrokeopacity{0.000000}%
\pgfsetdash{}{0pt}%
\pgfpathmoveto{\pgfqpoint{3.454822in}{0.422992in}}%
\pgfpathlineto{\pgfqpoint{5.674512in}{0.422992in}}%
\pgfpathlineto{\pgfqpoint{5.674512in}{3.574193in}}%
\pgfpathlineto{\pgfqpoint{3.454822in}{3.574193in}}%
\pgfpathlineto{\pgfqpoint{3.454822in}{0.422992in}}%
\pgfpathclose%
\pgfusepath{fill}%
\end{pgfscope}%
\begin{pgfscope}%
\pgfsetbuttcap%
\pgfsetroundjoin%
\definecolor{currentfill}{rgb}{0.000000,0.000000,0.000000}%
\pgfsetfillcolor{currentfill}%
\pgfsetlinewidth{0.501875pt}%
\definecolor{currentstroke}{rgb}{0.000000,0.000000,0.000000}%
\pgfsetstrokecolor{currentstroke}%
\pgfsetdash{}{0pt}%
\pgfsys@defobject{currentmarker}{\pgfqpoint{0.000000in}{0.000000in}}{\pgfqpoint{0.000000in}{0.041667in}}{%
\pgfpathmoveto{\pgfqpoint{0.000000in}{0.000000in}}%
\pgfpathlineto{\pgfqpoint{0.000000in}{0.041667in}}%
\pgfusepath{stroke,fill}%
}%
\begin{pgfscope}%
\pgfsys@transformshift{3.454822in}{0.422992in}%
\pgfsys@useobject{currentmarker}{}%
\end{pgfscope}%
\end{pgfscope}%
\begin{pgfscope}%
\pgfsetbuttcap%
\pgfsetroundjoin%
\definecolor{currentfill}{rgb}{0.000000,0.000000,0.000000}%
\pgfsetfillcolor{currentfill}%
\pgfsetlinewidth{0.501875pt}%
\definecolor{currentstroke}{rgb}{0.000000,0.000000,0.000000}%
\pgfsetstrokecolor{currentstroke}%
\pgfsetdash{}{0pt}%
\pgfsys@defobject{currentmarker}{\pgfqpoint{0.000000in}{-0.041667in}}{\pgfqpoint{0.000000in}{0.000000in}}{%
\pgfpathmoveto{\pgfqpoint{0.000000in}{0.000000in}}%
\pgfpathlineto{\pgfqpoint{0.000000in}{-0.041667in}}%
\pgfusepath{stroke,fill}%
}%
\begin{pgfscope}%
\pgfsys@transformshift{3.454822in}{3.574193in}%
\pgfsys@useobject{currentmarker}{}%
\end{pgfscope}%
\end{pgfscope}%
\begin{pgfscope}%
\definecolor{textcolor}{rgb}{0.000000,0.000000,0.000000}%
\pgfsetstrokecolor{textcolor}%
\pgfsetfillcolor{textcolor}%
\pgftext[x=3.454822in,y=0.374381in,,top]{\color{textcolor}\rmfamily\fontsize{10.000000}{12.000000}\selectfont \(\displaystyle {0}\)}%
\end{pgfscope}%
\begin{pgfscope}%
\pgfsetbuttcap%
\pgfsetroundjoin%
\definecolor{currentfill}{rgb}{0.000000,0.000000,0.000000}%
\pgfsetfillcolor{currentfill}%
\pgfsetlinewidth{0.501875pt}%
\definecolor{currentstroke}{rgb}{0.000000,0.000000,0.000000}%
\pgfsetstrokecolor{currentstroke}%
\pgfsetdash{}{0pt}%
\pgfsys@defobject{currentmarker}{\pgfqpoint{0.000000in}{0.000000in}}{\pgfqpoint{0.000000in}{0.041667in}}{%
\pgfpathmoveto{\pgfqpoint{0.000000in}{0.000000in}}%
\pgfpathlineto{\pgfqpoint{0.000000in}{0.041667in}}%
\pgfusepath{stroke,fill}%
}%
\begin{pgfscope}%
\pgfsys@transformshift{3.890055in}{0.422992in}%
\pgfsys@useobject{currentmarker}{}%
\end{pgfscope}%
\end{pgfscope}%
\begin{pgfscope}%
\pgfsetbuttcap%
\pgfsetroundjoin%
\definecolor{currentfill}{rgb}{0.000000,0.000000,0.000000}%
\pgfsetfillcolor{currentfill}%
\pgfsetlinewidth{0.501875pt}%
\definecolor{currentstroke}{rgb}{0.000000,0.000000,0.000000}%
\pgfsetstrokecolor{currentstroke}%
\pgfsetdash{}{0pt}%
\pgfsys@defobject{currentmarker}{\pgfqpoint{0.000000in}{-0.041667in}}{\pgfqpoint{0.000000in}{0.000000in}}{%
\pgfpathmoveto{\pgfqpoint{0.000000in}{0.000000in}}%
\pgfpathlineto{\pgfqpoint{0.000000in}{-0.041667in}}%
\pgfusepath{stroke,fill}%
}%
\begin{pgfscope}%
\pgfsys@transformshift{3.890055in}{3.574193in}%
\pgfsys@useobject{currentmarker}{}%
\end{pgfscope}%
\end{pgfscope}%
\begin{pgfscope}%
\definecolor{textcolor}{rgb}{0.000000,0.000000,0.000000}%
\pgfsetstrokecolor{textcolor}%
\pgfsetfillcolor{textcolor}%
\pgftext[x=3.890055in,y=0.374381in,,top]{\color{textcolor}\rmfamily\fontsize{10.000000}{12.000000}\selectfont \(\displaystyle {20}\)}%
\end{pgfscope}%
\begin{pgfscope}%
\pgfsetbuttcap%
\pgfsetroundjoin%
\definecolor{currentfill}{rgb}{0.000000,0.000000,0.000000}%
\pgfsetfillcolor{currentfill}%
\pgfsetlinewidth{0.501875pt}%
\definecolor{currentstroke}{rgb}{0.000000,0.000000,0.000000}%
\pgfsetstrokecolor{currentstroke}%
\pgfsetdash{}{0pt}%
\pgfsys@defobject{currentmarker}{\pgfqpoint{0.000000in}{0.000000in}}{\pgfqpoint{0.000000in}{0.041667in}}{%
\pgfpathmoveto{\pgfqpoint{0.000000in}{0.000000in}}%
\pgfpathlineto{\pgfqpoint{0.000000in}{0.041667in}}%
\pgfusepath{stroke,fill}%
}%
\begin{pgfscope}%
\pgfsys@transformshift{4.325289in}{0.422992in}%
\pgfsys@useobject{currentmarker}{}%
\end{pgfscope}%
\end{pgfscope}%
\begin{pgfscope}%
\pgfsetbuttcap%
\pgfsetroundjoin%
\definecolor{currentfill}{rgb}{0.000000,0.000000,0.000000}%
\pgfsetfillcolor{currentfill}%
\pgfsetlinewidth{0.501875pt}%
\definecolor{currentstroke}{rgb}{0.000000,0.000000,0.000000}%
\pgfsetstrokecolor{currentstroke}%
\pgfsetdash{}{0pt}%
\pgfsys@defobject{currentmarker}{\pgfqpoint{0.000000in}{-0.041667in}}{\pgfqpoint{0.000000in}{0.000000in}}{%
\pgfpathmoveto{\pgfqpoint{0.000000in}{0.000000in}}%
\pgfpathlineto{\pgfqpoint{0.000000in}{-0.041667in}}%
\pgfusepath{stroke,fill}%
}%
\begin{pgfscope}%
\pgfsys@transformshift{4.325289in}{3.574193in}%
\pgfsys@useobject{currentmarker}{}%
\end{pgfscope}%
\end{pgfscope}%
\begin{pgfscope}%
\definecolor{textcolor}{rgb}{0.000000,0.000000,0.000000}%
\pgfsetstrokecolor{textcolor}%
\pgfsetfillcolor{textcolor}%
\pgftext[x=4.325289in,y=0.374381in,,top]{\color{textcolor}\rmfamily\fontsize{10.000000}{12.000000}\selectfont \(\displaystyle {40}\)}%
\end{pgfscope}%
\begin{pgfscope}%
\pgfsetbuttcap%
\pgfsetroundjoin%
\definecolor{currentfill}{rgb}{0.000000,0.000000,0.000000}%
\pgfsetfillcolor{currentfill}%
\pgfsetlinewidth{0.501875pt}%
\definecolor{currentstroke}{rgb}{0.000000,0.000000,0.000000}%
\pgfsetstrokecolor{currentstroke}%
\pgfsetdash{}{0pt}%
\pgfsys@defobject{currentmarker}{\pgfqpoint{0.000000in}{0.000000in}}{\pgfqpoint{0.000000in}{0.041667in}}{%
\pgfpathmoveto{\pgfqpoint{0.000000in}{0.000000in}}%
\pgfpathlineto{\pgfqpoint{0.000000in}{0.041667in}}%
\pgfusepath{stroke,fill}%
}%
\begin{pgfscope}%
\pgfsys@transformshift{4.760522in}{0.422992in}%
\pgfsys@useobject{currentmarker}{}%
\end{pgfscope}%
\end{pgfscope}%
\begin{pgfscope}%
\pgfsetbuttcap%
\pgfsetroundjoin%
\definecolor{currentfill}{rgb}{0.000000,0.000000,0.000000}%
\pgfsetfillcolor{currentfill}%
\pgfsetlinewidth{0.501875pt}%
\definecolor{currentstroke}{rgb}{0.000000,0.000000,0.000000}%
\pgfsetstrokecolor{currentstroke}%
\pgfsetdash{}{0pt}%
\pgfsys@defobject{currentmarker}{\pgfqpoint{0.000000in}{-0.041667in}}{\pgfqpoint{0.000000in}{0.000000in}}{%
\pgfpathmoveto{\pgfqpoint{0.000000in}{0.000000in}}%
\pgfpathlineto{\pgfqpoint{0.000000in}{-0.041667in}}%
\pgfusepath{stroke,fill}%
}%
\begin{pgfscope}%
\pgfsys@transformshift{4.760522in}{3.574193in}%
\pgfsys@useobject{currentmarker}{}%
\end{pgfscope}%
\end{pgfscope}%
\begin{pgfscope}%
\definecolor{textcolor}{rgb}{0.000000,0.000000,0.000000}%
\pgfsetstrokecolor{textcolor}%
\pgfsetfillcolor{textcolor}%
\pgftext[x=4.760522in,y=0.374381in,,top]{\color{textcolor}\rmfamily\fontsize{10.000000}{12.000000}\selectfont \(\displaystyle {60}\)}%
\end{pgfscope}%
\begin{pgfscope}%
\pgfsetbuttcap%
\pgfsetroundjoin%
\definecolor{currentfill}{rgb}{0.000000,0.000000,0.000000}%
\pgfsetfillcolor{currentfill}%
\pgfsetlinewidth{0.501875pt}%
\definecolor{currentstroke}{rgb}{0.000000,0.000000,0.000000}%
\pgfsetstrokecolor{currentstroke}%
\pgfsetdash{}{0pt}%
\pgfsys@defobject{currentmarker}{\pgfqpoint{0.000000in}{0.000000in}}{\pgfqpoint{0.000000in}{0.041667in}}{%
\pgfpathmoveto{\pgfqpoint{0.000000in}{0.000000in}}%
\pgfpathlineto{\pgfqpoint{0.000000in}{0.041667in}}%
\pgfusepath{stroke,fill}%
}%
\begin{pgfscope}%
\pgfsys@transformshift{5.195755in}{0.422992in}%
\pgfsys@useobject{currentmarker}{}%
\end{pgfscope}%
\end{pgfscope}%
\begin{pgfscope}%
\pgfsetbuttcap%
\pgfsetroundjoin%
\definecolor{currentfill}{rgb}{0.000000,0.000000,0.000000}%
\pgfsetfillcolor{currentfill}%
\pgfsetlinewidth{0.501875pt}%
\definecolor{currentstroke}{rgb}{0.000000,0.000000,0.000000}%
\pgfsetstrokecolor{currentstroke}%
\pgfsetdash{}{0pt}%
\pgfsys@defobject{currentmarker}{\pgfqpoint{0.000000in}{-0.041667in}}{\pgfqpoint{0.000000in}{0.000000in}}{%
\pgfpathmoveto{\pgfqpoint{0.000000in}{0.000000in}}%
\pgfpathlineto{\pgfqpoint{0.000000in}{-0.041667in}}%
\pgfusepath{stroke,fill}%
}%
\begin{pgfscope}%
\pgfsys@transformshift{5.195755in}{3.574193in}%
\pgfsys@useobject{currentmarker}{}%
\end{pgfscope}%
\end{pgfscope}%
\begin{pgfscope}%
\definecolor{textcolor}{rgb}{0.000000,0.000000,0.000000}%
\pgfsetstrokecolor{textcolor}%
\pgfsetfillcolor{textcolor}%
\pgftext[x=5.195755in,y=0.374381in,,top]{\color{textcolor}\rmfamily\fontsize{10.000000}{12.000000}\selectfont \(\displaystyle {80}\)}%
\end{pgfscope}%
\begin{pgfscope}%
\pgfsetbuttcap%
\pgfsetroundjoin%
\definecolor{currentfill}{rgb}{0.000000,0.000000,0.000000}%
\pgfsetfillcolor{currentfill}%
\pgfsetlinewidth{0.501875pt}%
\definecolor{currentstroke}{rgb}{0.000000,0.000000,0.000000}%
\pgfsetstrokecolor{currentstroke}%
\pgfsetdash{}{0pt}%
\pgfsys@defobject{currentmarker}{\pgfqpoint{0.000000in}{0.000000in}}{\pgfqpoint{0.000000in}{0.041667in}}{%
\pgfpathmoveto{\pgfqpoint{0.000000in}{0.000000in}}%
\pgfpathlineto{\pgfqpoint{0.000000in}{0.041667in}}%
\pgfusepath{stroke,fill}%
}%
\begin{pgfscope}%
\pgfsys@transformshift{5.630989in}{0.422992in}%
\pgfsys@useobject{currentmarker}{}%
\end{pgfscope}%
\end{pgfscope}%
\begin{pgfscope}%
\pgfsetbuttcap%
\pgfsetroundjoin%
\definecolor{currentfill}{rgb}{0.000000,0.000000,0.000000}%
\pgfsetfillcolor{currentfill}%
\pgfsetlinewidth{0.501875pt}%
\definecolor{currentstroke}{rgb}{0.000000,0.000000,0.000000}%
\pgfsetstrokecolor{currentstroke}%
\pgfsetdash{}{0pt}%
\pgfsys@defobject{currentmarker}{\pgfqpoint{0.000000in}{-0.041667in}}{\pgfqpoint{0.000000in}{0.000000in}}{%
\pgfpathmoveto{\pgfqpoint{0.000000in}{0.000000in}}%
\pgfpathlineto{\pgfqpoint{0.000000in}{-0.041667in}}%
\pgfusepath{stroke,fill}%
}%
\begin{pgfscope}%
\pgfsys@transformshift{5.630989in}{3.574193in}%
\pgfsys@useobject{currentmarker}{}%
\end{pgfscope}%
\end{pgfscope}%
\begin{pgfscope}%
\definecolor{textcolor}{rgb}{0.000000,0.000000,0.000000}%
\pgfsetstrokecolor{textcolor}%
\pgfsetfillcolor{textcolor}%
\pgftext[x=5.630989in,y=0.374381in,,top]{\color{textcolor}\rmfamily\fontsize{10.000000}{12.000000}\selectfont \(\displaystyle {100}\)}%
\end{pgfscope}%
\begin{pgfscope}%
\pgfsetbuttcap%
\pgfsetroundjoin%
\definecolor{currentfill}{rgb}{0.000000,0.000000,0.000000}%
\pgfsetfillcolor{currentfill}%
\pgfsetlinewidth{0.501875pt}%
\definecolor{currentstroke}{rgb}{0.000000,0.000000,0.000000}%
\pgfsetstrokecolor{currentstroke}%
\pgfsetdash{}{0pt}%
\pgfsys@defobject{currentmarker}{\pgfqpoint{0.000000in}{0.000000in}}{\pgfqpoint{0.000000in}{0.020833in}}{%
\pgfpathmoveto{\pgfqpoint{0.000000in}{0.000000in}}%
\pgfpathlineto{\pgfqpoint{0.000000in}{0.020833in}}%
\pgfusepath{stroke,fill}%
}%
\begin{pgfscope}%
\pgfsys@transformshift{3.563630in}{0.422992in}%
\pgfsys@useobject{currentmarker}{}%
\end{pgfscope}%
\end{pgfscope}%
\begin{pgfscope}%
\pgfsetbuttcap%
\pgfsetroundjoin%
\definecolor{currentfill}{rgb}{0.000000,0.000000,0.000000}%
\pgfsetfillcolor{currentfill}%
\pgfsetlinewidth{0.501875pt}%
\definecolor{currentstroke}{rgb}{0.000000,0.000000,0.000000}%
\pgfsetstrokecolor{currentstroke}%
\pgfsetdash{}{0pt}%
\pgfsys@defobject{currentmarker}{\pgfqpoint{0.000000in}{-0.020833in}}{\pgfqpoint{0.000000in}{0.000000in}}{%
\pgfpathmoveto{\pgfqpoint{0.000000in}{0.000000in}}%
\pgfpathlineto{\pgfqpoint{0.000000in}{-0.020833in}}%
\pgfusepath{stroke,fill}%
}%
\begin{pgfscope}%
\pgfsys@transformshift{3.563630in}{3.574193in}%
\pgfsys@useobject{currentmarker}{}%
\end{pgfscope}%
\end{pgfscope}%
\begin{pgfscope}%
\pgfsetbuttcap%
\pgfsetroundjoin%
\definecolor{currentfill}{rgb}{0.000000,0.000000,0.000000}%
\pgfsetfillcolor{currentfill}%
\pgfsetlinewidth{0.501875pt}%
\definecolor{currentstroke}{rgb}{0.000000,0.000000,0.000000}%
\pgfsetstrokecolor{currentstroke}%
\pgfsetdash{}{0pt}%
\pgfsys@defobject{currentmarker}{\pgfqpoint{0.000000in}{0.000000in}}{\pgfqpoint{0.000000in}{0.020833in}}{%
\pgfpathmoveto{\pgfqpoint{0.000000in}{0.000000in}}%
\pgfpathlineto{\pgfqpoint{0.000000in}{0.020833in}}%
\pgfusepath{stroke,fill}%
}%
\begin{pgfscope}%
\pgfsys@transformshift{3.672439in}{0.422992in}%
\pgfsys@useobject{currentmarker}{}%
\end{pgfscope}%
\end{pgfscope}%
\begin{pgfscope}%
\pgfsetbuttcap%
\pgfsetroundjoin%
\definecolor{currentfill}{rgb}{0.000000,0.000000,0.000000}%
\pgfsetfillcolor{currentfill}%
\pgfsetlinewidth{0.501875pt}%
\definecolor{currentstroke}{rgb}{0.000000,0.000000,0.000000}%
\pgfsetstrokecolor{currentstroke}%
\pgfsetdash{}{0pt}%
\pgfsys@defobject{currentmarker}{\pgfqpoint{0.000000in}{-0.020833in}}{\pgfqpoint{0.000000in}{0.000000in}}{%
\pgfpathmoveto{\pgfqpoint{0.000000in}{0.000000in}}%
\pgfpathlineto{\pgfqpoint{0.000000in}{-0.020833in}}%
\pgfusepath{stroke,fill}%
}%
\begin{pgfscope}%
\pgfsys@transformshift{3.672439in}{3.574193in}%
\pgfsys@useobject{currentmarker}{}%
\end{pgfscope}%
\end{pgfscope}%
\begin{pgfscope}%
\pgfsetbuttcap%
\pgfsetroundjoin%
\definecolor{currentfill}{rgb}{0.000000,0.000000,0.000000}%
\pgfsetfillcolor{currentfill}%
\pgfsetlinewidth{0.501875pt}%
\definecolor{currentstroke}{rgb}{0.000000,0.000000,0.000000}%
\pgfsetstrokecolor{currentstroke}%
\pgfsetdash{}{0pt}%
\pgfsys@defobject{currentmarker}{\pgfqpoint{0.000000in}{0.000000in}}{\pgfqpoint{0.000000in}{0.020833in}}{%
\pgfpathmoveto{\pgfqpoint{0.000000in}{0.000000in}}%
\pgfpathlineto{\pgfqpoint{0.000000in}{0.020833in}}%
\pgfusepath{stroke,fill}%
}%
\begin{pgfscope}%
\pgfsys@transformshift{3.781247in}{0.422992in}%
\pgfsys@useobject{currentmarker}{}%
\end{pgfscope}%
\end{pgfscope}%
\begin{pgfscope}%
\pgfsetbuttcap%
\pgfsetroundjoin%
\definecolor{currentfill}{rgb}{0.000000,0.000000,0.000000}%
\pgfsetfillcolor{currentfill}%
\pgfsetlinewidth{0.501875pt}%
\definecolor{currentstroke}{rgb}{0.000000,0.000000,0.000000}%
\pgfsetstrokecolor{currentstroke}%
\pgfsetdash{}{0pt}%
\pgfsys@defobject{currentmarker}{\pgfqpoint{0.000000in}{-0.020833in}}{\pgfqpoint{0.000000in}{0.000000in}}{%
\pgfpathmoveto{\pgfqpoint{0.000000in}{0.000000in}}%
\pgfpathlineto{\pgfqpoint{0.000000in}{-0.020833in}}%
\pgfusepath{stroke,fill}%
}%
\begin{pgfscope}%
\pgfsys@transformshift{3.781247in}{3.574193in}%
\pgfsys@useobject{currentmarker}{}%
\end{pgfscope}%
\end{pgfscope}%
\begin{pgfscope}%
\pgfsetbuttcap%
\pgfsetroundjoin%
\definecolor{currentfill}{rgb}{0.000000,0.000000,0.000000}%
\pgfsetfillcolor{currentfill}%
\pgfsetlinewidth{0.501875pt}%
\definecolor{currentstroke}{rgb}{0.000000,0.000000,0.000000}%
\pgfsetstrokecolor{currentstroke}%
\pgfsetdash{}{0pt}%
\pgfsys@defobject{currentmarker}{\pgfqpoint{0.000000in}{0.000000in}}{\pgfqpoint{0.000000in}{0.020833in}}{%
\pgfpathmoveto{\pgfqpoint{0.000000in}{0.000000in}}%
\pgfpathlineto{\pgfqpoint{0.000000in}{0.020833in}}%
\pgfusepath{stroke,fill}%
}%
\begin{pgfscope}%
\pgfsys@transformshift{3.998864in}{0.422992in}%
\pgfsys@useobject{currentmarker}{}%
\end{pgfscope}%
\end{pgfscope}%
\begin{pgfscope}%
\pgfsetbuttcap%
\pgfsetroundjoin%
\definecolor{currentfill}{rgb}{0.000000,0.000000,0.000000}%
\pgfsetfillcolor{currentfill}%
\pgfsetlinewidth{0.501875pt}%
\definecolor{currentstroke}{rgb}{0.000000,0.000000,0.000000}%
\pgfsetstrokecolor{currentstroke}%
\pgfsetdash{}{0pt}%
\pgfsys@defobject{currentmarker}{\pgfqpoint{0.000000in}{-0.020833in}}{\pgfqpoint{0.000000in}{0.000000in}}{%
\pgfpathmoveto{\pgfqpoint{0.000000in}{0.000000in}}%
\pgfpathlineto{\pgfqpoint{0.000000in}{-0.020833in}}%
\pgfusepath{stroke,fill}%
}%
\begin{pgfscope}%
\pgfsys@transformshift{3.998864in}{3.574193in}%
\pgfsys@useobject{currentmarker}{}%
\end{pgfscope}%
\end{pgfscope}%
\begin{pgfscope}%
\pgfsetbuttcap%
\pgfsetroundjoin%
\definecolor{currentfill}{rgb}{0.000000,0.000000,0.000000}%
\pgfsetfillcolor{currentfill}%
\pgfsetlinewidth{0.501875pt}%
\definecolor{currentstroke}{rgb}{0.000000,0.000000,0.000000}%
\pgfsetstrokecolor{currentstroke}%
\pgfsetdash{}{0pt}%
\pgfsys@defobject{currentmarker}{\pgfqpoint{0.000000in}{0.000000in}}{\pgfqpoint{0.000000in}{0.020833in}}{%
\pgfpathmoveto{\pgfqpoint{0.000000in}{0.000000in}}%
\pgfpathlineto{\pgfqpoint{0.000000in}{0.020833in}}%
\pgfusepath{stroke,fill}%
}%
\begin{pgfscope}%
\pgfsys@transformshift{4.107672in}{0.422992in}%
\pgfsys@useobject{currentmarker}{}%
\end{pgfscope}%
\end{pgfscope}%
\begin{pgfscope}%
\pgfsetbuttcap%
\pgfsetroundjoin%
\definecolor{currentfill}{rgb}{0.000000,0.000000,0.000000}%
\pgfsetfillcolor{currentfill}%
\pgfsetlinewidth{0.501875pt}%
\definecolor{currentstroke}{rgb}{0.000000,0.000000,0.000000}%
\pgfsetstrokecolor{currentstroke}%
\pgfsetdash{}{0pt}%
\pgfsys@defobject{currentmarker}{\pgfqpoint{0.000000in}{-0.020833in}}{\pgfqpoint{0.000000in}{0.000000in}}{%
\pgfpathmoveto{\pgfqpoint{0.000000in}{0.000000in}}%
\pgfpathlineto{\pgfqpoint{0.000000in}{-0.020833in}}%
\pgfusepath{stroke,fill}%
}%
\begin{pgfscope}%
\pgfsys@transformshift{4.107672in}{3.574193in}%
\pgfsys@useobject{currentmarker}{}%
\end{pgfscope}%
\end{pgfscope}%
\begin{pgfscope}%
\pgfsetbuttcap%
\pgfsetroundjoin%
\definecolor{currentfill}{rgb}{0.000000,0.000000,0.000000}%
\pgfsetfillcolor{currentfill}%
\pgfsetlinewidth{0.501875pt}%
\definecolor{currentstroke}{rgb}{0.000000,0.000000,0.000000}%
\pgfsetstrokecolor{currentstroke}%
\pgfsetdash{}{0pt}%
\pgfsys@defobject{currentmarker}{\pgfqpoint{0.000000in}{0.000000in}}{\pgfqpoint{0.000000in}{0.020833in}}{%
\pgfpathmoveto{\pgfqpoint{0.000000in}{0.000000in}}%
\pgfpathlineto{\pgfqpoint{0.000000in}{0.020833in}}%
\pgfusepath{stroke,fill}%
}%
\begin{pgfscope}%
\pgfsys@transformshift{4.216480in}{0.422992in}%
\pgfsys@useobject{currentmarker}{}%
\end{pgfscope}%
\end{pgfscope}%
\begin{pgfscope}%
\pgfsetbuttcap%
\pgfsetroundjoin%
\definecolor{currentfill}{rgb}{0.000000,0.000000,0.000000}%
\pgfsetfillcolor{currentfill}%
\pgfsetlinewidth{0.501875pt}%
\definecolor{currentstroke}{rgb}{0.000000,0.000000,0.000000}%
\pgfsetstrokecolor{currentstroke}%
\pgfsetdash{}{0pt}%
\pgfsys@defobject{currentmarker}{\pgfqpoint{0.000000in}{-0.020833in}}{\pgfqpoint{0.000000in}{0.000000in}}{%
\pgfpathmoveto{\pgfqpoint{0.000000in}{0.000000in}}%
\pgfpathlineto{\pgfqpoint{0.000000in}{-0.020833in}}%
\pgfusepath{stroke,fill}%
}%
\begin{pgfscope}%
\pgfsys@transformshift{4.216480in}{3.574193in}%
\pgfsys@useobject{currentmarker}{}%
\end{pgfscope}%
\end{pgfscope}%
\begin{pgfscope}%
\pgfsetbuttcap%
\pgfsetroundjoin%
\definecolor{currentfill}{rgb}{0.000000,0.000000,0.000000}%
\pgfsetfillcolor{currentfill}%
\pgfsetlinewidth{0.501875pt}%
\definecolor{currentstroke}{rgb}{0.000000,0.000000,0.000000}%
\pgfsetstrokecolor{currentstroke}%
\pgfsetdash{}{0pt}%
\pgfsys@defobject{currentmarker}{\pgfqpoint{0.000000in}{0.000000in}}{\pgfqpoint{0.000000in}{0.020833in}}{%
\pgfpathmoveto{\pgfqpoint{0.000000in}{0.000000in}}%
\pgfpathlineto{\pgfqpoint{0.000000in}{0.020833in}}%
\pgfusepath{stroke,fill}%
}%
\begin{pgfscope}%
\pgfsys@transformshift{4.434097in}{0.422992in}%
\pgfsys@useobject{currentmarker}{}%
\end{pgfscope}%
\end{pgfscope}%
\begin{pgfscope}%
\pgfsetbuttcap%
\pgfsetroundjoin%
\definecolor{currentfill}{rgb}{0.000000,0.000000,0.000000}%
\pgfsetfillcolor{currentfill}%
\pgfsetlinewidth{0.501875pt}%
\definecolor{currentstroke}{rgb}{0.000000,0.000000,0.000000}%
\pgfsetstrokecolor{currentstroke}%
\pgfsetdash{}{0pt}%
\pgfsys@defobject{currentmarker}{\pgfqpoint{0.000000in}{-0.020833in}}{\pgfqpoint{0.000000in}{0.000000in}}{%
\pgfpathmoveto{\pgfqpoint{0.000000in}{0.000000in}}%
\pgfpathlineto{\pgfqpoint{0.000000in}{-0.020833in}}%
\pgfusepath{stroke,fill}%
}%
\begin{pgfscope}%
\pgfsys@transformshift{4.434097in}{3.574193in}%
\pgfsys@useobject{currentmarker}{}%
\end{pgfscope}%
\end{pgfscope}%
\begin{pgfscope}%
\pgfsetbuttcap%
\pgfsetroundjoin%
\definecolor{currentfill}{rgb}{0.000000,0.000000,0.000000}%
\pgfsetfillcolor{currentfill}%
\pgfsetlinewidth{0.501875pt}%
\definecolor{currentstroke}{rgb}{0.000000,0.000000,0.000000}%
\pgfsetstrokecolor{currentstroke}%
\pgfsetdash{}{0pt}%
\pgfsys@defobject{currentmarker}{\pgfqpoint{0.000000in}{0.000000in}}{\pgfqpoint{0.000000in}{0.020833in}}{%
\pgfpathmoveto{\pgfqpoint{0.000000in}{0.000000in}}%
\pgfpathlineto{\pgfqpoint{0.000000in}{0.020833in}}%
\pgfusepath{stroke,fill}%
}%
\begin{pgfscope}%
\pgfsys@transformshift{4.542905in}{0.422992in}%
\pgfsys@useobject{currentmarker}{}%
\end{pgfscope}%
\end{pgfscope}%
\begin{pgfscope}%
\pgfsetbuttcap%
\pgfsetroundjoin%
\definecolor{currentfill}{rgb}{0.000000,0.000000,0.000000}%
\pgfsetfillcolor{currentfill}%
\pgfsetlinewidth{0.501875pt}%
\definecolor{currentstroke}{rgb}{0.000000,0.000000,0.000000}%
\pgfsetstrokecolor{currentstroke}%
\pgfsetdash{}{0pt}%
\pgfsys@defobject{currentmarker}{\pgfqpoint{0.000000in}{-0.020833in}}{\pgfqpoint{0.000000in}{0.000000in}}{%
\pgfpathmoveto{\pgfqpoint{0.000000in}{0.000000in}}%
\pgfpathlineto{\pgfqpoint{0.000000in}{-0.020833in}}%
\pgfusepath{stroke,fill}%
}%
\begin{pgfscope}%
\pgfsys@transformshift{4.542905in}{3.574193in}%
\pgfsys@useobject{currentmarker}{}%
\end{pgfscope}%
\end{pgfscope}%
\begin{pgfscope}%
\pgfsetbuttcap%
\pgfsetroundjoin%
\definecolor{currentfill}{rgb}{0.000000,0.000000,0.000000}%
\pgfsetfillcolor{currentfill}%
\pgfsetlinewidth{0.501875pt}%
\definecolor{currentstroke}{rgb}{0.000000,0.000000,0.000000}%
\pgfsetstrokecolor{currentstroke}%
\pgfsetdash{}{0pt}%
\pgfsys@defobject{currentmarker}{\pgfqpoint{0.000000in}{0.000000in}}{\pgfqpoint{0.000000in}{0.020833in}}{%
\pgfpathmoveto{\pgfqpoint{0.000000in}{0.000000in}}%
\pgfpathlineto{\pgfqpoint{0.000000in}{0.020833in}}%
\pgfusepath{stroke,fill}%
}%
\begin{pgfscope}%
\pgfsys@transformshift{4.651714in}{0.422992in}%
\pgfsys@useobject{currentmarker}{}%
\end{pgfscope}%
\end{pgfscope}%
\begin{pgfscope}%
\pgfsetbuttcap%
\pgfsetroundjoin%
\definecolor{currentfill}{rgb}{0.000000,0.000000,0.000000}%
\pgfsetfillcolor{currentfill}%
\pgfsetlinewidth{0.501875pt}%
\definecolor{currentstroke}{rgb}{0.000000,0.000000,0.000000}%
\pgfsetstrokecolor{currentstroke}%
\pgfsetdash{}{0pt}%
\pgfsys@defobject{currentmarker}{\pgfqpoint{0.000000in}{-0.020833in}}{\pgfqpoint{0.000000in}{0.000000in}}{%
\pgfpathmoveto{\pgfqpoint{0.000000in}{0.000000in}}%
\pgfpathlineto{\pgfqpoint{0.000000in}{-0.020833in}}%
\pgfusepath{stroke,fill}%
}%
\begin{pgfscope}%
\pgfsys@transformshift{4.651714in}{3.574193in}%
\pgfsys@useobject{currentmarker}{}%
\end{pgfscope}%
\end{pgfscope}%
\begin{pgfscope}%
\pgfsetbuttcap%
\pgfsetroundjoin%
\definecolor{currentfill}{rgb}{0.000000,0.000000,0.000000}%
\pgfsetfillcolor{currentfill}%
\pgfsetlinewidth{0.501875pt}%
\definecolor{currentstroke}{rgb}{0.000000,0.000000,0.000000}%
\pgfsetstrokecolor{currentstroke}%
\pgfsetdash{}{0pt}%
\pgfsys@defobject{currentmarker}{\pgfqpoint{0.000000in}{0.000000in}}{\pgfqpoint{0.000000in}{0.020833in}}{%
\pgfpathmoveto{\pgfqpoint{0.000000in}{0.000000in}}%
\pgfpathlineto{\pgfqpoint{0.000000in}{0.020833in}}%
\pgfusepath{stroke,fill}%
}%
\begin{pgfscope}%
\pgfsys@transformshift{4.869330in}{0.422992in}%
\pgfsys@useobject{currentmarker}{}%
\end{pgfscope}%
\end{pgfscope}%
\begin{pgfscope}%
\pgfsetbuttcap%
\pgfsetroundjoin%
\definecolor{currentfill}{rgb}{0.000000,0.000000,0.000000}%
\pgfsetfillcolor{currentfill}%
\pgfsetlinewidth{0.501875pt}%
\definecolor{currentstroke}{rgb}{0.000000,0.000000,0.000000}%
\pgfsetstrokecolor{currentstroke}%
\pgfsetdash{}{0pt}%
\pgfsys@defobject{currentmarker}{\pgfqpoint{0.000000in}{-0.020833in}}{\pgfqpoint{0.000000in}{0.000000in}}{%
\pgfpathmoveto{\pgfqpoint{0.000000in}{0.000000in}}%
\pgfpathlineto{\pgfqpoint{0.000000in}{-0.020833in}}%
\pgfusepath{stroke,fill}%
}%
\begin{pgfscope}%
\pgfsys@transformshift{4.869330in}{3.574193in}%
\pgfsys@useobject{currentmarker}{}%
\end{pgfscope}%
\end{pgfscope}%
\begin{pgfscope}%
\pgfsetbuttcap%
\pgfsetroundjoin%
\definecolor{currentfill}{rgb}{0.000000,0.000000,0.000000}%
\pgfsetfillcolor{currentfill}%
\pgfsetlinewidth{0.501875pt}%
\definecolor{currentstroke}{rgb}{0.000000,0.000000,0.000000}%
\pgfsetstrokecolor{currentstroke}%
\pgfsetdash{}{0pt}%
\pgfsys@defobject{currentmarker}{\pgfqpoint{0.000000in}{0.000000in}}{\pgfqpoint{0.000000in}{0.020833in}}{%
\pgfpathmoveto{\pgfqpoint{0.000000in}{0.000000in}}%
\pgfpathlineto{\pgfqpoint{0.000000in}{0.020833in}}%
\pgfusepath{stroke,fill}%
}%
\begin{pgfscope}%
\pgfsys@transformshift{4.978139in}{0.422992in}%
\pgfsys@useobject{currentmarker}{}%
\end{pgfscope}%
\end{pgfscope}%
\begin{pgfscope}%
\pgfsetbuttcap%
\pgfsetroundjoin%
\definecolor{currentfill}{rgb}{0.000000,0.000000,0.000000}%
\pgfsetfillcolor{currentfill}%
\pgfsetlinewidth{0.501875pt}%
\definecolor{currentstroke}{rgb}{0.000000,0.000000,0.000000}%
\pgfsetstrokecolor{currentstroke}%
\pgfsetdash{}{0pt}%
\pgfsys@defobject{currentmarker}{\pgfqpoint{0.000000in}{-0.020833in}}{\pgfqpoint{0.000000in}{0.000000in}}{%
\pgfpathmoveto{\pgfqpoint{0.000000in}{0.000000in}}%
\pgfpathlineto{\pgfqpoint{0.000000in}{-0.020833in}}%
\pgfusepath{stroke,fill}%
}%
\begin{pgfscope}%
\pgfsys@transformshift{4.978139in}{3.574193in}%
\pgfsys@useobject{currentmarker}{}%
\end{pgfscope}%
\end{pgfscope}%
\begin{pgfscope}%
\pgfsetbuttcap%
\pgfsetroundjoin%
\definecolor{currentfill}{rgb}{0.000000,0.000000,0.000000}%
\pgfsetfillcolor{currentfill}%
\pgfsetlinewidth{0.501875pt}%
\definecolor{currentstroke}{rgb}{0.000000,0.000000,0.000000}%
\pgfsetstrokecolor{currentstroke}%
\pgfsetdash{}{0pt}%
\pgfsys@defobject{currentmarker}{\pgfqpoint{0.000000in}{0.000000in}}{\pgfqpoint{0.000000in}{0.020833in}}{%
\pgfpathmoveto{\pgfqpoint{0.000000in}{0.000000in}}%
\pgfpathlineto{\pgfqpoint{0.000000in}{0.020833in}}%
\pgfusepath{stroke,fill}%
}%
\begin{pgfscope}%
\pgfsys@transformshift{5.086947in}{0.422992in}%
\pgfsys@useobject{currentmarker}{}%
\end{pgfscope}%
\end{pgfscope}%
\begin{pgfscope}%
\pgfsetbuttcap%
\pgfsetroundjoin%
\definecolor{currentfill}{rgb}{0.000000,0.000000,0.000000}%
\pgfsetfillcolor{currentfill}%
\pgfsetlinewidth{0.501875pt}%
\definecolor{currentstroke}{rgb}{0.000000,0.000000,0.000000}%
\pgfsetstrokecolor{currentstroke}%
\pgfsetdash{}{0pt}%
\pgfsys@defobject{currentmarker}{\pgfqpoint{0.000000in}{-0.020833in}}{\pgfqpoint{0.000000in}{0.000000in}}{%
\pgfpathmoveto{\pgfqpoint{0.000000in}{0.000000in}}%
\pgfpathlineto{\pgfqpoint{0.000000in}{-0.020833in}}%
\pgfusepath{stroke,fill}%
}%
\begin{pgfscope}%
\pgfsys@transformshift{5.086947in}{3.574193in}%
\pgfsys@useobject{currentmarker}{}%
\end{pgfscope}%
\end{pgfscope}%
\begin{pgfscope}%
\pgfsetbuttcap%
\pgfsetroundjoin%
\definecolor{currentfill}{rgb}{0.000000,0.000000,0.000000}%
\pgfsetfillcolor{currentfill}%
\pgfsetlinewidth{0.501875pt}%
\definecolor{currentstroke}{rgb}{0.000000,0.000000,0.000000}%
\pgfsetstrokecolor{currentstroke}%
\pgfsetdash{}{0pt}%
\pgfsys@defobject{currentmarker}{\pgfqpoint{0.000000in}{0.000000in}}{\pgfqpoint{0.000000in}{0.020833in}}{%
\pgfpathmoveto{\pgfqpoint{0.000000in}{0.000000in}}%
\pgfpathlineto{\pgfqpoint{0.000000in}{0.020833in}}%
\pgfusepath{stroke,fill}%
}%
\begin{pgfscope}%
\pgfsys@transformshift{5.304564in}{0.422992in}%
\pgfsys@useobject{currentmarker}{}%
\end{pgfscope}%
\end{pgfscope}%
\begin{pgfscope}%
\pgfsetbuttcap%
\pgfsetroundjoin%
\definecolor{currentfill}{rgb}{0.000000,0.000000,0.000000}%
\pgfsetfillcolor{currentfill}%
\pgfsetlinewidth{0.501875pt}%
\definecolor{currentstroke}{rgb}{0.000000,0.000000,0.000000}%
\pgfsetstrokecolor{currentstroke}%
\pgfsetdash{}{0pt}%
\pgfsys@defobject{currentmarker}{\pgfqpoint{0.000000in}{-0.020833in}}{\pgfqpoint{0.000000in}{0.000000in}}{%
\pgfpathmoveto{\pgfqpoint{0.000000in}{0.000000in}}%
\pgfpathlineto{\pgfqpoint{0.000000in}{-0.020833in}}%
\pgfusepath{stroke,fill}%
}%
\begin{pgfscope}%
\pgfsys@transformshift{5.304564in}{3.574193in}%
\pgfsys@useobject{currentmarker}{}%
\end{pgfscope}%
\end{pgfscope}%
\begin{pgfscope}%
\pgfsetbuttcap%
\pgfsetroundjoin%
\definecolor{currentfill}{rgb}{0.000000,0.000000,0.000000}%
\pgfsetfillcolor{currentfill}%
\pgfsetlinewidth{0.501875pt}%
\definecolor{currentstroke}{rgb}{0.000000,0.000000,0.000000}%
\pgfsetstrokecolor{currentstroke}%
\pgfsetdash{}{0pt}%
\pgfsys@defobject{currentmarker}{\pgfqpoint{0.000000in}{0.000000in}}{\pgfqpoint{0.000000in}{0.020833in}}{%
\pgfpathmoveto{\pgfqpoint{0.000000in}{0.000000in}}%
\pgfpathlineto{\pgfqpoint{0.000000in}{0.020833in}}%
\pgfusepath{stroke,fill}%
}%
\begin{pgfscope}%
\pgfsys@transformshift{5.413372in}{0.422992in}%
\pgfsys@useobject{currentmarker}{}%
\end{pgfscope}%
\end{pgfscope}%
\begin{pgfscope}%
\pgfsetbuttcap%
\pgfsetroundjoin%
\definecolor{currentfill}{rgb}{0.000000,0.000000,0.000000}%
\pgfsetfillcolor{currentfill}%
\pgfsetlinewidth{0.501875pt}%
\definecolor{currentstroke}{rgb}{0.000000,0.000000,0.000000}%
\pgfsetstrokecolor{currentstroke}%
\pgfsetdash{}{0pt}%
\pgfsys@defobject{currentmarker}{\pgfqpoint{0.000000in}{-0.020833in}}{\pgfqpoint{0.000000in}{0.000000in}}{%
\pgfpathmoveto{\pgfqpoint{0.000000in}{0.000000in}}%
\pgfpathlineto{\pgfqpoint{0.000000in}{-0.020833in}}%
\pgfusepath{stroke,fill}%
}%
\begin{pgfscope}%
\pgfsys@transformshift{5.413372in}{3.574193in}%
\pgfsys@useobject{currentmarker}{}%
\end{pgfscope}%
\end{pgfscope}%
\begin{pgfscope}%
\pgfsetbuttcap%
\pgfsetroundjoin%
\definecolor{currentfill}{rgb}{0.000000,0.000000,0.000000}%
\pgfsetfillcolor{currentfill}%
\pgfsetlinewidth{0.501875pt}%
\definecolor{currentstroke}{rgb}{0.000000,0.000000,0.000000}%
\pgfsetstrokecolor{currentstroke}%
\pgfsetdash{}{0pt}%
\pgfsys@defobject{currentmarker}{\pgfqpoint{0.000000in}{0.000000in}}{\pgfqpoint{0.000000in}{0.020833in}}{%
\pgfpathmoveto{\pgfqpoint{0.000000in}{0.000000in}}%
\pgfpathlineto{\pgfqpoint{0.000000in}{0.020833in}}%
\pgfusepath{stroke,fill}%
}%
\begin{pgfscope}%
\pgfsys@transformshift{5.522180in}{0.422992in}%
\pgfsys@useobject{currentmarker}{}%
\end{pgfscope}%
\end{pgfscope}%
\begin{pgfscope}%
\pgfsetbuttcap%
\pgfsetroundjoin%
\definecolor{currentfill}{rgb}{0.000000,0.000000,0.000000}%
\pgfsetfillcolor{currentfill}%
\pgfsetlinewidth{0.501875pt}%
\definecolor{currentstroke}{rgb}{0.000000,0.000000,0.000000}%
\pgfsetstrokecolor{currentstroke}%
\pgfsetdash{}{0pt}%
\pgfsys@defobject{currentmarker}{\pgfqpoint{0.000000in}{-0.020833in}}{\pgfqpoint{0.000000in}{0.000000in}}{%
\pgfpathmoveto{\pgfqpoint{0.000000in}{0.000000in}}%
\pgfpathlineto{\pgfqpoint{0.000000in}{-0.020833in}}%
\pgfusepath{stroke,fill}%
}%
\begin{pgfscope}%
\pgfsys@transformshift{5.522180in}{3.574193in}%
\pgfsys@useobject{currentmarker}{}%
\end{pgfscope}%
\end{pgfscope}%
\begin{pgfscope}%
\definecolor{textcolor}{rgb}{0.000000,0.000000,0.000000}%
\pgfsetstrokecolor{textcolor}%
\pgfsetfillcolor{textcolor}%
\pgftext[x=4.564667in,y=0.184413in,,top]{\color{textcolor}\rmfamily\fontsize{10.000000}{12.000000}\selectfont \(\displaystyle K\)}%
\end{pgfscope}%
\begin{pgfscope}%
\pgfsetbuttcap%
\pgfsetroundjoin%
\definecolor{currentfill}{rgb}{0.000000,0.000000,0.000000}%
\pgfsetfillcolor{currentfill}%
\pgfsetlinewidth{0.501875pt}%
\definecolor{currentstroke}{rgb}{0.000000,0.000000,0.000000}%
\pgfsetstrokecolor{currentstroke}%
\pgfsetdash{}{0pt}%
\pgfsys@defobject{currentmarker}{\pgfqpoint{0.000000in}{0.000000in}}{\pgfqpoint{0.041667in}{0.000000in}}{%
\pgfpathmoveto{\pgfqpoint{0.000000in}{0.000000in}}%
\pgfpathlineto{\pgfqpoint{0.041667in}{0.000000in}}%
\pgfusepath{stroke,fill}%
}%
\begin{pgfscope}%
\pgfsys@transformshift{3.454822in}{0.715529in}%
\pgfsys@useobject{currentmarker}{}%
\end{pgfscope}%
\end{pgfscope}%
\begin{pgfscope}%
\pgfsetbuttcap%
\pgfsetroundjoin%
\definecolor{currentfill}{rgb}{0.000000,0.000000,0.000000}%
\pgfsetfillcolor{currentfill}%
\pgfsetlinewidth{0.501875pt}%
\definecolor{currentstroke}{rgb}{0.000000,0.000000,0.000000}%
\pgfsetstrokecolor{currentstroke}%
\pgfsetdash{}{0pt}%
\pgfsys@defobject{currentmarker}{\pgfqpoint{-0.041667in}{0.000000in}}{\pgfqpoint{-0.000000in}{0.000000in}}{%
\pgfpathmoveto{\pgfqpoint{-0.000000in}{0.000000in}}%
\pgfpathlineto{\pgfqpoint{-0.041667in}{0.000000in}}%
\pgfusepath{stroke,fill}%
}%
\begin{pgfscope}%
\pgfsys@transformshift{5.674512in}{0.715529in}%
\pgfsys@useobject{currentmarker}{}%
\end{pgfscope}%
\end{pgfscope}%
\begin{pgfscope}%
\definecolor{textcolor}{rgb}{0.000000,0.000000,0.000000}%
\pgfsetstrokecolor{textcolor}%
\pgfsetfillcolor{textcolor}%
\pgftext[x=3.228741in, y=0.662767in, left, base]{\color{textcolor}\rmfamily\fontsize{10.000000}{12.000000}\selectfont \(\displaystyle {0.1}\)}%
\end{pgfscope}%
\begin{pgfscope}%
\pgfsetbuttcap%
\pgfsetroundjoin%
\definecolor{currentfill}{rgb}{0.000000,0.000000,0.000000}%
\pgfsetfillcolor{currentfill}%
\pgfsetlinewidth{0.501875pt}%
\definecolor{currentstroke}{rgb}{0.000000,0.000000,0.000000}%
\pgfsetstrokecolor{currentstroke}%
\pgfsetdash{}{0pt}%
\pgfsys@defobject{currentmarker}{\pgfqpoint{0.000000in}{0.000000in}}{\pgfqpoint{0.041667in}{0.000000in}}{%
\pgfpathmoveto{\pgfqpoint{0.000000in}{0.000000in}}%
\pgfpathlineto{\pgfqpoint{0.041667in}{0.000000in}}%
\pgfusepath{stroke,fill}%
}%
\begin{pgfscope}%
\pgfsys@transformshift{3.454822in}{1.253255in}%
\pgfsys@useobject{currentmarker}{}%
\end{pgfscope}%
\end{pgfscope}%
\begin{pgfscope}%
\pgfsetbuttcap%
\pgfsetroundjoin%
\definecolor{currentfill}{rgb}{0.000000,0.000000,0.000000}%
\pgfsetfillcolor{currentfill}%
\pgfsetlinewidth{0.501875pt}%
\definecolor{currentstroke}{rgb}{0.000000,0.000000,0.000000}%
\pgfsetstrokecolor{currentstroke}%
\pgfsetdash{}{0pt}%
\pgfsys@defobject{currentmarker}{\pgfqpoint{-0.041667in}{0.000000in}}{\pgfqpoint{-0.000000in}{0.000000in}}{%
\pgfpathmoveto{\pgfqpoint{-0.000000in}{0.000000in}}%
\pgfpathlineto{\pgfqpoint{-0.041667in}{0.000000in}}%
\pgfusepath{stroke,fill}%
}%
\begin{pgfscope}%
\pgfsys@transformshift{5.674512in}{1.253255in}%
\pgfsys@useobject{currentmarker}{}%
\end{pgfscope}%
\end{pgfscope}%
\begin{pgfscope}%
\definecolor{textcolor}{rgb}{0.000000,0.000000,0.000000}%
\pgfsetstrokecolor{textcolor}%
\pgfsetfillcolor{textcolor}%
\pgftext[x=3.228741in, y=1.200494in, left, base]{\color{textcolor}\rmfamily\fontsize{10.000000}{12.000000}\selectfont \(\displaystyle {0.2}\)}%
\end{pgfscope}%
\begin{pgfscope}%
\pgfsetbuttcap%
\pgfsetroundjoin%
\definecolor{currentfill}{rgb}{0.000000,0.000000,0.000000}%
\pgfsetfillcolor{currentfill}%
\pgfsetlinewidth{0.501875pt}%
\definecolor{currentstroke}{rgb}{0.000000,0.000000,0.000000}%
\pgfsetstrokecolor{currentstroke}%
\pgfsetdash{}{0pt}%
\pgfsys@defobject{currentmarker}{\pgfqpoint{0.000000in}{0.000000in}}{\pgfqpoint{0.041667in}{0.000000in}}{%
\pgfpathmoveto{\pgfqpoint{0.000000in}{0.000000in}}%
\pgfpathlineto{\pgfqpoint{0.041667in}{0.000000in}}%
\pgfusepath{stroke,fill}%
}%
\begin{pgfscope}%
\pgfsys@transformshift{3.454822in}{1.790982in}%
\pgfsys@useobject{currentmarker}{}%
\end{pgfscope}%
\end{pgfscope}%
\begin{pgfscope}%
\pgfsetbuttcap%
\pgfsetroundjoin%
\definecolor{currentfill}{rgb}{0.000000,0.000000,0.000000}%
\pgfsetfillcolor{currentfill}%
\pgfsetlinewidth{0.501875pt}%
\definecolor{currentstroke}{rgb}{0.000000,0.000000,0.000000}%
\pgfsetstrokecolor{currentstroke}%
\pgfsetdash{}{0pt}%
\pgfsys@defobject{currentmarker}{\pgfqpoint{-0.041667in}{0.000000in}}{\pgfqpoint{-0.000000in}{0.000000in}}{%
\pgfpathmoveto{\pgfqpoint{-0.000000in}{0.000000in}}%
\pgfpathlineto{\pgfqpoint{-0.041667in}{0.000000in}}%
\pgfusepath{stroke,fill}%
}%
\begin{pgfscope}%
\pgfsys@transformshift{5.674512in}{1.790982in}%
\pgfsys@useobject{currentmarker}{}%
\end{pgfscope}%
\end{pgfscope}%
\begin{pgfscope}%
\definecolor{textcolor}{rgb}{0.000000,0.000000,0.000000}%
\pgfsetstrokecolor{textcolor}%
\pgfsetfillcolor{textcolor}%
\pgftext[x=3.228741in, y=1.738220in, left, base]{\color{textcolor}\rmfamily\fontsize{10.000000}{12.000000}\selectfont \(\displaystyle {0.3}\)}%
\end{pgfscope}%
\begin{pgfscope}%
\pgfsetbuttcap%
\pgfsetroundjoin%
\definecolor{currentfill}{rgb}{0.000000,0.000000,0.000000}%
\pgfsetfillcolor{currentfill}%
\pgfsetlinewidth{0.501875pt}%
\definecolor{currentstroke}{rgb}{0.000000,0.000000,0.000000}%
\pgfsetstrokecolor{currentstroke}%
\pgfsetdash{}{0pt}%
\pgfsys@defobject{currentmarker}{\pgfqpoint{0.000000in}{0.000000in}}{\pgfqpoint{0.041667in}{0.000000in}}{%
\pgfpathmoveto{\pgfqpoint{0.000000in}{0.000000in}}%
\pgfpathlineto{\pgfqpoint{0.041667in}{0.000000in}}%
\pgfusepath{stroke,fill}%
}%
\begin{pgfscope}%
\pgfsys@transformshift{3.454822in}{2.328708in}%
\pgfsys@useobject{currentmarker}{}%
\end{pgfscope}%
\end{pgfscope}%
\begin{pgfscope}%
\pgfsetbuttcap%
\pgfsetroundjoin%
\definecolor{currentfill}{rgb}{0.000000,0.000000,0.000000}%
\pgfsetfillcolor{currentfill}%
\pgfsetlinewidth{0.501875pt}%
\definecolor{currentstroke}{rgb}{0.000000,0.000000,0.000000}%
\pgfsetstrokecolor{currentstroke}%
\pgfsetdash{}{0pt}%
\pgfsys@defobject{currentmarker}{\pgfqpoint{-0.041667in}{0.000000in}}{\pgfqpoint{-0.000000in}{0.000000in}}{%
\pgfpathmoveto{\pgfqpoint{-0.000000in}{0.000000in}}%
\pgfpathlineto{\pgfqpoint{-0.041667in}{0.000000in}}%
\pgfusepath{stroke,fill}%
}%
\begin{pgfscope}%
\pgfsys@transformshift{5.674512in}{2.328708in}%
\pgfsys@useobject{currentmarker}{}%
\end{pgfscope}%
\end{pgfscope}%
\begin{pgfscope}%
\definecolor{textcolor}{rgb}{0.000000,0.000000,0.000000}%
\pgfsetstrokecolor{textcolor}%
\pgfsetfillcolor{textcolor}%
\pgftext[x=3.228741in, y=2.275946in, left, base]{\color{textcolor}\rmfamily\fontsize{10.000000}{12.000000}\selectfont \(\displaystyle {0.4}\)}%
\end{pgfscope}%
\begin{pgfscope}%
\pgfsetbuttcap%
\pgfsetroundjoin%
\definecolor{currentfill}{rgb}{0.000000,0.000000,0.000000}%
\pgfsetfillcolor{currentfill}%
\pgfsetlinewidth{0.501875pt}%
\definecolor{currentstroke}{rgb}{0.000000,0.000000,0.000000}%
\pgfsetstrokecolor{currentstroke}%
\pgfsetdash{}{0pt}%
\pgfsys@defobject{currentmarker}{\pgfqpoint{0.000000in}{0.000000in}}{\pgfqpoint{0.041667in}{0.000000in}}{%
\pgfpathmoveto{\pgfqpoint{0.000000in}{0.000000in}}%
\pgfpathlineto{\pgfqpoint{0.041667in}{0.000000in}}%
\pgfusepath{stroke,fill}%
}%
\begin{pgfscope}%
\pgfsys@transformshift{3.454822in}{2.866434in}%
\pgfsys@useobject{currentmarker}{}%
\end{pgfscope}%
\end{pgfscope}%
\begin{pgfscope}%
\pgfsetbuttcap%
\pgfsetroundjoin%
\definecolor{currentfill}{rgb}{0.000000,0.000000,0.000000}%
\pgfsetfillcolor{currentfill}%
\pgfsetlinewidth{0.501875pt}%
\definecolor{currentstroke}{rgb}{0.000000,0.000000,0.000000}%
\pgfsetstrokecolor{currentstroke}%
\pgfsetdash{}{0pt}%
\pgfsys@defobject{currentmarker}{\pgfqpoint{-0.041667in}{0.000000in}}{\pgfqpoint{-0.000000in}{0.000000in}}{%
\pgfpathmoveto{\pgfqpoint{-0.000000in}{0.000000in}}%
\pgfpathlineto{\pgfqpoint{-0.041667in}{0.000000in}}%
\pgfusepath{stroke,fill}%
}%
\begin{pgfscope}%
\pgfsys@transformshift{5.674512in}{2.866434in}%
\pgfsys@useobject{currentmarker}{}%
\end{pgfscope}%
\end{pgfscope}%
\begin{pgfscope}%
\definecolor{textcolor}{rgb}{0.000000,0.000000,0.000000}%
\pgfsetstrokecolor{textcolor}%
\pgfsetfillcolor{textcolor}%
\pgftext[x=3.228741in, y=2.813673in, left, base]{\color{textcolor}\rmfamily\fontsize{10.000000}{12.000000}\selectfont \(\displaystyle {0.5}\)}%
\end{pgfscope}%
\begin{pgfscope}%
\pgfsetbuttcap%
\pgfsetroundjoin%
\definecolor{currentfill}{rgb}{0.000000,0.000000,0.000000}%
\pgfsetfillcolor{currentfill}%
\pgfsetlinewidth{0.501875pt}%
\definecolor{currentstroke}{rgb}{0.000000,0.000000,0.000000}%
\pgfsetstrokecolor{currentstroke}%
\pgfsetdash{}{0pt}%
\pgfsys@defobject{currentmarker}{\pgfqpoint{0.000000in}{0.000000in}}{\pgfqpoint{0.041667in}{0.000000in}}{%
\pgfpathmoveto{\pgfqpoint{0.000000in}{0.000000in}}%
\pgfpathlineto{\pgfqpoint{0.041667in}{0.000000in}}%
\pgfusepath{stroke,fill}%
}%
\begin{pgfscope}%
\pgfsys@transformshift{3.454822in}{3.404161in}%
\pgfsys@useobject{currentmarker}{}%
\end{pgfscope}%
\end{pgfscope}%
\begin{pgfscope}%
\pgfsetbuttcap%
\pgfsetroundjoin%
\definecolor{currentfill}{rgb}{0.000000,0.000000,0.000000}%
\pgfsetfillcolor{currentfill}%
\pgfsetlinewidth{0.501875pt}%
\definecolor{currentstroke}{rgb}{0.000000,0.000000,0.000000}%
\pgfsetstrokecolor{currentstroke}%
\pgfsetdash{}{0pt}%
\pgfsys@defobject{currentmarker}{\pgfqpoint{-0.041667in}{0.000000in}}{\pgfqpoint{-0.000000in}{0.000000in}}{%
\pgfpathmoveto{\pgfqpoint{-0.000000in}{0.000000in}}%
\pgfpathlineto{\pgfqpoint{-0.041667in}{0.000000in}}%
\pgfusepath{stroke,fill}%
}%
\begin{pgfscope}%
\pgfsys@transformshift{5.674512in}{3.404161in}%
\pgfsys@useobject{currentmarker}{}%
\end{pgfscope}%
\end{pgfscope}%
\begin{pgfscope}%
\definecolor{textcolor}{rgb}{0.000000,0.000000,0.000000}%
\pgfsetstrokecolor{textcolor}%
\pgfsetfillcolor{textcolor}%
\pgftext[x=3.228741in, y=3.351399in, left, base]{\color{textcolor}\rmfamily\fontsize{10.000000}{12.000000}\selectfont \(\displaystyle {0.6}\)}%
\end{pgfscope}%
\begin{pgfscope}%
\pgfsetbuttcap%
\pgfsetroundjoin%
\definecolor{currentfill}{rgb}{0.000000,0.000000,0.000000}%
\pgfsetfillcolor{currentfill}%
\pgfsetlinewidth{0.501875pt}%
\definecolor{currentstroke}{rgb}{0.000000,0.000000,0.000000}%
\pgfsetstrokecolor{currentstroke}%
\pgfsetdash{}{0pt}%
\pgfsys@defobject{currentmarker}{\pgfqpoint{0.000000in}{0.000000in}}{\pgfqpoint{0.020833in}{0.000000in}}{%
\pgfpathmoveto{\pgfqpoint{0.000000in}{0.000000in}}%
\pgfpathlineto{\pgfqpoint{0.020833in}{0.000000in}}%
\pgfusepath{stroke,fill}%
}%
\begin{pgfscope}%
\pgfsys@transformshift{3.454822in}{0.500438in}%
\pgfsys@useobject{currentmarker}{}%
\end{pgfscope}%
\end{pgfscope}%
\begin{pgfscope}%
\pgfsetbuttcap%
\pgfsetroundjoin%
\definecolor{currentfill}{rgb}{0.000000,0.000000,0.000000}%
\pgfsetfillcolor{currentfill}%
\pgfsetlinewidth{0.501875pt}%
\definecolor{currentstroke}{rgb}{0.000000,0.000000,0.000000}%
\pgfsetstrokecolor{currentstroke}%
\pgfsetdash{}{0pt}%
\pgfsys@defobject{currentmarker}{\pgfqpoint{-0.020833in}{0.000000in}}{\pgfqpoint{-0.000000in}{0.000000in}}{%
\pgfpathmoveto{\pgfqpoint{-0.000000in}{0.000000in}}%
\pgfpathlineto{\pgfqpoint{-0.020833in}{0.000000in}}%
\pgfusepath{stroke,fill}%
}%
\begin{pgfscope}%
\pgfsys@transformshift{5.674512in}{0.500438in}%
\pgfsys@useobject{currentmarker}{}%
\end{pgfscope}%
\end{pgfscope}%
\begin{pgfscope}%
\pgfsetbuttcap%
\pgfsetroundjoin%
\definecolor{currentfill}{rgb}{0.000000,0.000000,0.000000}%
\pgfsetfillcolor{currentfill}%
\pgfsetlinewidth{0.501875pt}%
\definecolor{currentstroke}{rgb}{0.000000,0.000000,0.000000}%
\pgfsetstrokecolor{currentstroke}%
\pgfsetdash{}{0pt}%
\pgfsys@defobject{currentmarker}{\pgfqpoint{0.000000in}{0.000000in}}{\pgfqpoint{0.020833in}{0.000000in}}{%
\pgfpathmoveto{\pgfqpoint{0.000000in}{0.000000in}}%
\pgfpathlineto{\pgfqpoint{0.020833in}{0.000000in}}%
\pgfusepath{stroke,fill}%
}%
\begin{pgfscope}%
\pgfsys@transformshift{3.454822in}{0.607984in}%
\pgfsys@useobject{currentmarker}{}%
\end{pgfscope}%
\end{pgfscope}%
\begin{pgfscope}%
\pgfsetbuttcap%
\pgfsetroundjoin%
\definecolor{currentfill}{rgb}{0.000000,0.000000,0.000000}%
\pgfsetfillcolor{currentfill}%
\pgfsetlinewidth{0.501875pt}%
\definecolor{currentstroke}{rgb}{0.000000,0.000000,0.000000}%
\pgfsetstrokecolor{currentstroke}%
\pgfsetdash{}{0pt}%
\pgfsys@defobject{currentmarker}{\pgfqpoint{-0.020833in}{0.000000in}}{\pgfqpoint{-0.000000in}{0.000000in}}{%
\pgfpathmoveto{\pgfqpoint{-0.000000in}{0.000000in}}%
\pgfpathlineto{\pgfqpoint{-0.020833in}{0.000000in}}%
\pgfusepath{stroke,fill}%
}%
\begin{pgfscope}%
\pgfsys@transformshift{5.674512in}{0.607984in}%
\pgfsys@useobject{currentmarker}{}%
\end{pgfscope}%
\end{pgfscope}%
\begin{pgfscope}%
\pgfsetbuttcap%
\pgfsetroundjoin%
\definecolor{currentfill}{rgb}{0.000000,0.000000,0.000000}%
\pgfsetfillcolor{currentfill}%
\pgfsetlinewidth{0.501875pt}%
\definecolor{currentstroke}{rgb}{0.000000,0.000000,0.000000}%
\pgfsetstrokecolor{currentstroke}%
\pgfsetdash{}{0pt}%
\pgfsys@defobject{currentmarker}{\pgfqpoint{0.000000in}{0.000000in}}{\pgfqpoint{0.020833in}{0.000000in}}{%
\pgfpathmoveto{\pgfqpoint{0.000000in}{0.000000in}}%
\pgfpathlineto{\pgfqpoint{0.020833in}{0.000000in}}%
\pgfusepath{stroke,fill}%
}%
\begin{pgfscope}%
\pgfsys@transformshift{3.454822in}{0.823074in}%
\pgfsys@useobject{currentmarker}{}%
\end{pgfscope}%
\end{pgfscope}%
\begin{pgfscope}%
\pgfsetbuttcap%
\pgfsetroundjoin%
\definecolor{currentfill}{rgb}{0.000000,0.000000,0.000000}%
\pgfsetfillcolor{currentfill}%
\pgfsetlinewidth{0.501875pt}%
\definecolor{currentstroke}{rgb}{0.000000,0.000000,0.000000}%
\pgfsetstrokecolor{currentstroke}%
\pgfsetdash{}{0pt}%
\pgfsys@defobject{currentmarker}{\pgfqpoint{-0.020833in}{0.000000in}}{\pgfqpoint{-0.000000in}{0.000000in}}{%
\pgfpathmoveto{\pgfqpoint{-0.000000in}{0.000000in}}%
\pgfpathlineto{\pgfqpoint{-0.020833in}{0.000000in}}%
\pgfusepath{stroke,fill}%
}%
\begin{pgfscope}%
\pgfsys@transformshift{5.674512in}{0.823074in}%
\pgfsys@useobject{currentmarker}{}%
\end{pgfscope}%
\end{pgfscope}%
\begin{pgfscope}%
\pgfsetbuttcap%
\pgfsetroundjoin%
\definecolor{currentfill}{rgb}{0.000000,0.000000,0.000000}%
\pgfsetfillcolor{currentfill}%
\pgfsetlinewidth{0.501875pt}%
\definecolor{currentstroke}{rgb}{0.000000,0.000000,0.000000}%
\pgfsetstrokecolor{currentstroke}%
\pgfsetdash{}{0pt}%
\pgfsys@defobject{currentmarker}{\pgfqpoint{0.000000in}{0.000000in}}{\pgfqpoint{0.020833in}{0.000000in}}{%
\pgfpathmoveto{\pgfqpoint{0.000000in}{0.000000in}}%
\pgfpathlineto{\pgfqpoint{0.020833in}{0.000000in}}%
\pgfusepath{stroke,fill}%
}%
\begin{pgfscope}%
\pgfsys@transformshift{3.454822in}{0.930620in}%
\pgfsys@useobject{currentmarker}{}%
\end{pgfscope}%
\end{pgfscope}%
\begin{pgfscope}%
\pgfsetbuttcap%
\pgfsetroundjoin%
\definecolor{currentfill}{rgb}{0.000000,0.000000,0.000000}%
\pgfsetfillcolor{currentfill}%
\pgfsetlinewidth{0.501875pt}%
\definecolor{currentstroke}{rgb}{0.000000,0.000000,0.000000}%
\pgfsetstrokecolor{currentstroke}%
\pgfsetdash{}{0pt}%
\pgfsys@defobject{currentmarker}{\pgfqpoint{-0.020833in}{0.000000in}}{\pgfqpoint{-0.000000in}{0.000000in}}{%
\pgfpathmoveto{\pgfqpoint{-0.000000in}{0.000000in}}%
\pgfpathlineto{\pgfqpoint{-0.020833in}{0.000000in}}%
\pgfusepath{stroke,fill}%
}%
\begin{pgfscope}%
\pgfsys@transformshift{5.674512in}{0.930620in}%
\pgfsys@useobject{currentmarker}{}%
\end{pgfscope}%
\end{pgfscope}%
\begin{pgfscope}%
\pgfsetbuttcap%
\pgfsetroundjoin%
\definecolor{currentfill}{rgb}{0.000000,0.000000,0.000000}%
\pgfsetfillcolor{currentfill}%
\pgfsetlinewidth{0.501875pt}%
\definecolor{currentstroke}{rgb}{0.000000,0.000000,0.000000}%
\pgfsetstrokecolor{currentstroke}%
\pgfsetdash{}{0pt}%
\pgfsys@defobject{currentmarker}{\pgfqpoint{0.000000in}{0.000000in}}{\pgfqpoint{0.020833in}{0.000000in}}{%
\pgfpathmoveto{\pgfqpoint{0.000000in}{0.000000in}}%
\pgfpathlineto{\pgfqpoint{0.020833in}{0.000000in}}%
\pgfusepath{stroke,fill}%
}%
\begin{pgfscope}%
\pgfsys@transformshift{3.454822in}{1.038165in}%
\pgfsys@useobject{currentmarker}{}%
\end{pgfscope}%
\end{pgfscope}%
\begin{pgfscope}%
\pgfsetbuttcap%
\pgfsetroundjoin%
\definecolor{currentfill}{rgb}{0.000000,0.000000,0.000000}%
\pgfsetfillcolor{currentfill}%
\pgfsetlinewidth{0.501875pt}%
\definecolor{currentstroke}{rgb}{0.000000,0.000000,0.000000}%
\pgfsetstrokecolor{currentstroke}%
\pgfsetdash{}{0pt}%
\pgfsys@defobject{currentmarker}{\pgfqpoint{-0.020833in}{0.000000in}}{\pgfqpoint{-0.000000in}{0.000000in}}{%
\pgfpathmoveto{\pgfqpoint{-0.000000in}{0.000000in}}%
\pgfpathlineto{\pgfqpoint{-0.020833in}{0.000000in}}%
\pgfusepath{stroke,fill}%
}%
\begin{pgfscope}%
\pgfsys@transformshift{5.674512in}{1.038165in}%
\pgfsys@useobject{currentmarker}{}%
\end{pgfscope}%
\end{pgfscope}%
\begin{pgfscope}%
\pgfsetbuttcap%
\pgfsetroundjoin%
\definecolor{currentfill}{rgb}{0.000000,0.000000,0.000000}%
\pgfsetfillcolor{currentfill}%
\pgfsetlinewidth{0.501875pt}%
\definecolor{currentstroke}{rgb}{0.000000,0.000000,0.000000}%
\pgfsetstrokecolor{currentstroke}%
\pgfsetdash{}{0pt}%
\pgfsys@defobject{currentmarker}{\pgfqpoint{0.000000in}{0.000000in}}{\pgfqpoint{0.020833in}{0.000000in}}{%
\pgfpathmoveto{\pgfqpoint{0.000000in}{0.000000in}}%
\pgfpathlineto{\pgfqpoint{0.020833in}{0.000000in}}%
\pgfusepath{stroke,fill}%
}%
\begin{pgfscope}%
\pgfsys@transformshift{3.454822in}{1.145710in}%
\pgfsys@useobject{currentmarker}{}%
\end{pgfscope}%
\end{pgfscope}%
\begin{pgfscope}%
\pgfsetbuttcap%
\pgfsetroundjoin%
\definecolor{currentfill}{rgb}{0.000000,0.000000,0.000000}%
\pgfsetfillcolor{currentfill}%
\pgfsetlinewidth{0.501875pt}%
\definecolor{currentstroke}{rgb}{0.000000,0.000000,0.000000}%
\pgfsetstrokecolor{currentstroke}%
\pgfsetdash{}{0pt}%
\pgfsys@defobject{currentmarker}{\pgfqpoint{-0.020833in}{0.000000in}}{\pgfqpoint{-0.000000in}{0.000000in}}{%
\pgfpathmoveto{\pgfqpoint{-0.000000in}{0.000000in}}%
\pgfpathlineto{\pgfqpoint{-0.020833in}{0.000000in}}%
\pgfusepath{stroke,fill}%
}%
\begin{pgfscope}%
\pgfsys@transformshift{5.674512in}{1.145710in}%
\pgfsys@useobject{currentmarker}{}%
\end{pgfscope}%
\end{pgfscope}%
\begin{pgfscope}%
\pgfsetbuttcap%
\pgfsetroundjoin%
\definecolor{currentfill}{rgb}{0.000000,0.000000,0.000000}%
\pgfsetfillcolor{currentfill}%
\pgfsetlinewidth{0.501875pt}%
\definecolor{currentstroke}{rgb}{0.000000,0.000000,0.000000}%
\pgfsetstrokecolor{currentstroke}%
\pgfsetdash{}{0pt}%
\pgfsys@defobject{currentmarker}{\pgfqpoint{0.000000in}{0.000000in}}{\pgfqpoint{0.020833in}{0.000000in}}{%
\pgfpathmoveto{\pgfqpoint{0.000000in}{0.000000in}}%
\pgfpathlineto{\pgfqpoint{0.020833in}{0.000000in}}%
\pgfusepath{stroke,fill}%
}%
\begin{pgfscope}%
\pgfsys@transformshift{3.454822in}{1.360801in}%
\pgfsys@useobject{currentmarker}{}%
\end{pgfscope}%
\end{pgfscope}%
\begin{pgfscope}%
\pgfsetbuttcap%
\pgfsetroundjoin%
\definecolor{currentfill}{rgb}{0.000000,0.000000,0.000000}%
\pgfsetfillcolor{currentfill}%
\pgfsetlinewidth{0.501875pt}%
\definecolor{currentstroke}{rgb}{0.000000,0.000000,0.000000}%
\pgfsetstrokecolor{currentstroke}%
\pgfsetdash{}{0pt}%
\pgfsys@defobject{currentmarker}{\pgfqpoint{-0.020833in}{0.000000in}}{\pgfqpoint{-0.000000in}{0.000000in}}{%
\pgfpathmoveto{\pgfqpoint{-0.000000in}{0.000000in}}%
\pgfpathlineto{\pgfqpoint{-0.020833in}{0.000000in}}%
\pgfusepath{stroke,fill}%
}%
\begin{pgfscope}%
\pgfsys@transformshift{5.674512in}{1.360801in}%
\pgfsys@useobject{currentmarker}{}%
\end{pgfscope}%
\end{pgfscope}%
\begin{pgfscope}%
\pgfsetbuttcap%
\pgfsetroundjoin%
\definecolor{currentfill}{rgb}{0.000000,0.000000,0.000000}%
\pgfsetfillcolor{currentfill}%
\pgfsetlinewidth{0.501875pt}%
\definecolor{currentstroke}{rgb}{0.000000,0.000000,0.000000}%
\pgfsetstrokecolor{currentstroke}%
\pgfsetdash{}{0pt}%
\pgfsys@defobject{currentmarker}{\pgfqpoint{0.000000in}{0.000000in}}{\pgfqpoint{0.020833in}{0.000000in}}{%
\pgfpathmoveto{\pgfqpoint{0.000000in}{0.000000in}}%
\pgfpathlineto{\pgfqpoint{0.020833in}{0.000000in}}%
\pgfusepath{stroke,fill}%
}%
\begin{pgfscope}%
\pgfsys@transformshift{3.454822in}{1.468346in}%
\pgfsys@useobject{currentmarker}{}%
\end{pgfscope}%
\end{pgfscope}%
\begin{pgfscope}%
\pgfsetbuttcap%
\pgfsetroundjoin%
\definecolor{currentfill}{rgb}{0.000000,0.000000,0.000000}%
\pgfsetfillcolor{currentfill}%
\pgfsetlinewidth{0.501875pt}%
\definecolor{currentstroke}{rgb}{0.000000,0.000000,0.000000}%
\pgfsetstrokecolor{currentstroke}%
\pgfsetdash{}{0pt}%
\pgfsys@defobject{currentmarker}{\pgfqpoint{-0.020833in}{0.000000in}}{\pgfqpoint{-0.000000in}{0.000000in}}{%
\pgfpathmoveto{\pgfqpoint{-0.000000in}{0.000000in}}%
\pgfpathlineto{\pgfqpoint{-0.020833in}{0.000000in}}%
\pgfusepath{stroke,fill}%
}%
\begin{pgfscope}%
\pgfsys@transformshift{5.674512in}{1.468346in}%
\pgfsys@useobject{currentmarker}{}%
\end{pgfscope}%
\end{pgfscope}%
\begin{pgfscope}%
\pgfsetbuttcap%
\pgfsetroundjoin%
\definecolor{currentfill}{rgb}{0.000000,0.000000,0.000000}%
\pgfsetfillcolor{currentfill}%
\pgfsetlinewidth{0.501875pt}%
\definecolor{currentstroke}{rgb}{0.000000,0.000000,0.000000}%
\pgfsetstrokecolor{currentstroke}%
\pgfsetdash{}{0pt}%
\pgfsys@defobject{currentmarker}{\pgfqpoint{0.000000in}{0.000000in}}{\pgfqpoint{0.020833in}{0.000000in}}{%
\pgfpathmoveto{\pgfqpoint{0.000000in}{0.000000in}}%
\pgfpathlineto{\pgfqpoint{0.020833in}{0.000000in}}%
\pgfusepath{stroke,fill}%
}%
\begin{pgfscope}%
\pgfsys@transformshift{3.454822in}{1.575891in}%
\pgfsys@useobject{currentmarker}{}%
\end{pgfscope}%
\end{pgfscope}%
\begin{pgfscope}%
\pgfsetbuttcap%
\pgfsetroundjoin%
\definecolor{currentfill}{rgb}{0.000000,0.000000,0.000000}%
\pgfsetfillcolor{currentfill}%
\pgfsetlinewidth{0.501875pt}%
\definecolor{currentstroke}{rgb}{0.000000,0.000000,0.000000}%
\pgfsetstrokecolor{currentstroke}%
\pgfsetdash{}{0pt}%
\pgfsys@defobject{currentmarker}{\pgfqpoint{-0.020833in}{0.000000in}}{\pgfqpoint{-0.000000in}{0.000000in}}{%
\pgfpathmoveto{\pgfqpoint{-0.000000in}{0.000000in}}%
\pgfpathlineto{\pgfqpoint{-0.020833in}{0.000000in}}%
\pgfusepath{stroke,fill}%
}%
\begin{pgfscope}%
\pgfsys@transformshift{5.674512in}{1.575891in}%
\pgfsys@useobject{currentmarker}{}%
\end{pgfscope}%
\end{pgfscope}%
\begin{pgfscope}%
\pgfsetbuttcap%
\pgfsetroundjoin%
\definecolor{currentfill}{rgb}{0.000000,0.000000,0.000000}%
\pgfsetfillcolor{currentfill}%
\pgfsetlinewidth{0.501875pt}%
\definecolor{currentstroke}{rgb}{0.000000,0.000000,0.000000}%
\pgfsetstrokecolor{currentstroke}%
\pgfsetdash{}{0pt}%
\pgfsys@defobject{currentmarker}{\pgfqpoint{0.000000in}{0.000000in}}{\pgfqpoint{0.020833in}{0.000000in}}{%
\pgfpathmoveto{\pgfqpoint{0.000000in}{0.000000in}}%
\pgfpathlineto{\pgfqpoint{0.020833in}{0.000000in}}%
\pgfusepath{stroke,fill}%
}%
\begin{pgfscope}%
\pgfsys@transformshift{3.454822in}{1.683436in}%
\pgfsys@useobject{currentmarker}{}%
\end{pgfscope}%
\end{pgfscope}%
\begin{pgfscope}%
\pgfsetbuttcap%
\pgfsetroundjoin%
\definecolor{currentfill}{rgb}{0.000000,0.000000,0.000000}%
\pgfsetfillcolor{currentfill}%
\pgfsetlinewidth{0.501875pt}%
\definecolor{currentstroke}{rgb}{0.000000,0.000000,0.000000}%
\pgfsetstrokecolor{currentstroke}%
\pgfsetdash{}{0pt}%
\pgfsys@defobject{currentmarker}{\pgfqpoint{-0.020833in}{0.000000in}}{\pgfqpoint{-0.000000in}{0.000000in}}{%
\pgfpathmoveto{\pgfqpoint{-0.000000in}{0.000000in}}%
\pgfpathlineto{\pgfqpoint{-0.020833in}{0.000000in}}%
\pgfusepath{stroke,fill}%
}%
\begin{pgfscope}%
\pgfsys@transformshift{5.674512in}{1.683436in}%
\pgfsys@useobject{currentmarker}{}%
\end{pgfscope}%
\end{pgfscope}%
\begin{pgfscope}%
\pgfsetbuttcap%
\pgfsetroundjoin%
\definecolor{currentfill}{rgb}{0.000000,0.000000,0.000000}%
\pgfsetfillcolor{currentfill}%
\pgfsetlinewidth{0.501875pt}%
\definecolor{currentstroke}{rgb}{0.000000,0.000000,0.000000}%
\pgfsetstrokecolor{currentstroke}%
\pgfsetdash{}{0pt}%
\pgfsys@defobject{currentmarker}{\pgfqpoint{0.000000in}{0.000000in}}{\pgfqpoint{0.020833in}{0.000000in}}{%
\pgfpathmoveto{\pgfqpoint{0.000000in}{0.000000in}}%
\pgfpathlineto{\pgfqpoint{0.020833in}{0.000000in}}%
\pgfusepath{stroke,fill}%
}%
\begin{pgfscope}%
\pgfsys@transformshift{3.454822in}{1.898527in}%
\pgfsys@useobject{currentmarker}{}%
\end{pgfscope}%
\end{pgfscope}%
\begin{pgfscope}%
\pgfsetbuttcap%
\pgfsetroundjoin%
\definecolor{currentfill}{rgb}{0.000000,0.000000,0.000000}%
\pgfsetfillcolor{currentfill}%
\pgfsetlinewidth{0.501875pt}%
\definecolor{currentstroke}{rgb}{0.000000,0.000000,0.000000}%
\pgfsetstrokecolor{currentstroke}%
\pgfsetdash{}{0pt}%
\pgfsys@defobject{currentmarker}{\pgfqpoint{-0.020833in}{0.000000in}}{\pgfqpoint{-0.000000in}{0.000000in}}{%
\pgfpathmoveto{\pgfqpoint{-0.000000in}{0.000000in}}%
\pgfpathlineto{\pgfqpoint{-0.020833in}{0.000000in}}%
\pgfusepath{stroke,fill}%
}%
\begin{pgfscope}%
\pgfsys@transformshift{5.674512in}{1.898527in}%
\pgfsys@useobject{currentmarker}{}%
\end{pgfscope}%
\end{pgfscope}%
\begin{pgfscope}%
\pgfsetbuttcap%
\pgfsetroundjoin%
\definecolor{currentfill}{rgb}{0.000000,0.000000,0.000000}%
\pgfsetfillcolor{currentfill}%
\pgfsetlinewidth{0.501875pt}%
\definecolor{currentstroke}{rgb}{0.000000,0.000000,0.000000}%
\pgfsetstrokecolor{currentstroke}%
\pgfsetdash{}{0pt}%
\pgfsys@defobject{currentmarker}{\pgfqpoint{0.000000in}{0.000000in}}{\pgfqpoint{0.020833in}{0.000000in}}{%
\pgfpathmoveto{\pgfqpoint{0.000000in}{0.000000in}}%
\pgfpathlineto{\pgfqpoint{0.020833in}{0.000000in}}%
\pgfusepath{stroke,fill}%
}%
\begin{pgfscope}%
\pgfsys@transformshift{3.454822in}{2.006072in}%
\pgfsys@useobject{currentmarker}{}%
\end{pgfscope}%
\end{pgfscope}%
\begin{pgfscope}%
\pgfsetbuttcap%
\pgfsetroundjoin%
\definecolor{currentfill}{rgb}{0.000000,0.000000,0.000000}%
\pgfsetfillcolor{currentfill}%
\pgfsetlinewidth{0.501875pt}%
\definecolor{currentstroke}{rgb}{0.000000,0.000000,0.000000}%
\pgfsetstrokecolor{currentstroke}%
\pgfsetdash{}{0pt}%
\pgfsys@defobject{currentmarker}{\pgfqpoint{-0.020833in}{0.000000in}}{\pgfqpoint{-0.000000in}{0.000000in}}{%
\pgfpathmoveto{\pgfqpoint{-0.000000in}{0.000000in}}%
\pgfpathlineto{\pgfqpoint{-0.020833in}{0.000000in}}%
\pgfusepath{stroke,fill}%
}%
\begin{pgfscope}%
\pgfsys@transformshift{5.674512in}{2.006072in}%
\pgfsys@useobject{currentmarker}{}%
\end{pgfscope}%
\end{pgfscope}%
\begin{pgfscope}%
\pgfsetbuttcap%
\pgfsetroundjoin%
\definecolor{currentfill}{rgb}{0.000000,0.000000,0.000000}%
\pgfsetfillcolor{currentfill}%
\pgfsetlinewidth{0.501875pt}%
\definecolor{currentstroke}{rgb}{0.000000,0.000000,0.000000}%
\pgfsetstrokecolor{currentstroke}%
\pgfsetdash{}{0pt}%
\pgfsys@defobject{currentmarker}{\pgfqpoint{0.000000in}{0.000000in}}{\pgfqpoint{0.020833in}{0.000000in}}{%
\pgfpathmoveto{\pgfqpoint{0.000000in}{0.000000in}}%
\pgfpathlineto{\pgfqpoint{0.020833in}{0.000000in}}%
\pgfusepath{stroke,fill}%
}%
\begin{pgfscope}%
\pgfsys@transformshift{3.454822in}{2.113617in}%
\pgfsys@useobject{currentmarker}{}%
\end{pgfscope}%
\end{pgfscope}%
\begin{pgfscope}%
\pgfsetbuttcap%
\pgfsetroundjoin%
\definecolor{currentfill}{rgb}{0.000000,0.000000,0.000000}%
\pgfsetfillcolor{currentfill}%
\pgfsetlinewidth{0.501875pt}%
\definecolor{currentstroke}{rgb}{0.000000,0.000000,0.000000}%
\pgfsetstrokecolor{currentstroke}%
\pgfsetdash{}{0pt}%
\pgfsys@defobject{currentmarker}{\pgfqpoint{-0.020833in}{0.000000in}}{\pgfqpoint{-0.000000in}{0.000000in}}{%
\pgfpathmoveto{\pgfqpoint{-0.000000in}{0.000000in}}%
\pgfpathlineto{\pgfqpoint{-0.020833in}{0.000000in}}%
\pgfusepath{stroke,fill}%
}%
\begin{pgfscope}%
\pgfsys@transformshift{5.674512in}{2.113617in}%
\pgfsys@useobject{currentmarker}{}%
\end{pgfscope}%
\end{pgfscope}%
\begin{pgfscope}%
\pgfsetbuttcap%
\pgfsetroundjoin%
\definecolor{currentfill}{rgb}{0.000000,0.000000,0.000000}%
\pgfsetfillcolor{currentfill}%
\pgfsetlinewidth{0.501875pt}%
\definecolor{currentstroke}{rgb}{0.000000,0.000000,0.000000}%
\pgfsetstrokecolor{currentstroke}%
\pgfsetdash{}{0pt}%
\pgfsys@defobject{currentmarker}{\pgfqpoint{0.000000in}{0.000000in}}{\pgfqpoint{0.020833in}{0.000000in}}{%
\pgfpathmoveto{\pgfqpoint{0.000000in}{0.000000in}}%
\pgfpathlineto{\pgfqpoint{0.020833in}{0.000000in}}%
\pgfusepath{stroke,fill}%
}%
\begin{pgfscope}%
\pgfsys@transformshift{3.454822in}{2.221163in}%
\pgfsys@useobject{currentmarker}{}%
\end{pgfscope}%
\end{pgfscope}%
\begin{pgfscope}%
\pgfsetbuttcap%
\pgfsetroundjoin%
\definecolor{currentfill}{rgb}{0.000000,0.000000,0.000000}%
\pgfsetfillcolor{currentfill}%
\pgfsetlinewidth{0.501875pt}%
\definecolor{currentstroke}{rgb}{0.000000,0.000000,0.000000}%
\pgfsetstrokecolor{currentstroke}%
\pgfsetdash{}{0pt}%
\pgfsys@defobject{currentmarker}{\pgfqpoint{-0.020833in}{0.000000in}}{\pgfqpoint{-0.000000in}{0.000000in}}{%
\pgfpathmoveto{\pgfqpoint{-0.000000in}{0.000000in}}%
\pgfpathlineto{\pgfqpoint{-0.020833in}{0.000000in}}%
\pgfusepath{stroke,fill}%
}%
\begin{pgfscope}%
\pgfsys@transformshift{5.674512in}{2.221163in}%
\pgfsys@useobject{currentmarker}{}%
\end{pgfscope}%
\end{pgfscope}%
\begin{pgfscope}%
\pgfsetbuttcap%
\pgfsetroundjoin%
\definecolor{currentfill}{rgb}{0.000000,0.000000,0.000000}%
\pgfsetfillcolor{currentfill}%
\pgfsetlinewidth{0.501875pt}%
\definecolor{currentstroke}{rgb}{0.000000,0.000000,0.000000}%
\pgfsetstrokecolor{currentstroke}%
\pgfsetdash{}{0pt}%
\pgfsys@defobject{currentmarker}{\pgfqpoint{0.000000in}{0.000000in}}{\pgfqpoint{0.020833in}{0.000000in}}{%
\pgfpathmoveto{\pgfqpoint{0.000000in}{0.000000in}}%
\pgfpathlineto{\pgfqpoint{0.020833in}{0.000000in}}%
\pgfusepath{stroke,fill}%
}%
\begin{pgfscope}%
\pgfsys@transformshift{3.454822in}{2.436253in}%
\pgfsys@useobject{currentmarker}{}%
\end{pgfscope}%
\end{pgfscope}%
\begin{pgfscope}%
\pgfsetbuttcap%
\pgfsetroundjoin%
\definecolor{currentfill}{rgb}{0.000000,0.000000,0.000000}%
\pgfsetfillcolor{currentfill}%
\pgfsetlinewidth{0.501875pt}%
\definecolor{currentstroke}{rgb}{0.000000,0.000000,0.000000}%
\pgfsetstrokecolor{currentstroke}%
\pgfsetdash{}{0pt}%
\pgfsys@defobject{currentmarker}{\pgfqpoint{-0.020833in}{0.000000in}}{\pgfqpoint{-0.000000in}{0.000000in}}{%
\pgfpathmoveto{\pgfqpoint{-0.000000in}{0.000000in}}%
\pgfpathlineto{\pgfqpoint{-0.020833in}{0.000000in}}%
\pgfusepath{stroke,fill}%
}%
\begin{pgfscope}%
\pgfsys@transformshift{5.674512in}{2.436253in}%
\pgfsys@useobject{currentmarker}{}%
\end{pgfscope}%
\end{pgfscope}%
\begin{pgfscope}%
\pgfsetbuttcap%
\pgfsetroundjoin%
\definecolor{currentfill}{rgb}{0.000000,0.000000,0.000000}%
\pgfsetfillcolor{currentfill}%
\pgfsetlinewidth{0.501875pt}%
\definecolor{currentstroke}{rgb}{0.000000,0.000000,0.000000}%
\pgfsetstrokecolor{currentstroke}%
\pgfsetdash{}{0pt}%
\pgfsys@defobject{currentmarker}{\pgfqpoint{0.000000in}{0.000000in}}{\pgfqpoint{0.020833in}{0.000000in}}{%
\pgfpathmoveto{\pgfqpoint{0.000000in}{0.000000in}}%
\pgfpathlineto{\pgfqpoint{0.020833in}{0.000000in}}%
\pgfusepath{stroke,fill}%
}%
\begin{pgfscope}%
\pgfsys@transformshift{3.454822in}{2.543798in}%
\pgfsys@useobject{currentmarker}{}%
\end{pgfscope}%
\end{pgfscope}%
\begin{pgfscope}%
\pgfsetbuttcap%
\pgfsetroundjoin%
\definecolor{currentfill}{rgb}{0.000000,0.000000,0.000000}%
\pgfsetfillcolor{currentfill}%
\pgfsetlinewidth{0.501875pt}%
\definecolor{currentstroke}{rgb}{0.000000,0.000000,0.000000}%
\pgfsetstrokecolor{currentstroke}%
\pgfsetdash{}{0pt}%
\pgfsys@defobject{currentmarker}{\pgfqpoint{-0.020833in}{0.000000in}}{\pgfqpoint{-0.000000in}{0.000000in}}{%
\pgfpathmoveto{\pgfqpoint{-0.000000in}{0.000000in}}%
\pgfpathlineto{\pgfqpoint{-0.020833in}{0.000000in}}%
\pgfusepath{stroke,fill}%
}%
\begin{pgfscope}%
\pgfsys@transformshift{5.674512in}{2.543798in}%
\pgfsys@useobject{currentmarker}{}%
\end{pgfscope}%
\end{pgfscope}%
\begin{pgfscope}%
\pgfsetbuttcap%
\pgfsetroundjoin%
\definecolor{currentfill}{rgb}{0.000000,0.000000,0.000000}%
\pgfsetfillcolor{currentfill}%
\pgfsetlinewidth{0.501875pt}%
\definecolor{currentstroke}{rgb}{0.000000,0.000000,0.000000}%
\pgfsetstrokecolor{currentstroke}%
\pgfsetdash{}{0pt}%
\pgfsys@defobject{currentmarker}{\pgfqpoint{0.000000in}{0.000000in}}{\pgfqpoint{0.020833in}{0.000000in}}{%
\pgfpathmoveto{\pgfqpoint{0.000000in}{0.000000in}}%
\pgfpathlineto{\pgfqpoint{0.020833in}{0.000000in}}%
\pgfusepath{stroke,fill}%
}%
\begin{pgfscope}%
\pgfsys@transformshift{3.454822in}{2.651344in}%
\pgfsys@useobject{currentmarker}{}%
\end{pgfscope}%
\end{pgfscope}%
\begin{pgfscope}%
\pgfsetbuttcap%
\pgfsetroundjoin%
\definecolor{currentfill}{rgb}{0.000000,0.000000,0.000000}%
\pgfsetfillcolor{currentfill}%
\pgfsetlinewidth{0.501875pt}%
\definecolor{currentstroke}{rgb}{0.000000,0.000000,0.000000}%
\pgfsetstrokecolor{currentstroke}%
\pgfsetdash{}{0pt}%
\pgfsys@defobject{currentmarker}{\pgfqpoint{-0.020833in}{0.000000in}}{\pgfqpoint{-0.000000in}{0.000000in}}{%
\pgfpathmoveto{\pgfqpoint{-0.000000in}{0.000000in}}%
\pgfpathlineto{\pgfqpoint{-0.020833in}{0.000000in}}%
\pgfusepath{stroke,fill}%
}%
\begin{pgfscope}%
\pgfsys@transformshift{5.674512in}{2.651344in}%
\pgfsys@useobject{currentmarker}{}%
\end{pgfscope}%
\end{pgfscope}%
\begin{pgfscope}%
\pgfsetbuttcap%
\pgfsetroundjoin%
\definecolor{currentfill}{rgb}{0.000000,0.000000,0.000000}%
\pgfsetfillcolor{currentfill}%
\pgfsetlinewidth{0.501875pt}%
\definecolor{currentstroke}{rgb}{0.000000,0.000000,0.000000}%
\pgfsetstrokecolor{currentstroke}%
\pgfsetdash{}{0pt}%
\pgfsys@defobject{currentmarker}{\pgfqpoint{0.000000in}{0.000000in}}{\pgfqpoint{0.020833in}{0.000000in}}{%
\pgfpathmoveto{\pgfqpoint{0.000000in}{0.000000in}}%
\pgfpathlineto{\pgfqpoint{0.020833in}{0.000000in}}%
\pgfusepath{stroke,fill}%
}%
\begin{pgfscope}%
\pgfsys@transformshift{3.454822in}{2.758889in}%
\pgfsys@useobject{currentmarker}{}%
\end{pgfscope}%
\end{pgfscope}%
\begin{pgfscope}%
\pgfsetbuttcap%
\pgfsetroundjoin%
\definecolor{currentfill}{rgb}{0.000000,0.000000,0.000000}%
\pgfsetfillcolor{currentfill}%
\pgfsetlinewidth{0.501875pt}%
\definecolor{currentstroke}{rgb}{0.000000,0.000000,0.000000}%
\pgfsetstrokecolor{currentstroke}%
\pgfsetdash{}{0pt}%
\pgfsys@defobject{currentmarker}{\pgfqpoint{-0.020833in}{0.000000in}}{\pgfqpoint{-0.000000in}{0.000000in}}{%
\pgfpathmoveto{\pgfqpoint{-0.000000in}{0.000000in}}%
\pgfpathlineto{\pgfqpoint{-0.020833in}{0.000000in}}%
\pgfusepath{stroke,fill}%
}%
\begin{pgfscope}%
\pgfsys@transformshift{5.674512in}{2.758889in}%
\pgfsys@useobject{currentmarker}{}%
\end{pgfscope}%
\end{pgfscope}%
\begin{pgfscope}%
\pgfsetbuttcap%
\pgfsetroundjoin%
\definecolor{currentfill}{rgb}{0.000000,0.000000,0.000000}%
\pgfsetfillcolor{currentfill}%
\pgfsetlinewidth{0.501875pt}%
\definecolor{currentstroke}{rgb}{0.000000,0.000000,0.000000}%
\pgfsetstrokecolor{currentstroke}%
\pgfsetdash{}{0pt}%
\pgfsys@defobject{currentmarker}{\pgfqpoint{0.000000in}{0.000000in}}{\pgfqpoint{0.020833in}{0.000000in}}{%
\pgfpathmoveto{\pgfqpoint{0.000000in}{0.000000in}}%
\pgfpathlineto{\pgfqpoint{0.020833in}{0.000000in}}%
\pgfusepath{stroke,fill}%
}%
\begin{pgfscope}%
\pgfsys@transformshift{3.454822in}{2.973980in}%
\pgfsys@useobject{currentmarker}{}%
\end{pgfscope}%
\end{pgfscope}%
\begin{pgfscope}%
\pgfsetbuttcap%
\pgfsetroundjoin%
\definecolor{currentfill}{rgb}{0.000000,0.000000,0.000000}%
\pgfsetfillcolor{currentfill}%
\pgfsetlinewidth{0.501875pt}%
\definecolor{currentstroke}{rgb}{0.000000,0.000000,0.000000}%
\pgfsetstrokecolor{currentstroke}%
\pgfsetdash{}{0pt}%
\pgfsys@defobject{currentmarker}{\pgfqpoint{-0.020833in}{0.000000in}}{\pgfqpoint{-0.000000in}{0.000000in}}{%
\pgfpathmoveto{\pgfqpoint{-0.000000in}{0.000000in}}%
\pgfpathlineto{\pgfqpoint{-0.020833in}{0.000000in}}%
\pgfusepath{stroke,fill}%
}%
\begin{pgfscope}%
\pgfsys@transformshift{5.674512in}{2.973980in}%
\pgfsys@useobject{currentmarker}{}%
\end{pgfscope}%
\end{pgfscope}%
\begin{pgfscope}%
\pgfsetbuttcap%
\pgfsetroundjoin%
\definecolor{currentfill}{rgb}{0.000000,0.000000,0.000000}%
\pgfsetfillcolor{currentfill}%
\pgfsetlinewidth{0.501875pt}%
\definecolor{currentstroke}{rgb}{0.000000,0.000000,0.000000}%
\pgfsetstrokecolor{currentstroke}%
\pgfsetdash{}{0pt}%
\pgfsys@defobject{currentmarker}{\pgfqpoint{0.000000in}{0.000000in}}{\pgfqpoint{0.020833in}{0.000000in}}{%
\pgfpathmoveto{\pgfqpoint{0.000000in}{0.000000in}}%
\pgfpathlineto{\pgfqpoint{0.020833in}{0.000000in}}%
\pgfusepath{stroke,fill}%
}%
\begin{pgfscope}%
\pgfsys@transformshift{3.454822in}{3.081525in}%
\pgfsys@useobject{currentmarker}{}%
\end{pgfscope}%
\end{pgfscope}%
\begin{pgfscope}%
\pgfsetbuttcap%
\pgfsetroundjoin%
\definecolor{currentfill}{rgb}{0.000000,0.000000,0.000000}%
\pgfsetfillcolor{currentfill}%
\pgfsetlinewidth{0.501875pt}%
\definecolor{currentstroke}{rgb}{0.000000,0.000000,0.000000}%
\pgfsetstrokecolor{currentstroke}%
\pgfsetdash{}{0pt}%
\pgfsys@defobject{currentmarker}{\pgfqpoint{-0.020833in}{0.000000in}}{\pgfqpoint{-0.000000in}{0.000000in}}{%
\pgfpathmoveto{\pgfqpoint{-0.000000in}{0.000000in}}%
\pgfpathlineto{\pgfqpoint{-0.020833in}{0.000000in}}%
\pgfusepath{stroke,fill}%
}%
\begin{pgfscope}%
\pgfsys@transformshift{5.674512in}{3.081525in}%
\pgfsys@useobject{currentmarker}{}%
\end{pgfscope}%
\end{pgfscope}%
\begin{pgfscope}%
\pgfsetbuttcap%
\pgfsetroundjoin%
\definecolor{currentfill}{rgb}{0.000000,0.000000,0.000000}%
\pgfsetfillcolor{currentfill}%
\pgfsetlinewidth{0.501875pt}%
\definecolor{currentstroke}{rgb}{0.000000,0.000000,0.000000}%
\pgfsetstrokecolor{currentstroke}%
\pgfsetdash{}{0pt}%
\pgfsys@defobject{currentmarker}{\pgfqpoint{0.000000in}{0.000000in}}{\pgfqpoint{0.020833in}{0.000000in}}{%
\pgfpathmoveto{\pgfqpoint{0.000000in}{0.000000in}}%
\pgfpathlineto{\pgfqpoint{0.020833in}{0.000000in}}%
\pgfusepath{stroke,fill}%
}%
\begin{pgfscope}%
\pgfsys@transformshift{3.454822in}{3.189070in}%
\pgfsys@useobject{currentmarker}{}%
\end{pgfscope}%
\end{pgfscope}%
\begin{pgfscope}%
\pgfsetbuttcap%
\pgfsetroundjoin%
\definecolor{currentfill}{rgb}{0.000000,0.000000,0.000000}%
\pgfsetfillcolor{currentfill}%
\pgfsetlinewidth{0.501875pt}%
\definecolor{currentstroke}{rgb}{0.000000,0.000000,0.000000}%
\pgfsetstrokecolor{currentstroke}%
\pgfsetdash{}{0pt}%
\pgfsys@defobject{currentmarker}{\pgfqpoint{-0.020833in}{0.000000in}}{\pgfqpoint{-0.000000in}{0.000000in}}{%
\pgfpathmoveto{\pgfqpoint{-0.000000in}{0.000000in}}%
\pgfpathlineto{\pgfqpoint{-0.020833in}{0.000000in}}%
\pgfusepath{stroke,fill}%
}%
\begin{pgfscope}%
\pgfsys@transformshift{5.674512in}{3.189070in}%
\pgfsys@useobject{currentmarker}{}%
\end{pgfscope}%
\end{pgfscope}%
\begin{pgfscope}%
\pgfsetbuttcap%
\pgfsetroundjoin%
\definecolor{currentfill}{rgb}{0.000000,0.000000,0.000000}%
\pgfsetfillcolor{currentfill}%
\pgfsetlinewidth{0.501875pt}%
\definecolor{currentstroke}{rgb}{0.000000,0.000000,0.000000}%
\pgfsetstrokecolor{currentstroke}%
\pgfsetdash{}{0pt}%
\pgfsys@defobject{currentmarker}{\pgfqpoint{0.000000in}{0.000000in}}{\pgfqpoint{0.020833in}{0.000000in}}{%
\pgfpathmoveto{\pgfqpoint{0.000000in}{0.000000in}}%
\pgfpathlineto{\pgfqpoint{0.020833in}{0.000000in}}%
\pgfusepath{stroke,fill}%
}%
\begin{pgfscope}%
\pgfsys@transformshift{3.454822in}{3.296615in}%
\pgfsys@useobject{currentmarker}{}%
\end{pgfscope}%
\end{pgfscope}%
\begin{pgfscope}%
\pgfsetbuttcap%
\pgfsetroundjoin%
\definecolor{currentfill}{rgb}{0.000000,0.000000,0.000000}%
\pgfsetfillcolor{currentfill}%
\pgfsetlinewidth{0.501875pt}%
\definecolor{currentstroke}{rgb}{0.000000,0.000000,0.000000}%
\pgfsetstrokecolor{currentstroke}%
\pgfsetdash{}{0pt}%
\pgfsys@defobject{currentmarker}{\pgfqpoint{-0.020833in}{0.000000in}}{\pgfqpoint{-0.000000in}{0.000000in}}{%
\pgfpathmoveto{\pgfqpoint{-0.000000in}{0.000000in}}%
\pgfpathlineto{\pgfqpoint{-0.020833in}{0.000000in}}%
\pgfusepath{stroke,fill}%
}%
\begin{pgfscope}%
\pgfsys@transformshift{5.674512in}{3.296615in}%
\pgfsys@useobject{currentmarker}{}%
\end{pgfscope}%
\end{pgfscope}%
\begin{pgfscope}%
\pgfsetbuttcap%
\pgfsetroundjoin%
\definecolor{currentfill}{rgb}{0.000000,0.000000,0.000000}%
\pgfsetfillcolor{currentfill}%
\pgfsetlinewidth{0.501875pt}%
\definecolor{currentstroke}{rgb}{0.000000,0.000000,0.000000}%
\pgfsetstrokecolor{currentstroke}%
\pgfsetdash{}{0pt}%
\pgfsys@defobject{currentmarker}{\pgfqpoint{0.000000in}{0.000000in}}{\pgfqpoint{0.020833in}{0.000000in}}{%
\pgfpathmoveto{\pgfqpoint{0.000000in}{0.000000in}}%
\pgfpathlineto{\pgfqpoint{0.020833in}{0.000000in}}%
\pgfusepath{stroke,fill}%
}%
\begin{pgfscope}%
\pgfsys@transformshift{3.454822in}{3.511706in}%
\pgfsys@useobject{currentmarker}{}%
\end{pgfscope}%
\end{pgfscope}%
\begin{pgfscope}%
\pgfsetbuttcap%
\pgfsetroundjoin%
\definecolor{currentfill}{rgb}{0.000000,0.000000,0.000000}%
\pgfsetfillcolor{currentfill}%
\pgfsetlinewidth{0.501875pt}%
\definecolor{currentstroke}{rgb}{0.000000,0.000000,0.000000}%
\pgfsetstrokecolor{currentstroke}%
\pgfsetdash{}{0pt}%
\pgfsys@defobject{currentmarker}{\pgfqpoint{-0.020833in}{0.000000in}}{\pgfqpoint{-0.000000in}{0.000000in}}{%
\pgfpathmoveto{\pgfqpoint{-0.000000in}{0.000000in}}%
\pgfpathlineto{\pgfqpoint{-0.020833in}{0.000000in}}%
\pgfusepath{stroke,fill}%
}%
\begin{pgfscope}%
\pgfsys@transformshift{5.674512in}{3.511706in}%
\pgfsys@useobject{currentmarker}{}%
\end{pgfscope}%
\end{pgfscope}%
\begin{pgfscope}%
\definecolor{textcolor}{rgb}{0.000000,0.000000,0.000000}%
\pgfsetstrokecolor{textcolor}%
\pgfsetfillcolor{textcolor}%
\pgftext[x=3.173186in,y=1.998593in,,bottom,rotate=90.000000]{\color{textcolor}\rmfamily\fontsize{10.000000}{12.000000}\selectfont LCMC\(\displaystyle (K)\)}%
\end{pgfscope}%
\begin{pgfscope}%
\pgfpathrectangle{\pgfqpoint{3.454822in}{0.422992in}}{\pgfqpoint{2.219690in}{3.151201in}}%
\pgfusepath{clip}%
\pgfsetrectcap%
\pgfsetroundjoin%
\pgfsetlinewidth{1.003750pt}%
\definecolor{currentstroke}{rgb}{0.047059,0.364706,0.647059}%
\pgfsetstrokecolor{currentstroke}%
\pgfsetdash{}{0pt}%
\pgfpathmoveto{\pgfqpoint{3.476584in}{2.010662in}}%
\pgfpathlineto{\pgfqpoint{3.498345in}{2.086700in}}%
\pgfpathlineto{\pgfqpoint{3.520107in}{2.117638in}}%
\pgfpathlineto{\pgfqpoint{3.541869in}{2.153918in}}%
\pgfpathlineto{\pgfqpoint{3.563630in}{2.199347in}}%
\pgfpathlineto{\pgfqpoint{3.585392in}{2.233362in}}%
\pgfpathlineto{\pgfqpoint{3.607154in}{2.265278in}}%
\pgfpathlineto{\pgfqpoint{3.628915in}{2.288597in}}%
\pgfpathlineto{\pgfqpoint{3.650677in}{2.312064in}}%
\pgfpathlineto{\pgfqpoint{3.672439in}{2.333999in}}%
\pgfpathlineto{\pgfqpoint{3.694200in}{2.358966in}}%
\pgfpathlineto{\pgfqpoint{3.715962in}{2.378930in}}%
\pgfpathlineto{\pgfqpoint{3.737724in}{2.396898in}}%
\pgfpathlineto{\pgfqpoint{3.759485in}{2.412621in}}%
\pgfpathlineto{\pgfqpoint{3.781247in}{2.429116in}}%
\pgfpathlineto{\pgfqpoint{3.803009in}{2.443079in}}%
\pgfpathlineto{\pgfqpoint{3.824770in}{2.457372in}}%
\pgfpathlineto{\pgfqpoint{3.846532in}{2.470592in}}%
\pgfpathlineto{\pgfqpoint{3.868294in}{2.484243in}}%
\pgfpathlineto{\pgfqpoint{3.890055in}{2.494173in}}%
\pgfpathlineto{\pgfqpoint{3.911817in}{2.506087in}}%
\pgfpathlineto{\pgfqpoint{3.933579in}{2.518169in}}%
\pgfpathlineto{\pgfqpoint{3.955340in}{2.528144in}}%
\pgfpathlineto{\pgfqpoint{3.977102in}{2.539062in}}%
\pgfpathlineto{\pgfqpoint{3.998864in}{2.548806in}}%
\pgfpathlineto{\pgfqpoint{4.020625in}{2.557709in}}%
\pgfpathlineto{\pgfqpoint{4.042387in}{2.566239in}}%
\pgfpathlineto{\pgfqpoint{4.064149in}{2.574836in}}%
\pgfpathlineto{\pgfqpoint{4.085910in}{2.583412in}}%
\pgfpathlineto{\pgfqpoint{4.107672in}{2.592498in}}%
\pgfpathlineto{\pgfqpoint{4.129434in}{2.600880in}}%
\pgfpathlineto{\pgfqpoint{4.151195in}{2.608456in}}%
\pgfpathlineto{\pgfqpoint{4.172957in}{2.615306in}}%
\pgfpathlineto{\pgfqpoint{4.194719in}{2.623138in}}%
\pgfpathlineto{\pgfqpoint{4.216480in}{2.630830in}}%
\pgfpathlineto{\pgfqpoint{4.238242in}{2.638232in}}%
\pgfpathlineto{\pgfqpoint{4.260004in}{2.645768in}}%
\pgfpathlineto{\pgfqpoint{4.281765in}{2.654550in}}%
\pgfpathlineto{\pgfqpoint{4.303527in}{2.662407in}}%
\pgfpathlineto{\pgfqpoint{4.325289in}{2.669107in}}%
\pgfpathlineto{\pgfqpoint{4.347050in}{2.675229in}}%
\pgfpathlineto{\pgfqpoint{4.368812in}{2.682565in}}%
\pgfpathlineto{\pgfqpoint{4.390574in}{2.689105in}}%
\pgfpathlineto{\pgfqpoint{4.412335in}{2.695982in}}%
\pgfpathlineto{\pgfqpoint{4.434097in}{2.702469in}}%
\pgfpathlineto{\pgfqpoint{4.455859in}{2.708331in}}%
\pgfpathlineto{\pgfqpoint{4.477620in}{2.714160in}}%
\pgfpathlineto{\pgfqpoint{4.499382in}{2.720261in}}%
\pgfpathlineto{\pgfqpoint{4.521144in}{2.725897in}}%
\pgfpathlineto{\pgfqpoint{4.542905in}{2.731098in}}%
\pgfpathlineto{\pgfqpoint{4.564667in}{2.735972in}}%
\pgfpathlineto{\pgfqpoint{4.586429in}{2.740291in}}%
\pgfpathlineto{\pgfqpoint{4.608190in}{2.744913in}}%
\pgfpathlineto{\pgfqpoint{4.629952in}{2.749424in}}%
\pgfpathlineto{\pgfqpoint{4.651714in}{2.754400in}}%
\pgfpathlineto{\pgfqpoint{4.673475in}{2.758742in}}%
\pgfpathlineto{\pgfqpoint{4.695237in}{2.763569in}}%
\pgfpathlineto{\pgfqpoint{4.716999in}{2.768983in}}%
\pgfpathlineto{\pgfqpoint{4.738760in}{2.773655in}}%
\pgfpathlineto{\pgfqpoint{4.760522in}{2.778562in}}%
\pgfpathlineto{\pgfqpoint{4.782284in}{2.783030in}}%
\pgfpathlineto{\pgfqpoint{4.804045in}{2.788599in}}%
\pgfpathlineto{\pgfqpoint{4.825807in}{2.793042in}}%
\pgfpathlineto{\pgfqpoint{4.847569in}{2.797373in}}%
\pgfpathlineto{\pgfqpoint{4.869330in}{2.801674in}}%
\pgfpathlineto{\pgfqpoint{4.891092in}{2.805955in}}%
\pgfpathlineto{\pgfqpoint{4.912854in}{2.809880in}}%
\pgfpathlineto{\pgfqpoint{4.934615in}{2.814142in}}%
\pgfpathlineto{\pgfqpoint{4.956377in}{2.818375in}}%
\pgfpathlineto{\pgfqpoint{4.978139in}{2.822209in}}%
\pgfpathlineto{\pgfqpoint{4.999900in}{2.826384in}}%
\pgfpathlineto{\pgfqpoint{5.021662in}{2.830829in}}%
\pgfpathlineto{\pgfqpoint{5.043424in}{2.834992in}}%
\pgfpathlineto{\pgfqpoint{5.065185in}{2.839043in}}%
\pgfpathlineto{\pgfqpoint{5.086947in}{2.843437in}}%
\pgfpathlineto{\pgfqpoint{5.108709in}{2.847089in}}%
\pgfpathlineto{\pgfqpoint{5.130470in}{2.850445in}}%
\pgfpathlineto{\pgfqpoint{5.152232in}{2.854063in}}%
\pgfpathlineto{\pgfqpoint{5.173994in}{2.858139in}}%
\pgfpathlineto{\pgfqpoint{5.195755in}{2.861806in}}%
\pgfpathlineto{\pgfqpoint{5.217517in}{2.865678in}}%
\pgfpathlineto{\pgfqpoint{5.239279in}{2.868902in}}%
\pgfpathlineto{\pgfqpoint{5.261040in}{2.872473in}}%
\pgfpathlineto{\pgfqpoint{5.282802in}{2.875813in}}%
\pgfpathlineto{\pgfqpoint{5.304564in}{2.879763in}}%
\pgfpathlineto{\pgfqpoint{5.326325in}{2.883189in}}%
\pgfpathlineto{\pgfqpoint{5.348087in}{2.886152in}}%
\pgfpathlineto{\pgfqpoint{5.369849in}{2.889134in}}%
\pgfpathlineto{\pgfqpoint{5.391610in}{2.892355in}}%
\pgfpathlineto{\pgfqpoint{5.413372in}{2.895981in}}%
\pgfpathlineto{\pgfqpoint{5.435134in}{2.899132in}}%
\pgfpathlineto{\pgfqpoint{5.456895in}{2.902472in}}%
\pgfpathlineto{\pgfqpoint{5.478657in}{2.905159in}}%
\pgfpathlineto{\pgfqpoint{5.500419in}{2.908096in}}%
\pgfpathlineto{\pgfqpoint{5.522180in}{2.910980in}}%
\pgfpathlineto{\pgfqpoint{5.543942in}{2.914230in}}%
\pgfpathlineto{\pgfqpoint{5.565704in}{2.917072in}}%
\pgfpathlineto{\pgfqpoint{5.587465in}{2.919602in}}%
\pgfpathlineto{\pgfqpoint{5.609227in}{2.922449in}}%
\pgfpathlineto{\pgfqpoint{5.630989in}{2.925278in}}%
\pgfusepath{stroke}%
\end{pgfscope}%
\begin{pgfscope}%
\pgfpathrectangle{\pgfqpoint{3.454822in}{0.422992in}}{\pgfqpoint{2.219690in}{3.151201in}}%
\pgfusepath{clip}%
\pgfsetrectcap%
\pgfsetroundjoin%
\pgfsetlinewidth{1.003750pt}%
\definecolor{currentstroke}{rgb}{0.000000,0.725490,0.270588}%
\pgfsetstrokecolor{currentstroke}%
\pgfsetdash{}{0pt}%
\pgfpathmoveto{\pgfqpoint{3.476584in}{1.885735in}}%
\pgfpathlineto{\pgfqpoint{3.498345in}{1.950625in}}%
\pgfpathlineto{\pgfqpoint{3.520107in}{2.028182in}}%
\pgfpathlineto{\pgfqpoint{3.541869in}{2.075834in}}%
\pgfpathlineto{\pgfqpoint{3.563630in}{2.111810in}}%
\pgfpathlineto{\pgfqpoint{3.585392in}{2.144698in}}%
\pgfpathlineto{\pgfqpoint{3.607154in}{2.166703in}}%
\pgfpathlineto{\pgfqpoint{3.628915in}{2.189212in}}%
\pgfpathlineto{\pgfqpoint{3.650677in}{2.215762in}}%
\pgfpathlineto{\pgfqpoint{3.672439in}{2.234026in}}%
\pgfpathlineto{\pgfqpoint{3.694200in}{2.249621in}}%
\pgfpathlineto{\pgfqpoint{3.715962in}{2.262258in}}%
\pgfpathlineto{\pgfqpoint{3.737724in}{2.276729in}}%
\pgfpathlineto{\pgfqpoint{3.759485in}{2.292411in}}%
\pgfpathlineto{\pgfqpoint{3.781247in}{2.308057in}}%
\pgfpathlineto{\pgfqpoint{3.803009in}{2.323339in}}%
\pgfpathlineto{\pgfqpoint{3.824770in}{2.335156in}}%
\pgfpathlineto{\pgfqpoint{3.846532in}{2.345800in}}%
\pgfpathlineto{\pgfqpoint{3.868294in}{2.357192in}}%
\pgfpathlineto{\pgfqpoint{3.890055in}{2.364487in}}%
\pgfpathlineto{\pgfqpoint{3.911817in}{2.372094in}}%
\pgfpathlineto{\pgfqpoint{3.933579in}{2.378276in}}%
\pgfpathlineto{\pgfqpoint{3.955340in}{2.385760in}}%
\pgfpathlineto{\pgfqpoint{3.977102in}{2.394981in}}%
\pgfpathlineto{\pgfqpoint{3.998864in}{2.402847in}}%
\pgfpathlineto{\pgfqpoint{4.020625in}{2.409901in}}%
\pgfpathlineto{\pgfqpoint{4.042387in}{2.417601in}}%
\pgfpathlineto{\pgfqpoint{4.064149in}{2.424239in}}%
\pgfpathlineto{\pgfqpoint{4.085910in}{2.428813in}}%
\pgfpathlineto{\pgfqpoint{4.107672in}{2.435698in}}%
\pgfpathlineto{\pgfqpoint{4.129434in}{2.438253in}}%
\pgfpathlineto{\pgfqpoint{4.151195in}{2.441131in}}%
\pgfpathlineto{\pgfqpoint{4.172957in}{2.446409in}}%
\pgfpathlineto{\pgfqpoint{4.194719in}{2.451566in}}%
\pgfpathlineto{\pgfqpoint{4.216480in}{2.455957in}}%
\pgfpathlineto{\pgfqpoint{4.238242in}{2.458710in}}%
\pgfpathlineto{\pgfqpoint{4.260004in}{2.462332in}}%
\pgfpathlineto{\pgfqpoint{4.281765in}{2.466536in}}%
\pgfpathlineto{\pgfqpoint{4.303527in}{2.471527in}}%
\pgfpathlineto{\pgfqpoint{4.325289in}{2.476000in}}%
\pgfpathlineto{\pgfqpoint{4.347050in}{2.479607in}}%
\pgfpathlineto{\pgfqpoint{4.368812in}{2.483897in}}%
\pgfpathlineto{\pgfqpoint{4.390574in}{2.488245in}}%
\pgfpathlineto{\pgfqpoint{4.412335in}{2.492615in}}%
\pgfpathlineto{\pgfqpoint{4.434097in}{2.495541in}}%
\pgfpathlineto{\pgfqpoint{4.455859in}{2.498986in}}%
\pgfpathlineto{\pgfqpoint{4.477620in}{2.501865in}}%
\pgfpathlineto{\pgfqpoint{4.499382in}{2.504639in}}%
\pgfpathlineto{\pgfqpoint{4.521144in}{2.507212in}}%
\pgfpathlineto{\pgfqpoint{4.542905in}{2.509904in}}%
\pgfpathlineto{\pgfqpoint{4.564667in}{2.512413in}}%
\pgfpathlineto{\pgfqpoint{4.586429in}{2.515185in}}%
\pgfpathlineto{\pgfqpoint{4.608190in}{2.517730in}}%
\pgfpathlineto{\pgfqpoint{4.629952in}{2.520672in}}%
\pgfpathlineto{\pgfqpoint{4.651714in}{2.522236in}}%
\pgfpathlineto{\pgfqpoint{4.673475in}{2.523738in}}%
\pgfpathlineto{\pgfqpoint{4.695237in}{2.526356in}}%
\pgfpathlineto{\pgfqpoint{4.716999in}{2.528990in}}%
\pgfpathlineto{\pgfqpoint{4.738760in}{2.531067in}}%
\pgfpathlineto{\pgfqpoint{4.760522in}{2.533540in}}%
\pgfpathlineto{\pgfqpoint{4.782284in}{2.535615in}}%
\pgfpathlineto{\pgfqpoint{4.804045in}{2.538219in}}%
\pgfpathlineto{\pgfqpoint{4.825807in}{2.541087in}}%
\pgfpathlineto{\pgfqpoint{4.847569in}{2.543053in}}%
\pgfpathlineto{\pgfqpoint{4.869330in}{2.545240in}}%
\pgfpathlineto{\pgfqpoint{4.891092in}{2.547203in}}%
\pgfpathlineto{\pgfqpoint{4.912854in}{2.549038in}}%
\pgfpathlineto{\pgfqpoint{4.934615in}{2.552121in}}%
\pgfpathlineto{\pgfqpoint{4.956377in}{2.554258in}}%
\pgfpathlineto{\pgfqpoint{4.978139in}{2.555498in}}%
\pgfpathlineto{\pgfqpoint{4.999900in}{2.557123in}}%
\pgfpathlineto{\pgfqpoint{5.021662in}{2.559918in}}%
\pgfpathlineto{\pgfqpoint{5.043424in}{2.561934in}}%
\pgfpathlineto{\pgfqpoint{5.065185in}{2.563265in}}%
\pgfpathlineto{\pgfqpoint{5.086947in}{2.564394in}}%
\pgfpathlineto{\pgfqpoint{5.108709in}{2.566300in}}%
\pgfpathlineto{\pgfqpoint{5.130470in}{2.567504in}}%
\pgfpathlineto{\pgfqpoint{5.152232in}{2.569252in}}%
\pgfpathlineto{\pgfqpoint{5.173994in}{2.570806in}}%
\pgfpathlineto{\pgfqpoint{5.195755in}{2.572554in}}%
\pgfpathlineto{\pgfqpoint{5.217517in}{2.573936in}}%
\pgfpathlineto{\pgfqpoint{5.239279in}{2.574956in}}%
\pgfpathlineto{\pgfqpoint{5.261040in}{2.576492in}}%
\pgfpathlineto{\pgfqpoint{5.282802in}{2.578388in}}%
\pgfpathlineto{\pgfqpoint{5.304564in}{2.580518in}}%
\pgfpathlineto{\pgfqpoint{5.326325in}{2.582936in}}%
\pgfpathlineto{\pgfqpoint{5.348087in}{2.584626in}}%
\pgfpathlineto{\pgfqpoint{5.369849in}{2.585843in}}%
\pgfpathlineto{\pgfqpoint{5.391610in}{2.587442in}}%
\pgfpathlineto{\pgfqpoint{5.413372in}{2.589174in}}%
\pgfpathlineto{\pgfqpoint{5.435134in}{2.590048in}}%
\pgfpathlineto{\pgfqpoint{5.456895in}{2.591156in}}%
\pgfpathlineto{\pgfqpoint{5.478657in}{2.592688in}}%
\pgfpathlineto{\pgfqpoint{5.500419in}{2.593802in}}%
\pgfpathlineto{\pgfqpoint{5.522180in}{2.594515in}}%
\pgfpathlineto{\pgfqpoint{5.543942in}{2.595878in}}%
\pgfpathlineto{\pgfqpoint{5.565704in}{2.597741in}}%
\pgfpathlineto{\pgfqpoint{5.587465in}{2.598846in}}%
\pgfpathlineto{\pgfqpoint{5.609227in}{2.599555in}}%
\pgfpathlineto{\pgfqpoint{5.630989in}{2.601337in}}%
\pgfusepath{stroke}%
\end{pgfscope}%
\begin{pgfscope}%
\pgfpathrectangle{\pgfqpoint{3.454822in}{0.422992in}}{\pgfqpoint{2.219690in}{3.151201in}}%
\pgfusepath{clip}%
\pgfsetrectcap%
\pgfsetroundjoin%
\pgfsetlinewidth{1.003750pt}%
\definecolor{currentstroke}{rgb}{1.000000,0.584314,0.000000}%
\pgfsetstrokecolor{currentstroke}%
\pgfsetdash{}{0pt}%
\pgfpathmoveto{\pgfqpoint{3.476584in}{1.246587in}}%
\pgfpathlineto{\pgfqpoint{3.498345in}{1.127071in}}%
\pgfpathlineto{\pgfqpoint{3.520107in}{1.076657in}}%
\pgfpathlineto{\pgfqpoint{3.541869in}{1.063415in}}%
\pgfpathlineto{\pgfqpoint{3.563630in}{1.068645in}}%
\pgfpathlineto{\pgfqpoint{3.585392in}{1.086874in}}%
\pgfpathlineto{\pgfqpoint{3.607154in}{1.101163in}}%
\pgfpathlineto{\pgfqpoint{3.628915in}{1.120484in}}%
\pgfpathlineto{\pgfqpoint{3.650677in}{1.137005in}}%
\pgfpathlineto{\pgfqpoint{3.672439in}{1.154631in}}%
\pgfpathlineto{\pgfqpoint{3.694200in}{1.170275in}}%
\pgfpathlineto{\pgfqpoint{3.715962in}{1.184051in}}%
\pgfpathlineto{\pgfqpoint{3.737724in}{1.198375in}}%
\pgfpathlineto{\pgfqpoint{3.759485in}{1.214475in}}%
\pgfpathlineto{\pgfqpoint{3.781247in}{1.226242in}}%
\pgfpathlineto{\pgfqpoint{3.803009in}{1.240857in}}%
\pgfpathlineto{\pgfqpoint{3.824770in}{1.252249in}}%
\pgfpathlineto{\pgfqpoint{3.846532in}{1.263929in}}%
\pgfpathlineto{\pgfqpoint{3.868294in}{1.274975in}}%
\pgfpathlineto{\pgfqpoint{3.890055in}{1.284861in}}%
\pgfpathlineto{\pgfqpoint{3.911817in}{1.295215in}}%
\pgfpathlineto{\pgfqpoint{3.933579in}{1.307072in}}%
\pgfpathlineto{\pgfqpoint{3.955340in}{1.317301in}}%
\pgfpathlineto{\pgfqpoint{3.977102in}{1.326454in}}%
\pgfpathlineto{\pgfqpoint{3.998864in}{1.335617in}}%
\pgfpathlineto{\pgfqpoint{4.020625in}{1.344768in}}%
\pgfpathlineto{\pgfqpoint{4.042387in}{1.354307in}}%
\pgfpathlineto{\pgfqpoint{4.064149in}{1.361935in}}%
\pgfpathlineto{\pgfqpoint{4.085910in}{1.369936in}}%
\pgfpathlineto{\pgfqpoint{4.107672in}{1.376893in}}%
\pgfpathlineto{\pgfqpoint{4.129434in}{1.385527in}}%
\pgfpathlineto{\pgfqpoint{4.151195in}{1.393133in}}%
\pgfpathlineto{\pgfqpoint{4.172957in}{1.400115in}}%
\pgfpathlineto{\pgfqpoint{4.194719in}{1.407209in}}%
\pgfpathlineto{\pgfqpoint{4.216480in}{1.414005in}}%
\pgfpathlineto{\pgfqpoint{4.238242in}{1.420415in}}%
\pgfpathlineto{\pgfqpoint{4.260004in}{1.428020in}}%
\pgfpathlineto{\pgfqpoint{4.281765in}{1.435189in}}%
\pgfpathlineto{\pgfqpoint{4.303527in}{1.441536in}}%
\pgfpathlineto{\pgfqpoint{4.325289in}{1.447726in}}%
\pgfpathlineto{\pgfqpoint{4.347050in}{1.452946in}}%
\pgfpathlineto{\pgfqpoint{4.368812in}{1.458692in}}%
\pgfpathlineto{\pgfqpoint{4.390574in}{1.464333in}}%
\pgfpathlineto{\pgfqpoint{4.412335in}{1.469821in}}%
\pgfpathlineto{\pgfqpoint{4.434097in}{1.475460in}}%
\pgfpathlineto{\pgfqpoint{4.455859in}{1.481608in}}%
\pgfpathlineto{\pgfqpoint{4.477620in}{1.486510in}}%
\pgfpathlineto{\pgfqpoint{4.499382in}{1.492239in}}%
\pgfpathlineto{\pgfqpoint{4.521144in}{1.497454in}}%
\pgfpathlineto{\pgfqpoint{4.542905in}{1.501707in}}%
\pgfpathlineto{\pgfqpoint{4.564667in}{1.506348in}}%
\pgfpathlineto{\pgfqpoint{4.586429in}{1.511404in}}%
\pgfpathlineto{\pgfqpoint{4.608190in}{1.517097in}}%
\pgfpathlineto{\pgfqpoint{4.629952in}{1.522011in}}%
\pgfpathlineto{\pgfqpoint{4.651714in}{1.526257in}}%
\pgfpathlineto{\pgfqpoint{4.673475in}{1.530616in}}%
\pgfpathlineto{\pgfqpoint{4.695237in}{1.534714in}}%
\pgfpathlineto{\pgfqpoint{4.716999in}{1.539602in}}%
\pgfpathlineto{\pgfqpoint{4.738760in}{1.543927in}}%
\pgfpathlineto{\pgfqpoint{4.760522in}{1.547504in}}%
\pgfpathlineto{\pgfqpoint{4.782284in}{1.551523in}}%
\pgfpathlineto{\pgfqpoint{4.804045in}{1.554827in}}%
\pgfpathlineto{\pgfqpoint{4.825807in}{1.559900in}}%
\pgfpathlineto{\pgfqpoint{4.847569in}{1.564016in}}%
\pgfpathlineto{\pgfqpoint{4.869330in}{1.567890in}}%
\pgfpathlineto{\pgfqpoint{4.891092in}{1.572135in}}%
\pgfpathlineto{\pgfqpoint{4.912854in}{1.576542in}}%
\pgfpathlineto{\pgfqpoint{4.934615in}{1.580531in}}%
\pgfpathlineto{\pgfqpoint{4.956377in}{1.584759in}}%
\pgfpathlineto{\pgfqpoint{4.978139in}{1.588847in}}%
\pgfpathlineto{\pgfqpoint{4.999900in}{1.593115in}}%
\pgfpathlineto{\pgfqpoint{5.021662in}{1.596734in}}%
\pgfpathlineto{\pgfqpoint{5.043424in}{1.600641in}}%
\pgfpathlineto{\pgfqpoint{5.065185in}{1.604239in}}%
\pgfpathlineto{\pgfqpoint{5.086947in}{1.607838in}}%
\pgfpathlineto{\pgfqpoint{5.108709in}{1.611395in}}%
\pgfpathlineto{\pgfqpoint{5.130470in}{1.613980in}}%
\pgfpathlineto{\pgfqpoint{5.152232in}{1.617463in}}%
\pgfpathlineto{\pgfqpoint{5.173994in}{1.620716in}}%
\pgfpathlineto{\pgfqpoint{5.195755in}{1.623423in}}%
\pgfpathlineto{\pgfqpoint{5.217517in}{1.626160in}}%
\pgfpathlineto{\pgfqpoint{5.239279in}{1.629466in}}%
\pgfpathlineto{\pgfqpoint{5.261040in}{1.631789in}}%
\pgfpathlineto{\pgfqpoint{5.282802in}{1.634946in}}%
\pgfpathlineto{\pgfqpoint{5.304564in}{1.638330in}}%
\pgfpathlineto{\pgfqpoint{5.326325in}{1.641500in}}%
\pgfpathlineto{\pgfqpoint{5.348087in}{1.644211in}}%
\pgfpathlineto{\pgfqpoint{5.369849in}{1.647102in}}%
\pgfpathlineto{\pgfqpoint{5.391610in}{1.649865in}}%
\pgfpathlineto{\pgfqpoint{5.413372in}{1.652820in}}%
\pgfpathlineto{\pgfqpoint{5.435134in}{1.655157in}}%
\pgfpathlineto{\pgfqpoint{5.456895in}{1.657608in}}%
\pgfpathlineto{\pgfqpoint{5.478657in}{1.660245in}}%
\pgfpathlineto{\pgfqpoint{5.500419in}{1.663076in}}%
\pgfpathlineto{\pgfqpoint{5.522180in}{1.665552in}}%
\pgfpathlineto{\pgfqpoint{5.543942in}{1.667823in}}%
\pgfpathlineto{\pgfqpoint{5.565704in}{1.670615in}}%
\pgfpathlineto{\pgfqpoint{5.587465in}{1.672821in}}%
\pgfpathlineto{\pgfqpoint{5.609227in}{1.675162in}}%
\pgfpathlineto{\pgfqpoint{5.630989in}{1.677329in}}%
\pgfusepath{stroke}%
\end{pgfscope}%
\begin{pgfscope}%
\pgfpathrectangle{\pgfqpoint{3.454822in}{0.422992in}}{\pgfqpoint{2.219690in}{3.151201in}}%
\pgfusepath{clip}%
\pgfsetrectcap%
\pgfsetroundjoin%
\pgfsetlinewidth{1.003750pt}%
\definecolor{currentstroke}{rgb}{1.000000,0.172549,0.000000}%
\pgfsetstrokecolor{currentstroke}%
\pgfsetdash{}{0pt}%
\pgfpathmoveto{\pgfqpoint{3.476584in}{2.093906in}}%
\pgfpathlineto{\pgfqpoint{3.498345in}{2.256413in}}%
\pgfpathlineto{\pgfqpoint{3.520107in}{2.314741in}}%
\pgfpathlineto{\pgfqpoint{3.541869in}{2.360414in}}%
\pgfpathlineto{\pgfqpoint{3.563630in}{2.405584in}}%
\pgfpathlineto{\pgfqpoint{3.585392in}{2.439534in}}%
\pgfpathlineto{\pgfqpoint{3.607154in}{2.464306in}}%
\pgfpathlineto{\pgfqpoint{3.628915in}{2.489123in}}%
\pgfpathlineto{\pgfqpoint{3.650677in}{2.507470in}}%
\pgfpathlineto{\pgfqpoint{3.672439in}{2.524491in}}%
\pgfpathlineto{\pgfqpoint{3.694200in}{2.540471in}}%
\pgfpathlineto{\pgfqpoint{3.715962in}{2.555383in}}%
\pgfpathlineto{\pgfqpoint{3.737724in}{2.567438in}}%
\pgfpathlineto{\pgfqpoint{3.759485in}{2.580998in}}%
\pgfpathlineto{\pgfqpoint{3.781247in}{2.591430in}}%
\pgfpathlineto{\pgfqpoint{3.803009in}{2.599321in}}%
\pgfpathlineto{\pgfqpoint{3.824770in}{2.607512in}}%
\pgfpathlineto{\pgfqpoint{3.846532in}{2.615115in}}%
\pgfpathlineto{\pgfqpoint{3.868294in}{2.623932in}}%
\pgfpathlineto{\pgfqpoint{3.890055in}{2.632912in}}%
\pgfpathlineto{\pgfqpoint{3.911817in}{2.643351in}}%
\pgfpathlineto{\pgfqpoint{3.933579in}{2.650914in}}%
\pgfpathlineto{\pgfqpoint{3.955340in}{2.657157in}}%
\pgfpathlineto{\pgfqpoint{3.977102in}{2.664223in}}%
\pgfpathlineto{\pgfqpoint{3.998864in}{2.670397in}}%
\pgfpathlineto{\pgfqpoint{4.020625in}{2.675203in}}%
\pgfpathlineto{\pgfqpoint{4.042387in}{2.681971in}}%
\pgfpathlineto{\pgfqpoint{4.064149in}{2.688970in}}%
\pgfpathlineto{\pgfqpoint{4.085910in}{2.694514in}}%
\pgfpathlineto{\pgfqpoint{4.107672in}{2.700277in}}%
\pgfpathlineto{\pgfqpoint{4.129434in}{2.705869in}}%
\pgfpathlineto{\pgfqpoint{4.151195in}{2.710749in}}%
\pgfpathlineto{\pgfqpoint{4.172957in}{2.716911in}}%
\pgfpathlineto{\pgfqpoint{4.194719in}{2.721805in}}%
\pgfpathlineto{\pgfqpoint{4.216480in}{2.726475in}}%
\pgfpathlineto{\pgfqpoint{4.238242in}{2.732033in}}%
\pgfpathlineto{\pgfqpoint{4.260004in}{2.737116in}}%
\pgfpathlineto{\pgfqpoint{4.281765in}{2.741127in}}%
\pgfpathlineto{\pgfqpoint{4.303527in}{2.744889in}}%
\pgfpathlineto{\pgfqpoint{4.325289in}{2.748296in}}%
\pgfpathlineto{\pgfqpoint{4.347050in}{2.752817in}}%
\pgfpathlineto{\pgfqpoint{4.368812in}{2.757087in}}%
\pgfpathlineto{\pgfqpoint{4.390574in}{2.760798in}}%
\pgfpathlineto{\pgfqpoint{4.412335in}{2.764281in}}%
\pgfpathlineto{\pgfqpoint{4.434097in}{2.767825in}}%
\pgfpathlineto{\pgfqpoint{4.455859in}{2.771482in}}%
\pgfpathlineto{\pgfqpoint{4.477620in}{2.775294in}}%
\pgfpathlineto{\pgfqpoint{4.499382in}{2.779431in}}%
\pgfpathlineto{\pgfqpoint{4.521144in}{2.783566in}}%
\pgfpathlineto{\pgfqpoint{4.542905in}{2.786245in}}%
\pgfpathlineto{\pgfqpoint{4.564667in}{2.789430in}}%
\pgfpathlineto{\pgfqpoint{4.586429in}{2.792613in}}%
\pgfpathlineto{\pgfqpoint{4.608190in}{2.795059in}}%
\pgfpathlineto{\pgfqpoint{4.629952in}{2.797996in}}%
\pgfpathlineto{\pgfqpoint{4.651714in}{2.800975in}}%
\pgfpathlineto{\pgfqpoint{4.673475in}{2.804366in}}%
\pgfpathlineto{\pgfqpoint{4.695237in}{2.807646in}}%
\pgfpathlineto{\pgfqpoint{4.716999in}{2.810282in}}%
\pgfpathlineto{\pgfqpoint{4.738760in}{2.813474in}}%
\pgfpathlineto{\pgfqpoint{4.760522in}{2.815968in}}%
\pgfpathlineto{\pgfqpoint{4.782284in}{2.819050in}}%
\pgfpathlineto{\pgfqpoint{4.804045in}{2.821377in}}%
\pgfpathlineto{\pgfqpoint{4.825807in}{2.824723in}}%
\pgfpathlineto{\pgfqpoint{4.847569in}{2.827460in}}%
\pgfpathlineto{\pgfqpoint{4.869330in}{2.830153in}}%
\pgfpathlineto{\pgfqpoint{4.891092in}{2.833299in}}%
\pgfpathlineto{\pgfqpoint{4.912854in}{2.836443in}}%
\pgfpathlineto{\pgfqpoint{4.934615in}{2.838290in}}%
\pgfpathlineto{\pgfqpoint{4.956377in}{2.840963in}}%
\pgfpathlineto{\pgfqpoint{4.978139in}{2.843673in}}%
\pgfpathlineto{\pgfqpoint{4.999900in}{2.845931in}}%
\pgfpathlineto{\pgfqpoint{5.021662in}{2.848500in}}%
\pgfpathlineto{\pgfqpoint{5.043424in}{2.851234in}}%
\pgfpathlineto{\pgfqpoint{5.065185in}{2.853525in}}%
\pgfpathlineto{\pgfqpoint{5.086947in}{2.855565in}}%
\pgfpathlineto{\pgfqpoint{5.108709in}{2.857722in}}%
\pgfpathlineto{\pgfqpoint{5.130470in}{2.859798in}}%
\pgfpathlineto{\pgfqpoint{5.152232in}{2.862443in}}%
\pgfpathlineto{\pgfqpoint{5.173994in}{2.865120in}}%
\pgfpathlineto{\pgfqpoint{5.195755in}{2.867450in}}%
\pgfpathlineto{\pgfqpoint{5.217517in}{2.869614in}}%
\pgfpathlineto{\pgfqpoint{5.239279in}{2.871546in}}%
\pgfpathlineto{\pgfqpoint{5.261040in}{2.874093in}}%
\pgfpathlineto{\pgfqpoint{5.282802in}{2.876187in}}%
\pgfpathlineto{\pgfqpoint{5.304564in}{2.877898in}}%
\pgfpathlineto{\pgfqpoint{5.326325in}{2.880225in}}%
\pgfpathlineto{\pgfqpoint{5.348087in}{2.881937in}}%
\pgfpathlineto{\pgfqpoint{5.369849in}{2.883974in}}%
\pgfpathlineto{\pgfqpoint{5.391610in}{2.885279in}}%
\pgfpathlineto{\pgfqpoint{5.413372in}{2.886999in}}%
\pgfpathlineto{\pgfqpoint{5.435134in}{2.888795in}}%
\pgfpathlineto{\pgfqpoint{5.456895in}{2.890284in}}%
\pgfpathlineto{\pgfqpoint{5.478657in}{2.892096in}}%
\pgfpathlineto{\pgfqpoint{5.500419in}{2.893879in}}%
\pgfpathlineto{\pgfqpoint{5.522180in}{2.895389in}}%
\pgfpathlineto{\pgfqpoint{5.543942in}{2.896930in}}%
\pgfpathlineto{\pgfqpoint{5.565704in}{2.899110in}}%
\pgfpathlineto{\pgfqpoint{5.587465in}{2.901143in}}%
\pgfpathlineto{\pgfqpoint{5.609227in}{2.903067in}}%
\pgfpathlineto{\pgfqpoint{5.630989in}{2.905121in}}%
\pgfusepath{stroke}%
\end{pgfscope}%
\begin{pgfscope}%
\pgfpathrectangle{\pgfqpoint{3.454822in}{0.422992in}}{\pgfqpoint{2.219690in}{3.151201in}}%
\pgfusepath{clip}%
\pgfsetrectcap%
\pgfsetroundjoin%
\pgfsetlinewidth{1.003750pt}%
\definecolor{currentstroke}{rgb}{0.517647,0.356863,0.592157}%
\pgfsetstrokecolor{currentstroke}%
\pgfsetdash{}{0pt}%
\pgfpathmoveto{\pgfqpoint{3.476584in}{0.698773in}}%
\pgfpathlineto{\pgfqpoint{3.498345in}{0.643062in}}%
\pgfpathlineto{\pgfqpoint{3.520107in}{0.591582in}}%
\pgfpathlineto{\pgfqpoint{3.541869in}{0.568799in}}%
\pgfpathlineto{\pgfqpoint{3.563630in}{0.566229in}}%
\pgfpathlineto{\pgfqpoint{3.585392in}{0.569175in}}%
\pgfpathlineto{\pgfqpoint{3.607154in}{0.577150in}}%
\pgfpathlineto{\pgfqpoint{3.628915in}{0.583964in}}%
\pgfpathlineto{\pgfqpoint{3.650677in}{0.591271in}}%
\pgfpathlineto{\pgfqpoint{3.672439in}{0.597848in}}%
\pgfpathlineto{\pgfqpoint{3.694200in}{0.607180in}}%
\pgfpathlineto{\pgfqpoint{3.715962in}{0.615726in}}%
\pgfpathlineto{\pgfqpoint{3.737724in}{0.623670in}}%
\pgfpathlineto{\pgfqpoint{3.759485in}{0.631447in}}%
\pgfpathlineto{\pgfqpoint{3.781247in}{0.640094in}}%
\pgfpathlineto{\pgfqpoint{3.803009in}{0.648453in}}%
\pgfpathlineto{\pgfqpoint{3.824770in}{0.659106in}}%
\pgfpathlineto{\pgfqpoint{3.846532in}{0.668695in}}%
\pgfpathlineto{\pgfqpoint{3.868294in}{0.676651in}}%
\pgfpathlineto{\pgfqpoint{3.890055in}{0.684286in}}%
\pgfpathlineto{\pgfqpoint{3.911817in}{0.692729in}}%
\pgfpathlineto{\pgfqpoint{3.933579in}{0.700229in}}%
\pgfpathlineto{\pgfqpoint{3.955340in}{0.708377in}}%
\pgfpathlineto{\pgfqpoint{3.977102in}{0.715846in}}%
\pgfpathlineto{\pgfqpoint{3.998864in}{0.722700in}}%
\pgfpathlineto{\pgfqpoint{4.020625in}{0.730400in}}%
\pgfpathlineto{\pgfqpoint{4.042387in}{0.737315in}}%
\pgfpathlineto{\pgfqpoint{4.064149in}{0.744143in}}%
\pgfpathlineto{\pgfqpoint{4.085910in}{0.750782in}}%
\pgfpathlineto{\pgfqpoint{4.107672in}{0.757824in}}%
\pgfpathlineto{\pgfqpoint{4.129434in}{0.764142in}}%
\pgfpathlineto{\pgfqpoint{4.151195in}{0.770824in}}%
\pgfpathlineto{\pgfqpoint{4.172957in}{0.776404in}}%
\pgfpathlineto{\pgfqpoint{4.194719in}{0.783123in}}%
\pgfpathlineto{\pgfqpoint{4.216480in}{0.788684in}}%
\pgfpathlineto{\pgfqpoint{4.238242in}{0.794790in}}%
\pgfpathlineto{\pgfqpoint{4.260004in}{0.800282in}}%
\pgfpathlineto{\pgfqpoint{4.281765in}{0.806101in}}%
\pgfpathlineto{\pgfqpoint{4.303527in}{0.811617in}}%
\pgfpathlineto{\pgfqpoint{4.325289in}{0.817927in}}%
\pgfpathlineto{\pgfqpoint{4.347050in}{0.823740in}}%
\pgfpathlineto{\pgfqpoint{4.368812in}{0.828846in}}%
\pgfpathlineto{\pgfqpoint{4.390574in}{0.834960in}}%
\pgfpathlineto{\pgfqpoint{4.412335in}{0.839188in}}%
\pgfpathlineto{\pgfqpoint{4.434097in}{0.844246in}}%
\pgfpathlineto{\pgfqpoint{4.455859in}{0.850352in}}%
\pgfpathlineto{\pgfqpoint{4.477620in}{0.855543in}}%
\pgfpathlineto{\pgfqpoint{4.499382in}{0.861159in}}%
\pgfpathlineto{\pgfqpoint{4.521144in}{0.866558in}}%
\pgfpathlineto{\pgfqpoint{4.542905in}{0.872280in}}%
\pgfpathlineto{\pgfqpoint{4.564667in}{0.877233in}}%
\pgfpathlineto{\pgfqpoint{4.586429in}{0.881880in}}%
\pgfpathlineto{\pgfqpoint{4.608190in}{0.886424in}}%
\pgfpathlineto{\pgfqpoint{4.629952in}{0.891656in}}%
\pgfpathlineto{\pgfqpoint{4.651714in}{0.897833in}}%
\pgfpathlineto{\pgfqpoint{4.673475in}{0.903020in}}%
\pgfpathlineto{\pgfqpoint{4.695237in}{0.908075in}}%
\pgfpathlineto{\pgfqpoint{4.716999in}{0.912885in}}%
\pgfpathlineto{\pgfqpoint{4.738760in}{0.916937in}}%
\pgfpathlineto{\pgfqpoint{4.760522in}{0.921855in}}%
\pgfpathlineto{\pgfqpoint{4.782284in}{0.926069in}}%
\pgfpathlineto{\pgfqpoint{4.804045in}{0.931426in}}%
\pgfpathlineto{\pgfqpoint{4.825807in}{0.936057in}}%
\pgfpathlineto{\pgfqpoint{4.847569in}{0.940419in}}%
\pgfpathlineto{\pgfqpoint{4.869330in}{0.944492in}}%
\pgfpathlineto{\pgfqpoint{4.891092in}{0.948538in}}%
\pgfpathlineto{\pgfqpoint{4.912854in}{0.952840in}}%
\pgfpathlineto{\pgfqpoint{4.934615in}{0.957846in}}%
\pgfpathlineto{\pgfqpoint{4.956377in}{0.961605in}}%
\pgfpathlineto{\pgfqpoint{4.978139in}{0.966030in}}%
\pgfpathlineto{\pgfqpoint{4.999900in}{0.970461in}}%
\pgfpathlineto{\pgfqpoint{5.021662in}{0.974912in}}%
\pgfpathlineto{\pgfqpoint{5.043424in}{0.979070in}}%
\pgfpathlineto{\pgfqpoint{5.065185in}{0.982630in}}%
\pgfpathlineto{\pgfqpoint{5.086947in}{0.986543in}}%
\pgfpathlineto{\pgfqpoint{5.108709in}{0.989869in}}%
\pgfpathlineto{\pgfqpoint{5.130470in}{0.994271in}}%
\pgfpathlineto{\pgfqpoint{5.152232in}{0.998259in}}%
\pgfpathlineto{\pgfqpoint{5.173994in}{1.001615in}}%
\pgfpathlineto{\pgfqpoint{5.195755in}{1.005651in}}%
\pgfpathlineto{\pgfqpoint{5.217517in}{1.008828in}}%
\pgfpathlineto{\pgfqpoint{5.239279in}{1.012712in}}%
\pgfpathlineto{\pgfqpoint{5.261040in}{1.016264in}}%
\pgfpathlineto{\pgfqpoint{5.282802in}{1.019971in}}%
\pgfpathlineto{\pgfqpoint{5.304564in}{1.023754in}}%
\pgfpathlineto{\pgfqpoint{5.326325in}{1.027491in}}%
\pgfpathlineto{\pgfqpoint{5.348087in}{1.030672in}}%
\pgfpathlineto{\pgfqpoint{5.369849in}{1.034285in}}%
\pgfpathlineto{\pgfqpoint{5.391610in}{1.038027in}}%
\pgfpathlineto{\pgfqpoint{5.413372in}{1.041293in}}%
\pgfpathlineto{\pgfqpoint{5.435134in}{1.044949in}}%
\pgfpathlineto{\pgfqpoint{5.456895in}{1.048184in}}%
\pgfpathlineto{\pgfqpoint{5.478657in}{1.051578in}}%
\pgfpathlineto{\pgfqpoint{5.500419in}{1.054619in}}%
\pgfpathlineto{\pgfqpoint{5.522180in}{1.058064in}}%
\pgfpathlineto{\pgfqpoint{5.543942in}{1.061783in}}%
\pgfpathlineto{\pgfqpoint{5.565704in}{1.065248in}}%
\pgfpathlineto{\pgfqpoint{5.587465in}{1.068312in}}%
\pgfpathlineto{\pgfqpoint{5.609227in}{1.071352in}}%
\pgfpathlineto{\pgfqpoint{5.630989in}{1.074847in}}%
\pgfusepath{stroke}%
\end{pgfscope}%
\begin{pgfscope}%
\pgfpathrectangle{\pgfqpoint{3.454822in}{0.422992in}}{\pgfqpoint{2.219690in}{3.151201in}}%
\pgfusepath{clip}%
\pgfsetrectcap%
\pgfsetroundjoin%
\pgfsetlinewidth{1.003750pt}%
\definecolor{currentstroke}{rgb}{0.278431,0.278431,0.278431}%
\pgfsetstrokecolor{currentstroke}%
\pgfsetdash{}{0pt}%
\pgfpathmoveto{\pgfqpoint{3.476584in}{2.492054in}}%
\pgfpathlineto{\pgfqpoint{3.498345in}{2.734256in}}%
\pgfpathlineto{\pgfqpoint{3.520107in}{2.851341in}}%
\pgfpathlineto{\pgfqpoint{3.541869in}{2.928598in}}%
\pgfpathlineto{\pgfqpoint{3.563630in}{2.984158in}}%
\pgfpathlineto{\pgfqpoint{3.585392in}{3.033244in}}%
\pgfpathlineto{\pgfqpoint{3.607154in}{3.076540in}}%
\pgfpathlineto{\pgfqpoint{3.628915in}{3.103957in}}%
\pgfpathlineto{\pgfqpoint{3.650677in}{3.125473in}}%
\pgfpathlineto{\pgfqpoint{3.672439in}{3.146321in}}%
\pgfpathlineto{\pgfqpoint{3.694200in}{3.162107in}}%
\pgfpathlineto{\pgfqpoint{3.715962in}{3.177539in}}%
\pgfpathlineto{\pgfqpoint{3.737724in}{3.189124in}}%
\pgfpathlineto{\pgfqpoint{3.759485in}{3.200959in}}%
\pgfpathlineto{\pgfqpoint{3.781247in}{3.213181in}}%
\pgfpathlineto{\pgfqpoint{3.803009in}{3.222437in}}%
\pgfpathlineto{\pgfqpoint{3.824770in}{3.229427in}}%
\pgfpathlineto{\pgfqpoint{3.846532in}{3.236464in}}%
\pgfpathlineto{\pgfqpoint{3.868294in}{3.244493in}}%
\pgfpathlineto{\pgfqpoint{3.890055in}{3.252515in}}%
\pgfpathlineto{\pgfqpoint{3.911817in}{3.258411in}}%
\pgfpathlineto{\pgfqpoint{3.933579in}{3.264465in}}%
\pgfpathlineto{\pgfqpoint{3.955340in}{3.271844in}}%
\pgfpathlineto{\pgfqpoint{3.977102in}{3.280168in}}%
\pgfpathlineto{\pgfqpoint{3.998864in}{3.286277in}}%
\pgfpathlineto{\pgfqpoint{4.020625in}{3.292024in}}%
\pgfpathlineto{\pgfqpoint{4.042387in}{3.298516in}}%
\pgfpathlineto{\pgfqpoint{4.064149in}{3.302194in}}%
\pgfpathlineto{\pgfqpoint{4.085910in}{3.307168in}}%
\pgfpathlineto{\pgfqpoint{4.107672in}{3.311632in}}%
\pgfpathlineto{\pgfqpoint{4.129434in}{3.315759in}}%
\pgfpathlineto{\pgfqpoint{4.151195in}{3.320616in}}%
\pgfpathlineto{\pgfqpoint{4.172957in}{3.324996in}}%
\pgfpathlineto{\pgfqpoint{4.194719in}{3.328119in}}%
\pgfpathlineto{\pgfqpoint{4.216480in}{3.332631in}}%
\pgfpathlineto{\pgfqpoint{4.238242in}{3.336491in}}%
\pgfpathlineto{\pgfqpoint{4.260004in}{3.339754in}}%
\pgfpathlineto{\pgfqpoint{4.281765in}{3.341922in}}%
\pgfpathlineto{\pgfqpoint{4.303527in}{3.346058in}}%
\pgfpathlineto{\pgfqpoint{4.325289in}{3.349762in}}%
\pgfpathlineto{\pgfqpoint{4.347050in}{3.352172in}}%
\pgfpathlineto{\pgfqpoint{4.368812in}{3.354248in}}%
\pgfpathlineto{\pgfqpoint{4.390574in}{3.356407in}}%
\pgfpathlineto{\pgfqpoint{4.412335in}{3.359236in}}%
\pgfpathlineto{\pgfqpoint{4.434097in}{3.361408in}}%
\pgfpathlineto{\pgfqpoint{4.455859in}{3.363374in}}%
\pgfpathlineto{\pgfqpoint{4.477620in}{3.365008in}}%
\pgfpathlineto{\pgfqpoint{4.499382in}{3.366826in}}%
\pgfpathlineto{\pgfqpoint{4.521144in}{3.369592in}}%
\pgfpathlineto{\pgfqpoint{4.542905in}{3.372807in}}%
\pgfpathlineto{\pgfqpoint{4.564667in}{3.374968in}}%
\pgfpathlineto{\pgfqpoint{4.586429in}{3.377476in}}%
\pgfpathlineto{\pgfqpoint{4.608190in}{3.379058in}}%
\pgfpathlineto{\pgfqpoint{4.629952in}{3.381043in}}%
\pgfpathlineto{\pgfqpoint{4.651714in}{3.382474in}}%
\pgfpathlineto{\pgfqpoint{4.673475in}{3.383467in}}%
\pgfpathlineto{\pgfqpoint{4.695237in}{3.385432in}}%
\pgfpathlineto{\pgfqpoint{4.716999in}{3.386736in}}%
\pgfpathlineto{\pgfqpoint{4.738760in}{3.388284in}}%
\pgfpathlineto{\pgfqpoint{4.760522in}{3.388573in}}%
\pgfpathlineto{\pgfqpoint{4.782284in}{3.389881in}}%
\pgfpathlineto{\pgfqpoint{4.804045in}{3.391230in}}%
\pgfpathlineto{\pgfqpoint{4.825807in}{3.392281in}}%
\pgfpathlineto{\pgfqpoint{4.847569in}{3.393524in}}%
\pgfpathlineto{\pgfqpoint{4.869330in}{3.395225in}}%
\pgfpathlineto{\pgfqpoint{4.891092in}{3.396774in}}%
\pgfpathlineto{\pgfqpoint{4.912854in}{3.398090in}}%
\pgfpathlineto{\pgfqpoint{4.934615in}{3.400069in}}%
\pgfpathlineto{\pgfqpoint{4.956377in}{3.400879in}}%
\pgfpathlineto{\pgfqpoint{4.978139in}{3.402350in}}%
\pgfpathlineto{\pgfqpoint{4.999900in}{3.404477in}}%
\pgfpathlineto{\pgfqpoint{5.021662in}{3.405475in}}%
\pgfpathlineto{\pgfqpoint{5.043424in}{3.406858in}}%
\pgfpathlineto{\pgfqpoint{5.065185in}{3.407478in}}%
\pgfpathlineto{\pgfqpoint{5.086947in}{3.408688in}}%
\pgfpathlineto{\pgfqpoint{5.108709in}{3.410351in}}%
\pgfpathlineto{\pgfqpoint{5.130470in}{3.411485in}}%
\pgfpathlineto{\pgfqpoint{5.152232in}{3.412421in}}%
\pgfpathlineto{\pgfqpoint{5.173994in}{3.412849in}}%
\pgfpathlineto{\pgfqpoint{5.195755in}{3.413420in}}%
\pgfpathlineto{\pgfqpoint{5.217517in}{3.414290in}}%
\pgfpathlineto{\pgfqpoint{5.239279in}{3.415681in}}%
\pgfpathlineto{\pgfqpoint{5.261040in}{3.416645in}}%
\pgfpathlineto{\pgfqpoint{5.282802in}{3.417420in}}%
\pgfpathlineto{\pgfqpoint{5.304564in}{3.418172in}}%
\pgfpathlineto{\pgfqpoint{5.326325in}{3.419253in}}%
\pgfpathlineto{\pgfqpoint{5.348087in}{3.419830in}}%
\pgfpathlineto{\pgfqpoint{5.369849in}{3.420666in}}%
\pgfpathlineto{\pgfqpoint{5.391610in}{3.421161in}}%
\pgfpathlineto{\pgfqpoint{5.413372in}{3.422403in}}%
\pgfpathlineto{\pgfqpoint{5.435134in}{3.423220in}}%
\pgfpathlineto{\pgfqpoint{5.456895in}{3.423741in}}%
\pgfpathlineto{\pgfqpoint{5.478657in}{3.424411in}}%
\pgfpathlineto{\pgfqpoint{5.500419in}{3.425469in}}%
\pgfpathlineto{\pgfqpoint{5.522180in}{3.426394in}}%
\pgfpathlineto{\pgfqpoint{5.543942in}{3.427053in}}%
\pgfpathlineto{\pgfqpoint{5.565704in}{3.427983in}}%
\pgfpathlineto{\pgfqpoint{5.587465in}{3.428887in}}%
\pgfpathlineto{\pgfqpoint{5.609227in}{3.429833in}}%
\pgfpathlineto{\pgfqpoint{5.630989in}{3.430957in}}%
\pgfusepath{stroke}%
\end{pgfscope}%
\begin{pgfscope}%
\pgfsetrectcap%
\pgfsetmiterjoin%
\pgfsetlinewidth{0.501875pt}%
\definecolor{currentstroke}{rgb}{0.000000,0.000000,0.000000}%
\pgfsetstrokecolor{currentstroke}%
\pgfsetdash{}{0pt}%
\pgfpathmoveto{\pgfqpoint{3.454822in}{0.422992in}}%
\pgfpathlineto{\pgfqpoint{3.454822in}{3.574193in}}%
\pgfusepath{stroke}%
\end{pgfscope}%
\begin{pgfscope}%
\pgfsetrectcap%
\pgfsetmiterjoin%
\pgfsetlinewidth{0.501875pt}%
\definecolor{currentstroke}{rgb}{0.000000,0.000000,0.000000}%
\pgfsetstrokecolor{currentstroke}%
\pgfsetdash{}{0pt}%
\pgfpathmoveto{\pgfqpoint{5.674512in}{0.422992in}}%
\pgfpathlineto{\pgfqpoint{5.674512in}{3.574193in}}%
\pgfusepath{stroke}%
\end{pgfscope}%
\begin{pgfscope}%
\pgfsetrectcap%
\pgfsetmiterjoin%
\pgfsetlinewidth{0.501875pt}%
\definecolor{currentstroke}{rgb}{0.000000,0.000000,0.000000}%
\pgfsetstrokecolor{currentstroke}%
\pgfsetdash{}{0pt}%
\pgfpathmoveto{\pgfqpoint{3.454822in}{0.422992in}}%
\pgfpathlineto{\pgfqpoint{5.674512in}{0.422992in}}%
\pgfusepath{stroke}%
\end{pgfscope}%
\begin{pgfscope}%
\pgfsetrectcap%
\pgfsetmiterjoin%
\pgfsetlinewidth{0.501875pt}%
\definecolor{currentstroke}{rgb}{0.000000,0.000000,0.000000}%
\pgfsetstrokecolor{currentstroke}%
\pgfsetdash{}{0pt}%
\pgfpathmoveto{\pgfqpoint{3.454822in}{3.574193in}}%
\pgfpathlineto{\pgfqpoint{5.674512in}{3.574193in}}%
\pgfusepath{stroke}%
\end{pgfscope}%
\begin{pgfscope}%
\definecolor{textcolor}{rgb}{0.000000,0.000000,0.000000}%
\pgfsetstrokecolor{textcolor}%
\pgfsetfillcolor{textcolor}%
\pgftext[x=4.564667in,y=3.657526in,,base]{\color{textcolor}\rmfamily\fontsize{12.000000}{14.400000}\selectfont LCMC}%
\end{pgfscope}%
\begin{pgfscope}%
\pgfsetrectcap%
\pgfsetroundjoin%
\pgfsetlinewidth{1.003750pt}%
\definecolor{currentstroke}{rgb}{0.047059,0.364706,0.647059}%
\pgfsetstrokecolor{currentstroke}%
\pgfsetdash{}{0pt}%
\pgfpathmoveto{\pgfqpoint{3.579822in}{1.644779in}}%
\pgfpathlineto{\pgfqpoint{3.718711in}{1.644779in}}%
\pgfpathlineto{\pgfqpoint{3.857600in}{1.644779in}}%
\pgfusepath{stroke}%
\end{pgfscope}%
\begin{pgfscope}%
\definecolor{textcolor}{rgb}{0.000000,0.000000,0.000000}%
\pgfsetstrokecolor{textcolor}%
\pgfsetfillcolor{textcolor}%
\pgftext[x=3.968711in,y=1.596168in,left,base]{\color{textcolor}\rmfamily\fontsize{10.000000}{12.000000}\selectfont PCA}%
\end{pgfscope}%
\begin{pgfscope}%
\pgfsetrectcap%
\pgfsetroundjoin%
\pgfsetlinewidth{1.003750pt}%
\definecolor{currentstroke}{rgb}{0.000000,0.725490,0.270588}%
\pgfsetstrokecolor{currentstroke}%
\pgfsetdash{}{0pt}%
\pgfpathmoveto{\pgfqpoint{3.579822in}{1.440922in}}%
\pgfpathlineto{\pgfqpoint{3.718711in}{1.440922in}}%
\pgfpathlineto{\pgfqpoint{3.857600in}{1.440922in}}%
\pgfusepath{stroke}%
\end{pgfscope}%
\begin{pgfscope}%
\definecolor{textcolor}{rgb}{0.000000,0.000000,0.000000}%
\pgfsetstrokecolor{textcolor}%
\pgfsetfillcolor{textcolor}%
\pgftext[x=3.968711in,y=1.392311in,left,base]{\color{textcolor}\rmfamily\fontsize{10.000000}{12.000000}\selectfont KernelPCA}%
\end{pgfscope}%
\begin{pgfscope}%
\pgfsetrectcap%
\pgfsetroundjoin%
\pgfsetlinewidth{1.003750pt}%
\definecolor{currentstroke}{rgb}{1.000000,0.584314,0.000000}%
\pgfsetstrokecolor{currentstroke}%
\pgfsetdash{}{0pt}%
\pgfpathmoveto{\pgfqpoint{3.579822in}{1.237065in}}%
\pgfpathlineto{\pgfqpoint{3.718711in}{1.237065in}}%
\pgfpathlineto{\pgfqpoint{3.857600in}{1.237065in}}%
\pgfusepath{stroke}%
\end{pgfscope}%
\begin{pgfscope}%
\definecolor{textcolor}{rgb}{0.000000,0.000000,0.000000}%
\pgfsetstrokecolor{textcolor}%
\pgfsetfillcolor{textcolor}%
\pgftext[x=3.968711in,y=1.188454in,left,base]{\color{textcolor}\rmfamily\fontsize{10.000000}{12.000000}\selectfont LLE}%
\end{pgfscope}%
\begin{pgfscope}%
\pgfsetrectcap%
\pgfsetroundjoin%
\pgfsetlinewidth{1.003750pt}%
\definecolor{currentstroke}{rgb}{1.000000,0.172549,0.000000}%
\pgfsetstrokecolor{currentstroke}%
\pgfsetdash{}{0pt}%
\pgfpathmoveto{\pgfqpoint{3.579822in}{1.033208in}}%
\pgfpathlineto{\pgfqpoint{3.718711in}{1.033208in}}%
\pgfpathlineto{\pgfqpoint{3.857600in}{1.033208in}}%
\pgfusepath{stroke}%
\end{pgfscope}%
\begin{pgfscope}%
\definecolor{textcolor}{rgb}{0.000000,0.000000,0.000000}%
\pgfsetstrokecolor{textcolor}%
\pgfsetfillcolor{textcolor}%
\pgftext[x=3.968711in,y=0.984596in,left,base]{\color{textcolor}\rmfamily\fontsize{10.000000}{12.000000}\selectfont AE}%
\end{pgfscope}%
\begin{pgfscope}%
\pgfsetrectcap%
\pgfsetroundjoin%
\pgfsetlinewidth{1.003750pt}%
\definecolor{currentstroke}{rgb}{0.517647,0.356863,0.592157}%
\pgfsetstrokecolor{currentstroke}%
\pgfsetdash{}{0pt}%
\pgfpathmoveto{\pgfqpoint{3.579822in}{0.829350in}}%
\pgfpathlineto{\pgfqpoint{3.718711in}{0.829350in}}%
\pgfpathlineto{\pgfqpoint{3.857600in}{0.829350in}}%
\pgfusepath{stroke}%
\end{pgfscope}%
\begin{pgfscope}%
\definecolor{textcolor}{rgb}{0.000000,0.000000,0.000000}%
\pgfsetstrokecolor{textcolor}%
\pgfsetfillcolor{textcolor}%
\pgftext[x=3.968711in,y=0.780739in,left,base]{\color{textcolor}\rmfamily\fontsize{10.000000}{12.000000}\selectfont CAE}%
\end{pgfscope}%
\begin{pgfscope}%
\pgfsetrectcap%
\pgfsetroundjoin%
\pgfsetlinewidth{1.003750pt}%
\definecolor{currentstroke}{rgb}{0.278431,0.278431,0.278431}%
\pgfsetstrokecolor{currentstroke}%
\pgfsetdash{}{0pt}%
\pgfpathmoveto{\pgfqpoint{3.579822in}{0.625493in}}%
\pgfpathlineto{\pgfqpoint{3.718711in}{0.625493in}}%
\pgfpathlineto{\pgfqpoint{3.857600in}{0.625493in}}%
\pgfusepath{stroke}%
\end{pgfscope}%
\begin{pgfscope}%
\definecolor{textcolor}{rgb}{0.000000,0.000000,0.000000}%
\pgfsetstrokecolor{textcolor}%
\pgfsetfillcolor{textcolor}%
\pgftext[x=3.968711in,y=0.576882in,left,base]{\color{textcolor}\rmfamily\fontsize{10.000000}{12.000000}\selectfont ConvAE}%
\end{pgfscope}%
\end{pgfpicture}%
\makeatother%
\endgroup%

	\end{center}
	\caption[MNIST Qualitätskriterien]{Die Vertrauenswürdigkeit und Kontinuität der Dimensionsreduktion, sowie das Local Continuity Meta-Criterion (LCMC) für den MNIST Datensatz. Auf diesem Datensatz kann der der domänenspezifische Convolutional Autoencoder (ConvAE) seine starke Performance zeigen, da der Datensatz von hoher Qualität ist und eine hohe Stichprobengröße von 60 000 aufweisen kann. Nichtsdestotrotz kann die Hauptkomponentenanalyse nur knapp dahinter sehr gut mithalten und hinsichtlich der Kontinuität sogar übertreffen. Die Kernel PCA und der klassische vollvernetzte Autoencoder schneiden ebenfalls relativ gut ab, wobei letzteres aber für größere Werte von $K$ auf der Vertrauenswürdigkeit stärker abfällt, als die zuvor genannten Methoden. LLE und der Contractive Autoencoder können vor allem auf dem LCMC nicht mithalten. (Eigene Darstellung)}
	\label{fig:MNISTMetrics}
\end{figure}

\begin{figure}[ht]
	\begin{center}
		%% Creator: Matplotlib, PGF backend
%%
%% To include the figure in your LaTeX document, write
%%   \input{<filename>.pgf}
%%
%% Make sure the required packages are loaded in your preamble
%%   \usepackage{pgf}
%%
%% Also ensure that all the required font packages are loaded; for instance,
%% the lmodern package is sometimes necessary when using math font.
%%   \usepackage{lmodern}
%%
%% Figures using additional raster images can only be included by \input if
%% they are in the same directory as the main LaTeX file. For loading figures
%% from other directories you can use the `import` package
%%   \usepackage{import}
%%
%% and then include the figures with
%%   \import{<path to file>}{<filename>.pgf}
%%
%% Matplotlib used the following preamble
%%   
%%   \usepackage{fontspec}
%%   \setmainfont{DejaVuSerif.ttf}[Path=\detokenize{/Users/moritzmistol/.pyenv/versions/3.9.13/envs/thesis/lib/python3.9/site-packages/matplotlib/mpl-data/fonts/ttf/}]
%%   \setsansfont{DejaVuSans.ttf}[Path=\detokenize{/Users/moritzmistol/.pyenv/versions/3.9.13/envs/thesis/lib/python3.9/site-packages/matplotlib/mpl-data/fonts/ttf/}]
%%   \setmonofont{DejaVuSansMono.ttf}[Path=\detokenize{/Users/moritzmistol/.pyenv/versions/3.9.13/envs/thesis/lib/python3.9/site-packages/matplotlib/mpl-data/fonts/ttf/}]
%%   \makeatletter\@ifpackageloaded{underscore}{}{\usepackage[strings]{underscore}}\makeatother
%%
\begingroup%
\makeatletter%
\begin{pgfpicture}%
\pgfpathrectangle{\pgfpointorigin}{\pgfqpoint{5.641997in}{4.634154in}}%
\pgfusepath{use as bounding box, clip}%
\begin{pgfscope}%
\pgfsetbuttcap%
\pgfsetmiterjoin%
\definecolor{currentfill}{rgb}{1.000000,1.000000,1.000000}%
\pgfsetfillcolor{currentfill}%
\pgfsetlinewidth{0.000000pt}%
\definecolor{currentstroke}{rgb}{1.000000,1.000000,1.000000}%
\pgfsetstrokecolor{currentstroke}%
\pgfsetdash{}{0pt}%
\pgfpathmoveto{\pgfqpoint{0.000000in}{-0.000000in}}%
\pgfpathlineto{\pgfqpoint{5.641997in}{-0.000000in}}%
\pgfpathlineto{\pgfqpoint{5.641997in}{4.634154in}}%
\pgfpathlineto{\pgfqpoint{0.000000in}{4.634154in}}%
\pgfpathlineto{\pgfqpoint{0.000000in}{-0.000000in}}%
\pgfpathclose%
\pgfusepath{fill}%
\end{pgfscope}%
\begin{pgfscope}%
\pgfsetbuttcap%
\pgfsetmiterjoin%
\definecolor{currentfill}{rgb}{1.000000,1.000000,1.000000}%
\pgfsetfillcolor{currentfill}%
\pgfsetlinewidth{0.000000pt}%
\definecolor{currentstroke}{rgb}{0.000000,0.000000,0.000000}%
\pgfsetstrokecolor{currentstroke}%
\pgfsetstrokeopacity{0.000000}%
\pgfsetdash{}{0pt}%
\pgfpathmoveto{\pgfqpoint{0.470525in}{2.747992in}}%
\pgfpathlineto{\pgfqpoint{2.746589in}{2.747992in}}%
\pgfpathlineto{\pgfqpoint{2.746589in}{4.374193in}}%
\pgfpathlineto{\pgfqpoint{0.470525in}{4.374193in}}%
\pgfpathlineto{\pgfqpoint{0.470525in}{2.747992in}}%
\pgfpathclose%
\pgfusepath{fill}%
\end{pgfscope}%
\begin{pgfscope}%
\pgfsetbuttcap%
\pgfsetroundjoin%
\definecolor{currentfill}{rgb}{0.000000,0.000000,0.000000}%
\pgfsetfillcolor{currentfill}%
\pgfsetlinewidth{0.501875pt}%
\definecolor{currentstroke}{rgb}{0.000000,0.000000,0.000000}%
\pgfsetstrokecolor{currentstroke}%
\pgfsetdash{}{0pt}%
\pgfsys@defobject{currentmarker}{\pgfqpoint{0.000000in}{0.000000in}}{\pgfqpoint{0.000000in}{0.041667in}}{%
\pgfpathmoveto{\pgfqpoint{0.000000in}{0.000000in}}%
\pgfpathlineto{\pgfqpoint{0.000000in}{0.041667in}}%
\pgfusepath{stroke,fill}%
}%
\begin{pgfscope}%
\pgfsys@transformshift{0.470525in}{2.747992in}%
\pgfsys@useobject{currentmarker}{}%
\end{pgfscope}%
\end{pgfscope}%
\begin{pgfscope}%
\pgfsetbuttcap%
\pgfsetroundjoin%
\definecolor{currentfill}{rgb}{0.000000,0.000000,0.000000}%
\pgfsetfillcolor{currentfill}%
\pgfsetlinewidth{0.501875pt}%
\definecolor{currentstroke}{rgb}{0.000000,0.000000,0.000000}%
\pgfsetstrokecolor{currentstroke}%
\pgfsetdash{}{0pt}%
\pgfsys@defobject{currentmarker}{\pgfqpoint{0.000000in}{-0.041667in}}{\pgfqpoint{0.000000in}{0.000000in}}{%
\pgfpathmoveto{\pgfqpoint{0.000000in}{0.000000in}}%
\pgfpathlineto{\pgfqpoint{0.000000in}{-0.041667in}}%
\pgfusepath{stroke,fill}%
}%
\begin{pgfscope}%
\pgfsys@transformshift{0.470525in}{4.374193in}%
\pgfsys@useobject{currentmarker}{}%
\end{pgfscope}%
\end{pgfscope}%
\begin{pgfscope}%
\definecolor{textcolor}{rgb}{0.000000,0.000000,0.000000}%
\pgfsetstrokecolor{textcolor}%
\pgfsetfillcolor{textcolor}%
\pgftext[x=0.470525in,y=2.699381in,,top]{\color{textcolor}\rmfamily\fontsize{10.000000}{12.000000}\selectfont \(\displaystyle {0}\)}%
\end{pgfscope}%
\begin{pgfscope}%
\pgfsetbuttcap%
\pgfsetroundjoin%
\definecolor{currentfill}{rgb}{0.000000,0.000000,0.000000}%
\pgfsetfillcolor{currentfill}%
\pgfsetlinewidth{0.501875pt}%
\definecolor{currentstroke}{rgb}{0.000000,0.000000,0.000000}%
\pgfsetstrokecolor{currentstroke}%
\pgfsetdash{}{0pt}%
\pgfsys@defobject{currentmarker}{\pgfqpoint{0.000000in}{0.000000in}}{\pgfqpoint{0.000000in}{0.041667in}}{%
\pgfpathmoveto{\pgfqpoint{0.000000in}{0.000000in}}%
\pgfpathlineto{\pgfqpoint{0.000000in}{0.041667in}}%
\pgfusepath{stroke,fill}%
}%
\begin{pgfscope}%
\pgfsys@transformshift{0.921231in}{2.747992in}%
\pgfsys@useobject{currentmarker}{}%
\end{pgfscope}%
\end{pgfscope}%
\begin{pgfscope}%
\pgfsetbuttcap%
\pgfsetroundjoin%
\definecolor{currentfill}{rgb}{0.000000,0.000000,0.000000}%
\pgfsetfillcolor{currentfill}%
\pgfsetlinewidth{0.501875pt}%
\definecolor{currentstroke}{rgb}{0.000000,0.000000,0.000000}%
\pgfsetstrokecolor{currentstroke}%
\pgfsetdash{}{0pt}%
\pgfsys@defobject{currentmarker}{\pgfqpoint{0.000000in}{-0.041667in}}{\pgfqpoint{0.000000in}{0.000000in}}{%
\pgfpathmoveto{\pgfqpoint{0.000000in}{0.000000in}}%
\pgfpathlineto{\pgfqpoint{0.000000in}{-0.041667in}}%
\pgfusepath{stroke,fill}%
}%
\begin{pgfscope}%
\pgfsys@transformshift{0.921231in}{4.374193in}%
\pgfsys@useobject{currentmarker}{}%
\end{pgfscope}%
\end{pgfscope}%
\begin{pgfscope}%
\definecolor{textcolor}{rgb}{0.000000,0.000000,0.000000}%
\pgfsetstrokecolor{textcolor}%
\pgfsetfillcolor{textcolor}%
\pgftext[x=0.921231in,y=2.699381in,,top]{\color{textcolor}\rmfamily\fontsize{10.000000}{12.000000}\selectfont \(\displaystyle {20}\)}%
\end{pgfscope}%
\begin{pgfscope}%
\pgfsetbuttcap%
\pgfsetroundjoin%
\definecolor{currentfill}{rgb}{0.000000,0.000000,0.000000}%
\pgfsetfillcolor{currentfill}%
\pgfsetlinewidth{0.501875pt}%
\definecolor{currentstroke}{rgb}{0.000000,0.000000,0.000000}%
\pgfsetstrokecolor{currentstroke}%
\pgfsetdash{}{0pt}%
\pgfsys@defobject{currentmarker}{\pgfqpoint{0.000000in}{0.000000in}}{\pgfqpoint{0.000000in}{0.041667in}}{%
\pgfpathmoveto{\pgfqpoint{0.000000in}{0.000000in}}%
\pgfpathlineto{\pgfqpoint{0.000000in}{0.041667in}}%
\pgfusepath{stroke,fill}%
}%
\begin{pgfscope}%
\pgfsys@transformshift{1.371937in}{2.747992in}%
\pgfsys@useobject{currentmarker}{}%
\end{pgfscope}%
\end{pgfscope}%
\begin{pgfscope}%
\pgfsetbuttcap%
\pgfsetroundjoin%
\definecolor{currentfill}{rgb}{0.000000,0.000000,0.000000}%
\pgfsetfillcolor{currentfill}%
\pgfsetlinewidth{0.501875pt}%
\definecolor{currentstroke}{rgb}{0.000000,0.000000,0.000000}%
\pgfsetstrokecolor{currentstroke}%
\pgfsetdash{}{0pt}%
\pgfsys@defobject{currentmarker}{\pgfqpoint{0.000000in}{-0.041667in}}{\pgfqpoint{0.000000in}{0.000000in}}{%
\pgfpathmoveto{\pgfqpoint{0.000000in}{0.000000in}}%
\pgfpathlineto{\pgfqpoint{0.000000in}{-0.041667in}}%
\pgfusepath{stroke,fill}%
}%
\begin{pgfscope}%
\pgfsys@transformshift{1.371937in}{4.374193in}%
\pgfsys@useobject{currentmarker}{}%
\end{pgfscope}%
\end{pgfscope}%
\begin{pgfscope}%
\definecolor{textcolor}{rgb}{0.000000,0.000000,0.000000}%
\pgfsetstrokecolor{textcolor}%
\pgfsetfillcolor{textcolor}%
\pgftext[x=1.371937in,y=2.699381in,,top]{\color{textcolor}\rmfamily\fontsize{10.000000}{12.000000}\selectfont \(\displaystyle {40}\)}%
\end{pgfscope}%
\begin{pgfscope}%
\pgfsetbuttcap%
\pgfsetroundjoin%
\definecolor{currentfill}{rgb}{0.000000,0.000000,0.000000}%
\pgfsetfillcolor{currentfill}%
\pgfsetlinewidth{0.501875pt}%
\definecolor{currentstroke}{rgb}{0.000000,0.000000,0.000000}%
\pgfsetstrokecolor{currentstroke}%
\pgfsetdash{}{0pt}%
\pgfsys@defobject{currentmarker}{\pgfqpoint{0.000000in}{0.000000in}}{\pgfqpoint{0.000000in}{0.041667in}}{%
\pgfpathmoveto{\pgfqpoint{0.000000in}{0.000000in}}%
\pgfpathlineto{\pgfqpoint{0.000000in}{0.041667in}}%
\pgfusepath{stroke,fill}%
}%
\begin{pgfscope}%
\pgfsys@transformshift{1.822643in}{2.747992in}%
\pgfsys@useobject{currentmarker}{}%
\end{pgfscope}%
\end{pgfscope}%
\begin{pgfscope}%
\pgfsetbuttcap%
\pgfsetroundjoin%
\definecolor{currentfill}{rgb}{0.000000,0.000000,0.000000}%
\pgfsetfillcolor{currentfill}%
\pgfsetlinewidth{0.501875pt}%
\definecolor{currentstroke}{rgb}{0.000000,0.000000,0.000000}%
\pgfsetstrokecolor{currentstroke}%
\pgfsetdash{}{0pt}%
\pgfsys@defobject{currentmarker}{\pgfqpoint{0.000000in}{-0.041667in}}{\pgfqpoint{0.000000in}{0.000000in}}{%
\pgfpathmoveto{\pgfqpoint{0.000000in}{0.000000in}}%
\pgfpathlineto{\pgfqpoint{0.000000in}{-0.041667in}}%
\pgfusepath{stroke,fill}%
}%
\begin{pgfscope}%
\pgfsys@transformshift{1.822643in}{4.374193in}%
\pgfsys@useobject{currentmarker}{}%
\end{pgfscope}%
\end{pgfscope}%
\begin{pgfscope}%
\definecolor{textcolor}{rgb}{0.000000,0.000000,0.000000}%
\pgfsetstrokecolor{textcolor}%
\pgfsetfillcolor{textcolor}%
\pgftext[x=1.822643in,y=2.699381in,,top]{\color{textcolor}\rmfamily\fontsize{10.000000}{12.000000}\selectfont \(\displaystyle {60}\)}%
\end{pgfscope}%
\begin{pgfscope}%
\pgfsetbuttcap%
\pgfsetroundjoin%
\definecolor{currentfill}{rgb}{0.000000,0.000000,0.000000}%
\pgfsetfillcolor{currentfill}%
\pgfsetlinewidth{0.501875pt}%
\definecolor{currentstroke}{rgb}{0.000000,0.000000,0.000000}%
\pgfsetstrokecolor{currentstroke}%
\pgfsetdash{}{0pt}%
\pgfsys@defobject{currentmarker}{\pgfqpoint{0.000000in}{0.000000in}}{\pgfqpoint{0.000000in}{0.041667in}}{%
\pgfpathmoveto{\pgfqpoint{0.000000in}{0.000000in}}%
\pgfpathlineto{\pgfqpoint{0.000000in}{0.041667in}}%
\pgfusepath{stroke,fill}%
}%
\begin{pgfscope}%
\pgfsys@transformshift{2.273348in}{2.747992in}%
\pgfsys@useobject{currentmarker}{}%
\end{pgfscope}%
\end{pgfscope}%
\begin{pgfscope}%
\pgfsetbuttcap%
\pgfsetroundjoin%
\definecolor{currentfill}{rgb}{0.000000,0.000000,0.000000}%
\pgfsetfillcolor{currentfill}%
\pgfsetlinewidth{0.501875pt}%
\definecolor{currentstroke}{rgb}{0.000000,0.000000,0.000000}%
\pgfsetstrokecolor{currentstroke}%
\pgfsetdash{}{0pt}%
\pgfsys@defobject{currentmarker}{\pgfqpoint{0.000000in}{-0.041667in}}{\pgfqpoint{0.000000in}{0.000000in}}{%
\pgfpathmoveto{\pgfqpoint{0.000000in}{0.000000in}}%
\pgfpathlineto{\pgfqpoint{0.000000in}{-0.041667in}}%
\pgfusepath{stroke,fill}%
}%
\begin{pgfscope}%
\pgfsys@transformshift{2.273348in}{4.374193in}%
\pgfsys@useobject{currentmarker}{}%
\end{pgfscope}%
\end{pgfscope}%
\begin{pgfscope}%
\definecolor{textcolor}{rgb}{0.000000,0.000000,0.000000}%
\pgfsetstrokecolor{textcolor}%
\pgfsetfillcolor{textcolor}%
\pgftext[x=2.273348in,y=2.699381in,,top]{\color{textcolor}\rmfamily\fontsize{10.000000}{12.000000}\selectfont \(\displaystyle {80}\)}%
\end{pgfscope}%
\begin{pgfscope}%
\pgfsetbuttcap%
\pgfsetroundjoin%
\definecolor{currentfill}{rgb}{0.000000,0.000000,0.000000}%
\pgfsetfillcolor{currentfill}%
\pgfsetlinewidth{0.501875pt}%
\definecolor{currentstroke}{rgb}{0.000000,0.000000,0.000000}%
\pgfsetstrokecolor{currentstroke}%
\pgfsetdash{}{0pt}%
\pgfsys@defobject{currentmarker}{\pgfqpoint{0.000000in}{0.000000in}}{\pgfqpoint{0.000000in}{0.020833in}}{%
\pgfpathmoveto{\pgfqpoint{0.000000in}{0.000000in}}%
\pgfpathlineto{\pgfqpoint{0.000000in}{0.020833in}}%
\pgfusepath{stroke,fill}%
}%
\begin{pgfscope}%
\pgfsys@transformshift{0.583202in}{2.747992in}%
\pgfsys@useobject{currentmarker}{}%
\end{pgfscope}%
\end{pgfscope}%
\begin{pgfscope}%
\pgfsetbuttcap%
\pgfsetroundjoin%
\definecolor{currentfill}{rgb}{0.000000,0.000000,0.000000}%
\pgfsetfillcolor{currentfill}%
\pgfsetlinewidth{0.501875pt}%
\definecolor{currentstroke}{rgb}{0.000000,0.000000,0.000000}%
\pgfsetstrokecolor{currentstroke}%
\pgfsetdash{}{0pt}%
\pgfsys@defobject{currentmarker}{\pgfqpoint{0.000000in}{-0.020833in}}{\pgfqpoint{0.000000in}{0.000000in}}{%
\pgfpathmoveto{\pgfqpoint{0.000000in}{0.000000in}}%
\pgfpathlineto{\pgfqpoint{0.000000in}{-0.020833in}}%
\pgfusepath{stroke,fill}%
}%
\begin{pgfscope}%
\pgfsys@transformshift{0.583202in}{4.374193in}%
\pgfsys@useobject{currentmarker}{}%
\end{pgfscope}%
\end{pgfscope}%
\begin{pgfscope}%
\pgfsetbuttcap%
\pgfsetroundjoin%
\definecolor{currentfill}{rgb}{0.000000,0.000000,0.000000}%
\pgfsetfillcolor{currentfill}%
\pgfsetlinewidth{0.501875pt}%
\definecolor{currentstroke}{rgb}{0.000000,0.000000,0.000000}%
\pgfsetstrokecolor{currentstroke}%
\pgfsetdash{}{0pt}%
\pgfsys@defobject{currentmarker}{\pgfqpoint{0.000000in}{0.000000in}}{\pgfqpoint{0.000000in}{0.020833in}}{%
\pgfpathmoveto{\pgfqpoint{0.000000in}{0.000000in}}%
\pgfpathlineto{\pgfqpoint{0.000000in}{0.020833in}}%
\pgfusepath{stroke,fill}%
}%
\begin{pgfscope}%
\pgfsys@transformshift{0.695878in}{2.747992in}%
\pgfsys@useobject{currentmarker}{}%
\end{pgfscope}%
\end{pgfscope}%
\begin{pgfscope}%
\pgfsetbuttcap%
\pgfsetroundjoin%
\definecolor{currentfill}{rgb}{0.000000,0.000000,0.000000}%
\pgfsetfillcolor{currentfill}%
\pgfsetlinewidth{0.501875pt}%
\definecolor{currentstroke}{rgb}{0.000000,0.000000,0.000000}%
\pgfsetstrokecolor{currentstroke}%
\pgfsetdash{}{0pt}%
\pgfsys@defobject{currentmarker}{\pgfqpoint{0.000000in}{-0.020833in}}{\pgfqpoint{0.000000in}{0.000000in}}{%
\pgfpathmoveto{\pgfqpoint{0.000000in}{0.000000in}}%
\pgfpathlineto{\pgfqpoint{0.000000in}{-0.020833in}}%
\pgfusepath{stroke,fill}%
}%
\begin{pgfscope}%
\pgfsys@transformshift{0.695878in}{4.374193in}%
\pgfsys@useobject{currentmarker}{}%
\end{pgfscope}%
\end{pgfscope}%
\begin{pgfscope}%
\pgfsetbuttcap%
\pgfsetroundjoin%
\definecolor{currentfill}{rgb}{0.000000,0.000000,0.000000}%
\pgfsetfillcolor{currentfill}%
\pgfsetlinewidth{0.501875pt}%
\definecolor{currentstroke}{rgb}{0.000000,0.000000,0.000000}%
\pgfsetstrokecolor{currentstroke}%
\pgfsetdash{}{0pt}%
\pgfsys@defobject{currentmarker}{\pgfqpoint{0.000000in}{0.000000in}}{\pgfqpoint{0.000000in}{0.020833in}}{%
\pgfpathmoveto{\pgfqpoint{0.000000in}{0.000000in}}%
\pgfpathlineto{\pgfqpoint{0.000000in}{0.020833in}}%
\pgfusepath{stroke,fill}%
}%
\begin{pgfscope}%
\pgfsys@transformshift{0.808555in}{2.747992in}%
\pgfsys@useobject{currentmarker}{}%
\end{pgfscope}%
\end{pgfscope}%
\begin{pgfscope}%
\pgfsetbuttcap%
\pgfsetroundjoin%
\definecolor{currentfill}{rgb}{0.000000,0.000000,0.000000}%
\pgfsetfillcolor{currentfill}%
\pgfsetlinewidth{0.501875pt}%
\definecolor{currentstroke}{rgb}{0.000000,0.000000,0.000000}%
\pgfsetstrokecolor{currentstroke}%
\pgfsetdash{}{0pt}%
\pgfsys@defobject{currentmarker}{\pgfqpoint{0.000000in}{-0.020833in}}{\pgfqpoint{0.000000in}{0.000000in}}{%
\pgfpathmoveto{\pgfqpoint{0.000000in}{0.000000in}}%
\pgfpathlineto{\pgfqpoint{0.000000in}{-0.020833in}}%
\pgfusepath{stroke,fill}%
}%
\begin{pgfscope}%
\pgfsys@transformshift{0.808555in}{4.374193in}%
\pgfsys@useobject{currentmarker}{}%
\end{pgfscope}%
\end{pgfscope}%
\begin{pgfscope}%
\pgfsetbuttcap%
\pgfsetroundjoin%
\definecolor{currentfill}{rgb}{0.000000,0.000000,0.000000}%
\pgfsetfillcolor{currentfill}%
\pgfsetlinewidth{0.501875pt}%
\definecolor{currentstroke}{rgb}{0.000000,0.000000,0.000000}%
\pgfsetstrokecolor{currentstroke}%
\pgfsetdash{}{0pt}%
\pgfsys@defobject{currentmarker}{\pgfqpoint{0.000000in}{0.000000in}}{\pgfqpoint{0.000000in}{0.020833in}}{%
\pgfpathmoveto{\pgfqpoint{0.000000in}{0.000000in}}%
\pgfpathlineto{\pgfqpoint{0.000000in}{0.020833in}}%
\pgfusepath{stroke,fill}%
}%
\begin{pgfscope}%
\pgfsys@transformshift{1.033907in}{2.747992in}%
\pgfsys@useobject{currentmarker}{}%
\end{pgfscope}%
\end{pgfscope}%
\begin{pgfscope}%
\pgfsetbuttcap%
\pgfsetroundjoin%
\definecolor{currentfill}{rgb}{0.000000,0.000000,0.000000}%
\pgfsetfillcolor{currentfill}%
\pgfsetlinewidth{0.501875pt}%
\definecolor{currentstroke}{rgb}{0.000000,0.000000,0.000000}%
\pgfsetstrokecolor{currentstroke}%
\pgfsetdash{}{0pt}%
\pgfsys@defobject{currentmarker}{\pgfqpoint{0.000000in}{-0.020833in}}{\pgfqpoint{0.000000in}{0.000000in}}{%
\pgfpathmoveto{\pgfqpoint{0.000000in}{0.000000in}}%
\pgfpathlineto{\pgfqpoint{0.000000in}{-0.020833in}}%
\pgfusepath{stroke,fill}%
}%
\begin{pgfscope}%
\pgfsys@transformshift{1.033907in}{4.374193in}%
\pgfsys@useobject{currentmarker}{}%
\end{pgfscope}%
\end{pgfscope}%
\begin{pgfscope}%
\pgfsetbuttcap%
\pgfsetroundjoin%
\definecolor{currentfill}{rgb}{0.000000,0.000000,0.000000}%
\pgfsetfillcolor{currentfill}%
\pgfsetlinewidth{0.501875pt}%
\definecolor{currentstroke}{rgb}{0.000000,0.000000,0.000000}%
\pgfsetstrokecolor{currentstroke}%
\pgfsetdash{}{0pt}%
\pgfsys@defobject{currentmarker}{\pgfqpoint{0.000000in}{0.000000in}}{\pgfqpoint{0.000000in}{0.020833in}}{%
\pgfpathmoveto{\pgfqpoint{0.000000in}{0.000000in}}%
\pgfpathlineto{\pgfqpoint{0.000000in}{0.020833in}}%
\pgfusepath{stroke,fill}%
}%
\begin{pgfscope}%
\pgfsys@transformshift{1.146584in}{2.747992in}%
\pgfsys@useobject{currentmarker}{}%
\end{pgfscope}%
\end{pgfscope}%
\begin{pgfscope}%
\pgfsetbuttcap%
\pgfsetroundjoin%
\definecolor{currentfill}{rgb}{0.000000,0.000000,0.000000}%
\pgfsetfillcolor{currentfill}%
\pgfsetlinewidth{0.501875pt}%
\definecolor{currentstroke}{rgb}{0.000000,0.000000,0.000000}%
\pgfsetstrokecolor{currentstroke}%
\pgfsetdash{}{0pt}%
\pgfsys@defobject{currentmarker}{\pgfqpoint{0.000000in}{-0.020833in}}{\pgfqpoint{0.000000in}{0.000000in}}{%
\pgfpathmoveto{\pgfqpoint{0.000000in}{0.000000in}}%
\pgfpathlineto{\pgfqpoint{0.000000in}{-0.020833in}}%
\pgfusepath{stroke,fill}%
}%
\begin{pgfscope}%
\pgfsys@transformshift{1.146584in}{4.374193in}%
\pgfsys@useobject{currentmarker}{}%
\end{pgfscope}%
\end{pgfscope}%
\begin{pgfscope}%
\pgfsetbuttcap%
\pgfsetroundjoin%
\definecolor{currentfill}{rgb}{0.000000,0.000000,0.000000}%
\pgfsetfillcolor{currentfill}%
\pgfsetlinewidth{0.501875pt}%
\definecolor{currentstroke}{rgb}{0.000000,0.000000,0.000000}%
\pgfsetstrokecolor{currentstroke}%
\pgfsetdash{}{0pt}%
\pgfsys@defobject{currentmarker}{\pgfqpoint{0.000000in}{0.000000in}}{\pgfqpoint{0.000000in}{0.020833in}}{%
\pgfpathmoveto{\pgfqpoint{0.000000in}{0.000000in}}%
\pgfpathlineto{\pgfqpoint{0.000000in}{0.020833in}}%
\pgfusepath{stroke,fill}%
}%
\begin{pgfscope}%
\pgfsys@transformshift{1.259260in}{2.747992in}%
\pgfsys@useobject{currentmarker}{}%
\end{pgfscope}%
\end{pgfscope}%
\begin{pgfscope}%
\pgfsetbuttcap%
\pgfsetroundjoin%
\definecolor{currentfill}{rgb}{0.000000,0.000000,0.000000}%
\pgfsetfillcolor{currentfill}%
\pgfsetlinewidth{0.501875pt}%
\definecolor{currentstroke}{rgb}{0.000000,0.000000,0.000000}%
\pgfsetstrokecolor{currentstroke}%
\pgfsetdash{}{0pt}%
\pgfsys@defobject{currentmarker}{\pgfqpoint{0.000000in}{-0.020833in}}{\pgfqpoint{0.000000in}{0.000000in}}{%
\pgfpathmoveto{\pgfqpoint{0.000000in}{0.000000in}}%
\pgfpathlineto{\pgfqpoint{0.000000in}{-0.020833in}}%
\pgfusepath{stroke,fill}%
}%
\begin{pgfscope}%
\pgfsys@transformshift{1.259260in}{4.374193in}%
\pgfsys@useobject{currentmarker}{}%
\end{pgfscope}%
\end{pgfscope}%
\begin{pgfscope}%
\pgfsetbuttcap%
\pgfsetroundjoin%
\definecolor{currentfill}{rgb}{0.000000,0.000000,0.000000}%
\pgfsetfillcolor{currentfill}%
\pgfsetlinewidth{0.501875pt}%
\definecolor{currentstroke}{rgb}{0.000000,0.000000,0.000000}%
\pgfsetstrokecolor{currentstroke}%
\pgfsetdash{}{0pt}%
\pgfsys@defobject{currentmarker}{\pgfqpoint{0.000000in}{0.000000in}}{\pgfqpoint{0.000000in}{0.020833in}}{%
\pgfpathmoveto{\pgfqpoint{0.000000in}{0.000000in}}%
\pgfpathlineto{\pgfqpoint{0.000000in}{0.020833in}}%
\pgfusepath{stroke,fill}%
}%
\begin{pgfscope}%
\pgfsys@transformshift{1.484613in}{2.747992in}%
\pgfsys@useobject{currentmarker}{}%
\end{pgfscope}%
\end{pgfscope}%
\begin{pgfscope}%
\pgfsetbuttcap%
\pgfsetroundjoin%
\definecolor{currentfill}{rgb}{0.000000,0.000000,0.000000}%
\pgfsetfillcolor{currentfill}%
\pgfsetlinewidth{0.501875pt}%
\definecolor{currentstroke}{rgb}{0.000000,0.000000,0.000000}%
\pgfsetstrokecolor{currentstroke}%
\pgfsetdash{}{0pt}%
\pgfsys@defobject{currentmarker}{\pgfqpoint{0.000000in}{-0.020833in}}{\pgfqpoint{0.000000in}{0.000000in}}{%
\pgfpathmoveto{\pgfqpoint{0.000000in}{0.000000in}}%
\pgfpathlineto{\pgfqpoint{0.000000in}{-0.020833in}}%
\pgfusepath{stroke,fill}%
}%
\begin{pgfscope}%
\pgfsys@transformshift{1.484613in}{4.374193in}%
\pgfsys@useobject{currentmarker}{}%
\end{pgfscope}%
\end{pgfscope}%
\begin{pgfscope}%
\pgfsetbuttcap%
\pgfsetroundjoin%
\definecolor{currentfill}{rgb}{0.000000,0.000000,0.000000}%
\pgfsetfillcolor{currentfill}%
\pgfsetlinewidth{0.501875pt}%
\definecolor{currentstroke}{rgb}{0.000000,0.000000,0.000000}%
\pgfsetstrokecolor{currentstroke}%
\pgfsetdash{}{0pt}%
\pgfsys@defobject{currentmarker}{\pgfqpoint{0.000000in}{0.000000in}}{\pgfqpoint{0.000000in}{0.020833in}}{%
\pgfpathmoveto{\pgfqpoint{0.000000in}{0.000000in}}%
\pgfpathlineto{\pgfqpoint{0.000000in}{0.020833in}}%
\pgfusepath{stroke,fill}%
}%
\begin{pgfscope}%
\pgfsys@transformshift{1.597290in}{2.747992in}%
\pgfsys@useobject{currentmarker}{}%
\end{pgfscope}%
\end{pgfscope}%
\begin{pgfscope}%
\pgfsetbuttcap%
\pgfsetroundjoin%
\definecolor{currentfill}{rgb}{0.000000,0.000000,0.000000}%
\pgfsetfillcolor{currentfill}%
\pgfsetlinewidth{0.501875pt}%
\definecolor{currentstroke}{rgb}{0.000000,0.000000,0.000000}%
\pgfsetstrokecolor{currentstroke}%
\pgfsetdash{}{0pt}%
\pgfsys@defobject{currentmarker}{\pgfqpoint{0.000000in}{-0.020833in}}{\pgfqpoint{0.000000in}{0.000000in}}{%
\pgfpathmoveto{\pgfqpoint{0.000000in}{0.000000in}}%
\pgfpathlineto{\pgfqpoint{0.000000in}{-0.020833in}}%
\pgfusepath{stroke,fill}%
}%
\begin{pgfscope}%
\pgfsys@transformshift{1.597290in}{4.374193in}%
\pgfsys@useobject{currentmarker}{}%
\end{pgfscope}%
\end{pgfscope}%
\begin{pgfscope}%
\pgfsetbuttcap%
\pgfsetroundjoin%
\definecolor{currentfill}{rgb}{0.000000,0.000000,0.000000}%
\pgfsetfillcolor{currentfill}%
\pgfsetlinewidth{0.501875pt}%
\definecolor{currentstroke}{rgb}{0.000000,0.000000,0.000000}%
\pgfsetstrokecolor{currentstroke}%
\pgfsetdash{}{0pt}%
\pgfsys@defobject{currentmarker}{\pgfqpoint{0.000000in}{0.000000in}}{\pgfqpoint{0.000000in}{0.020833in}}{%
\pgfpathmoveto{\pgfqpoint{0.000000in}{0.000000in}}%
\pgfpathlineto{\pgfqpoint{0.000000in}{0.020833in}}%
\pgfusepath{stroke,fill}%
}%
\begin{pgfscope}%
\pgfsys@transformshift{1.709966in}{2.747992in}%
\pgfsys@useobject{currentmarker}{}%
\end{pgfscope}%
\end{pgfscope}%
\begin{pgfscope}%
\pgfsetbuttcap%
\pgfsetroundjoin%
\definecolor{currentfill}{rgb}{0.000000,0.000000,0.000000}%
\pgfsetfillcolor{currentfill}%
\pgfsetlinewidth{0.501875pt}%
\definecolor{currentstroke}{rgb}{0.000000,0.000000,0.000000}%
\pgfsetstrokecolor{currentstroke}%
\pgfsetdash{}{0pt}%
\pgfsys@defobject{currentmarker}{\pgfqpoint{0.000000in}{-0.020833in}}{\pgfqpoint{0.000000in}{0.000000in}}{%
\pgfpathmoveto{\pgfqpoint{0.000000in}{0.000000in}}%
\pgfpathlineto{\pgfqpoint{0.000000in}{-0.020833in}}%
\pgfusepath{stroke,fill}%
}%
\begin{pgfscope}%
\pgfsys@transformshift{1.709966in}{4.374193in}%
\pgfsys@useobject{currentmarker}{}%
\end{pgfscope}%
\end{pgfscope}%
\begin{pgfscope}%
\pgfsetbuttcap%
\pgfsetroundjoin%
\definecolor{currentfill}{rgb}{0.000000,0.000000,0.000000}%
\pgfsetfillcolor{currentfill}%
\pgfsetlinewidth{0.501875pt}%
\definecolor{currentstroke}{rgb}{0.000000,0.000000,0.000000}%
\pgfsetstrokecolor{currentstroke}%
\pgfsetdash{}{0pt}%
\pgfsys@defobject{currentmarker}{\pgfqpoint{0.000000in}{0.000000in}}{\pgfqpoint{0.000000in}{0.020833in}}{%
\pgfpathmoveto{\pgfqpoint{0.000000in}{0.000000in}}%
\pgfpathlineto{\pgfqpoint{0.000000in}{0.020833in}}%
\pgfusepath{stroke,fill}%
}%
\begin{pgfscope}%
\pgfsys@transformshift{1.935319in}{2.747992in}%
\pgfsys@useobject{currentmarker}{}%
\end{pgfscope}%
\end{pgfscope}%
\begin{pgfscope}%
\pgfsetbuttcap%
\pgfsetroundjoin%
\definecolor{currentfill}{rgb}{0.000000,0.000000,0.000000}%
\pgfsetfillcolor{currentfill}%
\pgfsetlinewidth{0.501875pt}%
\definecolor{currentstroke}{rgb}{0.000000,0.000000,0.000000}%
\pgfsetstrokecolor{currentstroke}%
\pgfsetdash{}{0pt}%
\pgfsys@defobject{currentmarker}{\pgfqpoint{0.000000in}{-0.020833in}}{\pgfqpoint{0.000000in}{0.000000in}}{%
\pgfpathmoveto{\pgfqpoint{0.000000in}{0.000000in}}%
\pgfpathlineto{\pgfqpoint{0.000000in}{-0.020833in}}%
\pgfusepath{stroke,fill}%
}%
\begin{pgfscope}%
\pgfsys@transformshift{1.935319in}{4.374193in}%
\pgfsys@useobject{currentmarker}{}%
\end{pgfscope}%
\end{pgfscope}%
\begin{pgfscope}%
\pgfsetbuttcap%
\pgfsetroundjoin%
\definecolor{currentfill}{rgb}{0.000000,0.000000,0.000000}%
\pgfsetfillcolor{currentfill}%
\pgfsetlinewidth{0.501875pt}%
\definecolor{currentstroke}{rgb}{0.000000,0.000000,0.000000}%
\pgfsetstrokecolor{currentstroke}%
\pgfsetdash{}{0pt}%
\pgfsys@defobject{currentmarker}{\pgfqpoint{0.000000in}{0.000000in}}{\pgfqpoint{0.000000in}{0.020833in}}{%
\pgfpathmoveto{\pgfqpoint{0.000000in}{0.000000in}}%
\pgfpathlineto{\pgfqpoint{0.000000in}{0.020833in}}%
\pgfusepath{stroke,fill}%
}%
\begin{pgfscope}%
\pgfsys@transformshift{2.047995in}{2.747992in}%
\pgfsys@useobject{currentmarker}{}%
\end{pgfscope}%
\end{pgfscope}%
\begin{pgfscope}%
\pgfsetbuttcap%
\pgfsetroundjoin%
\definecolor{currentfill}{rgb}{0.000000,0.000000,0.000000}%
\pgfsetfillcolor{currentfill}%
\pgfsetlinewidth{0.501875pt}%
\definecolor{currentstroke}{rgb}{0.000000,0.000000,0.000000}%
\pgfsetstrokecolor{currentstroke}%
\pgfsetdash{}{0pt}%
\pgfsys@defobject{currentmarker}{\pgfqpoint{0.000000in}{-0.020833in}}{\pgfqpoint{0.000000in}{0.000000in}}{%
\pgfpathmoveto{\pgfqpoint{0.000000in}{0.000000in}}%
\pgfpathlineto{\pgfqpoint{0.000000in}{-0.020833in}}%
\pgfusepath{stroke,fill}%
}%
\begin{pgfscope}%
\pgfsys@transformshift{2.047995in}{4.374193in}%
\pgfsys@useobject{currentmarker}{}%
\end{pgfscope}%
\end{pgfscope}%
\begin{pgfscope}%
\pgfsetbuttcap%
\pgfsetroundjoin%
\definecolor{currentfill}{rgb}{0.000000,0.000000,0.000000}%
\pgfsetfillcolor{currentfill}%
\pgfsetlinewidth{0.501875pt}%
\definecolor{currentstroke}{rgb}{0.000000,0.000000,0.000000}%
\pgfsetstrokecolor{currentstroke}%
\pgfsetdash{}{0pt}%
\pgfsys@defobject{currentmarker}{\pgfqpoint{0.000000in}{0.000000in}}{\pgfqpoint{0.000000in}{0.020833in}}{%
\pgfpathmoveto{\pgfqpoint{0.000000in}{0.000000in}}%
\pgfpathlineto{\pgfqpoint{0.000000in}{0.020833in}}%
\pgfusepath{stroke,fill}%
}%
\begin{pgfscope}%
\pgfsys@transformshift{2.160672in}{2.747992in}%
\pgfsys@useobject{currentmarker}{}%
\end{pgfscope}%
\end{pgfscope}%
\begin{pgfscope}%
\pgfsetbuttcap%
\pgfsetroundjoin%
\definecolor{currentfill}{rgb}{0.000000,0.000000,0.000000}%
\pgfsetfillcolor{currentfill}%
\pgfsetlinewidth{0.501875pt}%
\definecolor{currentstroke}{rgb}{0.000000,0.000000,0.000000}%
\pgfsetstrokecolor{currentstroke}%
\pgfsetdash{}{0pt}%
\pgfsys@defobject{currentmarker}{\pgfqpoint{0.000000in}{-0.020833in}}{\pgfqpoint{0.000000in}{0.000000in}}{%
\pgfpathmoveto{\pgfqpoint{0.000000in}{0.000000in}}%
\pgfpathlineto{\pgfqpoint{0.000000in}{-0.020833in}}%
\pgfusepath{stroke,fill}%
}%
\begin{pgfscope}%
\pgfsys@transformshift{2.160672in}{4.374193in}%
\pgfsys@useobject{currentmarker}{}%
\end{pgfscope}%
\end{pgfscope}%
\begin{pgfscope}%
\pgfsetbuttcap%
\pgfsetroundjoin%
\definecolor{currentfill}{rgb}{0.000000,0.000000,0.000000}%
\pgfsetfillcolor{currentfill}%
\pgfsetlinewidth{0.501875pt}%
\definecolor{currentstroke}{rgb}{0.000000,0.000000,0.000000}%
\pgfsetstrokecolor{currentstroke}%
\pgfsetdash{}{0pt}%
\pgfsys@defobject{currentmarker}{\pgfqpoint{0.000000in}{0.000000in}}{\pgfqpoint{0.000000in}{0.020833in}}{%
\pgfpathmoveto{\pgfqpoint{0.000000in}{0.000000in}}%
\pgfpathlineto{\pgfqpoint{0.000000in}{0.020833in}}%
\pgfusepath{stroke,fill}%
}%
\begin{pgfscope}%
\pgfsys@transformshift{2.386025in}{2.747992in}%
\pgfsys@useobject{currentmarker}{}%
\end{pgfscope}%
\end{pgfscope}%
\begin{pgfscope}%
\pgfsetbuttcap%
\pgfsetroundjoin%
\definecolor{currentfill}{rgb}{0.000000,0.000000,0.000000}%
\pgfsetfillcolor{currentfill}%
\pgfsetlinewidth{0.501875pt}%
\definecolor{currentstroke}{rgb}{0.000000,0.000000,0.000000}%
\pgfsetstrokecolor{currentstroke}%
\pgfsetdash{}{0pt}%
\pgfsys@defobject{currentmarker}{\pgfqpoint{0.000000in}{-0.020833in}}{\pgfqpoint{0.000000in}{0.000000in}}{%
\pgfpathmoveto{\pgfqpoint{0.000000in}{0.000000in}}%
\pgfpathlineto{\pgfqpoint{0.000000in}{-0.020833in}}%
\pgfusepath{stroke,fill}%
}%
\begin{pgfscope}%
\pgfsys@transformshift{2.386025in}{4.374193in}%
\pgfsys@useobject{currentmarker}{}%
\end{pgfscope}%
\end{pgfscope}%
\begin{pgfscope}%
\pgfsetbuttcap%
\pgfsetroundjoin%
\definecolor{currentfill}{rgb}{0.000000,0.000000,0.000000}%
\pgfsetfillcolor{currentfill}%
\pgfsetlinewidth{0.501875pt}%
\definecolor{currentstroke}{rgb}{0.000000,0.000000,0.000000}%
\pgfsetstrokecolor{currentstroke}%
\pgfsetdash{}{0pt}%
\pgfsys@defobject{currentmarker}{\pgfqpoint{0.000000in}{0.000000in}}{\pgfqpoint{0.000000in}{0.020833in}}{%
\pgfpathmoveto{\pgfqpoint{0.000000in}{0.000000in}}%
\pgfpathlineto{\pgfqpoint{0.000000in}{0.020833in}}%
\pgfusepath{stroke,fill}%
}%
\begin{pgfscope}%
\pgfsys@transformshift{2.498701in}{2.747992in}%
\pgfsys@useobject{currentmarker}{}%
\end{pgfscope}%
\end{pgfscope}%
\begin{pgfscope}%
\pgfsetbuttcap%
\pgfsetroundjoin%
\definecolor{currentfill}{rgb}{0.000000,0.000000,0.000000}%
\pgfsetfillcolor{currentfill}%
\pgfsetlinewidth{0.501875pt}%
\definecolor{currentstroke}{rgb}{0.000000,0.000000,0.000000}%
\pgfsetstrokecolor{currentstroke}%
\pgfsetdash{}{0pt}%
\pgfsys@defobject{currentmarker}{\pgfqpoint{0.000000in}{-0.020833in}}{\pgfqpoint{0.000000in}{0.000000in}}{%
\pgfpathmoveto{\pgfqpoint{0.000000in}{0.000000in}}%
\pgfpathlineto{\pgfqpoint{0.000000in}{-0.020833in}}%
\pgfusepath{stroke,fill}%
}%
\begin{pgfscope}%
\pgfsys@transformshift{2.498701in}{4.374193in}%
\pgfsys@useobject{currentmarker}{}%
\end{pgfscope}%
\end{pgfscope}%
\begin{pgfscope}%
\pgfsetbuttcap%
\pgfsetroundjoin%
\definecolor{currentfill}{rgb}{0.000000,0.000000,0.000000}%
\pgfsetfillcolor{currentfill}%
\pgfsetlinewidth{0.501875pt}%
\definecolor{currentstroke}{rgb}{0.000000,0.000000,0.000000}%
\pgfsetstrokecolor{currentstroke}%
\pgfsetdash{}{0pt}%
\pgfsys@defobject{currentmarker}{\pgfqpoint{0.000000in}{0.000000in}}{\pgfqpoint{0.000000in}{0.020833in}}{%
\pgfpathmoveto{\pgfqpoint{0.000000in}{0.000000in}}%
\pgfpathlineto{\pgfqpoint{0.000000in}{0.020833in}}%
\pgfusepath{stroke,fill}%
}%
\begin{pgfscope}%
\pgfsys@transformshift{2.611378in}{2.747992in}%
\pgfsys@useobject{currentmarker}{}%
\end{pgfscope}%
\end{pgfscope}%
\begin{pgfscope}%
\pgfsetbuttcap%
\pgfsetroundjoin%
\definecolor{currentfill}{rgb}{0.000000,0.000000,0.000000}%
\pgfsetfillcolor{currentfill}%
\pgfsetlinewidth{0.501875pt}%
\definecolor{currentstroke}{rgb}{0.000000,0.000000,0.000000}%
\pgfsetstrokecolor{currentstroke}%
\pgfsetdash{}{0pt}%
\pgfsys@defobject{currentmarker}{\pgfqpoint{0.000000in}{-0.020833in}}{\pgfqpoint{0.000000in}{0.000000in}}{%
\pgfpathmoveto{\pgfqpoint{0.000000in}{0.000000in}}%
\pgfpathlineto{\pgfqpoint{0.000000in}{-0.020833in}}%
\pgfusepath{stroke,fill}%
}%
\begin{pgfscope}%
\pgfsys@transformshift{2.611378in}{4.374193in}%
\pgfsys@useobject{currentmarker}{}%
\end{pgfscope}%
\end{pgfscope}%
\begin{pgfscope}%
\pgfsetbuttcap%
\pgfsetroundjoin%
\definecolor{currentfill}{rgb}{0.000000,0.000000,0.000000}%
\pgfsetfillcolor{currentfill}%
\pgfsetlinewidth{0.501875pt}%
\definecolor{currentstroke}{rgb}{0.000000,0.000000,0.000000}%
\pgfsetstrokecolor{currentstroke}%
\pgfsetdash{}{0pt}%
\pgfsys@defobject{currentmarker}{\pgfqpoint{0.000000in}{0.000000in}}{\pgfqpoint{0.000000in}{0.020833in}}{%
\pgfpathmoveto{\pgfqpoint{0.000000in}{0.000000in}}%
\pgfpathlineto{\pgfqpoint{0.000000in}{0.020833in}}%
\pgfusepath{stroke,fill}%
}%
\begin{pgfscope}%
\pgfsys@transformshift{2.724054in}{2.747992in}%
\pgfsys@useobject{currentmarker}{}%
\end{pgfscope}%
\end{pgfscope}%
\begin{pgfscope}%
\pgfsetbuttcap%
\pgfsetroundjoin%
\definecolor{currentfill}{rgb}{0.000000,0.000000,0.000000}%
\pgfsetfillcolor{currentfill}%
\pgfsetlinewidth{0.501875pt}%
\definecolor{currentstroke}{rgb}{0.000000,0.000000,0.000000}%
\pgfsetstrokecolor{currentstroke}%
\pgfsetdash{}{0pt}%
\pgfsys@defobject{currentmarker}{\pgfqpoint{0.000000in}{-0.020833in}}{\pgfqpoint{0.000000in}{0.000000in}}{%
\pgfpathmoveto{\pgfqpoint{0.000000in}{0.000000in}}%
\pgfpathlineto{\pgfqpoint{0.000000in}{-0.020833in}}%
\pgfusepath{stroke,fill}%
}%
\begin{pgfscope}%
\pgfsys@transformshift{2.724054in}{4.374193in}%
\pgfsys@useobject{currentmarker}{}%
\end{pgfscope}%
\end{pgfscope}%
\begin{pgfscope}%
\definecolor{textcolor}{rgb}{0.000000,0.000000,0.000000}%
\pgfsetstrokecolor{textcolor}%
\pgfsetfillcolor{textcolor}%
\pgftext[x=1.608557in,y=2.509413in,,top]{\color{textcolor}\rmfamily\fontsize{10.000000}{12.000000}\selectfont \(\displaystyle K\)}%
\end{pgfscope}%
\begin{pgfscope}%
\pgfsetbuttcap%
\pgfsetroundjoin%
\definecolor{currentfill}{rgb}{0.000000,0.000000,0.000000}%
\pgfsetfillcolor{currentfill}%
\pgfsetlinewidth{0.501875pt}%
\definecolor{currentstroke}{rgb}{0.000000,0.000000,0.000000}%
\pgfsetstrokecolor{currentstroke}%
\pgfsetdash{}{0pt}%
\pgfsys@defobject{currentmarker}{\pgfqpoint{0.000000in}{0.000000in}}{\pgfqpoint{0.041667in}{0.000000in}}{%
\pgfpathmoveto{\pgfqpoint{0.000000in}{0.000000in}}%
\pgfpathlineto{\pgfqpoint{0.041667in}{0.000000in}}%
\pgfusepath{stroke,fill}%
}%
\begin{pgfscope}%
\pgfsys@transformshift{0.470525in}{3.068095in}%
\pgfsys@useobject{currentmarker}{}%
\end{pgfscope}%
\end{pgfscope}%
\begin{pgfscope}%
\pgfsetbuttcap%
\pgfsetroundjoin%
\definecolor{currentfill}{rgb}{0.000000,0.000000,0.000000}%
\pgfsetfillcolor{currentfill}%
\pgfsetlinewidth{0.501875pt}%
\definecolor{currentstroke}{rgb}{0.000000,0.000000,0.000000}%
\pgfsetstrokecolor{currentstroke}%
\pgfsetdash{}{0pt}%
\pgfsys@defobject{currentmarker}{\pgfqpoint{-0.041667in}{0.000000in}}{\pgfqpoint{-0.000000in}{0.000000in}}{%
\pgfpathmoveto{\pgfqpoint{-0.000000in}{0.000000in}}%
\pgfpathlineto{\pgfqpoint{-0.041667in}{0.000000in}}%
\pgfusepath{stroke,fill}%
}%
\begin{pgfscope}%
\pgfsys@transformshift{2.746589in}{3.068095in}%
\pgfsys@useobject{currentmarker}{}%
\end{pgfscope}%
\end{pgfscope}%
\begin{pgfscope}%
\definecolor{textcolor}{rgb}{0.000000,0.000000,0.000000}%
\pgfsetstrokecolor{textcolor}%
\pgfsetfillcolor{textcolor}%
\pgftext[x=0.244444in, y=3.015333in, left, base]{\color{textcolor}\rmfamily\fontsize{10.000000}{12.000000}\selectfont \(\displaystyle {0.8}\)}%
\end{pgfscope}%
\begin{pgfscope}%
\pgfsetbuttcap%
\pgfsetroundjoin%
\definecolor{currentfill}{rgb}{0.000000,0.000000,0.000000}%
\pgfsetfillcolor{currentfill}%
\pgfsetlinewidth{0.501875pt}%
\definecolor{currentstroke}{rgb}{0.000000,0.000000,0.000000}%
\pgfsetstrokecolor{currentstroke}%
\pgfsetdash{}{0pt}%
\pgfsys@defobject{currentmarker}{\pgfqpoint{0.000000in}{0.000000in}}{\pgfqpoint{0.041667in}{0.000000in}}{%
\pgfpathmoveto{\pgfqpoint{0.000000in}{0.000000in}}%
\pgfpathlineto{\pgfqpoint{0.041667in}{0.000000in}}%
\pgfusepath{stroke,fill}%
}%
\begin{pgfscope}%
\pgfsys@transformshift{0.470525in}{3.717107in}%
\pgfsys@useobject{currentmarker}{}%
\end{pgfscope}%
\end{pgfscope}%
\begin{pgfscope}%
\pgfsetbuttcap%
\pgfsetroundjoin%
\definecolor{currentfill}{rgb}{0.000000,0.000000,0.000000}%
\pgfsetfillcolor{currentfill}%
\pgfsetlinewidth{0.501875pt}%
\definecolor{currentstroke}{rgb}{0.000000,0.000000,0.000000}%
\pgfsetstrokecolor{currentstroke}%
\pgfsetdash{}{0pt}%
\pgfsys@defobject{currentmarker}{\pgfqpoint{-0.041667in}{0.000000in}}{\pgfqpoint{-0.000000in}{0.000000in}}{%
\pgfpathmoveto{\pgfqpoint{-0.000000in}{0.000000in}}%
\pgfpathlineto{\pgfqpoint{-0.041667in}{0.000000in}}%
\pgfusepath{stroke,fill}%
}%
\begin{pgfscope}%
\pgfsys@transformshift{2.746589in}{3.717107in}%
\pgfsys@useobject{currentmarker}{}%
\end{pgfscope}%
\end{pgfscope}%
\begin{pgfscope}%
\definecolor{textcolor}{rgb}{0.000000,0.000000,0.000000}%
\pgfsetstrokecolor{textcolor}%
\pgfsetfillcolor{textcolor}%
\pgftext[x=0.244444in, y=3.664345in, left, base]{\color{textcolor}\rmfamily\fontsize{10.000000}{12.000000}\selectfont \(\displaystyle {0.9}\)}%
\end{pgfscope}%
\begin{pgfscope}%
\pgfsetbuttcap%
\pgfsetroundjoin%
\definecolor{currentfill}{rgb}{0.000000,0.000000,0.000000}%
\pgfsetfillcolor{currentfill}%
\pgfsetlinewidth{0.501875pt}%
\definecolor{currentstroke}{rgb}{0.000000,0.000000,0.000000}%
\pgfsetstrokecolor{currentstroke}%
\pgfsetdash{}{0pt}%
\pgfsys@defobject{currentmarker}{\pgfqpoint{0.000000in}{0.000000in}}{\pgfqpoint{0.041667in}{0.000000in}}{%
\pgfpathmoveto{\pgfqpoint{0.000000in}{0.000000in}}%
\pgfpathlineto{\pgfqpoint{0.041667in}{0.000000in}}%
\pgfusepath{stroke,fill}%
}%
\begin{pgfscope}%
\pgfsys@transformshift{0.470525in}{4.366119in}%
\pgfsys@useobject{currentmarker}{}%
\end{pgfscope}%
\end{pgfscope}%
\begin{pgfscope}%
\pgfsetbuttcap%
\pgfsetroundjoin%
\definecolor{currentfill}{rgb}{0.000000,0.000000,0.000000}%
\pgfsetfillcolor{currentfill}%
\pgfsetlinewidth{0.501875pt}%
\definecolor{currentstroke}{rgb}{0.000000,0.000000,0.000000}%
\pgfsetstrokecolor{currentstroke}%
\pgfsetdash{}{0pt}%
\pgfsys@defobject{currentmarker}{\pgfqpoint{-0.041667in}{0.000000in}}{\pgfqpoint{-0.000000in}{0.000000in}}{%
\pgfpathmoveto{\pgfqpoint{-0.000000in}{0.000000in}}%
\pgfpathlineto{\pgfqpoint{-0.041667in}{0.000000in}}%
\pgfusepath{stroke,fill}%
}%
\begin{pgfscope}%
\pgfsys@transformshift{2.746589in}{4.366119in}%
\pgfsys@useobject{currentmarker}{}%
\end{pgfscope}%
\end{pgfscope}%
\begin{pgfscope}%
\definecolor{textcolor}{rgb}{0.000000,0.000000,0.000000}%
\pgfsetstrokecolor{textcolor}%
\pgfsetfillcolor{textcolor}%
\pgftext[x=0.244444in, y=4.313358in, left, base]{\color{textcolor}\rmfamily\fontsize{10.000000}{12.000000}\selectfont \(\displaystyle {1.0}\)}%
\end{pgfscope}%
\begin{pgfscope}%
\pgfsetbuttcap%
\pgfsetroundjoin%
\definecolor{currentfill}{rgb}{0.000000,0.000000,0.000000}%
\pgfsetfillcolor{currentfill}%
\pgfsetlinewidth{0.501875pt}%
\definecolor{currentstroke}{rgb}{0.000000,0.000000,0.000000}%
\pgfsetstrokecolor{currentstroke}%
\pgfsetdash{}{0pt}%
\pgfsys@defobject{currentmarker}{\pgfqpoint{0.000000in}{0.000000in}}{\pgfqpoint{0.020833in}{0.000000in}}{%
\pgfpathmoveto{\pgfqpoint{0.000000in}{0.000000in}}%
\pgfpathlineto{\pgfqpoint{0.020833in}{0.000000in}}%
\pgfusepath{stroke,fill}%
}%
\begin{pgfscope}%
\pgfsys@transformshift{0.470525in}{2.808490in}%
\pgfsys@useobject{currentmarker}{}%
\end{pgfscope}%
\end{pgfscope}%
\begin{pgfscope}%
\pgfsetbuttcap%
\pgfsetroundjoin%
\definecolor{currentfill}{rgb}{0.000000,0.000000,0.000000}%
\pgfsetfillcolor{currentfill}%
\pgfsetlinewidth{0.501875pt}%
\definecolor{currentstroke}{rgb}{0.000000,0.000000,0.000000}%
\pgfsetstrokecolor{currentstroke}%
\pgfsetdash{}{0pt}%
\pgfsys@defobject{currentmarker}{\pgfqpoint{-0.020833in}{0.000000in}}{\pgfqpoint{-0.000000in}{0.000000in}}{%
\pgfpathmoveto{\pgfqpoint{-0.000000in}{0.000000in}}%
\pgfpathlineto{\pgfqpoint{-0.020833in}{0.000000in}}%
\pgfusepath{stroke,fill}%
}%
\begin{pgfscope}%
\pgfsys@transformshift{2.746589in}{2.808490in}%
\pgfsys@useobject{currentmarker}{}%
\end{pgfscope}%
\end{pgfscope}%
\begin{pgfscope}%
\pgfsetbuttcap%
\pgfsetroundjoin%
\definecolor{currentfill}{rgb}{0.000000,0.000000,0.000000}%
\pgfsetfillcolor{currentfill}%
\pgfsetlinewidth{0.501875pt}%
\definecolor{currentstroke}{rgb}{0.000000,0.000000,0.000000}%
\pgfsetstrokecolor{currentstroke}%
\pgfsetdash{}{0pt}%
\pgfsys@defobject{currentmarker}{\pgfqpoint{0.000000in}{0.000000in}}{\pgfqpoint{0.020833in}{0.000000in}}{%
\pgfpathmoveto{\pgfqpoint{0.000000in}{0.000000in}}%
\pgfpathlineto{\pgfqpoint{0.020833in}{0.000000in}}%
\pgfusepath{stroke,fill}%
}%
\begin{pgfscope}%
\pgfsys@transformshift{0.470525in}{2.938292in}%
\pgfsys@useobject{currentmarker}{}%
\end{pgfscope}%
\end{pgfscope}%
\begin{pgfscope}%
\pgfsetbuttcap%
\pgfsetroundjoin%
\definecolor{currentfill}{rgb}{0.000000,0.000000,0.000000}%
\pgfsetfillcolor{currentfill}%
\pgfsetlinewidth{0.501875pt}%
\definecolor{currentstroke}{rgb}{0.000000,0.000000,0.000000}%
\pgfsetstrokecolor{currentstroke}%
\pgfsetdash{}{0pt}%
\pgfsys@defobject{currentmarker}{\pgfqpoint{-0.020833in}{0.000000in}}{\pgfqpoint{-0.000000in}{0.000000in}}{%
\pgfpathmoveto{\pgfqpoint{-0.000000in}{0.000000in}}%
\pgfpathlineto{\pgfqpoint{-0.020833in}{0.000000in}}%
\pgfusepath{stroke,fill}%
}%
\begin{pgfscope}%
\pgfsys@transformshift{2.746589in}{2.938292in}%
\pgfsys@useobject{currentmarker}{}%
\end{pgfscope}%
\end{pgfscope}%
\begin{pgfscope}%
\pgfsetbuttcap%
\pgfsetroundjoin%
\definecolor{currentfill}{rgb}{0.000000,0.000000,0.000000}%
\pgfsetfillcolor{currentfill}%
\pgfsetlinewidth{0.501875pt}%
\definecolor{currentstroke}{rgb}{0.000000,0.000000,0.000000}%
\pgfsetstrokecolor{currentstroke}%
\pgfsetdash{}{0pt}%
\pgfsys@defobject{currentmarker}{\pgfqpoint{0.000000in}{0.000000in}}{\pgfqpoint{0.020833in}{0.000000in}}{%
\pgfpathmoveto{\pgfqpoint{0.000000in}{0.000000in}}%
\pgfpathlineto{\pgfqpoint{0.020833in}{0.000000in}}%
\pgfusepath{stroke,fill}%
}%
\begin{pgfscope}%
\pgfsys@transformshift{0.470525in}{3.197897in}%
\pgfsys@useobject{currentmarker}{}%
\end{pgfscope}%
\end{pgfscope}%
\begin{pgfscope}%
\pgfsetbuttcap%
\pgfsetroundjoin%
\definecolor{currentfill}{rgb}{0.000000,0.000000,0.000000}%
\pgfsetfillcolor{currentfill}%
\pgfsetlinewidth{0.501875pt}%
\definecolor{currentstroke}{rgb}{0.000000,0.000000,0.000000}%
\pgfsetstrokecolor{currentstroke}%
\pgfsetdash{}{0pt}%
\pgfsys@defobject{currentmarker}{\pgfqpoint{-0.020833in}{0.000000in}}{\pgfqpoint{-0.000000in}{0.000000in}}{%
\pgfpathmoveto{\pgfqpoint{-0.000000in}{0.000000in}}%
\pgfpathlineto{\pgfqpoint{-0.020833in}{0.000000in}}%
\pgfusepath{stroke,fill}%
}%
\begin{pgfscope}%
\pgfsys@transformshift{2.746589in}{3.197897in}%
\pgfsys@useobject{currentmarker}{}%
\end{pgfscope}%
\end{pgfscope}%
\begin{pgfscope}%
\pgfsetbuttcap%
\pgfsetroundjoin%
\definecolor{currentfill}{rgb}{0.000000,0.000000,0.000000}%
\pgfsetfillcolor{currentfill}%
\pgfsetlinewidth{0.501875pt}%
\definecolor{currentstroke}{rgb}{0.000000,0.000000,0.000000}%
\pgfsetstrokecolor{currentstroke}%
\pgfsetdash{}{0pt}%
\pgfsys@defobject{currentmarker}{\pgfqpoint{0.000000in}{0.000000in}}{\pgfqpoint{0.020833in}{0.000000in}}{%
\pgfpathmoveto{\pgfqpoint{0.000000in}{0.000000in}}%
\pgfpathlineto{\pgfqpoint{0.020833in}{0.000000in}}%
\pgfusepath{stroke,fill}%
}%
\begin{pgfscope}%
\pgfsys@transformshift{0.470525in}{3.327700in}%
\pgfsys@useobject{currentmarker}{}%
\end{pgfscope}%
\end{pgfscope}%
\begin{pgfscope}%
\pgfsetbuttcap%
\pgfsetroundjoin%
\definecolor{currentfill}{rgb}{0.000000,0.000000,0.000000}%
\pgfsetfillcolor{currentfill}%
\pgfsetlinewidth{0.501875pt}%
\definecolor{currentstroke}{rgb}{0.000000,0.000000,0.000000}%
\pgfsetstrokecolor{currentstroke}%
\pgfsetdash{}{0pt}%
\pgfsys@defobject{currentmarker}{\pgfqpoint{-0.020833in}{0.000000in}}{\pgfqpoint{-0.000000in}{0.000000in}}{%
\pgfpathmoveto{\pgfqpoint{-0.000000in}{0.000000in}}%
\pgfpathlineto{\pgfqpoint{-0.020833in}{0.000000in}}%
\pgfusepath{stroke,fill}%
}%
\begin{pgfscope}%
\pgfsys@transformshift{2.746589in}{3.327700in}%
\pgfsys@useobject{currentmarker}{}%
\end{pgfscope}%
\end{pgfscope}%
\begin{pgfscope}%
\pgfsetbuttcap%
\pgfsetroundjoin%
\definecolor{currentfill}{rgb}{0.000000,0.000000,0.000000}%
\pgfsetfillcolor{currentfill}%
\pgfsetlinewidth{0.501875pt}%
\definecolor{currentstroke}{rgb}{0.000000,0.000000,0.000000}%
\pgfsetstrokecolor{currentstroke}%
\pgfsetdash{}{0pt}%
\pgfsys@defobject{currentmarker}{\pgfqpoint{0.000000in}{0.000000in}}{\pgfqpoint{0.020833in}{0.000000in}}{%
\pgfpathmoveto{\pgfqpoint{0.000000in}{0.000000in}}%
\pgfpathlineto{\pgfqpoint{0.020833in}{0.000000in}}%
\pgfusepath{stroke,fill}%
}%
\begin{pgfscope}%
\pgfsys@transformshift{0.470525in}{3.457502in}%
\pgfsys@useobject{currentmarker}{}%
\end{pgfscope}%
\end{pgfscope}%
\begin{pgfscope}%
\pgfsetbuttcap%
\pgfsetroundjoin%
\definecolor{currentfill}{rgb}{0.000000,0.000000,0.000000}%
\pgfsetfillcolor{currentfill}%
\pgfsetlinewidth{0.501875pt}%
\definecolor{currentstroke}{rgb}{0.000000,0.000000,0.000000}%
\pgfsetstrokecolor{currentstroke}%
\pgfsetdash{}{0pt}%
\pgfsys@defobject{currentmarker}{\pgfqpoint{-0.020833in}{0.000000in}}{\pgfqpoint{-0.000000in}{0.000000in}}{%
\pgfpathmoveto{\pgfqpoint{-0.000000in}{0.000000in}}%
\pgfpathlineto{\pgfqpoint{-0.020833in}{0.000000in}}%
\pgfusepath{stroke,fill}%
}%
\begin{pgfscope}%
\pgfsys@transformshift{2.746589in}{3.457502in}%
\pgfsys@useobject{currentmarker}{}%
\end{pgfscope}%
\end{pgfscope}%
\begin{pgfscope}%
\pgfsetbuttcap%
\pgfsetroundjoin%
\definecolor{currentfill}{rgb}{0.000000,0.000000,0.000000}%
\pgfsetfillcolor{currentfill}%
\pgfsetlinewidth{0.501875pt}%
\definecolor{currentstroke}{rgb}{0.000000,0.000000,0.000000}%
\pgfsetstrokecolor{currentstroke}%
\pgfsetdash{}{0pt}%
\pgfsys@defobject{currentmarker}{\pgfqpoint{0.000000in}{0.000000in}}{\pgfqpoint{0.020833in}{0.000000in}}{%
\pgfpathmoveto{\pgfqpoint{0.000000in}{0.000000in}}%
\pgfpathlineto{\pgfqpoint{0.020833in}{0.000000in}}%
\pgfusepath{stroke,fill}%
}%
\begin{pgfscope}%
\pgfsys@transformshift{0.470525in}{3.587305in}%
\pgfsys@useobject{currentmarker}{}%
\end{pgfscope}%
\end{pgfscope}%
\begin{pgfscope}%
\pgfsetbuttcap%
\pgfsetroundjoin%
\definecolor{currentfill}{rgb}{0.000000,0.000000,0.000000}%
\pgfsetfillcolor{currentfill}%
\pgfsetlinewidth{0.501875pt}%
\definecolor{currentstroke}{rgb}{0.000000,0.000000,0.000000}%
\pgfsetstrokecolor{currentstroke}%
\pgfsetdash{}{0pt}%
\pgfsys@defobject{currentmarker}{\pgfqpoint{-0.020833in}{0.000000in}}{\pgfqpoint{-0.000000in}{0.000000in}}{%
\pgfpathmoveto{\pgfqpoint{-0.000000in}{0.000000in}}%
\pgfpathlineto{\pgfqpoint{-0.020833in}{0.000000in}}%
\pgfusepath{stroke,fill}%
}%
\begin{pgfscope}%
\pgfsys@transformshift{2.746589in}{3.587305in}%
\pgfsys@useobject{currentmarker}{}%
\end{pgfscope}%
\end{pgfscope}%
\begin{pgfscope}%
\pgfsetbuttcap%
\pgfsetroundjoin%
\definecolor{currentfill}{rgb}{0.000000,0.000000,0.000000}%
\pgfsetfillcolor{currentfill}%
\pgfsetlinewidth{0.501875pt}%
\definecolor{currentstroke}{rgb}{0.000000,0.000000,0.000000}%
\pgfsetstrokecolor{currentstroke}%
\pgfsetdash{}{0pt}%
\pgfsys@defobject{currentmarker}{\pgfqpoint{0.000000in}{0.000000in}}{\pgfqpoint{0.020833in}{0.000000in}}{%
\pgfpathmoveto{\pgfqpoint{0.000000in}{0.000000in}}%
\pgfpathlineto{\pgfqpoint{0.020833in}{0.000000in}}%
\pgfusepath{stroke,fill}%
}%
\begin{pgfscope}%
\pgfsys@transformshift{0.470525in}{3.846909in}%
\pgfsys@useobject{currentmarker}{}%
\end{pgfscope}%
\end{pgfscope}%
\begin{pgfscope}%
\pgfsetbuttcap%
\pgfsetroundjoin%
\definecolor{currentfill}{rgb}{0.000000,0.000000,0.000000}%
\pgfsetfillcolor{currentfill}%
\pgfsetlinewidth{0.501875pt}%
\definecolor{currentstroke}{rgb}{0.000000,0.000000,0.000000}%
\pgfsetstrokecolor{currentstroke}%
\pgfsetdash{}{0pt}%
\pgfsys@defobject{currentmarker}{\pgfqpoint{-0.020833in}{0.000000in}}{\pgfqpoint{-0.000000in}{0.000000in}}{%
\pgfpathmoveto{\pgfqpoint{-0.000000in}{0.000000in}}%
\pgfpathlineto{\pgfqpoint{-0.020833in}{0.000000in}}%
\pgfusepath{stroke,fill}%
}%
\begin{pgfscope}%
\pgfsys@transformshift{2.746589in}{3.846909in}%
\pgfsys@useobject{currentmarker}{}%
\end{pgfscope}%
\end{pgfscope}%
\begin{pgfscope}%
\pgfsetbuttcap%
\pgfsetroundjoin%
\definecolor{currentfill}{rgb}{0.000000,0.000000,0.000000}%
\pgfsetfillcolor{currentfill}%
\pgfsetlinewidth{0.501875pt}%
\definecolor{currentstroke}{rgb}{0.000000,0.000000,0.000000}%
\pgfsetstrokecolor{currentstroke}%
\pgfsetdash{}{0pt}%
\pgfsys@defobject{currentmarker}{\pgfqpoint{0.000000in}{0.000000in}}{\pgfqpoint{0.020833in}{0.000000in}}{%
\pgfpathmoveto{\pgfqpoint{0.000000in}{0.000000in}}%
\pgfpathlineto{\pgfqpoint{0.020833in}{0.000000in}}%
\pgfusepath{stroke,fill}%
}%
\begin{pgfscope}%
\pgfsys@transformshift{0.470525in}{3.976712in}%
\pgfsys@useobject{currentmarker}{}%
\end{pgfscope}%
\end{pgfscope}%
\begin{pgfscope}%
\pgfsetbuttcap%
\pgfsetroundjoin%
\definecolor{currentfill}{rgb}{0.000000,0.000000,0.000000}%
\pgfsetfillcolor{currentfill}%
\pgfsetlinewidth{0.501875pt}%
\definecolor{currentstroke}{rgb}{0.000000,0.000000,0.000000}%
\pgfsetstrokecolor{currentstroke}%
\pgfsetdash{}{0pt}%
\pgfsys@defobject{currentmarker}{\pgfqpoint{-0.020833in}{0.000000in}}{\pgfqpoint{-0.000000in}{0.000000in}}{%
\pgfpathmoveto{\pgfqpoint{-0.000000in}{0.000000in}}%
\pgfpathlineto{\pgfqpoint{-0.020833in}{0.000000in}}%
\pgfusepath{stroke,fill}%
}%
\begin{pgfscope}%
\pgfsys@transformshift{2.746589in}{3.976712in}%
\pgfsys@useobject{currentmarker}{}%
\end{pgfscope}%
\end{pgfscope}%
\begin{pgfscope}%
\pgfsetbuttcap%
\pgfsetroundjoin%
\definecolor{currentfill}{rgb}{0.000000,0.000000,0.000000}%
\pgfsetfillcolor{currentfill}%
\pgfsetlinewidth{0.501875pt}%
\definecolor{currentstroke}{rgb}{0.000000,0.000000,0.000000}%
\pgfsetstrokecolor{currentstroke}%
\pgfsetdash{}{0pt}%
\pgfsys@defobject{currentmarker}{\pgfqpoint{0.000000in}{0.000000in}}{\pgfqpoint{0.020833in}{0.000000in}}{%
\pgfpathmoveto{\pgfqpoint{0.000000in}{0.000000in}}%
\pgfpathlineto{\pgfqpoint{0.020833in}{0.000000in}}%
\pgfusepath{stroke,fill}%
}%
\begin{pgfscope}%
\pgfsys@transformshift{0.470525in}{4.106514in}%
\pgfsys@useobject{currentmarker}{}%
\end{pgfscope}%
\end{pgfscope}%
\begin{pgfscope}%
\pgfsetbuttcap%
\pgfsetroundjoin%
\definecolor{currentfill}{rgb}{0.000000,0.000000,0.000000}%
\pgfsetfillcolor{currentfill}%
\pgfsetlinewidth{0.501875pt}%
\definecolor{currentstroke}{rgb}{0.000000,0.000000,0.000000}%
\pgfsetstrokecolor{currentstroke}%
\pgfsetdash{}{0pt}%
\pgfsys@defobject{currentmarker}{\pgfqpoint{-0.020833in}{0.000000in}}{\pgfqpoint{-0.000000in}{0.000000in}}{%
\pgfpathmoveto{\pgfqpoint{-0.000000in}{0.000000in}}%
\pgfpathlineto{\pgfqpoint{-0.020833in}{0.000000in}}%
\pgfusepath{stroke,fill}%
}%
\begin{pgfscope}%
\pgfsys@transformshift{2.746589in}{4.106514in}%
\pgfsys@useobject{currentmarker}{}%
\end{pgfscope}%
\end{pgfscope}%
\begin{pgfscope}%
\pgfsetbuttcap%
\pgfsetroundjoin%
\definecolor{currentfill}{rgb}{0.000000,0.000000,0.000000}%
\pgfsetfillcolor{currentfill}%
\pgfsetlinewidth{0.501875pt}%
\definecolor{currentstroke}{rgb}{0.000000,0.000000,0.000000}%
\pgfsetstrokecolor{currentstroke}%
\pgfsetdash{}{0pt}%
\pgfsys@defobject{currentmarker}{\pgfqpoint{0.000000in}{0.000000in}}{\pgfqpoint{0.020833in}{0.000000in}}{%
\pgfpathmoveto{\pgfqpoint{0.000000in}{0.000000in}}%
\pgfpathlineto{\pgfqpoint{0.020833in}{0.000000in}}%
\pgfusepath{stroke,fill}%
}%
\begin{pgfscope}%
\pgfsys@transformshift{0.470525in}{4.236317in}%
\pgfsys@useobject{currentmarker}{}%
\end{pgfscope}%
\end{pgfscope}%
\begin{pgfscope}%
\pgfsetbuttcap%
\pgfsetroundjoin%
\definecolor{currentfill}{rgb}{0.000000,0.000000,0.000000}%
\pgfsetfillcolor{currentfill}%
\pgfsetlinewidth{0.501875pt}%
\definecolor{currentstroke}{rgb}{0.000000,0.000000,0.000000}%
\pgfsetstrokecolor{currentstroke}%
\pgfsetdash{}{0pt}%
\pgfsys@defobject{currentmarker}{\pgfqpoint{-0.020833in}{0.000000in}}{\pgfqpoint{-0.000000in}{0.000000in}}{%
\pgfpathmoveto{\pgfqpoint{-0.000000in}{0.000000in}}%
\pgfpathlineto{\pgfqpoint{-0.020833in}{0.000000in}}%
\pgfusepath{stroke,fill}%
}%
\begin{pgfscope}%
\pgfsys@transformshift{2.746589in}{4.236317in}%
\pgfsys@useobject{currentmarker}{}%
\end{pgfscope}%
\end{pgfscope}%
\begin{pgfscope}%
\definecolor{textcolor}{rgb}{0.000000,0.000000,0.000000}%
\pgfsetstrokecolor{textcolor}%
\pgfsetfillcolor{textcolor}%
\pgftext[x=0.188889in,y=3.561093in,,bottom,rotate=90.000000]{\color{textcolor}\rmfamily\fontsize{10.000000}{12.000000}\selectfont \(\displaystyle T(K)\)}%
\end{pgfscope}%
\begin{pgfscope}%
\pgfpathrectangle{\pgfqpoint{0.470525in}{2.747992in}}{\pgfqpoint{2.276064in}{1.626201in}}%
\pgfusepath{clip}%
\pgfsetrectcap%
\pgfsetroundjoin%
\pgfsetlinewidth{1.003750pt}%
\definecolor{currentstroke}{rgb}{0.047059,0.364706,0.647059}%
\pgfsetstrokecolor{currentstroke}%
\pgfsetdash{}{0pt}%
\pgfpathmoveto{\pgfqpoint{0.493061in}{4.300275in}}%
\pgfpathlineto{\pgfqpoint{0.515596in}{4.290474in}}%
\pgfpathlineto{\pgfqpoint{0.538131in}{4.283978in}}%
\pgfpathlineto{\pgfqpoint{0.560666in}{4.278700in}}%
\pgfpathlineto{\pgfqpoint{0.583202in}{4.274756in}}%
\pgfpathlineto{\pgfqpoint{0.605737in}{4.270177in}}%
\pgfpathlineto{\pgfqpoint{0.628272in}{4.266076in}}%
\pgfpathlineto{\pgfqpoint{0.650808in}{4.265991in}}%
\pgfpathlineto{\pgfqpoint{0.673343in}{4.264973in}}%
\pgfpathlineto{\pgfqpoint{0.695878in}{4.261073in}}%
\pgfpathlineto{\pgfqpoint{0.718413in}{4.259079in}}%
\pgfpathlineto{\pgfqpoint{0.740949in}{4.256976in}}%
\pgfpathlineto{\pgfqpoint{0.763484in}{4.256304in}}%
\pgfpathlineto{\pgfqpoint{0.786019in}{4.255196in}}%
\pgfpathlineto{\pgfqpoint{0.808555in}{4.254260in}}%
\pgfpathlineto{\pgfqpoint{0.831090in}{4.252821in}}%
\pgfpathlineto{\pgfqpoint{0.853625in}{4.253075in}}%
\pgfpathlineto{\pgfqpoint{0.876160in}{4.251656in}}%
\pgfpathlineto{\pgfqpoint{0.898696in}{4.249798in}}%
\pgfpathlineto{\pgfqpoint{0.921231in}{4.249910in}}%
\pgfpathlineto{\pgfqpoint{0.943766in}{4.249200in}}%
\pgfpathlineto{\pgfqpoint{0.966302in}{4.248218in}}%
\pgfpathlineto{\pgfqpoint{0.988837in}{4.247517in}}%
\pgfpathlineto{\pgfqpoint{1.011372in}{4.246692in}}%
\pgfpathlineto{\pgfqpoint{1.033907in}{4.245540in}}%
\pgfpathlineto{\pgfqpoint{1.056443in}{4.244709in}}%
\pgfpathlineto{\pgfqpoint{1.078978in}{4.244155in}}%
\pgfpathlineto{\pgfqpoint{1.101513in}{4.242649in}}%
\pgfpathlineto{\pgfqpoint{1.124049in}{4.242203in}}%
\pgfpathlineto{\pgfqpoint{1.146584in}{4.240441in}}%
\pgfpathlineto{\pgfqpoint{1.169119in}{4.239859in}}%
\pgfpathlineto{\pgfqpoint{1.191654in}{4.240204in}}%
\pgfpathlineto{\pgfqpoint{1.214190in}{4.237683in}}%
\pgfpathlineto{\pgfqpoint{1.236725in}{4.236450in}}%
\pgfpathlineto{\pgfqpoint{1.259260in}{4.236256in}}%
\pgfpathlineto{\pgfqpoint{1.281796in}{4.235159in}}%
\pgfpathlineto{\pgfqpoint{1.304331in}{4.235045in}}%
\pgfpathlineto{\pgfqpoint{1.326866in}{4.234095in}}%
\pgfpathlineto{\pgfqpoint{1.349402in}{4.233340in}}%
\pgfpathlineto{\pgfqpoint{1.371937in}{4.232871in}}%
\pgfpathlineto{\pgfqpoint{1.394472in}{4.232577in}}%
\pgfpathlineto{\pgfqpoint{1.417007in}{4.232629in}}%
\pgfpathlineto{\pgfqpoint{1.439543in}{4.232495in}}%
\pgfpathlineto{\pgfqpoint{1.462078in}{4.232251in}}%
\pgfpathlineto{\pgfqpoint{1.484613in}{4.232279in}}%
\pgfpathlineto{\pgfqpoint{1.507149in}{4.231928in}}%
\pgfpathlineto{\pgfqpoint{1.529684in}{4.231894in}}%
\pgfpathlineto{\pgfqpoint{1.552219in}{4.231898in}}%
\pgfpathlineto{\pgfqpoint{1.574754in}{4.230358in}}%
\pgfpathlineto{\pgfqpoint{1.597290in}{4.229548in}}%
\pgfpathlineto{\pgfqpoint{1.619825in}{4.229456in}}%
\pgfpathlineto{\pgfqpoint{1.642360in}{4.228545in}}%
\pgfpathlineto{\pgfqpoint{1.664896in}{4.228079in}}%
\pgfpathlineto{\pgfqpoint{1.687431in}{4.227863in}}%
\pgfpathlineto{\pgfqpoint{1.709966in}{4.226968in}}%
\pgfpathlineto{\pgfqpoint{1.732501in}{4.226477in}}%
\pgfpathlineto{\pgfqpoint{1.755037in}{4.226264in}}%
\pgfpathlineto{\pgfqpoint{1.777572in}{4.226138in}}%
\pgfpathlineto{\pgfqpoint{1.800107in}{4.225765in}}%
\pgfpathlineto{\pgfqpoint{1.822643in}{4.224693in}}%
\pgfpathlineto{\pgfqpoint{1.845178in}{4.224478in}}%
\pgfpathlineto{\pgfqpoint{1.867713in}{4.223691in}}%
\pgfpathlineto{\pgfqpoint{1.890248in}{4.223156in}}%
\pgfpathlineto{\pgfqpoint{1.912784in}{4.222556in}}%
\pgfpathlineto{\pgfqpoint{1.935319in}{4.222036in}}%
\pgfpathlineto{\pgfqpoint{1.957854in}{4.221852in}}%
\pgfpathlineto{\pgfqpoint{1.980390in}{4.221133in}}%
\pgfpathlineto{\pgfqpoint{2.002925in}{4.220367in}}%
\pgfpathlineto{\pgfqpoint{2.025460in}{4.219703in}}%
\pgfpathlineto{\pgfqpoint{2.047995in}{4.219017in}}%
\pgfpathlineto{\pgfqpoint{2.070531in}{4.218520in}}%
\pgfpathlineto{\pgfqpoint{2.093066in}{4.218075in}}%
\pgfpathlineto{\pgfqpoint{2.115601in}{4.217443in}}%
\pgfpathlineto{\pgfqpoint{2.138137in}{4.217126in}}%
\pgfpathlineto{\pgfqpoint{2.160672in}{4.216781in}}%
\pgfpathlineto{\pgfqpoint{2.183207in}{4.215903in}}%
\pgfpathlineto{\pgfqpoint{2.205742in}{4.215521in}}%
\pgfpathlineto{\pgfqpoint{2.228278in}{4.214516in}}%
\pgfpathlineto{\pgfqpoint{2.250813in}{4.213758in}}%
\pgfpathlineto{\pgfqpoint{2.273348in}{4.212715in}}%
\pgfpathlineto{\pgfqpoint{2.295884in}{4.211764in}}%
\pgfpathlineto{\pgfqpoint{2.318419in}{4.210543in}}%
\pgfpathlineto{\pgfqpoint{2.340954in}{4.209696in}}%
\pgfpathlineto{\pgfqpoint{2.363489in}{4.208550in}}%
\pgfpathlineto{\pgfqpoint{2.386025in}{4.207030in}}%
\pgfpathlineto{\pgfqpoint{2.408560in}{4.205463in}}%
\pgfpathlineto{\pgfqpoint{2.431095in}{4.204424in}}%
\pgfpathlineto{\pgfqpoint{2.453631in}{4.203436in}}%
\pgfpathlineto{\pgfqpoint{2.476166in}{4.202056in}}%
\pgfpathlineto{\pgfqpoint{2.498701in}{4.200677in}}%
\pgfpathlineto{\pgfqpoint{2.521236in}{4.200089in}}%
\pgfpathlineto{\pgfqpoint{2.543772in}{4.199055in}}%
\pgfpathlineto{\pgfqpoint{2.566307in}{4.197685in}}%
\pgfpathlineto{\pgfqpoint{2.588842in}{4.196692in}}%
\pgfpathlineto{\pgfqpoint{2.611378in}{4.195791in}}%
\pgfpathlineto{\pgfqpoint{2.633913in}{4.194265in}}%
\pgfpathlineto{\pgfqpoint{2.656448in}{4.192811in}}%
\pgfpathlineto{\pgfqpoint{2.678983in}{4.191906in}}%
\pgfpathlineto{\pgfqpoint{2.701519in}{4.190923in}}%
\pgfusepath{stroke}%
\end{pgfscope}%
\begin{pgfscope}%
\pgfpathrectangle{\pgfqpoint{0.470525in}{2.747992in}}{\pgfqpoint{2.276064in}{1.626201in}}%
\pgfusepath{clip}%
\pgfsetrectcap%
\pgfsetroundjoin%
\pgfsetlinewidth{1.003750pt}%
\definecolor{currentstroke}{rgb}{0.000000,0.725490,0.270588}%
\pgfsetstrokecolor{currentstroke}%
\pgfsetdash{}{0pt}%
\pgfpathmoveto{\pgfqpoint{0.493061in}{4.108552in}}%
\pgfpathlineto{\pgfqpoint{0.515596in}{4.080189in}}%
\pgfpathlineto{\pgfqpoint{0.538131in}{4.058030in}}%
\pgfpathlineto{\pgfqpoint{0.560666in}{4.037735in}}%
\pgfpathlineto{\pgfqpoint{0.583202in}{4.015052in}}%
\pgfpathlineto{\pgfqpoint{0.605737in}{3.999177in}}%
\pgfpathlineto{\pgfqpoint{0.628272in}{3.984698in}}%
\pgfpathlineto{\pgfqpoint{0.650808in}{3.975206in}}%
\pgfpathlineto{\pgfqpoint{0.673343in}{3.960505in}}%
\pgfpathlineto{\pgfqpoint{0.695878in}{3.953213in}}%
\pgfpathlineto{\pgfqpoint{0.718413in}{3.940190in}}%
\pgfpathlineto{\pgfqpoint{0.740949in}{3.929690in}}%
\pgfpathlineto{\pgfqpoint{0.763484in}{3.915292in}}%
\pgfpathlineto{\pgfqpoint{0.786019in}{3.902403in}}%
\pgfpathlineto{\pgfqpoint{0.808555in}{3.892251in}}%
\pgfpathlineto{\pgfqpoint{0.831090in}{3.884998in}}%
\pgfpathlineto{\pgfqpoint{0.853625in}{3.876114in}}%
\pgfpathlineto{\pgfqpoint{0.876160in}{3.866843in}}%
\pgfpathlineto{\pgfqpoint{0.898696in}{3.857923in}}%
\pgfpathlineto{\pgfqpoint{0.921231in}{3.849091in}}%
\pgfpathlineto{\pgfqpoint{0.943766in}{3.844468in}}%
\pgfpathlineto{\pgfqpoint{0.966302in}{3.837068in}}%
\pgfpathlineto{\pgfqpoint{0.988837in}{3.831432in}}%
\pgfpathlineto{\pgfqpoint{1.011372in}{3.822798in}}%
\pgfpathlineto{\pgfqpoint{1.033907in}{3.814460in}}%
\pgfpathlineto{\pgfqpoint{1.056443in}{3.808992in}}%
\pgfpathlineto{\pgfqpoint{1.078978in}{3.803257in}}%
\pgfpathlineto{\pgfqpoint{1.101513in}{3.799340in}}%
\pgfpathlineto{\pgfqpoint{1.124049in}{3.793287in}}%
\pgfpathlineto{\pgfqpoint{1.146584in}{3.788462in}}%
\pgfpathlineto{\pgfqpoint{1.169119in}{3.781186in}}%
\pgfpathlineto{\pgfqpoint{1.191654in}{3.775717in}}%
\pgfpathlineto{\pgfqpoint{1.214190in}{3.770656in}}%
\pgfpathlineto{\pgfqpoint{1.236725in}{3.766194in}}%
\pgfpathlineto{\pgfqpoint{1.259260in}{3.760912in}}%
\pgfpathlineto{\pgfqpoint{1.281796in}{3.755009in}}%
\pgfpathlineto{\pgfqpoint{1.304331in}{3.751600in}}%
\pgfpathlineto{\pgfqpoint{1.326866in}{3.745983in}}%
\pgfpathlineto{\pgfqpoint{1.349402in}{3.740644in}}%
\pgfpathlineto{\pgfqpoint{1.371937in}{3.736031in}}%
\pgfpathlineto{\pgfqpoint{1.394472in}{3.730222in}}%
\pgfpathlineto{\pgfqpoint{1.417007in}{3.724576in}}%
\pgfpathlineto{\pgfqpoint{1.439543in}{3.719720in}}%
\pgfpathlineto{\pgfqpoint{1.462078in}{3.713797in}}%
\pgfpathlineto{\pgfqpoint{1.484613in}{3.709703in}}%
\pgfpathlineto{\pgfqpoint{1.507149in}{3.704211in}}%
\pgfpathlineto{\pgfqpoint{1.529684in}{3.699503in}}%
\pgfpathlineto{\pgfqpoint{1.552219in}{3.693958in}}%
\pgfpathlineto{\pgfqpoint{1.574754in}{3.690201in}}%
\pgfpathlineto{\pgfqpoint{1.597290in}{3.685872in}}%
\pgfpathlineto{\pgfqpoint{1.619825in}{3.680752in}}%
\pgfpathlineto{\pgfqpoint{1.642360in}{3.676641in}}%
\pgfpathlineto{\pgfqpoint{1.664896in}{3.674058in}}%
\pgfpathlineto{\pgfqpoint{1.687431in}{3.670584in}}%
\pgfpathlineto{\pgfqpoint{1.709966in}{3.666494in}}%
\pgfpathlineto{\pgfqpoint{1.732501in}{3.661655in}}%
\pgfpathlineto{\pgfqpoint{1.755037in}{3.657780in}}%
\pgfpathlineto{\pgfqpoint{1.777572in}{3.653213in}}%
\pgfpathlineto{\pgfqpoint{1.800107in}{3.649248in}}%
\pgfpathlineto{\pgfqpoint{1.822643in}{3.645096in}}%
\pgfpathlineto{\pgfqpoint{1.845178in}{3.640100in}}%
\pgfpathlineto{\pgfqpoint{1.867713in}{3.635214in}}%
\pgfpathlineto{\pgfqpoint{1.890248in}{3.630102in}}%
\pgfpathlineto{\pgfqpoint{1.912784in}{3.625739in}}%
\pgfpathlineto{\pgfqpoint{1.935319in}{3.620750in}}%
\pgfpathlineto{\pgfqpoint{1.957854in}{3.615201in}}%
\pgfpathlineto{\pgfqpoint{1.980390in}{3.609402in}}%
\pgfpathlineto{\pgfqpoint{2.002925in}{3.604493in}}%
\pgfpathlineto{\pgfqpoint{2.025460in}{3.599106in}}%
\pgfpathlineto{\pgfqpoint{2.047995in}{3.594043in}}%
\pgfpathlineto{\pgfqpoint{2.070531in}{3.587400in}}%
\pgfpathlineto{\pgfqpoint{2.093066in}{3.582579in}}%
\pgfpathlineto{\pgfqpoint{2.115601in}{3.578026in}}%
\pgfpathlineto{\pgfqpoint{2.138137in}{3.573382in}}%
\pgfpathlineto{\pgfqpoint{2.160672in}{3.568309in}}%
\pgfpathlineto{\pgfqpoint{2.183207in}{3.562652in}}%
\pgfpathlineto{\pgfqpoint{2.205742in}{3.557158in}}%
\pgfpathlineto{\pgfqpoint{2.228278in}{3.551757in}}%
\pgfpathlineto{\pgfqpoint{2.250813in}{3.545646in}}%
\pgfpathlineto{\pgfqpoint{2.273348in}{3.539526in}}%
\pgfpathlineto{\pgfqpoint{2.295884in}{3.532502in}}%
\pgfpathlineto{\pgfqpoint{2.318419in}{3.525805in}}%
\pgfpathlineto{\pgfqpoint{2.340954in}{3.519496in}}%
\pgfpathlineto{\pgfqpoint{2.363489in}{3.512540in}}%
\pgfpathlineto{\pgfqpoint{2.386025in}{3.506576in}}%
\pgfpathlineto{\pgfqpoint{2.408560in}{3.500428in}}%
\pgfpathlineto{\pgfqpoint{2.431095in}{3.494676in}}%
\pgfpathlineto{\pgfqpoint{2.453631in}{3.488854in}}%
\pgfpathlineto{\pgfqpoint{2.476166in}{3.481960in}}%
\pgfpathlineto{\pgfqpoint{2.498701in}{3.475404in}}%
\pgfpathlineto{\pgfqpoint{2.521236in}{3.468873in}}%
\pgfpathlineto{\pgfqpoint{2.543772in}{3.462152in}}%
\pgfpathlineto{\pgfqpoint{2.566307in}{3.456218in}}%
\pgfpathlineto{\pgfqpoint{2.588842in}{3.450867in}}%
\pgfpathlineto{\pgfqpoint{2.611378in}{3.443701in}}%
\pgfpathlineto{\pgfqpoint{2.633913in}{3.437841in}}%
\pgfpathlineto{\pgfqpoint{2.656448in}{3.431405in}}%
\pgfpathlineto{\pgfqpoint{2.678983in}{3.425415in}}%
\pgfpathlineto{\pgfqpoint{2.701519in}{3.419429in}}%
\pgfusepath{stroke}%
\end{pgfscope}%
\begin{pgfscope}%
\pgfpathrectangle{\pgfqpoint{0.470525in}{2.747992in}}{\pgfqpoint{2.276064in}{1.626201in}}%
\pgfusepath{clip}%
\pgfsetrectcap%
\pgfsetroundjoin%
\pgfsetlinewidth{1.003750pt}%
\definecolor{currentstroke}{rgb}{1.000000,0.584314,0.000000}%
\pgfsetstrokecolor{currentstroke}%
\pgfsetdash{}{0pt}%
\pgfpathmoveto{\pgfqpoint{0.493061in}{4.239308in}}%
\pgfpathlineto{\pgfqpoint{0.515596in}{4.225003in}}%
\pgfpathlineto{\pgfqpoint{0.538131in}{4.220796in}}%
\pgfpathlineto{\pgfqpoint{0.560666in}{4.214318in}}%
\pgfpathlineto{\pgfqpoint{0.583202in}{4.207028in}}%
\pgfpathlineto{\pgfqpoint{0.605737in}{4.198537in}}%
\pgfpathlineto{\pgfqpoint{0.628272in}{4.195445in}}%
\pgfpathlineto{\pgfqpoint{0.650808in}{4.185911in}}%
\pgfpathlineto{\pgfqpoint{0.673343in}{4.179052in}}%
\pgfpathlineto{\pgfqpoint{0.695878in}{4.172340in}}%
\pgfpathlineto{\pgfqpoint{0.718413in}{4.169801in}}%
\pgfpathlineto{\pgfqpoint{0.740949in}{4.166485in}}%
\pgfpathlineto{\pgfqpoint{0.763484in}{4.165533in}}%
\pgfpathlineto{\pgfqpoint{0.786019in}{4.162175in}}%
\pgfpathlineto{\pgfqpoint{0.808555in}{4.159078in}}%
\pgfpathlineto{\pgfqpoint{0.831090in}{4.155936in}}%
\pgfpathlineto{\pgfqpoint{0.853625in}{4.156942in}}%
\pgfpathlineto{\pgfqpoint{0.876160in}{4.157265in}}%
\pgfpathlineto{\pgfqpoint{0.898696in}{4.155798in}}%
\pgfpathlineto{\pgfqpoint{0.921231in}{4.153827in}}%
\pgfpathlineto{\pgfqpoint{0.943766in}{4.152794in}}%
\pgfpathlineto{\pgfqpoint{0.966302in}{4.151175in}}%
\pgfpathlineto{\pgfqpoint{0.988837in}{4.149924in}}%
\pgfpathlineto{\pgfqpoint{1.011372in}{4.147867in}}%
\pgfpathlineto{\pgfqpoint{1.033907in}{4.148555in}}%
\pgfpathlineto{\pgfqpoint{1.056443in}{4.146639in}}%
\pgfpathlineto{\pgfqpoint{1.078978in}{4.145485in}}%
\pgfpathlineto{\pgfqpoint{1.101513in}{4.145845in}}%
\pgfpathlineto{\pgfqpoint{1.124049in}{4.143893in}}%
\pgfpathlineto{\pgfqpoint{1.146584in}{4.141529in}}%
\pgfpathlineto{\pgfqpoint{1.169119in}{4.140751in}}%
\pgfpathlineto{\pgfqpoint{1.191654in}{4.139615in}}%
\pgfpathlineto{\pgfqpoint{1.214190in}{4.137508in}}%
\pgfpathlineto{\pgfqpoint{1.236725in}{4.136709in}}%
\pgfpathlineto{\pgfqpoint{1.259260in}{4.135038in}}%
\pgfpathlineto{\pgfqpoint{1.281796in}{4.132692in}}%
\pgfpathlineto{\pgfqpoint{1.304331in}{4.130658in}}%
\pgfpathlineto{\pgfqpoint{1.326866in}{4.129248in}}%
\pgfpathlineto{\pgfqpoint{1.349402in}{4.128348in}}%
\pgfpathlineto{\pgfqpoint{1.371937in}{4.127211in}}%
\pgfpathlineto{\pgfqpoint{1.394472in}{4.125381in}}%
\pgfpathlineto{\pgfqpoint{1.417007in}{4.123096in}}%
\pgfpathlineto{\pgfqpoint{1.439543in}{4.122548in}}%
\pgfpathlineto{\pgfqpoint{1.462078in}{4.121076in}}%
\pgfpathlineto{\pgfqpoint{1.484613in}{4.119644in}}%
\pgfpathlineto{\pgfqpoint{1.507149in}{4.118089in}}%
\pgfpathlineto{\pgfqpoint{1.529684in}{4.116029in}}%
\pgfpathlineto{\pgfqpoint{1.552219in}{4.114831in}}%
\pgfpathlineto{\pgfqpoint{1.574754in}{4.113062in}}%
\pgfpathlineto{\pgfqpoint{1.597290in}{4.112136in}}%
\pgfpathlineto{\pgfqpoint{1.619825in}{4.110994in}}%
\pgfpathlineto{\pgfqpoint{1.642360in}{4.109557in}}%
\pgfpathlineto{\pgfqpoint{1.664896in}{4.108810in}}%
\pgfpathlineto{\pgfqpoint{1.687431in}{4.107261in}}%
\pgfpathlineto{\pgfqpoint{1.709966in}{4.106511in}}%
\pgfpathlineto{\pgfqpoint{1.732501in}{4.104747in}}%
\pgfpathlineto{\pgfqpoint{1.755037in}{4.103997in}}%
\pgfpathlineto{\pgfqpoint{1.777572in}{4.103687in}}%
\pgfpathlineto{\pgfqpoint{1.800107in}{4.103749in}}%
\pgfpathlineto{\pgfqpoint{1.822643in}{4.103008in}}%
\pgfpathlineto{\pgfqpoint{1.845178in}{4.101340in}}%
\pgfpathlineto{\pgfqpoint{1.867713in}{4.100237in}}%
\pgfpathlineto{\pgfqpoint{1.890248in}{4.099501in}}%
\pgfpathlineto{\pgfqpoint{1.912784in}{4.098320in}}%
\pgfpathlineto{\pgfqpoint{1.935319in}{4.097091in}}%
\pgfpathlineto{\pgfqpoint{1.957854in}{4.095753in}}%
\pgfpathlineto{\pgfqpoint{1.980390in}{4.095422in}}%
\pgfpathlineto{\pgfqpoint{2.002925in}{4.094185in}}%
\pgfpathlineto{\pgfqpoint{2.025460in}{4.093433in}}%
\pgfpathlineto{\pgfqpoint{2.047995in}{4.092902in}}%
\pgfpathlineto{\pgfqpoint{2.070531in}{4.091967in}}%
\pgfpathlineto{\pgfqpoint{2.093066in}{4.091155in}}%
\pgfpathlineto{\pgfqpoint{2.115601in}{4.089973in}}%
\pgfpathlineto{\pgfqpoint{2.138137in}{4.088848in}}%
\pgfpathlineto{\pgfqpoint{2.160672in}{4.087241in}}%
\pgfpathlineto{\pgfqpoint{2.183207in}{4.085433in}}%
\pgfpathlineto{\pgfqpoint{2.205742in}{4.082690in}}%
\pgfpathlineto{\pgfqpoint{2.228278in}{4.081012in}}%
\pgfpathlineto{\pgfqpoint{2.250813in}{4.078813in}}%
\pgfpathlineto{\pgfqpoint{2.273348in}{4.076658in}}%
\pgfpathlineto{\pgfqpoint{2.295884in}{4.074575in}}%
\pgfpathlineto{\pgfqpoint{2.318419in}{4.072143in}}%
\pgfpathlineto{\pgfqpoint{2.340954in}{4.069393in}}%
\pgfpathlineto{\pgfqpoint{2.363489in}{4.066340in}}%
\pgfpathlineto{\pgfqpoint{2.386025in}{4.063032in}}%
\pgfpathlineto{\pgfqpoint{2.408560in}{4.060553in}}%
\pgfpathlineto{\pgfqpoint{2.431095in}{4.058554in}}%
\pgfpathlineto{\pgfqpoint{2.453631in}{4.056049in}}%
\pgfpathlineto{\pgfqpoint{2.476166in}{4.055066in}}%
\pgfpathlineto{\pgfqpoint{2.498701in}{4.053167in}}%
\pgfpathlineto{\pgfqpoint{2.521236in}{4.051149in}}%
\pgfpathlineto{\pgfqpoint{2.543772in}{4.049665in}}%
\pgfpathlineto{\pgfqpoint{2.566307in}{4.047296in}}%
\pgfpathlineto{\pgfqpoint{2.588842in}{4.044782in}}%
\pgfpathlineto{\pgfqpoint{2.611378in}{4.042165in}}%
\pgfpathlineto{\pgfqpoint{2.633913in}{4.039319in}}%
\pgfpathlineto{\pgfqpoint{2.656448in}{4.036288in}}%
\pgfpathlineto{\pgfqpoint{2.678983in}{4.034271in}}%
\pgfpathlineto{\pgfqpoint{2.701519in}{4.032112in}}%
\pgfusepath{stroke}%
\end{pgfscope}%
\begin{pgfscope}%
\pgfpathrectangle{\pgfqpoint{0.470525in}{2.747992in}}{\pgfqpoint{2.276064in}{1.626201in}}%
\pgfusepath{clip}%
\pgfsetrectcap%
\pgfsetroundjoin%
\pgfsetlinewidth{1.003750pt}%
\definecolor{currentstroke}{rgb}{1.000000,0.172549,0.000000}%
\pgfsetstrokecolor{currentstroke}%
\pgfsetdash{}{0pt}%
\pgfpathmoveto{\pgfqpoint{0.493061in}{3.864473in}}%
\pgfpathlineto{\pgfqpoint{0.515596in}{3.822808in}}%
\pgfpathlineto{\pgfqpoint{0.538131in}{3.805301in}}%
\pgfpathlineto{\pgfqpoint{0.560666in}{3.802685in}}%
\pgfpathlineto{\pgfqpoint{0.583202in}{3.783790in}}%
\pgfpathlineto{\pgfqpoint{0.605737in}{3.778764in}}%
\pgfpathlineto{\pgfqpoint{0.628272in}{3.782219in}}%
\pgfpathlineto{\pgfqpoint{0.650808in}{3.780875in}}%
\pgfpathlineto{\pgfqpoint{0.673343in}{3.778665in}}%
\pgfpathlineto{\pgfqpoint{0.695878in}{3.774166in}}%
\pgfpathlineto{\pgfqpoint{0.718413in}{3.770912in}}%
\pgfpathlineto{\pgfqpoint{0.740949in}{3.763194in}}%
\pgfpathlineto{\pgfqpoint{0.763484in}{3.757586in}}%
\pgfpathlineto{\pgfqpoint{0.786019in}{3.754612in}}%
\pgfpathlineto{\pgfqpoint{0.808555in}{3.749812in}}%
\pgfpathlineto{\pgfqpoint{0.831090in}{3.741927in}}%
\pgfpathlineto{\pgfqpoint{0.853625in}{3.732922in}}%
\pgfpathlineto{\pgfqpoint{0.876160in}{3.730692in}}%
\pgfpathlineto{\pgfqpoint{0.898696in}{3.724570in}}%
\pgfpathlineto{\pgfqpoint{0.921231in}{3.717126in}}%
\pgfpathlineto{\pgfqpoint{0.943766in}{3.708132in}}%
\pgfpathlineto{\pgfqpoint{0.966302in}{3.705121in}}%
\pgfpathlineto{\pgfqpoint{0.988837in}{3.697544in}}%
\pgfpathlineto{\pgfqpoint{1.011372in}{3.692170in}}%
\pgfpathlineto{\pgfqpoint{1.033907in}{3.687913in}}%
\pgfpathlineto{\pgfqpoint{1.056443in}{3.679767in}}%
\pgfpathlineto{\pgfqpoint{1.078978in}{3.668564in}}%
\pgfpathlineto{\pgfqpoint{1.101513in}{3.660503in}}%
\pgfpathlineto{\pgfqpoint{1.124049in}{3.651609in}}%
\pgfpathlineto{\pgfqpoint{1.146584in}{3.642988in}}%
\pgfpathlineto{\pgfqpoint{1.169119in}{3.632690in}}%
\pgfpathlineto{\pgfqpoint{1.191654in}{3.622802in}}%
\pgfpathlineto{\pgfqpoint{1.214190in}{3.614246in}}%
\pgfpathlineto{\pgfqpoint{1.236725in}{3.605674in}}%
\pgfpathlineto{\pgfqpoint{1.259260in}{3.594331in}}%
\pgfpathlineto{\pgfqpoint{1.281796in}{3.587074in}}%
\pgfpathlineto{\pgfqpoint{1.304331in}{3.578462in}}%
\pgfpathlineto{\pgfqpoint{1.326866in}{3.569065in}}%
\pgfpathlineto{\pgfqpoint{1.349402in}{3.558396in}}%
\pgfpathlineto{\pgfqpoint{1.371937in}{3.546300in}}%
\pgfpathlineto{\pgfqpoint{1.394472in}{3.536061in}}%
\pgfpathlineto{\pgfqpoint{1.417007in}{3.523205in}}%
\pgfpathlineto{\pgfqpoint{1.439543in}{3.507842in}}%
\pgfpathlineto{\pgfqpoint{1.462078in}{3.493110in}}%
\pgfpathlineto{\pgfqpoint{1.484613in}{3.481381in}}%
\pgfpathlineto{\pgfqpoint{1.507149in}{3.466637in}}%
\pgfpathlineto{\pgfqpoint{1.529684in}{3.448656in}}%
\pgfpathlineto{\pgfqpoint{1.552219in}{3.435489in}}%
\pgfpathlineto{\pgfqpoint{1.574754in}{3.421873in}}%
\pgfpathlineto{\pgfqpoint{1.597290in}{3.407685in}}%
\pgfpathlineto{\pgfqpoint{1.619825in}{3.394001in}}%
\pgfpathlineto{\pgfqpoint{1.642360in}{3.381574in}}%
\pgfpathlineto{\pgfqpoint{1.664896in}{3.368313in}}%
\pgfpathlineto{\pgfqpoint{1.687431in}{3.354047in}}%
\pgfpathlineto{\pgfqpoint{1.709966in}{3.342981in}}%
\pgfpathlineto{\pgfqpoint{1.732501in}{3.327901in}}%
\pgfpathlineto{\pgfqpoint{1.755037in}{3.315694in}}%
\pgfpathlineto{\pgfqpoint{1.777572in}{3.300854in}}%
\pgfpathlineto{\pgfqpoint{1.800107in}{3.285563in}}%
\pgfpathlineto{\pgfqpoint{1.822643in}{3.272061in}}%
\pgfpathlineto{\pgfqpoint{1.845178in}{3.258496in}}%
\pgfpathlineto{\pgfqpoint{1.867713in}{3.245741in}}%
\pgfpathlineto{\pgfqpoint{1.890248in}{3.231745in}}%
\pgfpathlineto{\pgfqpoint{1.912784in}{3.216710in}}%
\pgfpathlineto{\pgfqpoint{1.935319in}{3.201542in}}%
\pgfpathlineto{\pgfqpoint{1.957854in}{3.184002in}}%
\pgfpathlineto{\pgfqpoint{1.980390in}{3.169958in}}%
\pgfpathlineto{\pgfqpoint{2.002925in}{3.155148in}}%
\pgfpathlineto{\pgfqpoint{2.025460in}{3.141060in}}%
\pgfpathlineto{\pgfqpoint{2.047995in}{3.125099in}}%
\pgfpathlineto{\pgfqpoint{2.070531in}{3.111896in}}%
\pgfpathlineto{\pgfqpoint{2.093066in}{3.098935in}}%
\pgfpathlineto{\pgfqpoint{2.115601in}{3.085968in}}%
\pgfpathlineto{\pgfqpoint{2.138137in}{3.074542in}}%
\pgfpathlineto{\pgfqpoint{2.160672in}{3.060498in}}%
\pgfpathlineto{\pgfqpoint{2.183207in}{3.047530in}}%
\pgfpathlineto{\pgfqpoint{2.205742in}{3.034589in}}%
\pgfpathlineto{\pgfqpoint{2.228278in}{3.021032in}}%
\pgfpathlineto{\pgfqpoint{2.250813in}{3.008262in}}%
\pgfpathlineto{\pgfqpoint{2.273348in}{2.994271in}}%
\pgfpathlineto{\pgfqpoint{2.295884in}{2.982226in}}%
\pgfpathlineto{\pgfqpoint{2.318419in}{2.970579in}}%
\pgfpathlineto{\pgfqpoint{2.340954in}{2.960070in}}%
\pgfpathlineto{\pgfqpoint{2.363489in}{2.951080in}}%
\pgfpathlineto{\pgfqpoint{2.386025in}{2.941265in}}%
\pgfpathlineto{\pgfqpoint{2.408560in}{2.932188in}}%
\pgfpathlineto{\pgfqpoint{2.431095in}{2.923387in}}%
\pgfpathlineto{\pgfqpoint{2.453631in}{2.915177in}}%
\pgfpathlineto{\pgfqpoint{2.476166in}{2.905952in}}%
\pgfpathlineto{\pgfqpoint{2.498701in}{2.895756in}}%
\pgfpathlineto{\pgfqpoint{2.521236in}{2.886072in}}%
\pgfpathlineto{\pgfqpoint{2.543772in}{2.876308in}}%
\pgfpathlineto{\pgfqpoint{2.566307in}{2.867698in}}%
\pgfpathlineto{\pgfqpoint{2.588842in}{2.858625in}}%
\pgfpathlineto{\pgfqpoint{2.611378in}{2.850450in}}%
\pgfpathlineto{\pgfqpoint{2.633913in}{2.843059in}}%
\pgfpathlineto{\pgfqpoint{2.656448in}{2.835465in}}%
\pgfpathlineto{\pgfqpoint{2.678983in}{2.828398in}}%
\pgfpathlineto{\pgfqpoint{2.701519in}{2.821910in}}%
\pgfusepath{stroke}%
\end{pgfscope}%
\begin{pgfscope}%
\pgfpathrectangle{\pgfqpoint{0.470525in}{2.747992in}}{\pgfqpoint{2.276064in}{1.626201in}}%
\pgfusepath{clip}%
\pgfsetrectcap%
\pgfsetroundjoin%
\pgfsetlinewidth{1.003750pt}%
\definecolor{currentstroke}{rgb}{0.517647,0.356863,0.592157}%
\pgfsetstrokecolor{currentstroke}%
\pgfsetdash{}{0pt}%
\pgfpathmoveto{\pgfqpoint{0.493061in}{3.880262in}}%
\pgfpathlineto{\pgfqpoint{0.515596in}{3.879790in}}%
\pgfpathlineto{\pgfqpoint{0.538131in}{3.868150in}}%
\pgfpathlineto{\pgfqpoint{0.560666in}{3.833206in}}%
\pgfpathlineto{\pgfqpoint{0.583202in}{3.821623in}}%
\pgfpathlineto{\pgfqpoint{0.605737in}{3.816518in}}%
\pgfpathlineto{\pgfqpoint{0.628272in}{3.813187in}}%
\pgfpathlineto{\pgfqpoint{0.650808in}{3.797452in}}%
\pgfpathlineto{\pgfqpoint{0.673343in}{3.793080in}}%
\pgfpathlineto{\pgfqpoint{0.695878in}{3.793293in}}%
\pgfpathlineto{\pgfqpoint{0.718413in}{3.787254in}}%
\pgfpathlineto{\pgfqpoint{0.740949in}{3.781682in}}%
\pgfpathlineto{\pgfqpoint{0.763484in}{3.778498in}}%
\pgfpathlineto{\pgfqpoint{0.786019in}{3.773030in}}%
\pgfpathlineto{\pgfqpoint{0.808555in}{3.771095in}}%
\pgfpathlineto{\pgfqpoint{0.831090in}{3.770678in}}%
\pgfpathlineto{\pgfqpoint{0.853625in}{3.767953in}}%
\pgfpathlineto{\pgfqpoint{0.876160in}{3.764998in}}%
\pgfpathlineto{\pgfqpoint{0.898696in}{3.762284in}}%
\pgfpathlineto{\pgfqpoint{0.921231in}{3.756680in}}%
\pgfpathlineto{\pgfqpoint{0.943766in}{3.752847in}}%
\pgfpathlineto{\pgfqpoint{0.966302in}{3.755241in}}%
\pgfpathlineto{\pgfqpoint{0.988837in}{3.752965in}}%
\pgfpathlineto{\pgfqpoint{1.011372in}{3.753958in}}%
\pgfpathlineto{\pgfqpoint{1.033907in}{3.757401in}}%
\pgfpathlineto{\pgfqpoint{1.056443in}{3.756438in}}%
\pgfpathlineto{\pgfqpoint{1.078978in}{3.757023in}}%
\pgfpathlineto{\pgfqpoint{1.101513in}{3.752846in}}%
\pgfpathlineto{\pgfqpoint{1.124049in}{3.747471in}}%
\pgfpathlineto{\pgfqpoint{1.146584in}{3.747394in}}%
\pgfpathlineto{\pgfqpoint{1.169119in}{3.746813in}}%
\pgfpathlineto{\pgfqpoint{1.191654in}{3.745538in}}%
\pgfpathlineto{\pgfqpoint{1.214190in}{3.743122in}}%
\pgfpathlineto{\pgfqpoint{1.236725in}{3.742720in}}%
\pgfpathlineto{\pgfqpoint{1.259260in}{3.741152in}}%
\pgfpathlineto{\pgfqpoint{1.281796in}{3.738843in}}%
\pgfpathlineto{\pgfqpoint{1.304331in}{3.737884in}}%
\pgfpathlineto{\pgfqpoint{1.326866in}{3.737446in}}%
\pgfpathlineto{\pgfqpoint{1.349402in}{3.736908in}}%
\pgfpathlineto{\pgfqpoint{1.371937in}{3.734538in}}%
\pgfpathlineto{\pgfqpoint{1.394472in}{3.731832in}}%
\pgfpathlineto{\pgfqpoint{1.417007in}{3.729988in}}%
\pgfpathlineto{\pgfqpoint{1.439543in}{3.729045in}}%
\pgfpathlineto{\pgfqpoint{1.462078in}{3.727762in}}%
\pgfpathlineto{\pgfqpoint{1.484613in}{3.725881in}}%
\pgfpathlineto{\pgfqpoint{1.507149in}{3.723559in}}%
\pgfpathlineto{\pgfqpoint{1.529684in}{3.722541in}}%
\pgfpathlineto{\pgfqpoint{1.552219in}{3.722079in}}%
\pgfpathlineto{\pgfqpoint{1.574754in}{3.718984in}}%
\pgfpathlineto{\pgfqpoint{1.597290in}{3.717314in}}%
\pgfpathlineto{\pgfqpoint{1.619825in}{3.715979in}}%
\pgfpathlineto{\pgfqpoint{1.642360in}{3.713951in}}%
\pgfpathlineto{\pgfqpoint{1.664896in}{3.711686in}}%
\pgfpathlineto{\pgfqpoint{1.687431in}{3.709064in}}%
\pgfpathlineto{\pgfqpoint{1.709966in}{3.707222in}}%
\pgfpathlineto{\pgfqpoint{1.732501in}{3.705997in}}%
\pgfpathlineto{\pgfqpoint{1.755037in}{3.706088in}}%
\pgfpathlineto{\pgfqpoint{1.777572in}{3.707138in}}%
\pgfpathlineto{\pgfqpoint{1.800107in}{3.707249in}}%
\pgfpathlineto{\pgfqpoint{1.822643in}{3.707382in}}%
\pgfpathlineto{\pgfqpoint{1.845178in}{3.706503in}}%
\pgfpathlineto{\pgfqpoint{1.867713in}{3.703675in}}%
\pgfpathlineto{\pgfqpoint{1.890248in}{3.700561in}}%
\pgfpathlineto{\pgfqpoint{1.912784in}{3.699235in}}%
\pgfpathlineto{\pgfqpoint{1.935319in}{3.697692in}}%
\pgfpathlineto{\pgfqpoint{1.957854in}{3.696789in}}%
\pgfpathlineto{\pgfqpoint{1.980390in}{3.694355in}}%
\pgfpathlineto{\pgfqpoint{2.002925in}{3.693036in}}%
\pgfpathlineto{\pgfqpoint{2.025460in}{3.688301in}}%
\pgfpathlineto{\pgfqpoint{2.047995in}{3.687275in}}%
\pgfpathlineto{\pgfqpoint{2.070531in}{3.683634in}}%
\pgfpathlineto{\pgfqpoint{2.093066in}{3.681535in}}%
\pgfpathlineto{\pgfqpoint{2.115601in}{3.678997in}}%
\pgfpathlineto{\pgfqpoint{2.138137in}{3.673284in}}%
\pgfpathlineto{\pgfqpoint{2.160672in}{3.668798in}}%
\pgfpathlineto{\pgfqpoint{2.183207in}{3.663544in}}%
\pgfpathlineto{\pgfqpoint{2.205742in}{3.661057in}}%
\pgfpathlineto{\pgfqpoint{2.228278in}{3.658825in}}%
\pgfpathlineto{\pgfqpoint{2.250813in}{3.656788in}}%
\pgfpathlineto{\pgfqpoint{2.273348in}{3.652958in}}%
\pgfpathlineto{\pgfqpoint{2.295884in}{3.648384in}}%
\pgfpathlineto{\pgfqpoint{2.318419in}{3.646631in}}%
\pgfpathlineto{\pgfqpoint{2.340954in}{3.643176in}}%
\pgfpathlineto{\pgfqpoint{2.363489in}{3.638930in}}%
\pgfpathlineto{\pgfqpoint{2.386025in}{3.635303in}}%
\pgfpathlineto{\pgfqpoint{2.408560in}{3.629790in}}%
\pgfpathlineto{\pgfqpoint{2.431095in}{3.627442in}}%
\pgfpathlineto{\pgfqpoint{2.453631in}{3.624250in}}%
\pgfpathlineto{\pgfqpoint{2.476166in}{3.621281in}}%
\pgfpathlineto{\pgfqpoint{2.498701in}{3.619508in}}%
\pgfpathlineto{\pgfqpoint{2.521236in}{3.618345in}}%
\pgfpathlineto{\pgfqpoint{2.543772in}{3.615326in}}%
\pgfpathlineto{\pgfqpoint{2.566307in}{3.611886in}}%
\pgfpathlineto{\pgfqpoint{2.588842in}{3.610641in}}%
\pgfpathlineto{\pgfqpoint{2.611378in}{3.607704in}}%
\pgfpathlineto{\pgfqpoint{2.633913in}{3.604810in}}%
\pgfpathlineto{\pgfqpoint{2.656448in}{3.602717in}}%
\pgfpathlineto{\pgfqpoint{2.678983in}{3.598176in}}%
\pgfpathlineto{\pgfqpoint{2.701519in}{3.596001in}}%
\pgfusepath{stroke}%
\end{pgfscope}%
\begin{pgfscope}%
\pgfsetrectcap%
\pgfsetmiterjoin%
\pgfsetlinewidth{0.501875pt}%
\definecolor{currentstroke}{rgb}{0.000000,0.000000,0.000000}%
\pgfsetstrokecolor{currentstroke}%
\pgfsetdash{}{0pt}%
\pgfpathmoveto{\pgfqpoint{0.470525in}{2.747992in}}%
\pgfpathlineto{\pgfqpoint{0.470525in}{4.374193in}}%
\pgfusepath{stroke}%
\end{pgfscope}%
\begin{pgfscope}%
\pgfsetrectcap%
\pgfsetmiterjoin%
\pgfsetlinewidth{0.501875pt}%
\definecolor{currentstroke}{rgb}{0.000000,0.000000,0.000000}%
\pgfsetstrokecolor{currentstroke}%
\pgfsetdash{}{0pt}%
\pgfpathmoveto{\pgfqpoint{2.746589in}{2.747992in}}%
\pgfpathlineto{\pgfqpoint{2.746589in}{4.374193in}}%
\pgfusepath{stroke}%
\end{pgfscope}%
\begin{pgfscope}%
\pgfsetrectcap%
\pgfsetmiterjoin%
\pgfsetlinewidth{0.501875pt}%
\definecolor{currentstroke}{rgb}{0.000000,0.000000,0.000000}%
\pgfsetstrokecolor{currentstroke}%
\pgfsetdash{}{0pt}%
\pgfpathmoveto{\pgfqpoint{0.470525in}{2.747992in}}%
\pgfpathlineto{\pgfqpoint{2.746589in}{2.747992in}}%
\pgfusepath{stroke}%
\end{pgfscope}%
\begin{pgfscope}%
\pgfsetrectcap%
\pgfsetmiterjoin%
\pgfsetlinewidth{0.501875pt}%
\definecolor{currentstroke}{rgb}{0.000000,0.000000,0.000000}%
\pgfsetstrokecolor{currentstroke}%
\pgfsetdash{}{0pt}%
\pgfpathmoveto{\pgfqpoint{0.470525in}{4.374193in}}%
\pgfpathlineto{\pgfqpoint{2.746589in}{4.374193in}}%
\pgfusepath{stroke}%
\end{pgfscope}%
\begin{pgfscope}%
\definecolor{textcolor}{rgb}{0.000000,0.000000,0.000000}%
\pgfsetstrokecolor{textcolor}%
\pgfsetfillcolor{textcolor}%
\pgftext[x=1.608557in,y=4.457526in,,base]{\color{textcolor}\rmfamily\fontsize{12.000000}{14.400000}\selectfont Trustworthiness}%
\end{pgfscope}%
\begin{pgfscope}%
\pgfsetbuttcap%
\pgfsetmiterjoin%
\definecolor{currentfill}{rgb}{1.000000,1.000000,1.000000}%
\pgfsetfillcolor{currentfill}%
\pgfsetlinewidth{0.000000pt}%
\definecolor{currentstroke}{rgb}{0.000000,0.000000,0.000000}%
\pgfsetstrokecolor{currentstroke}%
\pgfsetstrokeopacity{0.000000}%
\pgfsetdash{}{0pt}%
\pgfpathmoveto{\pgfqpoint{0.470525in}{0.422992in}}%
\pgfpathlineto{\pgfqpoint{2.746589in}{0.422992in}}%
\pgfpathlineto{\pgfqpoint{2.746589in}{2.049193in}}%
\pgfpathlineto{\pgfqpoint{0.470525in}{2.049193in}}%
\pgfpathlineto{\pgfqpoint{0.470525in}{0.422992in}}%
\pgfpathclose%
\pgfusepath{fill}%
\end{pgfscope}%
\begin{pgfscope}%
\pgfsetbuttcap%
\pgfsetroundjoin%
\definecolor{currentfill}{rgb}{0.000000,0.000000,0.000000}%
\pgfsetfillcolor{currentfill}%
\pgfsetlinewidth{0.501875pt}%
\definecolor{currentstroke}{rgb}{0.000000,0.000000,0.000000}%
\pgfsetstrokecolor{currentstroke}%
\pgfsetdash{}{0pt}%
\pgfsys@defobject{currentmarker}{\pgfqpoint{0.000000in}{0.000000in}}{\pgfqpoint{0.000000in}{0.041667in}}{%
\pgfpathmoveto{\pgfqpoint{0.000000in}{0.000000in}}%
\pgfpathlineto{\pgfqpoint{0.000000in}{0.041667in}}%
\pgfusepath{stroke,fill}%
}%
\begin{pgfscope}%
\pgfsys@transformshift{0.470525in}{0.422992in}%
\pgfsys@useobject{currentmarker}{}%
\end{pgfscope}%
\end{pgfscope}%
\begin{pgfscope}%
\pgfsetbuttcap%
\pgfsetroundjoin%
\definecolor{currentfill}{rgb}{0.000000,0.000000,0.000000}%
\pgfsetfillcolor{currentfill}%
\pgfsetlinewidth{0.501875pt}%
\definecolor{currentstroke}{rgb}{0.000000,0.000000,0.000000}%
\pgfsetstrokecolor{currentstroke}%
\pgfsetdash{}{0pt}%
\pgfsys@defobject{currentmarker}{\pgfqpoint{0.000000in}{-0.041667in}}{\pgfqpoint{0.000000in}{0.000000in}}{%
\pgfpathmoveto{\pgfqpoint{0.000000in}{0.000000in}}%
\pgfpathlineto{\pgfqpoint{0.000000in}{-0.041667in}}%
\pgfusepath{stroke,fill}%
}%
\begin{pgfscope}%
\pgfsys@transformshift{0.470525in}{2.049193in}%
\pgfsys@useobject{currentmarker}{}%
\end{pgfscope}%
\end{pgfscope}%
\begin{pgfscope}%
\definecolor{textcolor}{rgb}{0.000000,0.000000,0.000000}%
\pgfsetstrokecolor{textcolor}%
\pgfsetfillcolor{textcolor}%
\pgftext[x=0.470525in,y=0.374381in,,top]{\color{textcolor}\rmfamily\fontsize{10.000000}{12.000000}\selectfont \(\displaystyle {0}\)}%
\end{pgfscope}%
\begin{pgfscope}%
\pgfsetbuttcap%
\pgfsetroundjoin%
\definecolor{currentfill}{rgb}{0.000000,0.000000,0.000000}%
\pgfsetfillcolor{currentfill}%
\pgfsetlinewidth{0.501875pt}%
\definecolor{currentstroke}{rgb}{0.000000,0.000000,0.000000}%
\pgfsetstrokecolor{currentstroke}%
\pgfsetdash{}{0pt}%
\pgfsys@defobject{currentmarker}{\pgfqpoint{0.000000in}{0.000000in}}{\pgfqpoint{0.000000in}{0.041667in}}{%
\pgfpathmoveto{\pgfqpoint{0.000000in}{0.000000in}}%
\pgfpathlineto{\pgfqpoint{0.000000in}{0.041667in}}%
\pgfusepath{stroke,fill}%
}%
\begin{pgfscope}%
\pgfsys@transformshift{0.921231in}{0.422992in}%
\pgfsys@useobject{currentmarker}{}%
\end{pgfscope}%
\end{pgfscope}%
\begin{pgfscope}%
\pgfsetbuttcap%
\pgfsetroundjoin%
\definecolor{currentfill}{rgb}{0.000000,0.000000,0.000000}%
\pgfsetfillcolor{currentfill}%
\pgfsetlinewidth{0.501875pt}%
\definecolor{currentstroke}{rgb}{0.000000,0.000000,0.000000}%
\pgfsetstrokecolor{currentstroke}%
\pgfsetdash{}{0pt}%
\pgfsys@defobject{currentmarker}{\pgfqpoint{0.000000in}{-0.041667in}}{\pgfqpoint{0.000000in}{0.000000in}}{%
\pgfpathmoveto{\pgfqpoint{0.000000in}{0.000000in}}%
\pgfpathlineto{\pgfqpoint{0.000000in}{-0.041667in}}%
\pgfusepath{stroke,fill}%
}%
\begin{pgfscope}%
\pgfsys@transformshift{0.921231in}{2.049193in}%
\pgfsys@useobject{currentmarker}{}%
\end{pgfscope}%
\end{pgfscope}%
\begin{pgfscope}%
\definecolor{textcolor}{rgb}{0.000000,0.000000,0.000000}%
\pgfsetstrokecolor{textcolor}%
\pgfsetfillcolor{textcolor}%
\pgftext[x=0.921231in,y=0.374381in,,top]{\color{textcolor}\rmfamily\fontsize{10.000000}{12.000000}\selectfont \(\displaystyle {20}\)}%
\end{pgfscope}%
\begin{pgfscope}%
\pgfsetbuttcap%
\pgfsetroundjoin%
\definecolor{currentfill}{rgb}{0.000000,0.000000,0.000000}%
\pgfsetfillcolor{currentfill}%
\pgfsetlinewidth{0.501875pt}%
\definecolor{currentstroke}{rgb}{0.000000,0.000000,0.000000}%
\pgfsetstrokecolor{currentstroke}%
\pgfsetdash{}{0pt}%
\pgfsys@defobject{currentmarker}{\pgfqpoint{0.000000in}{0.000000in}}{\pgfqpoint{0.000000in}{0.041667in}}{%
\pgfpathmoveto{\pgfqpoint{0.000000in}{0.000000in}}%
\pgfpathlineto{\pgfqpoint{0.000000in}{0.041667in}}%
\pgfusepath{stroke,fill}%
}%
\begin{pgfscope}%
\pgfsys@transformshift{1.371937in}{0.422992in}%
\pgfsys@useobject{currentmarker}{}%
\end{pgfscope}%
\end{pgfscope}%
\begin{pgfscope}%
\pgfsetbuttcap%
\pgfsetroundjoin%
\definecolor{currentfill}{rgb}{0.000000,0.000000,0.000000}%
\pgfsetfillcolor{currentfill}%
\pgfsetlinewidth{0.501875pt}%
\definecolor{currentstroke}{rgb}{0.000000,0.000000,0.000000}%
\pgfsetstrokecolor{currentstroke}%
\pgfsetdash{}{0pt}%
\pgfsys@defobject{currentmarker}{\pgfqpoint{0.000000in}{-0.041667in}}{\pgfqpoint{0.000000in}{0.000000in}}{%
\pgfpathmoveto{\pgfqpoint{0.000000in}{0.000000in}}%
\pgfpathlineto{\pgfqpoint{0.000000in}{-0.041667in}}%
\pgfusepath{stroke,fill}%
}%
\begin{pgfscope}%
\pgfsys@transformshift{1.371937in}{2.049193in}%
\pgfsys@useobject{currentmarker}{}%
\end{pgfscope}%
\end{pgfscope}%
\begin{pgfscope}%
\definecolor{textcolor}{rgb}{0.000000,0.000000,0.000000}%
\pgfsetstrokecolor{textcolor}%
\pgfsetfillcolor{textcolor}%
\pgftext[x=1.371937in,y=0.374381in,,top]{\color{textcolor}\rmfamily\fontsize{10.000000}{12.000000}\selectfont \(\displaystyle {40}\)}%
\end{pgfscope}%
\begin{pgfscope}%
\pgfsetbuttcap%
\pgfsetroundjoin%
\definecolor{currentfill}{rgb}{0.000000,0.000000,0.000000}%
\pgfsetfillcolor{currentfill}%
\pgfsetlinewidth{0.501875pt}%
\definecolor{currentstroke}{rgb}{0.000000,0.000000,0.000000}%
\pgfsetstrokecolor{currentstroke}%
\pgfsetdash{}{0pt}%
\pgfsys@defobject{currentmarker}{\pgfqpoint{0.000000in}{0.000000in}}{\pgfqpoint{0.000000in}{0.041667in}}{%
\pgfpathmoveto{\pgfqpoint{0.000000in}{0.000000in}}%
\pgfpathlineto{\pgfqpoint{0.000000in}{0.041667in}}%
\pgfusepath{stroke,fill}%
}%
\begin{pgfscope}%
\pgfsys@transformshift{1.822643in}{0.422992in}%
\pgfsys@useobject{currentmarker}{}%
\end{pgfscope}%
\end{pgfscope}%
\begin{pgfscope}%
\pgfsetbuttcap%
\pgfsetroundjoin%
\definecolor{currentfill}{rgb}{0.000000,0.000000,0.000000}%
\pgfsetfillcolor{currentfill}%
\pgfsetlinewidth{0.501875pt}%
\definecolor{currentstroke}{rgb}{0.000000,0.000000,0.000000}%
\pgfsetstrokecolor{currentstroke}%
\pgfsetdash{}{0pt}%
\pgfsys@defobject{currentmarker}{\pgfqpoint{0.000000in}{-0.041667in}}{\pgfqpoint{0.000000in}{0.000000in}}{%
\pgfpathmoveto{\pgfqpoint{0.000000in}{0.000000in}}%
\pgfpathlineto{\pgfqpoint{0.000000in}{-0.041667in}}%
\pgfusepath{stroke,fill}%
}%
\begin{pgfscope}%
\pgfsys@transformshift{1.822643in}{2.049193in}%
\pgfsys@useobject{currentmarker}{}%
\end{pgfscope}%
\end{pgfscope}%
\begin{pgfscope}%
\definecolor{textcolor}{rgb}{0.000000,0.000000,0.000000}%
\pgfsetstrokecolor{textcolor}%
\pgfsetfillcolor{textcolor}%
\pgftext[x=1.822643in,y=0.374381in,,top]{\color{textcolor}\rmfamily\fontsize{10.000000}{12.000000}\selectfont \(\displaystyle {60}\)}%
\end{pgfscope}%
\begin{pgfscope}%
\pgfsetbuttcap%
\pgfsetroundjoin%
\definecolor{currentfill}{rgb}{0.000000,0.000000,0.000000}%
\pgfsetfillcolor{currentfill}%
\pgfsetlinewidth{0.501875pt}%
\definecolor{currentstroke}{rgb}{0.000000,0.000000,0.000000}%
\pgfsetstrokecolor{currentstroke}%
\pgfsetdash{}{0pt}%
\pgfsys@defobject{currentmarker}{\pgfqpoint{0.000000in}{0.000000in}}{\pgfqpoint{0.000000in}{0.041667in}}{%
\pgfpathmoveto{\pgfqpoint{0.000000in}{0.000000in}}%
\pgfpathlineto{\pgfqpoint{0.000000in}{0.041667in}}%
\pgfusepath{stroke,fill}%
}%
\begin{pgfscope}%
\pgfsys@transformshift{2.273348in}{0.422992in}%
\pgfsys@useobject{currentmarker}{}%
\end{pgfscope}%
\end{pgfscope}%
\begin{pgfscope}%
\pgfsetbuttcap%
\pgfsetroundjoin%
\definecolor{currentfill}{rgb}{0.000000,0.000000,0.000000}%
\pgfsetfillcolor{currentfill}%
\pgfsetlinewidth{0.501875pt}%
\definecolor{currentstroke}{rgb}{0.000000,0.000000,0.000000}%
\pgfsetstrokecolor{currentstroke}%
\pgfsetdash{}{0pt}%
\pgfsys@defobject{currentmarker}{\pgfqpoint{0.000000in}{-0.041667in}}{\pgfqpoint{0.000000in}{0.000000in}}{%
\pgfpathmoveto{\pgfqpoint{0.000000in}{0.000000in}}%
\pgfpathlineto{\pgfqpoint{0.000000in}{-0.041667in}}%
\pgfusepath{stroke,fill}%
}%
\begin{pgfscope}%
\pgfsys@transformshift{2.273348in}{2.049193in}%
\pgfsys@useobject{currentmarker}{}%
\end{pgfscope}%
\end{pgfscope}%
\begin{pgfscope}%
\definecolor{textcolor}{rgb}{0.000000,0.000000,0.000000}%
\pgfsetstrokecolor{textcolor}%
\pgfsetfillcolor{textcolor}%
\pgftext[x=2.273348in,y=0.374381in,,top]{\color{textcolor}\rmfamily\fontsize{10.000000}{12.000000}\selectfont \(\displaystyle {80}\)}%
\end{pgfscope}%
\begin{pgfscope}%
\pgfsetbuttcap%
\pgfsetroundjoin%
\definecolor{currentfill}{rgb}{0.000000,0.000000,0.000000}%
\pgfsetfillcolor{currentfill}%
\pgfsetlinewidth{0.501875pt}%
\definecolor{currentstroke}{rgb}{0.000000,0.000000,0.000000}%
\pgfsetstrokecolor{currentstroke}%
\pgfsetdash{}{0pt}%
\pgfsys@defobject{currentmarker}{\pgfqpoint{0.000000in}{0.000000in}}{\pgfqpoint{0.000000in}{0.020833in}}{%
\pgfpathmoveto{\pgfqpoint{0.000000in}{0.000000in}}%
\pgfpathlineto{\pgfqpoint{0.000000in}{0.020833in}}%
\pgfusepath{stroke,fill}%
}%
\begin{pgfscope}%
\pgfsys@transformshift{0.583202in}{0.422992in}%
\pgfsys@useobject{currentmarker}{}%
\end{pgfscope}%
\end{pgfscope}%
\begin{pgfscope}%
\pgfsetbuttcap%
\pgfsetroundjoin%
\definecolor{currentfill}{rgb}{0.000000,0.000000,0.000000}%
\pgfsetfillcolor{currentfill}%
\pgfsetlinewidth{0.501875pt}%
\definecolor{currentstroke}{rgb}{0.000000,0.000000,0.000000}%
\pgfsetstrokecolor{currentstroke}%
\pgfsetdash{}{0pt}%
\pgfsys@defobject{currentmarker}{\pgfqpoint{0.000000in}{-0.020833in}}{\pgfqpoint{0.000000in}{0.000000in}}{%
\pgfpathmoveto{\pgfqpoint{0.000000in}{0.000000in}}%
\pgfpathlineto{\pgfqpoint{0.000000in}{-0.020833in}}%
\pgfusepath{stroke,fill}%
}%
\begin{pgfscope}%
\pgfsys@transformshift{0.583202in}{2.049193in}%
\pgfsys@useobject{currentmarker}{}%
\end{pgfscope}%
\end{pgfscope}%
\begin{pgfscope}%
\pgfsetbuttcap%
\pgfsetroundjoin%
\definecolor{currentfill}{rgb}{0.000000,0.000000,0.000000}%
\pgfsetfillcolor{currentfill}%
\pgfsetlinewidth{0.501875pt}%
\definecolor{currentstroke}{rgb}{0.000000,0.000000,0.000000}%
\pgfsetstrokecolor{currentstroke}%
\pgfsetdash{}{0pt}%
\pgfsys@defobject{currentmarker}{\pgfqpoint{0.000000in}{0.000000in}}{\pgfqpoint{0.000000in}{0.020833in}}{%
\pgfpathmoveto{\pgfqpoint{0.000000in}{0.000000in}}%
\pgfpathlineto{\pgfqpoint{0.000000in}{0.020833in}}%
\pgfusepath{stroke,fill}%
}%
\begin{pgfscope}%
\pgfsys@transformshift{0.695878in}{0.422992in}%
\pgfsys@useobject{currentmarker}{}%
\end{pgfscope}%
\end{pgfscope}%
\begin{pgfscope}%
\pgfsetbuttcap%
\pgfsetroundjoin%
\definecolor{currentfill}{rgb}{0.000000,0.000000,0.000000}%
\pgfsetfillcolor{currentfill}%
\pgfsetlinewidth{0.501875pt}%
\definecolor{currentstroke}{rgb}{0.000000,0.000000,0.000000}%
\pgfsetstrokecolor{currentstroke}%
\pgfsetdash{}{0pt}%
\pgfsys@defobject{currentmarker}{\pgfqpoint{0.000000in}{-0.020833in}}{\pgfqpoint{0.000000in}{0.000000in}}{%
\pgfpathmoveto{\pgfqpoint{0.000000in}{0.000000in}}%
\pgfpathlineto{\pgfqpoint{0.000000in}{-0.020833in}}%
\pgfusepath{stroke,fill}%
}%
\begin{pgfscope}%
\pgfsys@transformshift{0.695878in}{2.049193in}%
\pgfsys@useobject{currentmarker}{}%
\end{pgfscope}%
\end{pgfscope}%
\begin{pgfscope}%
\pgfsetbuttcap%
\pgfsetroundjoin%
\definecolor{currentfill}{rgb}{0.000000,0.000000,0.000000}%
\pgfsetfillcolor{currentfill}%
\pgfsetlinewidth{0.501875pt}%
\definecolor{currentstroke}{rgb}{0.000000,0.000000,0.000000}%
\pgfsetstrokecolor{currentstroke}%
\pgfsetdash{}{0pt}%
\pgfsys@defobject{currentmarker}{\pgfqpoint{0.000000in}{0.000000in}}{\pgfqpoint{0.000000in}{0.020833in}}{%
\pgfpathmoveto{\pgfqpoint{0.000000in}{0.000000in}}%
\pgfpathlineto{\pgfqpoint{0.000000in}{0.020833in}}%
\pgfusepath{stroke,fill}%
}%
\begin{pgfscope}%
\pgfsys@transformshift{0.808555in}{0.422992in}%
\pgfsys@useobject{currentmarker}{}%
\end{pgfscope}%
\end{pgfscope}%
\begin{pgfscope}%
\pgfsetbuttcap%
\pgfsetroundjoin%
\definecolor{currentfill}{rgb}{0.000000,0.000000,0.000000}%
\pgfsetfillcolor{currentfill}%
\pgfsetlinewidth{0.501875pt}%
\definecolor{currentstroke}{rgb}{0.000000,0.000000,0.000000}%
\pgfsetstrokecolor{currentstroke}%
\pgfsetdash{}{0pt}%
\pgfsys@defobject{currentmarker}{\pgfqpoint{0.000000in}{-0.020833in}}{\pgfqpoint{0.000000in}{0.000000in}}{%
\pgfpathmoveto{\pgfqpoint{0.000000in}{0.000000in}}%
\pgfpathlineto{\pgfqpoint{0.000000in}{-0.020833in}}%
\pgfusepath{stroke,fill}%
}%
\begin{pgfscope}%
\pgfsys@transformshift{0.808555in}{2.049193in}%
\pgfsys@useobject{currentmarker}{}%
\end{pgfscope}%
\end{pgfscope}%
\begin{pgfscope}%
\pgfsetbuttcap%
\pgfsetroundjoin%
\definecolor{currentfill}{rgb}{0.000000,0.000000,0.000000}%
\pgfsetfillcolor{currentfill}%
\pgfsetlinewidth{0.501875pt}%
\definecolor{currentstroke}{rgb}{0.000000,0.000000,0.000000}%
\pgfsetstrokecolor{currentstroke}%
\pgfsetdash{}{0pt}%
\pgfsys@defobject{currentmarker}{\pgfqpoint{0.000000in}{0.000000in}}{\pgfqpoint{0.000000in}{0.020833in}}{%
\pgfpathmoveto{\pgfqpoint{0.000000in}{0.000000in}}%
\pgfpathlineto{\pgfqpoint{0.000000in}{0.020833in}}%
\pgfusepath{stroke,fill}%
}%
\begin{pgfscope}%
\pgfsys@transformshift{1.033907in}{0.422992in}%
\pgfsys@useobject{currentmarker}{}%
\end{pgfscope}%
\end{pgfscope}%
\begin{pgfscope}%
\pgfsetbuttcap%
\pgfsetroundjoin%
\definecolor{currentfill}{rgb}{0.000000,0.000000,0.000000}%
\pgfsetfillcolor{currentfill}%
\pgfsetlinewidth{0.501875pt}%
\definecolor{currentstroke}{rgb}{0.000000,0.000000,0.000000}%
\pgfsetstrokecolor{currentstroke}%
\pgfsetdash{}{0pt}%
\pgfsys@defobject{currentmarker}{\pgfqpoint{0.000000in}{-0.020833in}}{\pgfqpoint{0.000000in}{0.000000in}}{%
\pgfpathmoveto{\pgfqpoint{0.000000in}{0.000000in}}%
\pgfpathlineto{\pgfqpoint{0.000000in}{-0.020833in}}%
\pgfusepath{stroke,fill}%
}%
\begin{pgfscope}%
\pgfsys@transformshift{1.033907in}{2.049193in}%
\pgfsys@useobject{currentmarker}{}%
\end{pgfscope}%
\end{pgfscope}%
\begin{pgfscope}%
\pgfsetbuttcap%
\pgfsetroundjoin%
\definecolor{currentfill}{rgb}{0.000000,0.000000,0.000000}%
\pgfsetfillcolor{currentfill}%
\pgfsetlinewidth{0.501875pt}%
\definecolor{currentstroke}{rgb}{0.000000,0.000000,0.000000}%
\pgfsetstrokecolor{currentstroke}%
\pgfsetdash{}{0pt}%
\pgfsys@defobject{currentmarker}{\pgfqpoint{0.000000in}{0.000000in}}{\pgfqpoint{0.000000in}{0.020833in}}{%
\pgfpathmoveto{\pgfqpoint{0.000000in}{0.000000in}}%
\pgfpathlineto{\pgfqpoint{0.000000in}{0.020833in}}%
\pgfusepath{stroke,fill}%
}%
\begin{pgfscope}%
\pgfsys@transformshift{1.146584in}{0.422992in}%
\pgfsys@useobject{currentmarker}{}%
\end{pgfscope}%
\end{pgfscope}%
\begin{pgfscope}%
\pgfsetbuttcap%
\pgfsetroundjoin%
\definecolor{currentfill}{rgb}{0.000000,0.000000,0.000000}%
\pgfsetfillcolor{currentfill}%
\pgfsetlinewidth{0.501875pt}%
\definecolor{currentstroke}{rgb}{0.000000,0.000000,0.000000}%
\pgfsetstrokecolor{currentstroke}%
\pgfsetdash{}{0pt}%
\pgfsys@defobject{currentmarker}{\pgfqpoint{0.000000in}{-0.020833in}}{\pgfqpoint{0.000000in}{0.000000in}}{%
\pgfpathmoveto{\pgfqpoint{0.000000in}{0.000000in}}%
\pgfpathlineto{\pgfqpoint{0.000000in}{-0.020833in}}%
\pgfusepath{stroke,fill}%
}%
\begin{pgfscope}%
\pgfsys@transformshift{1.146584in}{2.049193in}%
\pgfsys@useobject{currentmarker}{}%
\end{pgfscope}%
\end{pgfscope}%
\begin{pgfscope}%
\pgfsetbuttcap%
\pgfsetroundjoin%
\definecolor{currentfill}{rgb}{0.000000,0.000000,0.000000}%
\pgfsetfillcolor{currentfill}%
\pgfsetlinewidth{0.501875pt}%
\definecolor{currentstroke}{rgb}{0.000000,0.000000,0.000000}%
\pgfsetstrokecolor{currentstroke}%
\pgfsetdash{}{0pt}%
\pgfsys@defobject{currentmarker}{\pgfqpoint{0.000000in}{0.000000in}}{\pgfqpoint{0.000000in}{0.020833in}}{%
\pgfpathmoveto{\pgfqpoint{0.000000in}{0.000000in}}%
\pgfpathlineto{\pgfqpoint{0.000000in}{0.020833in}}%
\pgfusepath{stroke,fill}%
}%
\begin{pgfscope}%
\pgfsys@transformshift{1.259260in}{0.422992in}%
\pgfsys@useobject{currentmarker}{}%
\end{pgfscope}%
\end{pgfscope}%
\begin{pgfscope}%
\pgfsetbuttcap%
\pgfsetroundjoin%
\definecolor{currentfill}{rgb}{0.000000,0.000000,0.000000}%
\pgfsetfillcolor{currentfill}%
\pgfsetlinewidth{0.501875pt}%
\definecolor{currentstroke}{rgb}{0.000000,0.000000,0.000000}%
\pgfsetstrokecolor{currentstroke}%
\pgfsetdash{}{0pt}%
\pgfsys@defobject{currentmarker}{\pgfqpoint{0.000000in}{-0.020833in}}{\pgfqpoint{0.000000in}{0.000000in}}{%
\pgfpathmoveto{\pgfqpoint{0.000000in}{0.000000in}}%
\pgfpathlineto{\pgfqpoint{0.000000in}{-0.020833in}}%
\pgfusepath{stroke,fill}%
}%
\begin{pgfscope}%
\pgfsys@transformshift{1.259260in}{2.049193in}%
\pgfsys@useobject{currentmarker}{}%
\end{pgfscope}%
\end{pgfscope}%
\begin{pgfscope}%
\pgfsetbuttcap%
\pgfsetroundjoin%
\definecolor{currentfill}{rgb}{0.000000,0.000000,0.000000}%
\pgfsetfillcolor{currentfill}%
\pgfsetlinewidth{0.501875pt}%
\definecolor{currentstroke}{rgb}{0.000000,0.000000,0.000000}%
\pgfsetstrokecolor{currentstroke}%
\pgfsetdash{}{0pt}%
\pgfsys@defobject{currentmarker}{\pgfqpoint{0.000000in}{0.000000in}}{\pgfqpoint{0.000000in}{0.020833in}}{%
\pgfpathmoveto{\pgfqpoint{0.000000in}{0.000000in}}%
\pgfpathlineto{\pgfqpoint{0.000000in}{0.020833in}}%
\pgfusepath{stroke,fill}%
}%
\begin{pgfscope}%
\pgfsys@transformshift{1.484613in}{0.422992in}%
\pgfsys@useobject{currentmarker}{}%
\end{pgfscope}%
\end{pgfscope}%
\begin{pgfscope}%
\pgfsetbuttcap%
\pgfsetroundjoin%
\definecolor{currentfill}{rgb}{0.000000,0.000000,0.000000}%
\pgfsetfillcolor{currentfill}%
\pgfsetlinewidth{0.501875pt}%
\definecolor{currentstroke}{rgb}{0.000000,0.000000,0.000000}%
\pgfsetstrokecolor{currentstroke}%
\pgfsetdash{}{0pt}%
\pgfsys@defobject{currentmarker}{\pgfqpoint{0.000000in}{-0.020833in}}{\pgfqpoint{0.000000in}{0.000000in}}{%
\pgfpathmoveto{\pgfqpoint{0.000000in}{0.000000in}}%
\pgfpathlineto{\pgfqpoint{0.000000in}{-0.020833in}}%
\pgfusepath{stroke,fill}%
}%
\begin{pgfscope}%
\pgfsys@transformshift{1.484613in}{2.049193in}%
\pgfsys@useobject{currentmarker}{}%
\end{pgfscope}%
\end{pgfscope}%
\begin{pgfscope}%
\pgfsetbuttcap%
\pgfsetroundjoin%
\definecolor{currentfill}{rgb}{0.000000,0.000000,0.000000}%
\pgfsetfillcolor{currentfill}%
\pgfsetlinewidth{0.501875pt}%
\definecolor{currentstroke}{rgb}{0.000000,0.000000,0.000000}%
\pgfsetstrokecolor{currentstroke}%
\pgfsetdash{}{0pt}%
\pgfsys@defobject{currentmarker}{\pgfqpoint{0.000000in}{0.000000in}}{\pgfqpoint{0.000000in}{0.020833in}}{%
\pgfpathmoveto{\pgfqpoint{0.000000in}{0.000000in}}%
\pgfpathlineto{\pgfqpoint{0.000000in}{0.020833in}}%
\pgfusepath{stroke,fill}%
}%
\begin{pgfscope}%
\pgfsys@transformshift{1.597290in}{0.422992in}%
\pgfsys@useobject{currentmarker}{}%
\end{pgfscope}%
\end{pgfscope}%
\begin{pgfscope}%
\pgfsetbuttcap%
\pgfsetroundjoin%
\definecolor{currentfill}{rgb}{0.000000,0.000000,0.000000}%
\pgfsetfillcolor{currentfill}%
\pgfsetlinewidth{0.501875pt}%
\definecolor{currentstroke}{rgb}{0.000000,0.000000,0.000000}%
\pgfsetstrokecolor{currentstroke}%
\pgfsetdash{}{0pt}%
\pgfsys@defobject{currentmarker}{\pgfqpoint{0.000000in}{-0.020833in}}{\pgfqpoint{0.000000in}{0.000000in}}{%
\pgfpathmoveto{\pgfqpoint{0.000000in}{0.000000in}}%
\pgfpathlineto{\pgfqpoint{0.000000in}{-0.020833in}}%
\pgfusepath{stroke,fill}%
}%
\begin{pgfscope}%
\pgfsys@transformshift{1.597290in}{2.049193in}%
\pgfsys@useobject{currentmarker}{}%
\end{pgfscope}%
\end{pgfscope}%
\begin{pgfscope}%
\pgfsetbuttcap%
\pgfsetroundjoin%
\definecolor{currentfill}{rgb}{0.000000,0.000000,0.000000}%
\pgfsetfillcolor{currentfill}%
\pgfsetlinewidth{0.501875pt}%
\definecolor{currentstroke}{rgb}{0.000000,0.000000,0.000000}%
\pgfsetstrokecolor{currentstroke}%
\pgfsetdash{}{0pt}%
\pgfsys@defobject{currentmarker}{\pgfqpoint{0.000000in}{0.000000in}}{\pgfqpoint{0.000000in}{0.020833in}}{%
\pgfpathmoveto{\pgfqpoint{0.000000in}{0.000000in}}%
\pgfpathlineto{\pgfqpoint{0.000000in}{0.020833in}}%
\pgfusepath{stroke,fill}%
}%
\begin{pgfscope}%
\pgfsys@transformshift{1.709966in}{0.422992in}%
\pgfsys@useobject{currentmarker}{}%
\end{pgfscope}%
\end{pgfscope}%
\begin{pgfscope}%
\pgfsetbuttcap%
\pgfsetroundjoin%
\definecolor{currentfill}{rgb}{0.000000,0.000000,0.000000}%
\pgfsetfillcolor{currentfill}%
\pgfsetlinewidth{0.501875pt}%
\definecolor{currentstroke}{rgb}{0.000000,0.000000,0.000000}%
\pgfsetstrokecolor{currentstroke}%
\pgfsetdash{}{0pt}%
\pgfsys@defobject{currentmarker}{\pgfqpoint{0.000000in}{-0.020833in}}{\pgfqpoint{0.000000in}{0.000000in}}{%
\pgfpathmoveto{\pgfqpoint{0.000000in}{0.000000in}}%
\pgfpathlineto{\pgfqpoint{0.000000in}{-0.020833in}}%
\pgfusepath{stroke,fill}%
}%
\begin{pgfscope}%
\pgfsys@transformshift{1.709966in}{2.049193in}%
\pgfsys@useobject{currentmarker}{}%
\end{pgfscope}%
\end{pgfscope}%
\begin{pgfscope}%
\pgfsetbuttcap%
\pgfsetroundjoin%
\definecolor{currentfill}{rgb}{0.000000,0.000000,0.000000}%
\pgfsetfillcolor{currentfill}%
\pgfsetlinewidth{0.501875pt}%
\definecolor{currentstroke}{rgb}{0.000000,0.000000,0.000000}%
\pgfsetstrokecolor{currentstroke}%
\pgfsetdash{}{0pt}%
\pgfsys@defobject{currentmarker}{\pgfqpoint{0.000000in}{0.000000in}}{\pgfqpoint{0.000000in}{0.020833in}}{%
\pgfpathmoveto{\pgfqpoint{0.000000in}{0.000000in}}%
\pgfpathlineto{\pgfqpoint{0.000000in}{0.020833in}}%
\pgfusepath{stroke,fill}%
}%
\begin{pgfscope}%
\pgfsys@transformshift{1.935319in}{0.422992in}%
\pgfsys@useobject{currentmarker}{}%
\end{pgfscope}%
\end{pgfscope}%
\begin{pgfscope}%
\pgfsetbuttcap%
\pgfsetroundjoin%
\definecolor{currentfill}{rgb}{0.000000,0.000000,0.000000}%
\pgfsetfillcolor{currentfill}%
\pgfsetlinewidth{0.501875pt}%
\definecolor{currentstroke}{rgb}{0.000000,0.000000,0.000000}%
\pgfsetstrokecolor{currentstroke}%
\pgfsetdash{}{0pt}%
\pgfsys@defobject{currentmarker}{\pgfqpoint{0.000000in}{-0.020833in}}{\pgfqpoint{0.000000in}{0.000000in}}{%
\pgfpathmoveto{\pgfqpoint{0.000000in}{0.000000in}}%
\pgfpathlineto{\pgfqpoint{0.000000in}{-0.020833in}}%
\pgfusepath{stroke,fill}%
}%
\begin{pgfscope}%
\pgfsys@transformshift{1.935319in}{2.049193in}%
\pgfsys@useobject{currentmarker}{}%
\end{pgfscope}%
\end{pgfscope}%
\begin{pgfscope}%
\pgfsetbuttcap%
\pgfsetroundjoin%
\definecolor{currentfill}{rgb}{0.000000,0.000000,0.000000}%
\pgfsetfillcolor{currentfill}%
\pgfsetlinewidth{0.501875pt}%
\definecolor{currentstroke}{rgb}{0.000000,0.000000,0.000000}%
\pgfsetstrokecolor{currentstroke}%
\pgfsetdash{}{0pt}%
\pgfsys@defobject{currentmarker}{\pgfqpoint{0.000000in}{0.000000in}}{\pgfqpoint{0.000000in}{0.020833in}}{%
\pgfpathmoveto{\pgfqpoint{0.000000in}{0.000000in}}%
\pgfpathlineto{\pgfqpoint{0.000000in}{0.020833in}}%
\pgfusepath{stroke,fill}%
}%
\begin{pgfscope}%
\pgfsys@transformshift{2.047995in}{0.422992in}%
\pgfsys@useobject{currentmarker}{}%
\end{pgfscope}%
\end{pgfscope}%
\begin{pgfscope}%
\pgfsetbuttcap%
\pgfsetroundjoin%
\definecolor{currentfill}{rgb}{0.000000,0.000000,0.000000}%
\pgfsetfillcolor{currentfill}%
\pgfsetlinewidth{0.501875pt}%
\definecolor{currentstroke}{rgb}{0.000000,0.000000,0.000000}%
\pgfsetstrokecolor{currentstroke}%
\pgfsetdash{}{0pt}%
\pgfsys@defobject{currentmarker}{\pgfqpoint{0.000000in}{-0.020833in}}{\pgfqpoint{0.000000in}{0.000000in}}{%
\pgfpathmoveto{\pgfqpoint{0.000000in}{0.000000in}}%
\pgfpathlineto{\pgfqpoint{0.000000in}{-0.020833in}}%
\pgfusepath{stroke,fill}%
}%
\begin{pgfscope}%
\pgfsys@transformshift{2.047995in}{2.049193in}%
\pgfsys@useobject{currentmarker}{}%
\end{pgfscope}%
\end{pgfscope}%
\begin{pgfscope}%
\pgfsetbuttcap%
\pgfsetroundjoin%
\definecolor{currentfill}{rgb}{0.000000,0.000000,0.000000}%
\pgfsetfillcolor{currentfill}%
\pgfsetlinewidth{0.501875pt}%
\definecolor{currentstroke}{rgb}{0.000000,0.000000,0.000000}%
\pgfsetstrokecolor{currentstroke}%
\pgfsetdash{}{0pt}%
\pgfsys@defobject{currentmarker}{\pgfqpoint{0.000000in}{0.000000in}}{\pgfqpoint{0.000000in}{0.020833in}}{%
\pgfpathmoveto{\pgfqpoint{0.000000in}{0.000000in}}%
\pgfpathlineto{\pgfqpoint{0.000000in}{0.020833in}}%
\pgfusepath{stroke,fill}%
}%
\begin{pgfscope}%
\pgfsys@transformshift{2.160672in}{0.422992in}%
\pgfsys@useobject{currentmarker}{}%
\end{pgfscope}%
\end{pgfscope}%
\begin{pgfscope}%
\pgfsetbuttcap%
\pgfsetroundjoin%
\definecolor{currentfill}{rgb}{0.000000,0.000000,0.000000}%
\pgfsetfillcolor{currentfill}%
\pgfsetlinewidth{0.501875pt}%
\definecolor{currentstroke}{rgb}{0.000000,0.000000,0.000000}%
\pgfsetstrokecolor{currentstroke}%
\pgfsetdash{}{0pt}%
\pgfsys@defobject{currentmarker}{\pgfqpoint{0.000000in}{-0.020833in}}{\pgfqpoint{0.000000in}{0.000000in}}{%
\pgfpathmoveto{\pgfqpoint{0.000000in}{0.000000in}}%
\pgfpathlineto{\pgfqpoint{0.000000in}{-0.020833in}}%
\pgfusepath{stroke,fill}%
}%
\begin{pgfscope}%
\pgfsys@transformshift{2.160672in}{2.049193in}%
\pgfsys@useobject{currentmarker}{}%
\end{pgfscope}%
\end{pgfscope}%
\begin{pgfscope}%
\pgfsetbuttcap%
\pgfsetroundjoin%
\definecolor{currentfill}{rgb}{0.000000,0.000000,0.000000}%
\pgfsetfillcolor{currentfill}%
\pgfsetlinewidth{0.501875pt}%
\definecolor{currentstroke}{rgb}{0.000000,0.000000,0.000000}%
\pgfsetstrokecolor{currentstroke}%
\pgfsetdash{}{0pt}%
\pgfsys@defobject{currentmarker}{\pgfqpoint{0.000000in}{0.000000in}}{\pgfqpoint{0.000000in}{0.020833in}}{%
\pgfpathmoveto{\pgfqpoint{0.000000in}{0.000000in}}%
\pgfpathlineto{\pgfqpoint{0.000000in}{0.020833in}}%
\pgfusepath{stroke,fill}%
}%
\begin{pgfscope}%
\pgfsys@transformshift{2.386025in}{0.422992in}%
\pgfsys@useobject{currentmarker}{}%
\end{pgfscope}%
\end{pgfscope}%
\begin{pgfscope}%
\pgfsetbuttcap%
\pgfsetroundjoin%
\definecolor{currentfill}{rgb}{0.000000,0.000000,0.000000}%
\pgfsetfillcolor{currentfill}%
\pgfsetlinewidth{0.501875pt}%
\definecolor{currentstroke}{rgb}{0.000000,0.000000,0.000000}%
\pgfsetstrokecolor{currentstroke}%
\pgfsetdash{}{0pt}%
\pgfsys@defobject{currentmarker}{\pgfqpoint{0.000000in}{-0.020833in}}{\pgfqpoint{0.000000in}{0.000000in}}{%
\pgfpathmoveto{\pgfqpoint{0.000000in}{0.000000in}}%
\pgfpathlineto{\pgfqpoint{0.000000in}{-0.020833in}}%
\pgfusepath{stroke,fill}%
}%
\begin{pgfscope}%
\pgfsys@transformshift{2.386025in}{2.049193in}%
\pgfsys@useobject{currentmarker}{}%
\end{pgfscope}%
\end{pgfscope}%
\begin{pgfscope}%
\pgfsetbuttcap%
\pgfsetroundjoin%
\definecolor{currentfill}{rgb}{0.000000,0.000000,0.000000}%
\pgfsetfillcolor{currentfill}%
\pgfsetlinewidth{0.501875pt}%
\definecolor{currentstroke}{rgb}{0.000000,0.000000,0.000000}%
\pgfsetstrokecolor{currentstroke}%
\pgfsetdash{}{0pt}%
\pgfsys@defobject{currentmarker}{\pgfqpoint{0.000000in}{0.000000in}}{\pgfqpoint{0.000000in}{0.020833in}}{%
\pgfpathmoveto{\pgfqpoint{0.000000in}{0.000000in}}%
\pgfpathlineto{\pgfqpoint{0.000000in}{0.020833in}}%
\pgfusepath{stroke,fill}%
}%
\begin{pgfscope}%
\pgfsys@transformshift{2.498701in}{0.422992in}%
\pgfsys@useobject{currentmarker}{}%
\end{pgfscope}%
\end{pgfscope}%
\begin{pgfscope}%
\pgfsetbuttcap%
\pgfsetroundjoin%
\definecolor{currentfill}{rgb}{0.000000,0.000000,0.000000}%
\pgfsetfillcolor{currentfill}%
\pgfsetlinewidth{0.501875pt}%
\definecolor{currentstroke}{rgb}{0.000000,0.000000,0.000000}%
\pgfsetstrokecolor{currentstroke}%
\pgfsetdash{}{0pt}%
\pgfsys@defobject{currentmarker}{\pgfqpoint{0.000000in}{-0.020833in}}{\pgfqpoint{0.000000in}{0.000000in}}{%
\pgfpathmoveto{\pgfqpoint{0.000000in}{0.000000in}}%
\pgfpathlineto{\pgfqpoint{0.000000in}{-0.020833in}}%
\pgfusepath{stroke,fill}%
}%
\begin{pgfscope}%
\pgfsys@transformshift{2.498701in}{2.049193in}%
\pgfsys@useobject{currentmarker}{}%
\end{pgfscope}%
\end{pgfscope}%
\begin{pgfscope}%
\pgfsetbuttcap%
\pgfsetroundjoin%
\definecolor{currentfill}{rgb}{0.000000,0.000000,0.000000}%
\pgfsetfillcolor{currentfill}%
\pgfsetlinewidth{0.501875pt}%
\definecolor{currentstroke}{rgb}{0.000000,0.000000,0.000000}%
\pgfsetstrokecolor{currentstroke}%
\pgfsetdash{}{0pt}%
\pgfsys@defobject{currentmarker}{\pgfqpoint{0.000000in}{0.000000in}}{\pgfqpoint{0.000000in}{0.020833in}}{%
\pgfpathmoveto{\pgfqpoint{0.000000in}{0.000000in}}%
\pgfpathlineto{\pgfqpoint{0.000000in}{0.020833in}}%
\pgfusepath{stroke,fill}%
}%
\begin{pgfscope}%
\pgfsys@transformshift{2.611378in}{0.422992in}%
\pgfsys@useobject{currentmarker}{}%
\end{pgfscope}%
\end{pgfscope}%
\begin{pgfscope}%
\pgfsetbuttcap%
\pgfsetroundjoin%
\definecolor{currentfill}{rgb}{0.000000,0.000000,0.000000}%
\pgfsetfillcolor{currentfill}%
\pgfsetlinewidth{0.501875pt}%
\definecolor{currentstroke}{rgb}{0.000000,0.000000,0.000000}%
\pgfsetstrokecolor{currentstroke}%
\pgfsetdash{}{0pt}%
\pgfsys@defobject{currentmarker}{\pgfqpoint{0.000000in}{-0.020833in}}{\pgfqpoint{0.000000in}{0.000000in}}{%
\pgfpathmoveto{\pgfqpoint{0.000000in}{0.000000in}}%
\pgfpathlineto{\pgfqpoint{0.000000in}{-0.020833in}}%
\pgfusepath{stroke,fill}%
}%
\begin{pgfscope}%
\pgfsys@transformshift{2.611378in}{2.049193in}%
\pgfsys@useobject{currentmarker}{}%
\end{pgfscope}%
\end{pgfscope}%
\begin{pgfscope}%
\pgfsetbuttcap%
\pgfsetroundjoin%
\definecolor{currentfill}{rgb}{0.000000,0.000000,0.000000}%
\pgfsetfillcolor{currentfill}%
\pgfsetlinewidth{0.501875pt}%
\definecolor{currentstroke}{rgb}{0.000000,0.000000,0.000000}%
\pgfsetstrokecolor{currentstroke}%
\pgfsetdash{}{0pt}%
\pgfsys@defobject{currentmarker}{\pgfqpoint{0.000000in}{0.000000in}}{\pgfqpoint{0.000000in}{0.020833in}}{%
\pgfpathmoveto{\pgfqpoint{0.000000in}{0.000000in}}%
\pgfpathlineto{\pgfqpoint{0.000000in}{0.020833in}}%
\pgfusepath{stroke,fill}%
}%
\begin{pgfscope}%
\pgfsys@transformshift{2.724054in}{0.422992in}%
\pgfsys@useobject{currentmarker}{}%
\end{pgfscope}%
\end{pgfscope}%
\begin{pgfscope}%
\pgfsetbuttcap%
\pgfsetroundjoin%
\definecolor{currentfill}{rgb}{0.000000,0.000000,0.000000}%
\pgfsetfillcolor{currentfill}%
\pgfsetlinewidth{0.501875pt}%
\definecolor{currentstroke}{rgb}{0.000000,0.000000,0.000000}%
\pgfsetstrokecolor{currentstroke}%
\pgfsetdash{}{0pt}%
\pgfsys@defobject{currentmarker}{\pgfqpoint{0.000000in}{-0.020833in}}{\pgfqpoint{0.000000in}{0.000000in}}{%
\pgfpathmoveto{\pgfqpoint{0.000000in}{0.000000in}}%
\pgfpathlineto{\pgfqpoint{0.000000in}{-0.020833in}}%
\pgfusepath{stroke,fill}%
}%
\begin{pgfscope}%
\pgfsys@transformshift{2.724054in}{2.049193in}%
\pgfsys@useobject{currentmarker}{}%
\end{pgfscope}%
\end{pgfscope}%
\begin{pgfscope}%
\definecolor{textcolor}{rgb}{0.000000,0.000000,0.000000}%
\pgfsetstrokecolor{textcolor}%
\pgfsetfillcolor{textcolor}%
\pgftext[x=1.608557in,y=0.184413in,,top]{\color{textcolor}\rmfamily\fontsize{10.000000}{12.000000}\selectfont \(\displaystyle K\)}%
\end{pgfscope}%
\begin{pgfscope}%
\pgfsetbuttcap%
\pgfsetroundjoin%
\definecolor{currentfill}{rgb}{0.000000,0.000000,0.000000}%
\pgfsetfillcolor{currentfill}%
\pgfsetlinewidth{0.501875pt}%
\definecolor{currentstroke}{rgb}{0.000000,0.000000,0.000000}%
\pgfsetstrokecolor{currentstroke}%
\pgfsetdash{}{0pt}%
\pgfsys@defobject{currentmarker}{\pgfqpoint{0.000000in}{0.000000in}}{\pgfqpoint{0.041667in}{0.000000in}}{%
\pgfpathmoveto{\pgfqpoint{0.000000in}{0.000000in}}%
\pgfpathlineto{\pgfqpoint{0.041667in}{0.000000in}}%
\pgfusepath{stroke,fill}%
}%
\begin{pgfscope}%
\pgfsys@transformshift{0.470525in}{0.519733in}%
\pgfsys@useobject{currentmarker}{}%
\end{pgfscope}%
\end{pgfscope}%
\begin{pgfscope}%
\pgfsetbuttcap%
\pgfsetroundjoin%
\definecolor{currentfill}{rgb}{0.000000,0.000000,0.000000}%
\pgfsetfillcolor{currentfill}%
\pgfsetlinewidth{0.501875pt}%
\definecolor{currentstroke}{rgb}{0.000000,0.000000,0.000000}%
\pgfsetstrokecolor{currentstroke}%
\pgfsetdash{}{0pt}%
\pgfsys@defobject{currentmarker}{\pgfqpoint{-0.041667in}{0.000000in}}{\pgfqpoint{-0.000000in}{0.000000in}}{%
\pgfpathmoveto{\pgfqpoint{-0.000000in}{0.000000in}}%
\pgfpathlineto{\pgfqpoint{-0.041667in}{0.000000in}}%
\pgfusepath{stroke,fill}%
}%
\begin{pgfscope}%
\pgfsys@transformshift{2.746589in}{0.519733in}%
\pgfsys@useobject{currentmarker}{}%
\end{pgfscope}%
\end{pgfscope}%
\begin{pgfscope}%
\definecolor{textcolor}{rgb}{0.000000,0.000000,0.000000}%
\pgfsetstrokecolor{textcolor}%
\pgfsetfillcolor{textcolor}%
\pgftext[x=0.244444in, y=0.466972in, left, base]{\color{textcolor}\rmfamily\fontsize{10.000000}{12.000000}\selectfont \(\displaystyle {0.7}\)}%
\end{pgfscope}%
\begin{pgfscope}%
\pgfsetbuttcap%
\pgfsetroundjoin%
\definecolor{currentfill}{rgb}{0.000000,0.000000,0.000000}%
\pgfsetfillcolor{currentfill}%
\pgfsetlinewidth{0.501875pt}%
\definecolor{currentstroke}{rgb}{0.000000,0.000000,0.000000}%
\pgfsetstrokecolor{currentstroke}%
\pgfsetdash{}{0pt}%
\pgfsys@defobject{currentmarker}{\pgfqpoint{0.000000in}{0.000000in}}{\pgfqpoint{0.041667in}{0.000000in}}{%
\pgfpathmoveto{\pgfqpoint{0.000000in}{0.000000in}}%
\pgfpathlineto{\pgfqpoint{0.041667in}{0.000000in}}%
\pgfusepath{stroke,fill}%
}%
\begin{pgfscope}%
\pgfsys@transformshift{0.470525in}{1.013734in}%
\pgfsys@useobject{currentmarker}{}%
\end{pgfscope}%
\end{pgfscope}%
\begin{pgfscope}%
\pgfsetbuttcap%
\pgfsetroundjoin%
\definecolor{currentfill}{rgb}{0.000000,0.000000,0.000000}%
\pgfsetfillcolor{currentfill}%
\pgfsetlinewidth{0.501875pt}%
\definecolor{currentstroke}{rgb}{0.000000,0.000000,0.000000}%
\pgfsetstrokecolor{currentstroke}%
\pgfsetdash{}{0pt}%
\pgfsys@defobject{currentmarker}{\pgfqpoint{-0.041667in}{0.000000in}}{\pgfqpoint{-0.000000in}{0.000000in}}{%
\pgfpathmoveto{\pgfqpoint{-0.000000in}{0.000000in}}%
\pgfpathlineto{\pgfqpoint{-0.041667in}{0.000000in}}%
\pgfusepath{stroke,fill}%
}%
\begin{pgfscope}%
\pgfsys@transformshift{2.746589in}{1.013734in}%
\pgfsys@useobject{currentmarker}{}%
\end{pgfscope}%
\end{pgfscope}%
\begin{pgfscope}%
\definecolor{textcolor}{rgb}{0.000000,0.000000,0.000000}%
\pgfsetstrokecolor{textcolor}%
\pgfsetfillcolor{textcolor}%
\pgftext[x=0.244444in, y=0.960972in, left, base]{\color{textcolor}\rmfamily\fontsize{10.000000}{12.000000}\selectfont \(\displaystyle {0.8}\)}%
\end{pgfscope}%
\begin{pgfscope}%
\pgfsetbuttcap%
\pgfsetroundjoin%
\definecolor{currentfill}{rgb}{0.000000,0.000000,0.000000}%
\pgfsetfillcolor{currentfill}%
\pgfsetlinewidth{0.501875pt}%
\definecolor{currentstroke}{rgb}{0.000000,0.000000,0.000000}%
\pgfsetstrokecolor{currentstroke}%
\pgfsetdash{}{0pt}%
\pgfsys@defobject{currentmarker}{\pgfqpoint{0.000000in}{0.000000in}}{\pgfqpoint{0.041667in}{0.000000in}}{%
\pgfpathmoveto{\pgfqpoint{0.000000in}{0.000000in}}%
\pgfpathlineto{\pgfqpoint{0.041667in}{0.000000in}}%
\pgfusepath{stroke,fill}%
}%
\begin{pgfscope}%
\pgfsys@transformshift{0.470525in}{1.507734in}%
\pgfsys@useobject{currentmarker}{}%
\end{pgfscope}%
\end{pgfscope}%
\begin{pgfscope}%
\pgfsetbuttcap%
\pgfsetroundjoin%
\definecolor{currentfill}{rgb}{0.000000,0.000000,0.000000}%
\pgfsetfillcolor{currentfill}%
\pgfsetlinewidth{0.501875pt}%
\definecolor{currentstroke}{rgb}{0.000000,0.000000,0.000000}%
\pgfsetstrokecolor{currentstroke}%
\pgfsetdash{}{0pt}%
\pgfsys@defobject{currentmarker}{\pgfqpoint{-0.041667in}{0.000000in}}{\pgfqpoint{-0.000000in}{0.000000in}}{%
\pgfpathmoveto{\pgfqpoint{-0.000000in}{0.000000in}}%
\pgfpathlineto{\pgfqpoint{-0.041667in}{0.000000in}}%
\pgfusepath{stroke,fill}%
}%
\begin{pgfscope}%
\pgfsys@transformshift{2.746589in}{1.507734in}%
\pgfsys@useobject{currentmarker}{}%
\end{pgfscope}%
\end{pgfscope}%
\begin{pgfscope}%
\definecolor{textcolor}{rgb}{0.000000,0.000000,0.000000}%
\pgfsetstrokecolor{textcolor}%
\pgfsetfillcolor{textcolor}%
\pgftext[x=0.244444in, y=1.454973in, left, base]{\color{textcolor}\rmfamily\fontsize{10.000000}{12.000000}\selectfont \(\displaystyle {0.9}\)}%
\end{pgfscope}%
\begin{pgfscope}%
\pgfsetbuttcap%
\pgfsetroundjoin%
\definecolor{currentfill}{rgb}{0.000000,0.000000,0.000000}%
\pgfsetfillcolor{currentfill}%
\pgfsetlinewidth{0.501875pt}%
\definecolor{currentstroke}{rgb}{0.000000,0.000000,0.000000}%
\pgfsetstrokecolor{currentstroke}%
\pgfsetdash{}{0pt}%
\pgfsys@defobject{currentmarker}{\pgfqpoint{0.000000in}{0.000000in}}{\pgfqpoint{0.041667in}{0.000000in}}{%
\pgfpathmoveto{\pgfqpoint{0.000000in}{0.000000in}}%
\pgfpathlineto{\pgfqpoint{0.041667in}{0.000000in}}%
\pgfusepath{stroke,fill}%
}%
\begin{pgfscope}%
\pgfsys@transformshift{0.470525in}{2.001735in}%
\pgfsys@useobject{currentmarker}{}%
\end{pgfscope}%
\end{pgfscope}%
\begin{pgfscope}%
\pgfsetbuttcap%
\pgfsetroundjoin%
\definecolor{currentfill}{rgb}{0.000000,0.000000,0.000000}%
\pgfsetfillcolor{currentfill}%
\pgfsetlinewidth{0.501875pt}%
\definecolor{currentstroke}{rgb}{0.000000,0.000000,0.000000}%
\pgfsetstrokecolor{currentstroke}%
\pgfsetdash{}{0pt}%
\pgfsys@defobject{currentmarker}{\pgfqpoint{-0.041667in}{0.000000in}}{\pgfqpoint{-0.000000in}{0.000000in}}{%
\pgfpathmoveto{\pgfqpoint{-0.000000in}{0.000000in}}%
\pgfpathlineto{\pgfqpoint{-0.041667in}{0.000000in}}%
\pgfusepath{stroke,fill}%
}%
\begin{pgfscope}%
\pgfsys@transformshift{2.746589in}{2.001735in}%
\pgfsys@useobject{currentmarker}{}%
\end{pgfscope}%
\end{pgfscope}%
\begin{pgfscope}%
\definecolor{textcolor}{rgb}{0.000000,0.000000,0.000000}%
\pgfsetstrokecolor{textcolor}%
\pgfsetfillcolor{textcolor}%
\pgftext[x=0.244444in, y=1.948973in, left, base]{\color{textcolor}\rmfamily\fontsize{10.000000}{12.000000}\selectfont \(\displaystyle {1.0}\)}%
\end{pgfscope}%
\begin{pgfscope}%
\pgfsetbuttcap%
\pgfsetroundjoin%
\definecolor{currentfill}{rgb}{0.000000,0.000000,0.000000}%
\pgfsetfillcolor{currentfill}%
\pgfsetlinewidth{0.501875pt}%
\definecolor{currentstroke}{rgb}{0.000000,0.000000,0.000000}%
\pgfsetstrokecolor{currentstroke}%
\pgfsetdash{}{0pt}%
\pgfsys@defobject{currentmarker}{\pgfqpoint{0.000000in}{0.000000in}}{\pgfqpoint{0.020833in}{0.000000in}}{%
\pgfpathmoveto{\pgfqpoint{0.000000in}{0.000000in}}%
\pgfpathlineto{\pgfqpoint{0.020833in}{0.000000in}}%
\pgfusepath{stroke,fill}%
}%
\begin{pgfscope}%
\pgfsys@transformshift{0.470525in}{0.618533in}%
\pgfsys@useobject{currentmarker}{}%
\end{pgfscope}%
\end{pgfscope}%
\begin{pgfscope}%
\pgfsetbuttcap%
\pgfsetroundjoin%
\definecolor{currentfill}{rgb}{0.000000,0.000000,0.000000}%
\pgfsetfillcolor{currentfill}%
\pgfsetlinewidth{0.501875pt}%
\definecolor{currentstroke}{rgb}{0.000000,0.000000,0.000000}%
\pgfsetstrokecolor{currentstroke}%
\pgfsetdash{}{0pt}%
\pgfsys@defobject{currentmarker}{\pgfqpoint{-0.020833in}{0.000000in}}{\pgfqpoint{-0.000000in}{0.000000in}}{%
\pgfpathmoveto{\pgfqpoint{-0.000000in}{0.000000in}}%
\pgfpathlineto{\pgfqpoint{-0.020833in}{0.000000in}}%
\pgfusepath{stroke,fill}%
}%
\begin{pgfscope}%
\pgfsys@transformshift{2.746589in}{0.618533in}%
\pgfsys@useobject{currentmarker}{}%
\end{pgfscope}%
\end{pgfscope}%
\begin{pgfscope}%
\pgfsetbuttcap%
\pgfsetroundjoin%
\definecolor{currentfill}{rgb}{0.000000,0.000000,0.000000}%
\pgfsetfillcolor{currentfill}%
\pgfsetlinewidth{0.501875pt}%
\definecolor{currentstroke}{rgb}{0.000000,0.000000,0.000000}%
\pgfsetstrokecolor{currentstroke}%
\pgfsetdash{}{0pt}%
\pgfsys@defobject{currentmarker}{\pgfqpoint{0.000000in}{0.000000in}}{\pgfqpoint{0.020833in}{0.000000in}}{%
\pgfpathmoveto{\pgfqpoint{0.000000in}{0.000000in}}%
\pgfpathlineto{\pgfqpoint{0.020833in}{0.000000in}}%
\pgfusepath{stroke,fill}%
}%
\begin{pgfscope}%
\pgfsys@transformshift{0.470525in}{0.717333in}%
\pgfsys@useobject{currentmarker}{}%
\end{pgfscope}%
\end{pgfscope}%
\begin{pgfscope}%
\pgfsetbuttcap%
\pgfsetroundjoin%
\definecolor{currentfill}{rgb}{0.000000,0.000000,0.000000}%
\pgfsetfillcolor{currentfill}%
\pgfsetlinewidth{0.501875pt}%
\definecolor{currentstroke}{rgb}{0.000000,0.000000,0.000000}%
\pgfsetstrokecolor{currentstroke}%
\pgfsetdash{}{0pt}%
\pgfsys@defobject{currentmarker}{\pgfqpoint{-0.020833in}{0.000000in}}{\pgfqpoint{-0.000000in}{0.000000in}}{%
\pgfpathmoveto{\pgfqpoint{-0.000000in}{0.000000in}}%
\pgfpathlineto{\pgfqpoint{-0.020833in}{0.000000in}}%
\pgfusepath{stroke,fill}%
}%
\begin{pgfscope}%
\pgfsys@transformshift{2.746589in}{0.717333in}%
\pgfsys@useobject{currentmarker}{}%
\end{pgfscope}%
\end{pgfscope}%
\begin{pgfscope}%
\pgfsetbuttcap%
\pgfsetroundjoin%
\definecolor{currentfill}{rgb}{0.000000,0.000000,0.000000}%
\pgfsetfillcolor{currentfill}%
\pgfsetlinewidth{0.501875pt}%
\definecolor{currentstroke}{rgb}{0.000000,0.000000,0.000000}%
\pgfsetstrokecolor{currentstroke}%
\pgfsetdash{}{0pt}%
\pgfsys@defobject{currentmarker}{\pgfqpoint{0.000000in}{0.000000in}}{\pgfqpoint{0.020833in}{0.000000in}}{%
\pgfpathmoveto{\pgfqpoint{0.000000in}{0.000000in}}%
\pgfpathlineto{\pgfqpoint{0.020833in}{0.000000in}}%
\pgfusepath{stroke,fill}%
}%
\begin{pgfscope}%
\pgfsys@transformshift{0.470525in}{0.816133in}%
\pgfsys@useobject{currentmarker}{}%
\end{pgfscope}%
\end{pgfscope}%
\begin{pgfscope}%
\pgfsetbuttcap%
\pgfsetroundjoin%
\definecolor{currentfill}{rgb}{0.000000,0.000000,0.000000}%
\pgfsetfillcolor{currentfill}%
\pgfsetlinewidth{0.501875pt}%
\definecolor{currentstroke}{rgb}{0.000000,0.000000,0.000000}%
\pgfsetstrokecolor{currentstroke}%
\pgfsetdash{}{0pt}%
\pgfsys@defobject{currentmarker}{\pgfqpoint{-0.020833in}{0.000000in}}{\pgfqpoint{-0.000000in}{0.000000in}}{%
\pgfpathmoveto{\pgfqpoint{-0.000000in}{0.000000in}}%
\pgfpathlineto{\pgfqpoint{-0.020833in}{0.000000in}}%
\pgfusepath{stroke,fill}%
}%
\begin{pgfscope}%
\pgfsys@transformshift{2.746589in}{0.816133in}%
\pgfsys@useobject{currentmarker}{}%
\end{pgfscope}%
\end{pgfscope}%
\begin{pgfscope}%
\pgfsetbuttcap%
\pgfsetroundjoin%
\definecolor{currentfill}{rgb}{0.000000,0.000000,0.000000}%
\pgfsetfillcolor{currentfill}%
\pgfsetlinewidth{0.501875pt}%
\definecolor{currentstroke}{rgb}{0.000000,0.000000,0.000000}%
\pgfsetstrokecolor{currentstroke}%
\pgfsetdash{}{0pt}%
\pgfsys@defobject{currentmarker}{\pgfqpoint{0.000000in}{0.000000in}}{\pgfqpoint{0.020833in}{0.000000in}}{%
\pgfpathmoveto{\pgfqpoint{0.000000in}{0.000000in}}%
\pgfpathlineto{\pgfqpoint{0.020833in}{0.000000in}}%
\pgfusepath{stroke,fill}%
}%
\begin{pgfscope}%
\pgfsys@transformshift{0.470525in}{0.914934in}%
\pgfsys@useobject{currentmarker}{}%
\end{pgfscope}%
\end{pgfscope}%
\begin{pgfscope}%
\pgfsetbuttcap%
\pgfsetroundjoin%
\definecolor{currentfill}{rgb}{0.000000,0.000000,0.000000}%
\pgfsetfillcolor{currentfill}%
\pgfsetlinewidth{0.501875pt}%
\definecolor{currentstroke}{rgb}{0.000000,0.000000,0.000000}%
\pgfsetstrokecolor{currentstroke}%
\pgfsetdash{}{0pt}%
\pgfsys@defobject{currentmarker}{\pgfqpoint{-0.020833in}{0.000000in}}{\pgfqpoint{-0.000000in}{0.000000in}}{%
\pgfpathmoveto{\pgfqpoint{-0.000000in}{0.000000in}}%
\pgfpathlineto{\pgfqpoint{-0.020833in}{0.000000in}}%
\pgfusepath{stroke,fill}%
}%
\begin{pgfscope}%
\pgfsys@transformshift{2.746589in}{0.914934in}%
\pgfsys@useobject{currentmarker}{}%
\end{pgfscope}%
\end{pgfscope}%
\begin{pgfscope}%
\pgfsetbuttcap%
\pgfsetroundjoin%
\definecolor{currentfill}{rgb}{0.000000,0.000000,0.000000}%
\pgfsetfillcolor{currentfill}%
\pgfsetlinewidth{0.501875pt}%
\definecolor{currentstroke}{rgb}{0.000000,0.000000,0.000000}%
\pgfsetstrokecolor{currentstroke}%
\pgfsetdash{}{0pt}%
\pgfsys@defobject{currentmarker}{\pgfqpoint{0.000000in}{0.000000in}}{\pgfqpoint{0.020833in}{0.000000in}}{%
\pgfpathmoveto{\pgfqpoint{0.000000in}{0.000000in}}%
\pgfpathlineto{\pgfqpoint{0.020833in}{0.000000in}}%
\pgfusepath{stroke,fill}%
}%
\begin{pgfscope}%
\pgfsys@transformshift{0.470525in}{1.112534in}%
\pgfsys@useobject{currentmarker}{}%
\end{pgfscope}%
\end{pgfscope}%
\begin{pgfscope}%
\pgfsetbuttcap%
\pgfsetroundjoin%
\definecolor{currentfill}{rgb}{0.000000,0.000000,0.000000}%
\pgfsetfillcolor{currentfill}%
\pgfsetlinewidth{0.501875pt}%
\definecolor{currentstroke}{rgb}{0.000000,0.000000,0.000000}%
\pgfsetstrokecolor{currentstroke}%
\pgfsetdash{}{0pt}%
\pgfsys@defobject{currentmarker}{\pgfqpoint{-0.020833in}{0.000000in}}{\pgfqpoint{-0.000000in}{0.000000in}}{%
\pgfpathmoveto{\pgfqpoint{-0.000000in}{0.000000in}}%
\pgfpathlineto{\pgfqpoint{-0.020833in}{0.000000in}}%
\pgfusepath{stroke,fill}%
}%
\begin{pgfscope}%
\pgfsys@transformshift{2.746589in}{1.112534in}%
\pgfsys@useobject{currentmarker}{}%
\end{pgfscope}%
\end{pgfscope}%
\begin{pgfscope}%
\pgfsetbuttcap%
\pgfsetroundjoin%
\definecolor{currentfill}{rgb}{0.000000,0.000000,0.000000}%
\pgfsetfillcolor{currentfill}%
\pgfsetlinewidth{0.501875pt}%
\definecolor{currentstroke}{rgb}{0.000000,0.000000,0.000000}%
\pgfsetstrokecolor{currentstroke}%
\pgfsetdash{}{0pt}%
\pgfsys@defobject{currentmarker}{\pgfqpoint{0.000000in}{0.000000in}}{\pgfqpoint{0.020833in}{0.000000in}}{%
\pgfpathmoveto{\pgfqpoint{0.000000in}{0.000000in}}%
\pgfpathlineto{\pgfqpoint{0.020833in}{0.000000in}}%
\pgfusepath{stroke,fill}%
}%
\begin{pgfscope}%
\pgfsys@transformshift{0.470525in}{1.211334in}%
\pgfsys@useobject{currentmarker}{}%
\end{pgfscope}%
\end{pgfscope}%
\begin{pgfscope}%
\pgfsetbuttcap%
\pgfsetroundjoin%
\definecolor{currentfill}{rgb}{0.000000,0.000000,0.000000}%
\pgfsetfillcolor{currentfill}%
\pgfsetlinewidth{0.501875pt}%
\definecolor{currentstroke}{rgb}{0.000000,0.000000,0.000000}%
\pgfsetstrokecolor{currentstroke}%
\pgfsetdash{}{0pt}%
\pgfsys@defobject{currentmarker}{\pgfqpoint{-0.020833in}{0.000000in}}{\pgfqpoint{-0.000000in}{0.000000in}}{%
\pgfpathmoveto{\pgfqpoint{-0.000000in}{0.000000in}}%
\pgfpathlineto{\pgfqpoint{-0.020833in}{0.000000in}}%
\pgfusepath{stroke,fill}%
}%
\begin{pgfscope}%
\pgfsys@transformshift{2.746589in}{1.211334in}%
\pgfsys@useobject{currentmarker}{}%
\end{pgfscope}%
\end{pgfscope}%
\begin{pgfscope}%
\pgfsetbuttcap%
\pgfsetroundjoin%
\definecolor{currentfill}{rgb}{0.000000,0.000000,0.000000}%
\pgfsetfillcolor{currentfill}%
\pgfsetlinewidth{0.501875pt}%
\definecolor{currentstroke}{rgb}{0.000000,0.000000,0.000000}%
\pgfsetstrokecolor{currentstroke}%
\pgfsetdash{}{0pt}%
\pgfsys@defobject{currentmarker}{\pgfqpoint{0.000000in}{0.000000in}}{\pgfqpoint{0.020833in}{0.000000in}}{%
\pgfpathmoveto{\pgfqpoint{0.000000in}{0.000000in}}%
\pgfpathlineto{\pgfqpoint{0.020833in}{0.000000in}}%
\pgfusepath{stroke,fill}%
}%
\begin{pgfscope}%
\pgfsys@transformshift{0.470525in}{1.310134in}%
\pgfsys@useobject{currentmarker}{}%
\end{pgfscope}%
\end{pgfscope}%
\begin{pgfscope}%
\pgfsetbuttcap%
\pgfsetroundjoin%
\definecolor{currentfill}{rgb}{0.000000,0.000000,0.000000}%
\pgfsetfillcolor{currentfill}%
\pgfsetlinewidth{0.501875pt}%
\definecolor{currentstroke}{rgb}{0.000000,0.000000,0.000000}%
\pgfsetstrokecolor{currentstroke}%
\pgfsetdash{}{0pt}%
\pgfsys@defobject{currentmarker}{\pgfqpoint{-0.020833in}{0.000000in}}{\pgfqpoint{-0.000000in}{0.000000in}}{%
\pgfpathmoveto{\pgfqpoint{-0.000000in}{0.000000in}}%
\pgfpathlineto{\pgfqpoint{-0.020833in}{0.000000in}}%
\pgfusepath{stroke,fill}%
}%
\begin{pgfscope}%
\pgfsys@transformshift{2.746589in}{1.310134in}%
\pgfsys@useobject{currentmarker}{}%
\end{pgfscope}%
\end{pgfscope}%
\begin{pgfscope}%
\pgfsetbuttcap%
\pgfsetroundjoin%
\definecolor{currentfill}{rgb}{0.000000,0.000000,0.000000}%
\pgfsetfillcolor{currentfill}%
\pgfsetlinewidth{0.501875pt}%
\definecolor{currentstroke}{rgb}{0.000000,0.000000,0.000000}%
\pgfsetstrokecolor{currentstroke}%
\pgfsetdash{}{0pt}%
\pgfsys@defobject{currentmarker}{\pgfqpoint{0.000000in}{0.000000in}}{\pgfqpoint{0.020833in}{0.000000in}}{%
\pgfpathmoveto{\pgfqpoint{0.000000in}{0.000000in}}%
\pgfpathlineto{\pgfqpoint{0.020833in}{0.000000in}}%
\pgfusepath{stroke,fill}%
}%
\begin{pgfscope}%
\pgfsys@transformshift{0.470525in}{1.408934in}%
\pgfsys@useobject{currentmarker}{}%
\end{pgfscope}%
\end{pgfscope}%
\begin{pgfscope}%
\pgfsetbuttcap%
\pgfsetroundjoin%
\definecolor{currentfill}{rgb}{0.000000,0.000000,0.000000}%
\pgfsetfillcolor{currentfill}%
\pgfsetlinewidth{0.501875pt}%
\definecolor{currentstroke}{rgb}{0.000000,0.000000,0.000000}%
\pgfsetstrokecolor{currentstroke}%
\pgfsetdash{}{0pt}%
\pgfsys@defobject{currentmarker}{\pgfqpoint{-0.020833in}{0.000000in}}{\pgfqpoint{-0.000000in}{0.000000in}}{%
\pgfpathmoveto{\pgfqpoint{-0.000000in}{0.000000in}}%
\pgfpathlineto{\pgfqpoint{-0.020833in}{0.000000in}}%
\pgfusepath{stroke,fill}%
}%
\begin{pgfscope}%
\pgfsys@transformshift{2.746589in}{1.408934in}%
\pgfsys@useobject{currentmarker}{}%
\end{pgfscope}%
\end{pgfscope}%
\begin{pgfscope}%
\pgfsetbuttcap%
\pgfsetroundjoin%
\definecolor{currentfill}{rgb}{0.000000,0.000000,0.000000}%
\pgfsetfillcolor{currentfill}%
\pgfsetlinewidth{0.501875pt}%
\definecolor{currentstroke}{rgb}{0.000000,0.000000,0.000000}%
\pgfsetstrokecolor{currentstroke}%
\pgfsetdash{}{0pt}%
\pgfsys@defobject{currentmarker}{\pgfqpoint{0.000000in}{0.000000in}}{\pgfqpoint{0.020833in}{0.000000in}}{%
\pgfpathmoveto{\pgfqpoint{0.000000in}{0.000000in}}%
\pgfpathlineto{\pgfqpoint{0.020833in}{0.000000in}}%
\pgfusepath{stroke,fill}%
}%
\begin{pgfscope}%
\pgfsys@transformshift{0.470525in}{1.606534in}%
\pgfsys@useobject{currentmarker}{}%
\end{pgfscope}%
\end{pgfscope}%
\begin{pgfscope}%
\pgfsetbuttcap%
\pgfsetroundjoin%
\definecolor{currentfill}{rgb}{0.000000,0.000000,0.000000}%
\pgfsetfillcolor{currentfill}%
\pgfsetlinewidth{0.501875pt}%
\definecolor{currentstroke}{rgb}{0.000000,0.000000,0.000000}%
\pgfsetstrokecolor{currentstroke}%
\pgfsetdash{}{0pt}%
\pgfsys@defobject{currentmarker}{\pgfqpoint{-0.020833in}{0.000000in}}{\pgfqpoint{-0.000000in}{0.000000in}}{%
\pgfpathmoveto{\pgfqpoint{-0.000000in}{0.000000in}}%
\pgfpathlineto{\pgfqpoint{-0.020833in}{0.000000in}}%
\pgfusepath{stroke,fill}%
}%
\begin{pgfscope}%
\pgfsys@transformshift{2.746589in}{1.606534in}%
\pgfsys@useobject{currentmarker}{}%
\end{pgfscope}%
\end{pgfscope}%
\begin{pgfscope}%
\pgfsetbuttcap%
\pgfsetroundjoin%
\definecolor{currentfill}{rgb}{0.000000,0.000000,0.000000}%
\pgfsetfillcolor{currentfill}%
\pgfsetlinewidth{0.501875pt}%
\definecolor{currentstroke}{rgb}{0.000000,0.000000,0.000000}%
\pgfsetstrokecolor{currentstroke}%
\pgfsetdash{}{0pt}%
\pgfsys@defobject{currentmarker}{\pgfqpoint{0.000000in}{0.000000in}}{\pgfqpoint{0.020833in}{0.000000in}}{%
\pgfpathmoveto{\pgfqpoint{0.000000in}{0.000000in}}%
\pgfpathlineto{\pgfqpoint{0.020833in}{0.000000in}}%
\pgfusepath{stroke,fill}%
}%
\begin{pgfscope}%
\pgfsys@transformshift{0.470525in}{1.705335in}%
\pgfsys@useobject{currentmarker}{}%
\end{pgfscope}%
\end{pgfscope}%
\begin{pgfscope}%
\pgfsetbuttcap%
\pgfsetroundjoin%
\definecolor{currentfill}{rgb}{0.000000,0.000000,0.000000}%
\pgfsetfillcolor{currentfill}%
\pgfsetlinewidth{0.501875pt}%
\definecolor{currentstroke}{rgb}{0.000000,0.000000,0.000000}%
\pgfsetstrokecolor{currentstroke}%
\pgfsetdash{}{0pt}%
\pgfsys@defobject{currentmarker}{\pgfqpoint{-0.020833in}{0.000000in}}{\pgfqpoint{-0.000000in}{0.000000in}}{%
\pgfpathmoveto{\pgfqpoint{-0.000000in}{0.000000in}}%
\pgfpathlineto{\pgfqpoint{-0.020833in}{0.000000in}}%
\pgfusepath{stroke,fill}%
}%
\begin{pgfscope}%
\pgfsys@transformshift{2.746589in}{1.705335in}%
\pgfsys@useobject{currentmarker}{}%
\end{pgfscope}%
\end{pgfscope}%
\begin{pgfscope}%
\pgfsetbuttcap%
\pgfsetroundjoin%
\definecolor{currentfill}{rgb}{0.000000,0.000000,0.000000}%
\pgfsetfillcolor{currentfill}%
\pgfsetlinewidth{0.501875pt}%
\definecolor{currentstroke}{rgb}{0.000000,0.000000,0.000000}%
\pgfsetstrokecolor{currentstroke}%
\pgfsetdash{}{0pt}%
\pgfsys@defobject{currentmarker}{\pgfqpoint{0.000000in}{0.000000in}}{\pgfqpoint{0.020833in}{0.000000in}}{%
\pgfpathmoveto{\pgfqpoint{0.000000in}{0.000000in}}%
\pgfpathlineto{\pgfqpoint{0.020833in}{0.000000in}}%
\pgfusepath{stroke,fill}%
}%
\begin{pgfscope}%
\pgfsys@transformshift{0.470525in}{1.804135in}%
\pgfsys@useobject{currentmarker}{}%
\end{pgfscope}%
\end{pgfscope}%
\begin{pgfscope}%
\pgfsetbuttcap%
\pgfsetroundjoin%
\definecolor{currentfill}{rgb}{0.000000,0.000000,0.000000}%
\pgfsetfillcolor{currentfill}%
\pgfsetlinewidth{0.501875pt}%
\definecolor{currentstroke}{rgb}{0.000000,0.000000,0.000000}%
\pgfsetstrokecolor{currentstroke}%
\pgfsetdash{}{0pt}%
\pgfsys@defobject{currentmarker}{\pgfqpoint{-0.020833in}{0.000000in}}{\pgfqpoint{-0.000000in}{0.000000in}}{%
\pgfpathmoveto{\pgfqpoint{-0.000000in}{0.000000in}}%
\pgfpathlineto{\pgfqpoint{-0.020833in}{0.000000in}}%
\pgfusepath{stroke,fill}%
}%
\begin{pgfscope}%
\pgfsys@transformshift{2.746589in}{1.804135in}%
\pgfsys@useobject{currentmarker}{}%
\end{pgfscope}%
\end{pgfscope}%
\begin{pgfscope}%
\pgfsetbuttcap%
\pgfsetroundjoin%
\definecolor{currentfill}{rgb}{0.000000,0.000000,0.000000}%
\pgfsetfillcolor{currentfill}%
\pgfsetlinewidth{0.501875pt}%
\definecolor{currentstroke}{rgb}{0.000000,0.000000,0.000000}%
\pgfsetstrokecolor{currentstroke}%
\pgfsetdash{}{0pt}%
\pgfsys@defobject{currentmarker}{\pgfqpoint{0.000000in}{0.000000in}}{\pgfqpoint{0.020833in}{0.000000in}}{%
\pgfpathmoveto{\pgfqpoint{0.000000in}{0.000000in}}%
\pgfpathlineto{\pgfqpoint{0.020833in}{0.000000in}}%
\pgfusepath{stroke,fill}%
}%
\begin{pgfscope}%
\pgfsys@transformshift{0.470525in}{1.902935in}%
\pgfsys@useobject{currentmarker}{}%
\end{pgfscope}%
\end{pgfscope}%
\begin{pgfscope}%
\pgfsetbuttcap%
\pgfsetroundjoin%
\definecolor{currentfill}{rgb}{0.000000,0.000000,0.000000}%
\pgfsetfillcolor{currentfill}%
\pgfsetlinewidth{0.501875pt}%
\definecolor{currentstroke}{rgb}{0.000000,0.000000,0.000000}%
\pgfsetstrokecolor{currentstroke}%
\pgfsetdash{}{0pt}%
\pgfsys@defobject{currentmarker}{\pgfqpoint{-0.020833in}{0.000000in}}{\pgfqpoint{-0.000000in}{0.000000in}}{%
\pgfpathmoveto{\pgfqpoint{-0.000000in}{0.000000in}}%
\pgfpathlineto{\pgfqpoint{-0.020833in}{0.000000in}}%
\pgfusepath{stroke,fill}%
}%
\begin{pgfscope}%
\pgfsys@transformshift{2.746589in}{1.902935in}%
\pgfsys@useobject{currentmarker}{}%
\end{pgfscope}%
\end{pgfscope}%
\begin{pgfscope}%
\definecolor{textcolor}{rgb}{0.000000,0.000000,0.000000}%
\pgfsetstrokecolor{textcolor}%
\pgfsetfillcolor{textcolor}%
\pgftext[x=0.188889in,y=1.236093in,,bottom,rotate=90.000000]{\color{textcolor}\rmfamily\fontsize{10.000000}{12.000000}\selectfont \(\displaystyle C(K)\)}%
\end{pgfscope}%
\begin{pgfscope}%
\pgfpathrectangle{\pgfqpoint{0.470525in}{0.422992in}}{\pgfqpoint{2.276064in}{1.626201in}}%
\pgfusepath{clip}%
\pgfsetrectcap%
\pgfsetroundjoin%
\pgfsetlinewidth{1.003750pt}%
\definecolor{currentstroke}{rgb}{0.047059,0.364706,0.647059}%
\pgfsetstrokecolor{currentstroke}%
\pgfsetdash{}{0pt}%
\pgfpathmoveto{\pgfqpoint{0.493061in}{1.975275in}}%
\pgfpathlineto{\pgfqpoint{0.515596in}{1.972727in}}%
\pgfpathlineto{\pgfqpoint{0.538131in}{1.970325in}}%
\pgfpathlineto{\pgfqpoint{0.560666in}{1.969412in}}%
\pgfpathlineto{\pgfqpoint{0.583202in}{1.968119in}}%
\pgfpathlineto{\pgfqpoint{0.605737in}{1.965530in}}%
\pgfpathlineto{\pgfqpoint{0.628272in}{1.964393in}}%
\pgfpathlineto{\pgfqpoint{0.650808in}{1.963179in}}%
\pgfpathlineto{\pgfqpoint{0.673343in}{1.962287in}}%
\pgfpathlineto{\pgfqpoint{0.695878in}{1.961090in}}%
\pgfpathlineto{\pgfqpoint{0.718413in}{1.961017in}}%
\pgfpathlineto{\pgfqpoint{0.740949in}{1.961004in}}%
\pgfpathlineto{\pgfqpoint{0.763484in}{1.960523in}}%
\pgfpathlineto{\pgfqpoint{0.786019in}{1.959983in}}%
\pgfpathlineto{\pgfqpoint{0.808555in}{1.959849in}}%
\pgfpathlineto{\pgfqpoint{0.831090in}{1.958847in}}%
\pgfpathlineto{\pgfqpoint{0.853625in}{1.958384in}}%
\pgfpathlineto{\pgfqpoint{0.876160in}{1.957955in}}%
\pgfpathlineto{\pgfqpoint{0.898696in}{1.957362in}}%
\pgfpathlineto{\pgfqpoint{0.921231in}{1.957206in}}%
\pgfpathlineto{\pgfqpoint{0.943766in}{1.956794in}}%
\pgfpathlineto{\pgfqpoint{0.966302in}{1.956337in}}%
\pgfpathlineto{\pgfqpoint{0.988837in}{1.956167in}}%
\pgfpathlineto{\pgfqpoint{1.011372in}{1.955632in}}%
\pgfpathlineto{\pgfqpoint{1.033907in}{1.955267in}}%
\pgfpathlineto{\pgfqpoint{1.056443in}{1.955004in}}%
\pgfpathlineto{\pgfqpoint{1.078978in}{1.954856in}}%
\pgfpathlineto{\pgfqpoint{1.101513in}{1.954609in}}%
\pgfpathlineto{\pgfqpoint{1.124049in}{1.954490in}}%
\pgfpathlineto{\pgfqpoint{1.146584in}{1.954452in}}%
\pgfpathlineto{\pgfqpoint{1.169119in}{1.954147in}}%
\pgfpathlineto{\pgfqpoint{1.191654in}{1.953819in}}%
\pgfpathlineto{\pgfqpoint{1.214190in}{1.953579in}}%
\pgfpathlineto{\pgfqpoint{1.236725in}{1.953221in}}%
\pgfpathlineto{\pgfqpoint{1.259260in}{1.953343in}}%
\pgfpathlineto{\pgfqpoint{1.281796in}{1.953064in}}%
\pgfpathlineto{\pgfqpoint{1.304331in}{1.953078in}}%
\pgfpathlineto{\pgfqpoint{1.326866in}{1.952991in}}%
\pgfpathlineto{\pgfqpoint{1.349402in}{1.953044in}}%
\pgfpathlineto{\pgfqpoint{1.371937in}{1.952860in}}%
\pgfpathlineto{\pgfqpoint{1.394472in}{1.952612in}}%
\pgfpathlineto{\pgfqpoint{1.417007in}{1.952758in}}%
\pgfpathlineto{\pgfqpoint{1.439543in}{1.952613in}}%
\pgfpathlineto{\pgfqpoint{1.462078in}{1.952339in}}%
\pgfpathlineto{\pgfqpoint{1.484613in}{1.952065in}}%
\pgfpathlineto{\pgfqpoint{1.507149in}{1.951941in}}%
\pgfpathlineto{\pgfqpoint{1.529684in}{1.951922in}}%
\pgfpathlineto{\pgfqpoint{1.552219in}{1.951859in}}%
\pgfpathlineto{\pgfqpoint{1.574754in}{1.951969in}}%
\pgfpathlineto{\pgfqpoint{1.597290in}{1.951794in}}%
\pgfpathlineto{\pgfqpoint{1.619825in}{1.951476in}}%
\pgfpathlineto{\pgfqpoint{1.642360in}{1.951221in}}%
\pgfpathlineto{\pgfqpoint{1.664896in}{1.951260in}}%
\pgfpathlineto{\pgfqpoint{1.687431in}{1.951071in}}%
\pgfpathlineto{\pgfqpoint{1.709966in}{1.951067in}}%
\pgfpathlineto{\pgfqpoint{1.732501in}{1.950973in}}%
\pgfpathlineto{\pgfqpoint{1.755037in}{1.950614in}}%
\pgfpathlineto{\pgfqpoint{1.777572in}{1.950617in}}%
\pgfpathlineto{\pgfqpoint{1.800107in}{1.950555in}}%
\pgfpathlineto{\pgfqpoint{1.822643in}{1.950494in}}%
\pgfpathlineto{\pgfqpoint{1.845178in}{1.950415in}}%
\pgfpathlineto{\pgfqpoint{1.867713in}{1.950297in}}%
\pgfpathlineto{\pgfqpoint{1.890248in}{1.950172in}}%
\pgfpathlineto{\pgfqpoint{1.912784in}{1.950082in}}%
\pgfpathlineto{\pgfqpoint{1.935319in}{1.949846in}}%
\pgfpathlineto{\pgfqpoint{1.957854in}{1.949705in}}%
\pgfpathlineto{\pgfqpoint{1.980390in}{1.949362in}}%
\pgfpathlineto{\pgfqpoint{2.002925in}{1.949183in}}%
\pgfpathlineto{\pgfqpoint{2.025460in}{1.949039in}}%
\pgfpathlineto{\pgfqpoint{2.047995in}{1.948918in}}%
\pgfpathlineto{\pgfqpoint{2.070531in}{1.948725in}}%
\pgfpathlineto{\pgfqpoint{2.093066in}{1.948408in}}%
\pgfpathlineto{\pgfqpoint{2.115601in}{1.948116in}}%
\pgfpathlineto{\pgfqpoint{2.138137in}{1.947980in}}%
\pgfpathlineto{\pgfqpoint{2.160672in}{1.947573in}}%
\pgfpathlineto{\pgfqpoint{2.183207in}{1.947483in}}%
\pgfpathlineto{\pgfqpoint{2.205742in}{1.947300in}}%
\pgfpathlineto{\pgfqpoint{2.228278in}{1.947027in}}%
\pgfpathlineto{\pgfqpoint{2.250813in}{1.946778in}}%
\pgfpathlineto{\pgfqpoint{2.273348in}{1.946513in}}%
\pgfpathlineto{\pgfqpoint{2.295884in}{1.946225in}}%
\pgfpathlineto{\pgfqpoint{2.318419in}{1.945900in}}%
\pgfpathlineto{\pgfqpoint{2.340954in}{1.945621in}}%
\pgfpathlineto{\pgfqpoint{2.363489in}{1.945316in}}%
\pgfpathlineto{\pgfqpoint{2.386025in}{1.944945in}}%
\pgfpathlineto{\pgfqpoint{2.408560in}{1.944658in}}%
\pgfpathlineto{\pgfqpoint{2.431095in}{1.944353in}}%
\pgfpathlineto{\pgfqpoint{2.453631in}{1.944239in}}%
\pgfpathlineto{\pgfqpoint{2.476166in}{1.943987in}}%
\pgfpathlineto{\pgfqpoint{2.498701in}{1.943481in}}%
\pgfpathlineto{\pgfqpoint{2.521236in}{1.942986in}}%
\pgfpathlineto{\pgfqpoint{2.543772in}{1.942642in}}%
\pgfpathlineto{\pgfqpoint{2.566307in}{1.942385in}}%
\pgfpathlineto{\pgfqpoint{2.588842in}{1.942033in}}%
\pgfpathlineto{\pgfqpoint{2.611378in}{1.941838in}}%
\pgfpathlineto{\pgfqpoint{2.633913in}{1.941322in}}%
\pgfpathlineto{\pgfqpoint{2.656448in}{1.941006in}}%
\pgfpathlineto{\pgfqpoint{2.678983in}{1.940606in}}%
\pgfpathlineto{\pgfqpoint{2.701519in}{1.940069in}}%
\pgfusepath{stroke}%
\end{pgfscope}%
\begin{pgfscope}%
\pgfpathrectangle{\pgfqpoint{0.470525in}{0.422992in}}{\pgfqpoint{2.276064in}{1.626201in}}%
\pgfusepath{clip}%
\pgfsetrectcap%
\pgfsetroundjoin%
\pgfsetlinewidth{1.003750pt}%
\definecolor{currentstroke}{rgb}{0.000000,0.725490,0.270588}%
\pgfsetstrokecolor{currentstroke}%
\pgfsetdash{}{0pt}%
\pgfpathmoveto{\pgfqpoint{0.493061in}{1.845507in}}%
\pgfpathlineto{\pgfqpoint{0.515596in}{1.838748in}}%
\pgfpathlineto{\pgfqpoint{0.538131in}{1.829286in}}%
\pgfpathlineto{\pgfqpoint{0.560666in}{1.822705in}}%
\pgfpathlineto{\pgfqpoint{0.583202in}{1.813249in}}%
\pgfpathlineto{\pgfqpoint{0.605737in}{1.800862in}}%
\pgfpathlineto{\pgfqpoint{0.628272in}{1.795585in}}%
\pgfpathlineto{\pgfqpoint{0.650808in}{1.791735in}}%
\pgfpathlineto{\pgfqpoint{0.673343in}{1.787366in}}%
\pgfpathlineto{\pgfqpoint{0.695878in}{1.781764in}}%
\pgfpathlineto{\pgfqpoint{0.718413in}{1.778201in}}%
\pgfpathlineto{\pgfqpoint{0.740949in}{1.773869in}}%
\pgfpathlineto{\pgfqpoint{0.763484in}{1.770367in}}%
\pgfpathlineto{\pgfqpoint{0.786019in}{1.766693in}}%
\pgfpathlineto{\pgfqpoint{0.808555in}{1.763659in}}%
\pgfpathlineto{\pgfqpoint{0.831090in}{1.758124in}}%
\pgfpathlineto{\pgfqpoint{0.853625in}{1.754131in}}%
\pgfpathlineto{\pgfqpoint{0.876160in}{1.750192in}}%
\pgfpathlineto{\pgfqpoint{0.898696in}{1.745834in}}%
\pgfpathlineto{\pgfqpoint{0.921231in}{1.742440in}}%
\pgfpathlineto{\pgfqpoint{0.943766in}{1.738262in}}%
\pgfpathlineto{\pgfqpoint{0.966302in}{1.734756in}}%
\pgfpathlineto{\pgfqpoint{0.988837in}{1.731913in}}%
\pgfpathlineto{\pgfqpoint{1.011372in}{1.728128in}}%
\pgfpathlineto{\pgfqpoint{1.033907in}{1.725607in}}%
\pgfpathlineto{\pgfqpoint{1.056443in}{1.722519in}}%
\pgfpathlineto{\pgfqpoint{1.078978in}{1.719836in}}%
\pgfpathlineto{\pgfqpoint{1.101513in}{1.717157in}}%
\pgfpathlineto{\pgfqpoint{1.124049in}{1.714783in}}%
\pgfpathlineto{\pgfqpoint{1.146584in}{1.711704in}}%
\pgfpathlineto{\pgfqpoint{1.169119in}{1.708975in}}%
\pgfpathlineto{\pgfqpoint{1.191654in}{1.706029in}}%
\pgfpathlineto{\pgfqpoint{1.214190in}{1.703630in}}%
\pgfpathlineto{\pgfqpoint{1.236725in}{1.700256in}}%
\pgfpathlineto{\pgfqpoint{1.259260in}{1.698810in}}%
\pgfpathlineto{\pgfqpoint{1.281796in}{1.695642in}}%
\pgfpathlineto{\pgfqpoint{1.304331in}{1.693994in}}%
\pgfpathlineto{\pgfqpoint{1.326866in}{1.691715in}}%
\pgfpathlineto{\pgfqpoint{1.349402in}{1.689499in}}%
\pgfpathlineto{\pgfqpoint{1.371937in}{1.686538in}}%
\pgfpathlineto{\pgfqpoint{1.394472in}{1.683968in}}%
\pgfpathlineto{\pgfqpoint{1.417007in}{1.682260in}}%
\pgfpathlineto{\pgfqpoint{1.439543in}{1.679611in}}%
\pgfpathlineto{\pgfqpoint{1.462078in}{1.676904in}}%
\pgfpathlineto{\pgfqpoint{1.484613in}{1.673796in}}%
\pgfpathlineto{\pgfqpoint{1.507149in}{1.671530in}}%
\pgfpathlineto{\pgfqpoint{1.529684in}{1.669718in}}%
\pgfpathlineto{\pgfqpoint{1.552219in}{1.667763in}}%
\pgfpathlineto{\pgfqpoint{1.574754in}{1.666610in}}%
\pgfpathlineto{\pgfqpoint{1.597290in}{1.664173in}}%
\pgfpathlineto{\pgfqpoint{1.619825in}{1.661702in}}%
\pgfpathlineto{\pgfqpoint{1.642360in}{1.659232in}}%
\pgfpathlineto{\pgfqpoint{1.664896in}{1.657323in}}%
\pgfpathlineto{\pgfqpoint{1.687431in}{1.655051in}}%
\pgfpathlineto{\pgfqpoint{1.709966in}{1.652812in}}%
\pgfpathlineto{\pgfqpoint{1.732501in}{1.650450in}}%
\pgfpathlineto{\pgfqpoint{1.755037in}{1.647274in}}%
\pgfpathlineto{\pgfqpoint{1.777572in}{1.645380in}}%
\pgfpathlineto{\pgfqpoint{1.800107in}{1.642833in}}%
\pgfpathlineto{\pgfqpoint{1.822643in}{1.640456in}}%
\pgfpathlineto{\pgfqpoint{1.845178in}{1.638128in}}%
\pgfpathlineto{\pgfqpoint{1.867713in}{1.635576in}}%
\pgfpathlineto{\pgfqpoint{1.890248in}{1.633075in}}%
\pgfpathlineto{\pgfqpoint{1.912784in}{1.631015in}}%
\pgfpathlineto{\pgfqpoint{1.935319in}{1.628225in}}%
\pgfpathlineto{\pgfqpoint{1.957854in}{1.625551in}}%
\pgfpathlineto{\pgfqpoint{1.980390in}{1.622408in}}%
\pgfpathlineto{\pgfqpoint{2.002925in}{1.620266in}}%
\pgfpathlineto{\pgfqpoint{2.025460in}{1.617980in}}%
\pgfpathlineto{\pgfqpoint{2.047995in}{1.615462in}}%
\pgfpathlineto{\pgfqpoint{2.070531in}{1.612511in}}%
\pgfpathlineto{\pgfqpoint{2.093066in}{1.609581in}}%
\pgfpathlineto{\pgfqpoint{2.115601in}{1.606522in}}%
\pgfpathlineto{\pgfqpoint{2.138137in}{1.603826in}}%
\pgfpathlineto{\pgfqpoint{2.160672in}{1.600067in}}%
\pgfpathlineto{\pgfqpoint{2.183207in}{1.597534in}}%
\pgfpathlineto{\pgfqpoint{2.205742in}{1.594454in}}%
\pgfpathlineto{\pgfqpoint{2.228278in}{1.591215in}}%
\pgfpathlineto{\pgfqpoint{2.250813in}{1.587779in}}%
\pgfpathlineto{\pgfqpoint{2.273348in}{1.584781in}}%
\pgfpathlineto{\pgfqpoint{2.295884in}{1.581553in}}%
\pgfpathlineto{\pgfqpoint{2.318419in}{1.578239in}}%
\pgfpathlineto{\pgfqpoint{2.340954in}{1.574951in}}%
\pgfpathlineto{\pgfqpoint{2.363489in}{1.571443in}}%
\pgfpathlineto{\pgfqpoint{2.386025in}{1.567774in}}%
\pgfpathlineto{\pgfqpoint{2.408560in}{1.564411in}}%
\pgfpathlineto{\pgfqpoint{2.431095in}{1.561069in}}%
\pgfpathlineto{\pgfqpoint{2.453631in}{1.558420in}}%
\pgfpathlineto{\pgfqpoint{2.476166in}{1.555296in}}%
\pgfpathlineto{\pgfqpoint{2.498701in}{1.551490in}}%
\pgfpathlineto{\pgfqpoint{2.521236in}{1.547403in}}%
\pgfpathlineto{\pgfqpoint{2.543772in}{1.543786in}}%
\pgfpathlineto{\pgfqpoint{2.566307in}{1.541025in}}%
\pgfpathlineto{\pgfqpoint{2.588842in}{1.537502in}}%
\pgfpathlineto{\pgfqpoint{2.611378in}{1.534613in}}%
\pgfpathlineto{\pgfqpoint{2.633913in}{1.530824in}}%
\pgfpathlineto{\pgfqpoint{2.656448in}{1.527405in}}%
\pgfpathlineto{\pgfqpoint{2.678983in}{1.523642in}}%
\pgfpathlineto{\pgfqpoint{2.701519in}{1.519854in}}%
\pgfusepath{stroke}%
\end{pgfscope}%
\begin{pgfscope}%
\pgfpathrectangle{\pgfqpoint{0.470525in}{0.422992in}}{\pgfqpoint{2.276064in}{1.626201in}}%
\pgfusepath{clip}%
\pgfsetrectcap%
\pgfsetroundjoin%
\pgfsetlinewidth{1.003750pt}%
\definecolor{currentstroke}{rgb}{1.000000,0.584314,0.000000}%
\pgfsetstrokecolor{currentstroke}%
\pgfsetdash{}{0pt}%
\pgfpathmoveto{\pgfqpoint{0.493061in}{1.951540in}}%
\pgfpathlineto{\pgfqpoint{0.515596in}{1.948375in}}%
\pgfpathlineto{\pgfqpoint{0.538131in}{1.943585in}}%
\pgfpathlineto{\pgfqpoint{0.560666in}{1.942049in}}%
\pgfpathlineto{\pgfqpoint{0.583202in}{1.939917in}}%
\pgfpathlineto{\pgfqpoint{0.605737in}{1.935458in}}%
\pgfpathlineto{\pgfqpoint{0.628272in}{1.934427in}}%
\pgfpathlineto{\pgfqpoint{0.650808in}{1.932782in}}%
\pgfpathlineto{\pgfqpoint{0.673343in}{1.931548in}}%
\pgfpathlineto{\pgfqpoint{0.695878in}{1.929319in}}%
\pgfpathlineto{\pgfqpoint{0.718413in}{1.928505in}}%
\pgfpathlineto{\pgfqpoint{0.740949in}{1.927535in}}%
\pgfpathlineto{\pgfqpoint{0.763484in}{1.926789in}}%
\pgfpathlineto{\pgfqpoint{0.786019in}{1.925549in}}%
\pgfpathlineto{\pgfqpoint{0.808555in}{1.924623in}}%
\pgfpathlineto{\pgfqpoint{0.831090in}{1.923055in}}%
\pgfpathlineto{\pgfqpoint{0.853625in}{1.922877in}}%
\pgfpathlineto{\pgfqpoint{0.876160in}{1.922421in}}%
\pgfpathlineto{\pgfqpoint{0.898696in}{1.920915in}}%
\pgfpathlineto{\pgfqpoint{0.921231in}{1.920423in}}%
\pgfpathlineto{\pgfqpoint{0.943766in}{1.919989in}}%
\pgfpathlineto{\pgfqpoint{0.966302in}{1.919004in}}%
\pgfpathlineto{\pgfqpoint{0.988837in}{1.918622in}}%
\pgfpathlineto{\pgfqpoint{1.011372in}{1.917173in}}%
\pgfpathlineto{\pgfqpoint{1.033907in}{1.916589in}}%
\pgfpathlineto{\pgfqpoint{1.056443in}{1.915860in}}%
\pgfpathlineto{\pgfqpoint{1.078978in}{1.915320in}}%
\pgfpathlineto{\pgfqpoint{1.101513in}{1.914330in}}%
\pgfpathlineto{\pgfqpoint{1.124049in}{1.913843in}}%
\pgfpathlineto{\pgfqpoint{1.146584in}{1.913384in}}%
\pgfpathlineto{\pgfqpoint{1.169119in}{1.912978in}}%
\pgfpathlineto{\pgfqpoint{1.191654in}{1.912290in}}%
\pgfpathlineto{\pgfqpoint{1.214190in}{1.911719in}}%
\pgfpathlineto{\pgfqpoint{1.236725in}{1.910784in}}%
\pgfpathlineto{\pgfqpoint{1.259260in}{1.910544in}}%
\pgfpathlineto{\pgfqpoint{1.281796in}{1.909420in}}%
\pgfpathlineto{\pgfqpoint{1.304331in}{1.909164in}}%
\pgfpathlineto{\pgfqpoint{1.326866in}{1.908667in}}%
\pgfpathlineto{\pgfqpoint{1.349402in}{1.908256in}}%
\pgfpathlineto{\pgfqpoint{1.371937in}{1.907676in}}%
\pgfpathlineto{\pgfqpoint{1.394472in}{1.907332in}}%
\pgfpathlineto{\pgfqpoint{1.417007in}{1.907106in}}%
\pgfpathlineto{\pgfqpoint{1.439543in}{1.906795in}}%
\pgfpathlineto{\pgfqpoint{1.462078in}{1.906247in}}%
\pgfpathlineto{\pgfqpoint{1.484613in}{1.905605in}}%
\pgfpathlineto{\pgfqpoint{1.507149in}{1.905502in}}%
\pgfpathlineto{\pgfqpoint{1.529684in}{1.904942in}}%
\pgfpathlineto{\pgfqpoint{1.552219in}{1.904776in}}%
\pgfpathlineto{\pgfqpoint{1.574754in}{1.904496in}}%
\pgfpathlineto{\pgfqpoint{1.597290in}{1.903843in}}%
\pgfpathlineto{\pgfqpoint{1.619825in}{1.903254in}}%
\pgfpathlineto{\pgfqpoint{1.642360in}{1.902532in}}%
\pgfpathlineto{\pgfqpoint{1.664896in}{1.902109in}}%
\pgfpathlineto{\pgfqpoint{1.687431in}{1.901461in}}%
\pgfpathlineto{\pgfqpoint{1.709966in}{1.900801in}}%
\pgfpathlineto{\pgfqpoint{1.732501in}{1.900130in}}%
\pgfpathlineto{\pgfqpoint{1.755037in}{1.899497in}}%
\pgfpathlineto{\pgfqpoint{1.777572in}{1.899042in}}%
\pgfpathlineto{\pgfqpoint{1.800107in}{1.898222in}}%
\pgfpathlineto{\pgfqpoint{1.822643in}{1.897792in}}%
\pgfpathlineto{\pgfqpoint{1.845178in}{1.897072in}}%
\pgfpathlineto{\pgfqpoint{1.867713in}{1.896669in}}%
\pgfpathlineto{\pgfqpoint{1.890248in}{1.896191in}}%
\pgfpathlineto{\pgfqpoint{1.912784in}{1.895544in}}%
\pgfpathlineto{\pgfqpoint{1.935319in}{1.894980in}}%
\pgfpathlineto{\pgfqpoint{1.957854in}{1.894690in}}%
\pgfpathlineto{\pgfqpoint{1.980390in}{1.893823in}}%
\pgfpathlineto{\pgfqpoint{2.002925in}{1.892842in}}%
\pgfpathlineto{\pgfqpoint{2.025460in}{1.892325in}}%
\pgfpathlineto{\pgfqpoint{2.047995in}{1.891848in}}%
\pgfpathlineto{\pgfqpoint{2.070531in}{1.890995in}}%
\pgfpathlineto{\pgfqpoint{2.093066in}{1.890235in}}%
\pgfpathlineto{\pgfqpoint{2.115601in}{1.889317in}}%
\pgfpathlineto{\pgfqpoint{2.138137in}{1.888801in}}%
\pgfpathlineto{\pgfqpoint{2.160672in}{1.887762in}}%
\pgfpathlineto{\pgfqpoint{2.183207in}{1.887171in}}%
\pgfpathlineto{\pgfqpoint{2.205742in}{1.886432in}}%
\pgfpathlineto{\pgfqpoint{2.228278in}{1.885135in}}%
\pgfpathlineto{\pgfqpoint{2.250813in}{1.884023in}}%
\pgfpathlineto{\pgfqpoint{2.273348in}{1.883430in}}%
\pgfpathlineto{\pgfqpoint{2.295884in}{1.882222in}}%
\pgfpathlineto{\pgfqpoint{2.318419in}{1.880809in}}%
\pgfpathlineto{\pgfqpoint{2.340954in}{1.879756in}}%
\pgfpathlineto{\pgfqpoint{2.363489in}{1.878741in}}%
\pgfpathlineto{\pgfqpoint{2.386025in}{1.877517in}}%
\pgfpathlineto{\pgfqpoint{2.408560in}{1.876340in}}%
\pgfpathlineto{\pgfqpoint{2.431095in}{1.875212in}}%
\pgfpathlineto{\pgfqpoint{2.453631in}{1.874252in}}%
\pgfpathlineto{\pgfqpoint{2.476166in}{1.873433in}}%
\pgfpathlineto{\pgfqpoint{2.498701in}{1.872322in}}%
\pgfpathlineto{\pgfqpoint{2.521236in}{1.870930in}}%
\pgfpathlineto{\pgfqpoint{2.543772in}{1.869905in}}%
\pgfpathlineto{\pgfqpoint{2.566307in}{1.868862in}}%
\pgfpathlineto{\pgfqpoint{2.588842in}{1.867880in}}%
\pgfpathlineto{\pgfqpoint{2.611378in}{1.866954in}}%
\pgfpathlineto{\pgfqpoint{2.633913in}{1.865865in}}%
\pgfpathlineto{\pgfqpoint{2.656448in}{1.864633in}}%
\pgfpathlineto{\pgfqpoint{2.678983in}{1.863703in}}%
\pgfpathlineto{\pgfqpoint{2.701519in}{1.862294in}}%
\pgfusepath{stroke}%
\end{pgfscope}%
\begin{pgfscope}%
\pgfpathrectangle{\pgfqpoint{0.470525in}{0.422992in}}{\pgfqpoint{2.276064in}{1.626201in}}%
\pgfusepath{clip}%
\pgfsetrectcap%
\pgfsetroundjoin%
\pgfsetlinewidth{1.003750pt}%
\definecolor{currentstroke}{rgb}{1.000000,0.172549,0.000000}%
\pgfsetstrokecolor{currentstroke}%
\pgfsetdash{}{0pt}%
\pgfpathmoveto{\pgfqpoint{0.493061in}{1.800452in}}%
\pgfpathlineto{\pgfqpoint{0.515596in}{1.745635in}}%
\pgfpathlineto{\pgfqpoint{0.538131in}{1.719798in}}%
\pgfpathlineto{\pgfqpoint{0.560666in}{1.693197in}}%
\pgfpathlineto{\pgfqpoint{0.583202in}{1.672981in}}%
\pgfpathlineto{\pgfqpoint{0.605737in}{1.641784in}}%
\pgfpathlineto{\pgfqpoint{0.628272in}{1.616455in}}%
\pgfpathlineto{\pgfqpoint{0.650808in}{1.587292in}}%
\pgfpathlineto{\pgfqpoint{0.673343in}{1.566343in}}%
\pgfpathlineto{\pgfqpoint{0.695878in}{1.543026in}}%
\pgfpathlineto{\pgfqpoint{0.718413in}{1.511936in}}%
\pgfpathlineto{\pgfqpoint{0.740949in}{1.484766in}}%
\pgfpathlineto{\pgfqpoint{0.763484in}{1.454657in}}%
\pgfpathlineto{\pgfqpoint{0.786019in}{1.427039in}}%
\pgfpathlineto{\pgfqpoint{0.808555in}{1.402129in}}%
\pgfpathlineto{\pgfqpoint{0.831090in}{1.376599in}}%
\pgfpathlineto{\pgfqpoint{0.853625in}{1.352246in}}%
\pgfpathlineto{\pgfqpoint{0.876160in}{1.331368in}}%
\pgfpathlineto{\pgfqpoint{0.898696in}{1.308236in}}%
\pgfpathlineto{\pgfqpoint{0.921231in}{1.293512in}}%
\pgfpathlineto{\pgfqpoint{0.943766in}{1.275994in}}%
\pgfpathlineto{\pgfqpoint{0.966302in}{1.260879in}}%
\pgfpathlineto{\pgfqpoint{0.988837in}{1.249286in}}%
\pgfpathlineto{\pgfqpoint{1.011372in}{1.231093in}}%
\pgfpathlineto{\pgfqpoint{1.033907in}{1.212259in}}%
\pgfpathlineto{\pgfqpoint{1.056443in}{1.197762in}}%
\pgfpathlineto{\pgfqpoint{1.078978in}{1.181344in}}%
\pgfpathlineto{\pgfqpoint{1.101513in}{1.166541in}}%
\pgfpathlineto{\pgfqpoint{1.124049in}{1.152829in}}%
\pgfpathlineto{\pgfqpoint{1.146584in}{1.137245in}}%
\pgfpathlineto{\pgfqpoint{1.169119in}{1.122078in}}%
\pgfpathlineto{\pgfqpoint{1.191654in}{1.106551in}}%
\pgfpathlineto{\pgfqpoint{1.214190in}{1.089981in}}%
\pgfpathlineto{\pgfqpoint{1.236725in}{1.074868in}}%
\pgfpathlineto{\pgfqpoint{1.259260in}{1.060766in}}%
\pgfpathlineto{\pgfqpoint{1.281796in}{1.046507in}}%
\pgfpathlineto{\pgfqpoint{1.304331in}{1.033943in}}%
\pgfpathlineto{\pgfqpoint{1.326866in}{1.024325in}}%
\pgfpathlineto{\pgfqpoint{1.349402in}{1.012913in}}%
\pgfpathlineto{\pgfqpoint{1.371937in}{1.001978in}}%
\pgfpathlineto{\pgfqpoint{1.394472in}{0.989497in}}%
\pgfpathlineto{\pgfqpoint{1.417007in}{0.976387in}}%
\pgfpathlineto{\pgfqpoint{1.439543in}{0.963811in}}%
\pgfpathlineto{\pgfqpoint{1.462078in}{0.948145in}}%
\pgfpathlineto{\pgfqpoint{1.484613in}{0.935979in}}%
\pgfpathlineto{\pgfqpoint{1.507149in}{0.923181in}}%
\pgfpathlineto{\pgfqpoint{1.529684in}{0.910945in}}%
\pgfpathlineto{\pgfqpoint{1.552219in}{0.898804in}}%
\pgfpathlineto{\pgfqpoint{1.574754in}{0.889277in}}%
\pgfpathlineto{\pgfqpoint{1.597290in}{0.877002in}}%
\pgfpathlineto{\pgfqpoint{1.619825in}{0.866191in}}%
\pgfpathlineto{\pgfqpoint{1.642360in}{0.854757in}}%
\pgfpathlineto{\pgfqpoint{1.664896in}{0.844058in}}%
\pgfpathlineto{\pgfqpoint{1.687431in}{0.834970in}}%
\pgfpathlineto{\pgfqpoint{1.709966in}{0.825337in}}%
\pgfpathlineto{\pgfqpoint{1.732501in}{0.817382in}}%
\pgfpathlineto{\pgfqpoint{1.755037in}{0.805987in}}%
\pgfpathlineto{\pgfqpoint{1.777572in}{0.798486in}}%
\pgfpathlineto{\pgfqpoint{1.800107in}{0.787195in}}%
\pgfpathlineto{\pgfqpoint{1.822643in}{0.776858in}}%
\pgfpathlineto{\pgfqpoint{1.845178in}{0.767284in}}%
\pgfpathlineto{\pgfqpoint{1.867713in}{0.757964in}}%
\pgfpathlineto{\pgfqpoint{1.890248in}{0.748609in}}%
\pgfpathlineto{\pgfqpoint{1.912784in}{0.741250in}}%
\pgfpathlineto{\pgfqpoint{1.935319in}{0.731758in}}%
\pgfpathlineto{\pgfqpoint{1.957854in}{0.722340in}}%
\pgfpathlineto{\pgfqpoint{1.980390in}{0.711999in}}%
\pgfpathlineto{\pgfqpoint{2.002925in}{0.705207in}}%
\pgfpathlineto{\pgfqpoint{2.025460in}{0.696600in}}%
\pgfpathlineto{\pgfqpoint{2.047995in}{0.689590in}}%
\pgfpathlineto{\pgfqpoint{2.070531in}{0.681252in}}%
\pgfpathlineto{\pgfqpoint{2.093066in}{0.671768in}}%
\pgfpathlineto{\pgfqpoint{2.115601in}{0.664177in}}%
\pgfpathlineto{\pgfqpoint{2.138137in}{0.656032in}}%
\pgfpathlineto{\pgfqpoint{2.160672in}{0.647628in}}%
\pgfpathlineto{\pgfqpoint{2.183207in}{0.640230in}}%
\pgfpathlineto{\pgfqpoint{2.205742in}{0.633466in}}%
\pgfpathlineto{\pgfqpoint{2.228278in}{0.625399in}}%
\pgfpathlineto{\pgfqpoint{2.250813in}{0.617980in}}%
\pgfpathlineto{\pgfqpoint{2.273348in}{0.611409in}}%
\pgfpathlineto{\pgfqpoint{2.295884in}{0.604446in}}%
\pgfpathlineto{\pgfqpoint{2.318419in}{0.598396in}}%
\pgfpathlineto{\pgfqpoint{2.340954in}{0.590684in}}%
\pgfpathlineto{\pgfqpoint{2.363489in}{0.583787in}}%
\pgfpathlineto{\pgfqpoint{2.386025in}{0.576099in}}%
\pgfpathlineto{\pgfqpoint{2.408560in}{0.568619in}}%
\pgfpathlineto{\pgfqpoint{2.431095in}{0.562743in}}%
\pgfpathlineto{\pgfqpoint{2.453631in}{0.557074in}}%
\pgfpathlineto{\pgfqpoint{2.476166in}{0.550228in}}%
\pgfpathlineto{\pgfqpoint{2.498701in}{0.544429in}}%
\pgfpathlineto{\pgfqpoint{2.521236in}{0.538855in}}%
\pgfpathlineto{\pgfqpoint{2.543772in}{0.532864in}}%
\pgfpathlineto{\pgfqpoint{2.566307in}{0.527587in}}%
\pgfpathlineto{\pgfqpoint{2.588842in}{0.521894in}}%
\pgfpathlineto{\pgfqpoint{2.611378in}{0.517769in}}%
\pgfpathlineto{\pgfqpoint{2.633913in}{0.513393in}}%
\pgfpathlineto{\pgfqpoint{2.656448in}{0.507784in}}%
\pgfpathlineto{\pgfqpoint{2.678983in}{0.502303in}}%
\pgfpathlineto{\pgfqpoint{2.701519in}{0.496910in}}%
\pgfusepath{stroke}%
\end{pgfscope}%
\begin{pgfscope}%
\pgfpathrectangle{\pgfqpoint{0.470525in}{0.422992in}}{\pgfqpoint{2.276064in}{1.626201in}}%
\pgfusepath{clip}%
\pgfsetrectcap%
\pgfsetroundjoin%
\pgfsetlinewidth{1.003750pt}%
\definecolor{currentstroke}{rgb}{0.517647,0.356863,0.592157}%
\pgfsetstrokecolor{currentstroke}%
\pgfsetdash{}{0pt}%
\pgfpathmoveto{\pgfqpoint{0.493061in}{1.725681in}}%
\pgfpathlineto{\pgfqpoint{0.515596in}{1.722834in}}%
\pgfpathlineto{\pgfqpoint{0.538131in}{1.724604in}}%
\pgfpathlineto{\pgfqpoint{0.560666in}{1.721330in}}%
\pgfpathlineto{\pgfqpoint{0.583202in}{1.716359in}}%
\pgfpathlineto{\pgfqpoint{0.605737in}{1.712356in}}%
\pgfpathlineto{\pgfqpoint{0.628272in}{1.708905in}}%
\pgfpathlineto{\pgfqpoint{0.650808in}{1.708663in}}%
\pgfpathlineto{\pgfqpoint{0.673343in}{1.707296in}}%
\pgfpathlineto{\pgfqpoint{0.695878in}{1.705062in}}%
\pgfpathlineto{\pgfqpoint{0.718413in}{1.705135in}}%
\pgfpathlineto{\pgfqpoint{0.740949in}{1.701912in}}%
\pgfpathlineto{\pgfqpoint{0.763484in}{1.698715in}}%
\pgfpathlineto{\pgfqpoint{0.786019in}{1.698215in}}%
\pgfpathlineto{\pgfqpoint{0.808555in}{1.696758in}}%
\pgfpathlineto{\pgfqpoint{0.831090in}{1.695399in}}%
\pgfpathlineto{\pgfqpoint{0.853625in}{1.694421in}}%
\pgfpathlineto{\pgfqpoint{0.876160in}{1.692848in}}%
\pgfpathlineto{\pgfqpoint{0.898696in}{1.690900in}}%
\pgfpathlineto{\pgfqpoint{0.921231in}{1.690708in}}%
\pgfpathlineto{\pgfqpoint{0.943766in}{1.690970in}}%
\pgfpathlineto{\pgfqpoint{0.966302in}{1.690591in}}%
\pgfpathlineto{\pgfqpoint{0.988837in}{1.690308in}}%
\pgfpathlineto{\pgfqpoint{1.011372in}{1.690986in}}%
\pgfpathlineto{\pgfqpoint{1.033907in}{1.690726in}}%
\pgfpathlineto{\pgfqpoint{1.056443in}{1.690830in}}%
\pgfpathlineto{\pgfqpoint{1.078978in}{1.690608in}}%
\pgfpathlineto{\pgfqpoint{1.101513in}{1.688690in}}%
\pgfpathlineto{\pgfqpoint{1.124049in}{1.688201in}}%
\pgfpathlineto{\pgfqpoint{1.146584in}{1.688044in}}%
\pgfpathlineto{\pgfqpoint{1.169119in}{1.687980in}}%
\pgfpathlineto{\pgfqpoint{1.191654in}{1.687585in}}%
\pgfpathlineto{\pgfqpoint{1.214190in}{1.686981in}}%
\pgfpathlineto{\pgfqpoint{1.236725in}{1.686649in}}%
\pgfpathlineto{\pgfqpoint{1.259260in}{1.686732in}}%
\pgfpathlineto{\pgfqpoint{1.281796in}{1.685560in}}%
\pgfpathlineto{\pgfqpoint{1.304331in}{1.685046in}}%
\pgfpathlineto{\pgfqpoint{1.326866in}{1.684433in}}%
\pgfpathlineto{\pgfqpoint{1.349402in}{1.684425in}}%
\pgfpathlineto{\pgfqpoint{1.371937in}{1.683689in}}%
\pgfpathlineto{\pgfqpoint{1.394472in}{1.683272in}}%
\pgfpathlineto{\pgfqpoint{1.417007in}{1.682614in}}%
\pgfpathlineto{\pgfqpoint{1.439543in}{1.681167in}}%
\pgfpathlineto{\pgfqpoint{1.462078in}{1.680285in}}%
\pgfpathlineto{\pgfqpoint{1.484613in}{1.679600in}}%
\pgfpathlineto{\pgfqpoint{1.507149in}{1.678661in}}%
\pgfpathlineto{\pgfqpoint{1.529684in}{1.678361in}}%
\pgfpathlineto{\pgfqpoint{1.552219in}{1.678456in}}%
\pgfpathlineto{\pgfqpoint{1.574754in}{1.678077in}}%
\pgfpathlineto{\pgfqpoint{1.597290in}{1.677928in}}%
\pgfpathlineto{\pgfqpoint{1.619825in}{1.676677in}}%
\pgfpathlineto{\pgfqpoint{1.642360in}{1.674896in}}%
\pgfpathlineto{\pgfqpoint{1.664896in}{1.674851in}}%
\pgfpathlineto{\pgfqpoint{1.687431in}{1.673598in}}%
\pgfpathlineto{\pgfqpoint{1.709966in}{1.673553in}}%
\pgfpathlineto{\pgfqpoint{1.732501in}{1.672603in}}%
\pgfpathlineto{\pgfqpoint{1.755037in}{1.671534in}}%
\pgfpathlineto{\pgfqpoint{1.777572in}{1.670997in}}%
\pgfpathlineto{\pgfqpoint{1.800107in}{1.670065in}}%
\pgfpathlineto{\pgfqpoint{1.822643in}{1.668643in}}%
\pgfpathlineto{\pgfqpoint{1.845178in}{1.667706in}}%
\pgfpathlineto{\pgfqpoint{1.867713in}{1.667284in}}%
\pgfpathlineto{\pgfqpoint{1.890248in}{1.666365in}}%
\pgfpathlineto{\pgfqpoint{1.912784in}{1.664869in}}%
\pgfpathlineto{\pgfqpoint{1.935319in}{1.663565in}}%
\pgfpathlineto{\pgfqpoint{1.957854in}{1.662632in}}%
\pgfpathlineto{\pgfqpoint{1.980390in}{1.661725in}}%
\pgfpathlineto{\pgfqpoint{2.002925in}{1.660578in}}%
\pgfpathlineto{\pgfqpoint{2.025460in}{1.659296in}}%
\pgfpathlineto{\pgfqpoint{2.047995in}{1.657633in}}%
\pgfpathlineto{\pgfqpoint{2.070531in}{1.655821in}}%
\pgfpathlineto{\pgfqpoint{2.093066in}{1.654184in}}%
\pgfpathlineto{\pgfqpoint{2.115601in}{1.652502in}}%
\pgfpathlineto{\pgfqpoint{2.138137in}{1.650517in}}%
\pgfpathlineto{\pgfqpoint{2.160672in}{1.647950in}}%
\pgfpathlineto{\pgfqpoint{2.183207in}{1.646218in}}%
\pgfpathlineto{\pgfqpoint{2.205742in}{1.643416in}}%
\pgfpathlineto{\pgfqpoint{2.228278in}{1.640973in}}%
\pgfpathlineto{\pgfqpoint{2.250813in}{1.638113in}}%
\pgfpathlineto{\pgfqpoint{2.273348in}{1.636357in}}%
\pgfpathlineto{\pgfqpoint{2.295884in}{1.634259in}}%
\pgfpathlineto{\pgfqpoint{2.318419in}{1.631744in}}%
\pgfpathlineto{\pgfqpoint{2.340954in}{1.629923in}}%
\pgfpathlineto{\pgfqpoint{2.363489in}{1.627280in}}%
\pgfpathlineto{\pgfqpoint{2.386025in}{1.623991in}}%
\pgfpathlineto{\pgfqpoint{2.408560in}{1.621455in}}%
\pgfpathlineto{\pgfqpoint{2.431095in}{1.617952in}}%
\pgfpathlineto{\pgfqpoint{2.453631in}{1.615718in}}%
\pgfpathlineto{\pgfqpoint{2.476166in}{1.613385in}}%
\pgfpathlineto{\pgfqpoint{2.498701in}{1.610742in}}%
\pgfpathlineto{\pgfqpoint{2.521236in}{1.608267in}}%
\pgfpathlineto{\pgfqpoint{2.543772in}{1.605632in}}%
\pgfpathlineto{\pgfqpoint{2.566307in}{1.603729in}}%
\pgfpathlineto{\pgfqpoint{2.588842in}{1.601694in}}%
\pgfpathlineto{\pgfqpoint{2.611378in}{1.599288in}}%
\pgfpathlineto{\pgfqpoint{2.633913in}{1.596229in}}%
\pgfpathlineto{\pgfqpoint{2.656448in}{1.593552in}}%
\pgfpathlineto{\pgfqpoint{2.678983in}{1.590945in}}%
\pgfpathlineto{\pgfqpoint{2.701519in}{1.587575in}}%
\pgfusepath{stroke}%
\end{pgfscope}%
\begin{pgfscope}%
\pgfsetrectcap%
\pgfsetmiterjoin%
\pgfsetlinewidth{0.501875pt}%
\definecolor{currentstroke}{rgb}{0.000000,0.000000,0.000000}%
\pgfsetstrokecolor{currentstroke}%
\pgfsetdash{}{0pt}%
\pgfpathmoveto{\pgfqpoint{0.470525in}{0.422992in}}%
\pgfpathlineto{\pgfqpoint{0.470525in}{2.049193in}}%
\pgfusepath{stroke}%
\end{pgfscope}%
\begin{pgfscope}%
\pgfsetrectcap%
\pgfsetmiterjoin%
\pgfsetlinewidth{0.501875pt}%
\definecolor{currentstroke}{rgb}{0.000000,0.000000,0.000000}%
\pgfsetstrokecolor{currentstroke}%
\pgfsetdash{}{0pt}%
\pgfpathmoveto{\pgfqpoint{2.746589in}{0.422992in}}%
\pgfpathlineto{\pgfqpoint{2.746589in}{2.049193in}}%
\pgfusepath{stroke}%
\end{pgfscope}%
\begin{pgfscope}%
\pgfsetrectcap%
\pgfsetmiterjoin%
\pgfsetlinewidth{0.501875pt}%
\definecolor{currentstroke}{rgb}{0.000000,0.000000,0.000000}%
\pgfsetstrokecolor{currentstroke}%
\pgfsetdash{}{0pt}%
\pgfpathmoveto{\pgfqpoint{0.470525in}{0.422992in}}%
\pgfpathlineto{\pgfqpoint{2.746589in}{0.422992in}}%
\pgfusepath{stroke}%
\end{pgfscope}%
\begin{pgfscope}%
\pgfsetrectcap%
\pgfsetmiterjoin%
\pgfsetlinewidth{0.501875pt}%
\definecolor{currentstroke}{rgb}{0.000000,0.000000,0.000000}%
\pgfsetstrokecolor{currentstroke}%
\pgfsetdash{}{0pt}%
\pgfpathmoveto{\pgfqpoint{0.470525in}{2.049193in}}%
\pgfpathlineto{\pgfqpoint{2.746589in}{2.049193in}}%
\pgfusepath{stroke}%
\end{pgfscope}%
\begin{pgfscope}%
\definecolor{textcolor}{rgb}{0.000000,0.000000,0.000000}%
\pgfsetstrokecolor{textcolor}%
\pgfsetfillcolor{textcolor}%
\pgftext[x=1.608557in,y=2.132526in,,base]{\color{textcolor}\rmfamily\fontsize{12.000000}{14.400000}\selectfont Continuity}%
\end{pgfscope}%
\begin{pgfscope}%
\pgfsetbuttcap%
\pgfsetmiterjoin%
\definecolor{currentfill}{rgb}{1.000000,1.000000,1.000000}%
\pgfsetfillcolor{currentfill}%
\pgfsetlinewidth{0.000000pt}%
\definecolor{currentstroke}{rgb}{0.000000,0.000000,0.000000}%
\pgfsetstrokecolor{currentstroke}%
\pgfsetstrokeopacity{0.000000}%
\pgfsetdash{}{0pt}%
\pgfpathmoveto{\pgfqpoint{3.315933in}{0.422992in}}%
\pgfpathlineto{\pgfqpoint{5.591997in}{0.422992in}}%
\pgfpathlineto{\pgfqpoint{5.591997in}{4.374193in}}%
\pgfpathlineto{\pgfqpoint{3.315933in}{4.374193in}}%
\pgfpathlineto{\pgfqpoint{3.315933in}{0.422992in}}%
\pgfpathclose%
\pgfusepath{fill}%
\end{pgfscope}%
\begin{pgfscope}%
\pgfsetbuttcap%
\pgfsetroundjoin%
\definecolor{currentfill}{rgb}{0.000000,0.000000,0.000000}%
\pgfsetfillcolor{currentfill}%
\pgfsetlinewidth{0.501875pt}%
\definecolor{currentstroke}{rgb}{0.000000,0.000000,0.000000}%
\pgfsetstrokecolor{currentstroke}%
\pgfsetdash{}{0pt}%
\pgfsys@defobject{currentmarker}{\pgfqpoint{0.000000in}{0.000000in}}{\pgfqpoint{0.000000in}{0.041667in}}{%
\pgfpathmoveto{\pgfqpoint{0.000000in}{0.000000in}}%
\pgfpathlineto{\pgfqpoint{0.000000in}{0.041667in}}%
\pgfusepath{stroke,fill}%
}%
\begin{pgfscope}%
\pgfsys@transformshift{3.315933in}{0.422992in}%
\pgfsys@useobject{currentmarker}{}%
\end{pgfscope}%
\end{pgfscope}%
\begin{pgfscope}%
\pgfsetbuttcap%
\pgfsetroundjoin%
\definecolor{currentfill}{rgb}{0.000000,0.000000,0.000000}%
\pgfsetfillcolor{currentfill}%
\pgfsetlinewidth{0.501875pt}%
\definecolor{currentstroke}{rgb}{0.000000,0.000000,0.000000}%
\pgfsetstrokecolor{currentstroke}%
\pgfsetdash{}{0pt}%
\pgfsys@defobject{currentmarker}{\pgfqpoint{0.000000in}{-0.041667in}}{\pgfqpoint{0.000000in}{0.000000in}}{%
\pgfpathmoveto{\pgfqpoint{0.000000in}{0.000000in}}%
\pgfpathlineto{\pgfqpoint{0.000000in}{-0.041667in}}%
\pgfusepath{stroke,fill}%
}%
\begin{pgfscope}%
\pgfsys@transformshift{3.315933in}{4.374193in}%
\pgfsys@useobject{currentmarker}{}%
\end{pgfscope}%
\end{pgfscope}%
\begin{pgfscope}%
\definecolor{textcolor}{rgb}{0.000000,0.000000,0.000000}%
\pgfsetstrokecolor{textcolor}%
\pgfsetfillcolor{textcolor}%
\pgftext[x=3.315933in,y=0.374381in,,top]{\color{textcolor}\rmfamily\fontsize{10.000000}{12.000000}\selectfont \(\displaystyle {0}\)}%
\end{pgfscope}%
\begin{pgfscope}%
\pgfsetbuttcap%
\pgfsetroundjoin%
\definecolor{currentfill}{rgb}{0.000000,0.000000,0.000000}%
\pgfsetfillcolor{currentfill}%
\pgfsetlinewidth{0.501875pt}%
\definecolor{currentstroke}{rgb}{0.000000,0.000000,0.000000}%
\pgfsetstrokecolor{currentstroke}%
\pgfsetdash{}{0pt}%
\pgfsys@defobject{currentmarker}{\pgfqpoint{0.000000in}{0.000000in}}{\pgfqpoint{0.000000in}{0.041667in}}{%
\pgfpathmoveto{\pgfqpoint{0.000000in}{0.000000in}}%
\pgfpathlineto{\pgfqpoint{0.000000in}{0.041667in}}%
\pgfusepath{stroke,fill}%
}%
\begin{pgfscope}%
\pgfsys@transformshift{3.766638in}{0.422992in}%
\pgfsys@useobject{currentmarker}{}%
\end{pgfscope}%
\end{pgfscope}%
\begin{pgfscope}%
\pgfsetbuttcap%
\pgfsetroundjoin%
\definecolor{currentfill}{rgb}{0.000000,0.000000,0.000000}%
\pgfsetfillcolor{currentfill}%
\pgfsetlinewidth{0.501875pt}%
\definecolor{currentstroke}{rgb}{0.000000,0.000000,0.000000}%
\pgfsetstrokecolor{currentstroke}%
\pgfsetdash{}{0pt}%
\pgfsys@defobject{currentmarker}{\pgfqpoint{0.000000in}{-0.041667in}}{\pgfqpoint{0.000000in}{0.000000in}}{%
\pgfpathmoveto{\pgfqpoint{0.000000in}{0.000000in}}%
\pgfpathlineto{\pgfqpoint{0.000000in}{-0.041667in}}%
\pgfusepath{stroke,fill}%
}%
\begin{pgfscope}%
\pgfsys@transformshift{3.766638in}{4.374193in}%
\pgfsys@useobject{currentmarker}{}%
\end{pgfscope}%
\end{pgfscope}%
\begin{pgfscope}%
\definecolor{textcolor}{rgb}{0.000000,0.000000,0.000000}%
\pgfsetstrokecolor{textcolor}%
\pgfsetfillcolor{textcolor}%
\pgftext[x=3.766638in,y=0.374381in,,top]{\color{textcolor}\rmfamily\fontsize{10.000000}{12.000000}\selectfont \(\displaystyle {20}\)}%
\end{pgfscope}%
\begin{pgfscope}%
\pgfsetbuttcap%
\pgfsetroundjoin%
\definecolor{currentfill}{rgb}{0.000000,0.000000,0.000000}%
\pgfsetfillcolor{currentfill}%
\pgfsetlinewidth{0.501875pt}%
\definecolor{currentstroke}{rgb}{0.000000,0.000000,0.000000}%
\pgfsetstrokecolor{currentstroke}%
\pgfsetdash{}{0pt}%
\pgfsys@defobject{currentmarker}{\pgfqpoint{0.000000in}{0.000000in}}{\pgfqpoint{0.000000in}{0.041667in}}{%
\pgfpathmoveto{\pgfqpoint{0.000000in}{0.000000in}}%
\pgfpathlineto{\pgfqpoint{0.000000in}{0.041667in}}%
\pgfusepath{stroke,fill}%
}%
\begin{pgfscope}%
\pgfsys@transformshift{4.217344in}{0.422992in}%
\pgfsys@useobject{currentmarker}{}%
\end{pgfscope}%
\end{pgfscope}%
\begin{pgfscope}%
\pgfsetbuttcap%
\pgfsetroundjoin%
\definecolor{currentfill}{rgb}{0.000000,0.000000,0.000000}%
\pgfsetfillcolor{currentfill}%
\pgfsetlinewidth{0.501875pt}%
\definecolor{currentstroke}{rgb}{0.000000,0.000000,0.000000}%
\pgfsetstrokecolor{currentstroke}%
\pgfsetdash{}{0pt}%
\pgfsys@defobject{currentmarker}{\pgfqpoint{0.000000in}{-0.041667in}}{\pgfqpoint{0.000000in}{0.000000in}}{%
\pgfpathmoveto{\pgfqpoint{0.000000in}{0.000000in}}%
\pgfpathlineto{\pgfqpoint{0.000000in}{-0.041667in}}%
\pgfusepath{stroke,fill}%
}%
\begin{pgfscope}%
\pgfsys@transformshift{4.217344in}{4.374193in}%
\pgfsys@useobject{currentmarker}{}%
\end{pgfscope}%
\end{pgfscope}%
\begin{pgfscope}%
\definecolor{textcolor}{rgb}{0.000000,0.000000,0.000000}%
\pgfsetstrokecolor{textcolor}%
\pgfsetfillcolor{textcolor}%
\pgftext[x=4.217344in,y=0.374381in,,top]{\color{textcolor}\rmfamily\fontsize{10.000000}{12.000000}\selectfont \(\displaystyle {40}\)}%
\end{pgfscope}%
\begin{pgfscope}%
\pgfsetbuttcap%
\pgfsetroundjoin%
\definecolor{currentfill}{rgb}{0.000000,0.000000,0.000000}%
\pgfsetfillcolor{currentfill}%
\pgfsetlinewidth{0.501875pt}%
\definecolor{currentstroke}{rgb}{0.000000,0.000000,0.000000}%
\pgfsetstrokecolor{currentstroke}%
\pgfsetdash{}{0pt}%
\pgfsys@defobject{currentmarker}{\pgfqpoint{0.000000in}{0.000000in}}{\pgfqpoint{0.000000in}{0.041667in}}{%
\pgfpathmoveto{\pgfqpoint{0.000000in}{0.000000in}}%
\pgfpathlineto{\pgfqpoint{0.000000in}{0.041667in}}%
\pgfusepath{stroke,fill}%
}%
\begin{pgfscope}%
\pgfsys@transformshift{4.668050in}{0.422992in}%
\pgfsys@useobject{currentmarker}{}%
\end{pgfscope}%
\end{pgfscope}%
\begin{pgfscope}%
\pgfsetbuttcap%
\pgfsetroundjoin%
\definecolor{currentfill}{rgb}{0.000000,0.000000,0.000000}%
\pgfsetfillcolor{currentfill}%
\pgfsetlinewidth{0.501875pt}%
\definecolor{currentstroke}{rgb}{0.000000,0.000000,0.000000}%
\pgfsetstrokecolor{currentstroke}%
\pgfsetdash{}{0pt}%
\pgfsys@defobject{currentmarker}{\pgfqpoint{0.000000in}{-0.041667in}}{\pgfqpoint{0.000000in}{0.000000in}}{%
\pgfpathmoveto{\pgfqpoint{0.000000in}{0.000000in}}%
\pgfpathlineto{\pgfqpoint{0.000000in}{-0.041667in}}%
\pgfusepath{stroke,fill}%
}%
\begin{pgfscope}%
\pgfsys@transformshift{4.668050in}{4.374193in}%
\pgfsys@useobject{currentmarker}{}%
\end{pgfscope}%
\end{pgfscope}%
\begin{pgfscope}%
\definecolor{textcolor}{rgb}{0.000000,0.000000,0.000000}%
\pgfsetstrokecolor{textcolor}%
\pgfsetfillcolor{textcolor}%
\pgftext[x=4.668050in,y=0.374381in,,top]{\color{textcolor}\rmfamily\fontsize{10.000000}{12.000000}\selectfont \(\displaystyle {60}\)}%
\end{pgfscope}%
\begin{pgfscope}%
\pgfsetbuttcap%
\pgfsetroundjoin%
\definecolor{currentfill}{rgb}{0.000000,0.000000,0.000000}%
\pgfsetfillcolor{currentfill}%
\pgfsetlinewidth{0.501875pt}%
\definecolor{currentstroke}{rgb}{0.000000,0.000000,0.000000}%
\pgfsetstrokecolor{currentstroke}%
\pgfsetdash{}{0pt}%
\pgfsys@defobject{currentmarker}{\pgfqpoint{0.000000in}{0.000000in}}{\pgfqpoint{0.000000in}{0.041667in}}{%
\pgfpathmoveto{\pgfqpoint{0.000000in}{0.000000in}}%
\pgfpathlineto{\pgfqpoint{0.000000in}{0.041667in}}%
\pgfusepath{stroke,fill}%
}%
\begin{pgfscope}%
\pgfsys@transformshift{5.118756in}{0.422992in}%
\pgfsys@useobject{currentmarker}{}%
\end{pgfscope}%
\end{pgfscope}%
\begin{pgfscope}%
\pgfsetbuttcap%
\pgfsetroundjoin%
\definecolor{currentfill}{rgb}{0.000000,0.000000,0.000000}%
\pgfsetfillcolor{currentfill}%
\pgfsetlinewidth{0.501875pt}%
\definecolor{currentstroke}{rgb}{0.000000,0.000000,0.000000}%
\pgfsetstrokecolor{currentstroke}%
\pgfsetdash{}{0pt}%
\pgfsys@defobject{currentmarker}{\pgfqpoint{0.000000in}{-0.041667in}}{\pgfqpoint{0.000000in}{0.000000in}}{%
\pgfpathmoveto{\pgfqpoint{0.000000in}{0.000000in}}%
\pgfpathlineto{\pgfqpoint{0.000000in}{-0.041667in}}%
\pgfusepath{stroke,fill}%
}%
\begin{pgfscope}%
\pgfsys@transformshift{5.118756in}{4.374193in}%
\pgfsys@useobject{currentmarker}{}%
\end{pgfscope}%
\end{pgfscope}%
\begin{pgfscope}%
\definecolor{textcolor}{rgb}{0.000000,0.000000,0.000000}%
\pgfsetstrokecolor{textcolor}%
\pgfsetfillcolor{textcolor}%
\pgftext[x=5.118756in,y=0.374381in,,top]{\color{textcolor}\rmfamily\fontsize{10.000000}{12.000000}\selectfont \(\displaystyle {80}\)}%
\end{pgfscope}%
\begin{pgfscope}%
\pgfsetbuttcap%
\pgfsetroundjoin%
\definecolor{currentfill}{rgb}{0.000000,0.000000,0.000000}%
\pgfsetfillcolor{currentfill}%
\pgfsetlinewidth{0.501875pt}%
\definecolor{currentstroke}{rgb}{0.000000,0.000000,0.000000}%
\pgfsetstrokecolor{currentstroke}%
\pgfsetdash{}{0pt}%
\pgfsys@defobject{currentmarker}{\pgfqpoint{0.000000in}{0.000000in}}{\pgfqpoint{0.000000in}{0.020833in}}{%
\pgfpathmoveto{\pgfqpoint{0.000000in}{0.000000in}}%
\pgfpathlineto{\pgfqpoint{0.000000in}{0.020833in}}%
\pgfusepath{stroke,fill}%
}%
\begin{pgfscope}%
\pgfsys@transformshift{3.428609in}{0.422992in}%
\pgfsys@useobject{currentmarker}{}%
\end{pgfscope}%
\end{pgfscope}%
\begin{pgfscope}%
\pgfsetbuttcap%
\pgfsetroundjoin%
\definecolor{currentfill}{rgb}{0.000000,0.000000,0.000000}%
\pgfsetfillcolor{currentfill}%
\pgfsetlinewidth{0.501875pt}%
\definecolor{currentstroke}{rgb}{0.000000,0.000000,0.000000}%
\pgfsetstrokecolor{currentstroke}%
\pgfsetdash{}{0pt}%
\pgfsys@defobject{currentmarker}{\pgfqpoint{0.000000in}{-0.020833in}}{\pgfqpoint{0.000000in}{0.000000in}}{%
\pgfpathmoveto{\pgfqpoint{0.000000in}{0.000000in}}%
\pgfpathlineto{\pgfqpoint{0.000000in}{-0.020833in}}%
\pgfusepath{stroke,fill}%
}%
\begin{pgfscope}%
\pgfsys@transformshift{3.428609in}{4.374193in}%
\pgfsys@useobject{currentmarker}{}%
\end{pgfscope}%
\end{pgfscope}%
\begin{pgfscope}%
\pgfsetbuttcap%
\pgfsetroundjoin%
\definecolor{currentfill}{rgb}{0.000000,0.000000,0.000000}%
\pgfsetfillcolor{currentfill}%
\pgfsetlinewidth{0.501875pt}%
\definecolor{currentstroke}{rgb}{0.000000,0.000000,0.000000}%
\pgfsetstrokecolor{currentstroke}%
\pgfsetdash{}{0pt}%
\pgfsys@defobject{currentmarker}{\pgfqpoint{0.000000in}{0.000000in}}{\pgfqpoint{0.000000in}{0.020833in}}{%
\pgfpathmoveto{\pgfqpoint{0.000000in}{0.000000in}}%
\pgfpathlineto{\pgfqpoint{0.000000in}{0.020833in}}%
\pgfusepath{stroke,fill}%
}%
\begin{pgfscope}%
\pgfsys@transformshift{3.541286in}{0.422992in}%
\pgfsys@useobject{currentmarker}{}%
\end{pgfscope}%
\end{pgfscope}%
\begin{pgfscope}%
\pgfsetbuttcap%
\pgfsetroundjoin%
\definecolor{currentfill}{rgb}{0.000000,0.000000,0.000000}%
\pgfsetfillcolor{currentfill}%
\pgfsetlinewidth{0.501875pt}%
\definecolor{currentstroke}{rgb}{0.000000,0.000000,0.000000}%
\pgfsetstrokecolor{currentstroke}%
\pgfsetdash{}{0pt}%
\pgfsys@defobject{currentmarker}{\pgfqpoint{0.000000in}{-0.020833in}}{\pgfqpoint{0.000000in}{0.000000in}}{%
\pgfpathmoveto{\pgfqpoint{0.000000in}{0.000000in}}%
\pgfpathlineto{\pgfqpoint{0.000000in}{-0.020833in}}%
\pgfusepath{stroke,fill}%
}%
\begin{pgfscope}%
\pgfsys@transformshift{3.541286in}{4.374193in}%
\pgfsys@useobject{currentmarker}{}%
\end{pgfscope}%
\end{pgfscope}%
\begin{pgfscope}%
\pgfsetbuttcap%
\pgfsetroundjoin%
\definecolor{currentfill}{rgb}{0.000000,0.000000,0.000000}%
\pgfsetfillcolor{currentfill}%
\pgfsetlinewidth{0.501875pt}%
\definecolor{currentstroke}{rgb}{0.000000,0.000000,0.000000}%
\pgfsetstrokecolor{currentstroke}%
\pgfsetdash{}{0pt}%
\pgfsys@defobject{currentmarker}{\pgfqpoint{0.000000in}{0.000000in}}{\pgfqpoint{0.000000in}{0.020833in}}{%
\pgfpathmoveto{\pgfqpoint{0.000000in}{0.000000in}}%
\pgfpathlineto{\pgfqpoint{0.000000in}{0.020833in}}%
\pgfusepath{stroke,fill}%
}%
\begin{pgfscope}%
\pgfsys@transformshift{3.653962in}{0.422992in}%
\pgfsys@useobject{currentmarker}{}%
\end{pgfscope}%
\end{pgfscope}%
\begin{pgfscope}%
\pgfsetbuttcap%
\pgfsetroundjoin%
\definecolor{currentfill}{rgb}{0.000000,0.000000,0.000000}%
\pgfsetfillcolor{currentfill}%
\pgfsetlinewidth{0.501875pt}%
\definecolor{currentstroke}{rgb}{0.000000,0.000000,0.000000}%
\pgfsetstrokecolor{currentstroke}%
\pgfsetdash{}{0pt}%
\pgfsys@defobject{currentmarker}{\pgfqpoint{0.000000in}{-0.020833in}}{\pgfqpoint{0.000000in}{0.000000in}}{%
\pgfpathmoveto{\pgfqpoint{0.000000in}{0.000000in}}%
\pgfpathlineto{\pgfqpoint{0.000000in}{-0.020833in}}%
\pgfusepath{stroke,fill}%
}%
\begin{pgfscope}%
\pgfsys@transformshift{3.653962in}{4.374193in}%
\pgfsys@useobject{currentmarker}{}%
\end{pgfscope}%
\end{pgfscope}%
\begin{pgfscope}%
\pgfsetbuttcap%
\pgfsetroundjoin%
\definecolor{currentfill}{rgb}{0.000000,0.000000,0.000000}%
\pgfsetfillcolor{currentfill}%
\pgfsetlinewidth{0.501875pt}%
\definecolor{currentstroke}{rgb}{0.000000,0.000000,0.000000}%
\pgfsetstrokecolor{currentstroke}%
\pgfsetdash{}{0pt}%
\pgfsys@defobject{currentmarker}{\pgfqpoint{0.000000in}{0.000000in}}{\pgfqpoint{0.000000in}{0.020833in}}{%
\pgfpathmoveto{\pgfqpoint{0.000000in}{0.000000in}}%
\pgfpathlineto{\pgfqpoint{0.000000in}{0.020833in}}%
\pgfusepath{stroke,fill}%
}%
\begin{pgfscope}%
\pgfsys@transformshift{3.879315in}{0.422992in}%
\pgfsys@useobject{currentmarker}{}%
\end{pgfscope}%
\end{pgfscope}%
\begin{pgfscope}%
\pgfsetbuttcap%
\pgfsetroundjoin%
\definecolor{currentfill}{rgb}{0.000000,0.000000,0.000000}%
\pgfsetfillcolor{currentfill}%
\pgfsetlinewidth{0.501875pt}%
\definecolor{currentstroke}{rgb}{0.000000,0.000000,0.000000}%
\pgfsetstrokecolor{currentstroke}%
\pgfsetdash{}{0pt}%
\pgfsys@defobject{currentmarker}{\pgfqpoint{0.000000in}{-0.020833in}}{\pgfqpoint{0.000000in}{0.000000in}}{%
\pgfpathmoveto{\pgfqpoint{0.000000in}{0.000000in}}%
\pgfpathlineto{\pgfqpoint{0.000000in}{-0.020833in}}%
\pgfusepath{stroke,fill}%
}%
\begin{pgfscope}%
\pgfsys@transformshift{3.879315in}{4.374193in}%
\pgfsys@useobject{currentmarker}{}%
\end{pgfscope}%
\end{pgfscope}%
\begin{pgfscope}%
\pgfsetbuttcap%
\pgfsetroundjoin%
\definecolor{currentfill}{rgb}{0.000000,0.000000,0.000000}%
\pgfsetfillcolor{currentfill}%
\pgfsetlinewidth{0.501875pt}%
\definecolor{currentstroke}{rgb}{0.000000,0.000000,0.000000}%
\pgfsetstrokecolor{currentstroke}%
\pgfsetdash{}{0pt}%
\pgfsys@defobject{currentmarker}{\pgfqpoint{0.000000in}{0.000000in}}{\pgfqpoint{0.000000in}{0.020833in}}{%
\pgfpathmoveto{\pgfqpoint{0.000000in}{0.000000in}}%
\pgfpathlineto{\pgfqpoint{0.000000in}{0.020833in}}%
\pgfusepath{stroke,fill}%
}%
\begin{pgfscope}%
\pgfsys@transformshift{3.991991in}{0.422992in}%
\pgfsys@useobject{currentmarker}{}%
\end{pgfscope}%
\end{pgfscope}%
\begin{pgfscope}%
\pgfsetbuttcap%
\pgfsetroundjoin%
\definecolor{currentfill}{rgb}{0.000000,0.000000,0.000000}%
\pgfsetfillcolor{currentfill}%
\pgfsetlinewidth{0.501875pt}%
\definecolor{currentstroke}{rgb}{0.000000,0.000000,0.000000}%
\pgfsetstrokecolor{currentstroke}%
\pgfsetdash{}{0pt}%
\pgfsys@defobject{currentmarker}{\pgfqpoint{0.000000in}{-0.020833in}}{\pgfqpoint{0.000000in}{0.000000in}}{%
\pgfpathmoveto{\pgfqpoint{0.000000in}{0.000000in}}%
\pgfpathlineto{\pgfqpoint{0.000000in}{-0.020833in}}%
\pgfusepath{stroke,fill}%
}%
\begin{pgfscope}%
\pgfsys@transformshift{3.991991in}{4.374193in}%
\pgfsys@useobject{currentmarker}{}%
\end{pgfscope}%
\end{pgfscope}%
\begin{pgfscope}%
\pgfsetbuttcap%
\pgfsetroundjoin%
\definecolor{currentfill}{rgb}{0.000000,0.000000,0.000000}%
\pgfsetfillcolor{currentfill}%
\pgfsetlinewidth{0.501875pt}%
\definecolor{currentstroke}{rgb}{0.000000,0.000000,0.000000}%
\pgfsetstrokecolor{currentstroke}%
\pgfsetdash{}{0pt}%
\pgfsys@defobject{currentmarker}{\pgfqpoint{0.000000in}{0.000000in}}{\pgfqpoint{0.000000in}{0.020833in}}{%
\pgfpathmoveto{\pgfqpoint{0.000000in}{0.000000in}}%
\pgfpathlineto{\pgfqpoint{0.000000in}{0.020833in}}%
\pgfusepath{stroke,fill}%
}%
\begin{pgfscope}%
\pgfsys@transformshift{4.104668in}{0.422992in}%
\pgfsys@useobject{currentmarker}{}%
\end{pgfscope}%
\end{pgfscope}%
\begin{pgfscope}%
\pgfsetbuttcap%
\pgfsetroundjoin%
\definecolor{currentfill}{rgb}{0.000000,0.000000,0.000000}%
\pgfsetfillcolor{currentfill}%
\pgfsetlinewidth{0.501875pt}%
\definecolor{currentstroke}{rgb}{0.000000,0.000000,0.000000}%
\pgfsetstrokecolor{currentstroke}%
\pgfsetdash{}{0pt}%
\pgfsys@defobject{currentmarker}{\pgfqpoint{0.000000in}{-0.020833in}}{\pgfqpoint{0.000000in}{0.000000in}}{%
\pgfpathmoveto{\pgfqpoint{0.000000in}{0.000000in}}%
\pgfpathlineto{\pgfqpoint{0.000000in}{-0.020833in}}%
\pgfusepath{stroke,fill}%
}%
\begin{pgfscope}%
\pgfsys@transformshift{4.104668in}{4.374193in}%
\pgfsys@useobject{currentmarker}{}%
\end{pgfscope}%
\end{pgfscope}%
\begin{pgfscope}%
\pgfsetbuttcap%
\pgfsetroundjoin%
\definecolor{currentfill}{rgb}{0.000000,0.000000,0.000000}%
\pgfsetfillcolor{currentfill}%
\pgfsetlinewidth{0.501875pt}%
\definecolor{currentstroke}{rgb}{0.000000,0.000000,0.000000}%
\pgfsetstrokecolor{currentstroke}%
\pgfsetdash{}{0pt}%
\pgfsys@defobject{currentmarker}{\pgfqpoint{0.000000in}{0.000000in}}{\pgfqpoint{0.000000in}{0.020833in}}{%
\pgfpathmoveto{\pgfqpoint{0.000000in}{0.000000in}}%
\pgfpathlineto{\pgfqpoint{0.000000in}{0.020833in}}%
\pgfusepath{stroke,fill}%
}%
\begin{pgfscope}%
\pgfsys@transformshift{4.330021in}{0.422992in}%
\pgfsys@useobject{currentmarker}{}%
\end{pgfscope}%
\end{pgfscope}%
\begin{pgfscope}%
\pgfsetbuttcap%
\pgfsetroundjoin%
\definecolor{currentfill}{rgb}{0.000000,0.000000,0.000000}%
\pgfsetfillcolor{currentfill}%
\pgfsetlinewidth{0.501875pt}%
\definecolor{currentstroke}{rgb}{0.000000,0.000000,0.000000}%
\pgfsetstrokecolor{currentstroke}%
\pgfsetdash{}{0pt}%
\pgfsys@defobject{currentmarker}{\pgfqpoint{0.000000in}{-0.020833in}}{\pgfqpoint{0.000000in}{0.000000in}}{%
\pgfpathmoveto{\pgfqpoint{0.000000in}{0.000000in}}%
\pgfpathlineto{\pgfqpoint{0.000000in}{-0.020833in}}%
\pgfusepath{stroke,fill}%
}%
\begin{pgfscope}%
\pgfsys@transformshift{4.330021in}{4.374193in}%
\pgfsys@useobject{currentmarker}{}%
\end{pgfscope}%
\end{pgfscope}%
\begin{pgfscope}%
\pgfsetbuttcap%
\pgfsetroundjoin%
\definecolor{currentfill}{rgb}{0.000000,0.000000,0.000000}%
\pgfsetfillcolor{currentfill}%
\pgfsetlinewidth{0.501875pt}%
\definecolor{currentstroke}{rgb}{0.000000,0.000000,0.000000}%
\pgfsetstrokecolor{currentstroke}%
\pgfsetdash{}{0pt}%
\pgfsys@defobject{currentmarker}{\pgfqpoint{0.000000in}{0.000000in}}{\pgfqpoint{0.000000in}{0.020833in}}{%
\pgfpathmoveto{\pgfqpoint{0.000000in}{0.000000in}}%
\pgfpathlineto{\pgfqpoint{0.000000in}{0.020833in}}%
\pgfusepath{stroke,fill}%
}%
\begin{pgfscope}%
\pgfsys@transformshift{4.442697in}{0.422992in}%
\pgfsys@useobject{currentmarker}{}%
\end{pgfscope}%
\end{pgfscope}%
\begin{pgfscope}%
\pgfsetbuttcap%
\pgfsetroundjoin%
\definecolor{currentfill}{rgb}{0.000000,0.000000,0.000000}%
\pgfsetfillcolor{currentfill}%
\pgfsetlinewidth{0.501875pt}%
\definecolor{currentstroke}{rgb}{0.000000,0.000000,0.000000}%
\pgfsetstrokecolor{currentstroke}%
\pgfsetdash{}{0pt}%
\pgfsys@defobject{currentmarker}{\pgfqpoint{0.000000in}{-0.020833in}}{\pgfqpoint{0.000000in}{0.000000in}}{%
\pgfpathmoveto{\pgfqpoint{0.000000in}{0.000000in}}%
\pgfpathlineto{\pgfqpoint{0.000000in}{-0.020833in}}%
\pgfusepath{stroke,fill}%
}%
\begin{pgfscope}%
\pgfsys@transformshift{4.442697in}{4.374193in}%
\pgfsys@useobject{currentmarker}{}%
\end{pgfscope}%
\end{pgfscope}%
\begin{pgfscope}%
\pgfsetbuttcap%
\pgfsetroundjoin%
\definecolor{currentfill}{rgb}{0.000000,0.000000,0.000000}%
\pgfsetfillcolor{currentfill}%
\pgfsetlinewidth{0.501875pt}%
\definecolor{currentstroke}{rgb}{0.000000,0.000000,0.000000}%
\pgfsetstrokecolor{currentstroke}%
\pgfsetdash{}{0pt}%
\pgfsys@defobject{currentmarker}{\pgfqpoint{0.000000in}{0.000000in}}{\pgfqpoint{0.000000in}{0.020833in}}{%
\pgfpathmoveto{\pgfqpoint{0.000000in}{0.000000in}}%
\pgfpathlineto{\pgfqpoint{0.000000in}{0.020833in}}%
\pgfusepath{stroke,fill}%
}%
\begin{pgfscope}%
\pgfsys@transformshift{4.555374in}{0.422992in}%
\pgfsys@useobject{currentmarker}{}%
\end{pgfscope}%
\end{pgfscope}%
\begin{pgfscope}%
\pgfsetbuttcap%
\pgfsetroundjoin%
\definecolor{currentfill}{rgb}{0.000000,0.000000,0.000000}%
\pgfsetfillcolor{currentfill}%
\pgfsetlinewidth{0.501875pt}%
\definecolor{currentstroke}{rgb}{0.000000,0.000000,0.000000}%
\pgfsetstrokecolor{currentstroke}%
\pgfsetdash{}{0pt}%
\pgfsys@defobject{currentmarker}{\pgfqpoint{0.000000in}{-0.020833in}}{\pgfqpoint{0.000000in}{0.000000in}}{%
\pgfpathmoveto{\pgfqpoint{0.000000in}{0.000000in}}%
\pgfpathlineto{\pgfqpoint{0.000000in}{-0.020833in}}%
\pgfusepath{stroke,fill}%
}%
\begin{pgfscope}%
\pgfsys@transformshift{4.555374in}{4.374193in}%
\pgfsys@useobject{currentmarker}{}%
\end{pgfscope}%
\end{pgfscope}%
\begin{pgfscope}%
\pgfsetbuttcap%
\pgfsetroundjoin%
\definecolor{currentfill}{rgb}{0.000000,0.000000,0.000000}%
\pgfsetfillcolor{currentfill}%
\pgfsetlinewidth{0.501875pt}%
\definecolor{currentstroke}{rgb}{0.000000,0.000000,0.000000}%
\pgfsetstrokecolor{currentstroke}%
\pgfsetdash{}{0pt}%
\pgfsys@defobject{currentmarker}{\pgfqpoint{0.000000in}{0.000000in}}{\pgfqpoint{0.000000in}{0.020833in}}{%
\pgfpathmoveto{\pgfqpoint{0.000000in}{0.000000in}}%
\pgfpathlineto{\pgfqpoint{0.000000in}{0.020833in}}%
\pgfusepath{stroke,fill}%
}%
\begin{pgfscope}%
\pgfsys@transformshift{4.780726in}{0.422992in}%
\pgfsys@useobject{currentmarker}{}%
\end{pgfscope}%
\end{pgfscope}%
\begin{pgfscope}%
\pgfsetbuttcap%
\pgfsetroundjoin%
\definecolor{currentfill}{rgb}{0.000000,0.000000,0.000000}%
\pgfsetfillcolor{currentfill}%
\pgfsetlinewidth{0.501875pt}%
\definecolor{currentstroke}{rgb}{0.000000,0.000000,0.000000}%
\pgfsetstrokecolor{currentstroke}%
\pgfsetdash{}{0pt}%
\pgfsys@defobject{currentmarker}{\pgfqpoint{0.000000in}{-0.020833in}}{\pgfqpoint{0.000000in}{0.000000in}}{%
\pgfpathmoveto{\pgfqpoint{0.000000in}{0.000000in}}%
\pgfpathlineto{\pgfqpoint{0.000000in}{-0.020833in}}%
\pgfusepath{stroke,fill}%
}%
\begin{pgfscope}%
\pgfsys@transformshift{4.780726in}{4.374193in}%
\pgfsys@useobject{currentmarker}{}%
\end{pgfscope}%
\end{pgfscope}%
\begin{pgfscope}%
\pgfsetbuttcap%
\pgfsetroundjoin%
\definecolor{currentfill}{rgb}{0.000000,0.000000,0.000000}%
\pgfsetfillcolor{currentfill}%
\pgfsetlinewidth{0.501875pt}%
\definecolor{currentstroke}{rgb}{0.000000,0.000000,0.000000}%
\pgfsetstrokecolor{currentstroke}%
\pgfsetdash{}{0pt}%
\pgfsys@defobject{currentmarker}{\pgfqpoint{0.000000in}{0.000000in}}{\pgfqpoint{0.000000in}{0.020833in}}{%
\pgfpathmoveto{\pgfqpoint{0.000000in}{0.000000in}}%
\pgfpathlineto{\pgfqpoint{0.000000in}{0.020833in}}%
\pgfusepath{stroke,fill}%
}%
\begin{pgfscope}%
\pgfsys@transformshift{4.893403in}{0.422992in}%
\pgfsys@useobject{currentmarker}{}%
\end{pgfscope}%
\end{pgfscope}%
\begin{pgfscope}%
\pgfsetbuttcap%
\pgfsetroundjoin%
\definecolor{currentfill}{rgb}{0.000000,0.000000,0.000000}%
\pgfsetfillcolor{currentfill}%
\pgfsetlinewidth{0.501875pt}%
\definecolor{currentstroke}{rgb}{0.000000,0.000000,0.000000}%
\pgfsetstrokecolor{currentstroke}%
\pgfsetdash{}{0pt}%
\pgfsys@defobject{currentmarker}{\pgfqpoint{0.000000in}{-0.020833in}}{\pgfqpoint{0.000000in}{0.000000in}}{%
\pgfpathmoveto{\pgfqpoint{0.000000in}{0.000000in}}%
\pgfpathlineto{\pgfqpoint{0.000000in}{-0.020833in}}%
\pgfusepath{stroke,fill}%
}%
\begin{pgfscope}%
\pgfsys@transformshift{4.893403in}{4.374193in}%
\pgfsys@useobject{currentmarker}{}%
\end{pgfscope}%
\end{pgfscope}%
\begin{pgfscope}%
\pgfsetbuttcap%
\pgfsetroundjoin%
\definecolor{currentfill}{rgb}{0.000000,0.000000,0.000000}%
\pgfsetfillcolor{currentfill}%
\pgfsetlinewidth{0.501875pt}%
\definecolor{currentstroke}{rgb}{0.000000,0.000000,0.000000}%
\pgfsetstrokecolor{currentstroke}%
\pgfsetdash{}{0pt}%
\pgfsys@defobject{currentmarker}{\pgfqpoint{0.000000in}{0.000000in}}{\pgfqpoint{0.000000in}{0.020833in}}{%
\pgfpathmoveto{\pgfqpoint{0.000000in}{0.000000in}}%
\pgfpathlineto{\pgfqpoint{0.000000in}{0.020833in}}%
\pgfusepath{stroke,fill}%
}%
\begin{pgfscope}%
\pgfsys@transformshift{5.006079in}{0.422992in}%
\pgfsys@useobject{currentmarker}{}%
\end{pgfscope}%
\end{pgfscope}%
\begin{pgfscope}%
\pgfsetbuttcap%
\pgfsetroundjoin%
\definecolor{currentfill}{rgb}{0.000000,0.000000,0.000000}%
\pgfsetfillcolor{currentfill}%
\pgfsetlinewidth{0.501875pt}%
\definecolor{currentstroke}{rgb}{0.000000,0.000000,0.000000}%
\pgfsetstrokecolor{currentstroke}%
\pgfsetdash{}{0pt}%
\pgfsys@defobject{currentmarker}{\pgfqpoint{0.000000in}{-0.020833in}}{\pgfqpoint{0.000000in}{0.000000in}}{%
\pgfpathmoveto{\pgfqpoint{0.000000in}{0.000000in}}%
\pgfpathlineto{\pgfqpoint{0.000000in}{-0.020833in}}%
\pgfusepath{stroke,fill}%
}%
\begin{pgfscope}%
\pgfsys@transformshift{5.006079in}{4.374193in}%
\pgfsys@useobject{currentmarker}{}%
\end{pgfscope}%
\end{pgfscope}%
\begin{pgfscope}%
\pgfsetbuttcap%
\pgfsetroundjoin%
\definecolor{currentfill}{rgb}{0.000000,0.000000,0.000000}%
\pgfsetfillcolor{currentfill}%
\pgfsetlinewidth{0.501875pt}%
\definecolor{currentstroke}{rgb}{0.000000,0.000000,0.000000}%
\pgfsetstrokecolor{currentstroke}%
\pgfsetdash{}{0pt}%
\pgfsys@defobject{currentmarker}{\pgfqpoint{0.000000in}{0.000000in}}{\pgfqpoint{0.000000in}{0.020833in}}{%
\pgfpathmoveto{\pgfqpoint{0.000000in}{0.000000in}}%
\pgfpathlineto{\pgfqpoint{0.000000in}{0.020833in}}%
\pgfusepath{stroke,fill}%
}%
\begin{pgfscope}%
\pgfsys@transformshift{5.231432in}{0.422992in}%
\pgfsys@useobject{currentmarker}{}%
\end{pgfscope}%
\end{pgfscope}%
\begin{pgfscope}%
\pgfsetbuttcap%
\pgfsetroundjoin%
\definecolor{currentfill}{rgb}{0.000000,0.000000,0.000000}%
\pgfsetfillcolor{currentfill}%
\pgfsetlinewidth{0.501875pt}%
\definecolor{currentstroke}{rgb}{0.000000,0.000000,0.000000}%
\pgfsetstrokecolor{currentstroke}%
\pgfsetdash{}{0pt}%
\pgfsys@defobject{currentmarker}{\pgfqpoint{0.000000in}{-0.020833in}}{\pgfqpoint{0.000000in}{0.000000in}}{%
\pgfpathmoveto{\pgfqpoint{0.000000in}{0.000000in}}%
\pgfpathlineto{\pgfqpoint{0.000000in}{-0.020833in}}%
\pgfusepath{stroke,fill}%
}%
\begin{pgfscope}%
\pgfsys@transformshift{5.231432in}{4.374193in}%
\pgfsys@useobject{currentmarker}{}%
\end{pgfscope}%
\end{pgfscope}%
\begin{pgfscope}%
\pgfsetbuttcap%
\pgfsetroundjoin%
\definecolor{currentfill}{rgb}{0.000000,0.000000,0.000000}%
\pgfsetfillcolor{currentfill}%
\pgfsetlinewidth{0.501875pt}%
\definecolor{currentstroke}{rgb}{0.000000,0.000000,0.000000}%
\pgfsetstrokecolor{currentstroke}%
\pgfsetdash{}{0pt}%
\pgfsys@defobject{currentmarker}{\pgfqpoint{0.000000in}{0.000000in}}{\pgfqpoint{0.000000in}{0.020833in}}{%
\pgfpathmoveto{\pgfqpoint{0.000000in}{0.000000in}}%
\pgfpathlineto{\pgfqpoint{0.000000in}{0.020833in}}%
\pgfusepath{stroke,fill}%
}%
\begin{pgfscope}%
\pgfsys@transformshift{5.344109in}{0.422992in}%
\pgfsys@useobject{currentmarker}{}%
\end{pgfscope}%
\end{pgfscope}%
\begin{pgfscope}%
\pgfsetbuttcap%
\pgfsetroundjoin%
\definecolor{currentfill}{rgb}{0.000000,0.000000,0.000000}%
\pgfsetfillcolor{currentfill}%
\pgfsetlinewidth{0.501875pt}%
\definecolor{currentstroke}{rgb}{0.000000,0.000000,0.000000}%
\pgfsetstrokecolor{currentstroke}%
\pgfsetdash{}{0pt}%
\pgfsys@defobject{currentmarker}{\pgfqpoint{0.000000in}{-0.020833in}}{\pgfqpoint{0.000000in}{0.000000in}}{%
\pgfpathmoveto{\pgfqpoint{0.000000in}{0.000000in}}%
\pgfpathlineto{\pgfqpoint{0.000000in}{-0.020833in}}%
\pgfusepath{stroke,fill}%
}%
\begin{pgfscope}%
\pgfsys@transformshift{5.344109in}{4.374193in}%
\pgfsys@useobject{currentmarker}{}%
\end{pgfscope}%
\end{pgfscope}%
\begin{pgfscope}%
\pgfsetbuttcap%
\pgfsetroundjoin%
\definecolor{currentfill}{rgb}{0.000000,0.000000,0.000000}%
\pgfsetfillcolor{currentfill}%
\pgfsetlinewidth{0.501875pt}%
\definecolor{currentstroke}{rgb}{0.000000,0.000000,0.000000}%
\pgfsetstrokecolor{currentstroke}%
\pgfsetdash{}{0pt}%
\pgfsys@defobject{currentmarker}{\pgfqpoint{0.000000in}{0.000000in}}{\pgfqpoint{0.000000in}{0.020833in}}{%
\pgfpathmoveto{\pgfqpoint{0.000000in}{0.000000in}}%
\pgfpathlineto{\pgfqpoint{0.000000in}{0.020833in}}%
\pgfusepath{stroke,fill}%
}%
\begin{pgfscope}%
\pgfsys@transformshift{5.456785in}{0.422992in}%
\pgfsys@useobject{currentmarker}{}%
\end{pgfscope}%
\end{pgfscope}%
\begin{pgfscope}%
\pgfsetbuttcap%
\pgfsetroundjoin%
\definecolor{currentfill}{rgb}{0.000000,0.000000,0.000000}%
\pgfsetfillcolor{currentfill}%
\pgfsetlinewidth{0.501875pt}%
\definecolor{currentstroke}{rgb}{0.000000,0.000000,0.000000}%
\pgfsetstrokecolor{currentstroke}%
\pgfsetdash{}{0pt}%
\pgfsys@defobject{currentmarker}{\pgfqpoint{0.000000in}{-0.020833in}}{\pgfqpoint{0.000000in}{0.000000in}}{%
\pgfpathmoveto{\pgfqpoint{0.000000in}{0.000000in}}%
\pgfpathlineto{\pgfqpoint{0.000000in}{-0.020833in}}%
\pgfusepath{stroke,fill}%
}%
\begin{pgfscope}%
\pgfsys@transformshift{5.456785in}{4.374193in}%
\pgfsys@useobject{currentmarker}{}%
\end{pgfscope}%
\end{pgfscope}%
\begin{pgfscope}%
\pgfsetbuttcap%
\pgfsetroundjoin%
\definecolor{currentfill}{rgb}{0.000000,0.000000,0.000000}%
\pgfsetfillcolor{currentfill}%
\pgfsetlinewidth{0.501875pt}%
\definecolor{currentstroke}{rgb}{0.000000,0.000000,0.000000}%
\pgfsetstrokecolor{currentstroke}%
\pgfsetdash{}{0pt}%
\pgfsys@defobject{currentmarker}{\pgfqpoint{0.000000in}{0.000000in}}{\pgfqpoint{0.000000in}{0.020833in}}{%
\pgfpathmoveto{\pgfqpoint{0.000000in}{0.000000in}}%
\pgfpathlineto{\pgfqpoint{0.000000in}{0.020833in}}%
\pgfusepath{stroke,fill}%
}%
\begin{pgfscope}%
\pgfsys@transformshift{5.569461in}{0.422992in}%
\pgfsys@useobject{currentmarker}{}%
\end{pgfscope}%
\end{pgfscope}%
\begin{pgfscope}%
\pgfsetbuttcap%
\pgfsetroundjoin%
\definecolor{currentfill}{rgb}{0.000000,0.000000,0.000000}%
\pgfsetfillcolor{currentfill}%
\pgfsetlinewidth{0.501875pt}%
\definecolor{currentstroke}{rgb}{0.000000,0.000000,0.000000}%
\pgfsetstrokecolor{currentstroke}%
\pgfsetdash{}{0pt}%
\pgfsys@defobject{currentmarker}{\pgfqpoint{0.000000in}{-0.020833in}}{\pgfqpoint{0.000000in}{0.000000in}}{%
\pgfpathmoveto{\pgfqpoint{0.000000in}{0.000000in}}%
\pgfpathlineto{\pgfqpoint{0.000000in}{-0.020833in}}%
\pgfusepath{stroke,fill}%
}%
\begin{pgfscope}%
\pgfsys@transformshift{5.569461in}{4.374193in}%
\pgfsys@useobject{currentmarker}{}%
\end{pgfscope}%
\end{pgfscope}%
\begin{pgfscope}%
\definecolor{textcolor}{rgb}{0.000000,0.000000,0.000000}%
\pgfsetstrokecolor{textcolor}%
\pgfsetfillcolor{textcolor}%
\pgftext[x=4.453965in,y=0.184413in,,top]{\color{textcolor}\rmfamily\fontsize{10.000000}{12.000000}\selectfont \(\displaystyle K\)}%
\end{pgfscope}%
\begin{pgfscope}%
\pgfsetbuttcap%
\pgfsetroundjoin%
\definecolor{currentfill}{rgb}{0.000000,0.000000,0.000000}%
\pgfsetfillcolor{currentfill}%
\pgfsetlinewidth{0.501875pt}%
\definecolor{currentstroke}{rgb}{0.000000,0.000000,0.000000}%
\pgfsetstrokecolor{currentstroke}%
\pgfsetdash{}{0pt}%
\pgfsys@defobject{currentmarker}{\pgfqpoint{0.000000in}{0.000000in}}{\pgfqpoint{0.041667in}{0.000000in}}{%
\pgfpathmoveto{\pgfqpoint{0.000000in}{0.000000in}}%
\pgfpathlineto{\pgfqpoint{0.041667in}{0.000000in}}%
\pgfusepath{stroke,fill}%
}%
\begin{pgfscope}%
\pgfsys@transformshift{3.315933in}{0.922738in}%
\pgfsys@useobject{currentmarker}{}%
\end{pgfscope}%
\end{pgfscope}%
\begin{pgfscope}%
\pgfsetbuttcap%
\pgfsetroundjoin%
\definecolor{currentfill}{rgb}{0.000000,0.000000,0.000000}%
\pgfsetfillcolor{currentfill}%
\pgfsetlinewidth{0.501875pt}%
\definecolor{currentstroke}{rgb}{0.000000,0.000000,0.000000}%
\pgfsetstrokecolor{currentstroke}%
\pgfsetdash{}{0pt}%
\pgfsys@defobject{currentmarker}{\pgfqpoint{-0.041667in}{0.000000in}}{\pgfqpoint{-0.000000in}{0.000000in}}{%
\pgfpathmoveto{\pgfqpoint{-0.000000in}{0.000000in}}%
\pgfpathlineto{\pgfqpoint{-0.041667in}{0.000000in}}%
\pgfusepath{stroke,fill}%
}%
\begin{pgfscope}%
\pgfsys@transformshift{5.591997in}{0.922738in}%
\pgfsys@useobject{currentmarker}{}%
\end{pgfscope}%
\end{pgfscope}%
\begin{pgfscope}%
\definecolor{textcolor}{rgb}{0.000000,0.000000,0.000000}%
\pgfsetstrokecolor{textcolor}%
\pgfsetfillcolor{textcolor}%
\pgftext[x=3.089852in, y=0.869977in, left, base]{\color{textcolor}\rmfamily\fontsize{10.000000}{12.000000}\selectfont \(\displaystyle {0.1}\)}%
\end{pgfscope}%
\begin{pgfscope}%
\pgfsetbuttcap%
\pgfsetroundjoin%
\definecolor{currentfill}{rgb}{0.000000,0.000000,0.000000}%
\pgfsetfillcolor{currentfill}%
\pgfsetlinewidth{0.501875pt}%
\definecolor{currentstroke}{rgb}{0.000000,0.000000,0.000000}%
\pgfsetstrokecolor{currentstroke}%
\pgfsetdash{}{0pt}%
\pgfsys@defobject{currentmarker}{\pgfqpoint{0.000000in}{0.000000in}}{\pgfqpoint{0.041667in}{0.000000in}}{%
\pgfpathmoveto{\pgfqpoint{0.000000in}{0.000000in}}%
\pgfpathlineto{\pgfqpoint{0.041667in}{0.000000in}}%
\pgfusepath{stroke,fill}%
}%
\begin{pgfscope}%
\pgfsys@transformshift{3.315933in}{1.479498in}%
\pgfsys@useobject{currentmarker}{}%
\end{pgfscope}%
\end{pgfscope}%
\begin{pgfscope}%
\pgfsetbuttcap%
\pgfsetroundjoin%
\definecolor{currentfill}{rgb}{0.000000,0.000000,0.000000}%
\pgfsetfillcolor{currentfill}%
\pgfsetlinewidth{0.501875pt}%
\definecolor{currentstroke}{rgb}{0.000000,0.000000,0.000000}%
\pgfsetstrokecolor{currentstroke}%
\pgfsetdash{}{0pt}%
\pgfsys@defobject{currentmarker}{\pgfqpoint{-0.041667in}{0.000000in}}{\pgfqpoint{-0.000000in}{0.000000in}}{%
\pgfpathmoveto{\pgfqpoint{-0.000000in}{0.000000in}}%
\pgfpathlineto{\pgfqpoint{-0.041667in}{0.000000in}}%
\pgfusepath{stroke,fill}%
}%
\begin{pgfscope}%
\pgfsys@transformshift{5.591997in}{1.479498in}%
\pgfsys@useobject{currentmarker}{}%
\end{pgfscope}%
\end{pgfscope}%
\begin{pgfscope}%
\definecolor{textcolor}{rgb}{0.000000,0.000000,0.000000}%
\pgfsetstrokecolor{textcolor}%
\pgfsetfillcolor{textcolor}%
\pgftext[x=3.089852in, y=1.426737in, left, base]{\color{textcolor}\rmfamily\fontsize{10.000000}{12.000000}\selectfont \(\displaystyle {0.2}\)}%
\end{pgfscope}%
\begin{pgfscope}%
\pgfsetbuttcap%
\pgfsetroundjoin%
\definecolor{currentfill}{rgb}{0.000000,0.000000,0.000000}%
\pgfsetfillcolor{currentfill}%
\pgfsetlinewidth{0.501875pt}%
\definecolor{currentstroke}{rgb}{0.000000,0.000000,0.000000}%
\pgfsetstrokecolor{currentstroke}%
\pgfsetdash{}{0pt}%
\pgfsys@defobject{currentmarker}{\pgfqpoint{0.000000in}{0.000000in}}{\pgfqpoint{0.041667in}{0.000000in}}{%
\pgfpathmoveto{\pgfqpoint{0.000000in}{0.000000in}}%
\pgfpathlineto{\pgfqpoint{0.041667in}{0.000000in}}%
\pgfusepath{stroke,fill}%
}%
\begin{pgfscope}%
\pgfsys@transformshift{3.315933in}{2.036258in}%
\pgfsys@useobject{currentmarker}{}%
\end{pgfscope}%
\end{pgfscope}%
\begin{pgfscope}%
\pgfsetbuttcap%
\pgfsetroundjoin%
\definecolor{currentfill}{rgb}{0.000000,0.000000,0.000000}%
\pgfsetfillcolor{currentfill}%
\pgfsetlinewidth{0.501875pt}%
\definecolor{currentstroke}{rgb}{0.000000,0.000000,0.000000}%
\pgfsetstrokecolor{currentstroke}%
\pgfsetdash{}{0pt}%
\pgfsys@defobject{currentmarker}{\pgfqpoint{-0.041667in}{0.000000in}}{\pgfqpoint{-0.000000in}{0.000000in}}{%
\pgfpathmoveto{\pgfqpoint{-0.000000in}{0.000000in}}%
\pgfpathlineto{\pgfqpoint{-0.041667in}{0.000000in}}%
\pgfusepath{stroke,fill}%
}%
\begin{pgfscope}%
\pgfsys@transformshift{5.591997in}{2.036258in}%
\pgfsys@useobject{currentmarker}{}%
\end{pgfscope}%
\end{pgfscope}%
\begin{pgfscope}%
\definecolor{textcolor}{rgb}{0.000000,0.000000,0.000000}%
\pgfsetstrokecolor{textcolor}%
\pgfsetfillcolor{textcolor}%
\pgftext[x=3.089852in, y=1.983497in, left, base]{\color{textcolor}\rmfamily\fontsize{10.000000}{12.000000}\selectfont \(\displaystyle {0.3}\)}%
\end{pgfscope}%
\begin{pgfscope}%
\pgfsetbuttcap%
\pgfsetroundjoin%
\definecolor{currentfill}{rgb}{0.000000,0.000000,0.000000}%
\pgfsetfillcolor{currentfill}%
\pgfsetlinewidth{0.501875pt}%
\definecolor{currentstroke}{rgb}{0.000000,0.000000,0.000000}%
\pgfsetstrokecolor{currentstroke}%
\pgfsetdash{}{0pt}%
\pgfsys@defobject{currentmarker}{\pgfqpoint{0.000000in}{0.000000in}}{\pgfqpoint{0.041667in}{0.000000in}}{%
\pgfpathmoveto{\pgfqpoint{0.000000in}{0.000000in}}%
\pgfpathlineto{\pgfqpoint{0.041667in}{0.000000in}}%
\pgfusepath{stroke,fill}%
}%
\begin{pgfscope}%
\pgfsys@transformshift{3.315933in}{2.593018in}%
\pgfsys@useobject{currentmarker}{}%
\end{pgfscope}%
\end{pgfscope}%
\begin{pgfscope}%
\pgfsetbuttcap%
\pgfsetroundjoin%
\definecolor{currentfill}{rgb}{0.000000,0.000000,0.000000}%
\pgfsetfillcolor{currentfill}%
\pgfsetlinewidth{0.501875pt}%
\definecolor{currentstroke}{rgb}{0.000000,0.000000,0.000000}%
\pgfsetstrokecolor{currentstroke}%
\pgfsetdash{}{0pt}%
\pgfsys@defobject{currentmarker}{\pgfqpoint{-0.041667in}{0.000000in}}{\pgfqpoint{-0.000000in}{0.000000in}}{%
\pgfpathmoveto{\pgfqpoint{-0.000000in}{0.000000in}}%
\pgfpathlineto{\pgfqpoint{-0.041667in}{0.000000in}}%
\pgfusepath{stroke,fill}%
}%
\begin{pgfscope}%
\pgfsys@transformshift{5.591997in}{2.593018in}%
\pgfsys@useobject{currentmarker}{}%
\end{pgfscope}%
\end{pgfscope}%
\begin{pgfscope}%
\definecolor{textcolor}{rgb}{0.000000,0.000000,0.000000}%
\pgfsetstrokecolor{textcolor}%
\pgfsetfillcolor{textcolor}%
\pgftext[x=3.089852in, y=2.540257in, left, base]{\color{textcolor}\rmfamily\fontsize{10.000000}{12.000000}\selectfont \(\displaystyle {0.4}\)}%
\end{pgfscope}%
\begin{pgfscope}%
\pgfsetbuttcap%
\pgfsetroundjoin%
\definecolor{currentfill}{rgb}{0.000000,0.000000,0.000000}%
\pgfsetfillcolor{currentfill}%
\pgfsetlinewidth{0.501875pt}%
\definecolor{currentstroke}{rgb}{0.000000,0.000000,0.000000}%
\pgfsetstrokecolor{currentstroke}%
\pgfsetdash{}{0pt}%
\pgfsys@defobject{currentmarker}{\pgfqpoint{0.000000in}{0.000000in}}{\pgfqpoint{0.041667in}{0.000000in}}{%
\pgfpathmoveto{\pgfqpoint{0.000000in}{0.000000in}}%
\pgfpathlineto{\pgfqpoint{0.041667in}{0.000000in}}%
\pgfusepath{stroke,fill}%
}%
\begin{pgfscope}%
\pgfsys@transformshift{3.315933in}{3.149778in}%
\pgfsys@useobject{currentmarker}{}%
\end{pgfscope}%
\end{pgfscope}%
\begin{pgfscope}%
\pgfsetbuttcap%
\pgfsetroundjoin%
\definecolor{currentfill}{rgb}{0.000000,0.000000,0.000000}%
\pgfsetfillcolor{currentfill}%
\pgfsetlinewidth{0.501875pt}%
\definecolor{currentstroke}{rgb}{0.000000,0.000000,0.000000}%
\pgfsetstrokecolor{currentstroke}%
\pgfsetdash{}{0pt}%
\pgfsys@defobject{currentmarker}{\pgfqpoint{-0.041667in}{0.000000in}}{\pgfqpoint{-0.000000in}{0.000000in}}{%
\pgfpathmoveto{\pgfqpoint{-0.000000in}{0.000000in}}%
\pgfpathlineto{\pgfqpoint{-0.041667in}{0.000000in}}%
\pgfusepath{stroke,fill}%
}%
\begin{pgfscope}%
\pgfsys@transformshift{5.591997in}{3.149778in}%
\pgfsys@useobject{currentmarker}{}%
\end{pgfscope}%
\end{pgfscope}%
\begin{pgfscope}%
\definecolor{textcolor}{rgb}{0.000000,0.000000,0.000000}%
\pgfsetstrokecolor{textcolor}%
\pgfsetfillcolor{textcolor}%
\pgftext[x=3.089852in, y=3.097017in, left, base]{\color{textcolor}\rmfamily\fontsize{10.000000}{12.000000}\selectfont \(\displaystyle {0.5}\)}%
\end{pgfscope}%
\begin{pgfscope}%
\pgfsetbuttcap%
\pgfsetroundjoin%
\definecolor{currentfill}{rgb}{0.000000,0.000000,0.000000}%
\pgfsetfillcolor{currentfill}%
\pgfsetlinewidth{0.501875pt}%
\definecolor{currentstroke}{rgb}{0.000000,0.000000,0.000000}%
\pgfsetstrokecolor{currentstroke}%
\pgfsetdash{}{0pt}%
\pgfsys@defobject{currentmarker}{\pgfqpoint{0.000000in}{0.000000in}}{\pgfqpoint{0.041667in}{0.000000in}}{%
\pgfpathmoveto{\pgfqpoint{0.000000in}{0.000000in}}%
\pgfpathlineto{\pgfqpoint{0.041667in}{0.000000in}}%
\pgfusepath{stroke,fill}%
}%
\begin{pgfscope}%
\pgfsys@transformshift{3.315933in}{3.706539in}%
\pgfsys@useobject{currentmarker}{}%
\end{pgfscope}%
\end{pgfscope}%
\begin{pgfscope}%
\pgfsetbuttcap%
\pgfsetroundjoin%
\definecolor{currentfill}{rgb}{0.000000,0.000000,0.000000}%
\pgfsetfillcolor{currentfill}%
\pgfsetlinewidth{0.501875pt}%
\definecolor{currentstroke}{rgb}{0.000000,0.000000,0.000000}%
\pgfsetstrokecolor{currentstroke}%
\pgfsetdash{}{0pt}%
\pgfsys@defobject{currentmarker}{\pgfqpoint{-0.041667in}{0.000000in}}{\pgfqpoint{-0.000000in}{0.000000in}}{%
\pgfpathmoveto{\pgfqpoint{-0.000000in}{0.000000in}}%
\pgfpathlineto{\pgfqpoint{-0.041667in}{0.000000in}}%
\pgfusepath{stroke,fill}%
}%
\begin{pgfscope}%
\pgfsys@transformshift{5.591997in}{3.706539in}%
\pgfsys@useobject{currentmarker}{}%
\end{pgfscope}%
\end{pgfscope}%
\begin{pgfscope}%
\definecolor{textcolor}{rgb}{0.000000,0.000000,0.000000}%
\pgfsetstrokecolor{textcolor}%
\pgfsetfillcolor{textcolor}%
\pgftext[x=3.089852in, y=3.653777in, left, base]{\color{textcolor}\rmfamily\fontsize{10.000000}{12.000000}\selectfont \(\displaystyle {0.6}\)}%
\end{pgfscope}%
\begin{pgfscope}%
\pgfsetbuttcap%
\pgfsetroundjoin%
\definecolor{currentfill}{rgb}{0.000000,0.000000,0.000000}%
\pgfsetfillcolor{currentfill}%
\pgfsetlinewidth{0.501875pt}%
\definecolor{currentstroke}{rgb}{0.000000,0.000000,0.000000}%
\pgfsetstrokecolor{currentstroke}%
\pgfsetdash{}{0pt}%
\pgfsys@defobject{currentmarker}{\pgfqpoint{0.000000in}{0.000000in}}{\pgfqpoint{0.041667in}{0.000000in}}{%
\pgfpathmoveto{\pgfqpoint{0.000000in}{0.000000in}}%
\pgfpathlineto{\pgfqpoint{0.041667in}{0.000000in}}%
\pgfusepath{stroke,fill}%
}%
\begin{pgfscope}%
\pgfsys@transformshift{3.315933in}{4.263299in}%
\pgfsys@useobject{currentmarker}{}%
\end{pgfscope}%
\end{pgfscope}%
\begin{pgfscope}%
\pgfsetbuttcap%
\pgfsetroundjoin%
\definecolor{currentfill}{rgb}{0.000000,0.000000,0.000000}%
\pgfsetfillcolor{currentfill}%
\pgfsetlinewidth{0.501875pt}%
\definecolor{currentstroke}{rgb}{0.000000,0.000000,0.000000}%
\pgfsetstrokecolor{currentstroke}%
\pgfsetdash{}{0pt}%
\pgfsys@defobject{currentmarker}{\pgfqpoint{-0.041667in}{0.000000in}}{\pgfqpoint{-0.000000in}{0.000000in}}{%
\pgfpathmoveto{\pgfqpoint{-0.000000in}{0.000000in}}%
\pgfpathlineto{\pgfqpoint{-0.041667in}{0.000000in}}%
\pgfusepath{stroke,fill}%
}%
\begin{pgfscope}%
\pgfsys@transformshift{5.591997in}{4.263299in}%
\pgfsys@useobject{currentmarker}{}%
\end{pgfscope}%
\end{pgfscope}%
\begin{pgfscope}%
\definecolor{textcolor}{rgb}{0.000000,0.000000,0.000000}%
\pgfsetstrokecolor{textcolor}%
\pgfsetfillcolor{textcolor}%
\pgftext[x=3.089852in, y=4.210537in, left, base]{\color{textcolor}\rmfamily\fontsize{10.000000}{12.000000}\selectfont \(\displaystyle {0.7}\)}%
\end{pgfscope}%
\begin{pgfscope}%
\pgfsetbuttcap%
\pgfsetroundjoin%
\definecolor{currentfill}{rgb}{0.000000,0.000000,0.000000}%
\pgfsetfillcolor{currentfill}%
\pgfsetlinewidth{0.501875pt}%
\definecolor{currentstroke}{rgb}{0.000000,0.000000,0.000000}%
\pgfsetstrokecolor{currentstroke}%
\pgfsetdash{}{0pt}%
\pgfsys@defobject{currentmarker}{\pgfqpoint{0.000000in}{0.000000in}}{\pgfqpoint{0.020833in}{0.000000in}}{%
\pgfpathmoveto{\pgfqpoint{0.000000in}{0.000000in}}%
\pgfpathlineto{\pgfqpoint{0.020833in}{0.000000in}}%
\pgfusepath{stroke,fill}%
}%
\begin{pgfscope}%
\pgfsys@transformshift{3.315933in}{0.477330in}%
\pgfsys@useobject{currentmarker}{}%
\end{pgfscope}%
\end{pgfscope}%
\begin{pgfscope}%
\pgfsetbuttcap%
\pgfsetroundjoin%
\definecolor{currentfill}{rgb}{0.000000,0.000000,0.000000}%
\pgfsetfillcolor{currentfill}%
\pgfsetlinewidth{0.501875pt}%
\definecolor{currentstroke}{rgb}{0.000000,0.000000,0.000000}%
\pgfsetstrokecolor{currentstroke}%
\pgfsetdash{}{0pt}%
\pgfsys@defobject{currentmarker}{\pgfqpoint{-0.020833in}{0.000000in}}{\pgfqpoint{-0.000000in}{0.000000in}}{%
\pgfpathmoveto{\pgfqpoint{-0.000000in}{0.000000in}}%
\pgfpathlineto{\pgfqpoint{-0.020833in}{0.000000in}}%
\pgfusepath{stroke,fill}%
}%
\begin{pgfscope}%
\pgfsys@transformshift{5.591997in}{0.477330in}%
\pgfsys@useobject{currentmarker}{}%
\end{pgfscope}%
\end{pgfscope}%
\begin{pgfscope}%
\pgfsetbuttcap%
\pgfsetroundjoin%
\definecolor{currentfill}{rgb}{0.000000,0.000000,0.000000}%
\pgfsetfillcolor{currentfill}%
\pgfsetlinewidth{0.501875pt}%
\definecolor{currentstroke}{rgb}{0.000000,0.000000,0.000000}%
\pgfsetstrokecolor{currentstroke}%
\pgfsetdash{}{0pt}%
\pgfsys@defobject{currentmarker}{\pgfqpoint{0.000000in}{0.000000in}}{\pgfqpoint{0.020833in}{0.000000in}}{%
\pgfpathmoveto{\pgfqpoint{0.000000in}{0.000000in}}%
\pgfpathlineto{\pgfqpoint{0.020833in}{0.000000in}}%
\pgfusepath{stroke,fill}%
}%
\begin{pgfscope}%
\pgfsys@transformshift{3.315933in}{0.588682in}%
\pgfsys@useobject{currentmarker}{}%
\end{pgfscope}%
\end{pgfscope}%
\begin{pgfscope}%
\pgfsetbuttcap%
\pgfsetroundjoin%
\definecolor{currentfill}{rgb}{0.000000,0.000000,0.000000}%
\pgfsetfillcolor{currentfill}%
\pgfsetlinewidth{0.501875pt}%
\definecolor{currentstroke}{rgb}{0.000000,0.000000,0.000000}%
\pgfsetstrokecolor{currentstroke}%
\pgfsetdash{}{0pt}%
\pgfsys@defobject{currentmarker}{\pgfqpoint{-0.020833in}{0.000000in}}{\pgfqpoint{-0.000000in}{0.000000in}}{%
\pgfpathmoveto{\pgfqpoint{-0.000000in}{0.000000in}}%
\pgfpathlineto{\pgfqpoint{-0.020833in}{0.000000in}}%
\pgfusepath{stroke,fill}%
}%
\begin{pgfscope}%
\pgfsys@transformshift{5.591997in}{0.588682in}%
\pgfsys@useobject{currentmarker}{}%
\end{pgfscope}%
\end{pgfscope}%
\begin{pgfscope}%
\pgfsetbuttcap%
\pgfsetroundjoin%
\definecolor{currentfill}{rgb}{0.000000,0.000000,0.000000}%
\pgfsetfillcolor{currentfill}%
\pgfsetlinewidth{0.501875pt}%
\definecolor{currentstroke}{rgb}{0.000000,0.000000,0.000000}%
\pgfsetstrokecolor{currentstroke}%
\pgfsetdash{}{0pt}%
\pgfsys@defobject{currentmarker}{\pgfqpoint{0.000000in}{0.000000in}}{\pgfqpoint{0.020833in}{0.000000in}}{%
\pgfpathmoveto{\pgfqpoint{0.000000in}{0.000000in}}%
\pgfpathlineto{\pgfqpoint{0.020833in}{0.000000in}}%
\pgfusepath{stroke,fill}%
}%
\begin{pgfscope}%
\pgfsys@transformshift{3.315933in}{0.700034in}%
\pgfsys@useobject{currentmarker}{}%
\end{pgfscope}%
\end{pgfscope}%
\begin{pgfscope}%
\pgfsetbuttcap%
\pgfsetroundjoin%
\definecolor{currentfill}{rgb}{0.000000,0.000000,0.000000}%
\pgfsetfillcolor{currentfill}%
\pgfsetlinewidth{0.501875pt}%
\definecolor{currentstroke}{rgb}{0.000000,0.000000,0.000000}%
\pgfsetstrokecolor{currentstroke}%
\pgfsetdash{}{0pt}%
\pgfsys@defobject{currentmarker}{\pgfqpoint{-0.020833in}{0.000000in}}{\pgfqpoint{-0.000000in}{0.000000in}}{%
\pgfpathmoveto{\pgfqpoint{-0.000000in}{0.000000in}}%
\pgfpathlineto{\pgfqpoint{-0.020833in}{0.000000in}}%
\pgfusepath{stroke,fill}%
}%
\begin{pgfscope}%
\pgfsys@transformshift{5.591997in}{0.700034in}%
\pgfsys@useobject{currentmarker}{}%
\end{pgfscope}%
\end{pgfscope}%
\begin{pgfscope}%
\pgfsetbuttcap%
\pgfsetroundjoin%
\definecolor{currentfill}{rgb}{0.000000,0.000000,0.000000}%
\pgfsetfillcolor{currentfill}%
\pgfsetlinewidth{0.501875pt}%
\definecolor{currentstroke}{rgb}{0.000000,0.000000,0.000000}%
\pgfsetstrokecolor{currentstroke}%
\pgfsetdash{}{0pt}%
\pgfsys@defobject{currentmarker}{\pgfqpoint{0.000000in}{0.000000in}}{\pgfqpoint{0.020833in}{0.000000in}}{%
\pgfpathmoveto{\pgfqpoint{0.000000in}{0.000000in}}%
\pgfpathlineto{\pgfqpoint{0.020833in}{0.000000in}}%
\pgfusepath{stroke,fill}%
}%
\begin{pgfscope}%
\pgfsys@transformshift{3.315933in}{0.811386in}%
\pgfsys@useobject{currentmarker}{}%
\end{pgfscope}%
\end{pgfscope}%
\begin{pgfscope}%
\pgfsetbuttcap%
\pgfsetroundjoin%
\definecolor{currentfill}{rgb}{0.000000,0.000000,0.000000}%
\pgfsetfillcolor{currentfill}%
\pgfsetlinewidth{0.501875pt}%
\definecolor{currentstroke}{rgb}{0.000000,0.000000,0.000000}%
\pgfsetstrokecolor{currentstroke}%
\pgfsetdash{}{0pt}%
\pgfsys@defobject{currentmarker}{\pgfqpoint{-0.020833in}{0.000000in}}{\pgfqpoint{-0.000000in}{0.000000in}}{%
\pgfpathmoveto{\pgfqpoint{-0.000000in}{0.000000in}}%
\pgfpathlineto{\pgfqpoint{-0.020833in}{0.000000in}}%
\pgfusepath{stroke,fill}%
}%
\begin{pgfscope}%
\pgfsys@transformshift{5.591997in}{0.811386in}%
\pgfsys@useobject{currentmarker}{}%
\end{pgfscope}%
\end{pgfscope}%
\begin{pgfscope}%
\pgfsetbuttcap%
\pgfsetroundjoin%
\definecolor{currentfill}{rgb}{0.000000,0.000000,0.000000}%
\pgfsetfillcolor{currentfill}%
\pgfsetlinewidth{0.501875pt}%
\definecolor{currentstroke}{rgb}{0.000000,0.000000,0.000000}%
\pgfsetstrokecolor{currentstroke}%
\pgfsetdash{}{0pt}%
\pgfsys@defobject{currentmarker}{\pgfqpoint{0.000000in}{0.000000in}}{\pgfqpoint{0.020833in}{0.000000in}}{%
\pgfpathmoveto{\pgfqpoint{0.000000in}{0.000000in}}%
\pgfpathlineto{\pgfqpoint{0.020833in}{0.000000in}}%
\pgfusepath{stroke,fill}%
}%
\begin{pgfscope}%
\pgfsys@transformshift{3.315933in}{1.034090in}%
\pgfsys@useobject{currentmarker}{}%
\end{pgfscope}%
\end{pgfscope}%
\begin{pgfscope}%
\pgfsetbuttcap%
\pgfsetroundjoin%
\definecolor{currentfill}{rgb}{0.000000,0.000000,0.000000}%
\pgfsetfillcolor{currentfill}%
\pgfsetlinewidth{0.501875pt}%
\definecolor{currentstroke}{rgb}{0.000000,0.000000,0.000000}%
\pgfsetstrokecolor{currentstroke}%
\pgfsetdash{}{0pt}%
\pgfsys@defobject{currentmarker}{\pgfqpoint{-0.020833in}{0.000000in}}{\pgfqpoint{-0.000000in}{0.000000in}}{%
\pgfpathmoveto{\pgfqpoint{-0.000000in}{0.000000in}}%
\pgfpathlineto{\pgfqpoint{-0.020833in}{0.000000in}}%
\pgfusepath{stroke,fill}%
}%
\begin{pgfscope}%
\pgfsys@transformshift{5.591997in}{1.034090in}%
\pgfsys@useobject{currentmarker}{}%
\end{pgfscope}%
\end{pgfscope}%
\begin{pgfscope}%
\pgfsetbuttcap%
\pgfsetroundjoin%
\definecolor{currentfill}{rgb}{0.000000,0.000000,0.000000}%
\pgfsetfillcolor{currentfill}%
\pgfsetlinewidth{0.501875pt}%
\definecolor{currentstroke}{rgb}{0.000000,0.000000,0.000000}%
\pgfsetstrokecolor{currentstroke}%
\pgfsetdash{}{0pt}%
\pgfsys@defobject{currentmarker}{\pgfqpoint{0.000000in}{0.000000in}}{\pgfqpoint{0.020833in}{0.000000in}}{%
\pgfpathmoveto{\pgfqpoint{0.000000in}{0.000000in}}%
\pgfpathlineto{\pgfqpoint{0.020833in}{0.000000in}}%
\pgfusepath{stroke,fill}%
}%
\begin{pgfscope}%
\pgfsys@transformshift{3.315933in}{1.145442in}%
\pgfsys@useobject{currentmarker}{}%
\end{pgfscope}%
\end{pgfscope}%
\begin{pgfscope}%
\pgfsetbuttcap%
\pgfsetroundjoin%
\definecolor{currentfill}{rgb}{0.000000,0.000000,0.000000}%
\pgfsetfillcolor{currentfill}%
\pgfsetlinewidth{0.501875pt}%
\definecolor{currentstroke}{rgb}{0.000000,0.000000,0.000000}%
\pgfsetstrokecolor{currentstroke}%
\pgfsetdash{}{0pt}%
\pgfsys@defobject{currentmarker}{\pgfqpoint{-0.020833in}{0.000000in}}{\pgfqpoint{-0.000000in}{0.000000in}}{%
\pgfpathmoveto{\pgfqpoint{-0.000000in}{0.000000in}}%
\pgfpathlineto{\pgfqpoint{-0.020833in}{0.000000in}}%
\pgfusepath{stroke,fill}%
}%
\begin{pgfscope}%
\pgfsys@transformshift{5.591997in}{1.145442in}%
\pgfsys@useobject{currentmarker}{}%
\end{pgfscope}%
\end{pgfscope}%
\begin{pgfscope}%
\pgfsetbuttcap%
\pgfsetroundjoin%
\definecolor{currentfill}{rgb}{0.000000,0.000000,0.000000}%
\pgfsetfillcolor{currentfill}%
\pgfsetlinewidth{0.501875pt}%
\definecolor{currentstroke}{rgb}{0.000000,0.000000,0.000000}%
\pgfsetstrokecolor{currentstroke}%
\pgfsetdash{}{0pt}%
\pgfsys@defobject{currentmarker}{\pgfqpoint{0.000000in}{0.000000in}}{\pgfqpoint{0.020833in}{0.000000in}}{%
\pgfpathmoveto{\pgfqpoint{0.000000in}{0.000000in}}%
\pgfpathlineto{\pgfqpoint{0.020833in}{0.000000in}}%
\pgfusepath{stroke,fill}%
}%
\begin{pgfscope}%
\pgfsys@transformshift{3.315933in}{1.256794in}%
\pgfsys@useobject{currentmarker}{}%
\end{pgfscope}%
\end{pgfscope}%
\begin{pgfscope}%
\pgfsetbuttcap%
\pgfsetroundjoin%
\definecolor{currentfill}{rgb}{0.000000,0.000000,0.000000}%
\pgfsetfillcolor{currentfill}%
\pgfsetlinewidth{0.501875pt}%
\definecolor{currentstroke}{rgb}{0.000000,0.000000,0.000000}%
\pgfsetstrokecolor{currentstroke}%
\pgfsetdash{}{0pt}%
\pgfsys@defobject{currentmarker}{\pgfqpoint{-0.020833in}{0.000000in}}{\pgfqpoint{-0.000000in}{0.000000in}}{%
\pgfpathmoveto{\pgfqpoint{-0.000000in}{0.000000in}}%
\pgfpathlineto{\pgfqpoint{-0.020833in}{0.000000in}}%
\pgfusepath{stroke,fill}%
}%
\begin{pgfscope}%
\pgfsys@transformshift{5.591997in}{1.256794in}%
\pgfsys@useobject{currentmarker}{}%
\end{pgfscope}%
\end{pgfscope}%
\begin{pgfscope}%
\pgfsetbuttcap%
\pgfsetroundjoin%
\definecolor{currentfill}{rgb}{0.000000,0.000000,0.000000}%
\pgfsetfillcolor{currentfill}%
\pgfsetlinewidth{0.501875pt}%
\definecolor{currentstroke}{rgb}{0.000000,0.000000,0.000000}%
\pgfsetstrokecolor{currentstroke}%
\pgfsetdash{}{0pt}%
\pgfsys@defobject{currentmarker}{\pgfqpoint{0.000000in}{0.000000in}}{\pgfqpoint{0.020833in}{0.000000in}}{%
\pgfpathmoveto{\pgfqpoint{0.000000in}{0.000000in}}%
\pgfpathlineto{\pgfqpoint{0.020833in}{0.000000in}}%
\pgfusepath{stroke,fill}%
}%
\begin{pgfscope}%
\pgfsys@transformshift{3.315933in}{1.368146in}%
\pgfsys@useobject{currentmarker}{}%
\end{pgfscope}%
\end{pgfscope}%
\begin{pgfscope}%
\pgfsetbuttcap%
\pgfsetroundjoin%
\definecolor{currentfill}{rgb}{0.000000,0.000000,0.000000}%
\pgfsetfillcolor{currentfill}%
\pgfsetlinewidth{0.501875pt}%
\definecolor{currentstroke}{rgb}{0.000000,0.000000,0.000000}%
\pgfsetstrokecolor{currentstroke}%
\pgfsetdash{}{0pt}%
\pgfsys@defobject{currentmarker}{\pgfqpoint{-0.020833in}{0.000000in}}{\pgfqpoint{-0.000000in}{0.000000in}}{%
\pgfpathmoveto{\pgfqpoint{-0.000000in}{0.000000in}}%
\pgfpathlineto{\pgfqpoint{-0.020833in}{0.000000in}}%
\pgfusepath{stroke,fill}%
}%
\begin{pgfscope}%
\pgfsys@transformshift{5.591997in}{1.368146in}%
\pgfsys@useobject{currentmarker}{}%
\end{pgfscope}%
\end{pgfscope}%
\begin{pgfscope}%
\pgfsetbuttcap%
\pgfsetroundjoin%
\definecolor{currentfill}{rgb}{0.000000,0.000000,0.000000}%
\pgfsetfillcolor{currentfill}%
\pgfsetlinewidth{0.501875pt}%
\definecolor{currentstroke}{rgb}{0.000000,0.000000,0.000000}%
\pgfsetstrokecolor{currentstroke}%
\pgfsetdash{}{0pt}%
\pgfsys@defobject{currentmarker}{\pgfqpoint{0.000000in}{0.000000in}}{\pgfqpoint{0.020833in}{0.000000in}}{%
\pgfpathmoveto{\pgfqpoint{0.000000in}{0.000000in}}%
\pgfpathlineto{\pgfqpoint{0.020833in}{0.000000in}}%
\pgfusepath{stroke,fill}%
}%
\begin{pgfscope}%
\pgfsys@transformshift{3.315933in}{1.590850in}%
\pgfsys@useobject{currentmarker}{}%
\end{pgfscope}%
\end{pgfscope}%
\begin{pgfscope}%
\pgfsetbuttcap%
\pgfsetroundjoin%
\definecolor{currentfill}{rgb}{0.000000,0.000000,0.000000}%
\pgfsetfillcolor{currentfill}%
\pgfsetlinewidth{0.501875pt}%
\definecolor{currentstroke}{rgb}{0.000000,0.000000,0.000000}%
\pgfsetstrokecolor{currentstroke}%
\pgfsetdash{}{0pt}%
\pgfsys@defobject{currentmarker}{\pgfqpoint{-0.020833in}{0.000000in}}{\pgfqpoint{-0.000000in}{0.000000in}}{%
\pgfpathmoveto{\pgfqpoint{-0.000000in}{0.000000in}}%
\pgfpathlineto{\pgfqpoint{-0.020833in}{0.000000in}}%
\pgfusepath{stroke,fill}%
}%
\begin{pgfscope}%
\pgfsys@transformshift{5.591997in}{1.590850in}%
\pgfsys@useobject{currentmarker}{}%
\end{pgfscope}%
\end{pgfscope}%
\begin{pgfscope}%
\pgfsetbuttcap%
\pgfsetroundjoin%
\definecolor{currentfill}{rgb}{0.000000,0.000000,0.000000}%
\pgfsetfillcolor{currentfill}%
\pgfsetlinewidth{0.501875pt}%
\definecolor{currentstroke}{rgb}{0.000000,0.000000,0.000000}%
\pgfsetstrokecolor{currentstroke}%
\pgfsetdash{}{0pt}%
\pgfsys@defobject{currentmarker}{\pgfqpoint{0.000000in}{0.000000in}}{\pgfqpoint{0.020833in}{0.000000in}}{%
\pgfpathmoveto{\pgfqpoint{0.000000in}{0.000000in}}%
\pgfpathlineto{\pgfqpoint{0.020833in}{0.000000in}}%
\pgfusepath{stroke,fill}%
}%
\begin{pgfscope}%
\pgfsys@transformshift{3.315933in}{1.702202in}%
\pgfsys@useobject{currentmarker}{}%
\end{pgfscope}%
\end{pgfscope}%
\begin{pgfscope}%
\pgfsetbuttcap%
\pgfsetroundjoin%
\definecolor{currentfill}{rgb}{0.000000,0.000000,0.000000}%
\pgfsetfillcolor{currentfill}%
\pgfsetlinewidth{0.501875pt}%
\definecolor{currentstroke}{rgb}{0.000000,0.000000,0.000000}%
\pgfsetstrokecolor{currentstroke}%
\pgfsetdash{}{0pt}%
\pgfsys@defobject{currentmarker}{\pgfqpoint{-0.020833in}{0.000000in}}{\pgfqpoint{-0.000000in}{0.000000in}}{%
\pgfpathmoveto{\pgfqpoint{-0.000000in}{0.000000in}}%
\pgfpathlineto{\pgfqpoint{-0.020833in}{0.000000in}}%
\pgfusepath{stroke,fill}%
}%
\begin{pgfscope}%
\pgfsys@transformshift{5.591997in}{1.702202in}%
\pgfsys@useobject{currentmarker}{}%
\end{pgfscope}%
\end{pgfscope}%
\begin{pgfscope}%
\pgfsetbuttcap%
\pgfsetroundjoin%
\definecolor{currentfill}{rgb}{0.000000,0.000000,0.000000}%
\pgfsetfillcolor{currentfill}%
\pgfsetlinewidth{0.501875pt}%
\definecolor{currentstroke}{rgb}{0.000000,0.000000,0.000000}%
\pgfsetstrokecolor{currentstroke}%
\pgfsetdash{}{0pt}%
\pgfsys@defobject{currentmarker}{\pgfqpoint{0.000000in}{0.000000in}}{\pgfqpoint{0.020833in}{0.000000in}}{%
\pgfpathmoveto{\pgfqpoint{0.000000in}{0.000000in}}%
\pgfpathlineto{\pgfqpoint{0.020833in}{0.000000in}}%
\pgfusepath{stroke,fill}%
}%
\begin{pgfscope}%
\pgfsys@transformshift{3.315933in}{1.813554in}%
\pgfsys@useobject{currentmarker}{}%
\end{pgfscope}%
\end{pgfscope}%
\begin{pgfscope}%
\pgfsetbuttcap%
\pgfsetroundjoin%
\definecolor{currentfill}{rgb}{0.000000,0.000000,0.000000}%
\pgfsetfillcolor{currentfill}%
\pgfsetlinewidth{0.501875pt}%
\definecolor{currentstroke}{rgb}{0.000000,0.000000,0.000000}%
\pgfsetstrokecolor{currentstroke}%
\pgfsetdash{}{0pt}%
\pgfsys@defobject{currentmarker}{\pgfqpoint{-0.020833in}{0.000000in}}{\pgfqpoint{-0.000000in}{0.000000in}}{%
\pgfpathmoveto{\pgfqpoint{-0.000000in}{0.000000in}}%
\pgfpathlineto{\pgfqpoint{-0.020833in}{0.000000in}}%
\pgfusepath{stroke,fill}%
}%
\begin{pgfscope}%
\pgfsys@transformshift{5.591997in}{1.813554in}%
\pgfsys@useobject{currentmarker}{}%
\end{pgfscope}%
\end{pgfscope}%
\begin{pgfscope}%
\pgfsetbuttcap%
\pgfsetroundjoin%
\definecolor{currentfill}{rgb}{0.000000,0.000000,0.000000}%
\pgfsetfillcolor{currentfill}%
\pgfsetlinewidth{0.501875pt}%
\definecolor{currentstroke}{rgb}{0.000000,0.000000,0.000000}%
\pgfsetstrokecolor{currentstroke}%
\pgfsetdash{}{0pt}%
\pgfsys@defobject{currentmarker}{\pgfqpoint{0.000000in}{0.000000in}}{\pgfqpoint{0.020833in}{0.000000in}}{%
\pgfpathmoveto{\pgfqpoint{0.000000in}{0.000000in}}%
\pgfpathlineto{\pgfqpoint{0.020833in}{0.000000in}}%
\pgfusepath{stroke,fill}%
}%
\begin{pgfscope}%
\pgfsys@transformshift{3.315933in}{1.924906in}%
\pgfsys@useobject{currentmarker}{}%
\end{pgfscope}%
\end{pgfscope}%
\begin{pgfscope}%
\pgfsetbuttcap%
\pgfsetroundjoin%
\definecolor{currentfill}{rgb}{0.000000,0.000000,0.000000}%
\pgfsetfillcolor{currentfill}%
\pgfsetlinewidth{0.501875pt}%
\definecolor{currentstroke}{rgb}{0.000000,0.000000,0.000000}%
\pgfsetstrokecolor{currentstroke}%
\pgfsetdash{}{0pt}%
\pgfsys@defobject{currentmarker}{\pgfqpoint{-0.020833in}{0.000000in}}{\pgfqpoint{-0.000000in}{0.000000in}}{%
\pgfpathmoveto{\pgfqpoint{-0.000000in}{0.000000in}}%
\pgfpathlineto{\pgfqpoint{-0.020833in}{0.000000in}}%
\pgfusepath{stroke,fill}%
}%
\begin{pgfscope}%
\pgfsys@transformshift{5.591997in}{1.924906in}%
\pgfsys@useobject{currentmarker}{}%
\end{pgfscope}%
\end{pgfscope}%
\begin{pgfscope}%
\pgfsetbuttcap%
\pgfsetroundjoin%
\definecolor{currentfill}{rgb}{0.000000,0.000000,0.000000}%
\pgfsetfillcolor{currentfill}%
\pgfsetlinewidth{0.501875pt}%
\definecolor{currentstroke}{rgb}{0.000000,0.000000,0.000000}%
\pgfsetstrokecolor{currentstroke}%
\pgfsetdash{}{0pt}%
\pgfsys@defobject{currentmarker}{\pgfqpoint{0.000000in}{0.000000in}}{\pgfqpoint{0.020833in}{0.000000in}}{%
\pgfpathmoveto{\pgfqpoint{0.000000in}{0.000000in}}%
\pgfpathlineto{\pgfqpoint{0.020833in}{0.000000in}}%
\pgfusepath{stroke,fill}%
}%
\begin{pgfscope}%
\pgfsys@transformshift{3.315933in}{2.147610in}%
\pgfsys@useobject{currentmarker}{}%
\end{pgfscope}%
\end{pgfscope}%
\begin{pgfscope}%
\pgfsetbuttcap%
\pgfsetroundjoin%
\definecolor{currentfill}{rgb}{0.000000,0.000000,0.000000}%
\pgfsetfillcolor{currentfill}%
\pgfsetlinewidth{0.501875pt}%
\definecolor{currentstroke}{rgb}{0.000000,0.000000,0.000000}%
\pgfsetstrokecolor{currentstroke}%
\pgfsetdash{}{0pt}%
\pgfsys@defobject{currentmarker}{\pgfqpoint{-0.020833in}{0.000000in}}{\pgfqpoint{-0.000000in}{0.000000in}}{%
\pgfpathmoveto{\pgfqpoint{-0.000000in}{0.000000in}}%
\pgfpathlineto{\pgfqpoint{-0.020833in}{0.000000in}}%
\pgfusepath{stroke,fill}%
}%
\begin{pgfscope}%
\pgfsys@transformshift{5.591997in}{2.147610in}%
\pgfsys@useobject{currentmarker}{}%
\end{pgfscope}%
\end{pgfscope}%
\begin{pgfscope}%
\pgfsetbuttcap%
\pgfsetroundjoin%
\definecolor{currentfill}{rgb}{0.000000,0.000000,0.000000}%
\pgfsetfillcolor{currentfill}%
\pgfsetlinewidth{0.501875pt}%
\definecolor{currentstroke}{rgb}{0.000000,0.000000,0.000000}%
\pgfsetstrokecolor{currentstroke}%
\pgfsetdash{}{0pt}%
\pgfsys@defobject{currentmarker}{\pgfqpoint{0.000000in}{0.000000in}}{\pgfqpoint{0.020833in}{0.000000in}}{%
\pgfpathmoveto{\pgfqpoint{0.000000in}{0.000000in}}%
\pgfpathlineto{\pgfqpoint{0.020833in}{0.000000in}}%
\pgfusepath{stroke,fill}%
}%
\begin{pgfscope}%
\pgfsys@transformshift{3.315933in}{2.258962in}%
\pgfsys@useobject{currentmarker}{}%
\end{pgfscope}%
\end{pgfscope}%
\begin{pgfscope}%
\pgfsetbuttcap%
\pgfsetroundjoin%
\definecolor{currentfill}{rgb}{0.000000,0.000000,0.000000}%
\pgfsetfillcolor{currentfill}%
\pgfsetlinewidth{0.501875pt}%
\definecolor{currentstroke}{rgb}{0.000000,0.000000,0.000000}%
\pgfsetstrokecolor{currentstroke}%
\pgfsetdash{}{0pt}%
\pgfsys@defobject{currentmarker}{\pgfqpoint{-0.020833in}{0.000000in}}{\pgfqpoint{-0.000000in}{0.000000in}}{%
\pgfpathmoveto{\pgfqpoint{-0.000000in}{0.000000in}}%
\pgfpathlineto{\pgfqpoint{-0.020833in}{0.000000in}}%
\pgfusepath{stroke,fill}%
}%
\begin{pgfscope}%
\pgfsys@transformshift{5.591997in}{2.258962in}%
\pgfsys@useobject{currentmarker}{}%
\end{pgfscope}%
\end{pgfscope}%
\begin{pgfscope}%
\pgfsetbuttcap%
\pgfsetroundjoin%
\definecolor{currentfill}{rgb}{0.000000,0.000000,0.000000}%
\pgfsetfillcolor{currentfill}%
\pgfsetlinewidth{0.501875pt}%
\definecolor{currentstroke}{rgb}{0.000000,0.000000,0.000000}%
\pgfsetstrokecolor{currentstroke}%
\pgfsetdash{}{0pt}%
\pgfsys@defobject{currentmarker}{\pgfqpoint{0.000000in}{0.000000in}}{\pgfqpoint{0.020833in}{0.000000in}}{%
\pgfpathmoveto{\pgfqpoint{0.000000in}{0.000000in}}%
\pgfpathlineto{\pgfqpoint{0.020833in}{0.000000in}}%
\pgfusepath{stroke,fill}%
}%
\begin{pgfscope}%
\pgfsys@transformshift{3.315933in}{2.370314in}%
\pgfsys@useobject{currentmarker}{}%
\end{pgfscope}%
\end{pgfscope}%
\begin{pgfscope}%
\pgfsetbuttcap%
\pgfsetroundjoin%
\definecolor{currentfill}{rgb}{0.000000,0.000000,0.000000}%
\pgfsetfillcolor{currentfill}%
\pgfsetlinewidth{0.501875pt}%
\definecolor{currentstroke}{rgb}{0.000000,0.000000,0.000000}%
\pgfsetstrokecolor{currentstroke}%
\pgfsetdash{}{0pt}%
\pgfsys@defobject{currentmarker}{\pgfqpoint{-0.020833in}{0.000000in}}{\pgfqpoint{-0.000000in}{0.000000in}}{%
\pgfpathmoveto{\pgfqpoint{-0.000000in}{0.000000in}}%
\pgfpathlineto{\pgfqpoint{-0.020833in}{0.000000in}}%
\pgfusepath{stroke,fill}%
}%
\begin{pgfscope}%
\pgfsys@transformshift{5.591997in}{2.370314in}%
\pgfsys@useobject{currentmarker}{}%
\end{pgfscope}%
\end{pgfscope}%
\begin{pgfscope}%
\pgfsetbuttcap%
\pgfsetroundjoin%
\definecolor{currentfill}{rgb}{0.000000,0.000000,0.000000}%
\pgfsetfillcolor{currentfill}%
\pgfsetlinewidth{0.501875pt}%
\definecolor{currentstroke}{rgb}{0.000000,0.000000,0.000000}%
\pgfsetstrokecolor{currentstroke}%
\pgfsetdash{}{0pt}%
\pgfsys@defobject{currentmarker}{\pgfqpoint{0.000000in}{0.000000in}}{\pgfqpoint{0.020833in}{0.000000in}}{%
\pgfpathmoveto{\pgfqpoint{0.000000in}{0.000000in}}%
\pgfpathlineto{\pgfqpoint{0.020833in}{0.000000in}}%
\pgfusepath{stroke,fill}%
}%
\begin{pgfscope}%
\pgfsys@transformshift{3.315933in}{2.481666in}%
\pgfsys@useobject{currentmarker}{}%
\end{pgfscope}%
\end{pgfscope}%
\begin{pgfscope}%
\pgfsetbuttcap%
\pgfsetroundjoin%
\definecolor{currentfill}{rgb}{0.000000,0.000000,0.000000}%
\pgfsetfillcolor{currentfill}%
\pgfsetlinewidth{0.501875pt}%
\definecolor{currentstroke}{rgb}{0.000000,0.000000,0.000000}%
\pgfsetstrokecolor{currentstroke}%
\pgfsetdash{}{0pt}%
\pgfsys@defobject{currentmarker}{\pgfqpoint{-0.020833in}{0.000000in}}{\pgfqpoint{-0.000000in}{0.000000in}}{%
\pgfpathmoveto{\pgfqpoint{-0.000000in}{0.000000in}}%
\pgfpathlineto{\pgfqpoint{-0.020833in}{0.000000in}}%
\pgfusepath{stroke,fill}%
}%
\begin{pgfscope}%
\pgfsys@transformshift{5.591997in}{2.481666in}%
\pgfsys@useobject{currentmarker}{}%
\end{pgfscope}%
\end{pgfscope}%
\begin{pgfscope}%
\pgfsetbuttcap%
\pgfsetroundjoin%
\definecolor{currentfill}{rgb}{0.000000,0.000000,0.000000}%
\pgfsetfillcolor{currentfill}%
\pgfsetlinewidth{0.501875pt}%
\definecolor{currentstroke}{rgb}{0.000000,0.000000,0.000000}%
\pgfsetstrokecolor{currentstroke}%
\pgfsetdash{}{0pt}%
\pgfsys@defobject{currentmarker}{\pgfqpoint{0.000000in}{0.000000in}}{\pgfqpoint{0.020833in}{0.000000in}}{%
\pgfpathmoveto{\pgfqpoint{0.000000in}{0.000000in}}%
\pgfpathlineto{\pgfqpoint{0.020833in}{0.000000in}}%
\pgfusepath{stroke,fill}%
}%
\begin{pgfscope}%
\pgfsys@transformshift{3.315933in}{2.704370in}%
\pgfsys@useobject{currentmarker}{}%
\end{pgfscope}%
\end{pgfscope}%
\begin{pgfscope}%
\pgfsetbuttcap%
\pgfsetroundjoin%
\definecolor{currentfill}{rgb}{0.000000,0.000000,0.000000}%
\pgfsetfillcolor{currentfill}%
\pgfsetlinewidth{0.501875pt}%
\definecolor{currentstroke}{rgb}{0.000000,0.000000,0.000000}%
\pgfsetstrokecolor{currentstroke}%
\pgfsetdash{}{0pt}%
\pgfsys@defobject{currentmarker}{\pgfqpoint{-0.020833in}{0.000000in}}{\pgfqpoint{-0.000000in}{0.000000in}}{%
\pgfpathmoveto{\pgfqpoint{-0.000000in}{0.000000in}}%
\pgfpathlineto{\pgfqpoint{-0.020833in}{0.000000in}}%
\pgfusepath{stroke,fill}%
}%
\begin{pgfscope}%
\pgfsys@transformshift{5.591997in}{2.704370in}%
\pgfsys@useobject{currentmarker}{}%
\end{pgfscope}%
\end{pgfscope}%
\begin{pgfscope}%
\pgfsetbuttcap%
\pgfsetroundjoin%
\definecolor{currentfill}{rgb}{0.000000,0.000000,0.000000}%
\pgfsetfillcolor{currentfill}%
\pgfsetlinewidth{0.501875pt}%
\definecolor{currentstroke}{rgb}{0.000000,0.000000,0.000000}%
\pgfsetstrokecolor{currentstroke}%
\pgfsetdash{}{0pt}%
\pgfsys@defobject{currentmarker}{\pgfqpoint{0.000000in}{0.000000in}}{\pgfqpoint{0.020833in}{0.000000in}}{%
\pgfpathmoveto{\pgfqpoint{0.000000in}{0.000000in}}%
\pgfpathlineto{\pgfqpoint{0.020833in}{0.000000in}}%
\pgfusepath{stroke,fill}%
}%
\begin{pgfscope}%
\pgfsys@transformshift{3.315933in}{2.815722in}%
\pgfsys@useobject{currentmarker}{}%
\end{pgfscope}%
\end{pgfscope}%
\begin{pgfscope}%
\pgfsetbuttcap%
\pgfsetroundjoin%
\definecolor{currentfill}{rgb}{0.000000,0.000000,0.000000}%
\pgfsetfillcolor{currentfill}%
\pgfsetlinewidth{0.501875pt}%
\definecolor{currentstroke}{rgb}{0.000000,0.000000,0.000000}%
\pgfsetstrokecolor{currentstroke}%
\pgfsetdash{}{0pt}%
\pgfsys@defobject{currentmarker}{\pgfqpoint{-0.020833in}{0.000000in}}{\pgfqpoint{-0.000000in}{0.000000in}}{%
\pgfpathmoveto{\pgfqpoint{-0.000000in}{0.000000in}}%
\pgfpathlineto{\pgfqpoint{-0.020833in}{0.000000in}}%
\pgfusepath{stroke,fill}%
}%
\begin{pgfscope}%
\pgfsys@transformshift{5.591997in}{2.815722in}%
\pgfsys@useobject{currentmarker}{}%
\end{pgfscope}%
\end{pgfscope}%
\begin{pgfscope}%
\pgfsetbuttcap%
\pgfsetroundjoin%
\definecolor{currentfill}{rgb}{0.000000,0.000000,0.000000}%
\pgfsetfillcolor{currentfill}%
\pgfsetlinewidth{0.501875pt}%
\definecolor{currentstroke}{rgb}{0.000000,0.000000,0.000000}%
\pgfsetstrokecolor{currentstroke}%
\pgfsetdash{}{0pt}%
\pgfsys@defobject{currentmarker}{\pgfqpoint{0.000000in}{0.000000in}}{\pgfqpoint{0.020833in}{0.000000in}}{%
\pgfpathmoveto{\pgfqpoint{0.000000in}{0.000000in}}%
\pgfpathlineto{\pgfqpoint{0.020833in}{0.000000in}}%
\pgfusepath{stroke,fill}%
}%
\begin{pgfscope}%
\pgfsys@transformshift{3.315933in}{2.927074in}%
\pgfsys@useobject{currentmarker}{}%
\end{pgfscope}%
\end{pgfscope}%
\begin{pgfscope}%
\pgfsetbuttcap%
\pgfsetroundjoin%
\definecolor{currentfill}{rgb}{0.000000,0.000000,0.000000}%
\pgfsetfillcolor{currentfill}%
\pgfsetlinewidth{0.501875pt}%
\definecolor{currentstroke}{rgb}{0.000000,0.000000,0.000000}%
\pgfsetstrokecolor{currentstroke}%
\pgfsetdash{}{0pt}%
\pgfsys@defobject{currentmarker}{\pgfqpoint{-0.020833in}{0.000000in}}{\pgfqpoint{-0.000000in}{0.000000in}}{%
\pgfpathmoveto{\pgfqpoint{-0.000000in}{0.000000in}}%
\pgfpathlineto{\pgfqpoint{-0.020833in}{0.000000in}}%
\pgfusepath{stroke,fill}%
}%
\begin{pgfscope}%
\pgfsys@transformshift{5.591997in}{2.927074in}%
\pgfsys@useobject{currentmarker}{}%
\end{pgfscope}%
\end{pgfscope}%
\begin{pgfscope}%
\pgfsetbuttcap%
\pgfsetroundjoin%
\definecolor{currentfill}{rgb}{0.000000,0.000000,0.000000}%
\pgfsetfillcolor{currentfill}%
\pgfsetlinewidth{0.501875pt}%
\definecolor{currentstroke}{rgb}{0.000000,0.000000,0.000000}%
\pgfsetstrokecolor{currentstroke}%
\pgfsetdash{}{0pt}%
\pgfsys@defobject{currentmarker}{\pgfqpoint{0.000000in}{0.000000in}}{\pgfqpoint{0.020833in}{0.000000in}}{%
\pgfpathmoveto{\pgfqpoint{0.000000in}{0.000000in}}%
\pgfpathlineto{\pgfqpoint{0.020833in}{0.000000in}}%
\pgfusepath{stroke,fill}%
}%
\begin{pgfscope}%
\pgfsys@transformshift{3.315933in}{3.038426in}%
\pgfsys@useobject{currentmarker}{}%
\end{pgfscope}%
\end{pgfscope}%
\begin{pgfscope}%
\pgfsetbuttcap%
\pgfsetroundjoin%
\definecolor{currentfill}{rgb}{0.000000,0.000000,0.000000}%
\pgfsetfillcolor{currentfill}%
\pgfsetlinewidth{0.501875pt}%
\definecolor{currentstroke}{rgb}{0.000000,0.000000,0.000000}%
\pgfsetstrokecolor{currentstroke}%
\pgfsetdash{}{0pt}%
\pgfsys@defobject{currentmarker}{\pgfqpoint{-0.020833in}{0.000000in}}{\pgfqpoint{-0.000000in}{0.000000in}}{%
\pgfpathmoveto{\pgfqpoint{-0.000000in}{0.000000in}}%
\pgfpathlineto{\pgfqpoint{-0.020833in}{0.000000in}}%
\pgfusepath{stroke,fill}%
}%
\begin{pgfscope}%
\pgfsys@transformshift{5.591997in}{3.038426in}%
\pgfsys@useobject{currentmarker}{}%
\end{pgfscope}%
\end{pgfscope}%
\begin{pgfscope}%
\pgfsetbuttcap%
\pgfsetroundjoin%
\definecolor{currentfill}{rgb}{0.000000,0.000000,0.000000}%
\pgfsetfillcolor{currentfill}%
\pgfsetlinewidth{0.501875pt}%
\definecolor{currentstroke}{rgb}{0.000000,0.000000,0.000000}%
\pgfsetstrokecolor{currentstroke}%
\pgfsetdash{}{0pt}%
\pgfsys@defobject{currentmarker}{\pgfqpoint{0.000000in}{0.000000in}}{\pgfqpoint{0.020833in}{0.000000in}}{%
\pgfpathmoveto{\pgfqpoint{0.000000in}{0.000000in}}%
\pgfpathlineto{\pgfqpoint{0.020833in}{0.000000in}}%
\pgfusepath{stroke,fill}%
}%
\begin{pgfscope}%
\pgfsys@transformshift{3.315933in}{3.261130in}%
\pgfsys@useobject{currentmarker}{}%
\end{pgfscope}%
\end{pgfscope}%
\begin{pgfscope}%
\pgfsetbuttcap%
\pgfsetroundjoin%
\definecolor{currentfill}{rgb}{0.000000,0.000000,0.000000}%
\pgfsetfillcolor{currentfill}%
\pgfsetlinewidth{0.501875pt}%
\definecolor{currentstroke}{rgb}{0.000000,0.000000,0.000000}%
\pgfsetstrokecolor{currentstroke}%
\pgfsetdash{}{0pt}%
\pgfsys@defobject{currentmarker}{\pgfqpoint{-0.020833in}{0.000000in}}{\pgfqpoint{-0.000000in}{0.000000in}}{%
\pgfpathmoveto{\pgfqpoint{-0.000000in}{0.000000in}}%
\pgfpathlineto{\pgfqpoint{-0.020833in}{0.000000in}}%
\pgfusepath{stroke,fill}%
}%
\begin{pgfscope}%
\pgfsys@transformshift{5.591997in}{3.261130in}%
\pgfsys@useobject{currentmarker}{}%
\end{pgfscope}%
\end{pgfscope}%
\begin{pgfscope}%
\pgfsetbuttcap%
\pgfsetroundjoin%
\definecolor{currentfill}{rgb}{0.000000,0.000000,0.000000}%
\pgfsetfillcolor{currentfill}%
\pgfsetlinewidth{0.501875pt}%
\definecolor{currentstroke}{rgb}{0.000000,0.000000,0.000000}%
\pgfsetstrokecolor{currentstroke}%
\pgfsetdash{}{0pt}%
\pgfsys@defobject{currentmarker}{\pgfqpoint{0.000000in}{0.000000in}}{\pgfqpoint{0.020833in}{0.000000in}}{%
\pgfpathmoveto{\pgfqpoint{0.000000in}{0.000000in}}%
\pgfpathlineto{\pgfqpoint{0.020833in}{0.000000in}}%
\pgfusepath{stroke,fill}%
}%
\begin{pgfscope}%
\pgfsys@transformshift{3.315933in}{3.372482in}%
\pgfsys@useobject{currentmarker}{}%
\end{pgfscope}%
\end{pgfscope}%
\begin{pgfscope}%
\pgfsetbuttcap%
\pgfsetroundjoin%
\definecolor{currentfill}{rgb}{0.000000,0.000000,0.000000}%
\pgfsetfillcolor{currentfill}%
\pgfsetlinewidth{0.501875pt}%
\definecolor{currentstroke}{rgb}{0.000000,0.000000,0.000000}%
\pgfsetstrokecolor{currentstroke}%
\pgfsetdash{}{0pt}%
\pgfsys@defobject{currentmarker}{\pgfqpoint{-0.020833in}{0.000000in}}{\pgfqpoint{-0.000000in}{0.000000in}}{%
\pgfpathmoveto{\pgfqpoint{-0.000000in}{0.000000in}}%
\pgfpathlineto{\pgfqpoint{-0.020833in}{0.000000in}}%
\pgfusepath{stroke,fill}%
}%
\begin{pgfscope}%
\pgfsys@transformshift{5.591997in}{3.372482in}%
\pgfsys@useobject{currentmarker}{}%
\end{pgfscope}%
\end{pgfscope}%
\begin{pgfscope}%
\pgfsetbuttcap%
\pgfsetroundjoin%
\definecolor{currentfill}{rgb}{0.000000,0.000000,0.000000}%
\pgfsetfillcolor{currentfill}%
\pgfsetlinewidth{0.501875pt}%
\definecolor{currentstroke}{rgb}{0.000000,0.000000,0.000000}%
\pgfsetstrokecolor{currentstroke}%
\pgfsetdash{}{0pt}%
\pgfsys@defobject{currentmarker}{\pgfqpoint{0.000000in}{0.000000in}}{\pgfqpoint{0.020833in}{0.000000in}}{%
\pgfpathmoveto{\pgfqpoint{0.000000in}{0.000000in}}%
\pgfpathlineto{\pgfqpoint{0.020833in}{0.000000in}}%
\pgfusepath{stroke,fill}%
}%
\begin{pgfscope}%
\pgfsys@transformshift{3.315933in}{3.483834in}%
\pgfsys@useobject{currentmarker}{}%
\end{pgfscope}%
\end{pgfscope}%
\begin{pgfscope}%
\pgfsetbuttcap%
\pgfsetroundjoin%
\definecolor{currentfill}{rgb}{0.000000,0.000000,0.000000}%
\pgfsetfillcolor{currentfill}%
\pgfsetlinewidth{0.501875pt}%
\definecolor{currentstroke}{rgb}{0.000000,0.000000,0.000000}%
\pgfsetstrokecolor{currentstroke}%
\pgfsetdash{}{0pt}%
\pgfsys@defobject{currentmarker}{\pgfqpoint{-0.020833in}{0.000000in}}{\pgfqpoint{-0.000000in}{0.000000in}}{%
\pgfpathmoveto{\pgfqpoint{-0.000000in}{0.000000in}}%
\pgfpathlineto{\pgfqpoint{-0.020833in}{0.000000in}}%
\pgfusepath{stroke,fill}%
}%
\begin{pgfscope}%
\pgfsys@transformshift{5.591997in}{3.483834in}%
\pgfsys@useobject{currentmarker}{}%
\end{pgfscope}%
\end{pgfscope}%
\begin{pgfscope}%
\pgfsetbuttcap%
\pgfsetroundjoin%
\definecolor{currentfill}{rgb}{0.000000,0.000000,0.000000}%
\pgfsetfillcolor{currentfill}%
\pgfsetlinewidth{0.501875pt}%
\definecolor{currentstroke}{rgb}{0.000000,0.000000,0.000000}%
\pgfsetstrokecolor{currentstroke}%
\pgfsetdash{}{0pt}%
\pgfsys@defobject{currentmarker}{\pgfqpoint{0.000000in}{0.000000in}}{\pgfqpoint{0.020833in}{0.000000in}}{%
\pgfpathmoveto{\pgfqpoint{0.000000in}{0.000000in}}%
\pgfpathlineto{\pgfqpoint{0.020833in}{0.000000in}}%
\pgfusepath{stroke,fill}%
}%
\begin{pgfscope}%
\pgfsys@transformshift{3.315933in}{3.595186in}%
\pgfsys@useobject{currentmarker}{}%
\end{pgfscope}%
\end{pgfscope}%
\begin{pgfscope}%
\pgfsetbuttcap%
\pgfsetroundjoin%
\definecolor{currentfill}{rgb}{0.000000,0.000000,0.000000}%
\pgfsetfillcolor{currentfill}%
\pgfsetlinewidth{0.501875pt}%
\definecolor{currentstroke}{rgb}{0.000000,0.000000,0.000000}%
\pgfsetstrokecolor{currentstroke}%
\pgfsetdash{}{0pt}%
\pgfsys@defobject{currentmarker}{\pgfqpoint{-0.020833in}{0.000000in}}{\pgfqpoint{-0.000000in}{0.000000in}}{%
\pgfpathmoveto{\pgfqpoint{-0.000000in}{0.000000in}}%
\pgfpathlineto{\pgfqpoint{-0.020833in}{0.000000in}}%
\pgfusepath{stroke,fill}%
}%
\begin{pgfscope}%
\pgfsys@transformshift{5.591997in}{3.595186in}%
\pgfsys@useobject{currentmarker}{}%
\end{pgfscope}%
\end{pgfscope}%
\begin{pgfscope}%
\pgfsetbuttcap%
\pgfsetroundjoin%
\definecolor{currentfill}{rgb}{0.000000,0.000000,0.000000}%
\pgfsetfillcolor{currentfill}%
\pgfsetlinewidth{0.501875pt}%
\definecolor{currentstroke}{rgb}{0.000000,0.000000,0.000000}%
\pgfsetstrokecolor{currentstroke}%
\pgfsetdash{}{0pt}%
\pgfsys@defobject{currentmarker}{\pgfqpoint{0.000000in}{0.000000in}}{\pgfqpoint{0.020833in}{0.000000in}}{%
\pgfpathmoveto{\pgfqpoint{0.000000in}{0.000000in}}%
\pgfpathlineto{\pgfqpoint{0.020833in}{0.000000in}}%
\pgfusepath{stroke,fill}%
}%
\begin{pgfscope}%
\pgfsys@transformshift{3.315933in}{3.817891in}%
\pgfsys@useobject{currentmarker}{}%
\end{pgfscope}%
\end{pgfscope}%
\begin{pgfscope}%
\pgfsetbuttcap%
\pgfsetroundjoin%
\definecolor{currentfill}{rgb}{0.000000,0.000000,0.000000}%
\pgfsetfillcolor{currentfill}%
\pgfsetlinewidth{0.501875pt}%
\definecolor{currentstroke}{rgb}{0.000000,0.000000,0.000000}%
\pgfsetstrokecolor{currentstroke}%
\pgfsetdash{}{0pt}%
\pgfsys@defobject{currentmarker}{\pgfqpoint{-0.020833in}{0.000000in}}{\pgfqpoint{-0.000000in}{0.000000in}}{%
\pgfpathmoveto{\pgfqpoint{-0.000000in}{0.000000in}}%
\pgfpathlineto{\pgfqpoint{-0.020833in}{0.000000in}}%
\pgfusepath{stroke,fill}%
}%
\begin{pgfscope}%
\pgfsys@transformshift{5.591997in}{3.817891in}%
\pgfsys@useobject{currentmarker}{}%
\end{pgfscope}%
\end{pgfscope}%
\begin{pgfscope}%
\pgfsetbuttcap%
\pgfsetroundjoin%
\definecolor{currentfill}{rgb}{0.000000,0.000000,0.000000}%
\pgfsetfillcolor{currentfill}%
\pgfsetlinewidth{0.501875pt}%
\definecolor{currentstroke}{rgb}{0.000000,0.000000,0.000000}%
\pgfsetstrokecolor{currentstroke}%
\pgfsetdash{}{0pt}%
\pgfsys@defobject{currentmarker}{\pgfqpoint{0.000000in}{0.000000in}}{\pgfqpoint{0.020833in}{0.000000in}}{%
\pgfpathmoveto{\pgfqpoint{0.000000in}{0.000000in}}%
\pgfpathlineto{\pgfqpoint{0.020833in}{0.000000in}}%
\pgfusepath{stroke,fill}%
}%
\begin{pgfscope}%
\pgfsys@transformshift{3.315933in}{3.929243in}%
\pgfsys@useobject{currentmarker}{}%
\end{pgfscope}%
\end{pgfscope}%
\begin{pgfscope}%
\pgfsetbuttcap%
\pgfsetroundjoin%
\definecolor{currentfill}{rgb}{0.000000,0.000000,0.000000}%
\pgfsetfillcolor{currentfill}%
\pgfsetlinewidth{0.501875pt}%
\definecolor{currentstroke}{rgb}{0.000000,0.000000,0.000000}%
\pgfsetstrokecolor{currentstroke}%
\pgfsetdash{}{0pt}%
\pgfsys@defobject{currentmarker}{\pgfqpoint{-0.020833in}{0.000000in}}{\pgfqpoint{-0.000000in}{0.000000in}}{%
\pgfpathmoveto{\pgfqpoint{-0.000000in}{0.000000in}}%
\pgfpathlineto{\pgfqpoint{-0.020833in}{0.000000in}}%
\pgfusepath{stroke,fill}%
}%
\begin{pgfscope}%
\pgfsys@transformshift{5.591997in}{3.929243in}%
\pgfsys@useobject{currentmarker}{}%
\end{pgfscope}%
\end{pgfscope}%
\begin{pgfscope}%
\pgfsetbuttcap%
\pgfsetroundjoin%
\definecolor{currentfill}{rgb}{0.000000,0.000000,0.000000}%
\pgfsetfillcolor{currentfill}%
\pgfsetlinewidth{0.501875pt}%
\definecolor{currentstroke}{rgb}{0.000000,0.000000,0.000000}%
\pgfsetstrokecolor{currentstroke}%
\pgfsetdash{}{0pt}%
\pgfsys@defobject{currentmarker}{\pgfqpoint{0.000000in}{0.000000in}}{\pgfqpoint{0.020833in}{0.000000in}}{%
\pgfpathmoveto{\pgfqpoint{0.000000in}{0.000000in}}%
\pgfpathlineto{\pgfqpoint{0.020833in}{0.000000in}}%
\pgfusepath{stroke,fill}%
}%
\begin{pgfscope}%
\pgfsys@transformshift{3.315933in}{4.040595in}%
\pgfsys@useobject{currentmarker}{}%
\end{pgfscope}%
\end{pgfscope}%
\begin{pgfscope}%
\pgfsetbuttcap%
\pgfsetroundjoin%
\definecolor{currentfill}{rgb}{0.000000,0.000000,0.000000}%
\pgfsetfillcolor{currentfill}%
\pgfsetlinewidth{0.501875pt}%
\definecolor{currentstroke}{rgb}{0.000000,0.000000,0.000000}%
\pgfsetstrokecolor{currentstroke}%
\pgfsetdash{}{0pt}%
\pgfsys@defobject{currentmarker}{\pgfqpoint{-0.020833in}{0.000000in}}{\pgfqpoint{-0.000000in}{0.000000in}}{%
\pgfpathmoveto{\pgfqpoint{-0.000000in}{0.000000in}}%
\pgfpathlineto{\pgfqpoint{-0.020833in}{0.000000in}}%
\pgfusepath{stroke,fill}%
}%
\begin{pgfscope}%
\pgfsys@transformshift{5.591997in}{4.040595in}%
\pgfsys@useobject{currentmarker}{}%
\end{pgfscope}%
\end{pgfscope}%
\begin{pgfscope}%
\pgfsetbuttcap%
\pgfsetroundjoin%
\definecolor{currentfill}{rgb}{0.000000,0.000000,0.000000}%
\pgfsetfillcolor{currentfill}%
\pgfsetlinewidth{0.501875pt}%
\definecolor{currentstroke}{rgb}{0.000000,0.000000,0.000000}%
\pgfsetstrokecolor{currentstroke}%
\pgfsetdash{}{0pt}%
\pgfsys@defobject{currentmarker}{\pgfqpoint{0.000000in}{0.000000in}}{\pgfqpoint{0.020833in}{0.000000in}}{%
\pgfpathmoveto{\pgfqpoint{0.000000in}{0.000000in}}%
\pgfpathlineto{\pgfqpoint{0.020833in}{0.000000in}}%
\pgfusepath{stroke,fill}%
}%
\begin{pgfscope}%
\pgfsys@transformshift{3.315933in}{4.151947in}%
\pgfsys@useobject{currentmarker}{}%
\end{pgfscope}%
\end{pgfscope}%
\begin{pgfscope}%
\pgfsetbuttcap%
\pgfsetroundjoin%
\definecolor{currentfill}{rgb}{0.000000,0.000000,0.000000}%
\pgfsetfillcolor{currentfill}%
\pgfsetlinewidth{0.501875pt}%
\definecolor{currentstroke}{rgb}{0.000000,0.000000,0.000000}%
\pgfsetstrokecolor{currentstroke}%
\pgfsetdash{}{0pt}%
\pgfsys@defobject{currentmarker}{\pgfqpoint{-0.020833in}{0.000000in}}{\pgfqpoint{-0.000000in}{0.000000in}}{%
\pgfpathmoveto{\pgfqpoint{-0.000000in}{0.000000in}}%
\pgfpathlineto{\pgfqpoint{-0.020833in}{0.000000in}}%
\pgfusepath{stroke,fill}%
}%
\begin{pgfscope}%
\pgfsys@transformshift{5.591997in}{4.151947in}%
\pgfsys@useobject{currentmarker}{}%
\end{pgfscope}%
\end{pgfscope}%
\begin{pgfscope}%
\definecolor{textcolor}{rgb}{0.000000,0.000000,0.000000}%
\pgfsetstrokecolor{textcolor}%
\pgfsetfillcolor{textcolor}%
\pgftext[x=3.034296in,y=2.398593in,,bottom,rotate=90.000000]{\color{textcolor}\rmfamily\fontsize{10.000000}{12.000000}\selectfont \(\displaystyle LCMC(K)\)}%
\end{pgfscope}%
\begin{pgfscope}%
\pgfpathrectangle{\pgfqpoint{3.315933in}{0.422992in}}{\pgfqpoint{2.276064in}{3.951201in}}%
\pgfusepath{clip}%
\pgfsetrectcap%
\pgfsetroundjoin%
\pgfsetlinewidth{1.003750pt}%
\definecolor{currentstroke}{rgb}{0.047059,0.364706,0.647059}%
\pgfsetstrokecolor{currentstroke}%
\pgfsetdash{}{0pt}%
\pgfpathmoveto{\pgfqpoint{3.338468in}{2.050169in}}%
\pgfpathlineto{\pgfqpoint{3.361003in}{2.425973in}}%
\pgfpathlineto{\pgfqpoint{3.383539in}{2.708984in}}%
\pgfpathlineto{\pgfqpoint{3.406074in}{2.900941in}}%
\pgfpathlineto{\pgfqpoint{3.428609in}{3.048126in}}%
\pgfpathlineto{\pgfqpoint{3.451144in}{3.090570in}}%
\pgfpathlineto{\pgfqpoint{3.473680in}{3.177555in}}%
\pgfpathlineto{\pgfqpoint{3.496215in}{3.267150in}}%
\pgfpathlineto{\pgfqpoint{3.518750in}{3.327554in}}%
\pgfpathlineto{\pgfqpoint{3.541286in}{3.355693in}}%
\pgfpathlineto{\pgfqpoint{3.563821in}{3.410348in}}%
\pgfpathlineto{\pgfqpoint{3.586356in}{3.440233in}}%
\pgfpathlineto{\pgfqpoint{3.608891in}{3.482115in}}%
\pgfpathlineto{\pgfqpoint{3.631427in}{3.521493in}}%
\pgfpathlineto{\pgfqpoint{3.653962in}{3.580209in}}%
\pgfpathlineto{\pgfqpoint{3.676497in}{3.600702in}}%
\pgfpathlineto{\pgfqpoint{3.699033in}{3.629017in}}%
\pgfpathlineto{\pgfqpoint{3.721568in}{3.662305in}}%
\pgfpathlineto{\pgfqpoint{3.744103in}{3.677436in}}%
\pgfpathlineto{\pgfqpoint{3.766638in}{3.700797in}}%
\pgfpathlineto{\pgfqpoint{3.789174in}{3.730880in}}%
\pgfpathlineto{\pgfqpoint{3.811709in}{3.748104in}}%
\pgfpathlineto{\pgfqpoint{3.834244in}{3.776538in}}%
\pgfpathlineto{\pgfqpoint{3.856780in}{3.791583in}}%
\pgfpathlineto{\pgfqpoint{3.879315in}{3.808765in}}%
\pgfpathlineto{\pgfqpoint{3.901850in}{3.830512in}}%
\pgfpathlineto{\pgfqpoint{3.924385in}{3.841885in}}%
\pgfpathlineto{\pgfqpoint{3.946921in}{3.850207in}}%
\pgfpathlineto{\pgfqpoint{3.969456in}{3.870674in}}%
\pgfpathlineto{\pgfqpoint{3.991991in}{3.880729in}}%
\pgfpathlineto{\pgfqpoint{4.014527in}{3.892828in}}%
\pgfpathlineto{\pgfqpoint{4.037062in}{3.908085in}}%
\pgfpathlineto{\pgfqpoint{4.059597in}{3.923261in}}%
\pgfpathlineto{\pgfqpoint{4.082132in}{3.934064in}}%
\pgfpathlineto{\pgfqpoint{4.104668in}{3.951208in}}%
\pgfpathlineto{\pgfqpoint{4.127203in}{3.958507in}}%
\pgfpathlineto{\pgfqpoint{4.149738in}{3.973311in}}%
\pgfpathlineto{\pgfqpoint{4.172274in}{3.978544in}}%
\pgfpathlineto{\pgfqpoint{4.194809in}{3.992609in}}%
\pgfpathlineto{\pgfqpoint{4.217344in}{4.002491in}}%
\pgfpathlineto{\pgfqpoint{4.239879in}{4.010363in}}%
\pgfpathlineto{\pgfqpoint{4.262415in}{4.024487in}}%
\pgfpathlineto{\pgfqpoint{4.284950in}{4.031804in}}%
\pgfpathlineto{\pgfqpoint{4.307485in}{4.039420in}}%
\pgfpathlineto{\pgfqpoint{4.330021in}{4.048863in}}%
\pgfpathlineto{\pgfqpoint{4.352556in}{4.055021in}}%
\pgfpathlineto{\pgfqpoint{4.375091in}{4.066098in}}%
\pgfpathlineto{\pgfqpoint{4.397626in}{4.071784in}}%
\pgfpathlineto{\pgfqpoint{4.420162in}{4.081357in}}%
\pgfpathlineto{\pgfqpoint{4.442697in}{4.091381in}}%
\pgfpathlineto{\pgfqpoint{4.465232in}{4.098828in}}%
\pgfpathlineto{\pgfqpoint{4.487768in}{4.102376in}}%
\pgfpathlineto{\pgfqpoint{4.510303in}{4.104738in}}%
\pgfpathlineto{\pgfqpoint{4.532838in}{4.115132in}}%
\pgfpathlineto{\pgfqpoint{4.555374in}{4.118948in}}%
\pgfpathlineto{\pgfqpoint{4.577909in}{4.128468in}}%
\pgfpathlineto{\pgfqpoint{4.600444in}{4.133624in}}%
\pgfpathlineto{\pgfqpoint{4.622979in}{4.141002in}}%
\pgfpathlineto{\pgfqpoint{4.645515in}{4.144827in}}%
\pgfpathlineto{\pgfqpoint{4.668050in}{4.146668in}}%
\pgfpathlineto{\pgfqpoint{4.690585in}{4.150959in}}%
\pgfpathlineto{\pgfqpoint{4.713121in}{4.155560in}}%
\pgfpathlineto{\pgfqpoint{4.735656in}{4.159352in}}%
\pgfpathlineto{\pgfqpoint{4.758191in}{4.164547in}}%
\pgfpathlineto{\pgfqpoint{4.780726in}{4.167548in}}%
\pgfpathlineto{\pgfqpoint{4.803262in}{4.175941in}}%
\pgfpathlineto{\pgfqpoint{4.825797in}{4.182109in}}%
\pgfpathlineto{\pgfqpoint{4.848332in}{4.187789in}}%
\pgfpathlineto{\pgfqpoint{4.870868in}{4.191186in}}%
\pgfpathlineto{\pgfqpoint{4.893403in}{4.192497in}}%
\pgfpathlineto{\pgfqpoint{4.915938in}{4.192497in}}%
\pgfpathlineto{\pgfqpoint{4.938473in}{4.194140in}}%
\pgfpathlineto{\pgfqpoint{4.961009in}{4.194593in}}%
\pgfpathlineto{\pgfqpoint{4.983544in}{4.194282in}}%
\pgfpathlineto{\pgfqpoint{5.006079in}{4.191844in}}%
\pgfpathlineto{\pgfqpoint{5.028615in}{4.190478in}}%
\pgfpathlineto{\pgfqpoint{5.051150in}{4.189418in}}%
\pgfpathlineto{\pgfqpoint{5.073685in}{4.187314in}}%
\pgfpathlineto{\pgfqpoint{5.096220in}{4.185615in}}%
\pgfpathlineto{\pgfqpoint{5.118756in}{4.184568in}}%
\pgfpathlineto{\pgfqpoint{5.141291in}{4.182774in}}%
\pgfpathlineto{\pgfqpoint{5.163826in}{4.181956in}}%
\pgfpathlineto{\pgfqpoint{5.186362in}{4.179900in}}%
\pgfpathlineto{\pgfqpoint{5.208897in}{4.176319in}}%
\pgfpathlineto{\pgfqpoint{5.231432in}{4.172903in}}%
\pgfpathlineto{\pgfqpoint{5.253967in}{4.169891in}}%
\pgfpathlineto{\pgfqpoint{5.276503in}{4.168308in}}%
\pgfpathlineto{\pgfqpoint{5.299038in}{4.167076in}}%
\pgfpathlineto{\pgfqpoint{5.321573in}{4.165638in}}%
\pgfpathlineto{\pgfqpoint{5.344109in}{4.164540in}}%
\pgfpathlineto{\pgfqpoint{5.366644in}{4.163237in}}%
\pgfpathlineto{\pgfqpoint{5.389179in}{4.160677in}}%
\pgfpathlineto{\pgfqpoint{5.411714in}{4.159892in}}%
\pgfpathlineto{\pgfqpoint{5.434250in}{4.158087in}}%
\pgfpathlineto{\pgfqpoint{5.456785in}{4.155734in}}%
\pgfpathlineto{\pgfqpoint{5.479320in}{4.152053in}}%
\pgfpathlineto{\pgfqpoint{5.501856in}{4.149093in}}%
\pgfpathlineto{\pgfqpoint{5.524391in}{4.146051in}}%
\pgfpathlineto{\pgfqpoint{5.546926in}{4.142719in}}%
\pgfusepath{stroke}%
\end{pgfscope}%
\begin{pgfscope}%
\pgfpathrectangle{\pgfqpoint{3.315933in}{0.422992in}}{\pgfqpoint{2.276064in}{3.951201in}}%
\pgfusepath{clip}%
\pgfsetrectcap%
\pgfsetroundjoin%
\pgfsetlinewidth{1.003750pt}%
\definecolor{currentstroke}{rgb}{0.000000,0.725490,0.270588}%
\pgfsetstrokecolor{currentstroke}%
\pgfsetdash{}{0pt}%
\pgfpathmoveto{\pgfqpoint{3.338468in}{1.382056in}}%
\pgfpathlineto{\pgfqpoint{3.361003in}{1.601272in}}%
\pgfpathlineto{\pgfqpoint{3.383539in}{1.783370in}}%
\pgfpathlineto{\pgfqpoint{3.406074in}{1.933571in}}%
\pgfpathlineto{\pgfqpoint{3.428609in}{2.069620in}}%
\pgfpathlineto{\pgfqpoint{3.451144in}{2.144078in}}%
\pgfpathlineto{\pgfqpoint{3.473680in}{2.226092in}}%
\pgfpathlineto{\pgfqpoint{3.496215in}{2.294560in}}%
\pgfpathlineto{\pgfqpoint{3.518750in}{2.336985in}}%
\pgfpathlineto{\pgfqpoint{3.541286in}{2.368835in}}%
\pgfpathlineto{\pgfqpoint{3.563821in}{2.423996in}}%
\pgfpathlineto{\pgfqpoint{3.586356in}{2.452564in}}%
\pgfpathlineto{\pgfqpoint{3.608891in}{2.484765in}}%
\pgfpathlineto{\pgfqpoint{3.631427in}{2.518330in}}%
\pgfpathlineto{\pgfqpoint{3.653962in}{2.558554in}}%
\pgfpathlineto{\pgfqpoint{3.676497in}{2.598534in}}%
\pgfpathlineto{\pgfqpoint{3.699033in}{2.641177in}}%
\pgfpathlineto{\pgfqpoint{3.721568in}{2.668256in}}%
\pgfpathlineto{\pgfqpoint{3.744103in}{2.683326in}}%
\pgfpathlineto{\pgfqpoint{3.766638in}{2.701064in}}%
\pgfpathlineto{\pgfqpoint{3.789174in}{2.733351in}}%
\pgfpathlineto{\pgfqpoint{3.811709in}{2.758273in}}%
\pgfpathlineto{\pgfqpoint{3.834244in}{2.781330in}}%
\pgfpathlineto{\pgfqpoint{3.856780in}{2.793475in}}%
\pgfpathlineto{\pgfqpoint{3.879315in}{2.816340in}}%
\pgfpathlineto{\pgfqpoint{3.901850in}{2.836642in}}%
\pgfpathlineto{\pgfqpoint{3.924385in}{2.850027in}}%
\pgfpathlineto{\pgfqpoint{3.946921in}{2.862455in}}%
\pgfpathlineto{\pgfqpoint{3.969456in}{2.870186in}}%
\pgfpathlineto{\pgfqpoint{3.991991in}{2.889232in}}%
\pgfpathlineto{\pgfqpoint{4.014527in}{2.900089in}}%
\pgfpathlineto{\pgfqpoint{4.037062in}{2.909397in}}%
\pgfpathlineto{\pgfqpoint{4.059597in}{2.920882in}}%
\pgfpathlineto{\pgfqpoint{4.082132in}{2.932714in}}%
\pgfpathlineto{\pgfqpoint{4.104668in}{2.950631in}}%
\pgfpathlineto{\pgfqpoint{4.127203in}{2.956725in}}%
\pgfpathlineto{\pgfqpoint{4.149738in}{2.973211in}}%
\pgfpathlineto{\pgfqpoint{4.172274in}{2.985533in}}%
\pgfpathlineto{\pgfqpoint{4.194809in}{2.999006in}}%
\pgfpathlineto{\pgfqpoint{4.217344in}{3.012502in}}%
\pgfpathlineto{\pgfqpoint{4.239879in}{3.023811in}}%
\pgfpathlineto{\pgfqpoint{4.262415in}{3.032095in}}%
\pgfpathlineto{\pgfqpoint{4.284950in}{3.040318in}}%
\pgfpathlineto{\pgfqpoint{4.307485in}{3.041523in}}%
\pgfpathlineto{\pgfqpoint{4.330021in}{3.050562in}}%
\pgfpathlineto{\pgfqpoint{4.352556in}{3.058602in}}%
\pgfpathlineto{\pgfqpoint{4.375091in}{3.067040in}}%
\pgfpathlineto{\pgfqpoint{4.397626in}{3.075416in}}%
\pgfpathlineto{\pgfqpoint{4.420162in}{3.085438in}}%
\pgfpathlineto{\pgfqpoint{4.442697in}{3.090605in}}%
\pgfpathlineto{\pgfqpoint{4.465232in}{3.098434in}}%
\pgfpathlineto{\pgfqpoint{4.487768in}{3.104089in}}%
\pgfpathlineto{\pgfqpoint{4.510303in}{3.112418in}}%
\pgfpathlineto{\pgfqpoint{4.532838in}{3.120439in}}%
\pgfpathlineto{\pgfqpoint{4.555374in}{3.126649in}}%
\pgfpathlineto{\pgfqpoint{4.577909in}{3.134253in}}%
\pgfpathlineto{\pgfqpoint{4.600444in}{3.139514in}}%
\pgfpathlineto{\pgfqpoint{4.622979in}{3.147953in}}%
\pgfpathlineto{\pgfqpoint{4.645515in}{3.149382in}}%
\pgfpathlineto{\pgfqpoint{4.668050in}{3.153547in}}%
\pgfpathlineto{\pgfqpoint{4.690585in}{3.156663in}}%
\pgfpathlineto{\pgfqpoint{4.713121in}{3.164729in}}%
\pgfpathlineto{\pgfqpoint{4.735656in}{3.170661in}}%
\pgfpathlineto{\pgfqpoint{4.758191in}{3.174558in}}%
\pgfpathlineto{\pgfqpoint{4.780726in}{3.175551in}}%
\pgfpathlineto{\pgfqpoint{4.803262in}{3.176514in}}%
\pgfpathlineto{\pgfqpoint{4.825797in}{3.180565in}}%
\pgfpathlineto{\pgfqpoint{4.848332in}{3.182039in}}%
\pgfpathlineto{\pgfqpoint{4.870868in}{3.187102in}}%
\pgfpathlineto{\pgfqpoint{4.893403in}{3.189434in}}%
\pgfpathlineto{\pgfqpoint{4.915938in}{3.189055in}}%
\pgfpathlineto{\pgfqpoint{4.938473in}{3.186172in}}%
\pgfpathlineto{\pgfqpoint{4.961009in}{3.185179in}}%
\pgfpathlineto{\pgfqpoint{4.983544in}{3.185906in}}%
\pgfpathlineto{\pgfqpoint{5.006079in}{3.182345in}}%
\pgfpathlineto{\pgfqpoint{5.028615in}{3.181167in}}%
\pgfpathlineto{\pgfqpoint{5.051150in}{3.179748in}}%
\pgfpathlineto{\pgfqpoint{5.073685in}{3.178543in}}%
\pgfpathlineto{\pgfqpoint{5.096220in}{3.176488in}}%
\pgfpathlineto{\pgfqpoint{5.118756in}{3.172135in}}%
\pgfpathlineto{\pgfqpoint{5.141291in}{3.167546in}}%
\pgfpathlineto{\pgfqpoint{5.163826in}{3.164002in}}%
\pgfpathlineto{\pgfqpoint{5.186362in}{3.159201in}}%
\pgfpathlineto{\pgfqpoint{5.208897in}{3.155758in}}%
\pgfpathlineto{\pgfqpoint{5.231432in}{3.151085in}}%
\pgfpathlineto{\pgfqpoint{5.253967in}{3.148220in}}%
\pgfpathlineto{\pgfqpoint{5.276503in}{3.145421in}}%
\pgfpathlineto{\pgfqpoint{5.299038in}{3.141736in}}%
\pgfpathlineto{\pgfqpoint{5.321573in}{3.137743in}}%
\pgfpathlineto{\pgfqpoint{5.344109in}{3.134070in}}%
\pgfpathlineto{\pgfqpoint{5.366644in}{3.131396in}}%
\pgfpathlineto{\pgfqpoint{5.389179in}{3.127191in}}%
\pgfpathlineto{\pgfqpoint{5.411714in}{3.123001in}}%
\pgfpathlineto{\pgfqpoint{5.434250in}{3.119345in}}%
\pgfpathlineto{\pgfqpoint{5.456785in}{3.115985in}}%
\pgfpathlineto{\pgfqpoint{5.479320in}{3.113275in}}%
\pgfpathlineto{\pgfqpoint{5.501856in}{3.108970in}}%
\pgfpathlineto{\pgfqpoint{5.524391in}{3.102694in}}%
\pgfpathlineto{\pgfqpoint{5.546926in}{3.097458in}}%
\pgfusepath{stroke}%
\end{pgfscope}%
\begin{pgfscope}%
\pgfpathrectangle{\pgfqpoint{3.315933in}{0.422992in}}{\pgfqpoint{2.276064in}{3.951201in}}%
\pgfusepath{clip}%
\pgfsetrectcap%
\pgfsetroundjoin%
\pgfsetlinewidth{1.003750pt}%
\definecolor{currentstroke}{rgb}{1.000000,0.584314,0.000000}%
\pgfsetstrokecolor{currentstroke}%
\pgfsetdash{}{0pt}%
\pgfpathmoveto{\pgfqpoint{3.338468in}{1.514287in}}%
\pgfpathlineto{\pgfqpoint{3.361003in}{1.938808in}}%
\pgfpathlineto{\pgfqpoint{3.383539in}{2.277495in}}%
\pgfpathlineto{\pgfqpoint{3.406074in}{2.420736in}}%
\pgfpathlineto{\pgfqpoint{3.428609in}{2.541474in}}%
\pgfpathlineto{\pgfqpoint{3.451144in}{2.592966in}}%
\pgfpathlineto{\pgfqpoint{3.473680in}{2.692379in}}%
\pgfpathlineto{\pgfqpoint{3.496215in}{2.755627in}}%
\pgfpathlineto{\pgfqpoint{3.518750in}{2.802498in}}%
\pgfpathlineto{\pgfqpoint{3.541286in}{2.851129in}}%
\pgfpathlineto{\pgfqpoint{3.563821in}{2.898508in}}%
\pgfpathlineto{\pgfqpoint{3.586356in}{2.936829in}}%
\pgfpathlineto{\pgfqpoint{3.608891in}{2.993879in}}%
\pgfpathlineto{\pgfqpoint{3.631427in}{3.045264in}}%
\pgfpathlineto{\pgfqpoint{3.653962in}{3.097683in}}%
\pgfpathlineto{\pgfqpoint{3.676497in}{3.117451in}}%
\pgfpathlineto{\pgfqpoint{3.699033in}{3.162321in}}%
\pgfpathlineto{\pgfqpoint{3.721568in}{3.208777in}}%
\pgfpathlineto{\pgfqpoint{3.744103in}{3.228365in}}%
\pgfpathlineto{\pgfqpoint{3.766638in}{3.245993in}}%
\pgfpathlineto{\pgfqpoint{3.789174in}{3.274535in}}%
\pgfpathlineto{\pgfqpoint{3.811709in}{3.303645in}}%
\pgfpathlineto{\pgfqpoint{3.834244in}{3.316606in}}%
\pgfpathlineto{\pgfqpoint{3.856780in}{3.328487in}}%
\pgfpathlineto{\pgfqpoint{3.879315in}{3.356119in}}%
\pgfpathlineto{\pgfqpoint{3.901850in}{3.389655in}}%
\pgfpathlineto{\pgfqpoint{3.924385in}{3.414004in}}%
\pgfpathlineto{\pgfqpoint{3.946921in}{3.438602in}}%
\pgfpathlineto{\pgfqpoint{3.969456in}{3.453344in}}%
\pgfpathlineto{\pgfqpoint{3.991991in}{3.465710in}}%
\pgfpathlineto{\pgfqpoint{4.014527in}{3.483564in}}%
\pgfpathlineto{\pgfqpoint{4.037062in}{3.491603in}}%
\pgfpathlineto{\pgfqpoint{4.059597in}{3.508433in}}%
\pgfpathlineto{\pgfqpoint{4.082132in}{3.526933in}}%
\pgfpathlineto{\pgfqpoint{4.104668in}{3.544177in}}%
\pgfpathlineto{\pgfqpoint{4.127203in}{3.556016in}}%
\pgfpathlineto{\pgfqpoint{4.149738in}{3.572857in}}%
\pgfpathlineto{\pgfqpoint{4.172274in}{3.583134in}}%
\pgfpathlineto{\pgfqpoint{4.194809in}{3.595917in}}%
\pgfpathlineto{\pgfqpoint{4.217344in}{3.608061in}}%
\pgfpathlineto{\pgfqpoint{4.239879in}{3.619272in}}%
\pgfpathlineto{\pgfqpoint{4.262415in}{3.632932in}}%
\pgfpathlineto{\pgfqpoint{4.284950in}{3.644823in}}%
\pgfpathlineto{\pgfqpoint{4.307485in}{3.654117in}}%
\pgfpathlineto{\pgfqpoint{4.330021in}{3.658977in}}%
\pgfpathlineto{\pgfqpoint{4.352556in}{3.668163in}}%
\pgfpathlineto{\pgfqpoint{4.375091in}{3.680512in}}%
\pgfpathlineto{\pgfqpoint{4.397626in}{3.695681in}}%
\pgfpathlineto{\pgfqpoint{4.420162in}{3.706254in}}%
\pgfpathlineto{\pgfqpoint{4.442697in}{3.711810in}}%
\pgfpathlineto{\pgfqpoint{4.465232in}{3.722879in}}%
\pgfpathlineto{\pgfqpoint{4.487768in}{3.729775in}}%
\pgfpathlineto{\pgfqpoint{4.510303in}{3.735491in}}%
\pgfpathlineto{\pgfqpoint{4.532838in}{3.740995in}}%
\pgfpathlineto{\pgfqpoint{4.555374in}{3.748576in}}%
\pgfpathlineto{\pgfqpoint{4.577909in}{3.756259in}}%
\pgfpathlineto{\pgfqpoint{4.600444in}{3.766724in}}%
\pgfpathlineto{\pgfqpoint{4.622979in}{3.777668in}}%
\pgfpathlineto{\pgfqpoint{4.645515in}{3.788359in}}%
\pgfpathlineto{\pgfqpoint{4.668050in}{3.796373in}}%
\pgfpathlineto{\pgfqpoint{4.690585in}{3.804239in}}%
\pgfpathlineto{\pgfqpoint{4.713121in}{3.812411in}}%
\pgfpathlineto{\pgfqpoint{4.735656in}{3.821871in}}%
\pgfpathlineto{\pgfqpoint{4.758191in}{3.823640in}}%
\pgfpathlineto{\pgfqpoint{4.780726in}{3.826746in}}%
\pgfpathlineto{\pgfqpoint{4.803262in}{3.834398in}}%
\pgfpathlineto{\pgfqpoint{4.825797in}{3.841302in}}%
\pgfpathlineto{\pgfqpoint{4.848332in}{3.847183in}}%
\pgfpathlineto{\pgfqpoint{4.870868in}{3.856828in}}%
\pgfpathlineto{\pgfqpoint{4.893403in}{3.863711in}}%
\pgfpathlineto{\pgfqpoint{4.915938in}{3.866479in}}%
\pgfpathlineto{\pgfqpoint{4.938473in}{3.869943in}}%
\pgfpathlineto{\pgfqpoint{4.961009in}{3.871596in}}%
\pgfpathlineto{\pgfqpoint{4.983544in}{3.870477in}}%
\pgfpathlineto{\pgfqpoint{5.006079in}{3.866882in}}%
\pgfpathlineto{\pgfqpoint{5.028615in}{3.867685in}}%
\pgfpathlineto{\pgfqpoint{5.051150in}{3.868557in}}%
\pgfpathlineto{\pgfqpoint{5.073685in}{3.866641in}}%
\pgfpathlineto{\pgfqpoint{5.096220in}{3.863805in}}%
\pgfpathlineto{\pgfqpoint{5.118756in}{3.864257in}}%
\pgfpathlineto{\pgfqpoint{5.141291in}{3.861434in}}%
\pgfpathlineto{\pgfqpoint{5.163826in}{3.858764in}}%
\pgfpathlineto{\pgfqpoint{5.186362in}{3.853810in}}%
\pgfpathlineto{\pgfqpoint{5.208897in}{3.853448in}}%
\pgfpathlineto{\pgfqpoint{5.231432in}{3.852275in}}%
\pgfpathlineto{\pgfqpoint{5.253967in}{3.847974in}}%
\pgfpathlineto{\pgfqpoint{5.276503in}{3.846651in}}%
\pgfpathlineto{\pgfqpoint{5.299038in}{3.843380in}}%
\pgfpathlineto{\pgfqpoint{5.321573in}{3.845266in}}%
\pgfpathlineto{\pgfqpoint{5.344109in}{3.846800in}}%
\pgfpathlineto{\pgfqpoint{5.366644in}{3.845777in}}%
\pgfpathlineto{\pgfqpoint{5.389179in}{3.844171in}}%
\pgfpathlineto{\pgfqpoint{5.411714in}{3.842524in}}%
\pgfpathlineto{\pgfqpoint{5.434250in}{3.840764in}}%
\pgfpathlineto{\pgfqpoint{5.456785in}{3.837062in}}%
\pgfpathlineto{\pgfqpoint{5.479320in}{3.832786in}}%
\pgfpathlineto{\pgfqpoint{5.501856in}{3.828095in}}%
\pgfpathlineto{\pgfqpoint{5.524391in}{3.825204in}}%
\pgfpathlineto{\pgfqpoint{5.546926in}{3.822511in}}%
\pgfusepath{stroke}%
\end{pgfscope}%
\begin{pgfscope}%
\pgfpathrectangle{\pgfqpoint{3.315933in}{0.422992in}}{\pgfqpoint{2.276064in}{3.951201in}}%
\pgfusepath{clip}%
\pgfsetrectcap%
\pgfsetroundjoin%
\pgfsetlinewidth{1.003750pt}%
\definecolor{currentstroke}{rgb}{1.000000,0.172549,0.000000}%
\pgfsetstrokecolor{currentstroke}%
\pgfsetdash{}{0pt}%
\pgfpathmoveto{\pgfqpoint{3.338468in}{0.950567in}}%
\pgfpathlineto{\pgfqpoint{3.361003in}{1.235898in}}%
\pgfpathlineto{\pgfqpoint{3.383539in}{1.419156in}}%
\pgfpathlineto{\pgfqpoint{3.406074in}{1.528180in}}%
\pgfpathlineto{\pgfqpoint{3.428609in}{1.646483in}}%
\pgfpathlineto{\pgfqpoint{3.451144in}{1.721868in}}%
\pgfpathlineto{\pgfqpoint{3.473680in}{1.756823in}}%
\pgfpathlineto{\pgfqpoint{3.496215in}{1.816094in}}%
\pgfpathlineto{\pgfqpoint{3.518750in}{1.869152in}}%
\pgfpathlineto{\pgfqpoint{3.541286in}{1.902549in}}%
\pgfpathlineto{\pgfqpoint{3.563821in}{1.934301in}}%
\pgfpathlineto{\pgfqpoint{3.586356in}{1.970618in}}%
\pgfpathlineto{\pgfqpoint{3.608891in}{1.992782in}}%
\pgfpathlineto{\pgfqpoint{3.631427in}{2.004321in}}%
\pgfpathlineto{\pgfqpoint{3.653962in}{2.029632in}}%
\pgfpathlineto{\pgfqpoint{3.676497in}{2.063957in}}%
\pgfpathlineto{\pgfqpoint{3.699033in}{2.088920in}}%
\pgfpathlineto{\pgfqpoint{3.721568in}{2.117296in}}%
\pgfpathlineto{\pgfqpoint{3.744103in}{2.132793in}}%
\pgfpathlineto{\pgfqpoint{3.766638in}{2.151264in}}%
\pgfpathlineto{\pgfqpoint{3.789174in}{2.176922in}}%
\pgfpathlineto{\pgfqpoint{3.811709in}{2.206891in}}%
\pgfpathlineto{\pgfqpoint{3.834244in}{2.231832in}}%
\pgfpathlineto{\pgfqpoint{3.856780in}{2.250054in}}%
\pgfpathlineto{\pgfqpoint{3.879315in}{2.271550in}}%
\pgfpathlineto{\pgfqpoint{3.901850in}{2.292998in}}%
\pgfpathlineto{\pgfqpoint{3.924385in}{2.309248in}}%
\pgfpathlineto{\pgfqpoint{3.946921in}{2.326325in}}%
\pgfpathlineto{\pgfqpoint{3.969456in}{2.345583in}}%
\pgfpathlineto{\pgfqpoint{3.991991in}{2.361006in}}%
\pgfpathlineto{\pgfqpoint{4.014527in}{2.373187in}}%
\pgfpathlineto{\pgfqpoint{4.037062in}{2.385912in}}%
\pgfpathlineto{\pgfqpoint{4.059597in}{2.398919in}}%
\pgfpathlineto{\pgfqpoint{4.082132in}{2.409319in}}%
\pgfpathlineto{\pgfqpoint{4.104668in}{2.415544in}}%
\pgfpathlineto{\pgfqpoint{4.127203in}{2.422197in}}%
\pgfpathlineto{\pgfqpoint{4.149738in}{2.431875in}}%
\pgfpathlineto{\pgfqpoint{4.172274in}{2.442325in}}%
\pgfpathlineto{\pgfqpoint{4.194809in}{2.450812in}}%
\pgfpathlineto{\pgfqpoint{4.217344in}{2.461135in}}%
\pgfpathlineto{\pgfqpoint{4.239879in}{2.465014in}}%
\pgfpathlineto{\pgfqpoint{4.262415in}{2.470033in}}%
\pgfpathlineto{\pgfqpoint{4.284950in}{2.470933in}}%
\pgfpathlineto{\pgfqpoint{4.307485in}{2.469737in}}%
\pgfpathlineto{\pgfqpoint{4.330021in}{2.471995in}}%
\pgfpathlineto{\pgfqpoint{4.352556in}{2.474911in}}%
\pgfpathlineto{\pgfqpoint{4.375091in}{2.472521in}}%
\pgfpathlineto{\pgfqpoint{4.397626in}{2.466314in}}%
\pgfpathlineto{\pgfqpoint{4.420162in}{2.466610in}}%
\pgfpathlineto{\pgfqpoint{4.442697in}{2.461466in}}%
\pgfpathlineto{\pgfqpoint{4.465232in}{2.457751in}}%
\pgfpathlineto{\pgfqpoint{4.487768in}{2.452305in}}%
\pgfpathlineto{\pgfqpoint{4.510303in}{2.447720in}}%
\pgfpathlineto{\pgfqpoint{4.532838in}{2.444465in}}%
\pgfpathlineto{\pgfqpoint{4.555374in}{2.442720in}}%
\pgfpathlineto{\pgfqpoint{4.577909in}{2.439297in}}%
\pgfpathlineto{\pgfqpoint{4.600444in}{2.431354in}}%
\pgfpathlineto{\pgfqpoint{4.622979in}{2.426085in}}%
\pgfpathlineto{\pgfqpoint{4.645515in}{2.416512in}}%
\pgfpathlineto{\pgfqpoint{4.668050in}{2.406909in}}%
\pgfpathlineto{\pgfqpoint{4.690585in}{2.398305in}}%
\pgfpathlineto{\pgfqpoint{4.713121in}{2.390877in}}%
\pgfpathlineto{\pgfqpoint{4.735656in}{2.383685in}}%
\pgfpathlineto{\pgfqpoint{4.758191in}{2.375520in}}%
\pgfpathlineto{\pgfqpoint{4.780726in}{2.366322in}}%
\pgfpathlineto{\pgfqpoint{4.803262in}{2.354555in}}%
\pgfpathlineto{\pgfqpoint{4.825797in}{2.347190in}}%
\pgfpathlineto{\pgfqpoint{4.848332in}{2.340758in}}%
\pgfpathlineto{\pgfqpoint{4.870868in}{2.331386in}}%
\pgfpathlineto{\pgfqpoint{4.893403in}{2.325164in}}%
\pgfpathlineto{\pgfqpoint{4.915938in}{2.315980in}}%
\pgfpathlineto{\pgfqpoint{4.938473in}{2.306278in}}%
\pgfpathlineto{\pgfqpoint{4.961009in}{2.297033in}}%
\pgfpathlineto{\pgfqpoint{4.983544in}{2.288601in}}%
\pgfpathlineto{\pgfqpoint{5.006079in}{2.278260in}}%
\pgfpathlineto{\pgfqpoint{5.028615in}{2.267366in}}%
\pgfpathlineto{\pgfqpoint{5.051150in}{2.256213in}}%
\pgfpathlineto{\pgfqpoint{5.073685in}{2.246327in}}%
\pgfpathlineto{\pgfqpoint{5.096220in}{2.237131in}}%
\pgfpathlineto{\pgfqpoint{5.118756in}{2.227731in}}%
\pgfpathlineto{\pgfqpoint{5.141291in}{2.216586in}}%
\pgfpathlineto{\pgfqpoint{5.163826in}{2.206985in}}%
\pgfpathlineto{\pgfqpoint{5.186362in}{2.196442in}}%
\pgfpathlineto{\pgfqpoint{5.208897in}{2.187393in}}%
\pgfpathlineto{\pgfqpoint{5.231432in}{2.178556in}}%
\pgfpathlineto{\pgfqpoint{5.253967in}{2.166445in}}%
\pgfpathlineto{\pgfqpoint{5.276503in}{2.157972in}}%
\pgfpathlineto{\pgfqpoint{5.299038in}{2.150798in}}%
\pgfpathlineto{\pgfqpoint{5.321573in}{2.140110in}}%
\pgfpathlineto{\pgfqpoint{5.344109in}{2.130742in}}%
\pgfpathlineto{\pgfqpoint{5.366644in}{2.120433in}}%
\pgfpathlineto{\pgfqpoint{5.389179in}{2.111331in}}%
\pgfpathlineto{\pgfqpoint{5.411714in}{2.103247in}}%
\pgfpathlineto{\pgfqpoint{5.434250in}{2.095039in}}%
\pgfpathlineto{\pgfqpoint{5.456785in}{2.088250in}}%
\pgfpathlineto{\pgfqpoint{5.479320in}{2.081819in}}%
\pgfpathlineto{\pgfqpoint{5.501856in}{2.072005in}}%
\pgfpathlineto{\pgfqpoint{5.524391in}{2.062035in}}%
\pgfpathlineto{\pgfqpoint{5.546926in}{2.053603in}}%
\pgfusepath{stroke}%
\end{pgfscope}%
\begin{pgfscope}%
\pgfpathrectangle{\pgfqpoint{3.315933in}{0.422992in}}{\pgfqpoint{2.276064in}{3.951201in}}%
\pgfusepath{clip}%
\pgfsetrectcap%
\pgfsetroundjoin%
\pgfsetlinewidth{1.003750pt}%
\definecolor{currentstroke}{rgb}{0.517647,0.356863,0.592157}%
\pgfsetstrokecolor{currentstroke}%
\pgfsetdash{}{0pt}%
\pgfpathmoveto{\pgfqpoint{3.338468in}{0.602592in}}%
\pgfpathlineto{\pgfqpoint{3.361003in}{0.713936in}}%
\pgfpathlineto{\pgfqpoint{3.383539in}{0.864716in}}%
\pgfpathlineto{\pgfqpoint{3.406074in}{0.950541in}}%
\pgfpathlineto{\pgfqpoint{3.428609in}{1.052141in}}%
\pgfpathlineto{\pgfqpoint{3.451144in}{1.131471in}}%
\pgfpathlineto{\pgfqpoint{3.473680in}{1.249774in}}%
\pgfpathlineto{\pgfqpoint{3.496215in}{1.292392in}}%
\pgfpathlineto{\pgfqpoint{3.518750in}{1.353375in}}%
\pgfpathlineto{\pgfqpoint{3.541286in}{1.407728in}}%
\pgfpathlineto{\pgfqpoint{3.563821in}{1.462320in}}%
\pgfpathlineto{\pgfqpoint{3.586356in}{1.508392in}}%
\pgfpathlineto{\pgfqpoint{3.608891in}{1.559151in}}%
\pgfpathlineto{\pgfqpoint{3.631427in}{1.622046in}}%
\pgfpathlineto{\pgfqpoint{3.653962in}{1.668666in}}%
\pgfpathlineto{\pgfqpoint{3.676497in}{1.706847in}}%
\pgfpathlineto{\pgfqpoint{3.699033in}{1.731530in}}%
\pgfpathlineto{\pgfqpoint{3.721568in}{1.755015in}}%
\pgfpathlineto{\pgfqpoint{3.744103in}{1.799470in}}%
\pgfpathlineto{\pgfqpoint{3.766638in}{1.834606in}}%
\pgfpathlineto{\pgfqpoint{3.789174in}{1.861756in}}%
\pgfpathlineto{\pgfqpoint{3.811709in}{1.903520in}}%
\pgfpathlineto{\pgfqpoint{3.834244in}{1.939533in}}%
\pgfpathlineto{\pgfqpoint{3.856780in}{1.974284in}}%
\pgfpathlineto{\pgfqpoint{3.879315in}{2.011543in}}%
\pgfpathlineto{\pgfqpoint{3.901850in}{2.051021in}}%
\pgfpathlineto{\pgfqpoint{3.924385in}{2.095308in}}%
\pgfpathlineto{\pgfqpoint{3.946921in}{2.113066in}}%
\pgfpathlineto{\pgfqpoint{3.969456in}{2.137518in}}%
\pgfpathlineto{\pgfqpoint{3.991991in}{2.166140in}}%
\pgfpathlineto{\pgfqpoint{4.014527in}{2.198526in}}%
\pgfpathlineto{\pgfqpoint{4.037062in}{2.225843in}}%
\pgfpathlineto{\pgfqpoint{4.059597in}{2.254457in}}%
\pgfpathlineto{\pgfqpoint{4.082132in}{2.295715in}}%
\pgfpathlineto{\pgfqpoint{4.104668in}{2.322287in}}%
\pgfpathlineto{\pgfqpoint{4.127203in}{2.351829in}}%
\pgfpathlineto{\pgfqpoint{4.149738in}{2.375447in}}%
\pgfpathlineto{\pgfqpoint{4.172274in}{2.400935in}}%
\pgfpathlineto{\pgfqpoint{4.194809in}{2.426007in}}%
\pgfpathlineto{\pgfqpoint{4.217344in}{2.446346in}}%
\pgfpathlineto{\pgfqpoint{4.239879in}{2.471804in}}%
\pgfpathlineto{\pgfqpoint{4.262415in}{2.495219in}}%
\pgfpathlineto{\pgfqpoint{4.284950in}{2.508968in}}%
\pgfpathlineto{\pgfqpoint{4.307485in}{2.524305in}}%
\pgfpathlineto{\pgfqpoint{4.330021in}{2.533857in}}%
\pgfpathlineto{\pgfqpoint{4.352556in}{2.548289in}}%
\pgfpathlineto{\pgfqpoint{4.375091in}{2.567140in}}%
\pgfpathlineto{\pgfqpoint{4.397626in}{2.589410in}}%
\pgfpathlineto{\pgfqpoint{4.420162in}{2.604096in}}%
\pgfpathlineto{\pgfqpoint{4.442697in}{2.630025in}}%
\pgfpathlineto{\pgfqpoint{4.465232in}{2.644566in}}%
\pgfpathlineto{\pgfqpoint{4.487768in}{2.659751in}}%
\pgfpathlineto{\pgfqpoint{4.510303in}{2.673707in}}%
\pgfpathlineto{\pgfqpoint{4.532838in}{2.698358in}}%
\pgfpathlineto{\pgfqpoint{4.555374in}{2.716418in}}%
\pgfpathlineto{\pgfqpoint{4.577909in}{2.738556in}}%
\pgfpathlineto{\pgfqpoint{4.600444in}{2.759550in}}%
\pgfpathlineto{\pgfqpoint{4.622979in}{2.779820in}}%
\pgfpathlineto{\pgfqpoint{4.645515in}{2.804474in}}%
\pgfpathlineto{\pgfqpoint{4.668050in}{2.820883in}}%
\pgfpathlineto{\pgfqpoint{4.690585in}{2.838351in}}%
\pgfpathlineto{\pgfqpoint{4.713121in}{2.848072in}}%
\pgfpathlineto{\pgfqpoint{4.735656in}{2.862675in}}%
\pgfpathlineto{\pgfqpoint{4.758191in}{2.877148in}}%
\pgfpathlineto{\pgfqpoint{4.780726in}{2.889783in}}%
\pgfpathlineto{\pgfqpoint{4.803262in}{2.902563in}}%
\pgfpathlineto{\pgfqpoint{4.825797in}{2.918285in}}%
\pgfpathlineto{\pgfqpoint{4.848332in}{2.930269in}}%
\pgfpathlineto{\pgfqpoint{4.870868in}{2.938577in}}%
\pgfpathlineto{\pgfqpoint{4.893403in}{2.954402in}}%
\pgfpathlineto{\pgfqpoint{4.915938in}{2.959195in}}%
\pgfpathlineto{\pgfqpoint{4.938473in}{2.971587in}}%
\pgfpathlineto{\pgfqpoint{4.961009in}{2.975536in}}%
\pgfpathlineto{\pgfqpoint{4.983544in}{2.975899in}}%
\pgfpathlineto{\pgfqpoint{5.006079in}{2.978385in}}%
\pgfpathlineto{\pgfqpoint{5.028615in}{2.985385in}}%
\pgfpathlineto{\pgfqpoint{5.051150in}{2.991480in}}%
\pgfpathlineto{\pgfqpoint{5.073685in}{3.000005in}}%
\pgfpathlineto{\pgfqpoint{5.096220in}{3.006024in}}%
\pgfpathlineto{\pgfqpoint{5.118756in}{3.011109in}}%
\pgfpathlineto{\pgfqpoint{5.141291in}{3.016241in}}%
\pgfpathlineto{\pgfqpoint{5.163826in}{3.021417in}}%
\pgfpathlineto{\pgfqpoint{5.186362in}{3.025294in}}%
\pgfpathlineto{\pgfqpoint{5.208897in}{3.030818in}}%
\pgfpathlineto{\pgfqpoint{5.231432in}{3.036785in}}%
\pgfpathlineto{\pgfqpoint{5.253967in}{3.038891in}}%
\pgfpathlineto{\pgfqpoint{5.276503in}{3.042948in}}%
\pgfpathlineto{\pgfqpoint{5.299038in}{3.045173in}}%
\pgfpathlineto{\pgfqpoint{5.321573in}{3.046878in}}%
\pgfpathlineto{\pgfqpoint{5.344109in}{3.052180in}}%
\pgfpathlineto{\pgfqpoint{5.366644in}{3.052165in}}%
\pgfpathlineto{\pgfqpoint{5.389179in}{3.054494in}}%
\pgfpathlineto{\pgfqpoint{5.411714in}{3.060815in}}%
\pgfpathlineto{\pgfqpoint{5.434250in}{3.064705in}}%
\pgfpathlineto{\pgfqpoint{5.456785in}{3.069246in}}%
\pgfpathlineto{\pgfqpoint{5.479320in}{3.070793in}}%
\pgfpathlineto{\pgfqpoint{5.501856in}{3.069581in}}%
\pgfpathlineto{\pgfqpoint{5.524391in}{3.068820in}}%
\pgfpathlineto{\pgfqpoint{5.546926in}{3.063926in}}%
\pgfusepath{stroke}%
\end{pgfscope}%
\begin{pgfscope}%
\pgfsetrectcap%
\pgfsetmiterjoin%
\pgfsetlinewidth{0.501875pt}%
\definecolor{currentstroke}{rgb}{0.000000,0.000000,0.000000}%
\pgfsetstrokecolor{currentstroke}%
\pgfsetdash{}{0pt}%
\pgfpathmoveto{\pgfqpoint{3.315933in}{0.422992in}}%
\pgfpathlineto{\pgfqpoint{3.315933in}{4.374193in}}%
\pgfusepath{stroke}%
\end{pgfscope}%
\begin{pgfscope}%
\pgfsetrectcap%
\pgfsetmiterjoin%
\pgfsetlinewidth{0.501875pt}%
\definecolor{currentstroke}{rgb}{0.000000,0.000000,0.000000}%
\pgfsetstrokecolor{currentstroke}%
\pgfsetdash{}{0pt}%
\pgfpathmoveto{\pgfqpoint{5.591997in}{0.422992in}}%
\pgfpathlineto{\pgfqpoint{5.591997in}{4.374193in}}%
\pgfusepath{stroke}%
\end{pgfscope}%
\begin{pgfscope}%
\pgfsetrectcap%
\pgfsetmiterjoin%
\pgfsetlinewidth{0.501875pt}%
\definecolor{currentstroke}{rgb}{0.000000,0.000000,0.000000}%
\pgfsetstrokecolor{currentstroke}%
\pgfsetdash{}{0pt}%
\pgfpathmoveto{\pgfqpoint{3.315933in}{0.422992in}}%
\pgfpathlineto{\pgfqpoint{5.591997in}{0.422992in}}%
\pgfusepath{stroke}%
\end{pgfscope}%
\begin{pgfscope}%
\pgfsetrectcap%
\pgfsetmiterjoin%
\pgfsetlinewidth{0.501875pt}%
\definecolor{currentstroke}{rgb}{0.000000,0.000000,0.000000}%
\pgfsetstrokecolor{currentstroke}%
\pgfsetdash{}{0pt}%
\pgfpathmoveto{\pgfqpoint{3.315933in}{4.374193in}}%
\pgfpathlineto{\pgfqpoint{5.591997in}{4.374193in}}%
\pgfusepath{stroke}%
\end{pgfscope}%
\begin{pgfscope}%
\definecolor{textcolor}{rgb}{0.000000,0.000000,0.000000}%
\pgfsetstrokecolor{textcolor}%
\pgfsetfillcolor{textcolor}%
\pgftext[x=4.453965in,y=4.457526in,,base]{\color{textcolor}\rmfamily\fontsize{12.000000}{14.400000}\selectfont LCMC}%
\end{pgfscope}%
\begin{pgfscope}%
\pgfsetrectcap%
\pgfsetroundjoin%
\pgfsetlinewidth{1.003750pt}%
\definecolor{currentstroke}{rgb}{0.047059,0.364706,0.647059}%
\pgfsetstrokecolor{currentstroke}%
\pgfsetdash{}{0pt}%
\pgfpathmoveto{\pgfqpoint{4.313405in}{1.440922in}}%
\pgfpathlineto{\pgfqpoint{4.452294in}{1.440922in}}%
\pgfpathlineto{\pgfqpoint{4.591183in}{1.440922in}}%
\pgfusepath{stroke}%
\end{pgfscope}%
\begin{pgfscope}%
\definecolor{textcolor}{rgb}{0.000000,0.000000,0.000000}%
\pgfsetstrokecolor{textcolor}%
\pgfsetfillcolor{textcolor}%
\pgftext[x=4.702294in,y=1.392311in,left,base]{\color{textcolor}\rmfamily\fontsize{10.000000}{12.000000}\selectfont PCA}%
\end{pgfscope}%
\begin{pgfscope}%
\pgfsetrectcap%
\pgfsetroundjoin%
\pgfsetlinewidth{1.003750pt}%
\definecolor{currentstroke}{rgb}{0.000000,0.725490,0.270588}%
\pgfsetstrokecolor{currentstroke}%
\pgfsetdash{}{0pt}%
\pgfpathmoveto{\pgfqpoint{4.313405in}{1.237065in}}%
\pgfpathlineto{\pgfqpoint{4.452294in}{1.237065in}}%
\pgfpathlineto{\pgfqpoint{4.591183in}{1.237065in}}%
\pgfusepath{stroke}%
\end{pgfscope}%
\begin{pgfscope}%
\definecolor{textcolor}{rgb}{0.000000,0.000000,0.000000}%
\pgfsetstrokecolor{textcolor}%
\pgfsetfillcolor{textcolor}%
\pgftext[x=4.702294in,y=1.188454in,left,base]{\color{textcolor}\rmfamily\fontsize{10.000000}{12.000000}\selectfont KernelPCA}%
\end{pgfscope}%
\begin{pgfscope}%
\pgfsetrectcap%
\pgfsetroundjoin%
\pgfsetlinewidth{1.003750pt}%
\definecolor{currentstroke}{rgb}{1.000000,0.584314,0.000000}%
\pgfsetstrokecolor{currentstroke}%
\pgfsetdash{}{0pt}%
\pgfpathmoveto{\pgfqpoint{4.313405in}{1.033208in}}%
\pgfpathlineto{\pgfqpoint{4.452294in}{1.033208in}}%
\pgfpathlineto{\pgfqpoint{4.591183in}{1.033208in}}%
\pgfusepath{stroke}%
\end{pgfscope}%
\begin{pgfscope}%
\definecolor{textcolor}{rgb}{0.000000,0.000000,0.000000}%
\pgfsetstrokecolor{textcolor}%
\pgfsetfillcolor{textcolor}%
\pgftext[x=4.702294in,y=0.984596in,left,base]{\color{textcolor}\rmfamily\fontsize{10.000000}{12.000000}\selectfont AE}%
\end{pgfscope}%
\begin{pgfscope}%
\pgfsetrectcap%
\pgfsetroundjoin%
\pgfsetlinewidth{1.003750pt}%
\definecolor{currentstroke}{rgb}{1.000000,0.172549,0.000000}%
\pgfsetstrokecolor{currentstroke}%
\pgfsetdash{}{0pt}%
\pgfpathmoveto{\pgfqpoint{4.313405in}{0.829350in}}%
\pgfpathlineto{\pgfqpoint{4.452294in}{0.829350in}}%
\pgfpathlineto{\pgfqpoint{4.591183in}{0.829350in}}%
\pgfusepath{stroke}%
\end{pgfscope}%
\begin{pgfscope}%
\definecolor{textcolor}{rgb}{0.000000,0.000000,0.000000}%
\pgfsetstrokecolor{textcolor}%
\pgfsetfillcolor{textcolor}%
\pgftext[x=4.702294in,y=0.780739in,left,base]{\color{textcolor}\rmfamily\fontsize{10.000000}{12.000000}\selectfont LLE}%
\end{pgfscope}%
\begin{pgfscope}%
\pgfsetrectcap%
\pgfsetroundjoin%
\pgfsetlinewidth{1.003750pt}%
\definecolor{currentstroke}{rgb}{0.517647,0.356863,0.592157}%
\pgfsetstrokecolor{currentstroke}%
\pgfsetdash{}{0pt}%
\pgfpathmoveto{\pgfqpoint{4.313405in}{0.625493in}}%
\pgfpathlineto{\pgfqpoint{4.452294in}{0.625493in}}%
\pgfpathlineto{\pgfqpoint{4.591183in}{0.625493in}}%
\pgfusepath{stroke}%
\end{pgfscope}%
\begin{pgfscope}%
\definecolor{textcolor}{rgb}{0.000000,0.000000,0.000000}%
\pgfsetstrokecolor{textcolor}%
\pgfsetfillcolor{textcolor}%
\pgftext[x=4.702294in,y=0.576882in,left,base]{\color{textcolor}\rmfamily\fontsize{10.000000}{12.000000}\selectfont CAE}%
\end{pgfscope}%
\end{pgfpicture}%
\makeatother%
\endgroup%

	\end{center}
	\caption[ICMR Qualitätskriterien]{Die Vertrauenswürdigkeit und Kontinuität der Dimensionsreduktion, sowie das Local Continuity Meta-Criterion (LCMC) für den ICMR Datensatz. Dieser Datensatz ist kein Bilddatensatz, weswegen hier der Convolutional Autoencoder nicht eingesetzt werden konnte. Auch hier schneidet die Hauptkomponentenanalyse wieder am besten ab, gefolgt vom vollvernetzten Autoencoder (AE) und der Kernel PCA. Am schlechtesten sind der Contractive Autoencoder und Locally Linear Embedding. (Eigene Darstellung)}
	\label{fig:ICMRMetrics}
\end{figure}

\begin{figure}[ht]
	\begin{center}
		%% Creator: Matplotlib, PGF backend
%%
%% To include the figure in your LaTeX document, write
%%   \input{<filename>.pgf}
%%
%% Make sure the required packages are loaded in your preamble
%%   \usepackage{pgf}
%%
%% Also ensure that all the required font packages are loaded; for instance,
%% the lmodern package is sometimes necessary when using math font.
%%   \usepackage{lmodern}
%%
%% Figures using additional raster images can only be included by \input if
%% they are in the same directory as the main LaTeX file. For loading figures
%% from other directories you can use the `import` package
%%   \usepackage{import}
%%
%% and then include the figures with
%%   \import{<path to file>}{<filename>.pgf}
%%
%% Matplotlib used the following preamble
%%   
%%   \usepackage{fontspec}
%%   \setmainfont{DejaVuSerif.ttf}[Path=\detokenize{/Users/moritzmistol/.pyenv/versions/3.9.13/envs/thesis/lib/python3.9/site-packages/matplotlib/mpl-data/fonts/ttf/}]
%%   \setsansfont{DejaVuSans.ttf}[Path=\detokenize{/Users/moritzmistol/.pyenv/versions/3.9.13/envs/thesis/lib/python3.9/site-packages/matplotlib/mpl-data/fonts/ttf/}]
%%   \setmonofont{DejaVuSansMono.ttf}[Path=\detokenize{/Users/moritzmistol/.pyenv/versions/3.9.13/envs/thesis/lib/python3.9/site-packages/matplotlib/mpl-data/fonts/ttf/}]
%%   \makeatletter\@ifpackageloaded{underscore}{}{\usepackage[strings]{underscore}}\makeatother
%%
\begingroup%
\makeatletter%
\begin{pgfpicture}%
\pgfpathrectangle{\pgfpointorigin}{\pgfqpoint{5.711441in}{4.634154in}}%
\pgfusepath{use as bounding box, clip}%
\begin{pgfscope}%
\pgfsetbuttcap%
\pgfsetmiterjoin%
\definecolor{currentfill}{rgb}{1.000000,1.000000,1.000000}%
\pgfsetfillcolor{currentfill}%
\pgfsetlinewidth{0.000000pt}%
\definecolor{currentstroke}{rgb}{1.000000,1.000000,1.000000}%
\pgfsetstrokecolor{currentstroke}%
\pgfsetdash{}{0pt}%
\pgfpathmoveto{\pgfqpoint{-0.000000in}{-0.000000in}}%
\pgfpathlineto{\pgfqpoint{5.711441in}{-0.000000in}}%
\pgfpathlineto{\pgfqpoint{5.711441in}{4.634154in}}%
\pgfpathlineto{\pgfqpoint{-0.000000in}{4.634154in}}%
\pgfpathlineto{\pgfqpoint{-0.000000in}{-0.000000in}}%
\pgfpathclose%
\pgfusepath{fill}%
\end{pgfscope}%
\begin{pgfscope}%
\pgfsetbuttcap%
\pgfsetmiterjoin%
\definecolor{currentfill}{rgb}{1.000000,1.000000,1.000000}%
\pgfsetfillcolor{currentfill}%
\pgfsetlinewidth{0.000000pt}%
\definecolor{currentstroke}{rgb}{0.000000,0.000000,0.000000}%
\pgfsetstrokecolor{currentstroke}%
\pgfsetstrokeopacity{0.000000}%
\pgfsetdash{}{0pt}%
\pgfpathmoveto{\pgfqpoint{0.539970in}{2.747992in}}%
\pgfpathlineto{\pgfqpoint{2.816034in}{2.747992in}}%
\pgfpathlineto{\pgfqpoint{2.816034in}{4.374193in}}%
\pgfpathlineto{\pgfqpoint{0.539970in}{4.374193in}}%
\pgfpathlineto{\pgfqpoint{0.539970in}{2.747992in}}%
\pgfpathclose%
\pgfusepath{fill}%
\end{pgfscope}%
\begin{pgfscope}%
\pgfsetbuttcap%
\pgfsetroundjoin%
\definecolor{currentfill}{rgb}{0.000000,0.000000,0.000000}%
\pgfsetfillcolor{currentfill}%
\pgfsetlinewidth{0.501875pt}%
\definecolor{currentstroke}{rgb}{0.000000,0.000000,0.000000}%
\pgfsetstrokecolor{currentstroke}%
\pgfsetdash{}{0pt}%
\pgfsys@defobject{currentmarker}{\pgfqpoint{0.000000in}{0.000000in}}{\pgfqpoint{0.000000in}{0.041667in}}{%
\pgfpathmoveto{\pgfqpoint{0.000000in}{0.000000in}}%
\pgfpathlineto{\pgfqpoint{0.000000in}{0.041667in}}%
\pgfusepath{stroke,fill}%
}%
\begin{pgfscope}%
\pgfsys@transformshift{0.539970in}{2.747992in}%
\pgfsys@useobject{currentmarker}{}%
\end{pgfscope}%
\end{pgfscope}%
\begin{pgfscope}%
\pgfsetbuttcap%
\pgfsetroundjoin%
\definecolor{currentfill}{rgb}{0.000000,0.000000,0.000000}%
\pgfsetfillcolor{currentfill}%
\pgfsetlinewidth{0.501875pt}%
\definecolor{currentstroke}{rgb}{0.000000,0.000000,0.000000}%
\pgfsetstrokecolor{currentstroke}%
\pgfsetdash{}{0pt}%
\pgfsys@defobject{currentmarker}{\pgfqpoint{0.000000in}{-0.041667in}}{\pgfqpoint{0.000000in}{0.000000in}}{%
\pgfpathmoveto{\pgfqpoint{0.000000in}{0.000000in}}%
\pgfpathlineto{\pgfqpoint{0.000000in}{-0.041667in}}%
\pgfusepath{stroke,fill}%
}%
\begin{pgfscope}%
\pgfsys@transformshift{0.539970in}{4.374193in}%
\pgfsys@useobject{currentmarker}{}%
\end{pgfscope}%
\end{pgfscope}%
\begin{pgfscope}%
\definecolor{textcolor}{rgb}{0.000000,0.000000,0.000000}%
\pgfsetstrokecolor{textcolor}%
\pgfsetfillcolor{textcolor}%
\pgftext[x=0.539970in,y=2.699381in,,top]{\color{textcolor}\rmfamily\fontsize{10.000000}{12.000000}\selectfont \(\displaystyle {0}\)}%
\end{pgfscope}%
\begin{pgfscope}%
\pgfsetbuttcap%
\pgfsetroundjoin%
\definecolor{currentfill}{rgb}{0.000000,0.000000,0.000000}%
\pgfsetfillcolor{currentfill}%
\pgfsetlinewidth{0.501875pt}%
\definecolor{currentstroke}{rgb}{0.000000,0.000000,0.000000}%
\pgfsetstrokecolor{currentstroke}%
\pgfsetdash{}{0pt}%
\pgfsys@defobject{currentmarker}{\pgfqpoint{0.000000in}{0.000000in}}{\pgfqpoint{0.000000in}{0.041667in}}{%
\pgfpathmoveto{\pgfqpoint{0.000000in}{0.000000in}}%
\pgfpathlineto{\pgfqpoint{0.000000in}{0.041667in}}%
\pgfusepath{stroke,fill}%
}%
\begin{pgfscope}%
\pgfsys@transformshift{0.990676in}{2.747992in}%
\pgfsys@useobject{currentmarker}{}%
\end{pgfscope}%
\end{pgfscope}%
\begin{pgfscope}%
\pgfsetbuttcap%
\pgfsetroundjoin%
\definecolor{currentfill}{rgb}{0.000000,0.000000,0.000000}%
\pgfsetfillcolor{currentfill}%
\pgfsetlinewidth{0.501875pt}%
\definecolor{currentstroke}{rgb}{0.000000,0.000000,0.000000}%
\pgfsetstrokecolor{currentstroke}%
\pgfsetdash{}{0pt}%
\pgfsys@defobject{currentmarker}{\pgfqpoint{0.000000in}{-0.041667in}}{\pgfqpoint{0.000000in}{0.000000in}}{%
\pgfpathmoveto{\pgfqpoint{0.000000in}{0.000000in}}%
\pgfpathlineto{\pgfqpoint{0.000000in}{-0.041667in}}%
\pgfusepath{stroke,fill}%
}%
\begin{pgfscope}%
\pgfsys@transformshift{0.990676in}{4.374193in}%
\pgfsys@useobject{currentmarker}{}%
\end{pgfscope}%
\end{pgfscope}%
\begin{pgfscope}%
\definecolor{textcolor}{rgb}{0.000000,0.000000,0.000000}%
\pgfsetstrokecolor{textcolor}%
\pgfsetfillcolor{textcolor}%
\pgftext[x=0.990676in,y=2.699381in,,top]{\color{textcolor}\rmfamily\fontsize{10.000000}{12.000000}\selectfont \(\displaystyle {20}\)}%
\end{pgfscope}%
\begin{pgfscope}%
\pgfsetbuttcap%
\pgfsetroundjoin%
\definecolor{currentfill}{rgb}{0.000000,0.000000,0.000000}%
\pgfsetfillcolor{currentfill}%
\pgfsetlinewidth{0.501875pt}%
\definecolor{currentstroke}{rgb}{0.000000,0.000000,0.000000}%
\pgfsetstrokecolor{currentstroke}%
\pgfsetdash{}{0pt}%
\pgfsys@defobject{currentmarker}{\pgfqpoint{0.000000in}{0.000000in}}{\pgfqpoint{0.000000in}{0.041667in}}{%
\pgfpathmoveto{\pgfqpoint{0.000000in}{0.000000in}}%
\pgfpathlineto{\pgfqpoint{0.000000in}{0.041667in}}%
\pgfusepath{stroke,fill}%
}%
\begin{pgfscope}%
\pgfsys@transformshift{1.441381in}{2.747992in}%
\pgfsys@useobject{currentmarker}{}%
\end{pgfscope}%
\end{pgfscope}%
\begin{pgfscope}%
\pgfsetbuttcap%
\pgfsetroundjoin%
\definecolor{currentfill}{rgb}{0.000000,0.000000,0.000000}%
\pgfsetfillcolor{currentfill}%
\pgfsetlinewidth{0.501875pt}%
\definecolor{currentstroke}{rgb}{0.000000,0.000000,0.000000}%
\pgfsetstrokecolor{currentstroke}%
\pgfsetdash{}{0pt}%
\pgfsys@defobject{currentmarker}{\pgfqpoint{0.000000in}{-0.041667in}}{\pgfqpoint{0.000000in}{0.000000in}}{%
\pgfpathmoveto{\pgfqpoint{0.000000in}{0.000000in}}%
\pgfpathlineto{\pgfqpoint{0.000000in}{-0.041667in}}%
\pgfusepath{stroke,fill}%
}%
\begin{pgfscope}%
\pgfsys@transformshift{1.441381in}{4.374193in}%
\pgfsys@useobject{currentmarker}{}%
\end{pgfscope}%
\end{pgfscope}%
\begin{pgfscope}%
\definecolor{textcolor}{rgb}{0.000000,0.000000,0.000000}%
\pgfsetstrokecolor{textcolor}%
\pgfsetfillcolor{textcolor}%
\pgftext[x=1.441381in,y=2.699381in,,top]{\color{textcolor}\rmfamily\fontsize{10.000000}{12.000000}\selectfont \(\displaystyle {40}\)}%
\end{pgfscope}%
\begin{pgfscope}%
\pgfsetbuttcap%
\pgfsetroundjoin%
\definecolor{currentfill}{rgb}{0.000000,0.000000,0.000000}%
\pgfsetfillcolor{currentfill}%
\pgfsetlinewidth{0.501875pt}%
\definecolor{currentstroke}{rgb}{0.000000,0.000000,0.000000}%
\pgfsetstrokecolor{currentstroke}%
\pgfsetdash{}{0pt}%
\pgfsys@defobject{currentmarker}{\pgfqpoint{0.000000in}{0.000000in}}{\pgfqpoint{0.000000in}{0.041667in}}{%
\pgfpathmoveto{\pgfqpoint{0.000000in}{0.000000in}}%
\pgfpathlineto{\pgfqpoint{0.000000in}{0.041667in}}%
\pgfusepath{stroke,fill}%
}%
\begin{pgfscope}%
\pgfsys@transformshift{1.892087in}{2.747992in}%
\pgfsys@useobject{currentmarker}{}%
\end{pgfscope}%
\end{pgfscope}%
\begin{pgfscope}%
\pgfsetbuttcap%
\pgfsetroundjoin%
\definecolor{currentfill}{rgb}{0.000000,0.000000,0.000000}%
\pgfsetfillcolor{currentfill}%
\pgfsetlinewidth{0.501875pt}%
\definecolor{currentstroke}{rgb}{0.000000,0.000000,0.000000}%
\pgfsetstrokecolor{currentstroke}%
\pgfsetdash{}{0pt}%
\pgfsys@defobject{currentmarker}{\pgfqpoint{0.000000in}{-0.041667in}}{\pgfqpoint{0.000000in}{0.000000in}}{%
\pgfpathmoveto{\pgfqpoint{0.000000in}{0.000000in}}%
\pgfpathlineto{\pgfqpoint{0.000000in}{-0.041667in}}%
\pgfusepath{stroke,fill}%
}%
\begin{pgfscope}%
\pgfsys@transformshift{1.892087in}{4.374193in}%
\pgfsys@useobject{currentmarker}{}%
\end{pgfscope}%
\end{pgfscope}%
\begin{pgfscope}%
\definecolor{textcolor}{rgb}{0.000000,0.000000,0.000000}%
\pgfsetstrokecolor{textcolor}%
\pgfsetfillcolor{textcolor}%
\pgftext[x=1.892087in,y=2.699381in,,top]{\color{textcolor}\rmfamily\fontsize{10.000000}{12.000000}\selectfont \(\displaystyle {60}\)}%
\end{pgfscope}%
\begin{pgfscope}%
\pgfsetbuttcap%
\pgfsetroundjoin%
\definecolor{currentfill}{rgb}{0.000000,0.000000,0.000000}%
\pgfsetfillcolor{currentfill}%
\pgfsetlinewidth{0.501875pt}%
\definecolor{currentstroke}{rgb}{0.000000,0.000000,0.000000}%
\pgfsetstrokecolor{currentstroke}%
\pgfsetdash{}{0pt}%
\pgfsys@defobject{currentmarker}{\pgfqpoint{0.000000in}{0.000000in}}{\pgfqpoint{0.000000in}{0.041667in}}{%
\pgfpathmoveto{\pgfqpoint{0.000000in}{0.000000in}}%
\pgfpathlineto{\pgfqpoint{0.000000in}{0.041667in}}%
\pgfusepath{stroke,fill}%
}%
\begin{pgfscope}%
\pgfsys@transformshift{2.342793in}{2.747992in}%
\pgfsys@useobject{currentmarker}{}%
\end{pgfscope}%
\end{pgfscope}%
\begin{pgfscope}%
\pgfsetbuttcap%
\pgfsetroundjoin%
\definecolor{currentfill}{rgb}{0.000000,0.000000,0.000000}%
\pgfsetfillcolor{currentfill}%
\pgfsetlinewidth{0.501875pt}%
\definecolor{currentstroke}{rgb}{0.000000,0.000000,0.000000}%
\pgfsetstrokecolor{currentstroke}%
\pgfsetdash{}{0pt}%
\pgfsys@defobject{currentmarker}{\pgfqpoint{0.000000in}{-0.041667in}}{\pgfqpoint{0.000000in}{0.000000in}}{%
\pgfpathmoveto{\pgfqpoint{0.000000in}{0.000000in}}%
\pgfpathlineto{\pgfqpoint{0.000000in}{-0.041667in}}%
\pgfusepath{stroke,fill}%
}%
\begin{pgfscope}%
\pgfsys@transformshift{2.342793in}{4.374193in}%
\pgfsys@useobject{currentmarker}{}%
\end{pgfscope}%
\end{pgfscope}%
\begin{pgfscope}%
\definecolor{textcolor}{rgb}{0.000000,0.000000,0.000000}%
\pgfsetstrokecolor{textcolor}%
\pgfsetfillcolor{textcolor}%
\pgftext[x=2.342793in,y=2.699381in,,top]{\color{textcolor}\rmfamily\fontsize{10.000000}{12.000000}\selectfont \(\displaystyle {80}\)}%
\end{pgfscope}%
\begin{pgfscope}%
\pgfsetbuttcap%
\pgfsetroundjoin%
\definecolor{currentfill}{rgb}{0.000000,0.000000,0.000000}%
\pgfsetfillcolor{currentfill}%
\pgfsetlinewidth{0.501875pt}%
\definecolor{currentstroke}{rgb}{0.000000,0.000000,0.000000}%
\pgfsetstrokecolor{currentstroke}%
\pgfsetdash{}{0pt}%
\pgfsys@defobject{currentmarker}{\pgfqpoint{0.000000in}{0.000000in}}{\pgfqpoint{0.000000in}{0.020833in}}{%
\pgfpathmoveto{\pgfqpoint{0.000000in}{0.000000in}}%
\pgfpathlineto{\pgfqpoint{0.000000in}{0.020833in}}%
\pgfusepath{stroke,fill}%
}%
\begin{pgfscope}%
\pgfsys@transformshift{0.652646in}{2.747992in}%
\pgfsys@useobject{currentmarker}{}%
\end{pgfscope}%
\end{pgfscope}%
\begin{pgfscope}%
\pgfsetbuttcap%
\pgfsetroundjoin%
\definecolor{currentfill}{rgb}{0.000000,0.000000,0.000000}%
\pgfsetfillcolor{currentfill}%
\pgfsetlinewidth{0.501875pt}%
\definecolor{currentstroke}{rgb}{0.000000,0.000000,0.000000}%
\pgfsetstrokecolor{currentstroke}%
\pgfsetdash{}{0pt}%
\pgfsys@defobject{currentmarker}{\pgfqpoint{0.000000in}{-0.020833in}}{\pgfqpoint{0.000000in}{0.000000in}}{%
\pgfpathmoveto{\pgfqpoint{0.000000in}{0.000000in}}%
\pgfpathlineto{\pgfqpoint{0.000000in}{-0.020833in}}%
\pgfusepath{stroke,fill}%
}%
\begin{pgfscope}%
\pgfsys@transformshift{0.652646in}{4.374193in}%
\pgfsys@useobject{currentmarker}{}%
\end{pgfscope}%
\end{pgfscope}%
\begin{pgfscope}%
\pgfsetbuttcap%
\pgfsetroundjoin%
\definecolor{currentfill}{rgb}{0.000000,0.000000,0.000000}%
\pgfsetfillcolor{currentfill}%
\pgfsetlinewidth{0.501875pt}%
\definecolor{currentstroke}{rgb}{0.000000,0.000000,0.000000}%
\pgfsetstrokecolor{currentstroke}%
\pgfsetdash{}{0pt}%
\pgfsys@defobject{currentmarker}{\pgfqpoint{0.000000in}{0.000000in}}{\pgfqpoint{0.000000in}{0.020833in}}{%
\pgfpathmoveto{\pgfqpoint{0.000000in}{0.000000in}}%
\pgfpathlineto{\pgfqpoint{0.000000in}{0.020833in}}%
\pgfusepath{stroke,fill}%
}%
\begin{pgfscope}%
\pgfsys@transformshift{0.765323in}{2.747992in}%
\pgfsys@useobject{currentmarker}{}%
\end{pgfscope}%
\end{pgfscope}%
\begin{pgfscope}%
\pgfsetbuttcap%
\pgfsetroundjoin%
\definecolor{currentfill}{rgb}{0.000000,0.000000,0.000000}%
\pgfsetfillcolor{currentfill}%
\pgfsetlinewidth{0.501875pt}%
\definecolor{currentstroke}{rgb}{0.000000,0.000000,0.000000}%
\pgfsetstrokecolor{currentstroke}%
\pgfsetdash{}{0pt}%
\pgfsys@defobject{currentmarker}{\pgfqpoint{0.000000in}{-0.020833in}}{\pgfqpoint{0.000000in}{0.000000in}}{%
\pgfpathmoveto{\pgfqpoint{0.000000in}{0.000000in}}%
\pgfpathlineto{\pgfqpoint{0.000000in}{-0.020833in}}%
\pgfusepath{stroke,fill}%
}%
\begin{pgfscope}%
\pgfsys@transformshift{0.765323in}{4.374193in}%
\pgfsys@useobject{currentmarker}{}%
\end{pgfscope}%
\end{pgfscope}%
\begin{pgfscope}%
\pgfsetbuttcap%
\pgfsetroundjoin%
\definecolor{currentfill}{rgb}{0.000000,0.000000,0.000000}%
\pgfsetfillcolor{currentfill}%
\pgfsetlinewidth{0.501875pt}%
\definecolor{currentstroke}{rgb}{0.000000,0.000000,0.000000}%
\pgfsetstrokecolor{currentstroke}%
\pgfsetdash{}{0pt}%
\pgfsys@defobject{currentmarker}{\pgfqpoint{0.000000in}{0.000000in}}{\pgfqpoint{0.000000in}{0.020833in}}{%
\pgfpathmoveto{\pgfqpoint{0.000000in}{0.000000in}}%
\pgfpathlineto{\pgfqpoint{0.000000in}{0.020833in}}%
\pgfusepath{stroke,fill}%
}%
\begin{pgfscope}%
\pgfsys@transformshift{0.877999in}{2.747992in}%
\pgfsys@useobject{currentmarker}{}%
\end{pgfscope}%
\end{pgfscope}%
\begin{pgfscope}%
\pgfsetbuttcap%
\pgfsetroundjoin%
\definecolor{currentfill}{rgb}{0.000000,0.000000,0.000000}%
\pgfsetfillcolor{currentfill}%
\pgfsetlinewidth{0.501875pt}%
\definecolor{currentstroke}{rgb}{0.000000,0.000000,0.000000}%
\pgfsetstrokecolor{currentstroke}%
\pgfsetdash{}{0pt}%
\pgfsys@defobject{currentmarker}{\pgfqpoint{0.000000in}{-0.020833in}}{\pgfqpoint{0.000000in}{0.000000in}}{%
\pgfpathmoveto{\pgfqpoint{0.000000in}{0.000000in}}%
\pgfpathlineto{\pgfqpoint{0.000000in}{-0.020833in}}%
\pgfusepath{stroke,fill}%
}%
\begin{pgfscope}%
\pgfsys@transformshift{0.877999in}{4.374193in}%
\pgfsys@useobject{currentmarker}{}%
\end{pgfscope}%
\end{pgfscope}%
\begin{pgfscope}%
\pgfsetbuttcap%
\pgfsetroundjoin%
\definecolor{currentfill}{rgb}{0.000000,0.000000,0.000000}%
\pgfsetfillcolor{currentfill}%
\pgfsetlinewidth{0.501875pt}%
\definecolor{currentstroke}{rgb}{0.000000,0.000000,0.000000}%
\pgfsetstrokecolor{currentstroke}%
\pgfsetdash{}{0pt}%
\pgfsys@defobject{currentmarker}{\pgfqpoint{0.000000in}{0.000000in}}{\pgfqpoint{0.000000in}{0.020833in}}{%
\pgfpathmoveto{\pgfqpoint{0.000000in}{0.000000in}}%
\pgfpathlineto{\pgfqpoint{0.000000in}{0.020833in}}%
\pgfusepath{stroke,fill}%
}%
\begin{pgfscope}%
\pgfsys@transformshift{1.103352in}{2.747992in}%
\pgfsys@useobject{currentmarker}{}%
\end{pgfscope}%
\end{pgfscope}%
\begin{pgfscope}%
\pgfsetbuttcap%
\pgfsetroundjoin%
\definecolor{currentfill}{rgb}{0.000000,0.000000,0.000000}%
\pgfsetfillcolor{currentfill}%
\pgfsetlinewidth{0.501875pt}%
\definecolor{currentstroke}{rgb}{0.000000,0.000000,0.000000}%
\pgfsetstrokecolor{currentstroke}%
\pgfsetdash{}{0pt}%
\pgfsys@defobject{currentmarker}{\pgfqpoint{0.000000in}{-0.020833in}}{\pgfqpoint{0.000000in}{0.000000in}}{%
\pgfpathmoveto{\pgfqpoint{0.000000in}{0.000000in}}%
\pgfpathlineto{\pgfqpoint{0.000000in}{-0.020833in}}%
\pgfusepath{stroke,fill}%
}%
\begin{pgfscope}%
\pgfsys@transformshift{1.103352in}{4.374193in}%
\pgfsys@useobject{currentmarker}{}%
\end{pgfscope}%
\end{pgfscope}%
\begin{pgfscope}%
\pgfsetbuttcap%
\pgfsetroundjoin%
\definecolor{currentfill}{rgb}{0.000000,0.000000,0.000000}%
\pgfsetfillcolor{currentfill}%
\pgfsetlinewidth{0.501875pt}%
\definecolor{currentstroke}{rgb}{0.000000,0.000000,0.000000}%
\pgfsetstrokecolor{currentstroke}%
\pgfsetdash{}{0pt}%
\pgfsys@defobject{currentmarker}{\pgfqpoint{0.000000in}{0.000000in}}{\pgfqpoint{0.000000in}{0.020833in}}{%
\pgfpathmoveto{\pgfqpoint{0.000000in}{0.000000in}}%
\pgfpathlineto{\pgfqpoint{0.000000in}{0.020833in}}%
\pgfusepath{stroke,fill}%
}%
\begin{pgfscope}%
\pgfsys@transformshift{1.216028in}{2.747992in}%
\pgfsys@useobject{currentmarker}{}%
\end{pgfscope}%
\end{pgfscope}%
\begin{pgfscope}%
\pgfsetbuttcap%
\pgfsetroundjoin%
\definecolor{currentfill}{rgb}{0.000000,0.000000,0.000000}%
\pgfsetfillcolor{currentfill}%
\pgfsetlinewidth{0.501875pt}%
\definecolor{currentstroke}{rgb}{0.000000,0.000000,0.000000}%
\pgfsetstrokecolor{currentstroke}%
\pgfsetdash{}{0pt}%
\pgfsys@defobject{currentmarker}{\pgfqpoint{0.000000in}{-0.020833in}}{\pgfqpoint{0.000000in}{0.000000in}}{%
\pgfpathmoveto{\pgfqpoint{0.000000in}{0.000000in}}%
\pgfpathlineto{\pgfqpoint{0.000000in}{-0.020833in}}%
\pgfusepath{stroke,fill}%
}%
\begin{pgfscope}%
\pgfsys@transformshift{1.216028in}{4.374193in}%
\pgfsys@useobject{currentmarker}{}%
\end{pgfscope}%
\end{pgfscope}%
\begin{pgfscope}%
\pgfsetbuttcap%
\pgfsetroundjoin%
\definecolor{currentfill}{rgb}{0.000000,0.000000,0.000000}%
\pgfsetfillcolor{currentfill}%
\pgfsetlinewidth{0.501875pt}%
\definecolor{currentstroke}{rgb}{0.000000,0.000000,0.000000}%
\pgfsetstrokecolor{currentstroke}%
\pgfsetdash{}{0pt}%
\pgfsys@defobject{currentmarker}{\pgfqpoint{0.000000in}{0.000000in}}{\pgfqpoint{0.000000in}{0.020833in}}{%
\pgfpathmoveto{\pgfqpoint{0.000000in}{0.000000in}}%
\pgfpathlineto{\pgfqpoint{0.000000in}{0.020833in}}%
\pgfusepath{stroke,fill}%
}%
\begin{pgfscope}%
\pgfsys@transformshift{1.328705in}{2.747992in}%
\pgfsys@useobject{currentmarker}{}%
\end{pgfscope}%
\end{pgfscope}%
\begin{pgfscope}%
\pgfsetbuttcap%
\pgfsetroundjoin%
\definecolor{currentfill}{rgb}{0.000000,0.000000,0.000000}%
\pgfsetfillcolor{currentfill}%
\pgfsetlinewidth{0.501875pt}%
\definecolor{currentstroke}{rgb}{0.000000,0.000000,0.000000}%
\pgfsetstrokecolor{currentstroke}%
\pgfsetdash{}{0pt}%
\pgfsys@defobject{currentmarker}{\pgfqpoint{0.000000in}{-0.020833in}}{\pgfqpoint{0.000000in}{0.000000in}}{%
\pgfpathmoveto{\pgfqpoint{0.000000in}{0.000000in}}%
\pgfpathlineto{\pgfqpoint{0.000000in}{-0.020833in}}%
\pgfusepath{stroke,fill}%
}%
\begin{pgfscope}%
\pgfsys@transformshift{1.328705in}{4.374193in}%
\pgfsys@useobject{currentmarker}{}%
\end{pgfscope}%
\end{pgfscope}%
\begin{pgfscope}%
\pgfsetbuttcap%
\pgfsetroundjoin%
\definecolor{currentfill}{rgb}{0.000000,0.000000,0.000000}%
\pgfsetfillcolor{currentfill}%
\pgfsetlinewidth{0.501875pt}%
\definecolor{currentstroke}{rgb}{0.000000,0.000000,0.000000}%
\pgfsetstrokecolor{currentstroke}%
\pgfsetdash{}{0pt}%
\pgfsys@defobject{currentmarker}{\pgfqpoint{0.000000in}{0.000000in}}{\pgfqpoint{0.000000in}{0.020833in}}{%
\pgfpathmoveto{\pgfqpoint{0.000000in}{0.000000in}}%
\pgfpathlineto{\pgfqpoint{0.000000in}{0.020833in}}%
\pgfusepath{stroke,fill}%
}%
\begin{pgfscope}%
\pgfsys@transformshift{1.554058in}{2.747992in}%
\pgfsys@useobject{currentmarker}{}%
\end{pgfscope}%
\end{pgfscope}%
\begin{pgfscope}%
\pgfsetbuttcap%
\pgfsetroundjoin%
\definecolor{currentfill}{rgb}{0.000000,0.000000,0.000000}%
\pgfsetfillcolor{currentfill}%
\pgfsetlinewidth{0.501875pt}%
\definecolor{currentstroke}{rgb}{0.000000,0.000000,0.000000}%
\pgfsetstrokecolor{currentstroke}%
\pgfsetdash{}{0pt}%
\pgfsys@defobject{currentmarker}{\pgfqpoint{0.000000in}{-0.020833in}}{\pgfqpoint{0.000000in}{0.000000in}}{%
\pgfpathmoveto{\pgfqpoint{0.000000in}{0.000000in}}%
\pgfpathlineto{\pgfqpoint{0.000000in}{-0.020833in}}%
\pgfusepath{stroke,fill}%
}%
\begin{pgfscope}%
\pgfsys@transformshift{1.554058in}{4.374193in}%
\pgfsys@useobject{currentmarker}{}%
\end{pgfscope}%
\end{pgfscope}%
\begin{pgfscope}%
\pgfsetbuttcap%
\pgfsetroundjoin%
\definecolor{currentfill}{rgb}{0.000000,0.000000,0.000000}%
\pgfsetfillcolor{currentfill}%
\pgfsetlinewidth{0.501875pt}%
\definecolor{currentstroke}{rgb}{0.000000,0.000000,0.000000}%
\pgfsetstrokecolor{currentstroke}%
\pgfsetdash{}{0pt}%
\pgfsys@defobject{currentmarker}{\pgfqpoint{0.000000in}{0.000000in}}{\pgfqpoint{0.000000in}{0.020833in}}{%
\pgfpathmoveto{\pgfqpoint{0.000000in}{0.000000in}}%
\pgfpathlineto{\pgfqpoint{0.000000in}{0.020833in}}%
\pgfusepath{stroke,fill}%
}%
\begin{pgfscope}%
\pgfsys@transformshift{1.666734in}{2.747992in}%
\pgfsys@useobject{currentmarker}{}%
\end{pgfscope}%
\end{pgfscope}%
\begin{pgfscope}%
\pgfsetbuttcap%
\pgfsetroundjoin%
\definecolor{currentfill}{rgb}{0.000000,0.000000,0.000000}%
\pgfsetfillcolor{currentfill}%
\pgfsetlinewidth{0.501875pt}%
\definecolor{currentstroke}{rgb}{0.000000,0.000000,0.000000}%
\pgfsetstrokecolor{currentstroke}%
\pgfsetdash{}{0pt}%
\pgfsys@defobject{currentmarker}{\pgfqpoint{0.000000in}{-0.020833in}}{\pgfqpoint{0.000000in}{0.000000in}}{%
\pgfpathmoveto{\pgfqpoint{0.000000in}{0.000000in}}%
\pgfpathlineto{\pgfqpoint{0.000000in}{-0.020833in}}%
\pgfusepath{stroke,fill}%
}%
\begin{pgfscope}%
\pgfsys@transformshift{1.666734in}{4.374193in}%
\pgfsys@useobject{currentmarker}{}%
\end{pgfscope}%
\end{pgfscope}%
\begin{pgfscope}%
\pgfsetbuttcap%
\pgfsetroundjoin%
\definecolor{currentfill}{rgb}{0.000000,0.000000,0.000000}%
\pgfsetfillcolor{currentfill}%
\pgfsetlinewidth{0.501875pt}%
\definecolor{currentstroke}{rgb}{0.000000,0.000000,0.000000}%
\pgfsetstrokecolor{currentstroke}%
\pgfsetdash{}{0pt}%
\pgfsys@defobject{currentmarker}{\pgfqpoint{0.000000in}{0.000000in}}{\pgfqpoint{0.000000in}{0.020833in}}{%
\pgfpathmoveto{\pgfqpoint{0.000000in}{0.000000in}}%
\pgfpathlineto{\pgfqpoint{0.000000in}{0.020833in}}%
\pgfusepath{stroke,fill}%
}%
\begin{pgfscope}%
\pgfsys@transformshift{1.779411in}{2.747992in}%
\pgfsys@useobject{currentmarker}{}%
\end{pgfscope}%
\end{pgfscope}%
\begin{pgfscope}%
\pgfsetbuttcap%
\pgfsetroundjoin%
\definecolor{currentfill}{rgb}{0.000000,0.000000,0.000000}%
\pgfsetfillcolor{currentfill}%
\pgfsetlinewidth{0.501875pt}%
\definecolor{currentstroke}{rgb}{0.000000,0.000000,0.000000}%
\pgfsetstrokecolor{currentstroke}%
\pgfsetdash{}{0pt}%
\pgfsys@defobject{currentmarker}{\pgfqpoint{0.000000in}{-0.020833in}}{\pgfqpoint{0.000000in}{0.000000in}}{%
\pgfpathmoveto{\pgfqpoint{0.000000in}{0.000000in}}%
\pgfpathlineto{\pgfqpoint{0.000000in}{-0.020833in}}%
\pgfusepath{stroke,fill}%
}%
\begin{pgfscope}%
\pgfsys@transformshift{1.779411in}{4.374193in}%
\pgfsys@useobject{currentmarker}{}%
\end{pgfscope}%
\end{pgfscope}%
\begin{pgfscope}%
\pgfsetbuttcap%
\pgfsetroundjoin%
\definecolor{currentfill}{rgb}{0.000000,0.000000,0.000000}%
\pgfsetfillcolor{currentfill}%
\pgfsetlinewidth{0.501875pt}%
\definecolor{currentstroke}{rgb}{0.000000,0.000000,0.000000}%
\pgfsetstrokecolor{currentstroke}%
\pgfsetdash{}{0pt}%
\pgfsys@defobject{currentmarker}{\pgfqpoint{0.000000in}{0.000000in}}{\pgfqpoint{0.000000in}{0.020833in}}{%
\pgfpathmoveto{\pgfqpoint{0.000000in}{0.000000in}}%
\pgfpathlineto{\pgfqpoint{0.000000in}{0.020833in}}%
\pgfusepath{stroke,fill}%
}%
\begin{pgfscope}%
\pgfsys@transformshift{2.004764in}{2.747992in}%
\pgfsys@useobject{currentmarker}{}%
\end{pgfscope}%
\end{pgfscope}%
\begin{pgfscope}%
\pgfsetbuttcap%
\pgfsetroundjoin%
\definecolor{currentfill}{rgb}{0.000000,0.000000,0.000000}%
\pgfsetfillcolor{currentfill}%
\pgfsetlinewidth{0.501875pt}%
\definecolor{currentstroke}{rgb}{0.000000,0.000000,0.000000}%
\pgfsetstrokecolor{currentstroke}%
\pgfsetdash{}{0pt}%
\pgfsys@defobject{currentmarker}{\pgfqpoint{0.000000in}{-0.020833in}}{\pgfqpoint{0.000000in}{0.000000in}}{%
\pgfpathmoveto{\pgfqpoint{0.000000in}{0.000000in}}%
\pgfpathlineto{\pgfqpoint{0.000000in}{-0.020833in}}%
\pgfusepath{stroke,fill}%
}%
\begin{pgfscope}%
\pgfsys@transformshift{2.004764in}{4.374193in}%
\pgfsys@useobject{currentmarker}{}%
\end{pgfscope}%
\end{pgfscope}%
\begin{pgfscope}%
\pgfsetbuttcap%
\pgfsetroundjoin%
\definecolor{currentfill}{rgb}{0.000000,0.000000,0.000000}%
\pgfsetfillcolor{currentfill}%
\pgfsetlinewidth{0.501875pt}%
\definecolor{currentstroke}{rgb}{0.000000,0.000000,0.000000}%
\pgfsetstrokecolor{currentstroke}%
\pgfsetdash{}{0pt}%
\pgfsys@defobject{currentmarker}{\pgfqpoint{0.000000in}{0.000000in}}{\pgfqpoint{0.000000in}{0.020833in}}{%
\pgfpathmoveto{\pgfqpoint{0.000000in}{0.000000in}}%
\pgfpathlineto{\pgfqpoint{0.000000in}{0.020833in}}%
\pgfusepath{stroke,fill}%
}%
\begin{pgfscope}%
\pgfsys@transformshift{2.117440in}{2.747992in}%
\pgfsys@useobject{currentmarker}{}%
\end{pgfscope}%
\end{pgfscope}%
\begin{pgfscope}%
\pgfsetbuttcap%
\pgfsetroundjoin%
\definecolor{currentfill}{rgb}{0.000000,0.000000,0.000000}%
\pgfsetfillcolor{currentfill}%
\pgfsetlinewidth{0.501875pt}%
\definecolor{currentstroke}{rgb}{0.000000,0.000000,0.000000}%
\pgfsetstrokecolor{currentstroke}%
\pgfsetdash{}{0pt}%
\pgfsys@defobject{currentmarker}{\pgfqpoint{0.000000in}{-0.020833in}}{\pgfqpoint{0.000000in}{0.000000in}}{%
\pgfpathmoveto{\pgfqpoint{0.000000in}{0.000000in}}%
\pgfpathlineto{\pgfqpoint{0.000000in}{-0.020833in}}%
\pgfusepath{stroke,fill}%
}%
\begin{pgfscope}%
\pgfsys@transformshift{2.117440in}{4.374193in}%
\pgfsys@useobject{currentmarker}{}%
\end{pgfscope}%
\end{pgfscope}%
\begin{pgfscope}%
\pgfsetbuttcap%
\pgfsetroundjoin%
\definecolor{currentfill}{rgb}{0.000000,0.000000,0.000000}%
\pgfsetfillcolor{currentfill}%
\pgfsetlinewidth{0.501875pt}%
\definecolor{currentstroke}{rgb}{0.000000,0.000000,0.000000}%
\pgfsetstrokecolor{currentstroke}%
\pgfsetdash{}{0pt}%
\pgfsys@defobject{currentmarker}{\pgfqpoint{0.000000in}{0.000000in}}{\pgfqpoint{0.000000in}{0.020833in}}{%
\pgfpathmoveto{\pgfqpoint{0.000000in}{0.000000in}}%
\pgfpathlineto{\pgfqpoint{0.000000in}{0.020833in}}%
\pgfusepath{stroke,fill}%
}%
\begin{pgfscope}%
\pgfsys@transformshift{2.230116in}{2.747992in}%
\pgfsys@useobject{currentmarker}{}%
\end{pgfscope}%
\end{pgfscope}%
\begin{pgfscope}%
\pgfsetbuttcap%
\pgfsetroundjoin%
\definecolor{currentfill}{rgb}{0.000000,0.000000,0.000000}%
\pgfsetfillcolor{currentfill}%
\pgfsetlinewidth{0.501875pt}%
\definecolor{currentstroke}{rgb}{0.000000,0.000000,0.000000}%
\pgfsetstrokecolor{currentstroke}%
\pgfsetdash{}{0pt}%
\pgfsys@defobject{currentmarker}{\pgfqpoint{0.000000in}{-0.020833in}}{\pgfqpoint{0.000000in}{0.000000in}}{%
\pgfpathmoveto{\pgfqpoint{0.000000in}{0.000000in}}%
\pgfpathlineto{\pgfqpoint{0.000000in}{-0.020833in}}%
\pgfusepath{stroke,fill}%
}%
\begin{pgfscope}%
\pgfsys@transformshift{2.230116in}{4.374193in}%
\pgfsys@useobject{currentmarker}{}%
\end{pgfscope}%
\end{pgfscope}%
\begin{pgfscope}%
\pgfsetbuttcap%
\pgfsetroundjoin%
\definecolor{currentfill}{rgb}{0.000000,0.000000,0.000000}%
\pgfsetfillcolor{currentfill}%
\pgfsetlinewidth{0.501875pt}%
\definecolor{currentstroke}{rgb}{0.000000,0.000000,0.000000}%
\pgfsetstrokecolor{currentstroke}%
\pgfsetdash{}{0pt}%
\pgfsys@defobject{currentmarker}{\pgfqpoint{0.000000in}{0.000000in}}{\pgfqpoint{0.000000in}{0.020833in}}{%
\pgfpathmoveto{\pgfqpoint{0.000000in}{0.000000in}}%
\pgfpathlineto{\pgfqpoint{0.000000in}{0.020833in}}%
\pgfusepath{stroke,fill}%
}%
\begin{pgfscope}%
\pgfsys@transformshift{2.455469in}{2.747992in}%
\pgfsys@useobject{currentmarker}{}%
\end{pgfscope}%
\end{pgfscope}%
\begin{pgfscope}%
\pgfsetbuttcap%
\pgfsetroundjoin%
\definecolor{currentfill}{rgb}{0.000000,0.000000,0.000000}%
\pgfsetfillcolor{currentfill}%
\pgfsetlinewidth{0.501875pt}%
\definecolor{currentstroke}{rgb}{0.000000,0.000000,0.000000}%
\pgfsetstrokecolor{currentstroke}%
\pgfsetdash{}{0pt}%
\pgfsys@defobject{currentmarker}{\pgfqpoint{0.000000in}{-0.020833in}}{\pgfqpoint{0.000000in}{0.000000in}}{%
\pgfpathmoveto{\pgfqpoint{0.000000in}{0.000000in}}%
\pgfpathlineto{\pgfqpoint{0.000000in}{-0.020833in}}%
\pgfusepath{stroke,fill}%
}%
\begin{pgfscope}%
\pgfsys@transformshift{2.455469in}{4.374193in}%
\pgfsys@useobject{currentmarker}{}%
\end{pgfscope}%
\end{pgfscope}%
\begin{pgfscope}%
\pgfsetbuttcap%
\pgfsetroundjoin%
\definecolor{currentfill}{rgb}{0.000000,0.000000,0.000000}%
\pgfsetfillcolor{currentfill}%
\pgfsetlinewidth{0.501875pt}%
\definecolor{currentstroke}{rgb}{0.000000,0.000000,0.000000}%
\pgfsetstrokecolor{currentstroke}%
\pgfsetdash{}{0pt}%
\pgfsys@defobject{currentmarker}{\pgfqpoint{0.000000in}{0.000000in}}{\pgfqpoint{0.000000in}{0.020833in}}{%
\pgfpathmoveto{\pgfqpoint{0.000000in}{0.000000in}}%
\pgfpathlineto{\pgfqpoint{0.000000in}{0.020833in}}%
\pgfusepath{stroke,fill}%
}%
\begin{pgfscope}%
\pgfsys@transformshift{2.568146in}{2.747992in}%
\pgfsys@useobject{currentmarker}{}%
\end{pgfscope}%
\end{pgfscope}%
\begin{pgfscope}%
\pgfsetbuttcap%
\pgfsetroundjoin%
\definecolor{currentfill}{rgb}{0.000000,0.000000,0.000000}%
\pgfsetfillcolor{currentfill}%
\pgfsetlinewidth{0.501875pt}%
\definecolor{currentstroke}{rgb}{0.000000,0.000000,0.000000}%
\pgfsetstrokecolor{currentstroke}%
\pgfsetdash{}{0pt}%
\pgfsys@defobject{currentmarker}{\pgfqpoint{0.000000in}{-0.020833in}}{\pgfqpoint{0.000000in}{0.000000in}}{%
\pgfpathmoveto{\pgfqpoint{0.000000in}{0.000000in}}%
\pgfpathlineto{\pgfqpoint{0.000000in}{-0.020833in}}%
\pgfusepath{stroke,fill}%
}%
\begin{pgfscope}%
\pgfsys@transformshift{2.568146in}{4.374193in}%
\pgfsys@useobject{currentmarker}{}%
\end{pgfscope}%
\end{pgfscope}%
\begin{pgfscope}%
\pgfsetbuttcap%
\pgfsetroundjoin%
\definecolor{currentfill}{rgb}{0.000000,0.000000,0.000000}%
\pgfsetfillcolor{currentfill}%
\pgfsetlinewidth{0.501875pt}%
\definecolor{currentstroke}{rgb}{0.000000,0.000000,0.000000}%
\pgfsetstrokecolor{currentstroke}%
\pgfsetdash{}{0pt}%
\pgfsys@defobject{currentmarker}{\pgfqpoint{0.000000in}{0.000000in}}{\pgfqpoint{0.000000in}{0.020833in}}{%
\pgfpathmoveto{\pgfqpoint{0.000000in}{0.000000in}}%
\pgfpathlineto{\pgfqpoint{0.000000in}{0.020833in}}%
\pgfusepath{stroke,fill}%
}%
\begin{pgfscope}%
\pgfsys@transformshift{2.680822in}{2.747992in}%
\pgfsys@useobject{currentmarker}{}%
\end{pgfscope}%
\end{pgfscope}%
\begin{pgfscope}%
\pgfsetbuttcap%
\pgfsetroundjoin%
\definecolor{currentfill}{rgb}{0.000000,0.000000,0.000000}%
\pgfsetfillcolor{currentfill}%
\pgfsetlinewidth{0.501875pt}%
\definecolor{currentstroke}{rgb}{0.000000,0.000000,0.000000}%
\pgfsetstrokecolor{currentstroke}%
\pgfsetdash{}{0pt}%
\pgfsys@defobject{currentmarker}{\pgfqpoint{0.000000in}{-0.020833in}}{\pgfqpoint{0.000000in}{0.000000in}}{%
\pgfpathmoveto{\pgfqpoint{0.000000in}{0.000000in}}%
\pgfpathlineto{\pgfqpoint{0.000000in}{-0.020833in}}%
\pgfusepath{stroke,fill}%
}%
\begin{pgfscope}%
\pgfsys@transformshift{2.680822in}{4.374193in}%
\pgfsys@useobject{currentmarker}{}%
\end{pgfscope}%
\end{pgfscope}%
\begin{pgfscope}%
\pgfsetbuttcap%
\pgfsetroundjoin%
\definecolor{currentfill}{rgb}{0.000000,0.000000,0.000000}%
\pgfsetfillcolor{currentfill}%
\pgfsetlinewidth{0.501875pt}%
\definecolor{currentstroke}{rgb}{0.000000,0.000000,0.000000}%
\pgfsetstrokecolor{currentstroke}%
\pgfsetdash{}{0pt}%
\pgfsys@defobject{currentmarker}{\pgfqpoint{0.000000in}{0.000000in}}{\pgfqpoint{0.000000in}{0.020833in}}{%
\pgfpathmoveto{\pgfqpoint{0.000000in}{0.000000in}}%
\pgfpathlineto{\pgfqpoint{0.000000in}{0.020833in}}%
\pgfusepath{stroke,fill}%
}%
\begin{pgfscope}%
\pgfsys@transformshift{2.793499in}{2.747992in}%
\pgfsys@useobject{currentmarker}{}%
\end{pgfscope}%
\end{pgfscope}%
\begin{pgfscope}%
\pgfsetbuttcap%
\pgfsetroundjoin%
\definecolor{currentfill}{rgb}{0.000000,0.000000,0.000000}%
\pgfsetfillcolor{currentfill}%
\pgfsetlinewidth{0.501875pt}%
\definecolor{currentstroke}{rgb}{0.000000,0.000000,0.000000}%
\pgfsetstrokecolor{currentstroke}%
\pgfsetdash{}{0pt}%
\pgfsys@defobject{currentmarker}{\pgfqpoint{0.000000in}{-0.020833in}}{\pgfqpoint{0.000000in}{0.000000in}}{%
\pgfpathmoveto{\pgfqpoint{0.000000in}{0.000000in}}%
\pgfpathlineto{\pgfqpoint{0.000000in}{-0.020833in}}%
\pgfusepath{stroke,fill}%
}%
\begin{pgfscope}%
\pgfsys@transformshift{2.793499in}{4.374193in}%
\pgfsys@useobject{currentmarker}{}%
\end{pgfscope}%
\end{pgfscope}%
\begin{pgfscope}%
\definecolor{textcolor}{rgb}{0.000000,0.000000,0.000000}%
\pgfsetstrokecolor{textcolor}%
\pgfsetfillcolor{textcolor}%
\pgftext[x=1.678002in,y=2.509413in,,top]{\color{textcolor}\rmfamily\fontsize{10.000000}{12.000000}\selectfont \(\displaystyle K\)}%
\end{pgfscope}%
\begin{pgfscope}%
\pgfsetbuttcap%
\pgfsetroundjoin%
\definecolor{currentfill}{rgb}{0.000000,0.000000,0.000000}%
\pgfsetfillcolor{currentfill}%
\pgfsetlinewidth{0.501875pt}%
\definecolor{currentstroke}{rgb}{0.000000,0.000000,0.000000}%
\pgfsetstrokecolor{currentstroke}%
\pgfsetdash{}{0pt}%
\pgfsys@defobject{currentmarker}{\pgfqpoint{0.000000in}{0.000000in}}{\pgfqpoint{0.041667in}{0.000000in}}{%
\pgfpathmoveto{\pgfqpoint{0.000000in}{0.000000in}}%
\pgfpathlineto{\pgfqpoint{0.041667in}{0.000000in}}%
\pgfusepath{stroke,fill}%
}%
\begin{pgfscope}%
\pgfsys@transformshift{0.539970in}{2.989503in}%
\pgfsys@useobject{currentmarker}{}%
\end{pgfscope}%
\end{pgfscope}%
\begin{pgfscope}%
\pgfsetbuttcap%
\pgfsetroundjoin%
\definecolor{currentfill}{rgb}{0.000000,0.000000,0.000000}%
\pgfsetfillcolor{currentfill}%
\pgfsetlinewidth{0.501875pt}%
\definecolor{currentstroke}{rgb}{0.000000,0.000000,0.000000}%
\pgfsetstrokecolor{currentstroke}%
\pgfsetdash{}{0pt}%
\pgfsys@defobject{currentmarker}{\pgfqpoint{-0.041667in}{0.000000in}}{\pgfqpoint{-0.000000in}{0.000000in}}{%
\pgfpathmoveto{\pgfqpoint{-0.000000in}{0.000000in}}%
\pgfpathlineto{\pgfqpoint{-0.041667in}{0.000000in}}%
\pgfusepath{stroke,fill}%
}%
\begin{pgfscope}%
\pgfsys@transformshift{2.816034in}{2.989503in}%
\pgfsys@useobject{currentmarker}{}%
\end{pgfscope}%
\end{pgfscope}%
\begin{pgfscope}%
\definecolor{textcolor}{rgb}{0.000000,0.000000,0.000000}%
\pgfsetstrokecolor{textcolor}%
\pgfsetfillcolor{textcolor}%
\pgftext[x=0.244444in, y=2.936741in, left, base]{\color{textcolor}\rmfamily\fontsize{10.000000}{12.000000}\selectfont \(\displaystyle {0.80}\)}%
\end{pgfscope}%
\begin{pgfscope}%
\pgfsetbuttcap%
\pgfsetroundjoin%
\definecolor{currentfill}{rgb}{0.000000,0.000000,0.000000}%
\pgfsetfillcolor{currentfill}%
\pgfsetlinewidth{0.501875pt}%
\definecolor{currentstroke}{rgb}{0.000000,0.000000,0.000000}%
\pgfsetstrokecolor{currentstroke}%
\pgfsetdash{}{0pt}%
\pgfsys@defobject{currentmarker}{\pgfqpoint{0.000000in}{0.000000in}}{\pgfqpoint{0.041667in}{0.000000in}}{%
\pgfpathmoveto{\pgfqpoint{0.000000in}{0.000000in}}%
\pgfpathlineto{\pgfqpoint{0.041667in}{0.000000in}}%
\pgfusepath{stroke,fill}%
}%
\begin{pgfscope}%
\pgfsys@transformshift{0.539970in}{3.323223in}%
\pgfsys@useobject{currentmarker}{}%
\end{pgfscope}%
\end{pgfscope}%
\begin{pgfscope}%
\pgfsetbuttcap%
\pgfsetroundjoin%
\definecolor{currentfill}{rgb}{0.000000,0.000000,0.000000}%
\pgfsetfillcolor{currentfill}%
\pgfsetlinewidth{0.501875pt}%
\definecolor{currentstroke}{rgb}{0.000000,0.000000,0.000000}%
\pgfsetstrokecolor{currentstroke}%
\pgfsetdash{}{0pt}%
\pgfsys@defobject{currentmarker}{\pgfqpoint{-0.041667in}{0.000000in}}{\pgfqpoint{-0.000000in}{0.000000in}}{%
\pgfpathmoveto{\pgfqpoint{-0.000000in}{0.000000in}}%
\pgfpathlineto{\pgfqpoint{-0.041667in}{0.000000in}}%
\pgfusepath{stroke,fill}%
}%
\begin{pgfscope}%
\pgfsys@transformshift{2.816034in}{3.323223in}%
\pgfsys@useobject{currentmarker}{}%
\end{pgfscope}%
\end{pgfscope}%
\begin{pgfscope}%
\definecolor{textcolor}{rgb}{0.000000,0.000000,0.000000}%
\pgfsetstrokecolor{textcolor}%
\pgfsetfillcolor{textcolor}%
\pgftext[x=0.244444in, y=3.270461in, left, base]{\color{textcolor}\rmfamily\fontsize{10.000000}{12.000000}\selectfont \(\displaystyle {0.85}\)}%
\end{pgfscope}%
\begin{pgfscope}%
\pgfsetbuttcap%
\pgfsetroundjoin%
\definecolor{currentfill}{rgb}{0.000000,0.000000,0.000000}%
\pgfsetfillcolor{currentfill}%
\pgfsetlinewidth{0.501875pt}%
\definecolor{currentstroke}{rgb}{0.000000,0.000000,0.000000}%
\pgfsetstrokecolor{currentstroke}%
\pgfsetdash{}{0pt}%
\pgfsys@defobject{currentmarker}{\pgfqpoint{0.000000in}{0.000000in}}{\pgfqpoint{0.041667in}{0.000000in}}{%
\pgfpathmoveto{\pgfqpoint{0.000000in}{0.000000in}}%
\pgfpathlineto{\pgfqpoint{0.041667in}{0.000000in}}%
\pgfusepath{stroke,fill}%
}%
\begin{pgfscope}%
\pgfsys@transformshift{0.539970in}{3.656942in}%
\pgfsys@useobject{currentmarker}{}%
\end{pgfscope}%
\end{pgfscope}%
\begin{pgfscope}%
\pgfsetbuttcap%
\pgfsetroundjoin%
\definecolor{currentfill}{rgb}{0.000000,0.000000,0.000000}%
\pgfsetfillcolor{currentfill}%
\pgfsetlinewidth{0.501875pt}%
\definecolor{currentstroke}{rgb}{0.000000,0.000000,0.000000}%
\pgfsetstrokecolor{currentstroke}%
\pgfsetdash{}{0pt}%
\pgfsys@defobject{currentmarker}{\pgfqpoint{-0.041667in}{0.000000in}}{\pgfqpoint{-0.000000in}{0.000000in}}{%
\pgfpathmoveto{\pgfqpoint{-0.000000in}{0.000000in}}%
\pgfpathlineto{\pgfqpoint{-0.041667in}{0.000000in}}%
\pgfusepath{stroke,fill}%
}%
\begin{pgfscope}%
\pgfsys@transformshift{2.816034in}{3.656942in}%
\pgfsys@useobject{currentmarker}{}%
\end{pgfscope}%
\end{pgfscope}%
\begin{pgfscope}%
\definecolor{textcolor}{rgb}{0.000000,0.000000,0.000000}%
\pgfsetstrokecolor{textcolor}%
\pgfsetfillcolor{textcolor}%
\pgftext[x=0.244444in, y=3.604181in, left, base]{\color{textcolor}\rmfamily\fontsize{10.000000}{12.000000}\selectfont \(\displaystyle {0.90}\)}%
\end{pgfscope}%
\begin{pgfscope}%
\pgfsetbuttcap%
\pgfsetroundjoin%
\definecolor{currentfill}{rgb}{0.000000,0.000000,0.000000}%
\pgfsetfillcolor{currentfill}%
\pgfsetlinewidth{0.501875pt}%
\definecolor{currentstroke}{rgb}{0.000000,0.000000,0.000000}%
\pgfsetstrokecolor{currentstroke}%
\pgfsetdash{}{0pt}%
\pgfsys@defobject{currentmarker}{\pgfqpoint{0.000000in}{0.000000in}}{\pgfqpoint{0.041667in}{0.000000in}}{%
\pgfpathmoveto{\pgfqpoint{0.000000in}{0.000000in}}%
\pgfpathlineto{\pgfqpoint{0.041667in}{0.000000in}}%
\pgfusepath{stroke,fill}%
}%
\begin{pgfscope}%
\pgfsys@transformshift{0.539970in}{3.990662in}%
\pgfsys@useobject{currentmarker}{}%
\end{pgfscope}%
\end{pgfscope}%
\begin{pgfscope}%
\pgfsetbuttcap%
\pgfsetroundjoin%
\definecolor{currentfill}{rgb}{0.000000,0.000000,0.000000}%
\pgfsetfillcolor{currentfill}%
\pgfsetlinewidth{0.501875pt}%
\definecolor{currentstroke}{rgb}{0.000000,0.000000,0.000000}%
\pgfsetstrokecolor{currentstroke}%
\pgfsetdash{}{0pt}%
\pgfsys@defobject{currentmarker}{\pgfqpoint{-0.041667in}{0.000000in}}{\pgfqpoint{-0.000000in}{0.000000in}}{%
\pgfpathmoveto{\pgfqpoint{-0.000000in}{0.000000in}}%
\pgfpathlineto{\pgfqpoint{-0.041667in}{0.000000in}}%
\pgfusepath{stroke,fill}%
}%
\begin{pgfscope}%
\pgfsys@transformshift{2.816034in}{3.990662in}%
\pgfsys@useobject{currentmarker}{}%
\end{pgfscope}%
\end{pgfscope}%
\begin{pgfscope}%
\definecolor{textcolor}{rgb}{0.000000,0.000000,0.000000}%
\pgfsetstrokecolor{textcolor}%
\pgfsetfillcolor{textcolor}%
\pgftext[x=0.244444in, y=3.937900in, left, base]{\color{textcolor}\rmfamily\fontsize{10.000000}{12.000000}\selectfont \(\displaystyle {0.95}\)}%
\end{pgfscope}%
\begin{pgfscope}%
\pgfsetbuttcap%
\pgfsetroundjoin%
\definecolor{currentfill}{rgb}{0.000000,0.000000,0.000000}%
\pgfsetfillcolor{currentfill}%
\pgfsetlinewidth{0.501875pt}%
\definecolor{currentstroke}{rgb}{0.000000,0.000000,0.000000}%
\pgfsetstrokecolor{currentstroke}%
\pgfsetdash{}{0pt}%
\pgfsys@defobject{currentmarker}{\pgfqpoint{0.000000in}{0.000000in}}{\pgfqpoint{0.041667in}{0.000000in}}{%
\pgfpathmoveto{\pgfqpoint{0.000000in}{0.000000in}}%
\pgfpathlineto{\pgfqpoint{0.041667in}{0.000000in}}%
\pgfusepath{stroke,fill}%
}%
\begin{pgfscope}%
\pgfsys@transformshift{0.539970in}{4.324382in}%
\pgfsys@useobject{currentmarker}{}%
\end{pgfscope}%
\end{pgfscope}%
\begin{pgfscope}%
\pgfsetbuttcap%
\pgfsetroundjoin%
\definecolor{currentfill}{rgb}{0.000000,0.000000,0.000000}%
\pgfsetfillcolor{currentfill}%
\pgfsetlinewidth{0.501875pt}%
\definecolor{currentstroke}{rgb}{0.000000,0.000000,0.000000}%
\pgfsetstrokecolor{currentstroke}%
\pgfsetdash{}{0pt}%
\pgfsys@defobject{currentmarker}{\pgfqpoint{-0.041667in}{0.000000in}}{\pgfqpoint{-0.000000in}{0.000000in}}{%
\pgfpathmoveto{\pgfqpoint{-0.000000in}{0.000000in}}%
\pgfpathlineto{\pgfqpoint{-0.041667in}{0.000000in}}%
\pgfusepath{stroke,fill}%
}%
\begin{pgfscope}%
\pgfsys@transformshift{2.816034in}{4.324382in}%
\pgfsys@useobject{currentmarker}{}%
\end{pgfscope}%
\end{pgfscope}%
\begin{pgfscope}%
\definecolor{textcolor}{rgb}{0.000000,0.000000,0.000000}%
\pgfsetstrokecolor{textcolor}%
\pgfsetfillcolor{textcolor}%
\pgftext[x=0.244444in, y=4.271620in, left, base]{\color{textcolor}\rmfamily\fontsize{10.000000}{12.000000}\selectfont \(\displaystyle {1.00}\)}%
\end{pgfscope}%
\begin{pgfscope}%
\pgfsetbuttcap%
\pgfsetroundjoin%
\definecolor{currentfill}{rgb}{0.000000,0.000000,0.000000}%
\pgfsetfillcolor{currentfill}%
\pgfsetlinewidth{0.501875pt}%
\definecolor{currentstroke}{rgb}{0.000000,0.000000,0.000000}%
\pgfsetstrokecolor{currentstroke}%
\pgfsetdash{}{0pt}%
\pgfsys@defobject{currentmarker}{\pgfqpoint{0.000000in}{0.000000in}}{\pgfqpoint{0.020833in}{0.000000in}}{%
\pgfpathmoveto{\pgfqpoint{0.000000in}{0.000000in}}%
\pgfpathlineto{\pgfqpoint{0.020833in}{0.000000in}}%
\pgfusepath{stroke,fill}%
}%
\begin{pgfscope}%
\pgfsys@transformshift{0.539970in}{2.789271in}%
\pgfsys@useobject{currentmarker}{}%
\end{pgfscope}%
\end{pgfscope}%
\begin{pgfscope}%
\pgfsetbuttcap%
\pgfsetroundjoin%
\definecolor{currentfill}{rgb}{0.000000,0.000000,0.000000}%
\pgfsetfillcolor{currentfill}%
\pgfsetlinewidth{0.501875pt}%
\definecolor{currentstroke}{rgb}{0.000000,0.000000,0.000000}%
\pgfsetstrokecolor{currentstroke}%
\pgfsetdash{}{0pt}%
\pgfsys@defobject{currentmarker}{\pgfqpoint{-0.020833in}{0.000000in}}{\pgfqpoint{-0.000000in}{0.000000in}}{%
\pgfpathmoveto{\pgfqpoint{-0.000000in}{0.000000in}}%
\pgfpathlineto{\pgfqpoint{-0.020833in}{0.000000in}}%
\pgfusepath{stroke,fill}%
}%
\begin{pgfscope}%
\pgfsys@transformshift{2.816034in}{2.789271in}%
\pgfsys@useobject{currentmarker}{}%
\end{pgfscope}%
\end{pgfscope}%
\begin{pgfscope}%
\pgfsetbuttcap%
\pgfsetroundjoin%
\definecolor{currentfill}{rgb}{0.000000,0.000000,0.000000}%
\pgfsetfillcolor{currentfill}%
\pgfsetlinewidth{0.501875pt}%
\definecolor{currentstroke}{rgb}{0.000000,0.000000,0.000000}%
\pgfsetstrokecolor{currentstroke}%
\pgfsetdash{}{0pt}%
\pgfsys@defobject{currentmarker}{\pgfqpoint{0.000000in}{0.000000in}}{\pgfqpoint{0.020833in}{0.000000in}}{%
\pgfpathmoveto{\pgfqpoint{0.000000in}{0.000000in}}%
\pgfpathlineto{\pgfqpoint{0.020833in}{0.000000in}}%
\pgfusepath{stroke,fill}%
}%
\begin{pgfscope}%
\pgfsys@transformshift{0.539970in}{2.856015in}%
\pgfsys@useobject{currentmarker}{}%
\end{pgfscope}%
\end{pgfscope}%
\begin{pgfscope}%
\pgfsetbuttcap%
\pgfsetroundjoin%
\definecolor{currentfill}{rgb}{0.000000,0.000000,0.000000}%
\pgfsetfillcolor{currentfill}%
\pgfsetlinewidth{0.501875pt}%
\definecolor{currentstroke}{rgb}{0.000000,0.000000,0.000000}%
\pgfsetstrokecolor{currentstroke}%
\pgfsetdash{}{0pt}%
\pgfsys@defobject{currentmarker}{\pgfqpoint{-0.020833in}{0.000000in}}{\pgfqpoint{-0.000000in}{0.000000in}}{%
\pgfpathmoveto{\pgfqpoint{-0.000000in}{0.000000in}}%
\pgfpathlineto{\pgfqpoint{-0.020833in}{0.000000in}}%
\pgfusepath{stroke,fill}%
}%
\begin{pgfscope}%
\pgfsys@transformshift{2.816034in}{2.856015in}%
\pgfsys@useobject{currentmarker}{}%
\end{pgfscope}%
\end{pgfscope}%
\begin{pgfscope}%
\pgfsetbuttcap%
\pgfsetroundjoin%
\definecolor{currentfill}{rgb}{0.000000,0.000000,0.000000}%
\pgfsetfillcolor{currentfill}%
\pgfsetlinewidth{0.501875pt}%
\definecolor{currentstroke}{rgb}{0.000000,0.000000,0.000000}%
\pgfsetstrokecolor{currentstroke}%
\pgfsetdash{}{0pt}%
\pgfsys@defobject{currentmarker}{\pgfqpoint{0.000000in}{0.000000in}}{\pgfqpoint{0.020833in}{0.000000in}}{%
\pgfpathmoveto{\pgfqpoint{0.000000in}{0.000000in}}%
\pgfpathlineto{\pgfqpoint{0.020833in}{0.000000in}}%
\pgfusepath{stroke,fill}%
}%
\begin{pgfscope}%
\pgfsys@transformshift{0.539970in}{2.922759in}%
\pgfsys@useobject{currentmarker}{}%
\end{pgfscope}%
\end{pgfscope}%
\begin{pgfscope}%
\pgfsetbuttcap%
\pgfsetroundjoin%
\definecolor{currentfill}{rgb}{0.000000,0.000000,0.000000}%
\pgfsetfillcolor{currentfill}%
\pgfsetlinewidth{0.501875pt}%
\definecolor{currentstroke}{rgb}{0.000000,0.000000,0.000000}%
\pgfsetstrokecolor{currentstroke}%
\pgfsetdash{}{0pt}%
\pgfsys@defobject{currentmarker}{\pgfqpoint{-0.020833in}{0.000000in}}{\pgfqpoint{-0.000000in}{0.000000in}}{%
\pgfpathmoveto{\pgfqpoint{-0.000000in}{0.000000in}}%
\pgfpathlineto{\pgfqpoint{-0.020833in}{0.000000in}}%
\pgfusepath{stroke,fill}%
}%
\begin{pgfscope}%
\pgfsys@transformshift{2.816034in}{2.922759in}%
\pgfsys@useobject{currentmarker}{}%
\end{pgfscope}%
\end{pgfscope}%
\begin{pgfscope}%
\pgfsetbuttcap%
\pgfsetroundjoin%
\definecolor{currentfill}{rgb}{0.000000,0.000000,0.000000}%
\pgfsetfillcolor{currentfill}%
\pgfsetlinewidth{0.501875pt}%
\definecolor{currentstroke}{rgb}{0.000000,0.000000,0.000000}%
\pgfsetstrokecolor{currentstroke}%
\pgfsetdash{}{0pt}%
\pgfsys@defobject{currentmarker}{\pgfqpoint{0.000000in}{0.000000in}}{\pgfqpoint{0.020833in}{0.000000in}}{%
\pgfpathmoveto{\pgfqpoint{0.000000in}{0.000000in}}%
\pgfpathlineto{\pgfqpoint{0.020833in}{0.000000in}}%
\pgfusepath{stroke,fill}%
}%
\begin{pgfscope}%
\pgfsys@transformshift{0.539970in}{3.056247in}%
\pgfsys@useobject{currentmarker}{}%
\end{pgfscope}%
\end{pgfscope}%
\begin{pgfscope}%
\pgfsetbuttcap%
\pgfsetroundjoin%
\definecolor{currentfill}{rgb}{0.000000,0.000000,0.000000}%
\pgfsetfillcolor{currentfill}%
\pgfsetlinewidth{0.501875pt}%
\definecolor{currentstroke}{rgb}{0.000000,0.000000,0.000000}%
\pgfsetstrokecolor{currentstroke}%
\pgfsetdash{}{0pt}%
\pgfsys@defobject{currentmarker}{\pgfqpoint{-0.020833in}{0.000000in}}{\pgfqpoint{-0.000000in}{0.000000in}}{%
\pgfpathmoveto{\pgfqpoint{-0.000000in}{0.000000in}}%
\pgfpathlineto{\pgfqpoint{-0.020833in}{0.000000in}}%
\pgfusepath{stroke,fill}%
}%
\begin{pgfscope}%
\pgfsys@transformshift{2.816034in}{3.056247in}%
\pgfsys@useobject{currentmarker}{}%
\end{pgfscope}%
\end{pgfscope}%
\begin{pgfscope}%
\pgfsetbuttcap%
\pgfsetroundjoin%
\definecolor{currentfill}{rgb}{0.000000,0.000000,0.000000}%
\pgfsetfillcolor{currentfill}%
\pgfsetlinewidth{0.501875pt}%
\definecolor{currentstroke}{rgb}{0.000000,0.000000,0.000000}%
\pgfsetstrokecolor{currentstroke}%
\pgfsetdash{}{0pt}%
\pgfsys@defobject{currentmarker}{\pgfqpoint{0.000000in}{0.000000in}}{\pgfqpoint{0.020833in}{0.000000in}}{%
\pgfpathmoveto{\pgfqpoint{0.000000in}{0.000000in}}%
\pgfpathlineto{\pgfqpoint{0.020833in}{0.000000in}}%
\pgfusepath{stroke,fill}%
}%
\begin{pgfscope}%
\pgfsys@transformshift{0.539970in}{3.122991in}%
\pgfsys@useobject{currentmarker}{}%
\end{pgfscope}%
\end{pgfscope}%
\begin{pgfscope}%
\pgfsetbuttcap%
\pgfsetroundjoin%
\definecolor{currentfill}{rgb}{0.000000,0.000000,0.000000}%
\pgfsetfillcolor{currentfill}%
\pgfsetlinewidth{0.501875pt}%
\definecolor{currentstroke}{rgb}{0.000000,0.000000,0.000000}%
\pgfsetstrokecolor{currentstroke}%
\pgfsetdash{}{0pt}%
\pgfsys@defobject{currentmarker}{\pgfqpoint{-0.020833in}{0.000000in}}{\pgfqpoint{-0.000000in}{0.000000in}}{%
\pgfpathmoveto{\pgfqpoint{-0.000000in}{0.000000in}}%
\pgfpathlineto{\pgfqpoint{-0.020833in}{0.000000in}}%
\pgfusepath{stroke,fill}%
}%
\begin{pgfscope}%
\pgfsys@transformshift{2.816034in}{3.122991in}%
\pgfsys@useobject{currentmarker}{}%
\end{pgfscope}%
\end{pgfscope}%
\begin{pgfscope}%
\pgfsetbuttcap%
\pgfsetroundjoin%
\definecolor{currentfill}{rgb}{0.000000,0.000000,0.000000}%
\pgfsetfillcolor{currentfill}%
\pgfsetlinewidth{0.501875pt}%
\definecolor{currentstroke}{rgb}{0.000000,0.000000,0.000000}%
\pgfsetstrokecolor{currentstroke}%
\pgfsetdash{}{0pt}%
\pgfsys@defobject{currentmarker}{\pgfqpoint{0.000000in}{0.000000in}}{\pgfqpoint{0.020833in}{0.000000in}}{%
\pgfpathmoveto{\pgfqpoint{0.000000in}{0.000000in}}%
\pgfpathlineto{\pgfqpoint{0.020833in}{0.000000in}}%
\pgfusepath{stroke,fill}%
}%
\begin{pgfscope}%
\pgfsys@transformshift{0.539970in}{3.189735in}%
\pgfsys@useobject{currentmarker}{}%
\end{pgfscope}%
\end{pgfscope}%
\begin{pgfscope}%
\pgfsetbuttcap%
\pgfsetroundjoin%
\definecolor{currentfill}{rgb}{0.000000,0.000000,0.000000}%
\pgfsetfillcolor{currentfill}%
\pgfsetlinewidth{0.501875pt}%
\definecolor{currentstroke}{rgb}{0.000000,0.000000,0.000000}%
\pgfsetstrokecolor{currentstroke}%
\pgfsetdash{}{0pt}%
\pgfsys@defobject{currentmarker}{\pgfqpoint{-0.020833in}{0.000000in}}{\pgfqpoint{-0.000000in}{0.000000in}}{%
\pgfpathmoveto{\pgfqpoint{-0.000000in}{0.000000in}}%
\pgfpathlineto{\pgfqpoint{-0.020833in}{0.000000in}}%
\pgfusepath{stroke,fill}%
}%
\begin{pgfscope}%
\pgfsys@transformshift{2.816034in}{3.189735in}%
\pgfsys@useobject{currentmarker}{}%
\end{pgfscope}%
\end{pgfscope}%
\begin{pgfscope}%
\pgfsetbuttcap%
\pgfsetroundjoin%
\definecolor{currentfill}{rgb}{0.000000,0.000000,0.000000}%
\pgfsetfillcolor{currentfill}%
\pgfsetlinewidth{0.501875pt}%
\definecolor{currentstroke}{rgb}{0.000000,0.000000,0.000000}%
\pgfsetstrokecolor{currentstroke}%
\pgfsetdash{}{0pt}%
\pgfsys@defobject{currentmarker}{\pgfqpoint{0.000000in}{0.000000in}}{\pgfqpoint{0.020833in}{0.000000in}}{%
\pgfpathmoveto{\pgfqpoint{0.000000in}{0.000000in}}%
\pgfpathlineto{\pgfqpoint{0.020833in}{0.000000in}}%
\pgfusepath{stroke,fill}%
}%
\begin{pgfscope}%
\pgfsys@transformshift{0.539970in}{3.256479in}%
\pgfsys@useobject{currentmarker}{}%
\end{pgfscope}%
\end{pgfscope}%
\begin{pgfscope}%
\pgfsetbuttcap%
\pgfsetroundjoin%
\definecolor{currentfill}{rgb}{0.000000,0.000000,0.000000}%
\pgfsetfillcolor{currentfill}%
\pgfsetlinewidth{0.501875pt}%
\definecolor{currentstroke}{rgb}{0.000000,0.000000,0.000000}%
\pgfsetstrokecolor{currentstroke}%
\pgfsetdash{}{0pt}%
\pgfsys@defobject{currentmarker}{\pgfqpoint{-0.020833in}{0.000000in}}{\pgfqpoint{-0.000000in}{0.000000in}}{%
\pgfpathmoveto{\pgfqpoint{-0.000000in}{0.000000in}}%
\pgfpathlineto{\pgfqpoint{-0.020833in}{0.000000in}}%
\pgfusepath{stroke,fill}%
}%
\begin{pgfscope}%
\pgfsys@transformshift{2.816034in}{3.256479in}%
\pgfsys@useobject{currentmarker}{}%
\end{pgfscope}%
\end{pgfscope}%
\begin{pgfscope}%
\pgfsetbuttcap%
\pgfsetroundjoin%
\definecolor{currentfill}{rgb}{0.000000,0.000000,0.000000}%
\pgfsetfillcolor{currentfill}%
\pgfsetlinewidth{0.501875pt}%
\definecolor{currentstroke}{rgb}{0.000000,0.000000,0.000000}%
\pgfsetstrokecolor{currentstroke}%
\pgfsetdash{}{0pt}%
\pgfsys@defobject{currentmarker}{\pgfqpoint{0.000000in}{0.000000in}}{\pgfqpoint{0.020833in}{0.000000in}}{%
\pgfpathmoveto{\pgfqpoint{0.000000in}{0.000000in}}%
\pgfpathlineto{\pgfqpoint{0.020833in}{0.000000in}}%
\pgfusepath{stroke,fill}%
}%
\begin{pgfscope}%
\pgfsys@transformshift{0.539970in}{3.389966in}%
\pgfsys@useobject{currentmarker}{}%
\end{pgfscope}%
\end{pgfscope}%
\begin{pgfscope}%
\pgfsetbuttcap%
\pgfsetroundjoin%
\definecolor{currentfill}{rgb}{0.000000,0.000000,0.000000}%
\pgfsetfillcolor{currentfill}%
\pgfsetlinewidth{0.501875pt}%
\definecolor{currentstroke}{rgb}{0.000000,0.000000,0.000000}%
\pgfsetstrokecolor{currentstroke}%
\pgfsetdash{}{0pt}%
\pgfsys@defobject{currentmarker}{\pgfqpoint{-0.020833in}{0.000000in}}{\pgfqpoint{-0.000000in}{0.000000in}}{%
\pgfpathmoveto{\pgfqpoint{-0.000000in}{0.000000in}}%
\pgfpathlineto{\pgfqpoint{-0.020833in}{0.000000in}}%
\pgfusepath{stroke,fill}%
}%
\begin{pgfscope}%
\pgfsys@transformshift{2.816034in}{3.389966in}%
\pgfsys@useobject{currentmarker}{}%
\end{pgfscope}%
\end{pgfscope}%
\begin{pgfscope}%
\pgfsetbuttcap%
\pgfsetroundjoin%
\definecolor{currentfill}{rgb}{0.000000,0.000000,0.000000}%
\pgfsetfillcolor{currentfill}%
\pgfsetlinewidth{0.501875pt}%
\definecolor{currentstroke}{rgb}{0.000000,0.000000,0.000000}%
\pgfsetstrokecolor{currentstroke}%
\pgfsetdash{}{0pt}%
\pgfsys@defobject{currentmarker}{\pgfqpoint{0.000000in}{0.000000in}}{\pgfqpoint{0.020833in}{0.000000in}}{%
\pgfpathmoveto{\pgfqpoint{0.000000in}{0.000000in}}%
\pgfpathlineto{\pgfqpoint{0.020833in}{0.000000in}}%
\pgfusepath{stroke,fill}%
}%
\begin{pgfscope}%
\pgfsys@transformshift{0.539970in}{3.456710in}%
\pgfsys@useobject{currentmarker}{}%
\end{pgfscope}%
\end{pgfscope}%
\begin{pgfscope}%
\pgfsetbuttcap%
\pgfsetroundjoin%
\definecolor{currentfill}{rgb}{0.000000,0.000000,0.000000}%
\pgfsetfillcolor{currentfill}%
\pgfsetlinewidth{0.501875pt}%
\definecolor{currentstroke}{rgb}{0.000000,0.000000,0.000000}%
\pgfsetstrokecolor{currentstroke}%
\pgfsetdash{}{0pt}%
\pgfsys@defobject{currentmarker}{\pgfqpoint{-0.020833in}{0.000000in}}{\pgfqpoint{-0.000000in}{0.000000in}}{%
\pgfpathmoveto{\pgfqpoint{-0.000000in}{0.000000in}}%
\pgfpathlineto{\pgfqpoint{-0.020833in}{0.000000in}}%
\pgfusepath{stroke,fill}%
}%
\begin{pgfscope}%
\pgfsys@transformshift{2.816034in}{3.456710in}%
\pgfsys@useobject{currentmarker}{}%
\end{pgfscope}%
\end{pgfscope}%
\begin{pgfscope}%
\pgfsetbuttcap%
\pgfsetroundjoin%
\definecolor{currentfill}{rgb}{0.000000,0.000000,0.000000}%
\pgfsetfillcolor{currentfill}%
\pgfsetlinewidth{0.501875pt}%
\definecolor{currentstroke}{rgb}{0.000000,0.000000,0.000000}%
\pgfsetstrokecolor{currentstroke}%
\pgfsetdash{}{0pt}%
\pgfsys@defobject{currentmarker}{\pgfqpoint{0.000000in}{0.000000in}}{\pgfqpoint{0.020833in}{0.000000in}}{%
\pgfpathmoveto{\pgfqpoint{0.000000in}{0.000000in}}%
\pgfpathlineto{\pgfqpoint{0.020833in}{0.000000in}}%
\pgfusepath{stroke,fill}%
}%
\begin{pgfscope}%
\pgfsys@transformshift{0.539970in}{3.523454in}%
\pgfsys@useobject{currentmarker}{}%
\end{pgfscope}%
\end{pgfscope}%
\begin{pgfscope}%
\pgfsetbuttcap%
\pgfsetroundjoin%
\definecolor{currentfill}{rgb}{0.000000,0.000000,0.000000}%
\pgfsetfillcolor{currentfill}%
\pgfsetlinewidth{0.501875pt}%
\definecolor{currentstroke}{rgb}{0.000000,0.000000,0.000000}%
\pgfsetstrokecolor{currentstroke}%
\pgfsetdash{}{0pt}%
\pgfsys@defobject{currentmarker}{\pgfqpoint{-0.020833in}{0.000000in}}{\pgfqpoint{-0.000000in}{0.000000in}}{%
\pgfpathmoveto{\pgfqpoint{-0.000000in}{0.000000in}}%
\pgfpathlineto{\pgfqpoint{-0.020833in}{0.000000in}}%
\pgfusepath{stroke,fill}%
}%
\begin{pgfscope}%
\pgfsys@transformshift{2.816034in}{3.523454in}%
\pgfsys@useobject{currentmarker}{}%
\end{pgfscope}%
\end{pgfscope}%
\begin{pgfscope}%
\pgfsetbuttcap%
\pgfsetroundjoin%
\definecolor{currentfill}{rgb}{0.000000,0.000000,0.000000}%
\pgfsetfillcolor{currentfill}%
\pgfsetlinewidth{0.501875pt}%
\definecolor{currentstroke}{rgb}{0.000000,0.000000,0.000000}%
\pgfsetstrokecolor{currentstroke}%
\pgfsetdash{}{0pt}%
\pgfsys@defobject{currentmarker}{\pgfqpoint{0.000000in}{0.000000in}}{\pgfqpoint{0.020833in}{0.000000in}}{%
\pgfpathmoveto{\pgfqpoint{0.000000in}{0.000000in}}%
\pgfpathlineto{\pgfqpoint{0.020833in}{0.000000in}}%
\pgfusepath{stroke,fill}%
}%
\begin{pgfscope}%
\pgfsys@transformshift{0.539970in}{3.590198in}%
\pgfsys@useobject{currentmarker}{}%
\end{pgfscope}%
\end{pgfscope}%
\begin{pgfscope}%
\pgfsetbuttcap%
\pgfsetroundjoin%
\definecolor{currentfill}{rgb}{0.000000,0.000000,0.000000}%
\pgfsetfillcolor{currentfill}%
\pgfsetlinewidth{0.501875pt}%
\definecolor{currentstroke}{rgb}{0.000000,0.000000,0.000000}%
\pgfsetstrokecolor{currentstroke}%
\pgfsetdash{}{0pt}%
\pgfsys@defobject{currentmarker}{\pgfqpoint{-0.020833in}{0.000000in}}{\pgfqpoint{-0.000000in}{0.000000in}}{%
\pgfpathmoveto{\pgfqpoint{-0.000000in}{0.000000in}}%
\pgfpathlineto{\pgfqpoint{-0.020833in}{0.000000in}}%
\pgfusepath{stroke,fill}%
}%
\begin{pgfscope}%
\pgfsys@transformshift{2.816034in}{3.590198in}%
\pgfsys@useobject{currentmarker}{}%
\end{pgfscope}%
\end{pgfscope}%
\begin{pgfscope}%
\pgfsetbuttcap%
\pgfsetroundjoin%
\definecolor{currentfill}{rgb}{0.000000,0.000000,0.000000}%
\pgfsetfillcolor{currentfill}%
\pgfsetlinewidth{0.501875pt}%
\definecolor{currentstroke}{rgb}{0.000000,0.000000,0.000000}%
\pgfsetstrokecolor{currentstroke}%
\pgfsetdash{}{0pt}%
\pgfsys@defobject{currentmarker}{\pgfqpoint{0.000000in}{0.000000in}}{\pgfqpoint{0.020833in}{0.000000in}}{%
\pgfpathmoveto{\pgfqpoint{0.000000in}{0.000000in}}%
\pgfpathlineto{\pgfqpoint{0.020833in}{0.000000in}}%
\pgfusepath{stroke,fill}%
}%
\begin{pgfscope}%
\pgfsys@transformshift{0.539970in}{3.723686in}%
\pgfsys@useobject{currentmarker}{}%
\end{pgfscope}%
\end{pgfscope}%
\begin{pgfscope}%
\pgfsetbuttcap%
\pgfsetroundjoin%
\definecolor{currentfill}{rgb}{0.000000,0.000000,0.000000}%
\pgfsetfillcolor{currentfill}%
\pgfsetlinewidth{0.501875pt}%
\definecolor{currentstroke}{rgb}{0.000000,0.000000,0.000000}%
\pgfsetstrokecolor{currentstroke}%
\pgfsetdash{}{0pt}%
\pgfsys@defobject{currentmarker}{\pgfqpoint{-0.020833in}{0.000000in}}{\pgfqpoint{-0.000000in}{0.000000in}}{%
\pgfpathmoveto{\pgfqpoint{-0.000000in}{0.000000in}}%
\pgfpathlineto{\pgfqpoint{-0.020833in}{0.000000in}}%
\pgfusepath{stroke,fill}%
}%
\begin{pgfscope}%
\pgfsys@transformshift{2.816034in}{3.723686in}%
\pgfsys@useobject{currentmarker}{}%
\end{pgfscope}%
\end{pgfscope}%
\begin{pgfscope}%
\pgfsetbuttcap%
\pgfsetroundjoin%
\definecolor{currentfill}{rgb}{0.000000,0.000000,0.000000}%
\pgfsetfillcolor{currentfill}%
\pgfsetlinewidth{0.501875pt}%
\definecolor{currentstroke}{rgb}{0.000000,0.000000,0.000000}%
\pgfsetstrokecolor{currentstroke}%
\pgfsetdash{}{0pt}%
\pgfsys@defobject{currentmarker}{\pgfqpoint{0.000000in}{0.000000in}}{\pgfqpoint{0.020833in}{0.000000in}}{%
\pgfpathmoveto{\pgfqpoint{0.000000in}{0.000000in}}%
\pgfpathlineto{\pgfqpoint{0.020833in}{0.000000in}}%
\pgfusepath{stroke,fill}%
}%
\begin{pgfscope}%
\pgfsys@transformshift{0.539970in}{3.790430in}%
\pgfsys@useobject{currentmarker}{}%
\end{pgfscope}%
\end{pgfscope}%
\begin{pgfscope}%
\pgfsetbuttcap%
\pgfsetroundjoin%
\definecolor{currentfill}{rgb}{0.000000,0.000000,0.000000}%
\pgfsetfillcolor{currentfill}%
\pgfsetlinewidth{0.501875pt}%
\definecolor{currentstroke}{rgb}{0.000000,0.000000,0.000000}%
\pgfsetstrokecolor{currentstroke}%
\pgfsetdash{}{0pt}%
\pgfsys@defobject{currentmarker}{\pgfqpoint{-0.020833in}{0.000000in}}{\pgfqpoint{-0.000000in}{0.000000in}}{%
\pgfpathmoveto{\pgfqpoint{-0.000000in}{0.000000in}}%
\pgfpathlineto{\pgfqpoint{-0.020833in}{0.000000in}}%
\pgfusepath{stroke,fill}%
}%
\begin{pgfscope}%
\pgfsys@transformshift{2.816034in}{3.790430in}%
\pgfsys@useobject{currentmarker}{}%
\end{pgfscope}%
\end{pgfscope}%
\begin{pgfscope}%
\pgfsetbuttcap%
\pgfsetroundjoin%
\definecolor{currentfill}{rgb}{0.000000,0.000000,0.000000}%
\pgfsetfillcolor{currentfill}%
\pgfsetlinewidth{0.501875pt}%
\definecolor{currentstroke}{rgb}{0.000000,0.000000,0.000000}%
\pgfsetstrokecolor{currentstroke}%
\pgfsetdash{}{0pt}%
\pgfsys@defobject{currentmarker}{\pgfqpoint{0.000000in}{0.000000in}}{\pgfqpoint{0.020833in}{0.000000in}}{%
\pgfpathmoveto{\pgfqpoint{0.000000in}{0.000000in}}%
\pgfpathlineto{\pgfqpoint{0.020833in}{0.000000in}}%
\pgfusepath{stroke,fill}%
}%
\begin{pgfscope}%
\pgfsys@transformshift{0.539970in}{3.857174in}%
\pgfsys@useobject{currentmarker}{}%
\end{pgfscope}%
\end{pgfscope}%
\begin{pgfscope}%
\pgfsetbuttcap%
\pgfsetroundjoin%
\definecolor{currentfill}{rgb}{0.000000,0.000000,0.000000}%
\pgfsetfillcolor{currentfill}%
\pgfsetlinewidth{0.501875pt}%
\definecolor{currentstroke}{rgb}{0.000000,0.000000,0.000000}%
\pgfsetstrokecolor{currentstroke}%
\pgfsetdash{}{0pt}%
\pgfsys@defobject{currentmarker}{\pgfqpoint{-0.020833in}{0.000000in}}{\pgfqpoint{-0.000000in}{0.000000in}}{%
\pgfpathmoveto{\pgfqpoint{-0.000000in}{0.000000in}}%
\pgfpathlineto{\pgfqpoint{-0.020833in}{0.000000in}}%
\pgfusepath{stroke,fill}%
}%
\begin{pgfscope}%
\pgfsys@transformshift{2.816034in}{3.857174in}%
\pgfsys@useobject{currentmarker}{}%
\end{pgfscope}%
\end{pgfscope}%
\begin{pgfscope}%
\pgfsetbuttcap%
\pgfsetroundjoin%
\definecolor{currentfill}{rgb}{0.000000,0.000000,0.000000}%
\pgfsetfillcolor{currentfill}%
\pgfsetlinewidth{0.501875pt}%
\definecolor{currentstroke}{rgb}{0.000000,0.000000,0.000000}%
\pgfsetstrokecolor{currentstroke}%
\pgfsetdash{}{0pt}%
\pgfsys@defobject{currentmarker}{\pgfqpoint{0.000000in}{0.000000in}}{\pgfqpoint{0.020833in}{0.000000in}}{%
\pgfpathmoveto{\pgfqpoint{0.000000in}{0.000000in}}%
\pgfpathlineto{\pgfqpoint{0.020833in}{0.000000in}}%
\pgfusepath{stroke,fill}%
}%
\begin{pgfscope}%
\pgfsys@transformshift{0.539970in}{3.923918in}%
\pgfsys@useobject{currentmarker}{}%
\end{pgfscope}%
\end{pgfscope}%
\begin{pgfscope}%
\pgfsetbuttcap%
\pgfsetroundjoin%
\definecolor{currentfill}{rgb}{0.000000,0.000000,0.000000}%
\pgfsetfillcolor{currentfill}%
\pgfsetlinewidth{0.501875pt}%
\definecolor{currentstroke}{rgb}{0.000000,0.000000,0.000000}%
\pgfsetstrokecolor{currentstroke}%
\pgfsetdash{}{0pt}%
\pgfsys@defobject{currentmarker}{\pgfqpoint{-0.020833in}{0.000000in}}{\pgfqpoint{-0.000000in}{0.000000in}}{%
\pgfpathmoveto{\pgfqpoint{-0.000000in}{0.000000in}}%
\pgfpathlineto{\pgfqpoint{-0.020833in}{0.000000in}}%
\pgfusepath{stroke,fill}%
}%
\begin{pgfscope}%
\pgfsys@transformshift{2.816034in}{3.923918in}%
\pgfsys@useobject{currentmarker}{}%
\end{pgfscope}%
\end{pgfscope}%
\begin{pgfscope}%
\pgfsetbuttcap%
\pgfsetroundjoin%
\definecolor{currentfill}{rgb}{0.000000,0.000000,0.000000}%
\pgfsetfillcolor{currentfill}%
\pgfsetlinewidth{0.501875pt}%
\definecolor{currentstroke}{rgb}{0.000000,0.000000,0.000000}%
\pgfsetstrokecolor{currentstroke}%
\pgfsetdash{}{0pt}%
\pgfsys@defobject{currentmarker}{\pgfqpoint{0.000000in}{0.000000in}}{\pgfqpoint{0.020833in}{0.000000in}}{%
\pgfpathmoveto{\pgfqpoint{0.000000in}{0.000000in}}%
\pgfpathlineto{\pgfqpoint{0.020833in}{0.000000in}}%
\pgfusepath{stroke,fill}%
}%
\begin{pgfscope}%
\pgfsys@transformshift{0.539970in}{4.057406in}%
\pgfsys@useobject{currentmarker}{}%
\end{pgfscope}%
\end{pgfscope}%
\begin{pgfscope}%
\pgfsetbuttcap%
\pgfsetroundjoin%
\definecolor{currentfill}{rgb}{0.000000,0.000000,0.000000}%
\pgfsetfillcolor{currentfill}%
\pgfsetlinewidth{0.501875pt}%
\definecolor{currentstroke}{rgb}{0.000000,0.000000,0.000000}%
\pgfsetstrokecolor{currentstroke}%
\pgfsetdash{}{0pt}%
\pgfsys@defobject{currentmarker}{\pgfqpoint{-0.020833in}{0.000000in}}{\pgfqpoint{-0.000000in}{0.000000in}}{%
\pgfpathmoveto{\pgfqpoint{-0.000000in}{0.000000in}}%
\pgfpathlineto{\pgfqpoint{-0.020833in}{0.000000in}}%
\pgfusepath{stroke,fill}%
}%
\begin{pgfscope}%
\pgfsys@transformshift{2.816034in}{4.057406in}%
\pgfsys@useobject{currentmarker}{}%
\end{pgfscope}%
\end{pgfscope}%
\begin{pgfscope}%
\pgfsetbuttcap%
\pgfsetroundjoin%
\definecolor{currentfill}{rgb}{0.000000,0.000000,0.000000}%
\pgfsetfillcolor{currentfill}%
\pgfsetlinewidth{0.501875pt}%
\definecolor{currentstroke}{rgb}{0.000000,0.000000,0.000000}%
\pgfsetstrokecolor{currentstroke}%
\pgfsetdash{}{0pt}%
\pgfsys@defobject{currentmarker}{\pgfqpoint{0.000000in}{0.000000in}}{\pgfqpoint{0.020833in}{0.000000in}}{%
\pgfpathmoveto{\pgfqpoint{0.000000in}{0.000000in}}%
\pgfpathlineto{\pgfqpoint{0.020833in}{0.000000in}}%
\pgfusepath{stroke,fill}%
}%
\begin{pgfscope}%
\pgfsys@transformshift{0.539970in}{4.124150in}%
\pgfsys@useobject{currentmarker}{}%
\end{pgfscope}%
\end{pgfscope}%
\begin{pgfscope}%
\pgfsetbuttcap%
\pgfsetroundjoin%
\definecolor{currentfill}{rgb}{0.000000,0.000000,0.000000}%
\pgfsetfillcolor{currentfill}%
\pgfsetlinewidth{0.501875pt}%
\definecolor{currentstroke}{rgb}{0.000000,0.000000,0.000000}%
\pgfsetstrokecolor{currentstroke}%
\pgfsetdash{}{0pt}%
\pgfsys@defobject{currentmarker}{\pgfqpoint{-0.020833in}{0.000000in}}{\pgfqpoint{-0.000000in}{0.000000in}}{%
\pgfpathmoveto{\pgfqpoint{-0.000000in}{0.000000in}}%
\pgfpathlineto{\pgfqpoint{-0.020833in}{0.000000in}}%
\pgfusepath{stroke,fill}%
}%
\begin{pgfscope}%
\pgfsys@transformshift{2.816034in}{4.124150in}%
\pgfsys@useobject{currentmarker}{}%
\end{pgfscope}%
\end{pgfscope}%
\begin{pgfscope}%
\pgfsetbuttcap%
\pgfsetroundjoin%
\definecolor{currentfill}{rgb}{0.000000,0.000000,0.000000}%
\pgfsetfillcolor{currentfill}%
\pgfsetlinewidth{0.501875pt}%
\definecolor{currentstroke}{rgb}{0.000000,0.000000,0.000000}%
\pgfsetstrokecolor{currentstroke}%
\pgfsetdash{}{0pt}%
\pgfsys@defobject{currentmarker}{\pgfqpoint{0.000000in}{0.000000in}}{\pgfqpoint{0.020833in}{0.000000in}}{%
\pgfpathmoveto{\pgfqpoint{0.000000in}{0.000000in}}%
\pgfpathlineto{\pgfqpoint{0.020833in}{0.000000in}}%
\pgfusepath{stroke,fill}%
}%
\begin{pgfscope}%
\pgfsys@transformshift{0.539970in}{4.190894in}%
\pgfsys@useobject{currentmarker}{}%
\end{pgfscope}%
\end{pgfscope}%
\begin{pgfscope}%
\pgfsetbuttcap%
\pgfsetroundjoin%
\definecolor{currentfill}{rgb}{0.000000,0.000000,0.000000}%
\pgfsetfillcolor{currentfill}%
\pgfsetlinewidth{0.501875pt}%
\definecolor{currentstroke}{rgb}{0.000000,0.000000,0.000000}%
\pgfsetstrokecolor{currentstroke}%
\pgfsetdash{}{0pt}%
\pgfsys@defobject{currentmarker}{\pgfqpoint{-0.020833in}{0.000000in}}{\pgfqpoint{-0.000000in}{0.000000in}}{%
\pgfpathmoveto{\pgfqpoint{-0.000000in}{0.000000in}}%
\pgfpathlineto{\pgfqpoint{-0.020833in}{0.000000in}}%
\pgfusepath{stroke,fill}%
}%
\begin{pgfscope}%
\pgfsys@transformshift{2.816034in}{4.190894in}%
\pgfsys@useobject{currentmarker}{}%
\end{pgfscope}%
\end{pgfscope}%
\begin{pgfscope}%
\pgfsetbuttcap%
\pgfsetroundjoin%
\definecolor{currentfill}{rgb}{0.000000,0.000000,0.000000}%
\pgfsetfillcolor{currentfill}%
\pgfsetlinewidth{0.501875pt}%
\definecolor{currentstroke}{rgb}{0.000000,0.000000,0.000000}%
\pgfsetstrokecolor{currentstroke}%
\pgfsetdash{}{0pt}%
\pgfsys@defobject{currentmarker}{\pgfqpoint{0.000000in}{0.000000in}}{\pgfqpoint{0.020833in}{0.000000in}}{%
\pgfpathmoveto{\pgfqpoint{0.000000in}{0.000000in}}%
\pgfpathlineto{\pgfqpoint{0.020833in}{0.000000in}}%
\pgfusepath{stroke,fill}%
}%
\begin{pgfscope}%
\pgfsys@transformshift{0.539970in}{4.257638in}%
\pgfsys@useobject{currentmarker}{}%
\end{pgfscope}%
\end{pgfscope}%
\begin{pgfscope}%
\pgfsetbuttcap%
\pgfsetroundjoin%
\definecolor{currentfill}{rgb}{0.000000,0.000000,0.000000}%
\pgfsetfillcolor{currentfill}%
\pgfsetlinewidth{0.501875pt}%
\definecolor{currentstroke}{rgb}{0.000000,0.000000,0.000000}%
\pgfsetstrokecolor{currentstroke}%
\pgfsetdash{}{0pt}%
\pgfsys@defobject{currentmarker}{\pgfqpoint{-0.020833in}{0.000000in}}{\pgfqpoint{-0.000000in}{0.000000in}}{%
\pgfpathmoveto{\pgfqpoint{-0.000000in}{0.000000in}}%
\pgfpathlineto{\pgfqpoint{-0.020833in}{0.000000in}}%
\pgfusepath{stroke,fill}%
}%
\begin{pgfscope}%
\pgfsys@transformshift{2.816034in}{4.257638in}%
\pgfsys@useobject{currentmarker}{}%
\end{pgfscope}%
\end{pgfscope}%
\begin{pgfscope}%
\definecolor{textcolor}{rgb}{0.000000,0.000000,0.000000}%
\pgfsetstrokecolor{textcolor}%
\pgfsetfillcolor{textcolor}%
\pgftext[x=0.188889in,y=3.561093in,,bottom,rotate=90.000000]{\color{textcolor}\rmfamily\fontsize{10.000000}{12.000000}\selectfont \(\displaystyle T(K)\)}%
\end{pgfscope}%
\begin{pgfscope}%
\pgfpathrectangle{\pgfqpoint{0.539970in}{2.747992in}}{\pgfqpoint{2.276064in}{1.626201in}}%
\pgfusepath{clip}%
\pgfsetrectcap%
\pgfsetroundjoin%
\pgfsetlinewidth{1.003750pt}%
\definecolor{currentstroke}{rgb}{0.047059,0.364706,0.647059}%
\pgfsetstrokecolor{currentstroke}%
\pgfsetdash{}{0pt}%
\pgfpathmoveto{\pgfqpoint{0.562505in}{4.300275in}}%
\pgfpathlineto{\pgfqpoint{0.585040in}{4.289179in}}%
\pgfpathlineto{\pgfqpoint{0.607576in}{4.273408in}}%
\pgfpathlineto{\pgfqpoint{0.630111in}{4.261072in}}%
\pgfpathlineto{\pgfqpoint{0.652646in}{4.254505in}}%
\pgfpathlineto{\pgfqpoint{0.675182in}{4.251406in}}%
\pgfpathlineto{\pgfqpoint{0.697717in}{4.242422in}}%
\pgfpathlineto{\pgfqpoint{0.720252in}{4.232372in}}%
\pgfpathlineto{\pgfqpoint{0.742787in}{4.222681in}}%
\pgfpathlineto{\pgfqpoint{0.765323in}{4.218012in}}%
\pgfpathlineto{\pgfqpoint{0.787858in}{4.210792in}}%
\pgfpathlineto{\pgfqpoint{0.810393in}{4.204813in}}%
\pgfpathlineto{\pgfqpoint{0.832929in}{4.201074in}}%
\pgfpathlineto{\pgfqpoint{0.855464in}{4.195652in}}%
\pgfpathlineto{\pgfqpoint{0.877999in}{4.191876in}}%
\pgfpathlineto{\pgfqpoint{0.900534in}{4.187030in}}%
\pgfpathlineto{\pgfqpoint{0.923070in}{4.183810in}}%
\pgfpathlineto{\pgfqpoint{0.945605in}{4.182430in}}%
\pgfpathlineto{\pgfqpoint{0.968140in}{4.180246in}}%
\pgfpathlineto{\pgfqpoint{0.990676in}{4.175161in}}%
\pgfpathlineto{\pgfqpoint{1.013211in}{4.172727in}}%
\pgfpathlineto{\pgfqpoint{1.035746in}{4.172519in}}%
\pgfpathlineto{\pgfqpoint{1.058281in}{4.170578in}}%
\pgfpathlineto{\pgfqpoint{1.080817in}{4.168458in}}%
\pgfpathlineto{\pgfqpoint{1.103352in}{4.167072in}}%
\pgfpathlineto{\pgfqpoint{1.125887in}{4.165481in}}%
\pgfpathlineto{\pgfqpoint{1.148423in}{4.164268in}}%
\pgfpathlineto{\pgfqpoint{1.170958in}{4.163689in}}%
\pgfpathlineto{\pgfqpoint{1.193493in}{4.161823in}}%
\pgfpathlineto{\pgfqpoint{1.216028in}{4.161910in}}%
\pgfpathlineto{\pgfqpoint{1.238564in}{4.161924in}}%
\pgfpathlineto{\pgfqpoint{1.261099in}{4.161655in}}%
\pgfpathlineto{\pgfqpoint{1.283634in}{4.160467in}}%
\pgfpathlineto{\pgfqpoint{1.306170in}{4.160137in}}%
\pgfpathlineto{\pgfqpoint{1.328705in}{4.160533in}}%
\pgfpathlineto{\pgfqpoint{1.351240in}{4.161011in}}%
\pgfpathlineto{\pgfqpoint{1.373776in}{4.160538in}}%
\pgfpathlineto{\pgfqpoint{1.396311in}{4.159681in}}%
\pgfpathlineto{\pgfqpoint{1.418846in}{4.158726in}}%
\pgfpathlineto{\pgfqpoint{1.441381in}{4.157470in}}%
\pgfpathlineto{\pgfqpoint{1.463917in}{4.157843in}}%
\pgfpathlineto{\pgfqpoint{1.486452in}{4.158020in}}%
\pgfpathlineto{\pgfqpoint{1.508987in}{4.158020in}}%
\pgfpathlineto{\pgfqpoint{1.531523in}{4.156879in}}%
\pgfpathlineto{\pgfqpoint{1.554058in}{4.156207in}}%
\pgfpathlineto{\pgfqpoint{1.576593in}{4.156022in}}%
\pgfpathlineto{\pgfqpoint{1.599128in}{4.156088in}}%
\pgfpathlineto{\pgfqpoint{1.621664in}{4.155596in}}%
\pgfpathlineto{\pgfqpoint{1.644199in}{4.155465in}}%
\pgfpathlineto{\pgfqpoint{1.666734in}{4.155784in}}%
\pgfpathlineto{\pgfqpoint{1.689270in}{4.155882in}}%
\pgfpathlineto{\pgfqpoint{1.711805in}{4.155342in}}%
\pgfpathlineto{\pgfqpoint{1.734340in}{4.153956in}}%
\pgfpathlineto{\pgfqpoint{1.756875in}{4.153457in}}%
\pgfpathlineto{\pgfqpoint{1.779411in}{4.153822in}}%
\pgfpathlineto{\pgfqpoint{1.801946in}{4.152963in}}%
\pgfpathlineto{\pgfqpoint{1.824481in}{4.152771in}}%
\pgfpathlineto{\pgfqpoint{1.847017in}{4.152121in}}%
\pgfpathlineto{\pgfqpoint{1.869552in}{4.152452in}}%
\pgfpathlineto{\pgfqpoint{1.892087in}{4.152527in}}%
\pgfpathlineto{\pgfqpoint{1.914622in}{4.152929in}}%
\pgfpathlineto{\pgfqpoint{1.937158in}{4.153228in}}%
\pgfpathlineto{\pgfqpoint{1.959693in}{4.153546in}}%
\pgfpathlineto{\pgfqpoint{1.982228in}{4.154347in}}%
\pgfpathlineto{\pgfqpoint{2.004764in}{4.154201in}}%
\pgfpathlineto{\pgfqpoint{2.027299in}{4.154190in}}%
\pgfpathlineto{\pgfqpoint{2.049834in}{4.153901in}}%
\pgfpathlineto{\pgfqpoint{2.072369in}{4.153575in}}%
\pgfpathlineto{\pgfqpoint{2.094905in}{4.153626in}}%
\pgfpathlineto{\pgfqpoint{2.117440in}{4.153740in}}%
\pgfpathlineto{\pgfqpoint{2.139975in}{4.153668in}}%
\pgfpathlineto{\pgfqpoint{2.162511in}{4.153531in}}%
\pgfpathlineto{\pgfqpoint{2.185046in}{4.153543in}}%
\pgfpathlineto{\pgfqpoint{2.207581in}{4.153608in}}%
\pgfpathlineto{\pgfqpoint{2.230116in}{4.154691in}}%
\pgfpathlineto{\pgfqpoint{2.252652in}{4.155175in}}%
\pgfpathlineto{\pgfqpoint{2.275187in}{4.156223in}}%
\pgfpathlineto{\pgfqpoint{2.297722in}{4.156127in}}%
\pgfpathlineto{\pgfqpoint{2.320258in}{4.156653in}}%
\pgfpathlineto{\pgfqpoint{2.342793in}{4.157507in}}%
\pgfpathlineto{\pgfqpoint{2.365328in}{4.157566in}}%
\pgfpathlineto{\pgfqpoint{2.387863in}{4.157733in}}%
\pgfpathlineto{\pgfqpoint{2.410399in}{4.158101in}}%
\pgfpathlineto{\pgfqpoint{2.432934in}{4.158194in}}%
\pgfpathlineto{\pgfqpoint{2.455469in}{4.158440in}}%
\pgfpathlineto{\pgfqpoint{2.478005in}{4.158096in}}%
\pgfpathlineto{\pgfqpoint{2.500540in}{4.158826in}}%
\pgfpathlineto{\pgfqpoint{2.523075in}{4.159800in}}%
\pgfpathlineto{\pgfqpoint{2.545610in}{4.160676in}}%
\pgfpathlineto{\pgfqpoint{2.568146in}{4.161828in}}%
\pgfpathlineto{\pgfqpoint{2.590681in}{4.162362in}}%
\pgfpathlineto{\pgfqpoint{2.613216in}{4.162544in}}%
\pgfpathlineto{\pgfqpoint{2.635752in}{4.163145in}}%
\pgfpathlineto{\pgfqpoint{2.658287in}{4.163084in}}%
\pgfpathlineto{\pgfqpoint{2.680822in}{4.163662in}}%
\pgfpathlineto{\pgfqpoint{2.703357in}{4.164169in}}%
\pgfpathlineto{\pgfqpoint{2.725893in}{4.163889in}}%
\pgfpathlineto{\pgfqpoint{2.748428in}{4.164386in}}%
\pgfpathlineto{\pgfqpoint{2.770963in}{4.164468in}}%
\pgfusepath{stroke}%
\end{pgfscope}%
\begin{pgfscope}%
\pgfpathrectangle{\pgfqpoint{0.539970in}{2.747992in}}{\pgfqpoint{2.276064in}{1.626201in}}%
\pgfusepath{clip}%
\pgfsetrectcap%
\pgfsetroundjoin%
\pgfsetlinewidth{1.003750pt}%
\definecolor{currentstroke}{rgb}{0.000000,0.725490,0.270588}%
\pgfsetstrokecolor{currentstroke}%
\pgfsetdash{}{0pt}%
\pgfpathmoveto{\pgfqpoint{0.562505in}{4.280025in}}%
\pgfpathlineto{\pgfqpoint{0.585040in}{4.240068in}}%
\pgfpathlineto{\pgfqpoint{0.607576in}{4.218887in}}%
\pgfpathlineto{\pgfqpoint{0.630111in}{4.190777in}}%
\pgfpathlineto{\pgfqpoint{0.652646in}{4.162647in}}%
\pgfpathlineto{\pgfqpoint{0.675182in}{4.137346in}}%
\pgfpathlineto{\pgfqpoint{0.697717in}{4.113469in}}%
\pgfpathlineto{\pgfqpoint{0.720252in}{4.090379in}}%
\pgfpathlineto{\pgfqpoint{0.742787in}{4.075850in}}%
\pgfpathlineto{\pgfqpoint{0.765323in}{4.058113in}}%
\pgfpathlineto{\pgfqpoint{0.787858in}{4.041239in}}%
\pgfpathlineto{\pgfqpoint{0.810393in}{4.027391in}}%
\pgfpathlineto{\pgfqpoint{0.832929in}{4.016799in}}%
\pgfpathlineto{\pgfqpoint{0.855464in}{4.007272in}}%
\pgfpathlineto{\pgfqpoint{0.877999in}{3.997136in}}%
\pgfpathlineto{\pgfqpoint{0.900534in}{3.983649in}}%
\pgfpathlineto{\pgfqpoint{0.923070in}{3.976519in}}%
\pgfpathlineto{\pgfqpoint{0.945605in}{3.969544in}}%
\pgfpathlineto{\pgfqpoint{0.968140in}{3.961963in}}%
\pgfpathlineto{\pgfqpoint{0.990676in}{3.955059in}}%
\pgfpathlineto{\pgfqpoint{1.013211in}{3.944624in}}%
\pgfpathlineto{\pgfqpoint{1.035746in}{3.939977in}}%
\pgfpathlineto{\pgfqpoint{1.058281in}{3.932031in}}%
\pgfpathlineto{\pgfqpoint{1.080817in}{3.926573in}}%
\pgfpathlineto{\pgfqpoint{1.103352in}{3.922023in}}%
\pgfpathlineto{\pgfqpoint{1.125887in}{3.913616in}}%
\pgfpathlineto{\pgfqpoint{1.148423in}{3.909322in}}%
\pgfpathlineto{\pgfqpoint{1.170958in}{3.905155in}}%
\pgfpathlineto{\pgfqpoint{1.193493in}{3.895403in}}%
\pgfpathlineto{\pgfqpoint{1.216028in}{3.892217in}}%
\pgfpathlineto{\pgfqpoint{1.238564in}{3.887711in}}%
\pgfpathlineto{\pgfqpoint{1.261099in}{3.882905in}}%
\pgfpathlineto{\pgfqpoint{1.283634in}{3.878952in}}%
\pgfpathlineto{\pgfqpoint{1.306170in}{3.876306in}}%
\pgfpathlineto{\pgfqpoint{1.328705in}{3.872302in}}%
\pgfpathlineto{\pgfqpoint{1.351240in}{3.867874in}}%
\pgfpathlineto{\pgfqpoint{1.373776in}{3.866203in}}%
\pgfpathlineto{\pgfqpoint{1.396311in}{3.862889in}}%
\pgfpathlineto{\pgfqpoint{1.418846in}{3.857429in}}%
\pgfpathlineto{\pgfqpoint{1.441381in}{3.854139in}}%
\pgfpathlineto{\pgfqpoint{1.463917in}{3.850918in}}%
\pgfpathlineto{\pgfqpoint{1.486452in}{3.849557in}}%
\pgfpathlineto{\pgfqpoint{1.508987in}{3.845416in}}%
\pgfpathlineto{\pgfqpoint{1.531523in}{3.839767in}}%
\pgfpathlineto{\pgfqpoint{1.554058in}{3.835334in}}%
\pgfpathlineto{\pgfqpoint{1.576593in}{3.833700in}}%
\pgfpathlineto{\pgfqpoint{1.599128in}{3.830106in}}%
\pgfpathlineto{\pgfqpoint{1.621664in}{3.829326in}}%
\pgfpathlineto{\pgfqpoint{1.644199in}{3.826757in}}%
\pgfpathlineto{\pgfqpoint{1.666734in}{3.822460in}}%
\pgfpathlineto{\pgfqpoint{1.689270in}{3.818969in}}%
\pgfpathlineto{\pgfqpoint{1.711805in}{3.814030in}}%
\pgfpathlineto{\pgfqpoint{1.734340in}{3.812411in}}%
\pgfpathlineto{\pgfqpoint{1.756875in}{3.810053in}}%
\pgfpathlineto{\pgfqpoint{1.779411in}{3.807309in}}%
\pgfpathlineto{\pgfqpoint{1.801946in}{3.804470in}}%
\pgfpathlineto{\pgfqpoint{1.824481in}{3.803111in}}%
\pgfpathlineto{\pgfqpoint{1.847017in}{3.799528in}}%
\pgfpathlineto{\pgfqpoint{1.869552in}{3.796597in}}%
\pgfpathlineto{\pgfqpoint{1.892087in}{3.792976in}}%
\pgfpathlineto{\pgfqpoint{1.914622in}{3.792065in}}%
\pgfpathlineto{\pgfqpoint{1.937158in}{3.789578in}}%
\pgfpathlineto{\pgfqpoint{1.959693in}{3.788811in}}%
\pgfpathlineto{\pgfqpoint{1.982228in}{3.788008in}}%
\pgfpathlineto{\pgfqpoint{2.004764in}{3.785854in}}%
\pgfpathlineto{\pgfqpoint{2.027299in}{3.782981in}}%
\pgfpathlineto{\pgfqpoint{2.049834in}{3.782860in}}%
\pgfpathlineto{\pgfqpoint{2.072369in}{3.781712in}}%
\pgfpathlineto{\pgfqpoint{2.094905in}{3.779117in}}%
\pgfpathlineto{\pgfqpoint{2.117440in}{3.776758in}}%
\pgfpathlineto{\pgfqpoint{2.139975in}{3.774100in}}%
\pgfpathlineto{\pgfqpoint{2.162511in}{3.772745in}}%
\pgfpathlineto{\pgfqpoint{2.185046in}{3.772006in}}%
\pgfpathlineto{\pgfqpoint{2.207581in}{3.768526in}}%
\pgfpathlineto{\pgfqpoint{2.230116in}{3.768096in}}%
\pgfpathlineto{\pgfqpoint{2.252652in}{3.765811in}}%
\pgfpathlineto{\pgfqpoint{2.275187in}{3.764529in}}%
\pgfpathlineto{\pgfqpoint{2.297722in}{3.761644in}}%
\pgfpathlineto{\pgfqpoint{2.320258in}{3.758488in}}%
\pgfpathlineto{\pgfqpoint{2.342793in}{3.757048in}}%
\pgfpathlineto{\pgfqpoint{2.365328in}{3.753516in}}%
\pgfpathlineto{\pgfqpoint{2.387863in}{3.750390in}}%
\pgfpathlineto{\pgfqpoint{2.410399in}{3.746416in}}%
\pgfpathlineto{\pgfqpoint{2.432934in}{3.742951in}}%
\pgfpathlineto{\pgfqpoint{2.455469in}{3.737677in}}%
\pgfpathlineto{\pgfqpoint{2.478005in}{3.733535in}}%
\pgfpathlineto{\pgfqpoint{2.500540in}{3.729297in}}%
\pgfpathlineto{\pgfqpoint{2.523075in}{3.725884in}}%
\pgfpathlineto{\pgfqpoint{2.545610in}{3.720944in}}%
\pgfpathlineto{\pgfqpoint{2.568146in}{3.717622in}}%
\pgfpathlineto{\pgfqpoint{2.590681in}{3.713736in}}%
\pgfpathlineto{\pgfqpoint{2.613216in}{3.710262in}}%
\pgfpathlineto{\pgfqpoint{2.635752in}{3.707550in}}%
\pgfpathlineto{\pgfqpoint{2.658287in}{3.703758in}}%
\pgfpathlineto{\pgfqpoint{2.680822in}{3.701927in}}%
\pgfpathlineto{\pgfqpoint{2.703357in}{3.697927in}}%
\pgfpathlineto{\pgfqpoint{2.725893in}{3.694805in}}%
\pgfpathlineto{\pgfqpoint{2.748428in}{3.691688in}}%
\pgfpathlineto{\pgfqpoint{2.770963in}{3.689631in}}%
\pgfusepath{stroke}%
\end{pgfscope}%
\begin{pgfscope}%
\pgfpathrectangle{\pgfqpoint{0.539970in}{2.747992in}}{\pgfqpoint{2.276064in}{1.626201in}}%
\pgfusepath{clip}%
\pgfsetrectcap%
\pgfsetroundjoin%
\pgfsetlinewidth{1.003750pt}%
\definecolor{currentstroke}{rgb}{1.000000,0.584314,0.000000}%
\pgfsetstrokecolor{currentstroke}%
\pgfsetdash{}{0pt}%
\pgfpathmoveto{\pgfqpoint{0.562505in}{4.219193in}}%
\pgfpathlineto{\pgfqpoint{0.585040in}{4.151904in}}%
\pgfpathlineto{\pgfqpoint{0.607576in}{4.109252in}}%
\pgfpathlineto{\pgfqpoint{0.630111in}{4.066724in}}%
\pgfpathlineto{\pgfqpoint{0.652646in}{4.033066in}}%
\pgfpathlineto{\pgfqpoint{0.675182in}{4.003609in}}%
\pgfpathlineto{\pgfqpoint{0.697717in}{3.984602in}}%
\pgfpathlineto{\pgfqpoint{0.720252in}{3.956789in}}%
\pgfpathlineto{\pgfqpoint{0.742787in}{3.948817in}}%
\pgfpathlineto{\pgfqpoint{0.765323in}{3.933313in}}%
\pgfpathlineto{\pgfqpoint{0.787858in}{3.924104in}}%
\pgfpathlineto{\pgfqpoint{0.810393in}{3.914525in}}%
\pgfpathlineto{\pgfqpoint{0.832929in}{3.903395in}}%
\pgfpathlineto{\pgfqpoint{0.855464in}{3.896809in}}%
\pgfpathlineto{\pgfqpoint{0.877999in}{3.891712in}}%
\pgfpathlineto{\pgfqpoint{0.900534in}{3.883753in}}%
\pgfpathlineto{\pgfqpoint{0.923070in}{3.881972in}}%
\pgfpathlineto{\pgfqpoint{0.945605in}{3.875002in}}%
\pgfpathlineto{\pgfqpoint{0.968140in}{3.874627in}}%
\pgfpathlineto{\pgfqpoint{0.990676in}{3.873420in}}%
\pgfpathlineto{\pgfqpoint{1.013211in}{3.871040in}}%
\pgfpathlineto{\pgfqpoint{1.035746in}{3.869589in}}%
\pgfpathlineto{\pgfqpoint{1.058281in}{3.870823in}}%
\pgfpathlineto{\pgfqpoint{1.080817in}{3.869620in}}%
\pgfpathlineto{\pgfqpoint{1.103352in}{3.871725in}}%
\pgfpathlineto{\pgfqpoint{1.125887in}{3.871095in}}%
\pgfpathlineto{\pgfqpoint{1.148423in}{3.869233in}}%
\pgfpathlineto{\pgfqpoint{1.170958in}{3.865160in}}%
\pgfpathlineto{\pgfqpoint{1.193493in}{3.861678in}}%
\pgfpathlineto{\pgfqpoint{1.216028in}{3.862465in}}%
\pgfpathlineto{\pgfqpoint{1.238564in}{3.862546in}}%
\pgfpathlineto{\pgfqpoint{1.261099in}{3.863194in}}%
\pgfpathlineto{\pgfqpoint{1.283634in}{3.861485in}}%
\pgfpathlineto{\pgfqpoint{1.306170in}{3.863704in}}%
\pgfpathlineto{\pgfqpoint{1.328705in}{3.862676in}}%
\pgfpathlineto{\pgfqpoint{1.351240in}{3.862115in}}%
\pgfpathlineto{\pgfqpoint{1.373776in}{3.861711in}}%
\pgfpathlineto{\pgfqpoint{1.396311in}{3.861646in}}%
\pgfpathlineto{\pgfqpoint{1.418846in}{3.859854in}}%
\pgfpathlineto{\pgfqpoint{1.441381in}{3.859008in}}%
\pgfpathlineto{\pgfqpoint{1.463917in}{3.857488in}}%
\pgfpathlineto{\pgfqpoint{1.486452in}{3.857233in}}%
\pgfpathlineto{\pgfqpoint{1.508987in}{3.855728in}}%
\pgfpathlineto{\pgfqpoint{1.531523in}{3.853878in}}%
\pgfpathlineto{\pgfqpoint{1.554058in}{3.852637in}}%
\pgfpathlineto{\pgfqpoint{1.576593in}{3.852010in}}%
\pgfpathlineto{\pgfqpoint{1.599128in}{3.852349in}}%
\pgfpathlineto{\pgfqpoint{1.621664in}{3.851595in}}%
\pgfpathlineto{\pgfqpoint{1.644199in}{3.850627in}}%
\pgfpathlineto{\pgfqpoint{1.666734in}{3.849710in}}%
\pgfpathlineto{\pgfqpoint{1.689270in}{3.848495in}}%
\pgfpathlineto{\pgfqpoint{1.711805in}{3.848389in}}%
\pgfpathlineto{\pgfqpoint{1.734340in}{3.850171in}}%
\pgfpathlineto{\pgfqpoint{1.756875in}{3.849998in}}%
\pgfpathlineto{\pgfqpoint{1.779411in}{3.846789in}}%
\pgfpathlineto{\pgfqpoint{1.801946in}{3.846004in}}%
\pgfpathlineto{\pgfqpoint{1.824481in}{3.845128in}}%
\pgfpathlineto{\pgfqpoint{1.847017in}{3.844750in}}%
\pgfpathlineto{\pgfqpoint{1.869552in}{3.844026in}}%
\pgfpathlineto{\pgfqpoint{1.892087in}{3.843070in}}%
\pgfpathlineto{\pgfqpoint{1.914622in}{3.842321in}}%
\pgfpathlineto{\pgfqpoint{1.937158in}{3.841536in}}%
\pgfpathlineto{\pgfqpoint{1.959693in}{3.839936in}}%
\pgfpathlineto{\pgfqpoint{1.982228in}{3.838160in}}%
\pgfpathlineto{\pgfqpoint{2.004764in}{3.837778in}}%
\pgfpathlineto{\pgfqpoint{2.027299in}{3.838069in}}%
\pgfpathlineto{\pgfqpoint{2.049834in}{3.838479in}}%
\pgfpathlineto{\pgfqpoint{2.072369in}{3.838446in}}%
\pgfpathlineto{\pgfqpoint{2.094905in}{3.837737in}}%
\pgfpathlineto{\pgfqpoint{2.117440in}{3.837679in}}%
\pgfpathlineto{\pgfqpoint{2.139975in}{3.836827in}}%
\pgfpathlineto{\pgfqpoint{2.162511in}{3.836831in}}%
\pgfpathlineto{\pgfqpoint{2.185046in}{3.835843in}}%
\pgfpathlineto{\pgfqpoint{2.207581in}{3.835647in}}%
\pgfpathlineto{\pgfqpoint{2.230116in}{3.836148in}}%
\pgfpathlineto{\pgfqpoint{2.252652in}{3.835509in}}%
\pgfpathlineto{\pgfqpoint{2.275187in}{3.836857in}}%
\pgfpathlineto{\pgfqpoint{2.297722in}{3.835708in}}%
\pgfpathlineto{\pgfqpoint{2.320258in}{3.833657in}}%
\pgfpathlineto{\pgfqpoint{2.342793in}{3.833964in}}%
\pgfpathlineto{\pgfqpoint{2.365328in}{3.833231in}}%
\pgfpathlineto{\pgfqpoint{2.387863in}{3.831894in}}%
\pgfpathlineto{\pgfqpoint{2.410399in}{3.831751in}}%
\pgfpathlineto{\pgfqpoint{2.432934in}{3.831257in}}%
\pgfpathlineto{\pgfqpoint{2.455469in}{3.832270in}}%
\pgfpathlineto{\pgfqpoint{2.478005in}{3.831491in}}%
\pgfpathlineto{\pgfqpoint{2.500540in}{3.832518in}}%
\pgfpathlineto{\pgfqpoint{2.523075in}{3.832480in}}%
\pgfpathlineto{\pgfqpoint{2.545610in}{3.831658in}}%
\pgfpathlineto{\pgfqpoint{2.568146in}{3.831074in}}%
\pgfpathlineto{\pgfqpoint{2.590681in}{3.831322in}}%
\pgfpathlineto{\pgfqpoint{2.613216in}{3.831509in}}%
\pgfpathlineto{\pgfqpoint{2.635752in}{3.829701in}}%
\pgfpathlineto{\pgfqpoint{2.658287in}{3.828909in}}%
\pgfpathlineto{\pgfqpoint{2.680822in}{3.828349in}}%
\pgfpathlineto{\pgfqpoint{2.703357in}{3.828073in}}%
\pgfpathlineto{\pgfqpoint{2.725893in}{3.827496in}}%
\pgfpathlineto{\pgfqpoint{2.748428in}{3.828411in}}%
\pgfpathlineto{\pgfqpoint{2.770963in}{3.827286in}}%
\pgfusepath{stroke}%
\end{pgfscope}%
\begin{pgfscope}%
\pgfpathrectangle{\pgfqpoint{0.539970in}{2.747992in}}{\pgfqpoint{2.276064in}{1.626201in}}%
\pgfusepath{clip}%
\pgfsetrectcap%
\pgfsetroundjoin%
\pgfsetlinewidth{1.003750pt}%
\definecolor{currentstroke}{rgb}{1.000000,0.172549,0.000000}%
\pgfsetstrokecolor{currentstroke}%
\pgfsetdash{}{0pt}%
\pgfpathmoveto{\pgfqpoint{0.562505in}{4.121173in}}%
\pgfpathlineto{\pgfqpoint{0.585040in}{4.035649in}}%
\pgfpathlineto{\pgfqpoint{0.607576in}{3.976567in}}%
\pgfpathlineto{\pgfqpoint{0.630111in}{3.915872in}}%
\pgfpathlineto{\pgfqpoint{0.652646in}{3.864146in}}%
\pgfpathlineto{\pgfqpoint{0.675182in}{3.818304in}}%
\pgfpathlineto{\pgfqpoint{0.697717in}{3.770123in}}%
\pgfpathlineto{\pgfqpoint{0.720252in}{3.733945in}}%
\pgfpathlineto{\pgfqpoint{0.742787in}{3.705396in}}%
\pgfpathlineto{\pgfqpoint{0.765323in}{3.675976in}}%
\pgfpathlineto{\pgfqpoint{0.787858in}{3.637757in}}%
\pgfpathlineto{\pgfqpoint{0.810393in}{3.603061in}}%
\pgfpathlineto{\pgfqpoint{0.832929in}{3.578988in}}%
\pgfpathlineto{\pgfqpoint{0.855464in}{3.546114in}}%
\pgfpathlineto{\pgfqpoint{0.877999in}{3.518332in}}%
\pgfpathlineto{\pgfqpoint{0.900534in}{3.487958in}}%
\pgfpathlineto{\pgfqpoint{0.923070in}{3.464872in}}%
\pgfpathlineto{\pgfqpoint{0.945605in}{3.437182in}}%
\pgfpathlineto{\pgfqpoint{0.968140in}{3.419655in}}%
\pgfpathlineto{\pgfqpoint{0.990676in}{3.397598in}}%
\pgfpathlineto{\pgfqpoint{1.013211in}{3.380866in}}%
\pgfpathlineto{\pgfqpoint{1.035746in}{3.359330in}}%
\pgfpathlineto{\pgfqpoint{1.058281in}{3.336796in}}%
\pgfpathlineto{\pgfqpoint{1.080817in}{3.320434in}}%
\pgfpathlineto{\pgfqpoint{1.103352in}{3.305862in}}%
\pgfpathlineto{\pgfqpoint{1.125887in}{3.286746in}}%
\pgfpathlineto{\pgfqpoint{1.148423in}{3.271546in}}%
\pgfpathlineto{\pgfqpoint{1.170958in}{3.258656in}}%
\pgfpathlineto{\pgfqpoint{1.193493in}{3.244745in}}%
\pgfpathlineto{\pgfqpoint{1.216028in}{3.231625in}}%
\pgfpathlineto{\pgfqpoint{1.238564in}{3.221506in}}%
\pgfpathlineto{\pgfqpoint{1.261099in}{3.207944in}}%
\pgfpathlineto{\pgfqpoint{1.283634in}{3.192341in}}%
\pgfpathlineto{\pgfqpoint{1.306170in}{3.178363in}}%
\pgfpathlineto{\pgfqpoint{1.328705in}{3.163464in}}%
\pgfpathlineto{\pgfqpoint{1.351240in}{3.149516in}}%
\pgfpathlineto{\pgfqpoint{1.373776in}{3.141339in}}%
\pgfpathlineto{\pgfqpoint{1.396311in}{3.131100in}}%
\pgfpathlineto{\pgfqpoint{1.418846in}{3.120361in}}%
\pgfpathlineto{\pgfqpoint{1.441381in}{3.113073in}}%
\pgfpathlineto{\pgfqpoint{1.463917in}{3.105935in}}%
\pgfpathlineto{\pgfqpoint{1.486452in}{3.097909in}}%
\pgfpathlineto{\pgfqpoint{1.508987in}{3.088480in}}%
\pgfpathlineto{\pgfqpoint{1.531523in}{3.078701in}}%
\pgfpathlineto{\pgfqpoint{1.554058in}{3.069624in}}%
\pgfpathlineto{\pgfqpoint{1.576593in}{3.062381in}}%
\pgfpathlineto{\pgfqpoint{1.599128in}{3.058212in}}%
\pgfpathlineto{\pgfqpoint{1.621664in}{3.055384in}}%
\pgfpathlineto{\pgfqpoint{1.644199in}{3.050644in}}%
\pgfpathlineto{\pgfqpoint{1.666734in}{3.040804in}}%
\pgfpathlineto{\pgfqpoint{1.689270in}{3.031467in}}%
\pgfpathlineto{\pgfqpoint{1.711805in}{3.023369in}}%
\pgfpathlineto{\pgfqpoint{1.734340in}{3.015292in}}%
\pgfpathlineto{\pgfqpoint{1.756875in}{3.010290in}}%
\pgfpathlineto{\pgfqpoint{1.779411in}{3.002342in}}%
\pgfpathlineto{\pgfqpoint{1.801946in}{2.994847in}}%
\pgfpathlineto{\pgfqpoint{1.824481in}{2.989799in}}%
\pgfpathlineto{\pgfqpoint{1.847017in}{2.981015in}}%
\pgfpathlineto{\pgfqpoint{1.869552in}{2.977551in}}%
\pgfpathlineto{\pgfqpoint{1.892087in}{2.973878in}}%
\pgfpathlineto{\pgfqpoint{1.914622in}{2.969714in}}%
\pgfpathlineto{\pgfqpoint{1.937158in}{2.961906in}}%
\pgfpathlineto{\pgfqpoint{1.959693in}{2.960940in}}%
\pgfpathlineto{\pgfqpoint{1.982228in}{2.957160in}}%
\pgfpathlineto{\pgfqpoint{2.004764in}{2.951544in}}%
\pgfpathlineto{\pgfqpoint{2.027299in}{2.948666in}}%
\pgfpathlineto{\pgfqpoint{2.049834in}{2.945665in}}%
\pgfpathlineto{\pgfqpoint{2.072369in}{2.942595in}}%
\pgfpathlineto{\pgfqpoint{2.094905in}{2.938844in}}%
\pgfpathlineto{\pgfqpoint{2.117440in}{2.938106in}}%
\pgfpathlineto{\pgfqpoint{2.139975in}{2.933675in}}%
\pgfpathlineto{\pgfqpoint{2.162511in}{2.929017in}}%
\pgfpathlineto{\pgfqpoint{2.185046in}{2.923217in}}%
\pgfpathlineto{\pgfqpoint{2.207581in}{2.918225in}}%
\pgfpathlineto{\pgfqpoint{2.230116in}{2.914770in}}%
\pgfpathlineto{\pgfqpoint{2.252652in}{2.910921in}}%
\pgfpathlineto{\pgfqpoint{2.275187in}{2.906118in}}%
\pgfpathlineto{\pgfqpoint{2.297722in}{2.903442in}}%
\pgfpathlineto{\pgfqpoint{2.320258in}{2.897633in}}%
\pgfpathlineto{\pgfqpoint{2.342793in}{2.895093in}}%
\pgfpathlineto{\pgfqpoint{2.365328in}{2.890428in}}%
\pgfpathlineto{\pgfqpoint{2.387863in}{2.886601in}}%
\pgfpathlineto{\pgfqpoint{2.410399in}{2.884765in}}%
\pgfpathlineto{\pgfqpoint{2.432934in}{2.880595in}}%
\pgfpathlineto{\pgfqpoint{2.455469in}{2.875301in}}%
\pgfpathlineto{\pgfqpoint{2.478005in}{2.869771in}}%
\pgfpathlineto{\pgfqpoint{2.500540in}{2.865423in}}%
\pgfpathlineto{\pgfqpoint{2.523075in}{2.859215in}}%
\pgfpathlineto{\pgfqpoint{2.545610in}{2.856153in}}%
\pgfpathlineto{\pgfqpoint{2.568146in}{2.850419in}}%
\pgfpathlineto{\pgfqpoint{2.590681in}{2.847587in}}%
\pgfpathlineto{\pgfqpoint{2.613216in}{2.843837in}}%
\pgfpathlineto{\pgfqpoint{2.635752in}{2.840427in}}%
\pgfpathlineto{\pgfqpoint{2.658287in}{2.836704in}}%
\pgfpathlineto{\pgfqpoint{2.680822in}{2.832103in}}%
\pgfpathlineto{\pgfqpoint{2.703357in}{2.829730in}}%
\pgfpathlineto{\pgfqpoint{2.725893in}{2.827589in}}%
\pgfpathlineto{\pgfqpoint{2.748428in}{2.824783in}}%
\pgfpathlineto{\pgfqpoint{2.770963in}{2.821910in}}%
\pgfusepath{stroke}%
\end{pgfscope}%
\begin{pgfscope}%
\pgfpathrectangle{\pgfqpoint{0.539970in}{2.747992in}}{\pgfqpoint{2.276064in}{1.626201in}}%
\pgfusepath{clip}%
\pgfsetrectcap%
\pgfsetroundjoin%
\pgfsetlinewidth{1.003750pt}%
\definecolor{currentstroke}{rgb}{0.517647,0.356863,0.592157}%
\pgfsetstrokecolor{currentstroke}%
\pgfsetdash{}{0pt}%
\pgfpathmoveto{\pgfqpoint{0.562505in}{3.581059in}}%
\pgfpathlineto{\pgfqpoint{0.585040in}{3.548157in}}%
\pgfpathlineto{\pgfqpoint{0.607576in}{3.579314in}}%
\pgfpathlineto{\pgfqpoint{0.630111in}{3.555459in}}%
\pgfpathlineto{\pgfqpoint{0.652646in}{3.531346in}}%
\pgfpathlineto{\pgfqpoint{0.675182in}{3.480290in}}%
\pgfpathlineto{\pgfqpoint{0.697717in}{3.452948in}}%
\pgfpathlineto{\pgfqpoint{0.720252in}{3.445730in}}%
\pgfpathlineto{\pgfqpoint{0.742787in}{3.433886in}}%
\pgfpathlineto{\pgfqpoint{0.765323in}{3.400295in}}%
\pgfpathlineto{\pgfqpoint{0.787858in}{3.391143in}}%
\pgfpathlineto{\pgfqpoint{0.810393in}{3.357652in}}%
\pgfpathlineto{\pgfqpoint{0.832929in}{3.323719in}}%
\pgfpathlineto{\pgfqpoint{0.855464in}{3.315020in}}%
\pgfpathlineto{\pgfqpoint{0.877999in}{3.303456in}}%
\pgfpathlineto{\pgfqpoint{0.900534in}{3.268149in}}%
\pgfpathlineto{\pgfqpoint{0.923070in}{3.262137in}}%
\pgfpathlineto{\pgfqpoint{0.945605in}{3.258671in}}%
\pgfpathlineto{\pgfqpoint{0.968140in}{3.235922in}}%
\pgfpathlineto{\pgfqpoint{0.990676in}{3.213368in}}%
\pgfpathlineto{\pgfqpoint{1.013211in}{3.178457in}}%
\pgfpathlineto{\pgfqpoint{1.035746in}{3.157503in}}%
\pgfpathlineto{\pgfqpoint{1.058281in}{3.133929in}}%
\pgfpathlineto{\pgfqpoint{1.080817in}{3.123949in}}%
\pgfpathlineto{\pgfqpoint{1.103352in}{3.095882in}}%
\pgfpathlineto{\pgfqpoint{1.125887in}{3.081355in}}%
\pgfpathlineto{\pgfqpoint{1.148423in}{3.072692in}}%
\pgfpathlineto{\pgfqpoint{1.170958in}{3.056592in}}%
\pgfpathlineto{\pgfqpoint{1.193493in}{3.057312in}}%
\pgfpathlineto{\pgfqpoint{1.216028in}{3.045518in}}%
\pgfpathlineto{\pgfqpoint{1.238564in}{3.035877in}}%
\pgfpathlineto{\pgfqpoint{1.261099in}{3.028291in}}%
\pgfpathlineto{\pgfqpoint{1.283634in}{3.026823in}}%
\pgfpathlineto{\pgfqpoint{1.306170in}{3.026322in}}%
\pgfpathlineto{\pgfqpoint{1.328705in}{3.034843in}}%
\pgfpathlineto{\pgfqpoint{1.351240in}{3.044188in}}%
\pgfpathlineto{\pgfqpoint{1.373776in}{3.046952in}}%
\pgfpathlineto{\pgfqpoint{1.396311in}{3.055580in}}%
\pgfpathlineto{\pgfqpoint{1.418846in}{3.060498in}}%
\pgfpathlineto{\pgfqpoint{1.441381in}{3.070592in}}%
\pgfpathlineto{\pgfqpoint{1.463917in}{3.072056in}}%
\pgfpathlineto{\pgfqpoint{1.486452in}{3.070458in}}%
\pgfpathlineto{\pgfqpoint{1.508987in}{3.070039in}}%
\pgfpathlineto{\pgfqpoint{1.531523in}{3.068571in}}%
\pgfpathlineto{\pgfqpoint{1.554058in}{3.068033in}}%
\pgfpathlineto{\pgfqpoint{1.576593in}{3.065648in}}%
\pgfpathlineto{\pgfqpoint{1.599128in}{3.059024in}}%
\pgfpathlineto{\pgfqpoint{1.621664in}{3.058662in}}%
\pgfpathlineto{\pgfqpoint{1.644199in}{3.058682in}}%
\pgfpathlineto{\pgfqpoint{1.666734in}{3.063776in}}%
\pgfpathlineto{\pgfqpoint{1.689270in}{3.069985in}}%
\pgfpathlineto{\pgfqpoint{1.711805in}{3.067235in}}%
\pgfpathlineto{\pgfqpoint{1.734340in}{3.067280in}}%
\pgfpathlineto{\pgfqpoint{1.756875in}{3.067886in}}%
\pgfpathlineto{\pgfqpoint{1.779411in}{3.064274in}}%
\pgfpathlineto{\pgfqpoint{1.801946in}{3.065901in}}%
\pgfpathlineto{\pgfqpoint{1.824481in}{3.072735in}}%
\pgfpathlineto{\pgfqpoint{1.847017in}{3.072424in}}%
\pgfpathlineto{\pgfqpoint{1.869552in}{3.074014in}}%
\pgfpathlineto{\pgfqpoint{1.892087in}{3.077397in}}%
\pgfpathlineto{\pgfqpoint{1.914622in}{3.085428in}}%
\pgfpathlineto{\pgfqpoint{1.937158in}{3.086832in}}%
\pgfpathlineto{\pgfqpoint{1.959693in}{3.088994in}}%
\pgfpathlineto{\pgfqpoint{1.982228in}{3.084119in}}%
\pgfpathlineto{\pgfqpoint{2.004764in}{3.085852in}}%
\pgfpathlineto{\pgfqpoint{2.027299in}{3.089781in}}%
\pgfpathlineto{\pgfqpoint{2.049834in}{3.087123in}}%
\pgfpathlineto{\pgfqpoint{2.072369in}{3.084570in}}%
\pgfpathlineto{\pgfqpoint{2.094905in}{3.089493in}}%
\pgfpathlineto{\pgfqpoint{2.117440in}{3.088503in}}%
\pgfpathlineto{\pgfqpoint{2.139975in}{3.079914in}}%
\pgfpathlineto{\pgfqpoint{2.162511in}{3.081676in}}%
\pgfpathlineto{\pgfqpoint{2.185046in}{3.082006in}}%
\pgfpathlineto{\pgfqpoint{2.207581in}{3.078377in}}%
\pgfpathlineto{\pgfqpoint{2.230116in}{3.077113in}}%
\pgfpathlineto{\pgfqpoint{2.252652in}{3.081530in}}%
\pgfpathlineto{\pgfqpoint{2.275187in}{3.076213in}}%
\pgfpathlineto{\pgfqpoint{2.297722in}{3.075932in}}%
\pgfpathlineto{\pgfqpoint{2.320258in}{3.071164in}}%
\pgfpathlineto{\pgfqpoint{2.342793in}{3.071013in}}%
\pgfpathlineto{\pgfqpoint{2.365328in}{3.067734in}}%
\pgfpathlineto{\pgfqpoint{2.387863in}{3.060897in}}%
\pgfpathlineto{\pgfqpoint{2.410399in}{3.060020in}}%
\pgfpathlineto{\pgfqpoint{2.432934in}{3.059776in}}%
\pgfpathlineto{\pgfqpoint{2.455469in}{3.055851in}}%
\pgfpathlineto{\pgfqpoint{2.478005in}{3.054198in}}%
\pgfpathlineto{\pgfqpoint{2.500540in}{3.052236in}}%
\pgfpathlineto{\pgfqpoint{2.523075in}{3.048155in}}%
\pgfpathlineto{\pgfqpoint{2.545610in}{3.048521in}}%
\pgfpathlineto{\pgfqpoint{2.568146in}{3.045169in}}%
\pgfpathlineto{\pgfqpoint{2.590681in}{3.049200in}}%
\pgfpathlineto{\pgfqpoint{2.613216in}{3.049582in}}%
\pgfpathlineto{\pgfqpoint{2.635752in}{3.044571in}}%
\pgfpathlineto{\pgfqpoint{2.658287in}{3.044349in}}%
\pgfpathlineto{\pgfqpoint{2.680822in}{3.045535in}}%
\pgfpathlineto{\pgfqpoint{2.703357in}{3.044828in}}%
\pgfpathlineto{\pgfqpoint{2.725893in}{3.040777in}}%
\pgfpathlineto{\pgfqpoint{2.748428in}{3.041875in}}%
\pgfpathlineto{\pgfqpoint{2.770963in}{3.038711in}}%
\pgfusepath{stroke}%
\end{pgfscope}%
\begin{pgfscope}%
\pgfsetrectcap%
\pgfsetmiterjoin%
\pgfsetlinewidth{0.501875pt}%
\definecolor{currentstroke}{rgb}{0.000000,0.000000,0.000000}%
\pgfsetstrokecolor{currentstroke}%
\pgfsetdash{}{0pt}%
\pgfpathmoveto{\pgfqpoint{0.539970in}{2.747992in}}%
\pgfpathlineto{\pgfqpoint{0.539970in}{4.374193in}}%
\pgfusepath{stroke}%
\end{pgfscope}%
\begin{pgfscope}%
\pgfsetrectcap%
\pgfsetmiterjoin%
\pgfsetlinewidth{0.501875pt}%
\definecolor{currentstroke}{rgb}{0.000000,0.000000,0.000000}%
\pgfsetstrokecolor{currentstroke}%
\pgfsetdash{}{0pt}%
\pgfpathmoveto{\pgfqpoint{2.816034in}{2.747992in}}%
\pgfpathlineto{\pgfqpoint{2.816034in}{4.374193in}}%
\pgfusepath{stroke}%
\end{pgfscope}%
\begin{pgfscope}%
\pgfsetrectcap%
\pgfsetmiterjoin%
\pgfsetlinewidth{0.501875pt}%
\definecolor{currentstroke}{rgb}{0.000000,0.000000,0.000000}%
\pgfsetstrokecolor{currentstroke}%
\pgfsetdash{}{0pt}%
\pgfpathmoveto{\pgfqpoint{0.539970in}{2.747992in}}%
\pgfpathlineto{\pgfqpoint{2.816034in}{2.747992in}}%
\pgfusepath{stroke}%
\end{pgfscope}%
\begin{pgfscope}%
\pgfsetrectcap%
\pgfsetmiterjoin%
\pgfsetlinewidth{0.501875pt}%
\definecolor{currentstroke}{rgb}{0.000000,0.000000,0.000000}%
\pgfsetstrokecolor{currentstroke}%
\pgfsetdash{}{0pt}%
\pgfpathmoveto{\pgfqpoint{0.539970in}{4.374193in}}%
\pgfpathlineto{\pgfqpoint{2.816034in}{4.374193in}}%
\pgfusepath{stroke}%
\end{pgfscope}%
\begin{pgfscope}%
\definecolor{textcolor}{rgb}{0.000000,0.000000,0.000000}%
\pgfsetstrokecolor{textcolor}%
\pgfsetfillcolor{textcolor}%
\pgftext[x=1.678002in,y=4.457526in,,base]{\color{textcolor}\rmfamily\fontsize{12.000000}{14.400000}\selectfont Trustworthiness}%
\end{pgfscope}%
\begin{pgfscope}%
\pgfsetbuttcap%
\pgfsetmiterjoin%
\definecolor{currentfill}{rgb}{1.000000,1.000000,1.000000}%
\pgfsetfillcolor{currentfill}%
\pgfsetlinewidth{0.000000pt}%
\definecolor{currentstroke}{rgb}{0.000000,0.000000,0.000000}%
\pgfsetstrokecolor{currentstroke}%
\pgfsetstrokeopacity{0.000000}%
\pgfsetdash{}{0pt}%
\pgfpathmoveto{\pgfqpoint{0.539970in}{0.422992in}}%
\pgfpathlineto{\pgfqpoint{2.816034in}{0.422992in}}%
\pgfpathlineto{\pgfqpoint{2.816034in}{2.049193in}}%
\pgfpathlineto{\pgfqpoint{0.539970in}{2.049193in}}%
\pgfpathlineto{\pgfqpoint{0.539970in}{0.422992in}}%
\pgfpathclose%
\pgfusepath{fill}%
\end{pgfscope}%
\begin{pgfscope}%
\pgfsetbuttcap%
\pgfsetroundjoin%
\definecolor{currentfill}{rgb}{0.000000,0.000000,0.000000}%
\pgfsetfillcolor{currentfill}%
\pgfsetlinewidth{0.501875pt}%
\definecolor{currentstroke}{rgb}{0.000000,0.000000,0.000000}%
\pgfsetstrokecolor{currentstroke}%
\pgfsetdash{}{0pt}%
\pgfsys@defobject{currentmarker}{\pgfqpoint{0.000000in}{0.000000in}}{\pgfqpoint{0.000000in}{0.041667in}}{%
\pgfpathmoveto{\pgfqpoint{0.000000in}{0.000000in}}%
\pgfpathlineto{\pgfqpoint{0.000000in}{0.041667in}}%
\pgfusepath{stroke,fill}%
}%
\begin{pgfscope}%
\pgfsys@transformshift{0.539970in}{0.422992in}%
\pgfsys@useobject{currentmarker}{}%
\end{pgfscope}%
\end{pgfscope}%
\begin{pgfscope}%
\pgfsetbuttcap%
\pgfsetroundjoin%
\definecolor{currentfill}{rgb}{0.000000,0.000000,0.000000}%
\pgfsetfillcolor{currentfill}%
\pgfsetlinewidth{0.501875pt}%
\definecolor{currentstroke}{rgb}{0.000000,0.000000,0.000000}%
\pgfsetstrokecolor{currentstroke}%
\pgfsetdash{}{0pt}%
\pgfsys@defobject{currentmarker}{\pgfqpoint{0.000000in}{-0.041667in}}{\pgfqpoint{0.000000in}{0.000000in}}{%
\pgfpathmoveto{\pgfqpoint{0.000000in}{0.000000in}}%
\pgfpathlineto{\pgfqpoint{0.000000in}{-0.041667in}}%
\pgfusepath{stroke,fill}%
}%
\begin{pgfscope}%
\pgfsys@transformshift{0.539970in}{2.049193in}%
\pgfsys@useobject{currentmarker}{}%
\end{pgfscope}%
\end{pgfscope}%
\begin{pgfscope}%
\definecolor{textcolor}{rgb}{0.000000,0.000000,0.000000}%
\pgfsetstrokecolor{textcolor}%
\pgfsetfillcolor{textcolor}%
\pgftext[x=0.539970in,y=0.374381in,,top]{\color{textcolor}\rmfamily\fontsize{10.000000}{12.000000}\selectfont \(\displaystyle {0}\)}%
\end{pgfscope}%
\begin{pgfscope}%
\pgfsetbuttcap%
\pgfsetroundjoin%
\definecolor{currentfill}{rgb}{0.000000,0.000000,0.000000}%
\pgfsetfillcolor{currentfill}%
\pgfsetlinewidth{0.501875pt}%
\definecolor{currentstroke}{rgb}{0.000000,0.000000,0.000000}%
\pgfsetstrokecolor{currentstroke}%
\pgfsetdash{}{0pt}%
\pgfsys@defobject{currentmarker}{\pgfqpoint{0.000000in}{0.000000in}}{\pgfqpoint{0.000000in}{0.041667in}}{%
\pgfpathmoveto{\pgfqpoint{0.000000in}{0.000000in}}%
\pgfpathlineto{\pgfqpoint{0.000000in}{0.041667in}}%
\pgfusepath{stroke,fill}%
}%
\begin{pgfscope}%
\pgfsys@transformshift{0.990676in}{0.422992in}%
\pgfsys@useobject{currentmarker}{}%
\end{pgfscope}%
\end{pgfscope}%
\begin{pgfscope}%
\pgfsetbuttcap%
\pgfsetroundjoin%
\definecolor{currentfill}{rgb}{0.000000,0.000000,0.000000}%
\pgfsetfillcolor{currentfill}%
\pgfsetlinewidth{0.501875pt}%
\definecolor{currentstroke}{rgb}{0.000000,0.000000,0.000000}%
\pgfsetstrokecolor{currentstroke}%
\pgfsetdash{}{0pt}%
\pgfsys@defobject{currentmarker}{\pgfqpoint{0.000000in}{-0.041667in}}{\pgfqpoint{0.000000in}{0.000000in}}{%
\pgfpathmoveto{\pgfqpoint{0.000000in}{0.000000in}}%
\pgfpathlineto{\pgfqpoint{0.000000in}{-0.041667in}}%
\pgfusepath{stroke,fill}%
}%
\begin{pgfscope}%
\pgfsys@transformshift{0.990676in}{2.049193in}%
\pgfsys@useobject{currentmarker}{}%
\end{pgfscope}%
\end{pgfscope}%
\begin{pgfscope}%
\definecolor{textcolor}{rgb}{0.000000,0.000000,0.000000}%
\pgfsetstrokecolor{textcolor}%
\pgfsetfillcolor{textcolor}%
\pgftext[x=0.990676in,y=0.374381in,,top]{\color{textcolor}\rmfamily\fontsize{10.000000}{12.000000}\selectfont \(\displaystyle {20}\)}%
\end{pgfscope}%
\begin{pgfscope}%
\pgfsetbuttcap%
\pgfsetroundjoin%
\definecolor{currentfill}{rgb}{0.000000,0.000000,0.000000}%
\pgfsetfillcolor{currentfill}%
\pgfsetlinewidth{0.501875pt}%
\definecolor{currentstroke}{rgb}{0.000000,0.000000,0.000000}%
\pgfsetstrokecolor{currentstroke}%
\pgfsetdash{}{0pt}%
\pgfsys@defobject{currentmarker}{\pgfqpoint{0.000000in}{0.000000in}}{\pgfqpoint{0.000000in}{0.041667in}}{%
\pgfpathmoveto{\pgfqpoint{0.000000in}{0.000000in}}%
\pgfpathlineto{\pgfqpoint{0.000000in}{0.041667in}}%
\pgfusepath{stroke,fill}%
}%
\begin{pgfscope}%
\pgfsys@transformshift{1.441381in}{0.422992in}%
\pgfsys@useobject{currentmarker}{}%
\end{pgfscope}%
\end{pgfscope}%
\begin{pgfscope}%
\pgfsetbuttcap%
\pgfsetroundjoin%
\definecolor{currentfill}{rgb}{0.000000,0.000000,0.000000}%
\pgfsetfillcolor{currentfill}%
\pgfsetlinewidth{0.501875pt}%
\definecolor{currentstroke}{rgb}{0.000000,0.000000,0.000000}%
\pgfsetstrokecolor{currentstroke}%
\pgfsetdash{}{0pt}%
\pgfsys@defobject{currentmarker}{\pgfqpoint{0.000000in}{-0.041667in}}{\pgfqpoint{0.000000in}{0.000000in}}{%
\pgfpathmoveto{\pgfqpoint{0.000000in}{0.000000in}}%
\pgfpathlineto{\pgfqpoint{0.000000in}{-0.041667in}}%
\pgfusepath{stroke,fill}%
}%
\begin{pgfscope}%
\pgfsys@transformshift{1.441381in}{2.049193in}%
\pgfsys@useobject{currentmarker}{}%
\end{pgfscope}%
\end{pgfscope}%
\begin{pgfscope}%
\definecolor{textcolor}{rgb}{0.000000,0.000000,0.000000}%
\pgfsetstrokecolor{textcolor}%
\pgfsetfillcolor{textcolor}%
\pgftext[x=1.441381in,y=0.374381in,,top]{\color{textcolor}\rmfamily\fontsize{10.000000}{12.000000}\selectfont \(\displaystyle {40}\)}%
\end{pgfscope}%
\begin{pgfscope}%
\pgfsetbuttcap%
\pgfsetroundjoin%
\definecolor{currentfill}{rgb}{0.000000,0.000000,0.000000}%
\pgfsetfillcolor{currentfill}%
\pgfsetlinewidth{0.501875pt}%
\definecolor{currentstroke}{rgb}{0.000000,0.000000,0.000000}%
\pgfsetstrokecolor{currentstroke}%
\pgfsetdash{}{0pt}%
\pgfsys@defobject{currentmarker}{\pgfqpoint{0.000000in}{0.000000in}}{\pgfqpoint{0.000000in}{0.041667in}}{%
\pgfpathmoveto{\pgfqpoint{0.000000in}{0.000000in}}%
\pgfpathlineto{\pgfqpoint{0.000000in}{0.041667in}}%
\pgfusepath{stroke,fill}%
}%
\begin{pgfscope}%
\pgfsys@transformshift{1.892087in}{0.422992in}%
\pgfsys@useobject{currentmarker}{}%
\end{pgfscope}%
\end{pgfscope}%
\begin{pgfscope}%
\pgfsetbuttcap%
\pgfsetroundjoin%
\definecolor{currentfill}{rgb}{0.000000,0.000000,0.000000}%
\pgfsetfillcolor{currentfill}%
\pgfsetlinewidth{0.501875pt}%
\definecolor{currentstroke}{rgb}{0.000000,0.000000,0.000000}%
\pgfsetstrokecolor{currentstroke}%
\pgfsetdash{}{0pt}%
\pgfsys@defobject{currentmarker}{\pgfqpoint{0.000000in}{-0.041667in}}{\pgfqpoint{0.000000in}{0.000000in}}{%
\pgfpathmoveto{\pgfqpoint{0.000000in}{0.000000in}}%
\pgfpathlineto{\pgfqpoint{0.000000in}{-0.041667in}}%
\pgfusepath{stroke,fill}%
}%
\begin{pgfscope}%
\pgfsys@transformshift{1.892087in}{2.049193in}%
\pgfsys@useobject{currentmarker}{}%
\end{pgfscope}%
\end{pgfscope}%
\begin{pgfscope}%
\definecolor{textcolor}{rgb}{0.000000,0.000000,0.000000}%
\pgfsetstrokecolor{textcolor}%
\pgfsetfillcolor{textcolor}%
\pgftext[x=1.892087in,y=0.374381in,,top]{\color{textcolor}\rmfamily\fontsize{10.000000}{12.000000}\selectfont \(\displaystyle {60}\)}%
\end{pgfscope}%
\begin{pgfscope}%
\pgfsetbuttcap%
\pgfsetroundjoin%
\definecolor{currentfill}{rgb}{0.000000,0.000000,0.000000}%
\pgfsetfillcolor{currentfill}%
\pgfsetlinewidth{0.501875pt}%
\definecolor{currentstroke}{rgb}{0.000000,0.000000,0.000000}%
\pgfsetstrokecolor{currentstroke}%
\pgfsetdash{}{0pt}%
\pgfsys@defobject{currentmarker}{\pgfqpoint{0.000000in}{0.000000in}}{\pgfqpoint{0.000000in}{0.041667in}}{%
\pgfpathmoveto{\pgfqpoint{0.000000in}{0.000000in}}%
\pgfpathlineto{\pgfqpoint{0.000000in}{0.041667in}}%
\pgfusepath{stroke,fill}%
}%
\begin{pgfscope}%
\pgfsys@transformshift{2.342793in}{0.422992in}%
\pgfsys@useobject{currentmarker}{}%
\end{pgfscope}%
\end{pgfscope}%
\begin{pgfscope}%
\pgfsetbuttcap%
\pgfsetroundjoin%
\definecolor{currentfill}{rgb}{0.000000,0.000000,0.000000}%
\pgfsetfillcolor{currentfill}%
\pgfsetlinewidth{0.501875pt}%
\definecolor{currentstroke}{rgb}{0.000000,0.000000,0.000000}%
\pgfsetstrokecolor{currentstroke}%
\pgfsetdash{}{0pt}%
\pgfsys@defobject{currentmarker}{\pgfqpoint{0.000000in}{-0.041667in}}{\pgfqpoint{0.000000in}{0.000000in}}{%
\pgfpathmoveto{\pgfqpoint{0.000000in}{0.000000in}}%
\pgfpathlineto{\pgfqpoint{0.000000in}{-0.041667in}}%
\pgfusepath{stroke,fill}%
}%
\begin{pgfscope}%
\pgfsys@transformshift{2.342793in}{2.049193in}%
\pgfsys@useobject{currentmarker}{}%
\end{pgfscope}%
\end{pgfscope}%
\begin{pgfscope}%
\definecolor{textcolor}{rgb}{0.000000,0.000000,0.000000}%
\pgfsetstrokecolor{textcolor}%
\pgfsetfillcolor{textcolor}%
\pgftext[x=2.342793in,y=0.374381in,,top]{\color{textcolor}\rmfamily\fontsize{10.000000}{12.000000}\selectfont \(\displaystyle {80}\)}%
\end{pgfscope}%
\begin{pgfscope}%
\pgfsetbuttcap%
\pgfsetroundjoin%
\definecolor{currentfill}{rgb}{0.000000,0.000000,0.000000}%
\pgfsetfillcolor{currentfill}%
\pgfsetlinewidth{0.501875pt}%
\definecolor{currentstroke}{rgb}{0.000000,0.000000,0.000000}%
\pgfsetstrokecolor{currentstroke}%
\pgfsetdash{}{0pt}%
\pgfsys@defobject{currentmarker}{\pgfqpoint{0.000000in}{0.000000in}}{\pgfqpoint{0.000000in}{0.020833in}}{%
\pgfpathmoveto{\pgfqpoint{0.000000in}{0.000000in}}%
\pgfpathlineto{\pgfqpoint{0.000000in}{0.020833in}}%
\pgfusepath{stroke,fill}%
}%
\begin{pgfscope}%
\pgfsys@transformshift{0.652646in}{0.422992in}%
\pgfsys@useobject{currentmarker}{}%
\end{pgfscope}%
\end{pgfscope}%
\begin{pgfscope}%
\pgfsetbuttcap%
\pgfsetroundjoin%
\definecolor{currentfill}{rgb}{0.000000,0.000000,0.000000}%
\pgfsetfillcolor{currentfill}%
\pgfsetlinewidth{0.501875pt}%
\definecolor{currentstroke}{rgb}{0.000000,0.000000,0.000000}%
\pgfsetstrokecolor{currentstroke}%
\pgfsetdash{}{0pt}%
\pgfsys@defobject{currentmarker}{\pgfqpoint{0.000000in}{-0.020833in}}{\pgfqpoint{0.000000in}{0.000000in}}{%
\pgfpathmoveto{\pgfqpoint{0.000000in}{0.000000in}}%
\pgfpathlineto{\pgfqpoint{0.000000in}{-0.020833in}}%
\pgfusepath{stroke,fill}%
}%
\begin{pgfscope}%
\pgfsys@transformshift{0.652646in}{2.049193in}%
\pgfsys@useobject{currentmarker}{}%
\end{pgfscope}%
\end{pgfscope}%
\begin{pgfscope}%
\pgfsetbuttcap%
\pgfsetroundjoin%
\definecolor{currentfill}{rgb}{0.000000,0.000000,0.000000}%
\pgfsetfillcolor{currentfill}%
\pgfsetlinewidth{0.501875pt}%
\definecolor{currentstroke}{rgb}{0.000000,0.000000,0.000000}%
\pgfsetstrokecolor{currentstroke}%
\pgfsetdash{}{0pt}%
\pgfsys@defobject{currentmarker}{\pgfqpoint{0.000000in}{0.000000in}}{\pgfqpoint{0.000000in}{0.020833in}}{%
\pgfpathmoveto{\pgfqpoint{0.000000in}{0.000000in}}%
\pgfpathlineto{\pgfqpoint{0.000000in}{0.020833in}}%
\pgfusepath{stroke,fill}%
}%
\begin{pgfscope}%
\pgfsys@transformshift{0.765323in}{0.422992in}%
\pgfsys@useobject{currentmarker}{}%
\end{pgfscope}%
\end{pgfscope}%
\begin{pgfscope}%
\pgfsetbuttcap%
\pgfsetroundjoin%
\definecolor{currentfill}{rgb}{0.000000,0.000000,0.000000}%
\pgfsetfillcolor{currentfill}%
\pgfsetlinewidth{0.501875pt}%
\definecolor{currentstroke}{rgb}{0.000000,0.000000,0.000000}%
\pgfsetstrokecolor{currentstroke}%
\pgfsetdash{}{0pt}%
\pgfsys@defobject{currentmarker}{\pgfqpoint{0.000000in}{-0.020833in}}{\pgfqpoint{0.000000in}{0.000000in}}{%
\pgfpathmoveto{\pgfqpoint{0.000000in}{0.000000in}}%
\pgfpathlineto{\pgfqpoint{0.000000in}{-0.020833in}}%
\pgfusepath{stroke,fill}%
}%
\begin{pgfscope}%
\pgfsys@transformshift{0.765323in}{2.049193in}%
\pgfsys@useobject{currentmarker}{}%
\end{pgfscope}%
\end{pgfscope}%
\begin{pgfscope}%
\pgfsetbuttcap%
\pgfsetroundjoin%
\definecolor{currentfill}{rgb}{0.000000,0.000000,0.000000}%
\pgfsetfillcolor{currentfill}%
\pgfsetlinewidth{0.501875pt}%
\definecolor{currentstroke}{rgb}{0.000000,0.000000,0.000000}%
\pgfsetstrokecolor{currentstroke}%
\pgfsetdash{}{0pt}%
\pgfsys@defobject{currentmarker}{\pgfqpoint{0.000000in}{0.000000in}}{\pgfqpoint{0.000000in}{0.020833in}}{%
\pgfpathmoveto{\pgfqpoint{0.000000in}{0.000000in}}%
\pgfpathlineto{\pgfqpoint{0.000000in}{0.020833in}}%
\pgfusepath{stroke,fill}%
}%
\begin{pgfscope}%
\pgfsys@transformshift{0.877999in}{0.422992in}%
\pgfsys@useobject{currentmarker}{}%
\end{pgfscope}%
\end{pgfscope}%
\begin{pgfscope}%
\pgfsetbuttcap%
\pgfsetroundjoin%
\definecolor{currentfill}{rgb}{0.000000,0.000000,0.000000}%
\pgfsetfillcolor{currentfill}%
\pgfsetlinewidth{0.501875pt}%
\definecolor{currentstroke}{rgb}{0.000000,0.000000,0.000000}%
\pgfsetstrokecolor{currentstroke}%
\pgfsetdash{}{0pt}%
\pgfsys@defobject{currentmarker}{\pgfqpoint{0.000000in}{-0.020833in}}{\pgfqpoint{0.000000in}{0.000000in}}{%
\pgfpathmoveto{\pgfqpoint{0.000000in}{0.000000in}}%
\pgfpathlineto{\pgfqpoint{0.000000in}{-0.020833in}}%
\pgfusepath{stroke,fill}%
}%
\begin{pgfscope}%
\pgfsys@transformshift{0.877999in}{2.049193in}%
\pgfsys@useobject{currentmarker}{}%
\end{pgfscope}%
\end{pgfscope}%
\begin{pgfscope}%
\pgfsetbuttcap%
\pgfsetroundjoin%
\definecolor{currentfill}{rgb}{0.000000,0.000000,0.000000}%
\pgfsetfillcolor{currentfill}%
\pgfsetlinewidth{0.501875pt}%
\definecolor{currentstroke}{rgb}{0.000000,0.000000,0.000000}%
\pgfsetstrokecolor{currentstroke}%
\pgfsetdash{}{0pt}%
\pgfsys@defobject{currentmarker}{\pgfqpoint{0.000000in}{0.000000in}}{\pgfqpoint{0.000000in}{0.020833in}}{%
\pgfpathmoveto{\pgfqpoint{0.000000in}{0.000000in}}%
\pgfpathlineto{\pgfqpoint{0.000000in}{0.020833in}}%
\pgfusepath{stroke,fill}%
}%
\begin{pgfscope}%
\pgfsys@transformshift{1.103352in}{0.422992in}%
\pgfsys@useobject{currentmarker}{}%
\end{pgfscope}%
\end{pgfscope}%
\begin{pgfscope}%
\pgfsetbuttcap%
\pgfsetroundjoin%
\definecolor{currentfill}{rgb}{0.000000,0.000000,0.000000}%
\pgfsetfillcolor{currentfill}%
\pgfsetlinewidth{0.501875pt}%
\definecolor{currentstroke}{rgb}{0.000000,0.000000,0.000000}%
\pgfsetstrokecolor{currentstroke}%
\pgfsetdash{}{0pt}%
\pgfsys@defobject{currentmarker}{\pgfqpoint{0.000000in}{-0.020833in}}{\pgfqpoint{0.000000in}{0.000000in}}{%
\pgfpathmoveto{\pgfqpoint{0.000000in}{0.000000in}}%
\pgfpathlineto{\pgfqpoint{0.000000in}{-0.020833in}}%
\pgfusepath{stroke,fill}%
}%
\begin{pgfscope}%
\pgfsys@transformshift{1.103352in}{2.049193in}%
\pgfsys@useobject{currentmarker}{}%
\end{pgfscope}%
\end{pgfscope}%
\begin{pgfscope}%
\pgfsetbuttcap%
\pgfsetroundjoin%
\definecolor{currentfill}{rgb}{0.000000,0.000000,0.000000}%
\pgfsetfillcolor{currentfill}%
\pgfsetlinewidth{0.501875pt}%
\definecolor{currentstroke}{rgb}{0.000000,0.000000,0.000000}%
\pgfsetstrokecolor{currentstroke}%
\pgfsetdash{}{0pt}%
\pgfsys@defobject{currentmarker}{\pgfqpoint{0.000000in}{0.000000in}}{\pgfqpoint{0.000000in}{0.020833in}}{%
\pgfpathmoveto{\pgfqpoint{0.000000in}{0.000000in}}%
\pgfpathlineto{\pgfqpoint{0.000000in}{0.020833in}}%
\pgfusepath{stroke,fill}%
}%
\begin{pgfscope}%
\pgfsys@transformshift{1.216028in}{0.422992in}%
\pgfsys@useobject{currentmarker}{}%
\end{pgfscope}%
\end{pgfscope}%
\begin{pgfscope}%
\pgfsetbuttcap%
\pgfsetroundjoin%
\definecolor{currentfill}{rgb}{0.000000,0.000000,0.000000}%
\pgfsetfillcolor{currentfill}%
\pgfsetlinewidth{0.501875pt}%
\definecolor{currentstroke}{rgb}{0.000000,0.000000,0.000000}%
\pgfsetstrokecolor{currentstroke}%
\pgfsetdash{}{0pt}%
\pgfsys@defobject{currentmarker}{\pgfqpoint{0.000000in}{-0.020833in}}{\pgfqpoint{0.000000in}{0.000000in}}{%
\pgfpathmoveto{\pgfqpoint{0.000000in}{0.000000in}}%
\pgfpathlineto{\pgfqpoint{0.000000in}{-0.020833in}}%
\pgfusepath{stroke,fill}%
}%
\begin{pgfscope}%
\pgfsys@transformshift{1.216028in}{2.049193in}%
\pgfsys@useobject{currentmarker}{}%
\end{pgfscope}%
\end{pgfscope}%
\begin{pgfscope}%
\pgfsetbuttcap%
\pgfsetroundjoin%
\definecolor{currentfill}{rgb}{0.000000,0.000000,0.000000}%
\pgfsetfillcolor{currentfill}%
\pgfsetlinewidth{0.501875pt}%
\definecolor{currentstroke}{rgb}{0.000000,0.000000,0.000000}%
\pgfsetstrokecolor{currentstroke}%
\pgfsetdash{}{0pt}%
\pgfsys@defobject{currentmarker}{\pgfqpoint{0.000000in}{0.000000in}}{\pgfqpoint{0.000000in}{0.020833in}}{%
\pgfpathmoveto{\pgfqpoint{0.000000in}{0.000000in}}%
\pgfpathlineto{\pgfqpoint{0.000000in}{0.020833in}}%
\pgfusepath{stroke,fill}%
}%
\begin{pgfscope}%
\pgfsys@transformshift{1.328705in}{0.422992in}%
\pgfsys@useobject{currentmarker}{}%
\end{pgfscope}%
\end{pgfscope}%
\begin{pgfscope}%
\pgfsetbuttcap%
\pgfsetroundjoin%
\definecolor{currentfill}{rgb}{0.000000,0.000000,0.000000}%
\pgfsetfillcolor{currentfill}%
\pgfsetlinewidth{0.501875pt}%
\definecolor{currentstroke}{rgb}{0.000000,0.000000,0.000000}%
\pgfsetstrokecolor{currentstroke}%
\pgfsetdash{}{0pt}%
\pgfsys@defobject{currentmarker}{\pgfqpoint{0.000000in}{-0.020833in}}{\pgfqpoint{0.000000in}{0.000000in}}{%
\pgfpathmoveto{\pgfqpoint{0.000000in}{0.000000in}}%
\pgfpathlineto{\pgfqpoint{0.000000in}{-0.020833in}}%
\pgfusepath{stroke,fill}%
}%
\begin{pgfscope}%
\pgfsys@transformshift{1.328705in}{2.049193in}%
\pgfsys@useobject{currentmarker}{}%
\end{pgfscope}%
\end{pgfscope}%
\begin{pgfscope}%
\pgfsetbuttcap%
\pgfsetroundjoin%
\definecolor{currentfill}{rgb}{0.000000,0.000000,0.000000}%
\pgfsetfillcolor{currentfill}%
\pgfsetlinewidth{0.501875pt}%
\definecolor{currentstroke}{rgb}{0.000000,0.000000,0.000000}%
\pgfsetstrokecolor{currentstroke}%
\pgfsetdash{}{0pt}%
\pgfsys@defobject{currentmarker}{\pgfqpoint{0.000000in}{0.000000in}}{\pgfqpoint{0.000000in}{0.020833in}}{%
\pgfpathmoveto{\pgfqpoint{0.000000in}{0.000000in}}%
\pgfpathlineto{\pgfqpoint{0.000000in}{0.020833in}}%
\pgfusepath{stroke,fill}%
}%
\begin{pgfscope}%
\pgfsys@transformshift{1.554058in}{0.422992in}%
\pgfsys@useobject{currentmarker}{}%
\end{pgfscope}%
\end{pgfscope}%
\begin{pgfscope}%
\pgfsetbuttcap%
\pgfsetroundjoin%
\definecolor{currentfill}{rgb}{0.000000,0.000000,0.000000}%
\pgfsetfillcolor{currentfill}%
\pgfsetlinewidth{0.501875pt}%
\definecolor{currentstroke}{rgb}{0.000000,0.000000,0.000000}%
\pgfsetstrokecolor{currentstroke}%
\pgfsetdash{}{0pt}%
\pgfsys@defobject{currentmarker}{\pgfqpoint{0.000000in}{-0.020833in}}{\pgfqpoint{0.000000in}{0.000000in}}{%
\pgfpathmoveto{\pgfqpoint{0.000000in}{0.000000in}}%
\pgfpathlineto{\pgfqpoint{0.000000in}{-0.020833in}}%
\pgfusepath{stroke,fill}%
}%
\begin{pgfscope}%
\pgfsys@transformshift{1.554058in}{2.049193in}%
\pgfsys@useobject{currentmarker}{}%
\end{pgfscope}%
\end{pgfscope}%
\begin{pgfscope}%
\pgfsetbuttcap%
\pgfsetroundjoin%
\definecolor{currentfill}{rgb}{0.000000,0.000000,0.000000}%
\pgfsetfillcolor{currentfill}%
\pgfsetlinewidth{0.501875pt}%
\definecolor{currentstroke}{rgb}{0.000000,0.000000,0.000000}%
\pgfsetstrokecolor{currentstroke}%
\pgfsetdash{}{0pt}%
\pgfsys@defobject{currentmarker}{\pgfqpoint{0.000000in}{0.000000in}}{\pgfqpoint{0.000000in}{0.020833in}}{%
\pgfpathmoveto{\pgfqpoint{0.000000in}{0.000000in}}%
\pgfpathlineto{\pgfqpoint{0.000000in}{0.020833in}}%
\pgfusepath{stroke,fill}%
}%
\begin{pgfscope}%
\pgfsys@transformshift{1.666734in}{0.422992in}%
\pgfsys@useobject{currentmarker}{}%
\end{pgfscope}%
\end{pgfscope}%
\begin{pgfscope}%
\pgfsetbuttcap%
\pgfsetroundjoin%
\definecolor{currentfill}{rgb}{0.000000,0.000000,0.000000}%
\pgfsetfillcolor{currentfill}%
\pgfsetlinewidth{0.501875pt}%
\definecolor{currentstroke}{rgb}{0.000000,0.000000,0.000000}%
\pgfsetstrokecolor{currentstroke}%
\pgfsetdash{}{0pt}%
\pgfsys@defobject{currentmarker}{\pgfqpoint{0.000000in}{-0.020833in}}{\pgfqpoint{0.000000in}{0.000000in}}{%
\pgfpathmoveto{\pgfqpoint{0.000000in}{0.000000in}}%
\pgfpathlineto{\pgfqpoint{0.000000in}{-0.020833in}}%
\pgfusepath{stroke,fill}%
}%
\begin{pgfscope}%
\pgfsys@transformshift{1.666734in}{2.049193in}%
\pgfsys@useobject{currentmarker}{}%
\end{pgfscope}%
\end{pgfscope}%
\begin{pgfscope}%
\pgfsetbuttcap%
\pgfsetroundjoin%
\definecolor{currentfill}{rgb}{0.000000,0.000000,0.000000}%
\pgfsetfillcolor{currentfill}%
\pgfsetlinewidth{0.501875pt}%
\definecolor{currentstroke}{rgb}{0.000000,0.000000,0.000000}%
\pgfsetstrokecolor{currentstroke}%
\pgfsetdash{}{0pt}%
\pgfsys@defobject{currentmarker}{\pgfqpoint{0.000000in}{0.000000in}}{\pgfqpoint{0.000000in}{0.020833in}}{%
\pgfpathmoveto{\pgfqpoint{0.000000in}{0.000000in}}%
\pgfpathlineto{\pgfqpoint{0.000000in}{0.020833in}}%
\pgfusepath{stroke,fill}%
}%
\begin{pgfscope}%
\pgfsys@transformshift{1.779411in}{0.422992in}%
\pgfsys@useobject{currentmarker}{}%
\end{pgfscope}%
\end{pgfscope}%
\begin{pgfscope}%
\pgfsetbuttcap%
\pgfsetroundjoin%
\definecolor{currentfill}{rgb}{0.000000,0.000000,0.000000}%
\pgfsetfillcolor{currentfill}%
\pgfsetlinewidth{0.501875pt}%
\definecolor{currentstroke}{rgb}{0.000000,0.000000,0.000000}%
\pgfsetstrokecolor{currentstroke}%
\pgfsetdash{}{0pt}%
\pgfsys@defobject{currentmarker}{\pgfqpoint{0.000000in}{-0.020833in}}{\pgfqpoint{0.000000in}{0.000000in}}{%
\pgfpathmoveto{\pgfqpoint{0.000000in}{0.000000in}}%
\pgfpathlineto{\pgfqpoint{0.000000in}{-0.020833in}}%
\pgfusepath{stroke,fill}%
}%
\begin{pgfscope}%
\pgfsys@transformshift{1.779411in}{2.049193in}%
\pgfsys@useobject{currentmarker}{}%
\end{pgfscope}%
\end{pgfscope}%
\begin{pgfscope}%
\pgfsetbuttcap%
\pgfsetroundjoin%
\definecolor{currentfill}{rgb}{0.000000,0.000000,0.000000}%
\pgfsetfillcolor{currentfill}%
\pgfsetlinewidth{0.501875pt}%
\definecolor{currentstroke}{rgb}{0.000000,0.000000,0.000000}%
\pgfsetstrokecolor{currentstroke}%
\pgfsetdash{}{0pt}%
\pgfsys@defobject{currentmarker}{\pgfqpoint{0.000000in}{0.000000in}}{\pgfqpoint{0.000000in}{0.020833in}}{%
\pgfpathmoveto{\pgfqpoint{0.000000in}{0.000000in}}%
\pgfpathlineto{\pgfqpoint{0.000000in}{0.020833in}}%
\pgfusepath{stroke,fill}%
}%
\begin{pgfscope}%
\pgfsys@transformshift{2.004764in}{0.422992in}%
\pgfsys@useobject{currentmarker}{}%
\end{pgfscope}%
\end{pgfscope}%
\begin{pgfscope}%
\pgfsetbuttcap%
\pgfsetroundjoin%
\definecolor{currentfill}{rgb}{0.000000,0.000000,0.000000}%
\pgfsetfillcolor{currentfill}%
\pgfsetlinewidth{0.501875pt}%
\definecolor{currentstroke}{rgb}{0.000000,0.000000,0.000000}%
\pgfsetstrokecolor{currentstroke}%
\pgfsetdash{}{0pt}%
\pgfsys@defobject{currentmarker}{\pgfqpoint{0.000000in}{-0.020833in}}{\pgfqpoint{0.000000in}{0.000000in}}{%
\pgfpathmoveto{\pgfqpoint{0.000000in}{0.000000in}}%
\pgfpathlineto{\pgfqpoint{0.000000in}{-0.020833in}}%
\pgfusepath{stroke,fill}%
}%
\begin{pgfscope}%
\pgfsys@transformshift{2.004764in}{2.049193in}%
\pgfsys@useobject{currentmarker}{}%
\end{pgfscope}%
\end{pgfscope}%
\begin{pgfscope}%
\pgfsetbuttcap%
\pgfsetroundjoin%
\definecolor{currentfill}{rgb}{0.000000,0.000000,0.000000}%
\pgfsetfillcolor{currentfill}%
\pgfsetlinewidth{0.501875pt}%
\definecolor{currentstroke}{rgb}{0.000000,0.000000,0.000000}%
\pgfsetstrokecolor{currentstroke}%
\pgfsetdash{}{0pt}%
\pgfsys@defobject{currentmarker}{\pgfqpoint{0.000000in}{0.000000in}}{\pgfqpoint{0.000000in}{0.020833in}}{%
\pgfpathmoveto{\pgfqpoint{0.000000in}{0.000000in}}%
\pgfpathlineto{\pgfqpoint{0.000000in}{0.020833in}}%
\pgfusepath{stroke,fill}%
}%
\begin{pgfscope}%
\pgfsys@transformshift{2.117440in}{0.422992in}%
\pgfsys@useobject{currentmarker}{}%
\end{pgfscope}%
\end{pgfscope}%
\begin{pgfscope}%
\pgfsetbuttcap%
\pgfsetroundjoin%
\definecolor{currentfill}{rgb}{0.000000,0.000000,0.000000}%
\pgfsetfillcolor{currentfill}%
\pgfsetlinewidth{0.501875pt}%
\definecolor{currentstroke}{rgb}{0.000000,0.000000,0.000000}%
\pgfsetstrokecolor{currentstroke}%
\pgfsetdash{}{0pt}%
\pgfsys@defobject{currentmarker}{\pgfqpoint{0.000000in}{-0.020833in}}{\pgfqpoint{0.000000in}{0.000000in}}{%
\pgfpathmoveto{\pgfqpoint{0.000000in}{0.000000in}}%
\pgfpathlineto{\pgfqpoint{0.000000in}{-0.020833in}}%
\pgfusepath{stroke,fill}%
}%
\begin{pgfscope}%
\pgfsys@transformshift{2.117440in}{2.049193in}%
\pgfsys@useobject{currentmarker}{}%
\end{pgfscope}%
\end{pgfscope}%
\begin{pgfscope}%
\pgfsetbuttcap%
\pgfsetroundjoin%
\definecolor{currentfill}{rgb}{0.000000,0.000000,0.000000}%
\pgfsetfillcolor{currentfill}%
\pgfsetlinewidth{0.501875pt}%
\definecolor{currentstroke}{rgb}{0.000000,0.000000,0.000000}%
\pgfsetstrokecolor{currentstroke}%
\pgfsetdash{}{0pt}%
\pgfsys@defobject{currentmarker}{\pgfqpoint{0.000000in}{0.000000in}}{\pgfqpoint{0.000000in}{0.020833in}}{%
\pgfpathmoveto{\pgfqpoint{0.000000in}{0.000000in}}%
\pgfpathlineto{\pgfqpoint{0.000000in}{0.020833in}}%
\pgfusepath{stroke,fill}%
}%
\begin{pgfscope}%
\pgfsys@transformshift{2.230116in}{0.422992in}%
\pgfsys@useobject{currentmarker}{}%
\end{pgfscope}%
\end{pgfscope}%
\begin{pgfscope}%
\pgfsetbuttcap%
\pgfsetroundjoin%
\definecolor{currentfill}{rgb}{0.000000,0.000000,0.000000}%
\pgfsetfillcolor{currentfill}%
\pgfsetlinewidth{0.501875pt}%
\definecolor{currentstroke}{rgb}{0.000000,0.000000,0.000000}%
\pgfsetstrokecolor{currentstroke}%
\pgfsetdash{}{0pt}%
\pgfsys@defobject{currentmarker}{\pgfqpoint{0.000000in}{-0.020833in}}{\pgfqpoint{0.000000in}{0.000000in}}{%
\pgfpathmoveto{\pgfqpoint{0.000000in}{0.000000in}}%
\pgfpathlineto{\pgfqpoint{0.000000in}{-0.020833in}}%
\pgfusepath{stroke,fill}%
}%
\begin{pgfscope}%
\pgfsys@transformshift{2.230116in}{2.049193in}%
\pgfsys@useobject{currentmarker}{}%
\end{pgfscope}%
\end{pgfscope}%
\begin{pgfscope}%
\pgfsetbuttcap%
\pgfsetroundjoin%
\definecolor{currentfill}{rgb}{0.000000,0.000000,0.000000}%
\pgfsetfillcolor{currentfill}%
\pgfsetlinewidth{0.501875pt}%
\definecolor{currentstroke}{rgb}{0.000000,0.000000,0.000000}%
\pgfsetstrokecolor{currentstroke}%
\pgfsetdash{}{0pt}%
\pgfsys@defobject{currentmarker}{\pgfqpoint{0.000000in}{0.000000in}}{\pgfqpoint{0.000000in}{0.020833in}}{%
\pgfpathmoveto{\pgfqpoint{0.000000in}{0.000000in}}%
\pgfpathlineto{\pgfqpoint{0.000000in}{0.020833in}}%
\pgfusepath{stroke,fill}%
}%
\begin{pgfscope}%
\pgfsys@transformshift{2.455469in}{0.422992in}%
\pgfsys@useobject{currentmarker}{}%
\end{pgfscope}%
\end{pgfscope}%
\begin{pgfscope}%
\pgfsetbuttcap%
\pgfsetroundjoin%
\definecolor{currentfill}{rgb}{0.000000,0.000000,0.000000}%
\pgfsetfillcolor{currentfill}%
\pgfsetlinewidth{0.501875pt}%
\definecolor{currentstroke}{rgb}{0.000000,0.000000,0.000000}%
\pgfsetstrokecolor{currentstroke}%
\pgfsetdash{}{0pt}%
\pgfsys@defobject{currentmarker}{\pgfqpoint{0.000000in}{-0.020833in}}{\pgfqpoint{0.000000in}{0.000000in}}{%
\pgfpathmoveto{\pgfqpoint{0.000000in}{0.000000in}}%
\pgfpathlineto{\pgfqpoint{0.000000in}{-0.020833in}}%
\pgfusepath{stroke,fill}%
}%
\begin{pgfscope}%
\pgfsys@transformshift{2.455469in}{2.049193in}%
\pgfsys@useobject{currentmarker}{}%
\end{pgfscope}%
\end{pgfscope}%
\begin{pgfscope}%
\pgfsetbuttcap%
\pgfsetroundjoin%
\definecolor{currentfill}{rgb}{0.000000,0.000000,0.000000}%
\pgfsetfillcolor{currentfill}%
\pgfsetlinewidth{0.501875pt}%
\definecolor{currentstroke}{rgb}{0.000000,0.000000,0.000000}%
\pgfsetstrokecolor{currentstroke}%
\pgfsetdash{}{0pt}%
\pgfsys@defobject{currentmarker}{\pgfqpoint{0.000000in}{0.000000in}}{\pgfqpoint{0.000000in}{0.020833in}}{%
\pgfpathmoveto{\pgfqpoint{0.000000in}{0.000000in}}%
\pgfpathlineto{\pgfqpoint{0.000000in}{0.020833in}}%
\pgfusepath{stroke,fill}%
}%
\begin{pgfscope}%
\pgfsys@transformshift{2.568146in}{0.422992in}%
\pgfsys@useobject{currentmarker}{}%
\end{pgfscope}%
\end{pgfscope}%
\begin{pgfscope}%
\pgfsetbuttcap%
\pgfsetroundjoin%
\definecolor{currentfill}{rgb}{0.000000,0.000000,0.000000}%
\pgfsetfillcolor{currentfill}%
\pgfsetlinewidth{0.501875pt}%
\definecolor{currentstroke}{rgb}{0.000000,0.000000,0.000000}%
\pgfsetstrokecolor{currentstroke}%
\pgfsetdash{}{0pt}%
\pgfsys@defobject{currentmarker}{\pgfqpoint{0.000000in}{-0.020833in}}{\pgfqpoint{0.000000in}{0.000000in}}{%
\pgfpathmoveto{\pgfqpoint{0.000000in}{0.000000in}}%
\pgfpathlineto{\pgfqpoint{0.000000in}{-0.020833in}}%
\pgfusepath{stroke,fill}%
}%
\begin{pgfscope}%
\pgfsys@transformshift{2.568146in}{2.049193in}%
\pgfsys@useobject{currentmarker}{}%
\end{pgfscope}%
\end{pgfscope}%
\begin{pgfscope}%
\pgfsetbuttcap%
\pgfsetroundjoin%
\definecolor{currentfill}{rgb}{0.000000,0.000000,0.000000}%
\pgfsetfillcolor{currentfill}%
\pgfsetlinewidth{0.501875pt}%
\definecolor{currentstroke}{rgb}{0.000000,0.000000,0.000000}%
\pgfsetstrokecolor{currentstroke}%
\pgfsetdash{}{0pt}%
\pgfsys@defobject{currentmarker}{\pgfqpoint{0.000000in}{0.000000in}}{\pgfqpoint{0.000000in}{0.020833in}}{%
\pgfpathmoveto{\pgfqpoint{0.000000in}{0.000000in}}%
\pgfpathlineto{\pgfqpoint{0.000000in}{0.020833in}}%
\pgfusepath{stroke,fill}%
}%
\begin{pgfscope}%
\pgfsys@transformshift{2.680822in}{0.422992in}%
\pgfsys@useobject{currentmarker}{}%
\end{pgfscope}%
\end{pgfscope}%
\begin{pgfscope}%
\pgfsetbuttcap%
\pgfsetroundjoin%
\definecolor{currentfill}{rgb}{0.000000,0.000000,0.000000}%
\pgfsetfillcolor{currentfill}%
\pgfsetlinewidth{0.501875pt}%
\definecolor{currentstroke}{rgb}{0.000000,0.000000,0.000000}%
\pgfsetstrokecolor{currentstroke}%
\pgfsetdash{}{0pt}%
\pgfsys@defobject{currentmarker}{\pgfqpoint{0.000000in}{-0.020833in}}{\pgfqpoint{0.000000in}{0.000000in}}{%
\pgfpathmoveto{\pgfqpoint{0.000000in}{0.000000in}}%
\pgfpathlineto{\pgfqpoint{0.000000in}{-0.020833in}}%
\pgfusepath{stroke,fill}%
}%
\begin{pgfscope}%
\pgfsys@transformshift{2.680822in}{2.049193in}%
\pgfsys@useobject{currentmarker}{}%
\end{pgfscope}%
\end{pgfscope}%
\begin{pgfscope}%
\pgfsetbuttcap%
\pgfsetroundjoin%
\definecolor{currentfill}{rgb}{0.000000,0.000000,0.000000}%
\pgfsetfillcolor{currentfill}%
\pgfsetlinewidth{0.501875pt}%
\definecolor{currentstroke}{rgb}{0.000000,0.000000,0.000000}%
\pgfsetstrokecolor{currentstroke}%
\pgfsetdash{}{0pt}%
\pgfsys@defobject{currentmarker}{\pgfqpoint{0.000000in}{0.000000in}}{\pgfqpoint{0.000000in}{0.020833in}}{%
\pgfpathmoveto{\pgfqpoint{0.000000in}{0.000000in}}%
\pgfpathlineto{\pgfqpoint{0.000000in}{0.020833in}}%
\pgfusepath{stroke,fill}%
}%
\begin{pgfscope}%
\pgfsys@transformshift{2.793499in}{0.422992in}%
\pgfsys@useobject{currentmarker}{}%
\end{pgfscope}%
\end{pgfscope}%
\begin{pgfscope}%
\pgfsetbuttcap%
\pgfsetroundjoin%
\definecolor{currentfill}{rgb}{0.000000,0.000000,0.000000}%
\pgfsetfillcolor{currentfill}%
\pgfsetlinewidth{0.501875pt}%
\definecolor{currentstroke}{rgb}{0.000000,0.000000,0.000000}%
\pgfsetstrokecolor{currentstroke}%
\pgfsetdash{}{0pt}%
\pgfsys@defobject{currentmarker}{\pgfqpoint{0.000000in}{-0.020833in}}{\pgfqpoint{0.000000in}{0.000000in}}{%
\pgfpathmoveto{\pgfqpoint{0.000000in}{0.000000in}}%
\pgfpathlineto{\pgfqpoint{0.000000in}{-0.020833in}}%
\pgfusepath{stroke,fill}%
}%
\begin{pgfscope}%
\pgfsys@transformshift{2.793499in}{2.049193in}%
\pgfsys@useobject{currentmarker}{}%
\end{pgfscope}%
\end{pgfscope}%
\begin{pgfscope}%
\definecolor{textcolor}{rgb}{0.000000,0.000000,0.000000}%
\pgfsetstrokecolor{textcolor}%
\pgfsetfillcolor{textcolor}%
\pgftext[x=1.678002in,y=0.184413in,,top]{\color{textcolor}\rmfamily\fontsize{10.000000}{12.000000}\selectfont \(\displaystyle K\)}%
\end{pgfscope}%
\begin{pgfscope}%
\pgfsetbuttcap%
\pgfsetroundjoin%
\definecolor{currentfill}{rgb}{0.000000,0.000000,0.000000}%
\pgfsetfillcolor{currentfill}%
\pgfsetlinewidth{0.501875pt}%
\definecolor{currentstroke}{rgb}{0.000000,0.000000,0.000000}%
\pgfsetstrokecolor{currentstroke}%
\pgfsetdash{}{0pt}%
\pgfsys@defobject{currentmarker}{\pgfqpoint{0.000000in}{0.000000in}}{\pgfqpoint{0.041667in}{0.000000in}}{%
\pgfpathmoveto{\pgfqpoint{0.000000in}{0.000000in}}%
\pgfpathlineto{\pgfqpoint{0.041667in}{0.000000in}}%
\pgfusepath{stroke,fill}%
}%
\begin{pgfscope}%
\pgfsys@transformshift{0.539970in}{0.928314in}%
\pgfsys@useobject{currentmarker}{}%
\end{pgfscope}%
\end{pgfscope}%
\begin{pgfscope}%
\pgfsetbuttcap%
\pgfsetroundjoin%
\definecolor{currentfill}{rgb}{0.000000,0.000000,0.000000}%
\pgfsetfillcolor{currentfill}%
\pgfsetlinewidth{0.501875pt}%
\definecolor{currentstroke}{rgb}{0.000000,0.000000,0.000000}%
\pgfsetstrokecolor{currentstroke}%
\pgfsetdash{}{0pt}%
\pgfsys@defobject{currentmarker}{\pgfqpoint{-0.041667in}{0.000000in}}{\pgfqpoint{-0.000000in}{0.000000in}}{%
\pgfpathmoveto{\pgfqpoint{-0.000000in}{0.000000in}}%
\pgfpathlineto{\pgfqpoint{-0.041667in}{0.000000in}}%
\pgfusepath{stroke,fill}%
}%
\begin{pgfscope}%
\pgfsys@transformshift{2.816034in}{0.928314in}%
\pgfsys@useobject{currentmarker}{}%
\end{pgfscope}%
\end{pgfscope}%
\begin{pgfscope}%
\definecolor{textcolor}{rgb}{0.000000,0.000000,0.000000}%
\pgfsetstrokecolor{textcolor}%
\pgfsetfillcolor{textcolor}%
\pgftext[x=0.313889in, y=0.875552in, left, base]{\color{textcolor}\rmfamily\fontsize{10.000000}{12.000000}\selectfont \(\displaystyle {0.8}\)}%
\end{pgfscope}%
\begin{pgfscope}%
\pgfsetbuttcap%
\pgfsetroundjoin%
\definecolor{currentfill}{rgb}{0.000000,0.000000,0.000000}%
\pgfsetfillcolor{currentfill}%
\pgfsetlinewidth{0.501875pt}%
\definecolor{currentstroke}{rgb}{0.000000,0.000000,0.000000}%
\pgfsetstrokecolor{currentstroke}%
\pgfsetdash{}{0pt}%
\pgfsys@defobject{currentmarker}{\pgfqpoint{0.000000in}{0.000000in}}{\pgfqpoint{0.041667in}{0.000000in}}{%
\pgfpathmoveto{\pgfqpoint{0.000000in}{0.000000in}}%
\pgfpathlineto{\pgfqpoint{0.041667in}{0.000000in}}%
\pgfusepath{stroke,fill}%
}%
\begin{pgfscope}%
\pgfsys@transformshift{0.539970in}{1.457545in}%
\pgfsys@useobject{currentmarker}{}%
\end{pgfscope}%
\end{pgfscope}%
\begin{pgfscope}%
\pgfsetbuttcap%
\pgfsetroundjoin%
\definecolor{currentfill}{rgb}{0.000000,0.000000,0.000000}%
\pgfsetfillcolor{currentfill}%
\pgfsetlinewidth{0.501875pt}%
\definecolor{currentstroke}{rgb}{0.000000,0.000000,0.000000}%
\pgfsetstrokecolor{currentstroke}%
\pgfsetdash{}{0pt}%
\pgfsys@defobject{currentmarker}{\pgfqpoint{-0.041667in}{0.000000in}}{\pgfqpoint{-0.000000in}{0.000000in}}{%
\pgfpathmoveto{\pgfqpoint{-0.000000in}{0.000000in}}%
\pgfpathlineto{\pgfqpoint{-0.041667in}{0.000000in}}%
\pgfusepath{stroke,fill}%
}%
\begin{pgfscope}%
\pgfsys@transformshift{2.816034in}{1.457545in}%
\pgfsys@useobject{currentmarker}{}%
\end{pgfscope}%
\end{pgfscope}%
\begin{pgfscope}%
\definecolor{textcolor}{rgb}{0.000000,0.000000,0.000000}%
\pgfsetstrokecolor{textcolor}%
\pgfsetfillcolor{textcolor}%
\pgftext[x=0.313889in, y=1.404784in, left, base]{\color{textcolor}\rmfamily\fontsize{10.000000}{12.000000}\selectfont \(\displaystyle {0.9}\)}%
\end{pgfscope}%
\begin{pgfscope}%
\pgfsetbuttcap%
\pgfsetroundjoin%
\definecolor{currentfill}{rgb}{0.000000,0.000000,0.000000}%
\pgfsetfillcolor{currentfill}%
\pgfsetlinewidth{0.501875pt}%
\definecolor{currentstroke}{rgb}{0.000000,0.000000,0.000000}%
\pgfsetstrokecolor{currentstroke}%
\pgfsetdash{}{0pt}%
\pgfsys@defobject{currentmarker}{\pgfqpoint{0.000000in}{0.000000in}}{\pgfqpoint{0.041667in}{0.000000in}}{%
\pgfpathmoveto{\pgfqpoint{0.000000in}{0.000000in}}%
\pgfpathlineto{\pgfqpoint{0.041667in}{0.000000in}}%
\pgfusepath{stroke,fill}%
}%
\begin{pgfscope}%
\pgfsys@transformshift{0.539970in}{1.986777in}%
\pgfsys@useobject{currentmarker}{}%
\end{pgfscope}%
\end{pgfscope}%
\begin{pgfscope}%
\pgfsetbuttcap%
\pgfsetroundjoin%
\definecolor{currentfill}{rgb}{0.000000,0.000000,0.000000}%
\pgfsetfillcolor{currentfill}%
\pgfsetlinewidth{0.501875pt}%
\definecolor{currentstroke}{rgb}{0.000000,0.000000,0.000000}%
\pgfsetstrokecolor{currentstroke}%
\pgfsetdash{}{0pt}%
\pgfsys@defobject{currentmarker}{\pgfqpoint{-0.041667in}{0.000000in}}{\pgfqpoint{-0.000000in}{0.000000in}}{%
\pgfpathmoveto{\pgfqpoint{-0.000000in}{0.000000in}}%
\pgfpathlineto{\pgfqpoint{-0.041667in}{0.000000in}}%
\pgfusepath{stroke,fill}%
}%
\begin{pgfscope}%
\pgfsys@transformshift{2.816034in}{1.986777in}%
\pgfsys@useobject{currentmarker}{}%
\end{pgfscope}%
\end{pgfscope}%
\begin{pgfscope}%
\definecolor{textcolor}{rgb}{0.000000,0.000000,0.000000}%
\pgfsetstrokecolor{textcolor}%
\pgfsetfillcolor{textcolor}%
\pgftext[x=0.313889in, y=1.934016in, left, base]{\color{textcolor}\rmfamily\fontsize{10.000000}{12.000000}\selectfont \(\displaystyle {1.0}\)}%
\end{pgfscope}%
\begin{pgfscope}%
\pgfsetbuttcap%
\pgfsetroundjoin%
\definecolor{currentfill}{rgb}{0.000000,0.000000,0.000000}%
\pgfsetfillcolor{currentfill}%
\pgfsetlinewidth{0.501875pt}%
\definecolor{currentstroke}{rgb}{0.000000,0.000000,0.000000}%
\pgfsetstrokecolor{currentstroke}%
\pgfsetdash{}{0pt}%
\pgfsys@defobject{currentmarker}{\pgfqpoint{0.000000in}{0.000000in}}{\pgfqpoint{0.020833in}{0.000000in}}{%
\pgfpathmoveto{\pgfqpoint{0.000000in}{0.000000in}}%
\pgfpathlineto{\pgfqpoint{0.020833in}{0.000000in}}%
\pgfusepath{stroke,fill}%
}%
\begin{pgfscope}%
\pgfsys@transformshift{0.539970in}{0.504929in}%
\pgfsys@useobject{currentmarker}{}%
\end{pgfscope}%
\end{pgfscope}%
\begin{pgfscope}%
\pgfsetbuttcap%
\pgfsetroundjoin%
\definecolor{currentfill}{rgb}{0.000000,0.000000,0.000000}%
\pgfsetfillcolor{currentfill}%
\pgfsetlinewidth{0.501875pt}%
\definecolor{currentstroke}{rgb}{0.000000,0.000000,0.000000}%
\pgfsetstrokecolor{currentstroke}%
\pgfsetdash{}{0pt}%
\pgfsys@defobject{currentmarker}{\pgfqpoint{-0.020833in}{0.000000in}}{\pgfqpoint{-0.000000in}{0.000000in}}{%
\pgfpathmoveto{\pgfqpoint{-0.000000in}{0.000000in}}%
\pgfpathlineto{\pgfqpoint{-0.020833in}{0.000000in}}%
\pgfusepath{stroke,fill}%
}%
\begin{pgfscope}%
\pgfsys@transformshift{2.816034in}{0.504929in}%
\pgfsys@useobject{currentmarker}{}%
\end{pgfscope}%
\end{pgfscope}%
\begin{pgfscope}%
\pgfsetbuttcap%
\pgfsetroundjoin%
\definecolor{currentfill}{rgb}{0.000000,0.000000,0.000000}%
\pgfsetfillcolor{currentfill}%
\pgfsetlinewidth{0.501875pt}%
\definecolor{currentstroke}{rgb}{0.000000,0.000000,0.000000}%
\pgfsetstrokecolor{currentstroke}%
\pgfsetdash{}{0pt}%
\pgfsys@defobject{currentmarker}{\pgfqpoint{0.000000in}{0.000000in}}{\pgfqpoint{0.020833in}{0.000000in}}{%
\pgfpathmoveto{\pgfqpoint{0.000000in}{0.000000in}}%
\pgfpathlineto{\pgfqpoint{0.020833in}{0.000000in}}%
\pgfusepath{stroke,fill}%
}%
\begin{pgfscope}%
\pgfsys@transformshift{0.539970in}{0.610775in}%
\pgfsys@useobject{currentmarker}{}%
\end{pgfscope}%
\end{pgfscope}%
\begin{pgfscope}%
\pgfsetbuttcap%
\pgfsetroundjoin%
\definecolor{currentfill}{rgb}{0.000000,0.000000,0.000000}%
\pgfsetfillcolor{currentfill}%
\pgfsetlinewidth{0.501875pt}%
\definecolor{currentstroke}{rgb}{0.000000,0.000000,0.000000}%
\pgfsetstrokecolor{currentstroke}%
\pgfsetdash{}{0pt}%
\pgfsys@defobject{currentmarker}{\pgfqpoint{-0.020833in}{0.000000in}}{\pgfqpoint{-0.000000in}{0.000000in}}{%
\pgfpathmoveto{\pgfqpoint{-0.000000in}{0.000000in}}%
\pgfpathlineto{\pgfqpoint{-0.020833in}{0.000000in}}%
\pgfusepath{stroke,fill}%
}%
\begin{pgfscope}%
\pgfsys@transformshift{2.816034in}{0.610775in}%
\pgfsys@useobject{currentmarker}{}%
\end{pgfscope}%
\end{pgfscope}%
\begin{pgfscope}%
\pgfsetbuttcap%
\pgfsetroundjoin%
\definecolor{currentfill}{rgb}{0.000000,0.000000,0.000000}%
\pgfsetfillcolor{currentfill}%
\pgfsetlinewidth{0.501875pt}%
\definecolor{currentstroke}{rgb}{0.000000,0.000000,0.000000}%
\pgfsetstrokecolor{currentstroke}%
\pgfsetdash{}{0pt}%
\pgfsys@defobject{currentmarker}{\pgfqpoint{0.000000in}{0.000000in}}{\pgfqpoint{0.020833in}{0.000000in}}{%
\pgfpathmoveto{\pgfqpoint{0.000000in}{0.000000in}}%
\pgfpathlineto{\pgfqpoint{0.020833in}{0.000000in}}%
\pgfusepath{stroke,fill}%
}%
\begin{pgfscope}%
\pgfsys@transformshift{0.539970in}{0.716621in}%
\pgfsys@useobject{currentmarker}{}%
\end{pgfscope}%
\end{pgfscope}%
\begin{pgfscope}%
\pgfsetbuttcap%
\pgfsetroundjoin%
\definecolor{currentfill}{rgb}{0.000000,0.000000,0.000000}%
\pgfsetfillcolor{currentfill}%
\pgfsetlinewidth{0.501875pt}%
\definecolor{currentstroke}{rgb}{0.000000,0.000000,0.000000}%
\pgfsetstrokecolor{currentstroke}%
\pgfsetdash{}{0pt}%
\pgfsys@defobject{currentmarker}{\pgfqpoint{-0.020833in}{0.000000in}}{\pgfqpoint{-0.000000in}{0.000000in}}{%
\pgfpathmoveto{\pgfqpoint{-0.000000in}{0.000000in}}%
\pgfpathlineto{\pgfqpoint{-0.020833in}{0.000000in}}%
\pgfusepath{stroke,fill}%
}%
\begin{pgfscope}%
\pgfsys@transformshift{2.816034in}{0.716621in}%
\pgfsys@useobject{currentmarker}{}%
\end{pgfscope}%
\end{pgfscope}%
\begin{pgfscope}%
\pgfsetbuttcap%
\pgfsetroundjoin%
\definecolor{currentfill}{rgb}{0.000000,0.000000,0.000000}%
\pgfsetfillcolor{currentfill}%
\pgfsetlinewidth{0.501875pt}%
\definecolor{currentstroke}{rgb}{0.000000,0.000000,0.000000}%
\pgfsetstrokecolor{currentstroke}%
\pgfsetdash{}{0pt}%
\pgfsys@defobject{currentmarker}{\pgfqpoint{0.000000in}{0.000000in}}{\pgfqpoint{0.020833in}{0.000000in}}{%
\pgfpathmoveto{\pgfqpoint{0.000000in}{0.000000in}}%
\pgfpathlineto{\pgfqpoint{0.020833in}{0.000000in}}%
\pgfusepath{stroke,fill}%
}%
\begin{pgfscope}%
\pgfsys@transformshift{0.539970in}{0.822468in}%
\pgfsys@useobject{currentmarker}{}%
\end{pgfscope}%
\end{pgfscope}%
\begin{pgfscope}%
\pgfsetbuttcap%
\pgfsetroundjoin%
\definecolor{currentfill}{rgb}{0.000000,0.000000,0.000000}%
\pgfsetfillcolor{currentfill}%
\pgfsetlinewidth{0.501875pt}%
\definecolor{currentstroke}{rgb}{0.000000,0.000000,0.000000}%
\pgfsetstrokecolor{currentstroke}%
\pgfsetdash{}{0pt}%
\pgfsys@defobject{currentmarker}{\pgfqpoint{-0.020833in}{0.000000in}}{\pgfqpoint{-0.000000in}{0.000000in}}{%
\pgfpathmoveto{\pgfqpoint{-0.000000in}{0.000000in}}%
\pgfpathlineto{\pgfqpoint{-0.020833in}{0.000000in}}%
\pgfusepath{stroke,fill}%
}%
\begin{pgfscope}%
\pgfsys@transformshift{2.816034in}{0.822468in}%
\pgfsys@useobject{currentmarker}{}%
\end{pgfscope}%
\end{pgfscope}%
\begin{pgfscope}%
\pgfsetbuttcap%
\pgfsetroundjoin%
\definecolor{currentfill}{rgb}{0.000000,0.000000,0.000000}%
\pgfsetfillcolor{currentfill}%
\pgfsetlinewidth{0.501875pt}%
\definecolor{currentstroke}{rgb}{0.000000,0.000000,0.000000}%
\pgfsetstrokecolor{currentstroke}%
\pgfsetdash{}{0pt}%
\pgfsys@defobject{currentmarker}{\pgfqpoint{0.000000in}{0.000000in}}{\pgfqpoint{0.020833in}{0.000000in}}{%
\pgfpathmoveto{\pgfqpoint{0.000000in}{0.000000in}}%
\pgfpathlineto{\pgfqpoint{0.020833in}{0.000000in}}%
\pgfusepath{stroke,fill}%
}%
\begin{pgfscope}%
\pgfsys@transformshift{0.539970in}{1.034160in}%
\pgfsys@useobject{currentmarker}{}%
\end{pgfscope}%
\end{pgfscope}%
\begin{pgfscope}%
\pgfsetbuttcap%
\pgfsetroundjoin%
\definecolor{currentfill}{rgb}{0.000000,0.000000,0.000000}%
\pgfsetfillcolor{currentfill}%
\pgfsetlinewidth{0.501875pt}%
\definecolor{currentstroke}{rgb}{0.000000,0.000000,0.000000}%
\pgfsetstrokecolor{currentstroke}%
\pgfsetdash{}{0pt}%
\pgfsys@defobject{currentmarker}{\pgfqpoint{-0.020833in}{0.000000in}}{\pgfqpoint{-0.000000in}{0.000000in}}{%
\pgfpathmoveto{\pgfqpoint{-0.000000in}{0.000000in}}%
\pgfpathlineto{\pgfqpoint{-0.020833in}{0.000000in}}%
\pgfusepath{stroke,fill}%
}%
\begin{pgfscope}%
\pgfsys@transformshift{2.816034in}{1.034160in}%
\pgfsys@useobject{currentmarker}{}%
\end{pgfscope}%
\end{pgfscope}%
\begin{pgfscope}%
\pgfsetbuttcap%
\pgfsetroundjoin%
\definecolor{currentfill}{rgb}{0.000000,0.000000,0.000000}%
\pgfsetfillcolor{currentfill}%
\pgfsetlinewidth{0.501875pt}%
\definecolor{currentstroke}{rgb}{0.000000,0.000000,0.000000}%
\pgfsetstrokecolor{currentstroke}%
\pgfsetdash{}{0pt}%
\pgfsys@defobject{currentmarker}{\pgfqpoint{0.000000in}{0.000000in}}{\pgfqpoint{0.020833in}{0.000000in}}{%
\pgfpathmoveto{\pgfqpoint{0.000000in}{0.000000in}}%
\pgfpathlineto{\pgfqpoint{0.020833in}{0.000000in}}%
\pgfusepath{stroke,fill}%
}%
\begin{pgfscope}%
\pgfsys@transformshift{0.539970in}{1.140007in}%
\pgfsys@useobject{currentmarker}{}%
\end{pgfscope}%
\end{pgfscope}%
\begin{pgfscope}%
\pgfsetbuttcap%
\pgfsetroundjoin%
\definecolor{currentfill}{rgb}{0.000000,0.000000,0.000000}%
\pgfsetfillcolor{currentfill}%
\pgfsetlinewidth{0.501875pt}%
\definecolor{currentstroke}{rgb}{0.000000,0.000000,0.000000}%
\pgfsetstrokecolor{currentstroke}%
\pgfsetdash{}{0pt}%
\pgfsys@defobject{currentmarker}{\pgfqpoint{-0.020833in}{0.000000in}}{\pgfqpoint{-0.000000in}{0.000000in}}{%
\pgfpathmoveto{\pgfqpoint{-0.000000in}{0.000000in}}%
\pgfpathlineto{\pgfqpoint{-0.020833in}{0.000000in}}%
\pgfusepath{stroke,fill}%
}%
\begin{pgfscope}%
\pgfsys@transformshift{2.816034in}{1.140007in}%
\pgfsys@useobject{currentmarker}{}%
\end{pgfscope}%
\end{pgfscope}%
\begin{pgfscope}%
\pgfsetbuttcap%
\pgfsetroundjoin%
\definecolor{currentfill}{rgb}{0.000000,0.000000,0.000000}%
\pgfsetfillcolor{currentfill}%
\pgfsetlinewidth{0.501875pt}%
\definecolor{currentstroke}{rgb}{0.000000,0.000000,0.000000}%
\pgfsetstrokecolor{currentstroke}%
\pgfsetdash{}{0pt}%
\pgfsys@defobject{currentmarker}{\pgfqpoint{0.000000in}{0.000000in}}{\pgfqpoint{0.020833in}{0.000000in}}{%
\pgfpathmoveto{\pgfqpoint{0.000000in}{0.000000in}}%
\pgfpathlineto{\pgfqpoint{0.020833in}{0.000000in}}%
\pgfusepath{stroke,fill}%
}%
\begin{pgfscope}%
\pgfsys@transformshift{0.539970in}{1.245853in}%
\pgfsys@useobject{currentmarker}{}%
\end{pgfscope}%
\end{pgfscope}%
\begin{pgfscope}%
\pgfsetbuttcap%
\pgfsetroundjoin%
\definecolor{currentfill}{rgb}{0.000000,0.000000,0.000000}%
\pgfsetfillcolor{currentfill}%
\pgfsetlinewidth{0.501875pt}%
\definecolor{currentstroke}{rgb}{0.000000,0.000000,0.000000}%
\pgfsetstrokecolor{currentstroke}%
\pgfsetdash{}{0pt}%
\pgfsys@defobject{currentmarker}{\pgfqpoint{-0.020833in}{0.000000in}}{\pgfqpoint{-0.000000in}{0.000000in}}{%
\pgfpathmoveto{\pgfqpoint{-0.000000in}{0.000000in}}%
\pgfpathlineto{\pgfqpoint{-0.020833in}{0.000000in}}%
\pgfusepath{stroke,fill}%
}%
\begin{pgfscope}%
\pgfsys@transformshift{2.816034in}{1.245853in}%
\pgfsys@useobject{currentmarker}{}%
\end{pgfscope}%
\end{pgfscope}%
\begin{pgfscope}%
\pgfsetbuttcap%
\pgfsetroundjoin%
\definecolor{currentfill}{rgb}{0.000000,0.000000,0.000000}%
\pgfsetfillcolor{currentfill}%
\pgfsetlinewidth{0.501875pt}%
\definecolor{currentstroke}{rgb}{0.000000,0.000000,0.000000}%
\pgfsetstrokecolor{currentstroke}%
\pgfsetdash{}{0pt}%
\pgfsys@defobject{currentmarker}{\pgfqpoint{0.000000in}{0.000000in}}{\pgfqpoint{0.020833in}{0.000000in}}{%
\pgfpathmoveto{\pgfqpoint{0.000000in}{0.000000in}}%
\pgfpathlineto{\pgfqpoint{0.020833in}{0.000000in}}%
\pgfusepath{stroke,fill}%
}%
\begin{pgfscope}%
\pgfsys@transformshift{0.539970in}{1.351699in}%
\pgfsys@useobject{currentmarker}{}%
\end{pgfscope}%
\end{pgfscope}%
\begin{pgfscope}%
\pgfsetbuttcap%
\pgfsetroundjoin%
\definecolor{currentfill}{rgb}{0.000000,0.000000,0.000000}%
\pgfsetfillcolor{currentfill}%
\pgfsetlinewidth{0.501875pt}%
\definecolor{currentstroke}{rgb}{0.000000,0.000000,0.000000}%
\pgfsetstrokecolor{currentstroke}%
\pgfsetdash{}{0pt}%
\pgfsys@defobject{currentmarker}{\pgfqpoint{-0.020833in}{0.000000in}}{\pgfqpoint{-0.000000in}{0.000000in}}{%
\pgfpathmoveto{\pgfqpoint{-0.000000in}{0.000000in}}%
\pgfpathlineto{\pgfqpoint{-0.020833in}{0.000000in}}%
\pgfusepath{stroke,fill}%
}%
\begin{pgfscope}%
\pgfsys@transformshift{2.816034in}{1.351699in}%
\pgfsys@useobject{currentmarker}{}%
\end{pgfscope}%
\end{pgfscope}%
\begin{pgfscope}%
\pgfsetbuttcap%
\pgfsetroundjoin%
\definecolor{currentfill}{rgb}{0.000000,0.000000,0.000000}%
\pgfsetfillcolor{currentfill}%
\pgfsetlinewidth{0.501875pt}%
\definecolor{currentstroke}{rgb}{0.000000,0.000000,0.000000}%
\pgfsetstrokecolor{currentstroke}%
\pgfsetdash{}{0pt}%
\pgfsys@defobject{currentmarker}{\pgfqpoint{0.000000in}{0.000000in}}{\pgfqpoint{0.020833in}{0.000000in}}{%
\pgfpathmoveto{\pgfqpoint{0.000000in}{0.000000in}}%
\pgfpathlineto{\pgfqpoint{0.020833in}{0.000000in}}%
\pgfusepath{stroke,fill}%
}%
\begin{pgfscope}%
\pgfsys@transformshift{0.539970in}{1.563392in}%
\pgfsys@useobject{currentmarker}{}%
\end{pgfscope}%
\end{pgfscope}%
\begin{pgfscope}%
\pgfsetbuttcap%
\pgfsetroundjoin%
\definecolor{currentfill}{rgb}{0.000000,0.000000,0.000000}%
\pgfsetfillcolor{currentfill}%
\pgfsetlinewidth{0.501875pt}%
\definecolor{currentstroke}{rgb}{0.000000,0.000000,0.000000}%
\pgfsetstrokecolor{currentstroke}%
\pgfsetdash{}{0pt}%
\pgfsys@defobject{currentmarker}{\pgfqpoint{-0.020833in}{0.000000in}}{\pgfqpoint{-0.000000in}{0.000000in}}{%
\pgfpathmoveto{\pgfqpoint{-0.000000in}{0.000000in}}%
\pgfpathlineto{\pgfqpoint{-0.020833in}{0.000000in}}%
\pgfusepath{stroke,fill}%
}%
\begin{pgfscope}%
\pgfsys@transformshift{2.816034in}{1.563392in}%
\pgfsys@useobject{currentmarker}{}%
\end{pgfscope}%
\end{pgfscope}%
\begin{pgfscope}%
\pgfsetbuttcap%
\pgfsetroundjoin%
\definecolor{currentfill}{rgb}{0.000000,0.000000,0.000000}%
\pgfsetfillcolor{currentfill}%
\pgfsetlinewidth{0.501875pt}%
\definecolor{currentstroke}{rgb}{0.000000,0.000000,0.000000}%
\pgfsetstrokecolor{currentstroke}%
\pgfsetdash{}{0pt}%
\pgfsys@defobject{currentmarker}{\pgfqpoint{0.000000in}{0.000000in}}{\pgfqpoint{0.020833in}{0.000000in}}{%
\pgfpathmoveto{\pgfqpoint{0.000000in}{0.000000in}}%
\pgfpathlineto{\pgfqpoint{0.020833in}{0.000000in}}%
\pgfusepath{stroke,fill}%
}%
\begin{pgfscope}%
\pgfsys@transformshift{0.539970in}{1.669238in}%
\pgfsys@useobject{currentmarker}{}%
\end{pgfscope}%
\end{pgfscope}%
\begin{pgfscope}%
\pgfsetbuttcap%
\pgfsetroundjoin%
\definecolor{currentfill}{rgb}{0.000000,0.000000,0.000000}%
\pgfsetfillcolor{currentfill}%
\pgfsetlinewidth{0.501875pt}%
\definecolor{currentstroke}{rgb}{0.000000,0.000000,0.000000}%
\pgfsetstrokecolor{currentstroke}%
\pgfsetdash{}{0pt}%
\pgfsys@defobject{currentmarker}{\pgfqpoint{-0.020833in}{0.000000in}}{\pgfqpoint{-0.000000in}{0.000000in}}{%
\pgfpathmoveto{\pgfqpoint{-0.000000in}{0.000000in}}%
\pgfpathlineto{\pgfqpoint{-0.020833in}{0.000000in}}%
\pgfusepath{stroke,fill}%
}%
\begin{pgfscope}%
\pgfsys@transformshift{2.816034in}{1.669238in}%
\pgfsys@useobject{currentmarker}{}%
\end{pgfscope}%
\end{pgfscope}%
\begin{pgfscope}%
\pgfsetbuttcap%
\pgfsetroundjoin%
\definecolor{currentfill}{rgb}{0.000000,0.000000,0.000000}%
\pgfsetfillcolor{currentfill}%
\pgfsetlinewidth{0.501875pt}%
\definecolor{currentstroke}{rgb}{0.000000,0.000000,0.000000}%
\pgfsetstrokecolor{currentstroke}%
\pgfsetdash{}{0pt}%
\pgfsys@defobject{currentmarker}{\pgfqpoint{0.000000in}{0.000000in}}{\pgfqpoint{0.020833in}{0.000000in}}{%
\pgfpathmoveto{\pgfqpoint{0.000000in}{0.000000in}}%
\pgfpathlineto{\pgfqpoint{0.020833in}{0.000000in}}%
\pgfusepath{stroke,fill}%
}%
\begin{pgfscope}%
\pgfsys@transformshift{0.539970in}{1.775084in}%
\pgfsys@useobject{currentmarker}{}%
\end{pgfscope}%
\end{pgfscope}%
\begin{pgfscope}%
\pgfsetbuttcap%
\pgfsetroundjoin%
\definecolor{currentfill}{rgb}{0.000000,0.000000,0.000000}%
\pgfsetfillcolor{currentfill}%
\pgfsetlinewidth{0.501875pt}%
\definecolor{currentstroke}{rgb}{0.000000,0.000000,0.000000}%
\pgfsetstrokecolor{currentstroke}%
\pgfsetdash{}{0pt}%
\pgfsys@defobject{currentmarker}{\pgfqpoint{-0.020833in}{0.000000in}}{\pgfqpoint{-0.000000in}{0.000000in}}{%
\pgfpathmoveto{\pgfqpoint{-0.000000in}{0.000000in}}%
\pgfpathlineto{\pgfqpoint{-0.020833in}{0.000000in}}%
\pgfusepath{stroke,fill}%
}%
\begin{pgfscope}%
\pgfsys@transformshift{2.816034in}{1.775084in}%
\pgfsys@useobject{currentmarker}{}%
\end{pgfscope}%
\end{pgfscope}%
\begin{pgfscope}%
\pgfsetbuttcap%
\pgfsetroundjoin%
\definecolor{currentfill}{rgb}{0.000000,0.000000,0.000000}%
\pgfsetfillcolor{currentfill}%
\pgfsetlinewidth{0.501875pt}%
\definecolor{currentstroke}{rgb}{0.000000,0.000000,0.000000}%
\pgfsetstrokecolor{currentstroke}%
\pgfsetdash{}{0pt}%
\pgfsys@defobject{currentmarker}{\pgfqpoint{0.000000in}{0.000000in}}{\pgfqpoint{0.020833in}{0.000000in}}{%
\pgfpathmoveto{\pgfqpoint{0.000000in}{0.000000in}}%
\pgfpathlineto{\pgfqpoint{0.020833in}{0.000000in}}%
\pgfusepath{stroke,fill}%
}%
\begin{pgfscope}%
\pgfsys@transformshift{0.539970in}{1.880931in}%
\pgfsys@useobject{currentmarker}{}%
\end{pgfscope}%
\end{pgfscope}%
\begin{pgfscope}%
\pgfsetbuttcap%
\pgfsetroundjoin%
\definecolor{currentfill}{rgb}{0.000000,0.000000,0.000000}%
\pgfsetfillcolor{currentfill}%
\pgfsetlinewidth{0.501875pt}%
\definecolor{currentstroke}{rgb}{0.000000,0.000000,0.000000}%
\pgfsetstrokecolor{currentstroke}%
\pgfsetdash{}{0pt}%
\pgfsys@defobject{currentmarker}{\pgfqpoint{-0.020833in}{0.000000in}}{\pgfqpoint{-0.000000in}{0.000000in}}{%
\pgfpathmoveto{\pgfqpoint{-0.000000in}{0.000000in}}%
\pgfpathlineto{\pgfqpoint{-0.020833in}{0.000000in}}%
\pgfusepath{stroke,fill}%
}%
\begin{pgfscope}%
\pgfsys@transformshift{2.816034in}{1.880931in}%
\pgfsys@useobject{currentmarker}{}%
\end{pgfscope}%
\end{pgfscope}%
\begin{pgfscope}%
\definecolor{textcolor}{rgb}{0.000000,0.000000,0.000000}%
\pgfsetstrokecolor{textcolor}%
\pgfsetfillcolor{textcolor}%
\pgftext[x=0.258333in,y=1.236093in,,bottom,rotate=90.000000]{\color{textcolor}\rmfamily\fontsize{10.000000}{12.000000}\selectfont \(\displaystyle C(K)\)}%
\end{pgfscope}%
\begin{pgfscope}%
\pgfpathrectangle{\pgfqpoint{0.539970in}{0.422992in}}{\pgfqpoint{2.276064in}{1.626201in}}%
\pgfusepath{clip}%
\pgfsetrectcap%
\pgfsetroundjoin%
\pgfsetlinewidth{1.003750pt}%
\definecolor{currentstroke}{rgb}{0.047059,0.364706,0.647059}%
\pgfsetstrokecolor{currentstroke}%
\pgfsetdash{}{0pt}%
\pgfpathmoveto{\pgfqpoint{0.562505in}{1.975275in}}%
\pgfpathlineto{\pgfqpoint{0.585040in}{1.973396in}}%
\pgfpathlineto{\pgfqpoint{0.607576in}{1.968745in}}%
\pgfpathlineto{\pgfqpoint{0.630111in}{1.965199in}}%
\pgfpathlineto{\pgfqpoint{0.652646in}{1.958689in}}%
\pgfpathlineto{\pgfqpoint{0.675182in}{1.954736in}}%
\pgfpathlineto{\pgfqpoint{0.697717in}{1.951220in}}%
\pgfpathlineto{\pgfqpoint{0.720252in}{1.948054in}}%
\pgfpathlineto{\pgfqpoint{0.742787in}{1.945333in}}%
\pgfpathlineto{\pgfqpoint{0.765323in}{1.942161in}}%
\pgfpathlineto{\pgfqpoint{0.787858in}{1.939127in}}%
\pgfpathlineto{\pgfqpoint{0.810393in}{1.936533in}}%
\pgfpathlineto{\pgfqpoint{0.832929in}{1.934465in}}%
\pgfpathlineto{\pgfqpoint{0.855464in}{1.933277in}}%
\pgfpathlineto{\pgfqpoint{0.877999in}{1.930735in}}%
\pgfpathlineto{\pgfqpoint{0.900534in}{1.928315in}}%
\pgfpathlineto{\pgfqpoint{0.923070in}{1.927191in}}%
\pgfpathlineto{\pgfqpoint{0.945605in}{1.926073in}}%
\pgfpathlineto{\pgfqpoint{0.968140in}{1.925098in}}%
\pgfpathlineto{\pgfqpoint{0.990676in}{1.924104in}}%
\pgfpathlineto{\pgfqpoint{1.013211in}{1.922474in}}%
\pgfpathlineto{\pgfqpoint{1.035746in}{1.921575in}}%
\pgfpathlineto{\pgfqpoint{1.058281in}{1.920680in}}%
\pgfpathlineto{\pgfqpoint{1.080817in}{1.919950in}}%
\pgfpathlineto{\pgfqpoint{1.103352in}{1.919664in}}%
\pgfpathlineto{\pgfqpoint{1.125887in}{1.918760in}}%
\pgfpathlineto{\pgfqpoint{1.148423in}{1.918509in}}%
\pgfpathlineto{\pgfqpoint{1.170958in}{1.918173in}}%
\pgfpathlineto{\pgfqpoint{1.193493in}{1.917209in}}%
\pgfpathlineto{\pgfqpoint{1.216028in}{1.916626in}}%
\pgfpathlineto{\pgfqpoint{1.238564in}{1.916458in}}%
\pgfpathlineto{\pgfqpoint{1.261099in}{1.916580in}}%
\pgfpathlineto{\pgfqpoint{1.283634in}{1.916700in}}%
\pgfpathlineto{\pgfqpoint{1.306170in}{1.915909in}}%
\pgfpathlineto{\pgfqpoint{1.328705in}{1.915551in}}%
\pgfpathlineto{\pgfqpoint{1.351240in}{1.915107in}}%
\pgfpathlineto{\pgfqpoint{1.373776in}{1.914736in}}%
\pgfpathlineto{\pgfqpoint{1.396311in}{1.913670in}}%
\pgfpathlineto{\pgfqpoint{1.418846in}{1.913434in}}%
\pgfpathlineto{\pgfqpoint{1.441381in}{1.913032in}}%
\pgfpathlineto{\pgfqpoint{1.463917in}{1.912731in}}%
\pgfpathlineto{\pgfqpoint{1.486452in}{1.912201in}}%
\pgfpathlineto{\pgfqpoint{1.508987in}{1.911803in}}%
\pgfpathlineto{\pgfqpoint{1.531523in}{1.911357in}}%
\pgfpathlineto{\pgfqpoint{1.554058in}{1.912230in}}%
\pgfpathlineto{\pgfqpoint{1.576593in}{1.911783in}}%
\pgfpathlineto{\pgfqpoint{1.599128in}{1.911719in}}%
\pgfpathlineto{\pgfqpoint{1.621664in}{1.911769in}}%
\pgfpathlineto{\pgfqpoint{1.644199in}{1.911571in}}%
\pgfpathlineto{\pgfqpoint{1.666734in}{1.911791in}}%
\pgfpathlineto{\pgfqpoint{1.689270in}{1.911019in}}%
\pgfpathlineto{\pgfqpoint{1.711805in}{1.911144in}}%
\pgfpathlineto{\pgfqpoint{1.734340in}{1.910937in}}%
\pgfpathlineto{\pgfqpoint{1.756875in}{1.910559in}}%
\pgfpathlineto{\pgfqpoint{1.779411in}{1.910111in}}%
\pgfpathlineto{\pgfqpoint{1.801946in}{1.910303in}}%
\pgfpathlineto{\pgfqpoint{1.824481in}{1.910227in}}%
\pgfpathlineto{\pgfqpoint{1.847017in}{1.909839in}}%
\pgfpathlineto{\pgfqpoint{1.869552in}{1.909703in}}%
\pgfpathlineto{\pgfqpoint{1.892087in}{1.909325in}}%
\pgfpathlineto{\pgfqpoint{1.914622in}{1.909387in}}%
\pgfpathlineto{\pgfqpoint{1.937158in}{1.909392in}}%
\pgfpathlineto{\pgfqpoint{1.959693in}{1.909115in}}%
\pgfpathlineto{\pgfqpoint{1.982228in}{1.909514in}}%
\pgfpathlineto{\pgfqpoint{2.004764in}{1.909240in}}%
\pgfpathlineto{\pgfqpoint{2.027299in}{1.909265in}}%
\pgfpathlineto{\pgfqpoint{2.049834in}{1.909656in}}%
\pgfpathlineto{\pgfqpoint{2.072369in}{1.909522in}}%
\pgfpathlineto{\pgfqpoint{2.094905in}{1.908949in}}%
\pgfpathlineto{\pgfqpoint{2.117440in}{1.908978in}}%
\pgfpathlineto{\pgfqpoint{2.139975in}{1.909344in}}%
\pgfpathlineto{\pgfqpoint{2.162511in}{1.909151in}}%
\pgfpathlineto{\pgfqpoint{2.185046in}{1.908676in}}%
\pgfpathlineto{\pgfqpoint{2.207581in}{1.908500in}}%
\pgfpathlineto{\pgfqpoint{2.230116in}{1.908556in}}%
\pgfpathlineto{\pgfqpoint{2.252652in}{1.908664in}}%
\pgfpathlineto{\pgfqpoint{2.275187in}{1.908225in}}%
\pgfpathlineto{\pgfqpoint{2.297722in}{1.908427in}}%
\pgfpathlineto{\pgfqpoint{2.320258in}{1.908750in}}%
\pgfpathlineto{\pgfqpoint{2.342793in}{1.908737in}}%
\pgfpathlineto{\pgfqpoint{2.365328in}{1.908684in}}%
\pgfpathlineto{\pgfqpoint{2.387863in}{1.908454in}}%
\pgfpathlineto{\pgfqpoint{2.410399in}{1.908234in}}%
\pgfpathlineto{\pgfqpoint{2.432934in}{1.908457in}}%
\pgfpathlineto{\pgfqpoint{2.455469in}{1.908167in}}%
\pgfpathlineto{\pgfqpoint{2.478005in}{1.908256in}}%
\pgfpathlineto{\pgfqpoint{2.500540in}{1.907838in}}%
\pgfpathlineto{\pgfqpoint{2.523075in}{1.907921in}}%
\pgfpathlineto{\pgfqpoint{2.545610in}{1.907703in}}%
\pgfpathlineto{\pgfqpoint{2.568146in}{1.907159in}}%
\pgfpathlineto{\pgfqpoint{2.590681in}{1.907230in}}%
\pgfpathlineto{\pgfqpoint{2.613216in}{1.907036in}}%
\pgfpathlineto{\pgfqpoint{2.635752in}{1.907100in}}%
\pgfpathlineto{\pgfqpoint{2.658287in}{1.907261in}}%
\pgfpathlineto{\pgfqpoint{2.680822in}{1.907165in}}%
\pgfpathlineto{\pgfqpoint{2.703357in}{1.907095in}}%
\pgfpathlineto{\pgfqpoint{2.725893in}{1.907252in}}%
\pgfpathlineto{\pgfqpoint{2.748428in}{1.907280in}}%
\pgfpathlineto{\pgfqpoint{2.770963in}{1.907537in}}%
\pgfusepath{stroke}%
\end{pgfscope}%
\begin{pgfscope}%
\pgfpathrectangle{\pgfqpoint{0.539970in}{0.422992in}}{\pgfqpoint{2.276064in}{1.626201in}}%
\pgfusepath{clip}%
\pgfsetrectcap%
\pgfsetroundjoin%
\pgfsetlinewidth{1.003750pt}%
\definecolor{currentstroke}{rgb}{0.000000,0.725490,0.270588}%
\pgfsetstrokecolor{currentstroke}%
\pgfsetdash{}{0pt}%
\pgfpathmoveto{\pgfqpoint{0.562505in}{1.967961in}}%
\pgfpathlineto{\pgfqpoint{0.585040in}{1.951573in}}%
\pgfpathlineto{\pgfqpoint{0.607576in}{1.940676in}}%
\pgfpathlineto{\pgfqpoint{0.630111in}{1.926465in}}%
\pgfpathlineto{\pgfqpoint{0.652646in}{1.905759in}}%
\pgfpathlineto{\pgfqpoint{0.675182in}{1.889333in}}%
\pgfpathlineto{\pgfqpoint{0.697717in}{1.869391in}}%
\pgfpathlineto{\pgfqpoint{0.720252in}{1.855476in}}%
\pgfpathlineto{\pgfqpoint{0.742787in}{1.841269in}}%
\pgfpathlineto{\pgfqpoint{0.765323in}{1.824398in}}%
\pgfpathlineto{\pgfqpoint{0.787858in}{1.810603in}}%
\pgfpathlineto{\pgfqpoint{0.810393in}{1.797552in}}%
\pgfpathlineto{\pgfqpoint{0.832929in}{1.788109in}}%
\pgfpathlineto{\pgfqpoint{0.855464in}{1.777544in}}%
\pgfpathlineto{\pgfqpoint{0.877999in}{1.766411in}}%
\pgfpathlineto{\pgfqpoint{0.900534in}{1.755210in}}%
\pgfpathlineto{\pgfqpoint{0.923070in}{1.748735in}}%
\pgfpathlineto{\pgfqpoint{0.945605in}{1.742608in}}%
\pgfpathlineto{\pgfqpoint{0.968140in}{1.736079in}}%
\pgfpathlineto{\pgfqpoint{0.990676in}{1.728062in}}%
\pgfpathlineto{\pgfqpoint{1.013211in}{1.719446in}}%
\pgfpathlineto{\pgfqpoint{1.035746in}{1.711520in}}%
\pgfpathlineto{\pgfqpoint{1.058281in}{1.705418in}}%
\pgfpathlineto{\pgfqpoint{1.080817in}{1.698942in}}%
\pgfpathlineto{\pgfqpoint{1.103352in}{1.694416in}}%
\pgfpathlineto{\pgfqpoint{1.125887in}{1.688612in}}%
\pgfpathlineto{\pgfqpoint{1.148423in}{1.684370in}}%
\pgfpathlineto{\pgfqpoint{1.170958in}{1.679790in}}%
\pgfpathlineto{\pgfqpoint{1.193493in}{1.674080in}}%
\pgfpathlineto{\pgfqpoint{1.216028in}{1.669731in}}%
\pgfpathlineto{\pgfqpoint{1.238564in}{1.664809in}}%
\pgfpathlineto{\pgfqpoint{1.261099in}{1.662693in}}%
\pgfpathlineto{\pgfqpoint{1.283634in}{1.659849in}}%
\pgfpathlineto{\pgfqpoint{1.306170in}{1.656677in}}%
\pgfpathlineto{\pgfqpoint{1.328705in}{1.653096in}}%
\pgfpathlineto{\pgfqpoint{1.351240in}{1.649693in}}%
\pgfpathlineto{\pgfqpoint{1.373776in}{1.646724in}}%
\pgfpathlineto{\pgfqpoint{1.396311in}{1.641645in}}%
\pgfpathlineto{\pgfqpoint{1.418846in}{1.638399in}}%
\pgfpathlineto{\pgfqpoint{1.441381in}{1.635418in}}%
\pgfpathlineto{\pgfqpoint{1.463917in}{1.633380in}}%
\pgfpathlineto{\pgfqpoint{1.486452in}{1.629208in}}%
\pgfpathlineto{\pgfqpoint{1.508987in}{1.624689in}}%
\pgfpathlineto{\pgfqpoint{1.531523in}{1.620884in}}%
\pgfpathlineto{\pgfqpoint{1.554058in}{1.619822in}}%
\pgfpathlineto{\pgfqpoint{1.576593in}{1.616094in}}%
\pgfpathlineto{\pgfqpoint{1.599128in}{1.614008in}}%
\pgfpathlineto{\pgfqpoint{1.621664in}{1.611877in}}%
\pgfpathlineto{\pgfqpoint{1.644199in}{1.608982in}}%
\pgfpathlineto{\pgfqpoint{1.666734in}{1.607898in}}%
\pgfpathlineto{\pgfqpoint{1.689270in}{1.602962in}}%
\pgfpathlineto{\pgfqpoint{1.711805in}{1.600581in}}%
\pgfpathlineto{\pgfqpoint{1.734340in}{1.597882in}}%
\pgfpathlineto{\pgfqpoint{1.756875in}{1.594455in}}%
\pgfpathlineto{\pgfqpoint{1.779411in}{1.590768in}}%
\pgfpathlineto{\pgfqpoint{1.801946in}{1.588256in}}%
\pgfpathlineto{\pgfqpoint{1.824481in}{1.586071in}}%
\pgfpathlineto{\pgfqpoint{1.847017in}{1.583122in}}%
\pgfpathlineto{\pgfqpoint{1.869552in}{1.581109in}}%
\pgfpathlineto{\pgfqpoint{1.892087in}{1.577886in}}%
\pgfpathlineto{\pgfqpoint{1.914622in}{1.575218in}}%
\pgfpathlineto{\pgfqpoint{1.937158in}{1.572122in}}%
\pgfpathlineto{\pgfqpoint{1.959693in}{1.568707in}}%
\pgfpathlineto{\pgfqpoint{1.982228in}{1.566802in}}%
\pgfpathlineto{\pgfqpoint{2.004764in}{1.564546in}}%
\pgfpathlineto{\pgfqpoint{2.027299in}{1.563001in}}%
\pgfpathlineto{\pgfqpoint{2.049834in}{1.561159in}}%
\pgfpathlineto{\pgfqpoint{2.072369in}{1.558412in}}%
\pgfpathlineto{\pgfqpoint{2.094905in}{1.554436in}}%
\pgfpathlineto{\pgfqpoint{2.117440in}{1.551484in}}%
\pgfpathlineto{\pgfqpoint{2.139975in}{1.550076in}}%
\pgfpathlineto{\pgfqpoint{2.162511in}{1.547212in}}%
\pgfpathlineto{\pgfqpoint{2.185046in}{1.542716in}}%
\pgfpathlineto{\pgfqpoint{2.207581in}{1.538646in}}%
\pgfpathlineto{\pgfqpoint{2.230116in}{1.536083in}}%
\pgfpathlineto{\pgfqpoint{2.252652in}{1.534097in}}%
\pgfpathlineto{\pgfqpoint{2.275187in}{1.530872in}}%
\pgfpathlineto{\pgfqpoint{2.297722in}{1.528695in}}%
\pgfpathlineto{\pgfqpoint{2.320258in}{1.525973in}}%
\pgfpathlineto{\pgfqpoint{2.342793in}{1.522933in}}%
\pgfpathlineto{\pgfqpoint{2.365328in}{1.519654in}}%
\pgfpathlineto{\pgfqpoint{2.387863in}{1.516310in}}%
\pgfpathlineto{\pgfqpoint{2.410399in}{1.514668in}}%
\pgfpathlineto{\pgfqpoint{2.432934in}{1.513220in}}%
\pgfpathlineto{\pgfqpoint{2.455469in}{1.510417in}}%
\pgfpathlineto{\pgfqpoint{2.478005in}{1.507469in}}%
\pgfpathlineto{\pgfqpoint{2.500540in}{1.503418in}}%
\pgfpathlineto{\pgfqpoint{2.523075in}{1.502043in}}%
\pgfpathlineto{\pgfqpoint{2.545610in}{1.499697in}}%
\pgfpathlineto{\pgfqpoint{2.568146in}{1.496992in}}%
\pgfpathlineto{\pgfqpoint{2.590681in}{1.493532in}}%
\pgfpathlineto{\pgfqpoint{2.613216in}{1.491904in}}%
\pgfpathlineto{\pgfqpoint{2.635752in}{1.489704in}}%
\pgfpathlineto{\pgfqpoint{2.658287in}{1.487352in}}%
\pgfpathlineto{\pgfqpoint{2.680822in}{1.485699in}}%
\pgfpathlineto{\pgfqpoint{2.703357in}{1.482904in}}%
\pgfpathlineto{\pgfqpoint{2.725893in}{1.481436in}}%
\pgfpathlineto{\pgfqpoint{2.748428in}{1.478867in}}%
\pgfpathlineto{\pgfqpoint{2.770963in}{1.476855in}}%
\pgfusepath{stroke}%
\end{pgfscope}%
\begin{pgfscope}%
\pgfpathrectangle{\pgfqpoint{0.539970in}{0.422992in}}{\pgfqpoint{2.276064in}{1.626201in}}%
\pgfusepath{clip}%
\pgfsetrectcap%
\pgfsetroundjoin%
\pgfsetlinewidth{1.003750pt}%
\definecolor{currentstroke}{rgb}{1.000000,0.584314,0.000000}%
\pgfsetstrokecolor{currentstroke}%
\pgfsetdash{}{0pt}%
\pgfpathmoveto{\pgfqpoint{0.562505in}{1.931926in}}%
\pgfpathlineto{\pgfqpoint{0.585040in}{1.896648in}}%
\pgfpathlineto{\pgfqpoint{0.607576in}{1.872054in}}%
\pgfpathlineto{\pgfqpoint{0.630111in}{1.854444in}}%
\pgfpathlineto{\pgfqpoint{0.652646in}{1.835001in}}%
\pgfpathlineto{\pgfqpoint{0.675182in}{1.815279in}}%
\pgfpathlineto{\pgfqpoint{0.697717in}{1.798742in}}%
\pgfpathlineto{\pgfqpoint{0.720252in}{1.780398in}}%
\pgfpathlineto{\pgfqpoint{0.742787in}{1.772578in}}%
\pgfpathlineto{\pgfqpoint{0.765323in}{1.764627in}}%
\pgfpathlineto{\pgfqpoint{0.787858in}{1.757093in}}%
\pgfpathlineto{\pgfqpoint{0.810393in}{1.752108in}}%
\pgfpathlineto{\pgfqpoint{0.832929in}{1.749525in}}%
\pgfpathlineto{\pgfqpoint{0.855464in}{1.744920in}}%
\pgfpathlineto{\pgfqpoint{0.877999in}{1.738768in}}%
\pgfpathlineto{\pgfqpoint{0.900534in}{1.733628in}}%
\pgfpathlineto{\pgfqpoint{0.923070in}{1.730125in}}%
\pgfpathlineto{\pgfqpoint{0.945605in}{1.726435in}}%
\pgfpathlineto{\pgfqpoint{0.968140in}{1.721666in}}%
\pgfpathlineto{\pgfqpoint{0.990676in}{1.716276in}}%
\pgfpathlineto{\pgfqpoint{1.013211in}{1.714195in}}%
\pgfpathlineto{\pgfqpoint{1.035746in}{1.710950in}}%
\pgfpathlineto{\pgfqpoint{1.058281in}{1.706453in}}%
\pgfpathlineto{\pgfqpoint{1.080817in}{1.701645in}}%
\pgfpathlineto{\pgfqpoint{1.103352in}{1.698803in}}%
\pgfpathlineto{\pgfqpoint{1.125887in}{1.696945in}}%
\pgfpathlineto{\pgfqpoint{1.148423in}{1.696169in}}%
\pgfpathlineto{\pgfqpoint{1.170958in}{1.694698in}}%
\pgfpathlineto{\pgfqpoint{1.193493in}{1.692979in}}%
\pgfpathlineto{\pgfqpoint{1.216028in}{1.690909in}}%
\pgfpathlineto{\pgfqpoint{1.238564in}{1.689127in}}%
\pgfpathlineto{\pgfqpoint{1.261099in}{1.686312in}}%
\pgfpathlineto{\pgfqpoint{1.283634in}{1.683964in}}%
\pgfpathlineto{\pgfqpoint{1.306170in}{1.682200in}}%
\pgfpathlineto{\pgfqpoint{1.328705in}{1.682051in}}%
\pgfpathlineto{\pgfqpoint{1.351240in}{1.679699in}}%
\pgfpathlineto{\pgfqpoint{1.373776in}{1.678177in}}%
\pgfpathlineto{\pgfqpoint{1.396311in}{1.676282in}}%
\pgfpathlineto{\pgfqpoint{1.418846in}{1.674413in}}%
\pgfpathlineto{\pgfqpoint{1.441381in}{1.672477in}}%
\pgfpathlineto{\pgfqpoint{1.463917in}{1.669183in}}%
\pgfpathlineto{\pgfqpoint{1.486452in}{1.666560in}}%
\pgfpathlineto{\pgfqpoint{1.508987in}{1.664681in}}%
\pgfpathlineto{\pgfqpoint{1.531523in}{1.662388in}}%
\pgfpathlineto{\pgfqpoint{1.554058in}{1.662443in}}%
\pgfpathlineto{\pgfqpoint{1.576593in}{1.661542in}}%
\pgfpathlineto{\pgfqpoint{1.599128in}{1.659382in}}%
\pgfpathlineto{\pgfqpoint{1.621664in}{1.657457in}}%
\pgfpathlineto{\pgfqpoint{1.644199in}{1.655755in}}%
\pgfpathlineto{\pgfqpoint{1.666734in}{1.655170in}}%
\pgfpathlineto{\pgfqpoint{1.689270in}{1.653988in}}%
\pgfpathlineto{\pgfqpoint{1.711805in}{1.652550in}}%
\pgfpathlineto{\pgfqpoint{1.734340in}{1.651705in}}%
\pgfpathlineto{\pgfqpoint{1.756875in}{1.651100in}}%
\pgfpathlineto{\pgfqpoint{1.779411in}{1.649977in}}%
\pgfpathlineto{\pgfqpoint{1.801946in}{1.649608in}}%
\pgfpathlineto{\pgfqpoint{1.824481in}{1.650301in}}%
\pgfpathlineto{\pgfqpoint{1.847017in}{1.648327in}}%
\pgfpathlineto{\pgfqpoint{1.869552in}{1.648673in}}%
\pgfpathlineto{\pgfqpoint{1.892087in}{1.646976in}}%
\pgfpathlineto{\pgfqpoint{1.914622in}{1.645379in}}%
\pgfpathlineto{\pgfqpoint{1.937158in}{1.645184in}}%
\pgfpathlineto{\pgfqpoint{1.959693in}{1.644481in}}%
\pgfpathlineto{\pgfqpoint{1.982228in}{1.644571in}}%
\pgfpathlineto{\pgfqpoint{2.004764in}{1.643831in}}%
\pgfpathlineto{\pgfqpoint{2.027299in}{1.642965in}}%
\pgfpathlineto{\pgfqpoint{2.049834in}{1.642270in}}%
\pgfpathlineto{\pgfqpoint{2.072369in}{1.641327in}}%
\pgfpathlineto{\pgfqpoint{2.094905in}{1.640369in}}%
\pgfpathlineto{\pgfqpoint{2.117440in}{1.639631in}}%
\pgfpathlineto{\pgfqpoint{2.139975in}{1.638322in}}%
\pgfpathlineto{\pgfqpoint{2.162511in}{1.637673in}}%
\pgfpathlineto{\pgfqpoint{2.185046in}{1.635793in}}%
\pgfpathlineto{\pgfqpoint{2.207581in}{1.635605in}}%
\pgfpathlineto{\pgfqpoint{2.230116in}{1.635469in}}%
\pgfpathlineto{\pgfqpoint{2.252652in}{1.635004in}}%
\pgfpathlineto{\pgfqpoint{2.275187in}{1.633216in}}%
\pgfpathlineto{\pgfqpoint{2.297722in}{1.631575in}}%
\pgfpathlineto{\pgfqpoint{2.320258in}{1.630300in}}%
\pgfpathlineto{\pgfqpoint{2.342793in}{1.628622in}}%
\pgfpathlineto{\pgfqpoint{2.365328in}{1.627522in}}%
\pgfpathlineto{\pgfqpoint{2.387863in}{1.626986in}}%
\pgfpathlineto{\pgfqpoint{2.410399in}{1.626512in}}%
\pgfpathlineto{\pgfqpoint{2.432934in}{1.625978in}}%
\pgfpathlineto{\pgfqpoint{2.455469in}{1.624986in}}%
\pgfpathlineto{\pgfqpoint{2.478005in}{1.623896in}}%
\pgfpathlineto{\pgfqpoint{2.500540in}{1.622443in}}%
\pgfpathlineto{\pgfqpoint{2.523075in}{1.622142in}}%
\pgfpathlineto{\pgfqpoint{2.545610in}{1.620324in}}%
\pgfpathlineto{\pgfqpoint{2.568146in}{1.618534in}}%
\pgfpathlineto{\pgfqpoint{2.590681in}{1.616947in}}%
\pgfpathlineto{\pgfqpoint{2.613216in}{1.616230in}}%
\pgfpathlineto{\pgfqpoint{2.635752in}{1.616399in}}%
\pgfpathlineto{\pgfqpoint{2.658287in}{1.615472in}}%
\pgfpathlineto{\pgfqpoint{2.680822in}{1.615493in}}%
\pgfpathlineto{\pgfqpoint{2.703357in}{1.615092in}}%
\pgfpathlineto{\pgfqpoint{2.725893in}{1.615844in}}%
\pgfpathlineto{\pgfqpoint{2.748428in}{1.615873in}}%
\pgfpathlineto{\pgfqpoint{2.770963in}{1.615779in}}%
\pgfusepath{stroke}%
\end{pgfscope}%
\begin{pgfscope}%
\pgfpathrectangle{\pgfqpoint{0.539970in}{0.422992in}}{\pgfqpoint{2.276064in}{1.626201in}}%
\pgfusepath{clip}%
\pgfsetrectcap%
\pgfsetroundjoin%
\pgfsetlinewidth{1.003750pt}%
\definecolor{currentstroke}{rgb}{1.000000,0.172549,0.000000}%
\pgfsetstrokecolor{currentstroke}%
\pgfsetdash{}{0pt}%
\pgfpathmoveto{\pgfqpoint{0.562505in}{1.942264in}}%
\pgfpathlineto{\pgfqpoint{0.585040in}{1.876076in}}%
\pgfpathlineto{\pgfqpoint{0.607576in}{1.804616in}}%
\pgfpathlineto{\pgfqpoint{0.630111in}{1.756953in}}%
\pgfpathlineto{\pgfqpoint{0.652646in}{1.686040in}}%
\pgfpathlineto{\pgfqpoint{0.675182in}{1.631325in}}%
\pgfpathlineto{\pgfqpoint{0.697717in}{1.576491in}}%
\pgfpathlineto{\pgfqpoint{0.720252in}{1.519790in}}%
\pgfpathlineto{\pgfqpoint{0.742787in}{1.468266in}}%
\pgfpathlineto{\pgfqpoint{0.765323in}{1.408786in}}%
\pgfpathlineto{\pgfqpoint{0.787858in}{1.366510in}}%
\pgfpathlineto{\pgfqpoint{0.810393in}{1.320053in}}%
\pgfpathlineto{\pgfqpoint{0.832929in}{1.277743in}}%
\pgfpathlineto{\pgfqpoint{0.855464in}{1.235781in}}%
\pgfpathlineto{\pgfqpoint{0.877999in}{1.204965in}}%
\pgfpathlineto{\pgfqpoint{0.900534in}{1.171491in}}%
\pgfpathlineto{\pgfqpoint{0.923070in}{1.134036in}}%
\pgfpathlineto{\pgfqpoint{0.945605in}{1.105946in}}%
\pgfpathlineto{\pgfqpoint{0.968140in}{1.082391in}}%
\pgfpathlineto{\pgfqpoint{0.990676in}{1.060466in}}%
\pgfpathlineto{\pgfqpoint{1.013211in}{1.045229in}}%
\pgfpathlineto{\pgfqpoint{1.035746in}{1.025727in}}%
\pgfpathlineto{\pgfqpoint{1.058281in}{1.007393in}}%
\pgfpathlineto{\pgfqpoint{1.080817in}{0.992298in}}%
\pgfpathlineto{\pgfqpoint{1.103352in}{0.974155in}}%
\pgfpathlineto{\pgfqpoint{1.125887in}{0.948398in}}%
\pgfpathlineto{\pgfqpoint{1.148423in}{0.932332in}}%
\pgfpathlineto{\pgfqpoint{1.170958in}{0.913112in}}%
\pgfpathlineto{\pgfqpoint{1.193493in}{0.896385in}}%
\pgfpathlineto{\pgfqpoint{1.216028in}{0.883859in}}%
\pgfpathlineto{\pgfqpoint{1.238564in}{0.865272in}}%
\pgfpathlineto{\pgfqpoint{1.261099in}{0.846654in}}%
\pgfpathlineto{\pgfqpoint{1.283634in}{0.833871in}}%
\pgfpathlineto{\pgfqpoint{1.306170in}{0.818157in}}%
\pgfpathlineto{\pgfqpoint{1.328705in}{0.805811in}}%
\pgfpathlineto{\pgfqpoint{1.351240in}{0.793106in}}%
\pgfpathlineto{\pgfqpoint{1.373776in}{0.779120in}}%
\pgfpathlineto{\pgfqpoint{1.396311in}{0.760839in}}%
\pgfpathlineto{\pgfqpoint{1.418846in}{0.747699in}}%
\pgfpathlineto{\pgfqpoint{1.441381in}{0.738533in}}%
\pgfpathlineto{\pgfqpoint{1.463917in}{0.725598in}}%
\pgfpathlineto{\pgfqpoint{1.486452in}{0.715417in}}%
\pgfpathlineto{\pgfqpoint{1.508987in}{0.705333in}}%
\pgfpathlineto{\pgfqpoint{1.531523in}{0.697284in}}%
\pgfpathlineto{\pgfqpoint{1.554058in}{0.690556in}}%
\pgfpathlineto{\pgfqpoint{1.576593in}{0.683505in}}%
\pgfpathlineto{\pgfqpoint{1.599128in}{0.673655in}}%
\pgfpathlineto{\pgfqpoint{1.621664in}{0.664904in}}%
\pgfpathlineto{\pgfqpoint{1.644199in}{0.657054in}}%
\pgfpathlineto{\pgfqpoint{1.666734in}{0.647015in}}%
\pgfpathlineto{\pgfqpoint{1.689270in}{0.643067in}}%
\pgfpathlineto{\pgfqpoint{1.711805in}{0.637916in}}%
\pgfpathlineto{\pgfqpoint{1.734340in}{0.631303in}}%
\pgfpathlineto{\pgfqpoint{1.756875in}{0.624972in}}%
\pgfpathlineto{\pgfqpoint{1.779411in}{0.620217in}}%
\pgfpathlineto{\pgfqpoint{1.801946in}{0.614355in}}%
\pgfpathlineto{\pgfqpoint{1.824481in}{0.605764in}}%
\pgfpathlineto{\pgfqpoint{1.847017in}{0.600145in}}%
\pgfpathlineto{\pgfqpoint{1.869552in}{0.592365in}}%
\pgfpathlineto{\pgfqpoint{1.892087in}{0.587516in}}%
\pgfpathlineto{\pgfqpoint{1.914622in}{0.583009in}}%
\pgfpathlineto{\pgfqpoint{1.937158in}{0.578032in}}%
\pgfpathlineto{\pgfqpoint{1.959693in}{0.573337in}}%
\pgfpathlineto{\pgfqpoint{1.982228in}{0.569292in}}%
\pgfpathlineto{\pgfqpoint{2.004764in}{0.567763in}}%
\pgfpathlineto{\pgfqpoint{2.027299in}{0.563039in}}%
\pgfpathlineto{\pgfqpoint{2.049834in}{0.559865in}}%
\pgfpathlineto{\pgfqpoint{2.072369in}{0.556262in}}%
\pgfpathlineto{\pgfqpoint{2.094905in}{0.552695in}}%
\pgfpathlineto{\pgfqpoint{2.117440in}{0.551418in}}%
\pgfpathlineto{\pgfqpoint{2.139975in}{0.549545in}}%
\pgfpathlineto{\pgfqpoint{2.162511in}{0.548060in}}%
\pgfpathlineto{\pgfqpoint{2.185046in}{0.544342in}}%
\pgfpathlineto{\pgfqpoint{2.207581in}{0.542447in}}%
\pgfpathlineto{\pgfqpoint{2.230116in}{0.541367in}}%
\pgfpathlineto{\pgfqpoint{2.252652in}{0.538512in}}%
\pgfpathlineto{\pgfqpoint{2.275187in}{0.535461in}}%
\pgfpathlineto{\pgfqpoint{2.297722in}{0.534584in}}%
\pgfpathlineto{\pgfqpoint{2.320258in}{0.534132in}}%
\pgfpathlineto{\pgfqpoint{2.342793in}{0.532457in}}%
\pgfpathlineto{\pgfqpoint{2.365328in}{0.530472in}}%
\pgfpathlineto{\pgfqpoint{2.387863in}{0.527158in}}%
\pgfpathlineto{\pgfqpoint{2.410399in}{0.525352in}}%
\pgfpathlineto{\pgfqpoint{2.432934in}{0.523621in}}%
\pgfpathlineto{\pgfqpoint{2.455469in}{0.521373in}}%
\pgfpathlineto{\pgfqpoint{2.478005in}{0.520258in}}%
\pgfpathlineto{\pgfqpoint{2.500540in}{0.516840in}}%
\pgfpathlineto{\pgfqpoint{2.523075in}{0.513600in}}%
\pgfpathlineto{\pgfqpoint{2.545610in}{0.511436in}}%
\pgfpathlineto{\pgfqpoint{2.568146in}{0.509155in}}%
\pgfpathlineto{\pgfqpoint{2.590681in}{0.506059in}}%
\pgfpathlineto{\pgfqpoint{2.613216in}{0.504976in}}%
\pgfpathlineto{\pgfqpoint{2.635752in}{0.503903in}}%
\pgfpathlineto{\pgfqpoint{2.658287in}{0.500670in}}%
\pgfpathlineto{\pgfqpoint{2.680822in}{0.500720in}}%
\pgfpathlineto{\pgfqpoint{2.703357in}{0.500344in}}%
\pgfpathlineto{\pgfqpoint{2.725893in}{0.499519in}}%
\pgfpathlineto{\pgfqpoint{2.748428in}{0.497554in}}%
\pgfpathlineto{\pgfqpoint{2.770963in}{0.496910in}}%
\pgfusepath{stroke}%
\end{pgfscope}%
\begin{pgfscope}%
\pgfpathrectangle{\pgfqpoint{0.539970in}{0.422992in}}{\pgfqpoint{2.276064in}{1.626201in}}%
\pgfusepath{clip}%
\pgfsetrectcap%
\pgfsetroundjoin%
\pgfsetlinewidth{1.003750pt}%
\definecolor{currentstroke}{rgb}{0.517647,0.356863,0.592157}%
\pgfsetstrokecolor{currentstroke}%
\pgfsetdash{}{0pt}%
\pgfpathmoveto{\pgfqpoint{0.562505in}{1.606741in}}%
\pgfpathlineto{\pgfqpoint{0.585040in}{1.556318in}}%
\pgfpathlineto{\pgfqpoint{0.607576in}{1.528020in}}%
\pgfpathlineto{\pgfqpoint{0.630111in}{1.506627in}}%
\pgfpathlineto{\pgfqpoint{0.652646in}{1.481212in}}%
\pgfpathlineto{\pgfqpoint{0.675182in}{1.444874in}}%
\pgfpathlineto{\pgfqpoint{0.697717in}{1.417251in}}%
\pgfpathlineto{\pgfqpoint{0.720252in}{1.392962in}}%
\pgfpathlineto{\pgfqpoint{0.742787in}{1.372216in}}%
\pgfpathlineto{\pgfqpoint{0.765323in}{1.342611in}}%
\pgfpathlineto{\pgfqpoint{0.787858in}{1.324437in}}%
\pgfpathlineto{\pgfqpoint{0.810393in}{1.307677in}}%
\pgfpathlineto{\pgfqpoint{0.832929in}{1.285310in}}%
\pgfpathlineto{\pgfqpoint{0.855464in}{1.262702in}}%
\pgfpathlineto{\pgfqpoint{0.877999in}{1.243396in}}%
\pgfpathlineto{\pgfqpoint{0.900534in}{1.224703in}}%
\pgfpathlineto{\pgfqpoint{0.923070in}{1.211132in}}%
\pgfpathlineto{\pgfqpoint{0.945605in}{1.205176in}}%
\pgfpathlineto{\pgfqpoint{0.968140in}{1.191571in}}%
\pgfpathlineto{\pgfqpoint{0.990676in}{1.183783in}}%
\pgfpathlineto{\pgfqpoint{1.013211in}{1.173515in}}%
\pgfpathlineto{\pgfqpoint{1.035746in}{1.169121in}}%
\pgfpathlineto{\pgfqpoint{1.058281in}{1.156917in}}%
\pgfpathlineto{\pgfqpoint{1.080817in}{1.154109in}}%
\pgfpathlineto{\pgfqpoint{1.103352in}{1.147685in}}%
\pgfpathlineto{\pgfqpoint{1.125887in}{1.138575in}}%
\pgfpathlineto{\pgfqpoint{1.148423in}{1.132114in}}%
\pgfpathlineto{\pgfqpoint{1.170958in}{1.130408in}}%
\pgfpathlineto{\pgfqpoint{1.193493in}{1.125512in}}%
\pgfpathlineto{\pgfqpoint{1.216028in}{1.118165in}}%
\pgfpathlineto{\pgfqpoint{1.238564in}{1.117077in}}%
\pgfpathlineto{\pgfqpoint{1.261099in}{1.115358in}}%
\pgfpathlineto{\pgfqpoint{1.283634in}{1.113393in}}%
\pgfpathlineto{\pgfqpoint{1.306170in}{1.106196in}}%
\pgfpathlineto{\pgfqpoint{1.328705in}{1.104720in}}%
\pgfpathlineto{\pgfqpoint{1.351240in}{1.105324in}}%
\pgfpathlineto{\pgfqpoint{1.373776in}{1.104257in}}%
\pgfpathlineto{\pgfqpoint{1.396311in}{1.101985in}}%
\pgfpathlineto{\pgfqpoint{1.418846in}{1.100408in}}%
\pgfpathlineto{\pgfqpoint{1.441381in}{1.097042in}}%
\pgfpathlineto{\pgfqpoint{1.463917in}{1.097694in}}%
\pgfpathlineto{\pgfqpoint{1.486452in}{1.096293in}}%
\pgfpathlineto{\pgfqpoint{1.508987in}{1.093234in}}%
\pgfpathlineto{\pgfqpoint{1.531523in}{1.091537in}}%
\pgfpathlineto{\pgfqpoint{1.554058in}{1.091688in}}%
\pgfpathlineto{\pgfqpoint{1.576593in}{1.088751in}}%
\pgfpathlineto{\pgfqpoint{1.599128in}{1.086645in}}%
\pgfpathlineto{\pgfqpoint{1.621664in}{1.084594in}}%
\pgfpathlineto{\pgfqpoint{1.644199in}{1.084140in}}%
\pgfpathlineto{\pgfqpoint{1.666734in}{1.084048in}}%
\pgfpathlineto{\pgfqpoint{1.689270in}{1.083117in}}%
\pgfpathlineto{\pgfqpoint{1.711805in}{1.081253in}}%
\pgfpathlineto{\pgfqpoint{1.734340in}{1.080361in}}%
\pgfpathlineto{\pgfqpoint{1.756875in}{1.077797in}}%
\pgfpathlineto{\pgfqpoint{1.779411in}{1.075031in}}%
\pgfpathlineto{\pgfqpoint{1.801946in}{1.072788in}}%
\pgfpathlineto{\pgfqpoint{1.824481in}{1.072139in}}%
\pgfpathlineto{\pgfqpoint{1.847017in}{1.070306in}}%
\pgfpathlineto{\pgfqpoint{1.869552in}{1.069902in}}%
\pgfpathlineto{\pgfqpoint{1.892087in}{1.068124in}}%
\pgfpathlineto{\pgfqpoint{1.914622in}{1.066836in}}%
\pgfpathlineto{\pgfqpoint{1.937158in}{1.064304in}}%
\pgfpathlineto{\pgfqpoint{1.959693in}{1.063638in}}%
\pgfpathlineto{\pgfqpoint{1.982228in}{1.063509in}}%
\pgfpathlineto{\pgfqpoint{2.004764in}{1.060018in}}%
\pgfpathlineto{\pgfqpoint{2.027299in}{1.058392in}}%
\pgfpathlineto{\pgfqpoint{2.049834in}{1.058484in}}%
\pgfpathlineto{\pgfqpoint{2.072369in}{1.056115in}}%
\pgfpathlineto{\pgfqpoint{2.094905in}{1.052781in}}%
\pgfpathlineto{\pgfqpoint{2.117440in}{1.051905in}}%
\pgfpathlineto{\pgfqpoint{2.139975in}{1.050948in}}%
\pgfpathlineto{\pgfqpoint{2.162511in}{1.049317in}}%
\pgfpathlineto{\pgfqpoint{2.185046in}{1.049678in}}%
\pgfpathlineto{\pgfqpoint{2.207581in}{1.048353in}}%
\pgfpathlineto{\pgfqpoint{2.230116in}{1.047611in}}%
\pgfpathlineto{\pgfqpoint{2.252652in}{1.047834in}}%
\pgfpathlineto{\pgfqpoint{2.275187in}{1.047945in}}%
\pgfpathlineto{\pgfqpoint{2.297722in}{1.046833in}}%
\pgfpathlineto{\pgfqpoint{2.320258in}{1.048112in}}%
\pgfpathlineto{\pgfqpoint{2.342793in}{1.046990in}}%
\pgfpathlineto{\pgfqpoint{2.365328in}{1.044655in}}%
\pgfpathlineto{\pgfqpoint{2.387863in}{1.044586in}}%
\pgfpathlineto{\pgfqpoint{2.410399in}{1.043439in}}%
\pgfpathlineto{\pgfqpoint{2.432934in}{1.042211in}}%
\pgfpathlineto{\pgfqpoint{2.455469in}{1.042479in}}%
\pgfpathlineto{\pgfqpoint{2.478005in}{1.042904in}}%
\pgfpathlineto{\pgfqpoint{2.500540in}{1.043382in}}%
\pgfpathlineto{\pgfqpoint{2.523075in}{1.043680in}}%
\pgfpathlineto{\pgfqpoint{2.545610in}{1.043498in}}%
\pgfpathlineto{\pgfqpoint{2.568146in}{1.043444in}}%
\pgfpathlineto{\pgfqpoint{2.590681in}{1.042782in}}%
\pgfpathlineto{\pgfqpoint{2.613216in}{1.042770in}}%
\pgfpathlineto{\pgfqpoint{2.635752in}{1.040999in}}%
\pgfpathlineto{\pgfqpoint{2.658287in}{1.041357in}}%
\pgfpathlineto{\pgfqpoint{2.680822in}{1.040397in}}%
\pgfpathlineto{\pgfqpoint{2.703357in}{1.041273in}}%
\pgfpathlineto{\pgfqpoint{2.725893in}{1.040869in}}%
\pgfpathlineto{\pgfqpoint{2.748428in}{1.039632in}}%
\pgfpathlineto{\pgfqpoint{2.770963in}{1.038853in}}%
\pgfusepath{stroke}%
\end{pgfscope}%
\begin{pgfscope}%
\pgfsetrectcap%
\pgfsetmiterjoin%
\pgfsetlinewidth{0.501875pt}%
\definecolor{currentstroke}{rgb}{0.000000,0.000000,0.000000}%
\pgfsetstrokecolor{currentstroke}%
\pgfsetdash{}{0pt}%
\pgfpathmoveto{\pgfqpoint{0.539970in}{0.422992in}}%
\pgfpathlineto{\pgfqpoint{0.539970in}{2.049193in}}%
\pgfusepath{stroke}%
\end{pgfscope}%
\begin{pgfscope}%
\pgfsetrectcap%
\pgfsetmiterjoin%
\pgfsetlinewidth{0.501875pt}%
\definecolor{currentstroke}{rgb}{0.000000,0.000000,0.000000}%
\pgfsetstrokecolor{currentstroke}%
\pgfsetdash{}{0pt}%
\pgfpathmoveto{\pgfqpoint{2.816034in}{0.422992in}}%
\pgfpathlineto{\pgfqpoint{2.816034in}{2.049193in}}%
\pgfusepath{stroke}%
\end{pgfscope}%
\begin{pgfscope}%
\pgfsetrectcap%
\pgfsetmiterjoin%
\pgfsetlinewidth{0.501875pt}%
\definecolor{currentstroke}{rgb}{0.000000,0.000000,0.000000}%
\pgfsetstrokecolor{currentstroke}%
\pgfsetdash{}{0pt}%
\pgfpathmoveto{\pgfqpoint{0.539970in}{0.422992in}}%
\pgfpathlineto{\pgfqpoint{2.816034in}{0.422992in}}%
\pgfusepath{stroke}%
\end{pgfscope}%
\begin{pgfscope}%
\pgfsetrectcap%
\pgfsetmiterjoin%
\pgfsetlinewidth{0.501875pt}%
\definecolor{currentstroke}{rgb}{0.000000,0.000000,0.000000}%
\pgfsetstrokecolor{currentstroke}%
\pgfsetdash{}{0pt}%
\pgfpathmoveto{\pgfqpoint{0.539970in}{2.049193in}}%
\pgfpathlineto{\pgfqpoint{2.816034in}{2.049193in}}%
\pgfusepath{stroke}%
\end{pgfscope}%
\begin{pgfscope}%
\definecolor{textcolor}{rgb}{0.000000,0.000000,0.000000}%
\pgfsetstrokecolor{textcolor}%
\pgfsetfillcolor{textcolor}%
\pgftext[x=1.678002in,y=2.132526in,,base]{\color{textcolor}\rmfamily\fontsize{12.000000}{14.400000}\selectfont Continuity}%
\end{pgfscope}%
\begin{pgfscope}%
\pgfsetbuttcap%
\pgfsetmiterjoin%
\definecolor{currentfill}{rgb}{1.000000,1.000000,1.000000}%
\pgfsetfillcolor{currentfill}%
\pgfsetlinewidth{0.000000pt}%
\definecolor{currentstroke}{rgb}{0.000000,0.000000,0.000000}%
\pgfsetstrokecolor{currentstroke}%
\pgfsetstrokeopacity{0.000000}%
\pgfsetdash{}{0pt}%
\pgfpathmoveto{\pgfqpoint{3.385377in}{0.422992in}}%
\pgfpathlineto{\pgfqpoint{5.661441in}{0.422992in}}%
\pgfpathlineto{\pgfqpoint{5.661441in}{4.374193in}}%
\pgfpathlineto{\pgfqpoint{3.385377in}{4.374193in}}%
\pgfpathlineto{\pgfqpoint{3.385377in}{0.422992in}}%
\pgfpathclose%
\pgfusepath{fill}%
\end{pgfscope}%
\begin{pgfscope}%
\pgfsetbuttcap%
\pgfsetroundjoin%
\definecolor{currentfill}{rgb}{0.000000,0.000000,0.000000}%
\pgfsetfillcolor{currentfill}%
\pgfsetlinewidth{0.501875pt}%
\definecolor{currentstroke}{rgb}{0.000000,0.000000,0.000000}%
\pgfsetstrokecolor{currentstroke}%
\pgfsetdash{}{0pt}%
\pgfsys@defobject{currentmarker}{\pgfqpoint{0.000000in}{0.000000in}}{\pgfqpoint{0.000000in}{0.041667in}}{%
\pgfpathmoveto{\pgfqpoint{0.000000in}{0.000000in}}%
\pgfpathlineto{\pgfqpoint{0.000000in}{0.041667in}}%
\pgfusepath{stroke,fill}%
}%
\begin{pgfscope}%
\pgfsys@transformshift{3.385377in}{0.422992in}%
\pgfsys@useobject{currentmarker}{}%
\end{pgfscope}%
\end{pgfscope}%
\begin{pgfscope}%
\pgfsetbuttcap%
\pgfsetroundjoin%
\definecolor{currentfill}{rgb}{0.000000,0.000000,0.000000}%
\pgfsetfillcolor{currentfill}%
\pgfsetlinewidth{0.501875pt}%
\definecolor{currentstroke}{rgb}{0.000000,0.000000,0.000000}%
\pgfsetstrokecolor{currentstroke}%
\pgfsetdash{}{0pt}%
\pgfsys@defobject{currentmarker}{\pgfqpoint{0.000000in}{-0.041667in}}{\pgfqpoint{0.000000in}{0.000000in}}{%
\pgfpathmoveto{\pgfqpoint{0.000000in}{0.000000in}}%
\pgfpathlineto{\pgfqpoint{0.000000in}{-0.041667in}}%
\pgfusepath{stroke,fill}%
}%
\begin{pgfscope}%
\pgfsys@transformshift{3.385377in}{4.374193in}%
\pgfsys@useobject{currentmarker}{}%
\end{pgfscope}%
\end{pgfscope}%
\begin{pgfscope}%
\definecolor{textcolor}{rgb}{0.000000,0.000000,0.000000}%
\pgfsetstrokecolor{textcolor}%
\pgfsetfillcolor{textcolor}%
\pgftext[x=3.385377in,y=0.374381in,,top]{\color{textcolor}\rmfamily\fontsize{10.000000}{12.000000}\selectfont \(\displaystyle {0}\)}%
\end{pgfscope}%
\begin{pgfscope}%
\pgfsetbuttcap%
\pgfsetroundjoin%
\definecolor{currentfill}{rgb}{0.000000,0.000000,0.000000}%
\pgfsetfillcolor{currentfill}%
\pgfsetlinewidth{0.501875pt}%
\definecolor{currentstroke}{rgb}{0.000000,0.000000,0.000000}%
\pgfsetstrokecolor{currentstroke}%
\pgfsetdash{}{0pt}%
\pgfsys@defobject{currentmarker}{\pgfqpoint{0.000000in}{0.000000in}}{\pgfqpoint{0.000000in}{0.041667in}}{%
\pgfpathmoveto{\pgfqpoint{0.000000in}{0.000000in}}%
\pgfpathlineto{\pgfqpoint{0.000000in}{0.041667in}}%
\pgfusepath{stroke,fill}%
}%
\begin{pgfscope}%
\pgfsys@transformshift{3.836083in}{0.422992in}%
\pgfsys@useobject{currentmarker}{}%
\end{pgfscope}%
\end{pgfscope}%
\begin{pgfscope}%
\pgfsetbuttcap%
\pgfsetroundjoin%
\definecolor{currentfill}{rgb}{0.000000,0.000000,0.000000}%
\pgfsetfillcolor{currentfill}%
\pgfsetlinewidth{0.501875pt}%
\definecolor{currentstroke}{rgb}{0.000000,0.000000,0.000000}%
\pgfsetstrokecolor{currentstroke}%
\pgfsetdash{}{0pt}%
\pgfsys@defobject{currentmarker}{\pgfqpoint{0.000000in}{-0.041667in}}{\pgfqpoint{0.000000in}{0.000000in}}{%
\pgfpathmoveto{\pgfqpoint{0.000000in}{0.000000in}}%
\pgfpathlineto{\pgfqpoint{0.000000in}{-0.041667in}}%
\pgfusepath{stroke,fill}%
}%
\begin{pgfscope}%
\pgfsys@transformshift{3.836083in}{4.374193in}%
\pgfsys@useobject{currentmarker}{}%
\end{pgfscope}%
\end{pgfscope}%
\begin{pgfscope}%
\definecolor{textcolor}{rgb}{0.000000,0.000000,0.000000}%
\pgfsetstrokecolor{textcolor}%
\pgfsetfillcolor{textcolor}%
\pgftext[x=3.836083in,y=0.374381in,,top]{\color{textcolor}\rmfamily\fontsize{10.000000}{12.000000}\selectfont \(\displaystyle {20}\)}%
\end{pgfscope}%
\begin{pgfscope}%
\pgfsetbuttcap%
\pgfsetroundjoin%
\definecolor{currentfill}{rgb}{0.000000,0.000000,0.000000}%
\pgfsetfillcolor{currentfill}%
\pgfsetlinewidth{0.501875pt}%
\definecolor{currentstroke}{rgb}{0.000000,0.000000,0.000000}%
\pgfsetstrokecolor{currentstroke}%
\pgfsetdash{}{0pt}%
\pgfsys@defobject{currentmarker}{\pgfqpoint{0.000000in}{0.000000in}}{\pgfqpoint{0.000000in}{0.041667in}}{%
\pgfpathmoveto{\pgfqpoint{0.000000in}{0.000000in}}%
\pgfpathlineto{\pgfqpoint{0.000000in}{0.041667in}}%
\pgfusepath{stroke,fill}%
}%
\begin{pgfscope}%
\pgfsys@transformshift{4.286789in}{0.422992in}%
\pgfsys@useobject{currentmarker}{}%
\end{pgfscope}%
\end{pgfscope}%
\begin{pgfscope}%
\pgfsetbuttcap%
\pgfsetroundjoin%
\definecolor{currentfill}{rgb}{0.000000,0.000000,0.000000}%
\pgfsetfillcolor{currentfill}%
\pgfsetlinewidth{0.501875pt}%
\definecolor{currentstroke}{rgb}{0.000000,0.000000,0.000000}%
\pgfsetstrokecolor{currentstroke}%
\pgfsetdash{}{0pt}%
\pgfsys@defobject{currentmarker}{\pgfqpoint{0.000000in}{-0.041667in}}{\pgfqpoint{0.000000in}{0.000000in}}{%
\pgfpathmoveto{\pgfqpoint{0.000000in}{0.000000in}}%
\pgfpathlineto{\pgfqpoint{0.000000in}{-0.041667in}}%
\pgfusepath{stroke,fill}%
}%
\begin{pgfscope}%
\pgfsys@transformshift{4.286789in}{4.374193in}%
\pgfsys@useobject{currentmarker}{}%
\end{pgfscope}%
\end{pgfscope}%
\begin{pgfscope}%
\definecolor{textcolor}{rgb}{0.000000,0.000000,0.000000}%
\pgfsetstrokecolor{textcolor}%
\pgfsetfillcolor{textcolor}%
\pgftext[x=4.286789in,y=0.374381in,,top]{\color{textcolor}\rmfamily\fontsize{10.000000}{12.000000}\selectfont \(\displaystyle {40}\)}%
\end{pgfscope}%
\begin{pgfscope}%
\pgfsetbuttcap%
\pgfsetroundjoin%
\definecolor{currentfill}{rgb}{0.000000,0.000000,0.000000}%
\pgfsetfillcolor{currentfill}%
\pgfsetlinewidth{0.501875pt}%
\definecolor{currentstroke}{rgb}{0.000000,0.000000,0.000000}%
\pgfsetstrokecolor{currentstroke}%
\pgfsetdash{}{0pt}%
\pgfsys@defobject{currentmarker}{\pgfqpoint{0.000000in}{0.000000in}}{\pgfqpoint{0.000000in}{0.041667in}}{%
\pgfpathmoveto{\pgfqpoint{0.000000in}{0.000000in}}%
\pgfpathlineto{\pgfqpoint{0.000000in}{0.041667in}}%
\pgfusepath{stroke,fill}%
}%
\begin{pgfscope}%
\pgfsys@transformshift{4.737495in}{0.422992in}%
\pgfsys@useobject{currentmarker}{}%
\end{pgfscope}%
\end{pgfscope}%
\begin{pgfscope}%
\pgfsetbuttcap%
\pgfsetroundjoin%
\definecolor{currentfill}{rgb}{0.000000,0.000000,0.000000}%
\pgfsetfillcolor{currentfill}%
\pgfsetlinewidth{0.501875pt}%
\definecolor{currentstroke}{rgb}{0.000000,0.000000,0.000000}%
\pgfsetstrokecolor{currentstroke}%
\pgfsetdash{}{0pt}%
\pgfsys@defobject{currentmarker}{\pgfqpoint{0.000000in}{-0.041667in}}{\pgfqpoint{0.000000in}{0.000000in}}{%
\pgfpathmoveto{\pgfqpoint{0.000000in}{0.000000in}}%
\pgfpathlineto{\pgfqpoint{0.000000in}{-0.041667in}}%
\pgfusepath{stroke,fill}%
}%
\begin{pgfscope}%
\pgfsys@transformshift{4.737495in}{4.374193in}%
\pgfsys@useobject{currentmarker}{}%
\end{pgfscope}%
\end{pgfscope}%
\begin{pgfscope}%
\definecolor{textcolor}{rgb}{0.000000,0.000000,0.000000}%
\pgfsetstrokecolor{textcolor}%
\pgfsetfillcolor{textcolor}%
\pgftext[x=4.737495in,y=0.374381in,,top]{\color{textcolor}\rmfamily\fontsize{10.000000}{12.000000}\selectfont \(\displaystyle {60}\)}%
\end{pgfscope}%
\begin{pgfscope}%
\pgfsetbuttcap%
\pgfsetroundjoin%
\definecolor{currentfill}{rgb}{0.000000,0.000000,0.000000}%
\pgfsetfillcolor{currentfill}%
\pgfsetlinewidth{0.501875pt}%
\definecolor{currentstroke}{rgb}{0.000000,0.000000,0.000000}%
\pgfsetstrokecolor{currentstroke}%
\pgfsetdash{}{0pt}%
\pgfsys@defobject{currentmarker}{\pgfqpoint{0.000000in}{0.000000in}}{\pgfqpoint{0.000000in}{0.041667in}}{%
\pgfpathmoveto{\pgfqpoint{0.000000in}{0.000000in}}%
\pgfpathlineto{\pgfqpoint{0.000000in}{0.041667in}}%
\pgfusepath{stroke,fill}%
}%
\begin{pgfscope}%
\pgfsys@transformshift{5.188200in}{0.422992in}%
\pgfsys@useobject{currentmarker}{}%
\end{pgfscope}%
\end{pgfscope}%
\begin{pgfscope}%
\pgfsetbuttcap%
\pgfsetroundjoin%
\definecolor{currentfill}{rgb}{0.000000,0.000000,0.000000}%
\pgfsetfillcolor{currentfill}%
\pgfsetlinewidth{0.501875pt}%
\definecolor{currentstroke}{rgb}{0.000000,0.000000,0.000000}%
\pgfsetstrokecolor{currentstroke}%
\pgfsetdash{}{0pt}%
\pgfsys@defobject{currentmarker}{\pgfqpoint{0.000000in}{-0.041667in}}{\pgfqpoint{0.000000in}{0.000000in}}{%
\pgfpathmoveto{\pgfqpoint{0.000000in}{0.000000in}}%
\pgfpathlineto{\pgfqpoint{0.000000in}{-0.041667in}}%
\pgfusepath{stroke,fill}%
}%
\begin{pgfscope}%
\pgfsys@transformshift{5.188200in}{4.374193in}%
\pgfsys@useobject{currentmarker}{}%
\end{pgfscope}%
\end{pgfscope}%
\begin{pgfscope}%
\definecolor{textcolor}{rgb}{0.000000,0.000000,0.000000}%
\pgfsetstrokecolor{textcolor}%
\pgfsetfillcolor{textcolor}%
\pgftext[x=5.188200in,y=0.374381in,,top]{\color{textcolor}\rmfamily\fontsize{10.000000}{12.000000}\selectfont \(\displaystyle {80}\)}%
\end{pgfscope}%
\begin{pgfscope}%
\pgfsetbuttcap%
\pgfsetroundjoin%
\definecolor{currentfill}{rgb}{0.000000,0.000000,0.000000}%
\pgfsetfillcolor{currentfill}%
\pgfsetlinewidth{0.501875pt}%
\definecolor{currentstroke}{rgb}{0.000000,0.000000,0.000000}%
\pgfsetstrokecolor{currentstroke}%
\pgfsetdash{}{0pt}%
\pgfsys@defobject{currentmarker}{\pgfqpoint{0.000000in}{0.000000in}}{\pgfqpoint{0.000000in}{0.020833in}}{%
\pgfpathmoveto{\pgfqpoint{0.000000in}{0.000000in}}%
\pgfpathlineto{\pgfqpoint{0.000000in}{0.020833in}}%
\pgfusepath{stroke,fill}%
}%
\begin{pgfscope}%
\pgfsys@transformshift{3.498054in}{0.422992in}%
\pgfsys@useobject{currentmarker}{}%
\end{pgfscope}%
\end{pgfscope}%
\begin{pgfscope}%
\pgfsetbuttcap%
\pgfsetroundjoin%
\definecolor{currentfill}{rgb}{0.000000,0.000000,0.000000}%
\pgfsetfillcolor{currentfill}%
\pgfsetlinewidth{0.501875pt}%
\definecolor{currentstroke}{rgb}{0.000000,0.000000,0.000000}%
\pgfsetstrokecolor{currentstroke}%
\pgfsetdash{}{0pt}%
\pgfsys@defobject{currentmarker}{\pgfqpoint{0.000000in}{-0.020833in}}{\pgfqpoint{0.000000in}{0.000000in}}{%
\pgfpathmoveto{\pgfqpoint{0.000000in}{0.000000in}}%
\pgfpathlineto{\pgfqpoint{0.000000in}{-0.020833in}}%
\pgfusepath{stroke,fill}%
}%
\begin{pgfscope}%
\pgfsys@transformshift{3.498054in}{4.374193in}%
\pgfsys@useobject{currentmarker}{}%
\end{pgfscope}%
\end{pgfscope}%
\begin{pgfscope}%
\pgfsetbuttcap%
\pgfsetroundjoin%
\definecolor{currentfill}{rgb}{0.000000,0.000000,0.000000}%
\pgfsetfillcolor{currentfill}%
\pgfsetlinewidth{0.501875pt}%
\definecolor{currentstroke}{rgb}{0.000000,0.000000,0.000000}%
\pgfsetstrokecolor{currentstroke}%
\pgfsetdash{}{0pt}%
\pgfsys@defobject{currentmarker}{\pgfqpoint{0.000000in}{0.000000in}}{\pgfqpoint{0.000000in}{0.020833in}}{%
\pgfpathmoveto{\pgfqpoint{0.000000in}{0.000000in}}%
\pgfpathlineto{\pgfqpoint{0.000000in}{0.020833in}}%
\pgfusepath{stroke,fill}%
}%
\begin{pgfscope}%
\pgfsys@transformshift{3.610730in}{0.422992in}%
\pgfsys@useobject{currentmarker}{}%
\end{pgfscope}%
\end{pgfscope}%
\begin{pgfscope}%
\pgfsetbuttcap%
\pgfsetroundjoin%
\definecolor{currentfill}{rgb}{0.000000,0.000000,0.000000}%
\pgfsetfillcolor{currentfill}%
\pgfsetlinewidth{0.501875pt}%
\definecolor{currentstroke}{rgb}{0.000000,0.000000,0.000000}%
\pgfsetstrokecolor{currentstroke}%
\pgfsetdash{}{0pt}%
\pgfsys@defobject{currentmarker}{\pgfqpoint{0.000000in}{-0.020833in}}{\pgfqpoint{0.000000in}{0.000000in}}{%
\pgfpathmoveto{\pgfqpoint{0.000000in}{0.000000in}}%
\pgfpathlineto{\pgfqpoint{0.000000in}{-0.020833in}}%
\pgfusepath{stroke,fill}%
}%
\begin{pgfscope}%
\pgfsys@transformshift{3.610730in}{4.374193in}%
\pgfsys@useobject{currentmarker}{}%
\end{pgfscope}%
\end{pgfscope}%
\begin{pgfscope}%
\pgfsetbuttcap%
\pgfsetroundjoin%
\definecolor{currentfill}{rgb}{0.000000,0.000000,0.000000}%
\pgfsetfillcolor{currentfill}%
\pgfsetlinewidth{0.501875pt}%
\definecolor{currentstroke}{rgb}{0.000000,0.000000,0.000000}%
\pgfsetstrokecolor{currentstroke}%
\pgfsetdash{}{0pt}%
\pgfsys@defobject{currentmarker}{\pgfqpoint{0.000000in}{0.000000in}}{\pgfqpoint{0.000000in}{0.020833in}}{%
\pgfpathmoveto{\pgfqpoint{0.000000in}{0.000000in}}%
\pgfpathlineto{\pgfqpoint{0.000000in}{0.020833in}}%
\pgfusepath{stroke,fill}%
}%
\begin{pgfscope}%
\pgfsys@transformshift{3.723407in}{0.422992in}%
\pgfsys@useobject{currentmarker}{}%
\end{pgfscope}%
\end{pgfscope}%
\begin{pgfscope}%
\pgfsetbuttcap%
\pgfsetroundjoin%
\definecolor{currentfill}{rgb}{0.000000,0.000000,0.000000}%
\pgfsetfillcolor{currentfill}%
\pgfsetlinewidth{0.501875pt}%
\definecolor{currentstroke}{rgb}{0.000000,0.000000,0.000000}%
\pgfsetstrokecolor{currentstroke}%
\pgfsetdash{}{0pt}%
\pgfsys@defobject{currentmarker}{\pgfqpoint{0.000000in}{-0.020833in}}{\pgfqpoint{0.000000in}{0.000000in}}{%
\pgfpathmoveto{\pgfqpoint{0.000000in}{0.000000in}}%
\pgfpathlineto{\pgfqpoint{0.000000in}{-0.020833in}}%
\pgfusepath{stroke,fill}%
}%
\begin{pgfscope}%
\pgfsys@transformshift{3.723407in}{4.374193in}%
\pgfsys@useobject{currentmarker}{}%
\end{pgfscope}%
\end{pgfscope}%
\begin{pgfscope}%
\pgfsetbuttcap%
\pgfsetroundjoin%
\definecolor{currentfill}{rgb}{0.000000,0.000000,0.000000}%
\pgfsetfillcolor{currentfill}%
\pgfsetlinewidth{0.501875pt}%
\definecolor{currentstroke}{rgb}{0.000000,0.000000,0.000000}%
\pgfsetstrokecolor{currentstroke}%
\pgfsetdash{}{0pt}%
\pgfsys@defobject{currentmarker}{\pgfqpoint{0.000000in}{0.000000in}}{\pgfqpoint{0.000000in}{0.020833in}}{%
\pgfpathmoveto{\pgfqpoint{0.000000in}{0.000000in}}%
\pgfpathlineto{\pgfqpoint{0.000000in}{0.020833in}}%
\pgfusepath{stroke,fill}%
}%
\begin{pgfscope}%
\pgfsys@transformshift{3.948759in}{0.422992in}%
\pgfsys@useobject{currentmarker}{}%
\end{pgfscope}%
\end{pgfscope}%
\begin{pgfscope}%
\pgfsetbuttcap%
\pgfsetroundjoin%
\definecolor{currentfill}{rgb}{0.000000,0.000000,0.000000}%
\pgfsetfillcolor{currentfill}%
\pgfsetlinewidth{0.501875pt}%
\definecolor{currentstroke}{rgb}{0.000000,0.000000,0.000000}%
\pgfsetstrokecolor{currentstroke}%
\pgfsetdash{}{0pt}%
\pgfsys@defobject{currentmarker}{\pgfqpoint{0.000000in}{-0.020833in}}{\pgfqpoint{0.000000in}{0.000000in}}{%
\pgfpathmoveto{\pgfqpoint{0.000000in}{0.000000in}}%
\pgfpathlineto{\pgfqpoint{0.000000in}{-0.020833in}}%
\pgfusepath{stroke,fill}%
}%
\begin{pgfscope}%
\pgfsys@transformshift{3.948759in}{4.374193in}%
\pgfsys@useobject{currentmarker}{}%
\end{pgfscope}%
\end{pgfscope}%
\begin{pgfscope}%
\pgfsetbuttcap%
\pgfsetroundjoin%
\definecolor{currentfill}{rgb}{0.000000,0.000000,0.000000}%
\pgfsetfillcolor{currentfill}%
\pgfsetlinewidth{0.501875pt}%
\definecolor{currentstroke}{rgb}{0.000000,0.000000,0.000000}%
\pgfsetstrokecolor{currentstroke}%
\pgfsetdash{}{0pt}%
\pgfsys@defobject{currentmarker}{\pgfqpoint{0.000000in}{0.000000in}}{\pgfqpoint{0.000000in}{0.020833in}}{%
\pgfpathmoveto{\pgfqpoint{0.000000in}{0.000000in}}%
\pgfpathlineto{\pgfqpoint{0.000000in}{0.020833in}}%
\pgfusepath{stroke,fill}%
}%
\begin{pgfscope}%
\pgfsys@transformshift{4.061436in}{0.422992in}%
\pgfsys@useobject{currentmarker}{}%
\end{pgfscope}%
\end{pgfscope}%
\begin{pgfscope}%
\pgfsetbuttcap%
\pgfsetroundjoin%
\definecolor{currentfill}{rgb}{0.000000,0.000000,0.000000}%
\pgfsetfillcolor{currentfill}%
\pgfsetlinewidth{0.501875pt}%
\definecolor{currentstroke}{rgb}{0.000000,0.000000,0.000000}%
\pgfsetstrokecolor{currentstroke}%
\pgfsetdash{}{0pt}%
\pgfsys@defobject{currentmarker}{\pgfqpoint{0.000000in}{-0.020833in}}{\pgfqpoint{0.000000in}{0.000000in}}{%
\pgfpathmoveto{\pgfqpoint{0.000000in}{0.000000in}}%
\pgfpathlineto{\pgfqpoint{0.000000in}{-0.020833in}}%
\pgfusepath{stroke,fill}%
}%
\begin{pgfscope}%
\pgfsys@transformshift{4.061436in}{4.374193in}%
\pgfsys@useobject{currentmarker}{}%
\end{pgfscope}%
\end{pgfscope}%
\begin{pgfscope}%
\pgfsetbuttcap%
\pgfsetroundjoin%
\definecolor{currentfill}{rgb}{0.000000,0.000000,0.000000}%
\pgfsetfillcolor{currentfill}%
\pgfsetlinewidth{0.501875pt}%
\definecolor{currentstroke}{rgb}{0.000000,0.000000,0.000000}%
\pgfsetstrokecolor{currentstroke}%
\pgfsetdash{}{0pt}%
\pgfsys@defobject{currentmarker}{\pgfqpoint{0.000000in}{0.000000in}}{\pgfqpoint{0.000000in}{0.020833in}}{%
\pgfpathmoveto{\pgfqpoint{0.000000in}{0.000000in}}%
\pgfpathlineto{\pgfqpoint{0.000000in}{0.020833in}}%
\pgfusepath{stroke,fill}%
}%
\begin{pgfscope}%
\pgfsys@transformshift{4.174112in}{0.422992in}%
\pgfsys@useobject{currentmarker}{}%
\end{pgfscope}%
\end{pgfscope}%
\begin{pgfscope}%
\pgfsetbuttcap%
\pgfsetroundjoin%
\definecolor{currentfill}{rgb}{0.000000,0.000000,0.000000}%
\pgfsetfillcolor{currentfill}%
\pgfsetlinewidth{0.501875pt}%
\definecolor{currentstroke}{rgb}{0.000000,0.000000,0.000000}%
\pgfsetstrokecolor{currentstroke}%
\pgfsetdash{}{0pt}%
\pgfsys@defobject{currentmarker}{\pgfqpoint{0.000000in}{-0.020833in}}{\pgfqpoint{0.000000in}{0.000000in}}{%
\pgfpathmoveto{\pgfqpoint{0.000000in}{0.000000in}}%
\pgfpathlineto{\pgfqpoint{0.000000in}{-0.020833in}}%
\pgfusepath{stroke,fill}%
}%
\begin{pgfscope}%
\pgfsys@transformshift{4.174112in}{4.374193in}%
\pgfsys@useobject{currentmarker}{}%
\end{pgfscope}%
\end{pgfscope}%
\begin{pgfscope}%
\pgfsetbuttcap%
\pgfsetroundjoin%
\definecolor{currentfill}{rgb}{0.000000,0.000000,0.000000}%
\pgfsetfillcolor{currentfill}%
\pgfsetlinewidth{0.501875pt}%
\definecolor{currentstroke}{rgb}{0.000000,0.000000,0.000000}%
\pgfsetstrokecolor{currentstroke}%
\pgfsetdash{}{0pt}%
\pgfsys@defobject{currentmarker}{\pgfqpoint{0.000000in}{0.000000in}}{\pgfqpoint{0.000000in}{0.020833in}}{%
\pgfpathmoveto{\pgfqpoint{0.000000in}{0.000000in}}%
\pgfpathlineto{\pgfqpoint{0.000000in}{0.020833in}}%
\pgfusepath{stroke,fill}%
}%
\begin{pgfscope}%
\pgfsys@transformshift{4.399465in}{0.422992in}%
\pgfsys@useobject{currentmarker}{}%
\end{pgfscope}%
\end{pgfscope}%
\begin{pgfscope}%
\pgfsetbuttcap%
\pgfsetroundjoin%
\definecolor{currentfill}{rgb}{0.000000,0.000000,0.000000}%
\pgfsetfillcolor{currentfill}%
\pgfsetlinewidth{0.501875pt}%
\definecolor{currentstroke}{rgb}{0.000000,0.000000,0.000000}%
\pgfsetstrokecolor{currentstroke}%
\pgfsetdash{}{0pt}%
\pgfsys@defobject{currentmarker}{\pgfqpoint{0.000000in}{-0.020833in}}{\pgfqpoint{0.000000in}{0.000000in}}{%
\pgfpathmoveto{\pgfqpoint{0.000000in}{0.000000in}}%
\pgfpathlineto{\pgfqpoint{0.000000in}{-0.020833in}}%
\pgfusepath{stroke,fill}%
}%
\begin{pgfscope}%
\pgfsys@transformshift{4.399465in}{4.374193in}%
\pgfsys@useobject{currentmarker}{}%
\end{pgfscope}%
\end{pgfscope}%
\begin{pgfscope}%
\pgfsetbuttcap%
\pgfsetroundjoin%
\definecolor{currentfill}{rgb}{0.000000,0.000000,0.000000}%
\pgfsetfillcolor{currentfill}%
\pgfsetlinewidth{0.501875pt}%
\definecolor{currentstroke}{rgb}{0.000000,0.000000,0.000000}%
\pgfsetstrokecolor{currentstroke}%
\pgfsetdash{}{0pt}%
\pgfsys@defobject{currentmarker}{\pgfqpoint{0.000000in}{0.000000in}}{\pgfqpoint{0.000000in}{0.020833in}}{%
\pgfpathmoveto{\pgfqpoint{0.000000in}{0.000000in}}%
\pgfpathlineto{\pgfqpoint{0.000000in}{0.020833in}}%
\pgfusepath{stroke,fill}%
}%
\begin{pgfscope}%
\pgfsys@transformshift{4.512142in}{0.422992in}%
\pgfsys@useobject{currentmarker}{}%
\end{pgfscope}%
\end{pgfscope}%
\begin{pgfscope}%
\pgfsetbuttcap%
\pgfsetroundjoin%
\definecolor{currentfill}{rgb}{0.000000,0.000000,0.000000}%
\pgfsetfillcolor{currentfill}%
\pgfsetlinewidth{0.501875pt}%
\definecolor{currentstroke}{rgb}{0.000000,0.000000,0.000000}%
\pgfsetstrokecolor{currentstroke}%
\pgfsetdash{}{0pt}%
\pgfsys@defobject{currentmarker}{\pgfqpoint{0.000000in}{-0.020833in}}{\pgfqpoint{0.000000in}{0.000000in}}{%
\pgfpathmoveto{\pgfqpoint{0.000000in}{0.000000in}}%
\pgfpathlineto{\pgfqpoint{0.000000in}{-0.020833in}}%
\pgfusepath{stroke,fill}%
}%
\begin{pgfscope}%
\pgfsys@transformshift{4.512142in}{4.374193in}%
\pgfsys@useobject{currentmarker}{}%
\end{pgfscope}%
\end{pgfscope}%
\begin{pgfscope}%
\pgfsetbuttcap%
\pgfsetroundjoin%
\definecolor{currentfill}{rgb}{0.000000,0.000000,0.000000}%
\pgfsetfillcolor{currentfill}%
\pgfsetlinewidth{0.501875pt}%
\definecolor{currentstroke}{rgb}{0.000000,0.000000,0.000000}%
\pgfsetstrokecolor{currentstroke}%
\pgfsetdash{}{0pt}%
\pgfsys@defobject{currentmarker}{\pgfqpoint{0.000000in}{0.000000in}}{\pgfqpoint{0.000000in}{0.020833in}}{%
\pgfpathmoveto{\pgfqpoint{0.000000in}{0.000000in}}%
\pgfpathlineto{\pgfqpoint{0.000000in}{0.020833in}}%
\pgfusepath{stroke,fill}%
}%
\begin{pgfscope}%
\pgfsys@transformshift{4.624818in}{0.422992in}%
\pgfsys@useobject{currentmarker}{}%
\end{pgfscope}%
\end{pgfscope}%
\begin{pgfscope}%
\pgfsetbuttcap%
\pgfsetroundjoin%
\definecolor{currentfill}{rgb}{0.000000,0.000000,0.000000}%
\pgfsetfillcolor{currentfill}%
\pgfsetlinewidth{0.501875pt}%
\definecolor{currentstroke}{rgb}{0.000000,0.000000,0.000000}%
\pgfsetstrokecolor{currentstroke}%
\pgfsetdash{}{0pt}%
\pgfsys@defobject{currentmarker}{\pgfqpoint{0.000000in}{-0.020833in}}{\pgfqpoint{0.000000in}{0.000000in}}{%
\pgfpathmoveto{\pgfqpoint{0.000000in}{0.000000in}}%
\pgfpathlineto{\pgfqpoint{0.000000in}{-0.020833in}}%
\pgfusepath{stroke,fill}%
}%
\begin{pgfscope}%
\pgfsys@transformshift{4.624818in}{4.374193in}%
\pgfsys@useobject{currentmarker}{}%
\end{pgfscope}%
\end{pgfscope}%
\begin{pgfscope}%
\pgfsetbuttcap%
\pgfsetroundjoin%
\definecolor{currentfill}{rgb}{0.000000,0.000000,0.000000}%
\pgfsetfillcolor{currentfill}%
\pgfsetlinewidth{0.501875pt}%
\definecolor{currentstroke}{rgb}{0.000000,0.000000,0.000000}%
\pgfsetstrokecolor{currentstroke}%
\pgfsetdash{}{0pt}%
\pgfsys@defobject{currentmarker}{\pgfqpoint{0.000000in}{0.000000in}}{\pgfqpoint{0.000000in}{0.020833in}}{%
\pgfpathmoveto{\pgfqpoint{0.000000in}{0.000000in}}%
\pgfpathlineto{\pgfqpoint{0.000000in}{0.020833in}}%
\pgfusepath{stroke,fill}%
}%
\begin{pgfscope}%
\pgfsys@transformshift{4.850171in}{0.422992in}%
\pgfsys@useobject{currentmarker}{}%
\end{pgfscope}%
\end{pgfscope}%
\begin{pgfscope}%
\pgfsetbuttcap%
\pgfsetroundjoin%
\definecolor{currentfill}{rgb}{0.000000,0.000000,0.000000}%
\pgfsetfillcolor{currentfill}%
\pgfsetlinewidth{0.501875pt}%
\definecolor{currentstroke}{rgb}{0.000000,0.000000,0.000000}%
\pgfsetstrokecolor{currentstroke}%
\pgfsetdash{}{0pt}%
\pgfsys@defobject{currentmarker}{\pgfqpoint{0.000000in}{-0.020833in}}{\pgfqpoint{0.000000in}{0.000000in}}{%
\pgfpathmoveto{\pgfqpoint{0.000000in}{0.000000in}}%
\pgfpathlineto{\pgfqpoint{0.000000in}{-0.020833in}}%
\pgfusepath{stroke,fill}%
}%
\begin{pgfscope}%
\pgfsys@transformshift{4.850171in}{4.374193in}%
\pgfsys@useobject{currentmarker}{}%
\end{pgfscope}%
\end{pgfscope}%
\begin{pgfscope}%
\pgfsetbuttcap%
\pgfsetroundjoin%
\definecolor{currentfill}{rgb}{0.000000,0.000000,0.000000}%
\pgfsetfillcolor{currentfill}%
\pgfsetlinewidth{0.501875pt}%
\definecolor{currentstroke}{rgb}{0.000000,0.000000,0.000000}%
\pgfsetstrokecolor{currentstroke}%
\pgfsetdash{}{0pt}%
\pgfsys@defobject{currentmarker}{\pgfqpoint{0.000000in}{0.000000in}}{\pgfqpoint{0.000000in}{0.020833in}}{%
\pgfpathmoveto{\pgfqpoint{0.000000in}{0.000000in}}%
\pgfpathlineto{\pgfqpoint{0.000000in}{0.020833in}}%
\pgfusepath{stroke,fill}%
}%
\begin{pgfscope}%
\pgfsys@transformshift{4.962847in}{0.422992in}%
\pgfsys@useobject{currentmarker}{}%
\end{pgfscope}%
\end{pgfscope}%
\begin{pgfscope}%
\pgfsetbuttcap%
\pgfsetroundjoin%
\definecolor{currentfill}{rgb}{0.000000,0.000000,0.000000}%
\pgfsetfillcolor{currentfill}%
\pgfsetlinewidth{0.501875pt}%
\definecolor{currentstroke}{rgb}{0.000000,0.000000,0.000000}%
\pgfsetstrokecolor{currentstroke}%
\pgfsetdash{}{0pt}%
\pgfsys@defobject{currentmarker}{\pgfqpoint{0.000000in}{-0.020833in}}{\pgfqpoint{0.000000in}{0.000000in}}{%
\pgfpathmoveto{\pgfqpoint{0.000000in}{0.000000in}}%
\pgfpathlineto{\pgfqpoint{0.000000in}{-0.020833in}}%
\pgfusepath{stroke,fill}%
}%
\begin{pgfscope}%
\pgfsys@transformshift{4.962847in}{4.374193in}%
\pgfsys@useobject{currentmarker}{}%
\end{pgfscope}%
\end{pgfscope}%
\begin{pgfscope}%
\pgfsetbuttcap%
\pgfsetroundjoin%
\definecolor{currentfill}{rgb}{0.000000,0.000000,0.000000}%
\pgfsetfillcolor{currentfill}%
\pgfsetlinewidth{0.501875pt}%
\definecolor{currentstroke}{rgb}{0.000000,0.000000,0.000000}%
\pgfsetstrokecolor{currentstroke}%
\pgfsetdash{}{0pt}%
\pgfsys@defobject{currentmarker}{\pgfqpoint{0.000000in}{0.000000in}}{\pgfqpoint{0.000000in}{0.020833in}}{%
\pgfpathmoveto{\pgfqpoint{0.000000in}{0.000000in}}%
\pgfpathlineto{\pgfqpoint{0.000000in}{0.020833in}}%
\pgfusepath{stroke,fill}%
}%
\begin{pgfscope}%
\pgfsys@transformshift{5.075524in}{0.422992in}%
\pgfsys@useobject{currentmarker}{}%
\end{pgfscope}%
\end{pgfscope}%
\begin{pgfscope}%
\pgfsetbuttcap%
\pgfsetroundjoin%
\definecolor{currentfill}{rgb}{0.000000,0.000000,0.000000}%
\pgfsetfillcolor{currentfill}%
\pgfsetlinewidth{0.501875pt}%
\definecolor{currentstroke}{rgb}{0.000000,0.000000,0.000000}%
\pgfsetstrokecolor{currentstroke}%
\pgfsetdash{}{0pt}%
\pgfsys@defobject{currentmarker}{\pgfqpoint{0.000000in}{-0.020833in}}{\pgfqpoint{0.000000in}{0.000000in}}{%
\pgfpathmoveto{\pgfqpoint{0.000000in}{0.000000in}}%
\pgfpathlineto{\pgfqpoint{0.000000in}{-0.020833in}}%
\pgfusepath{stroke,fill}%
}%
\begin{pgfscope}%
\pgfsys@transformshift{5.075524in}{4.374193in}%
\pgfsys@useobject{currentmarker}{}%
\end{pgfscope}%
\end{pgfscope}%
\begin{pgfscope}%
\pgfsetbuttcap%
\pgfsetroundjoin%
\definecolor{currentfill}{rgb}{0.000000,0.000000,0.000000}%
\pgfsetfillcolor{currentfill}%
\pgfsetlinewidth{0.501875pt}%
\definecolor{currentstroke}{rgb}{0.000000,0.000000,0.000000}%
\pgfsetstrokecolor{currentstroke}%
\pgfsetdash{}{0pt}%
\pgfsys@defobject{currentmarker}{\pgfqpoint{0.000000in}{0.000000in}}{\pgfqpoint{0.000000in}{0.020833in}}{%
\pgfpathmoveto{\pgfqpoint{0.000000in}{0.000000in}}%
\pgfpathlineto{\pgfqpoint{0.000000in}{0.020833in}}%
\pgfusepath{stroke,fill}%
}%
\begin{pgfscope}%
\pgfsys@transformshift{5.300877in}{0.422992in}%
\pgfsys@useobject{currentmarker}{}%
\end{pgfscope}%
\end{pgfscope}%
\begin{pgfscope}%
\pgfsetbuttcap%
\pgfsetroundjoin%
\definecolor{currentfill}{rgb}{0.000000,0.000000,0.000000}%
\pgfsetfillcolor{currentfill}%
\pgfsetlinewidth{0.501875pt}%
\definecolor{currentstroke}{rgb}{0.000000,0.000000,0.000000}%
\pgfsetstrokecolor{currentstroke}%
\pgfsetdash{}{0pt}%
\pgfsys@defobject{currentmarker}{\pgfqpoint{0.000000in}{-0.020833in}}{\pgfqpoint{0.000000in}{0.000000in}}{%
\pgfpathmoveto{\pgfqpoint{0.000000in}{0.000000in}}%
\pgfpathlineto{\pgfqpoint{0.000000in}{-0.020833in}}%
\pgfusepath{stroke,fill}%
}%
\begin{pgfscope}%
\pgfsys@transformshift{5.300877in}{4.374193in}%
\pgfsys@useobject{currentmarker}{}%
\end{pgfscope}%
\end{pgfscope}%
\begin{pgfscope}%
\pgfsetbuttcap%
\pgfsetroundjoin%
\definecolor{currentfill}{rgb}{0.000000,0.000000,0.000000}%
\pgfsetfillcolor{currentfill}%
\pgfsetlinewidth{0.501875pt}%
\definecolor{currentstroke}{rgb}{0.000000,0.000000,0.000000}%
\pgfsetstrokecolor{currentstroke}%
\pgfsetdash{}{0pt}%
\pgfsys@defobject{currentmarker}{\pgfqpoint{0.000000in}{0.000000in}}{\pgfqpoint{0.000000in}{0.020833in}}{%
\pgfpathmoveto{\pgfqpoint{0.000000in}{0.000000in}}%
\pgfpathlineto{\pgfqpoint{0.000000in}{0.020833in}}%
\pgfusepath{stroke,fill}%
}%
\begin{pgfscope}%
\pgfsys@transformshift{5.413553in}{0.422992in}%
\pgfsys@useobject{currentmarker}{}%
\end{pgfscope}%
\end{pgfscope}%
\begin{pgfscope}%
\pgfsetbuttcap%
\pgfsetroundjoin%
\definecolor{currentfill}{rgb}{0.000000,0.000000,0.000000}%
\pgfsetfillcolor{currentfill}%
\pgfsetlinewidth{0.501875pt}%
\definecolor{currentstroke}{rgb}{0.000000,0.000000,0.000000}%
\pgfsetstrokecolor{currentstroke}%
\pgfsetdash{}{0pt}%
\pgfsys@defobject{currentmarker}{\pgfqpoint{0.000000in}{-0.020833in}}{\pgfqpoint{0.000000in}{0.000000in}}{%
\pgfpathmoveto{\pgfqpoint{0.000000in}{0.000000in}}%
\pgfpathlineto{\pgfqpoint{0.000000in}{-0.020833in}}%
\pgfusepath{stroke,fill}%
}%
\begin{pgfscope}%
\pgfsys@transformshift{5.413553in}{4.374193in}%
\pgfsys@useobject{currentmarker}{}%
\end{pgfscope}%
\end{pgfscope}%
\begin{pgfscope}%
\pgfsetbuttcap%
\pgfsetroundjoin%
\definecolor{currentfill}{rgb}{0.000000,0.000000,0.000000}%
\pgfsetfillcolor{currentfill}%
\pgfsetlinewidth{0.501875pt}%
\definecolor{currentstroke}{rgb}{0.000000,0.000000,0.000000}%
\pgfsetstrokecolor{currentstroke}%
\pgfsetdash{}{0pt}%
\pgfsys@defobject{currentmarker}{\pgfqpoint{0.000000in}{0.000000in}}{\pgfqpoint{0.000000in}{0.020833in}}{%
\pgfpathmoveto{\pgfqpoint{0.000000in}{0.000000in}}%
\pgfpathlineto{\pgfqpoint{0.000000in}{0.020833in}}%
\pgfusepath{stroke,fill}%
}%
\begin{pgfscope}%
\pgfsys@transformshift{5.526230in}{0.422992in}%
\pgfsys@useobject{currentmarker}{}%
\end{pgfscope}%
\end{pgfscope}%
\begin{pgfscope}%
\pgfsetbuttcap%
\pgfsetroundjoin%
\definecolor{currentfill}{rgb}{0.000000,0.000000,0.000000}%
\pgfsetfillcolor{currentfill}%
\pgfsetlinewidth{0.501875pt}%
\definecolor{currentstroke}{rgb}{0.000000,0.000000,0.000000}%
\pgfsetstrokecolor{currentstroke}%
\pgfsetdash{}{0pt}%
\pgfsys@defobject{currentmarker}{\pgfqpoint{0.000000in}{-0.020833in}}{\pgfqpoint{0.000000in}{0.000000in}}{%
\pgfpathmoveto{\pgfqpoint{0.000000in}{0.000000in}}%
\pgfpathlineto{\pgfqpoint{0.000000in}{-0.020833in}}%
\pgfusepath{stroke,fill}%
}%
\begin{pgfscope}%
\pgfsys@transformshift{5.526230in}{4.374193in}%
\pgfsys@useobject{currentmarker}{}%
\end{pgfscope}%
\end{pgfscope}%
\begin{pgfscope}%
\pgfsetbuttcap%
\pgfsetroundjoin%
\definecolor{currentfill}{rgb}{0.000000,0.000000,0.000000}%
\pgfsetfillcolor{currentfill}%
\pgfsetlinewidth{0.501875pt}%
\definecolor{currentstroke}{rgb}{0.000000,0.000000,0.000000}%
\pgfsetstrokecolor{currentstroke}%
\pgfsetdash{}{0pt}%
\pgfsys@defobject{currentmarker}{\pgfqpoint{0.000000in}{0.000000in}}{\pgfqpoint{0.000000in}{0.020833in}}{%
\pgfpathmoveto{\pgfqpoint{0.000000in}{0.000000in}}%
\pgfpathlineto{\pgfqpoint{0.000000in}{0.020833in}}%
\pgfusepath{stroke,fill}%
}%
\begin{pgfscope}%
\pgfsys@transformshift{5.638906in}{0.422992in}%
\pgfsys@useobject{currentmarker}{}%
\end{pgfscope}%
\end{pgfscope}%
\begin{pgfscope}%
\pgfsetbuttcap%
\pgfsetroundjoin%
\definecolor{currentfill}{rgb}{0.000000,0.000000,0.000000}%
\pgfsetfillcolor{currentfill}%
\pgfsetlinewidth{0.501875pt}%
\definecolor{currentstroke}{rgb}{0.000000,0.000000,0.000000}%
\pgfsetstrokecolor{currentstroke}%
\pgfsetdash{}{0pt}%
\pgfsys@defobject{currentmarker}{\pgfqpoint{0.000000in}{-0.020833in}}{\pgfqpoint{0.000000in}{0.000000in}}{%
\pgfpathmoveto{\pgfqpoint{0.000000in}{0.000000in}}%
\pgfpathlineto{\pgfqpoint{0.000000in}{-0.020833in}}%
\pgfusepath{stroke,fill}%
}%
\begin{pgfscope}%
\pgfsys@transformshift{5.638906in}{4.374193in}%
\pgfsys@useobject{currentmarker}{}%
\end{pgfscope}%
\end{pgfscope}%
\begin{pgfscope}%
\definecolor{textcolor}{rgb}{0.000000,0.000000,0.000000}%
\pgfsetstrokecolor{textcolor}%
\pgfsetfillcolor{textcolor}%
\pgftext[x=4.523409in,y=0.184413in,,top]{\color{textcolor}\rmfamily\fontsize{10.000000}{12.000000}\selectfont \(\displaystyle K\)}%
\end{pgfscope}%
\begin{pgfscope}%
\pgfsetbuttcap%
\pgfsetroundjoin%
\definecolor{currentfill}{rgb}{0.000000,0.000000,0.000000}%
\pgfsetfillcolor{currentfill}%
\pgfsetlinewidth{0.501875pt}%
\definecolor{currentstroke}{rgb}{0.000000,0.000000,0.000000}%
\pgfsetstrokecolor{currentstroke}%
\pgfsetdash{}{0pt}%
\pgfsys@defobject{currentmarker}{\pgfqpoint{0.000000in}{0.000000in}}{\pgfqpoint{0.041667in}{0.000000in}}{%
\pgfpathmoveto{\pgfqpoint{0.000000in}{0.000000in}}%
\pgfpathlineto{\pgfqpoint{0.041667in}{0.000000in}}%
\pgfusepath{stroke,fill}%
}%
\begin{pgfscope}%
\pgfsys@transformshift{3.385377in}{0.727021in}%
\pgfsys@useobject{currentmarker}{}%
\end{pgfscope}%
\end{pgfscope}%
\begin{pgfscope}%
\pgfsetbuttcap%
\pgfsetroundjoin%
\definecolor{currentfill}{rgb}{0.000000,0.000000,0.000000}%
\pgfsetfillcolor{currentfill}%
\pgfsetlinewidth{0.501875pt}%
\definecolor{currentstroke}{rgb}{0.000000,0.000000,0.000000}%
\pgfsetstrokecolor{currentstroke}%
\pgfsetdash{}{0pt}%
\pgfsys@defobject{currentmarker}{\pgfqpoint{-0.041667in}{0.000000in}}{\pgfqpoint{-0.000000in}{0.000000in}}{%
\pgfpathmoveto{\pgfqpoint{-0.000000in}{0.000000in}}%
\pgfpathlineto{\pgfqpoint{-0.041667in}{0.000000in}}%
\pgfusepath{stroke,fill}%
}%
\begin{pgfscope}%
\pgfsys@transformshift{5.661441in}{0.727021in}%
\pgfsys@useobject{currentmarker}{}%
\end{pgfscope}%
\end{pgfscope}%
\begin{pgfscope}%
\definecolor{textcolor}{rgb}{0.000000,0.000000,0.000000}%
\pgfsetstrokecolor{textcolor}%
\pgfsetfillcolor{textcolor}%
\pgftext[x=3.159296in, y=0.674259in, left, base]{\color{textcolor}\rmfamily\fontsize{10.000000}{12.000000}\selectfont \(\displaystyle {0.1}\)}%
\end{pgfscope}%
\begin{pgfscope}%
\pgfsetbuttcap%
\pgfsetroundjoin%
\definecolor{currentfill}{rgb}{0.000000,0.000000,0.000000}%
\pgfsetfillcolor{currentfill}%
\pgfsetlinewidth{0.501875pt}%
\definecolor{currentstroke}{rgb}{0.000000,0.000000,0.000000}%
\pgfsetstrokecolor{currentstroke}%
\pgfsetdash{}{0pt}%
\pgfsys@defobject{currentmarker}{\pgfqpoint{0.000000in}{0.000000in}}{\pgfqpoint{0.041667in}{0.000000in}}{%
\pgfpathmoveto{\pgfqpoint{0.000000in}{0.000000in}}%
\pgfpathlineto{\pgfqpoint{0.041667in}{0.000000in}}%
\pgfusepath{stroke,fill}%
}%
\begin{pgfscope}%
\pgfsys@transformshift{3.385377in}{1.284695in}%
\pgfsys@useobject{currentmarker}{}%
\end{pgfscope}%
\end{pgfscope}%
\begin{pgfscope}%
\pgfsetbuttcap%
\pgfsetroundjoin%
\definecolor{currentfill}{rgb}{0.000000,0.000000,0.000000}%
\pgfsetfillcolor{currentfill}%
\pgfsetlinewidth{0.501875pt}%
\definecolor{currentstroke}{rgb}{0.000000,0.000000,0.000000}%
\pgfsetstrokecolor{currentstroke}%
\pgfsetdash{}{0pt}%
\pgfsys@defobject{currentmarker}{\pgfqpoint{-0.041667in}{0.000000in}}{\pgfqpoint{-0.000000in}{0.000000in}}{%
\pgfpathmoveto{\pgfqpoint{-0.000000in}{0.000000in}}%
\pgfpathlineto{\pgfqpoint{-0.041667in}{0.000000in}}%
\pgfusepath{stroke,fill}%
}%
\begin{pgfscope}%
\pgfsys@transformshift{5.661441in}{1.284695in}%
\pgfsys@useobject{currentmarker}{}%
\end{pgfscope}%
\end{pgfscope}%
\begin{pgfscope}%
\definecolor{textcolor}{rgb}{0.000000,0.000000,0.000000}%
\pgfsetstrokecolor{textcolor}%
\pgfsetfillcolor{textcolor}%
\pgftext[x=3.159296in, y=1.231933in, left, base]{\color{textcolor}\rmfamily\fontsize{10.000000}{12.000000}\selectfont \(\displaystyle {0.2}\)}%
\end{pgfscope}%
\begin{pgfscope}%
\pgfsetbuttcap%
\pgfsetroundjoin%
\definecolor{currentfill}{rgb}{0.000000,0.000000,0.000000}%
\pgfsetfillcolor{currentfill}%
\pgfsetlinewidth{0.501875pt}%
\definecolor{currentstroke}{rgb}{0.000000,0.000000,0.000000}%
\pgfsetstrokecolor{currentstroke}%
\pgfsetdash{}{0pt}%
\pgfsys@defobject{currentmarker}{\pgfqpoint{0.000000in}{0.000000in}}{\pgfqpoint{0.041667in}{0.000000in}}{%
\pgfpathmoveto{\pgfqpoint{0.000000in}{0.000000in}}%
\pgfpathlineto{\pgfqpoint{0.041667in}{0.000000in}}%
\pgfusepath{stroke,fill}%
}%
\begin{pgfscope}%
\pgfsys@transformshift{3.385377in}{1.842369in}%
\pgfsys@useobject{currentmarker}{}%
\end{pgfscope}%
\end{pgfscope}%
\begin{pgfscope}%
\pgfsetbuttcap%
\pgfsetroundjoin%
\definecolor{currentfill}{rgb}{0.000000,0.000000,0.000000}%
\pgfsetfillcolor{currentfill}%
\pgfsetlinewidth{0.501875pt}%
\definecolor{currentstroke}{rgb}{0.000000,0.000000,0.000000}%
\pgfsetstrokecolor{currentstroke}%
\pgfsetdash{}{0pt}%
\pgfsys@defobject{currentmarker}{\pgfqpoint{-0.041667in}{0.000000in}}{\pgfqpoint{-0.000000in}{0.000000in}}{%
\pgfpathmoveto{\pgfqpoint{-0.000000in}{0.000000in}}%
\pgfpathlineto{\pgfqpoint{-0.041667in}{0.000000in}}%
\pgfusepath{stroke,fill}%
}%
\begin{pgfscope}%
\pgfsys@transformshift{5.661441in}{1.842369in}%
\pgfsys@useobject{currentmarker}{}%
\end{pgfscope}%
\end{pgfscope}%
\begin{pgfscope}%
\definecolor{textcolor}{rgb}{0.000000,0.000000,0.000000}%
\pgfsetstrokecolor{textcolor}%
\pgfsetfillcolor{textcolor}%
\pgftext[x=3.159296in, y=1.789607in, left, base]{\color{textcolor}\rmfamily\fontsize{10.000000}{12.000000}\selectfont \(\displaystyle {0.3}\)}%
\end{pgfscope}%
\begin{pgfscope}%
\pgfsetbuttcap%
\pgfsetroundjoin%
\definecolor{currentfill}{rgb}{0.000000,0.000000,0.000000}%
\pgfsetfillcolor{currentfill}%
\pgfsetlinewidth{0.501875pt}%
\definecolor{currentstroke}{rgb}{0.000000,0.000000,0.000000}%
\pgfsetstrokecolor{currentstroke}%
\pgfsetdash{}{0pt}%
\pgfsys@defobject{currentmarker}{\pgfqpoint{0.000000in}{0.000000in}}{\pgfqpoint{0.041667in}{0.000000in}}{%
\pgfpathmoveto{\pgfqpoint{0.000000in}{0.000000in}}%
\pgfpathlineto{\pgfqpoint{0.041667in}{0.000000in}}%
\pgfusepath{stroke,fill}%
}%
\begin{pgfscope}%
\pgfsys@transformshift{3.385377in}{2.400043in}%
\pgfsys@useobject{currentmarker}{}%
\end{pgfscope}%
\end{pgfscope}%
\begin{pgfscope}%
\pgfsetbuttcap%
\pgfsetroundjoin%
\definecolor{currentfill}{rgb}{0.000000,0.000000,0.000000}%
\pgfsetfillcolor{currentfill}%
\pgfsetlinewidth{0.501875pt}%
\definecolor{currentstroke}{rgb}{0.000000,0.000000,0.000000}%
\pgfsetstrokecolor{currentstroke}%
\pgfsetdash{}{0pt}%
\pgfsys@defobject{currentmarker}{\pgfqpoint{-0.041667in}{0.000000in}}{\pgfqpoint{-0.000000in}{0.000000in}}{%
\pgfpathmoveto{\pgfqpoint{-0.000000in}{0.000000in}}%
\pgfpathlineto{\pgfqpoint{-0.041667in}{0.000000in}}%
\pgfusepath{stroke,fill}%
}%
\begin{pgfscope}%
\pgfsys@transformshift{5.661441in}{2.400043in}%
\pgfsys@useobject{currentmarker}{}%
\end{pgfscope}%
\end{pgfscope}%
\begin{pgfscope}%
\definecolor{textcolor}{rgb}{0.000000,0.000000,0.000000}%
\pgfsetstrokecolor{textcolor}%
\pgfsetfillcolor{textcolor}%
\pgftext[x=3.159296in, y=2.347282in, left, base]{\color{textcolor}\rmfamily\fontsize{10.000000}{12.000000}\selectfont \(\displaystyle {0.4}\)}%
\end{pgfscope}%
\begin{pgfscope}%
\pgfsetbuttcap%
\pgfsetroundjoin%
\definecolor{currentfill}{rgb}{0.000000,0.000000,0.000000}%
\pgfsetfillcolor{currentfill}%
\pgfsetlinewidth{0.501875pt}%
\definecolor{currentstroke}{rgb}{0.000000,0.000000,0.000000}%
\pgfsetstrokecolor{currentstroke}%
\pgfsetdash{}{0pt}%
\pgfsys@defobject{currentmarker}{\pgfqpoint{0.000000in}{0.000000in}}{\pgfqpoint{0.041667in}{0.000000in}}{%
\pgfpathmoveto{\pgfqpoint{0.000000in}{0.000000in}}%
\pgfpathlineto{\pgfqpoint{0.041667in}{0.000000in}}%
\pgfusepath{stroke,fill}%
}%
\begin{pgfscope}%
\pgfsys@transformshift{3.385377in}{2.957717in}%
\pgfsys@useobject{currentmarker}{}%
\end{pgfscope}%
\end{pgfscope}%
\begin{pgfscope}%
\pgfsetbuttcap%
\pgfsetroundjoin%
\definecolor{currentfill}{rgb}{0.000000,0.000000,0.000000}%
\pgfsetfillcolor{currentfill}%
\pgfsetlinewidth{0.501875pt}%
\definecolor{currentstroke}{rgb}{0.000000,0.000000,0.000000}%
\pgfsetstrokecolor{currentstroke}%
\pgfsetdash{}{0pt}%
\pgfsys@defobject{currentmarker}{\pgfqpoint{-0.041667in}{0.000000in}}{\pgfqpoint{-0.000000in}{0.000000in}}{%
\pgfpathmoveto{\pgfqpoint{-0.000000in}{0.000000in}}%
\pgfpathlineto{\pgfqpoint{-0.041667in}{0.000000in}}%
\pgfusepath{stroke,fill}%
}%
\begin{pgfscope}%
\pgfsys@transformshift{5.661441in}{2.957717in}%
\pgfsys@useobject{currentmarker}{}%
\end{pgfscope}%
\end{pgfscope}%
\begin{pgfscope}%
\definecolor{textcolor}{rgb}{0.000000,0.000000,0.000000}%
\pgfsetstrokecolor{textcolor}%
\pgfsetfillcolor{textcolor}%
\pgftext[x=3.159296in, y=2.904956in, left, base]{\color{textcolor}\rmfamily\fontsize{10.000000}{12.000000}\selectfont \(\displaystyle {0.5}\)}%
\end{pgfscope}%
\begin{pgfscope}%
\pgfsetbuttcap%
\pgfsetroundjoin%
\definecolor{currentfill}{rgb}{0.000000,0.000000,0.000000}%
\pgfsetfillcolor{currentfill}%
\pgfsetlinewidth{0.501875pt}%
\definecolor{currentstroke}{rgb}{0.000000,0.000000,0.000000}%
\pgfsetstrokecolor{currentstroke}%
\pgfsetdash{}{0pt}%
\pgfsys@defobject{currentmarker}{\pgfqpoint{0.000000in}{0.000000in}}{\pgfqpoint{0.041667in}{0.000000in}}{%
\pgfpathmoveto{\pgfqpoint{0.000000in}{0.000000in}}%
\pgfpathlineto{\pgfqpoint{0.041667in}{0.000000in}}%
\pgfusepath{stroke,fill}%
}%
\begin{pgfscope}%
\pgfsys@transformshift{3.385377in}{3.515391in}%
\pgfsys@useobject{currentmarker}{}%
\end{pgfscope}%
\end{pgfscope}%
\begin{pgfscope}%
\pgfsetbuttcap%
\pgfsetroundjoin%
\definecolor{currentfill}{rgb}{0.000000,0.000000,0.000000}%
\pgfsetfillcolor{currentfill}%
\pgfsetlinewidth{0.501875pt}%
\definecolor{currentstroke}{rgb}{0.000000,0.000000,0.000000}%
\pgfsetstrokecolor{currentstroke}%
\pgfsetdash{}{0pt}%
\pgfsys@defobject{currentmarker}{\pgfqpoint{-0.041667in}{0.000000in}}{\pgfqpoint{-0.000000in}{0.000000in}}{%
\pgfpathmoveto{\pgfqpoint{-0.000000in}{0.000000in}}%
\pgfpathlineto{\pgfqpoint{-0.041667in}{0.000000in}}%
\pgfusepath{stroke,fill}%
}%
\begin{pgfscope}%
\pgfsys@transformshift{5.661441in}{3.515391in}%
\pgfsys@useobject{currentmarker}{}%
\end{pgfscope}%
\end{pgfscope}%
\begin{pgfscope}%
\definecolor{textcolor}{rgb}{0.000000,0.000000,0.000000}%
\pgfsetstrokecolor{textcolor}%
\pgfsetfillcolor{textcolor}%
\pgftext[x=3.159296in, y=3.462630in, left, base]{\color{textcolor}\rmfamily\fontsize{10.000000}{12.000000}\selectfont \(\displaystyle {0.6}\)}%
\end{pgfscope}%
\begin{pgfscope}%
\pgfsetbuttcap%
\pgfsetroundjoin%
\definecolor{currentfill}{rgb}{0.000000,0.000000,0.000000}%
\pgfsetfillcolor{currentfill}%
\pgfsetlinewidth{0.501875pt}%
\definecolor{currentstroke}{rgb}{0.000000,0.000000,0.000000}%
\pgfsetstrokecolor{currentstroke}%
\pgfsetdash{}{0pt}%
\pgfsys@defobject{currentmarker}{\pgfqpoint{0.000000in}{0.000000in}}{\pgfqpoint{0.041667in}{0.000000in}}{%
\pgfpathmoveto{\pgfqpoint{0.000000in}{0.000000in}}%
\pgfpathlineto{\pgfqpoint{0.041667in}{0.000000in}}%
\pgfusepath{stroke,fill}%
}%
\begin{pgfscope}%
\pgfsys@transformshift{3.385377in}{4.073065in}%
\pgfsys@useobject{currentmarker}{}%
\end{pgfscope}%
\end{pgfscope}%
\begin{pgfscope}%
\pgfsetbuttcap%
\pgfsetroundjoin%
\definecolor{currentfill}{rgb}{0.000000,0.000000,0.000000}%
\pgfsetfillcolor{currentfill}%
\pgfsetlinewidth{0.501875pt}%
\definecolor{currentstroke}{rgb}{0.000000,0.000000,0.000000}%
\pgfsetstrokecolor{currentstroke}%
\pgfsetdash{}{0pt}%
\pgfsys@defobject{currentmarker}{\pgfqpoint{-0.041667in}{0.000000in}}{\pgfqpoint{-0.000000in}{0.000000in}}{%
\pgfpathmoveto{\pgfqpoint{-0.000000in}{0.000000in}}%
\pgfpathlineto{\pgfqpoint{-0.041667in}{0.000000in}}%
\pgfusepath{stroke,fill}%
}%
\begin{pgfscope}%
\pgfsys@transformshift{5.661441in}{4.073065in}%
\pgfsys@useobject{currentmarker}{}%
\end{pgfscope}%
\end{pgfscope}%
\begin{pgfscope}%
\definecolor{textcolor}{rgb}{0.000000,0.000000,0.000000}%
\pgfsetstrokecolor{textcolor}%
\pgfsetfillcolor{textcolor}%
\pgftext[x=3.159296in, y=4.020304in, left, base]{\color{textcolor}\rmfamily\fontsize{10.000000}{12.000000}\selectfont \(\displaystyle {0.7}\)}%
\end{pgfscope}%
\begin{pgfscope}%
\pgfsetbuttcap%
\pgfsetroundjoin%
\definecolor{currentfill}{rgb}{0.000000,0.000000,0.000000}%
\pgfsetfillcolor{currentfill}%
\pgfsetlinewidth{0.501875pt}%
\definecolor{currentstroke}{rgb}{0.000000,0.000000,0.000000}%
\pgfsetstrokecolor{currentstroke}%
\pgfsetdash{}{0pt}%
\pgfsys@defobject{currentmarker}{\pgfqpoint{0.000000in}{0.000000in}}{\pgfqpoint{0.020833in}{0.000000in}}{%
\pgfpathmoveto{\pgfqpoint{0.000000in}{0.000000in}}%
\pgfpathlineto{\pgfqpoint{0.020833in}{0.000000in}}%
\pgfusepath{stroke,fill}%
}%
\begin{pgfscope}%
\pgfsys@transformshift{3.385377in}{0.503951in}%
\pgfsys@useobject{currentmarker}{}%
\end{pgfscope}%
\end{pgfscope}%
\begin{pgfscope}%
\pgfsetbuttcap%
\pgfsetroundjoin%
\definecolor{currentfill}{rgb}{0.000000,0.000000,0.000000}%
\pgfsetfillcolor{currentfill}%
\pgfsetlinewidth{0.501875pt}%
\definecolor{currentstroke}{rgb}{0.000000,0.000000,0.000000}%
\pgfsetstrokecolor{currentstroke}%
\pgfsetdash{}{0pt}%
\pgfsys@defobject{currentmarker}{\pgfqpoint{-0.020833in}{0.000000in}}{\pgfqpoint{-0.000000in}{0.000000in}}{%
\pgfpathmoveto{\pgfqpoint{-0.000000in}{0.000000in}}%
\pgfpathlineto{\pgfqpoint{-0.020833in}{0.000000in}}%
\pgfusepath{stroke,fill}%
}%
\begin{pgfscope}%
\pgfsys@transformshift{5.661441in}{0.503951in}%
\pgfsys@useobject{currentmarker}{}%
\end{pgfscope}%
\end{pgfscope}%
\begin{pgfscope}%
\pgfsetbuttcap%
\pgfsetroundjoin%
\definecolor{currentfill}{rgb}{0.000000,0.000000,0.000000}%
\pgfsetfillcolor{currentfill}%
\pgfsetlinewidth{0.501875pt}%
\definecolor{currentstroke}{rgb}{0.000000,0.000000,0.000000}%
\pgfsetstrokecolor{currentstroke}%
\pgfsetdash{}{0pt}%
\pgfsys@defobject{currentmarker}{\pgfqpoint{0.000000in}{0.000000in}}{\pgfqpoint{0.020833in}{0.000000in}}{%
\pgfpathmoveto{\pgfqpoint{0.000000in}{0.000000in}}%
\pgfpathlineto{\pgfqpoint{0.020833in}{0.000000in}}%
\pgfusepath{stroke,fill}%
}%
\begin{pgfscope}%
\pgfsys@transformshift{3.385377in}{0.615486in}%
\pgfsys@useobject{currentmarker}{}%
\end{pgfscope}%
\end{pgfscope}%
\begin{pgfscope}%
\pgfsetbuttcap%
\pgfsetroundjoin%
\definecolor{currentfill}{rgb}{0.000000,0.000000,0.000000}%
\pgfsetfillcolor{currentfill}%
\pgfsetlinewidth{0.501875pt}%
\definecolor{currentstroke}{rgb}{0.000000,0.000000,0.000000}%
\pgfsetstrokecolor{currentstroke}%
\pgfsetdash{}{0pt}%
\pgfsys@defobject{currentmarker}{\pgfqpoint{-0.020833in}{0.000000in}}{\pgfqpoint{-0.000000in}{0.000000in}}{%
\pgfpathmoveto{\pgfqpoint{-0.000000in}{0.000000in}}%
\pgfpathlineto{\pgfqpoint{-0.020833in}{0.000000in}}%
\pgfusepath{stroke,fill}%
}%
\begin{pgfscope}%
\pgfsys@transformshift{5.661441in}{0.615486in}%
\pgfsys@useobject{currentmarker}{}%
\end{pgfscope}%
\end{pgfscope}%
\begin{pgfscope}%
\pgfsetbuttcap%
\pgfsetroundjoin%
\definecolor{currentfill}{rgb}{0.000000,0.000000,0.000000}%
\pgfsetfillcolor{currentfill}%
\pgfsetlinewidth{0.501875pt}%
\definecolor{currentstroke}{rgb}{0.000000,0.000000,0.000000}%
\pgfsetstrokecolor{currentstroke}%
\pgfsetdash{}{0pt}%
\pgfsys@defobject{currentmarker}{\pgfqpoint{0.000000in}{0.000000in}}{\pgfqpoint{0.020833in}{0.000000in}}{%
\pgfpathmoveto{\pgfqpoint{0.000000in}{0.000000in}}%
\pgfpathlineto{\pgfqpoint{0.020833in}{0.000000in}}%
\pgfusepath{stroke,fill}%
}%
\begin{pgfscope}%
\pgfsys@transformshift{3.385377in}{0.838556in}%
\pgfsys@useobject{currentmarker}{}%
\end{pgfscope}%
\end{pgfscope}%
\begin{pgfscope}%
\pgfsetbuttcap%
\pgfsetroundjoin%
\definecolor{currentfill}{rgb}{0.000000,0.000000,0.000000}%
\pgfsetfillcolor{currentfill}%
\pgfsetlinewidth{0.501875pt}%
\definecolor{currentstroke}{rgb}{0.000000,0.000000,0.000000}%
\pgfsetstrokecolor{currentstroke}%
\pgfsetdash{}{0pt}%
\pgfsys@defobject{currentmarker}{\pgfqpoint{-0.020833in}{0.000000in}}{\pgfqpoint{-0.000000in}{0.000000in}}{%
\pgfpathmoveto{\pgfqpoint{-0.000000in}{0.000000in}}%
\pgfpathlineto{\pgfqpoint{-0.020833in}{0.000000in}}%
\pgfusepath{stroke,fill}%
}%
\begin{pgfscope}%
\pgfsys@transformshift{5.661441in}{0.838556in}%
\pgfsys@useobject{currentmarker}{}%
\end{pgfscope}%
\end{pgfscope}%
\begin{pgfscope}%
\pgfsetbuttcap%
\pgfsetroundjoin%
\definecolor{currentfill}{rgb}{0.000000,0.000000,0.000000}%
\pgfsetfillcolor{currentfill}%
\pgfsetlinewidth{0.501875pt}%
\definecolor{currentstroke}{rgb}{0.000000,0.000000,0.000000}%
\pgfsetstrokecolor{currentstroke}%
\pgfsetdash{}{0pt}%
\pgfsys@defobject{currentmarker}{\pgfqpoint{0.000000in}{0.000000in}}{\pgfqpoint{0.020833in}{0.000000in}}{%
\pgfpathmoveto{\pgfqpoint{0.000000in}{0.000000in}}%
\pgfpathlineto{\pgfqpoint{0.020833in}{0.000000in}}%
\pgfusepath{stroke,fill}%
}%
\begin{pgfscope}%
\pgfsys@transformshift{3.385377in}{0.950090in}%
\pgfsys@useobject{currentmarker}{}%
\end{pgfscope}%
\end{pgfscope}%
\begin{pgfscope}%
\pgfsetbuttcap%
\pgfsetroundjoin%
\definecolor{currentfill}{rgb}{0.000000,0.000000,0.000000}%
\pgfsetfillcolor{currentfill}%
\pgfsetlinewidth{0.501875pt}%
\definecolor{currentstroke}{rgb}{0.000000,0.000000,0.000000}%
\pgfsetstrokecolor{currentstroke}%
\pgfsetdash{}{0pt}%
\pgfsys@defobject{currentmarker}{\pgfqpoint{-0.020833in}{0.000000in}}{\pgfqpoint{-0.000000in}{0.000000in}}{%
\pgfpathmoveto{\pgfqpoint{-0.000000in}{0.000000in}}%
\pgfpathlineto{\pgfqpoint{-0.020833in}{0.000000in}}%
\pgfusepath{stroke,fill}%
}%
\begin{pgfscope}%
\pgfsys@transformshift{5.661441in}{0.950090in}%
\pgfsys@useobject{currentmarker}{}%
\end{pgfscope}%
\end{pgfscope}%
\begin{pgfscope}%
\pgfsetbuttcap%
\pgfsetroundjoin%
\definecolor{currentfill}{rgb}{0.000000,0.000000,0.000000}%
\pgfsetfillcolor{currentfill}%
\pgfsetlinewidth{0.501875pt}%
\definecolor{currentstroke}{rgb}{0.000000,0.000000,0.000000}%
\pgfsetstrokecolor{currentstroke}%
\pgfsetdash{}{0pt}%
\pgfsys@defobject{currentmarker}{\pgfqpoint{0.000000in}{0.000000in}}{\pgfqpoint{0.020833in}{0.000000in}}{%
\pgfpathmoveto{\pgfqpoint{0.000000in}{0.000000in}}%
\pgfpathlineto{\pgfqpoint{0.020833in}{0.000000in}}%
\pgfusepath{stroke,fill}%
}%
\begin{pgfscope}%
\pgfsys@transformshift{3.385377in}{1.061625in}%
\pgfsys@useobject{currentmarker}{}%
\end{pgfscope}%
\end{pgfscope}%
\begin{pgfscope}%
\pgfsetbuttcap%
\pgfsetroundjoin%
\definecolor{currentfill}{rgb}{0.000000,0.000000,0.000000}%
\pgfsetfillcolor{currentfill}%
\pgfsetlinewidth{0.501875pt}%
\definecolor{currentstroke}{rgb}{0.000000,0.000000,0.000000}%
\pgfsetstrokecolor{currentstroke}%
\pgfsetdash{}{0pt}%
\pgfsys@defobject{currentmarker}{\pgfqpoint{-0.020833in}{0.000000in}}{\pgfqpoint{-0.000000in}{0.000000in}}{%
\pgfpathmoveto{\pgfqpoint{-0.000000in}{0.000000in}}%
\pgfpathlineto{\pgfqpoint{-0.020833in}{0.000000in}}%
\pgfusepath{stroke,fill}%
}%
\begin{pgfscope}%
\pgfsys@transformshift{5.661441in}{1.061625in}%
\pgfsys@useobject{currentmarker}{}%
\end{pgfscope}%
\end{pgfscope}%
\begin{pgfscope}%
\pgfsetbuttcap%
\pgfsetroundjoin%
\definecolor{currentfill}{rgb}{0.000000,0.000000,0.000000}%
\pgfsetfillcolor{currentfill}%
\pgfsetlinewidth{0.501875pt}%
\definecolor{currentstroke}{rgb}{0.000000,0.000000,0.000000}%
\pgfsetstrokecolor{currentstroke}%
\pgfsetdash{}{0pt}%
\pgfsys@defobject{currentmarker}{\pgfqpoint{0.000000in}{0.000000in}}{\pgfqpoint{0.020833in}{0.000000in}}{%
\pgfpathmoveto{\pgfqpoint{0.000000in}{0.000000in}}%
\pgfpathlineto{\pgfqpoint{0.020833in}{0.000000in}}%
\pgfusepath{stroke,fill}%
}%
\begin{pgfscope}%
\pgfsys@transformshift{3.385377in}{1.173160in}%
\pgfsys@useobject{currentmarker}{}%
\end{pgfscope}%
\end{pgfscope}%
\begin{pgfscope}%
\pgfsetbuttcap%
\pgfsetroundjoin%
\definecolor{currentfill}{rgb}{0.000000,0.000000,0.000000}%
\pgfsetfillcolor{currentfill}%
\pgfsetlinewidth{0.501875pt}%
\definecolor{currentstroke}{rgb}{0.000000,0.000000,0.000000}%
\pgfsetstrokecolor{currentstroke}%
\pgfsetdash{}{0pt}%
\pgfsys@defobject{currentmarker}{\pgfqpoint{-0.020833in}{0.000000in}}{\pgfqpoint{-0.000000in}{0.000000in}}{%
\pgfpathmoveto{\pgfqpoint{-0.000000in}{0.000000in}}%
\pgfpathlineto{\pgfqpoint{-0.020833in}{0.000000in}}%
\pgfusepath{stroke,fill}%
}%
\begin{pgfscope}%
\pgfsys@transformshift{5.661441in}{1.173160in}%
\pgfsys@useobject{currentmarker}{}%
\end{pgfscope}%
\end{pgfscope}%
\begin{pgfscope}%
\pgfsetbuttcap%
\pgfsetroundjoin%
\definecolor{currentfill}{rgb}{0.000000,0.000000,0.000000}%
\pgfsetfillcolor{currentfill}%
\pgfsetlinewidth{0.501875pt}%
\definecolor{currentstroke}{rgb}{0.000000,0.000000,0.000000}%
\pgfsetstrokecolor{currentstroke}%
\pgfsetdash{}{0pt}%
\pgfsys@defobject{currentmarker}{\pgfqpoint{0.000000in}{0.000000in}}{\pgfqpoint{0.020833in}{0.000000in}}{%
\pgfpathmoveto{\pgfqpoint{0.000000in}{0.000000in}}%
\pgfpathlineto{\pgfqpoint{0.020833in}{0.000000in}}%
\pgfusepath{stroke,fill}%
}%
\begin{pgfscope}%
\pgfsys@transformshift{3.385377in}{1.396230in}%
\pgfsys@useobject{currentmarker}{}%
\end{pgfscope}%
\end{pgfscope}%
\begin{pgfscope}%
\pgfsetbuttcap%
\pgfsetroundjoin%
\definecolor{currentfill}{rgb}{0.000000,0.000000,0.000000}%
\pgfsetfillcolor{currentfill}%
\pgfsetlinewidth{0.501875pt}%
\definecolor{currentstroke}{rgb}{0.000000,0.000000,0.000000}%
\pgfsetstrokecolor{currentstroke}%
\pgfsetdash{}{0pt}%
\pgfsys@defobject{currentmarker}{\pgfqpoint{-0.020833in}{0.000000in}}{\pgfqpoint{-0.000000in}{0.000000in}}{%
\pgfpathmoveto{\pgfqpoint{-0.000000in}{0.000000in}}%
\pgfpathlineto{\pgfqpoint{-0.020833in}{0.000000in}}%
\pgfusepath{stroke,fill}%
}%
\begin{pgfscope}%
\pgfsys@transformshift{5.661441in}{1.396230in}%
\pgfsys@useobject{currentmarker}{}%
\end{pgfscope}%
\end{pgfscope}%
\begin{pgfscope}%
\pgfsetbuttcap%
\pgfsetroundjoin%
\definecolor{currentfill}{rgb}{0.000000,0.000000,0.000000}%
\pgfsetfillcolor{currentfill}%
\pgfsetlinewidth{0.501875pt}%
\definecolor{currentstroke}{rgb}{0.000000,0.000000,0.000000}%
\pgfsetstrokecolor{currentstroke}%
\pgfsetdash{}{0pt}%
\pgfsys@defobject{currentmarker}{\pgfqpoint{0.000000in}{0.000000in}}{\pgfqpoint{0.020833in}{0.000000in}}{%
\pgfpathmoveto{\pgfqpoint{0.000000in}{0.000000in}}%
\pgfpathlineto{\pgfqpoint{0.020833in}{0.000000in}}%
\pgfusepath{stroke,fill}%
}%
\begin{pgfscope}%
\pgfsys@transformshift{3.385377in}{1.507765in}%
\pgfsys@useobject{currentmarker}{}%
\end{pgfscope}%
\end{pgfscope}%
\begin{pgfscope}%
\pgfsetbuttcap%
\pgfsetroundjoin%
\definecolor{currentfill}{rgb}{0.000000,0.000000,0.000000}%
\pgfsetfillcolor{currentfill}%
\pgfsetlinewidth{0.501875pt}%
\definecolor{currentstroke}{rgb}{0.000000,0.000000,0.000000}%
\pgfsetstrokecolor{currentstroke}%
\pgfsetdash{}{0pt}%
\pgfsys@defobject{currentmarker}{\pgfqpoint{-0.020833in}{0.000000in}}{\pgfqpoint{-0.000000in}{0.000000in}}{%
\pgfpathmoveto{\pgfqpoint{-0.000000in}{0.000000in}}%
\pgfpathlineto{\pgfqpoint{-0.020833in}{0.000000in}}%
\pgfusepath{stroke,fill}%
}%
\begin{pgfscope}%
\pgfsys@transformshift{5.661441in}{1.507765in}%
\pgfsys@useobject{currentmarker}{}%
\end{pgfscope}%
\end{pgfscope}%
\begin{pgfscope}%
\pgfsetbuttcap%
\pgfsetroundjoin%
\definecolor{currentfill}{rgb}{0.000000,0.000000,0.000000}%
\pgfsetfillcolor{currentfill}%
\pgfsetlinewidth{0.501875pt}%
\definecolor{currentstroke}{rgb}{0.000000,0.000000,0.000000}%
\pgfsetstrokecolor{currentstroke}%
\pgfsetdash{}{0pt}%
\pgfsys@defobject{currentmarker}{\pgfqpoint{0.000000in}{0.000000in}}{\pgfqpoint{0.020833in}{0.000000in}}{%
\pgfpathmoveto{\pgfqpoint{0.000000in}{0.000000in}}%
\pgfpathlineto{\pgfqpoint{0.020833in}{0.000000in}}%
\pgfusepath{stroke,fill}%
}%
\begin{pgfscope}%
\pgfsys@transformshift{3.385377in}{1.619299in}%
\pgfsys@useobject{currentmarker}{}%
\end{pgfscope}%
\end{pgfscope}%
\begin{pgfscope}%
\pgfsetbuttcap%
\pgfsetroundjoin%
\definecolor{currentfill}{rgb}{0.000000,0.000000,0.000000}%
\pgfsetfillcolor{currentfill}%
\pgfsetlinewidth{0.501875pt}%
\definecolor{currentstroke}{rgb}{0.000000,0.000000,0.000000}%
\pgfsetstrokecolor{currentstroke}%
\pgfsetdash{}{0pt}%
\pgfsys@defobject{currentmarker}{\pgfqpoint{-0.020833in}{0.000000in}}{\pgfqpoint{-0.000000in}{0.000000in}}{%
\pgfpathmoveto{\pgfqpoint{-0.000000in}{0.000000in}}%
\pgfpathlineto{\pgfqpoint{-0.020833in}{0.000000in}}%
\pgfusepath{stroke,fill}%
}%
\begin{pgfscope}%
\pgfsys@transformshift{5.661441in}{1.619299in}%
\pgfsys@useobject{currentmarker}{}%
\end{pgfscope}%
\end{pgfscope}%
\begin{pgfscope}%
\pgfsetbuttcap%
\pgfsetroundjoin%
\definecolor{currentfill}{rgb}{0.000000,0.000000,0.000000}%
\pgfsetfillcolor{currentfill}%
\pgfsetlinewidth{0.501875pt}%
\definecolor{currentstroke}{rgb}{0.000000,0.000000,0.000000}%
\pgfsetstrokecolor{currentstroke}%
\pgfsetdash{}{0pt}%
\pgfsys@defobject{currentmarker}{\pgfqpoint{0.000000in}{0.000000in}}{\pgfqpoint{0.020833in}{0.000000in}}{%
\pgfpathmoveto{\pgfqpoint{0.000000in}{0.000000in}}%
\pgfpathlineto{\pgfqpoint{0.020833in}{0.000000in}}%
\pgfusepath{stroke,fill}%
}%
\begin{pgfscope}%
\pgfsys@transformshift{3.385377in}{1.730834in}%
\pgfsys@useobject{currentmarker}{}%
\end{pgfscope}%
\end{pgfscope}%
\begin{pgfscope}%
\pgfsetbuttcap%
\pgfsetroundjoin%
\definecolor{currentfill}{rgb}{0.000000,0.000000,0.000000}%
\pgfsetfillcolor{currentfill}%
\pgfsetlinewidth{0.501875pt}%
\definecolor{currentstroke}{rgb}{0.000000,0.000000,0.000000}%
\pgfsetstrokecolor{currentstroke}%
\pgfsetdash{}{0pt}%
\pgfsys@defobject{currentmarker}{\pgfqpoint{-0.020833in}{0.000000in}}{\pgfqpoint{-0.000000in}{0.000000in}}{%
\pgfpathmoveto{\pgfqpoint{-0.000000in}{0.000000in}}%
\pgfpathlineto{\pgfqpoint{-0.020833in}{0.000000in}}%
\pgfusepath{stroke,fill}%
}%
\begin{pgfscope}%
\pgfsys@transformshift{5.661441in}{1.730834in}%
\pgfsys@useobject{currentmarker}{}%
\end{pgfscope}%
\end{pgfscope}%
\begin{pgfscope}%
\pgfsetbuttcap%
\pgfsetroundjoin%
\definecolor{currentfill}{rgb}{0.000000,0.000000,0.000000}%
\pgfsetfillcolor{currentfill}%
\pgfsetlinewidth{0.501875pt}%
\definecolor{currentstroke}{rgb}{0.000000,0.000000,0.000000}%
\pgfsetstrokecolor{currentstroke}%
\pgfsetdash{}{0pt}%
\pgfsys@defobject{currentmarker}{\pgfqpoint{0.000000in}{0.000000in}}{\pgfqpoint{0.020833in}{0.000000in}}{%
\pgfpathmoveto{\pgfqpoint{0.000000in}{0.000000in}}%
\pgfpathlineto{\pgfqpoint{0.020833in}{0.000000in}}%
\pgfusepath{stroke,fill}%
}%
\begin{pgfscope}%
\pgfsys@transformshift{3.385377in}{1.953904in}%
\pgfsys@useobject{currentmarker}{}%
\end{pgfscope}%
\end{pgfscope}%
\begin{pgfscope}%
\pgfsetbuttcap%
\pgfsetroundjoin%
\definecolor{currentfill}{rgb}{0.000000,0.000000,0.000000}%
\pgfsetfillcolor{currentfill}%
\pgfsetlinewidth{0.501875pt}%
\definecolor{currentstroke}{rgb}{0.000000,0.000000,0.000000}%
\pgfsetstrokecolor{currentstroke}%
\pgfsetdash{}{0pt}%
\pgfsys@defobject{currentmarker}{\pgfqpoint{-0.020833in}{0.000000in}}{\pgfqpoint{-0.000000in}{0.000000in}}{%
\pgfpathmoveto{\pgfqpoint{-0.000000in}{0.000000in}}%
\pgfpathlineto{\pgfqpoint{-0.020833in}{0.000000in}}%
\pgfusepath{stroke,fill}%
}%
\begin{pgfscope}%
\pgfsys@transformshift{5.661441in}{1.953904in}%
\pgfsys@useobject{currentmarker}{}%
\end{pgfscope}%
\end{pgfscope}%
\begin{pgfscope}%
\pgfsetbuttcap%
\pgfsetroundjoin%
\definecolor{currentfill}{rgb}{0.000000,0.000000,0.000000}%
\pgfsetfillcolor{currentfill}%
\pgfsetlinewidth{0.501875pt}%
\definecolor{currentstroke}{rgb}{0.000000,0.000000,0.000000}%
\pgfsetstrokecolor{currentstroke}%
\pgfsetdash{}{0pt}%
\pgfsys@defobject{currentmarker}{\pgfqpoint{0.000000in}{0.000000in}}{\pgfqpoint{0.020833in}{0.000000in}}{%
\pgfpathmoveto{\pgfqpoint{0.000000in}{0.000000in}}%
\pgfpathlineto{\pgfqpoint{0.020833in}{0.000000in}}%
\pgfusepath{stroke,fill}%
}%
\begin{pgfscope}%
\pgfsys@transformshift{3.385377in}{2.065439in}%
\pgfsys@useobject{currentmarker}{}%
\end{pgfscope}%
\end{pgfscope}%
\begin{pgfscope}%
\pgfsetbuttcap%
\pgfsetroundjoin%
\definecolor{currentfill}{rgb}{0.000000,0.000000,0.000000}%
\pgfsetfillcolor{currentfill}%
\pgfsetlinewidth{0.501875pt}%
\definecolor{currentstroke}{rgb}{0.000000,0.000000,0.000000}%
\pgfsetstrokecolor{currentstroke}%
\pgfsetdash{}{0pt}%
\pgfsys@defobject{currentmarker}{\pgfqpoint{-0.020833in}{0.000000in}}{\pgfqpoint{-0.000000in}{0.000000in}}{%
\pgfpathmoveto{\pgfqpoint{-0.000000in}{0.000000in}}%
\pgfpathlineto{\pgfqpoint{-0.020833in}{0.000000in}}%
\pgfusepath{stroke,fill}%
}%
\begin{pgfscope}%
\pgfsys@transformshift{5.661441in}{2.065439in}%
\pgfsys@useobject{currentmarker}{}%
\end{pgfscope}%
\end{pgfscope}%
\begin{pgfscope}%
\pgfsetbuttcap%
\pgfsetroundjoin%
\definecolor{currentfill}{rgb}{0.000000,0.000000,0.000000}%
\pgfsetfillcolor{currentfill}%
\pgfsetlinewidth{0.501875pt}%
\definecolor{currentstroke}{rgb}{0.000000,0.000000,0.000000}%
\pgfsetstrokecolor{currentstroke}%
\pgfsetdash{}{0pt}%
\pgfsys@defobject{currentmarker}{\pgfqpoint{0.000000in}{0.000000in}}{\pgfqpoint{0.020833in}{0.000000in}}{%
\pgfpathmoveto{\pgfqpoint{0.000000in}{0.000000in}}%
\pgfpathlineto{\pgfqpoint{0.020833in}{0.000000in}}%
\pgfusepath{stroke,fill}%
}%
\begin{pgfscope}%
\pgfsys@transformshift{3.385377in}{2.176973in}%
\pgfsys@useobject{currentmarker}{}%
\end{pgfscope}%
\end{pgfscope}%
\begin{pgfscope}%
\pgfsetbuttcap%
\pgfsetroundjoin%
\definecolor{currentfill}{rgb}{0.000000,0.000000,0.000000}%
\pgfsetfillcolor{currentfill}%
\pgfsetlinewidth{0.501875pt}%
\definecolor{currentstroke}{rgb}{0.000000,0.000000,0.000000}%
\pgfsetstrokecolor{currentstroke}%
\pgfsetdash{}{0pt}%
\pgfsys@defobject{currentmarker}{\pgfqpoint{-0.020833in}{0.000000in}}{\pgfqpoint{-0.000000in}{0.000000in}}{%
\pgfpathmoveto{\pgfqpoint{-0.000000in}{0.000000in}}%
\pgfpathlineto{\pgfqpoint{-0.020833in}{0.000000in}}%
\pgfusepath{stroke,fill}%
}%
\begin{pgfscope}%
\pgfsys@transformshift{5.661441in}{2.176973in}%
\pgfsys@useobject{currentmarker}{}%
\end{pgfscope}%
\end{pgfscope}%
\begin{pgfscope}%
\pgfsetbuttcap%
\pgfsetroundjoin%
\definecolor{currentfill}{rgb}{0.000000,0.000000,0.000000}%
\pgfsetfillcolor{currentfill}%
\pgfsetlinewidth{0.501875pt}%
\definecolor{currentstroke}{rgb}{0.000000,0.000000,0.000000}%
\pgfsetstrokecolor{currentstroke}%
\pgfsetdash{}{0pt}%
\pgfsys@defobject{currentmarker}{\pgfqpoint{0.000000in}{0.000000in}}{\pgfqpoint{0.020833in}{0.000000in}}{%
\pgfpathmoveto{\pgfqpoint{0.000000in}{0.000000in}}%
\pgfpathlineto{\pgfqpoint{0.020833in}{0.000000in}}%
\pgfusepath{stroke,fill}%
}%
\begin{pgfscope}%
\pgfsys@transformshift{3.385377in}{2.288508in}%
\pgfsys@useobject{currentmarker}{}%
\end{pgfscope}%
\end{pgfscope}%
\begin{pgfscope}%
\pgfsetbuttcap%
\pgfsetroundjoin%
\definecolor{currentfill}{rgb}{0.000000,0.000000,0.000000}%
\pgfsetfillcolor{currentfill}%
\pgfsetlinewidth{0.501875pt}%
\definecolor{currentstroke}{rgb}{0.000000,0.000000,0.000000}%
\pgfsetstrokecolor{currentstroke}%
\pgfsetdash{}{0pt}%
\pgfsys@defobject{currentmarker}{\pgfqpoint{-0.020833in}{0.000000in}}{\pgfqpoint{-0.000000in}{0.000000in}}{%
\pgfpathmoveto{\pgfqpoint{-0.000000in}{0.000000in}}%
\pgfpathlineto{\pgfqpoint{-0.020833in}{0.000000in}}%
\pgfusepath{stroke,fill}%
}%
\begin{pgfscope}%
\pgfsys@transformshift{5.661441in}{2.288508in}%
\pgfsys@useobject{currentmarker}{}%
\end{pgfscope}%
\end{pgfscope}%
\begin{pgfscope}%
\pgfsetbuttcap%
\pgfsetroundjoin%
\definecolor{currentfill}{rgb}{0.000000,0.000000,0.000000}%
\pgfsetfillcolor{currentfill}%
\pgfsetlinewidth{0.501875pt}%
\definecolor{currentstroke}{rgb}{0.000000,0.000000,0.000000}%
\pgfsetstrokecolor{currentstroke}%
\pgfsetdash{}{0pt}%
\pgfsys@defobject{currentmarker}{\pgfqpoint{0.000000in}{0.000000in}}{\pgfqpoint{0.020833in}{0.000000in}}{%
\pgfpathmoveto{\pgfqpoint{0.000000in}{0.000000in}}%
\pgfpathlineto{\pgfqpoint{0.020833in}{0.000000in}}%
\pgfusepath{stroke,fill}%
}%
\begin{pgfscope}%
\pgfsys@transformshift{3.385377in}{2.511578in}%
\pgfsys@useobject{currentmarker}{}%
\end{pgfscope}%
\end{pgfscope}%
\begin{pgfscope}%
\pgfsetbuttcap%
\pgfsetroundjoin%
\definecolor{currentfill}{rgb}{0.000000,0.000000,0.000000}%
\pgfsetfillcolor{currentfill}%
\pgfsetlinewidth{0.501875pt}%
\definecolor{currentstroke}{rgb}{0.000000,0.000000,0.000000}%
\pgfsetstrokecolor{currentstroke}%
\pgfsetdash{}{0pt}%
\pgfsys@defobject{currentmarker}{\pgfqpoint{-0.020833in}{0.000000in}}{\pgfqpoint{-0.000000in}{0.000000in}}{%
\pgfpathmoveto{\pgfqpoint{-0.000000in}{0.000000in}}%
\pgfpathlineto{\pgfqpoint{-0.020833in}{0.000000in}}%
\pgfusepath{stroke,fill}%
}%
\begin{pgfscope}%
\pgfsys@transformshift{5.661441in}{2.511578in}%
\pgfsys@useobject{currentmarker}{}%
\end{pgfscope}%
\end{pgfscope}%
\begin{pgfscope}%
\pgfsetbuttcap%
\pgfsetroundjoin%
\definecolor{currentfill}{rgb}{0.000000,0.000000,0.000000}%
\pgfsetfillcolor{currentfill}%
\pgfsetlinewidth{0.501875pt}%
\definecolor{currentstroke}{rgb}{0.000000,0.000000,0.000000}%
\pgfsetstrokecolor{currentstroke}%
\pgfsetdash{}{0pt}%
\pgfsys@defobject{currentmarker}{\pgfqpoint{0.000000in}{0.000000in}}{\pgfqpoint{0.020833in}{0.000000in}}{%
\pgfpathmoveto{\pgfqpoint{0.000000in}{0.000000in}}%
\pgfpathlineto{\pgfqpoint{0.020833in}{0.000000in}}%
\pgfusepath{stroke,fill}%
}%
\begin{pgfscope}%
\pgfsys@transformshift{3.385377in}{2.623113in}%
\pgfsys@useobject{currentmarker}{}%
\end{pgfscope}%
\end{pgfscope}%
\begin{pgfscope}%
\pgfsetbuttcap%
\pgfsetroundjoin%
\definecolor{currentfill}{rgb}{0.000000,0.000000,0.000000}%
\pgfsetfillcolor{currentfill}%
\pgfsetlinewidth{0.501875pt}%
\definecolor{currentstroke}{rgb}{0.000000,0.000000,0.000000}%
\pgfsetstrokecolor{currentstroke}%
\pgfsetdash{}{0pt}%
\pgfsys@defobject{currentmarker}{\pgfqpoint{-0.020833in}{0.000000in}}{\pgfqpoint{-0.000000in}{0.000000in}}{%
\pgfpathmoveto{\pgfqpoint{-0.000000in}{0.000000in}}%
\pgfpathlineto{\pgfqpoint{-0.020833in}{0.000000in}}%
\pgfusepath{stroke,fill}%
}%
\begin{pgfscope}%
\pgfsys@transformshift{5.661441in}{2.623113in}%
\pgfsys@useobject{currentmarker}{}%
\end{pgfscope}%
\end{pgfscope}%
\begin{pgfscope}%
\pgfsetbuttcap%
\pgfsetroundjoin%
\definecolor{currentfill}{rgb}{0.000000,0.000000,0.000000}%
\pgfsetfillcolor{currentfill}%
\pgfsetlinewidth{0.501875pt}%
\definecolor{currentstroke}{rgb}{0.000000,0.000000,0.000000}%
\pgfsetstrokecolor{currentstroke}%
\pgfsetdash{}{0pt}%
\pgfsys@defobject{currentmarker}{\pgfqpoint{0.000000in}{0.000000in}}{\pgfqpoint{0.020833in}{0.000000in}}{%
\pgfpathmoveto{\pgfqpoint{0.000000in}{0.000000in}}%
\pgfpathlineto{\pgfqpoint{0.020833in}{0.000000in}}%
\pgfusepath{stroke,fill}%
}%
\begin{pgfscope}%
\pgfsys@transformshift{3.385377in}{2.734647in}%
\pgfsys@useobject{currentmarker}{}%
\end{pgfscope}%
\end{pgfscope}%
\begin{pgfscope}%
\pgfsetbuttcap%
\pgfsetroundjoin%
\definecolor{currentfill}{rgb}{0.000000,0.000000,0.000000}%
\pgfsetfillcolor{currentfill}%
\pgfsetlinewidth{0.501875pt}%
\definecolor{currentstroke}{rgb}{0.000000,0.000000,0.000000}%
\pgfsetstrokecolor{currentstroke}%
\pgfsetdash{}{0pt}%
\pgfsys@defobject{currentmarker}{\pgfqpoint{-0.020833in}{0.000000in}}{\pgfqpoint{-0.000000in}{0.000000in}}{%
\pgfpathmoveto{\pgfqpoint{-0.000000in}{0.000000in}}%
\pgfpathlineto{\pgfqpoint{-0.020833in}{0.000000in}}%
\pgfusepath{stroke,fill}%
}%
\begin{pgfscope}%
\pgfsys@transformshift{5.661441in}{2.734647in}%
\pgfsys@useobject{currentmarker}{}%
\end{pgfscope}%
\end{pgfscope}%
\begin{pgfscope}%
\pgfsetbuttcap%
\pgfsetroundjoin%
\definecolor{currentfill}{rgb}{0.000000,0.000000,0.000000}%
\pgfsetfillcolor{currentfill}%
\pgfsetlinewidth{0.501875pt}%
\definecolor{currentstroke}{rgb}{0.000000,0.000000,0.000000}%
\pgfsetstrokecolor{currentstroke}%
\pgfsetdash{}{0pt}%
\pgfsys@defobject{currentmarker}{\pgfqpoint{0.000000in}{0.000000in}}{\pgfqpoint{0.020833in}{0.000000in}}{%
\pgfpathmoveto{\pgfqpoint{0.000000in}{0.000000in}}%
\pgfpathlineto{\pgfqpoint{0.020833in}{0.000000in}}%
\pgfusepath{stroke,fill}%
}%
\begin{pgfscope}%
\pgfsys@transformshift{3.385377in}{2.846182in}%
\pgfsys@useobject{currentmarker}{}%
\end{pgfscope}%
\end{pgfscope}%
\begin{pgfscope}%
\pgfsetbuttcap%
\pgfsetroundjoin%
\definecolor{currentfill}{rgb}{0.000000,0.000000,0.000000}%
\pgfsetfillcolor{currentfill}%
\pgfsetlinewidth{0.501875pt}%
\definecolor{currentstroke}{rgb}{0.000000,0.000000,0.000000}%
\pgfsetstrokecolor{currentstroke}%
\pgfsetdash{}{0pt}%
\pgfsys@defobject{currentmarker}{\pgfqpoint{-0.020833in}{0.000000in}}{\pgfqpoint{-0.000000in}{0.000000in}}{%
\pgfpathmoveto{\pgfqpoint{-0.000000in}{0.000000in}}%
\pgfpathlineto{\pgfqpoint{-0.020833in}{0.000000in}}%
\pgfusepath{stroke,fill}%
}%
\begin{pgfscope}%
\pgfsys@transformshift{5.661441in}{2.846182in}%
\pgfsys@useobject{currentmarker}{}%
\end{pgfscope}%
\end{pgfscope}%
\begin{pgfscope}%
\pgfsetbuttcap%
\pgfsetroundjoin%
\definecolor{currentfill}{rgb}{0.000000,0.000000,0.000000}%
\pgfsetfillcolor{currentfill}%
\pgfsetlinewidth{0.501875pt}%
\definecolor{currentstroke}{rgb}{0.000000,0.000000,0.000000}%
\pgfsetstrokecolor{currentstroke}%
\pgfsetdash{}{0pt}%
\pgfsys@defobject{currentmarker}{\pgfqpoint{0.000000in}{0.000000in}}{\pgfqpoint{0.020833in}{0.000000in}}{%
\pgfpathmoveto{\pgfqpoint{0.000000in}{0.000000in}}%
\pgfpathlineto{\pgfqpoint{0.020833in}{0.000000in}}%
\pgfusepath{stroke,fill}%
}%
\begin{pgfscope}%
\pgfsys@transformshift{3.385377in}{3.069252in}%
\pgfsys@useobject{currentmarker}{}%
\end{pgfscope}%
\end{pgfscope}%
\begin{pgfscope}%
\pgfsetbuttcap%
\pgfsetroundjoin%
\definecolor{currentfill}{rgb}{0.000000,0.000000,0.000000}%
\pgfsetfillcolor{currentfill}%
\pgfsetlinewidth{0.501875pt}%
\definecolor{currentstroke}{rgb}{0.000000,0.000000,0.000000}%
\pgfsetstrokecolor{currentstroke}%
\pgfsetdash{}{0pt}%
\pgfsys@defobject{currentmarker}{\pgfqpoint{-0.020833in}{0.000000in}}{\pgfqpoint{-0.000000in}{0.000000in}}{%
\pgfpathmoveto{\pgfqpoint{-0.000000in}{0.000000in}}%
\pgfpathlineto{\pgfqpoint{-0.020833in}{0.000000in}}%
\pgfusepath{stroke,fill}%
}%
\begin{pgfscope}%
\pgfsys@transformshift{5.661441in}{3.069252in}%
\pgfsys@useobject{currentmarker}{}%
\end{pgfscope}%
\end{pgfscope}%
\begin{pgfscope}%
\pgfsetbuttcap%
\pgfsetroundjoin%
\definecolor{currentfill}{rgb}{0.000000,0.000000,0.000000}%
\pgfsetfillcolor{currentfill}%
\pgfsetlinewidth{0.501875pt}%
\definecolor{currentstroke}{rgb}{0.000000,0.000000,0.000000}%
\pgfsetstrokecolor{currentstroke}%
\pgfsetdash{}{0pt}%
\pgfsys@defobject{currentmarker}{\pgfqpoint{0.000000in}{0.000000in}}{\pgfqpoint{0.020833in}{0.000000in}}{%
\pgfpathmoveto{\pgfqpoint{0.000000in}{0.000000in}}%
\pgfpathlineto{\pgfqpoint{0.020833in}{0.000000in}}%
\pgfusepath{stroke,fill}%
}%
\begin{pgfscope}%
\pgfsys@transformshift{3.385377in}{3.180787in}%
\pgfsys@useobject{currentmarker}{}%
\end{pgfscope}%
\end{pgfscope}%
\begin{pgfscope}%
\pgfsetbuttcap%
\pgfsetroundjoin%
\definecolor{currentfill}{rgb}{0.000000,0.000000,0.000000}%
\pgfsetfillcolor{currentfill}%
\pgfsetlinewidth{0.501875pt}%
\definecolor{currentstroke}{rgb}{0.000000,0.000000,0.000000}%
\pgfsetstrokecolor{currentstroke}%
\pgfsetdash{}{0pt}%
\pgfsys@defobject{currentmarker}{\pgfqpoint{-0.020833in}{0.000000in}}{\pgfqpoint{-0.000000in}{0.000000in}}{%
\pgfpathmoveto{\pgfqpoint{-0.000000in}{0.000000in}}%
\pgfpathlineto{\pgfqpoint{-0.020833in}{0.000000in}}%
\pgfusepath{stroke,fill}%
}%
\begin{pgfscope}%
\pgfsys@transformshift{5.661441in}{3.180787in}%
\pgfsys@useobject{currentmarker}{}%
\end{pgfscope}%
\end{pgfscope}%
\begin{pgfscope}%
\pgfsetbuttcap%
\pgfsetroundjoin%
\definecolor{currentfill}{rgb}{0.000000,0.000000,0.000000}%
\pgfsetfillcolor{currentfill}%
\pgfsetlinewidth{0.501875pt}%
\definecolor{currentstroke}{rgb}{0.000000,0.000000,0.000000}%
\pgfsetstrokecolor{currentstroke}%
\pgfsetdash{}{0pt}%
\pgfsys@defobject{currentmarker}{\pgfqpoint{0.000000in}{0.000000in}}{\pgfqpoint{0.020833in}{0.000000in}}{%
\pgfpathmoveto{\pgfqpoint{0.000000in}{0.000000in}}%
\pgfpathlineto{\pgfqpoint{0.020833in}{0.000000in}}%
\pgfusepath{stroke,fill}%
}%
\begin{pgfscope}%
\pgfsys@transformshift{3.385377in}{3.292322in}%
\pgfsys@useobject{currentmarker}{}%
\end{pgfscope}%
\end{pgfscope}%
\begin{pgfscope}%
\pgfsetbuttcap%
\pgfsetroundjoin%
\definecolor{currentfill}{rgb}{0.000000,0.000000,0.000000}%
\pgfsetfillcolor{currentfill}%
\pgfsetlinewidth{0.501875pt}%
\definecolor{currentstroke}{rgb}{0.000000,0.000000,0.000000}%
\pgfsetstrokecolor{currentstroke}%
\pgfsetdash{}{0pt}%
\pgfsys@defobject{currentmarker}{\pgfqpoint{-0.020833in}{0.000000in}}{\pgfqpoint{-0.000000in}{0.000000in}}{%
\pgfpathmoveto{\pgfqpoint{-0.000000in}{0.000000in}}%
\pgfpathlineto{\pgfqpoint{-0.020833in}{0.000000in}}%
\pgfusepath{stroke,fill}%
}%
\begin{pgfscope}%
\pgfsys@transformshift{5.661441in}{3.292322in}%
\pgfsys@useobject{currentmarker}{}%
\end{pgfscope}%
\end{pgfscope}%
\begin{pgfscope}%
\pgfsetbuttcap%
\pgfsetroundjoin%
\definecolor{currentfill}{rgb}{0.000000,0.000000,0.000000}%
\pgfsetfillcolor{currentfill}%
\pgfsetlinewidth{0.501875pt}%
\definecolor{currentstroke}{rgb}{0.000000,0.000000,0.000000}%
\pgfsetstrokecolor{currentstroke}%
\pgfsetdash{}{0pt}%
\pgfsys@defobject{currentmarker}{\pgfqpoint{0.000000in}{0.000000in}}{\pgfqpoint{0.020833in}{0.000000in}}{%
\pgfpathmoveto{\pgfqpoint{0.000000in}{0.000000in}}%
\pgfpathlineto{\pgfqpoint{0.020833in}{0.000000in}}%
\pgfusepath{stroke,fill}%
}%
\begin{pgfscope}%
\pgfsys@transformshift{3.385377in}{3.403856in}%
\pgfsys@useobject{currentmarker}{}%
\end{pgfscope}%
\end{pgfscope}%
\begin{pgfscope}%
\pgfsetbuttcap%
\pgfsetroundjoin%
\definecolor{currentfill}{rgb}{0.000000,0.000000,0.000000}%
\pgfsetfillcolor{currentfill}%
\pgfsetlinewidth{0.501875pt}%
\definecolor{currentstroke}{rgb}{0.000000,0.000000,0.000000}%
\pgfsetstrokecolor{currentstroke}%
\pgfsetdash{}{0pt}%
\pgfsys@defobject{currentmarker}{\pgfqpoint{-0.020833in}{0.000000in}}{\pgfqpoint{-0.000000in}{0.000000in}}{%
\pgfpathmoveto{\pgfqpoint{-0.000000in}{0.000000in}}%
\pgfpathlineto{\pgfqpoint{-0.020833in}{0.000000in}}%
\pgfusepath{stroke,fill}%
}%
\begin{pgfscope}%
\pgfsys@transformshift{5.661441in}{3.403856in}%
\pgfsys@useobject{currentmarker}{}%
\end{pgfscope}%
\end{pgfscope}%
\begin{pgfscope}%
\pgfsetbuttcap%
\pgfsetroundjoin%
\definecolor{currentfill}{rgb}{0.000000,0.000000,0.000000}%
\pgfsetfillcolor{currentfill}%
\pgfsetlinewidth{0.501875pt}%
\definecolor{currentstroke}{rgb}{0.000000,0.000000,0.000000}%
\pgfsetstrokecolor{currentstroke}%
\pgfsetdash{}{0pt}%
\pgfsys@defobject{currentmarker}{\pgfqpoint{0.000000in}{0.000000in}}{\pgfqpoint{0.020833in}{0.000000in}}{%
\pgfpathmoveto{\pgfqpoint{0.000000in}{0.000000in}}%
\pgfpathlineto{\pgfqpoint{0.020833in}{0.000000in}}%
\pgfusepath{stroke,fill}%
}%
\begin{pgfscope}%
\pgfsys@transformshift{3.385377in}{3.626926in}%
\pgfsys@useobject{currentmarker}{}%
\end{pgfscope}%
\end{pgfscope}%
\begin{pgfscope}%
\pgfsetbuttcap%
\pgfsetroundjoin%
\definecolor{currentfill}{rgb}{0.000000,0.000000,0.000000}%
\pgfsetfillcolor{currentfill}%
\pgfsetlinewidth{0.501875pt}%
\definecolor{currentstroke}{rgb}{0.000000,0.000000,0.000000}%
\pgfsetstrokecolor{currentstroke}%
\pgfsetdash{}{0pt}%
\pgfsys@defobject{currentmarker}{\pgfqpoint{-0.020833in}{0.000000in}}{\pgfqpoint{-0.000000in}{0.000000in}}{%
\pgfpathmoveto{\pgfqpoint{-0.000000in}{0.000000in}}%
\pgfpathlineto{\pgfqpoint{-0.020833in}{0.000000in}}%
\pgfusepath{stroke,fill}%
}%
\begin{pgfscope}%
\pgfsys@transformshift{5.661441in}{3.626926in}%
\pgfsys@useobject{currentmarker}{}%
\end{pgfscope}%
\end{pgfscope}%
\begin{pgfscope}%
\pgfsetbuttcap%
\pgfsetroundjoin%
\definecolor{currentfill}{rgb}{0.000000,0.000000,0.000000}%
\pgfsetfillcolor{currentfill}%
\pgfsetlinewidth{0.501875pt}%
\definecolor{currentstroke}{rgb}{0.000000,0.000000,0.000000}%
\pgfsetstrokecolor{currentstroke}%
\pgfsetdash{}{0pt}%
\pgfsys@defobject{currentmarker}{\pgfqpoint{0.000000in}{0.000000in}}{\pgfqpoint{0.020833in}{0.000000in}}{%
\pgfpathmoveto{\pgfqpoint{0.000000in}{0.000000in}}%
\pgfpathlineto{\pgfqpoint{0.020833in}{0.000000in}}%
\pgfusepath{stroke,fill}%
}%
\begin{pgfscope}%
\pgfsys@transformshift{3.385377in}{3.738461in}%
\pgfsys@useobject{currentmarker}{}%
\end{pgfscope}%
\end{pgfscope}%
\begin{pgfscope}%
\pgfsetbuttcap%
\pgfsetroundjoin%
\definecolor{currentfill}{rgb}{0.000000,0.000000,0.000000}%
\pgfsetfillcolor{currentfill}%
\pgfsetlinewidth{0.501875pt}%
\definecolor{currentstroke}{rgb}{0.000000,0.000000,0.000000}%
\pgfsetstrokecolor{currentstroke}%
\pgfsetdash{}{0pt}%
\pgfsys@defobject{currentmarker}{\pgfqpoint{-0.020833in}{0.000000in}}{\pgfqpoint{-0.000000in}{0.000000in}}{%
\pgfpathmoveto{\pgfqpoint{-0.000000in}{0.000000in}}%
\pgfpathlineto{\pgfqpoint{-0.020833in}{0.000000in}}%
\pgfusepath{stroke,fill}%
}%
\begin{pgfscope}%
\pgfsys@transformshift{5.661441in}{3.738461in}%
\pgfsys@useobject{currentmarker}{}%
\end{pgfscope}%
\end{pgfscope}%
\begin{pgfscope}%
\pgfsetbuttcap%
\pgfsetroundjoin%
\definecolor{currentfill}{rgb}{0.000000,0.000000,0.000000}%
\pgfsetfillcolor{currentfill}%
\pgfsetlinewidth{0.501875pt}%
\definecolor{currentstroke}{rgb}{0.000000,0.000000,0.000000}%
\pgfsetstrokecolor{currentstroke}%
\pgfsetdash{}{0pt}%
\pgfsys@defobject{currentmarker}{\pgfqpoint{0.000000in}{0.000000in}}{\pgfqpoint{0.020833in}{0.000000in}}{%
\pgfpathmoveto{\pgfqpoint{0.000000in}{0.000000in}}%
\pgfpathlineto{\pgfqpoint{0.020833in}{0.000000in}}%
\pgfusepath{stroke,fill}%
}%
\begin{pgfscope}%
\pgfsys@transformshift{3.385377in}{3.849996in}%
\pgfsys@useobject{currentmarker}{}%
\end{pgfscope}%
\end{pgfscope}%
\begin{pgfscope}%
\pgfsetbuttcap%
\pgfsetroundjoin%
\definecolor{currentfill}{rgb}{0.000000,0.000000,0.000000}%
\pgfsetfillcolor{currentfill}%
\pgfsetlinewidth{0.501875pt}%
\definecolor{currentstroke}{rgb}{0.000000,0.000000,0.000000}%
\pgfsetstrokecolor{currentstroke}%
\pgfsetdash{}{0pt}%
\pgfsys@defobject{currentmarker}{\pgfqpoint{-0.020833in}{0.000000in}}{\pgfqpoint{-0.000000in}{0.000000in}}{%
\pgfpathmoveto{\pgfqpoint{-0.000000in}{0.000000in}}%
\pgfpathlineto{\pgfqpoint{-0.020833in}{0.000000in}}%
\pgfusepath{stroke,fill}%
}%
\begin{pgfscope}%
\pgfsys@transformshift{5.661441in}{3.849996in}%
\pgfsys@useobject{currentmarker}{}%
\end{pgfscope}%
\end{pgfscope}%
\begin{pgfscope}%
\pgfsetbuttcap%
\pgfsetroundjoin%
\definecolor{currentfill}{rgb}{0.000000,0.000000,0.000000}%
\pgfsetfillcolor{currentfill}%
\pgfsetlinewidth{0.501875pt}%
\definecolor{currentstroke}{rgb}{0.000000,0.000000,0.000000}%
\pgfsetstrokecolor{currentstroke}%
\pgfsetdash{}{0pt}%
\pgfsys@defobject{currentmarker}{\pgfqpoint{0.000000in}{0.000000in}}{\pgfqpoint{0.020833in}{0.000000in}}{%
\pgfpathmoveto{\pgfqpoint{0.000000in}{0.000000in}}%
\pgfpathlineto{\pgfqpoint{0.020833in}{0.000000in}}%
\pgfusepath{stroke,fill}%
}%
\begin{pgfscope}%
\pgfsys@transformshift{3.385377in}{3.961530in}%
\pgfsys@useobject{currentmarker}{}%
\end{pgfscope}%
\end{pgfscope}%
\begin{pgfscope}%
\pgfsetbuttcap%
\pgfsetroundjoin%
\definecolor{currentfill}{rgb}{0.000000,0.000000,0.000000}%
\pgfsetfillcolor{currentfill}%
\pgfsetlinewidth{0.501875pt}%
\definecolor{currentstroke}{rgb}{0.000000,0.000000,0.000000}%
\pgfsetstrokecolor{currentstroke}%
\pgfsetdash{}{0pt}%
\pgfsys@defobject{currentmarker}{\pgfqpoint{-0.020833in}{0.000000in}}{\pgfqpoint{-0.000000in}{0.000000in}}{%
\pgfpathmoveto{\pgfqpoint{-0.000000in}{0.000000in}}%
\pgfpathlineto{\pgfqpoint{-0.020833in}{0.000000in}}%
\pgfusepath{stroke,fill}%
}%
\begin{pgfscope}%
\pgfsys@transformshift{5.661441in}{3.961530in}%
\pgfsys@useobject{currentmarker}{}%
\end{pgfscope}%
\end{pgfscope}%
\begin{pgfscope}%
\pgfsetbuttcap%
\pgfsetroundjoin%
\definecolor{currentfill}{rgb}{0.000000,0.000000,0.000000}%
\pgfsetfillcolor{currentfill}%
\pgfsetlinewidth{0.501875pt}%
\definecolor{currentstroke}{rgb}{0.000000,0.000000,0.000000}%
\pgfsetstrokecolor{currentstroke}%
\pgfsetdash{}{0pt}%
\pgfsys@defobject{currentmarker}{\pgfqpoint{0.000000in}{0.000000in}}{\pgfqpoint{0.020833in}{0.000000in}}{%
\pgfpathmoveto{\pgfqpoint{0.000000in}{0.000000in}}%
\pgfpathlineto{\pgfqpoint{0.020833in}{0.000000in}}%
\pgfusepath{stroke,fill}%
}%
\begin{pgfscope}%
\pgfsys@transformshift{3.385377in}{4.184600in}%
\pgfsys@useobject{currentmarker}{}%
\end{pgfscope}%
\end{pgfscope}%
\begin{pgfscope}%
\pgfsetbuttcap%
\pgfsetroundjoin%
\definecolor{currentfill}{rgb}{0.000000,0.000000,0.000000}%
\pgfsetfillcolor{currentfill}%
\pgfsetlinewidth{0.501875pt}%
\definecolor{currentstroke}{rgb}{0.000000,0.000000,0.000000}%
\pgfsetstrokecolor{currentstroke}%
\pgfsetdash{}{0pt}%
\pgfsys@defobject{currentmarker}{\pgfqpoint{-0.020833in}{0.000000in}}{\pgfqpoint{-0.000000in}{0.000000in}}{%
\pgfpathmoveto{\pgfqpoint{-0.000000in}{0.000000in}}%
\pgfpathlineto{\pgfqpoint{-0.020833in}{0.000000in}}%
\pgfusepath{stroke,fill}%
}%
\begin{pgfscope}%
\pgfsys@transformshift{5.661441in}{4.184600in}%
\pgfsys@useobject{currentmarker}{}%
\end{pgfscope}%
\end{pgfscope}%
\begin{pgfscope}%
\pgfsetbuttcap%
\pgfsetroundjoin%
\definecolor{currentfill}{rgb}{0.000000,0.000000,0.000000}%
\pgfsetfillcolor{currentfill}%
\pgfsetlinewidth{0.501875pt}%
\definecolor{currentstroke}{rgb}{0.000000,0.000000,0.000000}%
\pgfsetstrokecolor{currentstroke}%
\pgfsetdash{}{0pt}%
\pgfsys@defobject{currentmarker}{\pgfqpoint{0.000000in}{0.000000in}}{\pgfqpoint{0.020833in}{0.000000in}}{%
\pgfpathmoveto{\pgfqpoint{0.000000in}{0.000000in}}%
\pgfpathlineto{\pgfqpoint{0.020833in}{0.000000in}}%
\pgfusepath{stroke,fill}%
}%
\begin{pgfscope}%
\pgfsys@transformshift{3.385377in}{4.296135in}%
\pgfsys@useobject{currentmarker}{}%
\end{pgfscope}%
\end{pgfscope}%
\begin{pgfscope}%
\pgfsetbuttcap%
\pgfsetroundjoin%
\definecolor{currentfill}{rgb}{0.000000,0.000000,0.000000}%
\pgfsetfillcolor{currentfill}%
\pgfsetlinewidth{0.501875pt}%
\definecolor{currentstroke}{rgb}{0.000000,0.000000,0.000000}%
\pgfsetstrokecolor{currentstroke}%
\pgfsetdash{}{0pt}%
\pgfsys@defobject{currentmarker}{\pgfqpoint{-0.020833in}{0.000000in}}{\pgfqpoint{-0.000000in}{0.000000in}}{%
\pgfpathmoveto{\pgfqpoint{-0.000000in}{0.000000in}}%
\pgfpathlineto{\pgfqpoint{-0.020833in}{0.000000in}}%
\pgfusepath{stroke,fill}%
}%
\begin{pgfscope}%
\pgfsys@transformshift{5.661441in}{4.296135in}%
\pgfsys@useobject{currentmarker}{}%
\end{pgfscope}%
\end{pgfscope}%
\begin{pgfscope}%
\definecolor{textcolor}{rgb}{0.000000,0.000000,0.000000}%
\pgfsetstrokecolor{textcolor}%
\pgfsetfillcolor{textcolor}%
\pgftext[x=3.103741in,y=2.398593in,,bottom,rotate=90.000000]{\color{textcolor}\rmfamily\fontsize{10.000000}{12.000000}\selectfont \(\displaystyle LCMC(K)\)}%
\end{pgfscope}%
\begin{pgfscope}%
\pgfpathrectangle{\pgfqpoint{3.385377in}{0.422992in}}{\pgfqpoint{2.276064in}{3.951201in}}%
\pgfusepath{clip}%
\pgfsetrectcap%
\pgfsetroundjoin%
\pgfsetlinewidth{1.003750pt}%
\definecolor{currentstroke}{rgb}{0.047059,0.364706,0.647059}%
\pgfsetstrokecolor{currentstroke}%
\pgfsetdash{}{0pt}%
\pgfpathmoveto{\pgfqpoint{3.407913in}{3.943046in}}%
\pgfpathlineto{\pgfqpoint{3.430448in}{4.194593in}}%
\pgfpathlineto{\pgfqpoint{3.452983in}{4.120015in}}%
\pgfpathlineto{\pgfqpoint{3.475518in}{4.177052in}}%
\pgfpathlineto{\pgfqpoint{3.498054in}{4.040743in}}%
\pgfpathlineto{\pgfqpoint{3.520589in}{4.005766in}}%
\pgfpathlineto{\pgfqpoint{3.543124in}{3.924865in}}%
\pgfpathlineto{\pgfqpoint{3.565660in}{3.907858in}}%
\pgfpathlineto{\pgfqpoint{3.588195in}{3.902388in}}%
\pgfpathlineto{\pgfqpoint{3.610730in}{3.893811in}}%
\pgfpathlineto{\pgfqpoint{3.633265in}{3.866458in}}%
\pgfpathlineto{\pgfqpoint{3.655801in}{3.832010in}}%
\pgfpathlineto{\pgfqpoint{3.678336in}{3.824359in}}%
\pgfpathlineto{\pgfqpoint{3.700871in}{3.820792in}}%
\pgfpathlineto{\pgfqpoint{3.723407in}{3.821422in}}%
\pgfpathlineto{\pgfqpoint{3.745942in}{3.811487in}}%
\pgfpathlineto{\pgfqpoint{3.768477in}{3.823270in}}%
\pgfpathlineto{\pgfqpoint{3.791012in}{3.834517in}}%
\pgfpathlineto{\pgfqpoint{3.813548in}{3.836484in}}%
\pgfpathlineto{\pgfqpoint{3.836083in}{3.815190in}}%
\pgfpathlineto{\pgfqpoint{3.858618in}{3.794589in}}%
\pgfpathlineto{\pgfqpoint{3.881154in}{3.803811in}}%
\pgfpathlineto{\pgfqpoint{3.903689in}{3.798859in}}%
\pgfpathlineto{\pgfqpoint{3.926224in}{3.791405in}}%
\pgfpathlineto{\pgfqpoint{3.948759in}{3.792372in}}%
\pgfpathlineto{\pgfqpoint{3.971295in}{3.808313in}}%
\pgfpathlineto{\pgfqpoint{3.993830in}{3.807024in}}%
\pgfpathlineto{\pgfqpoint{4.016365in}{3.818802in}}%
\pgfpathlineto{\pgfqpoint{4.038901in}{3.814826in}}%
\pgfpathlineto{\pgfqpoint{4.061436in}{3.810646in}}%
\pgfpathlineto{\pgfqpoint{4.083971in}{3.816201in}}%
\pgfpathlineto{\pgfqpoint{4.106506in}{3.813982in}}%
\pgfpathlineto{\pgfqpoint{4.129042in}{3.816978in}}%
\pgfpathlineto{\pgfqpoint{4.151577in}{3.812807in}}%
\pgfpathlineto{\pgfqpoint{4.174112in}{3.814863in}}%
\pgfpathlineto{\pgfqpoint{4.196648in}{3.816414in}}%
\pgfpathlineto{\pgfqpoint{4.219183in}{3.813724in}}%
\pgfpathlineto{\pgfqpoint{4.241718in}{3.804186in}}%
\pgfpathlineto{\pgfqpoint{4.264253in}{3.802660in}}%
\pgfpathlineto{\pgfqpoint{4.286789in}{3.807849in}}%
\pgfpathlineto{\pgfqpoint{4.309324in}{3.809032in}}%
\pgfpathlineto{\pgfqpoint{4.331859in}{3.806497in}}%
\pgfpathlineto{\pgfqpoint{4.354395in}{3.800828in}}%
\pgfpathlineto{\pgfqpoint{4.376930in}{3.799544in}}%
\pgfpathlineto{\pgfqpoint{4.399465in}{3.801422in}}%
\pgfpathlineto{\pgfqpoint{4.422000in}{3.804432in}}%
\pgfpathlineto{\pgfqpoint{4.444536in}{3.800176in}}%
\pgfpathlineto{\pgfqpoint{4.467071in}{3.802792in}}%
\pgfpathlineto{\pgfqpoint{4.489606in}{3.801878in}}%
\pgfpathlineto{\pgfqpoint{4.512142in}{3.799880in}}%
\pgfpathlineto{\pgfqpoint{4.534677in}{3.789190in}}%
\pgfpathlineto{\pgfqpoint{4.557212in}{3.790198in}}%
\pgfpathlineto{\pgfqpoint{4.579748in}{3.776135in}}%
\pgfpathlineto{\pgfqpoint{4.602283in}{3.762592in}}%
\pgfpathlineto{\pgfqpoint{4.624818in}{3.755131in}}%
\pgfpathlineto{\pgfqpoint{4.647353in}{3.748434in}}%
\pgfpathlineto{\pgfqpoint{4.669889in}{3.742462in}}%
\pgfpathlineto{\pgfqpoint{4.692424in}{3.730910in}}%
\pgfpathlineto{\pgfqpoint{4.714959in}{3.727804in}}%
\pgfpathlineto{\pgfqpoint{4.737495in}{3.725964in}}%
\pgfpathlineto{\pgfqpoint{4.760030in}{3.720289in}}%
\pgfpathlineto{\pgfqpoint{4.782565in}{3.715922in}}%
\pgfpathlineto{\pgfqpoint{4.805100in}{3.711250in}}%
\pgfpathlineto{\pgfqpoint{4.827636in}{3.708906in}}%
\pgfpathlineto{\pgfqpoint{4.850171in}{3.703623in}}%
\pgfpathlineto{\pgfqpoint{4.872706in}{3.698499in}}%
\pgfpathlineto{\pgfqpoint{4.895242in}{3.693109in}}%
\pgfpathlineto{\pgfqpoint{4.917777in}{3.687672in}}%
\pgfpathlineto{\pgfqpoint{4.940312in}{3.679353in}}%
\pgfpathlineto{\pgfqpoint{4.962847in}{3.674664in}}%
\pgfpathlineto{\pgfqpoint{4.985383in}{3.670304in}}%
\pgfpathlineto{\pgfqpoint{5.007918in}{3.661987in}}%
\pgfpathlineto{\pgfqpoint{5.030453in}{3.650451in}}%
\pgfpathlineto{\pgfqpoint{5.052989in}{3.638848in}}%
\pgfpathlineto{\pgfqpoint{5.075524in}{3.637058in}}%
\pgfpathlineto{\pgfqpoint{5.098059in}{3.630532in}}%
\pgfpathlineto{\pgfqpoint{5.120594in}{3.622360in}}%
\pgfpathlineto{\pgfqpoint{5.143130in}{3.616188in}}%
\pgfpathlineto{\pgfqpoint{5.165665in}{3.611941in}}%
\pgfpathlineto{\pgfqpoint{5.188200in}{3.604479in}}%
\pgfpathlineto{\pgfqpoint{5.210736in}{3.597546in}}%
\pgfpathlineto{\pgfqpoint{5.233271in}{3.590782in}}%
\pgfpathlineto{\pgfqpoint{5.255806in}{3.584516in}}%
\pgfpathlineto{\pgfqpoint{5.278341in}{3.582725in}}%
\pgfpathlineto{\pgfqpoint{5.300877in}{3.572917in}}%
\pgfpathlineto{\pgfqpoint{5.323412in}{3.565450in}}%
\pgfpathlineto{\pgfqpoint{5.345947in}{3.558315in}}%
\pgfpathlineto{\pgfqpoint{5.368483in}{3.554199in}}%
\pgfpathlineto{\pgfqpoint{5.391018in}{3.547035in}}%
\pgfpathlineto{\pgfqpoint{5.413553in}{3.540805in}}%
\pgfpathlineto{\pgfqpoint{5.436088in}{3.531487in}}%
\pgfpathlineto{\pgfqpoint{5.458624in}{3.523585in}}%
\pgfpathlineto{\pgfqpoint{5.481159in}{3.515852in}}%
\pgfpathlineto{\pgfqpoint{5.503694in}{3.510068in}}%
\pgfpathlineto{\pgfqpoint{5.526230in}{3.504993in}}%
\pgfpathlineto{\pgfqpoint{5.548765in}{3.499878in}}%
\pgfpathlineto{\pgfqpoint{5.571300in}{3.491697in}}%
\pgfpathlineto{\pgfqpoint{5.593835in}{3.484966in}}%
\pgfpathlineto{\pgfqpoint{5.616371in}{3.474841in}}%
\pgfusepath{stroke}%
\end{pgfscope}%
\begin{pgfscope}%
\pgfpathrectangle{\pgfqpoint{3.385377in}{0.422992in}}{\pgfqpoint{2.276064in}{3.951201in}}%
\pgfusepath{clip}%
\pgfsetrectcap%
\pgfsetroundjoin%
\pgfsetlinewidth{1.003750pt}%
\definecolor{currentstroke}{rgb}{0.000000,0.725490,0.270588}%
\pgfsetstrokecolor{currentstroke}%
\pgfsetdash{}{0pt}%
\pgfpathmoveto{\pgfqpoint{3.407913in}{3.719417in}}%
\pgfpathlineto{\pgfqpoint{3.430448in}{3.824208in}}%
\pgfpathlineto{\pgfqpoint{3.452983in}{3.775254in}}%
\pgfpathlineto{\pgfqpoint{3.475518in}{3.733289in}}%
\pgfpathlineto{\pgfqpoint{3.498054in}{3.554351in}}%
\pgfpathlineto{\pgfqpoint{3.520589in}{3.460671in}}%
\pgfpathlineto{\pgfqpoint{3.543124in}{3.335843in}}%
\pgfpathlineto{\pgfqpoint{3.565660in}{3.259684in}}%
\pgfpathlineto{\pgfqpoint{3.588195in}{3.195783in}}%
\pgfpathlineto{\pgfqpoint{3.610730in}{3.139064in}}%
\pgfpathlineto{\pgfqpoint{3.633265in}{3.115523in}}%
\pgfpathlineto{\pgfqpoint{3.655801in}{3.078428in}}%
\pgfpathlineto{\pgfqpoint{3.678336in}{3.050260in}}%
\pgfpathlineto{\pgfqpoint{3.700871in}{3.046078in}}%
\pgfpathlineto{\pgfqpoint{3.723407in}{3.030336in}}%
\pgfpathlineto{\pgfqpoint{3.745942in}{2.994718in}}%
\pgfpathlineto{\pgfqpoint{3.768477in}{2.949310in}}%
\pgfpathlineto{\pgfqpoint{3.791012in}{2.960967in}}%
\pgfpathlineto{\pgfqpoint{3.813548in}{2.940498in}}%
\pgfpathlineto{\pgfqpoint{3.836083in}{2.936050in}}%
\pgfpathlineto{\pgfqpoint{3.858618in}{2.910058in}}%
\pgfpathlineto{\pgfqpoint{3.881154in}{2.911838in}}%
\pgfpathlineto{\pgfqpoint{3.903689in}{2.895229in}}%
\pgfpathlineto{\pgfqpoint{3.926224in}{2.894561in}}%
\pgfpathlineto{\pgfqpoint{3.948759in}{2.885558in}}%
\pgfpathlineto{\pgfqpoint{3.971295in}{2.870793in}}%
\pgfpathlineto{\pgfqpoint{3.993830in}{2.867473in}}%
\pgfpathlineto{\pgfqpoint{4.016365in}{2.870377in}}%
\pgfpathlineto{\pgfqpoint{4.038901in}{2.850427in}}%
\pgfpathlineto{\pgfqpoint{4.061436in}{2.850906in}}%
\pgfpathlineto{\pgfqpoint{4.083971in}{2.849097in}}%
\pgfpathlineto{\pgfqpoint{4.106506in}{2.841722in}}%
\pgfpathlineto{\pgfqpoint{4.129042in}{2.835214in}}%
\pgfpathlineto{\pgfqpoint{4.151577in}{2.830321in}}%
\pgfpathlineto{\pgfqpoint{4.174112in}{2.830098in}}%
\pgfpathlineto{\pgfqpoint{4.196648in}{2.821344in}}%
\pgfpathlineto{\pgfqpoint{4.219183in}{2.820994in}}%
\pgfpathlineto{\pgfqpoint{4.241718in}{2.811098in}}%
\pgfpathlineto{\pgfqpoint{4.264253in}{2.802065in}}%
\pgfpathlineto{\pgfqpoint{4.286789in}{2.802218in}}%
\pgfpathlineto{\pgfqpoint{4.309324in}{2.797249in}}%
\pgfpathlineto{\pgfqpoint{4.331859in}{2.796840in}}%
\pgfpathlineto{\pgfqpoint{4.354395in}{2.787023in}}%
\pgfpathlineto{\pgfqpoint{4.376930in}{2.781462in}}%
\pgfpathlineto{\pgfqpoint{4.399465in}{2.782980in}}%
\pgfpathlineto{\pgfqpoint{4.422000in}{2.781696in}}%
\pgfpathlineto{\pgfqpoint{4.444536in}{2.782546in}}%
\pgfpathlineto{\pgfqpoint{4.467071in}{2.779574in}}%
\pgfpathlineto{\pgfqpoint{4.489606in}{2.772729in}}%
\pgfpathlineto{\pgfqpoint{4.512142in}{2.765038in}}%
\pgfpathlineto{\pgfqpoint{4.534677in}{2.756277in}}%
\pgfpathlineto{\pgfqpoint{4.557212in}{2.751077in}}%
\pgfpathlineto{\pgfqpoint{4.579748in}{2.747654in}}%
\pgfpathlineto{\pgfqpoint{4.602283in}{2.742028in}}%
\pgfpathlineto{\pgfqpoint{4.624818in}{2.732284in}}%
\pgfpathlineto{\pgfqpoint{4.647353in}{2.727879in}}%
\pgfpathlineto{\pgfqpoint{4.669889in}{2.722401in}}%
\pgfpathlineto{\pgfqpoint{4.692424in}{2.713737in}}%
\pgfpathlineto{\pgfqpoint{4.714959in}{2.709393in}}%
\pgfpathlineto{\pgfqpoint{4.737495in}{2.705425in}}%
\pgfpathlineto{\pgfqpoint{4.760030in}{2.704107in}}%
\pgfpathlineto{\pgfqpoint{4.782565in}{2.696969in}}%
\pgfpathlineto{\pgfqpoint{4.805100in}{2.695159in}}%
\pgfpathlineto{\pgfqpoint{4.827636in}{2.701048in}}%
\pgfpathlineto{\pgfqpoint{4.850171in}{2.698154in}}%
\pgfpathlineto{\pgfqpoint{4.872706in}{2.696405in}}%
\pgfpathlineto{\pgfqpoint{4.895242in}{2.695959in}}%
\pgfpathlineto{\pgfqpoint{4.917777in}{2.694909in}}%
\pgfpathlineto{\pgfqpoint{4.940312in}{2.688823in}}%
\pgfpathlineto{\pgfqpoint{4.962847in}{2.684708in}}%
\pgfpathlineto{\pgfqpoint{4.985383in}{2.686220in}}%
\pgfpathlineto{\pgfqpoint{5.007918in}{2.680894in}}%
\pgfpathlineto{\pgfqpoint{5.030453in}{2.675522in}}%
\pgfpathlineto{\pgfqpoint{5.052989in}{2.664061in}}%
\pgfpathlineto{\pgfqpoint{5.075524in}{2.658123in}}%
\pgfpathlineto{\pgfqpoint{5.098059in}{2.657673in}}%
\pgfpathlineto{\pgfqpoint{5.120594in}{2.651426in}}%
\pgfpathlineto{\pgfqpoint{5.143130in}{2.639963in}}%
\pgfpathlineto{\pgfqpoint{5.165665in}{2.631796in}}%
\pgfpathlineto{\pgfqpoint{5.188200in}{2.624881in}}%
\pgfpathlineto{\pgfqpoint{5.210736in}{2.617273in}}%
\pgfpathlineto{\pgfqpoint{5.233271in}{2.608145in}}%
\pgfpathlineto{\pgfqpoint{5.255806in}{2.601930in}}%
\pgfpathlineto{\pgfqpoint{5.278341in}{2.596861in}}%
\pgfpathlineto{\pgfqpoint{5.300877in}{2.584676in}}%
\pgfpathlineto{\pgfqpoint{5.323412in}{2.576673in}}%
\pgfpathlineto{\pgfqpoint{5.345947in}{2.563713in}}%
\pgfpathlineto{\pgfqpoint{5.368483in}{2.559624in}}%
\pgfpathlineto{\pgfqpoint{5.391018in}{2.547616in}}%
\pgfpathlineto{\pgfqpoint{5.413553in}{2.539756in}}%
\pgfpathlineto{\pgfqpoint{5.436088in}{2.527922in}}%
\pgfpathlineto{\pgfqpoint{5.458624in}{2.521206in}}%
\pgfpathlineto{\pgfqpoint{5.481159in}{2.515084in}}%
\pgfpathlineto{\pgfqpoint{5.503694in}{2.508646in}}%
\pgfpathlineto{\pgfqpoint{5.526230in}{2.503372in}}%
\pgfpathlineto{\pgfqpoint{5.548765in}{2.494859in}}%
\pgfpathlineto{\pgfqpoint{5.571300in}{2.487097in}}%
\pgfpathlineto{\pgfqpoint{5.593835in}{2.482345in}}%
\pgfpathlineto{\pgfqpoint{5.616371in}{2.478818in}}%
\pgfusepath{stroke}%
\end{pgfscope}%
\begin{pgfscope}%
\pgfpathrectangle{\pgfqpoint{3.385377in}{0.422992in}}{\pgfqpoint{2.276064in}{3.951201in}}%
\pgfusepath{clip}%
\pgfsetrectcap%
\pgfsetroundjoin%
\pgfsetlinewidth{1.003750pt}%
\definecolor{currentstroke}{rgb}{1.000000,0.584314,0.000000}%
\pgfsetstrokecolor{currentstroke}%
\pgfsetdash{}{0pt}%
\pgfpathmoveto{\pgfqpoint{3.407913in}{2.838879in}}%
\pgfpathlineto{\pgfqpoint{3.430448in}{3.020542in}}%
\pgfpathlineto{\pgfqpoint{3.452983in}{2.950623in}}%
\pgfpathlineto{\pgfqpoint{3.475518in}{2.884198in}}%
\pgfpathlineto{\pgfqpoint{3.498054in}{2.813581in}}%
\pgfpathlineto{\pgfqpoint{3.520589in}{2.726889in}}%
\pgfpathlineto{\pgfqpoint{3.543124in}{2.676937in}}%
\pgfpathlineto{\pgfqpoint{3.565660in}{2.616752in}}%
\pgfpathlineto{\pgfqpoint{3.588195in}{2.614970in}}%
\pgfpathlineto{\pgfqpoint{3.610730in}{2.585583in}}%
\pgfpathlineto{\pgfqpoint{3.633265in}{2.566616in}}%
\pgfpathlineto{\pgfqpoint{3.655801in}{2.578758in}}%
\pgfpathlineto{\pgfqpoint{3.678336in}{2.592251in}}%
\pgfpathlineto{\pgfqpoint{3.700871in}{2.596824in}}%
\pgfpathlineto{\pgfqpoint{3.723407in}{2.584010in}}%
\pgfpathlineto{\pgfqpoint{3.745942in}{2.578035in}}%
\pgfpathlineto{\pgfqpoint{3.768477in}{2.579336in}}%
\pgfpathlineto{\pgfqpoint{3.791012in}{2.582818in}}%
\pgfpathlineto{\pgfqpoint{3.813548in}{2.588872in}}%
\pgfpathlineto{\pgfqpoint{3.836083in}{2.585931in}}%
\pgfpathlineto{\pgfqpoint{3.858618in}{2.589922in}}%
\pgfpathlineto{\pgfqpoint{3.881154in}{2.587830in}}%
\pgfpathlineto{\pgfqpoint{3.903689in}{2.591994in}}%
\pgfpathlineto{\pgfqpoint{3.926224in}{2.603960in}}%
\pgfpathlineto{\pgfqpoint{3.948759in}{2.618880in}}%
\pgfpathlineto{\pgfqpoint{3.971295in}{2.630500in}}%
\pgfpathlineto{\pgfqpoint{3.993830in}{2.646950in}}%
\pgfpathlineto{\pgfqpoint{4.016365in}{2.664219in}}%
\pgfpathlineto{\pgfqpoint{4.038901in}{2.683669in}}%
\pgfpathlineto{\pgfqpoint{4.061436in}{2.700422in}}%
\pgfpathlineto{\pgfqpoint{4.083971in}{2.710682in}}%
\pgfpathlineto{\pgfqpoint{4.106506in}{2.729907in}}%
\pgfpathlineto{\pgfqpoint{4.129042in}{2.740341in}}%
\pgfpathlineto{\pgfqpoint{4.151577in}{2.743994in}}%
\pgfpathlineto{\pgfqpoint{4.174112in}{2.757818in}}%
\pgfpathlineto{\pgfqpoint{4.196648in}{2.769319in}}%
\pgfpathlineto{\pgfqpoint{4.219183in}{2.780952in}}%
\pgfpathlineto{\pgfqpoint{4.241718in}{2.790132in}}%
\pgfpathlineto{\pgfqpoint{4.264253in}{2.795973in}}%
\pgfpathlineto{\pgfqpoint{4.286789in}{2.803616in}}%
\pgfpathlineto{\pgfqpoint{4.309324in}{2.797590in}}%
\pgfpathlineto{\pgfqpoint{4.331859in}{2.801832in}}%
\pgfpathlineto{\pgfqpoint{4.354395in}{2.803275in}}%
\pgfpathlineto{\pgfqpoint{4.376930in}{2.801157in}}%
\pgfpathlineto{\pgfqpoint{4.399465in}{2.811244in}}%
\pgfpathlineto{\pgfqpoint{4.422000in}{2.815726in}}%
\pgfpathlineto{\pgfqpoint{4.444536in}{2.818232in}}%
\pgfpathlineto{\pgfqpoint{4.467071in}{2.814516in}}%
\pgfpathlineto{\pgfqpoint{4.489606in}{2.818368in}}%
\pgfpathlineto{\pgfqpoint{4.512142in}{2.820945in}}%
\pgfpathlineto{\pgfqpoint{4.534677in}{2.820406in}}%
\pgfpathlineto{\pgfqpoint{4.557212in}{2.819886in}}%
\pgfpathlineto{\pgfqpoint{4.579748in}{2.823867in}}%
\pgfpathlineto{\pgfqpoint{4.602283in}{2.820194in}}%
\pgfpathlineto{\pgfqpoint{4.624818in}{2.809792in}}%
\pgfpathlineto{\pgfqpoint{4.647353in}{2.810492in}}%
\pgfpathlineto{\pgfqpoint{4.669889in}{2.814844in}}%
\pgfpathlineto{\pgfqpoint{4.692424in}{2.810852in}}%
\pgfpathlineto{\pgfqpoint{4.714959in}{2.813864in}}%
\pgfpathlineto{\pgfqpoint{4.737495in}{2.810717in}}%
\pgfpathlineto{\pgfqpoint{4.760030in}{2.806069in}}%
\pgfpathlineto{\pgfqpoint{4.782565in}{2.803148in}}%
\pgfpathlineto{\pgfqpoint{4.805100in}{2.794993in}}%
\pgfpathlineto{\pgfqpoint{4.827636in}{2.790369in}}%
\pgfpathlineto{\pgfqpoint{4.850171in}{2.786745in}}%
\pgfpathlineto{\pgfqpoint{4.872706in}{2.786619in}}%
\pgfpathlineto{\pgfqpoint{4.895242in}{2.787538in}}%
\pgfpathlineto{\pgfqpoint{4.917777in}{2.785141in}}%
\pgfpathlineto{\pgfqpoint{4.940312in}{2.780382in}}%
\pgfpathlineto{\pgfqpoint{4.962847in}{2.773760in}}%
\pgfpathlineto{\pgfqpoint{4.985383in}{2.772049in}}%
\pgfpathlineto{\pgfqpoint{5.007918in}{2.764755in}}%
\pgfpathlineto{\pgfqpoint{5.030453in}{2.760531in}}%
\pgfpathlineto{\pgfqpoint{5.052989in}{2.756232in}}%
\pgfpathlineto{\pgfqpoint{5.075524in}{2.754470in}}%
\pgfpathlineto{\pgfqpoint{5.098059in}{2.752017in}}%
\pgfpathlineto{\pgfqpoint{5.120594in}{2.749808in}}%
\pgfpathlineto{\pgfqpoint{5.143130in}{2.744072in}}%
\pgfpathlineto{\pgfqpoint{5.165665in}{2.737949in}}%
\pgfpathlineto{\pgfqpoint{5.188200in}{2.734074in}}%
\pgfpathlineto{\pgfqpoint{5.210736in}{2.729087in}}%
\pgfpathlineto{\pgfqpoint{5.233271in}{2.725243in}}%
\pgfpathlineto{\pgfqpoint{5.255806in}{2.724185in}}%
\pgfpathlineto{\pgfqpoint{5.278341in}{2.721155in}}%
\pgfpathlineto{\pgfqpoint{5.300877in}{2.719511in}}%
\pgfpathlineto{\pgfqpoint{5.323412in}{2.711728in}}%
\pgfpathlineto{\pgfqpoint{5.345947in}{2.707980in}}%
\pgfpathlineto{\pgfqpoint{5.368483in}{2.705268in}}%
\pgfpathlineto{\pgfqpoint{5.391018in}{2.700261in}}%
\pgfpathlineto{\pgfqpoint{5.413553in}{2.695054in}}%
\pgfpathlineto{\pgfqpoint{5.436088in}{2.691036in}}%
\pgfpathlineto{\pgfqpoint{5.458624in}{2.691054in}}%
\pgfpathlineto{\pgfqpoint{5.481159in}{2.685511in}}%
\pgfpathlineto{\pgfqpoint{5.503694in}{2.678151in}}%
\pgfpathlineto{\pgfqpoint{5.526230in}{2.675654in}}%
\pgfpathlineto{\pgfqpoint{5.548765in}{2.671898in}}%
\pgfpathlineto{\pgfqpoint{5.571300in}{2.669371in}}%
\pgfpathlineto{\pgfqpoint{5.593835in}{2.668607in}}%
\pgfpathlineto{\pgfqpoint{5.616371in}{2.664751in}}%
\pgfusepath{stroke}%
\end{pgfscope}%
\begin{pgfscope}%
\pgfpathrectangle{\pgfqpoint{3.385377in}{0.422992in}}{\pgfqpoint{2.276064in}{3.951201in}}%
\pgfusepath{clip}%
\pgfsetrectcap%
\pgfsetroundjoin%
\pgfsetlinewidth{1.003750pt}%
\definecolor{currentstroke}{rgb}{1.000000,0.172549,0.000000}%
\pgfsetstrokecolor{currentstroke}%
\pgfsetdash{}{0pt}%
\pgfpathmoveto{\pgfqpoint{3.407913in}{2.629227in}}%
\pgfpathlineto{\pgfqpoint{3.430448in}{3.118380in}}%
\pgfpathlineto{\pgfqpoint{3.452983in}{3.136981in}}%
\pgfpathlineto{\pgfqpoint{3.475518in}{3.177711in}}%
\pgfpathlineto{\pgfqpoint{3.498054in}{3.048391in}}%
\pgfpathlineto{\pgfqpoint{3.520589in}{2.973813in}}%
\pgfpathlineto{\pgfqpoint{3.543124in}{2.876606in}}%
\pgfpathlineto{\pgfqpoint{3.565660in}{2.833392in}}%
\pgfpathlineto{\pgfqpoint{3.588195in}{2.743867in}}%
\pgfpathlineto{\pgfqpoint{3.610730in}{2.659660in}}%
\pgfpathlineto{\pgfqpoint{3.633265in}{2.590758in}}%
\pgfpathlineto{\pgfqpoint{3.655801in}{2.524015in}}%
\pgfpathlineto{\pgfqpoint{3.678336in}{2.482587in}}%
\pgfpathlineto{\pgfqpoint{3.700871in}{2.421116in}}%
\pgfpathlineto{\pgfqpoint{3.723407in}{2.372494in}}%
\pgfpathlineto{\pgfqpoint{3.745942in}{2.321211in}}%
\pgfpathlineto{\pgfqpoint{3.768477in}{2.285823in}}%
\pgfpathlineto{\pgfqpoint{3.791012in}{2.260575in}}%
\pgfpathlineto{\pgfqpoint{3.813548in}{2.240923in}}%
\pgfpathlineto{\pgfqpoint{3.836083in}{2.215546in}}%
\pgfpathlineto{\pgfqpoint{3.858618in}{2.188589in}}%
\pgfpathlineto{\pgfqpoint{3.881154in}{2.176150in}}%
\pgfpathlineto{\pgfqpoint{3.903689in}{2.157498in}}%
\pgfpathlineto{\pgfqpoint{3.926224in}{2.139814in}}%
\pgfpathlineto{\pgfqpoint{3.948759in}{2.120747in}}%
\pgfpathlineto{\pgfqpoint{3.971295in}{2.092393in}}%
\pgfpathlineto{\pgfqpoint{3.993830in}{2.070796in}}%
\pgfpathlineto{\pgfqpoint{4.016365in}{2.046246in}}%
\pgfpathlineto{\pgfqpoint{4.038901in}{2.022904in}}%
\pgfpathlineto{\pgfqpoint{4.061436in}{2.007639in}}%
\pgfpathlineto{\pgfqpoint{4.083971in}{1.993808in}}%
\pgfpathlineto{\pgfqpoint{4.106506in}{1.977344in}}%
\pgfpathlineto{\pgfqpoint{4.129042in}{1.960182in}}%
\pgfpathlineto{\pgfqpoint{4.151577in}{1.941972in}}%
\pgfpathlineto{\pgfqpoint{4.174112in}{1.930791in}}%
\pgfpathlineto{\pgfqpoint{4.196648in}{1.912464in}}%
\pgfpathlineto{\pgfqpoint{4.219183in}{1.896637in}}%
\pgfpathlineto{\pgfqpoint{4.241718in}{1.877963in}}%
\pgfpathlineto{\pgfqpoint{4.264253in}{1.858094in}}%
\pgfpathlineto{\pgfqpoint{4.286789in}{1.849001in}}%
\pgfpathlineto{\pgfqpoint{4.309324in}{1.831486in}}%
\pgfpathlineto{\pgfqpoint{4.331859in}{1.815802in}}%
\pgfpathlineto{\pgfqpoint{4.354395in}{1.799221in}}%
\pgfpathlineto{\pgfqpoint{4.376930in}{1.790063in}}%
\pgfpathlineto{\pgfqpoint{4.399465in}{1.778204in}}%
\pgfpathlineto{\pgfqpoint{4.422000in}{1.765947in}}%
\pgfpathlineto{\pgfqpoint{4.444536in}{1.757780in}}%
\pgfpathlineto{\pgfqpoint{4.467071in}{1.758395in}}%
\pgfpathlineto{\pgfqpoint{4.489606in}{1.748430in}}%
\pgfpathlineto{\pgfqpoint{4.512142in}{1.736346in}}%
\pgfpathlineto{\pgfqpoint{4.534677in}{1.730216in}}%
\pgfpathlineto{\pgfqpoint{4.557212in}{1.722708in}}%
\pgfpathlineto{\pgfqpoint{4.579748in}{1.711526in}}%
\pgfpathlineto{\pgfqpoint{4.602283in}{1.704380in}}%
\pgfpathlineto{\pgfqpoint{4.624818in}{1.699526in}}%
\pgfpathlineto{\pgfqpoint{4.647353in}{1.693347in}}%
\pgfpathlineto{\pgfqpoint{4.669889in}{1.688609in}}%
\pgfpathlineto{\pgfqpoint{4.692424in}{1.680418in}}%
\pgfpathlineto{\pgfqpoint{4.714959in}{1.677716in}}%
\pgfpathlineto{\pgfqpoint{4.737495in}{1.671376in}}%
\pgfpathlineto{\pgfqpoint{4.760030in}{1.671658in}}%
\pgfpathlineto{\pgfqpoint{4.782565in}{1.665166in}}%
\pgfpathlineto{\pgfqpoint{4.805100in}{1.666423in}}%
\pgfpathlineto{\pgfqpoint{4.827636in}{1.663490in}}%
\pgfpathlineto{\pgfqpoint{4.850171in}{1.657420in}}%
\pgfpathlineto{\pgfqpoint{4.872706in}{1.652593in}}%
\pgfpathlineto{\pgfqpoint{4.895242in}{1.647700in}}%
\pgfpathlineto{\pgfqpoint{4.917777in}{1.644799in}}%
\pgfpathlineto{\pgfqpoint{4.940312in}{1.637931in}}%
\pgfpathlineto{\pgfqpoint{4.962847in}{1.637846in}}%
\pgfpathlineto{\pgfqpoint{4.985383in}{1.631464in}}%
\pgfpathlineto{\pgfqpoint{5.007918in}{1.625646in}}%
\pgfpathlineto{\pgfqpoint{5.030453in}{1.616923in}}%
\pgfpathlineto{\pgfqpoint{5.052989in}{1.613157in}}%
\pgfpathlineto{\pgfqpoint{5.075524in}{1.605391in}}%
\pgfpathlineto{\pgfqpoint{5.098059in}{1.596541in}}%
\pgfpathlineto{\pgfqpoint{5.120594in}{1.591731in}}%
\pgfpathlineto{\pgfqpoint{5.143130in}{1.590090in}}%
\pgfpathlineto{\pgfqpoint{5.165665in}{1.587429in}}%
\pgfpathlineto{\pgfqpoint{5.188200in}{1.582037in}}%
\pgfpathlineto{\pgfqpoint{5.210736in}{1.576260in}}%
\pgfpathlineto{\pgfqpoint{5.233271in}{1.569260in}}%
\pgfpathlineto{\pgfqpoint{5.255806in}{1.567479in}}%
\pgfpathlineto{\pgfqpoint{5.278341in}{1.564742in}}%
\pgfpathlineto{\pgfqpoint{5.300877in}{1.560095in}}%
\pgfpathlineto{\pgfqpoint{5.323412in}{1.554417in}}%
\pgfpathlineto{\pgfqpoint{5.345947in}{1.548388in}}%
\pgfpathlineto{\pgfqpoint{5.368483in}{1.542971in}}%
\pgfpathlineto{\pgfqpoint{5.391018in}{1.537203in}}%
\pgfpathlineto{\pgfqpoint{5.413553in}{1.530942in}}%
\pgfpathlineto{\pgfqpoint{5.436088in}{1.524818in}}%
\pgfpathlineto{\pgfqpoint{5.458624in}{1.522321in}}%
\pgfpathlineto{\pgfqpoint{5.481159in}{1.518674in}}%
\pgfpathlineto{\pgfqpoint{5.503694in}{1.513022in}}%
\pgfpathlineto{\pgfqpoint{5.526230in}{1.507636in}}%
\pgfpathlineto{\pgfqpoint{5.548765in}{1.505273in}}%
\pgfpathlineto{\pgfqpoint{5.571300in}{1.502381in}}%
\pgfpathlineto{\pgfqpoint{5.593835in}{1.497409in}}%
\pgfpathlineto{\pgfqpoint{5.616371in}{1.492818in}}%
\pgfusepath{stroke}%
\end{pgfscope}%
\begin{pgfscope}%
\pgfpathrectangle{\pgfqpoint{3.385377in}{0.422992in}}{\pgfqpoint{2.276064in}{3.951201in}}%
\pgfusepath{clip}%
\pgfsetrectcap%
\pgfsetroundjoin%
\pgfsetlinewidth{1.003750pt}%
\definecolor{currentstroke}{rgb}{0.517647,0.356863,0.592157}%
\pgfsetstrokecolor{currentstroke}%
\pgfsetdash{}{0pt}%
\pgfpathmoveto{\pgfqpoint{3.407913in}{0.602592in}}%
\pgfpathlineto{\pgfqpoint{3.430448in}{0.812209in}}%
\pgfpathlineto{\pgfqpoint{3.452983in}{1.045120in}}%
\pgfpathlineto{\pgfqpoint{3.475518in}{1.213972in}}%
\pgfpathlineto{\pgfqpoint{3.498054in}{1.264952in}}%
\pgfpathlineto{\pgfqpoint{3.520589in}{1.319892in}}%
\pgfpathlineto{\pgfqpoint{3.543124in}{1.357128in}}%
\pgfpathlineto{\pgfqpoint{3.565660in}{1.385047in}}%
\pgfpathlineto{\pgfqpoint{3.588195in}{1.409859in}}%
\pgfpathlineto{\pgfqpoint{3.610730in}{1.424112in}}%
\pgfpathlineto{\pgfqpoint{3.633265in}{1.472614in}}%
\pgfpathlineto{\pgfqpoint{3.655801in}{1.488568in}}%
\pgfpathlineto{\pgfqpoint{3.678336in}{1.478408in}}%
\pgfpathlineto{\pgfqpoint{3.700871in}{1.466700in}}%
\pgfpathlineto{\pgfqpoint{3.723407in}{1.473321in}}%
\pgfpathlineto{\pgfqpoint{3.745942in}{1.454650in}}%
\pgfpathlineto{\pgfqpoint{3.768477in}{1.462836in}}%
\pgfpathlineto{\pgfqpoint{3.791012in}{1.470109in}}%
\pgfpathlineto{\pgfqpoint{3.813548in}{1.468521in}}%
\pgfpathlineto{\pgfqpoint{3.836083in}{1.474776in}}%
\pgfpathlineto{\pgfqpoint{3.858618in}{1.463792in}}%
\pgfpathlineto{\pgfqpoint{3.881154in}{1.458251in}}%
\pgfpathlineto{\pgfqpoint{3.903689in}{1.468989in}}%
\pgfpathlineto{\pgfqpoint{3.926224in}{1.476499in}}%
\pgfpathlineto{\pgfqpoint{3.948759in}{1.468870in}}%
\pgfpathlineto{\pgfqpoint{3.971295in}{1.467738in}}%
\pgfpathlineto{\pgfqpoint{3.993830in}{1.463064in}}%
\pgfpathlineto{\pgfqpoint{4.016365in}{1.461716in}}%
\pgfpathlineto{\pgfqpoint{4.038901in}{1.480219in}}%
\pgfpathlineto{\pgfqpoint{4.061436in}{1.468135in}}%
\pgfpathlineto{\pgfqpoint{4.083971in}{1.469453in}}%
\pgfpathlineto{\pgfqpoint{4.106506in}{1.469812in}}%
\pgfpathlineto{\pgfqpoint{4.129042in}{1.474806in}}%
\pgfpathlineto{\pgfqpoint{4.151577in}{1.479094in}}%
\pgfpathlineto{\pgfqpoint{4.174112in}{1.496312in}}%
\pgfpathlineto{\pgfqpoint{4.196648in}{1.532373in}}%
\pgfpathlineto{\pgfqpoint{4.219183in}{1.561949in}}%
\pgfpathlineto{\pgfqpoint{4.241718in}{1.596588in}}%
\pgfpathlineto{\pgfqpoint{4.264253in}{1.606870in}}%
\pgfpathlineto{\pgfqpoint{4.286789in}{1.615239in}}%
\pgfpathlineto{\pgfqpoint{4.309324in}{1.633425in}}%
\pgfpathlineto{\pgfqpoint{4.331859in}{1.646417in}}%
\pgfpathlineto{\pgfqpoint{4.354395in}{1.654578in}}%
\pgfpathlineto{\pgfqpoint{4.376930in}{1.667766in}}%
\pgfpathlineto{\pgfqpoint{4.399465in}{1.664836in}}%
\pgfpathlineto{\pgfqpoint{4.422000in}{1.667198in}}%
\pgfpathlineto{\pgfqpoint{4.444536in}{1.674216in}}%
\pgfpathlineto{\pgfqpoint{4.467071in}{1.687055in}}%
\pgfpathlineto{\pgfqpoint{4.489606in}{1.693378in}}%
\pgfpathlineto{\pgfqpoint{4.512142in}{1.708672in}}%
\pgfpathlineto{\pgfqpoint{4.534677in}{1.716787in}}%
\pgfpathlineto{\pgfqpoint{4.557212in}{1.708462in}}%
\pgfpathlineto{\pgfqpoint{4.579748in}{1.718382in}}%
\pgfpathlineto{\pgfqpoint{4.602283in}{1.721204in}}%
\pgfpathlineto{\pgfqpoint{4.624818in}{1.713249in}}%
\pgfpathlineto{\pgfqpoint{4.647353in}{1.719303in}}%
\pgfpathlineto{\pgfqpoint{4.669889in}{1.729313in}}%
\pgfpathlineto{\pgfqpoint{4.692424in}{1.729819in}}%
\pgfpathlineto{\pgfqpoint{4.714959in}{1.741678in}}%
\pgfpathlineto{\pgfqpoint{4.737495in}{1.740328in}}%
\pgfpathlineto{\pgfqpoint{4.760030in}{1.755977in}}%
\pgfpathlineto{\pgfqpoint{4.782565in}{1.757368in}}%
\pgfpathlineto{\pgfqpoint{4.805100in}{1.763817in}}%
\pgfpathlineto{\pgfqpoint{4.827636in}{1.765913in}}%
\pgfpathlineto{\pgfqpoint{4.850171in}{1.769665in}}%
\pgfpathlineto{\pgfqpoint{4.872706in}{1.778596in}}%
\pgfpathlineto{\pgfqpoint{4.895242in}{1.777872in}}%
\pgfpathlineto{\pgfqpoint{4.917777in}{1.774290in}}%
\pgfpathlineto{\pgfqpoint{4.940312in}{1.775673in}}%
\pgfpathlineto{\pgfqpoint{4.962847in}{1.773022in}}%
\pgfpathlineto{\pgfqpoint{4.985383in}{1.765326in}}%
\pgfpathlineto{\pgfqpoint{5.007918in}{1.768326in}}%
\pgfpathlineto{\pgfqpoint{5.030453in}{1.766839in}}%
\pgfpathlineto{\pgfqpoint{5.052989in}{1.759158in}}%
\pgfpathlineto{\pgfqpoint{5.075524in}{1.760067in}}%
\pgfpathlineto{\pgfqpoint{5.098059in}{1.768676in}}%
\pgfpathlineto{\pgfqpoint{5.120594in}{1.761812in}}%
\pgfpathlineto{\pgfqpoint{5.143130in}{1.767846in}}%
\pgfpathlineto{\pgfqpoint{5.165665in}{1.761519in}}%
\pgfpathlineto{\pgfqpoint{5.188200in}{1.760940in}}%
\pgfpathlineto{\pgfqpoint{5.210736in}{1.753300in}}%
\pgfpathlineto{\pgfqpoint{5.233271in}{1.744993in}}%
\pgfpathlineto{\pgfqpoint{5.255806in}{1.746315in}}%
\pgfpathlineto{\pgfqpoint{5.278341in}{1.743112in}}%
\pgfpathlineto{\pgfqpoint{5.300877in}{1.741958in}}%
\pgfpathlineto{\pgfqpoint{5.323412in}{1.737903in}}%
\pgfpathlineto{\pgfqpoint{5.345947in}{1.738440in}}%
\pgfpathlineto{\pgfqpoint{5.368483in}{1.736263in}}%
\pgfpathlineto{\pgfqpoint{5.391018in}{1.738375in}}%
\pgfpathlineto{\pgfqpoint{5.413553in}{1.736557in}}%
\pgfpathlineto{\pgfqpoint{5.436088in}{1.741228in}}%
\pgfpathlineto{\pgfqpoint{5.458624in}{1.747772in}}%
\pgfpathlineto{\pgfqpoint{5.481159in}{1.743805in}}%
\pgfpathlineto{\pgfqpoint{5.503694in}{1.743044in}}%
\pgfpathlineto{\pgfqpoint{5.526230in}{1.744653in}}%
\pgfpathlineto{\pgfqpoint{5.548765in}{1.748556in}}%
\pgfpathlineto{\pgfqpoint{5.571300in}{1.745462in}}%
\pgfpathlineto{\pgfqpoint{5.593835in}{1.746139in}}%
\pgfpathlineto{\pgfqpoint{5.616371in}{1.737059in}}%
\pgfusepath{stroke}%
\end{pgfscope}%
\begin{pgfscope}%
\pgfsetrectcap%
\pgfsetmiterjoin%
\pgfsetlinewidth{0.501875pt}%
\definecolor{currentstroke}{rgb}{0.000000,0.000000,0.000000}%
\pgfsetstrokecolor{currentstroke}%
\pgfsetdash{}{0pt}%
\pgfpathmoveto{\pgfqpoint{3.385377in}{0.422992in}}%
\pgfpathlineto{\pgfqpoint{3.385377in}{4.374193in}}%
\pgfusepath{stroke}%
\end{pgfscope}%
\begin{pgfscope}%
\pgfsetrectcap%
\pgfsetmiterjoin%
\pgfsetlinewidth{0.501875pt}%
\definecolor{currentstroke}{rgb}{0.000000,0.000000,0.000000}%
\pgfsetstrokecolor{currentstroke}%
\pgfsetdash{}{0pt}%
\pgfpathmoveto{\pgfqpoint{5.661441in}{0.422992in}}%
\pgfpathlineto{\pgfqpoint{5.661441in}{4.374193in}}%
\pgfusepath{stroke}%
\end{pgfscope}%
\begin{pgfscope}%
\pgfsetrectcap%
\pgfsetmiterjoin%
\pgfsetlinewidth{0.501875pt}%
\definecolor{currentstroke}{rgb}{0.000000,0.000000,0.000000}%
\pgfsetstrokecolor{currentstroke}%
\pgfsetdash{}{0pt}%
\pgfpathmoveto{\pgfqpoint{3.385377in}{0.422992in}}%
\pgfpathlineto{\pgfqpoint{5.661441in}{0.422992in}}%
\pgfusepath{stroke}%
\end{pgfscope}%
\begin{pgfscope}%
\pgfsetrectcap%
\pgfsetmiterjoin%
\pgfsetlinewidth{0.501875pt}%
\definecolor{currentstroke}{rgb}{0.000000,0.000000,0.000000}%
\pgfsetstrokecolor{currentstroke}%
\pgfsetdash{}{0pt}%
\pgfpathmoveto{\pgfqpoint{3.385377in}{4.374193in}}%
\pgfpathlineto{\pgfqpoint{5.661441in}{4.374193in}}%
\pgfusepath{stroke}%
\end{pgfscope}%
\begin{pgfscope}%
\definecolor{textcolor}{rgb}{0.000000,0.000000,0.000000}%
\pgfsetstrokecolor{textcolor}%
\pgfsetfillcolor{textcolor}%
\pgftext[x=4.523409in,y=4.457526in,,base]{\color{textcolor}\rmfamily\fontsize{12.000000}{14.400000}\selectfont LCMC}%
\end{pgfscope}%
\begin{pgfscope}%
\pgfsetrectcap%
\pgfsetroundjoin%
\pgfsetlinewidth{1.003750pt}%
\definecolor{currentstroke}{rgb}{0.047059,0.364706,0.647059}%
\pgfsetstrokecolor{currentstroke}%
\pgfsetdash{}{0pt}%
\pgfpathmoveto{\pgfqpoint{4.382850in}{1.440922in}}%
\pgfpathlineto{\pgfqpoint{4.521739in}{1.440922in}}%
\pgfpathlineto{\pgfqpoint{4.660627in}{1.440922in}}%
\pgfusepath{stroke}%
\end{pgfscope}%
\begin{pgfscope}%
\definecolor{textcolor}{rgb}{0.000000,0.000000,0.000000}%
\pgfsetstrokecolor{textcolor}%
\pgfsetfillcolor{textcolor}%
\pgftext[x=4.771739in,y=1.392311in,left,base]{\color{textcolor}\rmfamily\fontsize{10.000000}{12.000000}\selectfont PCA}%
\end{pgfscope}%
\begin{pgfscope}%
\pgfsetrectcap%
\pgfsetroundjoin%
\pgfsetlinewidth{1.003750pt}%
\definecolor{currentstroke}{rgb}{0.000000,0.725490,0.270588}%
\pgfsetstrokecolor{currentstroke}%
\pgfsetdash{}{0pt}%
\pgfpathmoveto{\pgfqpoint{4.382850in}{1.237065in}}%
\pgfpathlineto{\pgfqpoint{4.521739in}{1.237065in}}%
\pgfpathlineto{\pgfqpoint{4.660627in}{1.237065in}}%
\pgfusepath{stroke}%
\end{pgfscope}%
\begin{pgfscope}%
\definecolor{textcolor}{rgb}{0.000000,0.000000,0.000000}%
\pgfsetstrokecolor{textcolor}%
\pgfsetfillcolor{textcolor}%
\pgftext[x=4.771739in,y=1.188454in,left,base]{\color{textcolor}\rmfamily\fontsize{10.000000}{12.000000}\selectfont KernelPCA}%
\end{pgfscope}%
\begin{pgfscope}%
\pgfsetrectcap%
\pgfsetroundjoin%
\pgfsetlinewidth{1.003750pt}%
\definecolor{currentstroke}{rgb}{1.000000,0.584314,0.000000}%
\pgfsetstrokecolor{currentstroke}%
\pgfsetdash{}{0pt}%
\pgfpathmoveto{\pgfqpoint{4.382850in}{1.033208in}}%
\pgfpathlineto{\pgfqpoint{4.521739in}{1.033208in}}%
\pgfpathlineto{\pgfqpoint{4.660627in}{1.033208in}}%
\pgfusepath{stroke}%
\end{pgfscope}%
\begin{pgfscope}%
\definecolor{textcolor}{rgb}{0.000000,0.000000,0.000000}%
\pgfsetstrokecolor{textcolor}%
\pgfsetfillcolor{textcolor}%
\pgftext[x=4.771739in,y=0.984596in,left,base]{\color{textcolor}\rmfamily\fontsize{10.000000}{12.000000}\selectfont AE}%
\end{pgfscope}%
\begin{pgfscope}%
\pgfsetrectcap%
\pgfsetroundjoin%
\pgfsetlinewidth{1.003750pt}%
\definecolor{currentstroke}{rgb}{1.000000,0.172549,0.000000}%
\pgfsetstrokecolor{currentstroke}%
\pgfsetdash{}{0pt}%
\pgfpathmoveto{\pgfqpoint{4.382850in}{0.829350in}}%
\pgfpathlineto{\pgfqpoint{4.521739in}{0.829350in}}%
\pgfpathlineto{\pgfqpoint{4.660627in}{0.829350in}}%
\pgfusepath{stroke}%
\end{pgfscope}%
\begin{pgfscope}%
\definecolor{textcolor}{rgb}{0.000000,0.000000,0.000000}%
\pgfsetstrokecolor{textcolor}%
\pgfsetfillcolor{textcolor}%
\pgftext[x=4.771739in,y=0.780739in,left,base]{\color{textcolor}\rmfamily\fontsize{10.000000}{12.000000}\selectfont LLE}%
\end{pgfscope}%
\begin{pgfscope}%
\pgfsetrectcap%
\pgfsetroundjoin%
\pgfsetlinewidth{1.003750pt}%
\definecolor{currentstroke}{rgb}{0.517647,0.356863,0.592157}%
\pgfsetstrokecolor{currentstroke}%
\pgfsetdash{}{0pt}%
\pgfpathmoveto{\pgfqpoint{4.382850in}{0.625493in}}%
\pgfpathlineto{\pgfqpoint{4.521739in}{0.625493in}}%
\pgfpathlineto{\pgfqpoint{4.660627in}{0.625493in}}%
\pgfusepath{stroke}%
\end{pgfscope}%
\begin{pgfscope}%
\definecolor{textcolor}{rgb}{0.000000,0.000000,0.000000}%
\pgfsetstrokecolor{textcolor}%
\pgfsetfillcolor{textcolor}%
\pgftext[x=4.771739in,y=0.576882in,left,base]{\color{textcolor}\rmfamily\fontsize{10.000000}{12.000000}\selectfont CAE}%
\end{pgfscope}%
\end{pgfpicture}%
\makeatother%
\endgroup%

	\end{center}
	\caption[Olivetti Faces Qualitätskriterien]{Die Vertrauenswürdigkeit und Kontinuität der Dimensionsreduktion, sowie das Local Continuity Meta-Criterion (LCMC) für den Olivetti Faces Datensatz. Hier schneiden alle Varianten der Autoencoder nicht so gut ab, wie es beispielweise auf dem MNIST Datensatz der Fall war. Dies könnte an der deutlich geringeren Stichprobengröße von 400 liegen, wodurch eine Konvergenz der Autoencoder schwieriger wird. Neben den Autoencoder schneidet hier auch LLE für größere Werte von $K$ schlecht ab, kann jedoch bei geringeren Nachbarschaftsgrößen sehr hohe Werte für die Vertrauenswürdigkeit und Kontinuität erreichen. PCA schneidet hinsichtlich allen drei Kriterien für alle Werte von $K$ am besten ab, gefolgt von der Kernel PCA. (Eigene Darstellung)}
	\label{fig:OlivettiFacesMetrics}
\end{figure}

\begin{figure}[ht]
	\begin{center}
		%% Creator: Matplotlib, PGF backend
%%
%% To include the figure in your LaTeX document, write
%%   \input{<filename>.pgf}
%%
%% Make sure the required packages are loaded in your preamble
%%   \usepackage{pgf}
%%
%% Also ensure that all the required font packages are loaded; for instance,
%% the lmodern package is sometimes necessary when using math font.
%%   \usepackage{lmodern}
%%
%% Figures using additional raster images can only be included by \input if
%% they are in the same directory as the main LaTeX file. For loading figures
%% from other directories you can use the `import` package
%%   \usepackage{import}
%%
%% and then include the figures with
%%   \import{<path to file>}{<filename>.pgf}
%%
%% Matplotlib used the following preamble
%%   
%%   \usepackage{fontspec}
%%   \setmainfont{DejaVuSerif.ttf}[Path=\detokenize{/Users/moritzmistol/.pyenv/versions/3.9.13/envs/thesis/lib/python3.9/site-packages/matplotlib/mpl-data/fonts/ttf/}]
%%   \setsansfont{DejaVuSans.ttf}[Path=\detokenize{/Users/moritzmistol/.pyenv/versions/3.9.13/envs/thesis/lib/python3.9/site-packages/matplotlib/mpl-data/fonts/ttf/}]
%%   \setmonofont{DejaVuSansMono.ttf}[Path=\detokenize{/Users/moritzmistol/.pyenv/versions/3.9.13/envs/thesis/lib/python3.9/site-packages/matplotlib/mpl-data/fonts/ttf/}]
%%   \makeatletter\@ifpackageloaded{underscore}{}{\usepackage[strings]{underscore}}\makeatother
%%
\begingroup%
\makeatletter%
\begin{pgfpicture}%
\pgfpathrectangle{\pgfpointorigin}{\pgfqpoint{5.711441in}{4.634154in}}%
\pgfusepath{use as bounding box, clip}%
\begin{pgfscope}%
\pgfsetbuttcap%
\pgfsetmiterjoin%
\definecolor{currentfill}{rgb}{1.000000,1.000000,1.000000}%
\pgfsetfillcolor{currentfill}%
\pgfsetlinewidth{0.000000pt}%
\definecolor{currentstroke}{rgb}{1.000000,1.000000,1.000000}%
\pgfsetstrokecolor{currentstroke}%
\pgfsetdash{}{0pt}%
\pgfpathmoveto{\pgfqpoint{-0.000000in}{-0.000000in}}%
\pgfpathlineto{\pgfqpoint{5.711441in}{-0.000000in}}%
\pgfpathlineto{\pgfqpoint{5.711441in}{4.634154in}}%
\pgfpathlineto{\pgfqpoint{-0.000000in}{4.634154in}}%
\pgfpathlineto{\pgfqpoint{-0.000000in}{-0.000000in}}%
\pgfpathclose%
\pgfusepath{fill}%
\end{pgfscope}%
\begin{pgfscope}%
\pgfsetbuttcap%
\pgfsetmiterjoin%
\definecolor{currentfill}{rgb}{1.000000,1.000000,1.000000}%
\pgfsetfillcolor{currentfill}%
\pgfsetlinewidth{0.000000pt}%
\definecolor{currentstroke}{rgb}{0.000000,0.000000,0.000000}%
\pgfsetstrokecolor{currentstroke}%
\pgfsetstrokeopacity{0.000000}%
\pgfsetdash{}{0pt}%
\pgfpathmoveto{\pgfqpoint{0.539970in}{2.747992in}}%
\pgfpathlineto{\pgfqpoint{2.816034in}{2.747992in}}%
\pgfpathlineto{\pgfqpoint{2.816034in}{4.374193in}}%
\pgfpathlineto{\pgfqpoint{0.539970in}{4.374193in}}%
\pgfpathlineto{\pgfqpoint{0.539970in}{2.747992in}}%
\pgfpathclose%
\pgfusepath{fill}%
\end{pgfscope}%
\begin{pgfscope}%
\pgfsetbuttcap%
\pgfsetroundjoin%
\definecolor{currentfill}{rgb}{0.000000,0.000000,0.000000}%
\pgfsetfillcolor{currentfill}%
\pgfsetlinewidth{0.501875pt}%
\definecolor{currentstroke}{rgb}{0.000000,0.000000,0.000000}%
\pgfsetstrokecolor{currentstroke}%
\pgfsetdash{}{0pt}%
\pgfsys@defobject{currentmarker}{\pgfqpoint{0.000000in}{0.000000in}}{\pgfqpoint{0.000000in}{0.041667in}}{%
\pgfpathmoveto{\pgfqpoint{0.000000in}{0.000000in}}%
\pgfpathlineto{\pgfqpoint{0.000000in}{0.041667in}}%
\pgfusepath{stroke,fill}%
}%
\begin{pgfscope}%
\pgfsys@transformshift{0.539970in}{2.747992in}%
\pgfsys@useobject{currentmarker}{}%
\end{pgfscope}%
\end{pgfscope}%
\begin{pgfscope}%
\pgfsetbuttcap%
\pgfsetroundjoin%
\definecolor{currentfill}{rgb}{0.000000,0.000000,0.000000}%
\pgfsetfillcolor{currentfill}%
\pgfsetlinewidth{0.501875pt}%
\definecolor{currentstroke}{rgb}{0.000000,0.000000,0.000000}%
\pgfsetstrokecolor{currentstroke}%
\pgfsetdash{}{0pt}%
\pgfsys@defobject{currentmarker}{\pgfqpoint{0.000000in}{-0.041667in}}{\pgfqpoint{0.000000in}{0.000000in}}{%
\pgfpathmoveto{\pgfqpoint{0.000000in}{0.000000in}}%
\pgfpathlineto{\pgfqpoint{0.000000in}{-0.041667in}}%
\pgfusepath{stroke,fill}%
}%
\begin{pgfscope}%
\pgfsys@transformshift{0.539970in}{4.374193in}%
\pgfsys@useobject{currentmarker}{}%
\end{pgfscope}%
\end{pgfscope}%
\begin{pgfscope}%
\definecolor{textcolor}{rgb}{0.000000,0.000000,0.000000}%
\pgfsetstrokecolor{textcolor}%
\pgfsetfillcolor{textcolor}%
\pgftext[x=0.539970in,y=2.699381in,,top]{\color{textcolor}\rmfamily\fontsize{10.000000}{12.000000}\selectfont \(\displaystyle {0}\)}%
\end{pgfscope}%
\begin{pgfscope}%
\pgfsetbuttcap%
\pgfsetroundjoin%
\definecolor{currentfill}{rgb}{0.000000,0.000000,0.000000}%
\pgfsetfillcolor{currentfill}%
\pgfsetlinewidth{0.501875pt}%
\definecolor{currentstroke}{rgb}{0.000000,0.000000,0.000000}%
\pgfsetstrokecolor{currentstroke}%
\pgfsetdash{}{0pt}%
\pgfsys@defobject{currentmarker}{\pgfqpoint{0.000000in}{0.000000in}}{\pgfqpoint{0.000000in}{0.041667in}}{%
\pgfpathmoveto{\pgfqpoint{0.000000in}{0.000000in}}%
\pgfpathlineto{\pgfqpoint{0.000000in}{0.041667in}}%
\pgfusepath{stroke,fill}%
}%
\begin{pgfscope}%
\pgfsys@transformshift{0.990676in}{2.747992in}%
\pgfsys@useobject{currentmarker}{}%
\end{pgfscope}%
\end{pgfscope}%
\begin{pgfscope}%
\pgfsetbuttcap%
\pgfsetroundjoin%
\definecolor{currentfill}{rgb}{0.000000,0.000000,0.000000}%
\pgfsetfillcolor{currentfill}%
\pgfsetlinewidth{0.501875pt}%
\definecolor{currentstroke}{rgb}{0.000000,0.000000,0.000000}%
\pgfsetstrokecolor{currentstroke}%
\pgfsetdash{}{0pt}%
\pgfsys@defobject{currentmarker}{\pgfqpoint{0.000000in}{-0.041667in}}{\pgfqpoint{0.000000in}{0.000000in}}{%
\pgfpathmoveto{\pgfqpoint{0.000000in}{0.000000in}}%
\pgfpathlineto{\pgfqpoint{0.000000in}{-0.041667in}}%
\pgfusepath{stroke,fill}%
}%
\begin{pgfscope}%
\pgfsys@transformshift{0.990676in}{4.374193in}%
\pgfsys@useobject{currentmarker}{}%
\end{pgfscope}%
\end{pgfscope}%
\begin{pgfscope}%
\definecolor{textcolor}{rgb}{0.000000,0.000000,0.000000}%
\pgfsetstrokecolor{textcolor}%
\pgfsetfillcolor{textcolor}%
\pgftext[x=0.990676in,y=2.699381in,,top]{\color{textcolor}\rmfamily\fontsize{10.000000}{12.000000}\selectfont \(\displaystyle {20}\)}%
\end{pgfscope}%
\begin{pgfscope}%
\pgfsetbuttcap%
\pgfsetroundjoin%
\definecolor{currentfill}{rgb}{0.000000,0.000000,0.000000}%
\pgfsetfillcolor{currentfill}%
\pgfsetlinewidth{0.501875pt}%
\definecolor{currentstroke}{rgb}{0.000000,0.000000,0.000000}%
\pgfsetstrokecolor{currentstroke}%
\pgfsetdash{}{0pt}%
\pgfsys@defobject{currentmarker}{\pgfqpoint{0.000000in}{0.000000in}}{\pgfqpoint{0.000000in}{0.041667in}}{%
\pgfpathmoveto{\pgfqpoint{0.000000in}{0.000000in}}%
\pgfpathlineto{\pgfqpoint{0.000000in}{0.041667in}}%
\pgfusepath{stroke,fill}%
}%
\begin{pgfscope}%
\pgfsys@transformshift{1.441381in}{2.747992in}%
\pgfsys@useobject{currentmarker}{}%
\end{pgfscope}%
\end{pgfscope}%
\begin{pgfscope}%
\pgfsetbuttcap%
\pgfsetroundjoin%
\definecolor{currentfill}{rgb}{0.000000,0.000000,0.000000}%
\pgfsetfillcolor{currentfill}%
\pgfsetlinewidth{0.501875pt}%
\definecolor{currentstroke}{rgb}{0.000000,0.000000,0.000000}%
\pgfsetstrokecolor{currentstroke}%
\pgfsetdash{}{0pt}%
\pgfsys@defobject{currentmarker}{\pgfqpoint{0.000000in}{-0.041667in}}{\pgfqpoint{0.000000in}{0.000000in}}{%
\pgfpathmoveto{\pgfqpoint{0.000000in}{0.000000in}}%
\pgfpathlineto{\pgfqpoint{0.000000in}{-0.041667in}}%
\pgfusepath{stroke,fill}%
}%
\begin{pgfscope}%
\pgfsys@transformshift{1.441381in}{4.374193in}%
\pgfsys@useobject{currentmarker}{}%
\end{pgfscope}%
\end{pgfscope}%
\begin{pgfscope}%
\definecolor{textcolor}{rgb}{0.000000,0.000000,0.000000}%
\pgfsetstrokecolor{textcolor}%
\pgfsetfillcolor{textcolor}%
\pgftext[x=1.441381in,y=2.699381in,,top]{\color{textcolor}\rmfamily\fontsize{10.000000}{12.000000}\selectfont \(\displaystyle {40}\)}%
\end{pgfscope}%
\begin{pgfscope}%
\pgfsetbuttcap%
\pgfsetroundjoin%
\definecolor{currentfill}{rgb}{0.000000,0.000000,0.000000}%
\pgfsetfillcolor{currentfill}%
\pgfsetlinewidth{0.501875pt}%
\definecolor{currentstroke}{rgb}{0.000000,0.000000,0.000000}%
\pgfsetstrokecolor{currentstroke}%
\pgfsetdash{}{0pt}%
\pgfsys@defobject{currentmarker}{\pgfqpoint{0.000000in}{0.000000in}}{\pgfqpoint{0.000000in}{0.041667in}}{%
\pgfpathmoveto{\pgfqpoint{0.000000in}{0.000000in}}%
\pgfpathlineto{\pgfqpoint{0.000000in}{0.041667in}}%
\pgfusepath{stroke,fill}%
}%
\begin{pgfscope}%
\pgfsys@transformshift{1.892087in}{2.747992in}%
\pgfsys@useobject{currentmarker}{}%
\end{pgfscope}%
\end{pgfscope}%
\begin{pgfscope}%
\pgfsetbuttcap%
\pgfsetroundjoin%
\definecolor{currentfill}{rgb}{0.000000,0.000000,0.000000}%
\pgfsetfillcolor{currentfill}%
\pgfsetlinewidth{0.501875pt}%
\definecolor{currentstroke}{rgb}{0.000000,0.000000,0.000000}%
\pgfsetstrokecolor{currentstroke}%
\pgfsetdash{}{0pt}%
\pgfsys@defobject{currentmarker}{\pgfqpoint{0.000000in}{-0.041667in}}{\pgfqpoint{0.000000in}{0.000000in}}{%
\pgfpathmoveto{\pgfqpoint{0.000000in}{0.000000in}}%
\pgfpathlineto{\pgfqpoint{0.000000in}{-0.041667in}}%
\pgfusepath{stroke,fill}%
}%
\begin{pgfscope}%
\pgfsys@transformshift{1.892087in}{4.374193in}%
\pgfsys@useobject{currentmarker}{}%
\end{pgfscope}%
\end{pgfscope}%
\begin{pgfscope}%
\definecolor{textcolor}{rgb}{0.000000,0.000000,0.000000}%
\pgfsetstrokecolor{textcolor}%
\pgfsetfillcolor{textcolor}%
\pgftext[x=1.892087in,y=2.699381in,,top]{\color{textcolor}\rmfamily\fontsize{10.000000}{12.000000}\selectfont \(\displaystyle {60}\)}%
\end{pgfscope}%
\begin{pgfscope}%
\pgfsetbuttcap%
\pgfsetroundjoin%
\definecolor{currentfill}{rgb}{0.000000,0.000000,0.000000}%
\pgfsetfillcolor{currentfill}%
\pgfsetlinewidth{0.501875pt}%
\definecolor{currentstroke}{rgb}{0.000000,0.000000,0.000000}%
\pgfsetstrokecolor{currentstroke}%
\pgfsetdash{}{0pt}%
\pgfsys@defobject{currentmarker}{\pgfqpoint{0.000000in}{0.000000in}}{\pgfqpoint{0.000000in}{0.041667in}}{%
\pgfpathmoveto{\pgfqpoint{0.000000in}{0.000000in}}%
\pgfpathlineto{\pgfqpoint{0.000000in}{0.041667in}}%
\pgfusepath{stroke,fill}%
}%
\begin{pgfscope}%
\pgfsys@transformshift{2.342793in}{2.747992in}%
\pgfsys@useobject{currentmarker}{}%
\end{pgfscope}%
\end{pgfscope}%
\begin{pgfscope}%
\pgfsetbuttcap%
\pgfsetroundjoin%
\definecolor{currentfill}{rgb}{0.000000,0.000000,0.000000}%
\pgfsetfillcolor{currentfill}%
\pgfsetlinewidth{0.501875pt}%
\definecolor{currentstroke}{rgb}{0.000000,0.000000,0.000000}%
\pgfsetstrokecolor{currentstroke}%
\pgfsetdash{}{0pt}%
\pgfsys@defobject{currentmarker}{\pgfqpoint{0.000000in}{-0.041667in}}{\pgfqpoint{0.000000in}{0.000000in}}{%
\pgfpathmoveto{\pgfqpoint{0.000000in}{0.000000in}}%
\pgfpathlineto{\pgfqpoint{0.000000in}{-0.041667in}}%
\pgfusepath{stroke,fill}%
}%
\begin{pgfscope}%
\pgfsys@transformshift{2.342793in}{4.374193in}%
\pgfsys@useobject{currentmarker}{}%
\end{pgfscope}%
\end{pgfscope}%
\begin{pgfscope}%
\definecolor{textcolor}{rgb}{0.000000,0.000000,0.000000}%
\pgfsetstrokecolor{textcolor}%
\pgfsetfillcolor{textcolor}%
\pgftext[x=2.342793in,y=2.699381in,,top]{\color{textcolor}\rmfamily\fontsize{10.000000}{12.000000}\selectfont \(\displaystyle {80}\)}%
\end{pgfscope}%
\begin{pgfscope}%
\pgfsetbuttcap%
\pgfsetroundjoin%
\definecolor{currentfill}{rgb}{0.000000,0.000000,0.000000}%
\pgfsetfillcolor{currentfill}%
\pgfsetlinewidth{0.501875pt}%
\definecolor{currentstroke}{rgb}{0.000000,0.000000,0.000000}%
\pgfsetstrokecolor{currentstroke}%
\pgfsetdash{}{0pt}%
\pgfsys@defobject{currentmarker}{\pgfqpoint{0.000000in}{0.000000in}}{\pgfqpoint{0.000000in}{0.020833in}}{%
\pgfpathmoveto{\pgfqpoint{0.000000in}{0.000000in}}%
\pgfpathlineto{\pgfqpoint{0.000000in}{0.020833in}}%
\pgfusepath{stroke,fill}%
}%
\begin{pgfscope}%
\pgfsys@transformshift{0.652646in}{2.747992in}%
\pgfsys@useobject{currentmarker}{}%
\end{pgfscope}%
\end{pgfscope}%
\begin{pgfscope}%
\pgfsetbuttcap%
\pgfsetroundjoin%
\definecolor{currentfill}{rgb}{0.000000,0.000000,0.000000}%
\pgfsetfillcolor{currentfill}%
\pgfsetlinewidth{0.501875pt}%
\definecolor{currentstroke}{rgb}{0.000000,0.000000,0.000000}%
\pgfsetstrokecolor{currentstroke}%
\pgfsetdash{}{0pt}%
\pgfsys@defobject{currentmarker}{\pgfqpoint{0.000000in}{-0.020833in}}{\pgfqpoint{0.000000in}{0.000000in}}{%
\pgfpathmoveto{\pgfqpoint{0.000000in}{0.000000in}}%
\pgfpathlineto{\pgfqpoint{0.000000in}{-0.020833in}}%
\pgfusepath{stroke,fill}%
}%
\begin{pgfscope}%
\pgfsys@transformshift{0.652646in}{4.374193in}%
\pgfsys@useobject{currentmarker}{}%
\end{pgfscope}%
\end{pgfscope}%
\begin{pgfscope}%
\pgfsetbuttcap%
\pgfsetroundjoin%
\definecolor{currentfill}{rgb}{0.000000,0.000000,0.000000}%
\pgfsetfillcolor{currentfill}%
\pgfsetlinewidth{0.501875pt}%
\definecolor{currentstroke}{rgb}{0.000000,0.000000,0.000000}%
\pgfsetstrokecolor{currentstroke}%
\pgfsetdash{}{0pt}%
\pgfsys@defobject{currentmarker}{\pgfqpoint{0.000000in}{0.000000in}}{\pgfqpoint{0.000000in}{0.020833in}}{%
\pgfpathmoveto{\pgfqpoint{0.000000in}{0.000000in}}%
\pgfpathlineto{\pgfqpoint{0.000000in}{0.020833in}}%
\pgfusepath{stroke,fill}%
}%
\begin{pgfscope}%
\pgfsys@transformshift{0.765323in}{2.747992in}%
\pgfsys@useobject{currentmarker}{}%
\end{pgfscope}%
\end{pgfscope}%
\begin{pgfscope}%
\pgfsetbuttcap%
\pgfsetroundjoin%
\definecolor{currentfill}{rgb}{0.000000,0.000000,0.000000}%
\pgfsetfillcolor{currentfill}%
\pgfsetlinewidth{0.501875pt}%
\definecolor{currentstroke}{rgb}{0.000000,0.000000,0.000000}%
\pgfsetstrokecolor{currentstroke}%
\pgfsetdash{}{0pt}%
\pgfsys@defobject{currentmarker}{\pgfqpoint{0.000000in}{-0.020833in}}{\pgfqpoint{0.000000in}{0.000000in}}{%
\pgfpathmoveto{\pgfqpoint{0.000000in}{0.000000in}}%
\pgfpathlineto{\pgfqpoint{0.000000in}{-0.020833in}}%
\pgfusepath{stroke,fill}%
}%
\begin{pgfscope}%
\pgfsys@transformshift{0.765323in}{4.374193in}%
\pgfsys@useobject{currentmarker}{}%
\end{pgfscope}%
\end{pgfscope}%
\begin{pgfscope}%
\pgfsetbuttcap%
\pgfsetroundjoin%
\definecolor{currentfill}{rgb}{0.000000,0.000000,0.000000}%
\pgfsetfillcolor{currentfill}%
\pgfsetlinewidth{0.501875pt}%
\definecolor{currentstroke}{rgb}{0.000000,0.000000,0.000000}%
\pgfsetstrokecolor{currentstroke}%
\pgfsetdash{}{0pt}%
\pgfsys@defobject{currentmarker}{\pgfqpoint{0.000000in}{0.000000in}}{\pgfqpoint{0.000000in}{0.020833in}}{%
\pgfpathmoveto{\pgfqpoint{0.000000in}{0.000000in}}%
\pgfpathlineto{\pgfqpoint{0.000000in}{0.020833in}}%
\pgfusepath{stroke,fill}%
}%
\begin{pgfscope}%
\pgfsys@transformshift{0.877999in}{2.747992in}%
\pgfsys@useobject{currentmarker}{}%
\end{pgfscope}%
\end{pgfscope}%
\begin{pgfscope}%
\pgfsetbuttcap%
\pgfsetroundjoin%
\definecolor{currentfill}{rgb}{0.000000,0.000000,0.000000}%
\pgfsetfillcolor{currentfill}%
\pgfsetlinewidth{0.501875pt}%
\definecolor{currentstroke}{rgb}{0.000000,0.000000,0.000000}%
\pgfsetstrokecolor{currentstroke}%
\pgfsetdash{}{0pt}%
\pgfsys@defobject{currentmarker}{\pgfqpoint{0.000000in}{-0.020833in}}{\pgfqpoint{0.000000in}{0.000000in}}{%
\pgfpathmoveto{\pgfqpoint{0.000000in}{0.000000in}}%
\pgfpathlineto{\pgfqpoint{0.000000in}{-0.020833in}}%
\pgfusepath{stroke,fill}%
}%
\begin{pgfscope}%
\pgfsys@transformshift{0.877999in}{4.374193in}%
\pgfsys@useobject{currentmarker}{}%
\end{pgfscope}%
\end{pgfscope}%
\begin{pgfscope}%
\pgfsetbuttcap%
\pgfsetroundjoin%
\definecolor{currentfill}{rgb}{0.000000,0.000000,0.000000}%
\pgfsetfillcolor{currentfill}%
\pgfsetlinewidth{0.501875pt}%
\definecolor{currentstroke}{rgb}{0.000000,0.000000,0.000000}%
\pgfsetstrokecolor{currentstroke}%
\pgfsetdash{}{0pt}%
\pgfsys@defobject{currentmarker}{\pgfqpoint{0.000000in}{0.000000in}}{\pgfqpoint{0.000000in}{0.020833in}}{%
\pgfpathmoveto{\pgfqpoint{0.000000in}{0.000000in}}%
\pgfpathlineto{\pgfqpoint{0.000000in}{0.020833in}}%
\pgfusepath{stroke,fill}%
}%
\begin{pgfscope}%
\pgfsys@transformshift{1.103352in}{2.747992in}%
\pgfsys@useobject{currentmarker}{}%
\end{pgfscope}%
\end{pgfscope}%
\begin{pgfscope}%
\pgfsetbuttcap%
\pgfsetroundjoin%
\definecolor{currentfill}{rgb}{0.000000,0.000000,0.000000}%
\pgfsetfillcolor{currentfill}%
\pgfsetlinewidth{0.501875pt}%
\definecolor{currentstroke}{rgb}{0.000000,0.000000,0.000000}%
\pgfsetstrokecolor{currentstroke}%
\pgfsetdash{}{0pt}%
\pgfsys@defobject{currentmarker}{\pgfqpoint{0.000000in}{-0.020833in}}{\pgfqpoint{0.000000in}{0.000000in}}{%
\pgfpathmoveto{\pgfqpoint{0.000000in}{0.000000in}}%
\pgfpathlineto{\pgfqpoint{0.000000in}{-0.020833in}}%
\pgfusepath{stroke,fill}%
}%
\begin{pgfscope}%
\pgfsys@transformshift{1.103352in}{4.374193in}%
\pgfsys@useobject{currentmarker}{}%
\end{pgfscope}%
\end{pgfscope}%
\begin{pgfscope}%
\pgfsetbuttcap%
\pgfsetroundjoin%
\definecolor{currentfill}{rgb}{0.000000,0.000000,0.000000}%
\pgfsetfillcolor{currentfill}%
\pgfsetlinewidth{0.501875pt}%
\definecolor{currentstroke}{rgb}{0.000000,0.000000,0.000000}%
\pgfsetstrokecolor{currentstroke}%
\pgfsetdash{}{0pt}%
\pgfsys@defobject{currentmarker}{\pgfqpoint{0.000000in}{0.000000in}}{\pgfqpoint{0.000000in}{0.020833in}}{%
\pgfpathmoveto{\pgfqpoint{0.000000in}{0.000000in}}%
\pgfpathlineto{\pgfqpoint{0.000000in}{0.020833in}}%
\pgfusepath{stroke,fill}%
}%
\begin{pgfscope}%
\pgfsys@transformshift{1.216028in}{2.747992in}%
\pgfsys@useobject{currentmarker}{}%
\end{pgfscope}%
\end{pgfscope}%
\begin{pgfscope}%
\pgfsetbuttcap%
\pgfsetroundjoin%
\definecolor{currentfill}{rgb}{0.000000,0.000000,0.000000}%
\pgfsetfillcolor{currentfill}%
\pgfsetlinewidth{0.501875pt}%
\definecolor{currentstroke}{rgb}{0.000000,0.000000,0.000000}%
\pgfsetstrokecolor{currentstroke}%
\pgfsetdash{}{0pt}%
\pgfsys@defobject{currentmarker}{\pgfqpoint{0.000000in}{-0.020833in}}{\pgfqpoint{0.000000in}{0.000000in}}{%
\pgfpathmoveto{\pgfqpoint{0.000000in}{0.000000in}}%
\pgfpathlineto{\pgfqpoint{0.000000in}{-0.020833in}}%
\pgfusepath{stroke,fill}%
}%
\begin{pgfscope}%
\pgfsys@transformshift{1.216028in}{4.374193in}%
\pgfsys@useobject{currentmarker}{}%
\end{pgfscope}%
\end{pgfscope}%
\begin{pgfscope}%
\pgfsetbuttcap%
\pgfsetroundjoin%
\definecolor{currentfill}{rgb}{0.000000,0.000000,0.000000}%
\pgfsetfillcolor{currentfill}%
\pgfsetlinewidth{0.501875pt}%
\definecolor{currentstroke}{rgb}{0.000000,0.000000,0.000000}%
\pgfsetstrokecolor{currentstroke}%
\pgfsetdash{}{0pt}%
\pgfsys@defobject{currentmarker}{\pgfqpoint{0.000000in}{0.000000in}}{\pgfqpoint{0.000000in}{0.020833in}}{%
\pgfpathmoveto{\pgfqpoint{0.000000in}{0.000000in}}%
\pgfpathlineto{\pgfqpoint{0.000000in}{0.020833in}}%
\pgfusepath{stroke,fill}%
}%
\begin{pgfscope}%
\pgfsys@transformshift{1.328705in}{2.747992in}%
\pgfsys@useobject{currentmarker}{}%
\end{pgfscope}%
\end{pgfscope}%
\begin{pgfscope}%
\pgfsetbuttcap%
\pgfsetroundjoin%
\definecolor{currentfill}{rgb}{0.000000,0.000000,0.000000}%
\pgfsetfillcolor{currentfill}%
\pgfsetlinewidth{0.501875pt}%
\definecolor{currentstroke}{rgb}{0.000000,0.000000,0.000000}%
\pgfsetstrokecolor{currentstroke}%
\pgfsetdash{}{0pt}%
\pgfsys@defobject{currentmarker}{\pgfqpoint{0.000000in}{-0.020833in}}{\pgfqpoint{0.000000in}{0.000000in}}{%
\pgfpathmoveto{\pgfqpoint{0.000000in}{0.000000in}}%
\pgfpathlineto{\pgfqpoint{0.000000in}{-0.020833in}}%
\pgfusepath{stroke,fill}%
}%
\begin{pgfscope}%
\pgfsys@transformshift{1.328705in}{4.374193in}%
\pgfsys@useobject{currentmarker}{}%
\end{pgfscope}%
\end{pgfscope}%
\begin{pgfscope}%
\pgfsetbuttcap%
\pgfsetroundjoin%
\definecolor{currentfill}{rgb}{0.000000,0.000000,0.000000}%
\pgfsetfillcolor{currentfill}%
\pgfsetlinewidth{0.501875pt}%
\definecolor{currentstroke}{rgb}{0.000000,0.000000,0.000000}%
\pgfsetstrokecolor{currentstroke}%
\pgfsetdash{}{0pt}%
\pgfsys@defobject{currentmarker}{\pgfqpoint{0.000000in}{0.000000in}}{\pgfqpoint{0.000000in}{0.020833in}}{%
\pgfpathmoveto{\pgfqpoint{0.000000in}{0.000000in}}%
\pgfpathlineto{\pgfqpoint{0.000000in}{0.020833in}}%
\pgfusepath{stroke,fill}%
}%
\begin{pgfscope}%
\pgfsys@transformshift{1.554058in}{2.747992in}%
\pgfsys@useobject{currentmarker}{}%
\end{pgfscope}%
\end{pgfscope}%
\begin{pgfscope}%
\pgfsetbuttcap%
\pgfsetroundjoin%
\definecolor{currentfill}{rgb}{0.000000,0.000000,0.000000}%
\pgfsetfillcolor{currentfill}%
\pgfsetlinewidth{0.501875pt}%
\definecolor{currentstroke}{rgb}{0.000000,0.000000,0.000000}%
\pgfsetstrokecolor{currentstroke}%
\pgfsetdash{}{0pt}%
\pgfsys@defobject{currentmarker}{\pgfqpoint{0.000000in}{-0.020833in}}{\pgfqpoint{0.000000in}{0.000000in}}{%
\pgfpathmoveto{\pgfqpoint{0.000000in}{0.000000in}}%
\pgfpathlineto{\pgfqpoint{0.000000in}{-0.020833in}}%
\pgfusepath{stroke,fill}%
}%
\begin{pgfscope}%
\pgfsys@transformshift{1.554058in}{4.374193in}%
\pgfsys@useobject{currentmarker}{}%
\end{pgfscope}%
\end{pgfscope}%
\begin{pgfscope}%
\pgfsetbuttcap%
\pgfsetroundjoin%
\definecolor{currentfill}{rgb}{0.000000,0.000000,0.000000}%
\pgfsetfillcolor{currentfill}%
\pgfsetlinewidth{0.501875pt}%
\definecolor{currentstroke}{rgb}{0.000000,0.000000,0.000000}%
\pgfsetstrokecolor{currentstroke}%
\pgfsetdash{}{0pt}%
\pgfsys@defobject{currentmarker}{\pgfqpoint{0.000000in}{0.000000in}}{\pgfqpoint{0.000000in}{0.020833in}}{%
\pgfpathmoveto{\pgfqpoint{0.000000in}{0.000000in}}%
\pgfpathlineto{\pgfqpoint{0.000000in}{0.020833in}}%
\pgfusepath{stroke,fill}%
}%
\begin{pgfscope}%
\pgfsys@transformshift{1.666734in}{2.747992in}%
\pgfsys@useobject{currentmarker}{}%
\end{pgfscope}%
\end{pgfscope}%
\begin{pgfscope}%
\pgfsetbuttcap%
\pgfsetroundjoin%
\definecolor{currentfill}{rgb}{0.000000,0.000000,0.000000}%
\pgfsetfillcolor{currentfill}%
\pgfsetlinewidth{0.501875pt}%
\definecolor{currentstroke}{rgb}{0.000000,0.000000,0.000000}%
\pgfsetstrokecolor{currentstroke}%
\pgfsetdash{}{0pt}%
\pgfsys@defobject{currentmarker}{\pgfqpoint{0.000000in}{-0.020833in}}{\pgfqpoint{0.000000in}{0.000000in}}{%
\pgfpathmoveto{\pgfqpoint{0.000000in}{0.000000in}}%
\pgfpathlineto{\pgfqpoint{0.000000in}{-0.020833in}}%
\pgfusepath{stroke,fill}%
}%
\begin{pgfscope}%
\pgfsys@transformshift{1.666734in}{4.374193in}%
\pgfsys@useobject{currentmarker}{}%
\end{pgfscope}%
\end{pgfscope}%
\begin{pgfscope}%
\pgfsetbuttcap%
\pgfsetroundjoin%
\definecolor{currentfill}{rgb}{0.000000,0.000000,0.000000}%
\pgfsetfillcolor{currentfill}%
\pgfsetlinewidth{0.501875pt}%
\definecolor{currentstroke}{rgb}{0.000000,0.000000,0.000000}%
\pgfsetstrokecolor{currentstroke}%
\pgfsetdash{}{0pt}%
\pgfsys@defobject{currentmarker}{\pgfqpoint{0.000000in}{0.000000in}}{\pgfqpoint{0.000000in}{0.020833in}}{%
\pgfpathmoveto{\pgfqpoint{0.000000in}{0.000000in}}%
\pgfpathlineto{\pgfqpoint{0.000000in}{0.020833in}}%
\pgfusepath{stroke,fill}%
}%
\begin{pgfscope}%
\pgfsys@transformshift{1.779411in}{2.747992in}%
\pgfsys@useobject{currentmarker}{}%
\end{pgfscope}%
\end{pgfscope}%
\begin{pgfscope}%
\pgfsetbuttcap%
\pgfsetroundjoin%
\definecolor{currentfill}{rgb}{0.000000,0.000000,0.000000}%
\pgfsetfillcolor{currentfill}%
\pgfsetlinewidth{0.501875pt}%
\definecolor{currentstroke}{rgb}{0.000000,0.000000,0.000000}%
\pgfsetstrokecolor{currentstroke}%
\pgfsetdash{}{0pt}%
\pgfsys@defobject{currentmarker}{\pgfqpoint{0.000000in}{-0.020833in}}{\pgfqpoint{0.000000in}{0.000000in}}{%
\pgfpathmoveto{\pgfqpoint{0.000000in}{0.000000in}}%
\pgfpathlineto{\pgfqpoint{0.000000in}{-0.020833in}}%
\pgfusepath{stroke,fill}%
}%
\begin{pgfscope}%
\pgfsys@transformshift{1.779411in}{4.374193in}%
\pgfsys@useobject{currentmarker}{}%
\end{pgfscope}%
\end{pgfscope}%
\begin{pgfscope}%
\pgfsetbuttcap%
\pgfsetroundjoin%
\definecolor{currentfill}{rgb}{0.000000,0.000000,0.000000}%
\pgfsetfillcolor{currentfill}%
\pgfsetlinewidth{0.501875pt}%
\definecolor{currentstroke}{rgb}{0.000000,0.000000,0.000000}%
\pgfsetstrokecolor{currentstroke}%
\pgfsetdash{}{0pt}%
\pgfsys@defobject{currentmarker}{\pgfqpoint{0.000000in}{0.000000in}}{\pgfqpoint{0.000000in}{0.020833in}}{%
\pgfpathmoveto{\pgfqpoint{0.000000in}{0.000000in}}%
\pgfpathlineto{\pgfqpoint{0.000000in}{0.020833in}}%
\pgfusepath{stroke,fill}%
}%
\begin{pgfscope}%
\pgfsys@transformshift{2.004764in}{2.747992in}%
\pgfsys@useobject{currentmarker}{}%
\end{pgfscope}%
\end{pgfscope}%
\begin{pgfscope}%
\pgfsetbuttcap%
\pgfsetroundjoin%
\definecolor{currentfill}{rgb}{0.000000,0.000000,0.000000}%
\pgfsetfillcolor{currentfill}%
\pgfsetlinewidth{0.501875pt}%
\definecolor{currentstroke}{rgb}{0.000000,0.000000,0.000000}%
\pgfsetstrokecolor{currentstroke}%
\pgfsetdash{}{0pt}%
\pgfsys@defobject{currentmarker}{\pgfqpoint{0.000000in}{-0.020833in}}{\pgfqpoint{0.000000in}{0.000000in}}{%
\pgfpathmoveto{\pgfqpoint{0.000000in}{0.000000in}}%
\pgfpathlineto{\pgfqpoint{0.000000in}{-0.020833in}}%
\pgfusepath{stroke,fill}%
}%
\begin{pgfscope}%
\pgfsys@transformshift{2.004764in}{4.374193in}%
\pgfsys@useobject{currentmarker}{}%
\end{pgfscope}%
\end{pgfscope}%
\begin{pgfscope}%
\pgfsetbuttcap%
\pgfsetroundjoin%
\definecolor{currentfill}{rgb}{0.000000,0.000000,0.000000}%
\pgfsetfillcolor{currentfill}%
\pgfsetlinewidth{0.501875pt}%
\definecolor{currentstroke}{rgb}{0.000000,0.000000,0.000000}%
\pgfsetstrokecolor{currentstroke}%
\pgfsetdash{}{0pt}%
\pgfsys@defobject{currentmarker}{\pgfqpoint{0.000000in}{0.000000in}}{\pgfqpoint{0.000000in}{0.020833in}}{%
\pgfpathmoveto{\pgfqpoint{0.000000in}{0.000000in}}%
\pgfpathlineto{\pgfqpoint{0.000000in}{0.020833in}}%
\pgfusepath{stroke,fill}%
}%
\begin{pgfscope}%
\pgfsys@transformshift{2.117440in}{2.747992in}%
\pgfsys@useobject{currentmarker}{}%
\end{pgfscope}%
\end{pgfscope}%
\begin{pgfscope}%
\pgfsetbuttcap%
\pgfsetroundjoin%
\definecolor{currentfill}{rgb}{0.000000,0.000000,0.000000}%
\pgfsetfillcolor{currentfill}%
\pgfsetlinewidth{0.501875pt}%
\definecolor{currentstroke}{rgb}{0.000000,0.000000,0.000000}%
\pgfsetstrokecolor{currentstroke}%
\pgfsetdash{}{0pt}%
\pgfsys@defobject{currentmarker}{\pgfqpoint{0.000000in}{-0.020833in}}{\pgfqpoint{0.000000in}{0.000000in}}{%
\pgfpathmoveto{\pgfqpoint{0.000000in}{0.000000in}}%
\pgfpathlineto{\pgfqpoint{0.000000in}{-0.020833in}}%
\pgfusepath{stroke,fill}%
}%
\begin{pgfscope}%
\pgfsys@transformshift{2.117440in}{4.374193in}%
\pgfsys@useobject{currentmarker}{}%
\end{pgfscope}%
\end{pgfscope}%
\begin{pgfscope}%
\pgfsetbuttcap%
\pgfsetroundjoin%
\definecolor{currentfill}{rgb}{0.000000,0.000000,0.000000}%
\pgfsetfillcolor{currentfill}%
\pgfsetlinewidth{0.501875pt}%
\definecolor{currentstroke}{rgb}{0.000000,0.000000,0.000000}%
\pgfsetstrokecolor{currentstroke}%
\pgfsetdash{}{0pt}%
\pgfsys@defobject{currentmarker}{\pgfqpoint{0.000000in}{0.000000in}}{\pgfqpoint{0.000000in}{0.020833in}}{%
\pgfpathmoveto{\pgfqpoint{0.000000in}{0.000000in}}%
\pgfpathlineto{\pgfqpoint{0.000000in}{0.020833in}}%
\pgfusepath{stroke,fill}%
}%
\begin{pgfscope}%
\pgfsys@transformshift{2.230116in}{2.747992in}%
\pgfsys@useobject{currentmarker}{}%
\end{pgfscope}%
\end{pgfscope}%
\begin{pgfscope}%
\pgfsetbuttcap%
\pgfsetroundjoin%
\definecolor{currentfill}{rgb}{0.000000,0.000000,0.000000}%
\pgfsetfillcolor{currentfill}%
\pgfsetlinewidth{0.501875pt}%
\definecolor{currentstroke}{rgb}{0.000000,0.000000,0.000000}%
\pgfsetstrokecolor{currentstroke}%
\pgfsetdash{}{0pt}%
\pgfsys@defobject{currentmarker}{\pgfqpoint{0.000000in}{-0.020833in}}{\pgfqpoint{0.000000in}{0.000000in}}{%
\pgfpathmoveto{\pgfqpoint{0.000000in}{0.000000in}}%
\pgfpathlineto{\pgfqpoint{0.000000in}{-0.020833in}}%
\pgfusepath{stroke,fill}%
}%
\begin{pgfscope}%
\pgfsys@transformshift{2.230116in}{4.374193in}%
\pgfsys@useobject{currentmarker}{}%
\end{pgfscope}%
\end{pgfscope}%
\begin{pgfscope}%
\pgfsetbuttcap%
\pgfsetroundjoin%
\definecolor{currentfill}{rgb}{0.000000,0.000000,0.000000}%
\pgfsetfillcolor{currentfill}%
\pgfsetlinewidth{0.501875pt}%
\definecolor{currentstroke}{rgb}{0.000000,0.000000,0.000000}%
\pgfsetstrokecolor{currentstroke}%
\pgfsetdash{}{0pt}%
\pgfsys@defobject{currentmarker}{\pgfqpoint{0.000000in}{0.000000in}}{\pgfqpoint{0.000000in}{0.020833in}}{%
\pgfpathmoveto{\pgfqpoint{0.000000in}{0.000000in}}%
\pgfpathlineto{\pgfqpoint{0.000000in}{0.020833in}}%
\pgfusepath{stroke,fill}%
}%
\begin{pgfscope}%
\pgfsys@transformshift{2.455469in}{2.747992in}%
\pgfsys@useobject{currentmarker}{}%
\end{pgfscope}%
\end{pgfscope}%
\begin{pgfscope}%
\pgfsetbuttcap%
\pgfsetroundjoin%
\definecolor{currentfill}{rgb}{0.000000,0.000000,0.000000}%
\pgfsetfillcolor{currentfill}%
\pgfsetlinewidth{0.501875pt}%
\definecolor{currentstroke}{rgb}{0.000000,0.000000,0.000000}%
\pgfsetstrokecolor{currentstroke}%
\pgfsetdash{}{0pt}%
\pgfsys@defobject{currentmarker}{\pgfqpoint{0.000000in}{-0.020833in}}{\pgfqpoint{0.000000in}{0.000000in}}{%
\pgfpathmoveto{\pgfqpoint{0.000000in}{0.000000in}}%
\pgfpathlineto{\pgfqpoint{0.000000in}{-0.020833in}}%
\pgfusepath{stroke,fill}%
}%
\begin{pgfscope}%
\pgfsys@transformshift{2.455469in}{4.374193in}%
\pgfsys@useobject{currentmarker}{}%
\end{pgfscope}%
\end{pgfscope}%
\begin{pgfscope}%
\pgfsetbuttcap%
\pgfsetroundjoin%
\definecolor{currentfill}{rgb}{0.000000,0.000000,0.000000}%
\pgfsetfillcolor{currentfill}%
\pgfsetlinewidth{0.501875pt}%
\definecolor{currentstroke}{rgb}{0.000000,0.000000,0.000000}%
\pgfsetstrokecolor{currentstroke}%
\pgfsetdash{}{0pt}%
\pgfsys@defobject{currentmarker}{\pgfqpoint{0.000000in}{0.000000in}}{\pgfqpoint{0.000000in}{0.020833in}}{%
\pgfpathmoveto{\pgfqpoint{0.000000in}{0.000000in}}%
\pgfpathlineto{\pgfqpoint{0.000000in}{0.020833in}}%
\pgfusepath{stroke,fill}%
}%
\begin{pgfscope}%
\pgfsys@transformshift{2.568146in}{2.747992in}%
\pgfsys@useobject{currentmarker}{}%
\end{pgfscope}%
\end{pgfscope}%
\begin{pgfscope}%
\pgfsetbuttcap%
\pgfsetroundjoin%
\definecolor{currentfill}{rgb}{0.000000,0.000000,0.000000}%
\pgfsetfillcolor{currentfill}%
\pgfsetlinewidth{0.501875pt}%
\definecolor{currentstroke}{rgb}{0.000000,0.000000,0.000000}%
\pgfsetstrokecolor{currentstroke}%
\pgfsetdash{}{0pt}%
\pgfsys@defobject{currentmarker}{\pgfqpoint{0.000000in}{-0.020833in}}{\pgfqpoint{0.000000in}{0.000000in}}{%
\pgfpathmoveto{\pgfqpoint{0.000000in}{0.000000in}}%
\pgfpathlineto{\pgfqpoint{0.000000in}{-0.020833in}}%
\pgfusepath{stroke,fill}%
}%
\begin{pgfscope}%
\pgfsys@transformshift{2.568146in}{4.374193in}%
\pgfsys@useobject{currentmarker}{}%
\end{pgfscope}%
\end{pgfscope}%
\begin{pgfscope}%
\pgfsetbuttcap%
\pgfsetroundjoin%
\definecolor{currentfill}{rgb}{0.000000,0.000000,0.000000}%
\pgfsetfillcolor{currentfill}%
\pgfsetlinewidth{0.501875pt}%
\definecolor{currentstroke}{rgb}{0.000000,0.000000,0.000000}%
\pgfsetstrokecolor{currentstroke}%
\pgfsetdash{}{0pt}%
\pgfsys@defobject{currentmarker}{\pgfqpoint{0.000000in}{0.000000in}}{\pgfqpoint{0.000000in}{0.020833in}}{%
\pgfpathmoveto{\pgfqpoint{0.000000in}{0.000000in}}%
\pgfpathlineto{\pgfqpoint{0.000000in}{0.020833in}}%
\pgfusepath{stroke,fill}%
}%
\begin{pgfscope}%
\pgfsys@transformshift{2.680822in}{2.747992in}%
\pgfsys@useobject{currentmarker}{}%
\end{pgfscope}%
\end{pgfscope}%
\begin{pgfscope}%
\pgfsetbuttcap%
\pgfsetroundjoin%
\definecolor{currentfill}{rgb}{0.000000,0.000000,0.000000}%
\pgfsetfillcolor{currentfill}%
\pgfsetlinewidth{0.501875pt}%
\definecolor{currentstroke}{rgb}{0.000000,0.000000,0.000000}%
\pgfsetstrokecolor{currentstroke}%
\pgfsetdash{}{0pt}%
\pgfsys@defobject{currentmarker}{\pgfqpoint{0.000000in}{-0.020833in}}{\pgfqpoint{0.000000in}{0.000000in}}{%
\pgfpathmoveto{\pgfqpoint{0.000000in}{0.000000in}}%
\pgfpathlineto{\pgfqpoint{0.000000in}{-0.020833in}}%
\pgfusepath{stroke,fill}%
}%
\begin{pgfscope}%
\pgfsys@transformshift{2.680822in}{4.374193in}%
\pgfsys@useobject{currentmarker}{}%
\end{pgfscope}%
\end{pgfscope}%
\begin{pgfscope}%
\pgfsetbuttcap%
\pgfsetroundjoin%
\definecolor{currentfill}{rgb}{0.000000,0.000000,0.000000}%
\pgfsetfillcolor{currentfill}%
\pgfsetlinewidth{0.501875pt}%
\definecolor{currentstroke}{rgb}{0.000000,0.000000,0.000000}%
\pgfsetstrokecolor{currentstroke}%
\pgfsetdash{}{0pt}%
\pgfsys@defobject{currentmarker}{\pgfqpoint{0.000000in}{0.000000in}}{\pgfqpoint{0.000000in}{0.020833in}}{%
\pgfpathmoveto{\pgfqpoint{0.000000in}{0.000000in}}%
\pgfpathlineto{\pgfqpoint{0.000000in}{0.020833in}}%
\pgfusepath{stroke,fill}%
}%
\begin{pgfscope}%
\pgfsys@transformshift{2.793499in}{2.747992in}%
\pgfsys@useobject{currentmarker}{}%
\end{pgfscope}%
\end{pgfscope}%
\begin{pgfscope}%
\pgfsetbuttcap%
\pgfsetroundjoin%
\definecolor{currentfill}{rgb}{0.000000,0.000000,0.000000}%
\pgfsetfillcolor{currentfill}%
\pgfsetlinewidth{0.501875pt}%
\definecolor{currentstroke}{rgb}{0.000000,0.000000,0.000000}%
\pgfsetstrokecolor{currentstroke}%
\pgfsetdash{}{0pt}%
\pgfsys@defobject{currentmarker}{\pgfqpoint{0.000000in}{-0.020833in}}{\pgfqpoint{0.000000in}{0.000000in}}{%
\pgfpathmoveto{\pgfqpoint{0.000000in}{0.000000in}}%
\pgfpathlineto{\pgfqpoint{0.000000in}{-0.020833in}}%
\pgfusepath{stroke,fill}%
}%
\begin{pgfscope}%
\pgfsys@transformshift{2.793499in}{4.374193in}%
\pgfsys@useobject{currentmarker}{}%
\end{pgfscope}%
\end{pgfscope}%
\begin{pgfscope}%
\definecolor{textcolor}{rgb}{0.000000,0.000000,0.000000}%
\pgfsetstrokecolor{textcolor}%
\pgfsetfillcolor{textcolor}%
\pgftext[x=1.678002in,y=2.509413in,,top]{\color{textcolor}\rmfamily\fontsize{10.000000}{12.000000}\selectfont \(\displaystyle K\)}%
\end{pgfscope}%
\begin{pgfscope}%
\pgfsetbuttcap%
\pgfsetroundjoin%
\definecolor{currentfill}{rgb}{0.000000,0.000000,0.000000}%
\pgfsetfillcolor{currentfill}%
\pgfsetlinewidth{0.501875pt}%
\definecolor{currentstroke}{rgb}{0.000000,0.000000,0.000000}%
\pgfsetstrokecolor{currentstroke}%
\pgfsetdash{}{0pt}%
\pgfsys@defobject{currentmarker}{\pgfqpoint{0.000000in}{0.000000in}}{\pgfqpoint{0.041667in}{0.000000in}}{%
\pgfpathmoveto{\pgfqpoint{0.000000in}{0.000000in}}%
\pgfpathlineto{\pgfqpoint{0.041667in}{0.000000in}}%
\pgfusepath{stroke,fill}%
}%
\begin{pgfscope}%
\pgfsys@transformshift{0.539970in}{3.076138in}%
\pgfsys@useobject{currentmarker}{}%
\end{pgfscope}%
\end{pgfscope}%
\begin{pgfscope}%
\pgfsetbuttcap%
\pgfsetroundjoin%
\definecolor{currentfill}{rgb}{0.000000,0.000000,0.000000}%
\pgfsetfillcolor{currentfill}%
\pgfsetlinewidth{0.501875pt}%
\definecolor{currentstroke}{rgb}{0.000000,0.000000,0.000000}%
\pgfsetstrokecolor{currentstroke}%
\pgfsetdash{}{0pt}%
\pgfsys@defobject{currentmarker}{\pgfqpoint{-0.041667in}{0.000000in}}{\pgfqpoint{-0.000000in}{0.000000in}}{%
\pgfpathmoveto{\pgfqpoint{-0.000000in}{0.000000in}}%
\pgfpathlineto{\pgfqpoint{-0.041667in}{0.000000in}}%
\pgfusepath{stroke,fill}%
}%
\begin{pgfscope}%
\pgfsys@transformshift{2.816034in}{3.076138in}%
\pgfsys@useobject{currentmarker}{}%
\end{pgfscope}%
\end{pgfscope}%
\begin{pgfscope}%
\definecolor{textcolor}{rgb}{0.000000,0.000000,0.000000}%
\pgfsetstrokecolor{textcolor}%
\pgfsetfillcolor{textcolor}%
\pgftext[x=0.244444in, y=3.023376in, left, base]{\color{textcolor}\rmfamily\fontsize{10.000000}{12.000000}\selectfont \(\displaystyle {0.85}\)}%
\end{pgfscope}%
\begin{pgfscope}%
\pgfsetbuttcap%
\pgfsetroundjoin%
\definecolor{currentfill}{rgb}{0.000000,0.000000,0.000000}%
\pgfsetfillcolor{currentfill}%
\pgfsetlinewidth{0.501875pt}%
\definecolor{currentstroke}{rgb}{0.000000,0.000000,0.000000}%
\pgfsetstrokecolor{currentstroke}%
\pgfsetdash{}{0pt}%
\pgfsys@defobject{currentmarker}{\pgfqpoint{0.000000in}{0.000000in}}{\pgfqpoint{0.041667in}{0.000000in}}{%
\pgfpathmoveto{\pgfqpoint{0.000000in}{0.000000in}}%
\pgfpathlineto{\pgfqpoint{0.041667in}{0.000000in}}%
\pgfusepath{stroke,fill}%
}%
\begin{pgfscope}%
\pgfsys@transformshift{0.539970in}{3.491305in}%
\pgfsys@useobject{currentmarker}{}%
\end{pgfscope}%
\end{pgfscope}%
\begin{pgfscope}%
\pgfsetbuttcap%
\pgfsetroundjoin%
\definecolor{currentfill}{rgb}{0.000000,0.000000,0.000000}%
\pgfsetfillcolor{currentfill}%
\pgfsetlinewidth{0.501875pt}%
\definecolor{currentstroke}{rgb}{0.000000,0.000000,0.000000}%
\pgfsetstrokecolor{currentstroke}%
\pgfsetdash{}{0pt}%
\pgfsys@defobject{currentmarker}{\pgfqpoint{-0.041667in}{0.000000in}}{\pgfqpoint{-0.000000in}{0.000000in}}{%
\pgfpathmoveto{\pgfqpoint{-0.000000in}{0.000000in}}%
\pgfpathlineto{\pgfqpoint{-0.041667in}{0.000000in}}%
\pgfusepath{stroke,fill}%
}%
\begin{pgfscope}%
\pgfsys@transformshift{2.816034in}{3.491305in}%
\pgfsys@useobject{currentmarker}{}%
\end{pgfscope}%
\end{pgfscope}%
\begin{pgfscope}%
\definecolor{textcolor}{rgb}{0.000000,0.000000,0.000000}%
\pgfsetstrokecolor{textcolor}%
\pgfsetfillcolor{textcolor}%
\pgftext[x=0.244444in, y=3.438544in, left, base]{\color{textcolor}\rmfamily\fontsize{10.000000}{12.000000}\selectfont \(\displaystyle {0.90}\)}%
\end{pgfscope}%
\begin{pgfscope}%
\pgfsetbuttcap%
\pgfsetroundjoin%
\definecolor{currentfill}{rgb}{0.000000,0.000000,0.000000}%
\pgfsetfillcolor{currentfill}%
\pgfsetlinewidth{0.501875pt}%
\definecolor{currentstroke}{rgb}{0.000000,0.000000,0.000000}%
\pgfsetstrokecolor{currentstroke}%
\pgfsetdash{}{0pt}%
\pgfsys@defobject{currentmarker}{\pgfqpoint{0.000000in}{0.000000in}}{\pgfqpoint{0.041667in}{0.000000in}}{%
\pgfpathmoveto{\pgfqpoint{0.000000in}{0.000000in}}%
\pgfpathlineto{\pgfqpoint{0.041667in}{0.000000in}}%
\pgfusepath{stroke,fill}%
}%
\begin{pgfscope}%
\pgfsys@transformshift{0.539970in}{3.906473in}%
\pgfsys@useobject{currentmarker}{}%
\end{pgfscope}%
\end{pgfscope}%
\begin{pgfscope}%
\pgfsetbuttcap%
\pgfsetroundjoin%
\definecolor{currentfill}{rgb}{0.000000,0.000000,0.000000}%
\pgfsetfillcolor{currentfill}%
\pgfsetlinewidth{0.501875pt}%
\definecolor{currentstroke}{rgb}{0.000000,0.000000,0.000000}%
\pgfsetstrokecolor{currentstroke}%
\pgfsetdash{}{0pt}%
\pgfsys@defobject{currentmarker}{\pgfqpoint{-0.041667in}{0.000000in}}{\pgfqpoint{-0.000000in}{0.000000in}}{%
\pgfpathmoveto{\pgfqpoint{-0.000000in}{0.000000in}}%
\pgfpathlineto{\pgfqpoint{-0.041667in}{0.000000in}}%
\pgfusepath{stroke,fill}%
}%
\begin{pgfscope}%
\pgfsys@transformshift{2.816034in}{3.906473in}%
\pgfsys@useobject{currentmarker}{}%
\end{pgfscope}%
\end{pgfscope}%
\begin{pgfscope}%
\definecolor{textcolor}{rgb}{0.000000,0.000000,0.000000}%
\pgfsetstrokecolor{textcolor}%
\pgfsetfillcolor{textcolor}%
\pgftext[x=0.244444in, y=3.853712in, left, base]{\color{textcolor}\rmfamily\fontsize{10.000000}{12.000000}\selectfont \(\displaystyle {0.95}\)}%
\end{pgfscope}%
\begin{pgfscope}%
\pgfsetbuttcap%
\pgfsetroundjoin%
\definecolor{currentfill}{rgb}{0.000000,0.000000,0.000000}%
\pgfsetfillcolor{currentfill}%
\pgfsetlinewidth{0.501875pt}%
\definecolor{currentstroke}{rgb}{0.000000,0.000000,0.000000}%
\pgfsetstrokecolor{currentstroke}%
\pgfsetdash{}{0pt}%
\pgfsys@defobject{currentmarker}{\pgfqpoint{0.000000in}{0.000000in}}{\pgfqpoint{0.041667in}{0.000000in}}{%
\pgfpathmoveto{\pgfqpoint{0.000000in}{0.000000in}}%
\pgfpathlineto{\pgfqpoint{0.041667in}{0.000000in}}%
\pgfusepath{stroke,fill}%
}%
\begin{pgfscope}%
\pgfsys@transformshift{0.539970in}{4.321641in}%
\pgfsys@useobject{currentmarker}{}%
\end{pgfscope}%
\end{pgfscope}%
\begin{pgfscope}%
\pgfsetbuttcap%
\pgfsetroundjoin%
\definecolor{currentfill}{rgb}{0.000000,0.000000,0.000000}%
\pgfsetfillcolor{currentfill}%
\pgfsetlinewidth{0.501875pt}%
\definecolor{currentstroke}{rgb}{0.000000,0.000000,0.000000}%
\pgfsetstrokecolor{currentstroke}%
\pgfsetdash{}{0pt}%
\pgfsys@defobject{currentmarker}{\pgfqpoint{-0.041667in}{0.000000in}}{\pgfqpoint{-0.000000in}{0.000000in}}{%
\pgfpathmoveto{\pgfqpoint{-0.000000in}{0.000000in}}%
\pgfpathlineto{\pgfqpoint{-0.041667in}{0.000000in}}%
\pgfusepath{stroke,fill}%
}%
\begin{pgfscope}%
\pgfsys@transformshift{2.816034in}{4.321641in}%
\pgfsys@useobject{currentmarker}{}%
\end{pgfscope}%
\end{pgfscope}%
\begin{pgfscope}%
\definecolor{textcolor}{rgb}{0.000000,0.000000,0.000000}%
\pgfsetstrokecolor{textcolor}%
\pgfsetfillcolor{textcolor}%
\pgftext[x=0.244444in, y=4.268879in, left, base]{\color{textcolor}\rmfamily\fontsize{10.000000}{12.000000}\selectfont \(\displaystyle {1.00}\)}%
\end{pgfscope}%
\begin{pgfscope}%
\pgfsetbuttcap%
\pgfsetroundjoin%
\definecolor{currentfill}{rgb}{0.000000,0.000000,0.000000}%
\pgfsetfillcolor{currentfill}%
\pgfsetlinewidth{0.501875pt}%
\definecolor{currentstroke}{rgb}{0.000000,0.000000,0.000000}%
\pgfsetstrokecolor{currentstroke}%
\pgfsetdash{}{0pt}%
\pgfsys@defobject{currentmarker}{\pgfqpoint{0.000000in}{0.000000in}}{\pgfqpoint{0.020833in}{0.000000in}}{%
\pgfpathmoveto{\pgfqpoint{0.000000in}{0.000000in}}%
\pgfpathlineto{\pgfqpoint{0.020833in}{0.000000in}}%
\pgfusepath{stroke,fill}%
}%
\begin{pgfscope}%
\pgfsys@transformshift{0.539970in}{2.827037in}%
\pgfsys@useobject{currentmarker}{}%
\end{pgfscope}%
\end{pgfscope}%
\begin{pgfscope}%
\pgfsetbuttcap%
\pgfsetroundjoin%
\definecolor{currentfill}{rgb}{0.000000,0.000000,0.000000}%
\pgfsetfillcolor{currentfill}%
\pgfsetlinewidth{0.501875pt}%
\definecolor{currentstroke}{rgb}{0.000000,0.000000,0.000000}%
\pgfsetstrokecolor{currentstroke}%
\pgfsetdash{}{0pt}%
\pgfsys@defobject{currentmarker}{\pgfqpoint{-0.020833in}{0.000000in}}{\pgfqpoint{-0.000000in}{0.000000in}}{%
\pgfpathmoveto{\pgfqpoint{-0.000000in}{0.000000in}}%
\pgfpathlineto{\pgfqpoint{-0.020833in}{0.000000in}}%
\pgfusepath{stroke,fill}%
}%
\begin{pgfscope}%
\pgfsys@transformshift{2.816034in}{2.827037in}%
\pgfsys@useobject{currentmarker}{}%
\end{pgfscope}%
\end{pgfscope}%
\begin{pgfscope}%
\pgfsetbuttcap%
\pgfsetroundjoin%
\definecolor{currentfill}{rgb}{0.000000,0.000000,0.000000}%
\pgfsetfillcolor{currentfill}%
\pgfsetlinewidth{0.501875pt}%
\definecolor{currentstroke}{rgb}{0.000000,0.000000,0.000000}%
\pgfsetstrokecolor{currentstroke}%
\pgfsetdash{}{0pt}%
\pgfsys@defobject{currentmarker}{\pgfqpoint{0.000000in}{0.000000in}}{\pgfqpoint{0.020833in}{0.000000in}}{%
\pgfpathmoveto{\pgfqpoint{0.000000in}{0.000000in}}%
\pgfpathlineto{\pgfqpoint{0.020833in}{0.000000in}}%
\pgfusepath{stroke,fill}%
}%
\begin{pgfscope}%
\pgfsys@transformshift{0.539970in}{2.910071in}%
\pgfsys@useobject{currentmarker}{}%
\end{pgfscope}%
\end{pgfscope}%
\begin{pgfscope}%
\pgfsetbuttcap%
\pgfsetroundjoin%
\definecolor{currentfill}{rgb}{0.000000,0.000000,0.000000}%
\pgfsetfillcolor{currentfill}%
\pgfsetlinewidth{0.501875pt}%
\definecolor{currentstroke}{rgb}{0.000000,0.000000,0.000000}%
\pgfsetstrokecolor{currentstroke}%
\pgfsetdash{}{0pt}%
\pgfsys@defobject{currentmarker}{\pgfqpoint{-0.020833in}{0.000000in}}{\pgfqpoint{-0.000000in}{0.000000in}}{%
\pgfpathmoveto{\pgfqpoint{-0.000000in}{0.000000in}}%
\pgfpathlineto{\pgfqpoint{-0.020833in}{0.000000in}}%
\pgfusepath{stroke,fill}%
}%
\begin{pgfscope}%
\pgfsys@transformshift{2.816034in}{2.910071in}%
\pgfsys@useobject{currentmarker}{}%
\end{pgfscope}%
\end{pgfscope}%
\begin{pgfscope}%
\pgfsetbuttcap%
\pgfsetroundjoin%
\definecolor{currentfill}{rgb}{0.000000,0.000000,0.000000}%
\pgfsetfillcolor{currentfill}%
\pgfsetlinewidth{0.501875pt}%
\definecolor{currentstroke}{rgb}{0.000000,0.000000,0.000000}%
\pgfsetstrokecolor{currentstroke}%
\pgfsetdash{}{0pt}%
\pgfsys@defobject{currentmarker}{\pgfqpoint{0.000000in}{0.000000in}}{\pgfqpoint{0.020833in}{0.000000in}}{%
\pgfpathmoveto{\pgfqpoint{0.000000in}{0.000000in}}%
\pgfpathlineto{\pgfqpoint{0.020833in}{0.000000in}}%
\pgfusepath{stroke,fill}%
}%
\begin{pgfscope}%
\pgfsys@transformshift{0.539970in}{2.993104in}%
\pgfsys@useobject{currentmarker}{}%
\end{pgfscope}%
\end{pgfscope}%
\begin{pgfscope}%
\pgfsetbuttcap%
\pgfsetroundjoin%
\definecolor{currentfill}{rgb}{0.000000,0.000000,0.000000}%
\pgfsetfillcolor{currentfill}%
\pgfsetlinewidth{0.501875pt}%
\definecolor{currentstroke}{rgb}{0.000000,0.000000,0.000000}%
\pgfsetstrokecolor{currentstroke}%
\pgfsetdash{}{0pt}%
\pgfsys@defobject{currentmarker}{\pgfqpoint{-0.020833in}{0.000000in}}{\pgfqpoint{-0.000000in}{0.000000in}}{%
\pgfpathmoveto{\pgfqpoint{-0.000000in}{0.000000in}}%
\pgfpathlineto{\pgfqpoint{-0.020833in}{0.000000in}}%
\pgfusepath{stroke,fill}%
}%
\begin{pgfscope}%
\pgfsys@transformshift{2.816034in}{2.993104in}%
\pgfsys@useobject{currentmarker}{}%
\end{pgfscope}%
\end{pgfscope}%
\begin{pgfscope}%
\pgfsetbuttcap%
\pgfsetroundjoin%
\definecolor{currentfill}{rgb}{0.000000,0.000000,0.000000}%
\pgfsetfillcolor{currentfill}%
\pgfsetlinewidth{0.501875pt}%
\definecolor{currentstroke}{rgb}{0.000000,0.000000,0.000000}%
\pgfsetstrokecolor{currentstroke}%
\pgfsetdash{}{0pt}%
\pgfsys@defobject{currentmarker}{\pgfqpoint{0.000000in}{0.000000in}}{\pgfqpoint{0.020833in}{0.000000in}}{%
\pgfpathmoveto{\pgfqpoint{0.000000in}{0.000000in}}%
\pgfpathlineto{\pgfqpoint{0.020833in}{0.000000in}}%
\pgfusepath{stroke,fill}%
}%
\begin{pgfscope}%
\pgfsys@transformshift{0.539970in}{3.159171in}%
\pgfsys@useobject{currentmarker}{}%
\end{pgfscope}%
\end{pgfscope}%
\begin{pgfscope}%
\pgfsetbuttcap%
\pgfsetroundjoin%
\definecolor{currentfill}{rgb}{0.000000,0.000000,0.000000}%
\pgfsetfillcolor{currentfill}%
\pgfsetlinewidth{0.501875pt}%
\definecolor{currentstroke}{rgb}{0.000000,0.000000,0.000000}%
\pgfsetstrokecolor{currentstroke}%
\pgfsetdash{}{0pt}%
\pgfsys@defobject{currentmarker}{\pgfqpoint{-0.020833in}{0.000000in}}{\pgfqpoint{-0.000000in}{0.000000in}}{%
\pgfpathmoveto{\pgfqpoint{-0.000000in}{0.000000in}}%
\pgfpathlineto{\pgfqpoint{-0.020833in}{0.000000in}}%
\pgfusepath{stroke,fill}%
}%
\begin{pgfscope}%
\pgfsys@transformshift{2.816034in}{3.159171in}%
\pgfsys@useobject{currentmarker}{}%
\end{pgfscope}%
\end{pgfscope}%
\begin{pgfscope}%
\pgfsetbuttcap%
\pgfsetroundjoin%
\definecolor{currentfill}{rgb}{0.000000,0.000000,0.000000}%
\pgfsetfillcolor{currentfill}%
\pgfsetlinewidth{0.501875pt}%
\definecolor{currentstroke}{rgb}{0.000000,0.000000,0.000000}%
\pgfsetstrokecolor{currentstroke}%
\pgfsetdash{}{0pt}%
\pgfsys@defobject{currentmarker}{\pgfqpoint{0.000000in}{0.000000in}}{\pgfqpoint{0.020833in}{0.000000in}}{%
\pgfpathmoveto{\pgfqpoint{0.000000in}{0.000000in}}%
\pgfpathlineto{\pgfqpoint{0.020833in}{0.000000in}}%
\pgfusepath{stroke,fill}%
}%
\begin{pgfscope}%
\pgfsys@transformshift{0.539970in}{3.242205in}%
\pgfsys@useobject{currentmarker}{}%
\end{pgfscope}%
\end{pgfscope}%
\begin{pgfscope}%
\pgfsetbuttcap%
\pgfsetroundjoin%
\definecolor{currentfill}{rgb}{0.000000,0.000000,0.000000}%
\pgfsetfillcolor{currentfill}%
\pgfsetlinewidth{0.501875pt}%
\definecolor{currentstroke}{rgb}{0.000000,0.000000,0.000000}%
\pgfsetstrokecolor{currentstroke}%
\pgfsetdash{}{0pt}%
\pgfsys@defobject{currentmarker}{\pgfqpoint{-0.020833in}{0.000000in}}{\pgfqpoint{-0.000000in}{0.000000in}}{%
\pgfpathmoveto{\pgfqpoint{-0.000000in}{0.000000in}}%
\pgfpathlineto{\pgfqpoint{-0.020833in}{0.000000in}}%
\pgfusepath{stroke,fill}%
}%
\begin{pgfscope}%
\pgfsys@transformshift{2.816034in}{3.242205in}%
\pgfsys@useobject{currentmarker}{}%
\end{pgfscope}%
\end{pgfscope}%
\begin{pgfscope}%
\pgfsetbuttcap%
\pgfsetroundjoin%
\definecolor{currentfill}{rgb}{0.000000,0.000000,0.000000}%
\pgfsetfillcolor{currentfill}%
\pgfsetlinewidth{0.501875pt}%
\definecolor{currentstroke}{rgb}{0.000000,0.000000,0.000000}%
\pgfsetstrokecolor{currentstroke}%
\pgfsetdash{}{0pt}%
\pgfsys@defobject{currentmarker}{\pgfqpoint{0.000000in}{0.000000in}}{\pgfqpoint{0.020833in}{0.000000in}}{%
\pgfpathmoveto{\pgfqpoint{0.000000in}{0.000000in}}%
\pgfpathlineto{\pgfqpoint{0.020833in}{0.000000in}}%
\pgfusepath{stroke,fill}%
}%
\begin{pgfscope}%
\pgfsys@transformshift{0.539970in}{3.325238in}%
\pgfsys@useobject{currentmarker}{}%
\end{pgfscope}%
\end{pgfscope}%
\begin{pgfscope}%
\pgfsetbuttcap%
\pgfsetroundjoin%
\definecolor{currentfill}{rgb}{0.000000,0.000000,0.000000}%
\pgfsetfillcolor{currentfill}%
\pgfsetlinewidth{0.501875pt}%
\definecolor{currentstroke}{rgb}{0.000000,0.000000,0.000000}%
\pgfsetstrokecolor{currentstroke}%
\pgfsetdash{}{0pt}%
\pgfsys@defobject{currentmarker}{\pgfqpoint{-0.020833in}{0.000000in}}{\pgfqpoint{-0.000000in}{0.000000in}}{%
\pgfpathmoveto{\pgfqpoint{-0.000000in}{0.000000in}}%
\pgfpathlineto{\pgfqpoint{-0.020833in}{0.000000in}}%
\pgfusepath{stroke,fill}%
}%
\begin{pgfscope}%
\pgfsys@transformshift{2.816034in}{3.325238in}%
\pgfsys@useobject{currentmarker}{}%
\end{pgfscope}%
\end{pgfscope}%
\begin{pgfscope}%
\pgfsetbuttcap%
\pgfsetroundjoin%
\definecolor{currentfill}{rgb}{0.000000,0.000000,0.000000}%
\pgfsetfillcolor{currentfill}%
\pgfsetlinewidth{0.501875pt}%
\definecolor{currentstroke}{rgb}{0.000000,0.000000,0.000000}%
\pgfsetstrokecolor{currentstroke}%
\pgfsetdash{}{0pt}%
\pgfsys@defobject{currentmarker}{\pgfqpoint{0.000000in}{0.000000in}}{\pgfqpoint{0.020833in}{0.000000in}}{%
\pgfpathmoveto{\pgfqpoint{0.000000in}{0.000000in}}%
\pgfpathlineto{\pgfqpoint{0.020833in}{0.000000in}}%
\pgfusepath{stroke,fill}%
}%
\begin{pgfscope}%
\pgfsys@transformshift{0.539970in}{3.408272in}%
\pgfsys@useobject{currentmarker}{}%
\end{pgfscope}%
\end{pgfscope}%
\begin{pgfscope}%
\pgfsetbuttcap%
\pgfsetroundjoin%
\definecolor{currentfill}{rgb}{0.000000,0.000000,0.000000}%
\pgfsetfillcolor{currentfill}%
\pgfsetlinewidth{0.501875pt}%
\definecolor{currentstroke}{rgb}{0.000000,0.000000,0.000000}%
\pgfsetstrokecolor{currentstroke}%
\pgfsetdash{}{0pt}%
\pgfsys@defobject{currentmarker}{\pgfqpoint{-0.020833in}{0.000000in}}{\pgfqpoint{-0.000000in}{0.000000in}}{%
\pgfpathmoveto{\pgfqpoint{-0.000000in}{0.000000in}}%
\pgfpathlineto{\pgfqpoint{-0.020833in}{0.000000in}}%
\pgfusepath{stroke,fill}%
}%
\begin{pgfscope}%
\pgfsys@transformshift{2.816034in}{3.408272in}%
\pgfsys@useobject{currentmarker}{}%
\end{pgfscope}%
\end{pgfscope}%
\begin{pgfscope}%
\pgfsetbuttcap%
\pgfsetroundjoin%
\definecolor{currentfill}{rgb}{0.000000,0.000000,0.000000}%
\pgfsetfillcolor{currentfill}%
\pgfsetlinewidth{0.501875pt}%
\definecolor{currentstroke}{rgb}{0.000000,0.000000,0.000000}%
\pgfsetstrokecolor{currentstroke}%
\pgfsetdash{}{0pt}%
\pgfsys@defobject{currentmarker}{\pgfqpoint{0.000000in}{0.000000in}}{\pgfqpoint{0.020833in}{0.000000in}}{%
\pgfpathmoveto{\pgfqpoint{0.000000in}{0.000000in}}%
\pgfpathlineto{\pgfqpoint{0.020833in}{0.000000in}}%
\pgfusepath{stroke,fill}%
}%
\begin{pgfscope}%
\pgfsys@transformshift{0.539970in}{3.574339in}%
\pgfsys@useobject{currentmarker}{}%
\end{pgfscope}%
\end{pgfscope}%
\begin{pgfscope}%
\pgfsetbuttcap%
\pgfsetroundjoin%
\definecolor{currentfill}{rgb}{0.000000,0.000000,0.000000}%
\pgfsetfillcolor{currentfill}%
\pgfsetlinewidth{0.501875pt}%
\definecolor{currentstroke}{rgb}{0.000000,0.000000,0.000000}%
\pgfsetstrokecolor{currentstroke}%
\pgfsetdash{}{0pt}%
\pgfsys@defobject{currentmarker}{\pgfqpoint{-0.020833in}{0.000000in}}{\pgfqpoint{-0.000000in}{0.000000in}}{%
\pgfpathmoveto{\pgfqpoint{-0.000000in}{0.000000in}}%
\pgfpathlineto{\pgfqpoint{-0.020833in}{0.000000in}}%
\pgfusepath{stroke,fill}%
}%
\begin{pgfscope}%
\pgfsys@transformshift{2.816034in}{3.574339in}%
\pgfsys@useobject{currentmarker}{}%
\end{pgfscope}%
\end{pgfscope}%
\begin{pgfscope}%
\pgfsetbuttcap%
\pgfsetroundjoin%
\definecolor{currentfill}{rgb}{0.000000,0.000000,0.000000}%
\pgfsetfillcolor{currentfill}%
\pgfsetlinewidth{0.501875pt}%
\definecolor{currentstroke}{rgb}{0.000000,0.000000,0.000000}%
\pgfsetstrokecolor{currentstroke}%
\pgfsetdash{}{0pt}%
\pgfsys@defobject{currentmarker}{\pgfqpoint{0.000000in}{0.000000in}}{\pgfqpoint{0.020833in}{0.000000in}}{%
\pgfpathmoveto{\pgfqpoint{0.000000in}{0.000000in}}%
\pgfpathlineto{\pgfqpoint{0.020833in}{0.000000in}}%
\pgfusepath{stroke,fill}%
}%
\begin{pgfscope}%
\pgfsys@transformshift{0.539970in}{3.657373in}%
\pgfsys@useobject{currentmarker}{}%
\end{pgfscope}%
\end{pgfscope}%
\begin{pgfscope}%
\pgfsetbuttcap%
\pgfsetroundjoin%
\definecolor{currentfill}{rgb}{0.000000,0.000000,0.000000}%
\pgfsetfillcolor{currentfill}%
\pgfsetlinewidth{0.501875pt}%
\definecolor{currentstroke}{rgb}{0.000000,0.000000,0.000000}%
\pgfsetstrokecolor{currentstroke}%
\pgfsetdash{}{0pt}%
\pgfsys@defobject{currentmarker}{\pgfqpoint{-0.020833in}{0.000000in}}{\pgfqpoint{-0.000000in}{0.000000in}}{%
\pgfpathmoveto{\pgfqpoint{-0.000000in}{0.000000in}}%
\pgfpathlineto{\pgfqpoint{-0.020833in}{0.000000in}}%
\pgfusepath{stroke,fill}%
}%
\begin{pgfscope}%
\pgfsys@transformshift{2.816034in}{3.657373in}%
\pgfsys@useobject{currentmarker}{}%
\end{pgfscope}%
\end{pgfscope}%
\begin{pgfscope}%
\pgfsetbuttcap%
\pgfsetroundjoin%
\definecolor{currentfill}{rgb}{0.000000,0.000000,0.000000}%
\pgfsetfillcolor{currentfill}%
\pgfsetlinewidth{0.501875pt}%
\definecolor{currentstroke}{rgb}{0.000000,0.000000,0.000000}%
\pgfsetstrokecolor{currentstroke}%
\pgfsetdash{}{0pt}%
\pgfsys@defobject{currentmarker}{\pgfqpoint{0.000000in}{0.000000in}}{\pgfqpoint{0.020833in}{0.000000in}}{%
\pgfpathmoveto{\pgfqpoint{0.000000in}{0.000000in}}%
\pgfpathlineto{\pgfqpoint{0.020833in}{0.000000in}}%
\pgfusepath{stroke,fill}%
}%
\begin{pgfscope}%
\pgfsys@transformshift{0.539970in}{3.740406in}%
\pgfsys@useobject{currentmarker}{}%
\end{pgfscope}%
\end{pgfscope}%
\begin{pgfscope}%
\pgfsetbuttcap%
\pgfsetroundjoin%
\definecolor{currentfill}{rgb}{0.000000,0.000000,0.000000}%
\pgfsetfillcolor{currentfill}%
\pgfsetlinewidth{0.501875pt}%
\definecolor{currentstroke}{rgb}{0.000000,0.000000,0.000000}%
\pgfsetstrokecolor{currentstroke}%
\pgfsetdash{}{0pt}%
\pgfsys@defobject{currentmarker}{\pgfqpoint{-0.020833in}{0.000000in}}{\pgfqpoint{-0.000000in}{0.000000in}}{%
\pgfpathmoveto{\pgfqpoint{-0.000000in}{0.000000in}}%
\pgfpathlineto{\pgfqpoint{-0.020833in}{0.000000in}}%
\pgfusepath{stroke,fill}%
}%
\begin{pgfscope}%
\pgfsys@transformshift{2.816034in}{3.740406in}%
\pgfsys@useobject{currentmarker}{}%
\end{pgfscope}%
\end{pgfscope}%
\begin{pgfscope}%
\pgfsetbuttcap%
\pgfsetroundjoin%
\definecolor{currentfill}{rgb}{0.000000,0.000000,0.000000}%
\pgfsetfillcolor{currentfill}%
\pgfsetlinewidth{0.501875pt}%
\definecolor{currentstroke}{rgb}{0.000000,0.000000,0.000000}%
\pgfsetstrokecolor{currentstroke}%
\pgfsetdash{}{0pt}%
\pgfsys@defobject{currentmarker}{\pgfqpoint{0.000000in}{0.000000in}}{\pgfqpoint{0.020833in}{0.000000in}}{%
\pgfpathmoveto{\pgfqpoint{0.000000in}{0.000000in}}%
\pgfpathlineto{\pgfqpoint{0.020833in}{0.000000in}}%
\pgfusepath{stroke,fill}%
}%
\begin{pgfscope}%
\pgfsys@transformshift{0.539970in}{3.823440in}%
\pgfsys@useobject{currentmarker}{}%
\end{pgfscope}%
\end{pgfscope}%
\begin{pgfscope}%
\pgfsetbuttcap%
\pgfsetroundjoin%
\definecolor{currentfill}{rgb}{0.000000,0.000000,0.000000}%
\pgfsetfillcolor{currentfill}%
\pgfsetlinewidth{0.501875pt}%
\definecolor{currentstroke}{rgb}{0.000000,0.000000,0.000000}%
\pgfsetstrokecolor{currentstroke}%
\pgfsetdash{}{0pt}%
\pgfsys@defobject{currentmarker}{\pgfqpoint{-0.020833in}{0.000000in}}{\pgfqpoint{-0.000000in}{0.000000in}}{%
\pgfpathmoveto{\pgfqpoint{-0.000000in}{0.000000in}}%
\pgfpathlineto{\pgfqpoint{-0.020833in}{0.000000in}}%
\pgfusepath{stroke,fill}%
}%
\begin{pgfscope}%
\pgfsys@transformshift{2.816034in}{3.823440in}%
\pgfsys@useobject{currentmarker}{}%
\end{pgfscope}%
\end{pgfscope}%
\begin{pgfscope}%
\pgfsetbuttcap%
\pgfsetroundjoin%
\definecolor{currentfill}{rgb}{0.000000,0.000000,0.000000}%
\pgfsetfillcolor{currentfill}%
\pgfsetlinewidth{0.501875pt}%
\definecolor{currentstroke}{rgb}{0.000000,0.000000,0.000000}%
\pgfsetstrokecolor{currentstroke}%
\pgfsetdash{}{0pt}%
\pgfsys@defobject{currentmarker}{\pgfqpoint{0.000000in}{0.000000in}}{\pgfqpoint{0.020833in}{0.000000in}}{%
\pgfpathmoveto{\pgfqpoint{0.000000in}{0.000000in}}%
\pgfpathlineto{\pgfqpoint{0.020833in}{0.000000in}}%
\pgfusepath{stroke,fill}%
}%
\begin{pgfscope}%
\pgfsys@transformshift{0.539970in}{3.989507in}%
\pgfsys@useobject{currentmarker}{}%
\end{pgfscope}%
\end{pgfscope}%
\begin{pgfscope}%
\pgfsetbuttcap%
\pgfsetroundjoin%
\definecolor{currentfill}{rgb}{0.000000,0.000000,0.000000}%
\pgfsetfillcolor{currentfill}%
\pgfsetlinewidth{0.501875pt}%
\definecolor{currentstroke}{rgb}{0.000000,0.000000,0.000000}%
\pgfsetstrokecolor{currentstroke}%
\pgfsetdash{}{0pt}%
\pgfsys@defobject{currentmarker}{\pgfqpoint{-0.020833in}{0.000000in}}{\pgfqpoint{-0.000000in}{0.000000in}}{%
\pgfpathmoveto{\pgfqpoint{-0.000000in}{0.000000in}}%
\pgfpathlineto{\pgfqpoint{-0.020833in}{0.000000in}}%
\pgfusepath{stroke,fill}%
}%
\begin{pgfscope}%
\pgfsys@transformshift{2.816034in}{3.989507in}%
\pgfsys@useobject{currentmarker}{}%
\end{pgfscope}%
\end{pgfscope}%
\begin{pgfscope}%
\pgfsetbuttcap%
\pgfsetroundjoin%
\definecolor{currentfill}{rgb}{0.000000,0.000000,0.000000}%
\pgfsetfillcolor{currentfill}%
\pgfsetlinewidth{0.501875pt}%
\definecolor{currentstroke}{rgb}{0.000000,0.000000,0.000000}%
\pgfsetstrokecolor{currentstroke}%
\pgfsetdash{}{0pt}%
\pgfsys@defobject{currentmarker}{\pgfqpoint{0.000000in}{0.000000in}}{\pgfqpoint{0.020833in}{0.000000in}}{%
\pgfpathmoveto{\pgfqpoint{0.000000in}{0.000000in}}%
\pgfpathlineto{\pgfqpoint{0.020833in}{0.000000in}}%
\pgfusepath{stroke,fill}%
}%
\begin{pgfscope}%
\pgfsys@transformshift{0.539970in}{4.072540in}%
\pgfsys@useobject{currentmarker}{}%
\end{pgfscope}%
\end{pgfscope}%
\begin{pgfscope}%
\pgfsetbuttcap%
\pgfsetroundjoin%
\definecolor{currentfill}{rgb}{0.000000,0.000000,0.000000}%
\pgfsetfillcolor{currentfill}%
\pgfsetlinewidth{0.501875pt}%
\definecolor{currentstroke}{rgb}{0.000000,0.000000,0.000000}%
\pgfsetstrokecolor{currentstroke}%
\pgfsetdash{}{0pt}%
\pgfsys@defobject{currentmarker}{\pgfqpoint{-0.020833in}{0.000000in}}{\pgfqpoint{-0.000000in}{0.000000in}}{%
\pgfpathmoveto{\pgfqpoint{-0.000000in}{0.000000in}}%
\pgfpathlineto{\pgfqpoint{-0.020833in}{0.000000in}}%
\pgfusepath{stroke,fill}%
}%
\begin{pgfscope}%
\pgfsys@transformshift{2.816034in}{4.072540in}%
\pgfsys@useobject{currentmarker}{}%
\end{pgfscope}%
\end{pgfscope}%
\begin{pgfscope}%
\pgfsetbuttcap%
\pgfsetroundjoin%
\definecolor{currentfill}{rgb}{0.000000,0.000000,0.000000}%
\pgfsetfillcolor{currentfill}%
\pgfsetlinewidth{0.501875pt}%
\definecolor{currentstroke}{rgb}{0.000000,0.000000,0.000000}%
\pgfsetstrokecolor{currentstroke}%
\pgfsetdash{}{0pt}%
\pgfsys@defobject{currentmarker}{\pgfqpoint{0.000000in}{0.000000in}}{\pgfqpoint{0.020833in}{0.000000in}}{%
\pgfpathmoveto{\pgfqpoint{0.000000in}{0.000000in}}%
\pgfpathlineto{\pgfqpoint{0.020833in}{0.000000in}}%
\pgfusepath{stroke,fill}%
}%
\begin{pgfscope}%
\pgfsys@transformshift{0.539970in}{4.155574in}%
\pgfsys@useobject{currentmarker}{}%
\end{pgfscope}%
\end{pgfscope}%
\begin{pgfscope}%
\pgfsetbuttcap%
\pgfsetroundjoin%
\definecolor{currentfill}{rgb}{0.000000,0.000000,0.000000}%
\pgfsetfillcolor{currentfill}%
\pgfsetlinewidth{0.501875pt}%
\definecolor{currentstroke}{rgb}{0.000000,0.000000,0.000000}%
\pgfsetstrokecolor{currentstroke}%
\pgfsetdash{}{0pt}%
\pgfsys@defobject{currentmarker}{\pgfqpoint{-0.020833in}{0.000000in}}{\pgfqpoint{-0.000000in}{0.000000in}}{%
\pgfpathmoveto{\pgfqpoint{-0.000000in}{0.000000in}}%
\pgfpathlineto{\pgfqpoint{-0.020833in}{0.000000in}}%
\pgfusepath{stroke,fill}%
}%
\begin{pgfscope}%
\pgfsys@transformshift{2.816034in}{4.155574in}%
\pgfsys@useobject{currentmarker}{}%
\end{pgfscope}%
\end{pgfscope}%
\begin{pgfscope}%
\pgfsetbuttcap%
\pgfsetroundjoin%
\definecolor{currentfill}{rgb}{0.000000,0.000000,0.000000}%
\pgfsetfillcolor{currentfill}%
\pgfsetlinewidth{0.501875pt}%
\definecolor{currentstroke}{rgb}{0.000000,0.000000,0.000000}%
\pgfsetstrokecolor{currentstroke}%
\pgfsetdash{}{0pt}%
\pgfsys@defobject{currentmarker}{\pgfqpoint{0.000000in}{0.000000in}}{\pgfqpoint{0.020833in}{0.000000in}}{%
\pgfpathmoveto{\pgfqpoint{0.000000in}{0.000000in}}%
\pgfpathlineto{\pgfqpoint{0.020833in}{0.000000in}}%
\pgfusepath{stroke,fill}%
}%
\begin{pgfscope}%
\pgfsys@transformshift{0.539970in}{4.238607in}%
\pgfsys@useobject{currentmarker}{}%
\end{pgfscope}%
\end{pgfscope}%
\begin{pgfscope}%
\pgfsetbuttcap%
\pgfsetroundjoin%
\definecolor{currentfill}{rgb}{0.000000,0.000000,0.000000}%
\pgfsetfillcolor{currentfill}%
\pgfsetlinewidth{0.501875pt}%
\definecolor{currentstroke}{rgb}{0.000000,0.000000,0.000000}%
\pgfsetstrokecolor{currentstroke}%
\pgfsetdash{}{0pt}%
\pgfsys@defobject{currentmarker}{\pgfqpoint{-0.020833in}{0.000000in}}{\pgfqpoint{-0.000000in}{0.000000in}}{%
\pgfpathmoveto{\pgfqpoint{-0.000000in}{0.000000in}}%
\pgfpathlineto{\pgfqpoint{-0.020833in}{0.000000in}}%
\pgfusepath{stroke,fill}%
}%
\begin{pgfscope}%
\pgfsys@transformshift{2.816034in}{4.238607in}%
\pgfsys@useobject{currentmarker}{}%
\end{pgfscope}%
\end{pgfscope}%
\begin{pgfscope}%
\definecolor{textcolor}{rgb}{0.000000,0.000000,0.000000}%
\pgfsetstrokecolor{textcolor}%
\pgfsetfillcolor{textcolor}%
\pgftext[x=0.188889in,y=3.561093in,,bottom,rotate=90.000000]{\color{textcolor}\rmfamily\fontsize{10.000000}{12.000000}\selectfont \(\displaystyle T(K)\)}%
\end{pgfscope}%
\begin{pgfscope}%
\pgfpathrectangle{\pgfqpoint{0.539970in}{2.747992in}}{\pgfqpoint{2.276064in}{1.626201in}}%
\pgfusepath{clip}%
\pgfsetrectcap%
\pgfsetroundjoin%
\pgfsetlinewidth{1.003750pt}%
\definecolor{currentstroke}{rgb}{0.047059,0.364706,0.647059}%
\pgfsetstrokecolor{currentstroke}%
\pgfsetdash{}{0pt}%
\pgfpathmoveto{\pgfqpoint{0.562505in}{4.300275in}}%
\pgfpathlineto{\pgfqpoint{0.585040in}{4.297813in}}%
\pgfpathlineto{\pgfqpoint{0.607576in}{4.295111in}}%
\pgfpathlineto{\pgfqpoint{0.630111in}{4.293822in}}%
\pgfpathlineto{\pgfqpoint{0.652646in}{4.291243in}}%
\pgfpathlineto{\pgfqpoint{0.675182in}{4.288598in}}%
\pgfpathlineto{\pgfqpoint{0.697717in}{4.286701in}}%
\pgfpathlineto{\pgfqpoint{0.720252in}{4.285060in}}%
\pgfpathlineto{\pgfqpoint{0.742787in}{4.283219in}}%
\pgfpathlineto{\pgfqpoint{0.765323in}{4.281841in}}%
\pgfpathlineto{\pgfqpoint{0.787858in}{4.280414in}}%
\pgfpathlineto{\pgfqpoint{0.810393in}{4.279544in}}%
\pgfpathlineto{\pgfqpoint{0.832929in}{4.278722in}}%
\pgfpathlineto{\pgfqpoint{0.855464in}{4.277510in}}%
\pgfpathlineto{\pgfqpoint{0.877999in}{4.276708in}}%
\pgfpathlineto{\pgfqpoint{0.900534in}{4.275677in}}%
\pgfpathlineto{\pgfqpoint{0.923070in}{4.274763in}}%
\pgfpathlineto{\pgfqpoint{0.945605in}{4.274300in}}%
\pgfpathlineto{\pgfqpoint{0.968140in}{4.273492in}}%
\pgfpathlineto{\pgfqpoint{0.990676in}{4.272786in}}%
\pgfpathlineto{\pgfqpoint{1.013211in}{4.271669in}}%
\pgfpathlineto{\pgfqpoint{1.035746in}{4.270862in}}%
\pgfpathlineto{\pgfqpoint{1.058281in}{4.270060in}}%
\pgfpathlineto{\pgfqpoint{1.080817in}{4.269419in}}%
\pgfpathlineto{\pgfqpoint{1.103352in}{4.268354in}}%
\pgfpathlineto{\pgfqpoint{1.125887in}{4.267783in}}%
\pgfpathlineto{\pgfqpoint{1.148423in}{4.267092in}}%
\pgfpathlineto{\pgfqpoint{1.170958in}{4.266517in}}%
\pgfpathlineto{\pgfqpoint{1.193493in}{4.266063in}}%
\pgfpathlineto{\pgfqpoint{1.216028in}{4.265481in}}%
\pgfpathlineto{\pgfqpoint{1.238564in}{4.264966in}}%
\pgfpathlineto{\pgfqpoint{1.261099in}{4.264215in}}%
\pgfpathlineto{\pgfqpoint{1.283634in}{4.263673in}}%
\pgfpathlineto{\pgfqpoint{1.306170in}{4.263087in}}%
\pgfpathlineto{\pgfqpoint{1.328705in}{4.262556in}}%
\pgfpathlineto{\pgfqpoint{1.351240in}{4.262192in}}%
\pgfpathlineto{\pgfqpoint{1.373776in}{4.262014in}}%
\pgfpathlineto{\pgfqpoint{1.396311in}{4.261549in}}%
\pgfpathlineto{\pgfqpoint{1.418846in}{4.261147in}}%
\pgfpathlineto{\pgfqpoint{1.441381in}{4.260539in}}%
\pgfpathlineto{\pgfqpoint{1.463917in}{4.260113in}}%
\pgfpathlineto{\pgfqpoint{1.486452in}{4.259672in}}%
\pgfpathlineto{\pgfqpoint{1.508987in}{4.259102in}}%
\pgfpathlineto{\pgfqpoint{1.531523in}{4.258752in}}%
\pgfpathlineto{\pgfqpoint{1.554058in}{4.258223in}}%
\pgfpathlineto{\pgfqpoint{1.576593in}{4.257769in}}%
\pgfpathlineto{\pgfqpoint{1.599128in}{4.257297in}}%
\pgfpathlineto{\pgfqpoint{1.621664in}{4.256968in}}%
\pgfpathlineto{\pgfqpoint{1.644199in}{4.256578in}}%
\pgfpathlineto{\pgfqpoint{1.666734in}{4.256229in}}%
\pgfpathlineto{\pgfqpoint{1.689270in}{4.255858in}}%
\pgfpathlineto{\pgfqpoint{1.711805in}{4.255390in}}%
\pgfpathlineto{\pgfqpoint{1.734340in}{4.255154in}}%
\pgfpathlineto{\pgfqpoint{1.756875in}{4.254640in}}%
\pgfpathlineto{\pgfqpoint{1.779411in}{4.254537in}}%
\pgfpathlineto{\pgfqpoint{1.801946in}{4.254487in}}%
\pgfpathlineto{\pgfqpoint{1.824481in}{4.254032in}}%
\pgfpathlineto{\pgfqpoint{1.847017in}{4.253765in}}%
\pgfpathlineto{\pgfqpoint{1.869552in}{4.253392in}}%
\pgfpathlineto{\pgfqpoint{1.892087in}{4.253066in}}%
\pgfpathlineto{\pgfqpoint{1.914622in}{4.252818in}}%
\pgfpathlineto{\pgfqpoint{1.937158in}{4.252441in}}%
\pgfpathlineto{\pgfqpoint{1.959693in}{4.252006in}}%
\pgfpathlineto{\pgfqpoint{1.982228in}{4.251796in}}%
\pgfpathlineto{\pgfqpoint{2.004764in}{4.251591in}}%
\pgfpathlineto{\pgfqpoint{2.027299in}{4.251343in}}%
\pgfpathlineto{\pgfqpoint{2.049834in}{4.251090in}}%
\pgfpathlineto{\pgfqpoint{2.072369in}{4.250786in}}%
\pgfpathlineto{\pgfqpoint{2.094905in}{4.250468in}}%
\pgfpathlineto{\pgfqpoint{2.117440in}{4.250279in}}%
\pgfpathlineto{\pgfqpoint{2.139975in}{4.249947in}}%
\pgfpathlineto{\pgfqpoint{2.162511in}{4.249691in}}%
\pgfpathlineto{\pgfqpoint{2.185046in}{4.249322in}}%
\pgfpathlineto{\pgfqpoint{2.207581in}{4.249109in}}%
\pgfpathlineto{\pgfqpoint{2.230116in}{4.248778in}}%
\pgfpathlineto{\pgfqpoint{2.252652in}{4.248528in}}%
\pgfpathlineto{\pgfqpoint{2.275187in}{4.248386in}}%
\pgfpathlineto{\pgfqpoint{2.297722in}{4.248297in}}%
\pgfpathlineto{\pgfqpoint{2.320258in}{4.247944in}}%
\pgfpathlineto{\pgfqpoint{2.342793in}{4.247658in}}%
\pgfpathlineto{\pgfqpoint{2.365328in}{4.247549in}}%
\pgfpathlineto{\pgfqpoint{2.387863in}{4.247224in}}%
\pgfpathlineto{\pgfqpoint{2.410399in}{4.246841in}}%
\pgfpathlineto{\pgfqpoint{2.432934in}{4.246618in}}%
\pgfpathlineto{\pgfqpoint{2.455469in}{4.246382in}}%
\pgfpathlineto{\pgfqpoint{2.478005in}{4.246222in}}%
\pgfpathlineto{\pgfqpoint{2.500540in}{4.245901in}}%
\pgfpathlineto{\pgfqpoint{2.523075in}{4.245692in}}%
\pgfpathlineto{\pgfqpoint{2.545610in}{4.245449in}}%
\pgfpathlineto{\pgfqpoint{2.568146in}{4.245226in}}%
\pgfpathlineto{\pgfqpoint{2.590681in}{4.245074in}}%
\pgfpathlineto{\pgfqpoint{2.613216in}{4.244795in}}%
\pgfpathlineto{\pgfqpoint{2.635752in}{4.244635in}}%
\pgfpathlineto{\pgfqpoint{2.658287in}{4.244466in}}%
\pgfpathlineto{\pgfqpoint{2.680822in}{4.244278in}}%
\pgfpathlineto{\pgfqpoint{2.703357in}{4.243980in}}%
\pgfpathlineto{\pgfqpoint{2.725893in}{4.243719in}}%
\pgfpathlineto{\pgfqpoint{2.748428in}{4.243523in}}%
\pgfpathlineto{\pgfqpoint{2.770963in}{4.243417in}}%
\pgfusepath{stroke}%
\end{pgfscope}%
\begin{pgfscope}%
\pgfpathrectangle{\pgfqpoint{0.539970in}{2.747992in}}{\pgfqpoint{2.276064in}{1.626201in}}%
\pgfusepath{clip}%
\pgfsetrectcap%
\pgfsetroundjoin%
\pgfsetlinewidth{1.003750pt}%
\definecolor{currentstroke}{rgb}{0.000000,0.725490,0.270588}%
\pgfsetstrokecolor{currentstroke}%
\pgfsetdash{}{0pt}%
\pgfpathmoveto{\pgfqpoint{0.562505in}{4.093147in}}%
\pgfpathlineto{\pgfqpoint{0.585040in}{4.063299in}}%
\pgfpathlineto{\pgfqpoint{0.607576in}{4.030229in}}%
\pgfpathlineto{\pgfqpoint{0.630111in}{4.009620in}}%
\pgfpathlineto{\pgfqpoint{0.652646in}{3.988373in}}%
\pgfpathlineto{\pgfqpoint{0.675182in}{3.969840in}}%
\pgfpathlineto{\pgfqpoint{0.697717in}{3.948868in}}%
\pgfpathlineto{\pgfqpoint{0.720252in}{3.932596in}}%
\pgfpathlineto{\pgfqpoint{0.742787in}{3.915654in}}%
\pgfpathlineto{\pgfqpoint{0.765323in}{3.902820in}}%
\pgfpathlineto{\pgfqpoint{0.787858in}{3.887639in}}%
\pgfpathlineto{\pgfqpoint{0.810393in}{3.876134in}}%
\pgfpathlineto{\pgfqpoint{0.832929in}{3.861862in}}%
\pgfpathlineto{\pgfqpoint{0.855464in}{3.850700in}}%
\pgfpathlineto{\pgfqpoint{0.877999in}{3.839969in}}%
\pgfpathlineto{\pgfqpoint{0.900534in}{3.828911in}}%
\pgfpathlineto{\pgfqpoint{0.923070in}{3.820261in}}%
\pgfpathlineto{\pgfqpoint{0.945605in}{3.811664in}}%
\pgfpathlineto{\pgfqpoint{0.968140in}{3.802441in}}%
\pgfpathlineto{\pgfqpoint{0.990676in}{3.794250in}}%
\pgfpathlineto{\pgfqpoint{1.013211in}{3.783878in}}%
\pgfpathlineto{\pgfqpoint{1.035746in}{3.776537in}}%
\pgfpathlineto{\pgfqpoint{1.058281in}{3.766567in}}%
\pgfpathlineto{\pgfqpoint{1.080817in}{3.757427in}}%
\pgfpathlineto{\pgfqpoint{1.103352in}{3.750271in}}%
\pgfpathlineto{\pgfqpoint{1.125887in}{3.743525in}}%
\pgfpathlineto{\pgfqpoint{1.148423in}{3.735925in}}%
\pgfpathlineto{\pgfqpoint{1.170958in}{3.728895in}}%
\pgfpathlineto{\pgfqpoint{1.193493in}{3.723704in}}%
\pgfpathlineto{\pgfqpoint{1.216028in}{3.717370in}}%
\pgfpathlineto{\pgfqpoint{1.238564in}{3.710756in}}%
\pgfpathlineto{\pgfqpoint{1.261099in}{3.704207in}}%
\pgfpathlineto{\pgfqpoint{1.283634in}{3.698363in}}%
\pgfpathlineto{\pgfqpoint{1.306170in}{3.692082in}}%
\pgfpathlineto{\pgfqpoint{1.328705in}{3.686926in}}%
\pgfpathlineto{\pgfqpoint{1.351240in}{3.681366in}}%
\pgfpathlineto{\pgfqpoint{1.373776in}{3.675954in}}%
\pgfpathlineto{\pgfqpoint{1.396311in}{3.669619in}}%
\pgfpathlineto{\pgfqpoint{1.418846in}{3.664370in}}%
\pgfpathlineto{\pgfqpoint{1.441381in}{3.658958in}}%
\pgfpathlineto{\pgfqpoint{1.463917in}{3.654223in}}%
\pgfpathlineto{\pgfqpoint{1.486452in}{3.648649in}}%
\pgfpathlineto{\pgfqpoint{1.508987in}{3.643242in}}%
\pgfpathlineto{\pgfqpoint{1.531523in}{3.638063in}}%
\pgfpathlineto{\pgfqpoint{1.554058in}{3.633897in}}%
\pgfpathlineto{\pgfqpoint{1.576593in}{3.629058in}}%
\pgfpathlineto{\pgfqpoint{1.599128in}{3.624158in}}%
\pgfpathlineto{\pgfqpoint{1.621664in}{3.618879in}}%
\pgfpathlineto{\pgfqpoint{1.644199in}{3.614346in}}%
\pgfpathlineto{\pgfqpoint{1.666734in}{3.610378in}}%
\pgfpathlineto{\pgfqpoint{1.689270in}{3.606320in}}%
\pgfpathlineto{\pgfqpoint{1.711805in}{3.602699in}}%
\pgfpathlineto{\pgfqpoint{1.734340in}{3.598263in}}%
\pgfpathlineto{\pgfqpoint{1.756875in}{3.594043in}}%
\pgfpathlineto{\pgfqpoint{1.779411in}{3.589443in}}%
\pgfpathlineto{\pgfqpoint{1.801946in}{3.586095in}}%
\pgfpathlineto{\pgfqpoint{1.824481in}{3.582101in}}%
\pgfpathlineto{\pgfqpoint{1.847017in}{3.577983in}}%
\pgfpathlineto{\pgfqpoint{1.869552in}{3.573946in}}%
\pgfpathlineto{\pgfqpoint{1.892087in}{3.570359in}}%
\pgfpathlineto{\pgfqpoint{1.914622in}{3.566661in}}%
\pgfpathlineto{\pgfqpoint{1.937158in}{3.562472in}}%
\pgfpathlineto{\pgfqpoint{1.959693in}{3.558547in}}%
\pgfpathlineto{\pgfqpoint{1.982228in}{3.555110in}}%
\pgfpathlineto{\pgfqpoint{2.004764in}{3.551107in}}%
\pgfpathlineto{\pgfqpoint{2.027299in}{3.547616in}}%
\pgfpathlineto{\pgfqpoint{2.049834in}{3.544306in}}%
\pgfpathlineto{\pgfqpoint{2.072369in}{3.540749in}}%
\pgfpathlineto{\pgfqpoint{2.094905in}{3.537732in}}%
\pgfpathlineto{\pgfqpoint{2.117440in}{3.534033in}}%
\pgfpathlineto{\pgfqpoint{2.139975in}{3.530655in}}%
\pgfpathlineto{\pgfqpoint{2.162511in}{3.527028in}}%
\pgfpathlineto{\pgfqpoint{2.185046in}{3.523579in}}%
\pgfpathlineto{\pgfqpoint{2.207581in}{3.520715in}}%
\pgfpathlineto{\pgfqpoint{2.230116in}{3.517087in}}%
\pgfpathlineto{\pgfqpoint{2.252652in}{3.513633in}}%
\pgfpathlineto{\pgfqpoint{2.275187in}{3.509696in}}%
\pgfpathlineto{\pgfqpoint{2.297722in}{3.506338in}}%
\pgfpathlineto{\pgfqpoint{2.320258in}{3.503190in}}%
\pgfpathlineto{\pgfqpoint{2.342793in}{3.500552in}}%
\pgfpathlineto{\pgfqpoint{2.365328in}{3.497588in}}%
\pgfpathlineto{\pgfqpoint{2.387863in}{3.494686in}}%
\pgfpathlineto{\pgfqpoint{2.410399in}{3.491877in}}%
\pgfpathlineto{\pgfqpoint{2.432934in}{3.488833in}}%
\pgfpathlineto{\pgfqpoint{2.455469in}{3.485658in}}%
\pgfpathlineto{\pgfqpoint{2.478005in}{3.482929in}}%
\pgfpathlineto{\pgfqpoint{2.500540in}{3.480256in}}%
\pgfpathlineto{\pgfqpoint{2.523075in}{3.477178in}}%
\pgfpathlineto{\pgfqpoint{2.545610in}{3.474577in}}%
\pgfpathlineto{\pgfqpoint{2.568146in}{3.471695in}}%
\pgfpathlineto{\pgfqpoint{2.590681in}{3.469261in}}%
\pgfpathlineto{\pgfqpoint{2.613216in}{3.466263in}}%
\pgfpathlineto{\pgfqpoint{2.635752in}{3.464041in}}%
\pgfpathlineto{\pgfqpoint{2.658287in}{3.461096in}}%
\pgfpathlineto{\pgfqpoint{2.680822in}{3.458444in}}%
\pgfpathlineto{\pgfqpoint{2.703357in}{3.455835in}}%
\pgfpathlineto{\pgfqpoint{2.725893in}{3.453199in}}%
\pgfpathlineto{\pgfqpoint{2.748428in}{3.450479in}}%
\pgfpathlineto{\pgfqpoint{2.770963in}{3.448752in}}%
\pgfusepath{stroke}%
\end{pgfscope}%
\begin{pgfscope}%
\pgfpathrectangle{\pgfqpoint{0.539970in}{2.747992in}}{\pgfqpoint{2.276064in}{1.626201in}}%
\pgfusepath{clip}%
\pgfsetrectcap%
\pgfsetroundjoin%
\pgfsetlinewidth{1.003750pt}%
\definecolor{currentstroke}{rgb}{1.000000,0.584314,0.000000}%
\pgfsetstrokecolor{currentstroke}%
\pgfsetdash{}{0pt}%
\pgfpathmoveto{\pgfqpoint{0.562505in}{4.271386in}}%
\pgfpathlineto{\pgfqpoint{0.585040in}{4.259709in}}%
\pgfpathlineto{\pgfqpoint{0.607576in}{4.253341in}}%
\pgfpathlineto{\pgfqpoint{0.630111in}{4.247007in}}%
\pgfpathlineto{\pgfqpoint{0.652646in}{4.240347in}}%
\pgfpathlineto{\pgfqpoint{0.675182in}{4.234697in}}%
\pgfpathlineto{\pgfqpoint{0.697717in}{4.231422in}}%
\pgfpathlineto{\pgfqpoint{0.720252in}{4.227072in}}%
\pgfpathlineto{\pgfqpoint{0.742787in}{4.223425in}}%
\pgfpathlineto{\pgfqpoint{0.765323in}{4.219719in}}%
\pgfpathlineto{\pgfqpoint{0.787858in}{4.216858in}}%
\pgfpathlineto{\pgfqpoint{0.810393in}{4.213895in}}%
\pgfpathlineto{\pgfqpoint{0.832929in}{4.210117in}}%
\pgfpathlineto{\pgfqpoint{0.855464in}{4.206861in}}%
\pgfpathlineto{\pgfqpoint{0.877999in}{4.203727in}}%
\pgfpathlineto{\pgfqpoint{0.900534in}{4.201331in}}%
\pgfpathlineto{\pgfqpoint{0.923070in}{4.199534in}}%
\pgfpathlineto{\pgfqpoint{0.945605in}{4.197082in}}%
\pgfpathlineto{\pgfqpoint{0.968140in}{4.194552in}}%
\pgfpathlineto{\pgfqpoint{0.990676in}{4.192123in}}%
\pgfpathlineto{\pgfqpoint{1.013211in}{4.190335in}}%
\pgfpathlineto{\pgfqpoint{1.035746in}{4.188276in}}%
\pgfpathlineto{\pgfqpoint{1.058281in}{4.185885in}}%
\pgfpathlineto{\pgfqpoint{1.080817in}{4.183673in}}%
\pgfpathlineto{\pgfqpoint{1.103352in}{4.181226in}}%
\pgfpathlineto{\pgfqpoint{1.125887in}{4.179187in}}%
\pgfpathlineto{\pgfqpoint{1.148423in}{4.177650in}}%
\pgfpathlineto{\pgfqpoint{1.170958in}{4.175712in}}%
\pgfpathlineto{\pgfqpoint{1.193493in}{4.174267in}}%
\pgfpathlineto{\pgfqpoint{1.216028in}{4.171791in}}%
\pgfpathlineto{\pgfqpoint{1.238564in}{4.170186in}}%
\pgfpathlineto{\pgfqpoint{1.261099in}{4.168210in}}%
\pgfpathlineto{\pgfqpoint{1.283634in}{4.166182in}}%
\pgfpathlineto{\pgfqpoint{1.306170in}{4.164687in}}%
\pgfpathlineto{\pgfqpoint{1.328705in}{4.163211in}}%
\pgfpathlineto{\pgfqpoint{1.351240in}{4.161475in}}%
\pgfpathlineto{\pgfqpoint{1.373776in}{4.160056in}}%
\pgfpathlineto{\pgfqpoint{1.396311in}{4.158461in}}%
\pgfpathlineto{\pgfqpoint{1.418846in}{4.156391in}}%
\pgfpathlineto{\pgfqpoint{1.441381in}{4.154858in}}%
\pgfpathlineto{\pgfqpoint{1.463917in}{4.153522in}}%
\pgfpathlineto{\pgfqpoint{1.486452in}{4.152303in}}%
\pgfpathlineto{\pgfqpoint{1.508987in}{4.151015in}}%
\pgfpathlineto{\pgfqpoint{1.531523in}{4.149752in}}%
\pgfpathlineto{\pgfqpoint{1.554058in}{4.148393in}}%
\pgfpathlineto{\pgfqpoint{1.576593in}{4.146964in}}%
\pgfpathlineto{\pgfqpoint{1.599128in}{4.145444in}}%
\pgfpathlineto{\pgfqpoint{1.621664in}{4.144306in}}%
\pgfpathlineto{\pgfqpoint{1.644199in}{4.142849in}}%
\pgfpathlineto{\pgfqpoint{1.666734in}{4.141597in}}%
\pgfpathlineto{\pgfqpoint{1.689270in}{4.140586in}}%
\pgfpathlineto{\pgfqpoint{1.711805in}{4.139580in}}%
\pgfpathlineto{\pgfqpoint{1.734340in}{4.138609in}}%
\pgfpathlineto{\pgfqpoint{1.756875in}{4.137548in}}%
\pgfpathlineto{\pgfqpoint{1.779411in}{4.136372in}}%
\pgfpathlineto{\pgfqpoint{1.801946in}{4.135140in}}%
\pgfpathlineto{\pgfqpoint{1.824481in}{4.134080in}}%
\pgfpathlineto{\pgfqpoint{1.847017in}{4.132863in}}%
\pgfpathlineto{\pgfqpoint{1.869552in}{4.132051in}}%
\pgfpathlineto{\pgfqpoint{1.892087in}{4.130939in}}%
\pgfpathlineto{\pgfqpoint{1.914622in}{4.129848in}}%
\pgfpathlineto{\pgfqpoint{1.937158in}{4.128737in}}%
\pgfpathlineto{\pgfqpoint{1.959693in}{4.127593in}}%
\pgfpathlineto{\pgfqpoint{1.982228in}{4.126409in}}%
\pgfpathlineto{\pgfqpoint{2.004764in}{4.125419in}}%
\pgfpathlineto{\pgfqpoint{2.027299in}{4.124272in}}%
\pgfpathlineto{\pgfqpoint{2.049834in}{4.123250in}}%
\pgfpathlineto{\pgfqpoint{2.072369in}{4.122024in}}%
\pgfpathlineto{\pgfqpoint{2.094905in}{4.121083in}}%
\pgfpathlineto{\pgfqpoint{2.117440in}{4.120367in}}%
\pgfpathlineto{\pgfqpoint{2.139975in}{4.119706in}}%
\pgfpathlineto{\pgfqpoint{2.162511in}{4.118710in}}%
\pgfpathlineto{\pgfqpoint{2.185046in}{4.117681in}}%
\pgfpathlineto{\pgfqpoint{2.207581in}{4.116729in}}%
\pgfpathlineto{\pgfqpoint{2.230116in}{4.115949in}}%
\pgfpathlineto{\pgfqpoint{2.252652in}{4.114938in}}%
\pgfpathlineto{\pgfqpoint{2.275187in}{4.113969in}}%
\pgfpathlineto{\pgfqpoint{2.297722in}{4.113035in}}%
\pgfpathlineto{\pgfqpoint{2.320258in}{4.112158in}}%
\pgfpathlineto{\pgfqpoint{2.342793in}{4.111235in}}%
\pgfpathlineto{\pgfqpoint{2.365328in}{4.110580in}}%
\pgfpathlineto{\pgfqpoint{2.387863in}{4.109707in}}%
\pgfpathlineto{\pgfqpoint{2.410399in}{4.109267in}}%
\pgfpathlineto{\pgfqpoint{2.432934in}{4.108541in}}%
\pgfpathlineto{\pgfqpoint{2.455469in}{4.107905in}}%
\pgfpathlineto{\pgfqpoint{2.478005in}{4.106935in}}%
\pgfpathlineto{\pgfqpoint{2.500540in}{4.106262in}}%
\pgfpathlineto{\pgfqpoint{2.523075in}{4.105487in}}%
\pgfpathlineto{\pgfqpoint{2.545610in}{4.104697in}}%
\pgfpathlineto{\pgfqpoint{2.568146in}{4.104033in}}%
\pgfpathlineto{\pgfqpoint{2.590681in}{4.103454in}}%
\pgfpathlineto{\pgfqpoint{2.613216in}{4.102710in}}%
\pgfpathlineto{\pgfqpoint{2.635752in}{4.102155in}}%
\pgfpathlineto{\pgfqpoint{2.658287in}{4.101551in}}%
\pgfpathlineto{\pgfqpoint{2.680822in}{4.100963in}}%
\pgfpathlineto{\pgfqpoint{2.703357in}{4.100198in}}%
\pgfpathlineto{\pgfqpoint{2.725893in}{4.099407in}}%
\pgfpathlineto{\pgfqpoint{2.748428in}{4.098824in}}%
\pgfpathlineto{\pgfqpoint{2.770963in}{4.097890in}}%
\pgfusepath{stroke}%
\end{pgfscope}%
\begin{pgfscope}%
\pgfpathrectangle{\pgfqpoint{0.539970in}{2.747992in}}{\pgfqpoint{2.276064in}{1.626201in}}%
\pgfusepath{clip}%
\pgfsetrectcap%
\pgfsetroundjoin%
\pgfsetlinewidth{1.003750pt}%
\definecolor{currentstroke}{rgb}{1.000000,0.172549,0.000000}%
\pgfsetstrokecolor{currentstroke}%
\pgfsetdash{}{0pt}%
\pgfpathmoveto{\pgfqpoint{0.562505in}{3.748748in}}%
\pgfpathlineto{\pgfqpoint{0.585040in}{3.696613in}}%
\pgfpathlineto{\pgfqpoint{0.607576in}{3.645545in}}%
\pgfpathlineto{\pgfqpoint{0.630111in}{3.609941in}}%
\pgfpathlineto{\pgfqpoint{0.652646in}{3.576584in}}%
\pgfpathlineto{\pgfqpoint{0.675182in}{3.556249in}}%
\pgfpathlineto{\pgfqpoint{0.697717in}{3.537576in}}%
\pgfpathlineto{\pgfqpoint{0.720252in}{3.514319in}}%
\pgfpathlineto{\pgfqpoint{0.742787in}{3.492162in}}%
\pgfpathlineto{\pgfqpoint{0.765323in}{3.471125in}}%
\pgfpathlineto{\pgfqpoint{0.787858in}{3.450235in}}%
\pgfpathlineto{\pgfqpoint{0.810393in}{3.432890in}}%
\pgfpathlineto{\pgfqpoint{0.832929in}{3.417932in}}%
\pgfpathlineto{\pgfqpoint{0.855464in}{3.402834in}}%
\pgfpathlineto{\pgfqpoint{0.877999in}{3.388936in}}%
\pgfpathlineto{\pgfqpoint{0.900534in}{3.376935in}}%
\pgfpathlineto{\pgfqpoint{0.923070in}{3.364790in}}%
\pgfpathlineto{\pgfqpoint{0.945605in}{3.349970in}}%
\pgfpathlineto{\pgfqpoint{0.968140in}{3.337094in}}%
\pgfpathlineto{\pgfqpoint{0.990676in}{3.326699in}}%
\pgfpathlineto{\pgfqpoint{1.013211in}{3.316855in}}%
\pgfpathlineto{\pgfqpoint{1.035746in}{3.305984in}}%
\pgfpathlineto{\pgfqpoint{1.058281in}{3.295822in}}%
\pgfpathlineto{\pgfqpoint{1.080817in}{3.285136in}}%
\pgfpathlineto{\pgfqpoint{1.103352in}{3.275554in}}%
\pgfpathlineto{\pgfqpoint{1.125887in}{3.263994in}}%
\pgfpathlineto{\pgfqpoint{1.148423in}{3.255237in}}%
\pgfpathlineto{\pgfqpoint{1.170958in}{3.246281in}}%
\pgfpathlineto{\pgfqpoint{1.193493in}{3.238262in}}%
\pgfpathlineto{\pgfqpoint{1.216028in}{3.231951in}}%
\pgfpathlineto{\pgfqpoint{1.238564in}{3.225518in}}%
\pgfpathlineto{\pgfqpoint{1.261099in}{3.216588in}}%
\pgfpathlineto{\pgfqpoint{1.283634in}{3.208404in}}%
\pgfpathlineto{\pgfqpoint{1.306170in}{3.201588in}}%
\pgfpathlineto{\pgfqpoint{1.328705in}{3.195785in}}%
\pgfpathlineto{\pgfqpoint{1.351240in}{3.188428in}}%
\pgfpathlineto{\pgfqpoint{1.373776in}{3.181163in}}%
\pgfpathlineto{\pgfqpoint{1.396311in}{3.174933in}}%
\pgfpathlineto{\pgfqpoint{1.418846in}{3.168160in}}%
\pgfpathlineto{\pgfqpoint{1.441381in}{3.161362in}}%
\pgfpathlineto{\pgfqpoint{1.463917in}{3.155108in}}%
\pgfpathlineto{\pgfqpoint{1.486452in}{3.149742in}}%
\pgfpathlineto{\pgfqpoint{1.508987in}{3.144140in}}%
\pgfpathlineto{\pgfqpoint{1.531523in}{3.136470in}}%
\pgfpathlineto{\pgfqpoint{1.554058in}{3.130564in}}%
\pgfpathlineto{\pgfqpoint{1.576593in}{3.124728in}}%
\pgfpathlineto{\pgfqpoint{1.599128in}{3.118492in}}%
\pgfpathlineto{\pgfqpoint{1.621664in}{3.112472in}}%
\pgfpathlineto{\pgfqpoint{1.644199in}{3.105724in}}%
\pgfpathlineto{\pgfqpoint{1.666734in}{3.100918in}}%
\pgfpathlineto{\pgfqpoint{1.689270in}{3.095260in}}%
\pgfpathlineto{\pgfqpoint{1.711805in}{3.089744in}}%
\pgfpathlineto{\pgfqpoint{1.734340in}{3.084646in}}%
\pgfpathlineto{\pgfqpoint{1.756875in}{3.080156in}}%
\pgfpathlineto{\pgfqpoint{1.779411in}{3.074930in}}%
\pgfpathlineto{\pgfqpoint{1.801946in}{3.070616in}}%
\pgfpathlineto{\pgfqpoint{1.824481in}{3.065202in}}%
\pgfpathlineto{\pgfqpoint{1.847017in}{3.059723in}}%
\pgfpathlineto{\pgfqpoint{1.869552in}{3.054189in}}%
\pgfpathlineto{\pgfqpoint{1.892087in}{3.049483in}}%
\pgfpathlineto{\pgfqpoint{1.914622in}{3.045014in}}%
\pgfpathlineto{\pgfqpoint{1.937158in}{3.040485in}}%
\pgfpathlineto{\pgfqpoint{1.959693in}{3.035538in}}%
\pgfpathlineto{\pgfqpoint{1.982228in}{3.030605in}}%
\pgfpathlineto{\pgfqpoint{2.004764in}{3.026375in}}%
\pgfpathlineto{\pgfqpoint{2.027299in}{3.022673in}}%
\pgfpathlineto{\pgfqpoint{2.049834in}{3.017741in}}%
\pgfpathlineto{\pgfqpoint{2.072369in}{3.013467in}}%
\pgfpathlineto{\pgfqpoint{2.094905in}{3.009751in}}%
\pgfpathlineto{\pgfqpoint{2.117440in}{3.006001in}}%
\pgfpathlineto{\pgfqpoint{2.139975in}{3.002414in}}%
\pgfpathlineto{\pgfqpoint{2.162511in}{2.997871in}}%
\pgfpathlineto{\pgfqpoint{2.185046in}{2.993943in}}%
\pgfpathlineto{\pgfqpoint{2.207581in}{2.988864in}}%
\pgfpathlineto{\pgfqpoint{2.230116in}{2.985371in}}%
\pgfpathlineto{\pgfqpoint{2.252652in}{2.981807in}}%
\pgfpathlineto{\pgfqpoint{2.275187in}{2.978738in}}%
\pgfpathlineto{\pgfqpoint{2.297722in}{2.974724in}}%
\pgfpathlineto{\pgfqpoint{2.320258in}{2.970851in}}%
\pgfpathlineto{\pgfqpoint{2.342793in}{2.966447in}}%
\pgfpathlineto{\pgfqpoint{2.365328in}{2.962347in}}%
\pgfpathlineto{\pgfqpoint{2.387863in}{2.959038in}}%
\pgfpathlineto{\pgfqpoint{2.410399in}{2.954866in}}%
\pgfpathlineto{\pgfqpoint{2.432934in}{2.951275in}}%
\pgfpathlineto{\pgfqpoint{2.455469in}{2.947519in}}%
\pgfpathlineto{\pgfqpoint{2.478005in}{2.944064in}}%
\pgfpathlineto{\pgfqpoint{2.500540in}{2.941083in}}%
\pgfpathlineto{\pgfqpoint{2.523075in}{2.937481in}}%
\pgfpathlineto{\pgfqpoint{2.545610in}{2.933408in}}%
\pgfpathlineto{\pgfqpoint{2.568146in}{2.929532in}}%
\pgfpathlineto{\pgfqpoint{2.590681in}{2.925688in}}%
\pgfpathlineto{\pgfqpoint{2.613216in}{2.922864in}}%
\pgfpathlineto{\pgfqpoint{2.635752in}{2.919247in}}%
\pgfpathlineto{\pgfqpoint{2.658287in}{2.915979in}}%
\pgfpathlineto{\pgfqpoint{2.680822in}{2.912897in}}%
\pgfpathlineto{\pgfqpoint{2.703357in}{2.909578in}}%
\pgfpathlineto{\pgfqpoint{2.725893in}{2.906615in}}%
\pgfpathlineto{\pgfqpoint{2.748428in}{2.903497in}}%
\pgfpathlineto{\pgfqpoint{2.770963in}{2.900440in}}%
\pgfusepath{stroke}%
\end{pgfscope}%
\begin{pgfscope}%
\pgfpathrectangle{\pgfqpoint{0.539970in}{2.747992in}}{\pgfqpoint{2.276064in}{1.626201in}}%
\pgfusepath{clip}%
\pgfsetrectcap%
\pgfsetroundjoin%
\pgfsetlinewidth{1.003750pt}%
\definecolor{currentstroke}{rgb}{0.517647,0.356863,0.592157}%
\pgfsetstrokecolor{currentstroke}%
\pgfsetdash{}{0pt}%
\pgfpathmoveto{\pgfqpoint{0.562505in}{3.230962in}}%
\pgfpathlineto{\pgfqpoint{0.585040in}{3.233293in}}%
\pgfpathlineto{\pgfqpoint{0.607576in}{3.208073in}}%
\pgfpathlineto{\pgfqpoint{0.630111in}{3.211602in}}%
\pgfpathlineto{\pgfqpoint{0.652646in}{3.185758in}}%
\pgfpathlineto{\pgfqpoint{0.675182in}{3.144792in}}%
\pgfpathlineto{\pgfqpoint{0.697717in}{3.132120in}}%
\pgfpathlineto{\pgfqpoint{0.720252in}{3.108103in}}%
\pgfpathlineto{\pgfqpoint{0.742787in}{3.092772in}}%
\pgfpathlineto{\pgfqpoint{0.765323in}{3.080845in}}%
\pgfpathlineto{\pgfqpoint{0.787858in}{3.065066in}}%
\pgfpathlineto{\pgfqpoint{0.810393in}{3.054308in}}%
\pgfpathlineto{\pgfqpoint{0.832929in}{3.045140in}}%
\pgfpathlineto{\pgfqpoint{0.855464in}{3.033875in}}%
\pgfpathlineto{\pgfqpoint{0.877999in}{3.028422in}}%
\pgfpathlineto{\pgfqpoint{0.900534in}{3.015704in}}%
\pgfpathlineto{\pgfqpoint{0.923070in}{3.011322in}}%
\pgfpathlineto{\pgfqpoint{0.945605in}{3.005264in}}%
\pgfpathlineto{\pgfqpoint{0.968140in}{2.999390in}}%
\pgfpathlineto{\pgfqpoint{0.990676in}{2.996949in}}%
\pgfpathlineto{\pgfqpoint{1.013211in}{2.988892in}}%
\pgfpathlineto{\pgfqpoint{1.035746in}{2.982734in}}%
\pgfpathlineto{\pgfqpoint{1.058281in}{2.974706in}}%
\pgfpathlineto{\pgfqpoint{1.080817in}{2.969644in}}%
\pgfpathlineto{\pgfqpoint{1.103352in}{2.966887in}}%
\pgfpathlineto{\pgfqpoint{1.125887in}{2.964513in}}%
\pgfpathlineto{\pgfqpoint{1.148423in}{2.959307in}}%
\pgfpathlineto{\pgfqpoint{1.170958in}{2.956528in}}%
\pgfpathlineto{\pgfqpoint{1.193493in}{2.952340in}}%
\pgfpathlineto{\pgfqpoint{1.216028in}{2.947798in}}%
\pgfpathlineto{\pgfqpoint{1.238564in}{2.941344in}}%
\pgfpathlineto{\pgfqpoint{1.261099in}{2.933815in}}%
\pgfpathlineto{\pgfqpoint{1.283634in}{2.931901in}}%
\pgfpathlineto{\pgfqpoint{1.306170in}{2.929241in}}%
\pgfpathlineto{\pgfqpoint{1.328705in}{2.927065in}}%
\pgfpathlineto{\pgfqpoint{1.351240in}{2.922181in}}%
\pgfpathlineto{\pgfqpoint{1.373776in}{2.919958in}}%
\pgfpathlineto{\pgfqpoint{1.396311in}{2.916492in}}%
\pgfpathlineto{\pgfqpoint{1.418846in}{2.910871in}}%
\pgfpathlineto{\pgfqpoint{1.441381in}{2.907498in}}%
\pgfpathlineto{\pgfqpoint{1.463917in}{2.902816in}}%
\pgfpathlineto{\pgfqpoint{1.486452in}{2.899079in}}%
\pgfpathlineto{\pgfqpoint{1.508987in}{2.896692in}}%
\pgfpathlineto{\pgfqpoint{1.531523in}{2.895924in}}%
\pgfpathlineto{\pgfqpoint{1.554058in}{2.895048in}}%
\pgfpathlineto{\pgfqpoint{1.576593in}{2.893131in}}%
\pgfpathlineto{\pgfqpoint{1.599128in}{2.892753in}}%
\pgfpathlineto{\pgfqpoint{1.621664in}{2.888376in}}%
\pgfpathlineto{\pgfqpoint{1.644199in}{2.885756in}}%
\pgfpathlineto{\pgfqpoint{1.666734in}{2.883836in}}%
\pgfpathlineto{\pgfqpoint{1.689270in}{2.880064in}}%
\pgfpathlineto{\pgfqpoint{1.711805in}{2.877594in}}%
\pgfpathlineto{\pgfqpoint{1.734340in}{2.875013in}}%
\pgfpathlineto{\pgfqpoint{1.756875in}{2.871793in}}%
\pgfpathlineto{\pgfqpoint{1.779411in}{2.871349in}}%
\pgfpathlineto{\pgfqpoint{1.801946in}{2.868804in}}%
\pgfpathlineto{\pgfqpoint{1.824481in}{2.867356in}}%
\pgfpathlineto{\pgfqpoint{1.847017in}{2.866695in}}%
\pgfpathlineto{\pgfqpoint{1.869552in}{2.866287in}}%
\pgfpathlineto{\pgfqpoint{1.892087in}{2.865860in}}%
\pgfpathlineto{\pgfqpoint{1.914622in}{2.867335in}}%
\pgfpathlineto{\pgfqpoint{1.937158in}{2.864468in}}%
\pgfpathlineto{\pgfqpoint{1.959693in}{2.862169in}}%
\pgfpathlineto{\pgfqpoint{1.982228in}{2.860825in}}%
\pgfpathlineto{\pgfqpoint{2.004764in}{2.859008in}}%
\pgfpathlineto{\pgfqpoint{2.027299in}{2.858502in}}%
\pgfpathlineto{\pgfqpoint{2.049834in}{2.858539in}}%
\pgfpathlineto{\pgfqpoint{2.072369in}{2.857089in}}%
\pgfpathlineto{\pgfqpoint{2.094905in}{2.855542in}}%
\pgfpathlineto{\pgfqpoint{2.117440in}{2.854933in}}%
\pgfpathlineto{\pgfqpoint{2.139975in}{2.852762in}}%
\pgfpathlineto{\pgfqpoint{2.162511in}{2.852464in}}%
\pgfpathlineto{\pgfqpoint{2.185046in}{2.849423in}}%
\pgfpathlineto{\pgfqpoint{2.207581in}{2.848824in}}%
\pgfpathlineto{\pgfqpoint{2.230116in}{2.848046in}}%
\pgfpathlineto{\pgfqpoint{2.252652in}{2.846931in}}%
\pgfpathlineto{\pgfqpoint{2.275187in}{2.844363in}}%
\pgfpathlineto{\pgfqpoint{2.297722in}{2.840853in}}%
\pgfpathlineto{\pgfqpoint{2.320258in}{2.840615in}}%
\pgfpathlineto{\pgfqpoint{2.342793in}{2.839688in}}%
\pgfpathlineto{\pgfqpoint{2.365328in}{2.838517in}}%
\pgfpathlineto{\pgfqpoint{2.387863in}{2.838128in}}%
\pgfpathlineto{\pgfqpoint{2.410399in}{2.837466in}}%
\pgfpathlineto{\pgfqpoint{2.432934in}{2.837547in}}%
\pgfpathlineto{\pgfqpoint{2.455469in}{2.835667in}}%
\pgfpathlineto{\pgfqpoint{2.478005in}{2.835157in}}%
\pgfpathlineto{\pgfqpoint{2.500540in}{2.833757in}}%
\pgfpathlineto{\pgfqpoint{2.523075in}{2.832930in}}%
\pgfpathlineto{\pgfqpoint{2.545610in}{2.831423in}}%
\pgfpathlineto{\pgfqpoint{2.568146in}{2.830354in}}%
\pgfpathlineto{\pgfqpoint{2.590681in}{2.828254in}}%
\pgfpathlineto{\pgfqpoint{2.613216in}{2.826889in}}%
\pgfpathlineto{\pgfqpoint{2.635752in}{2.826006in}}%
\pgfpathlineto{\pgfqpoint{2.658287in}{2.824889in}}%
\pgfpathlineto{\pgfqpoint{2.680822in}{2.824420in}}%
\pgfpathlineto{\pgfqpoint{2.703357in}{2.824788in}}%
\pgfpathlineto{\pgfqpoint{2.725893in}{2.823640in}}%
\pgfpathlineto{\pgfqpoint{2.748428in}{2.821910in}}%
\pgfpathlineto{\pgfqpoint{2.770963in}{2.821920in}}%
\pgfusepath{stroke}%
\end{pgfscope}%
\begin{pgfscope}%
\pgfsetrectcap%
\pgfsetmiterjoin%
\pgfsetlinewidth{0.501875pt}%
\definecolor{currentstroke}{rgb}{0.000000,0.000000,0.000000}%
\pgfsetstrokecolor{currentstroke}%
\pgfsetdash{}{0pt}%
\pgfpathmoveto{\pgfqpoint{0.539970in}{2.747992in}}%
\pgfpathlineto{\pgfqpoint{0.539970in}{4.374193in}}%
\pgfusepath{stroke}%
\end{pgfscope}%
\begin{pgfscope}%
\pgfsetrectcap%
\pgfsetmiterjoin%
\pgfsetlinewidth{0.501875pt}%
\definecolor{currentstroke}{rgb}{0.000000,0.000000,0.000000}%
\pgfsetstrokecolor{currentstroke}%
\pgfsetdash{}{0pt}%
\pgfpathmoveto{\pgfqpoint{2.816034in}{2.747992in}}%
\pgfpathlineto{\pgfqpoint{2.816034in}{4.374193in}}%
\pgfusepath{stroke}%
\end{pgfscope}%
\begin{pgfscope}%
\pgfsetrectcap%
\pgfsetmiterjoin%
\pgfsetlinewidth{0.501875pt}%
\definecolor{currentstroke}{rgb}{0.000000,0.000000,0.000000}%
\pgfsetstrokecolor{currentstroke}%
\pgfsetdash{}{0pt}%
\pgfpathmoveto{\pgfqpoint{0.539970in}{2.747992in}}%
\pgfpathlineto{\pgfqpoint{2.816034in}{2.747992in}}%
\pgfusepath{stroke}%
\end{pgfscope}%
\begin{pgfscope}%
\pgfsetrectcap%
\pgfsetmiterjoin%
\pgfsetlinewidth{0.501875pt}%
\definecolor{currentstroke}{rgb}{0.000000,0.000000,0.000000}%
\pgfsetstrokecolor{currentstroke}%
\pgfsetdash{}{0pt}%
\pgfpathmoveto{\pgfqpoint{0.539970in}{4.374193in}}%
\pgfpathlineto{\pgfqpoint{2.816034in}{4.374193in}}%
\pgfusepath{stroke}%
\end{pgfscope}%
\begin{pgfscope}%
\definecolor{textcolor}{rgb}{0.000000,0.000000,0.000000}%
\pgfsetstrokecolor{textcolor}%
\pgfsetfillcolor{textcolor}%
\pgftext[x=1.678002in,y=4.457526in,,base]{\color{textcolor}\rmfamily\fontsize{12.000000}{14.400000}\selectfont Trustworthiness}%
\end{pgfscope}%
\begin{pgfscope}%
\pgfsetbuttcap%
\pgfsetmiterjoin%
\definecolor{currentfill}{rgb}{1.000000,1.000000,1.000000}%
\pgfsetfillcolor{currentfill}%
\pgfsetlinewidth{0.000000pt}%
\definecolor{currentstroke}{rgb}{0.000000,0.000000,0.000000}%
\pgfsetstrokecolor{currentstroke}%
\pgfsetstrokeopacity{0.000000}%
\pgfsetdash{}{0pt}%
\pgfpathmoveto{\pgfqpoint{0.539970in}{0.422992in}}%
\pgfpathlineto{\pgfqpoint{2.816034in}{0.422992in}}%
\pgfpathlineto{\pgfqpoint{2.816034in}{2.049193in}}%
\pgfpathlineto{\pgfqpoint{0.539970in}{2.049193in}}%
\pgfpathlineto{\pgfqpoint{0.539970in}{0.422992in}}%
\pgfpathclose%
\pgfusepath{fill}%
\end{pgfscope}%
\begin{pgfscope}%
\pgfsetbuttcap%
\pgfsetroundjoin%
\definecolor{currentfill}{rgb}{0.000000,0.000000,0.000000}%
\pgfsetfillcolor{currentfill}%
\pgfsetlinewidth{0.501875pt}%
\definecolor{currentstroke}{rgb}{0.000000,0.000000,0.000000}%
\pgfsetstrokecolor{currentstroke}%
\pgfsetdash{}{0pt}%
\pgfsys@defobject{currentmarker}{\pgfqpoint{0.000000in}{0.000000in}}{\pgfqpoint{0.000000in}{0.041667in}}{%
\pgfpathmoveto{\pgfqpoint{0.000000in}{0.000000in}}%
\pgfpathlineto{\pgfqpoint{0.000000in}{0.041667in}}%
\pgfusepath{stroke,fill}%
}%
\begin{pgfscope}%
\pgfsys@transformshift{0.539970in}{0.422992in}%
\pgfsys@useobject{currentmarker}{}%
\end{pgfscope}%
\end{pgfscope}%
\begin{pgfscope}%
\pgfsetbuttcap%
\pgfsetroundjoin%
\definecolor{currentfill}{rgb}{0.000000,0.000000,0.000000}%
\pgfsetfillcolor{currentfill}%
\pgfsetlinewidth{0.501875pt}%
\definecolor{currentstroke}{rgb}{0.000000,0.000000,0.000000}%
\pgfsetstrokecolor{currentstroke}%
\pgfsetdash{}{0pt}%
\pgfsys@defobject{currentmarker}{\pgfqpoint{0.000000in}{-0.041667in}}{\pgfqpoint{0.000000in}{0.000000in}}{%
\pgfpathmoveto{\pgfqpoint{0.000000in}{0.000000in}}%
\pgfpathlineto{\pgfqpoint{0.000000in}{-0.041667in}}%
\pgfusepath{stroke,fill}%
}%
\begin{pgfscope}%
\pgfsys@transformshift{0.539970in}{2.049193in}%
\pgfsys@useobject{currentmarker}{}%
\end{pgfscope}%
\end{pgfscope}%
\begin{pgfscope}%
\definecolor{textcolor}{rgb}{0.000000,0.000000,0.000000}%
\pgfsetstrokecolor{textcolor}%
\pgfsetfillcolor{textcolor}%
\pgftext[x=0.539970in,y=0.374381in,,top]{\color{textcolor}\rmfamily\fontsize{10.000000}{12.000000}\selectfont \(\displaystyle {0}\)}%
\end{pgfscope}%
\begin{pgfscope}%
\pgfsetbuttcap%
\pgfsetroundjoin%
\definecolor{currentfill}{rgb}{0.000000,0.000000,0.000000}%
\pgfsetfillcolor{currentfill}%
\pgfsetlinewidth{0.501875pt}%
\definecolor{currentstroke}{rgb}{0.000000,0.000000,0.000000}%
\pgfsetstrokecolor{currentstroke}%
\pgfsetdash{}{0pt}%
\pgfsys@defobject{currentmarker}{\pgfqpoint{0.000000in}{0.000000in}}{\pgfqpoint{0.000000in}{0.041667in}}{%
\pgfpathmoveto{\pgfqpoint{0.000000in}{0.000000in}}%
\pgfpathlineto{\pgfqpoint{0.000000in}{0.041667in}}%
\pgfusepath{stroke,fill}%
}%
\begin{pgfscope}%
\pgfsys@transformshift{0.990676in}{0.422992in}%
\pgfsys@useobject{currentmarker}{}%
\end{pgfscope}%
\end{pgfscope}%
\begin{pgfscope}%
\pgfsetbuttcap%
\pgfsetroundjoin%
\definecolor{currentfill}{rgb}{0.000000,0.000000,0.000000}%
\pgfsetfillcolor{currentfill}%
\pgfsetlinewidth{0.501875pt}%
\definecolor{currentstroke}{rgb}{0.000000,0.000000,0.000000}%
\pgfsetstrokecolor{currentstroke}%
\pgfsetdash{}{0pt}%
\pgfsys@defobject{currentmarker}{\pgfqpoint{0.000000in}{-0.041667in}}{\pgfqpoint{0.000000in}{0.000000in}}{%
\pgfpathmoveto{\pgfqpoint{0.000000in}{0.000000in}}%
\pgfpathlineto{\pgfqpoint{0.000000in}{-0.041667in}}%
\pgfusepath{stroke,fill}%
}%
\begin{pgfscope}%
\pgfsys@transformshift{0.990676in}{2.049193in}%
\pgfsys@useobject{currentmarker}{}%
\end{pgfscope}%
\end{pgfscope}%
\begin{pgfscope}%
\definecolor{textcolor}{rgb}{0.000000,0.000000,0.000000}%
\pgfsetstrokecolor{textcolor}%
\pgfsetfillcolor{textcolor}%
\pgftext[x=0.990676in,y=0.374381in,,top]{\color{textcolor}\rmfamily\fontsize{10.000000}{12.000000}\selectfont \(\displaystyle {20}\)}%
\end{pgfscope}%
\begin{pgfscope}%
\pgfsetbuttcap%
\pgfsetroundjoin%
\definecolor{currentfill}{rgb}{0.000000,0.000000,0.000000}%
\pgfsetfillcolor{currentfill}%
\pgfsetlinewidth{0.501875pt}%
\definecolor{currentstroke}{rgb}{0.000000,0.000000,0.000000}%
\pgfsetstrokecolor{currentstroke}%
\pgfsetdash{}{0pt}%
\pgfsys@defobject{currentmarker}{\pgfqpoint{0.000000in}{0.000000in}}{\pgfqpoint{0.000000in}{0.041667in}}{%
\pgfpathmoveto{\pgfqpoint{0.000000in}{0.000000in}}%
\pgfpathlineto{\pgfqpoint{0.000000in}{0.041667in}}%
\pgfusepath{stroke,fill}%
}%
\begin{pgfscope}%
\pgfsys@transformshift{1.441381in}{0.422992in}%
\pgfsys@useobject{currentmarker}{}%
\end{pgfscope}%
\end{pgfscope}%
\begin{pgfscope}%
\pgfsetbuttcap%
\pgfsetroundjoin%
\definecolor{currentfill}{rgb}{0.000000,0.000000,0.000000}%
\pgfsetfillcolor{currentfill}%
\pgfsetlinewidth{0.501875pt}%
\definecolor{currentstroke}{rgb}{0.000000,0.000000,0.000000}%
\pgfsetstrokecolor{currentstroke}%
\pgfsetdash{}{0pt}%
\pgfsys@defobject{currentmarker}{\pgfqpoint{0.000000in}{-0.041667in}}{\pgfqpoint{0.000000in}{0.000000in}}{%
\pgfpathmoveto{\pgfqpoint{0.000000in}{0.000000in}}%
\pgfpathlineto{\pgfqpoint{0.000000in}{-0.041667in}}%
\pgfusepath{stroke,fill}%
}%
\begin{pgfscope}%
\pgfsys@transformshift{1.441381in}{2.049193in}%
\pgfsys@useobject{currentmarker}{}%
\end{pgfscope}%
\end{pgfscope}%
\begin{pgfscope}%
\definecolor{textcolor}{rgb}{0.000000,0.000000,0.000000}%
\pgfsetstrokecolor{textcolor}%
\pgfsetfillcolor{textcolor}%
\pgftext[x=1.441381in,y=0.374381in,,top]{\color{textcolor}\rmfamily\fontsize{10.000000}{12.000000}\selectfont \(\displaystyle {40}\)}%
\end{pgfscope}%
\begin{pgfscope}%
\pgfsetbuttcap%
\pgfsetroundjoin%
\definecolor{currentfill}{rgb}{0.000000,0.000000,0.000000}%
\pgfsetfillcolor{currentfill}%
\pgfsetlinewidth{0.501875pt}%
\definecolor{currentstroke}{rgb}{0.000000,0.000000,0.000000}%
\pgfsetstrokecolor{currentstroke}%
\pgfsetdash{}{0pt}%
\pgfsys@defobject{currentmarker}{\pgfqpoint{0.000000in}{0.000000in}}{\pgfqpoint{0.000000in}{0.041667in}}{%
\pgfpathmoveto{\pgfqpoint{0.000000in}{0.000000in}}%
\pgfpathlineto{\pgfqpoint{0.000000in}{0.041667in}}%
\pgfusepath{stroke,fill}%
}%
\begin{pgfscope}%
\pgfsys@transformshift{1.892087in}{0.422992in}%
\pgfsys@useobject{currentmarker}{}%
\end{pgfscope}%
\end{pgfscope}%
\begin{pgfscope}%
\pgfsetbuttcap%
\pgfsetroundjoin%
\definecolor{currentfill}{rgb}{0.000000,0.000000,0.000000}%
\pgfsetfillcolor{currentfill}%
\pgfsetlinewidth{0.501875pt}%
\definecolor{currentstroke}{rgb}{0.000000,0.000000,0.000000}%
\pgfsetstrokecolor{currentstroke}%
\pgfsetdash{}{0pt}%
\pgfsys@defobject{currentmarker}{\pgfqpoint{0.000000in}{-0.041667in}}{\pgfqpoint{0.000000in}{0.000000in}}{%
\pgfpathmoveto{\pgfqpoint{0.000000in}{0.000000in}}%
\pgfpathlineto{\pgfqpoint{0.000000in}{-0.041667in}}%
\pgfusepath{stroke,fill}%
}%
\begin{pgfscope}%
\pgfsys@transformshift{1.892087in}{2.049193in}%
\pgfsys@useobject{currentmarker}{}%
\end{pgfscope}%
\end{pgfscope}%
\begin{pgfscope}%
\definecolor{textcolor}{rgb}{0.000000,0.000000,0.000000}%
\pgfsetstrokecolor{textcolor}%
\pgfsetfillcolor{textcolor}%
\pgftext[x=1.892087in,y=0.374381in,,top]{\color{textcolor}\rmfamily\fontsize{10.000000}{12.000000}\selectfont \(\displaystyle {60}\)}%
\end{pgfscope}%
\begin{pgfscope}%
\pgfsetbuttcap%
\pgfsetroundjoin%
\definecolor{currentfill}{rgb}{0.000000,0.000000,0.000000}%
\pgfsetfillcolor{currentfill}%
\pgfsetlinewidth{0.501875pt}%
\definecolor{currentstroke}{rgb}{0.000000,0.000000,0.000000}%
\pgfsetstrokecolor{currentstroke}%
\pgfsetdash{}{0pt}%
\pgfsys@defobject{currentmarker}{\pgfqpoint{0.000000in}{0.000000in}}{\pgfqpoint{0.000000in}{0.041667in}}{%
\pgfpathmoveto{\pgfqpoint{0.000000in}{0.000000in}}%
\pgfpathlineto{\pgfqpoint{0.000000in}{0.041667in}}%
\pgfusepath{stroke,fill}%
}%
\begin{pgfscope}%
\pgfsys@transformshift{2.342793in}{0.422992in}%
\pgfsys@useobject{currentmarker}{}%
\end{pgfscope}%
\end{pgfscope}%
\begin{pgfscope}%
\pgfsetbuttcap%
\pgfsetroundjoin%
\definecolor{currentfill}{rgb}{0.000000,0.000000,0.000000}%
\pgfsetfillcolor{currentfill}%
\pgfsetlinewidth{0.501875pt}%
\definecolor{currentstroke}{rgb}{0.000000,0.000000,0.000000}%
\pgfsetstrokecolor{currentstroke}%
\pgfsetdash{}{0pt}%
\pgfsys@defobject{currentmarker}{\pgfqpoint{0.000000in}{-0.041667in}}{\pgfqpoint{0.000000in}{0.000000in}}{%
\pgfpathmoveto{\pgfqpoint{0.000000in}{0.000000in}}%
\pgfpathlineto{\pgfqpoint{0.000000in}{-0.041667in}}%
\pgfusepath{stroke,fill}%
}%
\begin{pgfscope}%
\pgfsys@transformshift{2.342793in}{2.049193in}%
\pgfsys@useobject{currentmarker}{}%
\end{pgfscope}%
\end{pgfscope}%
\begin{pgfscope}%
\definecolor{textcolor}{rgb}{0.000000,0.000000,0.000000}%
\pgfsetstrokecolor{textcolor}%
\pgfsetfillcolor{textcolor}%
\pgftext[x=2.342793in,y=0.374381in,,top]{\color{textcolor}\rmfamily\fontsize{10.000000}{12.000000}\selectfont \(\displaystyle {80}\)}%
\end{pgfscope}%
\begin{pgfscope}%
\pgfsetbuttcap%
\pgfsetroundjoin%
\definecolor{currentfill}{rgb}{0.000000,0.000000,0.000000}%
\pgfsetfillcolor{currentfill}%
\pgfsetlinewidth{0.501875pt}%
\definecolor{currentstroke}{rgb}{0.000000,0.000000,0.000000}%
\pgfsetstrokecolor{currentstroke}%
\pgfsetdash{}{0pt}%
\pgfsys@defobject{currentmarker}{\pgfqpoint{0.000000in}{0.000000in}}{\pgfqpoint{0.000000in}{0.020833in}}{%
\pgfpathmoveto{\pgfqpoint{0.000000in}{0.000000in}}%
\pgfpathlineto{\pgfqpoint{0.000000in}{0.020833in}}%
\pgfusepath{stroke,fill}%
}%
\begin{pgfscope}%
\pgfsys@transformshift{0.652646in}{0.422992in}%
\pgfsys@useobject{currentmarker}{}%
\end{pgfscope}%
\end{pgfscope}%
\begin{pgfscope}%
\pgfsetbuttcap%
\pgfsetroundjoin%
\definecolor{currentfill}{rgb}{0.000000,0.000000,0.000000}%
\pgfsetfillcolor{currentfill}%
\pgfsetlinewidth{0.501875pt}%
\definecolor{currentstroke}{rgb}{0.000000,0.000000,0.000000}%
\pgfsetstrokecolor{currentstroke}%
\pgfsetdash{}{0pt}%
\pgfsys@defobject{currentmarker}{\pgfqpoint{0.000000in}{-0.020833in}}{\pgfqpoint{0.000000in}{0.000000in}}{%
\pgfpathmoveto{\pgfqpoint{0.000000in}{0.000000in}}%
\pgfpathlineto{\pgfqpoint{0.000000in}{-0.020833in}}%
\pgfusepath{stroke,fill}%
}%
\begin{pgfscope}%
\pgfsys@transformshift{0.652646in}{2.049193in}%
\pgfsys@useobject{currentmarker}{}%
\end{pgfscope}%
\end{pgfscope}%
\begin{pgfscope}%
\pgfsetbuttcap%
\pgfsetroundjoin%
\definecolor{currentfill}{rgb}{0.000000,0.000000,0.000000}%
\pgfsetfillcolor{currentfill}%
\pgfsetlinewidth{0.501875pt}%
\definecolor{currentstroke}{rgb}{0.000000,0.000000,0.000000}%
\pgfsetstrokecolor{currentstroke}%
\pgfsetdash{}{0pt}%
\pgfsys@defobject{currentmarker}{\pgfqpoint{0.000000in}{0.000000in}}{\pgfqpoint{0.000000in}{0.020833in}}{%
\pgfpathmoveto{\pgfqpoint{0.000000in}{0.000000in}}%
\pgfpathlineto{\pgfqpoint{0.000000in}{0.020833in}}%
\pgfusepath{stroke,fill}%
}%
\begin{pgfscope}%
\pgfsys@transformshift{0.765323in}{0.422992in}%
\pgfsys@useobject{currentmarker}{}%
\end{pgfscope}%
\end{pgfscope}%
\begin{pgfscope}%
\pgfsetbuttcap%
\pgfsetroundjoin%
\definecolor{currentfill}{rgb}{0.000000,0.000000,0.000000}%
\pgfsetfillcolor{currentfill}%
\pgfsetlinewidth{0.501875pt}%
\definecolor{currentstroke}{rgb}{0.000000,0.000000,0.000000}%
\pgfsetstrokecolor{currentstroke}%
\pgfsetdash{}{0pt}%
\pgfsys@defobject{currentmarker}{\pgfqpoint{0.000000in}{-0.020833in}}{\pgfqpoint{0.000000in}{0.000000in}}{%
\pgfpathmoveto{\pgfqpoint{0.000000in}{0.000000in}}%
\pgfpathlineto{\pgfqpoint{0.000000in}{-0.020833in}}%
\pgfusepath{stroke,fill}%
}%
\begin{pgfscope}%
\pgfsys@transformshift{0.765323in}{2.049193in}%
\pgfsys@useobject{currentmarker}{}%
\end{pgfscope}%
\end{pgfscope}%
\begin{pgfscope}%
\pgfsetbuttcap%
\pgfsetroundjoin%
\definecolor{currentfill}{rgb}{0.000000,0.000000,0.000000}%
\pgfsetfillcolor{currentfill}%
\pgfsetlinewidth{0.501875pt}%
\definecolor{currentstroke}{rgb}{0.000000,0.000000,0.000000}%
\pgfsetstrokecolor{currentstroke}%
\pgfsetdash{}{0pt}%
\pgfsys@defobject{currentmarker}{\pgfqpoint{0.000000in}{0.000000in}}{\pgfqpoint{0.000000in}{0.020833in}}{%
\pgfpathmoveto{\pgfqpoint{0.000000in}{0.000000in}}%
\pgfpathlineto{\pgfqpoint{0.000000in}{0.020833in}}%
\pgfusepath{stroke,fill}%
}%
\begin{pgfscope}%
\pgfsys@transformshift{0.877999in}{0.422992in}%
\pgfsys@useobject{currentmarker}{}%
\end{pgfscope}%
\end{pgfscope}%
\begin{pgfscope}%
\pgfsetbuttcap%
\pgfsetroundjoin%
\definecolor{currentfill}{rgb}{0.000000,0.000000,0.000000}%
\pgfsetfillcolor{currentfill}%
\pgfsetlinewidth{0.501875pt}%
\definecolor{currentstroke}{rgb}{0.000000,0.000000,0.000000}%
\pgfsetstrokecolor{currentstroke}%
\pgfsetdash{}{0pt}%
\pgfsys@defobject{currentmarker}{\pgfqpoint{0.000000in}{-0.020833in}}{\pgfqpoint{0.000000in}{0.000000in}}{%
\pgfpathmoveto{\pgfqpoint{0.000000in}{0.000000in}}%
\pgfpathlineto{\pgfqpoint{0.000000in}{-0.020833in}}%
\pgfusepath{stroke,fill}%
}%
\begin{pgfscope}%
\pgfsys@transformshift{0.877999in}{2.049193in}%
\pgfsys@useobject{currentmarker}{}%
\end{pgfscope}%
\end{pgfscope}%
\begin{pgfscope}%
\pgfsetbuttcap%
\pgfsetroundjoin%
\definecolor{currentfill}{rgb}{0.000000,0.000000,0.000000}%
\pgfsetfillcolor{currentfill}%
\pgfsetlinewidth{0.501875pt}%
\definecolor{currentstroke}{rgb}{0.000000,0.000000,0.000000}%
\pgfsetstrokecolor{currentstroke}%
\pgfsetdash{}{0pt}%
\pgfsys@defobject{currentmarker}{\pgfqpoint{0.000000in}{0.000000in}}{\pgfqpoint{0.000000in}{0.020833in}}{%
\pgfpathmoveto{\pgfqpoint{0.000000in}{0.000000in}}%
\pgfpathlineto{\pgfqpoint{0.000000in}{0.020833in}}%
\pgfusepath{stroke,fill}%
}%
\begin{pgfscope}%
\pgfsys@transformshift{1.103352in}{0.422992in}%
\pgfsys@useobject{currentmarker}{}%
\end{pgfscope}%
\end{pgfscope}%
\begin{pgfscope}%
\pgfsetbuttcap%
\pgfsetroundjoin%
\definecolor{currentfill}{rgb}{0.000000,0.000000,0.000000}%
\pgfsetfillcolor{currentfill}%
\pgfsetlinewidth{0.501875pt}%
\definecolor{currentstroke}{rgb}{0.000000,0.000000,0.000000}%
\pgfsetstrokecolor{currentstroke}%
\pgfsetdash{}{0pt}%
\pgfsys@defobject{currentmarker}{\pgfqpoint{0.000000in}{-0.020833in}}{\pgfqpoint{0.000000in}{0.000000in}}{%
\pgfpathmoveto{\pgfqpoint{0.000000in}{0.000000in}}%
\pgfpathlineto{\pgfqpoint{0.000000in}{-0.020833in}}%
\pgfusepath{stroke,fill}%
}%
\begin{pgfscope}%
\pgfsys@transformshift{1.103352in}{2.049193in}%
\pgfsys@useobject{currentmarker}{}%
\end{pgfscope}%
\end{pgfscope}%
\begin{pgfscope}%
\pgfsetbuttcap%
\pgfsetroundjoin%
\definecolor{currentfill}{rgb}{0.000000,0.000000,0.000000}%
\pgfsetfillcolor{currentfill}%
\pgfsetlinewidth{0.501875pt}%
\definecolor{currentstroke}{rgb}{0.000000,0.000000,0.000000}%
\pgfsetstrokecolor{currentstroke}%
\pgfsetdash{}{0pt}%
\pgfsys@defobject{currentmarker}{\pgfqpoint{0.000000in}{0.000000in}}{\pgfqpoint{0.000000in}{0.020833in}}{%
\pgfpathmoveto{\pgfqpoint{0.000000in}{0.000000in}}%
\pgfpathlineto{\pgfqpoint{0.000000in}{0.020833in}}%
\pgfusepath{stroke,fill}%
}%
\begin{pgfscope}%
\pgfsys@transformshift{1.216028in}{0.422992in}%
\pgfsys@useobject{currentmarker}{}%
\end{pgfscope}%
\end{pgfscope}%
\begin{pgfscope}%
\pgfsetbuttcap%
\pgfsetroundjoin%
\definecolor{currentfill}{rgb}{0.000000,0.000000,0.000000}%
\pgfsetfillcolor{currentfill}%
\pgfsetlinewidth{0.501875pt}%
\definecolor{currentstroke}{rgb}{0.000000,0.000000,0.000000}%
\pgfsetstrokecolor{currentstroke}%
\pgfsetdash{}{0pt}%
\pgfsys@defobject{currentmarker}{\pgfqpoint{0.000000in}{-0.020833in}}{\pgfqpoint{0.000000in}{0.000000in}}{%
\pgfpathmoveto{\pgfqpoint{0.000000in}{0.000000in}}%
\pgfpathlineto{\pgfqpoint{0.000000in}{-0.020833in}}%
\pgfusepath{stroke,fill}%
}%
\begin{pgfscope}%
\pgfsys@transformshift{1.216028in}{2.049193in}%
\pgfsys@useobject{currentmarker}{}%
\end{pgfscope}%
\end{pgfscope}%
\begin{pgfscope}%
\pgfsetbuttcap%
\pgfsetroundjoin%
\definecolor{currentfill}{rgb}{0.000000,0.000000,0.000000}%
\pgfsetfillcolor{currentfill}%
\pgfsetlinewidth{0.501875pt}%
\definecolor{currentstroke}{rgb}{0.000000,0.000000,0.000000}%
\pgfsetstrokecolor{currentstroke}%
\pgfsetdash{}{0pt}%
\pgfsys@defobject{currentmarker}{\pgfqpoint{0.000000in}{0.000000in}}{\pgfqpoint{0.000000in}{0.020833in}}{%
\pgfpathmoveto{\pgfqpoint{0.000000in}{0.000000in}}%
\pgfpathlineto{\pgfqpoint{0.000000in}{0.020833in}}%
\pgfusepath{stroke,fill}%
}%
\begin{pgfscope}%
\pgfsys@transformshift{1.328705in}{0.422992in}%
\pgfsys@useobject{currentmarker}{}%
\end{pgfscope}%
\end{pgfscope}%
\begin{pgfscope}%
\pgfsetbuttcap%
\pgfsetroundjoin%
\definecolor{currentfill}{rgb}{0.000000,0.000000,0.000000}%
\pgfsetfillcolor{currentfill}%
\pgfsetlinewidth{0.501875pt}%
\definecolor{currentstroke}{rgb}{0.000000,0.000000,0.000000}%
\pgfsetstrokecolor{currentstroke}%
\pgfsetdash{}{0pt}%
\pgfsys@defobject{currentmarker}{\pgfqpoint{0.000000in}{-0.020833in}}{\pgfqpoint{0.000000in}{0.000000in}}{%
\pgfpathmoveto{\pgfqpoint{0.000000in}{0.000000in}}%
\pgfpathlineto{\pgfqpoint{0.000000in}{-0.020833in}}%
\pgfusepath{stroke,fill}%
}%
\begin{pgfscope}%
\pgfsys@transformshift{1.328705in}{2.049193in}%
\pgfsys@useobject{currentmarker}{}%
\end{pgfscope}%
\end{pgfscope}%
\begin{pgfscope}%
\pgfsetbuttcap%
\pgfsetroundjoin%
\definecolor{currentfill}{rgb}{0.000000,0.000000,0.000000}%
\pgfsetfillcolor{currentfill}%
\pgfsetlinewidth{0.501875pt}%
\definecolor{currentstroke}{rgb}{0.000000,0.000000,0.000000}%
\pgfsetstrokecolor{currentstroke}%
\pgfsetdash{}{0pt}%
\pgfsys@defobject{currentmarker}{\pgfqpoint{0.000000in}{0.000000in}}{\pgfqpoint{0.000000in}{0.020833in}}{%
\pgfpathmoveto{\pgfqpoint{0.000000in}{0.000000in}}%
\pgfpathlineto{\pgfqpoint{0.000000in}{0.020833in}}%
\pgfusepath{stroke,fill}%
}%
\begin{pgfscope}%
\pgfsys@transformshift{1.554058in}{0.422992in}%
\pgfsys@useobject{currentmarker}{}%
\end{pgfscope}%
\end{pgfscope}%
\begin{pgfscope}%
\pgfsetbuttcap%
\pgfsetroundjoin%
\definecolor{currentfill}{rgb}{0.000000,0.000000,0.000000}%
\pgfsetfillcolor{currentfill}%
\pgfsetlinewidth{0.501875pt}%
\definecolor{currentstroke}{rgb}{0.000000,0.000000,0.000000}%
\pgfsetstrokecolor{currentstroke}%
\pgfsetdash{}{0pt}%
\pgfsys@defobject{currentmarker}{\pgfqpoint{0.000000in}{-0.020833in}}{\pgfqpoint{0.000000in}{0.000000in}}{%
\pgfpathmoveto{\pgfqpoint{0.000000in}{0.000000in}}%
\pgfpathlineto{\pgfqpoint{0.000000in}{-0.020833in}}%
\pgfusepath{stroke,fill}%
}%
\begin{pgfscope}%
\pgfsys@transformshift{1.554058in}{2.049193in}%
\pgfsys@useobject{currentmarker}{}%
\end{pgfscope}%
\end{pgfscope}%
\begin{pgfscope}%
\pgfsetbuttcap%
\pgfsetroundjoin%
\definecolor{currentfill}{rgb}{0.000000,0.000000,0.000000}%
\pgfsetfillcolor{currentfill}%
\pgfsetlinewidth{0.501875pt}%
\definecolor{currentstroke}{rgb}{0.000000,0.000000,0.000000}%
\pgfsetstrokecolor{currentstroke}%
\pgfsetdash{}{0pt}%
\pgfsys@defobject{currentmarker}{\pgfqpoint{0.000000in}{0.000000in}}{\pgfqpoint{0.000000in}{0.020833in}}{%
\pgfpathmoveto{\pgfqpoint{0.000000in}{0.000000in}}%
\pgfpathlineto{\pgfqpoint{0.000000in}{0.020833in}}%
\pgfusepath{stroke,fill}%
}%
\begin{pgfscope}%
\pgfsys@transformshift{1.666734in}{0.422992in}%
\pgfsys@useobject{currentmarker}{}%
\end{pgfscope}%
\end{pgfscope}%
\begin{pgfscope}%
\pgfsetbuttcap%
\pgfsetroundjoin%
\definecolor{currentfill}{rgb}{0.000000,0.000000,0.000000}%
\pgfsetfillcolor{currentfill}%
\pgfsetlinewidth{0.501875pt}%
\definecolor{currentstroke}{rgb}{0.000000,0.000000,0.000000}%
\pgfsetstrokecolor{currentstroke}%
\pgfsetdash{}{0pt}%
\pgfsys@defobject{currentmarker}{\pgfqpoint{0.000000in}{-0.020833in}}{\pgfqpoint{0.000000in}{0.000000in}}{%
\pgfpathmoveto{\pgfqpoint{0.000000in}{0.000000in}}%
\pgfpathlineto{\pgfqpoint{0.000000in}{-0.020833in}}%
\pgfusepath{stroke,fill}%
}%
\begin{pgfscope}%
\pgfsys@transformshift{1.666734in}{2.049193in}%
\pgfsys@useobject{currentmarker}{}%
\end{pgfscope}%
\end{pgfscope}%
\begin{pgfscope}%
\pgfsetbuttcap%
\pgfsetroundjoin%
\definecolor{currentfill}{rgb}{0.000000,0.000000,0.000000}%
\pgfsetfillcolor{currentfill}%
\pgfsetlinewidth{0.501875pt}%
\definecolor{currentstroke}{rgb}{0.000000,0.000000,0.000000}%
\pgfsetstrokecolor{currentstroke}%
\pgfsetdash{}{0pt}%
\pgfsys@defobject{currentmarker}{\pgfqpoint{0.000000in}{0.000000in}}{\pgfqpoint{0.000000in}{0.020833in}}{%
\pgfpathmoveto{\pgfqpoint{0.000000in}{0.000000in}}%
\pgfpathlineto{\pgfqpoint{0.000000in}{0.020833in}}%
\pgfusepath{stroke,fill}%
}%
\begin{pgfscope}%
\pgfsys@transformshift{1.779411in}{0.422992in}%
\pgfsys@useobject{currentmarker}{}%
\end{pgfscope}%
\end{pgfscope}%
\begin{pgfscope}%
\pgfsetbuttcap%
\pgfsetroundjoin%
\definecolor{currentfill}{rgb}{0.000000,0.000000,0.000000}%
\pgfsetfillcolor{currentfill}%
\pgfsetlinewidth{0.501875pt}%
\definecolor{currentstroke}{rgb}{0.000000,0.000000,0.000000}%
\pgfsetstrokecolor{currentstroke}%
\pgfsetdash{}{0pt}%
\pgfsys@defobject{currentmarker}{\pgfqpoint{0.000000in}{-0.020833in}}{\pgfqpoint{0.000000in}{0.000000in}}{%
\pgfpathmoveto{\pgfqpoint{0.000000in}{0.000000in}}%
\pgfpathlineto{\pgfqpoint{0.000000in}{-0.020833in}}%
\pgfusepath{stroke,fill}%
}%
\begin{pgfscope}%
\pgfsys@transformshift{1.779411in}{2.049193in}%
\pgfsys@useobject{currentmarker}{}%
\end{pgfscope}%
\end{pgfscope}%
\begin{pgfscope}%
\pgfsetbuttcap%
\pgfsetroundjoin%
\definecolor{currentfill}{rgb}{0.000000,0.000000,0.000000}%
\pgfsetfillcolor{currentfill}%
\pgfsetlinewidth{0.501875pt}%
\definecolor{currentstroke}{rgb}{0.000000,0.000000,0.000000}%
\pgfsetstrokecolor{currentstroke}%
\pgfsetdash{}{0pt}%
\pgfsys@defobject{currentmarker}{\pgfqpoint{0.000000in}{0.000000in}}{\pgfqpoint{0.000000in}{0.020833in}}{%
\pgfpathmoveto{\pgfqpoint{0.000000in}{0.000000in}}%
\pgfpathlineto{\pgfqpoint{0.000000in}{0.020833in}}%
\pgfusepath{stroke,fill}%
}%
\begin{pgfscope}%
\pgfsys@transformshift{2.004764in}{0.422992in}%
\pgfsys@useobject{currentmarker}{}%
\end{pgfscope}%
\end{pgfscope}%
\begin{pgfscope}%
\pgfsetbuttcap%
\pgfsetroundjoin%
\definecolor{currentfill}{rgb}{0.000000,0.000000,0.000000}%
\pgfsetfillcolor{currentfill}%
\pgfsetlinewidth{0.501875pt}%
\definecolor{currentstroke}{rgb}{0.000000,0.000000,0.000000}%
\pgfsetstrokecolor{currentstroke}%
\pgfsetdash{}{0pt}%
\pgfsys@defobject{currentmarker}{\pgfqpoint{0.000000in}{-0.020833in}}{\pgfqpoint{0.000000in}{0.000000in}}{%
\pgfpathmoveto{\pgfqpoint{0.000000in}{0.000000in}}%
\pgfpathlineto{\pgfqpoint{0.000000in}{-0.020833in}}%
\pgfusepath{stroke,fill}%
}%
\begin{pgfscope}%
\pgfsys@transformshift{2.004764in}{2.049193in}%
\pgfsys@useobject{currentmarker}{}%
\end{pgfscope}%
\end{pgfscope}%
\begin{pgfscope}%
\pgfsetbuttcap%
\pgfsetroundjoin%
\definecolor{currentfill}{rgb}{0.000000,0.000000,0.000000}%
\pgfsetfillcolor{currentfill}%
\pgfsetlinewidth{0.501875pt}%
\definecolor{currentstroke}{rgb}{0.000000,0.000000,0.000000}%
\pgfsetstrokecolor{currentstroke}%
\pgfsetdash{}{0pt}%
\pgfsys@defobject{currentmarker}{\pgfqpoint{0.000000in}{0.000000in}}{\pgfqpoint{0.000000in}{0.020833in}}{%
\pgfpathmoveto{\pgfqpoint{0.000000in}{0.000000in}}%
\pgfpathlineto{\pgfqpoint{0.000000in}{0.020833in}}%
\pgfusepath{stroke,fill}%
}%
\begin{pgfscope}%
\pgfsys@transformshift{2.117440in}{0.422992in}%
\pgfsys@useobject{currentmarker}{}%
\end{pgfscope}%
\end{pgfscope}%
\begin{pgfscope}%
\pgfsetbuttcap%
\pgfsetroundjoin%
\definecolor{currentfill}{rgb}{0.000000,0.000000,0.000000}%
\pgfsetfillcolor{currentfill}%
\pgfsetlinewidth{0.501875pt}%
\definecolor{currentstroke}{rgb}{0.000000,0.000000,0.000000}%
\pgfsetstrokecolor{currentstroke}%
\pgfsetdash{}{0pt}%
\pgfsys@defobject{currentmarker}{\pgfqpoint{0.000000in}{-0.020833in}}{\pgfqpoint{0.000000in}{0.000000in}}{%
\pgfpathmoveto{\pgfqpoint{0.000000in}{0.000000in}}%
\pgfpathlineto{\pgfqpoint{0.000000in}{-0.020833in}}%
\pgfusepath{stroke,fill}%
}%
\begin{pgfscope}%
\pgfsys@transformshift{2.117440in}{2.049193in}%
\pgfsys@useobject{currentmarker}{}%
\end{pgfscope}%
\end{pgfscope}%
\begin{pgfscope}%
\pgfsetbuttcap%
\pgfsetroundjoin%
\definecolor{currentfill}{rgb}{0.000000,0.000000,0.000000}%
\pgfsetfillcolor{currentfill}%
\pgfsetlinewidth{0.501875pt}%
\definecolor{currentstroke}{rgb}{0.000000,0.000000,0.000000}%
\pgfsetstrokecolor{currentstroke}%
\pgfsetdash{}{0pt}%
\pgfsys@defobject{currentmarker}{\pgfqpoint{0.000000in}{0.000000in}}{\pgfqpoint{0.000000in}{0.020833in}}{%
\pgfpathmoveto{\pgfqpoint{0.000000in}{0.000000in}}%
\pgfpathlineto{\pgfqpoint{0.000000in}{0.020833in}}%
\pgfusepath{stroke,fill}%
}%
\begin{pgfscope}%
\pgfsys@transformshift{2.230116in}{0.422992in}%
\pgfsys@useobject{currentmarker}{}%
\end{pgfscope}%
\end{pgfscope}%
\begin{pgfscope}%
\pgfsetbuttcap%
\pgfsetroundjoin%
\definecolor{currentfill}{rgb}{0.000000,0.000000,0.000000}%
\pgfsetfillcolor{currentfill}%
\pgfsetlinewidth{0.501875pt}%
\definecolor{currentstroke}{rgb}{0.000000,0.000000,0.000000}%
\pgfsetstrokecolor{currentstroke}%
\pgfsetdash{}{0pt}%
\pgfsys@defobject{currentmarker}{\pgfqpoint{0.000000in}{-0.020833in}}{\pgfqpoint{0.000000in}{0.000000in}}{%
\pgfpathmoveto{\pgfqpoint{0.000000in}{0.000000in}}%
\pgfpathlineto{\pgfqpoint{0.000000in}{-0.020833in}}%
\pgfusepath{stroke,fill}%
}%
\begin{pgfscope}%
\pgfsys@transformshift{2.230116in}{2.049193in}%
\pgfsys@useobject{currentmarker}{}%
\end{pgfscope}%
\end{pgfscope}%
\begin{pgfscope}%
\pgfsetbuttcap%
\pgfsetroundjoin%
\definecolor{currentfill}{rgb}{0.000000,0.000000,0.000000}%
\pgfsetfillcolor{currentfill}%
\pgfsetlinewidth{0.501875pt}%
\definecolor{currentstroke}{rgb}{0.000000,0.000000,0.000000}%
\pgfsetstrokecolor{currentstroke}%
\pgfsetdash{}{0pt}%
\pgfsys@defobject{currentmarker}{\pgfqpoint{0.000000in}{0.000000in}}{\pgfqpoint{0.000000in}{0.020833in}}{%
\pgfpathmoveto{\pgfqpoint{0.000000in}{0.000000in}}%
\pgfpathlineto{\pgfqpoint{0.000000in}{0.020833in}}%
\pgfusepath{stroke,fill}%
}%
\begin{pgfscope}%
\pgfsys@transformshift{2.455469in}{0.422992in}%
\pgfsys@useobject{currentmarker}{}%
\end{pgfscope}%
\end{pgfscope}%
\begin{pgfscope}%
\pgfsetbuttcap%
\pgfsetroundjoin%
\definecolor{currentfill}{rgb}{0.000000,0.000000,0.000000}%
\pgfsetfillcolor{currentfill}%
\pgfsetlinewidth{0.501875pt}%
\definecolor{currentstroke}{rgb}{0.000000,0.000000,0.000000}%
\pgfsetstrokecolor{currentstroke}%
\pgfsetdash{}{0pt}%
\pgfsys@defobject{currentmarker}{\pgfqpoint{0.000000in}{-0.020833in}}{\pgfqpoint{0.000000in}{0.000000in}}{%
\pgfpathmoveto{\pgfqpoint{0.000000in}{0.000000in}}%
\pgfpathlineto{\pgfqpoint{0.000000in}{-0.020833in}}%
\pgfusepath{stroke,fill}%
}%
\begin{pgfscope}%
\pgfsys@transformshift{2.455469in}{2.049193in}%
\pgfsys@useobject{currentmarker}{}%
\end{pgfscope}%
\end{pgfscope}%
\begin{pgfscope}%
\pgfsetbuttcap%
\pgfsetroundjoin%
\definecolor{currentfill}{rgb}{0.000000,0.000000,0.000000}%
\pgfsetfillcolor{currentfill}%
\pgfsetlinewidth{0.501875pt}%
\definecolor{currentstroke}{rgb}{0.000000,0.000000,0.000000}%
\pgfsetstrokecolor{currentstroke}%
\pgfsetdash{}{0pt}%
\pgfsys@defobject{currentmarker}{\pgfqpoint{0.000000in}{0.000000in}}{\pgfqpoint{0.000000in}{0.020833in}}{%
\pgfpathmoveto{\pgfqpoint{0.000000in}{0.000000in}}%
\pgfpathlineto{\pgfqpoint{0.000000in}{0.020833in}}%
\pgfusepath{stroke,fill}%
}%
\begin{pgfscope}%
\pgfsys@transformshift{2.568146in}{0.422992in}%
\pgfsys@useobject{currentmarker}{}%
\end{pgfscope}%
\end{pgfscope}%
\begin{pgfscope}%
\pgfsetbuttcap%
\pgfsetroundjoin%
\definecolor{currentfill}{rgb}{0.000000,0.000000,0.000000}%
\pgfsetfillcolor{currentfill}%
\pgfsetlinewidth{0.501875pt}%
\definecolor{currentstroke}{rgb}{0.000000,0.000000,0.000000}%
\pgfsetstrokecolor{currentstroke}%
\pgfsetdash{}{0pt}%
\pgfsys@defobject{currentmarker}{\pgfqpoint{0.000000in}{-0.020833in}}{\pgfqpoint{0.000000in}{0.000000in}}{%
\pgfpathmoveto{\pgfqpoint{0.000000in}{0.000000in}}%
\pgfpathlineto{\pgfqpoint{0.000000in}{-0.020833in}}%
\pgfusepath{stroke,fill}%
}%
\begin{pgfscope}%
\pgfsys@transformshift{2.568146in}{2.049193in}%
\pgfsys@useobject{currentmarker}{}%
\end{pgfscope}%
\end{pgfscope}%
\begin{pgfscope}%
\pgfsetbuttcap%
\pgfsetroundjoin%
\definecolor{currentfill}{rgb}{0.000000,0.000000,0.000000}%
\pgfsetfillcolor{currentfill}%
\pgfsetlinewidth{0.501875pt}%
\definecolor{currentstroke}{rgb}{0.000000,0.000000,0.000000}%
\pgfsetstrokecolor{currentstroke}%
\pgfsetdash{}{0pt}%
\pgfsys@defobject{currentmarker}{\pgfqpoint{0.000000in}{0.000000in}}{\pgfqpoint{0.000000in}{0.020833in}}{%
\pgfpathmoveto{\pgfqpoint{0.000000in}{0.000000in}}%
\pgfpathlineto{\pgfqpoint{0.000000in}{0.020833in}}%
\pgfusepath{stroke,fill}%
}%
\begin{pgfscope}%
\pgfsys@transformshift{2.680822in}{0.422992in}%
\pgfsys@useobject{currentmarker}{}%
\end{pgfscope}%
\end{pgfscope}%
\begin{pgfscope}%
\pgfsetbuttcap%
\pgfsetroundjoin%
\definecolor{currentfill}{rgb}{0.000000,0.000000,0.000000}%
\pgfsetfillcolor{currentfill}%
\pgfsetlinewidth{0.501875pt}%
\definecolor{currentstroke}{rgb}{0.000000,0.000000,0.000000}%
\pgfsetstrokecolor{currentstroke}%
\pgfsetdash{}{0pt}%
\pgfsys@defobject{currentmarker}{\pgfqpoint{0.000000in}{-0.020833in}}{\pgfqpoint{0.000000in}{0.000000in}}{%
\pgfpathmoveto{\pgfqpoint{0.000000in}{0.000000in}}%
\pgfpathlineto{\pgfqpoint{0.000000in}{-0.020833in}}%
\pgfusepath{stroke,fill}%
}%
\begin{pgfscope}%
\pgfsys@transformshift{2.680822in}{2.049193in}%
\pgfsys@useobject{currentmarker}{}%
\end{pgfscope}%
\end{pgfscope}%
\begin{pgfscope}%
\pgfsetbuttcap%
\pgfsetroundjoin%
\definecolor{currentfill}{rgb}{0.000000,0.000000,0.000000}%
\pgfsetfillcolor{currentfill}%
\pgfsetlinewidth{0.501875pt}%
\definecolor{currentstroke}{rgb}{0.000000,0.000000,0.000000}%
\pgfsetstrokecolor{currentstroke}%
\pgfsetdash{}{0pt}%
\pgfsys@defobject{currentmarker}{\pgfqpoint{0.000000in}{0.000000in}}{\pgfqpoint{0.000000in}{0.020833in}}{%
\pgfpathmoveto{\pgfqpoint{0.000000in}{0.000000in}}%
\pgfpathlineto{\pgfqpoint{0.000000in}{0.020833in}}%
\pgfusepath{stroke,fill}%
}%
\begin{pgfscope}%
\pgfsys@transformshift{2.793499in}{0.422992in}%
\pgfsys@useobject{currentmarker}{}%
\end{pgfscope}%
\end{pgfscope}%
\begin{pgfscope}%
\pgfsetbuttcap%
\pgfsetroundjoin%
\definecolor{currentfill}{rgb}{0.000000,0.000000,0.000000}%
\pgfsetfillcolor{currentfill}%
\pgfsetlinewidth{0.501875pt}%
\definecolor{currentstroke}{rgb}{0.000000,0.000000,0.000000}%
\pgfsetstrokecolor{currentstroke}%
\pgfsetdash{}{0pt}%
\pgfsys@defobject{currentmarker}{\pgfqpoint{0.000000in}{-0.020833in}}{\pgfqpoint{0.000000in}{0.000000in}}{%
\pgfpathmoveto{\pgfqpoint{0.000000in}{0.000000in}}%
\pgfpathlineto{\pgfqpoint{0.000000in}{-0.020833in}}%
\pgfusepath{stroke,fill}%
}%
\begin{pgfscope}%
\pgfsys@transformshift{2.793499in}{2.049193in}%
\pgfsys@useobject{currentmarker}{}%
\end{pgfscope}%
\end{pgfscope}%
\begin{pgfscope}%
\definecolor{textcolor}{rgb}{0.000000,0.000000,0.000000}%
\pgfsetstrokecolor{textcolor}%
\pgfsetfillcolor{textcolor}%
\pgftext[x=1.678002in,y=0.184413in,,top]{\color{textcolor}\rmfamily\fontsize{10.000000}{12.000000}\selectfont \(\displaystyle K\)}%
\end{pgfscope}%
\begin{pgfscope}%
\pgfsetbuttcap%
\pgfsetroundjoin%
\definecolor{currentfill}{rgb}{0.000000,0.000000,0.000000}%
\pgfsetfillcolor{currentfill}%
\pgfsetlinewidth{0.501875pt}%
\definecolor{currentstroke}{rgb}{0.000000,0.000000,0.000000}%
\pgfsetstrokecolor{currentstroke}%
\pgfsetdash{}{0pt}%
\pgfsys@defobject{currentmarker}{\pgfqpoint{0.000000in}{0.000000in}}{\pgfqpoint{0.041667in}{0.000000in}}{%
\pgfpathmoveto{\pgfqpoint{0.000000in}{0.000000in}}%
\pgfpathlineto{\pgfqpoint{0.041667in}{0.000000in}}%
\pgfusepath{stroke,fill}%
}%
\begin{pgfscope}%
\pgfsys@transformshift{0.539970in}{0.838058in}%
\pgfsys@useobject{currentmarker}{}%
\end{pgfscope}%
\end{pgfscope}%
\begin{pgfscope}%
\pgfsetbuttcap%
\pgfsetroundjoin%
\definecolor{currentfill}{rgb}{0.000000,0.000000,0.000000}%
\pgfsetfillcolor{currentfill}%
\pgfsetlinewidth{0.501875pt}%
\definecolor{currentstroke}{rgb}{0.000000,0.000000,0.000000}%
\pgfsetstrokecolor{currentstroke}%
\pgfsetdash{}{0pt}%
\pgfsys@defobject{currentmarker}{\pgfqpoint{-0.041667in}{0.000000in}}{\pgfqpoint{-0.000000in}{0.000000in}}{%
\pgfpathmoveto{\pgfqpoint{-0.000000in}{0.000000in}}%
\pgfpathlineto{\pgfqpoint{-0.041667in}{0.000000in}}%
\pgfusepath{stroke,fill}%
}%
\begin{pgfscope}%
\pgfsys@transformshift{2.816034in}{0.838058in}%
\pgfsys@useobject{currentmarker}{}%
\end{pgfscope}%
\end{pgfscope}%
\begin{pgfscope}%
\definecolor{textcolor}{rgb}{0.000000,0.000000,0.000000}%
\pgfsetstrokecolor{textcolor}%
\pgfsetfillcolor{textcolor}%
\pgftext[x=0.313889in, y=0.785296in, left, base]{\color{textcolor}\rmfamily\fontsize{10.000000}{12.000000}\selectfont \(\displaystyle {0.8}\)}%
\end{pgfscope}%
\begin{pgfscope}%
\pgfsetbuttcap%
\pgfsetroundjoin%
\definecolor{currentfill}{rgb}{0.000000,0.000000,0.000000}%
\pgfsetfillcolor{currentfill}%
\pgfsetlinewidth{0.501875pt}%
\definecolor{currentstroke}{rgb}{0.000000,0.000000,0.000000}%
\pgfsetstrokecolor{currentstroke}%
\pgfsetdash{}{0pt}%
\pgfsys@defobject{currentmarker}{\pgfqpoint{0.000000in}{0.000000in}}{\pgfqpoint{0.041667in}{0.000000in}}{%
\pgfpathmoveto{\pgfqpoint{0.000000in}{0.000000in}}%
\pgfpathlineto{\pgfqpoint{0.041667in}{0.000000in}}%
\pgfusepath{stroke,fill}%
}%
\begin{pgfscope}%
\pgfsys@transformshift{0.539970in}{1.409607in}%
\pgfsys@useobject{currentmarker}{}%
\end{pgfscope}%
\end{pgfscope}%
\begin{pgfscope}%
\pgfsetbuttcap%
\pgfsetroundjoin%
\definecolor{currentfill}{rgb}{0.000000,0.000000,0.000000}%
\pgfsetfillcolor{currentfill}%
\pgfsetlinewidth{0.501875pt}%
\definecolor{currentstroke}{rgb}{0.000000,0.000000,0.000000}%
\pgfsetstrokecolor{currentstroke}%
\pgfsetdash{}{0pt}%
\pgfsys@defobject{currentmarker}{\pgfqpoint{-0.041667in}{0.000000in}}{\pgfqpoint{-0.000000in}{0.000000in}}{%
\pgfpathmoveto{\pgfqpoint{-0.000000in}{0.000000in}}%
\pgfpathlineto{\pgfqpoint{-0.041667in}{0.000000in}}%
\pgfusepath{stroke,fill}%
}%
\begin{pgfscope}%
\pgfsys@transformshift{2.816034in}{1.409607in}%
\pgfsys@useobject{currentmarker}{}%
\end{pgfscope}%
\end{pgfscope}%
\begin{pgfscope}%
\definecolor{textcolor}{rgb}{0.000000,0.000000,0.000000}%
\pgfsetstrokecolor{textcolor}%
\pgfsetfillcolor{textcolor}%
\pgftext[x=0.313889in, y=1.356846in, left, base]{\color{textcolor}\rmfamily\fontsize{10.000000}{12.000000}\selectfont \(\displaystyle {0.9}\)}%
\end{pgfscope}%
\begin{pgfscope}%
\pgfsetbuttcap%
\pgfsetroundjoin%
\definecolor{currentfill}{rgb}{0.000000,0.000000,0.000000}%
\pgfsetfillcolor{currentfill}%
\pgfsetlinewidth{0.501875pt}%
\definecolor{currentstroke}{rgb}{0.000000,0.000000,0.000000}%
\pgfsetstrokecolor{currentstroke}%
\pgfsetdash{}{0pt}%
\pgfsys@defobject{currentmarker}{\pgfqpoint{0.000000in}{0.000000in}}{\pgfqpoint{0.041667in}{0.000000in}}{%
\pgfpathmoveto{\pgfqpoint{0.000000in}{0.000000in}}%
\pgfpathlineto{\pgfqpoint{0.041667in}{0.000000in}}%
\pgfusepath{stroke,fill}%
}%
\begin{pgfscope}%
\pgfsys@transformshift{0.539970in}{1.981156in}%
\pgfsys@useobject{currentmarker}{}%
\end{pgfscope}%
\end{pgfscope}%
\begin{pgfscope}%
\pgfsetbuttcap%
\pgfsetroundjoin%
\definecolor{currentfill}{rgb}{0.000000,0.000000,0.000000}%
\pgfsetfillcolor{currentfill}%
\pgfsetlinewidth{0.501875pt}%
\definecolor{currentstroke}{rgb}{0.000000,0.000000,0.000000}%
\pgfsetstrokecolor{currentstroke}%
\pgfsetdash{}{0pt}%
\pgfsys@defobject{currentmarker}{\pgfqpoint{-0.041667in}{0.000000in}}{\pgfqpoint{-0.000000in}{0.000000in}}{%
\pgfpathmoveto{\pgfqpoint{-0.000000in}{0.000000in}}%
\pgfpathlineto{\pgfqpoint{-0.041667in}{0.000000in}}%
\pgfusepath{stroke,fill}%
}%
\begin{pgfscope}%
\pgfsys@transformshift{2.816034in}{1.981156in}%
\pgfsys@useobject{currentmarker}{}%
\end{pgfscope}%
\end{pgfscope}%
\begin{pgfscope}%
\definecolor{textcolor}{rgb}{0.000000,0.000000,0.000000}%
\pgfsetstrokecolor{textcolor}%
\pgfsetfillcolor{textcolor}%
\pgftext[x=0.313889in, y=1.928395in, left, base]{\color{textcolor}\rmfamily\fontsize{10.000000}{12.000000}\selectfont \(\displaystyle {1.0}\)}%
\end{pgfscope}%
\begin{pgfscope}%
\pgfsetbuttcap%
\pgfsetroundjoin%
\definecolor{currentfill}{rgb}{0.000000,0.000000,0.000000}%
\pgfsetfillcolor{currentfill}%
\pgfsetlinewidth{0.501875pt}%
\definecolor{currentstroke}{rgb}{0.000000,0.000000,0.000000}%
\pgfsetstrokecolor{currentstroke}%
\pgfsetdash{}{0pt}%
\pgfsys@defobject{currentmarker}{\pgfqpoint{0.000000in}{0.000000in}}{\pgfqpoint{0.020833in}{0.000000in}}{%
\pgfpathmoveto{\pgfqpoint{0.000000in}{0.000000in}}%
\pgfpathlineto{\pgfqpoint{0.020833in}{0.000000in}}%
\pgfusepath{stroke,fill}%
}%
\begin{pgfscope}%
\pgfsys@transformshift{0.539970in}{0.495129in}%
\pgfsys@useobject{currentmarker}{}%
\end{pgfscope}%
\end{pgfscope}%
\begin{pgfscope}%
\pgfsetbuttcap%
\pgfsetroundjoin%
\definecolor{currentfill}{rgb}{0.000000,0.000000,0.000000}%
\pgfsetfillcolor{currentfill}%
\pgfsetlinewidth{0.501875pt}%
\definecolor{currentstroke}{rgb}{0.000000,0.000000,0.000000}%
\pgfsetstrokecolor{currentstroke}%
\pgfsetdash{}{0pt}%
\pgfsys@defobject{currentmarker}{\pgfqpoint{-0.020833in}{0.000000in}}{\pgfqpoint{-0.000000in}{0.000000in}}{%
\pgfpathmoveto{\pgfqpoint{-0.000000in}{0.000000in}}%
\pgfpathlineto{\pgfqpoint{-0.020833in}{0.000000in}}%
\pgfusepath{stroke,fill}%
}%
\begin{pgfscope}%
\pgfsys@transformshift{2.816034in}{0.495129in}%
\pgfsys@useobject{currentmarker}{}%
\end{pgfscope}%
\end{pgfscope}%
\begin{pgfscope}%
\pgfsetbuttcap%
\pgfsetroundjoin%
\definecolor{currentfill}{rgb}{0.000000,0.000000,0.000000}%
\pgfsetfillcolor{currentfill}%
\pgfsetlinewidth{0.501875pt}%
\definecolor{currentstroke}{rgb}{0.000000,0.000000,0.000000}%
\pgfsetstrokecolor{currentstroke}%
\pgfsetdash{}{0pt}%
\pgfsys@defobject{currentmarker}{\pgfqpoint{0.000000in}{0.000000in}}{\pgfqpoint{0.020833in}{0.000000in}}{%
\pgfpathmoveto{\pgfqpoint{0.000000in}{0.000000in}}%
\pgfpathlineto{\pgfqpoint{0.020833in}{0.000000in}}%
\pgfusepath{stroke,fill}%
}%
\begin{pgfscope}%
\pgfsys@transformshift{0.539970in}{0.609438in}%
\pgfsys@useobject{currentmarker}{}%
\end{pgfscope}%
\end{pgfscope}%
\begin{pgfscope}%
\pgfsetbuttcap%
\pgfsetroundjoin%
\definecolor{currentfill}{rgb}{0.000000,0.000000,0.000000}%
\pgfsetfillcolor{currentfill}%
\pgfsetlinewidth{0.501875pt}%
\definecolor{currentstroke}{rgb}{0.000000,0.000000,0.000000}%
\pgfsetstrokecolor{currentstroke}%
\pgfsetdash{}{0pt}%
\pgfsys@defobject{currentmarker}{\pgfqpoint{-0.020833in}{0.000000in}}{\pgfqpoint{-0.000000in}{0.000000in}}{%
\pgfpathmoveto{\pgfqpoint{-0.000000in}{0.000000in}}%
\pgfpathlineto{\pgfqpoint{-0.020833in}{0.000000in}}%
\pgfusepath{stroke,fill}%
}%
\begin{pgfscope}%
\pgfsys@transformshift{2.816034in}{0.609438in}%
\pgfsys@useobject{currentmarker}{}%
\end{pgfscope}%
\end{pgfscope}%
\begin{pgfscope}%
\pgfsetbuttcap%
\pgfsetroundjoin%
\definecolor{currentfill}{rgb}{0.000000,0.000000,0.000000}%
\pgfsetfillcolor{currentfill}%
\pgfsetlinewidth{0.501875pt}%
\definecolor{currentstroke}{rgb}{0.000000,0.000000,0.000000}%
\pgfsetstrokecolor{currentstroke}%
\pgfsetdash{}{0pt}%
\pgfsys@defobject{currentmarker}{\pgfqpoint{0.000000in}{0.000000in}}{\pgfqpoint{0.020833in}{0.000000in}}{%
\pgfpathmoveto{\pgfqpoint{0.000000in}{0.000000in}}%
\pgfpathlineto{\pgfqpoint{0.020833in}{0.000000in}}%
\pgfusepath{stroke,fill}%
}%
\begin{pgfscope}%
\pgfsys@transformshift{0.539970in}{0.723748in}%
\pgfsys@useobject{currentmarker}{}%
\end{pgfscope}%
\end{pgfscope}%
\begin{pgfscope}%
\pgfsetbuttcap%
\pgfsetroundjoin%
\definecolor{currentfill}{rgb}{0.000000,0.000000,0.000000}%
\pgfsetfillcolor{currentfill}%
\pgfsetlinewidth{0.501875pt}%
\definecolor{currentstroke}{rgb}{0.000000,0.000000,0.000000}%
\pgfsetstrokecolor{currentstroke}%
\pgfsetdash{}{0pt}%
\pgfsys@defobject{currentmarker}{\pgfqpoint{-0.020833in}{0.000000in}}{\pgfqpoint{-0.000000in}{0.000000in}}{%
\pgfpathmoveto{\pgfqpoint{-0.000000in}{0.000000in}}%
\pgfpathlineto{\pgfqpoint{-0.020833in}{0.000000in}}%
\pgfusepath{stroke,fill}%
}%
\begin{pgfscope}%
\pgfsys@transformshift{2.816034in}{0.723748in}%
\pgfsys@useobject{currentmarker}{}%
\end{pgfscope}%
\end{pgfscope}%
\begin{pgfscope}%
\pgfsetbuttcap%
\pgfsetroundjoin%
\definecolor{currentfill}{rgb}{0.000000,0.000000,0.000000}%
\pgfsetfillcolor{currentfill}%
\pgfsetlinewidth{0.501875pt}%
\definecolor{currentstroke}{rgb}{0.000000,0.000000,0.000000}%
\pgfsetstrokecolor{currentstroke}%
\pgfsetdash{}{0pt}%
\pgfsys@defobject{currentmarker}{\pgfqpoint{0.000000in}{0.000000in}}{\pgfqpoint{0.020833in}{0.000000in}}{%
\pgfpathmoveto{\pgfqpoint{0.000000in}{0.000000in}}%
\pgfpathlineto{\pgfqpoint{0.020833in}{0.000000in}}%
\pgfusepath{stroke,fill}%
}%
\begin{pgfscope}%
\pgfsys@transformshift{0.539970in}{0.952368in}%
\pgfsys@useobject{currentmarker}{}%
\end{pgfscope}%
\end{pgfscope}%
\begin{pgfscope}%
\pgfsetbuttcap%
\pgfsetroundjoin%
\definecolor{currentfill}{rgb}{0.000000,0.000000,0.000000}%
\pgfsetfillcolor{currentfill}%
\pgfsetlinewidth{0.501875pt}%
\definecolor{currentstroke}{rgb}{0.000000,0.000000,0.000000}%
\pgfsetstrokecolor{currentstroke}%
\pgfsetdash{}{0pt}%
\pgfsys@defobject{currentmarker}{\pgfqpoint{-0.020833in}{0.000000in}}{\pgfqpoint{-0.000000in}{0.000000in}}{%
\pgfpathmoveto{\pgfqpoint{-0.000000in}{0.000000in}}%
\pgfpathlineto{\pgfqpoint{-0.020833in}{0.000000in}}%
\pgfusepath{stroke,fill}%
}%
\begin{pgfscope}%
\pgfsys@transformshift{2.816034in}{0.952368in}%
\pgfsys@useobject{currentmarker}{}%
\end{pgfscope}%
\end{pgfscope}%
\begin{pgfscope}%
\pgfsetbuttcap%
\pgfsetroundjoin%
\definecolor{currentfill}{rgb}{0.000000,0.000000,0.000000}%
\pgfsetfillcolor{currentfill}%
\pgfsetlinewidth{0.501875pt}%
\definecolor{currentstroke}{rgb}{0.000000,0.000000,0.000000}%
\pgfsetstrokecolor{currentstroke}%
\pgfsetdash{}{0pt}%
\pgfsys@defobject{currentmarker}{\pgfqpoint{0.000000in}{0.000000in}}{\pgfqpoint{0.020833in}{0.000000in}}{%
\pgfpathmoveto{\pgfqpoint{0.000000in}{0.000000in}}%
\pgfpathlineto{\pgfqpoint{0.020833in}{0.000000in}}%
\pgfusepath{stroke,fill}%
}%
\begin{pgfscope}%
\pgfsys@transformshift{0.539970in}{1.066678in}%
\pgfsys@useobject{currentmarker}{}%
\end{pgfscope}%
\end{pgfscope}%
\begin{pgfscope}%
\pgfsetbuttcap%
\pgfsetroundjoin%
\definecolor{currentfill}{rgb}{0.000000,0.000000,0.000000}%
\pgfsetfillcolor{currentfill}%
\pgfsetlinewidth{0.501875pt}%
\definecolor{currentstroke}{rgb}{0.000000,0.000000,0.000000}%
\pgfsetstrokecolor{currentstroke}%
\pgfsetdash{}{0pt}%
\pgfsys@defobject{currentmarker}{\pgfqpoint{-0.020833in}{0.000000in}}{\pgfqpoint{-0.000000in}{0.000000in}}{%
\pgfpathmoveto{\pgfqpoint{-0.000000in}{0.000000in}}%
\pgfpathlineto{\pgfqpoint{-0.020833in}{0.000000in}}%
\pgfusepath{stroke,fill}%
}%
\begin{pgfscope}%
\pgfsys@transformshift{2.816034in}{1.066678in}%
\pgfsys@useobject{currentmarker}{}%
\end{pgfscope}%
\end{pgfscope}%
\begin{pgfscope}%
\pgfsetbuttcap%
\pgfsetroundjoin%
\definecolor{currentfill}{rgb}{0.000000,0.000000,0.000000}%
\pgfsetfillcolor{currentfill}%
\pgfsetlinewidth{0.501875pt}%
\definecolor{currentstroke}{rgb}{0.000000,0.000000,0.000000}%
\pgfsetstrokecolor{currentstroke}%
\pgfsetdash{}{0pt}%
\pgfsys@defobject{currentmarker}{\pgfqpoint{0.000000in}{0.000000in}}{\pgfqpoint{0.020833in}{0.000000in}}{%
\pgfpathmoveto{\pgfqpoint{0.000000in}{0.000000in}}%
\pgfpathlineto{\pgfqpoint{0.020833in}{0.000000in}}%
\pgfusepath{stroke,fill}%
}%
\begin{pgfscope}%
\pgfsys@transformshift{0.539970in}{1.180987in}%
\pgfsys@useobject{currentmarker}{}%
\end{pgfscope}%
\end{pgfscope}%
\begin{pgfscope}%
\pgfsetbuttcap%
\pgfsetroundjoin%
\definecolor{currentfill}{rgb}{0.000000,0.000000,0.000000}%
\pgfsetfillcolor{currentfill}%
\pgfsetlinewidth{0.501875pt}%
\definecolor{currentstroke}{rgb}{0.000000,0.000000,0.000000}%
\pgfsetstrokecolor{currentstroke}%
\pgfsetdash{}{0pt}%
\pgfsys@defobject{currentmarker}{\pgfqpoint{-0.020833in}{0.000000in}}{\pgfqpoint{-0.000000in}{0.000000in}}{%
\pgfpathmoveto{\pgfqpoint{-0.000000in}{0.000000in}}%
\pgfpathlineto{\pgfqpoint{-0.020833in}{0.000000in}}%
\pgfusepath{stroke,fill}%
}%
\begin{pgfscope}%
\pgfsys@transformshift{2.816034in}{1.180987in}%
\pgfsys@useobject{currentmarker}{}%
\end{pgfscope}%
\end{pgfscope}%
\begin{pgfscope}%
\pgfsetbuttcap%
\pgfsetroundjoin%
\definecolor{currentfill}{rgb}{0.000000,0.000000,0.000000}%
\pgfsetfillcolor{currentfill}%
\pgfsetlinewidth{0.501875pt}%
\definecolor{currentstroke}{rgb}{0.000000,0.000000,0.000000}%
\pgfsetstrokecolor{currentstroke}%
\pgfsetdash{}{0pt}%
\pgfsys@defobject{currentmarker}{\pgfqpoint{0.000000in}{0.000000in}}{\pgfqpoint{0.020833in}{0.000000in}}{%
\pgfpathmoveto{\pgfqpoint{0.000000in}{0.000000in}}%
\pgfpathlineto{\pgfqpoint{0.020833in}{0.000000in}}%
\pgfusepath{stroke,fill}%
}%
\begin{pgfscope}%
\pgfsys@transformshift{0.539970in}{1.295297in}%
\pgfsys@useobject{currentmarker}{}%
\end{pgfscope}%
\end{pgfscope}%
\begin{pgfscope}%
\pgfsetbuttcap%
\pgfsetroundjoin%
\definecolor{currentfill}{rgb}{0.000000,0.000000,0.000000}%
\pgfsetfillcolor{currentfill}%
\pgfsetlinewidth{0.501875pt}%
\definecolor{currentstroke}{rgb}{0.000000,0.000000,0.000000}%
\pgfsetstrokecolor{currentstroke}%
\pgfsetdash{}{0pt}%
\pgfsys@defobject{currentmarker}{\pgfqpoint{-0.020833in}{0.000000in}}{\pgfqpoint{-0.000000in}{0.000000in}}{%
\pgfpathmoveto{\pgfqpoint{-0.000000in}{0.000000in}}%
\pgfpathlineto{\pgfqpoint{-0.020833in}{0.000000in}}%
\pgfusepath{stroke,fill}%
}%
\begin{pgfscope}%
\pgfsys@transformshift{2.816034in}{1.295297in}%
\pgfsys@useobject{currentmarker}{}%
\end{pgfscope}%
\end{pgfscope}%
\begin{pgfscope}%
\pgfsetbuttcap%
\pgfsetroundjoin%
\definecolor{currentfill}{rgb}{0.000000,0.000000,0.000000}%
\pgfsetfillcolor{currentfill}%
\pgfsetlinewidth{0.501875pt}%
\definecolor{currentstroke}{rgb}{0.000000,0.000000,0.000000}%
\pgfsetstrokecolor{currentstroke}%
\pgfsetdash{}{0pt}%
\pgfsys@defobject{currentmarker}{\pgfqpoint{0.000000in}{0.000000in}}{\pgfqpoint{0.020833in}{0.000000in}}{%
\pgfpathmoveto{\pgfqpoint{0.000000in}{0.000000in}}%
\pgfpathlineto{\pgfqpoint{0.020833in}{0.000000in}}%
\pgfusepath{stroke,fill}%
}%
\begin{pgfscope}%
\pgfsys@transformshift{0.539970in}{1.523917in}%
\pgfsys@useobject{currentmarker}{}%
\end{pgfscope}%
\end{pgfscope}%
\begin{pgfscope}%
\pgfsetbuttcap%
\pgfsetroundjoin%
\definecolor{currentfill}{rgb}{0.000000,0.000000,0.000000}%
\pgfsetfillcolor{currentfill}%
\pgfsetlinewidth{0.501875pt}%
\definecolor{currentstroke}{rgb}{0.000000,0.000000,0.000000}%
\pgfsetstrokecolor{currentstroke}%
\pgfsetdash{}{0pt}%
\pgfsys@defobject{currentmarker}{\pgfqpoint{-0.020833in}{0.000000in}}{\pgfqpoint{-0.000000in}{0.000000in}}{%
\pgfpathmoveto{\pgfqpoint{-0.000000in}{0.000000in}}%
\pgfpathlineto{\pgfqpoint{-0.020833in}{0.000000in}}%
\pgfusepath{stroke,fill}%
}%
\begin{pgfscope}%
\pgfsys@transformshift{2.816034in}{1.523917in}%
\pgfsys@useobject{currentmarker}{}%
\end{pgfscope}%
\end{pgfscope}%
\begin{pgfscope}%
\pgfsetbuttcap%
\pgfsetroundjoin%
\definecolor{currentfill}{rgb}{0.000000,0.000000,0.000000}%
\pgfsetfillcolor{currentfill}%
\pgfsetlinewidth{0.501875pt}%
\definecolor{currentstroke}{rgb}{0.000000,0.000000,0.000000}%
\pgfsetstrokecolor{currentstroke}%
\pgfsetdash{}{0pt}%
\pgfsys@defobject{currentmarker}{\pgfqpoint{0.000000in}{0.000000in}}{\pgfqpoint{0.020833in}{0.000000in}}{%
\pgfpathmoveto{\pgfqpoint{0.000000in}{0.000000in}}%
\pgfpathlineto{\pgfqpoint{0.020833in}{0.000000in}}%
\pgfusepath{stroke,fill}%
}%
\begin{pgfscope}%
\pgfsys@transformshift{0.539970in}{1.638227in}%
\pgfsys@useobject{currentmarker}{}%
\end{pgfscope}%
\end{pgfscope}%
\begin{pgfscope}%
\pgfsetbuttcap%
\pgfsetroundjoin%
\definecolor{currentfill}{rgb}{0.000000,0.000000,0.000000}%
\pgfsetfillcolor{currentfill}%
\pgfsetlinewidth{0.501875pt}%
\definecolor{currentstroke}{rgb}{0.000000,0.000000,0.000000}%
\pgfsetstrokecolor{currentstroke}%
\pgfsetdash{}{0pt}%
\pgfsys@defobject{currentmarker}{\pgfqpoint{-0.020833in}{0.000000in}}{\pgfqpoint{-0.000000in}{0.000000in}}{%
\pgfpathmoveto{\pgfqpoint{-0.000000in}{0.000000in}}%
\pgfpathlineto{\pgfqpoint{-0.020833in}{0.000000in}}%
\pgfusepath{stroke,fill}%
}%
\begin{pgfscope}%
\pgfsys@transformshift{2.816034in}{1.638227in}%
\pgfsys@useobject{currentmarker}{}%
\end{pgfscope}%
\end{pgfscope}%
\begin{pgfscope}%
\pgfsetbuttcap%
\pgfsetroundjoin%
\definecolor{currentfill}{rgb}{0.000000,0.000000,0.000000}%
\pgfsetfillcolor{currentfill}%
\pgfsetlinewidth{0.501875pt}%
\definecolor{currentstroke}{rgb}{0.000000,0.000000,0.000000}%
\pgfsetstrokecolor{currentstroke}%
\pgfsetdash{}{0pt}%
\pgfsys@defobject{currentmarker}{\pgfqpoint{0.000000in}{0.000000in}}{\pgfqpoint{0.020833in}{0.000000in}}{%
\pgfpathmoveto{\pgfqpoint{0.000000in}{0.000000in}}%
\pgfpathlineto{\pgfqpoint{0.020833in}{0.000000in}}%
\pgfusepath{stroke,fill}%
}%
\begin{pgfscope}%
\pgfsys@transformshift{0.539970in}{1.752537in}%
\pgfsys@useobject{currentmarker}{}%
\end{pgfscope}%
\end{pgfscope}%
\begin{pgfscope}%
\pgfsetbuttcap%
\pgfsetroundjoin%
\definecolor{currentfill}{rgb}{0.000000,0.000000,0.000000}%
\pgfsetfillcolor{currentfill}%
\pgfsetlinewidth{0.501875pt}%
\definecolor{currentstroke}{rgb}{0.000000,0.000000,0.000000}%
\pgfsetstrokecolor{currentstroke}%
\pgfsetdash{}{0pt}%
\pgfsys@defobject{currentmarker}{\pgfqpoint{-0.020833in}{0.000000in}}{\pgfqpoint{-0.000000in}{0.000000in}}{%
\pgfpathmoveto{\pgfqpoint{-0.000000in}{0.000000in}}%
\pgfpathlineto{\pgfqpoint{-0.020833in}{0.000000in}}%
\pgfusepath{stroke,fill}%
}%
\begin{pgfscope}%
\pgfsys@transformshift{2.816034in}{1.752537in}%
\pgfsys@useobject{currentmarker}{}%
\end{pgfscope}%
\end{pgfscope}%
\begin{pgfscope}%
\pgfsetbuttcap%
\pgfsetroundjoin%
\definecolor{currentfill}{rgb}{0.000000,0.000000,0.000000}%
\pgfsetfillcolor{currentfill}%
\pgfsetlinewidth{0.501875pt}%
\definecolor{currentstroke}{rgb}{0.000000,0.000000,0.000000}%
\pgfsetstrokecolor{currentstroke}%
\pgfsetdash{}{0pt}%
\pgfsys@defobject{currentmarker}{\pgfqpoint{0.000000in}{0.000000in}}{\pgfqpoint{0.020833in}{0.000000in}}{%
\pgfpathmoveto{\pgfqpoint{0.000000in}{0.000000in}}%
\pgfpathlineto{\pgfqpoint{0.020833in}{0.000000in}}%
\pgfusepath{stroke,fill}%
}%
\begin{pgfscope}%
\pgfsys@transformshift{0.539970in}{1.866846in}%
\pgfsys@useobject{currentmarker}{}%
\end{pgfscope}%
\end{pgfscope}%
\begin{pgfscope}%
\pgfsetbuttcap%
\pgfsetroundjoin%
\definecolor{currentfill}{rgb}{0.000000,0.000000,0.000000}%
\pgfsetfillcolor{currentfill}%
\pgfsetlinewidth{0.501875pt}%
\definecolor{currentstroke}{rgb}{0.000000,0.000000,0.000000}%
\pgfsetstrokecolor{currentstroke}%
\pgfsetdash{}{0pt}%
\pgfsys@defobject{currentmarker}{\pgfqpoint{-0.020833in}{0.000000in}}{\pgfqpoint{-0.000000in}{0.000000in}}{%
\pgfpathmoveto{\pgfqpoint{-0.000000in}{0.000000in}}%
\pgfpathlineto{\pgfqpoint{-0.020833in}{0.000000in}}%
\pgfusepath{stroke,fill}%
}%
\begin{pgfscope}%
\pgfsys@transformshift{2.816034in}{1.866846in}%
\pgfsys@useobject{currentmarker}{}%
\end{pgfscope}%
\end{pgfscope}%
\begin{pgfscope}%
\definecolor{textcolor}{rgb}{0.000000,0.000000,0.000000}%
\pgfsetstrokecolor{textcolor}%
\pgfsetfillcolor{textcolor}%
\pgftext[x=0.258333in,y=1.236093in,,bottom,rotate=90.000000]{\color{textcolor}\rmfamily\fontsize{10.000000}{12.000000}\selectfont \(\displaystyle C(K)\)}%
\end{pgfscope}%
\begin{pgfscope}%
\pgfpathrectangle{\pgfqpoint{0.539970in}{0.422992in}}{\pgfqpoint{2.276064in}{1.626201in}}%
\pgfusepath{clip}%
\pgfsetrectcap%
\pgfsetroundjoin%
\pgfsetlinewidth{1.003750pt}%
\definecolor{currentstroke}{rgb}{0.047059,0.364706,0.647059}%
\pgfsetstrokecolor{currentstroke}%
\pgfsetdash{}{0pt}%
\pgfpathmoveto{\pgfqpoint{0.562505in}{1.975275in}}%
\pgfpathlineto{\pgfqpoint{0.585040in}{1.974353in}}%
\pgfpathlineto{\pgfqpoint{0.607576in}{1.973541in}}%
\pgfpathlineto{\pgfqpoint{0.630111in}{1.972813in}}%
\pgfpathlineto{\pgfqpoint{0.652646in}{1.972065in}}%
\pgfpathlineto{\pgfqpoint{0.675182in}{1.971465in}}%
\pgfpathlineto{\pgfqpoint{0.697717in}{1.970871in}}%
\pgfpathlineto{\pgfqpoint{0.720252in}{1.970419in}}%
\pgfpathlineto{\pgfqpoint{0.742787in}{1.970051in}}%
\pgfpathlineto{\pgfqpoint{0.765323in}{1.969604in}}%
\pgfpathlineto{\pgfqpoint{0.787858in}{1.969161in}}%
\pgfpathlineto{\pgfqpoint{0.810393in}{1.968760in}}%
\pgfpathlineto{\pgfqpoint{0.832929in}{1.968391in}}%
\pgfpathlineto{\pgfqpoint{0.855464in}{1.968087in}}%
\pgfpathlineto{\pgfqpoint{0.877999in}{1.967716in}}%
\pgfpathlineto{\pgfqpoint{0.900534in}{1.967434in}}%
\pgfpathlineto{\pgfqpoint{0.923070in}{1.967076in}}%
\pgfpathlineto{\pgfqpoint{0.945605in}{1.966818in}}%
\pgfpathlineto{\pgfqpoint{0.968140in}{1.966461in}}%
\pgfpathlineto{\pgfqpoint{0.990676in}{1.966222in}}%
\pgfpathlineto{\pgfqpoint{1.013211in}{1.965933in}}%
\pgfpathlineto{\pgfqpoint{1.035746in}{1.965665in}}%
\pgfpathlineto{\pgfqpoint{1.058281in}{1.965358in}}%
\pgfpathlineto{\pgfqpoint{1.080817in}{1.965085in}}%
\pgfpathlineto{\pgfqpoint{1.103352in}{1.964820in}}%
\pgfpathlineto{\pgfqpoint{1.125887in}{1.964533in}}%
\pgfpathlineto{\pgfqpoint{1.148423in}{1.964379in}}%
\pgfpathlineto{\pgfqpoint{1.170958in}{1.964167in}}%
\pgfpathlineto{\pgfqpoint{1.193493in}{1.963963in}}%
\pgfpathlineto{\pgfqpoint{1.216028in}{1.963661in}}%
\pgfpathlineto{\pgfqpoint{1.238564in}{1.963405in}}%
\pgfpathlineto{\pgfqpoint{1.261099in}{1.963261in}}%
\pgfpathlineto{\pgfqpoint{1.283634in}{1.963122in}}%
\pgfpathlineto{\pgfqpoint{1.306170in}{1.962892in}}%
\pgfpathlineto{\pgfqpoint{1.328705in}{1.962717in}}%
\pgfpathlineto{\pgfqpoint{1.351240in}{1.962576in}}%
\pgfpathlineto{\pgfqpoint{1.373776in}{1.962397in}}%
\pgfpathlineto{\pgfqpoint{1.396311in}{1.962219in}}%
\pgfpathlineto{\pgfqpoint{1.418846in}{1.962032in}}%
\pgfpathlineto{\pgfqpoint{1.441381in}{1.961835in}}%
\pgfpathlineto{\pgfqpoint{1.463917in}{1.961621in}}%
\pgfpathlineto{\pgfqpoint{1.486452in}{1.961452in}}%
\pgfpathlineto{\pgfqpoint{1.508987in}{1.961264in}}%
\pgfpathlineto{\pgfqpoint{1.531523in}{1.961143in}}%
\pgfpathlineto{\pgfqpoint{1.554058in}{1.960962in}}%
\pgfpathlineto{\pgfqpoint{1.576593in}{1.960801in}}%
\pgfpathlineto{\pgfqpoint{1.599128in}{1.960637in}}%
\pgfpathlineto{\pgfqpoint{1.621664in}{1.960478in}}%
\pgfpathlineto{\pgfqpoint{1.644199in}{1.960334in}}%
\pgfpathlineto{\pgfqpoint{1.666734in}{1.960184in}}%
\pgfpathlineto{\pgfqpoint{1.689270in}{1.960071in}}%
\pgfpathlineto{\pgfqpoint{1.711805in}{1.959958in}}%
\pgfpathlineto{\pgfqpoint{1.734340in}{1.959754in}}%
\pgfpathlineto{\pgfqpoint{1.756875in}{1.959649in}}%
\pgfpathlineto{\pgfqpoint{1.779411in}{1.959519in}}%
\pgfpathlineto{\pgfqpoint{1.801946in}{1.959404in}}%
\pgfpathlineto{\pgfqpoint{1.824481in}{1.959233in}}%
\pgfpathlineto{\pgfqpoint{1.847017in}{1.959121in}}%
\pgfpathlineto{\pgfqpoint{1.869552in}{1.958993in}}%
\pgfpathlineto{\pgfqpoint{1.892087in}{1.958874in}}%
\pgfpathlineto{\pgfqpoint{1.914622in}{1.958746in}}%
\pgfpathlineto{\pgfqpoint{1.937158in}{1.958615in}}%
\pgfpathlineto{\pgfqpoint{1.959693in}{1.958479in}}%
\pgfpathlineto{\pgfqpoint{1.982228in}{1.958315in}}%
\pgfpathlineto{\pgfqpoint{2.004764in}{1.958211in}}%
\pgfpathlineto{\pgfqpoint{2.027299in}{1.958092in}}%
\pgfpathlineto{\pgfqpoint{2.049834in}{1.957964in}}%
\pgfpathlineto{\pgfqpoint{2.072369in}{1.957890in}}%
\pgfpathlineto{\pgfqpoint{2.094905in}{1.957780in}}%
\pgfpathlineto{\pgfqpoint{2.117440in}{1.957717in}}%
\pgfpathlineto{\pgfqpoint{2.139975in}{1.957625in}}%
\pgfpathlineto{\pgfqpoint{2.162511in}{1.957509in}}%
\pgfpathlineto{\pgfqpoint{2.185046in}{1.957418in}}%
\pgfpathlineto{\pgfqpoint{2.207581in}{1.957303in}}%
\pgfpathlineto{\pgfqpoint{2.230116in}{1.957254in}}%
\pgfpathlineto{\pgfqpoint{2.252652in}{1.957148in}}%
\pgfpathlineto{\pgfqpoint{2.275187in}{1.957074in}}%
\pgfpathlineto{\pgfqpoint{2.297722in}{1.956964in}}%
\pgfpathlineto{\pgfqpoint{2.320258in}{1.956856in}}%
\pgfpathlineto{\pgfqpoint{2.342793in}{1.956781in}}%
\pgfpathlineto{\pgfqpoint{2.365328in}{1.956665in}}%
\pgfpathlineto{\pgfqpoint{2.387863in}{1.956657in}}%
\pgfpathlineto{\pgfqpoint{2.410399in}{1.956567in}}%
\pgfpathlineto{\pgfqpoint{2.432934in}{1.956511in}}%
\pgfpathlineto{\pgfqpoint{2.455469in}{1.956375in}}%
\pgfpathlineto{\pgfqpoint{2.478005in}{1.956299in}}%
\pgfpathlineto{\pgfqpoint{2.500540in}{1.956222in}}%
\pgfpathlineto{\pgfqpoint{2.523075in}{1.956129in}}%
\pgfpathlineto{\pgfqpoint{2.545610in}{1.956023in}}%
\pgfpathlineto{\pgfqpoint{2.568146in}{1.955960in}}%
\pgfpathlineto{\pgfqpoint{2.590681in}{1.955881in}}%
\pgfpathlineto{\pgfqpoint{2.613216in}{1.955784in}}%
\pgfpathlineto{\pgfqpoint{2.635752in}{1.955656in}}%
\pgfpathlineto{\pgfqpoint{2.658287in}{1.955551in}}%
\pgfpathlineto{\pgfqpoint{2.680822in}{1.955486in}}%
\pgfpathlineto{\pgfqpoint{2.703357in}{1.955361in}}%
\pgfpathlineto{\pgfqpoint{2.725893in}{1.955270in}}%
\pgfpathlineto{\pgfqpoint{2.748428in}{1.955237in}}%
\pgfpathlineto{\pgfqpoint{2.770963in}{1.955204in}}%
\pgfusepath{stroke}%
\end{pgfscope}%
\begin{pgfscope}%
\pgfpathrectangle{\pgfqpoint{0.539970in}{0.422992in}}{\pgfqpoint{2.276064in}{1.626201in}}%
\pgfusepath{clip}%
\pgfsetrectcap%
\pgfsetroundjoin%
\pgfsetlinewidth{1.003750pt}%
\definecolor{currentstroke}{rgb}{0.000000,0.725490,0.270588}%
\pgfsetstrokecolor{currentstroke}%
\pgfsetdash{}{0pt}%
\pgfpathmoveto{\pgfqpoint{0.562505in}{1.887470in}}%
\pgfpathlineto{\pgfqpoint{0.585040in}{1.869810in}}%
\pgfpathlineto{\pgfqpoint{0.607576in}{1.853672in}}%
\pgfpathlineto{\pgfqpoint{0.630111in}{1.842293in}}%
\pgfpathlineto{\pgfqpoint{0.652646in}{1.831782in}}%
\pgfpathlineto{\pgfqpoint{0.675182in}{1.821702in}}%
\pgfpathlineto{\pgfqpoint{0.697717in}{1.813817in}}%
\pgfpathlineto{\pgfqpoint{0.720252in}{1.805953in}}%
\pgfpathlineto{\pgfqpoint{0.742787in}{1.800697in}}%
\pgfpathlineto{\pgfqpoint{0.765323in}{1.794049in}}%
\pgfpathlineto{\pgfqpoint{0.787858in}{1.787926in}}%
\pgfpathlineto{\pgfqpoint{0.810393in}{1.781680in}}%
\pgfpathlineto{\pgfqpoint{0.832929in}{1.776702in}}%
\pgfpathlineto{\pgfqpoint{0.855464in}{1.771508in}}%
\pgfpathlineto{\pgfqpoint{0.877999in}{1.766025in}}%
\pgfpathlineto{\pgfqpoint{0.900534in}{1.761616in}}%
\pgfpathlineto{\pgfqpoint{0.923070in}{1.756703in}}%
\pgfpathlineto{\pgfqpoint{0.945605in}{1.752124in}}%
\pgfpathlineto{\pgfqpoint{0.968140in}{1.747058in}}%
\pgfpathlineto{\pgfqpoint{0.990676in}{1.742866in}}%
\pgfpathlineto{\pgfqpoint{1.013211in}{1.738284in}}%
\pgfpathlineto{\pgfqpoint{1.035746in}{1.734790in}}%
\pgfpathlineto{\pgfqpoint{1.058281in}{1.730514in}}%
\pgfpathlineto{\pgfqpoint{1.080817in}{1.727032in}}%
\pgfpathlineto{\pgfqpoint{1.103352in}{1.723096in}}%
\pgfpathlineto{\pgfqpoint{1.125887in}{1.719214in}}%
\pgfpathlineto{\pgfqpoint{1.148423in}{1.715521in}}%
\pgfpathlineto{\pgfqpoint{1.170958in}{1.711766in}}%
\pgfpathlineto{\pgfqpoint{1.193493in}{1.708032in}}%
\pgfpathlineto{\pgfqpoint{1.216028in}{1.704041in}}%
\pgfpathlineto{\pgfqpoint{1.238564in}{1.700468in}}%
\pgfpathlineto{\pgfqpoint{1.261099in}{1.697415in}}%
\pgfpathlineto{\pgfqpoint{1.283634in}{1.694667in}}%
\pgfpathlineto{\pgfqpoint{1.306170in}{1.691601in}}%
\pgfpathlineto{\pgfqpoint{1.328705in}{1.688772in}}%
\pgfpathlineto{\pgfqpoint{1.351240in}{1.685830in}}%
\pgfpathlineto{\pgfqpoint{1.373776in}{1.683204in}}%
\pgfpathlineto{\pgfqpoint{1.396311in}{1.680301in}}%
\pgfpathlineto{\pgfqpoint{1.418846in}{1.677983in}}%
\pgfpathlineto{\pgfqpoint{1.441381in}{1.675320in}}%
\pgfpathlineto{\pgfqpoint{1.463917in}{1.672525in}}%
\pgfpathlineto{\pgfqpoint{1.486452in}{1.670124in}}%
\pgfpathlineto{\pgfqpoint{1.508987in}{1.667588in}}%
\pgfpathlineto{\pgfqpoint{1.531523in}{1.665292in}}%
\pgfpathlineto{\pgfqpoint{1.554058in}{1.662451in}}%
\pgfpathlineto{\pgfqpoint{1.576593in}{1.660322in}}%
\pgfpathlineto{\pgfqpoint{1.599128in}{1.657968in}}%
\pgfpathlineto{\pgfqpoint{1.621664in}{1.655543in}}%
\pgfpathlineto{\pgfqpoint{1.644199in}{1.653326in}}%
\pgfpathlineto{\pgfqpoint{1.666734in}{1.650884in}}%
\pgfpathlineto{\pgfqpoint{1.689270in}{1.648939in}}%
\pgfpathlineto{\pgfqpoint{1.711805in}{1.646730in}}%
\pgfpathlineto{\pgfqpoint{1.734340in}{1.644182in}}%
\pgfpathlineto{\pgfqpoint{1.756875in}{1.642077in}}%
\pgfpathlineto{\pgfqpoint{1.779411in}{1.639792in}}%
\pgfpathlineto{\pgfqpoint{1.801946in}{1.638055in}}%
\pgfpathlineto{\pgfqpoint{1.824481in}{1.635943in}}%
\pgfpathlineto{\pgfqpoint{1.847017in}{1.634065in}}%
\pgfpathlineto{\pgfqpoint{1.869552in}{1.631683in}}%
\pgfpathlineto{\pgfqpoint{1.892087in}{1.629700in}}%
\pgfpathlineto{\pgfqpoint{1.914622in}{1.627464in}}%
\pgfpathlineto{\pgfqpoint{1.937158in}{1.625873in}}%
\pgfpathlineto{\pgfqpoint{1.959693in}{1.623972in}}%
\pgfpathlineto{\pgfqpoint{1.982228in}{1.621850in}}%
\pgfpathlineto{\pgfqpoint{2.004764in}{1.620043in}}%
\pgfpathlineto{\pgfqpoint{2.027299in}{1.617899in}}%
\pgfpathlineto{\pgfqpoint{2.049834in}{1.615960in}}%
\pgfpathlineto{\pgfqpoint{2.072369in}{1.614426in}}%
\pgfpathlineto{\pgfqpoint{2.094905in}{1.612761in}}%
\pgfpathlineto{\pgfqpoint{2.117440in}{1.611399in}}%
\pgfpathlineto{\pgfqpoint{2.139975in}{1.609681in}}%
\pgfpathlineto{\pgfqpoint{2.162511in}{1.607991in}}%
\pgfpathlineto{\pgfqpoint{2.185046in}{1.606427in}}%
\pgfpathlineto{\pgfqpoint{2.207581in}{1.604717in}}%
\pgfpathlineto{\pgfqpoint{2.230116in}{1.603452in}}%
\pgfpathlineto{\pgfqpoint{2.252652in}{1.601747in}}%
\pgfpathlineto{\pgfqpoint{2.275187in}{1.600457in}}%
\pgfpathlineto{\pgfqpoint{2.297722in}{1.598789in}}%
\pgfpathlineto{\pgfqpoint{2.320258in}{1.596969in}}%
\pgfpathlineto{\pgfqpoint{2.342793in}{1.595498in}}%
\pgfpathlineto{\pgfqpoint{2.365328in}{1.593788in}}%
\pgfpathlineto{\pgfqpoint{2.387863in}{1.592380in}}%
\pgfpathlineto{\pgfqpoint{2.410399in}{1.590885in}}%
\pgfpathlineto{\pgfqpoint{2.432934in}{1.589321in}}%
\pgfpathlineto{\pgfqpoint{2.455469in}{1.587628in}}%
\pgfpathlineto{\pgfqpoint{2.478005in}{1.586181in}}%
\pgfpathlineto{\pgfqpoint{2.500540in}{1.584913in}}%
\pgfpathlineto{\pgfqpoint{2.523075in}{1.583308in}}%
\pgfpathlineto{\pgfqpoint{2.545610in}{1.581749in}}%
\pgfpathlineto{\pgfqpoint{2.568146in}{1.580351in}}%
\pgfpathlineto{\pgfqpoint{2.590681in}{1.578625in}}%
\pgfpathlineto{\pgfqpoint{2.613216in}{1.577125in}}%
\pgfpathlineto{\pgfqpoint{2.635752in}{1.575425in}}%
\pgfpathlineto{\pgfqpoint{2.658287in}{1.574054in}}%
\pgfpathlineto{\pgfqpoint{2.680822in}{1.572767in}}%
\pgfpathlineto{\pgfqpoint{2.703357in}{1.571020in}}%
\pgfpathlineto{\pgfqpoint{2.725893in}{1.569753in}}%
\pgfpathlineto{\pgfqpoint{2.748428in}{1.568563in}}%
\pgfpathlineto{\pgfqpoint{2.770963in}{1.567465in}}%
\pgfusepath{stroke}%
\end{pgfscope}%
\begin{pgfscope}%
\pgfpathrectangle{\pgfqpoint{0.539970in}{0.422992in}}{\pgfqpoint{2.276064in}{1.626201in}}%
\pgfusepath{clip}%
\pgfsetrectcap%
\pgfsetroundjoin%
\pgfsetlinewidth{1.003750pt}%
\definecolor{currentstroke}{rgb}{1.000000,0.584314,0.000000}%
\pgfsetstrokecolor{currentstroke}%
\pgfsetdash{}{0pt}%
\pgfpathmoveto{\pgfqpoint{0.562505in}{1.966135in}}%
\pgfpathlineto{\pgfqpoint{0.585040in}{1.963106in}}%
\pgfpathlineto{\pgfqpoint{0.607576in}{1.960302in}}%
\pgfpathlineto{\pgfqpoint{0.630111in}{1.956601in}}%
\pgfpathlineto{\pgfqpoint{0.652646in}{1.954509in}}%
\pgfpathlineto{\pgfqpoint{0.675182in}{1.951948in}}%
\pgfpathlineto{\pgfqpoint{0.697717in}{1.949518in}}%
\pgfpathlineto{\pgfqpoint{0.720252in}{1.947324in}}%
\pgfpathlineto{\pgfqpoint{0.742787in}{1.945261in}}%
\pgfpathlineto{\pgfqpoint{0.765323in}{1.943178in}}%
\pgfpathlineto{\pgfqpoint{0.787858in}{1.941383in}}%
\pgfpathlineto{\pgfqpoint{0.810393in}{1.939767in}}%
\pgfpathlineto{\pgfqpoint{0.832929in}{1.938152in}}%
\pgfpathlineto{\pgfqpoint{0.855464in}{1.936328in}}%
\pgfpathlineto{\pgfqpoint{0.877999in}{1.934838in}}%
\pgfpathlineto{\pgfqpoint{0.900534in}{1.933368in}}%
\pgfpathlineto{\pgfqpoint{0.923070in}{1.931924in}}%
\pgfpathlineto{\pgfqpoint{0.945605in}{1.930829in}}%
\pgfpathlineto{\pgfqpoint{0.968140in}{1.929726in}}%
\pgfpathlineto{\pgfqpoint{0.990676in}{1.928398in}}%
\pgfpathlineto{\pgfqpoint{1.013211in}{1.926961in}}%
\pgfpathlineto{\pgfqpoint{1.035746in}{1.925189in}}%
\pgfpathlineto{\pgfqpoint{1.058281in}{1.923885in}}%
\pgfpathlineto{\pgfqpoint{1.080817in}{1.922608in}}%
\pgfpathlineto{\pgfqpoint{1.103352in}{1.921284in}}%
\pgfpathlineto{\pgfqpoint{1.125887in}{1.920038in}}%
\pgfpathlineto{\pgfqpoint{1.148423in}{1.918740in}}%
\pgfpathlineto{\pgfqpoint{1.170958in}{1.917633in}}%
\pgfpathlineto{\pgfqpoint{1.193493in}{1.916673in}}%
\pgfpathlineto{\pgfqpoint{1.216028in}{1.915609in}}%
\pgfpathlineto{\pgfqpoint{1.238564in}{1.914489in}}%
\pgfpathlineto{\pgfqpoint{1.261099in}{1.913459in}}%
\pgfpathlineto{\pgfqpoint{1.283634in}{1.912729in}}%
\pgfpathlineto{\pgfqpoint{1.306170in}{1.911506in}}%
\pgfpathlineto{\pgfqpoint{1.328705in}{1.910349in}}%
\pgfpathlineto{\pgfqpoint{1.351240in}{1.909472in}}%
\pgfpathlineto{\pgfqpoint{1.373776in}{1.908326in}}%
\pgfpathlineto{\pgfqpoint{1.396311in}{1.907530in}}%
\pgfpathlineto{\pgfqpoint{1.418846in}{1.906447in}}%
\pgfpathlineto{\pgfqpoint{1.441381in}{1.905552in}}%
\pgfpathlineto{\pgfqpoint{1.463917in}{1.904458in}}%
\pgfpathlineto{\pgfqpoint{1.486452in}{1.903418in}}%
\pgfpathlineto{\pgfqpoint{1.508987in}{1.902441in}}%
\pgfpathlineto{\pgfqpoint{1.531523in}{1.901571in}}%
\pgfpathlineto{\pgfqpoint{1.554058in}{1.900796in}}%
\pgfpathlineto{\pgfqpoint{1.576593in}{1.899771in}}%
\pgfpathlineto{\pgfqpoint{1.599128in}{1.898854in}}%
\pgfpathlineto{\pgfqpoint{1.621664in}{1.897975in}}%
\pgfpathlineto{\pgfqpoint{1.644199in}{1.897258in}}%
\pgfpathlineto{\pgfqpoint{1.666734in}{1.896383in}}%
\pgfpathlineto{\pgfqpoint{1.689270in}{1.895583in}}%
\pgfpathlineto{\pgfqpoint{1.711805in}{1.894630in}}%
\pgfpathlineto{\pgfqpoint{1.734340in}{1.893896in}}%
\pgfpathlineto{\pgfqpoint{1.756875in}{1.893280in}}%
\pgfpathlineto{\pgfqpoint{1.779411in}{1.892588in}}%
\pgfpathlineto{\pgfqpoint{1.801946in}{1.891814in}}%
\pgfpathlineto{\pgfqpoint{1.824481in}{1.891166in}}%
\pgfpathlineto{\pgfqpoint{1.847017in}{1.890472in}}%
\pgfpathlineto{\pgfqpoint{1.869552in}{1.889709in}}%
\pgfpathlineto{\pgfqpoint{1.892087in}{1.888901in}}%
\pgfpathlineto{\pgfqpoint{1.914622in}{1.888210in}}%
\pgfpathlineto{\pgfqpoint{1.937158in}{1.887359in}}%
\pgfpathlineto{\pgfqpoint{1.959693in}{1.886654in}}%
\pgfpathlineto{\pgfqpoint{1.982228in}{1.886032in}}%
\pgfpathlineto{\pgfqpoint{2.004764in}{1.885334in}}%
\pgfpathlineto{\pgfqpoint{2.027299in}{1.884671in}}%
\pgfpathlineto{\pgfqpoint{2.049834in}{1.884006in}}%
\pgfpathlineto{\pgfqpoint{2.072369in}{1.883304in}}%
\pgfpathlineto{\pgfqpoint{2.094905in}{1.882638in}}%
\pgfpathlineto{\pgfqpoint{2.117440in}{1.882143in}}%
\pgfpathlineto{\pgfqpoint{2.139975in}{1.881531in}}%
\pgfpathlineto{\pgfqpoint{2.162511in}{1.880966in}}%
\pgfpathlineto{\pgfqpoint{2.185046in}{1.880335in}}%
\pgfpathlineto{\pgfqpoint{2.207581in}{1.879696in}}%
\pgfpathlineto{\pgfqpoint{2.230116in}{1.879106in}}%
\pgfpathlineto{\pgfqpoint{2.252652in}{1.878461in}}%
\pgfpathlineto{\pgfqpoint{2.275187in}{1.877809in}}%
\pgfpathlineto{\pgfqpoint{2.297722in}{1.877220in}}%
\pgfpathlineto{\pgfqpoint{2.320258in}{1.876437in}}%
\pgfpathlineto{\pgfqpoint{2.342793in}{1.875840in}}%
\pgfpathlineto{\pgfqpoint{2.365328in}{1.875346in}}%
\pgfpathlineto{\pgfqpoint{2.387863in}{1.874876in}}%
\pgfpathlineto{\pgfqpoint{2.410399in}{1.874310in}}%
\pgfpathlineto{\pgfqpoint{2.432934in}{1.873838in}}%
\pgfpathlineto{\pgfqpoint{2.455469in}{1.873263in}}%
\pgfpathlineto{\pgfqpoint{2.478005in}{1.872685in}}%
\pgfpathlineto{\pgfqpoint{2.500540in}{1.872293in}}%
\pgfpathlineto{\pgfqpoint{2.523075in}{1.871686in}}%
\pgfpathlineto{\pgfqpoint{2.545610in}{1.871166in}}%
\pgfpathlineto{\pgfqpoint{2.568146in}{1.870663in}}%
\pgfpathlineto{\pgfqpoint{2.590681in}{1.870200in}}%
\pgfpathlineto{\pgfqpoint{2.613216in}{1.869764in}}%
\pgfpathlineto{\pgfqpoint{2.635752in}{1.869207in}}%
\pgfpathlineto{\pgfqpoint{2.658287in}{1.868565in}}%
\pgfpathlineto{\pgfqpoint{2.680822in}{1.868127in}}%
\pgfpathlineto{\pgfqpoint{2.703357in}{1.867649in}}%
\pgfpathlineto{\pgfqpoint{2.725893in}{1.867071in}}%
\pgfpathlineto{\pgfqpoint{2.748428in}{1.866500in}}%
\pgfpathlineto{\pgfqpoint{2.770963in}{1.866023in}}%
\pgfusepath{stroke}%
\end{pgfscope}%
\begin{pgfscope}%
\pgfpathrectangle{\pgfqpoint{0.539970in}{0.422992in}}{\pgfqpoint{2.276064in}{1.626201in}}%
\pgfusepath{clip}%
\pgfsetrectcap%
\pgfsetroundjoin%
\pgfsetlinewidth{1.003750pt}%
\definecolor{currentstroke}{rgb}{1.000000,0.172549,0.000000}%
\pgfsetstrokecolor{currentstroke}%
\pgfsetdash{}{0pt}%
\pgfpathmoveto{\pgfqpoint{0.562505in}{1.622840in}}%
\pgfpathlineto{\pgfqpoint{0.585040in}{1.541489in}}%
\pgfpathlineto{\pgfqpoint{0.607576in}{1.474917in}}%
\pgfpathlineto{\pgfqpoint{0.630111in}{1.431296in}}%
\pgfpathlineto{\pgfqpoint{0.652646in}{1.381743in}}%
\pgfpathlineto{\pgfqpoint{0.675182in}{1.336748in}}%
\pgfpathlineto{\pgfqpoint{0.697717in}{1.298751in}}%
\pgfpathlineto{\pgfqpoint{0.720252in}{1.261898in}}%
\pgfpathlineto{\pgfqpoint{0.742787in}{1.234205in}}%
\pgfpathlineto{\pgfqpoint{0.765323in}{1.203013in}}%
\pgfpathlineto{\pgfqpoint{0.787858in}{1.171400in}}%
\pgfpathlineto{\pgfqpoint{0.810393in}{1.137301in}}%
\pgfpathlineto{\pgfqpoint{0.832929in}{1.109473in}}%
\pgfpathlineto{\pgfqpoint{0.855464in}{1.087327in}}%
\pgfpathlineto{\pgfqpoint{0.877999in}{1.067574in}}%
\pgfpathlineto{\pgfqpoint{0.900534in}{1.046510in}}%
\pgfpathlineto{\pgfqpoint{0.923070in}{1.024840in}}%
\pgfpathlineto{\pgfqpoint{0.945605in}{1.004953in}}%
\pgfpathlineto{\pgfqpoint{0.968140in}{0.986365in}}%
\pgfpathlineto{\pgfqpoint{0.990676in}{0.971117in}}%
\pgfpathlineto{\pgfqpoint{1.013211in}{0.960463in}}%
\pgfpathlineto{\pgfqpoint{1.035746in}{0.946338in}}%
\pgfpathlineto{\pgfqpoint{1.058281in}{0.931602in}}%
\pgfpathlineto{\pgfqpoint{1.080817in}{0.917301in}}%
\pgfpathlineto{\pgfqpoint{1.103352in}{0.904900in}}%
\pgfpathlineto{\pgfqpoint{1.125887in}{0.893138in}}%
\pgfpathlineto{\pgfqpoint{1.148423in}{0.881122in}}%
\pgfpathlineto{\pgfqpoint{1.170958in}{0.869289in}}%
\pgfpathlineto{\pgfqpoint{1.193493in}{0.857612in}}%
\pgfpathlineto{\pgfqpoint{1.216028in}{0.845529in}}%
\pgfpathlineto{\pgfqpoint{1.238564in}{0.837193in}}%
\pgfpathlineto{\pgfqpoint{1.261099in}{0.828023in}}%
\pgfpathlineto{\pgfqpoint{1.283634in}{0.820056in}}%
\pgfpathlineto{\pgfqpoint{1.306170in}{0.808871in}}%
\pgfpathlineto{\pgfqpoint{1.328705in}{0.801149in}}%
\pgfpathlineto{\pgfqpoint{1.351240in}{0.793525in}}%
\pgfpathlineto{\pgfqpoint{1.373776in}{0.785856in}}%
\pgfpathlineto{\pgfqpoint{1.396311in}{0.778258in}}%
\pgfpathlineto{\pgfqpoint{1.418846in}{0.769740in}}%
\pgfpathlineto{\pgfqpoint{1.441381in}{0.761739in}}%
\pgfpathlineto{\pgfqpoint{1.463917in}{0.755125in}}%
\pgfpathlineto{\pgfqpoint{1.486452in}{0.746628in}}%
\pgfpathlineto{\pgfqpoint{1.508987in}{0.739523in}}%
\pgfpathlineto{\pgfqpoint{1.531523in}{0.733943in}}%
\pgfpathlineto{\pgfqpoint{1.554058in}{0.725664in}}%
\pgfpathlineto{\pgfqpoint{1.576593in}{0.718741in}}%
\pgfpathlineto{\pgfqpoint{1.599128in}{0.711588in}}%
\pgfpathlineto{\pgfqpoint{1.621664in}{0.704789in}}%
\pgfpathlineto{\pgfqpoint{1.644199in}{0.698608in}}%
\pgfpathlineto{\pgfqpoint{1.666734in}{0.692202in}}%
\pgfpathlineto{\pgfqpoint{1.689270in}{0.686464in}}%
\pgfpathlineto{\pgfqpoint{1.711805in}{0.680373in}}%
\pgfpathlineto{\pgfqpoint{1.734340in}{0.674311in}}%
\pgfpathlineto{\pgfqpoint{1.756875in}{0.668573in}}%
\pgfpathlineto{\pgfqpoint{1.779411in}{0.663054in}}%
\pgfpathlineto{\pgfqpoint{1.801946in}{0.658019in}}%
\pgfpathlineto{\pgfqpoint{1.824481in}{0.653353in}}%
\pgfpathlineto{\pgfqpoint{1.847017in}{0.648547in}}%
\pgfpathlineto{\pgfqpoint{1.869552in}{0.643255in}}%
\pgfpathlineto{\pgfqpoint{1.892087in}{0.637967in}}%
\pgfpathlineto{\pgfqpoint{1.914622in}{0.633955in}}%
\pgfpathlineto{\pgfqpoint{1.937158in}{0.629591in}}%
\pgfpathlineto{\pgfqpoint{1.959693in}{0.624664in}}%
\pgfpathlineto{\pgfqpoint{1.982228in}{0.620103in}}%
\pgfpathlineto{\pgfqpoint{2.004764in}{0.615221in}}%
\pgfpathlineto{\pgfqpoint{2.027299in}{0.611083in}}%
\pgfpathlineto{\pgfqpoint{2.049834in}{0.607257in}}%
\pgfpathlineto{\pgfqpoint{2.072369in}{0.603776in}}%
\pgfpathlineto{\pgfqpoint{2.094905in}{0.599690in}}%
\pgfpathlineto{\pgfqpoint{2.117440in}{0.596154in}}%
\pgfpathlineto{\pgfqpoint{2.139975in}{0.592531in}}%
\pgfpathlineto{\pgfqpoint{2.162511in}{0.588851in}}%
\pgfpathlineto{\pgfqpoint{2.185046in}{0.584967in}}%
\pgfpathlineto{\pgfqpoint{2.207581in}{0.580615in}}%
\pgfpathlineto{\pgfqpoint{2.230116in}{0.576986in}}%
\pgfpathlineto{\pgfqpoint{2.252652in}{0.573104in}}%
\pgfpathlineto{\pgfqpoint{2.275187in}{0.569788in}}%
\pgfpathlineto{\pgfqpoint{2.297722in}{0.566173in}}%
\pgfpathlineto{\pgfqpoint{2.320258in}{0.562081in}}%
\pgfpathlineto{\pgfqpoint{2.342793in}{0.558209in}}%
\pgfpathlineto{\pgfqpoint{2.365328in}{0.554007in}}%
\pgfpathlineto{\pgfqpoint{2.387863in}{0.550868in}}%
\pgfpathlineto{\pgfqpoint{2.410399in}{0.548212in}}%
\pgfpathlineto{\pgfqpoint{2.432934in}{0.544791in}}%
\pgfpathlineto{\pgfqpoint{2.455469in}{0.540669in}}%
\pgfpathlineto{\pgfqpoint{2.478005in}{0.537579in}}%
\pgfpathlineto{\pgfqpoint{2.500540in}{0.534897in}}%
\pgfpathlineto{\pgfqpoint{2.523075in}{0.531483in}}%
\pgfpathlineto{\pgfqpoint{2.545610in}{0.527383in}}%
\pgfpathlineto{\pgfqpoint{2.568146in}{0.524818in}}%
\pgfpathlineto{\pgfqpoint{2.590681in}{0.521014in}}%
\pgfpathlineto{\pgfqpoint{2.613216in}{0.518314in}}%
\pgfpathlineto{\pgfqpoint{2.635752in}{0.515200in}}%
\pgfpathlineto{\pgfqpoint{2.658287in}{0.512305in}}%
\pgfpathlineto{\pgfqpoint{2.680822in}{0.509065in}}%
\pgfpathlineto{\pgfqpoint{2.703357in}{0.505862in}}%
\pgfpathlineto{\pgfqpoint{2.725893in}{0.502946in}}%
\pgfpathlineto{\pgfqpoint{2.748428in}{0.499930in}}%
\pgfpathlineto{\pgfqpoint{2.770963in}{0.496910in}}%
\pgfusepath{stroke}%
\end{pgfscope}%
\begin{pgfscope}%
\pgfpathrectangle{\pgfqpoint{0.539970in}{0.422992in}}{\pgfqpoint{2.276064in}{1.626201in}}%
\pgfusepath{clip}%
\pgfsetrectcap%
\pgfsetroundjoin%
\pgfsetlinewidth{1.003750pt}%
\definecolor{currentstroke}{rgb}{0.517647,0.356863,0.592157}%
\pgfsetstrokecolor{currentstroke}%
\pgfsetdash{}{0pt}%
\pgfpathmoveto{\pgfqpoint{0.562505in}{1.561103in}}%
\pgfpathlineto{\pgfqpoint{0.585040in}{1.532216in}}%
\pgfpathlineto{\pgfqpoint{0.607576in}{1.516814in}}%
\pgfpathlineto{\pgfqpoint{0.630111in}{1.506384in}}%
\pgfpathlineto{\pgfqpoint{0.652646in}{1.487444in}}%
\pgfpathlineto{\pgfqpoint{0.675182in}{1.472496in}}%
\pgfpathlineto{\pgfqpoint{0.697717in}{1.462784in}}%
\pgfpathlineto{\pgfqpoint{0.720252in}{1.450749in}}%
\pgfpathlineto{\pgfqpoint{0.742787in}{1.443591in}}%
\pgfpathlineto{\pgfqpoint{0.765323in}{1.436007in}}%
\pgfpathlineto{\pgfqpoint{0.787858in}{1.430481in}}%
\pgfpathlineto{\pgfqpoint{0.810393in}{1.422738in}}%
\pgfpathlineto{\pgfqpoint{0.832929in}{1.415660in}}%
\pgfpathlineto{\pgfqpoint{0.855464in}{1.409599in}}%
\pgfpathlineto{\pgfqpoint{0.877999in}{1.404137in}}%
\pgfpathlineto{\pgfqpoint{0.900534in}{1.399822in}}%
\pgfpathlineto{\pgfqpoint{0.923070in}{1.396568in}}%
\pgfpathlineto{\pgfqpoint{0.945605in}{1.392163in}}%
\pgfpathlineto{\pgfqpoint{0.968140in}{1.387600in}}%
\pgfpathlineto{\pgfqpoint{0.990676in}{1.385154in}}%
\pgfpathlineto{\pgfqpoint{1.013211in}{1.381074in}}%
\pgfpathlineto{\pgfqpoint{1.035746in}{1.377412in}}%
\pgfpathlineto{\pgfqpoint{1.058281in}{1.374137in}}%
\pgfpathlineto{\pgfqpoint{1.080817in}{1.370464in}}%
\pgfpathlineto{\pgfqpoint{1.103352in}{1.365160in}}%
\pgfpathlineto{\pgfqpoint{1.125887in}{1.361144in}}%
\pgfpathlineto{\pgfqpoint{1.148423in}{1.358197in}}%
\pgfpathlineto{\pgfqpoint{1.170958in}{1.354557in}}%
\pgfpathlineto{\pgfqpoint{1.193493in}{1.351696in}}%
\pgfpathlineto{\pgfqpoint{1.216028in}{1.347670in}}%
\pgfpathlineto{\pgfqpoint{1.238564in}{1.344396in}}%
\pgfpathlineto{\pgfqpoint{1.261099in}{1.342077in}}%
\pgfpathlineto{\pgfqpoint{1.283634in}{1.338943in}}%
\pgfpathlineto{\pgfqpoint{1.306170in}{1.336135in}}%
\pgfpathlineto{\pgfqpoint{1.328705in}{1.333792in}}%
\pgfpathlineto{\pgfqpoint{1.351240in}{1.331834in}}%
\pgfpathlineto{\pgfqpoint{1.373776in}{1.329690in}}%
\pgfpathlineto{\pgfqpoint{1.396311in}{1.326823in}}%
\pgfpathlineto{\pgfqpoint{1.418846in}{1.324734in}}%
\pgfpathlineto{\pgfqpoint{1.441381in}{1.323146in}}%
\pgfpathlineto{\pgfqpoint{1.463917in}{1.320525in}}%
\pgfpathlineto{\pgfqpoint{1.486452in}{1.318506in}}%
\pgfpathlineto{\pgfqpoint{1.508987in}{1.316287in}}%
\pgfpathlineto{\pgfqpoint{1.531523in}{1.314491in}}%
\pgfpathlineto{\pgfqpoint{1.554058in}{1.311500in}}%
\pgfpathlineto{\pgfqpoint{1.576593in}{1.309389in}}%
\pgfpathlineto{\pgfqpoint{1.599128in}{1.307539in}}%
\pgfpathlineto{\pgfqpoint{1.621664in}{1.304843in}}%
\pgfpathlineto{\pgfqpoint{1.644199in}{1.303049in}}%
\pgfpathlineto{\pgfqpoint{1.666734in}{1.301048in}}%
\pgfpathlineto{\pgfqpoint{1.689270in}{1.299486in}}%
\pgfpathlineto{\pgfqpoint{1.711805in}{1.297672in}}%
\pgfpathlineto{\pgfqpoint{1.734340in}{1.295281in}}%
\pgfpathlineto{\pgfqpoint{1.756875in}{1.293046in}}%
\pgfpathlineto{\pgfqpoint{1.779411in}{1.291505in}}%
\pgfpathlineto{\pgfqpoint{1.801946in}{1.290156in}}%
\pgfpathlineto{\pgfqpoint{1.824481in}{1.288324in}}%
\pgfpathlineto{\pgfqpoint{1.847017in}{1.286906in}}%
\pgfpathlineto{\pgfqpoint{1.869552in}{1.285384in}}%
\pgfpathlineto{\pgfqpoint{1.892087in}{1.284422in}}%
\pgfpathlineto{\pgfqpoint{1.914622in}{1.283124in}}%
\pgfpathlineto{\pgfqpoint{1.937158in}{1.281352in}}%
\pgfpathlineto{\pgfqpoint{1.959693in}{1.279694in}}%
\pgfpathlineto{\pgfqpoint{1.982228in}{1.278398in}}%
\pgfpathlineto{\pgfqpoint{2.004764in}{1.277178in}}%
\pgfpathlineto{\pgfqpoint{2.027299in}{1.275684in}}%
\pgfpathlineto{\pgfqpoint{2.049834in}{1.274044in}}%
\pgfpathlineto{\pgfqpoint{2.072369in}{1.273017in}}%
\pgfpathlineto{\pgfqpoint{2.094905in}{1.271173in}}%
\pgfpathlineto{\pgfqpoint{2.117440in}{1.269949in}}%
\pgfpathlineto{\pgfqpoint{2.139975in}{1.268560in}}%
\pgfpathlineto{\pgfqpoint{2.162511in}{1.267367in}}%
\pgfpathlineto{\pgfqpoint{2.185046in}{1.266305in}}%
\pgfpathlineto{\pgfqpoint{2.207581in}{1.265019in}}%
\pgfpathlineto{\pgfqpoint{2.230116in}{1.263898in}}%
\pgfpathlineto{\pgfqpoint{2.252652in}{1.262678in}}%
\pgfpathlineto{\pgfqpoint{2.275187in}{1.261677in}}%
\pgfpathlineto{\pgfqpoint{2.297722in}{1.260120in}}%
\pgfpathlineto{\pgfqpoint{2.320258in}{1.258747in}}%
\pgfpathlineto{\pgfqpoint{2.342793in}{1.257202in}}%
\pgfpathlineto{\pgfqpoint{2.365328in}{1.255958in}}%
\pgfpathlineto{\pgfqpoint{2.387863in}{1.254838in}}%
\pgfpathlineto{\pgfqpoint{2.410399in}{1.254030in}}%
\pgfpathlineto{\pgfqpoint{2.432934in}{1.253038in}}%
\pgfpathlineto{\pgfqpoint{2.455469in}{1.251978in}}%
\pgfpathlineto{\pgfqpoint{2.478005in}{1.251213in}}%
\pgfpathlineto{\pgfqpoint{2.500540in}{1.250235in}}%
\pgfpathlineto{\pgfqpoint{2.523075in}{1.249343in}}%
\pgfpathlineto{\pgfqpoint{2.545610in}{1.248367in}}%
\pgfpathlineto{\pgfqpoint{2.568146in}{1.247698in}}%
\pgfpathlineto{\pgfqpoint{2.590681in}{1.246377in}}%
\pgfpathlineto{\pgfqpoint{2.613216in}{1.245438in}}%
\pgfpathlineto{\pgfqpoint{2.635752in}{1.244282in}}%
\pgfpathlineto{\pgfqpoint{2.658287in}{1.243436in}}%
\pgfpathlineto{\pgfqpoint{2.680822in}{1.242340in}}%
\pgfpathlineto{\pgfqpoint{2.703357in}{1.241134in}}%
\pgfpathlineto{\pgfqpoint{2.725893in}{1.240385in}}%
\pgfpathlineto{\pgfqpoint{2.748428in}{1.239419in}}%
\pgfpathlineto{\pgfqpoint{2.770963in}{1.238776in}}%
\pgfusepath{stroke}%
\end{pgfscope}%
\begin{pgfscope}%
\pgfsetrectcap%
\pgfsetmiterjoin%
\pgfsetlinewidth{0.501875pt}%
\definecolor{currentstroke}{rgb}{0.000000,0.000000,0.000000}%
\pgfsetstrokecolor{currentstroke}%
\pgfsetdash{}{0pt}%
\pgfpathmoveto{\pgfqpoint{0.539970in}{0.422992in}}%
\pgfpathlineto{\pgfqpoint{0.539970in}{2.049193in}}%
\pgfusepath{stroke}%
\end{pgfscope}%
\begin{pgfscope}%
\pgfsetrectcap%
\pgfsetmiterjoin%
\pgfsetlinewidth{0.501875pt}%
\definecolor{currentstroke}{rgb}{0.000000,0.000000,0.000000}%
\pgfsetstrokecolor{currentstroke}%
\pgfsetdash{}{0pt}%
\pgfpathmoveto{\pgfqpoint{2.816034in}{0.422992in}}%
\pgfpathlineto{\pgfqpoint{2.816034in}{2.049193in}}%
\pgfusepath{stroke}%
\end{pgfscope}%
\begin{pgfscope}%
\pgfsetrectcap%
\pgfsetmiterjoin%
\pgfsetlinewidth{0.501875pt}%
\definecolor{currentstroke}{rgb}{0.000000,0.000000,0.000000}%
\pgfsetstrokecolor{currentstroke}%
\pgfsetdash{}{0pt}%
\pgfpathmoveto{\pgfqpoint{0.539970in}{0.422992in}}%
\pgfpathlineto{\pgfqpoint{2.816034in}{0.422992in}}%
\pgfusepath{stroke}%
\end{pgfscope}%
\begin{pgfscope}%
\pgfsetrectcap%
\pgfsetmiterjoin%
\pgfsetlinewidth{0.501875pt}%
\definecolor{currentstroke}{rgb}{0.000000,0.000000,0.000000}%
\pgfsetstrokecolor{currentstroke}%
\pgfsetdash{}{0pt}%
\pgfpathmoveto{\pgfqpoint{0.539970in}{2.049193in}}%
\pgfpathlineto{\pgfqpoint{2.816034in}{2.049193in}}%
\pgfusepath{stroke}%
\end{pgfscope}%
\begin{pgfscope}%
\definecolor{textcolor}{rgb}{0.000000,0.000000,0.000000}%
\pgfsetstrokecolor{textcolor}%
\pgfsetfillcolor{textcolor}%
\pgftext[x=1.678002in,y=2.132526in,,base]{\color{textcolor}\rmfamily\fontsize{12.000000}{14.400000}\selectfont Continuity}%
\end{pgfscope}%
\begin{pgfscope}%
\pgfsetbuttcap%
\pgfsetmiterjoin%
\definecolor{currentfill}{rgb}{1.000000,1.000000,1.000000}%
\pgfsetfillcolor{currentfill}%
\pgfsetlinewidth{0.000000pt}%
\definecolor{currentstroke}{rgb}{0.000000,0.000000,0.000000}%
\pgfsetstrokecolor{currentstroke}%
\pgfsetstrokeopacity{0.000000}%
\pgfsetdash{}{0pt}%
\pgfpathmoveto{\pgfqpoint{3.385377in}{0.422992in}}%
\pgfpathlineto{\pgfqpoint{5.661441in}{0.422992in}}%
\pgfpathlineto{\pgfqpoint{5.661441in}{4.374193in}}%
\pgfpathlineto{\pgfqpoint{3.385377in}{4.374193in}}%
\pgfpathlineto{\pgfqpoint{3.385377in}{0.422992in}}%
\pgfpathclose%
\pgfusepath{fill}%
\end{pgfscope}%
\begin{pgfscope}%
\pgfsetbuttcap%
\pgfsetroundjoin%
\definecolor{currentfill}{rgb}{0.000000,0.000000,0.000000}%
\pgfsetfillcolor{currentfill}%
\pgfsetlinewidth{0.501875pt}%
\definecolor{currentstroke}{rgb}{0.000000,0.000000,0.000000}%
\pgfsetstrokecolor{currentstroke}%
\pgfsetdash{}{0pt}%
\pgfsys@defobject{currentmarker}{\pgfqpoint{0.000000in}{0.000000in}}{\pgfqpoint{0.000000in}{0.041667in}}{%
\pgfpathmoveto{\pgfqpoint{0.000000in}{0.000000in}}%
\pgfpathlineto{\pgfqpoint{0.000000in}{0.041667in}}%
\pgfusepath{stroke,fill}%
}%
\begin{pgfscope}%
\pgfsys@transformshift{3.385377in}{0.422992in}%
\pgfsys@useobject{currentmarker}{}%
\end{pgfscope}%
\end{pgfscope}%
\begin{pgfscope}%
\pgfsetbuttcap%
\pgfsetroundjoin%
\definecolor{currentfill}{rgb}{0.000000,0.000000,0.000000}%
\pgfsetfillcolor{currentfill}%
\pgfsetlinewidth{0.501875pt}%
\definecolor{currentstroke}{rgb}{0.000000,0.000000,0.000000}%
\pgfsetstrokecolor{currentstroke}%
\pgfsetdash{}{0pt}%
\pgfsys@defobject{currentmarker}{\pgfqpoint{0.000000in}{-0.041667in}}{\pgfqpoint{0.000000in}{0.000000in}}{%
\pgfpathmoveto{\pgfqpoint{0.000000in}{0.000000in}}%
\pgfpathlineto{\pgfqpoint{0.000000in}{-0.041667in}}%
\pgfusepath{stroke,fill}%
}%
\begin{pgfscope}%
\pgfsys@transformshift{3.385377in}{4.374193in}%
\pgfsys@useobject{currentmarker}{}%
\end{pgfscope}%
\end{pgfscope}%
\begin{pgfscope}%
\definecolor{textcolor}{rgb}{0.000000,0.000000,0.000000}%
\pgfsetstrokecolor{textcolor}%
\pgfsetfillcolor{textcolor}%
\pgftext[x=3.385377in,y=0.374381in,,top]{\color{textcolor}\rmfamily\fontsize{10.000000}{12.000000}\selectfont \(\displaystyle {0}\)}%
\end{pgfscope}%
\begin{pgfscope}%
\pgfsetbuttcap%
\pgfsetroundjoin%
\definecolor{currentfill}{rgb}{0.000000,0.000000,0.000000}%
\pgfsetfillcolor{currentfill}%
\pgfsetlinewidth{0.501875pt}%
\definecolor{currentstroke}{rgb}{0.000000,0.000000,0.000000}%
\pgfsetstrokecolor{currentstroke}%
\pgfsetdash{}{0pt}%
\pgfsys@defobject{currentmarker}{\pgfqpoint{0.000000in}{0.000000in}}{\pgfqpoint{0.000000in}{0.041667in}}{%
\pgfpathmoveto{\pgfqpoint{0.000000in}{0.000000in}}%
\pgfpathlineto{\pgfqpoint{0.000000in}{0.041667in}}%
\pgfusepath{stroke,fill}%
}%
\begin{pgfscope}%
\pgfsys@transformshift{3.836083in}{0.422992in}%
\pgfsys@useobject{currentmarker}{}%
\end{pgfscope}%
\end{pgfscope}%
\begin{pgfscope}%
\pgfsetbuttcap%
\pgfsetroundjoin%
\definecolor{currentfill}{rgb}{0.000000,0.000000,0.000000}%
\pgfsetfillcolor{currentfill}%
\pgfsetlinewidth{0.501875pt}%
\definecolor{currentstroke}{rgb}{0.000000,0.000000,0.000000}%
\pgfsetstrokecolor{currentstroke}%
\pgfsetdash{}{0pt}%
\pgfsys@defobject{currentmarker}{\pgfqpoint{0.000000in}{-0.041667in}}{\pgfqpoint{0.000000in}{0.000000in}}{%
\pgfpathmoveto{\pgfqpoint{0.000000in}{0.000000in}}%
\pgfpathlineto{\pgfqpoint{0.000000in}{-0.041667in}}%
\pgfusepath{stroke,fill}%
}%
\begin{pgfscope}%
\pgfsys@transformshift{3.836083in}{4.374193in}%
\pgfsys@useobject{currentmarker}{}%
\end{pgfscope}%
\end{pgfscope}%
\begin{pgfscope}%
\definecolor{textcolor}{rgb}{0.000000,0.000000,0.000000}%
\pgfsetstrokecolor{textcolor}%
\pgfsetfillcolor{textcolor}%
\pgftext[x=3.836083in,y=0.374381in,,top]{\color{textcolor}\rmfamily\fontsize{10.000000}{12.000000}\selectfont \(\displaystyle {20}\)}%
\end{pgfscope}%
\begin{pgfscope}%
\pgfsetbuttcap%
\pgfsetroundjoin%
\definecolor{currentfill}{rgb}{0.000000,0.000000,0.000000}%
\pgfsetfillcolor{currentfill}%
\pgfsetlinewidth{0.501875pt}%
\definecolor{currentstroke}{rgb}{0.000000,0.000000,0.000000}%
\pgfsetstrokecolor{currentstroke}%
\pgfsetdash{}{0pt}%
\pgfsys@defobject{currentmarker}{\pgfqpoint{0.000000in}{0.000000in}}{\pgfqpoint{0.000000in}{0.041667in}}{%
\pgfpathmoveto{\pgfqpoint{0.000000in}{0.000000in}}%
\pgfpathlineto{\pgfqpoint{0.000000in}{0.041667in}}%
\pgfusepath{stroke,fill}%
}%
\begin{pgfscope}%
\pgfsys@transformshift{4.286789in}{0.422992in}%
\pgfsys@useobject{currentmarker}{}%
\end{pgfscope}%
\end{pgfscope}%
\begin{pgfscope}%
\pgfsetbuttcap%
\pgfsetroundjoin%
\definecolor{currentfill}{rgb}{0.000000,0.000000,0.000000}%
\pgfsetfillcolor{currentfill}%
\pgfsetlinewidth{0.501875pt}%
\definecolor{currentstroke}{rgb}{0.000000,0.000000,0.000000}%
\pgfsetstrokecolor{currentstroke}%
\pgfsetdash{}{0pt}%
\pgfsys@defobject{currentmarker}{\pgfqpoint{0.000000in}{-0.041667in}}{\pgfqpoint{0.000000in}{0.000000in}}{%
\pgfpathmoveto{\pgfqpoint{0.000000in}{0.000000in}}%
\pgfpathlineto{\pgfqpoint{0.000000in}{-0.041667in}}%
\pgfusepath{stroke,fill}%
}%
\begin{pgfscope}%
\pgfsys@transformshift{4.286789in}{4.374193in}%
\pgfsys@useobject{currentmarker}{}%
\end{pgfscope}%
\end{pgfscope}%
\begin{pgfscope}%
\definecolor{textcolor}{rgb}{0.000000,0.000000,0.000000}%
\pgfsetstrokecolor{textcolor}%
\pgfsetfillcolor{textcolor}%
\pgftext[x=4.286789in,y=0.374381in,,top]{\color{textcolor}\rmfamily\fontsize{10.000000}{12.000000}\selectfont \(\displaystyle {40}\)}%
\end{pgfscope}%
\begin{pgfscope}%
\pgfsetbuttcap%
\pgfsetroundjoin%
\definecolor{currentfill}{rgb}{0.000000,0.000000,0.000000}%
\pgfsetfillcolor{currentfill}%
\pgfsetlinewidth{0.501875pt}%
\definecolor{currentstroke}{rgb}{0.000000,0.000000,0.000000}%
\pgfsetstrokecolor{currentstroke}%
\pgfsetdash{}{0pt}%
\pgfsys@defobject{currentmarker}{\pgfqpoint{0.000000in}{0.000000in}}{\pgfqpoint{0.000000in}{0.041667in}}{%
\pgfpathmoveto{\pgfqpoint{0.000000in}{0.000000in}}%
\pgfpathlineto{\pgfqpoint{0.000000in}{0.041667in}}%
\pgfusepath{stroke,fill}%
}%
\begin{pgfscope}%
\pgfsys@transformshift{4.737495in}{0.422992in}%
\pgfsys@useobject{currentmarker}{}%
\end{pgfscope}%
\end{pgfscope}%
\begin{pgfscope}%
\pgfsetbuttcap%
\pgfsetroundjoin%
\definecolor{currentfill}{rgb}{0.000000,0.000000,0.000000}%
\pgfsetfillcolor{currentfill}%
\pgfsetlinewidth{0.501875pt}%
\definecolor{currentstroke}{rgb}{0.000000,0.000000,0.000000}%
\pgfsetstrokecolor{currentstroke}%
\pgfsetdash{}{0pt}%
\pgfsys@defobject{currentmarker}{\pgfqpoint{0.000000in}{-0.041667in}}{\pgfqpoint{0.000000in}{0.000000in}}{%
\pgfpathmoveto{\pgfqpoint{0.000000in}{0.000000in}}%
\pgfpathlineto{\pgfqpoint{0.000000in}{-0.041667in}}%
\pgfusepath{stroke,fill}%
}%
\begin{pgfscope}%
\pgfsys@transformshift{4.737495in}{4.374193in}%
\pgfsys@useobject{currentmarker}{}%
\end{pgfscope}%
\end{pgfscope}%
\begin{pgfscope}%
\definecolor{textcolor}{rgb}{0.000000,0.000000,0.000000}%
\pgfsetstrokecolor{textcolor}%
\pgfsetfillcolor{textcolor}%
\pgftext[x=4.737495in,y=0.374381in,,top]{\color{textcolor}\rmfamily\fontsize{10.000000}{12.000000}\selectfont \(\displaystyle {60}\)}%
\end{pgfscope}%
\begin{pgfscope}%
\pgfsetbuttcap%
\pgfsetroundjoin%
\definecolor{currentfill}{rgb}{0.000000,0.000000,0.000000}%
\pgfsetfillcolor{currentfill}%
\pgfsetlinewidth{0.501875pt}%
\definecolor{currentstroke}{rgb}{0.000000,0.000000,0.000000}%
\pgfsetstrokecolor{currentstroke}%
\pgfsetdash{}{0pt}%
\pgfsys@defobject{currentmarker}{\pgfqpoint{0.000000in}{0.000000in}}{\pgfqpoint{0.000000in}{0.041667in}}{%
\pgfpathmoveto{\pgfqpoint{0.000000in}{0.000000in}}%
\pgfpathlineto{\pgfqpoint{0.000000in}{0.041667in}}%
\pgfusepath{stroke,fill}%
}%
\begin{pgfscope}%
\pgfsys@transformshift{5.188200in}{0.422992in}%
\pgfsys@useobject{currentmarker}{}%
\end{pgfscope}%
\end{pgfscope}%
\begin{pgfscope}%
\pgfsetbuttcap%
\pgfsetroundjoin%
\definecolor{currentfill}{rgb}{0.000000,0.000000,0.000000}%
\pgfsetfillcolor{currentfill}%
\pgfsetlinewidth{0.501875pt}%
\definecolor{currentstroke}{rgb}{0.000000,0.000000,0.000000}%
\pgfsetstrokecolor{currentstroke}%
\pgfsetdash{}{0pt}%
\pgfsys@defobject{currentmarker}{\pgfqpoint{0.000000in}{-0.041667in}}{\pgfqpoint{0.000000in}{0.000000in}}{%
\pgfpathmoveto{\pgfqpoint{0.000000in}{0.000000in}}%
\pgfpathlineto{\pgfqpoint{0.000000in}{-0.041667in}}%
\pgfusepath{stroke,fill}%
}%
\begin{pgfscope}%
\pgfsys@transformshift{5.188200in}{4.374193in}%
\pgfsys@useobject{currentmarker}{}%
\end{pgfscope}%
\end{pgfscope}%
\begin{pgfscope}%
\definecolor{textcolor}{rgb}{0.000000,0.000000,0.000000}%
\pgfsetstrokecolor{textcolor}%
\pgfsetfillcolor{textcolor}%
\pgftext[x=5.188200in,y=0.374381in,,top]{\color{textcolor}\rmfamily\fontsize{10.000000}{12.000000}\selectfont \(\displaystyle {80}\)}%
\end{pgfscope}%
\begin{pgfscope}%
\pgfsetbuttcap%
\pgfsetroundjoin%
\definecolor{currentfill}{rgb}{0.000000,0.000000,0.000000}%
\pgfsetfillcolor{currentfill}%
\pgfsetlinewidth{0.501875pt}%
\definecolor{currentstroke}{rgb}{0.000000,0.000000,0.000000}%
\pgfsetstrokecolor{currentstroke}%
\pgfsetdash{}{0pt}%
\pgfsys@defobject{currentmarker}{\pgfqpoint{0.000000in}{0.000000in}}{\pgfqpoint{0.000000in}{0.020833in}}{%
\pgfpathmoveto{\pgfqpoint{0.000000in}{0.000000in}}%
\pgfpathlineto{\pgfqpoint{0.000000in}{0.020833in}}%
\pgfusepath{stroke,fill}%
}%
\begin{pgfscope}%
\pgfsys@transformshift{3.498054in}{0.422992in}%
\pgfsys@useobject{currentmarker}{}%
\end{pgfscope}%
\end{pgfscope}%
\begin{pgfscope}%
\pgfsetbuttcap%
\pgfsetroundjoin%
\definecolor{currentfill}{rgb}{0.000000,0.000000,0.000000}%
\pgfsetfillcolor{currentfill}%
\pgfsetlinewidth{0.501875pt}%
\definecolor{currentstroke}{rgb}{0.000000,0.000000,0.000000}%
\pgfsetstrokecolor{currentstroke}%
\pgfsetdash{}{0pt}%
\pgfsys@defobject{currentmarker}{\pgfqpoint{0.000000in}{-0.020833in}}{\pgfqpoint{0.000000in}{0.000000in}}{%
\pgfpathmoveto{\pgfqpoint{0.000000in}{0.000000in}}%
\pgfpathlineto{\pgfqpoint{0.000000in}{-0.020833in}}%
\pgfusepath{stroke,fill}%
}%
\begin{pgfscope}%
\pgfsys@transformshift{3.498054in}{4.374193in}%
\pgfsys@useobject{currentmarker}{}%
\end{pgfscope}%
\end{pgfscope}%
\begin{pgfscope}%
\pgfsetbuttcap%
\pgfsetroundjoin%
\definecolor{currentfill}{rgb}{0.000000,0.000000,0.000000}%
\pgfsetfillcolor{currentfill}%
\pgfsetlinewidth{0.501875pt}%
\definecolor{currentstroke}{rgb}{0.000000,0.000000,0.000000}%
\pgfsetstrokecolor{currentstroke}%
\pgfsetdash{}{0pt}%
\pgfsys@defobject{currentmarker}{\pgfqpoint{0.000000in}{0.000000in}}{\pgfqpoint{0.000000in}{0.020833in}}{%
\pgfpathmoveto{\pgfqpoint{0.000000in}{0.000000in}}%
\pgfpathlineto{\pgfqpoint{0.000000in}{0.020833in}}%
\pgfusepath{stroke,fill}%
}%
\begin{pgfscope}%
\pgfsys@transformshift{3.610730in}{0.422992in}%
\pgfsys@useobject{currentmarker}{}%
\end{pgfscope}%
\end{pgfscope}%
\begin{pgfscope}%
\pgfsetbuttcap%
\pgfsetroundjoin%
\definecolor{currentfill}{rgb}{0.000000,0.000000,0.000000}%
\pgfsetfillcolor{currentfill}%
\pgfsetlinewidth{0.501875pt}%
\definecolor{currentstroke}{rgb}{0.000000,0.000000,0.000000}%
\pgfsetstrokecolor{currentstroke}%
\pgfsetdash{}{0pt}%
\pgfsys@defobject{currentmarker}{\pgfqpoint{0.000000in}{-0.020833in}}{\pgfqpoint{0.000000in}{0.000000in}}{%
\pgfpathmoveto{\pgfqpoint{0.000000in}{0.000000in}}%
\pgfpathlineto{\pgfqpoint{0.000000in}{-0.020833in}}%
\pgfusepath{stroke,fill}%
}%
\begin{pgfscope}%
\pgfsys@transformshift{3.610730in}{4.374193in}%
\pgfsys@useobject{currentmarker}{}%
\end{pgfscope}%
\end{pgfscope}%
\begin{pgfscope}%
\pgfsetbuttcap%
\pgfsetroundjoin%
\definecolor{currentfill}{rgb}{0.000000,0.000000,0.000000}%
\pgfsetfillcolor{currentfill}%
\pgfsetlinewidth{0.501875pt}%
\definecolor{currentstroke}{rgb}{0.000000,0.000000,0.000000}%
\pgfsetstrokecolor{currentstroke}%
\pgfsetdash{}{0pt}%
\pgfsys@defobject{currentmarker}{\pgfqpoint{0.000000in}{0.000000in}}{\pgfqpoint{0.000000in}{0.020833in}}{%
\pgfpathmoveto{\pgfqpoint{0.000000in}{0.000000in}}%
\pgfpathlineto{\pgfqpoint{0.000000in}{0.020833in}}%
\pgfusepath{stroke,fill}%
}%
\begin{pgfscope}%
\pgfsys@transformshift{3.723407in}{0.422992in}%
\pgfsys@useobject{currentmarker}{}%
\end{pgfscope}%
\end{pgfscope}%
\begin{pgfscope}%
\pgfsetbuttcap%
\pgfsetroundjoin%
\definecolor{currentfill}{rgb}{0.000000,0.000000,0.000000}%
\pgfsetfillcolor{currentfill}%
\pgfsetlinewidth{0.501875pt}%
\definecolor{currentstroke}{rgb}{0.000000,0.000000,0.000000}%
\pgfsetstrokecolor{currentstroke}%
\pgfsetdash{}{0pt}%
\pgfsys@defobject{currentmarker}{\pgfqpoint{0.000000in}{-0.020833in}}{\pgfqpoint{0.000000in}{0.000000in}}{%
\pgfpathmoveto{\pgfqpoint{0.000000in}{0.000000in}}%
\pgfpathlineto{\pgfqpoint{0.000000in}{-0.020833in}}%
\pgfusepath{stroke,fill}%
}%
\begin{pgfscope}%
\pgfsys@transformshift{3.723407in}{4.374193in}%
\pgfsys@useobject{currentmarker}{}%
\end{pgfscope}%
\end{pgfscope}%
\begin{pgfscope}%
\pgfsetbuttcap%
\pgfsetroundjoin%
\definecolor{currentfill}{rgb}{0.000000,0.000000,0.000000}%
\pgfsetfillcolor{currentfill}%
\pgfsetlinewidth{0.501875pt}%
\definecolor{currentstroke}{rgb}{0.000000,0.000000,0.000000}%
\pgfsetstrokecolor{currentstroke}%
\pgfsetdash{}{0pt}%
\pgfsys@defobject{currentmarker}{\pgfqpoint{0.000000in}{0.000000in}}{\pgfqpoint{0.000000in}{0.020833in}}{%
\pgfpathmoveto{\pgfqpoint{0.000000in}{0.000000in}}%
\pgfpathlineto{\pgfqpoint{0.000000in}{0.020833in}}%
\pgfusepath{stroke,fill}%
}%
\begin{pgfscope}%
\pgfsys@transformshift{3.948759in}{0.422992in}%
\pgfsys@useobject{currentmarker}{}%
\end{pgfscope}%
\end{pgfscope}%
\begin{pgfscope}%
\pgfsetbuttcap%
\pgfsetroundjoin%
\definecolor{currentfill}{rgb}{0.000000,0.000000,0.000000}%
\pgfsetfillcolor{currentfill}%
\pgfsetlinewidth{0.501875pt}%
\definecolor{currentstroke}{rgb}{0.000000,0.000000,0.000000}%
\pgfsetstrokecolor{currentstroke}%
\pgfsetdash{}{0pt}%
\pgfsys@defobject{currentmarker}{\pgfqpoint{0.000000in}{-0.020833in}}{\pgfqpoint{0.000000in}{0.000000in}}{%
\pgfpathmoveto{\pgfqpoint{0.000000in}{0.000000in}}%
\pgfpathlineto{\pgfqpoint{0.000000in}{-0.020833in}}%
\pgfusepath{stroke,fill}%
}%
\begin{pgfscope}%
\pgfsys@transformshift{3.948759in}{4.374193in}%
\pgfsys@useobject{currentmarker}{}%
\end{pgfscope}%
\end{pgfscope}%
\begin{pgfscope}%
\pgfsetbuttcap%
\pgfsetroundjoin%
\definecolor{currentfill}{rgb}{0.000000,0.000000,0.000000}%
\pgfsetfillcolor{currentfill}%
\pgfsetlinewidth{0.501875pt}%
\definecolor{currentstroke}{rgb}{0.000000,0.000000,0.000000}%
\pgfsetstrokecolor{currentstroke}%
\pgfsetdash{}{0pt}%
\pgfsys@defobject{currentmarker}{\pgfqpoint{0.000000in}{0.000000in}}{\pgfqpoint{0.000000in}{0.020833in}}{%
\pgfpathmoveto{\pgfqpoint{0.000000in}{0.000000in}}%
\pgfpathlineto{\pgfqpoint{0.000000in}{0.020833in}}%
\pgfusepath{stroke,fill}%
}%
\begin{pgfscope}%
\pgfsys@transformshift{4.061436in}{0.422992in}%
\pgfsys@useobject{currentmarker}{}%
\end{pgfscope}%
\end{pgfscope}%
\begin{pgfscope}%
\pgfsetbuttcap%
\pgfsetroundjoin%
\definecolor{currentfill}{rgb}{0.000000,0.000000,0.000000}%
\pgfsetfillcolor{currentfill}%
\pgfsetlinewidth{0.501875pt}%
\definecolor{currentstroke}{rgb}{0.000000,0.000000,0.000000}%
\pgfsetstrokecolor{currentstroke}%
\pgfsetdash{}{0pt}%
\pgfsys@defobject{currentmarker}{\pgfqpoint{0.000000in}{-0.020833in}}{\pgfqpoint{0.000000in}{0.000000in}}{%
\pgfpathmoveto{\pgfqpoint{0.000000in}{0.000000in}}%
\pgfpathlineto{\pgfqpoint{0.000000in}{-0.020833in}}%
\pgfusepath{stroke,fill}%
}%
\begin{pgfscope}%
\pgfsys@transformshift{4.061436in}{4.374193in}%
\pgfsys@useobject{currentmarker}{}%
\end{pgfscope}%
\end{pgfscope}%
\begin{pgfscope}%
\pgfsetbuttcap%
\pgfsetroundjoin%
\definecolor{currentfill}{rgb}{0.000000,0.000000,0.000000}%
\pgfsetfillcolor{currentfill}%
\pgfsetlinewidth{0.501875pt}%
\definecolor{currentstroke}{rgb}{0.000000,0.000000,0.000000}%
\pgfsetstrokecolor{currentstroke}%
\pgfsetdash{}{0pt}%
\pgfsys@defobject{currentmarker}{\pgfqpoint{0.000000in}{0.000000in}}{\pgfqpoint{0.000000in}{0.020833in}}{%
\pgfpathmoveto{\pgfqpoint{0.000000in}{0.000000in}}%
\pgfpathlineto{\pgfqpoint{0.000000in}{0.020833in}}%
\pgfusepath{stroke,fill}%
}%
\begin{pgfscope}%
\pgfsys@transformshift{4.174112in}{0.422992in}%
\pgfsys@useobject{currentmarker}{}%
\end{pgfscope}%
\end{pgfscope}%
\begin{pgfscope}%
\pgfsetbuttcap%
\pgfsetroundjoin%
\definecolor{currentfill}{rgb}{0.000000,0.000000,0.000000}%
\pgfsetfillcolor{currentfill}%
\pgfsetlinewidth{0.501875pt}%
\definecolor{currentstroke}{rgb}{0.000000,0.000000,0.000000}%
\pgfsetstrokecolor{currentstroke}%
\pgfsetdash{}{0pt}%
\pgfsys@defobject{currentmarker}{\pgfqpoint{0.000000in}{-0.020833in}}{\pgfqpoint{0.000000in}{0.000000in}}{%
\pgfpathmoveto{\pgfqpoint{0.000000in}{0.000000in}}%
\pgfpathlineto{\pgfqpoint{0.000000in}{-0.020833in}}%
\pgfusepath{stroke,fill}%
}%
\begin{pgfscope}%
\pgfsys@transformshift{4.174112in}{4.374193in}%
\pgfsys@useobject{currentmarker}{}%
\end{pgfscope}%
\end{pgfscope}%
\begin{pgfscope}%
\pgfsetbuttcap%
\pgfsetroundjoin%
\definecolor{currentfill}{rgb}{0.000000,0.000000,0.000000}%
\pgfsetfillcolor{currentfill}%
\pgfsetlinewidth{0.501875pt}%
\definecolor{currentstroke}{rgb}{0.000000,0.000000,0.000000}%
\pgfsetstrokecolor{currentstroke}%
\pgfsetdash{}{0pt}%
\pgfsys@defobject{currentmarker}{\pgfqpoint{0.000000in}{0.000000in}}{\pgfqpoint{0.000000in}{0.020833in}}{%
\pgfpathmoveto{\pgfqpoint{0.000000in}{0.000000in}}%
\pgfpathlineto{\pgfqpoint{0.000000in}{0.020833in}}%
\pgfusepath{stroke,fill}%
}%
\begin{pgfscope}%
\pgfsys@transformshift{4.399465in}{0.422992in}%
\pgfsys@useobject{currentmarker}{}%
\end{pgfscope}%
\end{pgfscope}%
\begin{pgfscope}%
\pgfsetbuttcap%
\pgfsetroundjoin%
\definecolor{currentfill}{rgb}{0.000000,0.000000,0.000000}%
\pgfsetfillcolor{currentfill}%
\pgfsetlinewidth{0.501875pt}%
\definecolor{currentstroke}{rgb}{0.000000,0.000000,0.000000}%
\pgfsetstrokecolor{currentstroke}%
\pgfsetdash{}{0pt}%
\pgfsys@defobject{currentmarker}{\pgfqpoint{0.000000in}{-0.020833in}}{\pgfqpoint{0.000000in}{0.000000in}}{%
\pgfpathmoveto{\pgfqpoint{0.000000in}{0.000000in}}%
\pgfpathlineto{\pgfqpoint{0.000000in}{-0.020833in}}%
\pgfusepath{stroke,fill}%
}%
\begin{pgfscope}%
\pgfsys@transformshift{4.399465in}{4.374193in}%
\pgfsys@useobject{currentmarker}{}%
\end{pgfscope}%
\end{pgfscope}%
\begin{pgfscope}%
\pgfsetbuttcap%
\pgfsetroundjoin%
\definecolor{currentfill}{rgb}{0.000000,0.000000,0.000000}%
\pgfsetfillcolor{currentfill}%
\pgfsetlinewidth{0.501875pt}%
\definecolor{currentstroke}{rgb}{0.000000,0.000000,0.000000}%
\pgfsetstrokecolor{currentstroke}%
\pgfsetdash{}{0pt}%
\pgfsys@defobject{currentmarker}{\pgfqpoint{0.000000in}{0.000000in}}{\pgfqpoint{0.000000in}{0.020833in}}{%
\pgfpathmoveto{\pgfqpoint{0.000000in}{0.000000in}}%
\pgfpathlineto{\pgfqpoint{0.000000in}{0.020833in}}%
\pgfusepath{stroke,fill}%
}%
\begin{pgfscope}%
\pgfsys@transformshift{4.512142in}{0.422992in}%
\pgfsys@useobject{currentmarker}{}%
\end{pgfscope}%
\end{pgfscope}%
\begin{pgfscope}%
\pgfsetbuttcap%
\pgfsetroundjoin%
\definecolor{currentfill}{rgb}{0.000000,0.000000,0.000000}%
\pgfsetfillcolor{currentfill}%
\pgfsetlinewidth{0.501875pt}%
\definecolor{currentstroke}{rgb}{0.000000,0.000000,0.000000}%
\pgfsetstrokecolor{currentstroke}%
\pgfsetdash{}{0pt}%
\pgfsys@defobject{currentmarker}{\pgfqpoint{0.000000in}{-0.020833in}}{\pgfqpoint{0.000000in}{0.000000in}}{%
\pgfpathmoveto{\pgfqpoint{0.000000in}{0.000000in}}%
\pgfpathlineto{\pgfqpoint{0.000000in}{-0.020833in}}%
\pgfusepath{stroke,fill}%
}%
\begin{pgfscope}%
\pgfsys@transformshift{4.512142in}{4.374193in}%
\pgfsys@useobject{currentmarker}{}%
\end{pgfscope}%
\end{pgfscope}%
\begin{pgfscope}%
\pgfsetbuttcap%
\pgfsetroundjoin%
\definecolor{currentfill}{rgb}{0.000000,0.000000,0.000000}%
\pgfsetfillcolor{currentfill}%
\pgfsetlinewidth{0.501875pt}%
\definecolor{currentstroke}{rgb}{0.000000,0.000000,0.000000}%
\pgfsetstrokecolor{currentstroke}%
\pgfsetdash{}{0pt}%
\pgfsys@defobject{currentmarker}{\pgfqpoint{0.000000in}{0.000000in}}{\pgfqpoint{0.000000in}{0.020833in}}{%
\pgfpathmoveto{\pgfqpoint{0.000000in}{0.000000in}}%
\pgfpathlineto{\pgfqpoint{0.000000in}{0.020833in}}%
\pgfusepath{stroke,fill}%
}%
\begin{pgfscope}%
\pgfsys@transformshift{4.624818in}{0.422992in}%
\pgfsys@useobject{currentmarker}{}%
\end{pgfscope}%
\end{pgfscope}%
\begin{pgfscope}%
\pgfsetbuttcap%
\pgfsetroundjoin%
\definecolor{currentfill}{rgb}{0.000000,0.000000,0.000000}%
\pgfsetfillcolor{currentfill}%
\pgfsetlinewidth{0.501875pt}%
\definecolor{currentstroke}{rgb}{0.000000,0.000000,0.000000}%
\pgfsetstrokecolor{currentstroke}%
\pgfsetdash{}{0pt}%
\pgfsys@defobject{currentmarker}{\pgfqpoint{0.000000in}{-0.020833in}}{\pgfqpoint{0.000000in}{0.000000in}}{%
\pgfpathmoveto{\pgfqpoint{0.000000in}{0.000000in}}%
\pgfpathlineto{\pgfqpoint{0.000000in}{-0.020833in}}%
\pgfusepath{stroke,fill}%
}%
\begin{pgfscope}%
\pgfsys@transformshift{4.624818in}{4.374193in}%
\pgfsys@useobject{currentmarker}{}%
\end{pgfscope}%
\end{pgfscope}%
\begin{pgfscope}%
\pgfsetbuttcap%
\pgfsetroundjoin%
\definecolor{currentfill}{rgb}{0.000000,0.000000,0.000000}%
\pgfsetfillcolor{currentfill}%
\pgfsetlinewidth{0.501875pt}%
\definecolor{currentstroke}{rgb}{0.000000,0.000000,0.000000}%
\pgfsetstrokecolor{currentstroke}%
\pgfsetdash{}{0pt}%
\pgfsys@defobject{currentmarker}{\pgfqpoint{0.000000in}{0.000000in}}{\pgfqpoint{0.000000in}{0.020833in}}{%
\pgfpathmoveto{\pgfqpoint{0.000000in}{0.000000in}}%
\pgfpathlineto{\pgfqpoint{0.000000in}{0.020833in}}%
\pgfusepath{stroke,fill}%
}%
\begin{pgfscope}%
\pgfsys@transformshift{4.850171in}{0.422992in}%
\pgfsys@useobject{currentmarker}{}%
\end{pgfscope}%
\end{pgfscope}%
\begin{pgfscope}%
\pgfsetbuttcap%
\pgfsetroundjoin%
\definecolor{currentfill}{rgb}{0.000000,0.000000,0.000000}%
\pgfsetfillcolor{currentfill}%
\pgfsetlinewidth{0.501875pt}%
\definecolor{currentstroke}{rgb}{0.000000,0.000000,0.000000}%
\pgfsetstrokecolor{currentstroke}%
\pgfsetdash{}{0pt}%
\pgfsys@defobject{currentmarker}{\pgfqpoint{0.000000in}{-0.020833in}}{\pgfqpoint{0.000000in}{0.000000in}}{%
\pgfpathmoveto{\pgfqpoint{0.000000in}{0.000000in}}%
\pgfpathlineto{\pgfqpoint{0.000000in}{-0.020833in}}%
\pgfusepath{stroke,fill}%
}%
\begin{pgfscope}%
\pgfsys@transformshift{4.850171in}{4.374193in}%
\pgfsys@useobject{currentmarker}{}%
\end{pgfscope}%
\end{pgfscope}%
\begin{pgfscope}%
\pgfsetbuttcap%
\pgfsetroundjoin%
\definecolor{currentfill}{rgb}{0.000000,0.000000,0.000000}%
\pgfsetfillcolor{currentfill}%
\pgfsetlinewidth{0.501875pt}%
\definecolor{currentstroke}{rgb}{0.000000,0.000000,0.000000}%
\pgfsetstrokecolor{currentstroke}%
\pgfsetdash{}{0pt}%
\pgfsys@defobject{currentmarker}{\pgfqpoint{0.000000in}{0.000000in}}{\pgfqpoint{0.000000in}{0.020833in}}{%
\pgfpathmoveto{\pgfqpoint{0.000000in}{0.000000in}}%
\pgfpathlineto{\pgfqpoint{0.000000in}{0.020833in}}%
\pgfusepath{stroke,fill}%
}%
\begin{pgfscope}%
\pgfsys@transformshift{4.962847in}{0.422992in}%
\pgfsys@useobject{currentmarker}{}%
\end{pgfscope}%
\end{pgfscope}%
\begin{pgfscope}%
\pgfsetbuttcap%
\pgfsetroundjoin%
\definecolor{currentfill}{rgb}{0.000000,0.000000,0.000000}%
\pgfsetfillcolor{currentfill}%
\pgfsetlinewidth{0.501875pt}%
\definecolor{currentstroke}{rgb}{0.000000,0.000000,0.000000}%
\pgfsetstrokecolor{currentstroke}%
\pgfsetdash{}{0pt}%
\pgfsys@defobject{currentmarker}{\pgfqpoint{0.000000in}{-0.020833in}}{\pgfqpoint{0.000000in}{0.000000in}}{%
\pgfpathmoveto{\pgfqpoint{0.000000in}{0.000000in}}%
\pgfpathlineto{\pgfqpoint{0.000000in}{-0.020833in}}%
\pgfusepath{stroke,fill}%
}%
\begin{pgfscope}%
\pgfsys@transformshift{4.962847in}{4.374193in}%
\pgfsys@useobject{currentmarker}{}%
\end{pgfscope}%
\end{pgfscope}%
\begin{pgfscope}%
\pgfsetbuttcap%
\pgfsetroundjoin%
\definecolor{currentfill}{rgb}{0.000000,0.000000,0.000000}%
\pgfsetfillcolor{currentfill}%
\pgfsetlinewidth{0.501875pt}%
\definecolor{currentstroke}{rgb}{0.000000,0.000000,0.000000}%
\pgfsetstrokecolor{currentstroke}%
\pgfsetdash{}{0pt}%
\pgfsys@defobject{currentmarker}{\pgfqpoint{0.000000in}{0.000000in}}{\pgfqpoint{0.000000in}{0.020833in}}{%
\pgfpathmoveto{\pgfqpoint{0.000000in}{0.000000in}}%
\pgfpathlineto{\pgfqpoint{0.000000in}{0.020833in}}%
\pgfusepath{stroke,fill}%
}%
\begin{pgfscope}%
\pgfsys@transformshift{5.075524in}{0.422992in}%
\pgfsys@useobject{currentmarker}{}%
\end{pgfscope}%
\end{pgfscope}%
\begin{pgfscope}%
\pgfsetbuttcap%
\pgfsetroundjoin%
\definecolor{currentfill}{rgb}{0.000000,0.000000,0.000000}%
\pgfsetfillcolor{currentfill}%
\pgfsetlinewidth{0.501875pt}%
\definecolor{currentstroke}{rgb}{0.000000,0.000000,0.000000}%
\pgfsetstrokecolor{currentstroke}%
\pgfsetdash{}{0pt}%
\pgfsys@defobject{currentmarker}{\pgfqpoint{0.000000in}{-0.020833in}}{\pgfqpoint{0.000000in}{0.000000in}}{%
\pgfpathmoveto{\pgfqpoint{0.000000in}{0.000000in}}%
\pgfpathlineto{\pgfqpoint{0.000000in}{-0.020833in}}%
\pgfusepath{stroke,fill}%
}%
\begin{pgfscope}%
\pgfsys@transformshift{5.075524in}{4.374193in}%
\pgfsys@useobject{currentmarker}{}%
\end{pgfscope}%
\end{pgfscope}%
\begin{pgfscope}%
\pgfsetbuttcap%
\pgfsetroundjoin%
\definecolor{currentfill}{rgb}{0.000000,0.000000,0.000000}%
\pgfsetfillcolor{currentfill}%
\pgfsetlinewidth{0.501875pt}%
\definecolor{currentstroke}{rgb}{0.000000,0.000000,0.000000}%
\pgfsetstrokecolor{currentstroke}%
\pgfsetdash{}{0pt}%
\pgfsys@defobject{currentmarker}{\pgfqpoint{0.000000in}{0.000000in}}{\pgfqpoint{0.000000in}{0.020833in}}{%
\pgfpathmoveto{\pgfqpoint{0.000000in}{0.000000in}}%
\pgfpathlineto{\pgfqpoint{0.000000in}{0.020833in}}%
\pgfusepath{stroke,fill}%
}%
\begin{pgfscope}%
\pgfsys@transformshift{5.300877in}{0.422992in}%
\pgfsys@useobject{currentmarker}{}%
\end{pgfscope}%
\end{pgfscope}%
\begin{pgfscope}%
\pgfsetbuttcap%
\pgfsetroundjoin%
\definecolor{currentfill}{rgb}{0.000000,0.000000,0.000000}%
\pgfsetfillcolor{currentfill}%
\pgfsetlinewidth{0.501875pt}%
\definecolor{currentstroke}{rgb}{0.000000,0.000000,0.000000}%
\pgfsetstrokecolor{currentstroke}%
\pgfsetdash{}{0pt}%
\pgfsys@defobject{currentmarker}{\pgfqpoint{0.000000in}{-0.020833in}}{\pgfqpoint{0.000000in}{0.000000in}}{%
\pgfpathmoveto{\pgfqpoint{0.000000in}{0.000000in}}%
\pgfpathlineto{\pgfqpoint{0.000000in}{-0.020833in}}%
\pgfusepath{stroke,fill}%
}%
\begin{pgfscope}%
\pgfsys@transformshift{5.300877in}{4.374193in}%
\pgfsys@useobject{currentmarker}{}%
\end{pgfscope}%
\end{pgfscope}%
\begin{pgfscope}%
\pgfsetbuttcap%
\pgfsetroundjoin%
\definecolor{currentfill}{rgb}{0.000000,0.000000,0.000000}%
\pgfsetfillcolor{currentfill}%
\pgfsetlinewidth{0.501875pt}%
\definecolor{currentstroke}{rgb}{0.000000,0.000000,0.000000}%
\pgfsetstrokecolor{currentstroke}%
\pgfsetdash{}{0pt}%
\pgfsys@defobject{currentmarker}{\pgfqpoint{0.000000in}{0.000000in}}{\pgfqpoint{0.000000in}{0.020833in}}{%
\pgfpathmoveto{\pgfqpoint{0.000000in}{0.000000in}}%
\pgfpathlineto{\pgfqpoint{0.000000in}{0.020833in}}%
\pgfusepath{stroke,fill}%
}%
\begin{pgfscope}%
\pgfsys@transformshift{5.413553in}{0.422992in}%
\pgfsys@useobject{currentmarker}{}%
\end{pgfscope}%
\end{pgfscope}%
\begin{pgfscope}%
\pgfsetbuttcap%
\pgfsetroundjoin%
\definecolor{currentfill}{rgb}{0.000000,0.000000,0.000000}%
\pgfsetfillcolor{currentfill}%
\pgfsetlinewidth{0.501875pt}%
\definecolor{currentstroke}{rgb}{0.000000,0.000000,0.000000}%
\pgfsetstrokecolor{currentstroke}%
\pgfsetdash{}{0pt}%
\pgfsys@defobject{currentmarker}{\pgfqpoint{0.000000in}{-0.020833in}}{\pgfqpoint{0.000000in}{0.000000in}}{%
\pgfpathmoveto{\pgfqpoint{0.000000in}{0.000000in}}%
\pgfpathlineto{\pgfqpoint{0.000000in}{-0.020833in}}%
\pgfusepath{stroke,fill}%
}%
\begin{pgfscope}%
\pgfsys@transformshift{5.413553in}{4.374193in}%
\pgfsys@useobject{currentmarker}{}%
\end{pgfscope}%
\end{pgfscope}%
\begin{pgfscope}%
\pgfsetbuttcap%
\pgfsetroundjoin%
\definecolor{currentfill}{rgb}{0.000000,0.000000,0.000000}%
\pgfsetfillcolor{currentfill}%
\pgfsetlinewidth{0.501875pt}%
\definecolor{currentstroke}{rgb}{0.000000,0.000000,0.000000}%
\pgfsetstrokecolor{currentstroke}%
\pgfsetdash{}{0pt}%
\pgfsys@defobject{currentmarker}{\pgfqpoint{0.000000in}{0.000000in}}{\pgfqpoint{0.000000in}{0.020833in}}{%
\pgfpathmoveto{\pgfqpoint{0.000000in}{0.000000in}}%
\pgfpathlineto{\pgfqpoint{0.000000in}{0.020833in}}%
\pgfusepath{stroke,fill}%
}%
\begin{pgfscope}%
\pgfsys@transformshift{5.526230in}{0.422992in}%
\pgfsys@useobject{currentmarker}{}%
\end{pgfscope}%
\end{pgfscope}%
\begin{pgfscope}%
\pgfsetbuttcap%
\pgfsetroundjoin%
\definecolor{currentfill}{rgb}{0.000000,0.000000,0.000000}%
\pgfsetfillcolor{currentfill}%
\pgfsetlinewidth{0.501875pt}%
\definecolor{currentstroke}{rgb}{0.000000,0.000000,0.000000}%
\pgfsetstrokecolor{currentstroke}%
\pgfsetdash{}{0pt}%
\pgfsys@defobject{currentmarker}{\pgfqpoint{0.000000in}{-0.020833in}}{\pgfqpoint{0.000000in}{0.000000in}}{%
\pgfpathmoveto{\pgfqpoint{0.000000in}{0.000000in}}%
\pgfpathlineto{\pgfqpoint{0.000000in}{-0.020833in}}%
\pgfusepath{stroke,fill}%
}%
\begin{pgfscope}%
\pgfsys@transformshift{5.526230in}{4.374193in}%
\pgfsys@useobject{currentmarker}{}%
\end{pgfscope}%
\end{pgfscope}%
\begin{pgfscope}%
\pgfsetbuttcap%
\pgfsetroundjoin%
\definecolor{currentfill}{rgb}{0.000000,0.000000,0.000000}%
\pgfsetfillcolor{currentfill}%
\pgfsetlinewidth{0.501875pt}%
\definecolor{currentstroke}{rgb}{0.000000,0.000000,0.000000}%
\pgfsetstrokecolor{currentstroke}%
\pgfsetdash{}{0pt}%
\pgfsys@defobject{currentmarker}{\pgfqpoint{0.000000in}{0.000000in}}{\pgfqpoint{0.000000in}{0.020833in}}{%
\pgfpathmoveto{\pgfqpoint{0.000000in}{0.000000in}}%
\pgfpathlineto{\pgfqpoint{0.000000in}{0.020833in}}%
\pgfusepath{stroke,fill}%
}%
\begin{pgfscope}%
\pgfsys@transformshift{5.638906in}{0.422992in}%
\pgfsys@useobject{currentmarker}{}%
\end{pgfscope}%
\end{pgfscope}%
\begin{pgfscope}%
\pgfsetbuttcap%
\pgfsetroundjoin%
\definecolor{currentfill}{rgb}{0.000000,0.000000,0.000000}%
\pgfsetfillcolor{currentfill}%
\pgfsetlinewidth{0.501875pt}%
\definecolor{currentstroke}{rgb}{0.000000,0.000000,0.000000}%
\pgfsetstrokecolor{currentstroke}%
\pgfsetdash{}{0pt}%
\pgfsys@defobject{currentmarker}{\pgfqpoint{0.000000in}{-0.020833in}}{\pgfqpoint{0.000000in}{0.000000in}}{%
\pgfpathmoveto{\pgfqpoint{0.000000in}{0.000000in}}%
\pgfpathlineto{\pgfqpoint{0.000000in}{-0.020833in}}%
\pgfusepath{stroke,fill}%
}%
\begin{pgfscope}%
\pgfsys@transformshift{5.638906in}{4.374193in}%
\pgfsys@useobject{currentmarker}{}%
\end{pgfscope}%
\end{pgfscope}%
\begin{pgfscope}%
\definecolor{textcolor}{rgb}{0.000000,0.000000,0.000000}%
\pgfsetstrokecolor{textcolor}%
\pgfsetfillcolor{textcolor}%
\pgftext[x=4.523409in,y=0.184413in,,top]{\color{textcolor}\rmfamily\fontsize{10.000000}{12.000000}\selectfont \(\displaystyle K\)}%
\end{pgfscope}%
\begin{pgfscope}%
\pgfsetbuttcap%
\pgfsetroundjoin%
\definecolor{currentfill}{rgb}{0.000000,0.000000,0.000000}%
\pgfsetfillcolor{currentfill}%
\pgfsetlinewidth{0.501875pt}%
\definecolor{currentstroke}{rgb}{0.000000,0.000000,0.000000}%
\pgfsetstrokecolor{currentstroke}%
\pgfsetdash{}{0pt}%
\pgfsys@defobject{currentmarker}{\pgfqpoint{0.000000in}{0.000000in}}{\pgfqpoint{0.041667in}{0.000000in}}{%
\pgfpathmoveto{\pgfqpoint{0.000000in}{0.000000in}}%
\pgfpathlineto{\pgfqpoint{0.041667in}{0.000000in}}%
\pgfusepath{stroke,fill}%
}%
\begin{pgfscope}%
\pgfsys@transformshift{3.385377in}{0.460551in}%
\pgfsys@useobject{currentmarker}{}%
\end{pgfscope}%
\end{pgfscope}%
\begin{pgfscope}%
\pgfsetbuttcap%
\pgfsetroundjoin%
\definecolor{currentfill}{rgb}{0.000000,0.000000,0.000000}%
\pgfsetfillcolor{currentfill}%
\pgfsetlinewidth{0.501875pt}%
\definecolor{currentstroke}{rgb}{0.000000,0.000000,0.000000}%
\pgfsetstrokecolor{currentstroke}%
\pgfsetdash{}{0pt}%
\pgfsys@defobject{currentmarker}{\pgfqpoint{-0.041667in}{0.000000in}}{\pgfqpoint{-0.000000in}{0.000000in}}{%
\pgfpathmoveto{\pgfqpoint{-0.000000in}{0.000000in}}%
\pgfpathlineto{\pgfqpoint{-0.041667in}{0.000000in}}%
\pgfusepath{stroke,fill}%
}%
\begin{pgfscope}%
\pgfsys@transformshift{5.661441in}{0.460551in}%
\pgfsys@useobject{currentmarker}{}%
\end{pgfscope}%
\end{pgfscope}%
\begin{pgfscope}%
\definecolor{textcolor}{rgb}{0.000000,0.000000,0.000000}%
\pgfsetstrokecolor{textcolor}%
\pgfsetfillcolor{textcolor}%
\pgftext[x=3.159296in, y=0.407789in, left, base]{\color{textcolor}\rmfamily\fontsize{10.000000}{12.000000}\selectfont \(\displaystyle {0.0}\)}%
\end{pgfscope}%
\begin{pgfscope}%
\pgfsetbuttcap%
\pgfsetroundjoin%
\definecolor{currentfill}{rgb}{0.000000,0.000000,0.000000}%
\pgfsetfillcolor{currentfill}%
\pgfsetlinewidth{0.501875pt}%
\definecolor{currentstroke}{rgb}{0.000000,0.000000,0.000000}%
\pgfsetstrokecolor{currentstroke}%
\pgfsetdash{}{0pt}%
\pgfsys@defobject{currentmarker}{\pgfqpoint{0.000000in}{0.000000in}}{\pgfqpoint{0.041667in}{0.000000in}}{%
\pgfpathmoveto{\pgfqpoint{0.000000in}{0.000000in}}%
\pgfpathlineto{\pgfqpoint{0.041667in}{0.000000in}}%
\pgfusepath{stroke,fill}%
}%
\begin{pgfscope}%
\pgfsys@transformshift{3.385377in}{0.978240in}%
\pgfsys@useobject{currentmarker}{}%
\end{pgfscope}%
\end{pgfscope}%
\begin{pgfscope}%
\pgfsetbuttcap%
\pgfsetroundjoin%
\definecolor{currentfill}{rgb}{0.000000,0.000000,0.000000}%
\pgfsetfillcolor{currentfill}%
\pgfsetlinewidth{0.501875pt}%
\definecolor{currentstroke}{rgb}{0.000000,0.000000,0.000000}%
\pgfsetstrokecolor{currentstroke}%
\pgfsetdash{}{0pt}%
\pgfsys@defobject{currentmarker}{\pgfqpoint{-0.041667in}{0.000000in}}{\pgfqpoint{-0.000000in}{0.000000in}}{%
\pgfpathmoveto{\pgfqpoint{-0.000000in}{0.000000in}}%
\pgfpathlineto{\pgfqpoint{-0.041667in}{0.000000in}}%
\pgfusepath{stroke,fill}%
}%
\begin{pgfscope}%
\pgfsys@transformshift{5.661441in}{0.978240in}%
\pgfsys@useobject{currentmarker}{}%
\end{pgfscope}%
\end{pgfscope}%
\begin{pgfscope}%
\definecolor{textcolor}{rgb}{0.000000,0.000000,0.000000}%
\pgfsetstrokecolor{textcolor}%
\pgfsetfillcolor{textcolor}%
\pgftext[x=3.159296in, y=0.925479in, left, base]{\color{textcolor}\rmfamily\fontsize{10.000000}{12.000000}\selectfont \(\displaystyle {0.1}\)}%
\end{pgfscope}%
\begin{pgfscope}%
\pgfsetbuttcap%
\pgfsetroundjoin%
\definecolor{currentfill}{rgb}{0.000000,0.000000,0.000000}%
\pgfsetfillcolor{currentfill}%
\pgfsetlinewidth{0.501875pt}%
\definecolor{currentstroke}{rgb}{0.000000,0.000000,0.000000}%
\pgfsetstrokecolor{currentstroke}%
\pgfsetdash{}{0pt}%
\pgfsys@defobject{currentmarker}{\pgfqpoint{0.000000in}{0.000000in}}{\pgfqpoint{0.041667in}{0.000000in}}{%
\pgfpathmoveto{\pgfqpoint{0.000000in}{0.000000in}}%
\pgfpathlineto{\pgfqpoint{0.041667in}{0.000000in}}%
\pgfusepath{stroke,fill}%
}%
\begin{pgfscope}%
\pgfsys@transformshift{3.385377in}{1.495930in}%
\pgfsys@useobject{currentmarker}{}%
\end{pgfscope}%
\end{pgfscope}%
\begin{pgfscope}%
\pgfsetbuttcap%
\pgfsetroundjoin%
\definecolor{currentfill}{rgb}{0.000000,0.000000,0.000000}%
\pgfsetfillcolor{currentfill}%
\pgfsetlinewidth{0.501875pt}%
\definecolor{currentstroke}{rgb}{0.000000,0.000000,0.000000}%
\pgfsetstrokecolor{currentstroke}%
\pgfsetdash{}{0pt}%
\pgfsys@defobject{currentmarker}{\pgfqpoint{-0.041667in}{0.000000in}}{\pgfqpoint{-0.000000in}{0.000000in}}{%
\pgfpathmoveto{\pgfqpoint{-0.000000in}{0.000000in}}%
\pgfpathlineto{\pgfqpoint{-0.041667in}{0.000000in}}%
\pgfusepath{stroke,fill}%
}%
\begin{pgfscope}%
\pgfsys@transformshift{5.661441in}{1.495930in}%
\pgfsys@useobject{currentmarker}{}%
\end{pgfscope}%
\end{pgfscope}%
\begin{pgfscope}%
\definecolor{textcolor}{rgb}{0.000000,0.000000,0.000000}%
\pgfsetstrokecolor{textcolor}%
\pgfsetfillcolor{textcolor}%
\pgftext[x=3.159296in, y=1.443168in, left, base]{\color{textcolor}\rmfamily\fontsize{10.000000}{12.000000}\selectfont \(\displaystyle {0.2}\)}%
\end{pgfscope}%
\begin{pgfscope}%
\pgfsetbuttcap%
\pgfsetroundjoin%
\definecolor{currentfill}{rgb}{0.000000,0.000000,0.000000}%
\pgfsetfillcolor{currentfill}%
\pgfsetlinewidth{0.501875pt}%
\definecolor{currentstroke}{rgb}{0.000000,0.000000,0.000000}%
\pgfsetstrokecolor{currentstroke}%
\pgfsetdash{}{0pt}%
\pgfsys@defobject{currentmarker}{\pgfqpoint{0.000000in}{0.000000in}}{\pgfqpoint{0.041667in}{0.000000in}}{%
\pgfpathmoveto{\pgfqpoint{0.000000in}{0.000000in}}%
\pgfpathlineto{\pgfqpoint{0.041667in}{0.000000in}}%
\pgfusepath{stroke,fill}%
}%
\begin{pgfscope}%
\pgfsys@transformshift{3.385377in}{2.013619in}%
\pgfsys@useobject{currentmarker}{}%
\end{pgfscope}%
\end{pgfscope}%
\begin{pgfscope}%
\pgfsetbuttcap%
\pgfsetroundjoin%
\definecolor{currentfill}{rgb}{0.000000,0.000000,0.000000}%
\pgfsetfillcolor{currentfill}%
\pgfsetlinewidth{0.501875pt}%
\definecolor{currentstroke}{rgb}{0.000000,0.000000,0.000000}%
\pgfsetstrokecolor{currentstroke}%
\pgfsetdash{}{0pt}%
\pgfsys@defobject{currentmarker}{\pgfqpoint{-0.041667in}{0.000000in}}{\pgfqpoint{-0.000000in}{0.000000in}}{%
\pgfpathmoveto{\pgfqpoint{-0.000000in}{0.000000in}}%
\pgfpathlineto{\pgfqpoint{-0.041667in}{0.000000in}}%
\pgfusepath{stroke,fill}%
}%
\begin{pgfscope}%
\pgfsys@transformshift{5.661441in}{2.013619in}%
\pgfsys@useobject{currentmarker}{}%
\end{pgfscope}%
\end{pgfscope}%
\begin{pgfscope}%
\definecolor{textcolor}{rgb}{0.000000,0.000000,0.000000}%
\pgfsetstrokecolor{textcolor}%
\pgfsetfillcolor{textcolor}%
\pgftext[x=3.159296in, y=1.960858in, left, base]{\color{textcolor}\rmfamily\fontsize{10.000000}{12.000000}\selectfont \(\displaystyle {0.3}\)}%
\end{pgfscope}%
\begin{pgfscope}%
\pgfsetbuttcap%
\pgfsetroundjoin%
\definecolor{currentfill}{rgb}{0.000000,0.000000,0.000000}%
\pgfsetfillcolor{currentfill}%
\pgfsetlinewidth{0.501875pt}%
\definecolor{currentstroke}{rgb}{0.000000,0.000000,0.000000}%
\pgfsetstrokecolor{currentstroke}%
\pgfsetdash{}{0pt}%
\pgfsys@defobject{currentmarker}{\pgfqpoint{0.000000in}{0.000000in}}{\pgfqpoint{0.041667in}{0.000000in}}{%
\pgfpathmoveto{\pgfqpoint{0.000000in}{0.000000in}}%
\pgfpathlineto{\pgfqpoint{0.041667in}{0.000000in}}%
\pgfusepath{stroke,fill}%
}%
\begin{pgfscope}%
\pgfsys@transformshift{3.385377in}{2.531309in}%
\pgfsys@useobject{currentmarker}{}%
\end{pgfscope}%
\end{pgfscope}%
\begin{pgfscope}%
\pgfsetbuttcap%
\pgfsetroundjoin%
\definecolor{currentfill}{rgb}{0.000000,0.000000,0.000000}%
\pgfsetfillcolor{currentfill}%
\pgfsetlinewidth{0.501875pt}%
\definecolor{currentstroke}{rgb}{0.000000,0.000000,0.000000}%
\pgfsetstrokecolor{currentstroke}%
\pgfsetdash{}{0pt}%
\pgfsys@defobject{currentmarker}{\pgfqpoint{-0.041667in}{0.000000in}}{\pgfqpoint{-0.000000in}{0.000000in}}{%
\pgfpathmoveto{\pgfqpoint{-0.000000in}{0.000000in}}%
\pgfpathlineto{\pgfqpoint{-0.041667in}{0.000000in}}%
\pgfusepath{stroke,fill}%
}%
\begin{pgfscope}%
\pgfsys@transformshift{5.661441in}{2.531309in}%
\pgfsys@useobject{currentmarker}{}%
\end{pgfscope}%
\end{pgfscope}%
\begin{pgfscope}%
\definecolor{textcolor}{rgb}{0.000000,0.000000,0.000000}%
\pgfsetstrokecolor{textcolor}%
\pgfsetfillcolor{textcolor}%
\pgftext[x=3.159296in, y=2.478547in, left, base]{\color{textcolor}\rmfamily\fontsize{10.000000}{12.000000}\selectfont \(\displaystyle {0.4}\)}%
\end{pgfscope}%
\begin{pgfscope}%
\pgfsetbuttcap%
\pgfsetroundjoin%
\definecolor{currentfill}{rgb}{0.000000,0.000000,0.000000}%
\pgfsetfillcolor{currentfill}%
\pgfsetlinewidth{0.501875pt}%
\definecolor{currentstroke}{rgb}{0.000000,0.000000,0.000000}%
\pgfsetstrokecolor{currentstroke}%
\pgfsetdash{}{0pt}%
\pgfsys@defobject{currentmarker}{\pgfqpoint{0.000000in}{0.000000in}}{\pgfqpoint{0.041667in}{0.000000in}}{%
\pgfpathmoveto{\pgfqpoint{0.000000in}{0.000000in}}%
\pgfpathlineto{\pgfqpoint{0.041667in}{0.000000in}}%
\pgfusepath{stroke,fill}%
}%
\begin{pgfscope}%
\pgfsys@transformshift{3.385377in}{3.048998in}%
\pgfsys@useobject{currentmarker}{}%
\end{pgfscope}%
\end{pgfscope}%
\begin{pgfscope}%
\pgfsetbuttcap%
\pgfsetroundjoin%
\definecolor{currentfill}{rgb}{0.000000,0.000000,0.000000}%
\pgfsetfillcolor{currentfill}%
\pgfsetlinewidth{0.501875pt}%
\definecolor{currentstroke}{rgb}{0.000000,0.000000,0.000000}%
\pgfsetstrokecolor{currentstroke}%
\pgfsetdash{}{0pt}%
\pgfsys@defobject{currentmarker}{\pgfqpoint{-0.041667in}{0.000000in}}{\pgfqpoint{-0.000000in}{0.000000in}}{%
\pgfpathmoveto{\pgfqpoint{-0.000000in}{0.000000in}}%
\pgfpathlineto{\pgfqpoint{-0.041667in}{0.000000in}}%
\pgfusepath{stroke,fill}%
}%
\begin{pgfscope}%
\pgfsys@transformshift{5.661441in}{3.048998in}%
\pgfsys@useobject{currentmarker}{}%
\end{pgfscope}%
\end{pgfscope}%
\begin{pgfscope}%
\definecolor{textcolor}{rgb}{0.000000,0.000000,0.000000}%
\pgfsetstrokecolor{textcolor}%
\pgfsetfillcolor{textcolor}%
\pgftext[x=3.159296in, y=2.996237in, left, base]{\color{textcolor}\rmfamily\fontsize{10.000000}{12.000000}\selectfont \(\displaystyle {0.5}\)}%
\end{pgfscope}%
\begin{pgfscope}%
\pgfsetbuttcap%
\pgfsetroundjoin%
\definecolor{currentfill}{rgb}{0.000000,0.000000,0.000000}%
\pgfsetfillcolor{currentfill}%
\pgfsetlinewidth{0.501875pt}%
\definecolor{currentstroke}{rgb}{0.000000,0.000000,0.000000}%
\pgfsetstrokecolor{currentstroke}%
\pgfsetdash{}{0pt}%
\pgfsys@defobject{currentmarker}{\pgfqpoint{0.000000in}{0.000000in}}{\pgfqpoint{0.041667in}{0.000000in}}{%
\pgfpathmoveto{\pgfqpoint{0.000000in}{0.000000in}}%
\pgfpathlineto{\pgfqpoint{0.041667in}{0.000000in}}%
\pgfusepath{stroke,fill}%
}%
\begin{pgfscope}%
\pgfsys@transformshift{3.385377in}{3.566688in}%
\pgfsys@useobject{currentmarker}{}%
\end{pgfscope}%
\end{pgfscope}%
\begin{pgfscope}%
\pgfsetbuttcap%
\pgfsetroundjoin%
\definecolor{currentfill}{rgb}{0.000000,0.000000,0.000000}%
\pgfsetfillcolor{currentfill}%
\pgfsetlinewidth{0.501875pt}%
\definecolor{currentstroke}{rgb}{0.000000,0.000000,0.000000}%
\pgfsetstrokecolor{currentstroke}%
\pgfsetdash{}{0pt}%
\pgfsys@defobject{currentmarker}{\pgfqpoint{-0.041667in}{0.000000in}}{\pgfqpoint{-0.000000in}{0.000000in}}{%
\pgfpathmoveto{\pgfqpoint{-0.000000in}{0.000000in}}%
\pgfpathlineto{\pgfqpoint{-0.041667in}{0.000000in}}%
\pgfusepath{stroke,fill}%
}%
\begin{pgfscope}%
\pgfsys@transformshift{5.661441in}{3.566688in}%
\pgfsys@useobject{currentmarker}{}%
\end{pgfscope}%
\end{pgfscope}%
\begin{pgfscope}%
\definecolor{textcolor}{rgb}{0.000000,0.000000,0.000000}%
\pgfsetstrokecolor{textcolor}%
\pgfsetfillcolor{textcolor}%
\pgftext[x=3.159296in, y=3.513926in, left, base]{\color{textcolor}\rmfamily\fontsize{10.000000}{12.000000}\selectfont \(\displaystyle {0.6}\)}%
\end{pgfscope}%
\begin{pgfscope}%
\pgfsetbuttcap%
\pgfsetroundjoin%
\definecolor{currentfill}{rgb}{0.000000,0.000000,0.000000}%
\pgfsetfillcolor{currentfill}%
\pgfsetlinewidth{0.501875pt}%
\definecolor{currentstroke}{rgb}{0.000000,0.000000,0.000000}%
\pgfsetstrokecolor{currentstroke}%
\pgfsetdash{}{0pt}%
\pgfsys@defobject{currentmarker}{\pgfqpoint{0.000000in}{0.000000in}}{\pgfqpoint{0.041667in}{0.000000in}}{%
\pgfpathmoveto{\pgfqpoint{0.000000in}{0.000000in}}%
\pgfpathlineto{\pgfqpoint{0.041667in}{0.000000in}}%
\pgfusepath{stroke,fill}%
}%
\begin{pgfscope}%
\pgfsys@transformshift{3.385377in}{4.084377in}%
\pgfsys@useobject{currentmarker}{}%
\end{pgfscope}%
\end{pgfscope}%
\begin{pgfscope}%
\pgfsetbuttcap%
\pgfsetroundjoin%
\definecolor{currentfill}{rgb}{0.000000,0.000000,0.000000}%
\pgfsetfillcolor{currentfill}%
\pgfsetlinewidth{0.501875pt}%
\definecolor{currentstroke}{rgb}{0.000000,0.000000,0.000000}%
\pgfsetstrokecolor{currentstroke}%
\pgfsetdash{}{0pt}%
\pgfsys@defobject{currentmarker}{\pgfqpoint{-0.041667in}{0.000000in}}{\pgfqpoint{-0.000000in}{0.000000in}}{%
\pgfpathmoveto{\pgfqpoint{-0.000000in}{0.000000in}}%
\pgfpathlineto{\pgfqpoint{-0.041667in}{0.000000in}}%
\pgfusepath{stroke,fill}%
}%
\begin{pgfscope}%
\pgfsys@transformshift{5.661441in}{4.084377in}%
\pgfsys@useobject{currentmarker}{}%
\end{pgfscope}%
\end{pgfscope}%
\begin{pgfscope}%
\definecolor{textcolor}{rgb}{0.000000,0.000000,0.000000}%
\pgfsetstrokecolor{textcolor}%
\pgfsetfillcolor{textcolor}%
\pgftext[x=3.159296in, y=4.031616in, left, base]{\color{textcolor}\rmfamily\fontsize{10.000000}{12.000000}\selectfont \(\displaystyle {0.7}\)}%
\end{pgfscope}%
\begin{pgfscope}%
\pgfsetbuttcap%
\pgfsetroundjoin%
\definecolor{currentfill}{rgb}{0.000000,0.000000,0.000000}%
\pgfsetfillcolor{currentfill}%
\pgfsetlinewidth{0.501875pt}%
\definecolor{currentstroke}{rgb}{0.000000,0.000000,0.000000}%
\pgfsetstrokecolor{currentstroke}%
\pgfsetdash{}{0pt}%
\pgfsys@defobject{currentmarker}{\pgfqpoint{0.000000in}{0.000000in}}{\pgfqpoint{0.020833in}{0.000000in}}{%
\pgfpathmoveto{\pgfqpoint{0.000000in}{0.000000in}}%
\pgfpathlineto{\pgfqpoint{0.020833in}{0.000000in}}%
\pgfusepath{stroke,fill}%
}%
\begin{pgfscope}%
\pgfsys@transformshift{3.385377in}{0.564089in}%
\pgfsys@useobject{currentmarker}{}%
\end{pgfscope}%
\end{pgfscope}%
\begin{pgfscope}%
\pgfsetbuttcap%
\pgfsetroundjoin%
\definecolor{currentfill}{rgb}{0.000000,0.000000,0.000000}%
\pgfsetfillcolor{currentfill}%
\pgfsetlinewidth{0.501875pt}%
\definecolor{currentstroke}{rgb}{0.000000,0.000000,0.000000}%
\pgfsetstrokecolor{currentstroke}%
\pgfsetdash{}{0pt}%
\pgfsys@defobject{currentmarker}{\pgfqpoint{-0.020833in}{0.000000in}}{\pgfqpoint{-0.000000in}{0.000000in}}{%
\pgfpathmoveto{\pgfqpoint{-0.000000in}{0.000000in}}%
\pgfpathlineto{\pgfqpoint{-0.020833in}{0.000000in}}%
\pgfusepath{stroke,fill}%
}%
\begin{pgfscope}%
\pgfsys@transformshift{5.661441in}{0.564089in}%
\pgfsys@useobject{currentmarker}{}%
\end{pgfscope}%
\end{pgfscope}%
\begin{pgfscope}%
\pgfsetbuttcap%
\pgfsetroundjoin%
\definecolor{currentfill}{rgb}{0.000000,0.000000,0.000000}%
\pgfsetfillcolor{currentfill}%
\pgfsetlinewidth{0.501875pt}%
\definecolor{currentstroke}{rgb}{0.000000,0.000000,0.000000}%
\pgfsetstrokecolor{currentstroke}%
\pgfsetdash{}{0pt}%
\pgfsys@defobject{currentmarker}{\pgfqpoint{0.000000in}{0.000000in}}{\pgfqpoint{0.020833in}{0.000000in}}{%
\pgfpathmoveto{\pgfqpoint{0.000000in}{0.000000in}}%
\pgfpathlineto{\pgfqpoint{0.020833in}{0.000000in}}%
\pgfusepath{stroke,fill}%
}%
\begin{pgfscope}%
\pgfsys@transformshift{3.385377in}{0.667627in}%
\pgfsys@useobject{currentmarker}{}%
\end{pgfscope}%
\end{pgfscope}%
\begin{pgfscope}%
\pgfsetbuttcap%
\pgfsetroundjoin%
\definecolor{currentfill}{rgb}{0.000000,0.000000,0.000000}%
\pgfsetfillcolor{currentfill}%
\pgfsetlinewidth{0.501875pt}%
\definecolor{currentstroke}{rgb}{0.000000,0.000000,0.000000}%
\pgfsetstrokecolor{currentstroke}%
\pgfsetdash{}{0pt}%
\pgfsys@defobject{currentmarker}{\pgfqpoint{-0.020833in}{0.000000in}}{\pgfqpoint{-0.000000in}{0.000000in}}{%
\pgfpathmoveto{\pgfqpoint{-0.000000in}{0.000000in}}%
\pgfpathlineto{\pgfqpoint{-0.020833in}{0.000000in}}%
\pgfusepath{stroke,fill}%
}%
\begin{pgfscope}%
\pgfsys@transformshift{5.661441in}{0.667627in}%
\pgfsys@useobject{currentmarker}{}%
\end{pgfscope}%
\end{pgfscope}%
\begin{pgfscope}%
\pgfsetbuttcap%
\pgfsetroundjoin%
\definecolor{currentfill}{rgb}{0.000000,0.000000,0.000000}%
\pgfsetfillcolor{currentfill}%
\pgfsetlinewidth{0.501875pt}%
\definecolor{currentstroke}{rgb}{0.000000,0.000000,0.000000}%
\pgfsetstrokecolor{currentstroke}%
\pgfsetdash{}{0pt}%
\pgfsys@defobject{currentmarker}{\pgfqpoint{0.000000in}{0.000000in}}{\pgfqpoint{0.020833in}{0.000000in}}{%
\pgfpathmoveto{\pgfqpoint{0.000000in}{0.000000in}}%
\pgfpathlineto{\pgfqpoint{0.020833in}{0.000000in}}%
\pgfusepath{stroke,fill}%
}%
\begin{pgfscope}%
\pgfsys@transformshift{3.385377in}{0.771165in}%
\pgfsys@useobject{currentmarker}{}%
\end{pgfscope}%
\end{pgfscope}%
\begin{pgfscope}%
\pgfsetbuttcap%
\pgfsetroundjoin%
\definecolor{currentfill}{rgb}{0.000000,0.000000,0.000000}%
\pgfsetfillcolor{currentfill}%
\pgfsetlinewidth{0.501875pt}%
\definecolor{currentstroke}{rgb}{0.000000,0.000000,0.000000}%
\pgfsetstrokecolor{currentstroke}%
\pgfsetdash{}{0pt}%
\pgfsys@defobject{currentmarker}{\pgfqpoint{-0.020833in}{0.000000in}}{\pgfqpoint{-0.000000in}{0.000000in}}{%
\pgfpathmoveto{\pgfqpoint{-0.000000in}{0.000000in}}%
\pgfpathlineto{\pgfqpoint{-0.020833in}{0.000000in}}%
\pgfusepath{stroke,fill}%
}%
\begin{pgfscope}%
\pgfsys@transformshift{5.661441in}{0.771165in}%
\pgfsys@useobject{currentmarker}{}%
\end{pgfscope}%
\end{pgfscope}%
\begin{pgfscope}%
\pgfsetbuttcap%
\pgfsetroundjoin%
\definecolor{currentfill}{rgb}{0.000000,0.000000,0.000000}%
\pgfsetfillcolor{currentfill}%
\pgfsetlinewidth{0.501875pt}%
\definecolor{currentstroke}{rgb}{0.000000,0.000000,0.000000}%
\pgfsetstrokecolor{currentstroke}%
\pgfsetdash{}{0pt}%
\pgfsys@defobject{currentmarker}{\pgfqpoint{0.000000in}{0.000000in}}{\pgfqpoint{0.020833in}{0.000000in}}{%
\pgfpathmoveto{\pgfqpoint{0.000000in}{0.000000in}}%
\pgfpathlineto{\pgfqpoint{0.020833in}{0.000000in}}%
\pgfusepath{stroke,fill}%
}%
\begin{pgfscope}%
\pgfsys@transformshift{3.385377in}{0.874702in}%
\pgfsys@useobject{currentmarker}{}%
\end{pgfscope}%
\end{pgfscope}%
\begin{pgfscope}%
\pgfsetbuttcap%
\pgfsetroundjoin%
\definecolor{currentfill}{rgb}{0.000000,0.000000,0.000000}%
\pgfsetfillcolor{currentfill}%
\pgfsetlinewidth{0.501875pt}%
\definecolor{currentstroke}{rgb}{0.000000,0.000000,0.000000}%
\pgfsetstrokecolor{currentstroke}%
\pgfsetdash{}{0pt}%
\pgfsys@defobject{currentmarker}{\pgfqpoint{-0.020833in}{0.000000in}}{\pgfqpoint{-0.000000in}{0.000000in}}{%
\pgfpathmoveto{\pgfqpoint{-0.000000in}{0.000000in}}%
\pgfpathlineto{\pgfqpoint{-0.020833in}{0.000000in}}%
\pgfusepath{stroke,fill}%
}%
\begin{pgfscope}%
\pgfsys@transformshift{5.661441in}{0.874702in}%
\pgfsys@useobject{currentmarker}{}%
\end{pgfscope}%
\end{pgfscope}%
\begin{pgfscope}%
\pgfsetbuttcap%
\pgfsetroundjoin%
\definecolor{currentfill}{rgb}{0.000000,0.000000,0.000000}%
\pgfsetfillcolor{currentfill}%
\pgfsetlinewidth{0.501875pt}%
\definecolor{currentstroke}{rgb}{0.000000,0.000000,0.000000}%
\pgfsetstrokecolor{currentstroke}%
\pgfsetdash{}{0pt}%
\pgfsys@defobject{currentmarker}{\pgfqpoint{0.000000in}{0.000000in}}{\pgfqpoint{0.020833in}{0.000000in}}{%
\pgfpathmoveto{\pgfqpoint{0.000000in}{0.000000in}}%
\pgfpathlineto{\pgfqpoint{0.020833in}{0.000000in}}%
\pgfusepath{stroke,fill}%
}%
\begin{pgfscope}%
\pgfsys@transformshift{3.385377in}{1.081778in}%
\pgfsys@useobject{currentmarker}{}%
\end{pgfscope}%
\end{pgfscope}%
\begin{pgfscope}%
\pgfsetbuttcap%
\pgfsetroundjoin%
\definecolor{currentfill}{rgb}{0.000000,0.000000,0.000000}%
\pgfsetfillcolor{currentfill}%
\pgfsetlinewidth{0.501875pt}%
\definecolor{currentstroke}{rgb}{0.000000,0.000000,0.000000}%
\pgfsetstrokecolor{currentstroke}%
\pgfsetdash{}{0pt}%
\pgfsys@defobject{currentmarker}{\pgfqpoint{-0.020833in}{0.000000in}}{\pgfqpoint{-0.000000in}{0.000000in}}{%
\pgfpathmoveto{\pgfqpoint{-0.000000in}{0.000000in}}%
\pgfpathlineto{\pgfqpoint{-0.020833in}{0.000000in}}%
\pgfusepath{stroke,fill}%
}%
\begin{pgfscope}%
\pgfsys@transformshift{5.661441in}{1.081778in}%
\pgfsys@useobject{currentmarker}{}%
\end{pgfscope}%
\end{pgfscope}%
\begin{pgfscope}%
\pgfsetbuttcap%
\pgfsetroundjoin%
\definecolor{currentfill}{rgb}{0.000000,0.000000,0.000000}%
\pgfsetfillcolor{currentfill}%
\pgfsetlinewidth{0.501875pt}%
\definecolor{currentstroke}{rgb}{0.000000,0.000000,0.000000}%
\pgfsetstrokecolor{currentstroke}%
\pgfsetdash{}{0pt}%
\pgfsys@defobject{currentmarker}{\pgfqpoint{0.000000in}{0.000000in}}{\pgfqpoint{0.020833in}{0.000000in}}{%
\pgfpathmoveto{\pgfqpoint{0.000000in}{0.000000in}}%
\pgfpathlineto{\pgfqpoint{0.020833in}{0.000000in}}%
\pgfusepath{stroke,fill}%
}%
\begin{pgfscope}%
\pgfsys@transformshift{3.385377in}{1.185316in}%
\pgfsys@useobject{currentmarker}{}%
\end{pgfscope}%
\end{pgfscope}%
\begin{pgfscope}%
\pgfsetbuttcap%
\pgfsetroundjoin%
\definecolor{currentfill}{rgb}{0.000000,0.000000,0.000000}%
\pgfsetfillcolor{currentfill}%
\pgfsetlinewidth{0.501875pt}%
\definecolor{currentstroke}{rgb}{0.000000,0.000000,0.000000}%
\pgfsetstrokecolor{currentstroke}%
\pgfsetdash{}{0pt}%
\pgfsys@defobject{currentmarker}{\pgfqpoint{-0.020833in}{0.000000in}}{\pgfqpoint{-0.000000in}{0.000000in}}{%
\pgfpathmoveto{\pgfqpoint{-0.000000in}{0.000000in}}%
\pgfpathlineto{\pgfqpoint{-0.020833in}{0.000000in}}%
\pgfusepath{stroke,fill}%
}%
\begin{pgfscope}%
\pgfsys@transformshift{5.661441in}{1.185316in}%
\pgfsys@useobject{currentmarker}{}%
\end{pgfscope}%
\end{pgfscope}%
\begin{pgfscope}%
\pgfsetbuttcap%
\pgfsetroundjoin%
\definecolor{currentfill}{rgb}{0.000000,0.000000,0.000000}%
\pgfsetfillcolor{currentfill}%
\pgfsetlinewidth{0.501875pt}%
\definecolor{currentstroke}{rgb}{0.000000,0.000000,0.000000}%
\pgfsetstrokecolor{currentstroke}%
\pgfsetdash{}{0pt}%
\pgfsys@defobject{currentmarker}{\pgfqpoint{0.000000in}{0.000000in}}{\pgfqpoint{0.020833in}{0.000000in}}{%
\pgfpathmoveto{\pgfqpoint{0.000000in}{0.000000in}}%
\pgfpathlineto{\pgfqpoint{0.020833in}{0.000000in}}%
\pgfusepath{stroke,fill}%
}%
\begin{pgfscope}%
\pgfsys@transformshift{3.385377in}{1.288854in}%
\pgfsys@useobject{currentmarker}{}%
\end{pgfscope}%
\end{pgfscope}%
\begin{pgfscope}%
\pgfsetbuttcap%
\pgfsetroundjoin%
\definecolor{currentfill}{rgb}{0.000000,0.000000,0.000000}%
\pgfsetfillcolor{currentfill}%
\pgfsetlinewidth{0.501875pt}%
\definecolor{currentstroke}{rgb}{0.000000,0.000000,0.000000}%
\pgfsetstrokecolor{currentstroke}%
\pgfsetdash{}{0pt}%
\pgfsys@defobject{currentmarker}{\pgfqpoint{-0.020833in}{0.000000in}}{\pgfqpoint{-0.000000in}{0.000000in}}{%
\pgfpathmoveto{\pgfqpoint{-0.000000in}{0.000000in}}%
\pgfpathlineto{\pgfqpoint{-0.020833in}{0.000000in}}%
\pgfusepath{stroke,fill}%
}%
\begin{pgfscope}%
\pgfsys@transformshift{5.661441in}{1.288854in}%
\pgfsys@useobject{currentmarker}{}%
\end{pgfscope}%
\end{pgfscope}%
\begin{pgfscope}%
\pgfsetbuttcap%
\pgfsetroundjoin%
\definecolor{currentfill}{rgb}{0.000000,0.000000,0.000000}%
\pgfsetfillcolor{currentfill}%
\pgfsetlinewidth{0.501875pt}%
\definecolor{currentstroke}{rgb}{0.000000,0.000000,0.000000}%
\pgfsetstrokecolor{currentstroke}%
\pgfsetdash{}{0pt}%
\pgfsys@defobject{currentmarker}{\pgfqpoint{0.000000in}{0.000000in}}{\pgfqpoint{0.020833in}{0.000000in}}{%
\pgfpathmoveto{\pgfqpoint{0.000000in}{0.000000in}}%
\pgfpathlineto{\pgfqpoint{0.020833in}{0.000000in}}%
\pgfusepath{stroke,fill}%
}%
\begin{pgfscope}%
\pgfsys@transformshift{3.385377in}{1.392392in}%
\pgfsys@useobject{currentmarker}{}%
\end{pgfscope}%
\end{pgfscope}%
\begin{pgfscope}%
\pgfsetbuttcap%
\pgfsetroundjoin%
\definecolor{currentfill}{rgb}{0.000000,0.000000,0.000000}%
\pgfsetfillcolor{currentfill}%
\pgfsetlinewidth{0.501875pt}%
\definecolor{currentstroke}{rgb}{0.000000,0.000000,0.000000}%
\pgfsetstrokecolor{currentstroke}%
\pgfsetdash{}{0pt}%
\pgfsys@defobject{currentmarker}{\pgfqpoint{-0.020833in}{0.000000in}}{\pgfqpoint{-0.000000in}{0.000000in}}{%
\pgfpathmoveto{\pgfqpoint{-0.000000in}{0.000000in}}%
\pgfpathlineto{\pgfqpoint{-0.020833in}{0.000000in}}%
\pgfusepath{stroke,fill}%
}%
\begin{pgfscope}%
\pgfsys@transformshift{5.661441in}{1.392392in}%
\pgfsys@useobject{currentmarker}{}%
\end{pgfscope}%
\end{pgfscope}%
\begin{pgfscope}%
\pgfsetbuttcap%
\pgfsetroundjoin%
\definecolor{currentfill}{rgb}{0.000000,0.000000,0.000000}%
\pgfsetfillcolor{currentfill}%
\pgfsetlinewidth{0.501875pt}%
\definecolor{currentstroke}{rgb}{0.000000,0.000000,0.000000}%
\pgfsetstrokecolor{currentstroke}%
\pgfsetdash{}{0pt}%
\pgfsys@defobject{currentmarker}{\pgfqpoint{0.000000in}{0.000000in}}{\pgfqpoint{0.020833in}{0.000000in}}{%
\pgfpathmoveto{\pgfqpoint{0.000000in}{0.000000in}}%
\pgfpathlineto{\pgfqpoint{0.020833in}{0.000000in}}%
\pgfusepath{stroke,fill}%
}%
\begin{pgfscope}%
\pgfsys@transformshift{3.385377in}{1.599468in}%
\pgfsys@useobject{currentmarker}{}%
\end{pgfscope}%
\end{pgfscope}%
\begin{pgfscope}%
\pgfsetbuttcap%
\pgfsetroundjoin%
\definecolor{currentfill}{rgb}{0.000000,0.000000,0.000000}%
\pgfsetfillcolor{currentfill}%
\pgfsetlinewidth{0.501875pt}%
\definecolor{currentstroke}{rgb}{0.000000,0.000000,0.000000}%
\pgfsetstrokecolor{currentstroke}%
\pgfsetdash{}{0pt}%
\pgfsys@defobject{currentmarker}{\pgfqpoint{-0.020833in}{0.000000in}}{\pgfqpoint{-0.000000in}{0.000000in}}{%
\pgfpathmoveto{\pgfqpoint{-0.000000in}{0.000000in}}%
\pgfpathlineto{\pgfqpoint{-0.020833in}{0.000000in}}%
\pgfusepath{stroke,fill}%
}%
\begin{pgfscope}%
\pgfsys@transformshift{5.661441in}{1.599468in}%
\pgfsys@useobject{currentmarker}{}%
\end{pgfscope}%
\end{pgfscope}%
\begin{pgfscope}%
\pgfsetbuttcap%
\pgfsetroundjoin%
\definecolor{currentfill}{rgb}{0.000000,0.000000,0.000000}%
\pgfsetfillcolor{currentfill}%
\pgfsetlinewidth{0.501875pt}%
\definecolor{currentstroke}{rgb}{0.000000,0.000000,0.000000}%
\pgfsetstrokecolor{currentstroke}%
\pgfsetdash{}{0pt}%
\pgfsys@defobject{currentmarker}{\pgfqpoint{0.000000in}{0.000000in}}{\pgfqpoint{0.020833in}{0.000000in}}{%
\pgfpathmoveto{\pgfqpoint{0.000000in}{0.000000in}}%
\pgfpathlineto{\pgfqpoint{0.020833in}{0.000000in}}%
\pgfusepath{stroke,fill}%
}%
\begin{pgfscope}%
\pgfsys@transformshift{3.385377in}{1.703006in}%
\pgfsys@useobject{currentmarker}{}%
\end{pgfscope}%
\end{pgfscope}%
\begin{pgfscope}%
\pgfsetbuttcap%
\pgfsetroundjoin%
\definecolor{currentfill}{rgb}{0.000000,0.000000,0.000000}%
\pgfsetfillcolor{currentfill}%
\pgfsetlinewidth{0.501875pt}%
\definecolor{currentstroke}{rgb}{0.000000,0.000000,0.000000}%
\pgfsetstrokecolor{currentstroke}%
\pgfsetdash{}{0pt}%
\pgfsys@defobject{currentmarker}{\pgfqpoint{-0.020833in}{0.000000in}}{\pgfqpoint{-0.000000in}{0.000000in}}{%
\pgfpathmoveto{\pgfqpoint{-0.000000in}{0.000000in}}%
\pgfpathlineto{\pgfqpoint{-0.020833in}{0.000000in}}%
\pgfusepath{stroke,fill}%
}%
\begin{pgfscope}%
\pgfsys@transformshift{5.661441in}{1.703006in}%
\pgfsys@useobject{currentmarker}{}%
\end{pgfscope}%
\end{pgfscope}%
\begin{pgfscope}%
\pgfsetbuttcap%
\pgfsetroundjoin%
\definecolor{currentfill}{rgb}{0.000000,0.000000,0.000000}%
\pgfsetfillcolor{currentfill}%
\pgfsetlinewidth{0.501875pt}%
\definecolor{currentstroke}{rgb}{0.000000,0.000000,0.000000}%
\pgfsetstrokecolor{currentstroke}%
\pgfsetdash{}{0pt}%
\pgfsys@defobject{currentmarker}{\pgfqpoint{0.000000in}{0.000000in}}{\pgfqpoint{0.020833in}{0.000000in}}{%
\pgfpathmoveto{\pgfqpoint{0.000000in}{0.000000in}}%
\pgfpathlineto{\pgfqpoint{0.020833in}{0.000000in}}%
\pgfusepath{stroke,fill}%
}%
\begin{pgfscope}%
\pgfsys@transformshift{3.385377in}{1.806544in}%
\pgfsys@useobject{currentmarker}{}%
\end{pgfscope}%
\end{pgfscope}%
\begin{pgfscope}%
\pgfsetbuttcap%
\pgfsetroundjoin%
\definecolor{currentfill}{rgb}{0.000000,0.000000,0.000000}%
\pgfsetfillcolor{currentfill}%
\pgfsetlinewidth{0.501875pt}%
\definecolor{currentstroke}{rgb}{0.000000,0.000000,0.000000}%
\pgfsetstrokecolor{currentstroke}%
\pgfsetdash{}{0pt}%
\pgfsys@defobject{currentmarker}{\pgfqpoint{-0.020833in}{0.000000in}}{\pgfqpoint{-0.000000in}{0.000000in}}{%
\pgfpathmoveto{\pgfqpoint{-0.000000in}{0.000000in}}%
\pgfpathlineto{\pgfqpoint{-0.020833in}{0.000000in}}%
\pgfusepath{stroke,fill}%
}%
\begin{pgfscope}%
\pgfsys@transformshift{5.661441in}{1.806544in}%
\pgfsys@useobject{currentmarker}{}%
\end{pgfscope}%
\end{pgfscope}%
\begin{pgfscope}%
\pgfsetbuttcap%
\pgfsetroundjoin%
\definecolor{currentfill}{rgb}{0.000000,0.000000,0.000000}%
\pgfsetfillcolor{currentfill}%
\pgfsetlinewidth{0.501875pt}%
\definecolor{currentstroke}{rgb}{0.000000,0.000000,0.000000}%
\pgfsetstrokecolor{currentstroke}%
\pgfsetdash{}{0pt}%
\pgfsys@defobject{currentmarker}{\pgfqpoint{0.000000in}{0.000000in}}{\pgfqpoint{0.020833in}{0.000000in}}{%
\pgfpathmoveto{\pgfqpoint{0.000000in}{0.000000in}}%
\pgfpathlineto{\pgfqpoint{0.020833in}{0.000000in}}%
\pgfusepath{stroke,fill}%
}%
\begin{pgfscope}%
\pgfsys@transformshift{3.385377in}{1.910081in}%
\pgfsys@useobject{currentmarker}{}%
\end{pgfscope}%
\end{pgfscope}%
\begin{pgfscope}%
\pgfsetbuttcap%
\pgfsetroundjoin%
\definecolor{currentfill}{rgb}{0.000000,0.000000,0.000000}%
\pgfsetfillcolor{currentfill}%
\pgfsetlinewidth{0.501875pt}%
\definecolor{currentstroke}{rgb}{0.000000,0.000000,0.000000}%
\pgfsetstrokecolor{currentstroke}%
\pgfsetdash{}{0pt}%
\pgfsys@defobject{currentmarker}{\pgfqpoint{-0.020833in}{0.000000in}}{\pgfqpoint{-0.000000in}{0.000000in}}{%
\pgfpathmoveto{\pgfqpoint{-0.000000in}{0.000000in}}%
\pgfpathlineto{\pgfqpoint{-0.020833in}{0.000000in}}%
\pgfusepath{stroke,fill}%
}%
\begin{pgfscope}%
\pgfsys@transformshift{5.661441in}{1.910081in}%
\pgfsys@useobject{currentmarker}{}%
\end{pgfscope}%
\end{pgfscope}%
\begin{pgfscope}%
\pgfsetbuttcap%
\pgfsetroundjoin%
\definecolor{currentfill}{rgb}{0.000000,0.000000,0.000000}%
\pgfsetfillcolor{currentfill}%
\pgfsetlinewidth{0.501875pt}%
\definecolor{currentstroke}{rgb}{0.000000,0.000000,0.000000}%
\pgfsetstrokecolor{currentstroke}%
\pgfsetdash{}{0pt}%
\pgfsys@defobject{currentmarker}{\pgfqpoint{0.000000in}{0.000000in}}{\pgfqpoint{0.020833in}{0.000000in}}{%
\pgfpathmoveto{\pgfqpoint{0.000000in}{0.000000in}}%
\pgfpathlineto{\pgfqpoint{0.020833in}{0.000000in}}%
\pgfusepath{stroke,fill}%
}%
\begin{pgfscope}%
\pgfsys@transformshift{3.385377in}{2.117157in}%
\pgfsys@useobject{currentmarker}{}%
\end{pgfscope}%
\end{pgfscope}%
\begin{pgfscope}%
\pgfsetbuttcap%
\pgfsetroundjoin%
\definecolor{currentfill}{rgb}{0.000000,0.000000,0.000000}%
\pgfsetfillcolor{currentfill}%
\pgfsetlinewidth{0.501875pt}%
\definecolor{currentstroke}{rgb}{0.000000,0.000000,0.000000}%
\pgfsetstrokecolor{currentstroke}%
\pgfsetdash{}{0pt}%
\pgfsys@defobject{currentmarker}{\pgfqpoint{-0.020833in}{0.000000in}}{\pgfqpoint{-0.000000in}{0.000000in}}{%
\pgfpathmoveto{\pgfqpoint{-0.000000in}{0.000000in}}%
\pgfpathlineto{\pgfqpoint{-0.020833in}{0.000000in}}%
\pgfusepath{stroke,fill}%
}%
\begin{pgfscope}%
\pgfsys@transformshift{5.661441in}{2.117157in}%
\pgfsys@useobject{currentmarker}{}%
\end{pgfscope}%
\end{pgfscope}%
\begin{pgfscope}%
\pgfsetbuttcap%
\pgfsetroundjoin%
\definecolor{currentfill}{rgb}{0.000000,0.000000,0.000000}%
\pgfsetfillcolor{currentfill}%
\pgfsetlinewidth{0.501875pt}%
\definecolor{currentstroke}{rgb}{0.000000,0.000000,0.000000}%
\pgfsetstrokecolor{currentstroke}%
\pgfsetdash{}{0pt}%
\pgfsys@defobject{currentmarker}{\pgfqpoint{0.000000in}{0.000000in}}{\pgfqpoint{0.020833in}{0.000000in}}{%
\pgfpathmoveto{\pgfqpoint{0.000000in}{0.000000in}}%
\pgfpathlineto{\pgfqpoint{0.020833in}{0.000000in}}%
\pgfusepath{stroke,fill}%
}%
\begin{pgfscope}%
\pgfsys@transformshift{3.385377in}{2.220695in}%
\pgfsys@useobject{currentmarker}{}%
\end{pgfscope}%
\end{pgfscope}%
\begin{pgfscope}%
\pgfsetbuttcap%
\pgfsetroundjoin%
\definecolor{currentfill}{rgb}{0.000000,0.000000,0.000000}%
\pgfsetfillcolor{currentfill}%
\pgfsetlinewidth{0.501875pt}%
\definecolor{currentstroke}{rgb}{0.000000,0.000000,0.000000}%
\pgfsetstrokecolor{currentstroke}%
\pgfsetdash{}{0pt}%
\pgfsys@defobject{currentmarker}{\pgfqpoint{-0.020833in}{0.000000in}}{\pgfqpoint{-0.000000in}{0.000000in}}{%
\pgfpathmoveto{\pgfqpoint{-0.000000in}{0.000000in}}%
\pgfpathlineto{\pgfqpoint{-0.020833in}{0.000000in}}%
\pgfusepath{stroke,fill}%
}%
\begin{pgfscope}%
\pgfsys@transformshift{5.661441in}{2.220695in}%
\pgfsys@useobject{currentmarker}{}%
\end{pgfscope}%
\end{pgfscope}%
\begin{pgfscope}%
\pgfsetbuttcap%
\pgfsetroundjoin%
\definecolor{currentfill}{rgb}{0.000000,0.000000,0.000000}%
\pgfsetfillcolor{currentfill}%
\pgfsetlinewidth{0.501875pt}%
\definecolor{currentstroke}{rgb}{0.000000,0.000000,0.000000}%
\pgfsetstrokecolor{currentstroke}%
\pgfsetdash{}{0pt}%
\pgfsys@defobject{currentmarker}{\pgfqpoint{0.000000in}{0.000000in}}{\pgfqpoint{0.020833in}{0.000000in}}{%
\pgfpathmoveto{\pgfqpoint{0.000000in}{0.000000in}}%
\pgfpathlineto{\pgfqpoint{0.020833in}{0.000000in}}%
\pgfusepath{stroke,fill}%
}%
\begin{pgfscope}%
\pgfsys@transformshift{3.385377in}{2.324233in}%
\pgfsys@useobject{currentmarker}{}%
\end{pgfscope}%
\end{pgfscope}%
\begin{pgfscope}%
\pgfsetbuttcap%
\pgfsetroundjoin%
\definecolor{currentfill}{rgb}{0.000000,0.000000,0.000000}%
\pgfsetfillcolor{currentfill}%
\pgfsetlinewidth{0.501875pt}%
\definecolor{currentstroke}{rgb}{0.000000,0.000000,0.000000}%
\pgfsetstrokecolor{currentstroke}%
\pgfsetdash{}{0pt}%
\pgfsys@defobject{currentmarker}{\pgfqpoint{-0.020833in}{0.000000in}}{\pgfqpoint{-0.000000in}{0.000000in}}{%
\pgfpathmoveto{\pgfqpoint{-0.000000in}{0.000000in}}%
\pgfpathlineto{\pgfqpoint{-0.020833in}{0.000000in}}%
\pgfusepath{stroke,fill}%
}%
\begin{pgfscope}%
\pgfsys@transformshift{5.661441in}{2.324233in}%
\pgfsys@useobject{currentmarker}{}%
\end{pgfscope}%
\end{pgfscope}%
\begin{pgfscope}%
\pgfsetbuttcap%
\pgfsetroundjoin%
\definecolor{currentfill}{rgb}{0.000000,0.000000,0.000000}%
\pgfsetfillcolor{currentfill}%
\pgfsetlinewidth{0.501875pt}%
\definecolor{currentstroke}{rgb}{0.000000,0.000000,0.000000}%
\pgfsetstrokecolor{currentstroke}%
\pgfsetdash{}{0pt}%
\pgfsys@defobject{currentmarker}{\pgfqpoint{0.000000in}{0.000000in}}{\pgfqpoint{0.020833in}{0.000000in}}{%
\pgfpathmoveto{\pgfqpoint{0.000000in}{0.000000in}}%
\pgfpathlineto{\pgfqpoint{0.020833in}{0.000000in}}%
\pgfusepath{stroke,fill}%
}%
\begin{pgfscope}%
\pgfsys@transformshift{3.385377in}{2.427771in}%
\pgfsys@useobject{currentmarker}{}%
\end{pgfscope}%
\end{pgfscope}%
\begin{pgfscope}%
\pgfsetbuttcap%
\pgfsetroundjoin%
\definecolor{currentfill}{rgb}{0.000000,0.000000,0.000000}%
\pgfsetfillcolor{currentfill}%
\pgfsetlinewidth{0.501875pt}%
\definecolor{currentstroke}{rgb}{0.000000,0.000000,0.000000}%
\pgfsetstrokecolor{currentstroke}%
\pgfsetdash{}{0pt}%
\pgfsys@defobject{currentmarker}{\pgfqpoint{-0.020833in}{0.000000in}}{\pgfqpoint{-0.000000in}{0.000000in}}{%
\pgfpathmoveto{\pgfqpoint{-0.000000in}{0.000000in}}%
\pgfpathlineto{\pgfqpoint{-0.020833in}{0.000000in}}%
\pgfusepath{stroke,fill}%
}%
\begin{pgfscope}%
\pgfsys@transformshift{5.661441in}{2.427771in}%
\pgfsys@useobject{currentmarker}{}%
\end{pgfscope}%
\end{pgfscope}%
\begin{pgfscope}%
\pgfsetbuttcap%
\pgfsetroundjoin%
\definecolor{currentfill}{rgb}{0.000000,0.000000,0.000000}%
\pgfsetfillcolor{currentfill}%
\pgfsetlinewidth{0.501875pt}%
\definecolor{currentstroke}{rgb}{0.000000,0.000000,0.000000}%
\pgfsetstrokecolor{currentstroke}%
\pgfsetdash{}{0pt}%
\pgfsys@defobject{currentmarker}{\pgfqpoint{0.000000in}{0.000000in}}{\pgfqpoint{0.020833in}{0.000000in}}{%
\pgfpathmoveto{\pgfqpoint{0.000000in}{0.000000in}}%
\pgfpathlineto{\pgfqpoint{0.020833in}{0.000000in}}%
\pgfusepath{stroke,fill}%
}%
\begin{pgfscope}%
\pgfsys@transformshift{3.385377in}{2.634847in}%
\pgfsys@useobject{currentmarker}{}%
\end{pgfscope}%
\end{pgfscope}%
\begin{pgfscope}%
\pgfsetbuttcap%
\pgfsetroundjoin%
\definecolor{currentfill}{rgb}{0.000000,0.000000,0.000000}%
\pgfsetfillcolor{currentfill}%
\pgfsetlinewidth{0.501875pt}%
\definecolor{currentstroke}{rgb}{0.000000,0.000000,0.000000}%
\pgfsetstrokecolor{currentstroke}%
\pgfsetdash{}{0pt}%
\pgfsys@defobject{currentmarker}{\pgfqpoint{-0.020833in}{0.000000in}}{\pgfqpoint{-0.000000in}{0.000000in}}{%
\pgfpathmoveto{\pgfqpoint{-0.000000in}{0.000000in}}%
\pgfpathlineto{\pgfqpoint{-0.020833in}{0.000000in}}%
\pgfusepath{stroke,fill}%
}%
\begin{pgfscope}%
\pgfsys@transformshift{5.661441in}{2.634847in}%
\pgfsys@useobject{currentmarker}{}%
\end{pgfscope}%
\end{pgfscope}%
\begin{pgfscope}%
\pgfsetbuttcap%
\pgfsetroundjoin%
\definecolor{currentfill}{rgb}{0.000000,0.000000,0.000000}%
\pgfsetfillcolor{currentfill}%
\pgfsetlinewidth{0.501875pt}%
\definecolor{currentstroke}{rgb}{0.000000,0.000000,0.000000}%
\pgfsetstrokecolor{currentstroke}%
\pgfsetdash{}{0pt}%
\pgfsys@defobject{currentmarker}{\pgfqpoint{0.000000in}{0.000000in}}{\pgfqpoint{0.020833in}{0.000000in}}{%
\pgfpathmoveto{\pgfqpoint{0.000000in}{0.000000in}}%
\pgfpathlineto{\pgfqpoint{0.020833in}{0.000000in}}%
\pgfusepath{stroke,fill}%
}%
\begin{pgfscope}%
\pgfsys@transformshift{3.385377in}{2.738385in}%
\pgfsys@useobject{currentmarker}{}%
\end{pgfscope}%
\end{pgfscope}%
\begin{pgfscope}%
\pgfsetbuttcap%
\pgfsetroundjoin%
\definecolor{currentfill}{rgb}{0.000000,0.000000,0.000000}%
\pgfsetfillcolor{currentfill}%
\pgfsetlinewidth{0.501875pt}%
\definecolor{currentstroke}{rgb}{0.000000,0.000000,0.000000}%
\pgfsetstrokecolor{currentstroke}%
\pgfsetdash{}{0pt}%
\pgfsys@defobject{currentmarker}{\pgfqpoint{-0.020833in}{0.000000in}}{\pgfqpoint{-0.000000in}{0.000000in}}{%
\pgfpathmoveto{\pgfqpoint{-0.000000in}{0.000000in}}%
\pgfpathlineto{\pgfqpoint{-0.020833in}{0.000000in}}%
\pgfusepath{stroke,fill}%
}%
\begin{pgfscope}%
\pgfsys@transformshift{5.661441in}{2.738385in}%
\pgfsys@useobject{currentmarker}{}%
\end{pgfscope}%
\end{pgfscope}%
\begin{pgfscope}%
\pgfsetbuttcap%
\pgfsetroundjoin%
\definecolor{currentfill}{rgb}{0.000000,0.000000,0.000000}%
\pgfsetfillcolor{currentfill}%
\pgfsetlinewidth{0.501875pt}%
\definecolor{currentstroke}{rgb}{0.000000,0.000000,0.000000}%
\pgfsetstrokecolor{currentstroke}%
\pgfsetdash{}{0pt}%
\pgfsys@defobject{currentmarker}{\pgfqpoint{0.000000in}{0.000000in}}{\pgfqpoint{0.020833in}{0.000000in}}{%
\pgfpathmoveto{\pgfqpoint{0.000000in}{0.000000in}}%
\pgfpathlineto{\pgfqpoint{0.020833in}{0.000000in}}%
\pgfusepath{stroke,fill}%
}%
\begin{pgfscope}%
\pgfsys@transformshift{3.385377in}{2.841922in}%
\pgfsys@useobject{currentmarker}{}%
\end{pgfscope}%
\end{pgfscope}%
\begin{pgfscope}%
\pgfsetbuttcap%
\pgfsetroundjoin%
\definecolor{currentfill}{rgb}{0.000000,0.000000,0.000000}%
\pgfsetfillcolor{currentfill}%
\pgfsetlinewidth{0.501875pt}%
\definecolor{currentstroke}{rgb}{0.000000,0.000000,0.000000}%
\pgfsetstrokecolor{currentstroke}%
\pgfsetdash{}{0pt}%
\pgfsys@defobject{currentmarker}{\pgfqpoint{-0.020833in}{0.000000in}}{\pgfqpoint{-0.000000in}{0.000000in}}{%
\pgfpathmoveto{\pgfqpoint{-0.000000in}{0.000000in}}%
\pgfpathlineto{\pgfqpoint{-0.020833in}{0.000000in}}%
\pgfusepath{stroke,fill}%
}%
\begin{pgfscope}%
\pgfsys@transformshift{5.661441in}{2.841922in}%
\pgfsys@useobject{currentmarker}{}%
\end{pgfscope}%
\end{pgfscope}%
\begin{pgfscope}%
\pgfsetbuttcap%
\pgfsetroundjoin%
\definecolor{currentfill}{rgb}{0.000000,0.000000,0.000000}%
\pgfsetfillcolor{currentfill}%
\pgfsetlinewidth{0.501875pt}%
\definecolor{currentstroke}{rgb}{0.000000,0.000000,0.000000}%
\pgfsetstrokecolor{currentstroke}%
\pgfsetdash{}{0pt}%
\pgfsys@defobject{currentmarker}{\pgfqpoint{0.000000in}{0.000000in}}{\pgfqpoint{0.020833in}{0.000000in}}{%
\pgfpathmoveto{\pgfqpoint{0.000000in}{0.000000in}}%
\pgfpathlineto{\pgfqpoint{0.020833in}{0.000000in}}%
\pgfusepath{stroke,fill}%
}%
\begin{pgfscope}%
\pgfsys@transformshift{3.385377in}{2.945460in}%
\pgfsys@useobject{currentmarker}{}%
\end{pgfscope}%
\end{pgfscope}%
\begin{pgfscope}%
\pgfsetbuttcap%
\pgfsetroundjoin%
\definecolor{currentfill}{rgb}{0.000000,0.000000,0.000000}%
\pgfsetfillcolor{currentfill}%
\pgfsetlinewidth{0.501875pt}%
\definecolor{currentstroke}{rgb}{0.000000,0.000000,0.000000}%
\pgfsetstrokecolor{currentstroke}%
\pgfsetdash{}{0pt}%
\pgfsys@defobject{currentmarker}{\pgfqpoint{-0.020833in}{0.000000in}}{\pgfqpoint{-0.000000in}{0.000000in}}{%
\pgfpathmoveto{\pgfqpoint{-0.000000in}{0.000000in}}%
\pgfpathlineto{\pgfqpoint{-0.020833in}{0.000000in}}%
\pgfusepath{stroke,fill}%
}%
\begin{pgfscope}%
\pgfsys@transformshift{5.661441in}{2.945460in}%
\pgfsys@useobject{currentmarker}{}%
\end{pgfscope}%
\end{pgfscope}%
\begin{pgfscope}%
\pgfsetbuttcap%
\pgfsetroundjoin%
\definecolor{currentfill}{rgb}{0.000000,0.000000,0.000000}%
\pgfsetfillcolor{currentfill}%
\pgfsetlinewidth{0.501875pt}%
\definecolor{currentstroke}{rgb}{0.000000,0.000000,0.000000}%
\pgfsetstrokecolor{currentstroke}%
\pgfsetdash{}{0pt}%
\pgfsys@defobject{currentmarker}{\pgfqpoint{0.000000in}{0.000000in}}{\pgfqpoint{0.020833in}{0.000000in}}{%
\pgfpathmoveto{\pgfqpoint{0.000000in}{0.000000in}}%
\pgfpathlineto{\pgfqpoint{0.020833in}{0.000000in}}%
\pgfusepath{stroke,fill}%
}%
\begin{pgfscope}%
\pgfsys@transformshift{3.385377in}{3.152536in}%
\pgfsys@useobject{currentmarker}{}%
\end{pgfscope}%
\end{pgfscope}%
\begin{pgfscope}%
\pgfsetbuttcap%
\pgfsetroundjoin%
\definecolor{currentfill}{rgb}{0.000000,0.000000,0.000000}%
\pgfsetfillcolor{currentfill}%
\pgfsetlinewidth{0.501875pt}%
\definecolor{currentstroke}{rgb}{0.000000,0.000000,0.000000}%
\pgfsetstrokecolor{currentstroke}%
\pgfsetdash{}{0pt}%
\pgfsys@defobject{currentmarker}{\pgfqpoint{-0.020833in}{0.000000in}}{\pgfqpoint{-0.000000in}{0.000000in}}{%
\pgfpathmoveto{\pgfqpoint{-0.000000in}{0.000000in}}%
\pgfpathlineto{\pgfqpoint{-0.020833in}{0.000000in}}%
\pgfusepath{stroke,fill}%
}%
\begin{pgfscope}%
\pgfsys@transformshift{5.661441in}{3.152536in}%
\pgfsys@useobject{currentmarker}{}%
\end{pgfscope}%
\end{pgfscope}%
\begin{pgfscope}%
\pgfsetbuttcap%
\pgfsetroundjoin%
\definecolor{currentfill}{rgb}{0.000000,0.000000,0.000000}%
\pgfsetfillcolor{currentfill}%
\pgfsetlinewidth{0.501875pt}%
\definecolor{currentstroke}{rgb}{0.000000,0.000000,0.000000}%
\pgfsetstrokecolor{currentstroke}%
\pgfsetdash{}{0pt}%
\pgfsys@defobject{currentmarker}{\pgfqpoint{0.000000in}{0.000000in}}{\pgfqpoint{0.020833in}{0.000000in}}{%
\pgfpathmoveto{\pgfqpoint{0.000000in}{0.000000in}}%
\pgfpathlineto{\pgfqpoint{0.020833in}{0.000000in}}%
\pgfusepath{stroke,fill}%
}%
\begin{pgfscope}%
\pgfsys@transformshift{3.385377in}{3.256074in}%
\pgfsys@useobject{currentmarker}{}%
\end{pgfscope}%
\end{pgfscope}%
\begin{pgfscope}%
\pgfsetbuttcap%
\pgfsetroundjoin%
\definecolor{currentfill}{rgb}{0.000000,0.000000,0.000000}%
\pgfsetfillcolor{currentfill}%
\pgfsetlinewidth{0.501875pt}%
\definecolor{currentstroke}{rgb}{0.000000,0.000000,0.000000}%
\pgfsetstrokecolor{currentstroke}%
\pgfsetdash{}{0pt}%
\pgfsys@defobject{currentmarker}{\pgfqpoint{-0.020833in}{0.000000in}}{\pgfqpoint{-0.000000in}{0.000000in}}{%
\pgfpathmoveto{\pgfqpoint{-0.000000in}{0.000000in}}%
\pgfpathlineto{\pgfqpoint{-0.020833in}{0.000000in}}%
\pgfusepath{stroke,fill}%
}%
\begin{pgfscope}%
\pgfsys@transformshift{5.661441in}{3.256074in}%
\pgfsys@useobject{currentmarker}{}%
\end{pgfscope}%
\end{pgfscope}%
\begin{pgfscope}%
\pgfsetbuttcap%
\pgfsetroundjoin%
\definecolor{currentfill}{rgb}{0.000000,0.000000,0.000000}%
\pgfsetfillcolor{currentfill}%
\pgfsetlinewidth{0.501875pt}%
\definecolor{currentstroke}{rgb}{0.000000,0.000000,0.000000}%
\pgfsetstrokecolor{currentstroke}%
\pgfsetdash{}{0pt}%
\pgfsys@defobject{currentmarker}{\pgfqpoint{0.000000in}{0.000000in}}{\pgfqpoint{0.020833in}{0.000000in}}{%
\pgfpathmoveto{\pgfqpoint{0.000000in}{0.000000in}}%
\pgfpathlineto{\pgfqpoint{0.020833in}{0.000000in}}%
\pgfusepath{stroke,fill}%
}%
\begin{pgfscope}%
\pgfsys@transformshift{3.385377in}{3.359612in}%
\pgfsys@useobject{currentmarker}{}%
\end{pgfscope}%
\end{pgfscope}%
\begin{pgfscope}%
\pgfsetbuttcap%
\pgfsetroundjoin%
\definecolor{currentfill}{rgb}{0.000000,0.000000,0.000000}%
\pgfsetfillcolor{currentfill}%
\pgfsetlinewidth{0.501875pt}%
\definecolor{currentstroke}{rgb}{0.000000,0.000000,0.000000}%
\pgfsetstrokecolor{currentstroke}%
\pgfsetdash{}{0pt}%
\pgfsys@defobject{currentmarker}{\pgfqpoint{-0.020833in}{0.000000in}}{\pgfqpoint{-0.000000in}{0.000000in}}{%
\pgfpathmoveto{\pgfqpoint{-0.000000in}{0.000000in}}%
\pgfpathlineto{\pgfqpoint{-0.020833in}{0.000000in}}%
\pgfusepath{stroke,fill}%
}%
\begin{pgfscope}%
\pgfsys@transformshift{5.661441in}{3.359612in}%
\pgfsys@useobject{currentmarker}{}%
\end{pgfscope}%
\end{pgfscope}%
\begin{pgfscope}%
\pgfsetbuttcap%
\pgfsetroundjoin%
\definecolor{currentfill}{rgb}{0.000000,0.000000,0.000000}%
\pgfsetfillcolor{currentfill}%
\pgfsetlinewidth{0.501875pt}%
\definecolor{currentstroke}{rgb}{0.000000,0.000000,0.000000}%
\pgfsetstrokecolor{currentstroke}%
\pgfsetdash{}{0pt}%
\pgfsys@defobject{currentmarker}{\pgfqpoint{0.000000in}{0.000000in}}{\pgfqpoint{0.020833in}{0.000000in}}{%
\pgfpathmoveto{\pgfqpoint{0.000000in}{0.000000in}}%
\pgfpathlineto{\pgfqpoint{0.020833in}{0.000000in}}%
\pgfusepath{stroke,fill}%
}%
\begin{pgfscope}%
\pgfsys@transformshift{3.385377in}{3.463150in}%
\pgfsys@useobject{currentmarker}{}%
\end{pgfscope}%
\end{pgfscope}%
\begin{pgfscope}%
\pgfsetbuttcap%
\pgfsetroundjoin%
\definecolor{currentfill}{rgb}{0.000000,0.000000,0.000000}%
\pgfsetfillcolor{currentfill}%
\pgfsetlinewidth{0.501875pt}%
\definecolor{currentstroke}{rgb}{0.000000,0.000000,0.000000}%
\pgfsetstrokecolor{currentstroke}%
\pgfsetdash{}{0pt}%
\pgfsys@defobject{currentmarker}{\pgfqpoint{-0.020833in}{0.000000in}}{\pgfqpoint{-0.000000in}{0.000000in}}{%
\pgfpathmoveto{\pgfqpoint{-0.000000in}{0.000000in}}%
\pgfpathlineto{\pgfqpoint{-0.020833in}{0.000000in}}%
\pgfusepath{stroke,fill}%
}%
\begin{pgfscope}%
\pgfsys@transformshift{5.661441in}{3.463150in}%
\pgfsys@useobject{currentmarker}{}%
\end{pgfscope}%
\end{pgfscope}%
\begin{pgfscope}%
\pgfsetbuttcap%
\pgfsetroundjoin%
\definecolor{currentfill}{rgb}{0.000000,0.000000,0.000000}%
\pgfsetfillcolor{currentfill}%
\pgfsetlinewidth{0.501875pt}%
\definecolor{currentstroke}{rgb}{0.000000,0.000000,0.000000}%
\pgfsetstrokecolor{currentstroke}%
\pgfsetdash{}{0pt}%
\pgfsys@defobject{currentmarker}{\pgfqpoint{0.000000in}{0.000000in}}{\pgfqpoint{0.020833in}{0.000000in}}{%
\pgfpathmoveto{\pgfqpoint{0.000000in}{0.000000in}}%
\pgfpathlineto{\pgfqpoint{0.020833in}{0.000000in}}%
\pgfusepath{stroke,fill}%
}%
\begin{pgfscope}%
\pgfsys@transformshift{3.385377in}{3.670226in}%
\pgfsys@useobject{currentmarker}{}%
\end{pgfscope}%
\end{pgfscope}%
\begin{pgfscope}%
\pgfsetbuttcap%
\pgfsetroundjoin%
\definecolor{currentfill}{rgb}{0.000000,0.000000,0.000000}%
\pgfsetfillcolor{currentfill}%
\pgfsetlinewidth{0.501875pt}%
\definecolor{currentstroke}{rgb}{0.000000,0.000000,0.000000}%
\pgfsetstrokecolor{currentstroke}%
\pgfsetdash{}{0pt}%
\pgfsys@defobject{currentmarker}{\pgfqpoint{-0.020833in}{0.000000in}}{\pgfqpoint{-0.000000in}{0.000000in}}{%
\pgfpathmoveto{\pgfqpoint{-0.000000in}{0.000000in}}%
\pgfpathlineto{\pgfqpoint{-0.020833in}{0.000000in}}%
\pgfusepath{stroke,fill}%
}%
\begin{pgfscope}%
\pgfsys@transformshift{5.661441in}{3.670226in}%
\pgfsys@useobject{currentmarker}{}%
\end{pgfscope}%
\end{pgfscope}%
\begin{pgfscope}%
\pgfsetbuttcap%
\pgfsetroundjoin%
\definecolor{currentfill}{rgb}{0.000000,0.000000,0.000000}%
\pgfsetfillcolor{currentfill}%
\pgfsetlinewidth{0.501875pt}%
\definecolor{currentstroke}{rgb}{0.000000,0.000000,0.000000}%
\pgfsetstrokecolor{currentstroke}%
\pgfsetdash{}{0pt}%
\pgfsys@defobject{currentmarker}{\pgfqpoint{0.000000in}{0.000000in}}{\pgfqpoint{0.020833in}{0.000000in}}{%
\pgfpathmoveto{\pgfqpoint{0.000000in}{0.000000in}}%
\pgfpathlineto{\pgfqpoint{0.020833in}{0.000000in}}%
\pgfusepath{stroke,fill}%
}%
\begin{pgfscope}%
\pgfsys@transformshift{3.385377in}{3.773763in}%
\pgfsys@useobject{currentmarker}{}%
\end{pgfscope}%
\end{pgfscope}%
\begin{pgfscope}%
\pgfsetbuttcap%
\pgfsetroundjoin%
\definecolor{currentfill}{rgb}{0.000000,0.000000,0.000000}%
\pgfsetfillcolor{currentfill}%
\pgfsetlinewidth{0.501875pt}%
\definecolor{currentstroke}{rgb}{0.000000,0.000000,0.000000}%
\pgfsetstrokecolor{currentstroke}%
\pgfsetdash{}{0pt}%
\pgfsys@defobject{currentmarker}{\pgfqpoint{-0.020833in}{0.000000in}}{\pgfqpoint{-0.000000in}{0.000000in}}{%
\pgfpathmoveto{\pgfqpoint{-0.000000in}{0.000000in}}%
\pgfpathlineto{\pgfqpoint{-0.020833in}{0.000000in}}%
\pgfusepath{stroke,fill}%
}%
\begin{pgfscope}%
\pgfsys@transformshift{5.661441in}{3.773763in}%
\pgfsys@useobject{currentmarker}{}%
\end{pgfscope}%
\end{pgfscope}%
\begin{pgfscope}%
\pgfsetbuttcap%
\pgfsetroundjoin%
\definecolor{currentfill}{rgb}{0.000000,0.000000,0.000000}%
\pgfsetfillcolor{currentfill}%
\pgfsetlinewidth{0.501875pt}%
\definecolor{currentstroke}{rgb}{0.000000,0.000000,0.000000}%
\pgfsetstrokecolor{currentstroke}%
\pgfsetdash{}{0pt}%
\pgfsys@defobject{currentmarker}{\pgfqpoint{0.000000in}{0.000000in}}{\pgfqpoint{0.020833in}{0.000000in}}{%
\pgfpathmoveto{\pgfqpoint{0.000000in}{0.000000in}}%
\pgfpathlineto{\pgfqpoint{0.020833in}{0.000000in}}%
\pgfusepath{stroke,fill}%
}%
\begin{pgfscope}%
\pgfsys@transformshift{3.385377in}{3.877301in}%
\pgfsys@useobject{currentmarker}{}%
\end{pgfscope}%
\end{pgfscope}%
\begin{pgfscope}%
\pgfsetbuttcap%
\pgfsetroundjoin%
\definecolor{currentfill}{rgb}{0.000000,0.000000,0.000000}%
\pgfsetfillcolor{currentfill}%
\pgfsetlinewidth{0.501875pt}%
\definecolor{currentstroke}{rgb}{0.000000,0.000000,0.000000}%
\pgfsetstrokecolor{currentstroke}%
\pgfsetdash{}{0pt}%
\pgfsys@defobject{currentmarker}{\pgfqpoint{-0.020833in}{0.000000in}}{\pgfqpoint{-0.000000in}{0.000000in}}{%
\pgfpathmoveto{\pgfqpoint{-0.000000in}{0.000000in}}%
\pgfpathlineto{\pgfqpoint{-0.020833in}{0.000000in}}%
\pgfusepath{stroke,fill}%
}%
\begin{pgfscope}%
\pgfsys@transformshift{5.661441in}{3.877301in}%
\pgfsys@useobject{currentmarker}{}%
\end{pgfscope}%
\end{pgfscope}%
\begin{pgfscope}%
\pgfsetbuttcap%
\pgfsetroundjoin%
\definecolor{currentfill}{rgb}{0.000000,0.000000,0.000000}%
\pgfsetfillcolor{currentfill}%
\pgfsetlinewidth{0.501875pt}%
\definecolor{currentstroke}{rgb}{0.000000,0.000000,0.000000}%
\pgfsetstrokecolor{currentstroke}%
\pgfsetdash{}{0pt}%
\pgfsys@defobject{currentmarker}{\pgfqpoint{0.000000in}{0.000000in}}{\pgfqpoint{0.020833in}{0.000000in}}{%
\pgfpathmoveto{\pgfqpoint{0.000000in}{0.000000in}}%
\pgfpathlineto{\pgfqpoint{0.020833in}{0.000000in}}%
\pgfusepath{stroke,fill}%
}%
\begin{pgfscope}%
\pgfsys@transformshift{3.385377in}{3.980839in}%
\pgfsys@useobject{currentmarker}{}%
\end{pgfscope}%
\end{pgfscope}%
\begin{pgfscope}%
\pgfsetbuttcap%
\pgfsetroundjoin%
\definecolor{currentfill}{rgb}{0.000000,0.000000,0.000000}%
\pgfsetfillcolor{currentfill}%
\pgfsetlinewidth{0.501875pt}%
\definecolor{currentstroke}{rgb}{0.000000,0.000000,0.000000}%
\pgfsetstrokecolor{currentstroke}%
\pgfsetdash{}{0pt}%
\pgfsys@defobject{currentmarker}{\pgfqpoint{-0.020833in}{0.000000in}}{\pgfqpoint{-0.000000in}{0.000000in}}{%
\pgfpathmoveto{\pgfqpoint{-0.000000in}{0.000000in}}%
\pgfpathlineto{\pgfqpoint{-0.020833in}{0.000000in}}%
\pgfusepath{stroke,fill}%
}%
\begin{pgfscope}%
\pgfsys@transformshift{5.661441in}{3.980839in}%
\pgfsys@useobject{currentmarker}{}%
\end{pgfscope}%
\end{pgfscope}%
\begin{pgfscope}%
\pgfsetbuttcap%
\pgfsetroundjoin%
\definecolor{currentfill}{rgb}{0.000000,0.000000,0.000000}%
\pgfsetfillcolor{currentfill}%
\pgfsetlinewidth{0.501875pt}%
\definecolor{currentstroke}{rgb}{0.000000,0.000000,0.000000}%
\pgfsetstrokecolor{currentstroke}%
\pgfsetdash{}{0pt}%
\pgfsys@defobject{currentmarker}{\pgfqpoint{0.000000in}{0.000000in}}{\pgfqpoint{0.020833in}{0.000000in}}{%
\pgfpathmoveto{\pgfqpoint{0.000000in}{0.000000in}}%
\pgfpathlineto{\pgfqpoint{0.020833in}{0.000000in}}%
\pgfusepath{stroke,fill}%
}%
\begin{pgfscope}%
\pgfsys@transformshift{3.385377in}{4.187915in}%
\pgfsys@useobject{currentmarker}{}%
\end{pgfscope}%
\end{pgfscope}%
\begin{pgfscope}%
\pgfsetbuttcap%
\pgfsetroundjoin%
\definecolor{currentfill}{rgb}{0.000000,0.000000,0.000000}%
\pgfsetfillcolor{currentfill}%
\pgfsetlinewidth{0.501875pt}%
\definecolor{currentstroke}{rgb}{0.000000,0.000000,0.000000}%
\pgfsetstrokecolor{currentstroke}%
\pgfsetdash{}{0pt}%
\pgfsys@defobject{currentmarker}{\pgfqpoint{-0.020833in}{0.000000in}}{\pgfqpoint{-0.000000in}{0.000000in}}{%
\pgfpathmoveto{\pgfqpoint{-0.000000in}{0.000000in}}%
\pgfpathlineto{\pgfqpoint{-0.020833in}{0.000000in}}%
\pgfusepath{stroke,fill}%
}%
\begin{pgfscope}%
\pgfsys@transformshift{5.661441in}{4.187915in}%
\pgfsys@useobject{currentmarker}{}%
\end{pgfscope}%
\end{pgfscope}%
\begin{pgfscope}%
\pgfsetbuttcap%
\pgfsetroundjoin%
\definecolor{currentfill}{rgb}{0.000000,0.000000,0.000000}%
\pgfsetfillcolor{currentfill}%
\pgfsetlinewidth{0.501875pt}%
\definecolor{currentstroke}{rgb}{0.000000,0.000000,0.000000}%
\pgfsetstrokecolor{currentstroke}%
\pgfsetdash{}{0pt}%
\pgfsys@defobject{currentmarker}{\pgfqpoint{0.000000in}{0.000000in}}{\pgfqpoint{0.020833in}{0.000000in}}{%
\pgfpathmoveto{\pgfqpoint{0.000000in}{0.000000in}}%
\pgfpathlineto{\pgfqpoint{0.020833in}{0.000000in}}%
\pgfusepath{stroke,fill}%
}%
\begin{pgfscope}%
\pgfsys@transformshift{3.385377in}{4.291453in}%
\pgfsys@useobject{currentmarker}{}%
\end{pgfscope}%
\end{pgfscope}%
\begin{pgfscope}%
\pgfsetbuttcap%
\pgfsetroundjoin%
\definecolor{currentfill}{rgb}{0.000000,0.000000,0.000000}%
\pgfsetfillcolor{currentfill}%
\pgfsetlinewidth{0.501875pt}%
\definecolor{currentstroke}{rgb}{0.000000,0.000000,0.000000}%
\pgfsetstrokecolor{currentstroke}%
\pgfsetdash{}{0pt}%
\pgfsys@defobject{currentmarker}{\pgfqpoint{-0.020833in}{0.000000in}}{\pgfqpoint{-0.000000in}{0.000000in}}{%
\pgfpathmoveto{\pgfqpoint{-0.000000in}{0.000000in}}%
\pgfpathlineto{\pgfqpoint{-0.020833in}{0.000000in}}%
\pgfusepath{stroke,fill}%
}%
\begin{pgfscope}%
\pgfsys@transformshift{5.661441in}{4.291453in}%
\pgfsys@useobject{currentmarker}{}%
\end{pgfscope}%
\end{pgfscope}%
\begin{pgfscope}%
\definecolor{textcolor}{rgb}{0.000000,0.000000,0.000000}%
\pgfsetstrokecolor{textcolor}%
\pgfsetfillcolor{textcolor}%
\pgftext[x=3.103741in,y=2.398593in,,bottom,rotate=90.000000]{\color{textcolor}\rmfamily\fontsize{10.000000}{12.000000}\selectfont \(\displaystyle LCMC(K)\)}%
\end{pgfscope}%
\begin{pgfscope}%
\pgfpathrectangle{\pgfqpoint{3.385377in}{0.422992in}}{\pgfqpoint{2.276064in}{3.951201in}}%
\pgfusepath{clip}%
\pgfsetrectcap%
\pgfsetroundjoin%
\pgfsetlinewidth{1.003750pt}%
\definecolor{currentstroke}{rgb}{0.047059,0.364706,0.647059}%
\pgfsetstrokecolor{currentstroke}%
\pgfsetdash{}{0pt}%
\pgfpathmoveto{\pgfqpoint{3.407913in}{2.696077in}}%
\pgfpathlineto{\pgfqpoint{3.430448in}{2.972512in}}%
\pgfpathlineto{\pgfqpoint{3.452983in}{3.146240in}}%
\pgfpathlineto{\pgfqpoint{3.475518in}{3.253317in}}%
\pgfpathlineto{\pgfqpoint{3.498054in}{3.332860in}}%
\pgfpathlineto{\pgfqpoint{3.520589in}{3.371684in}}%
\pgfpathlineto{\pgfqpoint{3.543124in}{3.418770in}}%
\pgfpathlineto{\pgfqpoint{3.565660in}{3.474025in}}%
\pgfpathlineto{\pgfqpoint{3.588195in}{3.512145in}}%
\pgfpathlineto{\pgfqpoint{3.610730in}{3.545700in}}%
\pgfpathlineto{\pgfqpoint{3.633265in}{3.582888in}}%
\pgfpathlineto{\pgfqpoint{3.655801in}{3.613696in}}%
\pgfpathlineto{\pgfqpoint{3.678336in}{3.644807in}}%
\pgfpathlineto{\pgfqpoint{3.700871in}{3.665386in}}%
\pgfpathlineto{\pgfqpoint{3.723407in}{3.684678in}}%
\pgfpathlineto{\pgfqpoint{3.745942in}{3.702514in}}%
\pgfpathlineto{\pgfqpoint{3.768477in}{3.713368in}}%
\pgfpathlineto{\pgfqpoint{3.791012in}{3.739647in}}%
\pgfpathlineto{\pgfqpoint{3.813548in}{3.759020in}}%
\pgfpathlineto{\pgfqpoint{3.836083in}{3.778859in}}%
\pgfpathlineto{\pgfqpoint{3.858618in}{3.794727in}}%
\pgfpathlineto{\pgfqpoint{3.881154in}{3.810841in}}%
\pgfpathlineto{\pgfqpoint{3.903689in}{3.821183in}}%
\pgfpathlineto{\pgfqpoint{3.926224in}{3.834215in}}%
\pgfpathlineto{\pgfqpoint{3.948759in}{3.844980in}}%
\pgfpathlineto{\pgfqpoint{3.971295in}{3.856178in}}%
\pgfpathlineto{\pgfqpoint{3.993830in}{3.869541in}}%
\pgfpathlineto{\pgfqpoint{4.016365in}{3.880232in}}%
\pgfpathlineto{\pgfqpoint{4.038901in}{3.889206in}}%
\pgfpathlineto{\pgfqpoint{4.061436in}{3.896781in}}%
\pgfpathlineto{\pgfqpoint{4.083971in}{3.909295in}}%
\pgfpathlineto{\pgfqpoint{4.106506in}{3.921368in}}%
\pgfpathlineto{\pgfqpoint{4.129042in}{3.934298in}}%
\pgfpathlineto{\pgfqpoint{4.151577in}{3.943576in}}%
\pgfpathlineto{\pgfqpoint{4.174112in}{3.954883in}}%
\pgfpathlineto{\pgfqpoint{4.196648in}{3.969023in}}%
\pgfpathlineto{\pgfqpoint{4.219183in}{3.979681in}}%
\pgfpathlineto{\pgfqpoint{4.241718in}{3.988340in}}%
\pgfpathlineto{\pgfqpoint{4.264253in}{3.992913in}}%
\pgfpathlineto{\pgfqpoint{4.286789in}{3.998405in}}%
\pgfpathlineto{\pgfqpoint{4.309324in}{4.003362in}}%
\pgfpathlineto{\pgfqpoint{4.331859in}{4.008707in}}%
\pgfpathlineto{\pgfqpoint{4.354395in}{4.015278in}}%
\pgfpathlineto{\pgfqpoint{4.376930in}{4.022990in}}%
\pgfpathlineto{\pgfqpoint{4.399465in}{4.028077in}}%
\pgfpathlineto{\pgfqpoint{4.422000in}{4.032658in}}%
\pgfpathlineto{\pgfqpoint{4.444536in}{4.039740in}}%
\pgfpathlineto{\pgfqpoint{4.467071in}{4.045435in}}%
\pgfpathlineto{\pgfqpoint{4.489606in}{4.052324in}}%
\pgfpathlineto{\pgfqpoint{4.512142in}{4.057277in}}%
\pgfpathlineto{\pgfqpoint{4.534677in}{4.060708in}}%
\pgfpathlineto{\pgfqpoint{4.557212in}{4.062283in}}%
\pgfpathlineto{\pgfqpoint{4.579748in}{4.066974in}}%
\pgfpathlineto{\pgfqpoint{4.602283in}{4.068699in}}%
\pgfpathlineto{\pgfqpoint{4.624818in}{4.074850in}}%
\pgfpathlineto{\pgfqpoint{4.647353in}{4.082967in}}%
\pgfpathlineto{\pgfqpoint{4.669889in}{4.086659in}}%
\pgfpathlineto{\pgfqpoint{4.692424in}{4.091015in}}%
\pgfpathlineto{\pgfqpoint{4.714959in}{4.092297in}}%
\pgfpathlineto{\pgfqpoint{4.737495in}{4.096923in}}%
\pgfpathlineto{\pgfqpoint{4.760030in}{4.101971in}}%
\pgfpathlineto{\pgfqpoint{4.782565in}{4.103684in}}%
\pgfpathlineto{\pgfqpoint{4.805100in}{4.105308in}}%
\pgfpathlineto{\pgfqpoint{4.827636in}{4.109100in}}%
\pgfpathlineto{\pgfqpoint{4.850171in}{4.114524in}}%
\pgfpathlineto{\pgfqpoint{4.872706in}{4.117036in}}%
\pgfpathlineto{\pgfqpoint{4.895242in}{4.121266in}}%
\pgfpathlineto{\pgfqpoint{4.917777in}{4.126882in}}%
\pgfpathlineto{\pgfqpoint{4.940312in}{4.130182in}}%
\pgfpathlineto{\pgfqpoint{4.962847in}{4.133294in}}%
\pgfpathlineto{\pgfqpoint{4.985383in}{4.136102in}}%
\pgfpathlineto{\pgfqpoint{5.007918in}{4.139956in}}%
\pgfpathlineto{\pgfqpoint{5.030453in}{4.143854in}}%
\pgfpathlineto{\pgfqpoint{5.052989in}{4.146081in}}%
\pgfpathlineto{\pgfqpoint{5.075524in}{4.148744in}}%
\pgfpathlineto{\pgfqpoint{5.098059in}{4.151078in}}%
\pgfpathlineto{\pgfqpoint{5.120594in}{4.153721in}}%
\pgfpathlineto{\pgfqpoint{5.143130in}{4.157724in}}%
\pgfpathlineto{\pgfqpoint{5.165665in}{4.158888in}}%
\pgfpathlineto{\pgfqpoint{5.188200in}{4.161798in}}%
\pgfpathlineto{\pgfqpoint{5.210736in}{4.165365in}}%
\pgfpathlineto{\pgfqpoint{5.233271in}{4.169084in}}%
\pgfpathlineto{\pgfqpoint{5.255806in}{4.169871in}}%
\pgfpathlineto{\pgfqpoint{5.278341in}{4.171523in}}%
\pgfpathlineto{\pgfqpoint{5.300877in}{4.173522in}}%
\pgfpathlineto{\pgfqpoint{5.323412in}{4.176211in}}%
\pgfpathlineto{\pgfqpoint{5.345947in}{4.177884in}}%
\pgfpathlineto{\pgfqpoint{5.368483in}{4.179668in}}%
\pgfpathlineto{\pgfqpoint{5.391018in}{4.181387in}}%
\pgfpathlineto{\pgfqpoint{5.413553in}{4.183359in}}%
\pgfpathlineto{\pgfqpoint{5.436088in}{4.185384in}}%
\pgfpathlineto{\pgfqpoint{5.458624in}{4.185370in}}%
\pgfpathlineto{\pgfqpoint{5.481159in}{4.187166in}}%
\pgfpathlineto{\pgfqpoint{5.503694in}{4.187760in}}%
\pgfpathlineto{\pgfqpoint{5.526230in}{4.190114in}}%
\pgfpathlineto{\pgfqpoint{5.548765in}{4.190392in}}%
\pgfpathlineto{\pgfqpoint{5.571300in}{4.191746in}}%
\pgfpathlineto{\pgfqpoint{5.593835in}{4.193207in}}%
\pgfpathlineto{\pgfqpoint{5.616371in}{4.194593in}}%
\pgfusepath{stroke}%
\end{pgfscope}%
\begin{pgfscope}%
\pgfpathrectangle{\pgfqpoint{3.385377in}{0.422992in}}{\pgfqpoint{2.276064in}{3.951201in}}%
\pgfusepath{clip}%
\pgfsetrectcap%
\pgfsetroundjoin%
\pgfsetlinewidth{1.003750pt}%
\definecolor{currentstroke}{rgb}{0.000000,0.725490,0.270588}%
\pgfsetstrokecolor{currentstroke}%
\pgfsetdash{}{0pt}%
\pgfpathmoveto{\pgfqpoint{3.407913in}{1.452661in}}%
\pgfpathlineto{\pgfqpoint{3.430448in}{1.606721in}}%
\pgfpathlineto{\pgfqpoint{3.452983in}{1.682112in}}%
\pgfpathlineto{\pgfqpoint{3.475518in}{1.767336in}}%
\pgfpathlineto{\pgfqpoint{3.498054in}{1.814974in}}%
\pgfpathlineto{\pgfqpoint{3.520589in}{1.853288in}}%
\pgfpathlineto{\pgfqpoint{3.543124in}{1.885338in}}%
\pgfpathlineto{\pgfqpoint{3.565660in}{1.918389in}}%
\pgfpathlineto{\pgfqpoint{3.588195in}{1.950894in}}%
\pgfpathlineto{\pgfqpoint{3.610730in}{1.968812in}}%
\pgfpathlineto{\pgfqpoint{3.633265in}{1.989432in}}%
\pgfpathlineto{\pgfqpoint{3.655801in}{2.009893in}}%
\pgfpathlineto{\pgfqpoint{3.678336in}{2.038469in}}%
\pgfpathlineto{\pgfqpoint{3.700871in}{2.062026in}}%
\pgfpathlineto{\pgfqpoint{3.723407in}{2.071515in}}%
\pgfpathlineto{\pgfqpoint{3.745942in}{2.092110in}}%
\pgfpathlineto{\pgfqpoint{3.768477in}{2.105141in}}%
\pgfpathlineto{\pgfqpoint{3.791012in}{2.120851in}}%
\pgfpathlineto{\pgfqpoint{3.813548in}{2.133987in}}%
\pgfpathlineto{\pgfqpoint{3.836083in}{2.148322in}}%
\pgfpathlineto{\pgfqpoint{3.858618in}{2.156402in}}%
\pgfpathlineto{\pgfqpoint{3.881154in}{2.171693in}}%
\pgfpathlineto{\pgfqpoint{3.903689in}{2.180714in}}%
\pgfpathlineto{\pgfqpoint{3.926224in}{2.192261in}}%
\pgfpathlineto{\pgfqpoint{3.948759in}{2.199388in}}%
\pgfpathlineto{\pgfqpoint{3.971295in}{2.210420in}}%
\pgfpathlineto{\pgfqpoint{3.993830in}{2.220879in}}%
\pgfpathlineto{\pgfqpoint{4.016365in}{2.226610in}}%
\pgfpathlineto{\pgfqpoint{4.038901in}{2.233679in}}%
\pgfpathlineto{\pgfqpoint{4.061436in}{2.242025in}}%
\pgfpathlineto{\pgfqpoint{4.083971in}{2.251947in}}%
\pgfpathlineto{\pgfqpoint{4.106506in}{2.261454in}}%
\pgfpathlineto{\pgfqpoint{4.129042in}{2.268133in}}%
\pgfpathlineto{\pgfqpoint{4.151577in}{2.273005in}}%
\pgfpathlineto{\pgfqpoint{4.174112in}{2.281844in}}%
\pgfpathlineto{\pgfqpoint{4.196648in}{2.292499in}}%
\pgfpathlineto{\pgfqpoint{4.219183in}{2.300038in}}%
\pgfpathlineto{\pgfqpoint{4.241718in}{2.305283in}}%
\pgfpathlineto{\pgfqpoint{4.264253in}{2.312276in}}%
\pgfpathlineto{\pgfqpoint{4.286789in}{2.317116in}}%
\pgfpathlineto{\pgfqpoint{4.309324in}{2.324172in}}%
\pgfpathlineto{\pgfqpoint{4.331859in}{2.329487in}}%
\pgfpathlineto{\pgfqpoint{4.354395in}{2.335520in}}%
\pgfpathlineto{\pgfqpoint{4.376930in}{2.341478in}}%
\pgfpathlineto{\pgfqpoint{4.399465in}{2.348433in}}%
\pgfpathlineto{\pgfqpoint{4.422000in}{2.352331in}}%
\pgfpathlineto{\pgfqpoint{4.444536in}{2.354947in}}%
\pgfpathlineto{\pgfqpoint{4.467071in}{2.361369in}}%
\pgfpathlineto{\pgfqpoint{4.489606in}{2.367306in}}%
\pgfpathlineto{\pgfqpoint{4.512142in}{2.372219in}}%
\pgfpathlineto{\pgfqpoint{4.534677in}{2.379210in}}%
\pgfpathlineto{\pgfqpoint{4.557212in}{2.382108in}}%
\pgfpathlineto{\pgfqpoint{4.579748in}{2.385762in}}%
\pgfpathlineto{\pgfqpoint{4.602283in}{2.390495in}}%
\pgfpathlineto{\pgfqpoint{4.624818in}{2.393825in}}%
\pgfpathlineto{\pgfqpoint{4.647353in}{2.399220in}}%
\pgfpathlineto{\pgfqpoint{4.669889in}{2.402778in}}%
\pgfpathlineto{\pgfqpoint{4.692424in}{2.405987in}}%
\pgfpathlineto{\pgfqpoint{4.714959in}{2.408235in}}%
\pgfpathlineto{\pgfqpoint{4.737495in}{2.411829in}}%
\pgfpathlineto{\pgfqpoint{4.760030in}{2.416057in}}%
\pgfpathlineto{\pgfqpoint{4.782565in}{2.418245in}}%
\pgfpathlineto{\pgfqpoint{4.805100in}{2.420711in}}%
\pgfpathlineto{\pgfqpoint{4.827636in}{2.423065in}}%
\pgfpathlineto{\pgfqpoint{4.850171in}{2.426188in}}%
\pgfpathlineto{\pgfqpoint{4.872706in}{2.429315in}}%
\pgfpathlineto{\pgfqpoint{4.895242in}{2.433947in}}%
\pgfpathlineto{\pgfqpoint{4.917777in}{2.437157in}}%
\pgfpathlineto{\pgfqpoint{4.940312in}{2.441510in}}%
\pgfpathlineto{\pgfqpoint{4.962847in}{2.445488in}}%
\pgfpathlineto{\pgfqpoint{4.985383in}{2.448523in}}%
\pgfpathlineto{\pgfqpoint{5.007918in}{2.451717in}}%
\pgfpathlineto{\pgfqpoint{5.030453in}{2.455691in}}%
\pgfpathlineto{\pgfqpoint{5.052989in}{2.457550in}}%
\pgfpathlineto{\pgfqpoint{5.075524in}{2.459067in}}%
\pgfpathlineto{\pgfqpoint{5.098059in}{2.461667in}}%
\pgfpathlineto{\pgfqpoint{5.120594in}{2.463091in}}%
\pgfpathlineto{\pgfqpoint{5.143130in}{2.465264in}}%
\pgfpathlineto{\pgfqpoint{5.165665in}{2.466635in}}%
\pgfpathlineto{\pgfqpoint{5.188200in}{2.468108in}}%
\pgfpathlineto{\pgfqpoint{5.210736in}{2.470165in}}%
\pgfpathlineto{\pgfqpoint{5.233271in}{2.473851in}}%
\pgfpathlineto{\pgfqpoint{5.255806in}{2.476632in}}%
\pgfpathlineto{\pgfqpoint{5.278341in}{2.478644in}}%
\pgfpathlineto{\pgfqpoint{5.300877in}{2.480352in}}%
\pgfpathlineto{\pgfqpoint{5.323412in}{2.482401in}}%
\pgfpathlineto{\pgfqpoint{5.345947in}{2.484856in}}%
\pgfpathlineto{\pgfqpoint{5.368483in}{2.486882in}}%
\pgfpathlineto{\pgfqpoint{5.391018in}{2.488150in}}%
\pgfpathlineto{\pgfqpoint{5.413553in}{2.489171in}}%
\pgfpathlineto{\pgfqpoint{5.436088in}{2.490555in}}%
\pgfpathlineto{\pgfqpoint{5.458624in}{2.493214in}}%
\pgfpathlineto{\pgfqpoint{5.481159in}{2.495229in}}%
\pgfpathlineto{\pgfqpoint{5.503694in}{2.496830in}}%
\pgfpathlineto{\pgfqpoint{5.526230in}{2.498143in}}%
\pgfpathlineto{\pgfqpoint{5.548765in}{2.499293in}}%
\pgfpathlineto{\pgfqpoint{5.571300in}{2.500441in}}%
\pgfpathlineto{\pgfqpoint{5.593835in}{2.501543in}}%
\pgfpathlineto{\pgfqpoint{5.616371in}{2.502910in}}%
\pgfusepath{stroke}%
\end{pgfscope}%
\begin{pgfscope}%
\pgfpathrectangle{\pgfqpoint{3.385377in}{0.422992in}}{\pgfqpoint{2.276064in}{3.951201in}}%
\pgfusepath{clip}%
\pgfsetrectcap%
\pgfsetroundjoin%
\pgfsetlinewidth{1.003750pt}%
\definecolor{currentstroke}{rgb}{1.000000,0.584314,0.000000}%
\pgfsetstrokecolor{currentstroke}%
\pgfsetdash{}{0pt}%
\pgfpathmoveto{\pgfqpoint{3.407913in}{2.239356in}}%
\pgfpathlineto{\pgfqpoint{3.430448in}{2.414177in}}%
\pgfpathlineto{\pgfqpoint{3.452983in}{2.535094in}}%
\pgfpathlineto{\pgfqpoint{3.475518in}{2.616312in}}%
\pgfpathlineto{\pgfqpoint{3.498054in}{2.668102in}}%
\pgfpathlineto{\pgfqpoint{3.520589in}{2.729944in}}%
\pgfpathlineto{\pgfqpoint{3.543124in}{2.770995in}}%
\pgfpathlineto{\pgfqpoint{3.565660in}{2.817626in}}%
\pgfpathlineto{\pgfqpoint{3.588195in}{2.841511in}}%
\pgfpathlineto{\pgfqpoint{3.610730in}{2.875479in}}%
\pgfpathlineto{\pgfqpoint{3.633265in}{2.894529in}}%
\pgfpathlineto{\pgfqpoint{3.655801in}{2.920603in}}%
\pgfpathlineto{\pgfqpoint{3.678336in}{2.945858in}}%
\pgfpathlineto{\pgfqpoint{3.700871in}{2.969847in}}%
\pgfpathlineto{\pgfqpoint{3.723407in}{2.988161in}}%
\pgfpathlineto{\pgfqpoint{3.745942in}{3.003366in}}%
\pgfpathlineto{\pgfqpoint{3.768477in}{3.019867in}}%
\pgfpathlineto{\pgfqpoint{3.791012in}{3.038177in}}%
\pgfpathlineto{\pgfqpoint{3.813548in}{3.048808in}}%
\pgfpathlineto{\pgfqpoint{3.836083in}{3.062091in}}%
\pgfpathlineto{\pgfqpoint{3.858618in}{3.075878in}}%
\pgfpathlineto{\pgfqpoint{3.881154in}{3.088710in}}%
\pgfpathlineto{\pgfqpoint{3.903689in}{3.100996in}}%
\pgfpathlineto{\pgfqpoint{3.926224in}{3.111893in}}%
\pgfpathlineto{\pgfqpoint{3.948759in}{3.120870in}}%
\pgfpathlineto{\pgfqpoint{3.971295in}{3.126887in}}%
\pgfpathlineto{\pgfqpoint{3.993830in}{3.139743in}}%
\pgfpathlineto{\pgfqpoint{4.016365in}{3.148090in}}%
\pgfpathlineto{\pgfqpoint{4.038901in}{3.159930in}}%
\pgfpathlineto{\pgfqpoint{4.061436in}{3.166756in}}%
\pgfpathlineto{\pgfqpoint{4.083971in}{3.172437in}}%
\pgfpathlineto{\pgfqpoint{4.106506in}{3.181246in}}%
\pgfpathlineto{\pgfqpoint{4.129042in}{3.194818in}}%
\pgfpathlineto{\pgfqpoint{4.151577in}{3.201678in}}%
\pgfpathlineto{\pgfqpoint{4.174112in}{3.207960in}}%
\pgfpathlineto{\pgfqpoint{4.196648in}{3.214377in}}%
\pgfpathlineto{\pgfqpoint{4.219183in}{3.220626in}}%
\pgfpathlineto{\pgfqpoint{4.241718in}{3.230628in}}%
\pgfpathlineto{\pgfqpoint{4.264253in}{3.232048in}}%
\pgfpathlineto{\pgfqpoint{4.286789in}{3.237550in}}%
\pgfpathlineto{\pgfqpoint{4.309324in}{3.244062in}}%
\pgfpathlineto{\pgfqpoint{4.331859in}{3.251773in}}%
\pgfpathlineto{\pgfqpoint{4.354395in}{3.256889in}}%
\pgfpathlineto{\pgfqpoint{4.376930in}{3.261226in}}%
\pgfpathlineto{\pgfqpoint{4.399465in}{3.266488in}}%
\pgfpathlineto{\pgfqpoint{4.422000in}{3.271330in}}%
\pgfpathlineto{\pgfqpoint{4.444536in}{3.274572in}}%
\pgfpathlineto{\pgfqpoint{4.467071in}{3.277906in}}%
\pgfpathlineto{\pgfqpoint{4.489606in}{3.283468in}}%
\pgfpathlineto{\pgfqpoint{4.512142in}{3.286796in}}%
\pgfpathlineto{\pgfqpoint{4.534677in}{3.290123in}}%
\pgfpathlineto{\pgfqpoint{4.557212in}{3.293406in}}%
\pgfpathlineto{\pgfqpoint{4.579748in}{3.297018in}}%
\pgfpathlineto{\pgfqpoint{4.602283in}{3.303208in}}%
\pgfpathlineto{\pgfqpoint{4.624818in}{3.308736in}}%
\pgfpathlineto{\pgfqpoint{4.647353in}{3.311881in}}%
\pgfpathlineto{\pgfqpoint{4.669889in}{3.316142in}}%
\pgfpathlineto{\pgfqpoint{4.692424in}{3.319541in}}%
\pgfpathlineto{\pgfqpoint{4.714959in}{3.323861in}}%
\pgfpathlineto{\pgfqpoint{4.737495in}{3.324250in}}%
\pgfpathlineto{\pgfqpoint{4.760030in}{3.326918in}}%
\pgfpathlineto{\pgfqpoint{4.782565in}{3.329360in}}%
\pgfpathlineto{\pgfqpoint{4.805100in}{3.331204in}}%
\pgfpathlineto{\pgfqpoint{4.827636in}{3.334150in}}%
\pgfpathlineto{\pgfqpoint{4.850171in}{3.335628in}}%
\pgfpathlineto{\pgfqpoint{4.872706in}{3.339048in}}%
\pgfpathlineto{\pgfqpoint{4.895242in}{3.341909in}}%
\pgfpathlineto{\pgfqpoint{4.917777in}{3.343047in}}%
\pgfpathlineto{\pgfqpoint{4.940312in}{3.346495in}}%
\pgfpathlineto{\pgfqpoint{4.962847in}{3.349689in}}%
\pgfpathlineto{\pgfqpoint{4.985383in}{3.351807in}}%
\pgfpathlineto{\pgfqpoint{5.007918in}{3.356872in}}%
\pgfpathlineto{\pgfqpoint{5.030453in}{3.358864in}}%
\pgfpathlineto{\pgfqpoint{5.052989in}{3.360153in}}%
\pgfpathlineto{\pgfqpoint{5.075524in}{3.363301in}}%
\pgfpathlineto{\pgfqpoint{5.098059in}{3.366482in}}%
\pgfpathlineto{\pgfqpoint{5.120594in}{3.369239in}}%
\pgfpathlineto{\pgfqpoint{5.143130in}{3.371393in}}%
\pgfpathlineto{\pgfqpoint{5.165665in}{3.373410in}}%
\pgfpathlineto{\pgfqpoint{5.188200in}{3.376086in}}%
\pgfpathlineto{\pgfqpoint{5.210736in}{3.378939in}}%
\pgfpathlineto{\pgfqpoint{5.233271in}{3.383082in}}%
\pgfpathlineto{\pgfqpoint{5.255806in}{3.385887in}}%
\pgfpathlineto{\pgfqpoint{5.278341in}{3.388027in}}%
\pgfpathlineto{\pgfqpoint{5.300877in}{3.390194in}}%
\pgfpathlineto{\pgfqpoint{5.323412in}{3.390354in}}%
\pgfpathlineto{\pgfqpoint{5.345947in}{3.394202in}}%
\pgfpathlineto{\pgfqpoint{5.368483in}{3.396176in}}%
\pgfpathlineto{\pgfqpoint{5.391018in}{3.396729in}}%
\pgfpathlineto{\pgfqpoint{5.413553in}{3.400112in}}%
\pgfpathlineto{\pgfqpoint{5.436088in}{3.402819in}}%
\pgfpathlineto{\pgfqpoint{5.458624in}{3.405254in}}%
\pgfpathlineto{\pgfqpoint{5.481159in}{3.408248in}}%
\pgfpathlineto{\pgfqpoint{5.503694in}{3.409504in}}%
\pgfpathlineto{\pgfqpoint{5.526230in}{3.411101in}}%
\pgfpathlineto{\pgfqpoint{5.548765in}{3.413712in}}%
\pgfpathlineto{\pgfqpoint{5.571300in}{3.414851in}}%
\pgfpathlineto{\pgfqpoint{5.593835in}{3.416590in}}%
\pgfpathlineto{\pgfqpoint{5.616371in}{3.418493in}}%
\pgfusepath{stroke}%
\end{pgfscope}%
\begin{pgfscope}%
\pgfpathrectangle{\pgfqpoint{3.385377in}{0.422992in}}{\pgfqpoint{2.276064in}{3.951201in}}%
\pgfusepath{clip}%
\pgfsetrectcap%
\pgfsetroundjoin%
\pgfsetlinewidth{1.003750pt}%
\definecolor{currentstroke}{rgb}{1.000000,0.172549,0.000000}%
\pgfsetstrokecolor{currentstroke}%
\pgfsetdash{}{0pt}%
\pgfpathmoveto{\pgfqpoint{3.407913in}{1.266913in}}%
\pgfpathlineto{\pgfqpoint{3.430448in}{1.311710in}}%
\pgfpathlineto{\pgfqpoint{3.452983in}{1.360878in}}%
\pgfpathlineto{\pgfqpoint{3.475518in}{1.412777in}}%
\pgfpathlineto{\pgfqpoint{3.498054in}{1.430367in}}%
\pgfpathlineto{\pgfqpoint{3.520589in}{1.439909in}}%
\pgfpathlineto{\pgfqpoint{3.543124in}{1.457338in}}%
\pgfpathlineto{\pgfqpoint{3.565660in}{1.467132in}}%
\pgfpathlineto{\pgfqpoint{3.588195in}{1.483004in}}%
\pgfpathlineto{\pgfqpoint{3.610730in}{1.490457in}}%
\pgfpathlineto{\pgfqpoint{3.633265in}{1.494966in}}%
\pgfpathlineto{\pgfqpoint{3.655801in}{1.500180in}}%
\pgfpathlineto{\pgfqpoint{3.678336in}{1.504087in}}%
\pgfpathlineto{\pgfqpoint{3.700871in}{1.509466in}}%
\pgfpathlineto{\pgfqpoint{3.723407in}{1.515583in}}%
\pgfpathlineto{\pgfqpoint{3.745942in}{1.516156in}}%
\pgfpathlineto{\pgfqpoint{3.768477in}{1.525017in}}%
\pgfpathlineto{\pgfqpoint{3.791012in}{1.526580in}}%
\pgfpathlineto{\pgfqpoint{3.813548in}{1.525793in}}%
\pgfpathlineto{\pgfqpoint{3.836083in}{1.527161in}}%
\pgfpathlineto{\pgfqpoint{3.858618in}{1.531624in}}%
\pgfpathlineto{\pgfqpoint{3.881154in}{1.535483in}}%
\pgfpathlineto{\pgfqpoint{3.903689in}{1.540716in}}%
\pgfpathlineto{\pgfqpoint{3.926224in}{1.545604in}}%
\pgfpathlineto{\pgfqpoint{3.948759in}{1.548615in}}%
\pgfpathlineto{\pgfqpoint{3.971295in}{1.549966in}}%
\pgfpathlineto{\pgfqpoint{3.993830in}{1.552673in}}%
\pgfpathlineto{\pgfqpoint{4.016365in}{1.552221in}}%
\pgfpathlineto{\pgfqpoint{4.038901in}{1.556699in}}%
\pgfpathlineto{\pgfqpoint{4.061436in}{1.557672in}}%
\pgfpathlineto{\pgfqpoint{4.083971in}{1.562742in}}%
\pgfpathlineto{\pgfqpoint{4.106506in}{1.566061in}}%
\pgfpathlineto{\pgfqpoint{4.129042in}{1.568185in}}%
\pgfpathlineto{\pgfqpoint{4.151577in}{1.571084in}}%
\pgfpathlineto{\pgfqpoint{4.174112in}{1.574442in}}%
\pgfpathlineto{\pgfqpoint{4.196648in}{1.577431in}}%
\pgfpathlineto{\pgfqpoint{4.219183in}{1.579963in}}%
\pgfpathlineto{\pgfqpoint{4.241718in}{1.582937in}}%
\pgfpathlineto{\pgfqpoint{4.264253in}{1.584022in}}%
\pgfpathlineto{\pgfqpoint{4.286789in}{1.588439in}}%
\pgfpathlineto{\pgfqpoint{4.309324in}{1.591575in}}%
\pgfpathlineto{\pgfqpoint{4.331859in}{1.593000in}}%
\pgfpathlineto{\pgfqpoint{4.354395in}{1.595935in}}%
\pgfpathlineto{\pgfqpoint{4.376930in}{1.597345in}}%
\pgfpathlineto{\pgfqpoint{4.399465in}{1.599227in}}%
\pgfpathlineto{\pgfqpoint{4.422000in}{1.600505in}}%
\pgfpathlineto{\pgfqpoint{4.444536in}{1.603402in}}%
\pgfpathlineto{\pgfqpoint{4.467071in}{1.605495in}}%
\pgfpathlineto{\pgfqpoint{4.489606in}{1.607101in}}%
\pgfpathlineto{\pgfqpoint{4.512142in}{1.608731in}}%
\pgfpathlineto{\pgfqpoint{4.534677in}{1.610296in}}%
\pgfpathlineto{\pgfqpoint{4.557212in}{1.613273in}}%
\pgfpathlineto{\pgfqpoint{4.579748in}{1.616631in}}%
\pgfpathlineto{\pgfqpoint{4.602283in}{1.617964in}}%
\pgfpathlineto{\pgfqpoint{4.624818in}{1.619009in}}%
\pgfpathlineto{\pgfqpoint{4.647353in}{1.621305in}}%
\pgfpathlineto{\pgfqpoint{4.669889in}{1.623405in}}%
\pgfpathlineto{\pgfqpoint{4.692424in}{1.626487in}}%
\pgfpathlineto{\pgfqpoint{4.714959in}{1.626651in}}%
\pgfpathlineto{\pgfqpoint{4.737495in}{1.628302in}}%
\pgfpathlineto{\pgfqpoint{4.760030in}{1.631296in}}%
\pgfpathlineto{\pgfqpoint{4.782565in}{1.632607in}}%
\pgfpathlineto{\pgfqpoint{4.805100in}{1.634397in}}%
\pgfpathlineto{\pgfqpoint{4.827636in}{1.636267in}}%
\pgfpathlineto{\pgfqpoint{4.850171in}{1.637811in}}%
\pgfpathlineto{\pgfqpoint{4.872706in}{1.639706in}}%
\pgfpathlineto{\pgfqpoint{4.895242in}{1.641218in}}%
\pgfpathlineto{\pgfqpoint{4.917777in}{1.643038in}}%
\pgfpathlineto{\pgfqpoint{4.940312in}{1.645313in}}%
\pgfpathlineto{\pgfqpoint{4.962847in}{1.646336in}}%
\pgfpathlineto{\pgfqpoint{4.985383in}{1.647762in}}%
\pgfpathlineto{\pgfqpoint{5.007918in}{1.649906in}}%
\pgfpathlineto{\pgfqpoint{5.030453in}{1.651154in}}%
\pgfpathlineto{\pgfqpoint{5.052989in}{1.651305in}}%
\pgfpathlineto{\pgfqpoint{5.075524in}{1.651189in}}%
\pgfpathlineto{\pgfqpoint{5.098059in}{1.652141in}}%
\pgfpathlineto{\pgfqpoint{5.120594in}{1.654288in}}%
\pgfpathlineto{\pgfqpoint{5.143130in}{1.655847in}}%
\pgfpathlineto{\pgfqpoint{5.165665in}{1.657506in}}%
\pgfpathlineto{\pgfqpoint{5.188200in}{1.658331in}}%
\pgfpathlineto{\pgfqpoint{5.210736in}{1.658541in}}%
\pgfpathlineto{\pgfqpoint{5.233271in}{1.659520in}}%
\pgfpathlineto{\pgfqpoint{5.255806in}{1.662318in}}%
\pgfpathlineto{\pgfqpoint{5.278341in}{1.663592in}}%
\pgfpathlineto{\pgfqpoint{5.300877in}{1.664245in}}%
\pgfpathlineto{\pgfqpoint{5.323412in}{1.665646in}}%
\pgfpathlineto{\pgfqpoint{5.345947in}{1.667616in}}%
\pgfpathlineto{\pgfqpoint{5.368483in}{1.667680in}}%
\pgfpathlineto{\pgfqpoint{5.391018in}{1.668331in}}%
\pgfpathlineto{\pgfqpoint{5.413553in}{1.669041in}}%
\pgfpathlineto{\pgfqpoint{5.436088in}{1.670312in}}%
\pgfpathlineto{\pgfqpoint{5.458624in}{1.671935in}}%
\pgfpathlineto{\pgfqpoint{5.481159in}{1.672747in}}%
\pgfpathlineto{\pgfqpoint{5.503694in}{1.673589in}}%
\pgfpathlineto{\pgfqpoint{5.526230in}{1.674390in}}%
\pgfpathlineto{\pgfqpoint{5.548765in}{1.675607in}}%
\pgfpathlineto{\pgfqpoint{5.571300in}{1.676483in}}%
\pgfpathlineto{\pgfqpoint{5.593835in}{1.677943in}}%
\pgfpathlineto{\pgfqpoint{5.616371in}{1.678889in}}%
\pgfusepath{stroke}%
\end{pgfscope}%
\begin{pgfscope}%
\pgfpathrectangle{\pgfqpoint{3.385377in}{0.422992in}}{\pgfqpoint{2.276064in}{3.951201in}}%
\pgfusepath{clip}%
\pgfsetrectcap%
\pgfsetroundjoin%
\pgfsetlinewidth{1.003750pt}%
\definecolor{currentstroke}{rgb}{0.517647,0.356863,0.592157}%
\pgfsetstrokecolor{currentstroke}%
\pgfsetdash{}{0pt}%
\pgfpathmoveto{\pgfqpoint{3.407913in}{0.602592in}}%
\pgfpathlineto{\pgfqpoint{3.430448in}{0.688909in}}%
\pgfpathlineto{\pgfqpoint{3.452983in}{0.735892in}}%
\pgfpathlineto{\pgfqpoint{3.475518in}{0.780689in}}%
\pgfpathlineto{\pgfqpoint{3.498054in}{0.806692in}}%
\pgfpathlineto{\pgfqpoint{3.520589in}{0.834590in}}%
\pgfpathlineto{\pgfqpoint{3.543124in}{0.856702in}}%
\pgfpathlineto{\pgfqpoint{3.565660in}{0.868369in}}%
\pgfpathlineto{\pgfqpoint{3.588195in}{0.893711in}}%
\pgfpathlineto{\pgfqpoint{3.610730in}{0.916388in}}%
\pgfpathlineto{\pgfqpoint{3.633265in}{0.942491in}}%
\pgfpathlineto{\pgfqpoint{3.655801in}{0.962423in}}%
\pgfpathlineto{\pgfqpoint{3.678336in}{0.987188in}}%
\pgfpathlineto{\pgfqpoint{3.700871in}{1.007011in}}%
\pgfpathlineto{\pgfqpoint{3.723407in}{1.027686in}}%
\pgfpathlineto{\pgfqpoint{3.745942in}{1.044139in}}%
\pgfpathlineto{\pgfqpoint{3.768477in}{1.060841in}}%
\pgfpathlineto{\pgfqpoint{3.791012in}{1.076901in}}%
\pgfpathlineto{\pgfqpoint{3.813548in}{1.087935in}}%
\pgfpathlineto{\pgfqpoint{3.836083in}{1.101362in}}%
\pgfpathlineto{\pgfqpoint{3.858618in}{1.113302in}}%
\pgfpathlineto{\pgfqpoint{3.881154in}{1.125745in}}%
\pgfpathlineto{\pgfqpoint{3.903689in}{1.135017in}}%
\pgfpathlineto{\pgfqpoint{3.926224in}{1.150617in}}%
\pgfpathlineto{\pgfqpoint{3.948759in}{1.163309in}}%
\pgfpathlineto{\pgfqpoint{3.971295in}{1.175193in}}%
\pgfpathlineto{\pgfqpoint{3.993830in}{1.187815in}}%
\pgfpathlineto{\pgfqpoint{4.016365in}{1.200940in}}%
\pgfpathlineto{\pgfqpoint{4.038901in}{1.212934in}}%
\pgfpathlineto{\pgfqpoint{4.061436in}{1.226969in}}%
\pgfpathlineto{\pgfqpoint{4.083971in}{1.239252in}}%
\pgfpathlineto{\pgfqpoint{4.106506in}{1.247149in}}%
\pgfpathlineto{\pgfqpoint{4.129042in}{1.258474in}}%
\pgfpathlineto{\pgfqpoint{4.151577in}{1.264440in}}%
\pgfpathlineto{\pgfqpoint{4.174112in}{1.275186in}}%
\pgfpathlineto{\pgfqpoint{4.196648in}{1.284848in}}%
\pgfpathlineto{\pgfqpoint{4.219183in}{1.296529in}}%
\pgfpathlineto{\pgfqpoint{4.241718in}{1.305754in}}%
\pgfpathlineto{\pgfqpoint{4.264253in}{1.312657in}}%
\pgfpathlineto{\pgfqpoint{4.286789in}{1.319542in}}%
\pgfpathlineto{\pgfqpoint{4.309324in}{1.328650in}}%
\pgfpathlineto{\pgfqpoint{4.331859in}{1.337324in}}%
\pgfpathlineto{\pgfqpoint{4.354395in}{1.347678in}}%
\pgfpathlineto{\pgfqpoint{4.376930in}{1.354880in}}%
\pgfpathlineto{\pgfqpoint{4.399465in}{1.361033in}}%
\pgfpathlineto{\pgfqpoint{4.422000in}{1.370054in}}%
\pgfpathlineto{\pgfqpoint{4.444536in}{1.379110in}}%
\pgfpathlineto{\pgfqpoint{4.467071in}{1.387515in}}%
\pgfpathlineto{\pgfqpoint{4.489606in}{1.396647in}}%
\pgfpathlineto{\pgfqpoint{4.512142in}{1.403447in}}%
\pgfpathlineto{\pgfqpoint{4.534677in}{1.407238in}}%
\pgfpathlineto{\pgfqpoint{4.557212in}{1.413489in}}%
\pgfpathlineto{\pgfqpoint{4.579748in}{1.418061in}}%
\pgfpathlineto{\pgfqpoint{4.602283in}{1.421654in}}%
\pgfpathlineto{\pgfqpoint{4.624818in}{1.431632in}}%
\pgfpathlineto{\pgfqpoint{4.647353in}{1.439303in}}%
\pgfpathlineto{\pgfqpoint{4.669889in}{1.441989in}}%
\pgfpathlineto{\pgfqpoint{4.692424in}{1.447183in}}%
\pgfpathlineto{\pgfqpoint{4.714959in}{1.454792in}}%
\pgfpathlineto{\pgfqpoint{4.737495in}{1.461274in}}%
\pgfpathlineto{\pgfqpoint{4.760030in}{1.469013in}}%
\pgfpathlineto{\pgfqpoint{4.782565in}{1.474809in}}%
\pgfpathlineto{\pgfqpoint{4.805100in}{1.480214in}}%
\pgfpathlineto{\pgfqpoint{4.827636in}{1.487430in}}%
\pgfpathlineto{\pgfqpoint{4.850171in}{1.493584in}}%
\pgfpathlineto{\pgfqpoint{4.872706in}{1.499716in}}%
\pgfpathlineto{\pgfqpoint{4.895242in}{1.506546in}}%
\pgfpathlineto{\pgfqpoint{4.917777in}{1.512854in}}%
\pgfpathlineto{\pgfqpoint{4.940312in}{1.520151in}}%
\pgfpathlineto{\pgfqpoint{4.962847in}{1.525835in}}%
\pgfpathlineto{\pgfqpoint{4.985383in}{1.529727in}}%
\pgfpathlineto{\pgfqpoint{5.007918in}{1.536971in}}%
\pgfpathlineto{\pgfqpoint{5.030453in}{1.543507in}}%
\pgfpathlineto{\pgfqpoint{5.052989in}{1.548686in}}%
\pgfpathlineto{\pgfqpoint{5.075524in}{1.553348in}}%
\pgfpathlineto{\pgfqpoint{5.098059in}{1.558634in}}%
\pgfpathlineto{\pgfqpoint{5.120594in}{1.564209in}}%
\pgfpathlineto{\pgfqpoint{5.143130in}{1.567428in}}%
\pgfpathlineto{\pgfqpoint{5.165665in}{1.574023in}}%
\pgfpathlineto{\pgfqpoint{5.188200in}{1.576438in}}%
\pgfpathlineto{\pgfqpoint{5.210736in}{1.582435in}}%
\pgfpathlineto{\pgfqpoint{5.233271in}{1.588659in}}%
\pgfpathlineto{\pgfqpoint{5.255806in}{1.596102in}}%
\pgfpathlineto{\pgfqpoint{5.278341in}{1.601754in}}%
\pgfpathlineto{\pgfqpoint{5.300877in}{1.607814in}}%
\pgfpathlineto{\pgfqpoint{5.323412in}{1.613326in}}%
\pgfpathlineto{\pgfqpoint{5.345947in}{1.617154in}}%
\pgfpathlineto{\pgfqpoint{5.368483in}{1.623279in}}%
\pgfpathlineto{\pgfqpoint{5.391018in}{1.626566in}}%
\pgfpathlineto{\pgfqpoint{5.413553in}{1.632110in}}%
\pgfpathlineto{\pgfqpoint{5.436088in}{1.636020in}}%
\pgfpathlineto{\pgfqpoint{5.458624in}{1.641080in}}%
\pgfpathlineto{\pgfqpoint{5.481159in}{1.645138in}}%
\pgfpathlineto{\pgfqpoint{5.503694in}{1.648691in}}%
\pgfpathlineto{\pgfqpoint{5.526230in}{1.653572in}}%
\pgfpathlineto{\pgfqpoint{5.548765in}{1.658580in}}%
\pgfpathlineto{\pgfqpoint{5.571300in}{1.662560in}}%
\pgfpathlineto{\pgfqpoint{5.593835in}{1.664943in}}%
\pgfpathlineto{\pgfqpoint{5.616371in}{1.670170in}}%
\pgfusepath{stroke}%
\end{pgfscope}%
\begin{pgfscope}%
\pgfsetrectcap%
\pgfsetmiterjoin%
\pgfsetlinewidth{0.501875pt}%
\definecolor{currentstroke}{rgb}{0.000000,0.000000,0.000000}%
\pgfsetstrokecolor{currentstroke}%
\pgfsetdash{}{0pt}%
\pgfpathmoveto{\pgfqpoint{3.385377in}{0.422992in}}%
\pgfpathlineto{\pgfqpoint{3.385377in}{4.374193in}}%
\pgfusepath{stroke}%
\end{pgfscope}%
\begin{pgfscope}%
\pgfsetrectcap%
\pgfsetmiterjoin%
\pgfsetlinewidth{0.501875pt}%
\definecolor{currentstroke}{rgb}{0.000000,0.000000,0.000000}%
\pgfsetstrokecolor{currentstroke}%
\pgfsetdash{}{0pt}%
\pgfpathmoveto{\pgfqpoint{5.661441in}{0.422992in}}%
\pgfpathlineto{\pgfqpoint{5.661441in}{4.374193in}}%
\pgfusepath{stroke}%
\end{pgfscope}%
\begin{pgfscope}%
\pgfsetrectcap%
\pgfsetmiterjoin%
\pgfsetlinewidth{0.501875pt}%
\definecolor{currentstroke}{rgb}{0.000000,0.000000,0.000000}%
\pgfsetstrokecolor{currentstroke}%
\pgfsetdash{}{0pt}%
\pgfpathmoveto{\pgfqpoint{3.385377in}{0.422992in}}%
\pgfpathlineto{\pgfqpoint{5.661441in}{0.422992in}}%
\pgfusepath{stroke}%
\end{pgfscope}%
\begin{pgfscope}%
\pgfsetrectcap%
\pgfsetmiterjoin%
\pgfsetlinewidth{0.501875pt}%
\definecolor{currentstroke}{rgb}{0.000000,0.000000,0.000000}%
\pgfsetstrokecolor{currentstroke}%
\pgfsetdash{}{0pt}%
\pgfpathmoveto{\pgfqpoint{3.385377in}{4.374193in}}%
\pgfpathlineto{\pgfqpoint{5.661441in}{4.374193in}}%
\pgfusepath{stroke}%
\end{pgfscope}%
\begin{pgfscope}%
\definecolor{textcolor}{rgb}{0.000000,0.000000,0.000000}%
\pgfsetstrokecolor{textcolor}%
\pgfsetfillcolor{textcolor}%
\pgftext[x=4.523409in,y=4.457526in,,base]{\color{textcolor}\rmfamily\fontsize{12.000000}{14.400000}\selectfont LCMC}%
\end{pgfscope}%
\begin{pgfscope}%
\pgfsetrectcap%
\pgfsetroundjoin%
\pgfsetlinewidth{1.003750pt}%
\definecolor{currentstroke}{rgb}{0.047059,0.364706,0.647059}%
\pgfsetstrokecolor{currentstroke}%
\pgfsetdash{}{0pt}%
\pgfpathmoveto{\pgfqpoint{4.382850in}{1.440922in}}%
\pgfpathlineto{\pgfqpoint{4.521739in}{1.440922in}}%
\pgfpathlineto{\pgfqpoint{4.660627in}{1.440922in}}%
\pgfusepath{stroke}%
\end{pgfscope}%
\begin{pgfscope}%
\definecolor{textcolor}{rgb}{0.000000,0.000000,0.000000}%
\pgfsetstrokecolor{textcolor}%
\pgfsetfillcolor{textcolor}%
\pgftext[x=4.771739in,y=1.392311in,left,base]{\color{textcolor}\rmfamily\fontsize{10.000000}{12.000000}\selectfont PCA}%
\end{pgfscope}%
\begin{pgfscope}%
\pgfsetrectcap%
\pgfsetroundjoin%
\pgfsetlinewidth{1.003750pt}%
\definecolor{currentstroke}{rgb}{0.000000,0.725490,0.270588}%
\pgfsetstrokecolor{currentstroke}%
\pgfsetdash{}{0pt}%
\pgfpathmoveto{\pgfqpoint{4.382850in}{1.237065in}}%
\pgfpathlineto{\pgfqpoint{4.521739in}{1.237065in}}%
\pgfpathlineto{\pgfqpoint{4.660627in}{1.237065in}}%
\pgfusepath{stroke}%
\end{pgfscope}%
\begin{pgfscope}%
\definecolor{textcolor}{rgb}{0.000000,0.000000,0.000000}%
\pgfsetstrokecolor{textcolor}%
\pgfsetfillcolor{textcolor}%
\pgftext[x=4.771739in,y=1.188454in,left,base]{\color{textcolor}\rmfamily\fontsize{10.000000}{12.000000}\selectfont KernelPCA}%
\end{pgfscope}%
\begin{pgfscope}%
\pgfsetrectcap%
\pgfsetroundjoin%
\pgfsetlinewidth{1.003750pt}%
\definecolor{currentstroke}{rgb}{1.000000,0.584314,0.000000}%
\pgfsetstrokecolor{currentstroke}%
\pgfsetdash{}{0pt}%
\pgfpathmoveto{\pgfqpoint{4.382850in}{1.033208in}}%
\pgfpathlineto{\pgfqpoint{4.521739in}{1.033208in}}%
\pgfpathlineto{\pgfqpoint{4.660627in}{1.033208in}}%
\pgfusepath{stroke}%
\end{pgfscope}%
\begin{pgfscope}%
\definecolor{textcolor}{rgb}{0.000000,0.000000,0.000000}%
\pgfsetstrokecolor{textcolor}%
\pgfsetfillcolor{textcolor}%
\pgftext[x=4.771739in,y=0.984596in,left,base]{\color{textcolor}\rmfamily\fontsize{10.000000}{12.000000}\selectfont AE}%
\end{pgfscope}%
\begin{pgfscope}%
\pgfsetrectcap%
\pgfsetroundjoin%
\pgfsetlinewidth{1.003750pt}%
\definecolor{currentstroke}{rgb}{1.000000,0.172549,0.000000}%
\pgfsetstrokecolor{currentstroke}%
\pgfsetdash{}{0pt}%
\pgfpathmoveto{\pgfqpoint{4.382850in}{0.829350in}}%
\pgfpathlineto{\pgfqpoint{4.521739in}{0.829350in}}%
\pgfpathlineto{\pgfqpoint{4.660627in}{0.829350in}}%
\pgfusepath{stroke}%
\end{pgfscope}%
\begin{pgfscope}%
\definecolor{textcolor}{rgb}{0.000000,0.000000,0.000000}%
\pgfsetstrokecolor{textcolor}%
\pgfsetfillcolor{textcolor}%
\pgftext[x=4.771739in,y=0.780739in,left,base]{\color{textcolor}\rmfamily\fontsize{10.000000}{12.000000}\selectfont LLE}%
\end{pgfscope}%
\begin{pgfscope}%
\pgfsetrectcap%
\pgfsetroundjoin%
\pgfsetlinewidth{1.003750pt}%
\definecolor{currentstroke}{rgb}{0.517647,0.356863,0.592157}%
\pgfsetstrokecolor{currentstroke}%
\pgfsetdash{}{0pt}%
\pgfpathmoveto{\pgfqpoint{4.382850in}{0.625493in}}%
\pgfpathlineto{\pgfqpoint{4.521739in}{0.625493in}}%
\pgfpathlineto{\pgfqpoint{4.660627in}{0.625493in}}%
\pgfusepath{stroke}%
\end{pgfscope}%
\begin{pgfscope}%
\definecolor{textcolor}{rgb}{0.000000,0.000000,0.000000}%
\pgfsetstrokecolor{textcolor}%
\pgfsetfillcolor{textcolor}%
\pgftext[x=4.771739in,y=0.576882in,left,base]{\color{textcolor}\rmfamily\fontsize{10.000000}{12.000000}\selectfont CAE}%
\end{pgfscope}%
\end{pgfpicture}%
\makeatother%
\endgroup%

	\end{center}
	\caption[LFW People Qualitätskriterien]{Die Vertrauenswürdigkeit und Kontinuität der Dimensionsreduktion, sowie das Local Continuity Meta-Criterion (LCMC) für den LFW People Datensatz. Die Kriterien auf diesem Datensatz zeigen ein ähnliches Bild wie auf dem Olivetti Faces Datensatz, jedoch bereitet hier die Stichprobengröße keine Probleme beim Trainieren der Autoencoder. Daher können auch der Autoencoder und der Convolutional Autoencoder wieder eine bessere Performance aufweisen. Trotzdem ist auch hier die Hauptkomponentenanalyse wieder für alle Werte der Nachbarschaftsgröße am besten. (Eigene Darstellung)}
	\label{fig:LfwPeopleMetrics}
\end{figure}

\begin{figure}[ht]
	\begin{center}
		%% Creator: Matplotlib, PGF backend
%%
%% To include the figure in your LaTeX document, write
%%   \input{<filename>.pgf}
%%
%% Make sure the required packages are loaded in your preamble
%%   \usepackage{pgf}
%%
%% Also ensure that all the required font packages are loaded; for instance,
%% the lmodern package is sometimes necessary when using math font.
%%   \usepackage{lmodern}
%%
%% Figures using additional raster images can only be included by \input if
%% they are in the same directory as the main LaTeX file. For loading figures
%% from other directories you can use the `import` package
%%   \usepackage{import}
%%
%% and then include the figures with
%%   \import{<path to file>}{<filename>.pgf}
%%
%% Matplotlib used the following preamble
%%   
%%   \usepackage{fontspec}
%%   \setmainfont{DejaVuSerif.ttf}[Path=\detokenize{/Users/moritzmistol/.pyenv/versions/3.9.13/envs/thesis/lib/python3.9/site-packages/matplotlib/mpl-data/fonts/ttf/}]
%%   \setsansfont{DejaVuSans.ttf}[Path=\detokenize{/Users/moritzmistol/.pyenv/versions/3.9.13/envs/thesis/lib/python3.9/site-packages/matplotlib/mpl-data/fonts/ttf/}]
%%   \setmonofont{DejaVuSansMono.ttf}[Path=\detokenize{/Users/moritzmistol/.pyenv/versions/3.9.13/envs/thesis/lib/python3.9/site-packages/matplotlib/mpl-data/fonts/ttf/}]
%%   \makeatletter\@ifpackageloaded{underscore}{}{\usepackage[strings]{underscore}}\makeatother
%%
\begingroup%
\makeatletter%
\begin{pgfpicture}%
\pgfpathrectangle{\pgfpointorigin}{\pgfqpoint{5.641997in}{4.634154in}}%
\pgfusepath{use as bounding box, clip}%
\begin{pgfscope}%
\pgfsetbuttcap%
\pgfsetmiterjoin%
\definecolor{currentfill}{rgb}{1.000000,1.000000,1.000000}%
\pgfsetfillcolor{currentfill}%
\pgfsetlinewidth{0.000000pt}%
\definecolor{currentstroke}{rgb}{1.000000,1.000000,1.000000}%
\pgfsetstrokecolor{currentstroke}%
\pgfsetdash{}{0pt}%
\pgfpathmoveto{\pgfqpoint{0.000000in}{-0.000000in}}%
\pgfpathlineto{\pgfqpoint{5.641997in}{-0.000000in}}%
\pgfpathlineto{\pgfqpoint{5.641997in}{4.634154in}}%
\pgfpathlineto{\pgfqpoint{0.000000in}{4.634154in}}%
\pgfpathlineto{\pgfqpoint{0.000000in}{-0.000000in}}%
\pgfpathclose%
\pgfusepath{fill}%
\end{pgfscope}%
\begin{pgfscope}%
\pgfsetbuttcap%
\pgfsetmiterjoin%
\definecolor{currentfill}{rgb}{1.000000,1.000000,1.000000}%
\pgfsetfillcolor{currentfill}%
\pgfsetlinewidth{0.000000pt}%
\definecolor{currentstroke}{rgb}{0.000000,0.000000,0.000000}%
\pgfsetstrokecolor{currentstroke}%
\pgfsetstrokeopacity{0.000000}%
\pgfsetdash{}{0pt}%
\pgfpathmoveto{\pgfqpoint{0.470525in}{2.747992in}}%
\pgfpathlineto{\pgfqpoint{2.711997in}{2.747992in}}%
\pgfpathlineto{\pgfqpoint{2.711997in}{4.374193in}}%
\pgfpathlineto{\pgfqpoint{0.470525in}{4.374193in}}%
\pgfpathlineto{\pgfqpoint{0.470525in}{2.747992in}}%
\pgfpathclose%
\pgfusepath{fill}%
\end{pgfscope}%
\begin{pgfscope}%
\pgfsetbuttcap%
\pgfsetroundjoin%
\definecolor{currentfill}{rgb}{0.000000,0.000000,0.000000}%
\pgfsetfillcolor{currentfill}%
\pgfsetlinewidth{0.501875pt}%
\definecolor{currentstroke}{rgb}{0.000000,0.000000,0.000000}%
\pgfsetstrokecolor{currentstroke}%
\pgfsetdash{}{0pt}%
\pgfsys@defobject{currentmarker}{\pgfqpoint{0.000000in}{0.000000in}}{\pgfqpoint{0.000000in}{0.041667in}}{%
\pgfpathmoveto{\pgfqpoint{0.000000in}{0.000000in}}%
\pgfpathlineto{\pgfqpoint{0.000000in}{0.041667in}}%
\pgfusepath{stroke,fill}%
}%
\begin{pgfscope}%
\pgfsys@transformshift{0.470525in}{2.747992in}%
\pgfsys@useobject{currentmarker}{}%
\end{pgfscope}%
\end{pgfscope}%
\begin{pgfscope}%
\pgfsetbuttcap%
\pgfsetroundjoin%
\definecolor{currentfill}{rgb}{0.000000,0.000000,0.000000}%
\pgfsetfillcolor{currentfill}%
\pgfsetlinewidth{0.501875pt}%
\definecolor{currentstroke}{rgb}{0.000000,0.000000,0.000000}%
\pgfsetstrokecolor{currentstroke}%
\pgfsetdash{}{0pt}%
\pgfsys@defobject{currentmarker}{\pgfqpoint{0.000000in}{-0.041667in}}{\pgfqpoint{0.000000in}{0.000000in}}{%
\pgfpathmoveto{\pgfqpoint{0.000000in}{0.000000in}}%
\pgfpathlineto{\pgfqpoint{0.000000in}{-0.041667in}}%
\pgfusepath{stroke,fill}%
}%
\begin{pgfscope}%
\pgfsys@transformshift{0.470525in}{4.374193in}%
\pgfsys@useobject{currentmarker}{}%
\end{pgfscope}%
\end{pgfscope}%
\begin{pgfscope}%
\definecolor{textcolor}{rgb}{0.000000,0.000000,0.000000}%
\pgfsetstrokecolor{textcolor}%
\pgfsetfillcolor{textcolor}%
\pgftext[x=0.470525in,y=2.699381in,,top]{\color{textcolor}\rmfamily\fontsize{10.000000}{12.000000}\selectfont \(\displaystyle {0}\)}%
\end{pgfscope}%
\begin{pgfscope}%
\pgfsetbuttcap%
\pgfsetroundjoin%
\definecolor{currentfill}{rgb}{0.000000,0.000000,0.000000}%
\pgfsetfillcolor{currentfill}%
\pgfsetlinewidth{0.501875pt}%
\definecolor{currentstroke}{rgb}{0.000000,0.000000,0.000000}%
\pgfsetstrokecolor{currentstroke}%
\pgfsetdash{}{0pt}%
\pgfsys@defobject{currentmarker}{\pgfqpoint{0.000000in}{0.000000in}}{\pgfqpoint{0.000000in}{0.041667in}}{%
\pgfpathmoveto{\pgfqpoint{0.000000in}{0.000000in}}%
\pgfpathlineto{\pgfqpoint{0.000000in}{0.041667in}}%
\pgfusepath{stroke,fill}%
}%
\begin{pgfscope}%
\pgfsys@transformshift{0.914381in}{2.747992in}%
\pgfsys@useobject{currentmarker}{}%
\end{pgfscope}%
\end{pgfscope}%
\begin{pgfscope}%
\pgfsetbuttcap%
\pgfsetroundjoin%
\definecolor{currentfill}{rgb}{0.000000,0.000000,0.000000}%
\pgfsetfillcolor{currentfill}%
\pgfsetlinewidth{0.501875pt}%
\definecolor{currentstroke}{rgb}{0.000000,0.000000,0.000000}%
\pgfsetstrokecolor{currentstroke}%
\pgfsetdash{}{0pt}%
\pgfsys@defobject{currentmarker}{\pgfqpoint{0.000000in}{-0.041667in}}{\pgfqpoint{0.000000in}{0.000000in}}{%
\pgfpathmoveto{\pgfqpoint{0.000000in}{0.000000in}}%
\pgfpathlineto{\pgfqpoint{0.000000in}{-0.041667in}}%
\pgfusepath{stroke,fill}%
}%
\begin{pgfscope}%
\pgfsys@transformshift{0.914381in}{4.374193in}%
\pgfsys@useobject{currentmarker}{}%
\end{pgfscope}%
\end{pgfscope}%
\begin{pgfscope}%
\definecolor{textcolor}{rgb}{0.000000,0.000000,0.000000}%
\pgfsetstrokecolor{textcolor}%
\pgfsetfillcolor{textcolor}%
\pgftext[x=0.914381in,y=2.699381in,,top]{\color{textcolor}\rmfamily\fontsize{10.000000}{12.000000}\selectfont \(\displaystyle {20}\)}%
\end{pgfscope}%
\begin{pgfscope}%
\pgfsetbuttcap%
\pgfsetroundjoin%
\definecolor{currentfill}{rgb}{0.000000,0.000000,0.000000}%
\pgfsetfillcolor{currentfill}%
\pgfsetlinewidth{0.501875pt}%
\definecolor{currentstroke}{rgb}{0.000000,0.000000,0.000000}%
\pgfsetstrokecolor{currentstroke}%
\pgfsetdash{}{0pt}%
\pgfsys@defobject{currentmarker}{\pgfqpoint{0.000000in}{0.000000in}}{\pgfqpoint{0.000000in}{0.041667in}}{%
\pgfpathmoveto{\pgfqpoint{0.000000in}{0.000000in}}%
\pgfpathlineto{\pgfqpoint{0.000000in}{0.041667in}}%
\pgfusepath{stroke,fill}%
}%
\begin{pgfscope}%
\pgfsys@transformshift{1.358237in}{2.747992in}%
\pgfsys@useobject{currentmarker}{}%
\end{pgfscope}%
\end{pgfscope}%
\begin{pgfscope}%
\pgfsetbuttcap%
\pgfsetroundjoin%
\definecolor{currentfill}{rgb}{0.000000,0.000000,0.000000}%
\pgfsetfillcolor{currentfill}%
\pgfsetlinewidth{0.501875pt}%
\definecolor{currentstroke}{rgb}{0.000000,0.000000,0.000000}%
\pgfsetstrokecolor{currentstroke}%
\pgfsetdash{}{0pt}%
\pgfsys@defobject{currentmarker}{\pgfqpoint{0.000000in}{-0.041667in}}{\pgfqpoint{0.000000in}{0.000000in}}{%
\pgfpathmoveto{\pgfqpoint{0.000000in}{0.000000in}}%
\pgfpathlineto{\pgfqpoint{0.000000in}{-0.041667in}}%
\pgfusepath{stroke,fill}%
}%
\begin{pgfscope}%
\pgfsys@transformshift{1.358237in}{4.374193in}%
\pgfsys@useobject{currentmarker}{}%
\end{pgfscope}%
\end{pgfscope}%
\begin{pgfscope}%
\definecolor{textcolor}{rgb}{0.000000,0.000000,0.000000}%
\pgfsetstrokecolor{textcolor}%
\pgfsetfillcolor{textcolor}%
\pgftext[x=1.358237in,y=2.699381in,,top]{\color{textcolor}\rmfamily\fontsize{10.000000}{12.000000}\selectfont \(\displaystyle {40}\)}%
\end{pgfscope}%
\begin{pgfscope}%
\pgfsetbuttcap%
\pgfsetroundjoin%
\definecolor{currentfill}{rgb}{0.000000,0.000000,0.000000}%
\pgfsetfillcolor{currentfill}%
\pgfsetlinewidth{0.501875pt}%
\definecolor{currentstroke}{rgb}{0.000000,0.000000,0.000000}%
\pgfsetstrokecolor{currentstroke}%
\pgfsetdash{}{0pt}%
\pgfsys@defobject{currentmarker}{\pgfqpoint{0.000000in}{0.000000in}}{\pgfqpoint{0.000000in}{0.041667in}}{%
\pgfpathmoveto{\pgfqpoint{0.000000in}{0.000000in}}%
\pgfpathlineto{\pgfqpoint{0.000000in}{0.041667in}}%
\pgfusepath{stroke,fill}%
}%
\begin{pgfscope}%
\pgfsys@transformshift{1.802092in}{2.747992in}%
\pgfsys@useobject{currentmarker}{}%
\end{pgfscope}%
\end{pgfscope}%
\begin{pgfscope}%
\pgfsetbuttcap%
\pgfsetroundjoin%
\definecolor{currentfill}{rgb}{0.000000,0.000000,0.000000}%
\pgfsetfillcolor{currentfill}%
\pgfsetlinewidth{0.501875pt}%
\definecolor{currentstroke}{rgb}{0.000000,0.000000,0.000000}%
\pgfsetstrokecolor{currentstroke}%
\pgfsetdash{}{0pt}%
\pgfsys@defobject{currentmarker}{\pgfqpoint{0.000000in}{-0.041667in}}{\pgfqpoint{0.000000in}{0.000000in}}{%
\pgfpathmoveto{\pgfqpoint{0.000000in}{0.000000in}}%
\pgfpathlineto{\pgfqpoint{0.000000in}{-0.041667in}}%
\pgfusepath{stroke,fill}%
}%
\begin{pgfscope}%
\pgfsys@transformshift{1.802092in}{4.374193in}%
\pgfsys@useobject{currentmarker}{}%
\end{pgfscope}%
\end{pgfscope}%
\begin{pgfscope}%
\definecolor{textcolor}{rgb}{0.000000,0.000000,0.000000}%
\pgfsetstrokecolor{textcolor}%
\pgfsetfillcolor{textcolor}%
\pgftext[x=1.802092in,y=2.699381in,,top]{\color{textcolor}\rmfamily\fontsize{10.000000}{12.000000}\selectfont \(\displaystyle {60}\)}%
\end{pgfscope}%
\begin{pgfscope}%
\pgfsetbuttcap%
\pgfsetroundjoin%
\definecolor{currentfill}{rgb}{0.000000,0.000000,0.000000}%
\pgfsetfillcolor{currentfill}%
\pgfsetlinewidth{0.501875pt}%
\definecolor{currentstroke}{rgb}{0.000000,0.000000,0.000000}%
\pgfsetstrokecolor{currentstroke}%
\pgfsetdash{}{0pt}%
\pgfsys@defobject{currentmarker}{\pgfqpoint{0.000000in}{0.000000in}}{\pgfqpoint{0.000000in}{0.041667in}}{%
\pgfpathmoveto{\pgfqpoint{0.000000in}{0.000000in}}%
\pgfpathlineto{\pgfqpoint{0.000000in}{0.041667in}}%
\pgfusepath{stroke,fill}%
}%
\begin{pgfscope}%
\pgfsys@transformshift{2.245948in}{2.747992in}%
\pgfsys@useobject{currentmarker}{}%
\end{pgfscope}%
\end{pgfscope}%
\begin{pgfscope}%
\pgfsetbuttcap%
\pgfsetroundjoin%
\definecolor{currentfill}{rgb}{0.000000,0.000000,0.000000}%
\pgfsetfillcolor{currentfill}%
\pgfsetlinewidth{0.501875pt}%
\definecolor{currentstroke}{rgb}{0.000000,0.000000,0.000000}%
\pgfsetstrokecolor{currentstroke}%
\pgfsetdash{}{0pt}%
\pgfsys@defobject{currentmarker}{\pgfqpoint{0.000000in}{-0.041667in}}{\pgfqpoint{0.000000in}{0.000000in}}{%
\pgfpathmoveto{\pgfqpoint{0.000000in}{0.000000in}}%
\pgfpathlineto{\pgfqpoint{0.000000in}{-0.041667in}}%
\pgfusepath{stroke,fill}%
}%
\begin{pgfscope}%
\pgfsys@transformshift{2.245948in}{4.374193in}%
\pgfsys@useobject{currentmarker}{}%
\end{pgfscope}%
\end{pgfscope}%
\begin{pgfscope}%
\definecolor{textcolor}{rgb}{0.000000,0.000000,0.000000}%
\pgfsetstrokecolor{textcolor}%
\pgfsetfillcolor{textcolor}%
\pgftext[x=2.245948in,y=2.699381in,,top]{\color{textcolor}\rmfamily\fontsize{10.000000}{12.000000}\selectfont \(\displaystyle {80}\)}%
\end{pgfscope}%
\begin{pgfscope}%
\pgfsetbuttcap%
\pgfsetroundjoin%
\definecolor{currentfill}{rgb}{0.000000,0.000000,0.000000}%
\pgfsetfillcolor{currentfill}%
\pgfsetlinewidth{0.501875pt}%
\definecolor{currentstroke}{rgb}{0.000000,0.000000,0.000000}%
\pgfsetstrokecolor{currentstroke}%
\pgfsetdash{}{0pt}%
\pgfsys@defobject{currentmarker}{\pgfqpoint{0.000000in}{0.000000in}}{\pgfqpoint{0.000000in}{0.020833in}}{%
\pgfpathmoveto{\pgfqpoint{0.000000in}{0.000000in}}%
\pgfpathlineto{\pgfqpoint{0.000000in}{0.020833in}}%
\pgfusepath{stroke,fill}%
}%
\begin{pgfscope}%
\pgfsys@transformshift{0.581489in}{2.747992in}%
\pgfsys@useobject{currentmarker}{}%
\end{pgfscope}%
\end{pgfscope}%
\begin{pgfscope}%
\pgfsetbuttcap%
\pgfsetroundjoin%
\definecolor{currentfill}{rgb}{0.000000,0.000000,0.000000}%
\pgfsetfillcolor{currentfill}%
\pgfsetlinewidth{0.501875pt}%
\definecolor{currentstroke}{rgb}{0.000000,0.000000,0.000000}%
\pgfsetstrokecolor{currentstroke}%
\pgfsetdash{}{0pt}%
\pgfsys@defobject{currentmarker}{\pgfqpoint{0.000000in}{-0.020833in}}{\pgfqpoint{0.000000in}{0.000000in}}{%
\pgfpathmoveto{\pgfqpoint{0.000000in}{0.000000in}}%
\pgfpathlineto{\pgfqpoint{0.000000in}{-0.020833in}}%
\pgfusepath{stroke,fill}%
}%
\begin{pgfscope}%
\pgfsys@transformshift{0.581489in}{4.374193in}%
\pgfsys@useobject{currentmarker}{}%
\end{pgfscope}%
\end{pgfscope}%
\begin{pgfscope}%
\pgfsetbuttcap%
\pgfsetroundjoin%
\definecolor{currentfill}{rgb}{0.000000,0.000000,0.000000}%
\pgfsetfillcolor{currentfill}%
\pgfsetlinewidth{0.501875pt}%
\definecolor{currentstroke}{rgb}{0.000000,0.000000,0.000000}%
\pgfsetstrokecolor{currentstroke}%
\pgfsetdash{}{0pt}%
\pgfsys@defobject{currentmarker}{\pgfqpoint{0.000000in}{0.000000in}}{\pgfqpoint{0.000000in}{0.020833in}}{%
\pgfpathmoveto{\pgfqpoint{0.000000in}{0.000000in}}%
\pgfpathlineto{\pgfqpoint{0.000000in}{0.020833in}}%
\pgfusepath{stroke,fill}%
}%
\begin{pgfscope}%
\pgfsys@transformshift{0.692453in}{2.747992in}%
\pgfsys@useobject{currentmarker}{}%
\end{pgfscope}%
\end{pgfscope}%
\begin{pgfscope}%
\pgfsetbuttcap%
\pgfsetroundjoin%
\definecolor{currentfill}{rgb}{0.000000,0.000000,0.000000}%
\pgfsetfillcolor{currentfill}%
\pgfsetlinewidth{0.501875pt}%
\definecolor{currentstroke}{rgb}{0.000000,0.000000,0.000000}%
\pgfsetstrokecolor{currentstroke}%
\pgfsetdash{}{0pt}%
\pgfsys@defobject{currentmarker}{\pgfqpoint{0.000000in}{-0.020833in}}{\pgfqpoint{0.000000in}{0.000000in}}{%
\pgfpathmoveto{\pgfqpoint{0.000000in}{0.000000in}}%
\pgfpathlineto{\pgfqpoint{0.000000in}{-0.020833in}}%
\pgfusepath{stroke,fill}%
}%
\begin{pgfscope}%
\pgfsys@transformshift{0.692453in}{4.374193in}%
\pgfsys@useobject{currentmarker}{}%
\end{pgfscope}%
\end{pgfscope}%
\begin{pgfscope}%
\pgfsetbuttcap%
\pgfsetroundjoin%
\definecolor{currentfill}{rgb}{0.000000,0.000000,0.000000}%
\pgfsetfillcolor{currentfill}%
\pgfsetlinewidth{0.501875pt}%
\definecolor{currentstroke}{rgb}{0.000000,0.000000,0.000000}%
\pgfsetstrokecolor{currentstroke}%
\pgfsetdash{}{0pt}%
\pgfsys@defobject{currentmarker}{\pgfqpoint{0.000000in}{0.000000in}}{\pgfqpoint{0.000000in}{0.020833in}}{%
\pgfpathmoveto{\pgfqpoint{0.000000in}{0.000000in}}%
\pgfpathlineto{\pgfqpoint{0.000000in}{0.020833in}}%
\pgfusepath{stroke,fill}%
}%
\begin{pgfscope}%
\pgfsys@transformshift{0.803417in}{2.747992in}%
\pgfsys@useobject{currentmarker}{}%
\end{pgfscope}%
\end{pgfscope}%
\begin{pgfscope}%
\pgfsetbuttcap%
\pgfsetroundjoin%
\definecolor{currentfill}{rgb}{0.000000,0.000000,0.000000}%
\pgfsetfillcolor{currentfill}%
\pgfsetlinewidth{0.501875pt}%
\definecolor{currentstroke}{rgb}{0.000000,0.000000,0.000000}%
\pgfsetstrokecolor{currentstroke}%
\pgfsetdash{}{0pt}%
\pgfsys@defobject{currentmarker}{\pgfqpoint{0.000000in}{-0.020833in}}{\pgfqpoint{0.000000in}{0.000000in}}{%
\pgfpathmoveto{\pgfqpoint{0.000000in}{0.000000in}}%
\pgfpathlineto{\pgfqpoint{0.000000in}{-0.020833in}}%
\pgfusepath{stroke,fill}%
}%
\begin{pgfscope}%
\pgfsys@transformshift{0.803417in}{4.374193in}%
\pgfsys@useobject{currentmarker}{}%
\end{pgfscope}%
\end{pgfscope}%
\begin{pgfscope}%
\pgfsetbuttcap%
\pgfsetroundjoin%
\definecolor{currentfill}{rgb}{0.000000,0.000000,0.000000}%
\pgfsetfillcolor{currentfill}%
\pgfsetlinewidth{0.501875pt}%
\definecolor{currentstroke}{rgb}{0.000000,0.000000,0.000000}%
\pgfsetstrokecolor{currentstroke}%
\pgfsetdash{}{0pt}%
\pgfsys@defobject{currentmarker}{\pgfqpoint{0.000000in}{0.000000in}}{\pgfqpoint{0.000000in}{0.020833in}}{%
\pgfpathmoveto{\pgfqpoint{0.000000in}{0.000000in}}%
\pgfpathlineto{\pgfqpoint{0.000000in}{0.020833in}}%
\pgfusepath{stroke,fill}%
}%
\begin{pgfscope}%
\pgfsys@transformshift{1.025345in}{2.747992in}%
\pgfsys@useobject{currentmarker}{}%
\end{pgfscope}%
\end{pgfscope}%
\begin{pgfscope}%
\pgfsetbuttcap%
\pgfsetroundjoin%
\definecolor{currentfill}{rgb}{0.000000,0.000000,0.000000}%
\pgfsetfillcolor{currentfill}%
\pgfsetlinewidth{0.501875pt}%
\definecolor{currentstroke}{rgb}{0.000000,0.000000,0.000000}%
\pgfsetstrokecolor{currentstroke}%
\pgfsetdash{}{0pt}%
\pgfsys@defobject{currentmarker}{\pgfqpoint{0.000000in}{-0.020833in}}{\pgfqpoint{0.000000in}{0.000000in}}{%
\pgfpathmoveto{\pgfqpoint{0.000000in}{0.000000in}}%
\pgfpathlineto{\pgfqpoint{0.000000in}{-0.020833in}}%
\pgfusepath{stroke,fill}%
}%
\begin{pgfscope}%
\pgfsys@transformshift{1.025345in}{4.374193in}%
\pgfsys@useobject{currentmarker}{}%
\end{pgfscope}%
\end{pgfscope}%
\begin{pgfscope}%
\pgfsetbuttcap%
\pgfsetroundjoin%
\definecolor{currentfill}{rgb}{0.000000,0.000000,0.000000}%
\pgfsetfillcolor{currentfill}%
\pgfsetlinewidth{0.501875pt}%
\definecolor{currentstroke}{rgb}{0.000000,0.000000,0.000000}%
\pgfsetstrokecolor{currentstroke}%
\pgfsetdash{}{0pt}%
\pgfsys@defobject{currentmarker}{\pgfqpoint{0.000000in}{0.000000in}}{\pgfqpoint{0.000000in}{0.020833in}}{%
\pgfpathmoveto{\pgfqpoint{0.000000in}{0.000000in}}%
\pgfpathlineto{\pgfqpoint{0.000000in}{0.020833in}}%
\pgfusepath{stroke,fill}%
}%
\begin{pgfscope}%
\pgfsys@transformshift{1.136309in}{2.747992in}%
\pgfsys@useobject{currentmarker}{}%
\end{pgfscope}%
\end{pgfscope}%
\begin{pgfscope}%
\pgfsetbuttcap%
\pgfsetroundjoin%
\definecolor{currentfill}{rgb}{0.000000,0.000000,0.000000}%
\pgfsetfillcolor{currentfill}%
\pgfsetlinewidth{0.501875pt}%
\definecolor{currentstroke}{rgb}{0.000000,0.000000,0.000000}%
\pgfsetstrokecolor{currentstroke}%
\pgfsetdash{}{0pt}%
\pgfsys@defobject{currentmarker}{\pgfqpoint{0.000000in}{-0.020833in}}{\pgfqpoint{0.000000in}{0.000000in}}{%
\pgfpathmoveto{\pgfqpoint{0.000000in}{0.000000in}}%
\pgfpathlineto{\pgfqpoint{0.000000in}{-0.020833in}}%
\pgfusepath{stroke,fill}%
}%
\begin{pgfscope}%
\pgfsys@transformshift{1.136309in}{4.374193in}%
\pgfsys@useobject{currentmarker}{}%
\end{pgfscope}%
\end{pgfscope}%
\begin{pgfscope}%
\pgfsetbuttcap%
\pgfsetroundjoin%
\definecolor{currentfill}{rgb}{0.000000,0.000000,0.000000}%
\pgfsetfillcolor{currentfill}%
\pgfsetlinewidth{0.501875pt}%
\definecolor{currentstroke}{rgb}{0.000000,0.000000,0.000000}%
\pgfsetstrokecolor{currentstroke}%
\pgfsetdash{}{0pt}%
\pgfsys@defobject{currentmarker}{\pgfqpoint{0.000000in}{0.000000in}}{\pgfqpoint{0.000000in}{0.020833in}}{%
\pgfpathmoveto{\pgfqpoint{0.000000in}{0.000000in}}%
\pgfpathlineto{\pgfqpoint{0.000000in}{0.020833in}}%
\pgfusepath{stroke,fill}%
}%
\begin{pgfscope}%
\pgfsys@transformshift{1.247273in}{2.747992in}%
\pgfsys@useobject{currentmarker}{}%
\end{pgfscope}%
\end{pgfscope}%
\begin{pgfscope}%
\pgfsetbuttcap%
\pgfsetroundjoin%
\definecolor{currentfill}{rgb}{0.000000,0.000000,0.000000}%
\pgfsetfillcolor{currentfill}%
\pgfsetlinewidth{0.501875pt}%
\definecolor{currentstroke}{rgb}{0.000000,0.000000,0.000000}%
\pgfsetstrokecolor{currentstroke}%
\pgfsetdash{}{0pt}%
\pgfsys@defobject{currentmarker}{\pgfqpoint{0.000000in}{-0.020833in}}{\pgfqpoint{0.000000in}{0.000000in}}{%
\pgfpathmoveto{\pgfqpoint{0.000000in}{0.000000in}}%
\pgfpathlineto{\pgfqpoint{0.000000in}{-0.020833in}}%
\pgfusepath{stroke,fill}%
}%
\begin{pgfscope}%
\pgfsys@transformshift{1.247273in}{4.374193in}%
\pgfsys@useobject{currentmarker}{}%
\end{pgfscope}%
\end{pgfscope}%
\begin{pgfscope}%
\pgfsetbuttcap%
\pgfsetroundjoin%
\definecolor{currentfill}{rgb}{0.000000,0.000000,0.000000}%
\pgfsetfillcolor{currentfill}%
\pgfsetlinewidth{0.501875pt}%
\definecolor{currentstroke}{rgb}{0.000000,0.000000,0.000000}%
\pgfsetstrokecolor{currentstroke}%
\pgfsetdash{}{0pt}%
\pgfsys@defobject{currentmarker}{\pgfqpoint{0.000000in}{0.000000in}}{\pgfqpoint{0.000000in}{0.020833in}}{%
\pgfpathmoveto{\pgfqpoint{0.000000in}{0.000000in}}%
\pgfpathlineto{\pgfqpoint{0.000000in}{0.020833in}}%
\pgfusepath{stroke,fill}%
}%
\begin{pgfscope}%
\pgfsys@transformshift{1.469201in}{2.747992in}%
\pgfsys@useobject{currentmarker}{}%
\end{pgfscope}%
\end{pgfscope}%
\begin{pgfscope}%
\pgfsetbuttcap%
\pgfsetroundjoin%
\definecolor{currentfill}{rgb}{0.000000,0.000000,0.000000}%
\pgfsetfillcolor{currentfill}%
\pgfsetlinewidth{0.501875pt}%
\definecolor{currentstroke}{rgb}{0.000000,0.000000,0.000000}%
\pgfsetstrokecolor{currentstroke}%
\pgfsetdash{}{0pt}%
\pgfsys@defobject{currentmarker}{\pgfqpoint{0.000000in}{-0.020833in}}{\pgfqpoint{0.000000in}{0.000000in}}{%
\pgfpathmoveto{\pgfqpoint{0.000000in}{0.000000in}}%
\pgfpathlineto{\pgfqpoint{0.000000in}{-0.020833in}}%
\pgfusepath{stroke,fill}%
}%
\begin{pgfscope}%
\pgfsys@transformshift{1.469201in}{4.374193in}%
\pgfsys@useobject{currentmarker}{}%
\end{pgfscope}%
\end{pgfscope}%
\begin{pgfscope}%
\pgfsetbuttcap%
\pgfsetroundjoin%
\definecolor{currentfill}{rgb}{0.000000,0.000000,0.000000}%
\pgfsetfillcolor{currentfill}%
\pgfsetlinewidth{0.501875pt}%
\definecolor{currentstroke}{rgb}{0.000000,0.000000,0.000000}%
\pgfsetstrokecolor{currentstroke}%
\pgfsetdash{}{0pt}%
\pgfsys@defobject{currentmarker}{\pgfqpoint{0.000000in}{0.000000in}}{\pgfqpoint{0.000000in}{0.020833in}}{%
\pgfpathmoveto{\pgfqpoint{0.000000in}{0.000000in}}%
\pgfpathlineto{\pgfqpoint{0.000000in}{0.020833in}}%
\pgfusepath{stroke,fill}%
}%
\begin{pgfscope}%
\pgfsys@transformshift{1.580165in}{2.747992in}%
\pgfsys@useobject{currentmarker}{}%
\end{pgfscope}%
\end{pgfscope}%
\begin{pgfscope}%
\pgfsetbuttcap%
\pgfsetroundjoin%
\definecolor{currentfill}{rgb}{0.000000,0.000000,0.000000}%
\pgfsetfillcolor{currentfill}%
\pgfsetlinewidth{0.501875pt}%
\definecolor{currentstroke}{rgb}{0.000000,0.000000,0.000000}%
\pgfsetstrokecolor{currentstroke}%
\pgfsetdash{}{0pt}%
\pgfsys@defobject{currentmarker}{\pgfqpoint{0.000000in}{-0.020833in}}{\pgfqpoint{0.000000in}{0.000000in}}{%
\pgfpathmoveto{\pgfqpoint{0.000000in}{0.000000in}}%
\pgfpathlineto{\pgfqpoint{0.000000in}{-0.020833in}}%
\pgfusepath{stroke,fill}%
}%
\begin{pgfscope}%
\pgfsys@transformshift{1.580165in}{4.374193in}%
\pgfsys@useobject{currentmarker}{}%
\end{pgfscope}%
\end{pgfscope}%
\begin{pgfscope}%
\pgfsetbuttcap%
\pgfsetroundjoin%
\definecolor{currentfill}{rgb}{0.000000,0.000000,0.000000}%
\pgfsetfillcolor{currentfill}%
\pgfsetlinewidth{0.501875pt}%
\definecolor{currentstroke}{rgb}{0.000000,0.000000,0.000000}%
\pgfsetstrokecolor{currentstroke}%
\pgfsetdash{}{0pt}%
\pgfsys@defobject{currentmarker}{\pgfqpoint{0.000000in}{0.000000in}}{\pgfqpoint{0.000000in}{0.020833in}}{%
\pgfpathmoveto{\pgfqpoint{0.000000in}{0.000000in}}%
\pgfpathlineto{\pgfqpoint{0.000000in}{0.020833in}}%
\pgfusepath{stroke,fill}%
}%
\begin{pgfscope}%
\pgfsys@transformshift{1.691129in}{2.747992in}%
\pgfsys@useobject{currentmarker}{}%
\end{pgfscope}%
\end{pgfscope}%
\begin{pgfscope}%
\pgfsetbuttcap%
\pgfsetroundjoin%
\definecolor{currentfill}{rgb}{0.000000,0.000000,0.000000}%
\pgfsetfillcolor{currentfill}%
\pgfsetlinewidth{0.501875pt}%
\definecolor{currentstroke}{rgb}{0.000000,0.000000,0.000000}%
\pgfsetstrokecolor{currentstroke}%
\pgfsetdash{}{0pt}%
\pgfsys@defobject{currentmarker}{\pgfqpoint{0.000000in}{-0.020833in}}{\pgfqpoint{0.000000in}{0.000000in}}{%
\pgfpathmoveto{\pgfqpoint{0.000000in}{0.000000in}}%
\pgfpathlineto{\pgfqpoint{0.000000in}{-0.020833in}}%
\pgfusepath{stroke,fill}%
}%
\begin{pgfscope}%
\pgfsys@transformshift{1.691129in}{4.374193in}%
\pgfsys@useobject{currentmarker}{}%
\end{pgfscope}%
\end{pgfscope}%
\begin{pgfscope}%
\pgfsetbuttcap%
\pgfsetroundjoin%
\definecolor{currentfill}{rgb}{0.000000,0.000000,0.000000}%
\pgfsetfillcolor{currentfill}%
\pgfsetlinewidth{0.501875pt}%
\definecolor{currentstroke}{rgb}{0.000000,0.000000,0.000000}%
\pgfsetstrokecolor{currentstroke}%
\pgfsetdash{}{0pt}%
\pgfsys@defobject{currentmarker}{\pgfqpoint{0.000000in}{0.000000in}}{\pgfqpoint{0.000000in}{0.020833in}}{%
\pgfpathmoveto{\pgfqpoint{0.000000in}{0.000000in}}%
\pgfpathlineto{\pgfqpoint{0.000000in}{0.020833in}}%
\pgfusepath{stroke,fill}%
}%
\begin{pgfscope}%
\pgfsys@transformshift{1.913056in}{2.747992in}%
\pgfsys@useobject{currentmarker}{}%
\end{pgfscope}%
\end{pgfscope}%
\begin{pgfscope}%
\pgfsetbuttcap%
\pgfsetroundjoin%
\definecolor{currentfill}{rgb}{0.000000,0.000000,0.000000}%
\pgfsetfillcolor{currentfill}%
\pgfsetlinewidth{0.501875pt}%
\definecolor{currentstroke}{rgb}{0.000000,0.000000,0.000000}%
\pgfsetstrokecolor{currentstroke}%
\pgfsetdash{}{0pt}%
\pgfsys@defobject{currentmarker}{\pgfqpoint{0.000000in}{-0.020833in}}{\pgfqpoint{0.000000in}{0.000000in}}{%
\pgfpathmoveto{\pgfqpoint{0.000000in}{0.000000in}}%
\pgfpathlineto{\pgfqpoint{0.000000in}{-0.020833in}}%
\pgfusepath{stroke,fill}%
}%
\begin{pgfscope}%
\pgfsys@transformshift{1.913056in}{4.374193in}%
\pgfsys@useobject{currentmarker}{}%
\end{pgfscope}%
\end{pgfscope}%
\begin{pgfscope}%
\pgfsetbuttcap%
\pgfsetroundjoin%
\definecolor{currentfill}{rgb}{0.000000,0.000000,0.000000}%
\pgfsetfillcolor{currentfill}%
\pgfsetlinewidth{0.501875pt}%
\definecolor{currentstroke}{rgb}{0.000000,0.000000,0.000000}%
\pgfsetstrokecolor{currentstroke}%
\pgfsetdash{}{0pt}%
\pgfsys@defobject{currentmarker}{\pgfqpoint{0.000000in}{0.000000in}}{\pgfqpoint{0.000000in}{0.020833in}}{%
\pgfpathmoveto{\pgfqpoint{0.000000in}{0.000000in}}%
\pgfpathlineto{\pgfqpoint{0.000000in}{0.020833in}}%
\pgfusepath{stroke,fill}%
}%
\begin{pgfscope}%
\pgfsys@transformshift{2.024020in}{2.747992in}%
\pgfsys@useobject{currentmarker}{}%
\end{pgfscope}%
\end{pgfscope}%
\begin{pgfscope}%
\pgfsetbuttcap%
\pgfsetroundjoin%
\definecolor{currentfill}{rgb}{0.000000,0.000000,0.000000}%
\pgfsetfillcolor{currentfill}%
\pgfsetlinewidth{0.501875pt}%
\definecolor{currentstroke}{rgb}{0.000000,0.000000,0.000000}%
\pgfsetstrokecolor{currentstroke}%
\pgfsetdash{}{0pt}%
\pgfsys@defobject{currentmarker}{\pgfqpoint{0.000000in}{-0.020833in}}{\pgfqpoint{0.000000in}{0.000000in}}{%
\pgfpathmoveto{\pgfqpoint{0.000000in}{0.000000in}}%
\pgfpathlineto{\pgfqpoint{0.000000in}{-0.020833in}}%
\pgfusepath{stroke,fill}%
}%
\begin{pgfscope}%
\pgfsys@transformshift{2.024020in}{4.374193in}%
\pgfsys@useobject{currentmarker}{}%
\end{pgfscope}%
\end{pgfscope}%
\begin{pgfscope}%
\pgfsetbuttcap%
\pgfsetroundjoin%
\definecolor{currentfill}{rgb}{0.000000,0.000000,0.000000}%
\pgfsetfillcolor{currentfill}%
\pgfsetlinewidth{0.501875pt}%
\definecolor{currentstroke}{rgb}{0.000000,0.000000,0.000000}%
\pgfsetstrokecolor{currentstroke}%
\pgfsetdash{}{0pt}%
\pgfsys@defobject{currentmarker}{\pgfqpoint{0.000000in}{0.000000in}}{\pgfqpoint{0.000000in}{0.020833in}}{%
\pgfpathmoveto{\pgfqpoint{0.000000in}{0.000000in}}%
\pgfpathlineto{\pgfqpoint{0.000000in}{0.020833in}}%
\pgfusepath{stroke,fill}%
}%
\begin{pgfscope}%
\pgfsys@transformshift{2.134984in}{2.747992in}%
\pgfsys@useobject{currentmarker}{}%
\end{pgfscope}%
\end{pgfscope}%
\begin{pgfscope}%
\pgfsetbuttcap%
\pgfsetroundjoin%
\definecolor{currentfill}{rgb}{0.000000,0.000000,0.000000}%
\pgfsetfillcolor{currentfill}%
\pgfsetlinewidth{0.501875pt}%
\definecolor{currentstroke}{rgb}{0.000000,0.000000,0.000000}%
\pgfsetstrokecolor{currentstroke}%
\pgfsetdash{}{0pt}%
\pgfsys@defobject{currentmarker}{\pgfqpoint{0.000000in}{-0.020833in}}{\pgfqpoint{0.000000in}{0.000000in}}{%
\pgfpathmoveto{\pgfqpoint{0.000000in}{0.000000in}}%
\pgfpathlineto{\pgfqpoint{0.000000in}{-0.020833in}}%
\pgfusepath{stroke,fill}%
}%
\begin{pgfscope}%
\pgfsys@transformshift{2.134984in}{4.374193in}%
\pgfsys@useobject{currentmarker}{}%
\end{pgfscope}%
\end{pgfscope}%
\begin{pgfscope}%
\pgfsetbuttcap%
\pgfsetroundjoin%
\definecolor{currentfill}{rgb}{0.000000,0.000000,0.000000}%
\pgfsetfillcolor{currentfill}%
\pgfsetlinewidth{0.501875pt}%
\definecolor{currentstroke}{rgb}{0.000000,0.000000,0.000000}%
\pgfsetstrokecolor{currentstroke}%
\pgfsetdash{}{0pt}%
\pgfsys@defobject{currentmarker}{\pgfqpoint{0.000000in}{0.000000in}}{\pgfqpoint{0.000000in}{0.020833in}}{%
\pgfpathmoveto{\pgfqpoint{0.000000in}{0.000000in}}%
\pgfpathlineto{\pgfqpoint{0.000000in}{0.020833in}}%
\pgfusepath{stroke,fill}%
}%
\begin{pgfscope}%
\pgfsys@transformshift{2.356912in}{2.747992in}%
\pgfsys@useobject{currentmarker}{}%
\end{pgfscope}%
\end{pgfscope}%
\begin{pgfscope}%
\pgfsetbuttcap%
\pgfsetroundjoin%
\definecolor{currentfill}{rgb}{0.000000,0.000000,0.000000}%
\pgfsetfillcolor{currentfill}%
\pgfsetlinewidth{0.501875pt}%
\definecolor{currentstroke}{rgb}{0.000000,0.000000,0.000000}%
\pgfsetstrokecolor{currentstroke}%
\pgfsetdash{}{0pt}%
\pgfsys@defobject{currentmarker}{\pgfqpoint{0.000000in}{-0.020833in}}{\pgfqpoint{0.000000in}{0.000000in}}{%
\pgfpathmoveto{\pgfqpoint{0.000000in}{0.000000in}}%
\pgfpathlineto{\pgfqpoint{0.000000in}{-0.020833in}}%
\pgfusepath{stroke,fill}%
}%
\begin{pgfscope}%
\pgfsys@transformshift{2.356912in}{4.374193in}%
\pgfsys@useobject{currentmarker}{}%
\end{pgfscope}%
\end{pgfscope}%
\begin{pgfscope}%
\pgfsetbuttcap%
\pgfsetroundjoin%
\definecolor{currentfill}{rgb}{0.000000,0.000000,0.000000}%
\pgfsetfillcolor{currentfill}%
\pgfsetlinewidth{0.501875pt}%
\definecolor{currentstroke}{rgb}{0.000000,0.000000,0.000000}%
\pgfsetstrokecolor{currentstroke}%
\pgfsetdash{}{0pt}%
\pgfsys@defobject{currentmarker}{\pgfqpoint{0.000000in}{0.000000in}}{\pgfqpoint{0.000000in}{0.020833in}}{%
\pgfpathmoveto{\pgfqpoint{0.000000in}{0.000000in}}%
\pgfpathlineto{\pgfqpoint{0.000000in}{0.020833in}}%
\pgfusepath{stroke,fill}%
}%
\begin{pgfscope}%
\pgfsys@transformshift{2.467876in}{2.747992in}%
\pgfsys@useobject{currentmarker}{}%
\end{pgfscope}%
\end{pgfscope}%
\begin{pgfscope}%
\pgfsetbuttcap%
\pgfsetroundjoin%
\definecolor{currentfill}{rgb}{0.000000,0.000000,0.000000}%
\pgfsetfillcolor{currentfill}%
\pgfsetlinewidth{0.501875pt}%
\definecolor{currentstroke}{rgb}{0.000000,0.000000,0.000000}%
\pgfsetstrokecolor{currentstroke}%
\pgfsetdash{}{0pt}%
\pgfsys@defobject{currentmarker}{\pgfqpoint{0.000000in}{-0.020833in}}{\pgfqpoint{0.000000in}{0.000000in}}{%
\pgfpathmoveto{\pgfqpoint{0.000000in}{0.000000in}}%
\pgfpathlineto{\pgfqpoint{0.000000in}{-0.020833in}}%
\pgfusepath{stroke,fill}%
}%
\begin{pgfscope}%
\pgfsys@transformshift{2.467876in}{4.374193in}%
\pgfsys@useobject{currentmarker}{}%
\end{pgfscope}%
\end{pgfscope}%
\begin{pgfscope}%
\pgfsetbuttcap%
\pgfsetroundjoin%
\definecolor{currentfill}{rgb}{0.000000,0.000000,0.000000}%
\pgfsetfillcolor{currentfill}%
\pgfsetlinewidth{0.501875pt}%
\definecolor{currentstroke}{rgb}{0.000000,0.000000,0.000000}%
\pgfsetstrokecolor{currentstroke}%
\pgfsetdash{}{0pt}%
\pgfsys@defobject{currentmarker}{\pgfqpoint{0.000000in}{0.000000in}}{\pgfqpoint{0.000000in}{0.020833in}}{%
\pgfpathmoveto{\pgfqpoint{0.000000in}{0.000000in}}%
\pgfpathlineto{\pgfqpoint{0.000000in}{0.020833in}}%
\pgfusepath{stroke,fill}%
}%
\begin{pgfscope}%
\pgfsys@transformshift{2.578840in}{2.747992in}%
\pgfsys@useobject{currentmarker}{}%
\end{pgfscope}%
\end{pgfscope}%
\begin{pgfscope}%
\pgfsetbuttcap%
\pgfsetroundjoin%
\definecolor{currentfill}{rgb}{0.000000,0.000000,0.000000}%
\pgfsetfillcolor{currentfill}%
\pgfsetlinewidth{0.501875pt}%
\definecolor{currentstroke}{rgb}{0.000000,0.000000,0.000000}%
\pgfsetstrokecolor{currentstroke}%
\pgfsetdash{}{0pt}%
\pgfsys@defobject{currentmarker}{\pgfqpoint{0.000000in}{-0.020833in}}{\pgfqpoint{0.000000in}{0.000000in}}{%
\pgfpathmoveto{\pgfqpoint{0.000000in}{0.000000in}}%
\pgfpathlineto{\pgfqpoint{0.000000in}{-0.020833in}}%
\pgfusepath{stroke,fill}%
}%
\begin{pgfscope}%
\pgfsys@transformshift{2.578840in}{4.374193in}%
\pgfsys@useobject{currentmarker}{}%
\end{pgfscope}%
\end{pgfscope}%
\begin{pgfscope}%
\pgfsetbuttcap%
\pgfsetroundjoin%
\definecolor{currentfill}{rgb}{0.000000,0.000000,0.000000}%
\pgfsetfillcolor{currentfill}%
\pgfsetlinewidth{0.501875pt}%
\definecolor{currentstroke}{rgb}{0.000000,0.000000,0.000000}%
\pgfsetstrokecolor{currentstroke}%
\pgfsetdash{}{0pt}%
\pgfsys@defobject{currentmarker}{\pgfqpoint{0.000000in}{0.000000in}}{\pgfqpoint{0.000000in}{0.020833in}}{%
\pgfpathmoveto{\pgfqpoint{0.000000in}{0.000000in}}%
\pgfpathlineto{\pgfqpoint{0.000000in}{0.020833in}}%
\pgfusepath{stroke,fill}%
}%
\begin{pgfscope}%
\pgfsys@transformshift{2.689804in}{2.747992in}%
\pgfsys@useobject{currentmarker}{}%
\end{pgfscope}%
\end{pgfscope}%
\begin{pgfscope}%
\pgfsetbuttcap%
\pgfsetroundjoin%
\definecolor{currentfill}{rgb}{0.000000,0.000000,0.000000}%
\pgfsetfillcolor{currentfill}%
\pgfsetlinewidth{0.501875pt}%
\definecolor{currentstroke}{rgb}{0.000000,0.000000,0.000000}%
\pgfsetstrokecolor{currentstroke}%
\pgfsetdash{}{0pt}%
\pgfsys@defobject{currentmarker}{\pgfqpoint{0.000000in}{-0.020833in}}{\pgfqpoint{0.000000in}{0.000000in}}{%
\pgfpathmoveto{\pgfqpoint{0.000000in}{0.000000in}}%
\pgfpathlineto{\pgfqpoint{0.000000in}{-0.020833in}}%
\pgfusepath{stroke,fill}%
}%
\begin{pgfscope}%
\pgfsys@transformshift{2.689804in}{4.374193in}%
\pgfsys@useobject{currentmarker}{}%
\end{pgfscope}%
\end{pgfscope}%
\begin{pgfscope}%
\definecolor{textcolor}{rgb}{0.000000,0.000000,0.000000}%
\pgfsetstrokecolor{textcolor}%
\pgfsetfillcolor{textcolor}%
\pgftext[x=1.591261in,y=2.509413in,,top]{\color{textcolor}\rmfamily\fontsize{10.000000}{12.000000}\selectfont \(\displaystyle K\)}%
\end{pgfscope}%
\begin{pgfscope}%
\pgfsetbuttcap%
\pgfsetroundjoin%
\definecolor{currentfill}{rgb}{0.000000,0.000000,0.000000}%
\pgfsetfillcolor{currentfill}%
\pgfsetlinewidth{0.501875pt}%
\definecolor{currentstroke}{rgb}{0.000000,0.000000,0.000000}%
\pgfsetstrokecolor{currentstroke}%
\pgfsetdash{}{0pt}%
\pgfsys@defobject{currentmarker}{\pgfqpoint{0.000000in}{0.000000in}}{\pgfqpoint{0.041667in}{0.000000in}}{%
\pgfpathmoveto{\pgfqpoint{0.000000in}{0.000000in}}%
\pgfpathlineto{\pgfqpoint{0.041667in}{0.000000in}}%
\pgfusepath{stroke,fill}%
}%
\begin{pgfscope}%
\pgfsys@transformshift{0.470525in}{2.796635in}%
\pgfsys@useobject{currentmarker}{}%
\end{pgfscope}%
\end{pgfscope}%
\begin{pgfscope}%
\pgfsetbuttcap%
\pgfsetroundjoin%
\definecolor{currentfill}{rgb}{0.000000,0.000000,0.000000}%
\pgfsetfillcolor{currentfill}%
\pgfsetlinewidth{0.501875pt}%
\definecolor{currentstroke}{rgb}{0.000000,0.000000,0.000000}%
\pgfsetstrokecolor{currentstroke}%
\pgfsetdash{}{0pt}%
\pgfsys@defobject{currentmarker}{\pgfqpoint{-0.041667in}{0.000000in}}{\pgfqpoint{-0.000000in}{0.000000in}}{%
\pgfpathmoveto{\pgfqpoint{-0.000000in}{0.000000in}}%
\pgfpathlineto{\pgfqpoint{-0.041667in}{0.000000in}}%
\pgfusepath{stroke,fill}%
}%
\begin{pgfscope}%
\pgfsys@transformshift{2.711997in}{2.796635in}%
\pgfsys@useobject{currentmarker}{}%
\end{pgfscope}%
\end{pgfscope}%
\begin{pgfscope}%
\definecolor{textcolor}{rgb}{0.000000,0.000000,0.000000}%
\pgfsetstrokecolor{textcolor}%
\pgfsetfillcolor{textcolor}%
\pgftext[x=0.244444in, y=2.743873in, left, base]{\color{textcolor}\rmfamily\fontsize{10.000000}{12.000000}\selectfont \(\displaystyle {0.7}\)}%
\end{pgfscope}%
\begin{pgfscope}%
\pgfsetbuttcap%
\pgfsetroundjoin%
\definecolor{currentfill}{rgb}{0.000000,0.000000,0.000000}%
\pgfsetfillcolor{currentfill}%
\pgfsetlinewidth{0.501875pt}%
\definecolor{currentstroke}{rgb}{0.000000,0.000000,0.000000}%
\pgfsetstrokecolor{currentstroke}%
\pgfsetdash{}{0pt}%
\pgfsys@defobject{currentmarker}{\pgfqpoint{0.000000in}{0.000000in}}{\pgfqpoint{0.041667in}{0.000000in}}{%
\pgfpathmoveto{\pgfqpoint{0.000000in}{0.000000in}}%
\pgfpathlineto{\pgfqpoint{0.041667in}{0.000000in}}%
\pgfusepath{stroke,fill}%
}%
\begin{pgfscope}%
\pgfsys@transformshift{0.470525in}{3.361135in}%
\pgfsys@useobject{currentmarker}{}%
\end{pgfscope}%
\end{pgfscope}%
\begin{pgfscope}%
\pgfsetbuttcap%
\pgfsetroundjoin%
\definecolor{currentfill}{rgb}{0.000000,0.000000,0.000000}%
\pgfsetfillcolor{currentfill}%
\pgfsetlinewidth{0.501875pt}%
\definecolor{currentstroke}{rgb}{0.000000,0.000000,0.000000}%
\pgfsetstrokecolor{currentstroke}%
\pgfsetdash{}{0pt}%
\pgfsys@defobject{currentmarker}{\pgfqpoint{-0.041667in}{0.000000in}}{\pgfqpoint{-0.000000in}{0.000000in}}{%
\pgfpathmoveto{\pgfqpoint{-0.000000in}{0.000000in}}%
\pgfpathlineto{\pgfqpoint{-0.041667in}{0.000000in}}%
\pgfusepath{stroke,fill}%
}%
\begin{pgfscope}%
\pgfsys@transformshift{2.711997in}{3.361135in}%
\pgfsys@useobject{currentmarker}{}%
\end{pgfscope}%
\end{pgfscope}%
\begin{pgfscope}%
\definecolor{textcolor}{rgb}{0.000000,0.000000,0.000000}%
\pgfsetstrokecolor{textcolor}%
\pgfsetfillcolor{textcolor}%
\pgftext[x=0.244444in, y=3.308373in, left, base]{\color{textcolor}\rmfamily\fontsize{10.000000}{12.000000}\selectfont \(\displaystyle {0.8}\)}%
\end{pgfscope}%
\begin{pgfscope}%
\pgfsetbuttcap%
\pgfsetroundjoin%
\definecolor{currentfill}{rgb}{0.000000,0.000000,0.000000}%
\pgfsetfillcolor{currentfill}%
\pgfsetlinewidth{0.501875pt}%
\definecolor{currentstroke}{rgb}{0.000000,0.000000,0.000000}%
\pgfsetstrokecolor{currentstroke}%
\pgfsetdash{}{0pt}%
\pgfsys@defobject{currentmarker}{\pgfqpoint{0.000000in}{0.000000in}}{\pgfqpoint{0.041667in}{0.000000in}}{%
\pgfpathmoveto{\pgfqpoint{0.000000in}{0.000000in}}%
\pgfpathlineto{\pgfqpoint{0.041667in}{0.000000in}}%
\pgfusepath{stroke,fill}%
}%
\begin{pgfscope}%
\pgfsys@transformshift{0.470525in}{3.925635in}%
\pgfsys@useobject{currentmarker}{}%
\end{pgfscope}%
\end{pgfscope}%
\begin{pgfscope}%
\pgfsetbuttcap%
\pgfsetroundjoin%
\definecolor{currentfill}{rgb}{0.000000,0.000000,0.000000}%
\pgfsetfillcolor{currentfill}%
\pgfsetlinewidth{0.501875pt}%
\definecolor{currentstroke}{rgb}{0.000000,0.000000,0.000000}%
\pgfsetstrokecolor{currentstroke}%
\pgfsetdash{}{0pt}%
\pgfsys@defobject{currentmarker}{\pgfqpoint{-0.041667in}{0.000000in}}{\pgfqpoint{-0.000000in}{0.000000in}}{%
\pgfpathmoveto{\pgfqpoint{-0.000000in}{0.000000in}}%
\pgfpathlineto{\pgfqpoint{-0.041667in}{0.000000in}}%
\pgfusepath{stroke,fill}%
}%
\begin{pgfscope}%
\pgfsys@transformshift{2.711997in}{3.925635in}%
\pgfsys@useobject{currentmarker}{}%
\end{pgfscope}%
\end{pgfscope}%
\begin{pgfscope}%
\definecolor{textcolor}{rgb}{0.000000,0.000000,0.000000}%
\pgfsetstrokecolor{textcolor}%
\pgfsetfillcolor{textcolor}%
\pgftext[x=0.244444in, y=3.872873in, left, base]{\color{textcolor}\rmfamily\fontsize{10.000000}{12.000000}\selectfont \(\displaystyle {0.9}\)}%
\end{pgfscope}%
\begin{pgfscope}%
\pgfsetbuttcap%
\pgfsetroundjoin%
\definecolor{currentfill}{rgb}{0.000000,0.000000,0.000000}%
\pgfsetfillcolor{currentfill}%
\pgfsetlinewidth{0.501875pt}%
\definecolor{currentstroke}{rgb}{0.000000,0.000000,0.000000}%
\pgfsetstrokecolor{currentstroke}%
\pgfsetdash{}{0pt}%
\pgfsys@defobject{currentmarker}{\pgfqpoint{0.000000in}{0.000000in}}{\pgfqpoint{0.020833in}{0.000000in}}{%
\pgfpathmoveto{\pgfqpoint{0.000000in}{0.000000in}}%
\pgfpathlineto{\pgfqpoint{0.020833in}{0.000000in}}%
\pgfusepath{stroke,fill}%
}%
\begin{pgfscope}%
\pgfsys@transformshift{0.470525in}{2.909535in}%
\pgfsys@useobject{currentmarker}{}%
\end{pgfscope}%
\end{pgfscope}%
\begin{pgfscope}%
\pgfsetbuttcap%
\pgfsetroundjoin%
\definecolor{currentfill}{rgb}{0.000000,0.000000,0.000000}%
\pgfsetfillcolor{currentfill}%
\pgfsetlinewidth{0.501875pt}%
\definecolor{currentstroke}{rgb}{0.000000,0.000000,0.000000}%
\pgfsetstrokecolor{currentstroke}%
\pgfsetdash{}{0pt}%
\pgfsys@defobject{currentmarker}{\pgfqpoint{-0.020833in}{0.000000in}}{\pgfqpoint{-0.000000in}{0.000000in}}{%
\pgfpathmoveto{\pgfqpoint{-0.000000in}{0.000000in}}%
\pgfpathlineto{\pgfqpoint{-0.020833in}{0.000000in}}%
\pgfusepath{stroke,fill}%
}%
\begin{pgfscope}%
\pgfsys@transformshift{2.711997in}{2.909535in}%
\pgfsys@useobject{currentmarker}{}%
\end{pgfscope}%
\end{pgfscope}%
\begin{pgfscope}%
\pgfsetbuttcap%
\pgfsetroundjoin%
\definecolor{currentfill}{rgb}{0.000000,0.000000,0.000000}%
\pgfsetfillcolor{currentfill}%
\pgfsetlinewidth{0.501875pt}%
\definecolor{currentstroke}{rgb}{0.000000,0.000000,0.000000}%
\pgfsetstrokecolor{currentstroke}%
\pgfsetdash{}{0pt}%
\pgfsys@defobject{currentmarker}{\pgfqpoint{0.000000in}{0.000000in}}{\pgfqpoint{0.020833in}{0.000000in}}{%
\pgfpathmoveto{\pgfqpoint{0.000000in}{0.000000in}}%
\pgfpathlineto{\pgfqpoint{0.020833in}{0.000000in}}%
\pgfusepath{stroke,fill}%
}%
\begin{pgfscope}%
\pgfsys@transformshift{0.470525in}{3.022435in}%
\pgfsys@useobject{currentmarker}{}%
\end{pgfscope}%
\end{pgfscope}%
\begin{pgfscope}%
\pgfsetbuttcap%
\pgfsetroundjoin%
\definecolor{currentfill}{rgb}{0.000000,0.000000,0.000000}%
\pgfsetfillcolor{currentfill}%
\pgfsetlinewidth{0.501875pt}%
\definecolor{currentstroke}{rgb}{0.000000,0.000000,0.000000}%
\pgfsetstrokecolor{currentstroke}%
\pgfsetdash{}{0pt}%
\pgfsys@defobject{currentmarker}{\pgfqpoint{-0.020833in}{0.000000in}}{\pgfqpoint{-0.000000in}{0.000000in}}{%
\pgfpathmoveto{\pgfqpoint{-0.000000in}{0.000000in}}%
\pgfpathlineto{\pgfqpoint{-0.020833in}{0.000000in}}%
\pgfusepath{stroke,fill}%
}%
\begin{pgfscope}%
\pgfsys@transformshift{2.711997in}{3.022435in}%
\pgfsys@useobject{currentmarker}{}%
\end{pgfscope}%
\end{pgfscope}%
\begin{pgfscope}%
\pgfsetbuttcap%
\pgfsetroundjoin%
\definecolor{currentfill}{rgb}{0.000000,0.000000,0.000000}%
\pgfsetfillcolor{currentfill}%
\pgfsetlinewidth{0.501875pt}%
\definecolor{currentstroke}{rgb}{0.000000,0.000000,0.000000}%
\pgfsetstrokecolor{currentstroke}%
\pgfsetdash{}{0pt}%
\pgfsys@defobject{currentmarker}{\pgfqpoint{0.000000in}{0.000000in}}{\pgfqpoint{0.020833in}{0.000000in}}{%
\pgfpathmoveto{\pgfqpoint{0.000000in}{0.000000in}}%
\pgfpathlineto{\pgfqpoint{0.020833in}{0.000000in}}%
\pgfusepath{stroke,fill}%
}%
\begin{pgfscope}%
\pgfsys@transformshift{0.470525in}{3.135335in}%
\pgfsys@useobject{currentmarker}{}%
\end{pgfscope}%
\end{pgfscope}%
\begin{pgfscope}%
\pgfsetbuttcap%
\pgfsetroundjoin%
\definecolor{currentfill}{rgb}{0.000000,0.000000,0.000000}%
\pgfsetfillcolor{currentfill}%
\pgfsetlinewidth{0.501875pt}%
\definecolor{currentstroke}{rgb}{0.000000,0.000000,0.000000}%
\pgfsetstrokecolor{currentstroke}%
\pgfsetdash{}{0pt}%
\pgfsys@defobject{currentmarker}{\pgfqpoint{-0.020833in}{0.000000in}}{\pgfqpoint{-0.000000in}{0.000000in}}{%
\pgfpathmoveto{\pgfqpoint{-0.000000in}{0.000000in}}%
\pgfpathlineto{\pgfqpoint{-0.020833in}{0.000000in}}%
\pgfusepath{stroke,fill}%
}%
\begin{pgfscope}%
\pgfsys@transformshift{2.711997in}{3.135335in}%
\pgfsys@useobject{currentmarker}{}%
\end{pgfscope}%
\end{pgfscope}%
\begin{pgfscope}%
\pgfsetbuttcap%
\pgfsetroundjoin%
\definecolor{currentfill}{rgb}{0.000000,0.000000,0.000000}%
\pgfsetfillcolor{currentfill}%
\pgfsetlinewidth{0.501875pt}%
\definecolor{currentstroke}{rgb}{0.000000,0.000000,0.000000}%
\pgfsetstrokecolor{currentstroke}%
\pgfsetdash{}{0pt}%
\pgfsys@defobject{currentmarker}{\pgfqpoint{0.000000in}{0.000000in}}{\pgfqpoint{0.020833in}{0.000000in}}{%
\pgfpathmoveto{\pgfqpoint{0.000000in}{0.000000in}}%
\pgfpathlineto{\pgfqpoint{0.020833in}{0.000000in}}%
\pgfusepath{stroke,fill}%
}%
\begin{pgfscope}%
\pgfsys@transformshift{0.470525in}{3.248235in}%
\pgfsys@useobject{currentmarker}{}%
\end{pgfscope}%
\end{pgfscope}%
\begin{pgfscope}%
\pgfsetbuttcap%
\pgfsetroundjoin%
\definecolor{currentfill}{rgb}{0.000000,0.000000,0.000000}%
\pgfsetfillcolor{currentfill}%
\pgfsetlinewidth{0.501875pt}%
\definecolor{currentstroke}{rgb}{0.000000,0.000000,0.000000}%
\pgfsetstrokecolor{currentstroke}%
\pgfsetdash{}{0pt}%
\pgfsys@defobject{currentmarker}{\pgfqpoint{-0.020833in}{0.000000in}}{\pgfqpoint{-0.000000in}{0.000000in}}{%
\pgfpathmoveto{\pgfqpoint{-0.000000in}{0.000000in}}%
\pgfpathlineto{\pgfqpoint{-0.020833in}{0.000000in}}%
\pgfusepath{stroke,fill}%
}%
\begin{pgfscope}%
\pgfsys@transformshift{2.711997in}{3.248235in}%
\pgfsys@useobject{currentmarker}{}%
\end{pgfscope}%
\end{pgfscope}%
\begin{pgfscope}%
\pgfsetbuttcap%
\pgfsetroundjoin%
\definecolor{currentfill}{rgb}{0.000000,0.000000,0.000000}%
\pgfsetfillcolor{currentfill}%
\pgfsetlinewidth{0.501875pt}%
\definecolor{currentstroke}{rgb}{0.000000,0.000000,0.000000}%
\pgfsetstrokecolor{currentstroke}%
\pgfsetdash{}{0pt}%
\pgfsys@defobject{currentmarker}{\pgfqpoint{0.000000in}{0.000000in}}{\pgfqpoint{0.020833in}{0.000000in}}{%
\pgfpathmoveto{\pgfqpoint{0.000000in}{0.000000in}}%
\pgfpathlineto{\pgfqpoint{0.020833in}{0.000000in}}%
\pgfusepath{stroke,fill}%
}%
\begin{pgfscope}%
\pgfsys@transformshift{0.470525in}{3.474035in}%
\pgfsys@useobject{currentmarker}{}%
\end{pgfscope}%
\end{pgfscope}%
\begin{pgfscope}%
\pgfsetbuttcap%
\pgfsetroundjoin%
\definecolor{currentfill}{rgb}{0.000000,0.000000,0.000000}%
\pgfsetfillcolor{currentfill}%
\pgfsetlinewidth{0.501875pt}%
\definecolor{currentstroke}{rgb}{0.000000,0.000000,0.000000}%
\pgfsetstrokecolor{currentstroke}%
\pgfsetdash{}{0pt}%
\pgfsys@defobject{currentmarker}{\pgfqpoint{-0.020833in}{0.000000in}}{\pgfqpoint{-0.000000in}{0.000000in}}{%
\pgfpathmoveto{\pgfqpoint{-0.000000in}{0.000000in}}%
\pgfpathlineto{\pgfqpoint{-0.020833in}{0.000000in}}%
\pgfusepath{stroke,fill}%
}%
\begin{pgfscope}%
\pgfsys@transformshift{2.711997in}{3.474035in}%
\pgfsys@useobject{currentmarker}{}%
\end{pgfscope}%
\end{pgfscope}%
\begin{pgfscope}%
\pgfsetbuttcap%
\pgfsetroundjoin%
\definecolor{currentfill}{rgb}{0.000000,0.000000,0.000000}%
\pgfsetfillcolor{currentfill}%
\pgfsetlinewidth{0.501875pt}%
\definecolor{currentstroke}{rgb}{0.000000,0.000000,0.000000}%
\pgfsetstrokecolor{currentstroke}%
\pgfsetdash{}{0pt}%
\pgfsys@defobject{currentmarker}{\pgfqpoint{0.000000in}{0.000000in}}{\pgfqpoint{0.020833in}{0.000000in}}{%
\pgfpathmoveto{\pgfqpoint{0.000000in}{0.000000in}}%
\pgfpathlineto{\pgfqpoint{0.020833in}{0.000000in}}%
\pgfusepath{stroke,fill}%
}%
\begin{pgfscope}%
\pgfsys@transformshift{0.470525in}{3.586935in}%
\pgfsys@useobject{currentmarker}{}%
\end{pgfscope}%
\end{pgfscope}%
\begin{pgfscope}%
\pgfsetbuttcap%
\pgfsetroundjoin%
\definecolor{currentfill}{rgb}{0.000000,0.000000,0.000000}%
\pgfsetfillcolor{currentfill}%
\pgfsetlinewidth{0.501875pt}%
\definecolor{currentstroke}{rgb}{0.000000,0.000000,0.000000}%
\pgfsetstrokecolor{currentstroke}%
\pgfsetdash{}{0pt}%
\pgfsys@defobject{currentmarker}{\pgfqpoint{-0.020833in}{0.000000in}}{\pgfqpoint{-0.000000in}{0.000000in}}{%
\pgfpathmoveto{\pgfqpoint{-0.000000in}{0.000000in}}%
\pgfpathlineto{\pgfqpoint{-0.020833in}{0.000000in}}%
\pgfusepath{stroke,fill}%
}%
\begin{pgfscope}%
\pgfsys@transformshift{2.711997in}{3.586935in}%
\pgfsys@useobject{currentmarker}{}%
\end{pgfscope}%
\end{pgfscope}%
\begin{pgfscope}%
\pgfsetbuttcap%
\pgfsetroundjoin%
\definecolor{currentfill}{rgb}{0.000000,0.000000,0.000000}%
\pgfsetfillcolor{currentfill}%
\pgfsetlinewidth{0.501875pt}%
\definecolor{currentstroke}{rgb}{0.000000,0.000000,0.000000}%
\pgfsetstrokecolor{currentstroke}%
\pgfsetdash{}{0pt}%
\pgfsys@defobject{currentmarker}{\pgfqpoint{0.000000in}{0.000000in}}{\pgfqpoint{0.020833in}{0.000000in}}{%
\pgfpathmoveto{\pgfqpoint{0.000000in}{0.000000in}}%
\pgfpathlineto{\pgfqpoint{0.020833in}{0.000000in}}%
\pgfusepath{stroke,fill}%
}%
\begin{pgfscope}%
\pgfsys@transformshift{0.470525in}{3.699835in}%
\pgfsys@useobject{currentmarker}{}%
\end{pgfscope}%
\end{pgfscope}%
\begin{pgfscope}%
\pgfsetbuttcap%
\pgfsetroundjoin%
\definecolor{currentfill}{rgb}{0.000000,0.000000,0.000000}%
\pgfsetfillcolor{currentfill}%
\pgfsetlinewidth{0.501875pt}%
\definecolor{currentstroke}{rgb}{0.000000,0.000000,0.000000}%
\pgfsetstrokecolor{currentstroke}%
\pgfsetdash{}{0pt}%
\pgfsys@defobject{currentmarker}{\pgfqpoint{-0.020833in}{0.000000in}}{\pgfqpoint{-0.000000in}{0.000000in}}{%
\pgfpathmoveto{\pgfqpoint{-0.000000in}{0.000000in}}%
\pgfpathlineto{\pgfqpoint{-0.020833in}{0.000000in}}%
\pgfusepath{stroke,fill}%
}%
\begin{pgfscope}%
\pgfsys@transformshift{2.711997in}{3.699835in}%
\pgfsys@useobject{currentmarker}{}%
\end{pgfscope}%
\end{pgfscope}%
\begin{pgfscope}%
\pgfsetbuttcap%
\pgfsetroundjoin%
\definecolor{currentfill}{rgb}{0.000000,0.000000,0.000000}%
\pgfsetfillcolor{currentfill}%
\pgfsetlinewidth{0.501875pt}%
\definecolor{currentstroke}{rgb}{0.000000,0.000000,0.000000}%
\pgfsetstrokecolor{currentstroke}%
\pgfsetdash{}{0pt}%
\pgfsys@defobject{currentmarker}{\pgfqpoint{0.000000in}{0.000000in}}{\pgfqpoint{0.020833in}{0.000000in}}{%
\pgfpathmoveto{\pgfqpoint{0.000000in}{0.000000in}}%
\pgfpathlineto{\pgfqpoint{0.020833in}{0.000000in}}%
\pgfusepath{stroke,fill}%
}%
\begin{pgfscope}%
\pgfsys@transformshift{0.470525in}{3.812735in}%
\pgfsys@useobject{currentmarker}{}%
\end{pgfscope}%
\end{pgfscope}%
\begin{pgfscope}%
\pgfsetbuttcap%
\pgfsetroundjoin%
\definecolor{currentfill}{rgb}{0.000000,0.000000,0.000000}%
\pgfsetfillcolor{currentfill}%
\pgfsetlinewidth{0.501875pt}%
\definecolor{currentstroke}{rgb}{0.000000,0.000000,0.000000}%
\pgfsetstrokecolor{currentstroke}%
\pgfsetdash{}{0pt}%
\pgfsys@defobject{currentmarker}{\pgfqpoint{-0.020833in}{0.000000in}}{\pgfqpoint{-0.000000in}{0.000000in}}{%
\pgfpathmoveto{\pgfqpoint{-0.000000in}{0.000000in}}%
\pgfpathlineto{\pgfqpoint{-0.020833in}{0.000000in}}%
\pgfusepath{stroke,fill}%
}%
\begin{pgfscope}%
\pgfsys@transformshift{2.711997in}{3.812735in}%
\pgfsys@useobject{currentmarker}{}%
\end{pgfscope}%
\end{pgfscope}%
\begin{pgfscope}%
\pgfsetbuttcap%
\pgfsetroundjoin%
\definecolor{currentfill}{rgb}{0.000000,0.000000,0.000000}%
\pgfsetfillcolor{currentfill}%
\pgfsetlinewidth{0.501875pt}%
\definecolor{currentstroke}{rgb}{0.000000,0.000000,0.000000}%
\pgfsetstrokecolor{currentstroke}%
\pgfsetdash{}{0pt}%
\pgfsys@defobject{currentmarker}{\pgfqpoint{0.000000in}{0.000000in}}{\pgfqpoint{0.020833in}{0.000000in}}{%
\pgfpathmoveto{\pgfqpoint{0.000000in}{0.000000in}}%
\pgfpathlineto{\pgfqpoint{0.020833in}{0.000000in}}%
\pgfusepath{stroke,fill}%
}%
\begin{pgfscope}%
\pgfsys@transformshift{0.470525in}{4.038535in}%
\pgfsys@useobject{currentmarker}{}%
\end{pgfscope}%
\end{pgfscope}%
\begin{pgfscope}%
\pgfsetbuttcap%
\pgfsetroundjoin%
\definecolor{currentfill}{rgb}{0.000000,0.000000,0.000000}%
\pgfsetfillcolor{currentfill}%
\pgfsetlinewidth{0.501875pt}%
\definecolor{currentstroke}{rgb}{0.000000,0.000000,0.000000}%
\pgfsetstrokecolor{currentstroke}%
\pgfsetdash{}{0pt}%
\pgfsys@defobject{currentmarker}{\pgfqpoint{-0.020833in}{0.000000in}}{\pgfqpoint{-0.000000in}{0.000000in}}{%
\pgfpathmoveto{\pgfqpoint{-0.000000in}{0.000000in}}%
\pgfpathlineto{\pgfqpoint{-0.020833in}{0.000000in}}%
\pgfusepath{stroke,fill}%
}%
\begin{pgfscope}%
\pgfsys@transformshift{2.711997in}{4.038535in}%
\pgfsys@useobject{currentmarker}{}%
\end{pgfscope}%
\end{pgfscope}%
\begin{pgfscope}%
\pgfsetbuttcap%
\pgfsetroundjoin%
\definecolor{currentfill}{rgb}{0.000000,0.000000,0.000000}%
\pgfsetfillcolor{currentfill}%
\pgfsetlinewidth{0.501875pt}%
\definecolor{currentstroke}{rgb}{0.000000,0.000000,0.000000}%
\pgfsetstrokecolor{currentstroke}%
\pgfsetdash{}{0pt}%
\pgfsys@defobject{currentmarker}{\pgfqpoint{0.000000in}{0.000000in}}{\pgfqpoint{0.020833in}{0.000000in}}{%
\pgfpathmoveto{\pgfqpoint{0.000000in}{0.000000in}}%
\pgfpathlineto{\pgfqpoint{0.020833in}{0.000000in}}%
\pgfusepath{stroke,fill}%
}%
\begin{pgfscope}%
\pgfsys@transformshift{0.470525in}{4.151435in}%
\pgfsys@useobject{currentmarker}{}%
\end{pgfscope}%
\end{pgfscope}%
\begin{pgfscope}%
\pgfsetbuttcap%
\pgfsetroundjoin%
\definecolor{currentfill}{rgb}{0.000000,0.000000,0.000000}%
\pgfsetfillcolor{currentfill}%
\pgfsetlinewidth{0.501875pt}%
\definecolor{currentstroke}{rgb}{0.000000,0.000000,0.000000}%
\pgfsetstrokecolor{currentstroke}%
\pgfsetdash{}{0pt}%
\pgfsys@defobject{currentmarker}{\pgfqpoint{-0.020833in}{0.000000in}}{\pgfqpoint{-0.000000in}{0.000000in}}{%
\pgfpathmoveto{\pgfqpoint{-0.000000in}{0.000000in}}%
\pgfpathlineto{\pgfqpoint{-0.020833in}{0.000000in}}%
\pgfusepath{stroke,fill}%
}%
\begin{pgfscope}%
\pgfsys@transformshift{2.711997in}{4.151435in}%
\pgfsys@useobject{currentmarker}{}%
\end{pgfscope}%
\end{pgfscope}%
\begin{pgfscope}%
\pgfsetbuttcap%
\pgfsetroundjoin%
\definecolor{currentfill}{rgb}{0.000000,0.000000,0.000000}%
\pgfsetfillcolor{currentfill}%
\pgfsetlinewidth{0.501875pt}%
\definecolor{currentstroke}{rgb}{0.000000,0.000000,0.000000}%
\pgfsetstrokecolor{currentstroke}%
\pgfsetdash{}{0pt}%
\pgfsys@defobject{currentmarker}{\pgfqpoint{0.000000in}{0.000000in}}{\pgfqpoint{0.020833in}{0.000000in}}{%
\pgfpathmoveto{\pgfqpoint{0.000000in}{0.000000in}}%
\pgfpathlineto{\pgfqpoint{0.020833in}{0.000000in}}%
\pgfusepath{stroke,fill}%
}%
\begin{pgfscope}%
\pgfsys@transformshift{0.470525in}{4.264335in}%
\pgfsys@useobject{currentmarker}{}%
\end{pgfscope}%
\end{pgfscope}%
\begin{pgfscope}%
\pgfsetbuttcap%
\pgfsetroundjoin%
\definecolor{currentfill}{rgb}{0.000000,0.000000,0.000000}%
\pgfsetfillcolor{currentfill}%
\pgfsetlinewidth{0.501875pt}%
\definecolor{currentstroke}{rgb}{0.000000,0.000000,0.000000}%
\pgfsetstrokecolor{currentstroke}%
\pgfsetdash{}{0pt}%
\pgfsys@defobject{currentmarker}{\pgfqpoint{-0.020833in}{0.000000in}}{\pgfqpoint{-0.000000in}{0.000000in}}{%
\pgfpathmoveto{\pgfqpoint{-0.000000in}{0.000000in}}%
\pgfpathlineto{\pgfqpoint{-0.020833in}{0.000000in}}%
\pgfusepath{stroke,fill}%
}%
\begin{pgfscope}%
\pgfsys@transformshift{2.711997in}{4.264335in}%
\pgfsys@useobject{currentmarker}{}%
\end{pgfscope}%
\end{pgfscope}%
\begin{pgfscope}%
\definecolor{textcolor}{rgb}{0.000000,0.000000,0.000000}%
\pgfsetstrokecolor{textcolor}%
\pgfsetfillcolor{textcolor}%
\pgftext[x=0.188889in,y=3.561093in,,bottom,rotate=90.000000]{\color{textcolor}\rmfamily\fontsize{10.000000}{12.000000}\selectfont \(\displaystyle T(K)\)}%
\end{pgfscope}%
\begin{pgfscope}%
\pgfpathrectangle{\pgfqpoint{0.470525in}{2.747992in}}{\pgfqpoint{2.241471in}{1.626201in}}%
\pgfusepath{clip}%
\pgfsetrectcap%
\pgfsetroundjoin%
\pgfsetlinewidth{1.003750pt}%
\definecolor{currentstroke}{rgb}{0.047059,0.364706,0.647059}%
\pgfsetstrokecolor{currentstroke}%
\pgfsetdash{}{0pt}%
\pgfpathmoveto{\pgfqpoint{0.492718in}{4.300275in}}%
\pgfpathlineto{\pgfqpoint{0.514911in}{4.276069in}}%
\pgfpathlineto{\pgfqpoint{0.537104in}{4.258925in}}%
\pgfpathlineto{\pgfqpoint{0.559296in}{4.248305in}}%
\pgfpathlineto{\pgfqpoint{0.581489in}{4.241481in}}%
\pgfpathlineto{\pgfqpoint{0.603682in}{4.238170in}}%
\pgfpathlineto{\pgfqpoint{0.625875in}{4.236341in}}%
\pgfpathlineto{\pgfqpoint{0.648068in}{4.235350in}}%
\pgfpathlineto{\pgfqpoint{0.670260in}{4.233398in}}%
\pgfpathlineto{\pgfqpoint{0.692453in}{4.230488in}}%
\pgfpathlineto{\pgfqpoint{0.714646in}{4.229157in}}%
\pgfpathlineto{\pgfqpoint{0.736839in}{4.227543in}}%
\pgfpathlineto{\pgfqpoint{0.759032in}{4.225620in}}%
\pgfpathlineto{\pgfqpoint{0.781224in}{4.224547in}}%
\pgfpathlineto{\pgfqpoint{0.803417in}{4.223877in}}%
\pgfpathlineto{\pgfqpoint{0.825610in}{4.222503in}}%
\pgfpathlineto{\pgfqpoint{0.847803in}{4.220866in}}%
\pgfpathlineto{\pgfqpoint{0.869995in}{4.219796in}}%
\pgfpathlineto{\pgfqpoint{0.892188in}{4.219297in}}%
\pgfpathlineto{\pgfqpoint{0.914381in}{4.218641in}}%
\pgfpathlineto{\pgfqpoint{0.936574in}{4.218297in}}%
\pgfpathlineto{\pgfqpoint{0.958767in}{4.217025in}}%
\pgfpathlineto{\pgfqpoint{0.980959in}{4.216245in}}%
\pgfpathlineto{\pgfqpoint{1.003152in}{4.214429in}}%
\pgfpathlineto{\pgfqpoint{1.025345in}{4.213570in}}%
\pgfpathlineto{\pgfqpoint{1.047538in}{4.212574in}}%
\pgfpathlineto{\pgfqpoint{1.069731in}{4.211956in}}%
\pgfpathlineto{\pgfqpoint{1.091923in}{4.211776in}}%
\pgfpathlineto{\pgfqpoint{1.114116in}{4.211133in}}%
\pgfpathlineto{\pgfqpoint{1.136309in}{4.210600in}}%
\pgfpathlineto{\pgfqpoint{1.158502in}{4.210191in}}%
\pgfpathlineto{\pgfqpoint{1.180694in}{4.209868in}}%
\pgfpathlineto{\pgfqpoint{1.202887in}{4.209335in}}%
\pgfpathlineto{\pgfqpoint{1.225080in}{4.208873in}}%
\pgfpathlineto{\pgfqpoint{1.247273in}{4.208262in}}%
\pgfpathlineto{\pgfqpoint{1.269466in}{4.207661in}}%
\pgfpathlineto{\pgfqpoint{1.291658in}{4.207475in}}%
\pgfpathlineto{\pgfqpoint{1.313851in}{4.207172in}}%
\pgfpathlineto{\pgfqpoint{1.336044in}{4.206730in}}%
\pgfpathlineto{\pgfqpoint{1.358237in}{4.206078in}}%
\pgfpathlineto{\pgfqpoint{1.380430in}{4.205839in}}%
\pgfpathlineto{\pgfqpoint{1.402622in}{4.205423in}}%
\pgfpathlineto{\pgfqpoint{1.424815in}{4.204891in}}%
\pgfpathlineto{\pgfqpoint{1.447008in}{4.204983in}}%
\pgfpathlineto{\pgfqpoint{1.469201in}{4.204486in}}%
\pgfpathlineto{\pgfqpoint{1.491393in}{4.204368in}}%
\pgfpathlineto{\pgfqpoint{1.513586in}{4.204018in}}%
\pgfpathlineto{\pgfqpoint{1.535779in}{4.203818in}}%
\pgfpathlineto{\pgfqpoint{1.557972in}{4.203636in}}%
\pgfpathlineto{\pgfqpoint{1.580165in}{4.203320in}}%
\pgfpathlineto{\pgfqpoint{1.602357in}{4.203041in}}%
\pgfpathlineto{\pgfqpoint{1.624550in}{4.202830in}}%
\pgfpathlineto{\pgfqpoint{1.646743in}{4.202626in}}%
\pgfpathlineto{\pgfqpoint{1.668936in}{4.202523in}}%
\pgfpathlineto{\pgfqpoint{1.691129in}{4.202304in}}%
\pgfpathlineto{\pgfqpoint{1.713321in}{4.201947in}}%
\pgfpathlineto{\pgfqpoint{1.735514in}{4.201693in}}%
\pgfpathlineto{\pgfqpoint{1.757707in}{4.201538in}}%
\pgfpathlineto{\pgfqpoint{1.779900in}{4.201202in}}%
\pgfpathlineto{\pgfqpoint{1.802092in}{4.200939in}}%
\pgfpathlineto{\pgfqpoint{1.824285in}{4.200571in}}%
\pgfpathlineto{\pgfqpoint{1.846478in}{4.200398in}}%
\pgfpathlineto{\pgfqpoint{1.868671in}{4.199953in}}%
\pgfpathlineto{\pgfqpoint{1.890864in}{4.199713in}}%
\pgfpathlineto{\pgfqpoint{1.913056in}{4.199551in}}%
\pgfpathlineto{\pgfqpoint{1.935249in}{4.199331in}}%
\pgfpathlineto{\pgfqpoint{1.957442in}{4.199283in}}%
\pgfpathlineto{\pgfqpoint{1.979635in}{4.199036in}}%
\pgfpathlineto{\pgfqpoint{2.001828in}{4.198829in}}%
\pgfpathlineto{\pgfqpoint{2.024020in}{4.198834in}}%
\pgfpathlineto{\pgfqpoint{2.046213in}{4.198604in}}%
\pgfpathlineto{\pgfqpoint{2.068406in}{4.198349in}}%
\pgfpathlineto{\pgfqpoint{2.090599in}{4.198044in}}%
\pgfpathlineto{\pgfqpoint{2.112792in}{4.197961in}}%
\pgfpathlineto{\pgfqpoint{2.134984in}{4.197941in}}%
\pgfpathlineto{\pgfqpoint{2.157177in}{4.197716in}}%
\pgfpathlineto{\pgfqpoint{2.179370in}{4.197651in}}%
\pgfpathlineto{\pgfqpoint{2.201563in}{4.197503in}}%
\pgfpathlineto{\pgfqpoint{2.223755in}{4.197278in}}%
\pgfpathlineto{\pgfqpoint{2.245948in}{4.197135in}}%
\pgfpathlineto{\pgfqpoint{2.268141in}{4.196949in}}%
\pgfpathlineto{\pgfqpoint{2.290334in}{4.196691in}}%
\pgfpathlineto{\pgfqpoint{2.312527in}{4.196793in}}%
\pgfpathlineto{\pgfqpoint{2.334719in}{4.196665in}}%
\pgfpathlineto{\pgfqpoint{2.356912in}{4.196547in}}%
\pgfpathlineto{\pgfqpoint{2.379105in}{4.196552in}}%
\pgfpathlineto{\pgfqpoint{2.401298in}{4.196324in}}%
\pgfpathlineto{\pgfqpoint{2.423491in}{4.196115in}}%
\pgfpathlineto{\pgfqpoint{2.445683in}{4.195821in}}%
\pgfpathlineto{\pgfqpoint{2.467876in}{4.195650in}}%
\pgfpathlineto{\pgfqpoint{2.490069in}{4.195695in}}%
\pgfpathlineto{\pgfqpoint{2.512262in}{4.195593in}}%
\pgfpathlineto{\pgfqpoint{2.534454in}{4.195602in}}%
\pgfpathlineto{\pgfqpoint{2.556647in}{4.195410in}}%
\pgfpathlineto{\pgfqpoint{2.578840in}{4.195185in}}%
\pgfpathlineto{\pgfqpoint{2.601033in}{4.194980in}}%
\pgfpathlineto{\pgfqpoint{2.623226in}{4.194690in}}%
\pgfpathlineto{\pgfqpoint{2.645418in}{4.194499in}}%
\pgfpathlineto{\pgfqpoint{2.667611in}{4.194337in}}%
\pgfusepath{stroke}%
\end{pgfscope}%
\begin{pgfscope}%
\pgfpathrectangle{\pgfqpoint{0.470525in}{2.747992in}}{\pgfqpoint{2.241471in}{1.626201in}}%
\pgfusepath{clip}%
\pgfsetrectcap%
\pgfsetroundjoin%
\pgfsetlinewidth{1.003750pt}%
\definecolor{currentstroke}{rgb}{0.000000,0.725490,0.270588}%
\pgfsetstrokecolor{currentstroke}%
\pgfsetdash{}{0pt}%
\pgfpathmoveto{\pgfqpoint{0.492718in}{4.263297in}}%
\pgfpathlineto{\pgfqpoint{0.514911in}{4.189306in}}%
\pgfpathlineto{\pgfqpoint{0.537104in}{4.148947in}}%
\pgfpathlineto{\pgfqpoint{0.559296in}{4.123666in}}%
\pgfpathlineto{\pgfqpoint{0.581489in}{4.104605in}}%
\pgfpathlineto{\pgfqpoint{0.603682in}{4.090751in}}%
\pgfpathlineto{\pgfqpoint{0.625875in}{4.079636in}}%
\pgfpathlineto{\pgfqpoint{0.648068in}{4.070744in}}%
\pgfpathlineto{\pgfqpoint{0.670260in}{4.063475in}}%
\pgfpathlineto{\pgfqpoint{0.692453in}{4.058401in}}%
\pgfpathlineto{\pgfqpoint{0.714646in}{4.050706in}}%
\pgfpathlineto{\pgfqpoint{0.736839in}{4.044257in}}%
\pgfpathlineto{\pgfqpoint{0.759032in}{4.038583in}}%
\pgfpathlineto{\pgfqpoint{0.781224in}{4.034775in}}%
\pgfpathlineto{\pgfqpoint{0.803417in}{4.030714in}}%
\pgfpathlineto{\pgfqpoint{0.825610in}{4.025900in}}%
\pgfpathlineto{\pgfqpoint{0.847803in}{4.022286in}}%
\pgfpathlineto{\pgfqpoint{0.869995in}{4.018783in}}%
\pgfpathlineto{\pgfqpoint{0.892188in}{4.013991in}}%
\pgfpathlineto{\pgfqpoint{0.914381in}{4.010581in}}%
\pgfpathlineto{\pgfqpoint{0.936574in}{4.006776in}}%
\pgfpathlineto{\pgfqpoint{0.958767in}{4.003657in}}%
\pgfpathlineto{\pgfqpoint{0.980959in}{4.001354in}}%
\pgfpathlineto{\pgfqpoint{1.003152in}{3.998782in}}%
\pgfpathlineto{\pgfqpoint{1.025345in}{3.995981in}}%
\pgfpathlineto{\pgfqpoint{1.047538in}{3.993906in}}%
\pgfpathlineto{\pgfqpoint{1.069731in}{3.991792in}}%
\pgfpathlineto{\pgfqpoint{1.091923in}{3.989241in}}%
\pgfpathlineto{\pgfqpoint{1.114116in}{3.986520in}}%
\pgfpathlineto{\pgfqpoint{1.136309in}{3.984686in}}%
\pgfpathlineto{\pgfqpoint{1.158502in}{3.982799in}}%
\pgfpathlineto{\pgfqpoint{1.180694in}{3.980842in}}%
\pgfpathlineto{\pgfqpoint{1.202887in}{3.978305in}}%
\pgfpathlineto{\pgfqpoint{1.225080in}{3.976688in}}%
\pgfpathlineto{\pgfqpoint{1.247273in}{3.974737in}}%
\pgfpathlineto{\pgfqpoint{1.269466in}{3.972842in}}%
\pgfpathlineto{\pgfqpoint{1.291658in}{3.970942in}}%
\pgfpathlineto{\pgfqpoint{1.313851in}{3.969173in}}%
\pgfpathlineto{\pgfqpoint{1.336044in}{3.967367in}}%
\pgfpathlineto{\pgfqpoint{1.358237in}{3.965868in}}%
\pgfpathlineto{\pgfqpoint{1.380430in}{3.964010in}}%
\pgfpathlineto{\pgfqpoint{1.402622in}{3.962407in}}%
\pgfpathlineto{\pgfqpoint{1.424815in}{3.960404in}}%
\pgfpathlineto{\pgfqpoint{1.447008in}{3.958679in}}%
\pgfpathlineto{\pgfqpoint{1.469201in}{3.956991in}}%
\pgfpathlineto{\pgfqpoint{1.491393in}{3.955758in}}%
\pgfpathlineto{\pgfqpoint{1.513586in}{3.954499in}}%
\pgfpathlineto{\pgfqpoint{1.535779in}{3.952909in}}%
\pgfpathlineto{\pgfqpoint{1.557972in}{3.951458in}}%
\pgfpathlineto{\pgfqpoint{1.580165in}{3.950051in}}%
\pgfpathlineto{\pgfqpoint{1.602357in}{3.949051in}}%
\pgfpathlineto{\pgfqpoint{1.624550in}{3.948098in}}%
\pgfpathlineto{\pgfqpoint{1.646743in}{3.946965in}}%
\pgfpathlineto{\pgfqpoint{1.668936in}{3.945728in}}%
\pgfpathlineto{\pgfqpoint{1.691129in}{3.944676in}}%
\pgfpathlineto{\pgfqpoint{1.713321in}{3.943516in}}%
\pgfpathlineto{\pgfqpoint{1.735514in}{3.942134in}}%
\pgfpathlineto{\pgfqpoint{1.757707in}{3.940767in}}%
\pgfpathlineto{\pgfqpoint{1.779900in}{3.939858in}}%
\pgfpathlineto{\pgfqpoint{1.802092in}{3.938577in}}%
\pgfpathlineto{\pgfqpoint{1.824285in}{3.937234in}}%
\pgfpathlineto{\pgfqpoint{1.846478in}{3.936259in}}%
\pgfpathlineto{\pgfqpoint{1.868671in}{3.935296in}}%
\pgfpathlineto{\pgfqpoint{1.890864in}{3.934032in}}%
\pgfpathlineto{\pgfqpoint{1.913056in}{3.933150in}}%
\pgfpathlineto{\pgfqpoint{1.935249in}{3.932324in}}%
\pgfpathlineto{\pgfqpoint{1.957442in}{3.931172in}}%
\pgfpathlineto{\pgfqpoint{1.979635in}{3.930390in}}%
\pgfpathlineto{\pgfqpoint{2.001828in}{3.929476in}}%
\pgfpathlineto{\pgfqpoint{2.024020in}{3.928610in}}%
\pgfpathlineto{\pgfqpoint{2.046213in}{3.927921in}}%
\pgfpathlineto{\pgfqpoint{2.068406in}{3.926922in}}%
\pgfpathlineto{\pgfqpoint{2.090599in}{3.925889in}}%
\pgfpathlineto{\pgfqpoint{2.112792in}{3.925204in}}%
\pgfpathlineto{\pgfqpoint{2.134984in}{3.924291in}}%
\pgfpathlineto{\pgfqpoint{2.157177in}{3.923301in}}%
\pgfpathlineto{\pgfqpoint{2.179370in}{3.922269in}}%
\pgfpathlineto{\pgfqpoint{2.201563in}{3.921265in}}%
\pgfpathlineto{\pgfqpoint{2.223755in}{3.920121in}}%
\pgfpathlineto{\pgfqpoint{2.245948in}{3.919091in}}%
\pgfpathlineto{\pgfqpoint{2.268141in}{3.918246in}}%
\pgfpathlineto{\pgfqpoint{2.290334in}{3.917414in}}%
\pgfpathlineto{\pgfqpoint{2.312527in}{3.916410in}}%
\pgfpathlineto{\pgfqpoint{2.334719in}{3.915245in}}%
\pgfpathlineto{\pgfqpoint{2.356912in}{3.914169in}}%
\pgfpathlineto{\pgfqpoint{2.379105in}{3.913488in}}%
\pgfpathlineto{\pgfqpoint{2.401298in}{3.912707in}}%
\pgfpathlineto{\pgfqpoint{2.423491in}{3.911914in}}%
\pgfpathlineto{\pgfqpoint{2.445683in}{3.911100in}}%
\pgfpathlineto{\pgfqpoint{2.467876in}{3.910238in}}%
\pgfpathlineto{\pgfqpoint{2.490069in}{3.909326in}}%
\pgfpathlineto{\pgfqpoint{2.512262in}{3.908413in}}%
\pgfpathlineto{\pgfqpoint{2.534454in}{3.907733in}}%
\pgfpathlineto{\pgfqpoint{2.556647in}{3.907142in}}%
\pgfpathlineto{\pgfqpoint{2.578840in}{3.906419in}}%
\pgfpathlineto{\pgfqpoint{2.601033in}{3.905801in}}%
\pgfpathlineto{\pgfqpoint{2.623226in}{3.905262in}}%
\pgfpathlineto{\pgfqpoint{2.645418in}{3.904738in}}%
\pgfpathlineto{\pgfqpoint{2.667611in}{3.904170in}}%
\pgfusepath{stroke}%
\end{pgfscope}%
\begin{pgfscope}%
\pgfpathrectangle{\pgfqpoint{0.470525in}{2.747992in}}{\pgfqpoint{2.241471in}{1.626201in}}%
\pgfusepath{clip}%
\pgfsetrectcap%
\pgfsetroundjoin%
\pgfsetlinewidth{1.003750pt}%
\definecolor{currentstroke}{rgb}{1.000000,0.584314,0.000000}%
\pgfsetstrokecolor{currentstroke}%
\pgfsetdash{}{0pt}%
\pgfpathmoveto{\pgfqpoint{0.492718in}{4.190858in}}%
\pgfpathlineto{\pgfqpoint{0.514911in}{4.161422in}}%
\pgfpathlineto{\pgfqpoint{0.537104in}{4.144535in}}%
\pgfpathlineto{\pgfqpoint{0.559296in}{4.130233in}}%
\pgfpathlineto{\pgfqpoint{0.581489in}{4.123482in}}%
\pgfpathlineto{\pgfqpoint{0.603682in}{4.116264in}}%
\pgfpathlineto{\pgfqpoint{0.625875in}{4.112273in}}%
\pgfpathlineto{\pgfqpoint{0.648068in}{4.108995in}}%
\pgfpathlineto{\pgfqpoint{0.670260in}{4.104718in}}%
\pgfpathlineto{\pgfqpoint{0.692453in}{4.098248in}}%
\pgfpathlineto{\pgfqpoint{0.714646in}{4.094807in}}%
\pgfpathlineto{\pgfqpoint{0.736839in}{4.091175in}}%
\pgfpathlineto{\pgfqpoint{0.759032in}{4.089123in}}%
\pgfpathlineto{\pgfqpoint{0.781224in}{4.086425in}}%
\pgfpathlineto{\pgfqpoint{0.803417in}{4.084123in}}%
\pgfpathlineto{\pgfqpoint{0.825610in}{4.081343in}}%
\pgfpathlineto{\pgfqpoint{0.847803in}{4.078310in}}%
\pgfpathlineto{\pgfqpoint{0.869995in}{4.076707in}}%
\pgfpathlineto{\pgfqpoint{0.892188in}{4.073736in}}%
\pgfpathlineto{\pgfqpoint{0.914381in}{4.071643in}}%
\pgfpathlineto{\pgfqpoint{0.936574in}{4.070182in}}%
\pgfpathlineto{\pgfqpoint{0.958767in}{4.068871in}}%
\pgfpathlineto{\pgfqpoint{0.980959in}{4.067567in}}%
\pgfpathlineto{\pgfqpoint{1.003152in}{4.065968in}}%
\pgfpathlineto{\pgfqpoint{1.025345in}{4.064220in}}%
\pgfpathlineto{\pgfqpoint{1.047538in}{4.063125in}}%
\pgfpathlineto{\pgfqpoint{1.069731in}{4.061276in}}%
\pgfpathlineto{\pgfqpoint{1.091923in}{4.059666in}}%
\pgfpathlineto{\pgfqpoint{1.114116in}{4.057649in}}%
\pgfpathlineto{\pgfqpoint{1.136309in}{4.056553in}}%
\pgfpathlineto{\pgfqpoint{1.158502in}{4.055297in}}%
\pgfpathlineto{\pgfqpoint{1.180694in}{4.053805in}}%
\pgfpathlineto{\pgfqpoint{1.202887in}{4.051924in}}%
\pgfpathlineto{\pgfqpoint{1.225080in}{4.050260in}}%
\pgfpathlineto{\pgfqpoint{1.247273in}{4.048524in}}%
\pgfpathlineto{\pgfqpoint{1.269466in}{4.047251in}}%
\pgfpathlineto{\pgfqpoint{1.291658in}{4.046303in}}%
\pgfpathlineto{\pgfqpoint{1.313851in}{4.045423in}}%
\pgfpathlineto{\pgfqpoint{1.336044in}{4.044184in}}%
\pgfpathlineto{\pgfqpoint{1.358237in}{4.042650in}}%
\pgfpathlineto{\pgfqpoint{1.380430in}{4.041011in}}%
\pgfpathlineto{\pgfqpoint{1.402622in}{4.039603in}}%
\pgfpathlineto{\pgfqpoint{1.424815in}{4.038803in}}%
\pgfpathlineto{\pgfqpoint{1.447008in}{4.037956in}}%
\pgfpathlineto{\pgfqpoint{1.469201in}{4.037073in}}%
\pgfpathlineto{\pgfqpoint{1.491393in}{4.036498in}}%
\pgfpathlineto{\pgfqpoint{1.513586in}{4.035590in}}%
\pgfpathlineto{\pgfqpoint{1.535779in}{4.034785in}}%
\pgfpathlineto{\pgfqpoint{1.557972in}{4.034404in}}%
\pgfpathlineto{\pgfqpoint{1.580165in}{4.033671in}}%
\pgfpathlineto{\pgfqpoint{1.602357in}{4.032938in}}%
\pgfpathlineto{\pgfqpoint{1.624550in}{4.032525in}}%
\pgfpathlineto{\pgfqpoint{1.646743in}{4.031610in}}%
\pgfpathlineto{\pgfqpoint{1.668936in}{4.030652in}}%
\pgfpathlineto{\pgfqpoint{1.691129in}{4.029490in}}%
\pgfpathlineto{\pgfqpoint{1.713321in}{4.028539in}}%
\pgfpathlineto{\pgfqpoint{1.735514in}{4.027863in}}%
\pgfpathlineto{\pgfqpoint{1.757707in}{4.027024in}}%
\pgfpathlineto{\pgfqpoint{1.779900in}{4.026034in}}%
\pgfpathlineto{\pgfqpoint{1.802092in}{4.025647in}}%
\pgfpathlineto{\pgfqpoint{1.824285in}{4.024684in}}%
\pgfpathlineto{\pgfqpoint{1.846478in}{4.024021in}}%
\pgfpathlineto{\pgfqpoint{1.868671in}{4.023536in}}%
\pgfpathlineto{\pgfqpoint{1.890864in}{4.022723in}}%
\pgfpathlineto{\pgfqpoint{1.913056in}{4.021807in}}%
\pgfpathlineto{\pgfqpoint{1.935249in}{4.021478in}}%
\pgfpathlineto{\pgfqpoint{1.957442in}{4.020834in}}%
\pgfpathlineto{\pgfqpoint{1.979635in}{4.020042in}}%
\pgfpathlineto{\pgfqpoint{2.001828in}{4.019346in}}%
\pgfpathlineto{\pgfqpoint{2.024020in}{4.018754in}}%
\pgfpathlineto{\pgfqpoint{2.046213in}{4.018075in}}%
\pgfpathlineto{\pgfqpoint{2.068406in}{4.017407in}}%
\pgfpathlineto{\pgfqpoint{2.090599in}{4.016800in}}%
\pgfpathlineto{\pgfqpoint{2.112792in}{4.016513in}}%
\pgfpathlineto{\pgfqpoint{2.134984in}{4.015973in}}%
\pgfpathlineto{\pgfqpoint{2.157177in}{4.015363in}}%
\pgfpathlineto{\pgfqpoint{2.179370in}{4.014873in}}%
\pgfpathlineto{\pgfqpoint{2.201563in}{4.014330in}}%
\pgfpathlineto{\pgfqpoint{2.223755in}{4.014115in}}%
\pgfpathlineto{\pgfqpoint{2.245948in}{4.013532in}}%
\pgfpathlineto{\pgfqpoint{2.268141in}{4.013053in}}%
\pgfpathlineto{\pgfqpoint{2.290334in}{4.012391in}}%
\pgfpathlineto{\pgfqpoint{2.312527in}{4.011784in}}%
\pgfpathlineto{\pgfqpoint{2.334719in}{4.011445in}}%
\pgfpathlineto{\pgfqpoint{2.356912in}{4.011083in}}%
\pgfpathlineto{\pgfqpoint{2.379105in}{4.010534in}}%
\pgfpathlineto{\pgfqpoint{2.401298in}{4.010005in}}%
\pgfpathlineto{\pgfqpoint{2.423491in}{4.009475in}}%
\pgfpathlineto{\pgfqpoint{2.445683in}{4.008891in}}%
\pgfpathlineto{\pgfqpoint{2.467876in}{4.008300in}}%
\pgfpathlineto{\pgfqpoint{2.490069in}{4.007938in}}%
\pgfpathlineto{\pgfqpoint{2.512262in}{4.007526in}}%
\pgfpathlineto{\pgfqpoint{2.534454in}{4.006807in}}%
\pgfpathlineto{\pgfqpoint{2.556647in}{4.006297in}}%
\pgfpathlineto{\pgfqpoint{2.578840in}{4.005872in}}%
\pgfpathlineto{\pgfqpoint{2.601033in}{4.005388in}}%
\pgfpathlineto{\pgfqpoint{2.623226in}{4.004978in}}%
\pgfpathlineto{\pgfqpoint{2.645418in}{4.004526in}}%
\pgfpathlineto{\pgfqpoint{2.667611in}{4.004051in}}%
\pgfusepath{stroke}%
\end{pgfscope}%
\begin{pgfscope}%
\pgfpathrectangle{\pgfqpoint{0.470525in}{2.747992in}}{\pgfqpoint{2.241471in}{1.626201in}}%
\pgfusepath{clip}%
\pgfsetrectcap%
\pgfsetroundjoin%
\pgfsetlinewidth{1.003750pt}%
\definecolor{currentstroke}{rgb}{1.000000,0.172549,0.000000}%
\pgfsetstrokecolor{currentstroke}%
\pgfsetdash{}{0pt}%
\pgfpathmoveto{\pgfqpoint{0.492718in}{3.452228in}}%
\pgfpathlineto{\pgfqpoint{0.514911in}{3.247082in}}%
\pgfpathlineto{\pgfqpoint{0.537104in}{3.138654in}}%
\pgfpathlineto{\pgfqpoint{0.559296in}{3.074149in}}%
\pgfpathlineto{\pgfqpoint{0.581489in}{3.031381in}}%
\pgfpathlineto{\pgfqpoint{0.603682in}{3.001471in}}%
\pgfpathlineto{\pgfqpoint{0.625875in}{2.976159in}}%
\pgfpathlineto{\pgfqpoint{0.648068in}{2.954311in}}%
\pgfpathlineto{\pgfqpoint{0.670260in}{2.945459in}}%
\pgfpathlineto{\pgfqpoint{0.692453in}{2.936739in}}%
\pgfpathlineto{\pgfqpoint{0.714646in}{2.928572in}}%
\pgfpathlineto{\pgfqpoint{0.736839in}{2.924023in}}%
\pgfpathlineto{\pgfqpoint{0.759032in}{2.919543in}}%
\pgfpathlineto{\pgfqpoint{0.781224in}{2.913100in}}%
\pgfpathlineto{\pgfqpoint{0.803417in}{2.906994in}}%
\pgfpathlineto{\pgfqpoint{0.825610in}{2.903560in}}%
\pgfpathlineto{\pgfqpoint{0.847803in}{2.900578in}}%
\pgfpathlineto{\pgfqpoint{0.869995in}{2.898437in}}%
\pgfpathlineto{\pgfqpoint{0.892188in}{2.898157in}}%
\pgfpathlineto{\pgfqpoint{0.914381in}{2.893083in}}%
\pgfpathlineto{\pgfqpoint{0.936574in}{2.889596in}}%
\pgfpathlineto{\pgfqpoint{0.958767in}{2.886066in}}%
\pgfpathlineto{\pgfqpoint{0.980959in}{2.882970in}}%
\pgfpathlineto{\pgfqpoint{1.003152in}{2.881082in}}%
\pgfpathlineto{\pgfqpoint{1.025345in}{2.879123in}}%
\pgfpathlineto{\pgfqpoint{1.047538in}{2.877826in}}%
\pgfpathlineto{\pgfqpoint{1.069731in}{2.877162in}}%
\pgfpathlineto{\pgfqpoint{1.091923in}{2.876175in}}%
\pgfpathlineto{\pgfqpoint{1.114116in}{2.875486in}}%
\pgfpathlineto{\pgfqpoint{1.136309in}{2.874222in}}%
\pgfpathlineto{\pgfqpoint{1.158502in}{2.872709in}}%
\pgfpathlineto{\pgfqpoint{1.180694in}{2.872017in}}%
\pgfpathlineto{\pgfqpoint{1.202887in}{2.871244in}}%
\pgfpathlineto{\pgfqpoint{1.225080in}{2.869755in}}%
\pgfpathlineto{\pgfqpoint{1.247273in}{2.868122in}}%
\pgfpathlineto{\pgfqpoint{1.269466in}{2.867304in}}%
\pgfpathlineto{\pgfqpoint{1.291658in}{2.866585in}}%
\pgfpathlineto{\pgfqpoint{1.313851in}{2.865708in}}%
\pgfpathlineto{\pgfqpoint{1.336044in}{2.864541in}}%
\pgfpathlineto{\pgfqpoint{1.358237in}{2.863638in}}%
\pgfpathlineto{\pgfqpoint{1.380430in}{2.863015in}}%
\pgfpathlineto{\pgfqpoint{1.402622in}{2.862085in}}%
\pgfpathlineto{\pgfqpoint{1.424815in}{2.860924in}}%
\pgfpathlineto{\pgfqpoint{1.447008in}{2.859189in}}%
\pgfpathlineto{\pgfqpoint{1.469201in}{2.857742in}}%
\pgfpathlineto{\pgfqpoint{1.491393in}{2.856659in}}%
\pgfpathlineto{\pgfqpoint{1.513586in}{2.856224in}}%
\pgfpathlineto{\pgfqpoint{1.535779in}{2.856119in}}%
\pgfpathlineto{\pgfqpoint{1.557972in}{2.855648in}}%
\pgfpathlineto{\pgfqpoint{1.580165in}{2.855270in}}%
\pgfpathlineto{\pgfqpoint{1.602357in}{2.853933in}}%
\pgfpathlineto{\pgfqpoint{1.624550in}{2.853464in}}%
\pgfpathlineto{\pgfqpoint{1.646743in}{2.852177in}}%
\pgfpathlineto{\pgfqpoint{1.668936in}{2.851220in}}%
\pgfpathlineto{\pgfqpoint{1.691129in}{2.850932in}}%
\pgfpathlineto{\pgfqpoint{1.713321in}{2.850426in}}%
\pgfpathlineto{\pgfqpoint{1.735514in}{2.850042in}}%
\pgfpathlineto{\pgfqpoint{1.757707in}{2.849127in}}%
\pgfpathlineto{\pgfqpoint{1.779900in}{2.848192in}}%
\pgfpathlineto{\pgfqpoint{1.802092in}{2.848112in}}%
\pgfpathlineto{\pgfqpoint{1.824285in}{2.847719in}}%
\pgfpathlineto{\pgfqpoint{1.846478in}{2.846870in}}%
\pgfpathlineto{\pgfqpoint{1.868671in}{2.846107in}}%
\pgfpathlineto{\pgfqpoint{1.890864in}{2.844841in}}%
\pgfpathlineto{\pgfqpoint{1.913056in}{2.843876in}}%
\pgfpathlineto{\pgfqpoint{1.935249in}{2.843386in}}%
\pgfpathlineto{\pgfqpoint{1.957442in}{2.842655in}}%
\pgfpathlineto{\pgfqpoint{1.979635in}{2.841111in}}%
\pgfpathlineto{\pgfqpoint{2.001828in}{2.840169in}}%
\pgfpathlineto{\pgfqpoint{2.024020in}{2.839350in}}%
\pgfpathlineto{\pgfqpoint{2.046213in}{2.838433in}}%
\pgfpathlineto{\pgfqpoint{2.068406in}{2.838069in}}%
\pgfpathlineto{\pgfqpoint{2.090599in}{2.837298in}}%
\pgfpathlineto{\pgfqpoint{2.112792in}{2.836710in}}%
\pgfpathlineto{\pgfqpoint{2.134984in}{2.835715in}}%
\pgfpathlineto{\pgfqpoint{2.157177in}{2.834930in}}%
\pgfpathlineto{\pgfqpoint{2.179370in}{2.834425in}}%
\pgfpathlineto{\pgfqpoint{2.201563in}{2.833616in}}%
\pgfpathlineto{\pgfqpoint{2.223755in}{2.833125in}}%
\pgfpathlineto{\pgfqpoint{2.245948in}{2.832252in}}%
\pgfpathlineto{\pgfqpoint{2.268141in}{2.831542in}}%
\pgfpathlineto{\pgfqpoint{2.290334in}{2.830591in}}%
\pgfpathlineto{\pgfqpoint{2.312527in}{2.830099in}}%
\pgfpathlineto{\pgfqpoint{2.334719in}{2.829935in}}%
\pgfpathlineto{\pgfqpoint{2.356912in}{2.829622in}}%
\pgfpathlineto{\pgfqpoint{2.379105in}{2.829024in}}%
\pgfpathlineto{\pgfqpoint{2.401298in}{2.828189in}}%
\pgfpathlineto{\pgfqpoint{2.423491in}{2.827577in}}%
\pgfpathlineto{\pgfqpoint{2.445683in}{2.826965in}}%
\pgfpathlineto{\pgfqpoint{2.467876in}{2.826495in}}%
\pgfpathlineto{\pgfqpoint{2.490069in}{2.825802in}}%
\pgfpathlineto{\pgfqpoint{2.512262in}{2.825367in}}%
\pgfpathlineto{\pgfqpoint{2.534454in}{2.824914in}}%
\pgfpathlineto{\pgfqpoint{2.556647in}{2.824612in}}%
\pgfpathlineto{\pgfqpoint{2.578840in}{2.823859in}}%
\pgfpathlineto{\pgfqpoint{2.601033in}{2.823683in}}%
\pgfpathlineto{\pgfqpoint{2.623226in}{2.823077in}}%
\pgfpathlineto{\pgfqpoint{2.645418in}{2.822620in}}%
\pgfpathlineto{\pgfqpoint{2.667611in}{2.821910in}}%
\pgfusepath{stroke}%
\end{pgfscope}%
\begin{pgfscope}%
\pgfpathrectangle{\pgfqpoint{0.470525in}{2.747992in}}{\pgfqpoint{2.241471in}{1.626201in}}%
\pgfusepath{clip}%
\pgfsetrectcap%
\pgfsetroundjoin%
\pgfsetlinewidth{1.003750pt}%
\definecolor{currentstroke}{rgb}{0.517647,0.356863,0.592157}%
\pgfsetstrokecolor{currentstroke}%
\pgfsetdash{}{0pt}%
\pgfpathmoveto{\pgfqpoint{0.492718in}{3.365687in}}%
\pgfpathlineto{\pgfqpoint{0.514911in}{3.282782in}}%
\pgfpathlineto{\pgfqpoint{0.537104in}{3.357676in}}%
\pgfpathlineto{\pgfqpoint{0.559296in}{3.376622in}}%
\pgfpathlineto{\pgfqpoint{0.581489in}{3.347561in}}%
\pgfpathlineto{\pgfqpoint{0.603682in}{3.291911in}}%
\pgfpathlineto{\pgfqpoint{0.625875in}{3.256271in}}%
\pgfpathlineto{\pgfqpoint{0.648068in}{3.248702in}}%
\pgfpathlineto{\pgfqpoint{0.670260in}{3.240254in}}%
\pgfpathlineto{\pgfqpoint{0.692453in}{3.259997in}}%
\pgfpathlineto{\pgfqpoint{0.714646in}{3.271704in}}%
\pgfpathlineto{\pgfqpoint{0.736839in}{3.286336in}}%
\pgfpathlineto{\pgfqpoint{0.759032in}{3.278456in}}%
\pgfpathlineto{\pgfqpoint{0.781224in}{3.262434in}}%
\pgfpathlineto{\pgfqpoint{0.803417in}{3.241888in}}%
\pgfpathlineto{\pgfqpoint{0.825610in}{3.248573in}}%
\pgfpathlineto{\pgfqpoint{0.847803in}{3.258519in}}%
\pgfpathlineto{\pgfqpoint{0.869995in}{3.215678in}}%
\pgfpathlineto{\pgfqpoint{0.892188in}{3.203859in}}%
\pgfpathlineto{\pgfqpoint{0.914381in}{3.218861in}}%
\pgfpathlineto{\pgfqpoint{0.936574in}{3.220150in}}%
\pgfpathlineto{\pgfqpoint{0.958767in}{3.227540in}}%
\pgfpathlineto{\pgfqpoint{0.980959in}{3.210774in}}%
\pgfpathlineto{\pgfqpoint{1.003152in}{3.216941in}}%
\pgfpathlineto{\pgfqpoint{1.025345in}{3.230156in}}%
\pgfpathlineto{\pgfqpoint{1.047538in}{3.224953in}}%
\pgfpathlineto{\pgfqpoint{1.069731in}{3.240354in}}%
\pgfpathlineto{\pgfqpoint{1.091923in}{3.249905in}}%
\pgfpathlineto{\pgfqpoint{1.114116in}{3.236476in}}%
\pgfpathlineto{\pgfqpoint{1.136309in}{3.244386in}}%
\pgfpathlineto{\pgfqpoint{1.158502in}{3.258795in}}%
\pgfpathlineto{\pgfqpoint{1.180694in}{3.248567in}}%
\pgfpathlineto{\pgfqpoint{1.202887in}{3.236549in}}%
\pgfpathlineto{\pgfqpoint{1.225080in}{3.237846in}}%
\pgfpathlineto{\pgfqpoint{1.247273in}{3.240747in}}%
\pgfpathlineto{\pgfqpoint{1.269466in}{3.243046in}}%
\pgfpathlineto{\pgfqpoint{1.291658in}{3.249003in}}%
\pgfpathlineto{\pgfqpoint{1.313851in}{3.254627in}}%
\pgfpathlineto{\pgfqpoint{1.336044in}{3.254758in}}%
\pgfpathlineto{\pgfqpoint{1.358237in}{3.254238in}}%
\pgfpathlineto{\pgfqpoint{1.380430in}{3.260695in}}%
\pgfpathlineto{\pgfqpoint{1.402622in}{3.258466in}}%
\pgfpathlineto{\pgfqpoint{1.424815in}{3.256577in}}%
\pgfpathlineto{\pgfqpoint{1.447008in}{3.247635in}}%
\pgfpathlineto{\pgfqpoint{1.469201in}{3.252370in}}%
\pgfpathlineto{\pgfqpoint{1.491393in}{3.248725in}}%
\pgfpathlineto{\pgfqpoint{1.513586in}{3.253798in}}%
\pgfpathlineto{\pgfqpoint{1.535779in}{3.257430in}}%
\pgfpathlineto{\pgfqpoint{1.557972in}{3.256718in}}%
\pgfpathlineto{\pgfqpoint{1.580165in}{3.259147in}}%
\pgfpathlineto{\pgfqpoint{1.602357in}{3.267749in}}%
\pgfpathlineto{\pgfqpoint{1.624550in}{3.266730in}}%
\pgfpathlineto{\pgfqpoint{1.646743in}{3.262237in}}%
\pgfpathlineto{\pgfqpoint{1.668936in}{3.262382in}}%
\pgfpathlineto{\pgfqpoint{1.691129in}{3.264050in}}%
\pgfpathlineto{\pgfqpoint{1.713321in}{3.266226in}}%
\pgfpathlineto{\pgfqpoint{1.735514in}{3.265778in}}%
\pgfpathlineto{\pgfqpoint{1.757707in}{3.269952in}}%
\pgfpathlineto{\pgfqpoint{1.779900in}{3.276063in}}%
\pgfpathlineto{\pgfqpoint{1.802092in}{3.271881in}}%
\pgfpathlineto{\pgfqpoint{1.824285in}{3.271007in}}%
\pgfpathlineto{\pgfqpoint{1.846478in}{3.269088in}}%
\pgfpathlineto{\pgfqpoint{1.868671in}{3.271465in}}%
\pgfpathlineto{\pgfqpoint{1.890864in}{3.263049in}}%
\pgfpathlineto{\pgfqpoint{1.913056in}{3.266205in}}%
\pgfpathlineto{\pgfqpoint{1.935249in}{3.267717in}}%
\pgfpathlineto{\pgfqpoint{1.957442in}{3.274000in}}%
\pgfpathlineto{\pgfqpoint{1.979635in}{3.272227in}}%
\pgfpathlineto{\pgfqpoint{2.001828in}{3.275323in}}%
\pgfpathlineto{\pgfqpoint{2.024020in}{3.274191in}}%
\pgfpathlineto{\pgfqpoint{2.046213in}{3.272205in}}%
\pgfpathlineto{\pgfqpoint{2.068406in}{3.274159in}}%
\pgfpathlineto{\pgfqpoint{2.090599in}{3.274471in}}%
\pgfpathlineto{\pgfqpoint{2.112792in}{3.270620in}}%
\pgfpathlineto{\pgfqpoint{2.134984in}{3.274299in}}%
\pgfpathlineto{\pgfqpoint{2.157177in}{3.276638in}}%
\pgfpathlineto{\pgfqpoint{2.179370in}{3.275489in}}%
\pgfpathlineto{\pgfqpoint{2.201563in}{3.275457in}}%
\pgfpathlineto{\pgfqpoint{2.223755in}{3.273113in}}%
\pgfpathlineto{\pgfqpoint{2.245948in}{3.274008in}}%
\pgfpathlineto{\pgfqpoint{2.268141in}{3.275597in}}%
\pgfpathlineto{\pgfqpoint{2.290334in}{3.275843in}}%
\pgfpathlineto{\pgfqpoint{2.312527in}{3.276433in}}%
\pgfpathlineto{\pgfqpoint{2.334719in}{3.278935in}}%
\pgfpathlineto{\pgfqpoint{2.356912in}{3.275891in}}%
\pgfpathlineto{\pgfqpoint{2.379105in}{3.276400in}}%
\pgfpathlineto{\pgfqpoint{2.401298in}{3.279601in}}%
\pgfpathlineto{\pgfqpoint{2.423491in}{3.279849in}}%
\pgfpathlineto{\pgfqpoint{2.445683in}{3.283361in}}%
\pgfpathlineto{\pgfqpoint{2.467876in}{3.282388in}}%
\pgfpathlineto{\pgfqpoint{2.490069in}{3.284532in}}%
\pgfpathlineto{\pgfqpoint{2.512262in}{3.286227in}}%
\pgfpathlineto{\pgfqpoint{2.534454in}{3.286012in}}%
\pgfpathlineto{\pgfqpoint{2.556647in}{3.282862in}}%
\pgfpathlineto{\pgfqpoint{2.578840in}{3.280886in}}%
\pgfpathlineto{\pgfqpoint{2.601033in}{3.280617in}}%
\pgfpathlineto{\pgfqpoint{2.623226in}{3.281454in}}%
\pgfpathlineto{\pgfqpoint{2.645418in}{3.283180in}}%
\pgfpathlineto{\pgfqpoint{2.667611in}{3.284201in}}%
\pgfusepath{stroke}%
\end{pgfscope}%
\begin{pgfscope}%
\pgfsetrectcap%
\pgfsetmiterjoin%
\pgfsetlinewidth{0.501875pt}%
\definecolor{currentstroke}{rgb}{0.000000,0.000000,0.000000}%
\pgfsetstrokecolor{currentstroke}%
\pgfsetdash{}{0pt}%
\pgfpathmoveto{\pgfqpoint{0.470525in}{2.747992in}}%
\pgfpathlineto{\pgfqpoint{0.470525in}{4.374193in}}%
\pgfusepath{stroke}%
\end{pgfscope}%
\begin{pgfscope}%
\pgfsetrectcap%
\pgfsetmiterjoin%
\pgfsetlinewidth{0.501875pt}%
\definecolor{currentstroke}{rgb}{0.000000,0.000000,0.000000}%
\pgfsetstrokecolor{currentstroke}%
\pgfsetdash{}{0pt}%
\pgfpathmoveto{\pgfqpoint{2.711997in}{2.747992in}}%
\pgfpathlineto{\pgfqpoint{2.711997in}{4.374193in}}%
\pgfusepath{stroke}%
\end{pgfscope}%
\begin{pgfscope}%
\pgfsetrectcap%
\pgfsetmiterjoin%
\pgfsetlinewidth{0.501875pt}%
\definecolor{currentstroke}{rgb}{0.000000,0.000000,0.000000}%
\pgfsetstrokecolor{currentstroke}%
\pgfsetdash{}{0pt}%
\pgfpathmoveto{\pgfqpoint{0.470525in}{2.747992in}}%
\pgfpathlineto{\pgfqpoint{2.711997in}{2.747992in}}%
\pgfusepath{stroke}%
\end{pgfscope}%
\begin{pgfscope}%
\pgfsetrectcap%
\pgfsetmiterjoin%
\pgfsetlinewidth{0.501875pt}%
\definecolor{currentstroke}{rgb}{0.000000,0.000000,0.000000}%
\pgfsetstrokecolor{currentstroke}%
\pgfsetdash{}{0pt}%
\pgfpathmoveto{\pgfqpoint{0.470525in}{4.374193in}}%
\pgfpathlineto{\pgfqpoint{2.711997in}{4.374193in}}%
\pgfusepath{stroke}%
\end{pgfscope}%
\begin{pgfscope}%
\definecolor{textcolor}{rgb}{0.000000,0.000000,0.000000}%
\pgfsetstrokecolor{textcolor}%
\pgfsetfillcolor{textcolor}%
\pgftext[x=1.591261in,y=4.457526in,,base]{\color{textcolor}\rmfamily\fontsize{12.000000}{14.400000}\selectfont Trustworthiness}%
\end{pgfscope}%
\begin{pgfscope}%
\pgfsetbuttcap%
\pgfsetmiterjoin%
\definecolor{currentfill}{rgb}{1.000000,1.000000,1.000000}%
\pgfsetfillcolor{currentfill}%
\pgfsetlinewidth{0.000000pt}%
\definecolor{currentstroke}{rgb}{0.000000,0.000000,0.000000}%
\pgfsetstrokecolor{currentstroke}%
\pgfsetstrokeopacity{0.000000}%
\pgfsetdash{}{0pt}%
\pgfpathmoveto{\pgfqpoint{0.470525in}{0.422992in}}%
\pgfpathlineto{\pgfqpoint{2.711997in}{0.422992in}}%
\pgfpathlineto{\pgfqpoint{2.711997in}{2.049193in}}%
\pgfpathlineto{\pgfqpoint{0.470525in}{2.049193in}}%
\pgfpathlineto{\pgfqpoint{0.470525in}{0.422992in}}%
\pgfpathclose%
\pgfusepath{fill}%
\end{pgfscope}%
\begin{pgfscope}%
\pgfsetbuttcap%
\pgfsetroundjoin%
\definecolor{currentfill}{rgb}{0.000000,0.000000,0.000000}%
\pgfsetfillcolor{currentfill}%
\pgfsetlinewidth{0.501875pt}%
\definecolor{currentstroke}{rgb}{0.000000,0.000000,0.000000}%
\pgfsetstrokecolor{currentstroke}%
\pgfsetdash{}{0pt}%
\pgfsys@defobject{currentmarker}{\pgfqpoint{0.000000in}{0.000000in}}{\pgfqpoint{0.000000in}{0.041667in}}{%
\pgfpathmoveto{\pgfqpoint{0.000000in}{0.000000in}}%
\pgfpathlineto{\pgfqpoint{0.000000in}{0.041667in}}%
\pgfusepath{stroke,fill}%
}%
\begin{pgfscope}%
\pgfsys@transformshift{0.470525in}{0.422992in}%
\pgfsys@useobject{currentmarker}{}%
\end{pgfscope}%
\end{pgfscope}%
\begin{pgfscope}%
\pgfsetbuttcap%
\pgfsetroundjoin%
\definecolor{currentfill}{rgb}{0.000000,0.000000,0.000000}%
\pgfsetfillcolor{currentfill}%
\pgfsetlinewidth{0.501875pt}%
\definecolor{currentstroke}{rgb}{0.000000,0.000000,0.000000}%
\pgfsetstrokecolor{currentstroke}%
\pgfsetdash{}{0pt}%
\pgfsys@defobject{currentmarker}{\pgfqpoint{0.000000in}{-0.041667in}}{\pgfqpoint{0.000000in}{0.000000in}}{%
\pgfpathmoveto{\pgfqpoint{0.000000in}{0.000000in}}%
\pgfpathlineto{\pgfqpoint{0.000000in}{-0.041667in}}%
\pgfusepath{stroke,fill}%
}%
\begin{pgfscope}%
\pgfsys@transformshift{0.470525in}{2.049193in}%
\pgfsys@useobject{currentmarker}{}%
\end{pgfscope}%
\end{pgfscope}%
\begin{pgfscope}%
\definecolor{textcolor}{rgb}{0.000000,0.000000,0.000000}%
\pgfsetstrokecolor{textcolor}%
\pgfsetfillcolor{textcolor}%
\pgftext[x=0.470525in,y=0.374381in,,top]{\color{textcolor}\rmfamily\fontsize{10.000000}{12.000000}\selectfont \(\displaystyle {0}\)}%
\end{pgfscope}%
\begin{pgfscope}%
\pgfsetbuttcap%
\pgfsetroundjoin%
\definecolor{currentfill}{rgb}{0.000000,0.000000,0.000000}%
\pgfsetfillcolor{currentfill}%
\pgfsetlinewidth{0.501875pt}%
\definecolor{currentstroke}{rgb}{0.000000,0.000000,0.000000}%
\pgfsetstrokecolor{currentstroke}%
\pgfsetdash{}{0pt}%
\pgfsys@defobject{currentmarker}{\pgfqpoint{0.000000in}{0.000000in}}{\pgfqpoint{0.000000in}{0.041667in}}{%
\pgfpathmoveto{\pgfqpoint{0.000000in}{0.000000in}}%
\pgfpathlineto{\pgfqpoint{0.000000in}{0.041667in}}%
\pgfusepath{stroke,fill}%
}%
\begin{pgfscope}%
\pgfsys@transformshift{0.914381in}{0.422992in}%
\pgfsys@useobject{currentmarker}{}%
\end{pgfscope}%
\end{pgfscope}%
\begin{pgfscope}%
\pgfsetbuttcap%
\pgfsetroundjoin%
\definecolor{currentfill}{rgb}{0.000000,0.000000,0.000000}%
\pgfsetfillcolor{currentfill}%
\pgfsetlinewidth{0.501875pt}%
\definecolor{currentstroke}{rgb}{0.000000,0.000000,0.000000}%
\pgfsetstrokecolor{currentstroke}%
\pgfsetdash{}{0pt}%
\pgfsys@defobject{currentmarker}{\pgfqpoint{0.000000in}{-0.041667in}}{\pgfqpoint{0.000000in}{0.000000in}}{%
\pgfpathmoveto{\pgfqpoint{0.000000in}{0.000000in}}%
\pgfpathlineto{\pgfqpoint{0.000000in}{-0.041667in}}%
\pgfusepath{stroke,fill}%
}%
\begin{pgfscope}%
\pgfsys@transformshift{0.914381in}{2.049193in}%
\pgfsys@useobject{currentmarker}{}%
\end{pgfscope}%
\end{pgfscope}%
\begin{pgfscope}%
\definecolor{textcolor}{rgb}{0.000000,0.000000,0.000000}%
\pgfsetstrokecolor{textcolor}%
\pgfsetfillcolor{textcolor}%
\pgftext[x=0.914381in,y=0.374381in,,top]{\color{textcolor}\rmfamily\fontsize{10.000000}{12.000000}\selectfont \(\displaystyle {20}\)}%
\end{pgfscope}%
\begin{pgfscope}%
\pgfsetbuttcap%
\pgfsetroundjoin%
\definecolor{currentfill}{rgb}{0.000000,0.000000,0.000000}%
\pgfsetfillcolor{currentfill}%
\pgfsetlinewidth{0.501875pt}%
\definecolor{currentstroke}{rgb}{0.000000,0.000000,0.000000}%
\pgfsetstrokecolor{currentstroke}%
\pgfsetdash{}{0pt}%
\pgfsys@defobject{currentmarker}{\pgfqpoint{0.000000in}{0.000000in}}{\pgfqpoint{0.000000in}{0.041667in}}{%
\pgfpathmoveto{\pgfqpoint{0.000000in}{0.000000in}}%
\pgfpathlineto{\pgfqpoint{0.000000in}{0.041667in}}%
\pgfusepath{stroke,fill}%
}%
\begin{pgfscope}%
\pgfsys@transformshift{1.358237in}{0.422992in}%
\pgfsys@useobject{currentmarker}{}%
\end{pgfscope}%
\end{pgfscope}%
\begin{pgfscope}%
\pgfsetbuttcap%
\pgfsetroundjoin%
\definecolor{currentfill}{rgb}{0.000000,0.000000,0.000000}%
\pgfsetfillcolor{currentfill}%
\pgfsetlinewidth{0.501875pt}%
\definecolor{currentstroke}{rgb}{0.000000,0.000000,0.000000}%
\pgfsetstrokecolor{currentstroke}%
\pgfsetdash{}{0pt}%
\pgfsys@defobject{currentmarker}{\pgfqpoint{0.000000in}{-0.041667in}}{\pgfqpoint{0.000000in}{0.000000in}}{%
\pgfpathmoveto{\pgfqpoint{0.000000in}{0.000000in}}%
\pgfpathlineto{\pgfqpoint{0.000000in}{-0.041667in}}%
\pgfusepath{stroke,fill}%
}%
\begin{pgfscope}%
\pgfsys@transformshift{1.358237in}{2.049193in}%
\pgfsys@useobject{currentmarker}{}%
\end{pgfscope}%
\end{pgfscope}%
\begin{pgfscope}%
\definecolor{textcolor}{rgb}{0.000000,0.000000,0.000000}%
\pgfsetstrokecolor{textcolor}%
\pgfsetfillcolor{textcolor}%
\pgftext[x=1.358237in,y=0.374381in,,top]{\color{textcolor}\rmfamily\fontsize{10.000000}{12.000000}\selectfont \(\displaystyle {40}\)}%
\end{pgfscope}%
\begin{pgfscope}%
\pgfsetbuttcap%
\pgfsetroundjoin%
\definecolor{currentfill}{rgb}{0.000000,0.000000,0.000000}%
\pgfsetfillcolor{currentfill}%
\pgfsetlinewidth{0.501875pt}%
\definecolor{currentstroke}{rgb}{0.000000,0.000000,0.000000}%
\pgfsetstrokecolor{currentstroke}%
\pgfsetdash{}{0pt}%
\pgfsys@defobject{currentmarker}{\pgfqpoint{0.000000in}{0.000000in}}{\pgfqpoint{0.000000in}{0.041667in}}{%
\pgfpathmoveto{\pgfqpoint{0.000000in}{0.000000in}}%
\pgfpathlineto{\pgfqpoint{0.000000in}{0.041667in}}%
\pgfusepath{stroke,fill}%
}%
\begin{pgfscope}%
\pgfsys@transformshift{1.802092in}{0.422992in}%
\pgfsys@useobject{currentmarker}{}%
\end{pgfscope}%
\end{pgfscope}%
\begin{pgfscope}%
\pgfsetbuttcap%
\pgfsetroundjoin%
\definecolor{currentfill}{rgb}{0.000000,0.000000,0.000000}%
\pgfsetfillcolor{currentfill}%
\pgfsetlinewidth{0.501875pt}%
\definecolor{currentstroke}{rgb}{0.000000,0.000000,0.000000}%
\pgfsetstrokecolor{currentstroke}%
\pgfsetdash{}{0pt}%
\pgfsys@defobject{currentmarker}{\pgfqpoint{0.000000in}{-0.041667in}}{\pgfqpoint{0.000000in}{0.000000in}}{%
\pgfpathmoveto{\pgfqpoint{0.000000in}{0.000000in}}%
\pgfpathlineto{\pgfqpoint{0.000000in}{-0.041667in}}%
\pgfusepath{stroke,fill}%
}%
\begin{pgfscope}%
\pgfsys@transformshift{1.802092in}{2.049193in}%
\pgfsys@useobject{currentmarker}{}%
\end{pgfscope}%
\end{pgfscope}%
\begin{pgfscope}%
\definecolor{textcolor}{rgb}{0.000000,0.000000,0.000000}%
\pgfsetstrokecolor{textcolor}%
\pgfsetfillcolor{textcolor}%
\pgftext[x=1.802092in,y=0.374381in,,top]{\color{textcolor}\rmfamily\fontsize{10.000000}{12.000000}\selectfont \(\displaystyle {60}\)}%
\end{pgfscope}%
\begin{pgfscope}%
\pgfsetbuttcap%
\pgfsetroundjoin%
\definecolor{currentfill}{rgb}{0.000000,0.000000,0.000000}%
\pgfsetfillcolor{currentfill}%
\pgfsetlinewidth{0.501875pt}%
\definecolor{currentstroke}{rgb}{0.000000,0.000000,0.000000}%
\pgfsetstrokecolor{currentstroke}%
\pgfsetdash{}{0pt}%
\pgfsys@defobject{currentmarker}{\pgfqpoint{0.000000in}{0.000000in}}{\pgfqpoint{0.000000in}{0.041667in}}{%
\pgfpathmoveto{\pgfqpoint{0.000000in}{0.000000in}}%
\pgfpathlineto{\pgfqpoint{0.000000in}{0.041667in}}%
\pgfusepath{stroke,fill}%
}%
\begin{pgfscope}%
\pgfsys@transformshift{2.245948in}{0.422992in}%
\pgfsys@useobject{currentmarker}{}%
\end{pgfscope}%
\end{pgfscope}%
\begin{pgfscope}%
\pgfsetbuttcap%
\pgfsetroundjoin%
\definecolor{currentfill}{rgb}{0.000000,0.000000,0.000000}%
\pgfsetfillcolor{currentfill}%
\pgfsetlinewidth{0.501875pt}%
\definecolor{currentstroke}{rgb}{0.000000,0.000000,0.000000}%
\pgfsetstrokecolor{currentstroke}%
\pgfsetdash{}{0pt}%
\pgfsys@defobject{currentmarker}{\pgfqpoint{0.000000in}{-0.041667in}}{\pgfqpoint{0.000000in}{0.000000in}}{%
\pgfpathmoveto{\pgfqpoint{0.000000in}{0.000000in}}%
\pgfpathlineto{\pgfqpoint{0.000000in}{-0.041667in}}%
\pgfusepath{stroke,fill}%
}%
\begin{pgfscope}%
\pgfsys@transformshift{2.245948in}{2.049193in}%
\pgfsys@useobject{currentmarker}{}%
\end{pgfscope}%
\end{pgfscope}%
\begin{pgfscope}%
\definecolor{textcolor}{rgb}{0.000000,0.000000,0.000000}%
\pgfsetstrokecolor{textcolor}%
\pgfsetfillcolor{textcolor}%
\pgftext[x=2.245948in,y=0.374381in,,top]{\color{textcolor}\rmfamily\fontsize{10.000000}{12.000000}\selectfont \(\displaystyle {80}\)}%
\end{pgfscope}%
\begin{pgfscope}%
\pgfsetbuttcap%
\pgfsetroundjoin%
\definecolor{currentfill}{rgb}{0.000000,0.000000,0.000000}%
\pgfsetfillcolor{currentfill}%
\pgfsetlinewidth{0.501875pt}%
\definecolor{currentstroke}{rgb}{0.000000,0.000000,0.000000}%
\pgfsetstrokecolor{currentstroke}%
\pgfsetdash{}{0pt}%
\pgfsys@defobject{currentmarker}{\pgfqpoint{0.000000in}{0.000000in}}{\pgfqpoint{0.000000in}{0.020833in}}{%
\pgfpathmoveto{\pgfqpoint{0.000000in}{0.000000in}}%
\pgfpathlineto{\pgfqpoint{0.000000in}{0.020833in}}%
\pgfusepath{stroke,fill}%
}%
\begin{pgfscope}%
\pgfsys@transformshift{0.581489in}{0.422992in}%
\pgfsys@useobject{currentmarker}{}%
\end{pgfscope}%
\end{pgfscope}%
\begin{pgfscope}%
\pgfsetbuttcap%
\pgfsetroundjoin%
\definecolor{currentfill}{rgb}{0.000000,0.000000,0.000000}%
\pgfsetfillcolor{currentfill}%
\pgfsetlinewidth{0.501875pt}%
\definecolor{currentstroke}{rgb}{0.000000,0.000000,0.000000}%
\pgfsetstrokecolor{currentstroke}%
\pgfsetdash{}{0pt}%
\pgfsys@defobject{currentmarker}{\pgfqpoint{0.000000in}{-0.020833in}}{\pgfqpoint{0.000000in}{0.000000in}}{%
\pgfpathmoveto{\pgfqpoint{0.000000in}{0.000000in}}%
\pgfpathlineto{\pgfqpoint{0.000000in}{-0.020833in}}%
\pgfusepath{stroke,fill}%
}%
\begin{pgfscope}%
\pgfsys@transformshift{0.581489in}{2.049193in}%
\pgfsys@useobject{currentmarker}{}%
\end{pgfscope}%
\end{pgfscope}%
\begin{pgfscope}%
\pgfsetbuttcap%
\pgfsetroundjoin%
\definecolor{currentfill}{rgb}{0.000000,0.000000,0.000000}%
\pgfsetfillcolor{currentfill}%
\pgfsetlinewidth{0.501875pt}%
\definecolor{currentstroke}{rgb}{0.000000,0.000000,0.000000}%
\pgfsetstrokecolor{currentstroke}%
\pgfsetdash{}{0pt}%
\pgfsys@defobject{currentmarker}{\pgfqpoint{0.000000in}{0.000000in}}{\pgfqpoint{0.000000in}{0.020833in}}{%
\pgfpathmoveto{\pgfqpoint{0.000000in}{0.000000in}}%
\pgfpathlineto{\pgfqpoint{0.000000in}{0.020833in}}%
\pgfusepath{stroke,fill}%
}%
\begin{pgfscope}%
\pgfsys@transformshift{0.692453in}{0.422992in}%
\pgfsys@useobject{currentmarker}{}%
\end{pgfscope}%
\end{pgfscope}%
\begin{pgfscope}%
\pgfsetbuttcap%
\pgfsetroundjoin%
\definecolor{currentfill}{rgb}{0.000000,0.000000,0.000000}%
\pgfsetfillcolor{currentfill}%
\pgfsetlinewidth{0.501875pt}%
\definecolor{currentstroke}{rgb}{0.000000,0.000000,0.000000}%
\pgfsetstrokecolor{currentstroke}%
\pgfsetdash{}{0pt}%
\pgfsys@defobject{currentmarker}{\pgfqpoint{0.000000in}{-0.020833in}}{\pgfqpoint{0.000000in}{0.000000in}}{%
\pgfpathmoveto{\pgfqpoint{0.000000in}{0.000000in}}%
\pgfpathlineto{\pgfqpoint{0.000000in}{-0.020833in}}%
\pgfusepath{stroke,fill}%
}%
\begin{pgfscope}%
\pgfsys@transformshift{0.692453in}{2.049193in}%
\pgfsys@useobject{currentmarker}{}%
\end{pgfscope}%
\end{pgfscope}%
\begin{pgfscope}%
\pgfsetbuttcap%
\pgfsetroundjoin%
\definecolor{currentfill}{rgb}{0.000000,0.000000,0.000000}%
\pgfsetfillcolor{currentfill}%
\pgfsetlinewidth{0.501875pt}%
\definecolor{currentstroke}{rgb}{0.000000,0.000000,0.000000}%
\pgfsetstrokecolor{currentstroke}%
\pgfsetdash{}{0pt}%
\pgfsys@defobject{currentmarker}{\pgfqpoint{0.000000in}{0.000000in}}{\pgfqpoint{0.000000in}{0.020833in}}{%
\pgfpathmoveto{\pgfqpoint{0.000000in}{0.000000in}}%
\pgfpathlineto{\pgfqpoint{0.000000in}{0.020833in}}%
\pgfusepath{stroke,fill}%
}%
\begin{pgfscope}%
\pgfsys@transformshift{0.803417in}{0.422992in}%
\pgfsys@useobject{currentmarker}{}%
\end{pgfscope}%
\end{pgfscope}%
\begin{pgfscope}%
\pgfsetbuttcap%
\pgfsetroundjoin%
\definecolor{currentfill}{rgb}{0.000000,0.000000,0.000000}%
\pgfsetfillcolor{currentfill}%
\pgfsetlinewidth{0.501875pt}%
\definecolor{currentstroke}{rgb}{0.000000,0.000000,0.000000}%
\pgfsetstrokecolor{currentstroke}%
\pgfsetdash{}{0pt}%
\pgfsys@defobject{currentmarker}{\pgfqpoint{0.000000in}{-0.020833in}}{\pgfqpoint{0.000000in}{0.000000in}}{%
\pgfpathmoveto{\pgfqpoint{0.000000in}{0.000000in}}%
\pgfpathlineto{\pgfqpoint{0.000000in}{-0.020833in}}%
\pgfusepath{stroke,fill}%
}%
\begin{pgfscope}%
\pgfsys@transformshift{0.803417in}{2.049193in}%
\pgfsys@useobject{currentmarker}{}%
\end{pgfscope}%
\end{pgfscope}%
\begin{pgfscope}%
\pgfsetbuttcap%
\pgfsetroundjoin%
\definecolor{currentfill}{rgb}{0.000000,0.000000,0.000000}%
\pgfsetfillcolor{currentfill}%
\pgfsetlinewidth{0.501875pt}%
\definecolor{currentstroke}{rgb}{0.000000,0.000000,0.000000}%
\pgfsetstrokecolor{currentstroke}%
\pgfsetdash{}{0pt}%
\pgfsys@defobject{currentmarker}{\pgfqpoint{0.000000in}{0.000000in}}{\pgfqpoint{0.000000in}{0.020833in}}{%
\pgfpathmoveto{\pgfqpoint{0.000000in}{0.000000in}}%
\pgfpathlineto{\pgfqpoint{0.000000in}{0.020833in}}%
\pgfusepath{stroke,fill}%
}%
\begin{pgfscope}%
\pgfsys@transformshift{1.025345in}{0.422992in}%
\pgfsys@useobject{currentmarker}{}%
\end{pgfscope}%
\end{pgfscope}%
\begin{pgfscope}%
\pgfsetbuttcap%
\pgfsetroundjoin%
\definecolor{currentfill}{rgb}{0.000000,0.000000,0.000000}%
\pgfsetfillcolor{currentfill}%
\pgfsetlinewidth{0.501875pt}%
\definecolor{currentstroke}{rgb}{0.000000,0.000000,0.000000}%
\pgfsetstrokecolor{currentstroke}%
\pgfsetdash{}{0pt}%
\pgfsys@defobject{currentmarker}{\pgfqpoint{0.000000in}{-0.020833in}}{\pgfqpoint{0.000000in}{0.000000in}}{%
\pgfpathmoveto{\pgfqpoint{0.000000in}{0.000000in}}%
\pgfpathlineto{\pgfqpoint{0.000000in}{-0.020833in}}%
\pgfusepath{stroke,fill}%
}%
\begin{pgfscope}%
\pgfsys@transformshift{1.025345in}{2.049193in}%
\pgfsys@useobject{currentmarker}{}%
\end{pgfscope}%
\end{pgfscope}%
\begin{pgfscope}%
\pgfsetbuttcap%
\pgfsetroundjoin%
\definecolor{currentfill}{rgb}{0.000000,0.000000,0.000000}%
\pgfsetfillcolor{currentfill}%
\pgfsetlinewidth{0.501875pt}%
\definecolor{currentstroke}{rgb}{0.000000,0.000000,0.000000}%
\pgfsetstrokecolor{currentstroke}%
\pgfsetdash{}{0pt}%
\pgfsys@defobject{currentmarker}{\pgfqpoint{0.000000in}{0.000000in}}{\pgfqpoint{0.000000in}{0.020833in}}{%
\pgfpathmoveto{\pgfqpoint{0.000000in}{0.000000in}}%
\pgfpathlineto{\pgfqpoint{0.000000in}{0.020833in}}%
\pgfusepath{stroke,fill}%
}%
\begin{pgfscope}%
\pgfsys@transformshift{1.136309in}{0.422992in}%
\pgfsys@useobject{currentmarker}{}%
\end{pgfscope}%
\end{pgfscope}%
\begin{pgfscope}%
\pgfsetbuttcap%
\pgfsetroundjoin%
\definecolor{currentfill}{rgb}{0.000000,0.000000,0.000000}%
\pgfsetfillcolor{currentfill}%
\pgfsetlinewidth{0.501875pt}%
\definecolor{currentstroke}{rgb}{0.000000,0.000000,0.000000}%
\pgfsetstrokecolor{currentstroke}%
\pgfsetdash{}{0pt}%
\pgfsys@defobject{currentmarker}{\pgfqpoint{0.000000in}{-0.020833in}}{\pgfqpoint{0.000000in}{0.000000in}}{%
\pgfpathmoveto{\pgfqpoint{0.000000in}{0.000000in}}%
\pgfpathlineto{\pgfqpoint{0.000000in}{-0.020833in}}%
\pgfusepath{stroke,fill}%
}%
\begin{pgfscope}%
\pgfsys@transformshift{1.136309in}{2.049193in}%
\pgfsys@useobject{currentmarker}{}%
\end{pgfscope}%
\end{pgfscope}%
\begin{pgfscope}%
\pgfsetbuttcap%
\pgfsetroundjoin%
\definecolor{currentfill}{rgb}{0.000000,0.000000,0.000000}%
\pgfsetfillcolor{currentfill}%
\pgfsetlinewidth{0.501875pt}%
\definecolor{currentstroke}{rgb}{0.000000,0.000000,0.000000}%
\pgfsetstrokecolor{currentstroke}%
\pgfsetdash{}{0pt}%
\pgfsys@defobject{currentmarker}{\pgfqpoint{0.000000in}{0.000000in}}{\pgfqpoint{0.000000in}{0.020833in}}{%
\pgfpathmoveto{\pgfqpoint{0.000000in}{0.000000in}}%
\pgfpathlineto{\pgfqpoint{0.000000in}{0.020833in}}%
\pgfusepath{stroke,fill}%
}%
\begin{pgfscope}%
\pgfsys@transformshift{1.247273in}{0.422992in}%
\pgfsys@useobject{currentmarker}{}%
\end{pgfscope}%
\end{pgfscope}%
\begin{pgfscope}%
\pgfsetbuttcap%
\pgfsetroundjoin%
\definecolor{currentfill}{rgb}{0.000000,0.000000,0.000000}%
\pgfsetfillcolor{currentfill}%
\pgfsetlinewidth{0.501875pt}%
\definecolor{currentstroke}{rgb}{0.000000,0.000000,0.000000}%
\pgfsetstrokecolor{currentstroke}%
\pgfsetdash{}{0pt}%
\pgfsys@defobject{currentmarker}{\pgfqpoint{0.000000in}{-0.020833in}}{\pgfqpoint{0.000000in}{0.000000in}}{%
\pgfpathmoveto{\pgfqpoint{0.000000in}{0.000000in}}%
\pgfpathlineto{\pgfqpoint{0.000000in}{-0.020833in}}%
\pgfusepath{stroke,fill}%
}%
\begin{pgfscope}%
\pgfsys@transformshift{1.247273in}{2.049193in}%
\pgfsys@useobject{currentmarker}{}%
\end{pgfscope}%
\end{pgfscope}%
\begin{pgfscope}%
\pgfsetbuttcap%
\pgfsetroundjoin%
\definecolor{currentfill}{rgb}{0.000000,0.000000,0.000000}%
\pgfsetfillcolor{currentfill}%
\pgfsetlinewidth{0.501875pt}%
\definecolor{currentstroke}{rgb}{0.000000,0.000000,0.000000}%
\pgfsetstrokecolor{currentstroke}%
\pgfsetdash{}{0pt}%
\pgfsys@defobject{currentmarker}{\pgfqpoint{0.000000in}{0.000000in}}{\pgfqpoint{0.000000in}{0.020833in}}{%
\pgfpathmoveto{\pgfqpoint{0.000000in}{0.000000in}}%
\pgfpathlineto{\pgfqpoint{0.000000in}{0.020833in}}%
\pgfusepath{stroke,fill}%
}%
\begin{pgfscope}%
\pgfsys@transformshift{1.469201in}{0.422992in}%
\pgfsys@useobject{currentmarker}{}%
\end{pgfscope}%
\end{pgfscope}%
\begin{pgfscope}%
\pgfsetbuttcap%
\pgfsetroundjoin%
\definecolor{currentfill}{rgb}{0.000000,0.000000,0.000000}%
\pgfsetfillcolor{currentfill}%
\pgfsetlinewidth{0.501875pt}%
\definecolor{currentstroke}{rgb}{0.000000,0.000000,0.000000}%
\pgfsetstrokecolor{currentstroke}%
\pgfsetdash{}{0pt}%
\pgfsys@defobject{currentmarker}{\pgfqpoint{0.000000in}{-0.020833in}}{\pgfqpoint{0.000000in}{0.000000in}}{%
\pgfpathmoveto{\pgfqpoint{0.000000in}{0.000000in}}%
\pgfpathlineto{\pgfqpoint{0.000000in}{-0.020833in}}%
\pgfusepath{stroke,fill}%
}%
\begin{pgfscope}%
\pgfsys@transformshift{1.469201in}{2.049193in}%
\pgfsys@useobject{currentmarker}{}%
\end{pgfscope}%
\end{pgfscope}%
\begin{pgfscope}%
\pgfsetbuttcap%
\pgfsetroundjoin%
\definecolor{currentfill}{rgb}{0.000000,0.000000,0.000000}%
\pgfsetfillcolor{currentfill}%
\pgfsetlinewidth{0.501875pt}%
\definecolor{currentstroke}{rgb}{0.000000,0.000000,0.000000}%
\pgfsetstrokecolor{currentstroke}%
\pgfsetdash{}{0pt}%
\pgfsys@defobject{currentmarker}{\pgfqpoint{0.000000in}{0.000000in}}{\pgfqpoint{0.000000in}{0.020833in}}{%
\pgfpathmoveto{\pgfqpoint{0.000000in}{0.000000in}}%
\pgfpathlineto{\pgfqpoint{0.000000in}{0.020833in}}%
\pgfusepath{stroke,fill}%
}%
\begin{pgfscope}%
\pgfsys@transformshift{1.580165in}{0.422992in}%
\pgfsys@useobject{currentmarker}{}%
\end{pgfscope}%
\end{pgfscope}%
\begin{pgfscope}%
\pgfsetbuttcap%
\pgfsetroundjoin%
\definecolor{currentfill}{rgb}{0.000000,0.000000,0.000000}%
\pgfsetfillcolor{currentfill}%
\pgfsetlinewidth{0.501875pt}%
\definecolor{currentstroke}{rgb}{0.000000,0.000000,0.000000}%
\pgfsetstrokecolor{currentstroke}%
\pgfsetdash{}{0pt}%
\pgfsys@defobject{currentmarker}{\pgfqpoint{0.000000in}{-0.020833in}}{\pgfqpoint{0.000000in}{0.000000in}}{%
\pgfpathmoveto{\pgfqpoint{0.000000in}{0.000000in}}%
\pgfpathlineto{\pgfqpoint{0.000000in}{-0.020833in}}%
\pgfusepath{stroke,fill}%
}%
\begin{pgfscope}%
\pgfsys@transformshift{1.580165in}{2.049193in}%
\pgfsys@useobject{currentmarker}{}%
\end{pgfscope}%
\end{pgfscope}%
\begin{pgfscope}%
\pgfsetbuttcap%
\pgfsetroundjoin%
\definecolor{currentfill}{rgb}{0.000000,0.000000,0.000000}%
\pgfsetfillcolor{currentfill}%
\pgfsetlinewidth{0.501875pt}%
\definecolor{currentstroke}{rgb}{0.000000,0.000000,0.000000}%
\pgfsetstrokecolor{currentstroke}%
\pgfsetdash{}{0pt}%
\pgfsys@defobject{currentmarker}{\pgfqpoint{0.000000in}{0.000000in}}{\pgfqpoint{0.000000in}{0.020833in}}{%
\pgfpathmoveto{\pgfqpoint{0.000000in}{0.000000in}}%
\pgfpathlineto{\pgfqpoint{0.000000in}{0.020833in}}%
\pgfusepath{stroke,fill}%
}%
\begin{pgfscope}%
\pgfsys@transformshift{1.691129in}{0.422992in}%
\pgfsys@useobject{currentmarker}{}%
\end{pgfscope}%
\end{pgfscope}%
\begin{pgfscope}%
\pgfsetbuttcap%
\pgfsetroundjoin%
\definecolor{currentfill}{rgb}{0.000000,0.000000,0.000000}%
\pgfsetfillcolor{currentfill}%
\pgfsetlinewidth{0.501875pt}%
\definecolor{currentstroke}{rgb}{0.000000,0.000000,0.000000}%
\pgfsetstrokecolor{currentstroke}%
\pgfsetdash{}{0pt}%
\pgfsys@defobject{currentmarker}{\pgfqpoint{0.000000in}{-0.020833in}}{\pgfqpoint{0.000000in}{0.000000in}}{%
\pgfpathmoveto{\pgfqpoint{0.000000in}{0.000000in}}%
\pgfpathlineto{\pgfqpoint{0.000000in}{-0.020833in}}%
\pgfusepath{stroke,fill}%
}%
\begin{pgfscope}%
\pgfsys@transformshift{1.691129in}{2.049193in}%
\pgfsys@useobject{currentmarker}{}%
\end{pgfscope}%
\end{pgfscope}%
\begin{pgfscope}%
\pgfsetbuttcap%
\pgfsetroundjoin%
\definecolor{currentfill}{rgb}{0.000000,0.000000,0.000000}%
\pgfsetfillcolor{currentfill}%
\pgfsetlinewidth{0.501875pt}%
\definecolor{currentstroke}{rgb}{0.000000,0.000000,0.000000}%
\pgfsetstrokecolor{currentstroke}%
\pgfsetdash{}{0pt}%
\pgfsys@defobject{currentmarker}{\pgfqpoint{0.000000in}{0.000000in}}{\pgfqpoint{0.000000in}{0.020833in}}{%
\pgfpathmoveto{\pgfqpoint{0.000000in}{0.000000in}}%
\pgfpathlineto{\pgfqpoint{0.000000in}{0.020833in}}%
\pgfusepath{stroke,fill}%
}%
\begin{pgfscope}%
\pgfsys@transformshift{1.913056in}{0.422992in}%
\pgfsys@useobject{currentmarker}{}%
\end{pgfscope}%
\end{pgfscope}%
\begin{pgfscope}%
\pgfsetbuttcap%
\pgfsetroundjoin%
\definecolor{currentfill}{rgb}{0.000000,0.000000,0.000000}%
\pgfsetfillcolor{currentfill}%
\pgfsetlinewidth{0.501875pt}%
\definecolor{currentstroke}{rgb}{0.000000,0.000000,0.000000}%
\pgfsetstrokecolor{currentstroke}%
\pgfsetdash{}{0pt}%
\pgfsys@defobject{currentmarker}{\pgfqpoint{0.000000in}{-0.020833in}}{\pgfqpoint{0.000000in}{0.000000in}}{%
\pgfpathmoveto{\pgfqpoint{0.000000in}{0.000000in}}%
\pgfpathlineto{\pgfqpoint{0.000000in}{-0.020833in}}%
\pgfusepath{stroke,fill}%
}%
\begin{pgfscope}%
\pgfsys@transformshift{1.913056in}{2.049193in}%
\pgfsys@useobject{currentmarker}{}%
\end{pgfscope}%
\end{pgfscope}%
\begin{pgfscope}%
\pgfsetbuttcap%
\pgfsetroundjoin%
\definecolor{currentfill}{rgb}{0.000000,0.000000,0.000000}%
\pgfsetfillcolor{currentfill}%
\pgfsetlinewidth{0.501875pt}%
\definecolor{currentstroke}{rgb}{0.000000,0.000000,0.000000}%
\pgfsetstrokecolor{currentstroke}%
\pgfsetdash{}{0pt}%
\pgfsys@defobject{currentmarker}{\pgfqpoint{0.000000in}{0.000000in}}{\pgfqpoint{0.000000in}{0.020833in}}{%
\pgfpathmoveto{\pgfqpoint{0.000000in}{0.000000in}}%
\pgfpathlineto{\pgfqpoint{0.000000in}{0.020833in}}%
\pgfusepath{stroke,fill}%
}%
\begin{pgfscope}%
\pgfsys@transformshift{2.024020in}{0.422992in}%
\pgfsys@useobject{currentmarker}{}%
\end{pgfscope}%
\end{pgfscope}%
\begin{pgfscope}%
\pgfsetbuttcap%
\pgfsetroundjoin%
\definecolor{currentfill}{rgb}{0.000000,0.000000,0.000000}%
\pgfsetfillcolor{currentfill}%
\pgfsetlinewidth{0.501875pt}%
\definecolor{currentstroke}{rgb}{0.000000,0.000000,0.000000}%
\pgfsetstrokecolor{currentstroke}%
\pgfsetdash{}{0pt}%
\pgfsys@defobject{currentmarker}{\pgfqpoint{0.000000in}{-0.020833in}}{\pgfqpoint{0.000000in}{0.000000in}}{%
\pgfpathmoveto{\pgfqpoint{0.000000in}{0.000000in}}%
\pgfpathlineto{\pgfqpoint{0.000000in}{-0.020833in}}%
\pgfusepath{stroke,fill}%
}%
\begin{pgfscope}%
\pgfsys@transformshift{2.024020in}{2.049193in}%
\pgfsys@useobject{currentmarker}{}%
\end{pgfscope}%
\end{pgfscope}%
\begin{pgfscope}%
\pgfsetbuttcap%
\pgfsetroundjoin%
\definecolor{currentfill}{rgb}{0.000000,0.000000,0.000000}%
\pgfsetfillcolor{currentfill}%
\pgfsetlinewidth{0.501875pt}%
\definecolor{currentstroke}{rgb}{0.000000,0.000000,0.000000}%
\pgfsetstrokecolor{currentstroke}%
\pgfsetdash{}{0pt}%
\pgfsys@defobject{currentmarker}{\pgfqpoint{0.000000in}{0.000000in}}{\pgfqpoint{0.000000in}{0.020833in}}{%
\pgfpathmoveto{\pgfqpoint{0.000000in}{0.000000in}}%
\pgfpathlineto{\pgfqpoint{0.000000in}{0.020833in}}%
\pgfusepath{stroke,fill}%
}%
\begin{pgfscope}%
\pgfsys@transformshift{2.134984in}{0.422992in}%
\pgfsys@useobject{currentmarker}{}%
\end{pgfscope}%
\end{pgfscope}%
\begin{pgfscope}%
\pgfsetbuttcap%
\pgfsetroundjoin%
\definecolor{currentfill}{rgb}{0.000000,0.000000,0.000000}%
\pgfsetfillcolor{currentfill}%
\pgfsetlinewidth{0.501875pt}%
\definecolor{currentstroke}{rgb}{0.000000,0.000000,0.000000}%
\pgfsetstrokecolor{currentstroke}%
\pgfsetdash{}{0pt}%
\pgfsys@defobject{currentmarker}{\pgfqpoint{0.000000in}{-0.020833in}}{\pgfqpoint{0.000000in}{0.000000in}}{%
\pgfpathmoveto{\pgfqpoint{0.000000in}{0.000000in}}%
\pgfpathlineto{\pgfqpoint{0.000000in}{-0.020833in}}%
\pgfusepath{stroke,fill}%
}%
\begin{pgfscope}%
\pgfsys@transformshift{2.134984in}{2.049193in}%
\pgfsys@useobject{currentmarker}{}%
\end{pgfscope}%
\end{pgfscope}%
\begin{pgfscope}%
\pgfsetbuttcap%
\pgfsetroundjoin%
\definecolor{currentfill}{rgb}{0.000000,0.000000,0.000000}%
\pgfsetfillcolor{currentfill}%
\pgfsetlinewidth{0.501875pt}%
\definecolor{currentstroke}{rgb}{0.000000,0.000000,0.000000}%
\pgfsetstrokecolor{currentstroke}%
\pgfsetdash{}{0pt}%
\pgfsys@defobject{currentmarker}{\pgfqpoint{0.000000in}{0.000000in}}{\pgfqpoint{0.000000in}{0.020833in}}{%
\pgfpathmoveto{\pgfqpoint{0.000000in}{0.000000in}}%
\pgfpathlineto{\pgfqpoint{0.000000in}{0.020833in}}%
\pgfusepath{stroke,fill}%
}%
\begin{pgfscope}%
\pgfsys@transformshift{2.356912in}{0.422992in}%
\pgfsys@useobject{currentmarker}{}%
\end{pgfscope}%
\end{pgfscope}%
\begin{pgfscope}%
\pgfsetbuttcap%
\pgfsetroundjoin%
\definecolor{currentfill}{rgb}{0.000000,0.000000,0.000000}%
\pgfsetfillcolor{currentfill}%
\pgfsetlinewidth{0.501875pt}%
\definecolor{currentstroke}{rgb}{0.000000,0.000000,0.000000}%
\pgfsetstrokecolor{currentstroke}%
\pgfsetdash{}{0pt}%
\pgfsys@defobject{currentmarker}{\pgfqpoint{0.000000in}{-0.020833in}}{\pgfqpoint{0.000000in}{0.000000in}}{%
\pgfpathmoveto{\pgfqpoint{0.000000in}{0.000000in}}%
\pgfpathlineto{\pgfqpoint{0.000000in}{-0.020833in}}%
\pgfusepath{stroke,fill}%
}%
\begin{pgfscope}%
\pgfsys@transformshift{2.356912in}{2.049193in}%
\pgfsys@useobject{currentmarker}{}%
\end{pgfscope}%
\end{pgfscope}%
\begin{pgfscope}%
\pgfsetbuttcap%
\pgfsetroundjoin%
\definecolor{currentfill}{rgb}{0.000000,0.000000,0.000000}%
\pgfsetfillcolor{currentfill}%
\pgfsetlinewidth{0.501875pt}%
\definecolor{currentstroke}{rgb}{0.000000,0.000000,0.000000}%
\pgfsetstrokecolor{currentstroke}%
\pgfsetdash{}{0pt}%
\pgfsys@defobject{currentmarker}{\pgfqpoint{0.000000in}{0.000000in}}{\pgfqpoint{0.000000in}{0.020833in}}{%
\pgfpathmoveto{\pgfqpoint{0.000000in}{0.000000in}}%
\pgfpathlineto{\pgfqpoint{0.000000in}{0.020833in}}%
\pgfusepath{stroke,fill}%
}%
\begin{pgfscope}%
\pgfsys@transformshift{2.467876in}{0.422992in}%
\pgfsys@useobject{currentmarker}{}%
\end{pgfscope}%
\end{pgfscope}%
\begin{pgfscope}%
\pgfsetbuttcap%
\pgfsetroundjoin%
\definecolor{currentfill}{rgb}{0.000000,0.000000,0.000000}%
\pgfsetfillcolor{currentfill}%
\pgfsetlinewidth{0.501875pt}%
\definecolor{currentstroke}{rgb}{0.000000,0.000000,0.000000}%
\pgfsetstrokecolor{currentstroke}%
\pgfsetdash{}{0pt}%
\pgfsys@defobject{currentmarker}{\pgfqpoint{0.000000in}{-0.020833in}}{\pgfqpoint{0.000000in}{0.000000in}}{%
\pgfpathmoveto{\pgfqpoint{0.000000in}{0.000000in}}%
\pgfpathlineto{\pgfqpoint{0.000000in}{-0.020833in}}%
\pgfusepath{stroke,fill}%
}%
\begin{pgfscope}%
\pgfsys@transformshift{2.467876in}{2.049193in}%
\pgfsys@useobject{currentmarker}{}%
\end{pgfscope}%
\end{pgfscope}%
\begin{pgfscope}%
\pgfsetbuttcap%
\pgfsetroundjoin%
\definecolor{currentfill}{rgb}{0.000000,0.000000,0.000000}%
\pgfsetfillcolor{currentfill}%
\pgfsetlinewidth{0.501875pt}%
\definecolor{currentstroke}{rgb}{0.000000,0.000000,0.000000}%
\pgfsetstrokecolor{currentstroke}%
\pgfsetdash{}{0pt}%
\pgfsys@defobject{currentmarker}{\pgfqpoint{0.000000in}{0.000000in}}{\pgfqpoint{0.000000in}{0.020833in}}{%
\pgfpathmoveto{\pgfqpoint{0.000000in}{0.000000in}}%
\pgfpathlineto{\pgfqpoint{0.000000in}{0.020833in}}%
\pgfusepath{stroke,fill}%
}%
\begin{pgfscope}%
\pgfsys@transformshift{2.578840in}{0.422992in}%
\pgfsys@useobject{currentmarker}{}%
\end{pgfscope}%
\end{pgfscope}%
\begin{pgfscope}%
\pgfsetbuttcap%
\pgfsetroundjoin%
\definecolor{currentfill}{rgb}{0.000000,0.000000,0.000000}%
\pgfsetfillcolor{currentfill}%
\pgfsetlinewidth{0.501875pt}%
\definecolor{currentstroke}{rgb}{0.000000,0.000000,0.000000}%
\pgfsetstrokecolor{currentstroke}%
\pgfsetdash{}{0pt}%
\pgfsys@defobject{currentmarker}{\pgfqpoint{0.000000in}{-0.020833in}}{\pgfqpoint{0.000000in}{0.000000in}}{%
\pgfpathmoveto{\pgfqpoint{0.000000in}{0.000000in}}%
\pgfpathlineto{\pgfqpoint{0.000000in}{-0.020833in}}%
\pgfusepath{stroke,fill}%
}%
\begin{pgfscope}%
\pgfsys@transformshift{2.578840in}{2.049193in}%
\pgfsys@useobject{currentmarker}{}%
\end{pgfscope}%
\end{pgfscope}%
\begin{pgfscope}%
\pgfsetbuttcap%
\pgfsetroundjoin%
\definecolor{currentfill}{rgb}{0.000000,0.000000,0.000000}%
\pgfsetfillcolor{currentfill}%
\pgfsetlinewidth{0.501875pt}%
\definecolor{currentstroke}{rgb}{0.000000,0.000000,0.000000}%
\pgfsetstrokecolor{currentstroke}%
\pgfsetdash{}{0pt}%
\pgfsys@defobject{currentmarker}{\pgfqpoint{0.000000in}{0.000000in}}{\pgfqpoint{0.000000in}{0.020833in}}{%
\pgfpathmoveto{\pgfqpoint{0.000000in}{0.000000in}}%
\pgfpathlineto{\pgfqpoint{0.000000in}{0.020833in}}%
\pgfusepath{stroke,fill}%
}%
\begin{pgfscope}%
\pgfsys@transformshift{2.689804in}{0.422992in}%
\pgfsys@useobject{currentmarker}{}%
\end{pgfscope}%
\end{pgfscope}%
\begin{pgfscope}%
\pgfsetbuttcap%
\pgfsetroundjoin%
\definecolor{currentfill}{rgb}{0.000000,0.000000,0.000000}%
\pgfsetfillcolor{currentfill}%
\pgfsetlinewidth{0.501875pt}%
\definecolor{currentstroke}{rgb}{0.000000,0.000000,0.000000}%
\pgfsetstrokecolor{currentstroke}%
\pgfsetdash{}{0pt}%
\pgfsys@defobject{currentmarker}{\pgfqpoint{0.000000in}{-0.020833in}}{\pgfqpoint{0.000000in}{0.000000in}}{%
\pgfpathmoveto{\pgfqpoint{0.000000in}{0.000000in}}%
\pgfpathlineto{\pgfqpoint{0.000000in}{-0.020833in}}%
\pgfusepath{stroke,fill}%
}%
\begin{pgfscope}%
\pgfsys@transformshift{2.689804in}{2.049193in}%
\pgfsys@useobject{currentmarker}{}%
\end{pgfscope}%
\end{pgfscope}%
\begin{pgfscope}%
\definecolor{textcolor}{rgb}{0.000000,0.000000,0.000000}%
\pgfsetstrokecolor{textcolor}%
\pgfsetfillcolor{textcolor}%
\pgftext[x=1.591261in,y=0.184413in,,top]{\color{textcolor}\rmfamily\fontsize{10.000000}{12.000000}\selectfont \(\displaystyle K\)}%
\end{pgfscope}%
\begin{pgfscope}%
\pgfsetbuttcap%
\pgfsetroundjoin%
\definecolor{currentfill}{rgb}{0.000000,0.000000,0.000000}%
\pgfsetfillcolor{currentfill}%
\pgfsetlinewidth{0.501875pt}%
\definecolor{currentstroke}{rgb}{0.000000,0.000000,0.000000}%
\pgfsetstrokecolor{currentstroke}%
\pgfsetdash{}{0pt}%
\pgfsys@defobject{currentmarker}{\pgfqpoint{0.000000in}{0.000000in}}{\pgfqpoint{0.041667in}{0.000000in}}{%
\pgfpathmoveto{\pgfqpoint{0.000000in}{0.000000in}}%
\pgfpathlineto{\pgfqpoint{0.041667in}{0.000000in}}%
\pgfusepath{stroke,fill}%
}%
\begin{pgfscope}%
\pgfsys@transformshift{0.470525in}{0.427362in}%
\pgfsys@useobject{currentmarker}{}%
\end{pgfscope}%
\end{pgfscope}%
\begin{pgfscope}%
\pgfsetbuttcap%
\pgfsetroundjoin%
\definecolor{currentfill}{rgb}{0.000000,0.000000,0.000000}%
\pgfsetfillcolor{currentfill}%
\pgfsetlinewidth{0.501875pt}%
\definecolor{currentstroke}{rgb}{0.000000,0.000000,0.000000}%
\pgfsetstrokecolor{currentstroke}%
\pgfsetdash{}{0pt}%
\pgfsys@defobject{currentmarker}{\pgfqpoint{-0.041667in}{0.000000in}}{\pgfqpoint{-0.000000in}{0.000000in}}{%
\pgfpathmoveto{\pgfqpoint{-0.000000in}{0.000000in}}%
\pgfpathlineto{\pgfqpoint{-0.041667in}{0.000000in}}%
\pgfusepath{stroke,fill}%
}%
\begin{pgfscope}%
\pgfsys@transformshift{2.711997in}{0.427362in}%
\pgfsys@useobject{currentmarker}{}%
\end{pgfscope}%
\end{pgfscope}%
\begin{pgfscope}%
\definecolor{textcolor}{rgb}{0.000000,0.000000,0.000000}%
\pgfsetstrokecolor{textcolor}%
\pgfsetfillcolor{textcolor}%
\pgftext[x=0.244444in, y=0.374601in, left, base]{\color{textcolor}\rmfamily\fontsize{10.000000}{12.000000}\selectfont \(\displaystyle {0.6}\)}%
\end{pgfscope}%
\begin{pgfscope}%
\pgfsetbuttcap%
\pgfsetroundjoin%
\definecolor{currentfill}{rgb}{0.000000,0.000000,0.000000}%
\pgfsetfillcolor{currentfill}%
\pgfsetlinewidth{0.501875pt}%
\definecolor{currentstroke}{rgb}{0.000000,0.000000,0.000000}%
\pgfsetstrokecolor{currentstroke}%
\pgfsetdash{}{0pt}%
\pgfsys@defobject{currentmarker}{\pgfqpoint{0.000000in}{0.000000in}}{\pgfqpoint{0.041667in}{0.000000in}}{%
\pgfpathmoveto{\pgfqpoint{0.000000in}{0.000000in}}%
\pgfpathlineto{\pgfqpoint{0.041667in}{0.000000in}}%
\pgfusepath{stroke,fill}%
}%
\begin{pgfscope}%
\pgfsys@transformshift{0.470525in}{0.818752in}%
\pgfsys@useobject{currentmarker}{}%
\end{pgfscope}%
\end{pgfscope}%
\begin{pgfscope}%
\pgfsetbuttcap%
\pgfsetroundjoin%
\definecolor{currentfill}{rgb}{0.000000,0.000000,0.000000}%
\pgfsetfillcolor{currentfill}%
\pgfsetlinewidth{0.501875pt}%
\definecolor{currentstroke}{rgb}{0.000000,0.000000,0.000000}%
\pgfsetstrokecolor{currentstroke}%
\pgfsetdash{}{0pt}%
\pgfsys@defobject{currentmarker}{\pgfqpoint{-0.041667in}{0.000000in}}{\pgfqpoint{-0.000000in}{0.000000in}}{%
\pgfpathmoveto{\pgfqpoint{-0.000000in}{0.000000in}}%
\pgfpathlineto{\pgfqpoint{-0.041667in}{0.000000in}}%
\pgfusepath{stroke,fill}%
}%
\begin{pgfscope}%
\pgfsys@transformshift{2.711997in}{0.818752in}%
\pgfsys@useobject{currentmarker}{}%
\end{pgfscope}%
\end{pgfscope}%
\begin{pgfscope}%
\definecolor{textcolor}{rgb}{0.000000,0.000000,0.000000}%
\pgfsetstrokecolor{textcolor}%
\pgfsetfillcolor{textcolor}%
\pgftext[x=0.244444in, y=0.765991in, left, base]{\color{textcolor}\rmfamily\fontsize{10.000000}{12.000000}\selectfont \(\displaystyle {0.7}\)}%
\end{pgfscope}%
\begin{pgfscope}%
\pgfsetbuttcap%
\pgfsetroundjoin%
\definecolor{currentfill}{rgb}{0.000000,0.000000,0.000000}%
\pgfsetfillcolor{currentfill}%
\pgfsetlinewidth{0.501875pt}%
\definecolor{currentstroke}{rgb}{0.000000,0.000000,0.000000}%
\pgfsetstrokecolor{currentstroke}%
\pgfsetdash{}{0pt}%
\pgfsys@defobject{currentmarker}{\pgfqpoint{0.000000in}{0.000000in}}{\pgfqpoint{0.041667in}{0.000000in}}{%
\pgfpathmoveto{\pgfqpoint{0.000000in}{0.000000in}}%
\pgfpathlineto{\pgfqpoint{0.041667in}{0.000000in}}%
\pgfusepath{stroke,fill}%
}%
\begin{pgfscope}%
\pgfsys@transformshift{0.470525in}{1.210143in}%
\pgfsys@useobject{currentmarker}{}%
\end{pgfscope}%
\end{pgfscope}%
\begin{pgfscope}%
\pgfsetbuttcap%
\pgfsetroundjoin%
\definecolor{currentfill}{rgb}{0.000000,0.000000,0.000000}%
\pgfsetfillcolor{currentfill}%
\pgfsetlinewidth{0.501875pt}%
\definecolor{currentstroke}{rgb}{0.000000,0.000000,0.000000}%
\pgfsetstrokecolor{currentstroke}%
\pgfsetdash{}{0pt}%
\pgfsys@defobject{currentmarker}{\pgfqpoint{-0.041667in}{0.000000in}}{\pgfqpoint{-0.000000in}{0.000000in}}{%
\pgfpathmoveto{\pgfqpoint{-0.000000in}{0.000000in}}%
\pgfpathlineto{\pgfqpoint{-0.041667in}{0.000000in}}%
\pgfusepath{stroke,fill}%
}%
\begin{pgfscope}%
\pgfsys@transformshift{2.711997in}{1.210143in}%
\pgfsys@useobject{currentmarker}{}%
\end{pgfscope}%
\end{pgfscope}%
\begin{pgfscope}%
\definecolor{textcolor}{rgb}{0.000000,0.000000,0.000000}%
\pgfsetstrokecolor{textcolor}%
\pgfsetfillcolor{textcolor}%
\pgftext[x=0.244444in, y=1.157381in, left, base]{\color{textcolor}\rmfamily\fontsize{10.000000}{12.000000}\selectfont \(\displaystyle {0.8}\)}%
\end{pgfscope}%
\begin{pgfscope}%
\pgfsetbuttcap%
\pgfsetroundjoin%
\definecolor{currentfill}{rgb}{0.000000,0.000000,0.000000}%
\pgfsetfillcolor{currentfill}%
\pgfsetlinewidth{0.501875pt}%
\definecolor{currentstroke}{rgb}{0.000000,0.000000,0.000000}%
\pgfsetstrokecolor{currentstroke}%
\pgfsetdash{}{0pt}%
\pgfsys@defobject{currentmarker}{\pgfqpoint{0.000000in}{0.000000in}}{\pgfqpoint{0.041667in}{0.000000in}}{%
\pgfpathmoveto{\pgfqpoint{0.000000in}{0.000000in}}%
\pgfpathlineto{\pgfqpoint{0.041667in}{0.000000in}}%
\pgfusepath{stroke,fill}%
}%
\begin{pgfscope}%
\pgfsys@transformshift{0.470525in}{1.601533in}%
\pgfsys@useobject{currentmarker}{}%
\end{pgfscope}%
\end{pgfscope}%
\begin{pgfscope}%
\pgfsetbuttcap%
\pgfsetroundjoin%
\definecolor{currentfill}{rgb}{0.000000,0.000000,0.000000}%
\pgfsetfillcolor{currentfill}%
\pgfsetlinewidth{0.501875pt}%
\definecolor{currentstroke}{rgb}{0.000000,0.000000,0.000000}%
\pgfsetstrokecolor{currentstroke}%
\pgfsetdash{}{0pt}%
\pgfsys@defobject{currentmarker}{\pgfqpoint{-0.041667in}{0.000000in}}{\pgfqpoint{-0.000000in}{0.000000in}}{%
\pgfpathmoveto{\pgfqpoint{-0.000000in}{0.000000in}}%
\pgfpathlineto{\pgfqpoint{-0.041667in}{0.000000in}}%
\pgfusepath{stroke,fill}%
}%
\begin{pgfscope}%
\pgfsys@transformshift{2.711997in}{1.601533in}%
\pgfsys@useobject{currentmarker}{}%
\end{pgfscope}%
\end{pgfscope}%
\begin{pgfscope}%
\definecolor{textcolor}{rgb}{0.000000,0.000000,0.000000}%
\pgfsetstrokecolor{textcolor}%
\pgfsetfillcolor{textcolor}%
\pgftext[x=0.244444in, y=1.548771in, left, base]{\color{textcolor}\rmfamily\fontsize{10.000000}{12.000000}\selectfont \(\displaystyle {0.9}\)}%
\end{pgfscope}%
\begin{pgfscope}%
\pgfsetbuttcap%
\pgfsetroundjoin%
\definecolor{currentfill}{rgb}{0.000000,0.000000,0.000000}%
\pgfsetfillcolor{currentfill}%
\pgfsetlinewidth{0.501875pt}%
\definecolor{currentstroke}{rgb}{0.000000,0.000000,0.000000}%
\pgfsetstrokecolor{currentstroke}%
\pgfsetdash{}{0pt}%
\pgfsys@defobject{currentmarker}{\pgfqpoint{0.000000in}{0.000000in}}{\pgfqpoint{0.041667in}{0.000000in}}{%
\pgfpathmoveto{\pgfqpoint{0.000000in}{0.000000in}}%
\pgfpathlineto{\pgfqpoint{0.041667in}{0.000000in}}%
\pgfusepath{stroke,fill}%
}%
\begin{pgfscope}%
\pgfsys@transformshift{0.470525in}{1.992923in}%
\pgfsys@useobject{currentmarker}{}%
\end{pgfscope}%
\end{pgfscope}%
\begin{pgfscope}%
\pgfsetbuttcap%
\pgfsetroundjoin%
\definecolor{currentfill}{rgb}{0.000000,0.000000,0.000000}%
\pgfsetfillcolor{currentfill}%
\pgfsetlinewidth{0.501875pt}%
\definecolor{currentstroke}{rgb}{0.000000,0.000000,0.000000}%
\pgfsetstrokecolor{currentstroke}%
\pgfsetdash{}{0pt}%
\pgfsys@defobject{currentmarker}{\pgfqpoint{-0.041667in}{0.000000in}}{\pgfqpoint{-0.000000in}{0.000000in}}{%
\pgfpathmoveto{\pgfqpoint{-0.000000in}{0.000000in}}%
\pgfpathlineto{\pgfqpoint{-0.041667in}{0.000000in}}%
\pgfusepath{stroke,fill}%
}%
\begin{pgfscope}%
\pgfsys@transformshift{2.711997in}{1.992923in}%
\pgfsys@useobject{currentmarker}{}%
\end{pgfscope}%
\end{pgfscope}%
\begin{pgfscope}%
\definecolor{textcolor}{rgb}{0.000000,0.000000,0.000000}%
\pgfsetstrokecolor{textcolor}%
\pgfsetfillcolor{textcolor}%
\pgftext[x=0.244444in, y=1.940162in, left, base]{\color{textcolor}\rmfamily\fontsize{10.000000}{12.000000}\selectfont \(\displaystyle {1.0}\)}%
\end{pgfscope}%
\begin{pgfscope}%
\pgfsetbuttcap%
\pgfsetroundjoin%
\definecolor{currentfill}{rgb}{0.000000,0.000000,0.000000}%
\pgfsetfillcolor{currentfill}%
\pgfsetlinewidth{0.501875pt}%
\definecolor{currentstroke}{rgb}{0.000000,0.000000,0.000000}%
\pgfsetstrokecolor{currentstroke}%
\pgfsetdash{}{0pt}%
\pgfsys@defobject{currentmarker}{\pgfqpoint{0.000000in}{0.000000in}}{\pgfqpoint{0.020833in}{0.000000in}}{%
\pgfpathmoveto{\pgfqpoint{0.000000in}{0.000000in}}%
\pgfpathlineto{\pgfqpoint{0.020833in}{0.000000in}}%
\pgfusepath{stroke,fill}%
}%
\begin{pgfscope}%
\pgfsys@transformshift{0.470525in}{0.505640in}%
\pgfsys@useobject{currentmarker}{}%
\end{pgfscope}%
\end{pgfscope}%
\begin{pgfscope}%
\pgfsetbuttcap%
\pgfsetroundjoin%
\definecolor{currentfill}{rgb}{0.000000,0.000000,0.000000}%
\pgfsetfillcolor{currentfill}%
\pgfsetlinewidth{0.501875pt}%
\definecolor{currentstroke}{rgb}{0.000000,0.000000,0.000000}%
\pgfsetstrokecolor{currentstroke}%
\pgfsetdash{}{0pt}%
\pgfsys@defobject{currentmarker}{\pgfqpoint{-0.020833in}{0.000000in}}{\pgfqpoint{-0.000000in}{0.000000in}}{%
\pgfpathmoveto{\pgfqpoint{-0.000000in}{0.000000in}}%
\pgfpathlineto{\pgfqpoint{-0.020833in}{0.000000in}}%
\pgfusepath{stroke,fill}%
}%
\begin{pgfscope}%
\pgfsys@transformshift{2.711997in}{0.505640in}%
\pgfsys@useobject{currentmarker}{}%
\end{pgfscope}%
\end{pgfscope}%
\begin{pgfscope}%
\pgfsetbuttcap%
\pgfsetroundjoin%
\definecolor{currentfill}{rgb}{0.000000,0.000000,0.000000}%
\pgfsetfillcolor{currentfill}%
\pgfsetlinewidth{0.501875pt}%
\definecolor{currentstroke}{rgb}{0.000000,0.000000,0.000000}%
\pgfsetstrokecolor{currentstroke}%
\pgfsetdash{}{0pt}%
\pgfsys@defobject{currentmarker}{\pgfqpoint{0.000000in}{0.000000in}}{\pgfqpoint{0.020833in}{0.000000in}}{%
\pgfpathmoveto{\pgfqpoint{0.000000in}{0.000000in}}%
\pgfpathlineto{\pgfqpoint{0.020833in}{0.000000in}}%
\pgfusepath{stroke,fill}%
}%
\begin{pgfscope}%
\pgfsys@transformshift{0.470525in}{0.583918in}%
\pgfsys@useobject{currentmarker}{}%
\end{pgfscope}%
\end{pgfscope}%
\begin{pgfscope}%
\pgfsetbuttcap%
\pgfsetroundjoin%
\definecolor{currentfill}{rgb}{0.000000,0.000000,0.000000}%
\pgfsetfillcolor{currentfill}%
\pgfsetlinewidth{0.501875pt}%
\definecolor{currentstroke}{rgb}{0.000000,0.000000,0.000000}%
\pgfsetstrokecolor{currentstroke}%
\pgfsetdash{}{0pt}%
\pgfsys@defobject{currentmarker}{\pgfqpoint{-0.020833in}{0.000000in}}{\pgfqpoint{-0.000000in}{0.000000in}}{%
\pgfpathmoveto{\pgfqpoint{-0.000000in}{0.000000in}}%
\pgfpathlineto{\pgfqpoint{-0.020833in}{0.000000in}}%
\pgfusepath{stroke,fill}%
}%
\begin{pgfscope}%
\pgfsys@transformshift{2.711997in}{0.583918in}%
\pgfsys@useobject{currentmarker}{}%
\end{pgfscope}%
\end{pgfscope}%
\begin{pgfscope}%
\pgfsetbuttcap%
\pgfsetroundjoin%
\definecolor{currentfill}{rgb}{0.000000,0.000000,0.000000}%
\pgfsetfillcolor{currentfill}%
\pgfsetlinewidth{0.501875pt}%
\definecolor{currentstroke}{rgb}{0.000000,0.000000,0.000000}%
\pgfsetstrokecolor{currentstroke}%
\pgfsetdash{}{0pt}%
\pgfsys@defobject{currentmarker}{\pgfqpoint{0.000000in}{0.000000in}}{\pgfqpoint{0.020833in}{0.000000in}}{%
\pgfpathmoveto{\pgfqpoint{0.000000in}{0.000000in}}%
\pgfpathlineto{\pgfqpoint{0.020833in}{0.000000in}}%
\pgfusepath{stroke,fill}%
}%
\begin{pgfscope}%
\pgfsys@transformshift{0.470525in}{0.662196in}%
\pgfsys@useobject{currentmarker}{}%
\end{pgfscope}%
\end{pgfscope}%
\begin{pgfscope}%
\pgfsetbuttcap%
\pgfsetroundjoin%
\definecolor{currentfill}{rgb}{0.000000,0.000000,0.000000}%
\pgfsetfillcolor{currentfill}%
\pgfsetlinewidth{0.501875pt}%
\definecolor{currentstroke}{rgb}{0.000000,0.000000,0.000000}%
\pgfsetstrokecolor{currentstroke}%
\pgfsetdash{}{0pt}%
\pgfsys@defobject{currentmarker}{\pgfqpoint{-0.020833in}{0.000000in}}{\pgfqpoint{-0.000000in}{0.000000in}}{%
\pgfpathmoveto{\pgfqpoint{-0.000000in}{0.000000in}}%
\pgfpathlineto{\pgfqpoint{-0.020833in}{0.000000in}}%
\pgfusepath{stroke,fill}%
}%
\begin{pgfscope}%
\pgfsys@transformshift{2.711997in}{0.662196in}%
\pgfsys@useobject{currentmarker}{}%
\end{pgfscope}%
\end{pgfscope}%
\begin{pgfscope}%
\pgfsetbuttcap%
\pgfsetroundjoin%
\definecolor{currentfill}{rgb}{0.000000,0.000000,0.000000}%
\pgfsetfillcolor{currentfill}%
\pgfsetlinewidth{0.501875pt}%
\definecolor{currentstroke}{rgb}{0.000000,0.000000,0.000000}%
\pgfsetstrokecolor{currentstroke}%
\pgfsetdash{}{0pt}%
\pgfsys@defobject{currentmarker}{\pgfqpoint{0.000000in}{0.000000in}}{\pgfqpoint{0.020833in}{0.000000in}}{%
\pgfpathmoveto{\pgfqpoint{0.000000in}{0.000000in}}%
\pgfpathlineto{\pgfqpoint{0.020833in}{0.000000in}}%
\pgfusepath{stroke,fill}%
}%
\begin{pgfscope}%
\pgfsys@transformshift{0.470525in}{0.740474in}%
\pgfsys@useobject{currentmarker}{}%
\end{pgfscope}%
\end{pgfscope}%
\begin{pgfscope}%
\pgfsetbuttcap%
\pgfsetroundjoin%
\definecolor{currentfill}{rgb}{0.000000,0.000000,0.000000}%
\pgfsetfillcolor{currentfill}%
\pgfsetlinewidth{0.501875pt}%
\definecolor{currentstroke}{rgb}{0.000000,0.000000,0.000000}%
\pgfsetstrokecolor{currentstroke}%
\pgfsetdash{}{0pt}%
\pgfsys@defobject{currentmarker}{\pgfqpoint{-0.020833in}{0.000000in}}{\pgfqpoint{-0.000000in}{0.000000in}}{%
\pgfpathmoveto{\pgfqpoint{-0.000000in}{0.000000in}}%
\pgfpathlineto{\pgfqpoint{-0.020833in}{0.000000in}}%
\pgfusepath{stroke,fill}%
}%
\begin{pgfscope}%
\pgfsys@transformshift{2.711997in}{0.740474in}%
\pgfsys@useobject{currentmarker}{}%
\end{pgfscope}%
\end{pgfscope}%
\begin{pgfscope}%
\pgfsetbuttcap%
\pgfsetroundjoin%
\definecolor{currentfill}{rgb}{0.000000,0.000000,0.000000}%
\pgfsetfillcolor{currentfill}%
\pgfsetlinewidth{0.501875pt}%
\definecolor{currentstroke}{rgb}{0.000000,0.000000,0.000000}%
\pgfsetstrokecolor{currentstroke}%
\pgfsetdash{}{0pt}%
\pgfsys@defobject{currentmarker}{\pgfqpoint{0.000000in}{0.000000in}}{\pgfqpoint{0.020833in}{0.000000in}}{%
\pgfpathmoveto{\pgfqpoint{0.000000in}{0.000000in}}%
\pgfpathlineto{\pgfqpoint{0.020833in}{0.000000in}}%
\pgfusepath{stroke,fill}%
}%
\begin{pgfscope}%
\pgfsys@transformshift{0.470525in}{0.897030in}%
\pgfsys@useobject{currentmarker}{}%
\end{pgfscope}%
\end{pgfscope}%
\begin{pgfscope}%
\pgfsetbuttcap%
\pgfsetroundjoin%
\definecolor{currentfill}{rgb}{0.000000,0.000000,0.000000}%
\pgfsetfillcolor{currentfill}%
\pgfsetlinewidth{0.501875pt}%
\definecolor{currentstroke}{rgb}{0.000000,0.000000,0.000000}%
\pgfsetstrokecolor{currentstroke}%
\pgfsetdash{}{0pt}%
\pgfsys@defobject{currentmarker}{\pgfqpoint{-0.020833in}{0.000000in}}{\pgfqpoint{-0.000000in}{0.000000in}}{%
\pgfpathmoveto{\pgfqpoint{-0.000000in}{0.000000in}}%
\pgfpathlineto{\pgfqpoint{-0.020833in}{0.000000in}}%
\pgfusepath{stroke,fill}%
}%
\begin{pgfscope}%
\pgfsys@transformshift{2.711997in}{0.897030in}%
\pgfsys@useobject{currentmarker}{}%
\end{pgfscope}%
\end{pgfscope}%
\begin{pgfscope}%
\pgfsetbuttcap%
\pgfsetroundjoin%
\definecolor{currentfill}{rgb}{0.000000,0.000000,0.000000}%
\pgfsetfillcolor{currentfill}%
\pgfsetlinewidth{0.501875pt}%
\definecolor{currentstroke}{rgb}{0.000000,0.000000,0.000000}%
\pgfsetstrokecolor{currentstroke}%
\pgfsetdash{}{0pt}%
\pgfsys@defobject{currentmarker}{\pgfqpoint{0.000000in}{0.000000in}}{\pgfqpoint{0.020833in}{0.000000in}}{%
\pgfpathmoveto{\pgfqpoint{0.000000in}{0.000000in}}%
\pgfpathlineto{\pgfqpoint{0.020833in}{0.000000in}}%
\pgfusepath{stroke,fill}%
}%
\begin{pgfscope}%
\pgfsys@transformshift{0.470525in}{0.975309in}%
\pgfsys@useobject{currentmarker}{}%
\end{pgfscope}%
\end{pgfscope}%
\begin{pgfscope}%
\pgfsetbuttcap%
\pgfsetroundjoin%
\definecolor{currentfill}{rgb}{0.000000,0.000000,0.000000}%
\pgfsetfillcolor{currentfill}%
\pgfsetlinewidth{0.501875pt}%
\definecolor{currentstroke}{rgb}{0.000000,0.000000,0.000000}%
\pgfsetstrokecolor{currentstroke}%
\pgfsetdash{}{0pt}%
\pgfsys@defobject{currentmarker}{\pgfqpoint{-0.020833in}{0.000000in}}{\pgfqpoint{-0.000000in}{0.000000in}}{%
\pgfpathmoveto{\pgfqpoint{-0.000000in}{0.000000in}}%
\pgfpathlineto{\pgfqpoint{-0.020833in}{0.000000in}}%
\pgfusepath{stroke,fill}%
}%
\begin{pgfscope}%
\pgfsys@transformshift{2.711997in}{0.975309in}%
\pgfsys@useobject{currentmarker}{}%
\end{pgfscope}%
\end{pgfscope}%
\begin{pgfscope}%
\pgfsetbuttcap%
\pgfsetroundjoin%
\definecolor{currentfill}{rgb}{0.000000,0.000000,0.000000}%
\pgfsetfillcolor{currentfill}%
\pgfsetlinewidth{0.501875pt}%
\definecolor{currentstroke}{rgb}{0.000000,0.000000,0.000000}%
\pgfsetstrokecolor{currentstroke}%
\pgfsetdash{}{0pt}%
\pgfsys@defobject{currentmarker}{\pgfqpoint{0.000000in}{0.000000in}}{\pgfqpoint{0.020833in}{0.000000in}}{%
\pgfpathmoveto{\pgfqpoint{0.000000in}{0.000000in}}%
\pgfpathlineto{\pgfqpoint{0.020833in}{0.000000in}}%
\pgfusepath{stroke,fill}%
}%
\begin{pgfscope}%
\pgfsys@transformshift{0.470525in}{1.053587in}%
\pgfsys@useobject{currentmarker}{}%
\end{pgfscope}%
\end{pgfscope}%
\begin{pgfscope}%
\pgfsetbuttcap%
\pgfsetroundjoin%
\definecolor{currentfill}{rgb}{0.000000,0.000000,0.000000}%
\pgfsetfillcolor{currentfill}%
\pgfsetlinewidth{0.501875pt}%
\definecolor{currentstroke}{rgb}{0.000000,0.000000,0.000000}%
\pgfsetstrokecolor{currentstroke}%
\pgfsetdash{}{0pt}%
\pgfsys@defobject{currentmarker}{\pgfqpoint{-0.020833in}{0.000000in}}{\pgfqpoint{-0.000000in}{0.000000in}}{%
\pgfpathmoveto{\pgfqpoint{-0.000000in}{0.000000in}}%
\pgfpathlineto{\pgfqpoint{-0.020833in}{0.000000in}}%
\pgfusepath{stroke,fill}%
}%
\begin{pgfscope}%
\pgfsys@transformshift{2.711997in}{1.053587in}%
\pgfsys@useobject{currentmarker}{}%
\end{pgfscope}%
\end{pgfscope}%
\begin{pgfscope}%
\pgfsetbuttcap%
\pgfsetroundjoin%
\definecolor{currentfill}{rgb}{0.000000,0.000000,0.000000}%
\pgfsetfillcolor{currentfill}%
\pgfsetlinewidth{0.501875pt}%
\definecolor{currentstroke}{rgb}{0.000000,0.000000,0.000000}%
\pgfsetstrokecolor{currentstroke}%
\pgfsetdash{}{0pt}%
\pgfsys@defobject{currentmarker}{\pgfqpoint{0.000000in}{0.000000in}}{\pgfqpoint{0.020833in}{0.000000in}}{%
\pgfpathmoveto{\pgfqpoint{0.000000in}{0.000000in}}%
\pgfpathlineto{\pgfqpoint{0.020833in}{0.000000in}}%
\pgfusepath{stroke,fill}%
}%
\begin{pgfscope}%
\pgfsys@transformshift{0.470525in}{1.131865in}%
\pgfsys@useobject{currentmarker}{}%
\end{pgfscope}%
\end{pgfscope}%
\begin{pgfscope}%
\pgfsetbuttcap%
\pgfsetroundjoin%
\definecolor{currentfill}{rgb}{0.000000,0.000000,0.000000}%
\pgfsetfillcolor{currentfill}%
\pgfsetlinewidth{0.501875pt}%
\definecolor{currentstroke}{rgb}{0.000000,0.000000,0.000000}%
\pgfsetstrokecolor{currentstroke}%
\pgfsetdash{}{0pt}%
\pgfsys@defobject{currentmarker}{\pgfqpoint{-0.020833in}{0.000000in}}{\pgfqpoint{-0.000000in}{0.000000in}}{%
\pgfpathmoveto{\pgfqpoint{-0.000000in}{0.000000in}}%
\pgfpathlineto{\pgfqpoint{-0.020833in}{0.000000in}}%
\pgfusepath{stroke,fill}%
}%
\begin{pgfscope}%
\pgfsys@transformshift{2.711997in}{1.131865in}%
\pgfsys@useobject{currentmarker}{}%
\end{pgfscope}%
\end{pgfscope}%
\begin{pgfscope}%
\pgfsetbuttcap%
\pgfsetroundjoin%
\definecolor{currentfill}{rgb}{0.000000,0.000000,0.000000}%
\pgfsetfillcolor{currentfill}%
\pgfsetlinewidth{0.501875pt}%
\definecolor{currentstroke}{rgb}{0.000000,0.000000,0.000000}%
\pgfsetstrokecolor{currentstroke}%
\pgfsetdash{}{0pt}%
\pgfsys@defobject{currentmarker}{\pgfqpoint{0.000000in}{0.000000in}}{\pgfqpoint{0.020833in}{0.000000in}}{%
\pgfpathmoveto{\pgfqpoint{0.000000in}{0.000000in}}%
\pgfpathlineto{\pgfqpoint{0.020833in}{0.000000in}}%
\pgfusepath{stroke,fill}%
}%
\begin{pgfscope}%
\pgfsys@transformshift{0.470525in}{1.288421in}%
\pgfsys@useobject{currentmarker}{}%
\end{pgfscope}%
\end{pgfscope}%
\begin{pgfscope}%
\pgfsetbuttcap%
\pgfsetroundjoin%
\definecolor{currentfill}{rgb}{0.000000,0.000000,0.000000}%
\pgfsetfillcolor{currentfill}%
\pgfsetlinewidth{0.501875pt}%
\definecolor{currentstroke}{rgb}{0.000000,0.000000,0.000000}%
\pgfsetstrokecolor{currentstroke}%
\pgfsetdash{}{0pt}%
\pgfsys@defobject{currentmarker}{\pgfqpoint{-0.020833in}{0.000000in}}{\pgfqpoint{-0.000000in}{0.000000in}}{%
\pgfpathmoveto{\pgfqpoint{-0.000000in}{0.000000in}}%
\pgfpathlineto{\pgfqpoint{-0.020833in}{0.000000in}}%
\pgfusepath{stroke,fill}%
}%
\begin{pgfscope}%
\pgfsys@transformshift{2.711997in}{1.288421in}%
\pgfsys@useobject{currentmarker}{}%
\end{pgfscope}%
\end{pgfscope}%
\begin{pgfscope}%
\pgfsetbuttcap%
\pgfsetroundjoin%
\definecolor{currentfill}{rgb}{0.000000,0.000000,0.000000}%
\pgfsetfillcolor{currentfill}%
\pgfsetlinewidth{0.501875pt}%
\definecolor{currentstroke}{rgb}{0.000000,0.000000,0.000000}%
\pgfsetstrokecolor{currentstroke}%
\pgfsetdash{}{0pt}%
\pgfsys@defobject{currentmarker}{\pgfqpoint{0.000000in}{0.000000in}}{\pgfqpoint{0.020833in}{0.000000in}}{%
\pgfpathmoveto{\pgfqpoint{0.000000in}{0.000000in}}%
\pgfpathlineto{\pgfqpoint{0.020833in}{0.000000in}}%
\pgfusepath{stroke,fill}%
}%
\begin{pgfscope}%
\pgfsys@transformshift{0.470525in}{1.366699in}%
\pgfsys@useobject{currentmarker}{}%
\end{pgfscope}%
\end{pgfscope}%
\begin{pgfscope}%
\pgfsetbuttcap%
\pgfsetroundjoin%
\definecolor{currentfill}{rgb}{0.000000,0.000000,0.000000}%
\pgfsetfillcolor{currentfill}%
\pgfsetlinewidth{0.501875pt}%
\definecolor{currentstroke}{rgb}{0.000000,0.000000,0.000000}%
\pgfsetstrokecolor{currentstroke}%
\pgfsetdash{}{0pt}%
\pgfsys@defobject{currentmarker}{\pgfqpoint{-0.020833in}{0.000000in}}{\pgfqpoint{-0.000000in}{0.000000in}}{%
\pgfpathmoveto{\pgfqpoint{-0.000000in}{0.000000in}}%
\pgfpathlineto{\pgfqpoint{-0.020833in}{0.000000in}}%
\pgfusepath{stroke,fill}%
}%
\begin{pgfscope}%
\pgfsys@transformshift{2.711997in}{1.366699in}%
\pgfsys@useobject{currentmarker}{}%
\end{pgfscope}%
\end{pgfscope}%
\begin{pgfscope}%
\pgfsetbuttcap%
\pgfsetroundjoin%
\definecolor{currentfill}{rgb}{0.000000,0.000000,0.000000}%
\pgfsetfillcolor{currentfill}%
\pgfsetlinewidth{0.501875pt}%
\definecolor{currentstroke}{rgb}{0.000000,0.000000,0.000000}%
\pgfsetstrokecolor{currentstroke}%
\pgfsetdash{}{0pt}%
\pgfsys@defobject{currentmarker}{\pgfqpoint{0.000000in}{0.000000in}}{\pgfqpoint{0.020833in}{0.000000in}}{%
\pgfpathmoveto{\pgfqpoint{0.000000in}{0.000000in}}%
\pgfpathlineto{\pgfqpoint{0.020833in}{0.000000in}}%
\pgfusepath{stroke,fill}%
}%
\begin{pgfscope}%
\pgfsys@transformshift{0.470525in}{1.444977in}%
\pgfsys@useobject{currentmarker}{}%
\end{pgfscope}%
\end{pgfscope}%
\begin{pgfscope}%
\pgfsetbuttcap%
\pgfsetroundjoin%
\definecolor{currentfill}{rgb}{0.000000,0.000000,0.000000}%
\pgfsetfillcolor{currentfill}%
\pgfsetlinewidth{0.501875pt}%
\definecolor{currentstroke}{rgb}{0.000000,0.000000,0.000000}%
\pgfsetstrokecolor{currentstroke}%
\pgfsetdash{}{0pt}%
\pgfsys@defobject{currentmarker}{\pgfqpoint{-0.020833in}{0.000000in}}{\pgfqpoint{-0.000000in}{0.000000in}}{%
\pgfpathmoveto{\pgfqpoint{-0.000000in}{0.000000in}}%
\pgfpathlineto{\pgfqpoint{-0.020833in}{0.000000in}}%
\pgfusepath{stroke,fill}%
}%
\begin{pgfscope}%
\pgfsys@transformshift{2.711997in}{1.444977in}%
\pgfsys@useobject{currentmarker}{}%
\end{pgfscope}%
\end{pgfscope}%
\begin{pgfscope}%
\pgfsetbuttcap%
\pgfsetroundjoin%
\definecolor{currentfill}{rgb}{0.000000,0.000000,0.000000}%
\pgfsetfillcolor{currentfill}%
\pgfsetlinewidth{0.501875pt}%
\definecolor{currentstroke}{rgb}{0.000000,0.000000,0.000000}%
\pgfsetstrokecolor{currentstroke}%
\pgfsetdash{}{0pt}%
\pgfsys@defobject{currentmarker}{\pgfqpoint{0.000000in}{0.000000in}}{\pgfqpoint{0.020833in}{0.000000in}}{%
\pgfpathmoveto{\pgfqpoint{0.000000in}{0.000000in}}%
\pgfpathlineto{\pgfqpoint{0.020833in}{0.000000in}}%
\pgfusepath{stroke,fill}%
}%
\begin{pgfscope}%
\pgfsys@transformshift{0.470525in}{1.523255in}%
\pgfsys@useobject{currentmarker}{}%
\end{pgfscope}%
\end{pgfscope}%
\begin{pgfscope}%
\pgfsetbuttcap%
\pgfsetroundjoin%
\definecolor{currentfill}{rgb}{0.000000,0.000000,0.000000}%
\pgfsetfillcolor{currentfill}%
\pgfsetlinewidth{0.501875pt}%
\definecolor{currentstroke}{rgb}{0.000000,0.000000,0.000000}%
\pgfsetstrokecolor{currentstroke}%
\pgfsetdash{}{0pt}%
\pgfsys@defobject{currentmarker}{\pgfqpoint{-0.020833in}{0.000000in}}{\pgfqpoint{-0.000000in}{0.000000in}}{%
\pgfpathmoveto{\pgfqpoint{-0.000000in}{0.000000in}}%
\pgfpathlineto{\pgfqpoint{-0.020833in}{0.000000in}}%
\pgfusepath{stroke,fill}%
}%
\begin{pgfscope}%
\pgfsys@transformshift{2.711997in}{1.523255in}%
\pgfsys@useobject{currentmarker}{}%
\end{pgfscope}%
\end{pgfscope}%
\begin{pgfscope}%
\pgfsetbuttcap%
\pgfsetroundjoin%
\definecolor{currentfill}{rgb}{0.000000,0.000000,0.000000}%
\pgfsetfillcolor{currentfill}%
\pgfsetlinewidth{0.501875pt}%
\definecolor{currentstroke}{rgb}{0.000000,0.000000,0.000000}%
\pgfsetstrokecolor{currentstroke}%
\pgfsetdash{}{0pt}%
\pgfsys@defobject{currentmarker}{\pgfqpoint{0.000000in}{0.000000in}}{\pgfqpoint{0.020833in}{0.000000in}}{%
\pgfpathmoveto{\pgfqpoint{0.000000in}{0.000000in}}%
\pgfpathlineto{\pgfqpoint{0.020833in}{0.000000in}}%
\pgfusepath{stroke,fill}%
}%
\begin{pgfscope}%
\pgfsys@transformshift{0.470525in}{1.679811in}%
\pgfsys@useobject{currentmarker}{}%
\end{pgfscope}%
\end{pgfscope}%
\begin{pgfscope}%
\pgfsetbuttcap%
\pgfsetroundjoin%
\definecolor{currentfill}{rgb}{0.000000,0.000000,0.000000}%
\pgfsetfillcolor{currentfill}%
\pgfsetlinewidth{0.501875pt}%
\definecolor{currentstroke}{rgb}{0.000000,0.000000,0.000000}%
\pgfsetstrokecolor{currentstroke}%
\pgfsetdash{}{0pt}%
\pgfsys@defobject{currentmarker}{\pgfqpoint{-0.020833in}{0.000000in}}{\pgfqpoint{-0.000000in}{0.000000in}}{%
\pgfpathmoveto{\pgfqpoint{-0.000000in}{0.000000in}}%
\pgfpathlineto{\pgfqpoint{-0.020833in}{0.000000in}}%
\pgfusepath{stroke,fill}%
}%
\begin{pgfscope}%
\pgfsys@transformshift{2.711997in}{1.679811in}%
\pgfsys@useobject{currentmarker}{}%
\end{pgfscope}%
\end{pgfscope}%
\begin{pgfscope}%
\pgfsetbuttcap%
\pgfsetroundjoin%
\definecolor{currentfill}{rgb}{0.000000,0.000000,0.000000}%
\pgfsetfillcolor{currentfill}%
\pgfsetlinewidth{0.501875pt}%
\definecolor{currentstroke}{rgb}{0.000000,0.000000,0.000000}%
\pgfsetstrokecolor{currentstroke}%
\pgfsetdash{}{0pt}%
\pgfsys@defobject{currentmarker}{\pgfqpoint{0.000000in}{0.000000in}}{\pgfqpoint{0.020833in}{0.000000in}}{%
\pgfpathmoveto{\pgfqpoint{0.000000in}{0.000000in}}%
\pgfpathlineto{\pgfqpoint{0.020833in}{0.000000in}}%
\pgfusepath{stroke,fill}%
}%
\begin{pgfscope}%
\pgfsys@transformshift{0.470525in}{1.758089in}%
\pgfsys@useobject{currentmarker}{}%
\end{pgfscope}%
\end{pgfscope}%
\begin{pgfscope}%
\pgfsetbuttcap%
\pgfsetroundjoin%
\definecolor{currentfill}{rgb}{0.000000,0.000000,0.000000}%
\pgfsetfillcolor{currentfill}%
\pgfsetlinewidth{0.501875pt}%
\definecolor{currentstroke}{rgb}{0.000000,0.000000,0.000000}%
\pgfsetstrokecolor{currentstroke}%
\pgfsetdash{}{0pt}%
\pgfsys@defobject{currentmarker}{\pgfqpoint{-0.020833in}{0.000000in}}{\pgfqpoint{-0.000000in}{0.000000in}}{%
\pgfpathmoveto{\pgfqpoint{-0.000000in}{0.000000in}}%
\pgfpathlineto{\pgfqpoint{-0.020833in}{0.000000in}}%
\pgfusepath{stroke,fill}%
}%
\begin{pgfscope}%
\pgfsys@transformshift{2.711997in}{1.758089in}%
\pgfsys@useobject{currentmarker}{}%
\end{pgfscope}%
\end{pgfscope}%
\begin{pgfscope}%
\pgfsetbuttcap%
\pgfsetroundjoin%
\definecolor{currentfill}{rgb}{0.000000,0.000000,0.000000}%
\pgfsetfillcolor{currentfill}%
\pgfsetlinewidth{0.501875pt}%
\definecolor{currentstroke}{rgb}{0.000000,0.000000,0.000000}%
\pgfsetstrokecolor{currentstroke}%
\pgfsetdash{}{0pt}%
\pgfsys@defobject{currentmarker}{\pgfqpoint{0.000000in}{0.000000in}}{\pgfqpoint{0.020833in}{0.000000in}}{%
\pgfpathmoveto{\pgfqpoint{0.000000in}{0.000000in}}%
\pgfpathlineto{\pgfqpoint{0.020833in}{0.000000in}}%
\pgfusepath{stroke,fill}%
}%
\begin{pgfscope}%
\pgfsys@transformshift{0.470525in}{1.836367in}%
\pgfsys@useobject{currentmarker}{}%
\end{pgfscope}%
\end{pgfscope}%
\begin{pgfscope}%
\pgfsetbuttcap%
\pgfsetroundjoin%
\definecolor{currentfill}{rgb}{0.000000,0.000000,0.000000}%
\pgfsetfillcolor{currentfill}%
\pgfsetlinewidth{0.501875pt}%
\definecolor{currentstroke}{rgb}{0.000000,0.000000,0.000000}%
\pgfsetstrokecolor{currentstroke}%
\pgfsetdash{}{0pt}%
\pgfsys@defobject{currentmarker}{\pgfqpoint{-0.020833in}{0.000000in}}{\pgfqpoint{-0.000000in}{0.000000in}}{%
\pgfpathmoveto{\pgfqpoint{-0.000000in}{0.000000in}}%
\pgfpathlineto{\pgfqpoint{-0.020833in}{0.000000in}}%
\pgfusepath{stroke,fill}%
}%
\begin{pgfscope}%
\pgfsys@transformshift{2.711997in}{1.836367in}%
\pgfsys@useobject{currentmarker}{}%
\end{pgfscope}%
\end{pgfscope}%
\begin{pgfscope}%
\pgfsetbuttcap%
\pgfsetroundjoin%
\definecolor{currentfill}{rgb}{0.000000,0.000000,0.000000}%
\pgfsetfillcolor{currentfill}%
\pgfsetlinewidth{0.501875pt}%
\definecolor{currentstroke}{rgb}{0.000000,0.000000,0.000000}%
\pgfsetstrokecolor{currentstroke}%
\pgfsetdash{}{0pt}%
\pgfsys@defobject{currentmarker}{\pgfqpoint{0.000000in}{0.000000in}}{\pgfqpoint{0.020833in}{0.000000in}}{%
\pgfpathmoveto{\pgfqpoint{0.000000in}{0.000000in}}%
\pgfpathlineto{\pgfqpoint{0.020833in}{0.000000in}}%
\pgfusepath{stroke,fill}%
}%
\begin{pgfscope}%
\pgfsys@transformshift{0.470525in}{1.914645in}%
\pgfsys@useobject{currentmarker}{}%
\end{pgfscope}%
\end{pgfscope}%
\begin{pgfscope}%
\pgfsetbuttcap%
\pgfsetroundjoin%
\definecolor{currentfill}{rgb}{0.000000,0.000000,0.000000}%
\pgfsetfillcolor{currentfill}%
\pgfsetlinewidth{0.501875pt}%
\definecolor{currentstroke}{rgb}{0.000000,0.000000,0.000000}%
\pgfsetstrokecolor{currentstroke}%
\pgfsetdash{}{0pt}%
\pgfsys@defobject{currentmarker}{\pgfqpoint{-0.020833in}{0.000000in}}{\pgfqpoint{-0.000000in}{0.000000in}}{%
\pgfpathmoveto{\pgfqpoint{-0.000000in}{0.000000in}}%
\pgfpathlineto{\pgfqpoint{-0.020833in}{0.000000in}}%
\pgfusepath{stroke,fill}%
}%
\begin{pgfscope}%
\pgfsys@transformshift{2.711997in}{1.914645in}%
\pgfsys@useobject{currentmarker}{}%
\end{pgfscope}%
\end{pgfscope}%
\begin{pgfscope}%
\definecolor{textcolor}{rgb}{0.000000,0.000000,0.000000}%
\pgfsetstrokecolor{textcolor}%
\pgfsetfillcolor{textcolor}%
\pgftext[x=0.188889in,y=1.236093in,,bottom,rotate=90.000000]{\color{textcolor}\rmfamily\fontsize{10.000000}{12.000000}\selectfont \(\displaystyle C(K)\)}%
\end{pgfscope}%
\begin{pgfscope}%
\pgfpathrectangle{\pgfqpoint{0.470525in}{0.422992in}}{\pgfqpoint{2.241471in}{1.626201in}}%
\pgfusepath{clip}%
\pgfsetrectcap%
\pgfsetroundjoin%
\pgfsetlinewidth{1.003750pt}%
\definecolor{currentstroke}{rgb}{0.047059,0.364706,0.647059}%
\pgfsetstrokecolor{currentstroke}%
\pgfsetdash{}{0pt}%
\pgfpathmoveto{\pgfqpoint{0.492718in}{1.975275in}}%
\pgfpathlineto{\pgfqpoint{0.514911in}{1.971175in}}%
\pgfpathlineto{\pgfqpoint{0.537104in}{1.968276in}}%
\pgfpathlineto{\pgfqpoint{0.559296in}{1.965850in}}%
\pgfpathlineto{\pgfqpoint{0.581489in}{1.963935in}}%
\pgfpathlineto{\pgfqpoint{0.603682in}{1.961869in}}%
\pgfpathlineto{\pgfqpoint{0.625875in}{1.960357in}}%
\pgfpathlineto{\pgfqpoint{0.648068in}{1.958958in}}%
\pgfpathlineto{\pgfqpoint{0.670260in}{1.957769in}}%
\pgfpathlineto{\pgfqpoint{0.692453in}{1.956434in}}%
\pgfpathlineto{\pgfqpoint{0.714646in}{1.955260in}}%
\pgfpathlineto{\pgfqpoint{0.736839in}{1.954218in}}%
\pgfpathlineto{\pgfqpoint{0.759032in}{1.953203in}}%
\pgfpathlineto{\pgfqpoint{0.781224in}{1.952388in}}%
\pgfpathlineto{\pgfqpoint{0.803417in}{1.951401in}}%
\pgfpathlineto{\pgfqpoint{0.825610in}{1.950591in}}%
\pgfpathlineto{\pgfqpoint{0.847803in}{1.949838in}}%
\pgfpathlineto{\pgfqpoint{0.869995in}{1.949005in}}%
\pgfpathlineto{\pgfqpoint{0.892188in}{1.948194in}}%
\pgfpathlineto{\pgfqpoint{0.914381in}{1.947548in}}%
\pgfpathlineto{\pgfqpoint{0.936574in}{1.946835in}}%
\pgfpathlineto{\pgfqpoint{0.958767in}{1.946119in}}%
\pgfpathlineto{\pgfqpoint{0.980959in}{1.945500in}}%
\pgfpathlineto{\pgfqpoint{1.003152in}{1.944912in}}%
\pgfpathlineto{\pgfqpoint{1.025345in}{1.944359in}}%
\pgfpathlineto{\pgfqpoint{1.047538in}{1.943799in}}%
\pgfpathlineto{\pgfqpoint{1.069731in}{1.943259in}}%
\pgfpathlineto{\pgfqpoint{1.091923in}{1.942708in}}%
\pgfpathlineto{\pgfqpoint{1.114116in}{1.942131in}}%
\pgfpathlineto{\pgfqpoint{1.136309in}{1.941616in}}%
\pgfpathlineto{\pgfqpoint{1.158502in}{1.941144in}}%
\pgfpathlineto{\pgfqpoint{1.180694in}{1.940582in}}%
\pgfpathlineto{\pgfqpoint{1.202887in}{1.940021in}}%
\pgfpathlineto{\pgfqpoint{1.225080in}{1.939563in}}%
\pgfpathlineto{\pgfqpoint{1.247273in}{1.939142in}}%
\pgfpathlineto{\pgfqpoint{1.269466in}{1.938691in}}%
\pgfpathlineto{\pgfqpoint{1.291658in}{1.938264in}}%
\pgfpathlineto{\pgfqpoint{1.313851in}{1.937803in}}%
\pgfpathlineto{\pgfqpoint{1.336044in}{1.937343in}}%
\pgfpathlineto{\pgfqpoint{1.358237in}{1.936943in}}%
\pgfpathlineto{\pgfqpoint{1.380430in}{1.936576in}}%
\pgfpathlineto{\pgfqpoint{1.402622in}{1.936157in}}%
\pgfpathlineto{\pgfqpoint{1.424815in}{1.935785in}}%
\pgfpathlineto{\pgfqpoint{1.447008in}{1.935412in}}%
\pgfpathlineto{\pgfqpoint{1.469201in}{1.935024in}}%
\pgfpathlineto{\pgfqpoint{1.491393in}{1.934644in}}%
\pgfpathlineto{\pgfqpoint{1.513586in}{1.934316in}}%
\pgfpathlineto{\pgfqpoint{1.535779in}{1.934011in}}%
\pgfpathlineto{\pgfqpoint{1.557972in}{1.933638in}}%
\pgfpathlineto{\pgfqpoint{1.580165in}{1.933342in}}%
\pgfpathlineto{\pgfqpoint{1.602357in}{1.932940in}}%
\pgfpathlineto{\pgfqpoint{1.624550in}{1.932608in}}%
\pgfpathlineto{\pgfqpoint{1.646743in}{1.932269in}}%
\pgfpathlineto{\pgfqpoint{1.668936in}{1.931942in}}%
\pgfpathlineto{\pgfqpoint{1.691129in}{1.931585in}}%
\pgfpathlineto{\pgfqpoint{1.713321in}{1.931231in}}%
\pgfpathlineto{\pgfqpoint{1.735514in}{1.930903in}}%
\pgfpathlineto{\pgfqpoint{1.757707in}{1.930519in}}%
\pgfpathlineto{\pgfqpoint{1.779900in}{1.930245in}}%
\pgfpathlineto{\pgfqpoint{1.802092in}{1.929890in}}%
\pgfpathlineto{\pgfqpoint{1.824285in}{1.929579in}}%
\pgfpathlineto{\pgfqpoint{1.846478in}{1.929272in}}%
\pgfpathlineto{\pgfqpoint{1.868671in}{1.928993in}}%
\pgfpathlineto{\pgfqpoint{1.890864in}{1.928689in}}%
\pgfpathlineto{\pgfqpoint{1.913056in}{1.928381in}}%
\pgfpathlineto{\pgfqpoint{1.935249in}{1.928058in}}%
\pgfpathlineto{\pgfqpoint{1.957442in}{1.927797in}}%
\pgfpathlineto{\pgfqpoint{1.979635in}{1.927546in}}%
\pgfpathlineto{\pgfqpoint{2.001828in}{1.927278in}}%
\pgfpathlineto{\pgfqpoint{2.024020in}{1.927033in}}%
\pgfpathlineto{\pgfqpoint{2.046213in}{1.926756in}}%
\pgfpathlineto{\pgfqpoint{2.068406in}{1.926511in}}%
\pgfpathlineto{\pgfqpoint{2.090599in}{1.926300in}}%
\pgfpathlineto{\pgfqpoint{2.112792in}{1.926041in}}%
\pgfpathlineto{\pgfqpoint{2.134984in}{1.925753in}}%
\pgfpathlineto{\pgfqpoint{2.157177in}{1.925508in}}%
\pgfpathlineto{\pgfqpoint{2.179370in}{1.925306in}}%
\pgfpathlineto{\pgfqpoint{2.201563in}{1.925076in}}%
\pgfpathlineto{\pgfqpoint{2.223755in}{1.924852in}}%
\pgfpathlineto{\pgfqpoint{2.245948in}{1.924694in}}%
\pgfpathlineto{\pgfqpoint{2.268141in}{1.924467in}}%
\pgfpathlineto{\pgfqpoint{2.290334in}{1.924231in}}%
\pgfpathlineto{\pgfqpoint{2.312527in}{1.924030in}}%
\pgfpathlineto{\pgfqpoint{2.334719in}{1.923823in}}%
\pgfpathlineto{\pgfqpoint{2.356912in}{1.923603in}}%
\pgfpathlineto{\pgfqpoint{2.379105in}{1.923377in}}%
\pgfpathlineto{\pgfqpoint{2.401298in}{1.923141in}}%
\pgfpathlineto{\pgfqpoint{2.423491in}{1.922919in}}%
\pgfpathlineto{\pgfqpoint{2.445683in}{1.922707in}}%
\pgfpathlineto{\pgfqpoint{2.467876in}{1.922510in}}%
\pgfpathlineto{\pgfqpoint{2.490069in}{1.922321in}}%
\pgfpathlineto{\pgfqpoint{2.512262in}{1.922140in}}%
\pgfpathlineto{\pgfqpoint{2.534454in}{1.921960in}}%
\pgfpathlineto{\pgfqpoint{2.556647in}{1.921745in}}%
\pgfpathlineto{\pgfqpoint{2.578840in}{1.921542in}}%
\pgfpathlineto{\pgfqpoint{2.601033in}{1.921341in}}%
\pgfpathlineto{\pgfqpoint{2.623226in}{1.921122in}}%
\pgfpathlineto{\pgfqpoint{2.645418in}{1.920912in}}%
\pgfpathlineto{\pgfqpoint{2.667611in}{1.920757in}}%
\pgfusepath{stroke}%
\end{pgfscope}%
\begin{pgfscope}%
\pgfpathrectangle{\pgfqpoint{0.470525in}{0.422992in}}{\pgfqpoint{2.241471in}{1.626201in}}%
\pgfusepath{clip}%
\pgfsetrectcap%
\pgfsetroundjoin%
\pgfsetlinewidth{1.003750pt}%
\definecolor{currentstroke}{rgb}{0.000000,0.725490,0.270588}%
\pgfsetstrokecolor{currentstroke}%
\pgfsetdash{}{0pt}%
\pgfpathmoveto{\pgfqpoint{0.492718in}{1.838130in}}%
\pgfpathlineto{\pgfqpoint{0.514911in}{1.789683in}}%
\pgfpathlineto{\pgfqpoint{0.537104in}{1.759728in}}%
\pgfpathlineto{\pgfqpoint{0.559296in}{1.738809in}}%
\pgfpathlineto{\pgfqpoint{0.581489in}{1.719150in}}%
\pgfpathlineto{\pgfqpoint{0.603682in}{1.705994in}}%
\pgfpathlineto{\pgfqpoint{0.625875in}{1.696661in}}%
\pgfpathlineto{\pgfqpoint{0.648068in}{1.687068in}}%
\pgfpathlineto{\pgfqpoint{0.670260in}{1.677462in}}%
\pgfpathlineto{\pgfqpoint{0.692453in}{1.669826in}}%
\pgfpathlineto{\pgfqpoint{0.714646in}{1.664123in}}%
\pgfpathlineto{\pgfqpoint{0.736839in}{1.657570in}}%
\pgfpathlineto{\pgfqpoint{0.759032in}{1.652417in}}%
\pgfpathlineto{\pgfqpoint{0.781224in}{1.648391in}}%
\pgfpathlineto{\pgfqpoint{0.803417in}{1.644247in}}%
\pgfpathlineto{\pgfqpoint{0.825610in}{1.639466in}}%
\pgfpathlineto{\pgfqpoint{0.847803in}{1.636096in}}%
\pgfpathlineto{\pgfqpoint{0.869995in}{1.632285in}}%
\pgfpathlineto{\pgfqpoint{0.892188in}{1.628506in}}%
\pgfpathlineto{\pgfqpoint{0.914381in}{1.624754in}}%
\pgfpathlineto{\pgfqpoint{0.936574in}{1.621063in}}%
\pgfpathlineto{\pgfqpoint{0.958767in}{1.618122in}}%
\pgfpathlineto{\pgfqpoint{0.980959in}{1.615309in}}%
\pgfpathlineto{\pgfqpoint{1.003152in}{1.612443in}}%
\pgfpathlineto{\pgfqpoint{1.025345in}{1.609562in}}%
\pgfpathlineto{\pgfqpoint{1.047538in}{1.607036in}}%
\pgfpathlineto{\pgfqpoint{1.069731in}{1.604551in}}%
\pgfpathlineto{\pgfqpoint{1.091923in}{1.602451in}}%
\pgfpathlineto{\pgfqpoint{1.114116in}{1.600218in}}%
\pgfpathlineto{\pgfqpoint{1.136309in}{1.598043in}}%
\pgfpathlineto{\pgfqpoint{1.158502in}{1.596383in}}%
\pgfpathlineto{\pgfqpoint{1.180694in}{1.594610in}}%
\pgfpathlineto{\pgfqpoint{1.202887in}{1.592777in}}%
\pgfpathlineto{\pgfqpoint{1.225080in}{1.591114in}}%
\pgfpathlineto{\pgfqpoint{1.247273in}{1.589291in}}%
\pgfpathlineto{\pgfqpoint{1.269466in}{1.587385in}}%
\pgfpathlineto{\pgfqpoint{1.291658in}{1.585497in}}%
\pgfpathlineto{\pgfqpoint{1.313851in}{1.583320in}}%
\pgfpathlineto{\pgfqpoint{1.336044in}{1.581261in}}%
\pgfpathlineto{\pgfqpoint{1.358237in}{1.579770in}}%
\pgfpathlineto{\pgfqpoint{1.380430in}{1.578095in}}%
\pgfpathlineto{\pgfqpoint{1.402622in}{1.576615in}}%
\pgfpathlineto{\pgfqpoint{1.424815in}{1.575026in}}%
\pgfpathlineto{\pgfqpoint{1.447008in}{1.573398in}}%
\pgfpathlineto{\pgfqpoint{1.469201in}{1.571899in}}%
\pgfpathlineto{\pgfqpoint{1.491393in}{1.570566in}}%
\pgfpathlineto{\pgfqpoint{1.513586in}{1.569201in}}%
\pgfpathlineto{\pgfqpoint{1.535779in}{1.567884in}}%
\pgfpathlineto{\pgfqpoint{1.557972in}{1.566909in}}%
\pgfpathlineto{\pgfqpoint{1.580165in}{1.565915in}}%
\pgfpathlineto{\pgfqpoint{1.602357in}{1.564489in}}%
\pgfpathlineto{\pgfqpoint{1.624550in}{1.563259in}}%
\pgfpathlineto{\pgfqpoint{1.646743in}{1.561898in}}%
\pgfpathlineto{\pgfqpoint{1.668936in}{1.560636in}}%
\pgfpathlineto{\pgfqpoint{1.691129in}{1.559550in}}%
\pgfpathlineto{\pgfqpoint{1.713321in}{1.558387in}}%
\pgfpathlineto{\pgfqpoint{1.735514in}{1.557206in}}%
\pgfpathlineto{\pgfqpoint{1.757707in}{1.556116in}}%
\pgfpathlineto{\pgfqpoint{1.779900in}{1.555417in}}%
\pgfpathlineto{\pgfqpoint{1.802092in}{1.554432in}}%
\pgfpathlineto{\pgfqpoint{1.824285in}{1.553377in}}%
\pgfpathlineto{\pgfqpoint{1.846478in}{1.552387in}}%
\pgfpathlineto{\pgfqpoint{1.868671in}{1.551239in}}%
\pgfpathlineto{\pgfqpoint{1.890864in}{1.549873in}}%
\pgfpathlineto{\pgfqpoint{1.913056in}{1.548918in}}%
\pgfpathlineto{\pgfqpoint{1.935249in}{1.548153in}}%
\pgfpathlineto{\pgfqpoint{1.957442in}{1.547294in}}%
\pgfpathlineto{\pgfqpoint{1.979635in}{1.546587in}}%
\pgfpathlineto{\pgfqpoint{2.001828in}{1.545634in}}%
\pgfpathlineto{\pgfqpoint{2.024020in}{1.544829in}}%
\pgfpathlineto{\pgfqpoint{2.046213in}{1.543922in}}%
\pgfpathlineto{\pgfqpoint{2.068406in}{1.543235in}}%
\pgfpathlineto{\pgfqpoint{2.090599in}{1.542645in}}%
\pgfpathlineto{\pgfqpoint{2.112792in}{1.541635in}}%
\pgfpathlineto{\pgfqpoint{2.134984in}{1.540926in}}%
\pgfpathlineto{\pgfqpoint{2.157177in}{1.540056in}}%
\pgfpathlineto{\pgfqpoint{2.179370in}{1.539421in}}%
\pgfpathlineto{\pgfqpoint{2.201563in}{1.538752in}}%
\pgfpathlineto{\pgfqpoint{2.223755in}{1.538058in}}%
\pgfpathlineto{\pgfqpoint{2.245948in}{1.537419in}}%
\pgfpathlineto{\pgfqpoint{2.268141in}{1.536856in}}%
\pgfpathlineto{\pgfqpoint{2.290334in}{1.536074in}}%
\pgfpathlineto{\pgfqpoint{2.312527in}{1.535212in}}%
\pgfpathlineto{\pgfqpoint{2.334719in}{1.534657in}}%
\pgfpathlineto{\pgfqpoint{2.356912in}{1.533989in}}%
\pgfpathlineto{\pgfqpoint{2.379105in}{1.533339in}}%
\pgfpathlineto{\pgfqpoint{2.401298in}{1.532768in}}%
\pgfpathlineto{\pgfqpoint{2.423491in}{1.532129in}}%
\pgfpathlineto{\pgfqpoint{2.445683in}{1.531495in}}%
\pgfpathlineto{\pgfqpoint{2.467876in}{1.530846in}}%
\pgfpathlineto{\pgfqpoint{2.490069in}{1.530183in}}%
\pgfpathlineto{\pgfqpoint{2.512262in}{1.529464in}}%
\pgfpathlineto{\pgfqpoint{2.534454in}{1.528803in}}%
\pgfpathlineto{\pgfqpoint{2.556647in}{1.528270in}}%
\pgfpathlineto{\pgfqpoint{2.578840in}{1.527754in}}%
\pgfpathlineto{\pgfqpoint{2.601033in}{1.526868in}}%
\pgfpathlineto{\pgfqpoint{2.623226in}{1.526063in}}%
\pgfpathlineto{\pgfqpoint{2.645418in}{1.525450in}}%
\pgfpathlineto{\pgfqpoint{2.667611in}{1.524994in}}%
\pgfusepath{stroke}%
\end{pgfscope}%
\begin{pgfscope}%
\pgfpathrectangle{\pgfqpoint{0.470525in}{0.422992in}}{\pgfqpoint{2.241471in}{1.626201in}}%
\pgfusepath{clip}%
\pgfsetrectcap%
\pgfsetroundjoin%
\pgfsetlinewidth{1.003750pt}%
\definecolor{currentstroke}{rgb}{1.000000,0.584314,0.000000}%
\pgfsetstrokecolor{currentstroke}%
\pgfsetdash{}{0pt}%
\pgfpathmoveto{\pgfqpoint{0.492718in}{1.923693in}}%
\pgfpathlineto{\pgfqpoint{0.514911in}{1.908373in}}%
\pgfpathlineto{\pgfqpoint{0.537104in}{1.899841in}}%
\pgfpathlineto{\pgfqpoint{0.559296in}{1.892984in}}%
\pgfpathlineto{\pgfqpoint{0.581489in}{1.886937in}}%
\pgfpathlineto{\pgfqpoint{0.603682in}{1.882704in}}%
\pgfpathlineto{\pgfqpoint{0.625875in}{1.877120in}}%
\pgfpathlineto{\pgfqpoint{0.648068in}{1.872940in}}%
\pgfpathlineto{\pgfqpoint{0.670260in}{1.869312in}}%
\pgfpathlineto{\pgfqpoint{0.692453in}{1.865660in}}%
\pgfpathlineto{\pgfqpoint{0.714646in}{1.862932in}}%
\pgfpathlineto{\pgfqpoint{0.736839in}{1.860124in}}%
\pgfpathlineto{\pgfqpoint{0.759032in}{1.857879in}}%
\pgfpathlineto{\pgfqpoint{0.781224in}{1.855246in}}%
\pgfpathlineto{\pgfqpoint{0.803417in}{1.852583in}}%
\pgfpathlineto{\pgfqpoint{0.825610in}{1.850538in}}%
\pgfpathlineto{\pgfqpoint{0.847803in}{1.848285in}}%
\pgfpathlineto{\pgfqpoint{0.869995in}{1.846483in}}%
\pgfpathlineto{\pgfqpoint{0.892188in}{1.844510in}}%
\pgfpathlineto{\pgfqpoint{0.914381in}{1.842728in}}%
\pgfpathlineto{\pgfqpoint{0.936574in}{1.840873in}}%
\pgfpathlineto{\pgfqpoint{0.958767in}{1.839168in}}%
\pgfpathlineto{\pgfqpoint{0.980959in}{1.837617in}}%
\pgfpathlineto{\pgfqpoint{1.003152in}{1.836186in}}%
\pgfpathlineto{\pgfqpoint{1.025345in}{1.834429in}}%
\pgfpathlineto{\pgfqpoint{1.047538in}{1.832932in}}%
\pgfpathlineto{\pgfqpoint{1.069731in}{1.831732in}}%
\pgfpathlineto{\pgfqpoint{1.091923in}{1.830403in}}%
\pgfpathlineto{\pgfqpoint{1.114116in}{1.829142in}}%
\pgfpathlineto{\pgfqpoint{1.136309in}{1.827825in}}%
\pgfpathlineto{\pgfqpoint{1.158502in}{1.826505in}}%
\pgfpathlineto{\pgfqpoint{1.180694in}{1.825258in}}%
\pgfpathlineto{\pgfqpoint{1.202887in}{1.824007in}}%
\pgfpathlineto{\pgfqpoint{1.225080in}{1.822928in}}%
\pgfpathlineto{\pgfqpoint{1.247273in}{1.821568in}}%
\pgfpathlineto{\pgfqpoint{1.269466in}{1.820580in}}%
\pgfpathlineto{\pgfqpoint{1.291658in}{1.819467in}}%
\pgfpathlineto{\pgfqpoint{1.313851in}{1.818427in}}%
\pgfpathlineto{\pgfqpoint{1.336044in}{1.817618in}}%
\pgfpathlineto{\pgfqpoint{1.358237in}{1.816479in}}%
\pgfpathlineto{\pgfqpoint{1.380430in}{1.815376in}}%
\pgfpathlineto{\pgfqpoint{1.402622in}{1.814458in}}%
\pgfpathlineto{\pgfqpoint{1.424815in}{1.813457in}}%
\pgfpathlineto{\pgfqpoint{1.447008in}{1.812403in}}%
\pgfpathlineto{\pgfqpoint{1.469201in}{1.811326in}}%
\pgfpathlineto{\pgfqpoint{1.491393in}{1.810397in}}%
\pgfpathlineto{\pgfqpoint{1.513586in}{1.809541in}}%
\pgfpathlineto{\pgfqpoint{1.535779in}{1.808616in}}%
\pgfpathlineto{\pgfqpoint{1.557972in}{1.807746in}}%
\pgfpathlineto{\pgfqpoint{1.580165in}{1.807007in}}%
\pgfpathlineto{\pgfqpoint{1.602357in}{1.806181in}}%
\pgfpathlineto{\pgfqpoint{1.624550in}{1.805405in}}%
\pgfpathlineto{\pgfqpoint{1.646743in}{1.804741in}}%
\pgfpathlineto{\pgfqpoint{1.668936in}{1.804100in}}%
\pgfpathlineto{\pgfqpoint{1.691129in}{1.803396in}}%
\pgfpathlineto{\pgfqpoint{1.713321in}{1.802543in}}%
\pgfpathlineto{\pgfqpoint{1.735514in}{1.801630in}}%
\pgfpathlineto{\pgfqpoint{1.757707in}{1.800978in}}%
\pgfpathlineto{\pgfqpoint{1.779900in}{1.800276in}}%
\pgfpathlineto{\pgfqpoint{1.802092in}{1.799464in}}%
\pgfpathlineto{\pgfqpoint{1.824285in}{1.798820in}}%
\pgfpathlineto{\pgfqpoint{1.846478in}{1.798064in}}%
\pgfpathlineto{\pgfqpoint{1.868671in}{1.797492in}}%
\pgfpathlineto{\pgfqpoint{1.890864in}{1.796884in}}%
\pgfpathlineto{\pgfqpoint{1.913056in}{1.796278in}}%
\pgfpathlineto{\pgfqpoint{1.935249in}{1.795707in}}%
\pgfpathlineto{\pgfqpoint{1.957442in}{1.795106in}}%
\pgfpathlineto{\pgfqpoint{1.979635in}{1.794577in}}%
\pgfpathlineto{\pgfqpoint{2.001828in}{1.793851in}}%
\pgfpathlineto{\pgfqpoint{2.024020in}{1.793075in}}%
\pgfpathlineto{\pgfqpoint{2.046213in}{1.792585in}}%
\pgfpathlineto{\pgfqpoint{2.068406in}{1.792121in}}%
\pgfpathlineto{\pgfqpoint{2.090599in}{1.791496in}}%
\pgfpathlineto{\pgfqpoint{2.112792in}{1.790994in}}%
\pgfpathlineto{\pgfqpoint{2.134984in}{1.790353in}}%
\pgfpathlineto{\pgfqpoint{2.157177in}{1.789724in}}%
\pgfpathlineto{\pgfqpoint{2.179370in}{1.789217in}}%
\pgfpathlineto{\pgfqpoint{2.201563in}{1.788726in}}%
\pgfpathlineto{\pgfqpoint{2.223755in}{1.788212in}}%
\pgfpathlineto{\pgfqpoint{2.245948in}{1.787716in}}%
\pgfpathlineto{\pgfqpoint{2.268141in}{1.787040in}}%
\pgfpathlineto{\pgfqpoint{2.290334in}{1.786565in}}%
\pgfpathlineto{\pgfqpoint{2.312527in}{1.786032in}}%
\pgfpathlineto{\pgfqpoint{2.334719in}{1.785574in}}%
\pgfpathlineto{\pgfqpoint{2.356912in}{1.785047in}}%
\pgfpathlineto{\pgfqpoint{2.379105in}{1.784537in}}%
\pgfpathlineto{\pgfqpoint{2.401298in}{1.783967in}}%
\pgfpathlineto{\pgfqpoint{2.423491in}{1.783458in}}%
\pgfpathlineto{\pgfqpoint{2.445683in}{1.783005in}}%
\pgfpathlineto{\pgfqpoint{2.467876in}{1.782596in}}%
\pgfpathlineto{\pgfqpoint{2.490069in}{1.782125in}}%
\pgfpathlineto{\pgfqpoint{2.512262in}{1.781696in}}%
\pgfpathlineto{\pgfqpoint{2.534454in}{1.781279in}}%
\pgfpathlineto{\pgfqpoint{2.556647in}{1.780816in}}%
\pgfpathlineto{\pgfqpoint{2.578840in}{1.780358in}}%
\pgfpathlineto{\pgfqpoint{2.601033in}{1.779874in}}%
\pgfpathlineto{\pgfqpoint{2.623226in}{1.779379in}}%
\pgfpathlineto{\pgfqpoint{2.645418in}{1.778909in}}%
\pgfpathlineto{\pgfqpoint{2.667611in}{1.778443in}}%
\pgfusepath{stroke}%
\end{pgfscope}%
\begin{pgfscope}%
\pgfpathrectangle{\pgfqpoint{0.470525in}{0.422992in}}{\pgfqpoint{2.241471in}{1.626201in}}%
\pgfusepath{clip}%
\pgfsetrectcap%
\pgfsetroundjoin%
\pgfsetlinewidth{1.003750pt}%
\definecolor{currentstroke}{rgb}{1.000000,0.172549,0.000000}%
\pgfsetstrokecolor{currentstroke}%
\pgfsetdash{}{0pt}%
\pgfpathmoveto{\pgfqpoint{0.492718in}{1.011413in}}%
\pgfpathlineto{\pgfqpoint{0.514911in}{0.840551in}}%
\pgfpathlineto{\pgfqpoint{0.537104in}{0.744252in}}%
\pgfpathlineto{\pgfqpoint{0.559296in}{0.690334in}}%
\pgfpathlineto{\pgfqpoint{0.581489in}{0.650315in}}%
\pgfpathlineto{\pgfqpoint{0.603682in}{0.629553in}}%
\pgfpathlineto{\pgfqpoint{0.625875in}{0.613413in}}%
\pgfpathlineto{\pgfqpoint{0.648068in}{0.601044in}}%
\pgfpathlineto{\pgfqpoint{0.670260in}{0.587790in}}%
\pgfpathlineto{\pgfqpoint{0.692453in}{0.577491in}}%
\pgfpathlineto{\pgfqpoint{0.714646in}{0.572726in}}%
\pgfpathlineto{\pgfqpoint{0.736839in}{0.566573in}}%
\pgfpathlineto{\pgfqpoint{0.759032in}{0.561505in}}%
\pgfpathlineto{\pgfqpoint{0.781224in}{0.557859in}}%
\pgfpathlineto{\pgfqpoint{0.803417in}{0.554014in}}%
\pgfpathlineto{\pgfqpoint{0.825610in}{0.551605in}}%
\pgfpathlineto{\pgfqpoint{0.847803in}{0.548004in}}%
\pgfpathlineto{\pgfqpoint{0.869995in}{0.544724in}}%
\pgfpathlineto{\pgfqpoint{0.892188in}{0.541699in}}%
\pgfpathlineto{\pgfqpoint{0.914381in}{0.538071in}}%
\pgfpathlineto{\pgfqpoint{0.936574in}{0.536404in}}%
\pgfpathlineto{\pgfqpoint{0.958767in}{0.534175in}}%
\pgfpathlineto{\pgfqpoint{0.980959in}{0.531974in}}%
\pgfpathlineto{\pgfqpoint{1.003152in}{0.529020in}}%
\pgfpathlineto{\pgfqpoint{1.025345in}{0.527123in}}%
\pgfpathlineto{\pgfqpoint{1.047538in}{0.526377in}}%
\pgfpathlineto{\pgfqpoint{1.069731in}{0.525132in}}%
\pgfpathlineto{\pgfqpoint{1.091923in}{0.523017in}}%
\pgfpathlineto{\pgfqpoint{1.114116in}{0.522318in}}%
\pgfpathlineto{\pgfqpoint{1.136309in}{0.521573in}}%
\pgfpathlineto{\pgfqpoint{1.158502in}{0.520298in}}%
\pgfpathlineto{\pgfqpoint{1.180694in}{0.518964in}}%
\pgfpathlineto{\pgfqpoint{1.202887in}{0.517282in}}%
\pgfpathlineto{\pgfqpoint{1.225080in}{0.516268in}}%
\pgfpathlineto{\pgfqpoint{1.247273in}{0.514476in}}%
\pgfpathlineto{\pgfqpoint{1.269466in}{0.512231in}}%
\pgfpathlineto{\pgfqpoint{1.291658in}{0.511651in}}%
\pgfpathlineto{\pgfqpoint{1.313851in}{0.511146in}}%
\pgfpathlineto{\pgfqpoint{1.336044in}{0.510397in}}%
\pgfpathlineto{\pgfqpoint{1.358237in}{0.509089in}}%
\pgfpathlineto{\pgfqpoint{1.380430in}{0.507367in}}%
\pgfpathlineto{\pgfqpoint{1.402622in}{0.507480in}}%
\pgfpathlineto{\pgfqpoint{1.424815in}{0.506091in}}%
\pgfpathlineto{\pgfqpoint{1.447008in}{0.505225in}}%
\pgfpathlineto{\pgfqpoint{1.469201in}{0.504416in}}%
\pgfpathlineto{\pgfqpoint{1.491393in}{0.503755in}}%
\pgfpathlineto{\pgfqpoint{1.513586in}{0.502856in}}%
\pgfpathlineto{\pgfqpoint{1.535779in}{0.502705in}}%
\pgfpathlineto{\pgfqpoint{1.557972in}{0.502592in}}%
\pgfpathlineto{\pgfqpoint{1.580165in}{0.502718in}}%
\pgfpathlineto{\pgfqpoint{1.602357in}{0.502321in}}%
\pgfpathlineto{\pgfqpoint{1.624550in}{0.501896in}}%
\pgfpathlineto{\pgfqpoint{1.646743in}{0.501905in}}%
\pgfpathlineto{\pgfqpoint{1.668936in}{0.501457in}}%
\pgfpathlineto{\pgfqpoint{1.691129in}{0.501488in}}%
\pgfpathlineto{\pgfqpoint{1.713321in}{0.500876in}}%
\pgfpathlineto{\pgfqpoint{1.735514in}{0.500701in}}%
\pgfpathlineto{\pgfqpoint{1.757707in}{0.500249in}}%
\pgfpathlineto{\pgfqpoint{1.779900in}{0.500069in}}%
\pgfpathlineto{\pgfqpoint{1.802092in}{0.499783in}}%
\pgfpathlineto{\pgfqpoint{1.824285in}{0.499904in}}%
\pgfpathlineto{\pgfqpoint{1.846478in}{0.499554in}}%
\pgfpathlineto{\pgfqpoint{1.868671in}{0.499300in}}%
\pgfpathlineto{\pgfqpoint{1.890864in}{0.499131in}}%
\pgfpathlineto{\pgfqpoint{1.913056in}{0.498767in}}%
\pgfpathlineto{\pgfqpoint{1.935249in}{0.498564in}}%
\pgfpathlineto{\pgfqpoint{1.957442in}{0.498620in}}%
\pgfpathlineto{\pgfqpoint{1.979635in}{0.498751in}}%
\pgfpathlineto{\pgfqpoint{2.001828in}{0.498662in}}%
\pgfpathlineto{\pgfqpoint{2.024020in}{0.498685in}}%
\pgfpathlineto{\pgfqpoint{2.046213in}{0.498648in}}%
\pgfpathlineto{\pgfqpoint{2.068406in}{0.498668in}}%
\pgfpathlineto{\pgfqpoint{2.090599in}{0.498512in}}%
\pgfpathlineto{\pgfqpoint{2.112792in}{0.498069in}}%
\pgfpathlineto{\pgfqpoint{2.134984in}{0.498079in}}%
\pgfpathlineto{\pgfqpoint{2.157177in}{0.497871in}}%
\pgfpathlineto{\pgfqpoint{2.179370in}{0.497607in}}%
\pgfpathlineto{\pgfqpoint{2.201563in}{0.497627in}}%
\pgfpathlineto{\pgfqpoint{2.223755in}{0.497581in}}%
\pgfpathlineto{\pgfqpoint{2.245948in}{0.497745in}}%
\pgfpathlineto{\pgfqpoint{2.268141in}{0.498007in}}%
\pgfpathlineto{\pgfqpoint{2.290334in}{0.497883in}}%
\pgfpathlineto{\pgfqpoint{2.312527in}{0.498371in}}%
\pgfpathlineto{\pgfqpoint{2.334719in}{0.498514in}}%
\pgfpathlineto{\pgfqpoint{2.356912in}{0.498557in}}%
\pgfpathlineto{\pgfqpoint{2.379105in}{0.498501in}}%
\pgfpathlineto{\pgfqpoint{2.401298in}{0.498471in}}%
\pgfpathlineto{\pgfqpoint{2.423491in}{0.498241in}}%
\pgfpathlineto{\pgfqpoint{2.445683in}{0.498050in}}%
\pgfpathlineto{\pgfqpoint{2.467876in}{0.497943in}}%
\pgfpathlineto{\pgfqpoint{2.490069in}{0.497697in}}%
\pgfpathlineto{\pgfqpoint{2.512262in}{0.497451in}}%
\pgfpathlineto{\pgfqpoint{2.534454in}{0.497362in}}%
\pgfpathlineto{\pgfqpoint{2.556647in}{0.497277in}}%
\pgfpathlineto{\pgfqpoint{2.578840in}{0.497233in}}%
\pgfpathlineto{\pgfqpoint{2.601033in}{0.496910in}}%
\pgfpathlineto{\pgfqpoint{2.623226in}{0.497124in}}%
\pgfpathlineto{\pgfqpoint{2.645418in}{0.497141in}}%
\pgfpathlineto{\pgfqpoint{2.667611in}{0.497155in}}%
\pgfusepath{stroke}%
\end{pgfscope}%
\begin{pgfscope}%
\pgfpathrectangle{\pgfqpoint{0.470525in}{0.422992in}}{\pgfqpoint{2.241471in}{1.626201in}}%
\pgfusepath{clip}%
\pgfsetrectcap%
\pgfsetroundjoin%
\pgfsetlinewidth{1.003750pt}%
\definecolor{currentstroke}{rgb}{0.517647,0.356863,0.592157}%
\pgfsetstrokecolor{currentstroke}%
\pgfsetdash{}{0pt}%
\pgfpathmoveto{\pgfqpoint{0.492718in}{1.559432in}}%
\pgfpathlineto{\pgfqpoint{0.514911in}{1.523037in}}%
\pgfpathlineto{\pgfqpoint{0.537104in}{1.498434in}}%
\pgfpathlineto{\pgfqpoint{0.559296in}{1.484886in}}%
\pgfpathlineto{\pgfqpoint{0.581489in}{1.474697in}}%
\pgfpathlineto{\pgfqpoint{0.603682in}{1.463236in}}%
\pgfpathlineto{\pgfqpoint{0.625875in}{1.453379in}}%
\pgfpathlineto{\pgfqpoint{0.648068in}{1.446241in}}%
\pgfpathlineto{\pgfqpoint{0.670260in}{1.440429in}}%
\pgfpathlineto{\pgfqpoint{0.692453in}{1.435653in}}%
\pgfpathlineto{\pgfqpoint{0.714646in}{1.429842in}}%
\pgfpathlineto{\pgfqpoint{0.736839in}{1.425914in}}%
\pgfpathlineto{\pgfqpoint{0.759032in}{1.421911in}}%
\pgfpathlineto{\pgfqpoint{0.781224in}{1.418024in}}%
\pgfpathlineto{\pgfqpoint{0.803417in}{1.414475in}}%
\pgfpathlineto{\pgfqpoint{0.825610in}{1.411430in}}%
\pgfpathlineto{\pgfqpoint{0.847803in}{1.408402in}}%
\pgfpathlineto{\pgfqpoint{0.869995in}{1.405097in}}%
\pgfpathlineto{\pgfqpoint{0.892188in}{1.402192in}}%
\pgfpathlineto{\pgfqpoint{0.914381in}{1.400437in}}%
\pgfpathlineto{\pgfqpoint{0.936574in}{1.397604in}}%
\pgfpathlineto{\pgfqpoint{0.958767in}{1.395181in}}%
\pgfpathlineto{\pgfqpoint{0.980959in}{1.393317in}}%
\pgfpathlineto{\pgfqpoint{1.003152in}{1.391730in}}%
\pgfpathlineto{\pgfqpoint{1.025345in}{1.389669in}}%
\pgfpathlineto{\pgfqpoint{1.047538in}{1.387038in}}%
\pgfpathlineto{\pgfqpoint{1.069731in}{1.385442in}}%
\pgfpathlineto{\pgfqpoint{1.091923in}{1.383785in}}%
\pgfpathlineto{\pgfqpoint{1.114116in}{1.382304in}}%
\pgfpathlineto{\pgfqpoint{1.136309in}{1.380435in}}%
\pgfpathlineto{\pgfqpoint{1.158502in}{1.378853in}}%
\pgfpathlineto{\pgfqpoint{1.180694in}{1.376923in}}%
\pgfpathlineto{\pgfqpoint{1.202887in}{1.374914in}}%
\pgfpathlineto{\pgfqpoint{1.225080in}{1.373468in}}%
\pgfpathlineto{\pgfqpoint{1.247273in}{1.371547in}}%
\pgfpathlineto{\pgfqpoint{1.269466in}{1.370019in}}%
\pgfpathlineto{\pgfqpoint{1.291658in}{1.368767in}}%
\pgfpathlineto{\pgfqpoint{1.313851in}{1.367158in}}%
\pgfpathlineto{\pgfqpoint{1.336044in}{1.366000in}}%
\pgfpathlineto{\pgfqpoint{1.358237in}{1.365299in}}%
\pgfpathlineto{\pgfqpoint{1.380430in}{1.363805in}}%
\pgfpathlineto{\pgfqpoint{1.402622in}{1.362901in}}%
\pgfpathlineto{\pgfqpoint{1.424815in}{1.361713in}}%
\pgfpathlineto{\pgfqpoint{1.447008in}{1.360528in}}%
\pgfpathlineto{\pgfqpoint{1.469201in}{1.358962in}}%
\pgfpathlineto{\pgfqpoint{1.491393in}{1.357849in}}%
\pgfpathlineto{\pgfqpoint{1.513586in}{1.356555in}}%
\pgfpathlineto{\pgfqpoint{1.535779in}{1.355476in}}%
\pgfpathlineto{\pgfqpoint{1.557972in}{1.354396in}}%
\pgfpathlineto{\pgfqpoint{1.580165in}{1.353047in}}%
\pgfpathlineto{\pgfqpoint{1.602357in}{1.351958in}}%
\pgfpathlineto{\pgfqpoint{1.624550in}{1.351182in}}%
\pgfpathlineto{\pgfqpoint{1.646743in}{1.350302in}}%
\pgfpathlineto{\pgfqpoint{1.668936in}{1.349243in}}%
\pgfpathlineto{\pgfqpoint{1.691129in}{1.348430in}}%
\pgfpathlineto{\pgfqpoint{1.713321in}{1.347350in}}%
\pgfpathlineto{\pgfqpoint{1.735514in}{1.346329in}}%
\pgfpathlineto{\pgfqpoint{1.757707in}{1.345636in}}%
\pgfpathlineto{\pgfqpoint{1.779900in}{1.344660in}}%
\pgfpathlineto{\pgfqpoint{1.802092in}{1.343455in}}%
\pgfpathlineto{\pgfqpoint{1.824285in}{1.342495in}}%
\pgfpathlineto{\pgfqpoint{1.846478in}{1.341713in}}%
\pgfpathlineto{\pgfqpoint{1.868671in}{1.340647in}}%
\pgfpathlineto{\pgfqpoint{1.890864in}{1.339885in}}%
\pgfpathlineto{\pgfqpoint{1.913056in}{1.339090in}}%
\pgfpathlineto{\pgfqpoint{1.935249in}{1.338087in}}%
\pgfpathlineto{\pgfqpoint{1.957442in}{1.337364in}}%
\pgfpathlineto{\pgfqpoint{1.979635in}{1.336567in}}%
\pgfpathlineto{\pgfqpoint{2.001828in}{1.335945in}}%
\pgfpathlineto{\pgfqpoint{2.024020in}{1.335355in}}%
\pgfpathlineto{\pgfqpoint{2.046213in}{1.334743in}}%
\pgfpathlineto{\pgfqpoint{2.068406in}{1.334140in}}%
\pgfpathlineto{\pgfqpoint{2.090599in}{1.333629in}}%
\pgfpathlineto{\pgfqpoint{2.112792in}{1.333091in}}%
\pgfpathlineto{\pgfqpoint{2.134984in}{1.332223in}}%
\pgfpathlineto{\pgfqpoint{2.157177in}{1.331460in}}%
\pgfpathlineto{\pgfqpoint{2.179370in}{1.330930in}}%
\pgfpathlineto{\pgfqpoint{2.201563in}{1.330526in}}%
\pgfpathlineto{\pgfqpoint{2.223755in}{1.330138in}}%
\pgfpathlineto{\pgfqpoint{2.245948in}{1.329554in}}%
\pgfpathlineto{\pgfqpoint{2.268141in}{1.328915in}}%
\pgfpathlineto{\pgfqpoint{2.290334in}{1.328510in}}%
\pgfpathlineto{\pgfqpoint{2.312527in}{1.328014in}}%
\pgfpathlineto{\pgfqpoint{2.334719in}{1.327238in}}%
\pgfpathlineto{\pgfqpoint{2.356912in}{1.326427in}}%
\pgfpathlineto{\pgfqpoint{2.379105in}{1.325888in}}%
\pgfpathlineto{\pgfqpoint{2.401298in}{1.325323in}}%
\pgfpathlineto{\pgfqpoint{2.423491in}{1.324706in}}%
\pgfpathlineto{\pgfqpoint{2.445683in}{1.324126in}}%
\pgfpathlineto{\pgfqpoint{2.467876in}{1.323792in}}%
\pgfpathlineto{\pgfqpoint{2.490069in}{1.323239in}}%
\pgfpathlineto{\pgfqpoint{2.512262in}{1.322829in}}%
\pgfpathlineto{\pgfqpoint{2.534454in}{1.322294in}}%
\pgfpathlineto{\pgfqpoint{2.556647in}{1.321901in}}%
\pgfpathlineto{\pgfqpoint{2.578840in}{1.321350in}}%
\pgfpathlineto{\pgfqpoint{2.601033in}{1.320639in}}%
\pgfpathlineto{\pgfqpoint{2.623226in}{1.320134in}}%
\pgfpathlineto{\pgfqpoint{2.645418in}{1.319638in}}%
\pgfpathlineto{\pgfqpoint{2.667611in}{1.319183in}}%
\pgfusepath{stroke}%
\end{pgfscope}%
\begin{pgfscope}%
\pgfsetrectcap%
\pgfsetmiterjoin%
\pgfsetlinewidth{0.501875pt}%
\definecolor{currentstroke}{rgb}{0.000000,0.000000,0.000000}%
\pgfsetstrokecolor{currentstroke}%
\pgfsetdash{}{0pt}%
\pgfpathmoveto{\pgfqpoint{0.470525in}{0.422992in}}%
\pgfpathlineto{\pgfqpoint{0.470525in}{2.049193in}}%
\pgfusepath{stroke}%
\end{pgfscope}%
\begin{pgfscope}%
\pgfsetrectcap%
\pgfsetmiterjoin%
\pgfsetlinewidth{0.501875pt}%
\definecolor{currentstroke}{rgb}{0.000000,0.000000,0.000000}%
\pgfsetstrokecolor{currentstroke}%
\pgfsetdash{}{0pt}%
\pgfpathmoveto{\pgfqpoint{2.711997in}{0.422992in}}%
\pgfpathlineto{\pgfqpoint{2.711997in}{2.049193in}}%
\pgfusepath{stroke}%
\end{pgfscope}%
\begin{pgfscope}%
\pgfsetrectcap%
\pgfsetmiterjoin%
\pgfsetlinewidth{0.501875pt}%
\definecolor{currentstroke}{rgb}{0.000000,0.000000,0.000000}%
\pgfsetstrokecolor{currentstroke}%
\pgfsetdash{}{0pt}%
\pgfpathmoveto{\pgfqpoint{0.470525in}{0.422992in}}%
\pgfpathlineto{\pgfqpoint{2.711997in}{0.422992in}}%
\pgfusepath{stroke}%
\end{pgfscope}%
\begin{pgfscope}%
\pgfsetrectcap%
\pgfsetmiterjoin%
\pgfsetlinewidth{0.501875pt}%
\definecolor{currentstroke}{rgb}{0.000000,0.000000,0.000000}%
\pgfsetstrokecolor{currentstroke}%
\pgfsetdash{}{0pt}%
\pgfpathmoveto{\pgfqpoint{0.470525in}{2.049193in}}%
\pgfpathlineto{\pgfqpoint{2.711997in}{2.049193in}}%
\pgfusepath{stroke}%
\end{pgfscope}%
\begin{pgfscope}%
\definecolor{textcolor}{rgb}{0.000000,0.000000,0.000000}%
\pgfsetstrokecolor{textcolor}%
\pgfsetfillcolor{textcolor}%
\pgftext[x=1.591261in,y=2.132526in,,base]{\color{textcolor}\rmfamily\fontsize{12.000000}{14.400000}\selectfont Continuity}%
\end{pgfscope}%
\begin{pgfscope}%
\pgfsetbuttcap%
\pgfsetmiterjoin%
\definecolor{currentfill}{rgb}{1.000000,1.000000,1.000000}%
\pgfsetfillcolor{currentfill}%
\pgfsetlinewidth{0.000000pt}%
\definecolor{currentstroke}{rgb}{0.000000,0.000000,0.000000}%
\pgfsetstrokecolor{currentstroke}%
\pgfsetstrokeopacity{0.000000}%
\pgfsetdash{}{0pt}%
\pgfpathmoveto{\pgfqpoint{3.350525in}{0.422992in}}%
\pgfpathlineto{\pgfqpoint{5.591997in}{0.422992in}}%
\pgfpathlineto{\pgfqpoint{5.591997in}{4.374193in}}%
\pgfpathlineto{\pgfqpoint{3.350525in}{4.374193in}}%
\pgfpathlineto{\pgfqpoint{3.350525in}{0.422992in}}%
\pgfpathclose%
\pgfusepath{fill}%
\end{pgfscope}%
\begin{pgfscope}%
\pgfsetbuttcap%
\pgfsetroundjoin%
\definecolor{currentfill}{rgb}{0.000000,0.000000,0.000000}%
\pgfsetfillcolor{currentfill}%
\pgfsetlinewidth{0.501875pt}%
\definecolor{currentstroke}{rgb}{0.000000,0.000000,0.000000}%
\pgfsetstrokecolor{currentstroke}%
\pgfsetdash{}{0pt}%
\pgfsys@defobject{currentmarker}{\pgfqpoint{0.000000in}{0.000000in}}{\pgfqpoint{0.000000in}{0.041667in}}{%
\pgfpathmoveto{\pgfqpoint{0.000000in}{0.000000in}}%
\pgfpathlineto{\pgfqpoint{0.000000in}{0.041667in}}%
\pgfusepath{stroke,fill}%
}%
\begin{pgfscope}%
\pgfsys@transformshift{3.350525in}{0.422992in}%
\pgfsys@useobject{currentmarker}{}%
\end{pgfscope}%
\end{pgfscope}%
\begin{pgfscope}%
\pgfsetbuttcap%
\pgfsetroundjoin%
\definecolor{currentfill}{rgb}{0.000000,0.000000,0.000000}%
\pgfsetfillcolor{currentfill}%
\pgfsetlinewidth{0.501875pt}%
\definecolor{currentstroke}{rgb}{0.000000,0.000000,0.000000}%
\pgfsetstrokecolor{currentstroke}%
\pgfsetdash{}{0pt}%
\pgfsys@defobject{currentmarker}{\pgfqpoint{0.000000in}{-0.041667in}}{\pgfqpoint{0.000000in}{0.000000in}}{%
\pgfpathmoveto{\pgfqpoint{0.000000in}{0.000000in}}%
\pgfpathlineto{\pgfqpoint{0.000000in}{-0.041667in}}%
\pgfusepath{stroke,fill}%
}%
\begin{pgfscope}%
\pgfsys@transformshift{3.350525in}{4.374193in}%
\pgfsys@useobject{currentmarker}{}%
\end{pgfscope}%
\end{pgfscope}%
\begin{pgfscope}%
\definecolor{textcolor}{rgb}{0.000000,0.000000,0.000000}%
\pgfsetstrokecolor{textcolor}%
\pgfsetfillcolor{textcolor}%
\pgftext[x=3.350525in,y=0.374381in,,top]{\color{textcolor}\rmfamily\fontsize{10.000000}{12.000000}\selectfont \(\displaystyle {0}\)}%
\end{pgfscope}%
\begin{pgfscope}%
\pgfsetbuttcap%
\pgfsetroundjoin%
\definecolor{currentfill}{rgb}{0.000000,0.000000,0.000000}%
\pgfsetfillcolor{currentfill}%
\pgfsetlinewidth{0.501875pt}%
\definecolor{currentstroke}{rgb}{0.000000,0.000000,0.000000}%
\pgfsetstrokecolor{currentstroke}%
\pgfsetdash{}{0pt}%
\pgfsys@defobject{currentmarker}{\pgfqpoint{0.000000in}{0.000000in}}{\pgfqpoint{0.000000in}{0.041667in}}{%
\pgfpathmoveto{\pgfqpoint{0.000000in}{0.000000in}}%
\pgfpathlineto{\pgfqpoint{0.000000in}{0.041667in}}%
\pgfusepath{stroke,fill}%
}%
\begin{pgfscope}%
\pgfsys@transformshift{3.794381in}{0.422992in}%
\pgfsys@useobject{currentmarker}{}%
\end{pgfscope}%
\end{pgfscope}%
\begin{pgfscope}%
\pgfsetbuttcap%
\pgfsetroundjoin%
\definecolor{currentfill}{rgb}{0.000000,0.000000,0.000000}%
\pgfsetfillcolor{currentfill}%
\pgfsetlinewidth{0.501875pt}%
\definecolor{currentstroke}{rgb}{0.000000,0.000000,0.000000}%
\pgfsetstrokecolor{currentstroke}%
\pgfsetdash{}{0pt}%
\pgfsys@defobject{currentmarker}{\pgfqpoint{0.000000in}{-0.041667in}}{\pgfqpoint{0.000000in}{0.000000in}}{%
\pgfpathmoveto{\pgfqpoint{0.000000in}{0.000000in}}%
\pgfpathlineto{\pgfqpoint{0.000000in}{-0.041667in}}%
\pgfusepath{stroke,fill}%
}%
\begin{pgfscope}%
\pgfsys@transformshift{3.794381in}{4.374193in}%
\pgfsys@useobject{currentmarker}{}%
\end{pgfscope}%
\end{pgfscope}%
\begin{pgfscope}%
\definecolor{textcolor}{rgb}{0.000000,0.000000,0.000000}%
\pgfsetstrokecolor{textcolor}%
\pgfsetfillcolor{textcolor}%
\pgftext[x=3.794381in,y=0.374381in,,top]{\color{textcolor}\rmfamily\fontsize{10.000000}{12.000000}\selectfont \(\displaystyle {20}\)}%
\end{pgfscope}%
\begin{pgfscope}%
\pgfsetbuttcap%
\pgfsetroundjoin%
\definecolor{currentfill}{rgb}{0.000000,0.000000,0.000000}%
\pgfsetfillcolor{currentfill}%
\pgfsetlinewidth{0.501875pt}%
\definecolor{currentstroke}{rgb}{0.000000,0.000000,0.000000}%
\pgfsetstrokecolor{currentstroke}%
\pgfsetdash{}{0pt}%
\pgfsys@defobject{currentmarker}{\pgfqpoint{0.000000in}{0.000000in}}{\pgfqpoint{0.000000in}{0.041667in}}{%
\pgfpathmoveto{\pgfqpoint{0.000000in}{0.000000in}}%
\pgfpathlineto{\pgfqpoint{0.000000in}{0.041667in}}%
\pgfusepath{stroke,fill}%
}%
\begin{pgfscope}%
\pgfsys@transformshift{4.238237in}{0.422992in}%
\pgfsys@useobject{currentmarker}{}%
\end{pgfscope}%
\end{pgfscope}%
\begin{pgfscope}%
\pgfsetbuttcap%
\pgfsetroundjoin%
\definecolor{currentfill}{rgb}{0.000000,0.000000,0.000000}%
\pgfsetfillcolor{currentfill}%
\pgfsetlinewidth{0.501875pt}%
\definecolor{currentstroke}{rgb}{0.000000,0.000000,0.000000}%
\pgfsetstrokecolor{currentstroke}%
\pgfsetdash{}{0pt}%
\pgfsys@defobject{currentmarker}{\pgfqpoint{0.000000in}{-0.041667in}}{\pgfqpoint{0.000000in}{0.000000in}}{%
\pgfpathmoveto{\pgfqpoint{0.000000in}{0.000000in}}%
\pgfpathlineto{\pgfqpoint{0.000000in}{-0.041667in}}%
\pgfusepath{stroke,fill}%
}%
\begin{pgfscope}%
\pgfsys@transformshift{4.238237in}{4.374193in}%
\pgfsys@useobject{currentmarker}{}%
\end{pgfscope}%
\end{pgfscope}%
\begin{pgfscope}%
\definecolor{textcolor}{rgb}{0.000000,0.000000,0.000000}%
\pgfsetstrokecolor{textcolor}%
\pgfsetfillcolor{textcolor}%
\pgftext[x=4.238237in,y=0.374381in,,top]{\color{textcolor}\rmfamily\fontsize{10.000000}{12.000000}\selectfont \(\displaystyle {40}\)}%
\end{pgfscope}%
\begin{pgfscope}%
\pgfsetbuttcap%
\pgfsetroundjoin%
\definecolor{currentfill}{rgb}{0.000000,0.000000,0.000000}%
\pgfsetfillcolor{currentfill}%
\pgfsetlinewidth{0.501875pt}%
\definecolor{currentstroke}{rgb}{0.000000,0.000000,0.000000}%
\pgfsetstrokecolor{currentstroke}%
\pgfsetdash{}{0pt}%
\pgfsys@defobject{currentmarker}{\pgfqpoint{0.000000in}{0.000000in}}{\pgfqpoint{0.000000in}{0.041667in}}{%
\pgfpathmoveto{\pgfqpoint{0.000000in}{0.000000in}}%
\pgfpathlineto{\pgfqpoint{0.000000in}{0.041667in}}%
\pgfusepath{stroke,fill}%
}%
\begin{pgfscope}%
\pgfsys@transformshift{4.682092in}{0.422992in}%
\pgfsys@useobject{currentmarker}{}%
\end{pgfscope}%
\end{pgfscope}%
\begin{pgfscope}%
\pgfsetbuttcap%
\pgfsetroundjoin%
\definecolor{currentfill}{rgb}{0.000000,0.000000,0.000000}%
\pgfsetfillcolor{currentfill}%
\pgfsetlinewidth{0.501875pt}%
\definecolor{currentstroke}{rgb}{0.000000,0.000000,0.000000}%
\pgfsetstrokecolor{currentstroke}%
\pgfsetdash{}{0pt}%
\pgfsys@defobject{currentmarker}{\pgfqpoint{0.000000in}{-0.041667in}}{\pgfqpoint{0.000000in}{0.000000in}}{%
\pgfpathmoveto{\pgfqpoint{0.000000in}{0.000000in}}%
\pgfpathlineto{\pgfqpoint{0.000000in}{-0.041667in}}%
\pgfusepath{stroke,fill}%
}%
\begin{pgfscope}%
\pgfsys@transformshift{4.682092in}{4.374193in}%
\pgfsys@useobject{currentmarker}{}%
\end{pgfscope}%
\end{pgfscope}%
\begin{pgfscope}%
\definecolor{textcolor}{rgb}{0.000000,0.000000,0.000000}%
\pgfsetstrokecolor{textcolor}%
\pgfsetfillcolor{textcolor}%
\pgftext[x=4.682092in,y=0.374381in,,top]{\color{textcolor}\rmfamily\fontsize{10.000000}{12.000000}\selectfont \(\displaystyle {60}\)}%
\end{pgfscope}%
\begin{pgfscope}%
\pgfsetbuttcap%
\pgfsetroundjoin%
\definecolor{currentfill}{rgb}{0.000000,0.000000,0.000000}%
\pgfsetfillcolor{currentfill}%
\pgfsetlinewidth{0.501875pt}%
\definecolor{currentstroke}{rgb}{0.000000,0.000000,0.000000}%
\pgfsetstrokecolor{currentstroke}%
\pgfsetdash{}{0pt}%
\pgfsys@defobject{currentmarker}{\pgfqpoint{0.000000in}{0.000000in}}{\pgfqpoint{0.000000in}{0.041667in}}{%
\pgfpathmoveto{\pgfqpoint{0.000000in}{0.000000in}}%
\pgfpathlineto{\pgfqpoint{0.000000in}{0.041667in}}%
\pgfusepath{stroke,fill}%
}%
\begin{pgfscope}%
\pgfsys@transformshift{5.125948in}{0.422992in}%
\pgfsys@useobject{currentmarker}{}%
\end{pgfscope}%
\end{pgfscope}%
\begin{pgfscope}%
\pgfsetbuttcap%
\pgfsetroundjoin%
\definecolor{currentfill}{rgb}{0.000000,0.000000,0.000000}%
\pgfsetfillcolor{currentfill}%
\pgfsetlinewidth{0.501875pt}%
\definecolor{currentstroke}{rgb}{0.000000,0.000000,0.000000}%
\pgfsetstrokecolor{currentstroke}%
\pgfsetdash{}{0pt}%
\pgfsys@defobject{currentmarker}{\pgfqpoint{0.000000in}{-0.041667in}}{\pgfqpoint{0.000000in}{0.000000in}}{%
\pgfpathmoveto{\pgfqpoint{0.000000in}{0.000000in}}%
\pgfpathlineto{\pgfqpoint{0.000000in}{-0.041667in}}%
\pgfusepath{stroke,fill}%
}%
\begin{pgfscope}%
\pgfsys@transformshift{5.125948in}{4.374193in}%
\pgfsys@useobject{currentmarker}{}%
\end{pgfscope}%
\end{pgfscope}%
\begin{pgfscope}%
\definecolor{textcolor}{rgb}{0.000000,0.000000,0.000000}%
\pgfsetstrokecolor{textcolor}%
\pgfsetfillcolor{textcolor}%
\pgftext[x=5.125948in,y=0.374381in,,top]{\color{textcolor}\rmfamily\fontsize{10.000000}{12.000000}\selectfont \(\displaystyle {80}\)}%
\end{pgfscope}%
\begin{pgfscope}%
\pgfsetbuttcap%
\pgfsetroundjoin%
\definecolor{currentfill}{rgb}{0.000000,0.000000,0.000000}%
\pgfsetfillcolor{currentfill}%
\pgfsetlinewidth{0.501875pt}%
\definecolor{currentstroke}{rgb}{0.000000,0.000000,0.000000}%
\pgfsetstrokecolor{currentstroke}%
\pgfsetdash{}{0pt}%
\pgfsys@defobject{currentmarker}{\pgfqpoint{0.000000in}{0.000000in}}{\pgfqpoint{0.000000in}{0.020833in}}{%
\pgfpathmoveto{\pgfqpoint{0.000000in}{0.000000in}}%
\pgfpathlineto{\pgfqpoint{0.000000in}{0.020833in}}%
\pgfusepath{stroke,fill}%
}%
\begin{pgfscope}%
\pgfsys@transformshift{3.461489in}{0.422992in}%
\pgfsys@useobject{currentmarker}{}%
\end{pgfscope}%
\end{pgfscope}%
\begin{pgfscope}%
\pgfsetbuttcap%
\pgfsetroundjoin%
\definecolor{currentfill}{rgb}{0.000000,0.000000,0.000000}%
\pgfsetfillcolor{currentfill}%
\pgfsetlinewidth{0.501875pt}%
\definecolor{currentstroke}{rgb}{0.000000,0.000000,0.000000}%
\pgfsetstrokecolor{currentstroke}%
\pgfsetdash{}{0pt}%
\pgfsys@defobject{currentmarker}{\pgfqpoint{0.000000in}{-0.020833in}}{\pgfqpoint{0.000000in}{0.000000in}}{%
\pgfpathmoveto{\pgfqpoint{0.000000in}{0.000000in}}%
\pgfpathlineto{\pgfqpoint{0.000000in}{-0.020833in}}%
\pgfusepath{stroke,fill}%
}%
\begin{pgfscope}%
\pgfsys@transformshift{3.461489in}{4.374193in}%
\pgfsys@useobject{currentmarker}{}%
\end{pgfscope}%
\end{pgfscope}%
\begin{pgfscope}%
\pgfsetbuttcap%
\pgfsetroundjoin%
\definecolor{currentfill}{rgb}{0.000000,0.000000,0.000000}%
\pgfsetfillcolor{currentfill}%
\pgfsetlinewidth{0.501875pt}%
\definecolor{currentstroke}{rgb}{0.000000,0.000000,0.000000}%
\pgfsetstrokecolor{currentstroke}%
\pgfsetdash{}{0pt}%
\pgfsys@defobject{currentmarker}{\pgfqpoint{0.000000in}{0.000000in}}{\pgfqpoint{0.000000in}{0.020833in}}{%
\pgfpathmoveto{\pgfqpoint{0.000000in}{0.000000in}}%
\pgfpathlineto{\pgfqpoint{0.000000in}{0.020833in}}%
\pgfusepath{stroke,fill}%
}%
\begin{pgfscope}%
\pgfsys@transformshift{3.572453in}{0.422992in}%
\pgfsys@useobject{currentmarker}{}%
\end{pgfscope}%
\end{pgfscope}%
\begin{pgfscope}%
\pgfsetbuttcap%
\pgfsetroundjoin%
\definecolor{currentfill}{rgb}{0.000000,0.000000,0.000000}%
\pgfsetfillcolor{currentfill}%
\pgfsetlinewidth{0.501875pt}%
\definecolor{currentstroke}{rgb}{0.000000,0.000000,0.000000}%
\pgfsetstrokecolor{currentstroke}%
\pgfsetdash{}{0pt}%
\pgfsys@defobject{currentmarker}{\pgfqpoint{0.000000in}{-0.020833in}}{\pgfqpoint{0.000000in}{0.000000in}}{%
\pgfpathmoveto{\pgfqpoint{0.000000in}{0.000000in}}%
\pgfpathlineto{\pgfqpoint{0.000000in}{-0.020833in}}%
\pgfusepath{stroke,fill}%
}%
\begin{pgfscope}%
\pgfsys@transformshift{3.572453in}{4.374193in}%
\pgfsys@useobject{currentmarker}{}%
\end{pgfscope}%
\end{pgfscope}%
\begin{pgfscope}%
\pgfsetbuttcap%
\pgfsetroundjoin%
\definecolor{currentfill}{rgb}{0.000000,0.000000,0.000000}%
\pgfsetfillcolor{currentfill}%
\pgfsetlinewidth{0.501875pt}%
\definecolor{currentstroke}{rgb}{0.000000,0.000000,0.000000}%
\pgfsetstrokecolor{currentstroke}%
\pgfsetdash{}{0pt}%
\pgfsys@defobject{currentmarker}{\pgfqpoint{0.000000in}{0.000000in}}{\pgfqpoint{0.000000in}{0.020833in}}{%
\pgfpathmoveto{\pgfqpoint{0.000000in}{0.000000in}}%
\pgfpathlineto{\pgfqpoint{0.000000in}{0.020833in}}%
\pgfusepath{stroke,fill}%
}%
\begin{pgfscope}%
\pgfsys@transformshift{3.683417in}{0.422992in}%
\pgfsys@useobject{currentmarker}{}%
\end{pgfscope}%
\end{pgfscope}%
\begin{pgfscope}%
\pgfsetbuttcap%
\pgfsetroundjoin%
\definecolor{currentfill}{rgb}{0.000000,0.000000,0.000000}%
\pgfsetfillcolor{currentfill}%
\pgfsetlinewidth{0.501875pt}%
\definecolor{currentstroke}{rgb}{0.000000,0.000000,0.000000}%
\pgfsetstrokecolor{currentstroke}%
\pgfsetdash{}{0pt}%
\pgfsys@defobject{currentmarker}{\pgfqpoint{0.000000in}{-0.020833in}}{\pgfqpoint{0.000000in}{0.000000in}}{%
\pgfpathmoveto{\pgfqpoint{0.000000in}{0.000000in}}%
\pgfpathlineto{\pgfqpoint{0.000000in}{-0.020833in}}%
\pgfusepath{stroke,fill}%
}%
\begin{pgfscope}%
\pgfsys@transformshift{3.683417in}{4.374193in}%
\pgfsys@useobject{currentmarker}{}%
\end{pgfscope}%
\end{pgfscope}%
\begin{pgfscope}%
\pgfsetbuttcap%
\pgfsetroundjoin%
\definecolor{currentfill}{rgb}{0.000000,0.000000,0.000000}%
\pgfsetfillcolor{currentfill}%
\pgfsetlinewidth{0.501875pt}%
\definecolor{currentstroke}{rgb}{0.000000,0.000000,0.000000}%
\pgfsetstrokecolor{currentstroke}%
\pgfsetdash{}{0pt}%
\pgfsys@defobject{currentmarker}{\pgfqpoint{0.000000in}{0.000000in}}{\pgfqpoint{0.000000in}{0.020833in}}{%
\pgfpathmoveto{\pgfqpoint{0.000000in}{0.000000in}}%
\pgfpathlineto{\pgfqpoint{0.000000in}{0.020833in}}%
\pgfusepath{stroke,fill}%
}%
\begin{pgfscope}%
\pgfsys@transformshift{3.905345in}{0.422992in}%
\pgfsys@useobject{currentmarker}{}%
\end{pgfscope}%
\end{pgfscope}%
\begin{pgfscope}%
\pgfsetbuttcap%
\pgfsetroundjoin%
\definecolor{currentfill}{rgb}{0.000000,0.000000,0.000000}%
\pgfsetfillcolor{currentfill}%
\pgfsetlinewidth{0.501875pt}%
\definecolor{currentstroke}{rgb}{0.000000,0.000000,0.000000}%
\pgfsetstrokecolor{currentstroke}%
\pgfsetdash{}{0pt}%
\pgfsys@defobject{currentmarker}{\pgfqpoint{0.000000in}{-0.020833in}}{\pgfqpoint{0.000000in}{0.000000in}}{%
\pgfpathmoveto{\pgfqpoint{0.000000in}{0.000000in}}%
\pgfpathlineto{\pgfqpoint{0.000000in}{-0.020833in}}%
\pgfusepath{stroke,fill}%
}%
\begin{pgfscope}%
\pgfsys@transformshift{3.905345in}{4.374193in}%
\pgfsys@useobject{currentmarker}{}%
\end{pgfscope}%
\end{pgfscope}%
\begin{pgfscope}%
\pgfsetbuttcap%
\pgfsetroundjoin%
\definecolor{currentfill}{rgb}{0.000000,0.000000,0.000000}%
\pgfsetfillcolor{currentfill}%
\pgfsetlinewidth{0.501875pt}%
\definecolor{currentstroke}{rgb}{0.000000,0.000000,0.000000}%
\pgfsetstrokecolor{currentstroke}%
\pgfsetdash{}{0pt}%
\pgfsys@defobject{currentmarker}{\pgfqpoint{0.000000in}{0.000000in}}{\pgfqpoint{0.000000in}{0.020833in}}{%
\pgfpathmoveto{\pgfqpoint{0.000000in}{0.000000in}}%
\pgfpathlineto{\pgfqpoint{0.000000in}{0.020833in}}%
\pgfusepath{stroke,fill}%
}%
\begin{pgfscope}%
\pgfsys@transformshift{4.016309in}{0.422992in}%
\pgfsys@useobject{currentmarker}{}%
\end{pgfscope}%
\end{pgfscope}%
\begin{pgfscope}%
\pgfsetbuttcap%
\pgfsetroundjoin%
\definecolor{currentfill}{rgb}{0.000000,0.000000,0.000000}%
\pgfsetfillcolor{currentfill}%
\pgfsetlinewidth{0.501875pt}%
\definecolor{currentstroke}{rgb}{0.000000,0.000000,0.000000}%
\pgfsetstrokecolor{currentstroke}%
\pgfsetdash{}{0pt}%
\pgfsys@defobject{currentmarker}{\pgfqpoint{0.000000in}{-0.020833in}}{\pgfqpoint{0.000000in}{0.000000in}}{%
\pgfpathmoveto{\pgfqpoint{0.000000in}{0.000000in}}%
\pgfpathlineto{\pgfqpoint{0.000000in}{-0.020833in}}%
\pgfusepath{stroke,fill}%
}%
\begin{pgfscope}%
\pgfsys@transformshift{4.016309in}{4.374193in}%
\pgfsys@useobject{currentmarker}{}%
\end{pgfscope}%
\end{pgfscope}%
\begin{pgfscope}%
\pgfsetbuttcap%
\pgfsetroundjoin%
\definecolor{currentfill}{rgb}{0.000000,0.000000,0.000000}%
\pgfsetfillcolor{currentfill}%
\pgfsetlinewidth{0.501875pt}%
\definecolor{currentstroke}{rgb}{0.000000,0.000000,0.000000}%
\pgfsetstrokecolor{currentstroke}%
\pgfsetdash{}{0pt}%
\pgfsys@defobject{currentmarker}{\pgfqpoint{0.000000in}{0.000000in}}{\pgfqpoint{0.000000in}{0.020833in}}{%
\pgfpathmoveto{\pgfqpoint{0.000000in}{0.000000in}}%
\pgfpathlineto{\pgfqpoint{0.000000in}{0.020833in}}%
\pgfusepath{stroke,fill}%
}%
\begin{pgfscope}%
\pgfsys@transformshift{4.127273in}{0.422992in}%
\pgfsys@useobject{currentmarker}{}%
\end{pgfscope}%
\end{pgfscope}%
\begin{pgfscope}%
\pgfsetbuttcap%
\pgfsetroundjoin%
\definecolor{currentfill}{rgb}{0.000000,0.000000,0.000000}%
\pgfsetfillcolor{currentfill}%
\pgfsetlinewidth{0.501875pt}%
\definecolor{currentstroke}{rgb}{0.000000,0.000000,0.000000}%
\pgfsetstrokecolor{currentstroke}%
\pgfsetdash{}{0pt}%
\pgfsys@defobject{currentmarker}{\pgfqpoint{0.000000in}{-0.020833in}}{\pgfqpoint{0.000000in}{0.000000in}}{%
\pgfpathmoveto{\pgfqpoint{0.000000in}{0.000000in}}%
\pgfpathlineto{\pgfqpoint{0.000000in}{-0.020833in}}%
\pgfusepath{stroke,fill}%
}%
\begin{pgfscope}%
\pgfsys@transformshift{4.127273in}{4.374193in}%
\pgfsys@useobject{currentmarker}{}%
\end{pgfscope}%
\end{pgfscope}%
\begin{pgfscope}%
\pgfsetbuttcap%
\pgfsetroundjoin%
\definecolor{currentfill}{rgb}{0.000000,0.000000,0.000000}%
\pgfsetfillcolor{currentfill}%
\pgfsetlinewidth{0.501875pt}%
\definecolor{currentstroke}{rgb}{0.000000,0.000000,0.000000}%
\pgfsetstrokecolor{currentstroke}%
\pgfsetdash{}{0pt}%
\pgfsys@defobject{currentmarker}{\pgfqpoint{0.000000in}{0.000000in}}{\pgfqpoint{0.000000in}{0.020833in}}{%
\pgfpathmoveto{\pgfqpoint{0.000000in}{0.000000in}}%
\pgfpathlineto{\pgfqpoint{0.000000in}{0.020833in}}%
\pgfusepath{stroke,fill}%
}%
\begin{pgfscope}%
\pgfsys@transformshift{4.349201in}{0.422992in}%
\pgfsys@useobject{currentmarker}{}%
\end{pgfscope}%
\end{pgfscope}%
\begin{pgfscope}%
\pgfsetbuttcap%
\pgfsetroundjoin%
\definecolor{currentfill}{rgb}{0.000000,0.000000,0.000000}%
\pgfsetfillcolor{currentfill}%
\pgfsetlinewidth{0.501875pt}%
\definecolor{currentstroke}{rgb}{0.000000,0.000000,0.000000}%
\pgfsetstrokecolor{currentstroke}%
\pgfsetdash{}{0pt}%
\pgfsys@defobject{currentmarker}{\pgfqpoint{0.000000in}{-0.020833in}}{\pgfqpoint{0.000000in}{0.000000in}}{%
\pgfpathmoveto{\pgfqpoint{0.000000in}{0.000000in}}%
\pgfpathlineto{\pgfqpoint{0.000000in}{-0.020833in}}%
\pgfusepath{stroke,fill}%
}%
\begin{pgfscope}%
\pgfsys@transformshift{4.349201in}{4.374193in}%
\pgfsys@useobject{currentmarker}{}%
\end{pgfscope}%
\end{pgfscope}%
\begin{pgfscope}%
\pgfsetbuttcap%
\pgfsetroundjoin%
\definecolor{currentfill}{rgb}{0.000000,0.000000,0.000000}%
\pgfsetfillcolor{currentfill}%
\pgfsetlinewidth{0.501875pt}%
\definecolor{currentstroke}{rgb}{0.000000,0.000000,0.000000}%
\pgfsetstrokecolor{currentstroke}%
\pgfsetdash{}{0pt}%
\pgfsys@defobject{currentmarker}{\pgfqpoint{0.000000in}{0.000000in}}{\pgfqpoint{0.000000in}{0.020833in}}{%
\pgfpathmoveto{\pgfqpoint{0.000000in}{0.000000in}}%
\pgfpathlineto{\pgfqpoint{0.000000in}{0.020833in}}%
\pgfusepath{stroke,fill}%
}%
\begin{pgfscope}%
\pgfsys@transformshift{4.460165in}{0.422992in}%
\pgfsys@useobject{currentmarker}{}%
\end{pgfscope}%
\end{pgfscope}%
\begin{pgfscope}%
\pgfsetbuttcap%
\pgfsetroundjoin%
\definecolor{currentfill}{rgb}{0.000000,0.000000,0.000000}%
\pgfsetfillcolor{currentfill}%
\pgfsetlinewidth{0.501875pt}%
\definecolor{currentstroke}{rgb}{0.000000,0.000000,0.000000}%
\pgfsetstrokecolor{currentstroke}%
\pgfsetdash{}{0pt}%
\pgfsys@defobject{currentmarker}{\pgfqpoint{0.000000in}{-0.020833in}}{\pgfqpoint{0.000000in}{0.000000in}}{%
\pgfpathmoveto{\pgfqpoint{0.000000in}{0.000000in}}%
\pgfpathlineto{\pgfqpoint{0.000000in}{-0.020833in}}%
\pgfusepath{stroke,fill}%
}%
\begin{pgfscope}%
\pgfsys@transformshift{4.460165in}{4.374193in}%
\pgfsys@useobject{currentmarker}{}%
\end{pgfscope}%
\end{pgfscope}%
\begin{pgfscope}%
\pgfsetbuttcap%
\pgfsetroundjoin%
\definecolor{currentfill}{rgb}{0.000000,0.000000,0.000000}%
\pgfsetfillcolor{currentfill}%
\pgfsetlinewidth{0.501875pt}%
\definecolor{currentstroke}{rgb}{0.000000,0.000000,0.000000}%
\pgfsetstrokecolor{currentstroke}%
\pgfsetdash{}{0pt}%
\pgfsys@defobject{currentmarker}{\pgfqpoint{0.000000in}{0.000000in}}{\pgfqpoint{0.000000in}{0.020833in}}{%
\pgfpathmoveto{\pgfqpoint{0.000000in}{0.000000in}}%
\pgfpathlineto{\pgfqpoint{0.000000in}{0.020833in}}%
\pgfusepath{stroke,fill}%
}%
\begin{pgfscope}%
\pgfsys@transformshift{4.571129in}{0.422992in}%
\pgfsys@useobject{currentmarker}{}%
\end{pgfscope}%
\end{pgfscope}%
\begin{pgfscope}%
\pgfsetbuttcap%
\pgfsetroundjoin%
\definecolor{currentfill}{rgb}{0.000000,0.000000,0.000000}%
\pgfsetfillcolor{currentfill}%
\pgfsetlinewidth{0.501875pt}%
\definecolor{currentstroke}{rgb}{0.000000,0.000000,0.000000}%
\pgfsetstrokecolor{currentstroke}%
\pgfsetdash{}{0pt}%
\pgfsys@defobject{currentmarker}{\pgfqpoint{0.000000in}{-0.020833in}}{\pgfqpoint{0.000000in}{0.000000in}}{%
\pgfpathmoveto{\pgfqpoint{0.000000in}{0.000000in}}%
\pgfpathlineto{\pgfqpoint{0.000000in}{-0.020833in}}%
\pgfusepath{stroke,fill}%
}%
\begin{pgfscope}%
\pgfsys@transformshift{4.571129in}{4.374193in}%
\pgfsys@useobject{currentmarker}{}%
\end{pgfscope}%
\end{pgfscope}%
\begin{pgfscope}%
\pgfsetbuttcap%
\pgfsetroundjoin%
\definecolor{currentfill}{rgb}{0.000000,0.000000,0.000000}%
\pgfsetfillcolor{currentfill}%
\pgfsetlinewidth{0.501875pt}%
\definecolor{currentstroke}{rgb}{0.000000,0.000000,0.000000}%
\pgfsetstrokecolor{currentstroke}%
\pgfsetdash{}{0pt}%
\pgfsys@defobject{currentmarker}{\pgfqpoint{0.000000in}{0.000000in}}{\pgfqpoint{0.000000in}{0.020833in}}{%
\pgfpathmoveto{\pgfqpoint{0.000000in}{0.000000in}}%
\pgfpathlineto{\pgfqpoint{0.000000in}{0.020833in}}%
\pgfusepath{stroke,fill}%
}%
\begin{pgfscope}%
\pgfsys@transformshift{4.793056in}{0.422992in}%
\pgfsys@useobject{currentmarker}{}%
\end{pgfscope}%
\end{pgfscope}%
\begin{pgfscope}%
\pgfsetbuttcap%
\pgfsetroundjoin%
\definecolor{currentfill}{rgb}{0.000000,0.000000,0.000000}%
\pgfsetfillcolor{currentfill}%
\pgfsetlinewidth{0.501875pt}%
\definecolor{currentstroke}{rgb}{0.000000,0.000000,0.000000}%
\pgfsetstrokecolor{currentstroke}%
\pgfsetdash{}{0pt}%
\pgfsys@defobject{currentmarker}{\pgfqpoint{0.000000in}{-0.020833in}}{\pgfqpoint{0.000000in}{0.000000in}}{%
\pgfpathmoveto{\pgfqpoint{0.000000in}{0.000000in}}%
\pgfpathlineto{\pgfqpoint{0.000000in}{-0.020833in}}%
\pgfusepath{stroke,fill}%
}%
\begin{pgfscope}%
\pgfsys@transformshift{4.793056in}{4.374193in}%
\pgfsys@useobject{currentmarker}{}%
\end{pgfscope}%
\end{pgfscope}%
\begin{pgfscope}%
\pgfsetbuttcap%
\pgfsetroundjoin%
\definecolor{currentfill}{rgb}{0.000000,0.000000,0.000000}%
\pgfsetfillcolor{currentfill}%
\pgfsetlinewidth{0.501875pt}%
\definecolor{currentstroke}{rgb}{0.000000,0.000000,0.000000}%
\pgfsetstrokecolor{currentstroke}%
\pgfsetdash{}{0pt}%
\pgfsys@defobject{currentmarker}{\pgfqpoint{0.000000in}{0.000000in}}{\pgfqpoint{0.000000in}{0.020833in}}{%
\pgfpathmoveto{\pgfqpoint{0.000000in}{0.000000in}}%
\pgfpathlineto{\pgfqpoint{0.000000in}{0.020833in}}%
\pgfusepath{stroke,fill}%
}%
\begin{pgfscope}%
\pgfsys@transformshift{4.904020in}{0.422992in}%
\pgfsys@useobject{currentmarker}{}%
\end{pgfscope}%
\end{pgfscope}%
\begin{pgfscope}%
\pgfsetbuttcap%
\pgfsetroundjoin%
\definecolor{currentfill}{rgb}{0.000000,0.000000,0.000000}%
\pgfsetfillcolor{currentfill}%
\pgfsetlinewidth{0.501875pt}%
\definecolor{currentstroke}{rgb}{0.000000,0.000000,0.000000}%
\pgfsetstrokecolor{currentstroke}%
\pgfsetdash{}{0pt}%
\pgfsys@defobject{currentmarker}{\pgfqpoint{0.000000in}{-0.020833in}}{\pgfqpoint{0.000000in}{0.000000in}}{%
\pgfpathmoveto{\pgfqpoint{0.000000in}{0.000000in}}%
\pgfpathlineto{\pgfqpoint{0.000000in}{-0.020833in}}%
\pgfusepath{stroke,fill}%
}%
\begin{pgfscope}%
\pgfsys@transformshift{4.904020in}{4.374193in}%
\pgfsys@useobject{currentmarker}{}%
\end{pgfscope}%
\end{pgfscope}%
\begin{pgfscope}%
\pgfsetbuttcap%
\pgfsetroundjoin%
\definecolor{currentfill}{rgb}{0.000000,0.000000,0.000000}%
\pgfsetfillcolor{currentfill}%
\pgfsetlinewidth{0.501875pt}%
\definecolor{currentstroke}{rgb}{0.000000,0.000000,0.000000}%
\pgfsetstrokecolor{currentstroke}%
\pgfsetdash{}{0pt}%
\pgfsys@defobject{currentmarker}{\pgfqpoint{0.000000in}{0.000000in}}{\pgfqpoint{0.000000in}{0.020833in}}{%
\pgfpathmoveto{\pgfqpoint{0.000000in}{0.000000in}}%
\pgfpathlineto{\pgfqpoint{0.000000in}{0.020833in}}%
\pgfusepath{stroke,fill}%
}%
\begin{pgfscope}%
\pgfsys@transformshift{5.014984in}{0.422992in}%
\pgfsys@useobject{currentmarker}{}%
\end{pgfscope}%
\end{pgfscope}%
\begin{pgfscope}%
\pgfsetbuttcap%
\pgfsetroundjoin%
\definecolor{currentfill}{rgb}{0.000000,0.000000,0.000000}%
\pgfsetfillcolor{currentfill}%
\pgfsetlinewidth{0.501875pt}%
\definecolor{currentstroke}{rgb}{0.000000,0.000000,0.000000}%
\pgfsetstrokecolor{currentstroke}%
\pgfsetdash{}{0pt}%
\pgfsys@defobject{currentmarker}{\pgfqpoint{0.000000in}{-0.020833in}}{\pgfqpoint{0.000000in}{0.000000in}}{%
\pgfpathmoveto{\pgfqpoint{0.000000in}{0.000000in}}%
\pgfpathlineto{\pgfqpoint{0.000000in}{-0.020833in}}%
\pgfusepath{stroke,fill}%
}%
\begin{pgfscope}%
\pgfsys@transformshift{5.014984in}{4.374193in}%
\pgfsys@useobject{currentmarker}{}%
\end{pgfscope}%
\end{pgfscope}%
\begin{pgfscope}%
\pgfsetbuttcap%
\pgfsetroundjoin%
\definecolor{currentfill}{rgb}{0.000000,0.000000,0.000000}%
\pgfsetfillcolor{currentfill}%
\pgfsetlinewidth{0.501875pt}%
\definecolor{currentstroke}{rgb}{0.000000,0.000000,0.000000}%
\pgfsetstrokecolor{currentstroke}%
\pgfsetdash{}{0pt}%
\pgfsys@defobject{currentmarker}{\pgfqpoint{0.000000in}{0.000000in}}{\pgfqpoint{0.000000in}{0.020833in}}{%
\pgfpathmoveto{\pgfqpoint{0.000000in}{0.000000in}}%
\pgfpathlineto{\pgfqpoint{0.000000in}{0.020833in}}%
\pgfusepath{stroke,fill}%
}%
\begin{pgfscope}%
\pgfsys@transformshift{5.236912in}{0.422992in}%
\pgfsys@useobject{currentmarker}{}%
\end{pgfscope}%
\end{pgfscope}%
\begin{pgfscope}%
\pgfsetbuttcap%
\pgfsetroundjoin%
\definecolor{currentfill}{rgb}{0.000000,0.000000,0.000000}%
\pgfsetfillcolor{currentfill}%
\pgfsetlinewidth{0.501875pt}%
\definecolor{currentstroke}{rgb}{0.000000,0.000000,0.000000}%
\pgfsetstrokecolor{currentstroke}%
\pgfsetdash{}{0pt}%
\pgfsys@defobject{currentmarker}{\pgfqpoint{0.000000in}{-0.020833in}}{\pgfqpoint{0.000000in}{0.000000in}}{%
\pgfpathmoveto{\pgfqpoint{0.000000in}{0.000000in}}%
\pgfpathlineto{\pgfqpoint{0.000000in}{-0.020833in}}%
\pgfusepath{stroke,fill}%
}%
\begin{pgfscope}%
\pgfsys@transformshift{5.236912in}{4.374193in}%
\pgfsys@useobject{currentmarker}{}%
\end{pgfscope}%
\end{pgfscope}%
\begin{pgfscope}%
\pgfsetbuttcap%
\pgfsetroundjoin%
\definecolor{currentfill}{rgb}{0.000000,0.000000,0.000000}%
\pgfsetfillcolor{currentfill}%
\pgfsetlinewidth{0.501875pt}%
\definecolor{currentstroke}{rgb}{0.000000,0.000000,0.000000}%
\pgfsetstrokecolor{currentstroke}%
\pgfsetdash{}{0pt}%
\pgfsys@defobject{currentmarker}{\pgfqpoint{0.000000in}{0.000000in}}{\pgfqpoint{0.000000in}{0.020833in}}{%
\pgfpathmoveto{\pgfqpoint{0.000000in}{0.000000in}}%
\pgfpathlineto{\pgfqpoint{0.000000in}{0.020833in}}%
\pgfusepath{stroke,fill}%
}%
\begin{pgfscope}%
\pgfsys@transformshift{5.347876in}{0.422992in}%
\pgfsys@useobject{currentmarker}{}%
\end{pgfscope}%
\end{pgfscope}%
\begin{pgfscope}%
\pgfsetbuttcap%
\pgfsetroundjoin%
\definecolor{currentfill}{rgb}{0.000000,0.000000,0.000000}%
\pgfsetfillcolor{currentfill}%
\pgfsetlinewidth{0.501875pt}%
\definecolor{currentstroke}{rgb}{0.000000,0.000000,0.000000}%
\pgfsetstrokecolor{currentstroke}%
\pgfsetdash{}{0pt}%
\pgfsys@defobject{currentmarker}{\pgfqpoint{0.000000in}{-0.020833in}}{\pgfqpoint{0.000000in}{0.000000in}}{%
\pgfpathmoveto{\pgfqpoint{0.000000in}{0.000000in}}%
\pgfpathlineto{\pgfqpoint{0.000000in}{-0.020833in}}%
\pgfusepath{stroke,fill}%
}%
\begin{pgfscope}%
\pgfsys@transformshift{5.347876in}{4.374193in}%
\pgfsys@useobject{currentmarker}{}%
\end{pgfscope}%
\end{pgfscope}%
\begin{pgfscope}%
\pgfsetbuttcap%
\pgfsetroundjoin%
\definecolor{currentfill}{rgb}{0.000000,0.000000,0.000000}%
\pgfsetfillcolor{currentfill}%
\pgfsetlinewidth{0.501875pt}%
\definecolor{currentstroke}{rgb}{0.000000,0.000000,0.000000}%
\pgfsetstrokecolor{currentstroke}%
\pgfsetdash{}{0pt}%
\pgfsys@defobject{currentmarker}{\pgfqpoint{0.000000in}{0.000000in}}{\pgfqpoint{0.000000in}{0.020833in}}{%
\pgfpathmoveto{\pgfqpoint{0.000000in}{0.000000in}}%
\pgfpathlineto{\pgfqpoint{0.000000in}{0.020833in}}%
\pgfusepath{stroke,fill}%
}%
\begin{pgfscope}%
\pgfsys@transformshift{5.458840in}{0.422992in}%
\pgfsys@useobject{currentmarker}{}%
\end{pgfscope}%
\end{pgfscope}%
\begin{pgfscope}%
\pgfsetbuttcap%
\pgfsetroundjoin%
\definecolor{currentfill}{rgb}{0.000000,0.000000,0.000000}%
\pgfsetfillcolor{currentfill}%
\pgfsetlinewidth{0.501875pt}%
\definecolor{currentstroke}{rgb}{0.000000,0.000000,0.000000}%
\pgfsetstrokecolor{currentstroke}%
\pgfsetdash{}{0pt}%
\pgfsys@defobject{currentmarker}{\pgfqpoint{0.000000in}{-0.020833in}}{\pgfqpoint{0.000000in}{0.000000in}}{%
\pgfpathmoveto{\pgfqpoint{0.000000in}{0.000000in}}%
\pgfpathlineto{\pgfqpoint{0.000000in}{-0.020833in}}%
\pgfusepath{stroke,fill}%
}%
\begin{pgfscope}%
\pgfsys@transformshift{5.458840in}{4.374193in}%
\pgfsys@useobject{currentmarker}{}%
\end{pgfscope}%
\end{pgfscope}%
\begin{pgfscope}%
\pgfsetbuttcap%
\pgfsetroundjoin%
\definecolor{currentfill}{rgb}{0.000000,0.000000,0.000000}%
\pgfsetfillcolor{currentfill}%
\pgfsetlinewidth{0.501875pt}%
\definecolor{currentstroke}{rgb}{0.000000,0.000000,0.000000}%
\pgfsetstrokecolor{currentstroke}%
\pgfsetdash{}{0pt}%
\pgfsys@defobject{currentmarker}{\pgfqpoint{0.000000in}{0.000000in}}{\pgfqpoint{0.000000in}{0.020833in}}{%
\pgfpathmoveto{\pgfqpoint{0.000000in}{0.000000in}}%
\pgfpathlineto{\pgfqpoint{0.000000in}{0.020833in}}%
\pgfusepath{stroke,fill}%
}%
\begin{pgfscope}%
\pgfsys@transformshift{5.569804in}{0.422992in}%
\pgfsys@useobject{currentmarker}{}%
\end{pgfscope}%
\end{pgfscope}%
\begin{pgfscope}%
\pgfsetbuttcap%
\pgfsetroundjoin%
\definecolor{currentfill}{rgb}{0.000000,0.000000,0.000000}%
\pgfsetfillcolor{currentfill}%
\pgfsetlinewidth{0.501875pt}%
\definecolor{currentstroke}{rgb}{0.000000,0.000000,0.000000}%
\pgfsetstrokecolor{currentstroke}%
\pgfsetdash{}{0pt}%
\pgfsys@defobject{currentmarker}{\pgfqpoint{0.000000in}{-0.020833in}}{\pgfqpoint{0.000000in}{0.000000in}}{%
\pgfpathmoveto{\pgfqpoint{0.000000in}{0.000000in}}%
\pgfpathlineto{\pgfqpoint{0.000000in}{-0.020833in}}%
\pgfusepath{stroke,fill}%
}%
\begin{pgfscope}%
\pgfsys@transformshift{5.569804in}{4.374193in}%
\pgfsys@useobject{currentmarker}{}%
\end{pgfscope}%
\end{pgfscope}%
\begin{pgfscope}%
\definecolor{textcolor}{rgb}{0.000000,0.000000,0.000000}%
\pgfsetstrokecolor{textcolor}%
\pgfsetfillcolor{textcolor}%
\pgftext[x=4.471261in,y=0.184413in,,top]{\color{textcolor}\rmfamily\fontsize{10.000000}{12.000000}\selectfont \(\displaystyle K\)}%
\end{pgfscope}%
\begin{pgfscope}%
\pgfsetbuttcap%
\pgfsetroundjoin%
\definecolor{currentfill}{rgb}{0.000000,0.000000,0.000000}%
\pgfsetfillcolor{currentfill}%
\pgfsetlinewidth{0.501875pt}%
\definecolor{currentstroke}{rgb}{0.000000,0.000000,0.000000}%
\pgfsetstrokecolor{currentstroke}%
\pgfsetdash{}{0pt}%
\pgfsys@defobject{currentmarker}{\pgfqpoint{0.000000in}{0.000000in}}{\pgfqpoint{0.041667in}{0.000000in}}{%
\pgfpathmoveto{\pgfqpoint{0.000000in}{0.000000in}}%
\pgfpathlineto{\pgfqpoint{0.041667in}{0.000000in}}%
\pgfusepath{stroke,fill}%
}%
\begin{pgfscope}%
\pgfsys@transformshift{3.350525in}{0.496123in}%
\pgfsys@useobject{currentmarker}{}%
\end{pgfscope}%
\end{pgfscope}%
\begin{pgfscope}%
\pgfsetbuttcap%
\pgfsetroundjoin%
\definecolor{currentfill}{rgb}{0.000000,0.000000,0.000000}%
\pgfsetfillcolor{currentfill}%
\pgfsetlinewidth{0.501875pt}%
\definecolor{currentstroke}{rgb}{0.000000,0.000000,0.000000}%
\pgfsetstrokecolor{currentstroke}%
\pgfsetdash{}{0pt}%
\pgfsys@defobject{currentmarker}{\pgfqpoint{-0.041667in}{0.000000in}}{\pgfqpoint{-0.000000in}{0.000000in}}{%
\pgfpathmoveto{\pgfqpoint{-0.000000in}{0.000000in}}%
\pgfpathlineto{\pgfqpoint{-0.041667in}{0.000000in}}%
\pgfusepath{stroke,fill}%
}%
\begin{pgfscope}%
\pgfsys@transformshift{5.591997in}{0.496123in}%
\pgfsys@useobject{currentmarker}{}%
\end{pgfscope}%
\end{pgfscope}%
\begin{pgfscope}%
\definecolor{textcolor}{rgb}{0.000000,0.000000,0.000000}%
\pgfsetstrokecolor{textcolor}%
\pgfsetfillcolor{textcolor}%
\pgftext[x=3.055000in, y=0.443362in, left, base]{\color{textcolor}\rmfamily\fontsize{10.000000}{12.000000}\selectfont \(\displaystyle {0.00}\)}%
\end{pgfscope}%
\begin{pgfscope}%
\pgfsetbuttcap%
\pgfsetroundjoin%
\definecolor{currentfill}{rgb}{0.000000,0.000000,0.000000}%
\pgfsetfillcolor{currentfill}%
\pgfsetlinewidth{0.501875pt}%
\definecolor{currentstroke}{rgb}{0.000000,0.000000,0.000000}%
\pgfsetstrokecolor{currentstroke}%
\pgfsetdash{}{0pt}%
\pgfsys@defobject{currentmarker}{\pgfqpoint{0.000000in}{0.000000in}}{\pgfqpoint{0.041667in}{0.000000in}}{%
\pgfpathmoveto{\pgfqpoint{0.000000in}{0.000000in}}%
\pgfpathlineto{\pgfqpoint{0.041667in}{0.000000in}}%
\pgfusepath{stroke,fill}%
}%
\begin{pgfscope}%
\pgfsys@transformshift{3.350525in}{0.939659in}%
\pgfsys@useobject{currentmarker}{}%
\end{pgfscope}%
\end{pgfscope}%
\begin{pgfscope}%
\pgfsetbuttcap%
\pgfsetroundjoin%
\definecolor{currentfill}{rgb}{0.000000,0.000000,0.000000}%
\pgfsetfillcolor{currentfill}%
\pgfsetlinewidth{0.501875pt}%
\definecolor{currentstroke}{rgb}{0.000000,0.000000,0.000000}%
\pgfsetstrokecolor{currentstroke}%
\pgfsetdash{}{0pt}%
\pgfsys@defobject{currentmarker}{\pgfqpoint{-0.041667in}{0.000000in}}{\pgfqpoint{-0.000000in}{0.000000in}}{%
\pgfpathmoveto{\pgfqpoint{-0.000000in}{0.000000in}}%
\pgfpathlineto{\pgfqpoint{-0.041667in}{0.000000in}}%
\pgfusepath{stroke,fill}%
}%
\begin{pgfscope}%
\pgfsys@transformshift{5.591997in}{0.939659in}%
\pgfsys@useobject{currentmarker}{}%
\end{pgfscope}%
\end{pgfscope}%
\begin{pgfscope}%
\definecolor{textcolor}{rgb}{0.000000,0.000000,0.000000}%
\pgfsetstrokecolor{textcolor}%
\pgfsetfillcolor{textcolor}%
\pgftext[x=3.055000in, y=0.886898in, left, base]{\color{textcolor}\rmfamily\fontsize{10.000000}{12.000000}\selectfont \(\displaystyle {0.05}\)}%
\end{pgfscope}%
\begin{pgfscope}%
\pgfsetbuttcap%
\pgfsetroundjoin%
\definecolor{currentfill}{rgb}{0.000000,0.000000,0.000000}%
\pgfsetfillcolor{currentfill}%
\pgfsetlinewidth{0.501875pt}%
\definecolor{currentstroke}{rgb}{0.000000,0.000000,0.000000}%
\pgfsetstrokecolor{currentstroke}%
\pgfsetdash{}{0pt}%
\pgfsys@defobject{currentmarker}{\pgfqpoint{0.000000in}{0.000000in}}{\pgfqpoint{0.041667in}{0.000000in}}{%
\pgfpathmoveto{\pgfqpoint{0.000000in}{0.000000in}}%
\pgfpathlineto{\pgfqpoint{0.041667in}{0.000000in}}%
\pgfusepath{stroke,fill}%
}%
\begin{pgfscope}%
\pgfsys@transformshift{3.350525in}{1.383196in}%
\pgfsys@useobject{currentmarker}{}%
\end{pgfscope}%
\end{pgfscope}%
\begin{pgfscope}%
\pgfsetbuttcap%
\pgfsetroundjoin%
\definecolor{currentfill}{rgb}{0.000000,0.000000,0.000000}%
\pgfsetfillcolor{currentfill}%
\pgfsetlinewidth{0.501875pt}%
\definecolor{currentstroke}{rgb}{0.000000,0.000000,0.000000}%
\pgfsetstrokecolor{currentstroke}%
\pgfsetdash{}{0pt}%
\pgfsys@defobject{currentmarker}{\pgfqpoint{-0.041667in}{0.000000in}}{\pgfqpoint{-0.000000in}{0.000000in}}{%
\pgfpathmoveto{\pgfqpoint{-0.000000in}{0.000000in}}%
\pgfpathlineto{\pgfqpoint{-0.041667in}{0.000000in}}%
\pgfusepath{stroke,fill}%
}%
\begin{pgfscope}%
\pgfsys@transformshift{5.591997in}{1.383196in}%
\pgfsys@useobject{currentmarker}{}%
\end{pgfscope}%
\end{pgfscope}%
\begin{pgfscope}%
\definecolor{textcolor}{rgb}{0.000000,0.000000,0.000000}%
\pgfsetstrokecolor{textcolor}%
\pgfsetfillcolor{textcolor}%
\pgftext[x=3.055000in, y=1.330434in, left, base]{\color{textcolor}\rmfamily\fontsize{10.000000}{12.000000}\selectfont \(\displaystyle {0.10}\)}%
\end{pgfscope}%
\begin{pgfscope}%
\pgfsetbuttcap%
\pgfsetroundjoin%
\definecolor{currentfill}{rgb}{0.000000,0.000000,0.000000}%
\pgfsetfillcolor{currentfill}%
\pgfsetlinewidth{0.501875pt}%
\definecolor{currentstroke}{rgb}{0.000000,0.000000,0.000000}%
\pgfsetstrokecolor{currentstroke}%
\pgfsetdash{}{0pt}%
\pgfsys@defobject{currentmarker}{\pgfqpoint{0.000000in}{0.000000in}}{\pgfqpoint{0.041667in}{0.000000in}}{%
\pgfpathmoveto{\pgfqpoint{0.000000in}{0.000000in}}%
\pgfpathlineto{\pgfqpoint{0.041667in}{0.000000in}}%
\pgfusepath{stroke,fill}%
}%
\begin{pgfscope}%
\pgfsys@transformshift{3.350525in}{1.826732in}%
\pgfsys@useobject{currentmarker}{}%
\end{pgfscope}%
\end{pgfscope}%
\begin{pgfscope}%
\pgfsetbuttcap%
\pgfsetroundjoin%
\definecolor{currentfill}{rgb}{0.000000,0.000000,0.000000}%
\pgfsetfillcolor{currentfill}%
\pgfsetlinewidth{0.501875pt}%
\definecolor{currentstroke}{rgb}{0.000000,0.000000,0.000000}%
\pgfsetstrokecolor{currentstroke}%
\pgfsetdash{}{0pt}%
\pgfsys@defobject{currentmarker}{\pgfqpoint{-0.041667in}{0.000000in}}{\pgfqpoint{-0.000000in}{0.000000in}}{%
\pgfpathmoveto{\pgfqpoint{-0.000000in}{0.000000in}}%
\pgfpathlineto{\pgfqpoint{-0.041667in}{0.000000in}}%
\pgfusepath{stroke,fill}%
}%
\begin{pgfscope}%
\pgfsys@transformshift{5.591997in}{1.826732in}%
\pgfsys@useobject{currentmarker}{}%
\end{pgfscope}%
\end{pgfscope}%
\begin{pgfscope}%
\definecolor{textcolor}{rgb}{0.000000,0.000000,0.000000}%
\pgfsetstrokecolor{textcolor}%
\pgfsetfillcolor{textcolor}%
\pgftext[x=3.055000in, y=1.773970in, left, base]{\color{textcolor}\rmfamily\fontsize{10.000000}{12.000000}\selectfont \(\displaystyle {0.15}\)}%
\end{pgfscope}%
\begin{pgfscope}%
\pgfsetbuttcap%
\pgfsetroundjoin%
\definecolor{currentfill}{rgb}{0.000000,0.000000,0.000000}%
\pgfsetfillcolor{currentfill}%
\pgfsetlinewidth{0.501875pt}%
\definecolor{currentstroke}{rgb}{0.000000,0.000000,0.000000}%
\pgfsetstrokecolor{currentstroke}%
\pgfsetdash{}{0pt}%
\pgfsys@defobject{currentmarker}{\pgfqpoint{0.000000in}{0.000000in}}{\pgfqpoint{0.041667in}{0.000000in}}{%
\pgfpathmoveto{\pgfqpoint{0.000000in}{0.000000in}}%
\pgfpathlineto{\pgfqpoint{0.041667in}{0.000000in}}%
\pgfusepath{stroke,fill}%
}%
\begin{pgfscope}%
\pgfsys@transformshift{3.350525in}{2.270268in}%
\pgfsys@useobject{currentmarker}{}%
\end{pgfscope}%
\end{pgfscope}%
\begin{pgfscope}%
\pgfsetbuttcap%
\pgfsetroundjoin%
\definecolor{currentfill}{rgb}{0.000000,0.000000,0.000000}%
\pgfsetfillcolor{currentfill}%
\pgfsetlinewidth{0.501875pt}%
\definecolor{currentstroke}{rgb}{0.000000,0.000000,0.000000}%
\pgfsetstrokecolor{currentstroke}%
\pgfsetdash{}{0pt}%
\pgfsys@defobject{currentmarker}{\pgfqpoint{-0.041667in}{0.000000in}}{\pgfqpoint{-0.000000in}{0.000000in}}{%
\pgfpathmoveto{\pgfqpoint{-0.000000in}{0.000000in}}%
\pgfpathlineto{\pgfqpoint{-0.041667in}{0.000000in}}%
\pgfusepath{stroke,fill}%
}%
\begin{pgfscope}%
\pgfsys@transformshift{5.591997in}{2.270268in}%
\pgfsys@useobject{currentmarker}{}%
\end{pgfscope}%
\end{pgfscope}%
\begin{pgfscope}%
\definecolor{textcolor}{rgb}{0.000000,0.000000,0.000000}%
\pgfsetstrokecolor{textcolor}%
\pgfsetfillcolor{textcolor}%
\pgftext[x=3.055000in, y=2.217506in, left, base]{\color{textcolor}\rmfamily\fontsize{10.000000}{12.000000}\selectfont \(\displaystyle {0.20}\)}%
\end{pgfscope}%
\begin{pgfscope}%
\pgfsetbuttcap%
\pgfsetroundjoin%
\definecolor{currentfill}{rgb}{0.000000,0.000000,0.000000}%
\pgfsetfillcolor{currentfill}%
\pgfsetlinewidth{0.501875pt}%
\definecolor{currentstroke}{rgb}{0.000000,0.000000,0.000000}%
\pgfsetstrokecolor{currentstroke}%
\pgfsetdash{}{0pt}%
\pgfsys@defobject{currentmarker}{\pgfqpoint{0.000000in}{0.000000in}}{\pgfqpoint{0.041667in}{0.000000in}}{%
\pgfpathmoveto{\pgfqpoint{0.000000in}{0.000000in}}%
\pgfpathlineto{\pgfqpoint{0.041667in}{0.000000in}}%
\pgfusepath{stroke,fill}%
}%
\begin{pgfscope}%
\pgfsys@transformshift{3.350525in}{2.713804in}%
\pgfsys@useobject{currentmarker}{}%
\end{pgfscope}%
\end{pgfscope}%
\begin{pgfscope}%
\pgfsetbuttcap%
\pgfsetroundjoin%
\definecolor{currentfill}{rgb}{0.000000,0.000000,0.000000}%
\pgfsetfillcolor{currentfill}%
\pgfsetlinewidth{0.501875pt}%
\definecolor{currentstroke}{rgb}{0.000000,0.000000,0.000000}%
\pgfsetstrokecolor{currentstroke}%
\pgfsetdash{}{0pt}%
\pgfsys@defobject{currentmarker}{\pgfqpoint{-0.041667in}{0.000000in}}{\pgfqpoint{-0.000000in}{0.000000in}}{%
\pgfpathmoveto{\pgfqpoint{-0.000000in}{0.000000in}}%
\pgfpathlineto{\pgfqpoint{-0.041667in}{0.000000in}}%
\pgfusepath{stroke,fill}%
}%
\begin{pgfscope}%
\pgfsys@transformshift{5.591997in}{2.713804in}%
\pgfsys@useobject{currentmarker}{}%
\end{pgfscope}%
\end{pgfscope}%
\begin{pgfscope}%
\definecolor{textcolor}{rgb}{0.000000,0.000000,0.000000}%
\pgfsetstrokecolor{textcolor}%
\pgfsetfillcolor{textcolor}%
\pgftext[x=3.055000in, y=2.661042in, left, base]{\color{textcolor}\rmfamily\fontsize{10.000000}{12.000000}\selectfont \(\displaystyle {0.25}\)}%
\end{pgfscope}%
\begin{pgfscope}%
\pgfsetbuttcap%
\pgfsetroundjoin%
\definecolor{currentfill}{rgb}{0.000000,0.000000,0.000000}%
\pgfsetfillcolor{currentfill}%
\pgfsetlinewidth{0.501875pt}%
\definecolor{currentstroke}{rgb}{0.000000,0.000000,0.000000}%
\pgfsetstrokecolor{currentstroke}%
\pgfsetdash{}{0pt}%
\pgfsys@defobject{currentmarker}{\pgfqpoint{0.000000in}{0.000000in}}{\pgfqpoint{0.041667in}{0.000000in}}{%
\pgfpathmoveto{\pgfqpoint{0.000000in}{0.000000in}}%
\pgfpathlineto{\pgfqpoint{0.041667in}{0.000000in}}%
\pgfusepath{stroke,fill}%
}%
\begin{pgfscope}%
\pgfsys@transformshift{3.350525in}{3.157340in}%
\pgfsys@useobject{currentmarker}{}%
\end{pgfscope}%
\end{pgfscope}%
\begin{pgfscope}%
\pgfsetbuttcap%
\pgfsetroundjoin%
\definecolor{currentfill}{rgb}{0.000000,0.000000,0.000000}%
\pgfsetfillcolor{currentfill}%
\pgfsetlinewidth{0.501875pt}%
\definecolor{currentstroke}{rgb}{0.000000,0.000000,0.000000}%
\pgfsetstrokecolor{currentstroke}%
\pgfsetdash{}{0pt}%
\pgfsys@defobject{currentmarker}{\pgfqpoint{-0.041667in}{0.000000in}}{\pgfqpoint{-0.000000in}{0.000000in}}{%
\pgfpathmoveto{\pgfqpoint{-0.000000in}{0.000000in}}%
\pgfpathlineto{\pgfqpoint{-0.041667in}{0.000000in}}%
\pgfusepath{stroke,fill}%
}%
\begin{pgfscope}%
\pgfsys@transformshift{5.591997in}{3.157340in}%
\pgfsys@useobject{currentmarker}{}%
\end{pgfscope}%
\end{pgfscope}%
\begin{pgfscope}%
\definecolor{textcolor}{rgb}{0.000000,0.000000,0.000000}%
\pgfsetstrokecolor{textcolor}%
\pgfsetfillcolor{textcolor}%
\pgftext[x=3.055000in, y=3.104579in, left, base]{\color{textcolor}\rmfamily\fontsize{10.000000}{12.000000}\selectfont \(\displaystyle {0.30}\)}%
\end{pgfscope}%
\begin{pgfscope}%
\pgfsetbuttcap%
\pgfsetroundjoin%
\definecolor{currentfill}{rgb}{0.000000,0.000000,0.000000}%
\pgfsetfillcolor{currentfill}%
\pgfsetlinewidth{0.501875pt}%
\definecolor{currentstroke}{rgb}{0.000000,0.000000,0.000000}%
\pgfsetstrokecolor{currentstroke}%
\pgfsetdash{}{0pt}%
\pgfsys@defobject{currentmarker}{\pgfqpoint{0.000000in}{0.000000in}}{\pgfqpoint{0.041667in}{0.000000in}}{%
\pgfpathmoveto{\pgfqpoint{0.000000in}{0.000000in}}%
\pgfpathlineto{\pgfqpoint{0.041667in}{0.000000in}}%
\pgfusepath{stroke,fill}%
}%
\begin{pgfscope}%
\pgfsys@transformshift{3.350525in}{3.600876in}%
\pgfsys@useobject{currentmarker}{}%
\end{pgfscope}%
\end{pgfscope}%
\begin{pgfscope}%
\pgfsetbuttcap%
\pgfsetroundjoin%
\definecolor{currentfill}{rgb}{0.000000,0.000000,0.000000}%
\pgfsetfillcolor{currentfill}%
\pgfsetlinewidth{0.501875pt}%
\definecolor{currentstroke}{rgb}{0.000000,0.000000,0.000000}%
\pgfsetstrokecolor{currentstroke}%
\pgfsetdash{}{0pt}%
\pgfsys@defobject{currentmarker}{\pgfqpoint{-0.041667in}{0.000000in}}{\pgfqpoint{-0.000000in}{0.000000in}}{%
\pgfpathmoveto{\pgfqpoint{-0.000000in}{0.000000in}}%
\pgfpathlineto{\pgfqpoint{-0.041667in}{0.000000in}}%
\pgfusepath{stroke,fill}%
}%
\begin{pgfscope}%
\pgfsys@transformshift{5.591997in}{3.600876in}%
\pgfsys@useobject{currentmarker}{}%
\end{pgfscope}%
\end{pgfscope}%
\begin{pgfscope}%
\definecolor{textcolor}{rgb}{0.000000,0.000000,0.000000}%
\pgfsetstrokecolor{textcolor}%
\pgfsetfillcolor{textcolor}%
\pgftext[x=3.055000in, y=3.548115in, left, base]{\color{textcolor}\rmfamily\fontsize{10.000000}{12.000000}\selectfont \(\displaystyle {0.35}\)}%
\end{pgfscope}%
\begin{pgfscope}%
\pgfsetbuttcap%
\pgfsetroundjoin%
\definecolor{currentfill}{rgb}{0.000000,0.000000,0.000000}%
\pgfsetfillcolor{currentfill}%
\pgfsetlinewidth{0.501875pt}%
\definecolor{currentstroke}{rgb}{0.000000,0.000000,0.000000}%
\pgfsetstrokecolor{currentstroke}%
\pgfsetdash{}{0pt}%
\pgfsys@defobject{currentmarker}{\pgfqpoint{0.000000in}{0.000000in}}{\pgfqpoint{0.041667in}{0.000000in}}{%
\pgfpathmoveto{\pgfqpoint{0.000000in}{0.000000in}}%
\pgfpathlineto{\pgfqpoint{0.041667in}{0.000000in}}%
\pgfusepath{stroke,fill}%
}%
\begin{pgfscope}%
\pgfsys@transformshift{3.350525in}{4.044412in}%
\pgfsys@useobject{currentmarker}{}%
\end{pgfscope}%
\end{pgfscope}%
\begin{pgfscope}%
\pgfsetbuttcap%
\pgfsetroundjoin%
\definecolor{currentfill}{rgb}{0.000000,0.000000,0.000000}%
\pgfsetfillcolor{currentfill}%
\pgfsetlinewidth{0.501875pt}%
\definecolor{currentstroke}{rgb}{0.000000,0.000000,0.000000}%
\pgfsetstrokecolor{currentstroke}%
\pgfsetdash{}{0pt}%
\pgfsys@defobject{currentmarker}{\pgfqpoint{-0.041667in}{0.000000in}}{\pgfqpoint{-0.000000in}{0.000000in}}{%
\pgfpathmoveto{\pgfqpoint{-0.000000in}{0.000000in}}%
\pgfpathlineto{\pgfqpoint{-0.041667in}{0.000000in}}%
\pgfusepath{stroke,fill}%
}%
\begin{pgfscope}%
\pgfsys@transformshift{5.591997in}{4.044412in}%
\pgfsys@useobject{currentmarker}{}%
\end{pgfscope}%
\end{pgfscope}%
\begin{pgfscope}%
\definecolor{textcolor}{rgb}{0.000000,0.000000,0.000000}%
\pgfsetstrokecolor{textcolor}%
\pgfsetfillcolor{textcolor}%
\pgftext[x=3.055000in, y=3.991651in, left, base]{\color{textcolor}\rmfamily\fontsize{10.000000}{12.000000}\selectfont \(\displaystyle {0.40}\)}%
\end{pgfscope}%
\begin{pgfscope}%
\pgfsetbuttcap%
\pgfsetroundjoin%
\definecolor{currentfill}{rgb}{0.000000,0.000000,0.000000}%
\pgfsetfillcolor{currentfill}%
\pgfsetlinewidth{0.501875pt}%
\definecolor{currentstroke}{rgb}{0.000000,0.000000,0.000000}%
\pgfsetstrokecolor{currentstroke}%
\pgfsetdash{}{0pt}%
\pgfsys@defobject{currentmarker}{\pgfqpoint{0.000000in}{0.000000in}}{\pgfqpoint{0.020833in}{0.000000in}}{%
\pgfpathmoveto{\pgfqpoint{0.000000in}{0.000000in}}%
\pgfpathlineto{\pgfqpoint{0.020833in}{0.000000in}}%
\pgfusepath{stroke,fill}%
}%
\begin{pgfscope}%
\pgfsys@transformshift{3.350525in}{0.584831in}%
\pgfsys@useobject{currentmarker}{}%
\end{pgfscope}%
\end{pgfscope}%
\begin{pgfscope}%
\pgfsetbuttcap%
\pgfsetroundjoin%
\definecolor{currentfill}{rgb}{0.000000,0.000000,0.000000}%
\pgfsetfillcolor{currentfill}%
\pgfsetlinewidth{0.501875pt}%
\definecolor{currentstroke}{rgb}{0.000000,0.000000,0.000000}%
\pgfsetstrokecolor{currentstroke}%
\pgfsetdash{}{0pt}%
\pgfsys@defobject{currentmarker}{\pgfqpoint{-0.020833in}{0.000000in}}{\pgfqpoint{-0.000000in}{0.000000in}}{%
\pgfpathmoveto{\pgfqpoint{-0.000000in}{0.000000in}}%
\pgfpathlineto{\pgfqpoint{-0.020833in}{0.000000in}}%
\pgfusepath{stroke,fill}%
}%
\begin{pgfscope}%
\pgfsys@transformshift{5.591997in}{0.584831in}%
\pgfsys@useobject{currentmarker}{}%
\end{pgfscope}%
\end{pgfscope}%
\begin{pgfscope}%
\pgfsetbuttcap%
\pgfsetroundjoin%
\definecolor{currentfill}{rgb}{0.000000,0.000000,0.000000}%
\pgfsetfillcolor{currentfill}%
\pgfsetlinewidth{0.501875pt}%
\definecolor{currentstroke}{rgb}{0.000000,0.000000,0.000000}%
\pgfsetstrokecolor{currentstroke}%
\pgfsetdash{}{0pt}%
\pgfsys@defobject{currentmarker}{\pgfqpoint{0.000000in}{0.000000in}}{\pgfqpoint{0.020833in}{0.000000in}}{%
\pgfpathmoveto{\pgfqpoint{0.000000in}{0.000000in}}%
\pgfpathlineto{\pgfqpoint{0.020833in}{0.000000in}}%
\pgfusepath{stroke,fill}%
}%
\begin{pgfscope}%
\pgfsys@transformshift{3.350525in}{0.673538in}%
\pgfsys@useobject{currentmarker}{}%
\end{pgfscope}%
\end{pgfscope}%
\begin{pgfscope}%
\pgfsetbuttcap%
\pgfsetroundjoin%
\definecolor{currentfill}{rgb}{0.000000,0.000000,0.000000}%
\pgfsetfillcolor{currentfill}%
\pgfsetlinewidth{0.501875pt}%
\definecolor{currentstroke}{rgb}{0.000000,0.000000,0.000000}%
\pgfsetstrokecolor{currentstroke}%
\pgfsetdash{}{0pt}%
\pgfsys@defobject{currentmarker}{\pgfqpoint{-0.020833in}{0.000000in}}{\pgfqpoint{-0.000000in}{0.000000in}}{%
\pgfpathmoveto{\pgfqpoint{-0.000000in}{0.000000in}}%
\pgfpathlineto{\pgfqpoint{-0.020833in}{0.000000in}}%
\pgfusepath{stroke,fill}%
}%
\begin{pgfscope}%
\pgfsys@transformshift{5.591997in}{0.673538in}%
\pgfsys@useobject{currentmarker}{}%
\end{pgfscope}%
\end{pgfscope}%
\begin{pgfscope}%
\pgfsetbuttcap%
\pgfsetroundjoin%
\definecolor{currentfill}{rgb}{0.000000,0.000000,0.000000}%
\pgfsetfillcolor{currentfill}%
\pgfsetlinewidth{0.501875pt}%
\definecolor{currentstroke}{rgb}{0.000000,0.000000,0.000000}%
\pgfsetstrokecolor{currentstroke}%
\pgfsetdash{}{0pt}%
\pgfsys@defobject{currentmarker}{\pgfqpoint{0.000000in}{0.000000in}}{\pgfqpoint{0.020833in}{0.000000in}}{%
\pgfpathmoveto{\pgfqpoint{0.000000in}{0.000000in}}%
\pgfpathlineto{\pgfqpoint{0.020833in}{0.000000in}}%
\pgfusepath{stroke,fill}%
}%
\begin{pgfscope}%
\pgfsys@transformshift{3.350525in}{0.762245in}%
\pgfsys@useobject{currentmarker}{}%
\end{pgfscope}%
\end{pgfscope}%
\begin{pgfscope}%
\pgfsetbuttcap%
\pgfsetroundjoin%
\definecolor{currentfill}{rgb}{0.000000,0.000000,0.000000}%
\pgfsetfillcolor{currentfill}%
\pgfsetlinewidth{0.501875pt}%
\definecolor{currentstroke}{rgb}{0.000000,0.000000,0.000000}%
\pgfsetstrokecolor{currentstroke}%
\pgfsetdash{}{0pt}%
\pgfsys@defobject{currentmarker}{\pgfqpoint{-0.020833in}{0.000000in}}{\pgfqpoint{-0.000000in}{0.000000in}}{%
\pgfpathmoveto{\pgfqpoint{-0.000000in}{0.000000in}}%
\pgfpathlineto{\pgfqpoint{-0.020833in}{0.000000in}}%
\pgfusepath{stroke,fill}%
}%
\begin{pgfscope}%
\pgfsys@transformshift{5.591997in}{0.762245in}%
\pgfsys@useobject{currentmarker}{}%
\end{pgfscope}%
\end{pgfscope}%
\begin{pgfscope}%
\pgfsetbuttcap%
\pgfsetroundjoin%
\definecolor{currentfill}{rgb}{0.000000,0.000000,0.000000}%
\pgfsetfillcolor{currentfill}%
\pgfsetlinewidth{0.501875pt}%
\definecolor{currentstroke}{rgb}{0.000000,0.000000,0.000000}%
\pgfsetstrokecolor{currentstroke}%
\pgfsetdash{}{0pt}%
\pgfsys@defobject{currentmarker}{\pgfqpoint{0.000000in}{0.000000in}}{\pgfqpoint{0.020833in}{0.000000in}}{%
\pgfpathmoveto{\pgfqpoint{0.000000in}{0.000000in}}%
\pgfpathlineto{\pgfqpoint{0.020833in}{0.000000in}}%
\pgfusepath{stroke,fill}%
}%
\begin{pgfscope}%
\pgfsys@transformshift{3.350525in}{0.850952in}%
\pgfsys@useobject{currentmarker}{}%
\end{pgfscope}%
\end{pgfscope}%
\begin{pgfscope}%
\pgfsetbuttcap%
\pgfsetroundjoin%
\definecolor{currentfill}{rgb}{0.000000,0.000000,0.000000}%
\pgfsetfillcolor{currentfill}%
\pgfsetlinewidth{0.501875pt}%
\definecolor{currentstroke}{rgb}{0.000000,0.000000,0.000000}%
\pgfsetstrokecolor{currentstroke}%
\pgfsetdash{}{0pt}%
\pgfsys@defobject{currentmarker}{\pgfqpoint{-0.020833in}{0.000000in}}{\pgfqpoint{-0.000000in}{0.000000in}}{%
\pgfpathmoveto{\pgfqpoint{-0.000000in}{0.000000in}}%
\pgfpathlineto{\pgfqpoint{-0.020833in}{0.000000in}}%
\pgfusepath{stroke,fill}%
}%
\begin{pgfscope}%
\pgfsys@transformshift{5.591997in}{0.850952in}%
\pgfsys@useobject{currentmarker}{}%
\end{pgfscope}%
\end{pgfscope}%
\begin{pgfscope}%
\pgfsetbuttcap%
\pgfsetroundjoin%
\definecolor{currentfill}{rgb}{0.000000,0.000000,0.000000}%
\pgfsetfillcolor{currentfill}%
\pgfsetlinewidth{0.501875pt}%
\definecolor{currentstroke}{rgb}{0.000000,0.000000,0.000000}%
\pgfsetstrokecolor{currentstroke}%
\pgfsetdash{}{0pt}%
\pgfsys@defobject{currentmarker}{\pgfqpoint{0.000000in}{0.000000in}}{\pgfqpoint{0.020833in}{0.000000in}}{%
\pgfpathmoveto{\pgfqpoint{0.000000in}{0.000000in}}%
\pgfpathlineto{\pgfqpoint{0.020833in}{0.000000in}}%
\pgfusepath{stroke,fill}%
}%
\begin{pgfscope}%
\pgfsys@transformshift{3.350525in}{1.028367in}%
\pgfsys@useobject{currentmarker}{}%
\end{pgfscope}%
\end{pgfscope}%
\begin{pgfscope}%
\pgfsetbuttcap%
\pgfsetroundjoin%
\definecolor{currentfill}{rgb}{0.000000,0.000000,0.000000}%
\pgfsetfillcolor{currentfill}%
\pgfsetlinewidth{0.501875pt}%
\definecolor{currentstroke}{rgb}{0.000000,0.000000,0.000000}%
\pgfsetstrokecolor{currentstroke}%
\pgfsetdash{}{0pt}%
\pgfsys@defobject{currentmarker}{\pgfqpoint{-0.020833in}{0.000000in}}{\pgfqpoint{-0.000000in}{0.000000in}}{%
\pgfpathmoveto{\pgfqpoint{-0.000000in}{0.000000in}}%
\pgfpathlineto{\pgfqpoint{-0.020833in}{0.000000in}}%
\pgfusepath{stroke,fill}%
}%
\begin{pgfscope}%
\pgfsys@transformshift{5.591997in}{1.028367in}%
\pgfsys@useobject{currentmarker}{}%
\end{pgfscope}%
\end{pgfscope}%
\begin{pgfscope}%
\pgfsetbuttcap%
\pgfsetroundjoin%
\definecolor{currentfill}{rgb}{0.000000,0.000000,0.000000}%
\pgfsetfillcolor{currentfill}%
\pgfsetlinewidth{0.501875pt}%
\definecolor{currentstroke}{rgb}{0.000000,0.000000,0.000000}%
\pgfsetstrokecolor{currentstroke}%
\pgfsetdash{}{0pt}%
\pgfsys@defobject{currentmarker}{\pgfqpoint{0.000000in}{0.000000in}}{\pgfqpoint{0.020833in}{0.000000in}}{%
\pgfpathmoveto{\pgfqpoint{0.000000in}{0.000000in}}%
\pgfpathlineto{\pgfqpoint{0.020833in}{0.000000in}}%
\pgfusepath{stroke,fill}%
}%
\begin{pgfscope}%
\pgfsys@transformshift{3.350525in}{1.117074in}%
\pgfsys@useobject{currentmarker}{}%
\end{pgfscope}%
\end{pgfscope}%
\begin{pgfscope}%
\pgfsetbuttcap%
\pgfsetroundjoin%
\definecolor{currentfill}{rgb}{0.000000,0.000000,0.000000}%
\pgfsetfillcolor{currentfill}%
\pgfsetlinewidth{0.501875pt}%
\definecolor{currentstroke}{rgb}{0.000000,0.000000,0.000000}%
\pgfsetstrokecolor{currentstroke}%
\pgfsetdash{}{0pt}%
\pgfsys@defobject{currentmarker}{\pgfqpoint{-0.020833in}{0.000000in}}{\pgfqpoint{-0.000000in}{0.000000in}}{%
\pgfpathmoveto{\pgfqpoint{-0.000000in}{0.000000in}}%
\pgfpathlineto{\pgfqpoint{-0.020833in}{0.000000in}}%
\pgfusepath{stroke,fill}%
}%
\begin{pgfscope}%
\pgfsys@transformshift{5.591997in}{1.117074in}%
\pgfsys@useobject{currentmarker}{}%
\end{pgfscope}%
\end{pgfscope}%
\begin{pgfscope}%
\pgfsetbuttcap%
\pgfsetroundjoin%
\definecolor{currentfill}{rgb}{0.000000,0.000000,0.000000}%
\pgfsetfillcolor{currentfill}%
\pgfsetlinewidth{0.501875pt}%
\definecolor{currentstroke}{rgb}{0.000000,0.000000,0.000000}%
\pgfsetstrokecolor{currentstroke}%
\pgfsetdash{}{0pt}%
\pgfsys@defobject{currentmarker}{\pgfqpoint{0.000000in}{0.000000in}}{\pgfqpoint{0.020833in}{0.000000in}}{%
\pgfpathmoveto{\pgfqpoint{0.000000in}{0.000000in}}%
\pgfpathlineto{\pgfqpoint{0.020833in}{0.000000in}}%
\pgfusepath{stroke,fill}%
}%
\begin{pgfscope}%
\pgfsys@transformshift{3.350525in}{1.205781in}%
\pgfsys@useobject{currentmarker}{}%
\end{pgfscope}%
\end{pgfscope}%
\begin{pgfscope}%
\pgfsetbuttcap%
\pgfsetroundjoin%
\definecolor{currentfill}{rgb}{0.000000,0.000000,0.000000}%
\pgfsetfillcolor{currentfill}%
\pgfsetlinewidth{0.501875pt}%
\definecolor{currentstroke}{rgb}{0.000000,0.000000,0.000000}%
\pgfsetstrokecolor{currentstroke}%
\pgfsetdash{}{0pt}%
\pgfsys@defobject{currentmarker}{\pgfqpoint{-0.020833in}{0.000000in}}{\pgfqpoint{-0.000000in}{0.000000in}}{%
\pgfpathmoveto{\pgfqpoint{-0.000000in}{0.000000in}}%
\pgfpathlineto{\pgfqpoint{-0.020833in}{0.000000in}}%
\pgfusepath{stroke,fill}%
}%
\begin{pgfscope}%
\pgfsys@transformshift{5.591997in}{1.205781in}%
\pgfsys@useobject{currentmarker}{}%
\end{pgfscope}%
\end{pgfscope}%
\begin{pgfscope}%
\pgfsetbuttcap%
\pgfsetroundjoin%
\definecolor{currentfill}{rgb}{0.000000,0.000000,0.000000}%
\pgfsetfillcolor{currentfill}%
\pgfsetlinewidth{0.501875pt}%
\definecolor{currentstroke}{rgb}{0.000000,0.000000,0.000000}%
\pgfsetstrokecolor{currentstroke}%
\pgfsetdash{}{0pt}%
\pgfsys@defobject{currentmarker}{\pgfqpoint{0.000000in}{0.000000in}}{\pgfqpoint{0.020833in}{0.000000in}}{%
\pgfpathmoveto{\pgfqpoint{0.000000in}{0.000000in}}%
\pgfpathlineto{\pgfqpoint{0.020833in}{0.000000in}}%
\pgfusepath{stroke,fill}%
}%
\begin{pgfscope}%
\pgfsys@transformshift{3.350525in}{1.294488in}%
\pgfsys@useobject{currentmarker}{}%
\end{pgfscope}%
\end{pgfscope}%
\begin{pgfscope}%
\pgfsetbuttcap%
\pgfsetroundjoin%
\definecolor{currentfill}{rgb}{0.000000,0.000000,0.000000}%
\pgfsetfillcolor{currentfill}%
\pgfsetlinewidth{0.501875pt}%
\definecolor{currentstroke}{rgb}{0.000000,0.000000,0.000000}%
\pgfsetstrokecolor{currentstroke}%
\pgfsetdash{}{0pt}%
\pgfsys@defobject{currentmarker}{\pgfqpoint{-0.020833in}{0.000000in}}{\pgfqpoint{-0.000000in}{0.000000in}}{%
\pgfpathmoveto{\pgfqpoint{-0.000000in}{0.000000in}}%
\pgfpathlineto{\pgfqpoint{-0.020833in}{0.000000in}}%
\pgfusepath{stroke,fill}%
}%
\begin{pgfscope}%
\pgfsys@transformshift{5.591997in}{1.294488in}%
\pgfsys@useobject{currentmarker}{}%
\end{pgfscope}%
\end{pgfscope}%
\begin{pgfscope}%
\pgfsetbuttcap%
\pgfsetroundjoin%
\definecolor{currentfill}{rgb}{0.000000,0.000000,0.000000}%
\pgfsetfillcolor{currentfill}%
\pgfsetlinewidth{0.501875pt}%
\definecolor{currentstroke}{rgb}{0.000000,0.000000,0.000000}%
\pgfsetstrokecolor{currentstroke}%
\pgfsetdash{}{0pt}%
\pgfsys@defobject{currentmarker}{\pgfqpoint{0.000000in}{0.000000in}}{\pgfqpoint{0.020833in}{0.000000in}}{%
\pgfpathmoveto{\pgfqpoint{0.000000in}{0.000000in}}%
\pgfpathlineto{\pgfqpoint{0.020833in}{0.000000in}}%
\pgfusepath{stroke,fill}%
}%
\begin{pgfscope}%
\pgfsys@transformshift{3.350525in}{1.471903in}%
\pgfsys@useobject{currentmarker}{}%
\end{pgfscope}%
\end{pgfscope}%
\begin{pgfscope}%
\pgfsetbuttcap%
\pgfsetroundjoin%
\definecolor{currentfill}{rgb}{0.000000,0.000000,0.000000}%
\pgfsetfillcolor{currentfill}%
\pgfsetlinewidth{0.501875pt}%
\definecolor{currentstroke}{rgb}{0.000000,0.000000,0.000000}%
\pgfsetstrokecolor{currentstroke}%
\pgfsetdash{}{0pt}%
\pgfsys@defobject{currentmarker}{\pgfqpoint{-0.020833in}{0.000000in}}{\pgfqpoint{-0.000000in}{0.000000in}}{%
\pgfpathmoveto{\pgfqpoint{-0.000000in}{0.000000in}}%
\pgfpathlineto{\pgfqpoint{-0.020833in}{0.000000in}}%
\pgfusepath{stroke,fill}%
}%
\begin{pgfscope}%
\pgfsys@transformshift{5.591997in}{1.471903in}%
\pgfsys@useobject{currentmarker}{}%
\end{pgfscope}%
\end{pgfscope}%
\begin{pgfscope}%
\pgfsetbuttcap%
\pgfsetroundjoin%
\definecolor{currentfill}{rgb}{0.000000,0.000000,0.000000}%
\pgfsetfillcolor{currentfill}%
\pgfsetlinewidth{0.501875pt}%
\definecolor{currentstroke}{rgb}{0.000000,0.000000,0.000000}%
\pgfsetstrokecolor{currentstroke}%
\pgfsetdash{}{0pt}%
\pgfsys@defobject{currentmarker}{\pgfqpoint{0.000000in}{0.000000in}}{\pgfqpoint{0.020833in}{0.000000in}}{%
\pgfpathmoveto{\pgfqpoint{0.000000in}{0.000000in}}%
\pgfpathlineto{\pgfqpoint{0.020833in}{0.000000in}}%
\pgfusepath{stroke,fill}%
}%
\begin{pgfscope}%
\pgfsys@transformshift{3.350525in}{1.560610in}%
\pgfsys@useobject{currentmarker}{}%
\end{pgfscope}%
\end{pgfscope}%
\begin{pgfscope}%
\pgfsetbuttcap%
\pgfsetroundjoin%
\definecolor{currentfill}{rgb}{0.000000,0.000000,0.000000}%
\pgfsetfillcolor{currentfill}%
\pgfsetlinewidth{0.501875pt}%
\definecolor{currentstroke}{rgb}{0.000000,0.000000,0.000000}%
\pgfsetstrokecolor{currentstroke}%
\pgfsetdash{}{0pt}%
\pgfsys@defobject{currentmarker}{\pgfqpoint{-0.020833in}{0.000000in}}{\pgfqpoint{-0.000000in}{0.000000in}}{%
\pgfpathmoveto{\pgfqpoint{-0.000000in}{0.000000in}}%
\pgfpathlineto{\pgfqpoint{-0.020833in}{0.000000in}}%
\pgfusepath{stroke,fill}%
}%
\begin{pgfscope}%
\pgfsys@transformshift{5.591997in}{1.560610in}%
\pgfsys@useobject{currentmarker}{}%
\end{pgfscope}%
\end{pgfscope}%
\begin{pgfscope}%
\pgfsetbuttcap%
\pgfsetroundjoin%
\definecolor{currentfill}{rgb}{0.000000,0.000000,0.000000}%
\pgfsetfillcolor{currentfill}%
\pgfsetlinewidth{0.501875pt}%
\definecolor{currentstroke}{rgb}{0.000000,0.000000,0.000000}%
\pgfsetstrokecolor{currentstroke}%
\pgfsetdash{}{0pt}%
\pgfsys@defobject{currentmarker}{\pgfqpoint{0.000000in}{0.000000in}}{\pgfqpoint{0.020833in}{0.000000in}}{%
\pgfpathmoveto{\pgfqpoint{0.000000in}{0.000000in}}%
\pgfpathlineto{\pgfqpoint{0.020833in}{0.000000in}}%
\pgfusepath{stroke,fill}%
}%
\begin{pgfscope}%
\pgfsys@transformshift{3.350525in}{1.649317in}%
\pgfsys@useobject{currentmarker}{}%
\end{pgfscope}%
\end{pgfscope}%
\begin{pgfscope}%
\pgfsetbuttcap%
\pgfsetroundjoin%
\definecolor{currentfill}{rgb}{0.000000,0.000000,0.000000}%
\pgfsetfillcolor{currentfill}%
\pgfsetlinewidth{0.501875pt}%
\definecolor{currentstroke}{rgb}{0.000000,0.000000,0.000000}%
\pgfsetstrokecolor{currentstroke}%
\pgfsetdash{}{0pt}%
\pgfsys@defobject{currentmarker}{\pgfqpoint{-0.020833in}{0.000000in}}{\pgfqpoint{-0.000000in}{0.000000in}}{%
\pgfpathmoveto{\pgfqpoint{-0.000000in}{0.000000in}}%
\pgfpathlineto{\pgfqpoint{-0.020833in}{0.000000in}}%
\pgfusepath{stroke,fill}%
}%
\begin{pgfscope}%
\pgfsys@transformshift{5.591997in}{1.649317in}%
\pgfsys@useobject{currentmarker}{}%
\end{pgfscope}%
\end{pgfscope}%
\begin{pgfscope}%
\pgfsetbuttcap%
\pgfsetroundjoin%
\definecolor{currentfill}{rgb}{0.000000,0.000000,0.000000}%
\pgfsetfillcolor{currentfill}%
\pgfsetlinewidth{0.501875pt}%
\definecolor{currentstroke}{rgb}{0.000000,0.000000,0.000000}%
\pgfsetstrokecolor{currentstroke}%
\pgfsetdash{}{0pt}%
\pgfsys@defobject{currentmarker}{\pgfqpoint{0.000000in}{0.000000in}}{\pgfqpoint{0.020833in}{0.000000in}}{%
\pgfpathmoveto{\pgfqpoint{0.000000in}{0.000000in}}%
\pgfpathlineto{\pgfqpoint{0.020833in}{0.000000in}}%
\pgfusepath{stroke,fill}%
}%
\begin{pgfscope}%
\pgfsys@transformshift{3.350525in}{1.738025in}%
\pgfsys@useobject{currentmarker}{}%
\end{pgfscope}%
\end{pgfscope}%
\begin{pgfscope}%
\pgfsetbuttcap%
\pgfsetroundjoin%
\definecolor{currentfill}{rgb}{0.000000,0.000000,0.000000}%
\pgfsetfillcolor{currentfill}%
\pgfsetlinewidth{0.501875pt}%
\definecolor{currentstroke}{rgb}{0.000000,0.000000,0.000000}%
\pgfsetstrokecolor{currentstroke}%
\pgfsetdash{}{0pt}%
\pgfsys@defobject{currentmarker}{\pgfqpoint{-0.020833in}{0.000000in}}{\pgfqpoint{-0.000000in}{0.000000in}}{%
\pgfpathmoveto{\pgfqpoint{-0.000000in}{0.000000in}}%
\pgfpathlineto{\pgfqpoint{-0.020833in}{0.000000in}}%
\pgfusepath{stroke,fill}%
}%
\begin{pgfscope}%
\pgfsys@transformshift{5.591997in}{1.738025in}%
\pgfsys@useobject{currentmarker}{}%
\end{pgfscope}%
\end{pgfscope}%
\begin{pgfscope}%
\pgfsetbuttcap%
\pgfsetroundjoin%
\definecolor{currentfill}{rgb}{0.000000,0.000000,0.000000}%
\pgfsetfillcolor{currentfill}%
\pgfsetlinewidth{0.501875pt}%
\definecolor{currentstroke}{rgb}{0.000000,0.000000,0.000000}%
\pgfsetstrokecolor{currentstroke}%
\pgfsetdash{}{0pt}%
\pgfsys@defobject{currentmarker}{\pgfqpoint{0.000000in}{0.000000in}}{\pgfqpoint{0.020833in}{0.000000in}}{%
\pgfpathmoveto{\pgfqpoint{0.000000in}{0.000000in}}%
\pgfpathlineto{\pgfqpoint{0.020833in}{0.000000in}}%
\pgfusepath{stroke,fill}%
}%
\begin{pgfscope}%
\pgfsys@transformshift{3.350525in}{1.915439in}%
\pgfsys@useobject{currentmarker}{}%
\end{pgfscope}%
\end{pgfscope}%
\begin{pgfscope}%
\pgfsetbuttcap%
\pgfsetroundjoin%
\definecolor{currentfill}{rgb}{0.000000,0.000000,0.000000}%
\pgfsetfillcolor{currentfill}%
\pgfsetlinewidth{0.501875pt}%
\definecolor{currentstroke}{rgb}{0.000000,0.000000,0.000000}%
\pgfsetstrokecolor{currentstroke}%
\pgfsetdash{}{0pt}%
\pgfsys@defobject{currentmarker}{\pgfqpoint{-0.020833in}{0.000000in}}{\pgfqpoint{-0.000000in}{0.000000in}}{%
\pgfpathmoveto{\pgfqpoint{-0.000000in}{0.000000in}}%
\pgfpathlineto{\pgfqpoint{-0.020833in}{0.000000in}}%
\pgfusepath{stroke,fill}%
}%
\begin{pgfscope}%
\pgfsys@transformshift{5.591997in}{1.915439in}%
\pgfsys@useobject{currentmarker}{}%
\end{pgfscope}%
\end{pgfscope}%
\begin{pgfscope}%
\pgfsetbuttcap%
\pgfsetroundjoin%
\definecolor{currentfill}{rgb}{0.000000,0.000000,0.000000}%
\pgfsetfillcolor{currentfill}%
\pgfsetlinewidth{0.501875pt}%
\definecolor{currentstroke}{rgb}{0.000000,0.000000,0.000000}%
\pgfsetstrokecolor{currentstroke}%
\pgfsetdash{}{0pt}%
\pgfsys@defobject{currentmarker}{\pgfqpoint{0.000000in}{0.000000in}}{\pgfqpoint{0.020833in}{0.000000in}}{%
\pgfpathmoveto{\pgfqpoint{0.000000in}{0.000000in}}%
\pgfpathlineto{\pgfqpoint{0.020833in}{0.000000in}}%
\pgfusepath{stroke,fill}%
}%
\begin{pgfscope}%
\pgfsys@transformshift{3.350525in}{2.004146in}%
\pgfsys@useobject{currentmarker}{}%
\end{pgfscope}%
\end{pgfscope}%
\begin{pgfscope}%
\pgfsetbuttcap%
\pgfsetroundjoin%
\definecolor{currentfill}{rgb}{0.000000,0.000000,0.000000}%
\pgfsetfillcolor{currentfill}%
\pgfsetlinewidth{0.501875pt}%
\definecolor{currentstroke}{rgb}{0.000000,0.000000,0.000000}%
\pgfsetstrokecolor{currentstroke}%
\pgfsetdash{}{0pt}%
\pgfsys@defobject{currentmarker}{\pgfqpoint{-0.020833in}{0.000000in}}{\pgfqpoint{-0.000000in}{0.000000in}}{%
\pgfpathmoveto{\pgfqpoint{-0.000000in}{0.000000in}}%
\pgfpathlineto{\pgfqpoint{-0.020833in}{0.000000in}}%
\pgfusepath{stroke,fill}%
}%
\begin{pgfscope}%
\pgfsys@transformshift{5.591997in}{2.004146in}%
\pgfsys@useobject{currentmarker}{}%
\end{pgfscope}%
\end{pgfscope}%
\begin{pgfscope}%
\pgfsetbuttcap%
\pgfsetroundjoin%
\definecolor{currentfill}{rgb}{0.000000,0.000000,0.000000}%
\pgfsetfillcolor{currentfill}%
\pgfsetlinewidth{0.501875pt}%
\definecolor{currentstroke}{rgb}{0.000000,0.000000,0.000000}%
\pgfsetstrokecolor{currentstroke}%
\pgfsetdash{}{0pt}%
\pgfsys@defobject{currentmarker}{\pgfqpoint{0.000000in}{0.000000in}}{\pgfqpoint{0.020833in}{0.000000in}}{%
\pgfpathmoveto{\pgfqpoint{0.000000in}{0.000000in}}%
\pgfpathlineto{\pgfqpoint{0.020833in}{0.000000in}}%
\pgfusepath{stroke,fill}%
}%
\begin{pgfscope}%
\pgfsys@transformshift{3.350525in}{2.092853in}%
\pgfsys@useobject{currentmarker}{}%
\end{pgfscope}%
\end{pgfscope}%
\begin{pgfscope}%
\pgfsetbuttcap%
\pgfsetroundjoin%
\definecolor{currentfill}{rgb}{0.000000,0.000000,0.000000}%
\pgfsetfillcolor{currentfill}%
\pgfsetlinewidth{0.501875pt}%
\definecolor{currentstroke}{rgb}{0.000000,0.000000,0.000000}%
\pgfsetstrokecolor{currentstroke}%
\pgfsetdash{}{0pt}%
\pgfsys@defobject{currentmarker}{\pgfqpoint{-0.020833in}{0.000000in}}{\pgfqpoint{-0.000000in}{0.000000in}}{%
\pgfpathmoveto{\pgfqpoint{-0.000000in}{0.000000in}}%
\pgfpathlineto{\pgfqpoint{-0.020833in}{0.000000in}}%
\pgfusepath{stroke,fill}%
}%
\begin{pgfscope}%
\pgfsys@transformshift{5.591997in}{2.092853in}%
\pgfsys@useobject{currentmarker}{}%
\end{pgfscope}%
\end{pgfscope}%
\begin{pgfscope}%
\pgfsetbuttcap%
\pgfsetroundjoin%
\definecolor{currentfill}{rgb}{0.000000,0.000000,0.000000}%
\pgfsetfillcolor{currentfill}%
\pgfsetlinewidth{0.501875pt}%
\definecolor{currentstroke}{rgb}{0.000000,0.000000,0.000000}%
\pgfsetstrokecolor{currentstroke}%
\pgfsetdash{}{0pt}%
\pgfsys@defobject{currentmarker}{\pgfqpoint{0.000000in}{0.000000in}}{\pgfqpoint{0.020833in}{0.000000in}}{%
\pgfpathmoveto{\pgfqpoint{0.000000in}{0.000000in}}%
\pgfpathlineto{\pgfqpoint{0.020833in}{0.000000in}}%
\pgfusepath{stroke,fill}%
}%
\begin{pgfscope}%
\pgfsys@transformshift{3.350525in}{2.181561in}%
\pgfsys@useobject{currentmarker}{}%
\end{pgfscope}%
\end{pgfscope}%
\begin{pgfscope}%
\pgfsetbuttcap%
\pgfsetroundjoin%
\definecolor{currentfill}{rgb}{0.000000,0.000000,0.000000}%
\pgfsetfillcolor{currentfill}%
\pgfsetlinewidth{0.501875pt}%
\definecolor{currentstroke}{rgb}{0.000000,0.000000,0.000000}%
\pgfsetstrokecolor{currentstroke}%
\pgfsetdash{}{0pt}%
\pgfsys@defobject{currentmarker}{\pgfqpoint{-0.020833in}{0.000000in}}{\pgfqpoint{-0.000000in}{0.000000in}}{%
\pgfpathmoveto{\pgfqpoint{-0.000000in}{0.000000in}}%
\pgfpathlineto{\pgfqpoint{-0.020833in}{0.000000in}}%
\pgfusepath{stroke,fill}%
}%
\begin{pgfscope}%
\pgfsys@transformshift{5.591997in}{2.181561in}%
\pgfsys@useobject{currentmarker}{}%
\end{pgfscope}%
\end{pgfscope}%
\begin{pgfscope}%
\pgfsetbuttcap%
\pgfsetroundjoin%
\definecolor{currentfill}{rgb}{0.000000,0.000000,0.000000}%
\pgfsetfillcolor{currentfill}%
\pgfsetlinewidth{0.501875pt}%
\definecolor{currentstroke}{rgb}{0.000000,0.000000,0.000000}%
\pgfsetstrokecolor{currentstroke}%
\pgfsetdash{}{0pt}%
\pgfsys@defobject{currentmarker}{\pgfqpoint{0.000000in}{0.000000in}}{\pgfqpoint{0.020833in}{0.000000in}}{%
\pgfpathmoveto{\pgfqpoint{0.000000in}{0.000000in}}%
\pgfpathlineto{\pgfqpoint{0.020833in}{0.000000in}}%
\pgfusepath{stroke,fill}%
}%
\begin{pgfscope}%
\pgfsys@transformshift{3.350525in}{2.358975in}%
\pgfsys@useobject{currentmarker}{}%
\end{pgfscope}%
\end{pgfscope}%
\begin{pgfscope}%
\pgfsetbuttcap%
\pgfsetroundjoin%
\definecolor{currentfill}{rgb}{0.000000,0.000000,0.000000}%
\pgfsetfillcolor{currentfill}%
\pgfsetlinewidth{0.501875pt}%
\definecolor{currentstroke}{rgb}{0.000000,0.000000,0.000000}%
\pgfsetstrokecolor{currentstroke}%
\pgfsetdash{}{0pt}%
\pgfsys@defobject{currentmarker}{\pgfqpoint{-0.020833in}{0.000000in}}{\pgfqpoint{-0.000000in}{0.000000in}}{%
\pgfpathmoveto{\pgfqpoint{-0.000000in}{0.000000in}}%
\pgfpathlineto{\pgfqpoint{-0.020833in}{0.000000in}}%
\pgfusepath{stroke,fill}%
}%
\begin{pgfscope}%
\pgfsys@transformshift{5.591997in}{2.358975in}%
\pgfsys@useobject{currentmarker}{}%
\end{pgfscope}%
\end{pgfscope}%
\begin{pgfscope}%
\pgfsetbuttcap%
\pgfsetroundjoin%
\definecolor{currentfill}{rgb}{0.000000,0.000000,0.000000}%
\pgfsetfillcolor{currentfill}%
\pgfsetlinewidth{0.501875pt}%
\definecolor{currentstroke}{rgb}{0.000000,0.000000,0.000000}%
\pgfsetstrokecolor{currentstroke}%
\pgfsetdash{}{0pt}%
\pgfsys@defobject{currentmarker}{\pgfqpoint{0.000000in}{0.000000in}}{\pgfqpoint{0.020833in}{0.000000in}}{%
\pgfpathmoveto{\pgfqpoint{0.000000in}{0.000000in}}%
\pgfpathlineto{\pgfqpoint{0.020833in}{0.000000in}}%
\pgfusepath{stroke,fill}%
}%
\begin{pgfscope}%
\pgfsys@transformshift{3.350525in}{2.447682in}%
\pgfsys@useobject{currentmarker}{}%
\end{pgfscope}%
\end{pgfscope}%
\begin{pgfscope}%
\pgfsetbuttcap%
\pgfsetroundjoin%
\definecolor{currentfill}{rgb}{0.000000,0.000000,0.000000}%
\pgfsetfillcolor{currentfill}%
\pgfsetlinewidth{0.501875pt}%
\definecolor{currentstroke}{rgb}{0.000000,0.000000,0.000000}%
\pgfsetstrokecolor{currentstroke}%
\pgfsetdash{}{0pt}%
\pgfsys@defobject{currentmarker}{\pgfqpoint{-0.020833in}{0.000000in}}{\pgfqpoint{-0.000000in}{0.000000in}}{%
\pgfpathmoveto{\pgfqpoint{-0.000000in}{0.000000in}}%
\pgfpathlineto{\pgfqpoint{-0.020833in}{0.000000in}}%
\pgfusepath{stroke,fill}%
}%
\begin{pgfscope}%
\pgfsys@transformshift{5.591997in}{2.447682in}%
\pgfsys@useobject{currentmarker}{}%
\end{pgfscope}%
\end{pgfscope}%
\begin{pgfscope}%
\pgfsetbuttcap%
\pgfsetroundjoin%
\definecolor{currentfill}{rgb}{0.000000,0.000000,0.000000}%
\pgfsetfillcolor{currentfill}%
\pgfsetlinewidth{0.501875pt}%
\definecolor{currentstroke}{rgb}{0.000000,0.000000,0.000000}%
\pgfsetstrokecolor{currentstroke}%
\pgfsetdash{}{0pt}%
\pgfsys@defobject{currentmarker}{\pgfqpoint{0.000000in}{0.000000in}}{\pgfqpoint{0.020833in}{0.000000in}}{%
\pgfpathmoveto{\pgfqpoint{0.000000in}{0.000000in}}%
\pgfpathlineto{\pgfqpoint{0.020833in}{0.000000in}}%
\pgfusepath{stroke,fill}%
}%
\begin{pgfscope}%
\pgfsys@transformshift{3.350525in}{2.536390in}%
\pgfsys@useobject{currentmarker}{}%
\end{pgfscope}%
\end{pgfscope}%
\begin{pgfscope}%
\pgfsetbuttcap%
\pgfsetroundjoin%
\definecolor{currentfill}{rgb}{0.000000,0.000000,0.000000}%
\pgfsetfillcolor{currentfill}%
\pgfsetlinewidth{0.501875pt}%
\definecolor{currentstroke}{rgb}{0.000000,0.000000,0.000000}%
\pgfsetstrokecolor{currentstroke}%
\pgfsetdash{}{0pt}%
\pgfsys@defobject{currentmarker}{\pgfqpoint{-0.020833in}{0.000000in}}{\pgfqpoint{-0.000000in}{0.000000in}}{%
\pgfpathmoveto{\pgfqpoint{-0.000000in}{0.000000in}}%
\pgfpathlineto{\pgfqpoint{-0.020833in}{0.000000in}}%
\pgfusepath{stroke,fill}%
}%
\begin{pgfscope}%
\pgfsys@transformshift{5.591997in}{2.536390in}%
\pgfsys@useobject{currentmarker}{}%
\end{pgfscope}%
\end{pgfscope}%
\begin{pgfscope}%
\pgfsetbuttcap%
\pgfsetroundjoin%
\definecolor{currentfill}{rgb}{0.000000,0.000000,0.000000}%
\pgfsetfillcolor{currentfill}%
\pgfsetlinewidth{0.501875pt}%
\definecolor{currentstroke}{rgb}{0.000000,0.000000,0.000000}%
\pgfsetstrokecolor{currentstroke}%
\pgfsetdash{}{0pt}%
\pgfsys@defobject{currentmarker}{\pgfqpoint{0.000000in}{0.000000in}}{\pgfqpoint{0.020833in}{0.000000in}}{%
\pgfpathmoveto{\pgfqpoint{0.000000in}{0.000000in}}%
\pgfpathlineto{\pgfqpoint{0.020833in}{0.000000in}}%
\pgfusepath{stroke,fill}%
}%
\begin{pgfscope}%
\pgfsys@transformshift{3.350525in}{2.625097in}%
\pgfsys@useobject{currentmarker}{}%
\end{pgfscope}%
\end{pgfscope}%
\begin{pgfscope}%
\pgfsetbuttcap%
\pgfsetroundjoin%
\definecolor{currentfill}{rgb}{0.000000,0.000000,0.000000}%
\pgfsetfillcolor{currentfill}%
\pgfsetlinewidth{0.501875pt}%
\definecolor{currentstroke}{rgb}{0.000000,0.000000,0.000000}%
\pgfsetstrokecolor{currentstroke}%
\pgfsetdash{}{0pt}%
\pgfsys@defobject{currentmarker}{\pgfqpoint{-0.020833in}{0.000000in}}{\pgfqpoint{-0.000000in}{0.000000in}}{%
\pgfpathmoveto{\pgfqpoint{-0.000000in}{0.000000in}}%
\pgfpathlineto{\pgfqpoint{-0.020833in}{0.000000in}}%
\pgfusepath{stroke,fill}%
}%
\begin{pgfscope}%
\pgfsys@transformshift{5.591997in}{2.625097in}%
\pgfsys@useobject{currentmarker}{}%
\end{pgfscope}%
\end{pgfscope}%
\begin{pgfscope}%
\pgfsetbuttcap%
\pgfsetroundjoin%
\definecolor{currentfill}{rgb}{0.000000,0.000000,0.000000}%
\pgfsetfillcolor{currentfill}%
\pgfsetlinewidth{0.501875pt}%
\definecolor{currentstroke}{rgb}{0.000000,0.000000,0.000000}%
\pgfsetstrokecolor{currentstroke}%
\pgfsetdash{}{0pt}%
\pgfsys@defobject{currentmarker}{\pgfqpoint{0.000000in}{0.000000in}}{\pgfqpoint{0.020833in}{0.000000in}}{%
\pgfpathmoveto{\pgfqpoint{0.000000in}{0.000000in}}%
\pgfpathlineto{\pgfqpoint{0.020833in}{0.000000in}}%
\pgfusepath{stroke,fill}%
}%
\begin{pgfscope}%
\pgfsys@transformshift{3.350525in}{2.802511in}%
\pgfsys@useobject{currentmarker}{}%
\end{pgfscope}%
\end{pgfscope}%
\begin{pgfscope}%
\pgfsetbuttcap%
\pgfsetroundjoin%
\definecolor{currentfill}{rgb}{0.000000,0.000000,0.000000}%
\pgfsetfillcolor{currentfill}%
\pgfsetlinewidth{0.501875pt}%
\definecolor{currentstroke}{rgb}{0.000000,0.000000,0.000000}%
\pgfsetstrokecolor{currentstroke}%
\pgfsetdash{}{0pt}%
\pgfsys@defobject{currentmarker}{\pgfqpoint{-0.020833in}{0.000000in}}{\pgfqpoint{-0.000000in}{0.000000in}}{%
\pgfpathmoveto{\pgfqpoint{-0.000000in}{0.000000in}}%
\pgfpathlineto{\pgfqpoint{-0.020833in}{0.000000in}}%
\pgfusepath{stroke,fill}%
}%
\begin{pgfscope}%
\pgfsys@transformshift{5.591997in}{2.802511in}%
\pgfsys@useobject{currentmarker}{}%
\end{pgfscope}%
\end{pgfscope}%
\begin{pgfscope}%
\pgfsetbuttcap%
\pgfsetroundjoin%
\definecolor{currentfill}{rgb}{0.000000,0.000000,0.000000}%
\pgfsetfillcolor{currentfill}%
\pgfsetlinewidth{0.501875pt}%
\definecolor{currentstroke}{rgb}{0.000000,0.000000,0.000000}%
\pgfsetstrokecolor{currentstroke}%
\pgfsetdash{}{0pt}%
\pgfsys@defobject{currentmarker}{\pgfqpoint{0.000000in}{0.000000in}}{\pgfqpoint{0.020833in}{0.000000in}}{%
\pgfpathmoveto{\pgfqpoint{0.000000in}{0.000000in}}%
\pgfpathlineto{\pgfqpoint{0.020833in}{0.000000in}}%
\pgfusepath{stroke,fill}%
}%
\begin{pgfscope}%
\pgfsys@transformshift{3.350525in}{2.891218in}%
\pgfsys@useobject{currentmarker}{}%
\end{pgfscope}%
\end{pgfscope}%
\begin{pgfscope}%
\pgfsetbuttcap%
\pgfsetroundjoin%
\definecolor{currentfill}{rgb}{0.000000,0.000000,0.000000}%
\pgfsetfillcolor{currentfill}%
\pgfsetlinewidth{0.501875pt}%
\definecolor{currentstroke}{rgb}{0.000000,0.000000,0.000000}%
\pgfsetstrokecolor{currentstroke}%
\pgfsetdash{}{0pt}%
\pgfsys@defobject{currentmarker}{\pgfqpoint{-0.020833in}{0.000000in}}{\pgfqpoint{-0.000000in}{0.000000in}}{%
\pgfpathmoveto{\pgfqpoint{-0.000000in}{0.000000in}}%
\pgfpathlineto{\pgfqpoint{-0.020833in}{0.000000in}}%
\pgfusepath{stroke,fill}%
}%
\begin{pgfscope}%
\pgfsys@transformshift{5.591997in}{2.891218in}%
\pgfsys@useobject{currentmarker}{}%
\end{pgfscope}%
\end{pgfscope}%
\begin{pgfscope}%
\pgfsetbuttcap%
\pgfsetroundjoin%
\definecolor{currentfill}{rgb}{0.000000,0.000000,0.000000}%
\pgfsetfillcolor{currentfill}%
\pgfsetlinewidth{0.501875pt}%
\definecolor{currentstroke}{rgb}{0.000000,0.000000,0.000000}%
\pgfsetstrokecolor{currentstroke}%
\pgfsetdash{}{0pt}%
\pgfsys@defobject{currentmarker}{\pgfqpoint{0.000000in}{0.000000in}}{\pgfqpoint{0.020833in}{0.000000in}}{%
\pgfpathmoveto{\pgfqpoint{0.000000in}{0.000000in}}%
\pgfpathlineto{\pgfqpoint{0.020833in}{0.000000in}}%
\pgfusepath{stroke,fill}%
}%
\begin{pgfscope}%
\pgfsys@transformshift{3.350525in}{2.979926in}%
\pgfsys@useobject{currentmarker}{}%
\end{pgfscope}%
\end{pgfscope}%
\begin{pgfscope}%
\pgfsetbuttcap%
\pgfsetroundjoin%
\definecolor{currentfill}{rgb}{0.000000,0.000000,0.000000}%
\pgfsetfillcolor{currentfill}%
\pgfsetlinewidth{0.501875pt}%
\definecolor{currentstroke}{rgb}{0.000000,0.000000,0.000000}%
\pgfsetstrokecolor{currentstroke}%
\pgfsetdash{}{0pt}%
\pgfsys@defobject{currentmarker}{\pgfqpoint{-0.020833in}{0.000000in}}{\pgfqpoint{-0.000000in}{0.000000in}}{%
\pgfpathmoveto{\pgfqpoint{-0.000000in}{0.000000in}}%
\pgfpathlineto{\pgfqpoint{-0.020833in}{0.000000in}}%
\pgfusepath{stroke,fill}%
}%
\begin{pgfscope}%
\pgfsys@transformshift{5.591997in}{2.979926in}%
\pgfsys@useobject{currentmarker}{}%
\end{pgfscope}%
\end{pgfscope}%
\begin{pgfscope}%
\pgfsetbuttcap%
\pgfsetroundjoin%
\definecolor{currentfill}{rgb}{0.000000,0.000000,0.000000}%
\pgfsetfillcolor{currentfill}%
\pgfsetlinewidth{0.501875pt}%
\definecolor{currentstroke}{rgb}{0.000000,0.000000,0.000000}%
\pgfsetstrokecolor{currentstroke}%
\pgfsetdash{}{0pt}%
\pgfsys@defobject{currentmarker}{\pgfqpoint{0.000000in}{0.000000in}}{\pgfqpoint{0.020833in}{0.000000in}}{%
\pgfpathmoveto{\pgfqpoint{0.000000in}{0.000000in}}%
\pgfpathlineto{\pgfqpoint{0.020833in}{0.000000in}}%
\pgfusepath{stroke,fill}%
}%
\begin{pgfscope}%
\pgfsys@transformshift{3.350525in}{3.068633in}%
\pgfsys@useobject{currentmarker}{}%
\end{pgfscope}%
\end{pgfscope}%
\begin{pgfscope}%
\pgfsetbuttcap%
\pgfsetroundjoin%
\definecolor{currentfill}{rgb}{0.000000,0.000000,0.000000}%
\pgfsetfillcolor{currentfill}%
\pgfsetlinewidth{0.501875pt}%
\definecolor{currentstroke}{rgb}{0.000000,0.000000,0.000000}%
\pgfsetstrokecolor{currentstroke}%
\pgfsetdash{}{0pt}%
\pgfsys@defobject{currentmarker}{\pgfqpoint{-0.020833in}{0.000000in}}{\pgfqpoint{-0.000000in}{0.000000in}}{%
\pgfpathmoveto{\pgfqpoint{-0.000000in}{0.000000in}}%
\pgfpathlineto{\pgfqpoint{-0.020833in}{0.000000in}}%
\pgfusepath{stroke,fill}%
}%
\begin{pgfscope}%
\pgfsys@transformshift{5.591997in}{3.068633in}%
\pgfsys@useobject{currentmarker}{}%
\end{pgfscope}%
\end{pgfscope}%
\begin{pgfscope}%
\pgfsetbuttcap%
\pgfsetroundjoin%
\definecolor{currentfill}{rgb}{0.000000,0.000000,0.000000}%
\pgfsetfillcolor{currentfill}%
\pgfsetlinewidth{0.501875pt}%
\definecolor{currentstroke}{rgb}{0.000000,0.000000,0.000000}%
\pgfsetstrokecolor{currentstroke}%
\pgfsetdash{}{0pt}%
\pgfsys@defobject{currentmarker}{\pgfqpoint{0.000000in}{0.000000in}}{\pgfqpoint{0.020833in}{0.000000in}}{%
\pgfpathmoveto{\pgfqpoint{0.000000in}{0.000000in}}%
\pgfpathlineto{\pgfqpoint{0.020833in}{0.000000in}}%
\pgfusepath{stroke,fill}%
}%
\begin{pgfscope}%
\pgfsys@transformshift{3.350525in}{3.246047in}%
\pgfsys@useobject{currentmarker}{}%
\end{pgfscope}%
\end{pgfscope}%
\begin{pgfscope}%
\pgfsetbuttcap%
\pgfsetroundjoin%
\definecolor{currentfill}{rgb}{0.000000,0.000000,0.000000}%
\pgfsetfillcolor{currentfill}%
\pgfsetlinewidth{0.501875pt}%
\definecolor{currentstroke}{rgb}{0.000000,0.000000,0.000000}%
\pgfsetstrokecolor{currentstroke}%
\pgfsetdash{}{0pt}%
\pgfsys@defobject{currentmarker}{\pgfqpoint{-0.020833in}{0.000000in}}{\pgfqpoint{-0.000000in}{0.000000in}}{%
\pgfpathmoveto{\pgfqpoint{-0.000000in}{0.000000in}}%
\pgfpathlineto{\pgfqpoint{-0.020833in}{0.000000in}}%
\pgfusepath{stroke,fill}%
}%
\begin{pgfscope}%
\pgfsys@transformshift{5.591997in}{3.246047in}%
\pgfsys@useobject{currentmarker}{}%
\end{pgfscope}%
\end{pgfscope}%
\begin{pgfscope}%
\pgfsetbuttcap%
\pgfsetroundjoin%
\definecolor{currentfill}{rgb}{0.000000,0.000000,0.000000}%
\pgfsetfillcolor{currentfill}%
\pgfsetlinewidth{0.501875pt}%
\definecolor{currentstroke}{rgb}{0.000000,0.000000,0.000000}%
\pgfsetstrokecolor{currentstroke}%
\pgfsetdash{}{0pt}%
\pgfsys@defobject{currentmarker}{\pgfqpoint{0.000000in}{0.000000in}}{\pgfqpoint{0.020833in}{0.000000in}}{%
\pgfpathmoveto{\pgfqpoint{0.000000in}{0.000000in}}%
\pgfpathlineto{\pgfqpoint{0.020833in}{0.000000in}}%
\pgfusepath{stroke,fill}%
}%
\begin{pgfscope}%
\pgfsys@transformshift{3.350525in}{3.334755in}%
\pgfsys@useobject{currentmarker}{}%
\end{pgfscope}%
\end{pgfscope}%
\begin{pgfscope}%
\pgfsetbuttcap%
\pgfsetroundjoin%
\definecolor{currentfill}{rgb}{0.000000,0.000000,0.000000}%
\pgfsetfillcolor{currentfill}%
\pgfsetlinewidth{0.501875pt}%
\definecolor{currentstroke}{rgb}{0.000000,0.000000,0.000000}%
\pgfsetstrokecolor{currentstroke}%
\pgfsetdash{}{0pt}%
\pgfsys@defobject{currentmarker}{\pgfqpoint{-0.020833in}{0.000000in}}{\pgfqpoint{-0.000000in}{0.000000in}}{%
\pgfpathmoveto{\pgfqpoint{-0.000000in}{0.000000in}}%
\pgfpathlineto{\pgfqpoint{-0.020833in}{0.000000in}}%
\pgfusepath{stroke,fill}%
}%
\begin{pgfscope}%
\pgfsys@transformshift{5.591997in}{3.334755in}%
\pgfsys@useobject{currentmarker}{}%
\end{pgfscope}%
\end{pgfscope}%
\begin{pgfscope}%
\pgfsetbuttcap%
\pgfsetroundjoin%
\definecolor{currentfill}{rgb}{0.000000,0.000000,0.000000}%
\pgfsetfillcolor{currentfill}%
\pgfsetlinewidth{0.501875pt}%
\definecolor{currentstroke}{rgb}{0.000000,0.000000,0.000000}%
\pgfsetstrokecolor{currentstroke}%
\pgfsetdash{}{0pt}%
\pgfsys@defobject{currentmarker}{\pgfqpoint{0.000000in}{0.000000in}}{\pgfqpoint{0.020833in}{0.000000in}}{%
\pgfpathmoveto{\pgfqpoint{0.000000in}{0.000000in}}%
\pgfpathlineto{\pgfqpoint{0.020833in}{0.000000in}}%
\pgfusepath{stroke,fill}%
}%
\begin{pgfscope}%
\pgfsys@transformshift{3.350525in}{3.423462in}%
\pgfsys@useobject{currentmarker}{}%
\end{pgfscope}%
\end{pgfscope}%
\begin{pgfscope}%
\pgfsetbuttcap%
\pgfsetroundjoin%
\definecolor{currentfill}{rgb}{0.000000,0.000000,0.000000}%
\pgfsetfillcolor{currentfill}%
\pgfsetlinewidth{0.501875pt}%
\definecolor{currentstroke}{rgb}{0.000000,0.000000,0.000000}%
\pgfsetstrokecolor{currentstroke}%
\pgfsetdash{}{0pt}%
\pgfsys@defobject{currentmarker}{\pgfqpoint{-0.020833in}{0.000000in}}{\pgfqpoint{-0.000000in}{0.000000in}}{%
\pgfpathmoveto{\pgfqpoint{-0.000000in}{0.000000in}}%
\pgfpathlineto{\pgfqpoint{-0.020833in}{0.000000in}}%
\pgfusepath{stroke,fill}%
}%
\begin{pgfscope}%
\pgfsys@transformshift{5.591997in}{3.423462in}%
\pgfsys@useobject{currentmarker}{}%
\end{pgfscope}%
\end{pgfscope}%
\begin{pgfscope}%
\pgfsetbuttcap%
\pgfsetroundjoin%
\definecolor{currentfill}{rgb}{0.000000,0.000000,0.000000}%
\pgfsetfillcolor{currentfill}%
\pgfsetlinewidth{0.501875pt}%
\definecolor{currentstroke}{rgb}{0.000000,0.000000,0.000000}%
\pgfsetstrokecolor{currentstroke}%
\pgfsetdash{}{0pt}%
\pgfsys@defobject{currentmarker}{\pgfqpoint{0.000000in}{0.000000in}}{\pgfqpoint{0.020833in}{0.000000in}}{%
\pgfpathmoveto{\pgfqpoint{0.000000in}{0.000000in}}%
\pgfpathlineto{\pgfqpoint{0.020833in}{0.000000in}}%
\pgfusepath{stroke,fill}%
}%
\begin{pgfscope}%
\pgfsys@transformshift{3.350525in}{3.512169in}%
\pgfsys@useobject{currentmarker}{}%
\end{pgfscope}%
\end{pgfscope}%
\begin{pgfscope}%
\pgfsetbuttcap%
\pgfsetroundjoin%
\definecolor{currentfill}{rgb}{0.000000,0.000000,0.000000}%
\pgfsetfillcolor{currentfill}%
\pgfsetlinewidth{0.501875pt}%
\definecolor{currentstroke}{rgb}{0.000000,0.000000,0.000000}%
\pgfsetstrokecolor{currentstroke}%
\pgfsetdash{}{0pt}%
\pgfsys@defobject{currentmarker}{\pgfqpoint{-0.020833in}{0.000000in}}{\pgfqpoint{-0.000000in}{0.000000in}}{%
\pgfpathmoveto{\pgfqpoint{-0.000000in}{0.000000in}}%
\pgfpathlineto{\pgfqpoint{-0.020833in}{0.000000in}}%
\pgfusepath{stroke,fill}%
}%
\begin{pgfscope}%
\pgfsys@transformshift{5.591997in}{3.512169in}%
\pgfsys@useobject{currentmarker}{}%
\end{pgfscope}%
\end{pgfscope}%
\begin{pgfscope}%
\pgfsetbuttcap%
\pgfsetroundjoin%
\definecolor{currentfill}{rgb}{0.000000,0.000000,0.000000}%
\pgfsetfillcolor{currentfill}%
\pgfsetlinewidth{0.501875pt}%
\definecolor{currentstroke}{rgb}{0.000000,0.000000,0.000000}%
\pgfsetstrokecolor{currentstroke}%
\pgfsetdash{}{0pt}%
\pgfsys@defobject{currentmarker}{\pgfqpoint{0.000000in}{0.000000in}}{\pgfqpoint{0.020833in}{0.000000in}}{%
\pgfpathmoveto{\pgfqpoint{0.000000in}{0.000000in}}%
\pgfpathlineto{\pgfqpoint{0.020833in}{0.000000in}}%
\pgfusepath{stroke,fill}%
}%
\begin{pgfscope}%
\pgfsys@transformshift{3.350525in}{3.689583in}%
\pgfsys@useobject{currentmarker}{}%
\end{pgfscope}%
\end{pgfscope}%
\begin{pgfscope}%
\pgfsetbuttcap%
\pgfsetroundjoin%
\definecolor{currentfill}{rgb}{0.000000,0.000000,0.000000}%
\pgfsetfillcolor{currentfill}%
\pgfsetlinewidth{0.501875pt}%
\definecolor{currentstroke}{rgb}{0.000000,0.000000,0.000000}%
\pgfsetstrokecolor{currentstroke}%
\pgfsetdash{}{0pt}%
\pgfsys@defobject{currentmarker}{\pgfqpoint{-0.020833in}{0.000000in}}{\pgfqpoint{-0.000000in}{0.000000in}}{%
\pgfpathmoveto{\pgfqpoint{-0.000000in}{0.000000in}}%
\pgfpathlineto{\pgfqpoint{-0.020833in}{0.000000in}}%
\pgfusepath{stroke,fill}%
}%
\begin{pgfscope}%
\pgfsys@transformshift{5.591997in}{3.689583in}%
\pgfsys@useobject{currentmarker}{}%
\end{pgfscope}%
\end{pgfscope}%
\begin{pgfscope}%
\pgfsetbuttcap%
\pgfsetroundjoin%
\definecolor{currentfill}{rgb}{0.000000,0.000000,0.000000}%
\pgfsetfillcolor{currentfill}%
\pgfsetlinewidth{0.501875pt}%
\definecolor{currentstroke}{rgb}{0.000000,0.000000,0.000000}%
\pgfsetstrokecolor{currentstroke}%
\pgfsetdash{}{0pt}%
\pgfsys@defobject{currentmarker}{\pgfqpoint{0.000000in}{0.000000in}}{\pgfqpoint{0.020833in}{0.000000in}}{%
\pgfpathmoveto{\pgfqpoint{0.000000in}{0.000000in}}%
\pgfpathlineto{\pgfqpoint{0.020833in}{0.000000in}}%
\pgfusepath{stroke,fill}%
}%
\begin{pgfscope}%
\pgfsys@transformshift{3.350525in}{3.778291in}%
\pgfsys@useobject{currentmarker}{}%
\end{pgfscope}%
\end{pgfscope}%
\begin{pgfscope}%
\pgfsetbuttcap%
\pgfsetroundjoin%
\definecolor{currentfill}{rgb}{0.000000,0.000000,0.000000}%
\pgfsetfillcolor{currentfill}%
\pgfsetlinewidth{0.501875pt}%
\definecolor{currentstroke}{rgb}{0.000000,0.000000,0.000000}%
\pgfsetstrokecolor{currentstroke}%
\pgfsetdash{}{0pt}%
\pgfsys@defobject{currentmarker}{\pgfqpoint{-0.020833in}{0.000000in}}{\pgfqpoint{-0.000000in}{0.000000in}}{%
\pgfpathmoveto{\pgfqpoint{-0.000000in}{0.000000in}}%
\pgfpathlineto{\pgfqpoint{-0.020833in}{0.000000in}}%
\pgfusepath{stroke,fill}%
}%
\begin{pgfscope}%
\pgfsys@transformshift{5.591997in}{3.778291in}%
\pgfsys@useobject{currentmarker}{}%
\end{pgfscope}%
\end{pgfscope}%
\begin{pgfscope}%
\pgfsetbuttcap%
\pgfsetroundjoin%
\definecolor{currentfill}{rgb}{0.000000,0.000000,0.000000}%
\pgfsetfillcolor{currentfill}%
\pgfsetlinewidth{0.501875pt}%
\definecolor{currentstroke}{rgb}{0.000000,0.000000,0.000000}%
\pgfsetstrokecolor{currentstroke}%
\pgfsetdash{}{0pt}%
\pgfsys@defobject{currentmarker}{\pgfqpoint{0.000000in}{0.000000in}}{\pgfqpoint{0.020833in}{0.000000in}}{%
\pgfpathmoveto{\pgfqpoint{0.000000in}{0.000000in}}%
\pgfpathlineto{\pgfqpoint{0.020833in}{0.000000in}}%
\pgfusepath{stroke,fill}%
}%
\begin{pgfscope}%
\pgfsys@transformshift{3.350525in}{3.866998in}%
\pgfsys@useobject{currentmarker}{}%
\end{pgfscope}%
\end{pgfscope}%
\begin{pgfscope}%
\pgfsetbuttcap%
\pgfsetroundjoin%
\definecolor{currentfill}{rgb}{0.000000,0.000000,0.000000}%
\pgfsetfillcolor{currentfill}%
\pgfsetlinewidth{0.501875pt}%
\definecolor{currentstroke}{rgb}{0.000000,0.000000,0.000000}%
\pgfsetstrokecolor{currentstroke}%
\pgfsetdash{}{0pt}%
\pgfsys@defobject{currentmarker}{\pgfqpoint{-0.020833in}{0.000000in}}{\pgfqpoint{-0.000000in}{0.000000in}}{%
\pgfpathmoveto{\pgfqpoint{-0.000000in}{0.000000in}}%
\pgfpathlineto{\pgfqpoint{-0.020833in}{0.000000in}}%
\pgfusepath{stroke,fill}%
}%
\begin{pgfscope}%
\pgfsys@transformshift{5.591997in}{3.866998in}%
\pgfsys@useobject{currentmarker}{}%
\end{pgfscope}%
\end{pgfscope}%
\begin{pgfscope}%
\pgfsetbuttcap%
\pgfsetroundjoin%
\definecolor{currentfill}{rgb}{0.000000,0.000000,0.000000}%
\pgfsetfillcolor{currentfill}%
\pgfsetlinewidth{0.501875pt}%
\definecolor{currentstroke}{rgb}{0.000000,0.000000,0.000000}%
\pgfsetstrokecolor{currentstroke}%
\pgfsetdash{}{0pt}%
\pgfsys@defobject{currentmarker}{\pgfqpoint{0.000000in}{0.000000in}}{\pgfqpoint{0.020833in}{0.000000in}}{%
\pgfpathmoveto{\pgfqpoint{0.000000in}{0.000000in}}%
\pgfpathlineto{\pgfqpoint{0.020833in}{0.000000in}}%
\pgfusepath{stroke,fill}%
}%
\begin{pgfscope}%
\pgfsys@transformshift{3.350525in}{3.955705in}%
\pgfsys@useobject{currentmarker}{}%
\end{pgfscope}%
\end{pgfscope}%
\begin{pgfscope}%
\pgfsetbuttcap%
\pgfsetroundjoin%
\definecolor{currentfill}{rgb}{0.000000,0.000000,0.000000}%
\pgfsetfillcolor{currentfill}%
\pgfsetlinewidth{0.501875pt}%
\definecolor{currentstroke}{rgb}{0.000000,0.000000,0.000000}%
\pgfsetstrokecolor{currentstroke}%
\pgfsetdash{}{0pt}%
\pgfsys@defobject{currentmarker}{\pgfqpoint{-0.020833in}{0.000000in}}{\pgfqpoint{-0.000000in}{0.000000in}}{%
\pgfpathmoveto{\pgfqpoint{-0.000000in}{0.000000in}}%
\pgfpathlineto{\pgfqpoint{-0.020833in}{0.000000in}}%
\pgfusepath{stroke,fill}%
}%
\begin{pgfscope}%
\pgfsys@transformshift{5.591997in}{3.955705in}%
\pgfsys@useobject{currentmarker}{}%
\end{pgfscope}%
\end{pgfscope}%
\begin{pgfscope}%
\pgfsetbuttcap%
\pgfsetroundjoin%
\definecolor{currentfill}{rgb}{0.000000,0.000000,0.000000}%
\pgfsetfillcolor{currentfill}%
\pgfsetlinewidth{0.501875pt}%
\definecolor{currentstroke}{rgb}{0.000000,0.000000,0.000000}%
\pgfsetstrokecolor{currentstroke}%
\pgfsetdash{}{0pt}%
\pgfsys@defobject{currentmarker}{\pgfqpoint{0.000000in}{0.000000in}}{\pgfqpoint{0.020833in}{0.000000in}}{%
\pgfpathmoveto{\pgfqpoint{0.000000in}{0.000000in}}%
\pgfpathlineto{\pgfqpoint{0.020833in}{0.000000in}}%
\pgfusepath{stroke,fill}%
}%
\begin{pgfscope}%
\pgfsys@transformshift{3.350525in}{4.133120in}%
\pgfsys@useobject{currentmarker}{}%
\end{pgfscope}%
\end{pgfscope}%
\begin{pgfscope}%
\pgfsetbuttcap%
\pgfsetroundjoin%
\definecolor{currentfill}{rgb}{0.000000,0.000000,0.000000}%
\pgfsetfillcolor{currentfill}%
\pgfsetlinewidth{0.501875pt}%
\definecolor{currentstroke}{rgb}{0.000000,0.000000,0.000000}%
\pgfsetstrokecolor{currentstroke}%
\pgfsetdash{}{0pt}%
\pgfsys@defobject{currentmarker}{\pgfqpoint{-0.020833in}{0.000000in}}{\pgfqpoint{-0.000000in}{0.000000in}}{%
\pgfpathmoveto{\pgfqpoint{-0.000000in}{0.000000in}}%
\pgfpathlineto{\pgfqpoint{-0.020833in}{0.000000in}}%
\pgfusepath{stroke,fill}%
}%
\begin{pgfscope}%
\pgfsys@transformshift{5.591997in}{4.133120in}%
\pgfsys@useobject{currentmarker}{}%
\end{pgfscope}%
\end{pgfscope}%
\begin{pgfscope}%
\pgfsetbuttcap%
\pgfsetroundjoin%
\definecolor{currentfill}{rgb}{0.000000,0.000000,0.000000}%
\pgfsetfillcolor{currentfill}%
\pgfsetlinewidth{0.501875pt}%
\definecolor{currentstroke}{rgb}{0.000000,0.000000,0.000000}%
\pgfsetstrokecolor{currentstroke}%
\pgfsetdash{}{0pt}%
\pgfsys@defobject{currentmarker}{\pgfqpoint{0.000000in}{0.000000in}}{\pgfqpoint{0.020833in}{0.000000in}}{%
\pgfpathmoveto{\pgfqpoint{0.000000in}{0.000000in}}%
\pgfpathlineto{\pgfqpoint{0.020833in}{0.000000in}}%
\pgfusepath{stroke,fill}%
}%
\begin{pgfscope}%
\pgfsys@transformshift{3.350525in}{4.221827in}%
\pgfsys@useobject{currentmarker}{}%
\end{pgfscope}%
\end{pgfscope}%
\begin{pgfscope}%
\pgfsetbuttcap%
\pgfsetroundjoin%
\definecolor{currentfill}{rgb}{0.000000,0.000000,0.000000}%
\pgfsetfillcolor{currentfill}%
\pgfsetlinewidth{0.501875pt}%
\definecolor{currentstroke}{rgb}{0.000000,0.000000,0.000000}%
\pgfsetstrokecolor{currentstroke}%
\pgfsetdash{}{0pt}%
\pgfsys@defobject{currentmarker}{\pgfqpoint{-0.020833in}{0.000000in}}{\pgfqpoint{-0.000000in}{0.000000in}}{%
\pgfpathmoveto{\pgfqpoint{-0.000000in}{0.000000in}}%
\pgfpathlineto{\pgfqpoint{-0.020833in}{0.000000in}}%
\pgfusepath{stroke,fill}%
}%
\begin{pgfscope}%
\pgfsys@transformshift{5.591997in}{4.221827in}%
\pgfsys@useobject{currentmarker}{}%
\end{pgfscope}%
\end{pgfscope}%
\begin{pgfscope}%
\pgfsetbuttcap%
\pgfsetroundjoin%
\definecolor{currentfill}{rgb}{0.000000,0.000000,0.000000}%
\pgfsetfillcolor{currentfill}%
\pgfsetlinewidth{0.501875pt}%
\definecolor{currentstroke}{rgb}{0.000000,0.000000,0.000000}%
\pgfsetstrokecolor{currentstroke}%
\pgfsetdash{}{0pt}%
\pgfsys@defobject{currentmarker}{\pgfqpoint{0.000000in}{0.000000in}}{\pgfqpoint{0.020833in}{0.000000in}}{%
\pgfpathmoveto{\pgfqpoint{0.000000in}{0.000000in}}%
\pgfpathlineto{\pgfqpoint{0.020833in}{0.000000in}}%
\pgfusepath{stroke,fill}%
}%
\begin{pgfscope}%
\pgfsys@transformshift{3.350525in}{4.310534in}%
\pgfsys@useobject{currentmarker}{}%
\end{pgfscope}%
\end{pgfscope}%
\begin{pgfscope}%
\pgfsetbuttcap%
\pgfsetroundjoin%
\definecolor{currentfill}{rgb}{0.000000,0.000000,0.000000}%
\pgfsetfillcolor{currentfill}%
\pgfsetlinewidth{0.501875pt}%
\definecolor{currentstroke}{rgb}{0.000000,0.000000,0.000000}%
\pgfsetstrokecolor{currentstroke}%
\pgfsetdash{}{0pt}%
\pgfsys@defobject{currentmarker}{\pgfqpoint{-0.020833in}{0.000000in}}{\pgfqpoint{-0.000000in}{0.000000in}}{%
\pgfpathmoveto{\pgfqpoint{-0.000000in}{0.000000in}}%
\pgfpathlineto{\pgfqpoint{-0.020833in}{0.000000in}}%
\pgfusepath{stroke,fill}%
}%
\begin{pgfscope}%
\pgfsys@transformshift{5.591997in}{4.310534in}%
\pgfsys@useobject{currentmarker}{}%
\end{pgfscope}%
\end{pgfscope}%
\begin{pgfscope}%
\definecolor{textcolor}{rgb}{0.000000,0.000000,0.000000}%
\pgfsetstrokecolor{textcolor}%
\pgfsetfillcolor{textcolor}%
\pgftext[x=2.999444in,y=2.398593in,,bottom,rotate=90.000000]{\color{textcolor}\rmfamily\fontsize{10.000000}{12.000000}\selectfont \(\displaystyle LCMC(K)\)}%
\end{pgfscope}%
\begin{pgfscope}%
\pgfpathrectangle{\pgfqpoint{3.350525in}{0.422992in}}{\pgfqpoint{2.241471in}{3.951201in}}%
\pgfusepath{clip}%
\pgfsetrectcap%
\pgfsetroundjoin%
\pgfsetlinewidth{1.003750pt}%
\definecolor{currentstroke}{rgb}{0.047059,0.364706,0.647059}%
\pgfsetstrokecolor{currentstroke}%
\pgfsetdash{}{0pt}%
\pgfpathmoveto{\pgfqpoint{3.372718in}{2.022193in}}%
\pgfpathlineto{\pgfqpoint{3.394911in}{2.017756in}}%
\pgfpathlineto{\pgfqpoint{3.417104in}{2.045260in}}%
\pgfpathlineto{\pgfqpoint{3.439296in}{2.107367in}}%
\pgfpathlineto{\pgfqpoint{3.461489in}{2.168055in}}%
\pgfpathlineto{\pgfqpoint{3.483682in}{2.223596in}}%
\pgfpathlineto{\pgfqpoint{3.505875in}{2.301040in}}%
\pgfpathlineto{\pgfqpoint{3.528068in}{2.375094in}}%
\pgfpathlineto{\pgfqpoint{3.550260in}{2.437817in}}%
\pgfpathlineto{\pgfqpoint{3.572453in}{2.488883in}}%
\pgfpathlineto{\pgfqpoint{3.594646in}{2.539698in}}%
\pgfpathlineto{\pgfqpoint{3.616839in}{2.601858in}}%
\pgfpathlineto{\pgfqpoint{3.639032in}{2.640806in}}%
\pgfpathlineto{\pgfqpoint{3.661224in}{2.687752in}}%
\pgfpathlineto{\pgfqpoint{3.683417in}{2.734708in}}%
\pgfpathlineto{\pgfqpoint{3.705610in}{2.771802in}}%
\pgfpathlineto{\pgfqpoint{3.727803in}{2.816119in}}%
\pgfpathlineto{\pgfqpoint{3.749995in}{2.856398in}}%
\pgfpathlineto{\pgfqpoint{3.772188in}{2.891411in}}%
\pgfpathlineto{\pgfqpoint{3.794381in}{2.932593in}}%
\pgfpathlineto{\pgfqpoint{3.816574in}{2.970022in}}%
\pgfpathlineto{\pgfqpoint{3.838767in}{3.003645in}}%
\pgfpathlineto{\pgfqpoint{3.860959in}{3.037198in}}%
\pgfpathlineto{\pgfqpoint{3.883152in}{3.065516in}}%
\pgfpathlineto{\pgfqpoint{3.905345in}{3.091426in}}%
\pgfpathlineto{\pgfqpoint{3.927538in}{3.122169in}}%
\pgfpathlineto{\pgfqpoint{3.949731in}{3.148990in}}%
\pgfpathlineto{\pgfqpoint{3.971923in}{3.172312in}}%
\pgfpathlineto{\pgfqpoint{3.994116in}{3.194515in}}%
\pgfpathlineto{\pgfqpoint{4.016309in}{3.216125in}}%
\pgfpathlineto{\pgfqpoint{4.038502in}{3.241835in}}%
\pgfpathlineto{\pgfqpoint{4.060694in}{3.264054in}}%
\pgfpathlineto{\pgfqpoint{4.082887in}{3.284334in}}%
\pgfpathlineto{\pgfqpoint{4.105080in}{3.304257in}}%
\pgfpathlineto{\pgfqpoint{4.127273in}{3.323243in}}%
\pgfpathlineto{\pgfqpoint{4.149466in}{3.347435in}}%
\pgfpathlineto{\pgfqpoint{4.171658in}{3.368545in}}%
\pgfpathlineto{\pgfqpoint{4.193851in}{3.392933in}}%
\pgfpathlineto{\pgfqpoint{4.216044in}{3.413114in}}%
\pgfpathlineto{\pgfqpoint{4.238237in}{3.433438in}}%
\pgfpathlineto{\pgfqpoint{4.260430in}{3.452079in}}%
\pgfpathlineto{\pgfqpoint{4.282622in}{3.468606in}}%
\pgfpathlineto{\pgfqpoint{4.304815in}{3.488203in}}%
\pgfpathlineto{\pgfqpoint{4.327008in}{3.506102in}}%
\pgfpathlineto{\pgfqpoint{4.349201in}{3.523916in}}%
\pgfpathlineto{\pgfqpoint{4.371393in}{3.543501in}}%
\pgfpathlineto{\pgfqpoint{4.393586in}{3.565085in}}%
\pgfpathlineto{\pgfqpoint{4.415779in}{3.582146in}}%
\pgfpathlineto{\pgfqpoint{4.437972in}{3.599162in}}%
\pgfpathlineto{\pgfqpoint{4.460165in}{3.615569in}}%
\pgfpathlineto{\pgfqpoint{4.482357in}{3.631924in}}%
\pgfpathlineto{\pgfqpoint{4.504550in}{3.647889in}}%
\pgfpathlineto{\pgfqpoint{4.526743in}{3.665260in}}%
\pgfpathlineto{\pgfqpoint{4.548936in}{3.682349in}}%
\pgfpathlineto{\pgfqpoint{4.571129in}{3.697430in}}%
\pgfpathlineto{\pgfqpoint{4.593321in}{3.714380in}}%
\pgfpathlineto{\pgfqpoint{4.615514in}{3.729957in}}%
\pgfpathlineto{\pgfqpoint{4.637707in}{3.743926in}}%
\pgfpathlineto{\pgfqpoint{4.659900in}{3.756129in}}%
\pgfpathlineto{\pgfqpoint{4.682092in}{3.770882in}}%
\pgfpathlineto{\pgfqpoint{4.704285in}{3.785180in}}%
\pgfpathlineto{\pgfqpoint{4.726478in}{3.799046in}}%
\pgfpathlineto{\pgfqpoint{4.748671in}{3.810584in}}%
\pgfpathlineto{\pgfqpoint{4.770864in}{3.825477in}}%
\pgfpathlineto{\pgfqpoint{4.793056in}{3.838111in}}%
\pgfpathlineto{\pgfqpoint{4.815249in}{3.849043in}}%
\pgfpathlineto{\pgfqpoint{4.837442in}{3.861954in}}%
\pgfpathlineto{\pgfqpoint{4.859635in}{3.875633in}}%
\pgfpathlineto{\pgfqpoint{4.881828in}{3.886550in}}%
\pgfpathlineto{\pgfqpoint{4.904020in}{3.900197in}}%
\pgfpathlineto{\pgfqpoint{4.926213in}{3.911810in}}%
\pgfpathlineto{\pgfqpoint{4.948406in}{3.924086in}}%
\pgfpathlineto{\pgfqpoint{4.970599in}{3.937265in}}%
\pgfpathlineto{\pgfqpoint{4.992792in}{3.948602in}}%
\pgfpathlineto{\pgfqpoint{5.014984in}{3.958595in}}%
\pgfpathlineto{\pgfqpoint{5.037177in}{3.968021in}}%
\pgfpathlineto{\pgfqpoint{5.059370in}{3.978908in}}%
\pgfpathlineto{\pgfqpoint{5.081563in}{3.989585in}}%
\pgfpathlineto{\pgfqpoint{5.103755in}{3.999564in}}%
\pgfpathlineto{\pgfqpoint{5.125948in}{4.010425in}}%
\pgfpathlineto{\pgfqpoint{5.148141in}{4.022419in}}%
\pgfpathlineto{\pgfqpoint{5.170334in}{4.033645in}}%
\pgfpathlineto{\pgfqpoint{5.192527in}{4.044879in}}%
\pgfpathlineto{\pgfqpoint{5.214719in}{4.055063in}}%
\pgfpathlineto{\pgfqpoint{5.236912in}{4.064277in}}%
\pgfpathlineto{\pgfqpoint{5.259105in}{4.074990in}}%
\pgfpathlineto{\pgfqpoint{5.281298in}{4.084518in}}%
\pgfpathlineto{\pgfqpoint{5.303491in}{4.094051in}}%
\pgfpathlineto{\pgfqpoint{5.325683in}{4.103210in}}%
\pgfpathlineto{\pgfqpoint{5.347876in}{4.113014in}}%
\pgfpathlineto{\pgfqpoint{5.370069in}{4.122290in}}%
\pgfpathlineto{\pgfqpoint{5.392262in}{4.132329in}}%
\pgfpathlineto{\pgfqpoint{5.414454in}{4.142954in}}%
\pgfpathlineto{\pgfqpoint{5.436647in}{4.150936in}}%
\pgfpathlineto{\pgfqpoint{5.458840in}{4.158750in}}%
\pgfpathlineto{\pgfqpoint{5.481033in}{4.167621in}}%
\pgfpathlineto{\pgfqpoint{5.503226in}{4.175138in}}%
\pgfpathlineto{\pgfqpoint{5.525418in}{4.183263in}}%
\pgfpathlineto{\pgfqpoint{5.547611in}{4.194593in}}%
\pgfusepath{stroke}%
\end{pgfscope}%
\begin{pgfscope}%
\pgfpathrectangle{\pgfqpoint{3.350525in}{0.422992in}}{\pgfqpoint{2.241471in}{3.951201in}}%
\pgfusepath{clip}%
\pgfsetrectcap%
\pgfsetroundjoin%
\pgfsetlinewidth{1.003750pt}%
\definecolor{currentstroke}{rgb}{0.000000,0.725490,0.270588}%
\pgfsetstrokecolor{currentstroke}%
\pgfsetdash{}{0pt}%
\pgfpathmoveto{\pgfqpoint{3.372718in}{2.378867in}}%
\pgfpathlineto{\pgfqpoint{3.394911in}{2.177461in}}%
\pgfpathlineto{\pgfqpoint{3.417104in}{1.910990in}}%
\pgfpathlineto{\pgfqpoint{3.439296in}{1.772874in}}%
\pgfpathlineto{\pgfqpoint{3.461489in}{1.688940in}}%
\pgfpathlineto{\pgfqpoint{3.483682in}{1.634167in}}%
\pgfpathlineto{\pgfqpoint{3.505875in}{1.605437in}}%
\pgfpathlineto{\pgfqpoint{3.528068in}{1.599637in}}%
\pgfpathlineto{\pgfqpoint{3.550260in}{1.598873in}}%
\pgfpathlineto{\pgfqpoint{3.572453in}{1.602165in}}%
\pgfpathlineto{\pgfqpoint{3.594646in}{1.604536in}}%
\pgfpathlineto{\pgfqpoint{3.616839in}{1.604146in}}%
\pgfpathlineto{\pgfqpoint{3.639032in}{1.614190in}}%
\pgfpathlineto{\pgfqpoint{3.661224in}{1.634587in}}%
\pgfpathlineto{\pgfqpoint{3.683417in}{1.645875in}}%
\pgfpathlineto{\pgfqpoint{3.705610in}{1.658415in}}%
\pgfpathlineto{\pgfqpoint{3.727803in}{1.666974in}}%
\pgfpathlineto{\pgfqpoint{3.749995in}{1.678328in}}%
\pgfpathlineto{\pgfqpoint{3.772188in}{1.694651in}}%
\pgfpathlineto{\pgfqpoint{3.794381in}{1.711293in}}%
\pgfpathlineto{\pgfqpoint{3.816574in}{1.723816in}}%
\pgfpathlineto{\pgfqpoint{3.838767in}{1.737701in}}%
\pgfpathlineto{\pgfqpoint{3.860959in}{1.752306in}}%
\pgfpathlineto{\pgfqpoint{3.883152in}{1.773385in}}%
\pgfpathlineto{\pgfqpoint{3.905345in}{1.791499in}}%
\pgfpathlineto{\pgfqpoint{3.927538in}{1.811496in}}%
\pgfpathlineto{\pgfqpoint{3.949731in}{1.825937in}}%
\pgfpathlineto{\pgfqpoint{3.971923in}{1.838332in}}%
\pgfpathlineto{\pgfqpoint{3.994116in}{1.856665in}}%
\pgfpathlineto{\pgfqpoint{4.016309in}{1.874780in}}%
\pgfpathlineto{\pgfqpoint{4.038502in}{1.889610in}}%
\pgfpathlineto{\pgfqpoint{4.060694in}{1.904066in}}%
\pgfpathlineto{\pgfqpoint{4.082887in}{1.915496in}}%
\pgfpathlineto{\pgfqpoint{4.105080in}{1.927976in}}%
\pgfpathlineto{\pgfqpoint{4.127273in}{1.941415in}}%
\pgfpathlineto{\pgfqpoint{4.149466in}{1.957608in}}%
\pgfpathlineto{\pgfqpoint{4.171658in}{1.971535in}}%
\pgfpathlineto{\pgfqpoint{4.193851in}{1.986456in}}%
\pgfpathlineto{\pgfqpoint{4.216044in}{2.002023in}}%
\pgfpathlineto{\pgfqpoint{4.238237in}{2.016101in}}%
\pgfpathlineto{\pgfqpoint{4.260430in}{2.031657in}}%
\pgfpathlineto{\pgfqpoint{4.282622in}{2.042542in}}%
\pgfpathlineto{\pgfqpoint{4.304815in}{2.054407in}}%
\pgfpathlineto{\pgfqpoint{4.327008in}{2.065612in}}%
\pgfpathlineto{\pgfqpoint{4.349201in}{2.077857in}}%
\pgfpathlineto{\pgfqpoint{4.371393in}{2.088219in}}%
\pgfpathlineto{\pgfqpoint{4.393586in}{2.099990in}}%
\pgfpathlineto{\pgfqpoint{4.415779in}{2.111862in}}%
\pgfpathlineto{\pgfqpoint{4.437972in}{2.123068in}}%
\pgfpathlineto{\pgfqpoint{4.460165in}{2.135814in}}%
\pgfpathlineto{\pgfqpoint{4.482357in}{2.148964in}}%
\pgfpathlineto{\pgfqpoint{4.504550in}{2.159732in}}%
\pgfpathlineto{\pgfqpoint{4.526743in}{2.168620in}}%
\pgfpathlineto{\pgfqpoint{4.548936in}{2.178165in}}%
\pgfpathlineto{\pgfqpoint{4.571129in}{2.189622in}}%
\pgfpathlineto{\pgfqpoint{4.593321in}{2.199243in}}%
\pgfpathlineto{\pgfqpoint{4.615514in}{2.209429in}}%
\pgfpathlineto{\pgfqpoint{4.637707in}{2.219876in}}%
\pgfpathlineto{\pgfqpoint{4.659900in}{2.230661in}}%
\pgfpathlineto{\pgfqpoint{4.682092in}{2.243540in}}%
\pgfpathlineto{\pgfqpoint{4.704285in}{2.252362in}}%
\pgfpathlineto{\pgfqpoint{4.726478in}{2.259925in}}%
\pgfpathlineto{\pgfqpoint{4.748671in}{2.269333in}}%
\pgfpathlineto{\pgfqpoint{4.770864in}{2.278086in}}%
\pgfpathlineto{\pgfqpoint{4.793056in}{2.288454in}}%
\pgfpathlineto{\pgfqpoint{4.815249in}{2.296840in}}%
\pgfpathlineto{\pgfqpoint{4.837442in}{2.305056in}}%
\pgfpathlineto{\pgfqpoint{4.859635in}{2.314909in}}%
\pgfpathlineto{\pgfqpoint{4.881828in}{2.324888in}}%
\pgfpathlineto{\pgfqpoint{4.904020in}{2.335190in}}%
\pgfpathlineto{\pgfqpoint{4.926213in}{2.342852in}}%
\pgfpathlineto{\pgfqpoint{4.948406in}{2.350450in}}%
\pgfpathlineto{\pgfqpoint{4.970599in}{2.358009in}}%
\pgfpathlineto{\pgfqpoint{4.992792in}{2.367067in}}%
\pgfpathlineto{\pgfqpoint{5.014984in}{2.373919in}}%
\pgfpathlineto{\pgfqpoint{5.037177in}{2.382366in}}%
\pgfpathlineto{\pgfqpoint{5.059370in}{2.390754in}}%
\pgfpathlineto{\pgfqpoint{5.081563in}{2.398382in}}%
\pgfpathlineto{\pgfqpoint{5.103755in}{2.405704in}}%
\pgfpathlineto{\pgfqpoint{5.125948in}{2.414284in}}%
\pgfpathlineto{\pgfqpoint{5.148141in}{2.422894in}}%
\pgfpathlineto{\pgfqpoint{5.170334in}{2.429823in}}%
\pgfpathlineto{\pgfqpoint{5.192527in}{2.436178in}}%
\pgfpathlineto{\pgfqpoint{5.214719in}{2.443353in}}%
\pgfpathlineto{\pgfqpoint{5.236912in}{2.451466in}}%
\pgfpathlineto{\pgfqpoint{5.259105in}{2.458256in}}%
\pgfpathlineto{\pgfqpoint{5.281298in}{2.465440in}}%
\pgfpathlineto{\pgfqpoint{5.303491in}{2.471836in}}%
\pgfpathlineto{\pgfqpoint{5.325683in}{2.478447in}}%
\pgfpathlineto{\pgfqpoint{5.347876in}{2.483373in}}%
\pgfpathlineto{\pgfqpoint{5.370069in}{2.489400in}}%
\pgfpathlineto{\pgfqpoint{5.392262in}{2.494968in}}%
\pgfpathlineto{\pgfqpoint{5.414454in}{2.499767in}}%
\pgfpathlineto{\pgfqpoint{5.436647in}{2.507089in}}%
\pgfpathlineto{\pgfqpoint{5.458840in}{2.514573in}}%
\pgfpathlineto{\pgfqpoint{5.481033in}{2.522457in}}%
\pgfpathlineto{\pgfqpoint{5.503226in}{2.530049in}}%
\pgfpathlineto{\pgfqpoint{5.525418in}{2.537759in}}%
\pgfpathlineto{\pgfqpoint{5.547611in}{2.544793in}}%
\pgfusepath{stroke}%
\end{pgfscope}%
\begin{pgfscope}%
\pgfpathrectangle{\pgfqpoint{3.350525in}{0.422992in}}{\pgfqpoint{2.241471in}{3.951201in}}%
\pgfusepath{clip}%
\pgfsetrectcap%
\pgfsetroundjoin%
\pgfsetlinewidth{1.003750pt}%
\definecolor{currentstroke}{rgb}{1.000000,0.584314,0.000000}%
\pgfsetstrokecolor{currentstroke}%
\pgfsetdash{}{0pt}%
\pgfpathmoveto{\pgfqpoint{3.372718in}{1.770214in}}%
\pgfpathlineto{\pgfqpoint{3.394911in}{1.580342in}}%
\pgfpathlineto{\pgfqpoint{3.417104in}{1.518234in}}%
\pgfpathlineto{\pgfqpoint{3.439296in}{1.528881in}}%
\pgfpathlineto{\pgfqpoint{3.461489in}{1.577502in}}%
\pgfpathlineto{\pgfqpoint{3.483682in}{1.616718in}}%
\pgfpathlineto{\pgfqpoint{3.505875in}{1.642448in}}%
\pgfpathlineto{\pgfqpoint{3.528068in}{1.668843in}}%
\pgfpathlineto{\pgfqpoint{3.550260in}{1.702188in}}%
\pgfpathlineto{\pgfqpoint{3.572453in}{1.729042in}}%
\pgfpathlineto{\pgfqpoint{3.594646in}{1.757305in}}%
\pgfpathlineto{\pgfqpoint{3.616839in}{1.783518in}}%
\pgfpathlineto{\pgfqpoint{3.639032in}{1.815664in}}%
\pgfpathlineto{\pgfqpoint{3.661224in}{1.846132in}}%
\pgfpathlineto{\pgfqpoint{3.683417in}{1.886616in}}%
\pgfpathlineto{\pgfqpoint{3.705610in}{1.921595in}}%
\pgfpathlineto{\pgfqpoint{3.727803in}{1.952981in}}%
\pgfpathlineto{\pgfqpoint{3.749995in}{1.985218in}}%
\pgfpathlineto{\pgfqpoint{3.772188in}{2.009391in}}%
\pgfpathlineto{\pgfqpoint{3.794381in}{2.034785in}}%
\pgfpathlineto{\pgfqpoint{3.816574in}{2.058858in}}%
\pgfpathlineto{\pgfqpoint{3.838767in}{2.081792in}}%
\pgfpathlineto{\pgfqpoint{3.860959in}{2.110910in}}%
\pgfpathlineto{\pgfqpoint{3.883152in}{2.132425in}}%
\pgfpathlineto{\pgfqpoint{3.905345in}{2.151651in}}%
\pgfpathlineto{\pgfqpoint{3.927538in}{2.171515in}}%
\pgfpathlineto{\pgfqpoint{3.949731in}{2.188329in}}%
\pgfpathlineto{\pgfqpoint{3.971923in}{2.207681in}}%
\pgfpathlineto{\pgfqpoint{3.994116in}{2.221783in}}%
\pgfpathlineto{\pgfqpoint{4.016309in}{2.238730in}}%
\pgfpathlineto{\pgfqpoint{4.038502in}{2.257732in}}%
\pgfpathlineto{\pgfqpoint{4.060694in}{2.275214in}}%
\pgfpathlineto{\pgfqpoint{4.082887in}{2.290722in}}%
\pgfpathlineto{\pgfqpoint{4.105080in}{2.309023in}}%
\pgfpathlineto{\pgfqpoint{4.127273in}{2.327597in}}%
\pgfpathlineto{\pgfqpoint{4.149466in}{2.347406in}}%
\pgfpathlineto{\pgfqpoint{4.171658in}{2.367200in}}%
\pgfpathlineto{\pgfqpoint{4.193851in}{2.382169in}}%
\pgfpathlineto{\pgfqpoint{4.216044in}{2.397599in}}%
\pgfpathlineto{\pgfqpoint{4.238237in}{2.409596in}}%
\pgfpathlineto{\pgfqpoint{4.260430in}{2.427067in}}%
\pgfpathlineto{\pgfqpoint{4.282622in}{2.443241in}}%
\pgfpathlineto{\pgfqpoint{4.304815in}{2.456641in}}%
\pgfpathlineto{\pgfqpoint{4.327008in}{2.469472in}}%
\pgfpathlineto{\pgfqpoint{4.349201in}{2.481891in}}%
\pgfpathlineto{\pgfqpoint{4.371393in}{2.498591in}}%
\pgfpathlineto{\pgfqpoint{4.393586in}{2.511107in}}%
\pgfpathlineto{\pgfqpoint{4.415779in}{2.526429in}}%
\pgfpathlineto{\pgfqpoint{4.437972in}{2.542937in}}%
\pgfpathlineto{\pgfqpoint{4.460165in}{2.560026in}}%
\pgfpathlineto{\pgfqpoint{4.482357in}{2.571469in}}%
\pgfpathlineto{\pgfqpoint{4.504550in}{2.585476in}}%
\pgfpathlineto{\pgfqpoint{4.526743in}{2.598585in}}%
\pgfpathlineto{\pgfqpoint{4.548936in}{2.611899in}}%
\pgfpathlineto{\pgfqpoint{4.571129in}{2.625987in}}%
\pgfpathlineto{\pgfqpoint{4.593321in}{2.637924in}}%
\pgfpathlineto{\pgfqpoint{4.615514in}{2.652276in}}%
\pgfpathlineto{\pgfqpoint{4.637707in}{2.664755in}}%
\pgfpathlineto{\pgfqpoint{4.659900in}{2.676932in}}%
\pgfpathlineto{\pgfqpoint{4.682092in}{2.692370in}}%
\pgfpathlineto{\pgfqpoint{4.704285in}{2.705237in}}%
\pgfpathlineto{\pgfqpoint{4.726478in}{2.717403in}}%
\pgfpathlineto{\pgfqpoint{4.748671in}{2.730900in}}%
\pgfpathlineto{\pgfqpoint{4.770864in}{2.743754in}}%
\pgfpathlineto{\pgfqpoint{4.793056in}{2.754929in}}%
\pgfpathlineto{\pgfqpoint{4.815249in}{2.765819in}}%
\pgfpathlineto{\pgfqpoint{4.837442in}{2.776596in}}%
\pgfpathlineto{\pgfqpoint{4.859635in}{2.787187in}}%
\pgfpathlineto{\pgfqpoint{4.881828in}{2.798216in}}%
\pgfpathlineto{\pgfqpoint{4.904020in}{2.809057in}}%
\pgfpathlineto{\pgfqpoint{4.926213in}{2.819418in}}%
\pgfpathlineto{\pgfqpoint{4.948406in}{2.829121in}}%
\pgfpathlineto{\pgfqpoint{4.970599in}{2.837829in}}%
\pgfpathlineto{\pgfqpoint{4.992792in}{2.848076in}}%
\pgfpathlineto{\pgfqpoint{5.014984in}{2.858973in}}%
\pgfpathlineto{\pgfqpoint{5.037177in}{2.868182in}}%
\pgfpathlineto{\pgfqpoint{5.059370in}{2.876668in}}%
\pgfpathlineto{\pgfqpoint{5.081563in}{2.885209in}}%
\pgfpathlineto{\pgfqpoint{5.103755in}{2.896454in}}%
\pgfpathlineto{\pgfqpoint{5.125948in}{2.907351in}}%
\pgfpathlineto{\pgfqpoint{5.148141in}{2.918659in}}%
\pgfpathlineto{\pgfqpoint{5.170334in}{2.927981in}}%
\pgfpathlineto{\pgfqpoint{5.192527in}{2.937206in}}%
\pgfpathlineto{\pgfqpoint{5.214719in}{2.946424in}}%
\pgfpathlineto{\pgfqpoint{5.236912in}{2.954902in}}%
\pgfpathlineto{\pgfqpoint{5.259105in}{2.965474in}}%
\pgfpathlineto{\pgfqpoint{5.281298in}{2.974660in}}%
\pgfpathlineto{\pgfqpoint{5.303491in}{2.984323in}}%
\pgfpathlineto{\pgfqpoint{5.325683in}{2.993072in}}%
\pgfpathlineto{\pgfqpoint{5.347876in}{3.001409in}}%
\pgfpathlineto{\pgfqpoint{5.370069in}{3.009621in}}%
\pgfpathlineto{\pgfqpoint{5.392262in}{3.017095in}}%
\pgfpathlineto{\pgfqpoint{5.414454in}{3.024752in}}%
\pgfpathlineto{\pgfqpoint{5.436647in}{3.032416in}}%
\pgfpathlineto{\pgfqpoint{5.458840in}{3.041600in}}%
\pgfpathlineto{\pgfqpoint{5.481033in}{3.049076in}}%
\pgfpathlineto{\pgfqpoint{5.503226in}{3.057259in}}%
\pgfpathlineto{\pgfqpoint{5.525418in}{3.064966in}}%
\pgfpathlineto{\pgfqpoint{5.547611in}{3.071909in}}%
\pgfusepath{stroke}%
\end{pgfscope}%
\begin{pgfscope}%
\pgfpathrectangle{\pgfqpoint{3.350525in}{0.422992in}}{\pgfqpoint{2.241471in}{3.951201in}}%
\pgfusepath{clip}%
\pgfsetrectcap%
\pgfsetroundjoin%
\pgfsetlinewidth{1.003750pt}%
\definecolor{currentstroke}{rgb}{1.000000,0.172549,0.000000}%
\pgfsetstrokecolor{currentstroke}%
\pgfsetdash{}{0pt}%
\pgfpathmoveto{\pgfqpoint{3.372718in}{1.775537in}}%
\pgfpathlineto{\pgfqpoint{3.394911in}{1.595425in}}%
\pgfpathlineto{\pgfqpoint{3.417104in}{1.324814in}}%
\pgfpathlineto{\pgfqpoint{3.439296in}{1.153574in}}%
\pgfpathlineto{\pgfqpoint{3.461489in}{1.039473in}}%
\pgfpathlineto{\pgfqpoint{3.483682in}{0.960449in}}%
\pgfpathlineto{\pgfqpoint{3.505875in}{0.904509in}}%
\pgfpathlineto{\pgfqpoint{3.528068in}{0.871871in}}%
\pgfpathlineto{\pgfqpoint{3.550260in}{0.843528in}}%
\pgfpathlineto{\pgfqpoint{3.572453in}{0.826887in}}%
\pgfpathlineto{\pgfqpoint{3.594646in}{0.811012in}}%
\pgfpathlineto{\pgfqpoint{3.616839in}{0.798967in}}%
\pgfpathlineto{\pgfqpoint{3.639032in}{0.790140in}}%
\pgfpathlineto{\pgfqpoint{3.661224in}{0.784094in}}%
\pgfpathlineto{\pgfqpoint{3.683417in}{0.775661in}}%
\pgfpathlineto{\pgfqpoint{3.705610in}{0.771941in}}%
\pgfpathlineto{\pgfqpoint{3.727803in}{0.768764in}}%
\pgfpathlineto{\pgfqpoint{3.749995in}{0.764264in}}%
\pgfpathlineto{\pgfqpoint{3.772188in}{0.764159in}}%
\pgfpathlineto{\pgfqpoint{3.794381in}{0.763178in}}%
\pgfpathlineto{\pgfqpoint{3.816574in}{0.762629in}}%
\pgfpathlineto{\pgfqpoint{3.838767in}{0.761726in}}%
\pgfpathlineto{\pgfqpoint{3.860959in}{0.761210in}}%
\pgfpathlineto{\pgfqpoint{3.883152in}{0.760885in}}%
\pgfpathlineto{\pgfqpoint{3.905345in}{0.762573in}}%
\pgfpathlineto{\pgfqpoint{3.927538in}{0.764951in}}%
\pgfpathlineto{\pgfqpoint{3.949731in}{0.767218in}}%
\pgfpathlineto{\pgfqpoint{3.971923in}{0.768436in}}%
\pgfpathlineto{\pgfqpoint{3.994116in}{0.773180in}}%
\pgfpathlineto{\pgfqpoint{4.016309in}{0.772757in}}%
\pgfpathlineto{\pgfqpoint{4.038502in}{0.772992in}}%
\pgfpathlineto{\pgfqpoint{4.060694in}{0.774875in}}%
\pgfpathlineto{\pgfqpoint{4.082887in}{0.775784in}}%
\pgfpathlineto{\pgfqpoint{4.105080in}{0.777996in}}%
\pgfpathlineto{\pgfqpoint{4.127273in}{0.779422in}}%
\pgfpathlineto{\pgfqpoint{4.149466in}{0.781016in}}%
\pgfpathlineto{\pgfqpoint{4.171658in}{0.783387in}}%
\pgfpathlineto{\pgfqpoint{4.193851in}{0.785353in}}%
\pgfpathlineto{\pgfqpoint{4.216044in}{0.786763in}}%
\pgfpathlineto{\pgfqpoint{4.238237in}{0.788768in}}%
\pgfpathlineto{\pgfqpoint{4.260430in}{0.791152in}}%
\pgfpathlineto{\pgfqpoint{4.282622in}{0.792534in}}%
\pgfpathlineto{\pgfqpoint{4.304815in}{0.793564in}}%
\pgfpathlineto{\pgfqpoint{4.327008in}{0.795393in}}%
\pgfpathlineto{\pgfqpoint{4.349201in}{0.797378in}}%
\pgfpathlineto{\pgfqpoint{4.371393in}{0.798929in}}%
\pgfpathlineto{\pgfqpoint{4.393586in}{0.801585in}}%
\pgfpathlineto{\pgfqpoint{4.415779in}{0.803871in}}%
\pgfpathlineto{\pgfqpoint{4.437972in}{0.806499in}}%
\pgfpathlineto{\pgfqpoint{4.460165in}{0.809376in}}%
\pgfpathlineto{\pgfqpoint{4.482357in}{0.811375in}}%
\pgfpathlineto{\pgfqpoint{4.504550in}{0.812955in}}%
\pgfpathlineto{\pgfqpoint{4.526743in}{0.814912in}}%
\pgfpathlineto{\pgfqpoint{4.548936in}{0.818110in}}%
\pgfpathlineto{\pgfqpoint{4.571129in}{0.822354in}}%
\pgfpathlineto{\pgfqpoint{4.593321in}{0.824798in}}%
\pgfpathlineto{\pgfqpoint{4.615514in}{0.826752in}}%
\pgfpathlineto{\pgfqpoint{4.637707in}{0.829311in}}%
\pgfpathlineto{\pgfqpoint{4.659900in}{0.831272in}}%
\pgfpathlineto{\pgfqpoint{4.682092in}{0.833789in}}%
\pgfpathlineto{\pgfqpoint{4.704285in}{0.836195in}}%
\pgfpathlineto{\pgfqpoint{4.726478in}{0.839066in}}%
\pgfpathlineto{\pgfqpoint{4.748671in}{0.841903in}}%
\pgfpathlineto{\pgfqpoint{4.770864in}{0.844235in}}%
\pgfpathlineto{\pgfqpoint{4.793056in}{0.848161in}}%
\pgfpathlineto{\pgfqpoint{4.815249in}{0.851403in}}%
\pgfpathlineto{\pgfqpoint{4.837442in}{0.853886in}}%
\pgfpathlineto{\pgfqpoint{4.859635in}{0.856871in}}%
\pgfpathlineto{\pgfqpoint{4.881828in}{0.859691in}}%
\pgfpathlineto{\pgfqpoint{4.904020in}{0.861950in}}%
\pgfpathlineto{\pgfqpoint{4.926213in}{0.863944in}}%
\pgfpathlineto{\pgfqpoint{4.948406in}{0.865761in}}%
\pgfpathlineto{\pgfqpoint{4.970599in}{0.867964in}}%
\pgfpathlineto{\pgfqpoint{4.992792in}{0.870301in}}%
\pgfpathlineto{\pgfqpoint{5.014984in}{0.872149in}}%
\pgfpathlineto{\pgfqpoint{5.037177in}{0.874555in}}%
\pgfpathlineto{\pgfqpoint{5.059370in}{0.876784in}}%
\pgfpathlineto{\pgfqpoint{5.081563in}{0.877932in}}%
\pgfpathlineto{\pgfqpoint{5.103755in}{0.880039in}}%
\pgfpathlineto{\pgfqpoint{5.125948in}{0.882781in}}%
\pgfpathlineto{\pgfqpoint{5.148141in}{0.885542in}}%
\pgfpathlineto{\pgfqpoint{5.170334in}{0.888973in}}%
\pgfpathlineto{\pgfqpoint{5.192527in}{0.891572in}}%
\pgfpathlineto{\pgfqpoint{5.214719in}{0.894743in}}%
\pgfpathlineto{\pgfqpoint{5.236912in}{0.897861in}}%
\pgfpathlineto{\pgfqpoint{5.259105in}{0.900576in}}%
\pgfpathlineto{\pgfqpoint{5.281298in}{0.903432in}}%
\pgfpathlineto{\pgfqpoint{5.303491in}{0.905941in}}%
\pgfpathlineto{\pgfqpoint{5.325683in}{0.908513in}}%
\pgfpathlineto{\pgfqpoint{5.347876in}{0.912310in}}%
\pgfpathlineto{\pgfqpoint{5.370069in}{0.914678in}}%
\pgfpathlineto{\pgfqpoint{5.392262in}{0.916917in}}%
\pgfpathlineto{\pgfqpoint{5.414454in}{0.919452in}}%
\pgfpathlineto{\pgfqpoint{5.436647in}{0.921932in}}%
\pgfpathlineto{\pgfqpoint{5.458840in}{0.924529in}}%
\pgfpathlineto{\pgfqpoint{5.481033in}{0.926553in}}%
\pgfpathlineto{\pgfqpoint{5.503226in}{0.928097in}}%
\pgfpathlineto{\pgfqpoint{5.525418in}{0.929791in}}%
\pgfpathlineto{\pgfqpoint{5.547611in}{0.932328in}}%
\pgfusepath{stroke}%
\end{pgfscope}%
\begin{pgfscope}%
\pgfpathrectangle{\pgfqpoint{3.350525in}{0.422992in}}{\pgfqpoint{2.241471in}{3.951201in}}%
\pgfusepath{clip}%
\pgfsetrectcap%
\pgfsetroundjoin%
\pgfsetlinewidth{1.003750pt}%
\definecolor{currentstroke}{rgb}{0.517647,0.356863,0.592157}%
\pgfsetstrokecolor{currentstroke}%
\pgfsetdash{}{0pt}%
\pgfpathmoveto{\pgfqpoint{3.372718in}{0.652279in}}%
\pgfpathlineto{\pgfqpoint{3.394911in}{0.607029in}}%
\pgfpathlineto{\pgfqpoint{3.417104in}{0.602592in}}%
\pgfpathlineto{\pgfqpoint{3.439296in}{0.618119in}}%
\pgfpathlineto{\pgfqpoint{3.461489in}{0.635952in}}%
\pgfpathlineto{\pgfqpoint{3.483682in}{0.642517in}}%
\pgfpathlineto{\pgfqpoint{3.505875in}{0.650756in}}%
\pgfpathlineto{\pgfqpoint{3.528068in}{0.660705in}}%
\pgfpathlineto{\pgfqpoint{3.550260in}{0.666669in}}%
\pgfpathlineto{\pgfqpoint{3.572453in}{0.676409in}}%
\pgfpathlineto{\pgfqpoint{3.594646in}{0.688411in}}%
\pgfpathlineto{\pgfqpoint{3.616839in}{0.694715in}}%
\pgfpathlineto{\pgfqpoint{3.639032in}{0.701005in}}%
\pgfpathlineto{\pgfqpoint{3.661224in}{0.713241in}}%
\pgfpathlineto{\pgfqpoint{3.683417in}{0.717102in}}%
\pgfpathlineto{\pgfqpoint{3.705610in}{0.738780in}}%
\pgfpathlineto{\pgfqpoint{3.727803in}{0.751750in}}%
\pgfpathlineto{\pgfqpoint{3.749995in}{0.751546in}}%
\pgfpathlineto{\pgfqpoint{3.772188in}{0.761358in}}%
\pgfpathlineto{\pgfqpoint{3.794381in}{0.775156in}}%
\pgfpathlineto{\pgfqpoint{3.816574in}{0.785021in}}%
\pgfpathlineto{\pgfqpoint{3.838767in}{0.792538in}}%
\pgfpathlineto{\pgfqpoint{3.860959in}{0.798860in}}%
\pgfpathlineto{\pgfqpoint{3.883152in}{0.808649in}}%
\pgfpathlineto{\pgfqpoint{3.905345in}{0.823332in}}%
\pgfpathlineto{\pgfqpoint{3.927538in}{0.823919in}}%
\pgfpathlineto{\pgfqpoint{3.949731in}{0.833072in}}%
\pgfpathlineto{\pgfqpoint{3.971923in}{0.841000in}}%
\pgfpathlineto{\pgfqpoint{3.994116in}{0.842202in}}%
\pgfpathlineto{\pgfqpoint{4.016309in}{0.848824in}}%
\pgfpathlineto{\pgfqpoint{4.038502in}{0.866010in}}%
\pgfpathlineto{\pgfqpoint{4.060694in}{0.867703in}}%
\pgfpathlineto{\pgfqpoint{4.082887in}{0.871230in}}%
\pgfpathlineto{\pgfqpoint{4.105080in}{0.877524in}}%
\pgfpathlineto{\pgfqpoint{4.127273in}{0.880062in}}%
\pgfpathlineto{\pgfqpoint{4.149466in}{0.889113in}}%
\pgfpathlineto{\pgfqpoint{4.171658in}{0.904725in}}%
\pgfpathlineto{\pgfqpoint{4.193851in}{0.913771in}}%
\pgfpathlineto{\pgfqpoint{4.216044in}{0.919214in}}%
\pgfpathlineto{\pgfqpoint{4.238237in}{0.923764in}}%
\pgfpathlineto{\pgfqpoint{4.260430in}{0.930645in}}%
\pgfpathlineto{\pgfqpoint{4.282622in}{0.937959in}}%
\pgfpathlineto{\pgfqpoint{4.304815in}{0.938412in}}%
\pgfpathlineto{\pgfqpoint{4.327008in}{0.945096in}}%
\pgfpathlineto{\pgfqpoint{4.349201in}{0.955663in}}%
\pgfpathlineto{\pgfqpoint{4.371393in}{0.959946in}}%
\pgfpathlineto{\pgfqpoint{4.393586in}{0.977940in}}%
\pgfpathlineto{\pgfqpoint{4.415779in}{0.983576in}}%
\pgfpathlineto{\pgfqpoint{4.437972in}{0.990322in}}%
\pgfpathlineto{\pgfqpoint{4.460165in}{0.994669in}}%
\pgfpathlineto{\pgfqpoint{4.482357in}{1.008344in}}%
\pgfpathlineto{\pgfqpoint{4.504550in}{1.015555in}}%
\pgfpathlineto{\pgfqpoint{4.526743in}{1.016636in}}%
\pgfpathlineto{\pgfqpoint{4.548936in}{1.020863in}}%
\pgfpathlineto{\pgfqpoint{4.571129in}{1.028260in}}%
\pgfpathlineto{\pgfqpoint{4.593321in}{1.032414in}}%
\pgfpathlineto{\pgfqpoint{4.615514in}{1.039660in}}%
\pgfpathlineto{\pgfqpoint{4.637707in}{1.051215in}}%
\pgfpathlineto{\pgfqpoint{4.659900in}{1.058619in}}%
\pgfpathlineto{\pgfqpoint{4.682092in}{1.060245in}}%
\pgfpathlineto{\pgfqpoint{4.704285in}{1.067694in}}%
\pgfpathlineto{\pgfqpoint{4.726478in}{1.070925in}}%
\pgfpathlineto{\pgfqpoint{4.748671in}{1.080925in}}%
\pgfpathlineto{\pgfqpoint{4.770864in}{1.084763in}}%
\pgfpathlineto{\pgfqpoint{4.793056in}{1.089165in}}%
\pgfpathlineto{\pgfqpoint{4.815249in}{1.100693in}}%
\pgfpathlineto{\pgfqpoint{4.837442in}{1.114976in}}%
\pgfpathlineto{\pgfqpoint{4.859635in}{1.116469in}}%
\pgfpathlineto{\pgfqpoint{4.881828in}{1.126406in}}%
\pgfpathlineto{\pgfqpoint{4.904020in}{1.131167in}}%
\pgfpathlineto{\pgfqpoint{4.926213in}{1.139142in}}%
\pgfpathlineto{\pgfqpoint{4.948406in}{1.141202in}}%
\pgfpathlineto{\pgfqpoint{4.970599in}{1.147047in}}%
\pgfpathlineto{\pgfqpoint{4.992792in}{1.147099in}}%
\pgfpathlineto{\pgfqpoint{5.014984in}{1.154815in}}%
\pgfpathlineto{\pgfqpoint{5.037177in}{1.162444in}}%
\pgfpathlineto{\pgfqpoint{5.059370in}{1.163538in}}%
\pgfpathlineto{\pgfqpoint{5.081563in}{1.169859in}}%
\pgfpathlineto{\pgfqpoint{5.103755in}{1.174942in}}%
\pgfpathlineto{\pgfqpoint{5.125948in}{1.179920in}}%
\pgfpathlineto{\pgfqpoint{5.148141in}{1.183154in}}%
\pgfpathlineto{\pgfqpoint{5.170334in}{1.194425in}}%
\pgfpathlineto{\pgfqpoint{5.192527in}{1.201383in}}%
\pgfpathlineto{\pgfqpoint{5.214719in}{1.207900in}}%
\pgfpathlineto{\pgfqpoint{5.236912in}{1.212532in}}%
\pgfpathlineto{\pgfqpoint{5.259105in}{1.214352in}}%
\pgfpathlineto{\pgfqpoint{5.281298in}{1.222699in}}%
\pgfpathlineto{\pgfqpoint{5.303491in}{1.225472in}}%
\pgfpathlineto{\pgfqpoint{5.325683in}{1.235799in}}%
\pgfpathlineto{\pgfqpoint{5.347876in}{1.240770in}}%
\pgfpathlineto{\pgfqpoint{5.370069in}{1.246392in}}%
\pgfpathlineto{\pgfqpoint{5.392262in}{1.249269in}}%
\pgfpathlineto{\pgfqpoint{5.414454in}{1.254374in}}%
\pgfpathlineto{\pgfqpoint{5.436647in}{1.256387in}}%
\pgfpathlineto{\pgfqpoint{5.458840in}{1.258844in}}%
\pgfpathlineto{\pgfqpoint{5.481033in}{1.262451in}}%
\pgfpathlineto{\pgfqpoint{5.503226in}{1.275003in}}%
\pgfpathlineto{\pgfqpoint{5.525418in}{1.279005in}}%
\pgfpathlineto{\pgfqpoint{5.547611in}{1.286816in}}%
\pgfusepath{stroke}%
\end{pgfscope}%
\begin{pgfscope}%
\pgfsetrectcap%
\pgfsetmiterjoin%
\pgfsetlinewidth{0.501875pt}%
\definecolor{currentstroke}{rgb}{0.000000,0.000000,0.000000}%
\pgfsetstrokecolor{currentstroke}%
\pgfsetdash{}{0pt}%
\pgfpathmoveto{\pgfqpoint{3.350525in}{0.422992in}}%
\pgfpathlineto{\pgfqpoint{3.350525in}{4.374193in}}%
\pgfusepath{stroke}%
\end{pgfscope}%
\begin{pgfscope}%
\pgfsetrectcap%
\pgfsetmiterjoin%
\pgfsetlinewidth{0.501875pt}%
\definecolor{currentstroke}{rgb}{0.000000,0.000000,0.000000}%
\pgfsetstrokecolor{currentstroke}%
\pgfsetdash{}{0pt}%
\pgfpathmoveto{\pgfqpoint{5.591997in}{0.422992in}}%
\pgfpathlineto{\pgfqpoint{5.591997in}{4.374193in}}%
\pgfusepath{stroke}%
\end{pgfscope}%
\begin{pgfscope}%
\pgfsetrectcap%
\pgfsetmiterjoin%
\pgfsetlinewidth{0.501875pt}%
\definecolor{currentstroke}{rgb}{0.000000,0.000000,0.000000}%
\pgfsetstrokecolor{currentstroke}%
\pgfsetdash{}{0pt}%
\pgfpathmoveto{\pgfqpoint{3.350525in}{0.422992in}}%
\pgfpathlineto{\pgfqpoint{5.591997in}{0.422992in}}%
\pgfusepath{stroke}%
\end{pgfscope}%
\begin{pgfscope}%
\pgfsetrectcap%
\pgfsetmiterjoin%
\pgfsetlinewidth{0.501875pt}%
\definecolor{currentstroke}{rgb}{0.000000,0.000000,0.000000}%
\pgfsetstrokecolor{currentstroke}%
\pgfsetdash{}{0pt}%
\pgfpathmoveto{\pgfqpoint{3.350525in}{4.374193in}}%
\pgfpathlineto{\pgfqpoint{5.591997in}{4.374193in}}%
\pgfusepath{stroke}%
\end{pgfscope}%
\begin{pgfscope}%
\definecolor{textcolor}{rgb}{0.000000,0.000000,0.000000}%
\pgfsetstrokecolor{textcolor}%
\pgfsetfillcolor{textcolor}%
\pgftext[x=4.471261in,y=4.457526in,,base]{\color{textcolor}\rmfamily\fontsize{12.000000}{14.400000}\selectfont LCMC}%
\end{pgfscope}%
\begin{pgfscope}%
\pgfsetrectcap%
\pgfsetroundjoin%
\pgfsetlinewidth{1.003750pt}%
\definecolor{currentstroke}{rgb}{0.047059,0.364706,0.647059}%
\pgfsetstrokecolor{currentstroke}%
\pgfsetdash{}{0pt}%
\pgfpathmoveto{\pgfqpoint{3.475525in}{4.192281in}}%
\pgfpathlineto{\pgfqpoint{3.614414in}{4.192281in}}%
\pgfpathlineto{\pgfqpoint{3.753303in}{4.192281in}}%
\pgfusepath{stroke}%
\end{pgfscope}%
\begin{pgfscope}%
\definecolor{textcolor}{rgb}{0.000000,0.000000,0.000000}%
\pgfsetstrokecolor{textcolor}%
\pgfsetfillcolor{textcolor}%
\pgftext[x=3.864414in,y=4.143670in,left,base]{\color{textcolor}\rmfamily\fontsize{10.000000}{12.000000}\selectfont PCA}%
\end{pgfscope}%
\begin{pgfscope}%
\pgfsetrectcap%
\pgfsetroundjoin%
\pgfsetlinewidth{1.003750pt}%
\definecolor{currentstroke}{rgb}{0.000000,0.725490,0.270588}%
\pgfsetstrokecolor{currentstroke}%
\pgfsetdash{}{0pt}%
\pgfpathmoveto{\pgfqpoint{3.475525in}{3.988424in}}%
\pgfpathlineto{\pgfqpoint{3.614414in}{3.988424in}}%
\pgfpathlineto{\pgfqpoint{3.753303in}{3.988424in}}%
\pgfusepath{stroke}%
\end{pgfscope}%
\begin{pgfscope}%
\definecolor{textcolor}{rgb}{0.000000,0.000000,0.000000}%
\pgfsetstrokecolor{textcolor}%
\pgfsetfillcolor{textcolor}%
\pgftext[x=3.864414in,y=3.939813in,left,base]{\color{textcolor}\rmfamily\fontsize{10.000000}{12.000000}\selectfont KernelPCA}%
\end{pgfscope}%
\begin{pgfscope}%
\pgfsetrectcap%
\pgfsetroundjoin%
\pgfsetlinewidth{1.003750pt}%
\definecolor{currentstroke}{rgb}{1.000000,0.584314,0.000000}%
\pgfsetstrokecolor{currentstroke}%
\pgfsetdash{}{0pt}%
\pgfpathmoveto{\pgfqpoint{3.475525in}{3.784567in}}%
\pgfpathlineto{\pgfqpoint{3.614414in}{3.784567in}}%
\pgfpathlineto{\pgfqpoint{3.753303in}{3.784567in}}%
\pgfusepath{stroke}%
\end{pgfscope}%
\begin{pgfscope}%
\definecolor{textcolor}{rgb}{0.000000,0.000000,0.000000}%
\pgfsetstrokecolor{textcolor}%
\pgfsetfillcolor{textcolor}%
\pgftext[x=3.864414in,y=3.735956in,left,base]{\color{textcolor}\rmfamily\fontsize{10.000000}{12.000000}\selectfont AE}%
\end{pgfscope}%
\begin{pgfscope}%
\pgfsetrectcap%
\pgfsetroundjoin%
\pgfsetlinewidth{1.003750pt}%
\definecolor{currentstroke}{rgb}{1.000000,0.172549,0.000000}%
\pgfsetstrokecolor{currentstroke}%
\pgfsetdash{}{0pt}%
\pgfpathmoveto{\pgfqpoint{3.475525in}{3.580710in}}%
\pgfpathlineto{\pgfqpoint{3.614414in}{3.580710in}}%
\pgfpathlineto{\pgfqpoint{3.753303in}{3.580710in}}%
\pgfusepath{stroke}%
\end{pgfscope}%
\begin{pgfscope}%
\definecolor{textcolor}{rgb}{0.000000,0.000000,0.000000}%
\pgfsetstrokecolor{textcolor}%
\pgfsetfillcolor{textcolor}%
\pgftext[x=3.864414in,y=3.532098in,left,base]{\color{textcolor}\rmfamily\fontsize{10.000000}{12.000000}\selectfont LLE}%
\end{pgfscope}%
\begin{pgfscope}%
\pgfsetrectcap%
\pgfsetroundjoin%
\pgfsetlinewidth{1.003750pt}%
\definecolor{currentstroke}{rgb}{0.517647,0.356863,0.592157}%
\pgfsetstrokecolor{currentstroke}%
\pgfsetdash{}{0pt}%
\pgfpathmoveto{\pgfqpoint{3.475525in}{3.376852in}}%
\pgfpathlineto{\pgfqpoint{3.614414in}{3.376852in}}%
\pgfpathlineto{\pgfqpoint{3.753303in}{3.376852in}}%
\pgfusepath{stroke}%
\end{pgfscope}%
\begin{pgfscope}%
\definecolor{textcolor}{rgb}{0.000000,0.000000,0.000000}%
\pgfsetstrokecolor{textcolor}%
\pgfsetfillcolor{textcolor}%
\pgftext[x=3.864414in,y=3.328241in,left,base]{\color{textcolor}\rmfamily\fontsize{10.000000}{12.000000}\selectfont CAE}%
\end{pgfscope}%
\end{pgfpicture}%
\makeatother%
\endgroup%

	\end{center}
	\caption[FER2013 Qualitätskriterien]{Die Vertrauenswürdigkeit und Kontinuität der Dimensionsreduktion, sowie das Local Continuity Meta-Criterion (LCMC) für den FER 2013 Datensatz. Hier ist die Performance des Convolutional Autoencoders und der Hauptkomponentenanalyse nahezu identisch und im Vergleich mit den anderen Methoden am besten. Aber auch der Autoencoder und die Kernel PCA können einen konstant hohen Wert für $T(K)$ und $C(K)$ erreichen. Auch auf diesem Datensatz schneiden LLE und der Contractive Autoencoder mit Abstand am schlechtesten. Besonders LLE zeigt hier die Schwäche einer lokalen Erhaltung der Struktur, da die Qualitätskriterien für steigende Werte von $K$ sehr stark abfallen. (Eigene Darstellung)}
	\label{fig:FER2013Metrics}
\end{figure}